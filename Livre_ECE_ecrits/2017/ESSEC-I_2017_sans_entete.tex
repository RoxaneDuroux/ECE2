\chapter*{ESSEC-I 2017 : le sujet}
  
%

\noindent
Est-il possible que le marketing digital pose des problèmes de
sécurité des données personnelles ? De récents travaux % \footnote{Par
% exemple, A. Korolova. Privacy violations using microtargeted ads: A
% case study (2010)}
mettant en cause les outils de mesure de performance en temps réel des
différentes campagnes de publicité sur internet, démontrent que
certaines données très sensibles (préférences religieuses, sexuelles,
etc.) peuvent être obtenues par des segmentations précises des
audiences et sans aucune action de la part de l'utilisateur.\\[.2cm]
Dans ce problème, nous nous intéressons à une méthode proposée pour
protéger ces données, méthode baptisée {\bf confidentialité
  différentielle}.\\[.2cm]
Les parties I et II sont totalement indépendantes. Vous trouverez une
aide \Scilab{} en fin de sujet.\\[.2cm]
On considère un espace probabilisé $(\Omega, \A , \Prob)$ sur lequel
sont définies les variables aléatoires qui apparaissent dans l'énoncé.

\section*{Partie I - Lois de Laplace - propriétés et simulation}

\noindent 
Soit $\alpha\in\R$ et $\beta>0$. On dit qu'une variable aléatoire
réelle à densité suit une loi de Laplace de paramètre
$(\alpha,\beta)$, notée $\cal L(\alpha,\beta)$, si elle admet comme
densité la fonction $f$ donnée par:
\[
\forall t \in \R, \ f(t) = \dfrac{1}{2{\beta}} \ \exp\left(-
  \dfrac{|t-\alpha|}{\beta} \right)
\]
\begin{noliste}{1.}
  \setlength{\itemsep}{4mm}
\item Vérifier que $f$ est bien une densité de probabilité d'une
  variable aléatoire réelle.
  
  

\item Déterminer la fonction de répartition, notée $\Psi$, de la loi
  ${\cal L}(0,1)$.

  
  
\item On suppose que $X$ suit la loi ${\cal L} (0,1)$.
  
  \begin{noliste}{a)}
    \setlength{\itemsep}{2mm} %
  \item Montrer que $\beta X+\alpha$ suit la loi ${\cal
      L}(\alpha,\beta)$.

    
  
  
  
  %\newpage
  
  
  
\item En déduire la fonction de répartition de la loi ${\cal
    L}(\alpha,\beta)$.
  
  
\end{noliste}

\item {\em Espérance et variance.}
  \begin{noliste}{a)}
  \setlength{\itemsep}{2mm} %
  \item On suppose que $X$ suit la loi ${\cal L}(0,1)$.\\
    Montrer que $\E(X)$ et $\V(X)$ existent et valent respectivement
    $0$ et $2$.

    

  \item En déduire l'existence et les valeurs de l'espérance et de la
    variance d'une variable aléatoire réelle qui suit la loi ${\cal
      L}(\alpha,\beta)$.

    
  \end{noliste}

\item {\em Simulation à partir d'une loi exponentielle.}\\
  Soit $U$ une variable aléatoire qui suit la loi exponentielle de
  paramètre $1$ et $V$ une variable aléatoire qui suit la loi de
  Bernoulli de paramètre $\frac{1}{2}$ et indépendante de $U$.
  \begin{noliste}{a)}
    \setlength{\itemsep}{2mm} %
  \item En utilisant le système complet naturellement associé à $V$,
    montrer que $X = (2V-1) \ U$ suit la loi ${\cal L}(0,1)$.

    

  \item Compléter la définition \Scilab{} ci-dessous pour que la
    fonction ainsi définie réalise la simulation d'une variable
    aléatoire qui suit la loi ${\cal L}(\alpha,\beta)$ :

    \begin{scilab}
      & \tcFun{function} \tcVar{r} =
      \underline{Laplace}(\tcVar{alpha}, \tcVar{beta}) \nl %
      & \qquad \tcIf{if} ------ <= 1/2 \nl %
      & \qquad \qquad V = 1 \nl %
      & \qquad \tcIf{else} \nl %
      & \qquad \qquad V = 0 \nl %
      & \qquad \tcIf{end} \nl %
      & \qquad X = (2 \Sfois{} V - 1) \Sfois{} grand(1, 1,
      \ttq{}exp\ttq{}, 1) \nl %
      & \qquad \tcVar{r} = ------ \nl %
      & \tcFun{endfunction}      
    \end{scilab}
    
    
  \end{noliste}
\end{noliste}

\section*{Partie II - Lois $\eps$-différentielles}

\noindent
Soit $\eps>0$. On dit que $(X,Y)$, un couple de variables aléatoires,
est un couple $\eps$-différentiel si, pour tout intervalle $I$ de
$\R$:
\[
\ee^{-\eps} \ \Prob(\Ev{X\in I}) \ \leq \ \Prob(\Ev{Y \in I}) \ \leq \
\ee^{\eps} \ \Prob(\Ev{X\in I})
\]
Intuitivement, les lois de $X$ et $Y$ seront d'autant plus proches que
le plus petit $\eps$ tel que $(X,Y)$ soit un couple
$\eps$-différentiel est proche de $0$.
\begin{noliste}{1.}
  \setlength{\itemsep}{4mm} %
  \setcounter{enumi}{5}
\item Soit $(X,Y,Z)$ un triplet de variables aléatoires réelles.
  \begin{noliste}{a)}
    \setlength{\itemsep}{2mm} %
  \item Montrer que si $(X,Y)$ est $\eps$-différentiel alors $(Y,X)$
    l'est aussi.

    
    
    
    %\newpage
    
    

  \item Montrer que si $(X,Y)$ est $\eps$-différentiel et $(Y,Z)$ est
    $\eps'$-différentiel alors $(X,Z)$ est $(\eps +
    \eps')$-différentiel.

    
  \end{noliste}

\item Soit $(X,Y)$ un couple de variables aléatoires réelles
  discrètes.\\
  On suppose que $X(\Omega) \ \cup \ Y(\Omega) = \{z_n \ |
  \ n\in J\}$ où $J$ est un sous-ensemble non vide de $\N$.\\[.2cm]
  Montrer que $(X,Y)$ est $\eps$-différentiel si et seulement si
  \[
  \forall n\in J, \ \ee^{-\eps} \ \Prob(\Ev{X=z_n}) \ \leq \
  \Prob(\Ev{Y=z_n}) \ \leq \ \ee^{\eps} \ \Prob(\Ev{X = z_n})
  \]

  

\item {\em Premier exemple.}\\
  Dans cette question, on suppose que $X$ suit la loi géométrique de
  paramètre $\frac{1}{2}$, $Z$ suit la loi de Bernoulli de paramètre
  $p \in \ ]0,1[$ et elles sont indépendantes. On pose $Y = X + Z$.
  \begin{noliste}{a)}
    \setlength{\itemsep}{2mm} %
  \item Déterminer la loi de $Y$.

    

  \item Établir que pour tout $k\in\N^*$, $1-p \ \leq \
    \dfrac{\Prob(\Ev{Y = k})}{\Prob(\Ev{X = k})} \ \leq \
    \dfrac{1}{1-p}$.

    
    
    
    %\newpage
    

  \item En déduire que $(X,Y)$ est $-\ln(1-p)$-différentiel.

    

  \item Que ce passe-t-il lorsque $p$ s'approche de $0$ ou lorsqu'il
    s'approche de $1$? Était-ce prévisible?

    
  \end{noliste}
  
\item On suppose que $X$ et $Y$ sont deux variables à densité de
  densités respectives $f$ et $g$ et de fonction de répartition $F$ et
  $G$.
  \begin{noliste}{a)}
    \setlength{\itemsep}{2mm} %
  \item On suppose que pour tout $t\in\R$ : $\ee^{-\eps} f(t) \ \leq \
    g(t) \ \leq \ \ee^{\eps}f(t)$.\\
    Montrer que $(X,Y)$ est $\eps$-différentiel.

  
\newpage
    

  \item On suppose dans la suite de cette question que $(X,Y)$ est
    $\eps$-différentiel.\\
    Soit $h>0$ et $t \in \R$ où $f$ et $g$ sont continues.\\[.2cm]
    Montrer que:
    \[
    \ee^{-\eps} \ \dfrac{F(t+h)-F(t)}{h} \ \leq \
    \dfrac{G(t+h)-G(t)}{h} \ \leq \ \ee^{\eps} \
    \dfrac{F(t+h)-F(t)}{h}
    \]
    En conclure que: \ $\ee^{-\eps} f(t) \ \leq \ g(t) \ \leq \
    \ee^{\eps} \ f(t)$.
  \end{noliste}

\item {\em Deuxième exemple: lois de Cauchy.}
  \begin{noliste}{a)}
    \setlength{\itemsep}{2mm} %
  \item Montrer que $\dint{-\infty}{+\infty} \dfrac{1}{t^2+1} \dt$
    converge. On admet que cette intégrale est égale à $\pi$.

    
    
    
    %\newpage
    
    

  \item On définit, pour $a > 0$, la fonction $f_a$ sur $\R$ par, pour
    tout $t \in \R$, $f_a(t) = \dfrac{a}{\pi \ (t^2+a^2)}$.\\
    Montrer que $f_a$ est une densité de probabilité d'une variable
    aléatoire à densité.

    

  \item On suppose que $X$ et $Y$ sont deux variables aléatoires
    admettant comme densités respectives $f_1$ et $f_a$ avec $a>1$.\\
    Montrer que $(X,Y)$ est $\ln(a)$-différentiel.

    
  \end{noliste}

\item {\em Une première interprétation.}\\
  On suppose que $(X,Y)$ est un couple $\eps$-différentiel et que $U$
  est une variable de Bernoulli de paramètre $p \in \ ]0,1[$
  indépendante de $X$ et $Y$.\\
  On définit la variable aléatoire $Z$ par:
  \[
  \forall \omega\in\Omega, \ Z(\omega) = %
  \left\{
    \begin{array}{cR{2cm}}
      X(\omega) & si $U(\omega) = 1$ \nl
      Y(\omega) & sinon.
    \end{array}
  \right.
  \]

  \begin{noliste}{a)}
    \setlength{\itemsep}{2mm} %
  \item Soit $I$ un intervalle de $\R$ telle que $\Prob([Z\in I])\neq
    0$.\\[.2cm]
    Montrer que : 
    \[
    \Prob_{\Ev{Z\in I}}(\Ev{U = 1}) = p \ \dfrac{\Prob(\Ev{X\in I})}{p
      \ \Prob(\Ev{X\in I}) + (1-p) \ \Prob(\Ev{Y\in I})}
    \]
    En déduire que:
    \[
    \dfrac{p}{p + (1-p) \ \ee^{\eps}} \ \leq \ \Prob_{\Ev{Z\in
        I}}(\Ev{U = 1}) \ \leq \ \dfrac{p}{p + (1-p) \ \ee^{-\eps}}
    \]

    

  \item Si $\eps$ est proche de zéro, le fait de disposer d'une
    information sur la valeur de $Z$ change-t-il notablement le
    paramètre de la loi de $U$ et par conséquent la probabilité d'en
    déduire la valeur prise par $U$ ?

    
  \end{noliste}
\end{noliste}

%   \masque{
%   \item On considère une population qui comporte une proportion $p$,
%     inconnue, d'individus qui vérifie une certaine propriété (atteint
%     d'une anomalie génétique, soumis à l'ISF, etc...). On demande à
%     chaque individu de cette population de s'identifier et le lancer
%     une pièce de monnaie jusqu'à obtenir pile. Si $y$ est le nombre de
%     lancers effectués, l'individu saisit $y$ s'il ne vérifie pas la
%     propriété et $y+1$ s'il la vérifie.
%     \begin{noliste}
%     \item Si l'on choisit un individu au hasard, soit $Y$ le nombre
%       aléatoire de lancers qu'il effectuera et $Z$ la variable de
%       Bernoulli qui vaut $1$ s'il vérifie la propriété et $0$
%       sinon. On pose $Y=1+Z$ si $X=1$ et $Y=X-1+Z$ si $Z\eq 2$.
%     \item En utilisant, éventuellement, les calculs effectués dans la
%       question 8, montrer que:
%  $$\Prob_{[Y=k]}([Z=1])=\begin{cases}
%    \frac{2p}{1+p}\\
	
%  \end{cases}
% $$ 
% \end{noliste}
% }
	
%\suspend{noliste}


%\newpage


\section*{Partie III - Confidentialité différentielle}

\begin{noliste}{$\sbullet$}
\item Soit $d\in\N^*$. On considère $D = \llb 0 ,d \rrb$ et $n$ un
  entier naturel plus grand que $2$.
  
\item On dira que deux éléments de $D^n$, $a$ et $b$, sont voisins
  s'ils ne différent que d'une composante au plus. On note $\cal V$
  l'ensemble des couples de voisins.
\item On considère $q$ une application de $D^n$ dans $\R$.\medskip
\end{noliste}
Concrètement, un élément de $D^n$ représente une table d'une base de
données et $q$ une requête sur cette base. Étant donné $a=(a_1, \ldots
,a_n)$, on s'intéresse au problème de la confidentialité de certains
des $a_i$ lorsque les autres $a_i$ sont connus, ainsi que $D$, $q$ et
$q(a)$.


\newpage


\begin{noliste}{1.}
  \setlength{\itemsep}{4mm} %
  \setcounter{enumi}{11}
\item Dans cette question on suppose que $a_2,\ldots,a_n$ sont connus et
  on cherche à protéger $a_1$.
  \begin{noliste}{a)}
  \setlength{\itemsep}{2mm} %
  \item Quelle est probabilité d'obtenir la bonne valeur de $a_1$ si
    l'on choisit une valeur au hasard dans $\llb 0, d \rrb$ ?

    


    %\newpage


  \item Dans cette question $q(a_1,\ldots,a_n) = \Sum{i=1}{n} a_i$.\\
    Montrer que si $q(a)$ est publique alors on sait déterminer la
    valeur de $a_1$.

    
  \end{noliste}
\end{noliste}
On dit que l'on dispose d'un procédé de $\eps$-confidentialité de
$D^n$ pour $q$ si:
\begin{liste}{$\sbullet$}
\item[(c1)] pour tout $a\in D^n$, on dispose d'une variable aléatoire
  réelle $X_a$;
\item[(c2)] pour tout $(a,b) \in {\cal V}$, $(X_a,X_b)$ est
  $\eps$-différentiel.
\item[(c3)] pour tout $a\in D^n$, $\E(X_a) = q(a)$.
\end{liste}~\\[-.4cm]
\begin{noliste}{1.}
  \setcounter{enumi}{12} %
  \setlength{\itemsep}{4mm}
\item {\em Majoration de la probabilité de trouver $a_1$.}\\
  Dans cette question, nous allons justifier en partie la
  terminologie.\\
  On suppose à nouveau que $a_2, \ldots ,a_n$ sont connus, que l'on
  cherche à protéger $a_1$ et que :
  \begin{noliste}{$\stimes$}
  \item le public connaît des d'intervalles $I_0,\ldots,I_d$ disjoints
    de réunion $\R$ tels qu'avec les valeurs fixées de $a_2, \ldots,
    a_n$, si $q(a) \in I_j$ alors $a_1 = j$. Cela signifie que si
    $q(a)$ est publique alors $a_1$ aussi.
  \item on dispose d'un procédé de $\eps$-confidentialité de $D^n$
    pour $q$ et que l'on rend $X_a$ publique à la place de $q(a)$.
  \end{noliste}
  On considère alors que l'expérience aléatoire modélisée par
  $(\Omega, \A, \Prob)$ comporte comme première étape le choix au
  hasard de $a_1$ dans $\llb 0, d \rrb$ et on définit:
  \begin{noliste}{$\stimes$}
  \item $A_1$ la variable aléatoire associée à ce choix;
  \item pour tout $j \in \llb 0, d \rrb$, $Y_j = X_{(j, a_2, \ldots,
      a_n)}$. \\
    On suppose que $A_1$ et $Y_j$ sont indépendantes pour tout $j\in
    D$.
  \item la variable aléatoire réelle $R$ par:
    \[
    \forall \omega\in\Omega, \text{ si } A_1(\omega) = j \text{ alors
      on détermine l'unique } k \text{ tel que } Y_j(\omega)\in I_k
    \text{ et on pose } R(\omega)=k
    \]
  \item $\theta = \Prob(\Ev{R = A_1})$.
  \end{noliste}


  %\newpage


  \begin{noliste}{a)}
    \setlength{\itemsep}{2mm} %
  \item Montrer que $\theta = \Sum{j=0}{d} \Prob(\Ev{Y_j\in I_j} \cap
    \Ev{A_1=j})$.

    

  \item En déduire que $\theta = \dfrac{1}{d+1} \Sum{j=0}{d}
    \Prob(\Ev{Y_j \in I_j})$.

    
    
    
    
    %\newpage
    

  \item En conclure que:
    \[
    \theta \ \leq \ \dfrac{1}{d+1} \ \left(\ee^{\eps} - (\ee^{\eps}-1)
      \ \Prob(\Ev{Y_0\in I_0}) \right) \ \leq \
    \dfrac{\ee^{\eps}}{d+1}
    \]

    

  \item On pose $\rho = \dfrac{1}{d+1}$ et $\tau = \dfrac{\theta -
      \rho}{\rho}$.\\
    Donner une majoration de $\tau$. Que représente cette quantité ?\\
    Qu'en déduire concernant la méthode de confidentialité présentée
    dans cette question lorsque $\eps$ est proche de $0$ ?

    
  \end{noliste}
\end{noliste}


\newpage


On pose $\delta = \dmax{(a,b)\in\cal V} |q(a)-q(b)|$ et on suppose que
$\delta > 0$.
\begin{noliste}{1.}
  \setlength{\itemsep}{4mm} %
  \setcounter{enumi}{13}
\item Dans cette question, pour tout $a\in D^n$, on pose $X_a = q(a) +
  Y$ où $Y$ suit la loi de Laplace de paramètre $(0, \beta)$.
  \begin{noliste}{a)}
    \setlength{\itemsep}{2mm} %
  \item Pour tout $a\in D^n$, déterminer $\E(X_a)$ et une densité de
    probabilité $f_a$ de la loi de $X_a$ en fonction de $q(a)$ et de
    $\beta$.
    
    
    
  \item Montrer que pour tout $t\in\R$ et $(a,b)\in {\cal V}$, $f_a(t)
    \ \leq \ \exp\left(\dfrac{\delta}{\beta}\right) f_b(t)$.\\
    En déduire que pour tout $(a,b)\in{\cal V}$, $(X_a,X_b)$ est
    $\dfrac{\delta}{\beta}$-différentiel.

    
  
  \item Comment choisir $\beta$ pour disposer alors d'un procédé de
    $\eps$-confidentialité de $D^n$ pour $q$?

    
  \end{noliste}

\item Dans cette question, pour tout $a = (a_1, \ldots ,a_n)$
  appartenant à $D^n$, $q(a) = \Sum{k=1}{n} a_k$.
  \begin{noliste}{a)}
    \setlength{\itemsep}{2mm} %
  \item Quelle est la valeur de $\delta$ ?

    
  \end{noliste}
  On utilise dans la suite le procédé de $\eps$-confidentialité tel
  qu'il a été défini dans la question \itbf{14.} mais au lieu de 
  publier la
  valeur $X_a$, on procède ainsi :
  \begin{noliste}{$\stimes$}
  \item si $X_a < \frac{1}{2}$ on publie $0$;
  \item si $X_a\in [k-\frac{1}{2}, k + \frac{1}{2}[$ où $k \in \llb 1,
    nd-1 \rrb$, on publie $k$;
  \item sinon on publie $nd$.	
  \end{noliste}
  \begin{noliste}{a)}
    \setlength{\itemsep}{2mm} %
    \setcounter{enumii}{1}
  \item Montrer que la valeur aléatoire $Z_a$ publiée vérifie:
    \[
    Z_a = %
    \left\{
      \begin{array}{cR{3.4cm}}
        0 & si $X_a < \frac{1}{2}$ 
        \nl
        \nl[-.2cm]
        \left \lfloor X_a + \frac{1}{2} \right\rfloor & si $X_a \in
        [\frac{1}{2}, nd - \frac{1}{2}[$
        \nl
        \nl[-.2cm]
        nd & si $X_a\geq nd-\frac{1}{2}$
      \end{array}
    \right.
    \]

    

  \item Écrire un script qui pour $d$, $n$ et $\eps$ saisis par
    l'utilisateur, génère une valeur aléatoire de $a \in D^n$ puis
    affiche $q(a)$ et $Z_a$.

    


    %\newpage


  \item Pour $n = 1000$, $d = 4$ et $\eps$ choisi par l'utilisateur,
    écrire un script qui estime la valeur moyenne de $\dfrac{|Z_a -
      q(a)|}{q(a)}$ (on considèrera que $q(a)$ est toujours non nul).
      
      {\bf N.B.} À titre d'information, on obtient le tableau de valeurs
suivant:
\[
\scalebox{.9}{$
  \begin{array}{|c|c|c|c|c|c|c|c|c|c|c|c|c|}
    \hline
    \eps & 0.1 &  0.2 &  0.3 &  0.4 &  0.5 &  0.6 &  0.7 &  0.8 &  0.9
    &  1 &  1.1 &  1.2 \\
    \hline
    \text{\footnotesize Moyenne} & 1.91 \% &  1\% &  0.6 \% &  0.5 \% & 
 
0.3
    \% &  0.3 \% & 0.28 \% &  0.2 \% &  0.2 \% &  0.19 \% &  0.17 \% &
    0.16 \% \\
    \hline
  \end{array}
  $}
\]
% \fbox{
%   \begin{tabular}{R{15.4cm}}
%     {\bf Aide Scilab.}\\[.2cm]
%     La fonction \Scilab{} {\tt grand} permet de simuler, en particulier,
%     les lois exponentielles et uniformes discrètes. Par exemple :
%     \begin{noliste}{$\stimes$}
%     \item {\tt grand(3,2,\ttq{}exp\ttq{},0.5)} renvoie une matrice 
% aléatoire
%       $(3,2)$ dont les coefficients sont des variables indépendantes qui
%       suivent la loi exponentielle d'espérance $0,5$.
%     \item {\tt grand(1,2,\ttq{}uin\ttq{},-1,3)} renvoie une matrice
%       aléatoire $(1,2)$ dont les coefficients sont des variables
%       indépendantes qui suivent la loi uniforme discrètes sur $\llb
%       -1, 3 \rrb$.
%     \end{noliste}
%   \end{tabular}
% } 
    
  \end{noliste}
\end{noliste}%

