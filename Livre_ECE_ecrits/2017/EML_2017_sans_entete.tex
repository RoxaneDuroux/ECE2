\chapter*{EML 2017 : le sujet}
  
%

\section*{Exercice 1}
\noindent
On considère la fonction $f : \ ]0,+\infty[ \ \rightarrow \R$ définie,
pour tout $x$ de $]0,+\infty[$, par :
\[
f(x)=\ee^x-\ee \ln(x).
\]
On admet les encadrements numériques suivants :
\[
2,7<\ee<2,8 \qquad 7,3<\ee^2<7,4 \qquad 0,6<\ln(2)<0,7.
\]

\subsection*{Partie I : Étude de la fonction $f$}

\begin{noliste}{1.}
  \setlength{\itemsep}{2mm}
\item 
  \begin{noliste}{a)}
  \item Montrer que $f$ est deux fois dérivable sur $]0,+\infty[$ et
    calculer, pour tout $x$ de $]0,+\infty[$,\\
    $f'(x)$ et $f''(x)$.
    
    	
    
  \item Dresser le tableau de variations de $f'$ avec la limite de
    $f'$ en $0$ et la limite de $f'$ en $+\infty$ et \\ préciser
    $f'(1)$.
    
    
  \end{noliste}
  
  
  %\newpage
  

\item Dresser le tableau de variations de $f$ avec la limite de $f$ en
  $0$ et la limite de $f$ en $+\infty$ et\\
  préciser $f(1)$.
  
  
  
\item Tracer l'allure de la courbe représentative de $f$.
  
  
    
    
  \item 
    \begin{noliste}{a)}
    \item Étudier les variations de la fonction $u:
    \begin{array}[t]{ccl}
      ]0,+\infty[ & \rightarrow & \R\\
      x & \mapsto & f'(x)-x
    \end{array}$
    
    
    

    %\newpage

	
  \item En déduire que l'équation $f'(x)=x$, d'inconnue $x\in \
    ]0,+\infty[$, admet une solution et une seule, notée $\alpha$, et
    montrer : $1<\alpha<2$.
  \end{noliste}      
 
    
\end{noliste}


%\newpage


\subsection*{Partie II : Étude d'une suite, étude d'une série}
\noindent
On considère la suite réelle $(u_n)_{n\in\N}$ définie par :
\[
\left\{
  \begin{array}{l}
    u_0=2\\
    \forall n \in\N, \ u_{n+1}=f(u_n)
  \end{array}
\right.
\]
\begin{noliste}{1.}
  \setlength{\itemsep}{2mm}
  \setcounter{enumi}{4}
\item Montrer que, pour tout $n$ de $\N$, $u_n$ existe et $u_n\geq 2$.
  
  
  
\item 
  \begin{noliste}{a)}
  \item Étudier les variations, puis le signe, de la fonction $g:
    \begin{array}[t]{ccl}
      [2,+\infty[ & \rightarrow & \R\\
      x& \mapsto & f(x)-x
    \end{array}$
    
    
    
  \item En déduire que la suite $(u_n)_{n\in\N}$ est croissante.
	
    
  \end{noliste}
  
  
\item Démontrer que la suite $(u_n)_{n\in\N}$ admet $+\infty$ pour 
  limite.
  
  

\item Écrire un programme en \Scilab{} qui, étant donné un réel $A$,
  renvoie un entier naturel $N$ tel que $u_N\geq A$.




\item 
  \begin{noliste}{a)}
  \item Démontrer : $\forall x\in[2,+\infty[$, $2\ln(x)\leq x \leq
    \dfrac{\ee^x}{3}$.
    
    
    
  \item En déduire : $\forall n\in\N$, $u_{n+1} \geq
    \dfrac{6-\ee}{2}u_n$.
	
    
    

    %\newpage
	

  \item Déterminer la nature de la série de terme général
    $\dfrac{1}{u_n}$.
	
    
\end{noliste}
\end{noliste}


\newpage


\subsection*{Partie III : Étude d'intégrales généralisées}
\begin{noliste}{1.}
\setlength{\itemsep}{2mm}
\setcounter{enumi}{9}
\item Montrer que l'intégrale $\dint{0}{1} f(x)\dx$ converge et 
calculer 
cette intégrale.


 
\item L'intégrale $\dint{1}{+\infty} f(x)\dx$ converge-t-elle ?

  

%\newpage

\item Montrer que l'intégrale $\dint{2}{+\infty} \dfrac{1}{f(x)}\dx$
  converge.  On pourra utiliser le résultat de la question
  \itbf{9.a)}.


\end{noliste}

%\newpage

\subsection*{Partie IV : Étude d'une fonction de deux variables
  réelles}

\noindent
On considère la fonction $F: \ ]1,+\infty[^2 \rightarrow \R$, de classe 
$\Cont 2$ sur l'ouvert $]1,+\infty[^2$, définie, pour tout $(x,y)$ de 
$]1,+\infty[^2$, par :
\[
F(x,y)=f(x)+f(y)-xy.
\]
\begin{noliste}{1.}
\setlength{\itemsep}{2mm}
\setcounter{enumi}{12}
\item Montrer que $F$ admet un point critique et un seul et qu'il 
s'agit 
de $(\alpha,\alpha)$, le réel $\alpha$ ayant été défini à la question 
{\bf 4.} de la partie {\bf I}.



%\newpage

\item \begin{noliste}{a)}
	\item Déterminer la matrice hessienne de $F$ en 
	$(\alpha,\alpha)$.
	
	
	
	\item La fonction $F$ admet-elle un extremum local en 
	$(\alpha,\alpha)$ ? Si oui, s'agit-il d'un maximum local ou 
	s'agit-il d'un minimum local ?
	
	
	\end{noliste}
\end{noliste}


%\newpage


\section*{Exercice 2}

\noindent
On note $E = \R_2[X]$ l'espace vectoriel des polynômes de degré
inférieur ou égal à $2$ et $\B = (1,X,X^2)$ la base canonique de $E$.
Pour tout polynôme $P$ de $E$, on note indifféremment $P$ ou $P(X)$.\\
Pour tout $(\alpha,\beta,\gamma)\in\R^3$, la dérivée $P'$ du polynôme
$P=\alpha+\beta X+\gamma X^2$ est le polynôme $P' = \beta+2\gamma X$,
et la dérivée seconde $P''$ de $P$ est le polynôme $P'' = 2\gamma$.\\
On note, pour tout polynôme $P$ de $E$ :
\[
a(P) = P-XP', \quad b(P) = P-P', \quad c(P) = 2XP-(X^2-1) \ P'
\]
Par exemple : $a(X^2) = X^2-X(2X) = -X^2$.\\
Enfin, on note $f = b\circ a - a \circ b$.




\subsection*{Partie I : Étude de $a$}

\begin{noliste}{1.}
  \setlength{\itemsep}{2mm}
\item Montrer que $a$ est un endomorphisme de $E$.
  
  


%\newpage


\item
  \begin{noliste}{a)}
  \item Montrer que la matrice $A$ de $a$ dans la base $\B$ de $E$ est
    $A = 
    \begin{smatrix} 
      1 & 0 & 0 \\ 
      0 & 0 & 0 \\ 
      0 & 0 & -1
    \end{smatrix}$.
    
    
    
  \item Déterminer le rang de la matrice $A$.
    
    
  \end{noliste}
  
\item L'endomorphisme $a$ est-il bijectif ? Déterminer $\kr(a)$ et
  $\im(a)$.

  
  On admet, pour la suite de l'exercice, que $b$ et $c$ sont des
  endomorphismes de $E$.\\
  On note $B$ et $C$ les matrices, dans la base $\B$ de $E$, de $b$ et
  $c$ respectivement.
\end{noliste}


%\newpage


\subsection*{Partie II : Étude de $b$}

\begin{noliste}{1.}
  \setlength{\itemsep}{2mm} %
  \setcounter{enumi}{3}
\item Montrer que $b$ est bijectif et que, pour tout $Q$ de $E$, on a
  : $b^{-1}(Q) = Q + Q' + Q''$.

  


  %\newpage


\item 
  \begin{noliste}{a)}
  \item Montrer que $b$ admet une valeur propre et une seule et
    déterminer celle-ci.
	
    
    
  \item L'endomorphisme $b$ est-il diagonalisable ?
    
    
  \end{noliste}
\end{noliste}


\newpage


\subsection*{Partie III : Étude de $c$}

\begin{noliste}{1.}
\setlength{\itemsep}{2mm}
\setcounter{enumi}{5}
\item Montrer : $C=\begin{smatrix}
0 & 1 & 0\\ 
2 & 0 & 2\\ 
0 & 1 & 0
\end{smatrix}$.



\item L'endomorphisme $c$ est-il bijectif ?

  


%\newpage


\item 
  \begin{noliste}{a)}
  \item Déterminer une matrice $R$, carrée d'ordre trois, inversible,
    dont les coefficients de la première ligne sont tous égaux à $1$,
    et une matrice $D$, carrée d'ordre trois, diagonale, à
    coefficients diagonaux dans l'ordre croissant, telles que $C =
    RDR^{-1}$.
    
    
    

    %\newpage

	
  \item En déduire que l'endomorphisme $c$ est diagonalisable et
    déterminer une base de $E$ constituée de vecteurs propres de
    $c$.
    
    
  \end{noliste}
\end{noliste}


\subsection*{Partie IV : Étude de $f$}
\begin{noliste}{1.}
\setlength{\itemsep}{2mm}
\setcounter{enumi}{8}
\item Montrer : $\forall P\in E$, $f(P)=P'$.

  


%\newpage 


\item En déduire : $(BA-AB)^3=0$.

  
\end{noliste}



%\newpage

\section*{Exercice 3}

\noindent
On considère une urne contenant initialement une boule bleue et deux
boules rouges.\\
On effectue, dans cette urne, des tirages successifs de la façon
suivante : on pioche une boule au hasard et on note sa couleur, puis
on la replace dans l'urne en ajoutant une boule de la même couleur que
celle qui vient d'être obtenue.\\[.1cm]
Pour tout $k$ de $\N^*$, on note :
\begin{tabular}[t]{R{12cm}}
  $B_k$ l'événement : \og on obtient une boule bleue au $\eme{k}$ tirage \fg{}, 
  \nl
  \nl[-.3cm]
  $R_k$ l'événement : \og on obtient une boule rouge au $\eme{k}$
  tirage \fg{}.
\end{tabular}

\subsection*{Partie I : Simulation informatique}

\begin{noliste}{1.}
\item Recopier et compléter la fonction suivante afin qu'elle simule
  l'expérience étudiée et renvoie le nombre de boules rouges obtenues
  lors des $n$ premiers tirages, l'entier $n$ étant entré en argument.
  \begin{scilab}
    & \tcFun{function} \tcVar{s} = EML(\tcVar{n}) \nl %
    & \qquad b = 1 \commentaire{b désigne le nombre de boules bleues
      présentes dans l'urne} \nl %
    & \qquad r = 2 \commentaire{r désigne le nombre de boules rouges
      présentes dans l'urne} \nl %
    & \qquad \tcVar{s} = 0 \commentaire{s désigne le nombre de boules 
    rouges obtenues lors des n tirages} \nl %
    & \qquad \tcFor{for} k = 1:\tcVar{n} \nl %
    & \qquad \qquad x = rand() \nl %
    & \qquad \qquad \tcIf{if} ... \tcIf{then} \nl %
    & \qquad \qquad \qquad  ... \nl %
    & \qquad \qquad \tcIf{else} \nl %
    & \qquad \qquad \qquad  ... \nl %
    & \qquad \qquad \tcIf{end} \nl %
    & \qquad \tcFor{end} \nl %
    & \tcFun{endfunction}
  \end{scilab}

  
  \newpage
  

\item On exécute le programme suivant :
  \begin{scilab}
    & n = 10 \nl %
    & m = 0 \nl %
    & \tcFor{for} i = 1:1000 \nl % 
    & \qquad m = m + EML(n) \nl %
    & \tcFor{end} \nl %
    & disp(m/1000)
  \end{scilab}
  On obtient $6.657$. Comment interpréter ce résultat ?
  
  
\end{noliste}

\subsection*{Partie II : Rang d'apparition de la première boule bleue 
et rang d'apparition de la première boule rouge}

\noindent 
On définit la variable aléatoire $Y$ égale au rang d'apparition de la
première boule bleue et la variable aléatoire $Z$ égale au rang
d'apparition de la première boule rouge.

\begin{noliste}{1.}
  \setcounter{enumi}{2}
\item
  \begin{noliste}{a)}
  \item Montrer que : \ $\forall n \in \N^*, \ \Prob(\Ev{Y = n}) =
    \dfrac{2}{(n+1)(n+2)}$.

    


    %\newpage


  \item La variable aléatoire $Y$ admet-elle une espérance ? une
    variance ?
    
    
  \end{noliste}

\item Déterminer la loi de $Z$. La \var admet-elle une espérance ? une
  variance ?
 
  
\end{noliste}

\subsection*{Partie III : Nombre de boules rouges obtenues au cours de 
$n$ tirages}

\noindent 
On définit, pour tout $k$ de $\N^*$, la variable aléatoire $X_k$ égale
à $1$ si l'on obtient une boule rouge au $\eme{k}$ tirage et égale à
$0$ sinon.\\
On définit, pour tout $n$ de $\N^*$, la variable aléatoire $S_n$ égale
au nombre de boules rouges obtenues au cours des $n$ premiers tirages.
\begin{noliste}{1.}
  \setcounter{enumi}{4}
\item Donner, pour tout $n$ de $\N^*$, une relation entre $S_n$ et
  certaines variables aléatoires $X_k$ pour $k \in \N^*$.
  
  

\item Déterminer la loi de $X_1$, son espérance et sa variance.

  

\item
  \begin{noliste}{a)}
  \item Déterminer la loi du couple $(X_1, X_2)$.

    

  \item En déduire la loi de $X_2$.

    

  \item Les variables $X_1$ et $X_2$ sont-elles indépendantes ?

    
  \end{noliste}

\item Soit $n \in \N^*$ et $k \in \llb 0, n \rrb$.
  \begin{noliste}{a)}
  \item Calculer $\Prob(R_1 \cap \ldots \cap R_k \cap B_{k+1} \cap
    \ldots \cap B_n)$.

    


    %\newpage


  \item Justifier : $\Prob(\Ev{S_n = k}) = \dbinom{n}{k} \ \Prob(R_1
    \cap \ldots \cap R_k \cap B_{k+1} \cap \ldots \cap B_n)$,\\
    puis en déduire : $\Prob(\Ev{S_n = k}) =
    \dfrac{2(k+1)}{(n+1)(n+2)}$.

    
  \end{noliste}


  %\newpage


\item Montrer que, pour tout $n$ de $\N^*$, $S_n$ admet une espérance
  et : $\E(S_n) = \dfrac{2n}{3}$.

  

\item Soit $n \in \N^*$.
  \begin{noliste}{a)}
  \item Montrer : \ $\forall k \in \llb 0, n \rrb$, \ $\Prob_{\Ev{S_n
        = k}}(\Ev{X_{n+1} = 1}) = \dfrac{k+2}{n+3}$.
    
    
        

    %\newpage


  \item En déduire : \ $\Prob(\Ev{X_{n+1} = 1}) = \dfrac{\E(S_n) +
      2}{n+3}$.
      
      

  \item Déterminer alors la loi de la variable aléatoire
    $X_{n+1}$. Que remarque-t-on ?
    
            
  \end{noliste}
\end{noliste}


\newpage


\subsection*{Partie IV : Étude d'une convergence en loi}
\noindent
On s'intéresse dans cette partie à la proportion de boules rouges 
obtenues lors des $n$ premiers tirages.\\[.1cm]
On pose, pour tout $n$ de $\N^*$, $T_n=\dfrac{S_n}{n}$.
\begin{noliste}{1.}
\setcounter{enumi}{10}
\item Justifier, pour tout $n$ de $\N^*$ : $\forall x<0$, 
$\Prob(\Ev{T_n\leq 
x})=0$, et : $\forall x>1$, $\Prob(\Ev{T_n\leq x})=1$.



\item Soit $x\in[0,1]$. Montrer, pour tout $n$ de $\N^*$ :
\[
\Prob(\Ev{T_n\leq x})=\frac{(\lfloor nx\rfloor +1)(\lfloor nx\rfloor 
+2)}{(n+1)(n+2)},
\]
où $\lfloor . \rfloor$ désigne la fonction partie entière.



\item En déduire que la suite de variables aléatoires 
$(T_n)_{n\in\N^*}$ 
converge en loi vers une variable aléatoire à densité, dont on 
précisera 
la fonction de répartition et une densité.


\end{noliste}



