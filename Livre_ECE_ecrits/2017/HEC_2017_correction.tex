\documentclass[11pt]{article}%
\usepackage{geometry}%
\geometry{a4paper,
 lmargin = 2cm,rmargin = 2cm,tmargin = 2.5cm,bmargin = 2.5cm}

\usepackage{array}
\usepackage{paralist}

\usepackage[svgnames, usenames, dvipsnames]{xcolor}
\xdefinecolor{RecColor}{named}{Aqua}
\xdefinecolor{IncColor}{named}{Aqua}
\xdefinecolor{ImpColor}{named}{PaleGreen}

% \usepackage{frcursive}

\usepackage{adjustbox}

%%%%%%%%%%%
\newcommand{\cRB}[1]{{\color{Red} \pmb{#1}}} %
\newcommand{\cR}[1]{{\color{Red} {#1}}} %
\newcommand{\cBB}[1]{{\color{Blue} \pmb{#1}}}
\newcommand{\cB}[1]{{\color{Blue} {#1}}}
\newcommand{\cGB}[1]{{\color{LimeGreen} \pmb{#1}}}
\newcommand{\cG}[1]{{\color{LimeGreen} {#1}}}

%%%%%%%%%%

\usepackage{diagbox} %
\usepackage{colortbl} %
\usepackage{multirow} %
\usepackage{pgf} %
\usepackage{environ} %
\usepackage{fancybox} %
\usepackage{textcomp} %
\usepackage{marvosym} %

%%%%%%%%%% pour qu'une cellcolor ne recouvre pas le trait du tableau
\usepackage{hhline}%

\usepackage{pgfplots}
\pgfplotsset{compat=1.10}
\usepgfplotslibrary{patchplots}
\usepgfplotslibrary{fillbetween}
\usepackage{tikz,tkz-tab}
\usepackage{ifthen}
\usepackage{calc}
\usetikzlibrary{calc,decorations.pathreplacing,arrows,positioning} 
\usetikzlibrary{fit,shapes,backgrounds}
\usepackage[nomessages]{fp}% http://ctan.org/pkg/fp

\usetikzlibrary{matrix,arrows,decorations.pathmorphing,
  decorations.pathreplacing} 

\newcommand{\myunit}{1 cm}
\tikzset{
    node style sp/.style={draw,circle,minimum size=\myunit},
    node style ge/.style={circle,minimum size=\myunit},
    arrow style mul/.style={draw,sloped,midway,fill=white},
    arrow style plus/.style={midway,sloped,fill=white},
}

%%%%%%%%%%%%%%
%%%%% écrire des inférieur égal ou supérieur égal avec typographie
%%%%% francaise
%%%%%%%%%%%%%

\renewcommand{\geq}{\geqslant}
\renewcommand{\leq}{\leqslant}
\renewcommand{\emptyset}{\varnothing}

\newcommand{\Leq}{\leqslant}
\newcommand{\Geq}{\geqslant}

%%%%%%%%%%%%%%
%%%%% Macro Celia
%%%%%%%%%%%%%

\newcommand{\ff}[2]{\left[#1, #2\right]} %
\newcommand{\fo}[2]{\left[#1, #2\right[} %
\newcommand{\of}[2]{\left]#1, #2\right]} %
\newcommand{\soo}[2]{\left]#1, #2\right[} %
\newcommand{\abs}[1]{\left|#1\right|} %
\newcommand{\Ent}[1]{\left\lfloor #1 \right\rfloor} %


%%%%%%%%%%%%%%
%%%%% tikz : comment dessiner un "oeil"
%%%%%%%%%%%%%

\newcommand{\eye}[4]% size, x, y, rotation
{ \draw[rotate around={#4:(#2,#3)}] (#2,#3) -- ++(-.5*55:#1) (#2,#3)
  -- ++(.5*55:#1); \draw (#2,#3) ++(#4+55:.75*#1) arc
  (#4+55:#4-55:.75*#1);
  % IRIS
  \draw[fill=gray] (#2,#3) ++(#4+55/3:.75*#1) arc
  (#4+180-55:#4+180+55:.28*#1);
  % PUPIL, a filled arc
  \draw[fill=black] (#2,#3) ++(#4+55/3:.75*#1) arc
  (#4+55/3:#4-55/3:.75*#1);%
}


%%%%%%%%%%
%% discontinuité fonction
\newcommand\pointg[2]{%
  \draw[color = red, very thick] (#1+0.15, #2-.04)--(#1, #2-.04)--(#1,
  #2+.04)--(#1+0.15, #2+.04);%
}%

\newcommand\pointd[2]{%
  \draw[color = red, very thick] (#1-0.15, #2+.04)--(#1, #2+.04)--(#1,
  #2-.04)--(#1-0.15, #2-.04);%
}%

%%%%%%%%%%
%%% 1 : position abscisse, 2 : position ordonnée, 3 : taille, 4 : couleur
%%%%%%%%%%
% \newcommand\pointG[4]{%
%   \draw[color = #4, very thick] (#1+#3, #2-(#3/3.75))--(#1,
%   #2-(#3/3.75))--(#1, #2+(#3/3.75))--(#1+#3, #2+(#3/3.75)) %
% }%

\newcommand\pointG[4]{%
  \draw[color = #4, very thick] ({#1+#3/3.75}, {#2-#3})--(#1,
  {#2-#3})--(#1, {#2+#3})--({#1+#3/3.75}, {#2+#3}) %
}%

\newcommand\pointD[4]{%
  \draw[color = #4, very thick] ({#1-#3/3.75}, {#2+#3})--(#1,
  {#2+#3})--(#1, {#2-#3})--({#1-#3/3.75}, {#2-#3}) %
}%

\newcommand\spointG[4]{%
  \draw[color = #4, very thick] ({#1+#3/1.75}, {#2-#3})--(#1,
  {#2-#3})--(#1, {#2+#3})--({#1+#3/1.75}, {#2+#3}) %
}%

\newcommand\spointD[4]{%
  \draw[color = #4, very thick] ({#1-#3/2}, {#2+#3})--(#1,
  {#2+#3})--(#1, {#2-#3})--({#1-#3/2}, {#2-#3}) %
}%

%%%%%%%%%%

\newcommand{\Pb}{\mathtt{P}}

%%%%%%%%%%%%%%%
%%% Pour citer un précédent item
%%%%%%%%%%%%%%%
\newcommand{\itbf}[1]{{\small \bf \textit{#1}}}


%%%%%%%%%%%%%%%
%%% Quelques couleurs
%%%%%%%%%%%%%%%

\xdefinecolor{cancelcolor}{named}{Red}
\xdefinecolor{intI}{named}{ProcessBlue}
\xdefinecolor{intJ}{named}{ForestGreen}

%%%%%%%%%%%%%%%
%%%%%%%%%%%%%%%
% barrer du texte
\usetikzlibrary{shapes.misc}

\makeatletter
% \definecolor{cancelcolor}{rgb}{0.127,0.372,0.987}
\newcommand{\tikz@bcancel}[1]{%
  \begin{tikzpicture}[baseline=(textbox.base), inner sep=0pt]
    \node[strike out, draw] (textbox) {#1}[thick, color=cancelcolor];
    \useasboundingbox (textbox);
  \end{tikzpicture}%
}
\newcommand{\bcancel}[1]{%
  \relax\ifmmode
    \mathchoice{\tikz@bcancel{$\displaystyle#1$}}
               {\tikz@bcancel{$\textstyle#1$}}
               {\tikz@bcancel{$\scriptstyle#1$}}
               {\tikz@bcancel{$\scriptscriptstyle#1$}}
  \else
    \tikz@bcancel{\strut#1}%
  \fi
}
\newcommand{\tikz@xcancel}[1]{%
  \begin{tikzpicture}[baseline=(textbox.base),inner sep=0pt]
  \node[cross out,draw] (textbox) {#1}[thick, color=cancelcolor];
  \useasboundingbox (textbox);
  \end{tikzpicture}%
}
\newcommand{\xcancel}[1]{%
  \relax\ifmmode
    \mathchoice{\tikz@xcancel{$\displaystyle#1$}}
               {\tikz@xcancel{$\textstyle#1$}}
               {\tikz@xcancel{$\scriptstyle#1$}}
               {\tikz@xcancel{$\scriptscriptstyle#1$}}
  \else
    \tikz@xcancel{\strut#1}%
  \fi
}
\makeatother

\newcommand{\xcancelRA}{\xcancel{\rule[-.15cm]{0cm}{.5cm} \Rightarrow
    \rule[-.15cm]{0cm}{.5cm}}}

%%%%%%%%%%%%%%%%%%%%%%%%%%%%%%%%%%%
%%%%%%%%%%%%%%%%%%%%%%%%%%%%%%%%%%%

\newcommand{\vide}{\multicolumn{1}{c}{}}

%%%%%%%%%%%%%%%%%%%%%%%%%%%%%%%%%%%
%%%%%%%%%%%%%%%%%%%%%%%%%%%%%%%%%%%


\usepackage{multicol}
% \usepackage[latin1]{inputenc}
% \usepackage[T1]{fontenc}
\usepackage[utf8]{inputenc}
\usepackage[T1]{fontenc}
\usepackage[normalem]{ulem}
\usepackage[french]{babel}

\usepackage{url}    
\usepackage{hyperref}
\hypersetup{
  backref=true,
  pagebackref=true,
  hyperindex=true,
  colorlinks=true,
  breaklinks=true,
  urlcolor=blue,
  linkcolor=black,
  %%%%%%%%
  % ATTENTION : red changé en black pour le Livre !
  %%%%%%%%
  bookmarks=true,
  bookmarksopen=true
}

%%%%%%%%%%%%%%%%%%%%%%%%%%%%%%%%%%%%%%%%%%%
%% Pour faire des traits diagonaux dans les tableaux
%% Nécessite slashbox.sty
%\usepackage{slashbox}

\usepackage{tipa}
\usepackage{verbatim,listings}
\usepackage{graphicx}
\usepackage{fancyhdr}
\usepackage{mathrsfs}
\usepackage{pifont}
\usepackage{tablists}
\usepackage{dsfont,amsfonts,amssymb,amsmath,amsthm,stmaryrd,upgreek,manfnt}
\usepackage{enumerate}

%\newcolumntype{M}[1]{p{#1}}
\newcolumntype{C}[1]{>{\centering}m{#1}}
\newcolumntype{R}[1]{>{\raggedright}m{#1}}
\newcolumntype{L}[1]{>{\raggedleft}m{#1}}
\newcolumntype{P}[1]{>{\raggedright}p{#1}}
\newcolumntype{B}[1]{>{\raggedright}b{#1}}
\newcolumntype{Q}[1]{>{\raggedright}t{#1}}

\newcommand{\alias}[2]{
\providecommand{#1}{}
\renewcommand{#1}{#2}
}
\alias{\R}{\mathbb{R}}
\alias{\N}{\mathbb{N}}
\alias{\Z}{\mathbb{Z}}
\alias{\Q}{\mathbb{Q}}
\alias{\C}{\mathbb{C}}
\alias{\K}{\mathbb{K}}

%%%%%%%%%%%%
%% rendre +infty et -infty plus petits
%%%%%%%%%%%%
\newcommand{\sinfty}{{\scriptstyle \infty}}

%%%%%%%%%%%%%%%%%%%%%%%%%%%%%%%
%%%%% macros TP Scilab %%%%%%%%
\newcommand{\Scilab}{\textbf{Scilab}} %
\newcommand{\Scinotes}{\textbf{SciNotes}} %
\newcommand{\faire}{\noindent $\blacktriangleright$ } %
\newcommand{\fitem}{\scalebox{.8}{$\blacktriangleright$}} %
\newcommand{\entree}{{\small\texttt{ENTRÉE}}} %
\newcommand{\tab}{{\small\texttt{TAB}}} %
\newcommand{\mt}[1]{\mathtt{#1}} %
% guillemets droits

\newcommand{\ttq}{\textquotesingle} %

\newcommand{\reponse}[1]{\longboxed{
    \begin{array}{C{0.9\textwidth}}
      \nl[#1]
    \end{array}
  }} %

\newcommand{\reponseR}[1]{\longboxed{
    \begin{array}{R{0.9\textwidth}}
      #1
    \end{array}
  }} %

\newcommand{\reponseC}[1]{\longboxed{
    \begin{array}{C{0.9\textwidth}}
      #1
    \end{array}
  }} %

\colorlet{pyfunction}{Blue}
\colorlet{pyCle}{Magenta}
\colorlet{pycomment}{LimeGreen}
\colorlet{pydoc}{Cyan}
% \colorlet{SansCo}{white}
% \colorlet{AvecCo}{black}

\newcommand{\visible}[1]{{\color{ASCo}\colorlet{pydoc}{pyDo}\colorlet{pycomment}{pyCo}\colorlet{pyfunction}{pyF}\colorlet{pyCle}{pyC}\colorlet{function}{sciFun}\colorlet{var}{sciVar}\colorlet{if}{sciIf}\colorlet{comment}{sciComment}#1}} %

%%%% à changer ????
\newcommand{\invisible}[1]{{\color{ASCo}\colorlet{pydoc}{pyDo}\colorlet{pycomment}{pyCo}\colorlet{pyfunction}{pyF}\colorlet{pyCle}{pyC}\colorlet{function}{sciFun}\colorlet{var}{sciVar}\colorlet{if}{sciIf}\colorlet{comment}{sciComment}#1}} %

\newcommand{\invisibleCol}[2]{{\color{#1}#2}} %

\NewEnviron{solution} %
{ %
  \Boxed{
    \begin{array}{>{\color{ASCo}} R{0.9\textwidth}}
      \colorlet{pycomment}{pyCo}
      \colorlet{pydoc}{pyDo}
      \colorlet{pyfunction}{pyF}
      \colorlet{pyCle}{pyC}
      \colorlet{function}{sciFun}
      \colorlet{var}{sciVar}
      \colorlet{if}{sciIf}
      \colorlet{comment}{sciComment}
      \BODY
    \end{array}
  } %
} %

\NewEnviron{solutionC} %
{ %
  \Boxed{
    \begin{array}{>{\color{ASCo}} C{0.9\textwidth}}
      \colorlet{pycomment}{pyCo}
      \colorlet{pydoc}{pyDo}
      \colorlet{pyfunction}{pyF}
      \colorlet{pyCle}{pyC}
      \colorlet{function}{sciFun}
      \colorlet{var}{sciVar}
      \colorlet{if}{sciIf}
      \colorlet{comment}{sciComment}
      \BODY
    \end{array}
  } %
} %

\newcommand{\invite}{--\!\!>} %

%%%%% nouvel environnement tabular pour retour console %%%%
\colorlet{ConsoleColor}{Black!12}
\colorlet{function}{Red}
\colorlet{var}{Maroon}
\colorlet{if}{Magenta}
\colorlet{comment}{LimeGreen}

\newcommand{\tcVar}[1]{\textcolor{var}{\bf \small #1}} %
\newcommand{\tcFun}[1]{\textcolor{function}{#1}} %
\newcommand{\tcIf}[1]{\textcolor{if}{#1}} %
\newcommand{\tcFor}[1]{\textcolor{if}{#1}} %

\newcommand{\moins}{\!\!\!\!\!\!- }
\newcommand{\espn}{\!\!\!\!\!\!}

\usepackage{booktabs,varwidth} \newsavebox\TBox
\newenvironment{console}
{\begin{lrbox}{\TBox}\varwidth{\linewidth}
    \tabular{>{\tt\small}R{0.84\textwidth}}
    \nl[-.4cm]} {\endtabular\endvarwidth\end{lrbox}%
  \fboxsep=1pt\colorbox{ConsoleColor}{\usebox\TBox}}

\newcommand{\lInv}[1]{%
  $\invite$ #1} %

\newcommand{\lAns}[1]{%
  \qquad ans \ = \nl %
  \qquad \qquad #1} %

\newcommand{\lVar}[2]{%
  \qquad #1 \ = \nl %
  \qquad \qquad #2} %

\newcommand{\lDisp}[1]{%
  #1 %
} %

\newcommand{\ligne}[1]{\underline{\small \tt #1}} %

\newcommand{\ligneAns}[2]{%
  $\invite$ #1 \nl %
  \qquad ans \ = \nl %
  \qquad \qquad #2} %

\newcommand{\ligneVar}[3]{%
  $\invite$ #1 \nl %
  \qquad #2 \ = \nl %
  \qquad \qquad #3} %

\newcommand{\ligneErr}[3]{%
  $\invite$ #1 \nl %
  \quad !-{-}error #2 \nl %
  #3} %
%%%%%%%%%%%%%%%%%%%%%% 

\newcommand{\bs}[1]{\boldsymbol{#1}} %
\newcommand{\nll}{\nl[.4cm]} %
\newcommand{\nle}{\nl[.2cm]} %
%% opérateur puissance copiant l'affichage Scilab
%\newcommand{\puis}{\!\!\!~^{\scriptscriptstyle\pmb{\wedge}}}
\newcommand{\puis}{\mbox{$\hspace{-.1cm}~^{\scriptscriptstyle\pmb{\wedge}}
    \hspace{0.05cm}$}} %
\newcommand{\pointpuis}{.\mbox{$\hspace{-.15cm}~^{\scriptscriptstyle\pmb{\wedge}}$}} %
\newcommand{\Sfois}{\mbox{$\mt{\star}$}} %

%%%%% nouvel environnement tabular pour les encadrés Scilab %%%%
\newenvironment{encadre}
{\begin{lrbox}{\TBox}\varwidth{\linewidth}
    \tabular{>{\tt\small}C{0.1\textwidth}>{\small}R{0.7\textwidth}}}
  {\endtabular\endvarwidth\end{lrbox}%
  \fboxsep=1pt\longboxed{\usebox\TBox}}

\newenvironment{encadreL}
{\begin{lrbox}{\TBox}\varwidth{\linewidth}
    \tabular{>{\tt\small}C{0.25\textwidth}>{\small}R{0.6\textwidth}}}
  {\endtabular\endvarwidth\end{lrbox}%
  \fboxsep=1pt\longboxed{\usebox\TBox}}

\newenvironment{encadreF}
{\begin{lrbox}{\TBox}\varwidth{\linewidth}
    \tabular{>{\tt\small}C{0.2\textwidth}>{\small}R{0.70\textwidth}}}
  {\endtabular\endvarwidth\end{lrbox}%
  \fboxsep=1pt\longboxed{\usebox\TBox}}

\newenvironment{encadreLL}[2]
{\begin{lrbox}{\TBox}\varwidth{\linewidth}
    \tabular{>{\tt\small}C{#1\textwidth}>{\small}R{#2\textwidth}}}
  {\endtabular\endvarwidth\end{lrbox}%
  \fboxsep=1pt\longboxed{\usebox\TBox}}

%%%%% nouvel environnement tabular pour les script et fonctions %%%%
\newcommand{\commentaireDL}[1]{\multicolumn{1}{l}{\it
    \textcolor{comment}{$\slash\slash$ #1}}}

\newcommand{\commentaire}[1]{{\textcolor{comment}{$\slash\slash$ #1}}}

\newcounter{cptcol}

\newcommand{\nocount}{\multicolumn{1}{c}{}}

\newcommand{\sciNo}[1]{{\small \underbar #1}}

\NewEnviron{scilab}{ %
  \setcounter{cptcol}{0}
  \begin{center}
    \longboxed{
      \begin{tabular}{>{\stepcounter{cptcol}{\tiny \underbar
              \thecptcol}}c>{\tt}l}
        \BODY
      \end{tabular}
    }
  \end{center}
}

\NewEnviron{scilabNC}{ %
  \begin{center}
    \longboxed{
      \begin{tabular}{>{\tt}l} %
          \BODY
      \end{tabular}
    }
  \end{center}
}

\NewEnviron{scilabC}[1]{ %
  \setcounter{cptcol}{#1}
  \begin{center}
    \longboxed{
      \begin{tabular}{>{\stepcounter{cptcol}{\tiny \underbar
              \thecptcol}}c>{\tt}l}
        \BODY
      \end{tabular}
    }
  \end{center}
}

\newcommand{\scisol}[1]{ %
  \setcounter{cptcol}{0}
  \longboxed{
    \begin{tabular}{>{\stepcounter{cptcol}{\tiny \underbar
            \thecptcol}}c>{\tt}l}
      #1
    \end{tabular}
  }
}

\newcommand{\scisolNC}[1]{ %
  \longboxed{
    \begin{tabular}{>{\tt}l}
      #1
    \end{tabular}
  }
}

\newcommand{\scisolC}[2]{ %
  \setcounter{cptcol}{#1}
  \longboxed{
    \begin{tabular}{>{\stepcounter{cptcol}{\tiny \underbar
            \thecptcol}}c>{\tt}l}
      #2
    \end{tabular}
  }
}

\NewEnviron{syntaxe}{ %
  % \fcolorbox{black}{Yellow!20}{\setlength{\fboxsep}{3mm}
  \shadowbox{
    \setlength{\fboxsep}{3mm}
    \begin{tabular}{>{\tt}l}
      \BODY
    \end{tabular}
  }
}

%%%%% fin macros TP Scilab %%%%%%%%
%%%%%%%%%%%%%%%%%%%%%%%%%%%%%%%%%%%

%%%%%%%%%%%%%%%%%%%%%%%%%%%%%%%%%%%
%%%%% TP Python - listings %%%%%%%%
%%%%%%%%%%%%%%%%%%%%%%%%%%%%%%%%%%%
\newcommand{\Python}{\textbf{Python}} %

\lstset{% general command to set parameter(s)
basicstyle=\ttfamily\small, % print whole listing small
keywordstyle=\color{blue}\bfseries\underbar,
%% underlined bold black keywords
frame=lines,
xleftmargin=10mm,
numbers=left,
numberstyle=\tiny\underbar,
numbersep=10pt,
%identifierstyle=, % nothing happens
commentstyle=\color{green}, % white comments
%%stringstyle=\ttfamily, % typewriter type for strings
showstringspaces=false}

\newcommand{\pysolCpt}[2]{
  \setcounter{cptcol}{#1}
  \longboxed{
    \begin{tabular}{>{\stepcounter{cptcol}{\tiny \underbar
            \thecptcol}}c>{\tt}l}
        #2
      \end{tabular}
    }
} %

\newcommand{\pysol}[1]{
  \setcounter{cptcol}{0}
  \longboxed{
    \begin{tabular}{>{\stepcounter{cptcol}{\tiny \underbar
            \thecptcol}}c>{\tt}l}
        #1
      \end{tabular}
    }
} %

% \usepackage[labelsep=endash]{caption}

% avec un caption
\NewEnviron{pythonCap}[1]{ %
  \renewcommand{\tablename}{Programme}
  \setcounter{cptcol}{0}
  \begin{center}
    \longboxed{
      \begin{tabular}{>{\stepcounter{cptcol}{\tiny \underbar
              \thecptcol}}c>{\tt}l}
        \BODY
      \end{tabular}
    }
    \captionof{table}{#1}
  \end{center}
}

\NewEnviron{python}{ %
  \setcounter{cptcol}{0}
  \begin{center}
    \longboxed{
      \begin{tabular}{>{\stepcounter{cptcol}{\tiny \underbar
              \thecptcol}}c>{\tt}l}
        \BODY
      \end{tabular}
    }
  \end{center}
}

\newcommand{\pyVar}[1]{\textcolor{var}{\bf \small #1}} %
\newcommand{\pyFun}[1]{\textcolor{pyfunction}{#1}} %
\newcommand{\pyCle}[1]{\textcolor{pyCle}{#1}} %
\newcommand{\pyImp}[1]{{\bf #1}} %

%%%%% commentaire python %%%%
\newcommand{\pyComDL}[1]{\multicolumn{1}{l}{\textcolor{pycomment}{\#
      #1}}}

\newcommand{\pyCom}[1]{{\textcolor{pycomment}{\# #1}}}
\newcommand{\pyDoc}[1]{{\textcolor{pydoc}{#1}}}

\newcommand{\pyNo}[1]{{\small \underbar #1}}

%%%%%%%%%%%%%%%%%%%%%%%%%%%%%%%%%%%
%%%%%% Système linéaire paramétré : écrire les opérations au-dessus
%%%%%% d'un symbole équivalent
%%%%%%%%%%%%%%%%%%%%%%%%%%%%%%%%%%%

\usepackage{systeme}

\NewEnviron{arrayEq}{ %
  \stackrel{\scalebox{.6}{$
      \begin{array}{l} 
        \BODY \\[.1cm]
      \end{array}$}
  }{\Longleftrightarrow}
}

\NewEnviron{arrayEg}{ %
  \stackrel{\scalebox{.6}{$
      \begin{array}{l} 
        \BODY \\[.1cm]
      \end{array}$}
  }{=}
}

\NewEnviron{operationEq}{ %
  \scalebox{.6}{$
    \begin{array}{l} 
      \scalebox{1.6}{$\mbox{Opérations :}$} \\[.2cm]
      \BODY \\[.1cm]
    \end{array}$}
}

% \NewEnviron{arraySys}[1]{ %
%   \sysdelim\{.\systeme[#1]{ %
%     \BODY %
%   } %
% }

%%%%%

%%%%%%%%%%
%%%%%%%%%% ESSAI
\newlength\fboxseph
\newlength\fboxsepva
\newlength\fboxsepvb

\setlength\fboxsepva{0.2cm}
\setlength\fboxsepvb{0.2cm}
\setlength\fboxseph{0.2cm}

\makeatletter

\def\longboxed#1{\leavevmode\setbox\@tempboxa\hbox{\color@begingroup%
\kern\fboxseph{\m@th$\displaystyle #1 $}\kern\fboxseph%
\color@endgroup }\my@frameb@x\relax}

\def\my@frameb@x#1{%
  \@tempdima\fboxrule \advance\@tempdima \fboxsepva \advance\@tempdima
  \dp\@tempboxa\hbox {%
    \lower \@tempdima \hbox {%
      \vbox {\hrule\@height\fboxrule \hbox{\vrule\@width\fboxrule #1
          \vbox{%
            \vskip\fboxsepva \box\@tempboxa \vskip\fboxsepvb}#1
          \vrule\@width\fboxrule }%
        \hrule \@height \fboxrule }}}}

\newcommand{\boxedhv}[3]{\setlength\fboxseph{#1cm}
  \setlength\fboxsepva{#2cm}\setlength\fboxsepvb{#2cm}\longboxed{#3}}

\newcommand{\boxedhvv}[4]{\setlength\fboxseph{#1cm}
  \setlength\fboxsepva{#2cm}\setlength\fboxsepvb{#3cm}\longboxed{#4}}

\newcommand{\Boxed}[1]{{\setlength\fboxseph{0.2cm}
  \setlength\fboxsepva{0.2cm}\setlength\fboxsepvb{0.2cm}\longboxed{#1}}}

\newcommand{\mBoxed}[1]{{\setlength\fboxseph{0.2cm}
  \setlength\fboxsepva{0.2cm}\setlength\fboxsepvb{0.2cm}\longboxed{\mbox{#1}}}}

\newcommand{\mboxed}[1]{{\setlength\fboxseph{0.2cm}
  \setlength\fboxsepva{0.2cm}\setlength\fboxsepvb{0.2cm}\boxed{\mbox{#1}}}}

\newsavebox{\fmbox}
\newenvironment{fmpage}[1]
     {\begin{lrbox}{\fmbox}\begin{minipage}{#1}}
     {\end{minipage}\end{lrbox}\fbox{\usebox{\fmbox}}}

%%%%%%%%%%
%%%%%%%%%%

\DeclareMathOperator{\ch}{ch}
\DeclareMathOperator{\sh}{sh}

%%%%%%%%%%
%%%%%%%%%%

\newcommand{\norme}[1]{\Vert #1 \Vert}

%\newcommand*\widefbox[1]{\fbox{\hspace{2em}#1\hspace{2em}}}

\newcommand{\nl}{\tabularnewline}

\newcommand{\hand}{\noindent\ding{43}\ }
\newcommand{\ie}{\textit{i.e. }}
\newcommand{\cf}{\textit{cf }}

\newcommand{\Card}{\operatorname{Card}}

\newcommand{\aire}{\mathcal{A}}

\newcommand{\LL}[1]{\mathscr{L}(#1)} %
\newcommand{\B}{\mathscr{B}} %
\newcommand{\Bc}[1]{B_{#1}} %
\newcommand{\M}[1]{\mathscr{M}_{#1}(\mathbb{R})}

\DeclareMathOperator{\im}{Im}
\DeclareMathOperator{\kr}{Ker}
\DeclareMathOperator{\rg}{rg}
\DeclareMathOperator{\spc}{Sp}
\DeclareMathOperator{\sgn}{sgn}
\DeclareMathOperator{\supp}{Supp}

\newcommand{\Mat}{{\rm{Mat}}}
\newcommand{\Vect}[1]{{\rm{Vect}}\left(#1\right)}

\newenvironment{smatrix}{%
  \begin{adjustbox}{width=.9\width}
    $
    \begin{pmatrix}
    }{%      
    \end{pmatrix}
    $
  \end{adjustbox}
}

\newenvironment{sarray}[1]{%
  \begin{adjustbox}{width=.9\width}
    $
    \begin{array}{#1}
    }{%      
    \end{array}
    $
  \end{adjustbox}
}

\newcommand{\vd}[2]{
  \scalebox{.8}{
    $\left(\!
      \begin{array}{c}
        #1 \\
        #2
      \end{array}
    \!\right)$
    }}

\newcommand{\vt}[3]{
  \scalebox{.8}{
    $\left(\!
      \begin{array}{c}
        #1 \\
        #2 \\
        #3 
      \end{array}
    \!\right)$
    }}

\newcommand{\vq}[4]{
  \scalebox{.8}{
    $\left(\!
      \begin{array}{c}
        #1 \\
        #2 \\
        #3 \\
        #4 
      \end{array}
    \!\right)$
    }}

\newcommand{\vc}[5]{
  \scalebox{.8}{
    $\left(\!
      \begin{array}{c}
        #1 \\
        #2 \\
        #3 \\
        #4 \\
        #5 
      \end{array}
    \!\right)$
    }}

\newcommand{\ee}{\text{e}}

\newcommand{\dd}{\text{d}}

%%% Ensemble de définition
\newcommand{\Df}{\mathscr{D}}
\newcommand{\Cf}{\mathscr{C}}
\newcommand{\Ef}{\mathscr{C}}

\newcommand{\rond}[1]{\,\overset{\scriptscriptstyle \circ}{\!#1}}

\newcommand{\df}[2]{\dfrac{\partial #1}{\partial #2}} %
\newcommand{\dfn}[2]{\partial_{#2}(#1)} %
\newcommand{\ddfn}[2]{\partial^2_{#2}(#1)} %
\newcommand{\ddf}[2]{\dfrac{\partial^2 #1}{\partial #2^2}} %
\newcommand{\ddfr}[3]{\dfrac{\partial^2 #1}{\partial #2 \partial
    #3}} %


\newcommand{\dlim}[1]{{\displaystyle \lim_{#1} \ }}
\newcommand{\dlimPlus}[2]{
  \dlim{
    \scalebox{.6}{
      $
      \begin{array}{l}
        #1 \rightarrow #2\\
        #1 > #2
      \end{array}
      $}}}
\newcommand{\dlimMoins}[2]{
  \dlim{
    \scalebox{.6}{
      $
      \begin{array}{l}
        #1 \rightarrow #2\\
        #1 < #2
      \end{array}
      $}}}

%%%%%%%%%%%%%%
%% petit o, développement limité
%%%%%%%%%%%%%%

\newcommand{\oo}[2]{{\underset {{\overset {#1\rightarrow #2}{}}}{o}}} %
\newcommand{\oox}[1]{{\underset {{\overset {x\rightarrow #1}{}}}{o}}} %
\newcommand{\oon}{{\underset {{\overset {n\rightarrow +\infty}{}}}{o}}} %
\newcommand{\po}[1]{{\underset {{\overset {#1}{}}}{o}}} %
\newcommand{\neqx}[1]{{\ \underset {{\overset {x \to #1}{}}}{\not\sim}\ }} %
\newcommand{\eqx}[1]{{\ \underset {{\overset {x \to #1}{}}}{\sim}\ }} %
\newcommand{\eqn}{{\ \underset {{\overset {n \to +\infty}{}}}{\sim}\ }} %
\newcommand{\eq}[2]{{\ \underset {{\overset {#1 \to #2}{}}}{\sim}\ }} %
\newcommand{\DL}[1]{{\rm{DL}}_1 (#1)} %
\newcommand{\DLL}[1]{{\rm{DL}}_2 (#1)} %

\newcommand{\negl}{<<}

\newcommand{\neglP}[1]{\begin{array}{c}
    \vspace{-.2cm}\\
    << \\
    \vspace{-.7cm}\\
    {\scriptstyle #1}
  \end{array}}

%%%%%%%%%%%%%%
%% borne sup, inf, max, min
%%%%%%%%%%%%%%
\newcommand{\dsup}[1]{\displaystyle \sup_{#1} \ }
\newcommand{\dinf}[1]{\displaystyle \inf_{#1} \ }
\newcommand{\dmax}[1]{\max\limits_{#1} \ }
\newcommand{\dmin}[1]{\min\limits_{#1} \ }

\newcommand{\dcup}[2]{{\textstyle\bigcup\limits_{#1}^{#2}}\hspace{.1cm}}
%\displaystyle \bigcup_{#1}^{#2}}
\newcommand{\dcap}[2]{{\textstyle\bigcap\limits_{#1}^{#2}}\hspace{.1cm}}
% \displaystyle \bigcap_{#1}^{#2}
%%%%%%%%%%%%%%
%% opérateurs logiques
%%%%%%%%%%%%%%
\newcommand{\NON}[1]{\mathop{\small \tt{NON}} (#1)}
\newcommand{\ET}{\mathrel{\mathop{\small \mathtt{ET}}}}
\newcommand{\OU}{\mathrel{\mathop{\small \tt{OU}}}}
\newcommand{\XOR}{\mathrel{\mathop{\small \tt{XOR}}}}

\newcommand{\id}{{\rm{id}}}

\newcommand{\sbullet}{\scriptstyle \bullet}
\newcommand{\stimes}{\scriptstyle \times}

%%%%%%%%%%%%%%%%%%
%% Probabilités
%%%%%%%%%%%%%%%%%%
\newcommand{\Prob}{\mathbb{P}}
\newcommand{\Ev}[1]{\left[ {#1} \right]}
\newcommand{\Evmb}[1]{[ {#1} ]}
\newcommand{\E}{\mathbb{E}}
\newcommand{\V}{\mathbb{V}}
\newcommand{\Cov}{{\rm{Cov}}}
\newcommand{\U}[2]{\mathcal{U}(\llb #1, #2\rrb)}
\newcommand{\Uc}[2]{\mathcal{U}([#1, #2])}
\newcommand{\Ucof}[2]{\mathcal{U}(]#1, #2])}
\newcommand{\Ucoo}[2]{\mathcal{U}(]#1, #2[)}
\newcommand{\Ucfo}[2]{\mathcal{U}([#1, #2[)}
\newcommand{\Bern}[1]{\mathcal{B}\left(#1\right)}
\newcommand{\Bin}[2]{\mathcal{B}\left(#1, #2\right)}
\newcommand{\G}[1]{\mathcal{G}\left(#1\right)}
\newcommand{\Pois}[1]{\mathcal{P}\left(#1\right)}
\newcommand{\HG}[3]{\mathcal{H}\left(#1, #2, #3\right)}
\newcommand{\Exp}[1]{\mathcal{E}\left(#1\right)}
\newcommand{\Norm}[2]{\mathcal{N}\left(#1, #2\right)}

\DeclareMathOperator{\cov}{Cov}

\newcommand{\var}{v.a.r. }
\newcommand{\suit}{\hookrightarrow}

\newcommand{\flecheR}[1]{\rotatebox{90}{\scalebox{#1}{\color{red}
      $\curvearrowleft$}}}


\newcommand{\partie}[1]{\mathcal{P}(#1)}
\newcommand{\Cont}[1]{\mathcal{C}^{#1}}
\newcommand{\Contm}[1]{\mathcal{C}^{#1}_m}

\newcommand{\llb}{\llbracket}
\newcommand{\rrb}{\rrbracket}

%\newcommand{\im}[1]{{\rm{Im}}(#1)}
\newcommand{\imrec}[1]{#1^{- \mathds{1}}}

\newcommand{\unq}{\mathds{1}}

\newcommand{\Hyp}{\mathtt{H}}

\newcommand{\eme}[1]{#1^{\scriptsize \mbox{ème}}}
\newcommand{\er}[1]{#1^{\scriptsize \mbox{er}}}
\newcommand{\ere}[1]{#1^{\scriptsize \mbox{ère}}}
\newcommand{\nd}[1]{#1^{\scriptsize \mbox{nd}}}
\newcommand{\nde}[1]{#1^{\scriptsize \mbox{nde}}}

\newcommand{\truc}{\mathop{\top}}
\newcommand{\fois}{\mathop{\ast}}

\newcommand{\f}[1]{\overrightarrow{#1}}

\newcommand{\checked}{\textcolor{green}{\checkmark}}

\def\restriction#1#2{\mathchoice
              {\setbox1\hbox{${\displaystyle #1}_{\scriptstyle #2}$}
              \restrictionaux{#1}{#2}}
              {\setbox1\hbox{${\textstyle #1}_{\scriptstyle #2}$}
              \restrictionaux{#1}{#2}}
              {\setbox1\hbox{${\scriptstyle #1}_{\scriptscriptstyle #2}$}
              \restrictionaux{#1}{#2}}
              {\setbox1\hbox{${\scriptscriptstyle #1}_{\scriptscriptstyle #2}$}
              \restrictionaux{#1}{#2}}}
\def\restrictionaux#1#2{{#1\,\smash{\vrule height .8\ht1 depth .85\dp1}}_{\,#2}}

\makeatletter
\newcommand*{\da@rightarrow}{\mathchar"0\hexnumber@\symAMSa 4B }
\newcommand*{\da@leftarrow}{\mathchar"0\hexnumber@\symAMSa 4C }
\newcommand*{\xdashrightarrow}[2][]{%
  \mathrel{%
    \mathpalette{\da@xarrow{#1}{#2}{}\da@rightarrow{\,}{}}{}%
  }%
}
\newcommand{\xdashleftarrow}[2][]{%
  \mathrel{%
    \mathpalette{\da@xarrow{#1}{#2}\da@leftarrow{}{}{\,}}{}%
  }%
}
\newcommand*{\da@xarrow}[7]{%
  % #1: below
  % #2: above
  % #3: arrow left
  % #4: arrow right
  % #5: space left 
  % #6: space right
  % #7: math style 
  \sbox0{$\ifx#7\scriptstyle\scriptscriptstyle\else\scriptstyle\fi#5#1#6\m@th$}%
  \sbox2{$\ifx#7\scriptstyle\scriptscriptstyle\else\scriptstyle\fi#5#2#6\m@th$}%
  \sbox4{$#7\dabar@\m@th$}%
  \dimen@=\wd0 %
  \ifdim\wd2 >\dimen@
    \dimen@=\wd2 %   
  \fi
  \count@=2 %
  \def\da@bars{\dabar@\dabar@}%
  \@whiledim\count@\wd4<\dimen@\do{%
    \advance\count@\@ne
    \expandafter\def\expandafter\da@bars\expandafter{%
      \da@bars
      \dabar@ 
    }%
  }%  
  \mathrel{#3}%
  \mathrel{%   
    \mathop{\da@bars}\limits
    \ifx\\#1\\%
    \else
      _{\copy0}%
    \fi
    \ifx\\#2\\%
    \else
      ^{\copy2}%
    \fi
  }%   
  \mathrel{#4}%
}
\makeatother



\newcount\depth

\newcount\depth
\newcount\totaldepth

\makeatletter
\newcommand{\labelsymbol}{%
      \ifnum\depth=0
        %
      \else
        \rlap{\,$\bullet$}%
      \fi
}

\newcommand*\bernoulliTree[1]{%
    \depth=#1\relax            
    \totaldepth=#1\relax
    \draw node(root)[bernoulli/root] {\labelsymbol}[grow=right] \draw@bernoulli@tree;
    \draw \label@bernoulli@tree{root};                                   
}                                                                        

\def\draw@bernoulli@tree{%
    \ifnum\depth>0 
      child[parent anchor=east] foreach \type/\label in {left child/$E$,right child/$S$} {%
          node[bernoulli/\type] {\label\strut\labelsymbol} \draw@bernoulli@tree
      }
      coordinate[bernoulli/increment] (dummy)
   \fi%
}

\def\label@bernoulli@tree#1{%
    \ifnum\depth>0
      ($(#1)!0.5!(#1-1)$) node[fill=white,bernoulli/decrement] {\tiny$p$}
      \label@bernoulli@tree{#1-1}
      ($(#1)!0.5!(#1-2)$) node[fill=white] {\tiny$q$}
      \label@bernoulli@tree{#1-2}
      coordinate[bernoulli/increment] (dummy)
   \fi%
}

\makeatother

\tikzset{bernoulli/.cd,
         root/.style={},
         decrement/.code=\global\advance\depth by-1\relax,
         increment/.code=\global\advance\depth by 1\relax,
         left child/.style={bernoulli/decrement},
         right child/.style={}}


\newcommand{\eps}{\varepsilon}

% \newcommand{\tendi}[1]{\xrightarrow[\footnotesize #1 \rightarrow
%   +\infty]{}}%

\newcommand{\tend}{\rightarrow}%
\newcommand{\tendn}{\underset{n\to +\infty}{\longrightarrow}} %
\newcommand{\ntendn}{\underset{n\to
    +\infty}{\not\hspace{-.15cm}\longrightarrow}} %
% \newcommand{\tendn}{\xrightarrow[\footnotesize n \rightarrow
%   +\infty]{}}%
\newcommand{\Tendx}[1]{\xrightarrow[\footnotesize x \rightarrow
  #1]{}}%
\newcommand{\tendx}[1]{\underset{x\to #1}{\longrightarrow}}%
\newcommand{\ntendx}[1]{\underset{x\to #1}{\not\!\!\longrightarrow}}%
\newcommand{\tendd}[2]{\underset{#1\to #2}{\longrightarrow}}%
% \newcommand{\tendd}[2]{\xrightarrow[\footnotesize #1 \rightarrow
%   #2]{}}%
\newcommand{\tendash}[1]{\xdashrightarrow[\footnotesize #1 \rightarrow
  +\infty]{}}%
\newcommand{\tendashx}[1]{\xdashrightarrow[\footnotesize x \rightarrow
  #1]{}}%
\newcommand{\tendb}[1]{\underset{#1 \to +\infty}{\longrightarrow}}%
\newcommand{\tendL}{\overset{\mathscr L}{\underset{n \to
      +\infty}{\longrightarrow}}}%
\newcommand{\tendP}{\overset{\Prob}{\underset{n \to
      +\infty}{\longrightarrow}}}%
\newcommand{\tenddL}[1]{\overset{\mathscr L}{\underset{#1 \to
      +\infty}{\longrightarrow}}}%

\NewEnviron{attention}{ %
  ~\\[-.2cm]\noindent
  \begin{minipage}{\linewidth}
  \setlength{\fboxsep}{3mm}%
  \ \ \dbend \ \ %
  \fbox{\parbox[t]{.88\linewidth}{\BODY}} %
  \end{minipage}\\
}

\NewEnviron{sattention}[1]{ %
  ~\\[-.2cm]\noindent
  \begin{minipage}{#1\linewidth}
  \setlength{\fboxsep}{3mm}%
  \ \ \dbend \ \ %
  \fbox{\parbox[t]{.88\linewidth}{\BODY}} %
  \end{minipage}\\
}

%%%%% OBSOLETE %%%%%%

% \newcommand{\attention}[1]{
%   \noindent
%   \begin{tabular}{@{}l|p{11.5cm}|}
%     \cline{2-2}
%     \vspace{-.2cm} 
%     & \nl
%     \dbend & #1 \nl
%     \cline{2-2}
%   \end{tabular}
% }

% \newcommand{\attentionv}[2]{
%   \noindent
%   \begin{tabular}{@{}l|p{11.5cm}|}
%     \cline{2-2}
%     \vspace{-.2cm} 
%     & \nl
%     \dbend & #2 \nl[#1 cm]
%     \cline{2-2}
%   \end{tabular}
% }

\newcommand{\explainvb}[2]{
  \noindent
  \begin{tabular}{@{}l|p{11.5cm}|}
    \cline{2-2}
    \vspace{-.2cm} 
    & \nl
    \hand & #2 \nl [#1 cm]
    \cline{2-2}
  \end{tabular}
}


% \noindent
% \begin{tabular}{@{}l|lp{11cm}|}
%   \cline{3-3} 
%   \multicolumn{1}{@{}l@{\dbend}}{} & & #1 \nl
%   \multicolumn{1}{l}{} & & \nl [-.8cm]
%   & & #2 \nl
%   \cline{2-3}
% \end{tabular}

% \newcommand{\attention}[1]{
%   \noindent
%   \begin{tabular}{@{}@{}cp{11cm}}
%     \dbend & #1 \nl
%   \end{tabular}
% }

\newcommand{\PP}[1]{\mathcal{P}(#1)}
\newcommand{\HH}[1]{\mathcal{H}(#1)}
\newcommand{\FF}[1]{\mathcal{F}(#1)}

\newcommand{\DSum}[2]{\displaystyle\sum\limits_{#1}^{#2}\hspace{.1cm}}
\newcommand{\Sum}[2]{{\textstyle\sum\limits_{#1}^{#2}}\hspace{.1cm}}
\newcommand{\Serie}{\textstyle\sum\hspace{.1cm}}
\newcommand{\Prod}[2]{\textstyle\prod\limits_{#1}^{#2}}

\newcommand{\Prim}[3]{\left[\ {#1} \ \right]_{\scriptscriptstyle
   \hspace{-.15cm} ~_{#2}\, }^ {\scriptscriptstyle \hspace{-.15cm} ~^{#3}\, }}

% \newcommand{\Prim}[3]{\left[\ {#1} \ \right]_{\scriptscriptstyle
%     \!\!~_{#2}}^ {\scriptscriptstyle \!\!~^{#3}}}

\newcommand{\dint}[2]{\displaystyle \int_{#1}^{#2}\ }
\newcommand{\Int}[2]{{\rm{Int}}_{\scriptscriptstyle #1, #2}}
\newcommand{\dt}{\ dt}
\newcommand{\dx}{\ dx}

\newcommand{\llpar}[1]{\left(\!\!\!
    \begin{array}{c}
      \rule{0pt}{#1}
    \end{array}
  \!\!\!\right.}

\newcommand{\rrpar}[1]{\left.\!\!\!
    \begin{array}{c}
      \rule{0pt}{#1}
    \end{array}
  \!\!\!\right)}

\newcommand{\llacc}[1]{\left\{\!\!\!
    \begin{array}{c}
      \rule{0pt}{#1}
    \end{array}
  \!\!\!\right.}

\newcommand{\rracc}[1]{\left.\!\!\!
    \begin{array}{c}
      \rule{0pt}{#1}
    \end{array}
  \!\!\!\right\}}

\newcommand{\ttacc}[1]{\mbox{\rotatebox{-90}{\hspace{-.7cm}$\llacc{#1}$}}}
\newcommand{\bbacc}[1]{\mbox{\rotatebox{90}{\hspace{-.5cm}$\llacc{#1}$}}}

\newcommand{\comp}[1]{\overline{#1}}%

\newcommand{\dcomp}[2]{\stackrel{\mbox{\ \ \----}{\scriptscriptstyle
      #2}}{#1}}%

% \newcommand{\Comp}[2]{\stackrel{\mbox{\ \
%       \-------}{\scriptscriptstyle #2}}{#1}}

% \newcommand{\dcomp}[2]{\stackrel{\mbox{\ \
%       \-------}{\scriptscriptstyle #2}}{#1}}

\newcommand{\A}{\mathscr{A}}

\newcommand{\conc}[1]{
  \begin{center}
    \fbox{
      \begin{tabular}{c}
        #1
      \end{tabular}
    }
  \end{center}
}

\newcommand{\concC}[1]{
  \begin{center}
    \fbox{
    \begin{tabular}{C{10cm}}
      \quad #1 \quad
    \end{tabular}
    }
  \end{center}
}

\newcommand{\concL}[2]{
  \begin{center}
    \fbox{
    \begin{tabular}{C{#2cm}}
      \quad #1 \quad
    \end{tabular}
    }
  \end{center}
}


% \newcommand{\lims}[2]{\prod\limits_{#1}^{#2}}

\newtheorem{theorem}{Théorème}[]
\newtheorem{lemma}{Lemme}[]
\newtheorem{proposition}{Proposition}[]
\newtheorem{corollary}{Corollaire}[]

% \newenvironment{proof}[1][Démonstration]{\begin{trivlist}
% \item[\hskip \labelsep {\bfseries #1}]}{\end{trivlist}}
\newenvironment{definition}[1][Définition]{\begin{trivlist}
\item[\hskip \labelsep {\bfseries #1}]}{\end{trivlist}}
\newenvironment{example}[1][Exemple]{\begin{trivlist}
\item[\hskip \labelsep {\bfseries #1}]}{\end{trivlist}}
\newenvironment{examples}[1][Exemples]{\begin{trivlist}
\item[\hskip \labelsep {\bfseries #1}]}{\end{trivlist}}
\newenvironment{notation}[1][Notation]{\begin{trivlist}
\item[\hskip \labelsep {\bfseries #1}]}{\end{trivlist}}
\newenvironment{propriete}[1][Propriété]{\begin{trivlist}
\item[\hskip \labelsep {\bfseries #1}]}{\end{trivlist}}
\newenvironment{proprietes}[1][Propriétés]{\begin{trivlist}
\item[\hskip \labelsep {\bfseries #1}]}{\end{trivlist}}
% \newenvironment{remark}[1][Remarque]{\begin{trivlist}
% \item[\hskip \labelsep {\bfseries #1}]}{\end{trivlist}}
\newenvironment{application}[1][Application]{\begin{trivlist}
\item[\hskip \labelsep {\bfseries #1}]}{\end{trivlist}}

% Environnement pour les réponses des DS
\newenvironment{answer}{\par\emph{Réponse :}\par{}}
{\vspace{-.6cm}\hspace{\stretch{1}}\rule{1ex}{1ex}\vspace{.3cm}}

\newenvironment{answerTD}{\vspace{.2cm}\par\emph{Réponse :}\par{}}
{\hspace{\stretch{1}}\rule{1ex}{1ex}\vspace{.3cm}}

\newenvironment{answerCours}{\noindent\emph{Réponse :}}
{\rule{1ex}{1ex}}%\vspace{.3cm}}


% footnote in footer
\newcommand{\fancyfootnotetext}[2]{%
  \fancypagestyle{dingens}{%
    \fancyfoot[LO,RE]{\parbox{0.95\textwidth}{\footnotemark[#1]\footnotesize
        #2}}%
  }%
  \thispagestyle{dingens}%
}

%%% définit le style (arabic : 1,2,3...) et place des parenthèses
%%% autour de la numérotation
\renewcommand*{\thefootnote}{(\arabic{footnote})}
% http://www.tuteurs.ens.fr/logiciels/latex/footnote.html

%%%%%%%% tikz axis
% \pgfplotsset{every axis/.append style={
%                     axis x line=middle,    % put the x axis in the middle
%                     axis y line=middle,    % put the y axis in the middle
%                     axis line style={<->,color=blue}, % arrows on the axis
%                     xlabel={$x$},          % default put x on x-axis
%                     ylabel={$y$},          % default put y on y-axis
%             }}

%%%% s'utilise comme suit

% \begin{axis}[
%   xmin=-8,xmax=4,
%   ymin=-8,ymax=4,
%   grid=both,
%   ]
%   \addplot [domain=-3:3,samples=50]({x^3-3*x},{3*x^2-9}); 
% \end{axis}

%%%%%%%%



%%%%%%%%%%%% Pour avoir des numéros de section qui correspondent à
%%%%%%%%%%%% ceux du tableau
\renewcommand{\thesection}{\Roman{section}.\hspace{-.3cm}}
\renewcommand{\thesubsection}{\Roman{section}.\arabic{subsection}.\hspace{-.2cm}}
\renewcommand{\thesubsubsection}{\Roman{section}.\arabic{subsection}.\alph{subsubsection})\hspace{-.2cm}}
%%%%%%%%%%%% 

%%% Changer le nom des figures : Fig. au lieu de Figure
\usepackage[font=small,labelfont=bf,labelsep=space]{caption}
\captionsetup{%
  figurename=Fig.,
  tablename=Tab.
}
% \renewcommand{\thesection}{\Roman{section}.\hspace{-.2cm}}
% \renewcommand{\thesubsection}{\Roman{section}
%   .\hspace{.2cm}\arabic{subsection}\ .\hspace{-.3cm}}
% \renewcommand{\thesubsubsection}{\alph{subsection})}

\newenvironment{tabliste}[1]
{\begin{tabenum}[\bfseries\small\itshape #1]}{\end{tabenum}} 

%%%% ESSAI contre le too deeply nested %%%%
%%%% ATTENTION au package enumitem qui se comporte mal avec les
%%%% noliste, à redéfinir !
% \usepackage{enumitem}

% \setlistdepth{9}

% \newlist{myEnumerate}{enumerate}{9}
% \setlist[myEnumerate,1]{label=(\arabic*)}
% \setlist[myEnumerate,2]{label=(\Roman*)}
% \setlist[myEnumerate,3]{label=(\Alph*)}
% \setlist[myEnumerate,4]{label=(\roman*)}
% \setlist[myEnumerate,5]{label=(\alph*)}
% \setlist[myEnumerate,6]{label=(\arabic*)}
% \setlist[myEnumerate,7]{label=(\Roman*)}
% \setlist[myEnumerate,8]{label=(\Alph*)}
% \setlist[myEnumerate,9]{label=(\roman*)}

%%%%%

\newenvironment{noliste}[1] %
{\begin{enumerate}[\bfseries\small\itshape #1]} %
  {\end{enumerate}}

\newenvironment{nonoliste}[1] %
{\begin{enumerate}[\hspace{-12pt}\bfseries\small\itshape #1]} %
  {\end{enumerate}}

\newenvironment{arrayliste}[1]{ 
  % List with minimal white space to fit in small areas, e.g. table
  % cell
  \begin{minipage}[t]{\linewidth} %
    \begin{enumerate}[\bfseries\small\itshape #1] %
      {\leftmargin=0.5em \rightmargin=0em
        \topsep=0em \parskip=0em \parsep=0em
        \listparindent=0em \partopsep=0em \itemsep=0pt
        \itemindent=0em \labelwidth=\leftmargin\labelsep+0.25em}
    }{
    \end{enumerate}\end{minipage}
}

\newenvironment{nolistes}[2]
{\begin{enumerate}[\bfseries\small\itshape
    #1]\setlength{\itemsep}{#2 mm}}{\end{enumerate}}

\newenvironment{liste}[1]
{\begin{enumerate}[\hspace{12pt}\bfseries\small\itshape
    #1]}{\end{enumerate}}   


%%%%%%%% Pour les programmes de colle %%%%%%%

\newcommand{\cours}{{\small \tt (COURS)}} %
\newcommand{\poly}{{\small \tt (POLY)}} %
\newcommand{\exo}{{\small \tt (EXO)}} %
\newcommand{\culture}{{\small \tt (CULTURE)}} %
\newcommand{\methodo}{{\small \tt (MÉTHODO)}} %
\newcommand{\methodob}{\Boxed{\mbox{\tt MÉTHODO}}} %

%%%%%%%% Pour les TD %%%%%%%
\newtheoremstyle{exostyle} {\topsep} % espace avant
{.6cm} % espace apres
{} % Police utilisee par le style de thm
{} % Indentation (vide = aucune, \parindent = indentation paragraphe)
{\bfseries} % Police du titre de thm
{} % Signe de ponctuation apres le titre du thm
{ } % Espace apres le titre du thm (\newline = linebreak)
{\thmname{#1}\thmnumber{ #2}\thmnote{.
    \normalfont{\textit{#3}}}} % composants du titre du thm : \thmname
                               % = nom du thm, \thmnumber = numéro du
                               % thm, \thmnote = sous-titre du thm
 
\theoremstyle{exostyle}
\newtheorem{exercice}{Exercice}
\newtheorem*{exoCours}{Exercice}

%%%%%%%% Pour des théorèmes Sans Espaces APRÈS %%%%%%%
\newtheoremstyle{exostyleSE} {\topsep} % espace avant
{} % espace apres
{} % Police utilisee par le style de thm
{} % Indentation (vide = aucune, \parindent = indentation paragraphe)
{\bfseries} % Police du titre de thm
{} % Signe de ponctuation apres le titre du thm
{ } % Espace apres le titre du thm (\newline = linebreak)
{\thmname{#1}\thmnumber{ #2}\thmnote{.
    \normalfont{\textit{#3}}}} % composants du titre du thm : \thmname
                               % = nom du thm, \thmnumber = numéro du
                               % thm, \thmnote = sous-titre du thm
 
\theoremstyle{exostyleSE}
\newtheorem{exerciceSE}{Exercice}
\newtheorem*{exoCoursSE}{Exercice}

% \newcommand{\lims}[2]{\prod\limits_{#1}^{#2}}

\newtheorem{theoremSE}{Théorème}[]
\newtheorem{lemmaSE}{Lemme}[]
\newtheorem{propositionSE}{Proposition}[]
\newtheorem{corollarySE}{Corollaire}[]

% \newenvironment{proofSE}[1][Démonstration]{\begin{trivlist}
% \item[\hskip \labelsep {\bfseries #1}]}{\end{trivlist}}
\newenvironment{definitionSE}[1][Définition]{\begin{trivlist}
  \item[\hskip \labelsep {\bfseries #1}]}{\end{trivlist}}
\newenvironment{exampleSE}[1][Exemple]{\begin{trivlist} 
  \item[\hskip \labelsep {\bfseries #1}]}{\end{trivlist}}
\newenvironment{examplesSE}[1][Exemples]{\begin{trivlist}
\item[\hskip \labelsep {\bfseries #1}]}{\end{trivlist}}
\newenvironment{notationSE}[1][Notation]{\begin{trivlist}
\item[\hskip \labelsep {\bfseries #1}]}{\end{trivlist}}
\newenvironment{proprieteSE}[1][Propriété]{\begin{trivlist}
\item[\hskip \labelsep {\bfseries #1}]}{\end{trivlist}}
\newenvironment{proprietesSE}[1][Propriétés]{\begin{trivlist}
\item[\hskip \labelsep {\bfseries #1}]}{\end{trivlist}}
\newenvironment{remarkSE}[1][Remarque]{\begin{trivlist}
\item[\hskip \labelsep {\bfseries #1}]}{\end{trivlist}}
\newenvironment{applicationSE}[1][Application]{\begin{trivlist}
\item[\hskip \labelsep {\bfseries #1}]}{\end{trivlist}}

%%%%%%%%%%% Obtenir les étoiles sans charger le package MnSymbol
%%%%%%%%%%%
\DeclareFontFamily{U} {MnSymbolC}{}
\DeclareFontShape{U}{MnSymbolC}{m}{n}{
  <-6> MnSymbolC5
  <6-7> MnSymbolC6
  <7-8> MnSymbolC7
  <8-9> MnSymbolC8
  <9-10> MnSymbolC9
  <10-12> MnSymbolC10
  <12-> MnSymbolC12}{}
\DeclareFontShape{U}{MnSymbolC}{b}{n}{
  <-6> MnSymbolC-Bold5
  <6-7> MnSymbolC-Bold6
  <7-8> MnSymbolC-Bold7
  <8-9> MnSymbolC-Bold8
  <9-10> MnSymbolC-Bold9
  <10-12> MnSymbolC-Bold10
  <12-> MnSymbolC-Bold12}{}

\DeclareSymbolFont{MnSyC} {U} {MnSymbolC}{m}{n}

\DeclareMathSymbol{\filledlargestar}{\mathrel}{MnSyC}{205}
\DeclareMathSymbol{\largestar}{\mathrel}{MnSyC}{131}

\newcommand{\facile}{\rm{(}$\scriptstyle\largestar$\rm{)}} %
\newcommand{\moyen}{\rm{(}$\scriptstyle\filledlargestar$\rm{)}} %
\newcommand{\dur}{\rm{(}$\scriptstyle\filledlargestar\filledlargestar$\rm{)}} %
\newcommand{\costaud}{\rm{(}$\scriptstyle\filledlargestar\filledlargestar\filledlargestar$\rm{)}}

%%%%%%%%%%%%%%%%%%%%%%%%%

%%%%%%%%%%%%%%%%%%%%%%%%%
%%%%%%%% Fin de la partie TD

%%%%%%%%%%%%%%%%
%%%%%%%%%%%%%%%%
\makeatletter %
\newenvironment{myitemize}{%
  \setlength{\topsep}{0pt} %
  \setlength{\partopsep}{0pt} %
  \renewcommand*{\@listi}{\leftmargin\leftmargini \parsep\z@
    \topsep\z@ \itemsep\z@} \let\@listI\@listi %
  \itemize %
}{\enditemize} %
\makeatother
%%%%%%%%%%%%%%%%
%%%%%%%%%%%%%%%%

%% Commentaires dans la correction du livre

\newcommand{\Com}[1]{
% Define box and box title style
\tikzstyle{mybox} = [draw=black!50,
very thick,
    rectangle, rounded corners, inner sep=10pt, inner ysep=8pt]
\tikzstyle{fancytitle} =[rounded corners, fill=black!80, text=white]
\tikzstyle{fancylogo} =[ text=white]
\begin{center}

\begin{tikzpicture}
\node [mybox] (box){%

    \begin{minipage}{0.90\linewidth}
\vspace{6pt}  #1
    \end{minipage}
};
\node[fancytitle, right=10pt] at (box.north west) 
{\bfseries\normalsize{Commentaire}};

\end{tikzpicture}%

\end{center}
%
}

\NewEnviron{remark}{%
  % Define box and box title style
  \tikzstyle{mybox} = [draw=black!50, very thick, rectangle, rounded
  corners, inner sep=10pt, inner ysep=8pt] %
  \tikzstyle{fancytitle} = [rounded corners , fill=black!80,
  text=white] %
  \tikzstyle{fancylogo} =[ text=white]
  \begin{center}
    \begin{tikzpicture}
      \node [mybox] (box){%
        \begin{minipage}{0.90\linewidth}
          \vspace{6pt} \BODY
        \end{minipage}
      }; %
      \node[fancytitle, right=10pt] at (box.north west) %
      {\bfseries\normalsize{Commentaire}}; %
    \end{tikzpicture}%
  \end{center}
}

\NewEnviron{remarkST}{%
  % Define box and box title style
  \tikzstyle{mybox} = [draw=black!50, very thick, rectangle, rounded
  corners, inner sep=10pt, inner ysep=8pt] %
  \tikzstyle{fancytitle} = [rounded corners , fill=black!80,
  text=white] %
  \tikzstyle{fancylogo} =[ text=white]
  \begin{center}
    \begin{tikzpicture}
      \node [mybox] (box){%
        \begin{minipage}{0.90\linewidth}
          \vspace{6pt} \BODY
        \end{minipage}
      }; %
      % \node[fancytitle, right=10pt] at (box.north west) %
%       {\bfseries\normalsize{Commentaire}}; %
    \end{tikzpicture}%
  \end{center}
}

\NewEnviron{remarkL}[1]{%
  % Define box and box title style
  \tikzstyle{mybox} = [draw=black!50, very thick, rectangle, rounded
  corners, inner sep=10pt, inner ysep=8pt] %
  \tikzstyle{fancytitle} =[rounded corners, fill=black!80,
  text=white] %
  \tikzstyle{fancylogo} =[ text=white]
  \begin{center}
    \begin{tikzpicture}
      \node [mybox] (box){%
        \begin{minipage}{#1\linewidth}
          \vspace{6pt} \BODY
        \end{minipage}
      }; %
      \node[fancytitle, right=10pt] at (box.north west) %
      {\bfseries\normalsize{Commentaire}}; %
    \end{tikzpicture}%
  \end{center}
}

\NewEnviron{remarkSTL}[1]{%
  % Define box and box title style
  \tikzstyle{mybox} = [draw=black!50, very thick, rectangle, rounded
  corners, inner sep=10pt, inner ysep=8pt] %
  \tikzstyle{fancytitle} =[rounded corners, fill=black!80,
  text=white] %
  \tikzstyle{fancylogo} =[ text=white]
  \begin{center}
    \begin{tikzpicture}
      \node [mybox] (box){%
        \begin{minipage}{#1\linewidth}
          \vspace{6pt} \BODY
        \end{minipage}
      }; %
%       \node[fancytitle, right=10pt] at (box.north west) %
%       {\bfseries\normalsize{Commentaire}}; %
    \end{tikzpicture}%
  \end{center}
}

\NewEnviron{titre} %
{ %
  ~\\[-1.8cm]
  \begin{center}
    \bf \LARGE \BODY
  \end{center}
  ~\\[-.6cm]
  \hrule %
  \vspace*{.2cm}
} %

\NewEnviron{titreL}[2] %
{ %
  ~\\[-#1cm]
  \begin{center}
    \bf \LARGE \BODY
  \end{center}
  ~\\[-#2cm]
  \hrule %
  \vspace*{.2cm}
} %



%%%%%%%%%%% Redefinition \chapter



\usepackage[explicit]{titlesec}
\usepackage{color}
\titleformat{\chapter}
{\gdef\chapterlabel{}
\selectfont\huge\bf}
%\normalfont\sffamily\Huge\bfseries\scshape}
{\gdef\chapterlabel{\thechapter)\ }}{0pt}
{\begin{tikzpicture}[remember picture,overlay]
\node[yshift=-3cm] at (current page.north west)
{\begin{tikzpicture}[remember picture, overlay]
\draw (.1\paperwidth,0) -- (.9\paperwidth,0);
\draw (.1\paperwidth,2) -- (.9\paperwidth,2);
%(\paperwidth,3cm);
\node[anchor=center,xshift=.5\paperwidth,yshift=1cm, rectangle,
rounded corners=20pt,inner sep=11pt]
{\color{black}\chapterlabel#1};
\end{tikzpicture}
};
\end{tikzpicture}
}
\titlespacing*{\chapter}{0pt}{50pt}{-75pt}




%%%%%%%%%%%%%% Affichage chapter dans Table des matieres

\makeatletter
\renewcommand*\l@chapter[2]{%
  \ifnum \c@tocdepth >\m@ne
    \addpenalty{-\@highpenalty}%
    \vskip 1.0em \@plus\p@
    \setlength\@tempdima{1.5em}%
    \begingroup
      \parindent \z@ \rightskip \@pnumwidth
      \parfillskip -\@pnumwidth
      \leavevmode %\bfseries
      \advance\leftskip\@tempdima
      \hskip -\leftskip
      #1\nobreak\ 
       \leaders\hbox{$\m@th
        \mkern \@dotsep mu\hbox{.}\mkern \@dotsep
        mu$}\hfil\nobreak\hb@xt@\@pnumwidth{\hss #2}\par
      \penalty\@highpenalty
    \endgroup
  \fi}
\makeatother




%%%%%%%%%%%%%%%%% Redefinition part




\pagestyle{fancy} %
\lhead{ECE2 \hfill Mathématiques\\
} %
\chead{\hrule} %
\rhead{} %
\lfoot{} %
\cfoot{} %
\rfoot{\thepage} %

\renewcommand{\headrulewidth}{0pt}% : Trace un trait de séparation
 % de largeur 0,4 point. Mettre 0pt
 % pour supprimer le trait.

\renewcommand{\footrulewidth}{0.4pt}% : Trace un trait de séparation
 % de largeur 0,4 point. Mettre 0pt
 % pour supprimer le trait.

\setlength{\headheight}{14pt}

\title{\bf \vspace{-2cm} HEC 2017} %
\author{} %
\date{} %
\begin{document}

\maketitle %
\vspace{-1.4cm}\hrule %
\thispagestyle{fancy}

\vspace*{.2cm}

%%DEBUT

% \begin{remark}[Retour de commission]~\\
%   Le 31 mai 2017 s'est tenue la commission HEC. Les informations
%   suivantes ont été signalées.
%   \begin{noliste}{$\sbullet$}
%   \item {\bf Barème}. La répartition des points s'est faite de la
%     manière suivante :
%     \begin{noliste}{$\stimes$}
%     \item l'exercice a pesé pour 32\% des points attribués.
%     \item le problème a pesé 68 \% des points attribués. Plus
%       précisément : 
%       \begin{noliste}{}
%       \item partie I : 32 \%
%       \item partie II : 15 \%
%       \item partie III : 21 \%        
%       \end{noliste}
%       Pour obtenir 20, le jury estime qu'il fallait recueillir 75\%
%       des points attribués. Il précise que cela correspond par exemple
%       à traiter entièrement l'exercice puis la moitié du problème. Ou
%       encore de traiter tout le problème.    
%     \end{noliste}

%   \item {\bf Contenu mathématique}. Le jury a précisé cette année que
%     les candidats s'en sont plutôt bien sortis sur l'algèbre linéaire
%     (exercice 1). C'est un changement par rapport aux années
%     précédentes où le jury estimait insuffisant le niveau des
%     candidats sur ce chapitre.

%   \item {\bf Présentation des copies}. Le jury a souligné une forte
%     dégradation de la qualité de présentation des copies : lisibilité
%     / encadrement des résultats / numérotation approximative /
%     utilisation d'abréviations (dont certaines connues uniquement par
%     le candidat \ldots). Devant ce constat alarmant, le jury a promis
%     que la présentation des copies serait maintenant sanctionné.

%   \item {\bf Informatique}. Le jury a précisé que les questions
%     \Scilab{} représentent à elles seules 12\% des points
%     attribués. Ce poids est un signal fort du jury envers
%     l'informatique. Il valide l'importance de ce contenu dans la
%     formation. Si par le passé certains candidats ont pu estimer que
%     sécher la partie informatique représentait un bon calcul, il est
%     maintenant hautement déconseillé de le faire.
    
%   \end{noliste}
% \end{remark}

\section*{EXERCICE}

\noindent
Pour tout $n\in\N^*$, on note $\M{n}$ l'ensemble des matrices carrés à
$n$ lignes et $n$ colonnes à coefficients réels et $\Bc{n}$ l'ensemble
des matrices de $\M{n}$ dont tous les coefficients sont égaux à $0$ ou
à $1$.
\begin{noliste}{1.}
  \setlength{\itemsep}{4mm}
\item {\it Exemple 1}. Soit $A$ la matrice de $\Bc{2}$ définie par :
  $A = 
  \begin{smatrix} 
    0 & 1 \\ 
    1 & 0
  \end{smatrix}$.
  \begin{noliste}{a)}
    \setlength{\itemsep}{2mm}
  \item Calculer la matrice $A^2$.
    
    \begin{proof}~\\[-.6cm]%
      \conc{$A^2 = %
        \begin{smatrix}
          0 & 1 \\
          1 & 0
        \end{smatrix} 
        \begin{smatrix}
          0 & 1 \\ 
          1 & 0
        \end{smatrix} = I_2$.}~\\[-1cm]
    \end{proof}
    
  \item Quelles sont les valeurs propres de $A$ ?
	
    \begin{proof}~
      \begin{noliste}{$\sbullet$}
      \item D'après la question \itbf{1.a)}, $A^2 - I_2 = 0$.\\
        On en déduit que le polynôme $P(X) = X^2-1 = (X-1)(X+1)$ est
        un polynôme annulateur de $A$. Le polynôme $P$ admet donc $1$
        et $-1$ comme racines.\\
        Or, le spectre de $A$ est inclus dans l'ensemble des racines
        de $P$. %
        \conc{Autrement dit : $\spc(A) \subset \{-1, 1\}$.}

      \item Vérifions que $1$ est valeur propre de $A$.
	\[
	\det (A-I_2) = \det
        \left(
        \begin{smatrix}
          -1 & 1 \\
          1 & -1
	\end{smatrix}
        \right) = 1 - 1 = 0
	\]
	Ainsi, $A-I_2$ n'est pas inversible. Donc $1$ est valeur
        propre de $A$.
      \item Vérifions que $-1$ est valeur propre de $A$.
	\[
	\det (A+I_2) = \det \left(
          \begin{smatrix} 
            1 & 1 \\
            1 & 1
          \end{smatrix} \right) = 1 - 1 = 0
	\]
        Donc $A+I_2$ n'est pas inversible. Donc $-1$ est valeur propre
        de $A$.
      \end{noliste}
      \conc{$\spc(A)=\{-1,1\}$.}~\\[-1.4cm]      
      \begin{remark}%~\\
        \begin{noliste}{$\sbullet$}
        \item Étant en présence d'une matrice de $\M{2}$, on privilégie
          ici l'utilisation du déterminant. Cependant, on peut aussi
          rédiger à l'aide d'un calcul de rang. Par exemple :
          \[
          \rg (A+I_2) = \rg \left(
            \begin{smatrix} 
              1 & 1 \\
              1 & 1
            \end{smatrix} \right) < 2 \quad \text{ car $L_1 = L_2$}
          \]
        \item Au vu de la première question, introduire un polynôme
          annulateur est un bon réflexe. Cependant, on peut faire une
          démonstration directe.\\
          Détaillons la. Soit $\lambda \in \R$.
          \[
          \begin{array}{rcl}
            \lambda \in \spc(A) & \Leftrightarrow & \text{$A-\lambda \ I$
              non inversible}
            \\[.1cm]
            & \Leftrightarrow & \det(A - \lambda \ I_2) = 0
            \\[-.4cm]
          \end{array}
          \]
          Or : $\det(A-\lambda I_2) = \det \left(
            \begin{smatrix}
              -\lambda & 1 \\
              1 & -\lambda
            \end{smatrix}
          \right) = \lambda^2-1 = (\lambda-1)(\lambda+1)$.\\          
          On obtient bien : $\spc(A) = \{-1,1\}$.
        \end{noliste}
      \end{remark}~\\[-1.4cm]
    \end{proof}
	

    \newpage


  \item La matrice $A$ est-elle diagonalisable ?
	
    \begin{proof}~\\
      La matrice $A$ est carrée d'ordre $2$ et admet $2$ valeurs
      propres {\bf distinctes}. %
      \conc{Ainsi, la matrice $A$ est diagonalisable.}~\\[-1cm]
    \end{proof}
  \end{noliste}

\item {\it Exemple 2}. Soit $B$ la matrice de $\Bc{3}$ définie par :
  $B =
  \begin{smatrix} 
    0 & 1 & 0 \\
    1 & 0 & 0 \\
    0 & 0 & 1
  \end{smatrix}$.\\
  On considère les instructions et la sortie ({\tt
    ans}) % (-$\rightarrow$)
  \Scilab{} suivantes :
  \begin{scilab}
    & B = [0,1,0;1,0,0;0,0,1] \nl %
    & P = [1,1,0;1,-1,0;0,0,1] \nl %
    & inv(P) \Sfois{} B \Sfois{} P \nl %
  \end{scilab}
  % \hspace*{1cm}   -$\rightarrow$ \\
  % \\
  % \hspace*{1cm} $ \quad\begin{matrix}
  %   1. & 0. & 0.\\
  %   0. & -1. & 0.\\
  %   0. & 0. & 1.
  % \end{matrix}$
  \[
  \begin{console}
    \lDisp{\qquad ans \ =} \nl %
    \lDisp{\qquad \qquad 1. \quad 0. \quad 0.} \nl %
    \lDisp{\qquad \qquad 0. \quad \moins 1. \quad 0.} \nl %
    \lDisp{\qquad \qquad 0. \quad 0. \quad 1.} \nle %
  \end{console}
  \]

  \begin{noliste}{a)}
    \setlength{\itemsep}{2mm}
  \item Déduire les valeurs propres de $B$ de la séquence \Scilab{}
    précédente.
    
    \begin{proof}~\\[.2cm]
      Notons $D = 
      \begin{smatrix} 
        1 & 0 & 0 \\ 
        0 & -1 & 0 \\ 
        0 & 0 & 1
      \end{smatrix}$ et $P = 
      \begin{smatrix} 
        1 & 1 & 0 \\ 
        1 & -1 & 0 \\ 
        0 & 0 & 1
      \end{smatrix}$.\\[.2cm]
      D'après la séquence \Scilab{}, $P^{-1}BP = D$, d'où $B =
      PDP^{-1}$. \\
      Ainsi, $B$ est semblable à une matrice diagonale.\\
      Elle est donc diagonalisable et ses valeurs propres sont les
      coefficients diagonaux de $D$. %
      \conc{$\spc(B) = \{-1,1\}$}~\\[-1cm]
    \end{proof}
   	
  \item Déterminer une base de chacun des sous-espaces propres de $B$.
   	
    \begin{proof}~%
      \begin{noliste}{$\sbullet$}
      \item Le programme \Scilab{} précédent nous fournit la
        diagonalisation de la matrice $B$ :
        \[
        B = PDP^{-1}
        \]
        La famille ${\cal F} = \left(
          \begin{smatrix}
            1 \\
            1 \\
            0
          \end{smatrix},
          \begin{smatrix}
            1 \\
            -1 \\
            0
          \end{smatrix},
          \begin{smatrix}
            0 \\
            0 \\
            1
          \end{smatrix}\right)$ constituée des colonnes de la matrice
        $P$ est une base de vecteurs propres de $B$. Plus
        précisément, pour tout $i \in \llb 1, 3 \rrb$, le $\eme{i}$
        vecteur de ${\cal F}$ est un vecteur propre associé au
        coefficient diagonal $d_{i,i}$.

      \item On en déduit :
        \[
        E_{-1}(B) \supset \Vect{
          \begin{smatrix}
            1 \\
            -1 \\
            0
          \end{smatrix}} 
        \qquad \text{ et } \qquad
        E_1(B) \supset \Vect{
          \begin{smatrix}
            1 \\
            1 \\
            0
          \end{smatrix},
          \begin{smatrix}
            0 \\
            0 \\
            1
          \end{smatrix}}
        \]


        \newpage


      \item Ainsi, $\dim(E_{-1}(B)) \geq 1$ et $\dim(E_{1}(B)) \geq
        2$.\\
        Or, comme $B$ est diagonalisable :
        \[
        \dim(E_{-1}(B)) + \dim(E_{1}(B)) = \dim(\M{3,1}) = 3
        \]
        On en déduit : $\dim(E_{-1}(B)) = 1$ et $\dim(E_{1}(B)) = 2$.%
        \conc{Ainsi : $E_{-1}(B) = \Vect{
            \begin{smatrix}
              1 \\
              -1 \\
              0
            \end{smatrix}}$ \quad et \quad $E_1(B) = \Vect{
            \begin{smatrix}
              1 \\
              1 \\
              0
            \end{smatrix},
            \begin{smatrix}
              0 \\
              0 \\
              1
            \end{smatrix}}$.}
      \end{noliste}
      \begin{remark}%~\\
        \begin{noliste}{$\sbullet$}
        \item On aurait aussi pu déterminer ces deux sous-espaces
          propres \og à la main \fg{}.

        \item Détaillons la méthode. Soit $X =
          \begin{smatrix}
            x \\
            y \\
            z
          \end{smatrix}
          \in \M{3,1}$.
          \[
          \begin{array}{rcl}
            X \in E_{-1}(B)
            & \Leftrightarrow & (B + I_3) \ X = 0
            \\[.2cm]
            & \Leftrightarrow & 
            \begin{smatrix}
              1 & 1 & 0 \\ 
              1 & 1 & 0 \\
              0 & 0 & 2
            \end{smatrix} %
            \begin{smatrix}
              x \\
              y \\
              z
            \end{smatrix}
            = 
            \begin{smatrix}
              0 \\
              0 \\
              0
            \end{smatrix}
            \\[.8cm]
            & \Leftrightarrow & 
            \left\{
              \begin{array}{rcrcrcr}
                x & + & y & & & = & 0 \\[.2cm]
                x & + & y & & & = & 0 \\[.2cm]
                & & & & 2 \ z & = & 0
              \end{array}
            \right.
            \\[1cm]
            &            
            \Leftrightarrow 
            &
            \left\{
              \begin{array}{rcrcr}
                x & & & = & -y \\[.2cm]
                & & z & = & 0
              \end{array}
            \right.
          \end{array}
          \]
          Ainsi : 
          \[
          \begin{array}{rcl}
            E_{-1}(B) & = & \{
            X = 
            \begin{smatrix}
              x \\
              y \\
              z
            \end{smatrix}
            \in \M{3,1}
            \ | \ X \in E_{-1}(B) \}
            \\[.4cm]
            & = & \{
            \begin{smatrix}
              x \\
              y \\
              z
            \end{smatrix}
            \ | \ x = -y \ \ET{} \ z = 0 \}
            \\[.6cm]
            & = & \{
            \begin{smatrix}
              -y \\
              y \\
              0
            \end{smatrix}
            \ | \ y \in \R \}
            \\[.6cm]
            & = & \{
            y \cdot 
            \begin{smatrix}
              -1 \\
              1 \\
              0
            \end{smatrix}
            \ | \ y \in \R \}
            \ = \ \Vect{
              \begin{smatrix}
                -1 \\
                1 \\
                0
              \end{smatrix}
            }
          \end{array}          
          \]~\\[-.8cm]
          La famille $\left(
              \begin{smatrix}
                -1 \\
                1 \\
                0
              \end{smatrix}\right)$ est donc :
              \begin{noliste}{$\stimes$}
              \item génératrice de  $E_{-1}(B)$,
              \item libre car constituée d'un unique vecteur non nul.
              \end{noliste}
            C'est donc une base de $E_{-1}(B)$.~\\[-.8cm]

          \item De même, on démontre : $E_{1}(B) =
            \Vect{ \begin{smatrix}
                1 \\
                1 \\
                0
              \end{smatrix}
              , 
              \begin{smatrix}
                0 \\
                0 \\
                1
              \end{smatrix}
            }$.
          \end{noliste}
        \end{remark}~\\[-1.4cm]
      \end{proof}
    \end{noliste}


\newpage


\item 
  \begin{noliste}{a)}
    \setlength{\itemsep}{2mm}
  \item Combien existe-t-il de matrices appartenant à $\Bc{n}$ ?
	
    \begin{proof}~\\%
      Une matrice $M$ de $\Bc{n}$ est une matrice de $\M{n}$ dont
      chaque coefficient vaut soit $0$ soit $1$.\\
      Une telle matrice est entièrement déterminée par :
      \begin{noliste}{$\stimes$}
      \item le choix du coefficient $m_{11}$ : 2 possibilités.
      \item \ldots
      \item le choix du coefficient $m_{1n}$ : 2 possibilités.
      \item le choix du coefficient $m_{21}$ : 2 possibilités.
      \item \ldots
      \item le choix du coefficient $m_{2n}$ : 2 possibilités.
      \item \ldots
      \item le choix du coefficient $m_{n1}$ : 2 possibilités.
      \item \ldots
      \item le choix du coefficient $m_{nn}$ : 2 possibilités.
      \end{noliste}
      Il y a donc $2^{n^2}$ telles matrices.%
      \conc{On en déduit : $\Card(\Bc{n}) = 2^{n^2}$.}~\\[-1cm]
    \end{proof}
    
  \item Combien existe-t-il de matrices de $\Bc{n}$ dont chaque ligne
    et chaque colonne comporte exactement un coefficient égal à $1$
    ?
	
    \begin{proof}~\\
      Une matrice $M$ de $\Bc{n}$ dont chaque ligne et chaque colonne
      comporte exactement un coefficient égal à $1$ est entièrement
      déterminée par :
      \begin{noliste}{$\stimes$}
      \item la position du $1$ sur la $\ere{1}$ ligne : $n$
        possibilités.\\
        {\it (on peut le placer sur n'importe quelle colonne)}
      \item la position du $1$ sur la $\eme{2}$ ligne : $n-1$
        possibilités.\\
        {\it (on peut le placer sur n'importe quelle colonne non déjà
          choisie)}
      \item la position du $1$ sur la $\eme{3}$ ligne : $n-2$
        possibilités.\\
        {\it (on peut le placer sur n'importe quelle colonne non déjà
          choisie)}
      \item \ldots
      \item la position du $1$ sur la $\eme{n}$ ligne : $1$
        possibilité.\\
        {\it (on peut le placer sur n'importe quelle colonne non déjà
          choisie)}
      \end{noliste}
      Il existe donc $n \ (n-1) \ (n-2) \cdots 1 = n!$ telles
      matrices.~\\[-.8cm]%
      \concL{L'ensemble des matrices de $\Bc{n}$ dont chaque ligne et
        chaque colonne comporte exactement un coefficient égal à
        $1$ comporte exactement $n!$ éléments.}{15.4}%~\\[-1cm]
      \begin{remark}%~\\
        \begin{noliste}{$\sbullet$}
        \item Une matrice de $\Bc{n}$ dont chaque ligne et chaque
          colonne comporte exactement un coefficient égal à $1$
          peut être représentée par le $n$-uplet $(c_1, \ldots, c_n)$
          où $c_i \in \llb 1, n \rrb$ est le numéro de colonne où se
          situe le coefficient égal à $1$ de la ligne $i$.
        \item Dans cette question, on demande donc de dénombrer
          l'ensemble des $n$-uplets d'éléments distincts (il n'y a
          qu'un $1$ sur chaque ligne) de $\llb 1 , n \rrb$. Autrement
          dit, on s'intéresse aux $n$-arrangements de $\llb 1 , n
          \rrb$ ou encore aux permutations de $\llb 1 , n \rrb$.\\
          Il y en a exactement $n!$
        \end{noliste}
      \end{remark}~\\[-1.4cm]
    \end{proof}
  \end{noliste}


  \newpage

  
\item Dans cette question, $n$ est un entier supérieur ou égal à $2$.\\
  Soit $E$ un espace vectoriel de dimension $n$ et $u$ un
  endomorphisme de $E$. On note :
  \begin{noliste}{$-$}
  \item $\id$ l'endomorphisme identité de $E$ ;
  \item $F$ le noyau de l'endomorphisme $(u+\id)$ et $G$ le noyau de
    l'endomorphisme $(u-\id)$ ;
  \item $p$ la dimension de $F$ et $q$ la dimension de $G$.
  \end{noliste}
  On suppose que $u \circ u =\id$.
  \begin{noliste}{a)}
    \setlength{\itemsep}{2mm}
  \item Justifier que l'image de $(u-\id)$ est incluse dans $F$.
	
    \begin{proof}~\\
      Il s'agit de montrer que $\im(u-\id) \subset \kr(u + \id) = F$.\\[.2cm]
      Soit $y \in \im(u - \id)$. Par définition, il existe $x\in E$
      tel que $y = (u-\id)(x) = u(x)-x$.\\
      Démontrons que $y \in \kr(u + \id)$.
      \[
      \begin{array}{rcl@{\quad}>{\it}R{3.5cm}}
	(u + \id)(y) & = & u(y) + y
        \\[.2cm]
	& = & u(u(x)-x)+(u(x)-x) & (par définition de $y$) 
        \nl
        \nl[-.2cm]
	& = & u(u(x)) - \bcancel{u(x)}+\bcancel{u(x)}-x & (par
        linéarité de $u$) 
        \nl
        \nl[-.2cm]
	& = & \bcancel{x} - \bcancel{x} \ = \ 0_E & (car $u\circ u=\id$)
        % \nl
        % \nl[-.2cm]
        % & = & 0_E
      \end{array}
      \]
      Donc $y\in\kr(u+\id) = F$.%
      \conc{$\im(u-\id)\subset F$}~\\[-1cm]
    \end{proof}
    
  \item En déduire l'inégalité : $p + q \geq n$.
    
    \begin{proof}~
      \begin{noliste}{$\sbullet$}
      \item D'après la question \itbf{4.a)}, $\im(u-\id) \subset \kr(u
        + \id)$.%
	\conc{Ainsi : $\dim(\im(u-\id)) \leq \dim(\kr(u+\id)) = p$.}

      \item D'après le théorème du rang :
        % appliqué à l'endomorphisme $u-\id$,
	\[
        \begin{array}{ccccc}
          \dim(E) & = & \dim(\kr(u-\id)) & + & \dim(\im(u-\id))
          \\[.2cm]
          \shortparallel & & \shortparallel & & 
          \\[.2cm]
          n & & q & & 
        \end{array}        
	\]
	\conc{$\dim(\im(u-\id)) = n - q$}~

      \item En combinant ces deux résultats, on obtient : $n - q \leq
        p$.\\%
        \conc{Ainsi, on a bien : $p + q \geq n$.}~\\[-1.4cm]
      \end{noliste}
    \end{proof}
    {\it On suppose désormais que $1\leq p<q$.}\\
    Soit $(f_1, f_2, \ldots, f_p)$ une base de $F$ et $(g_1, g_2,
    \ldots, g_q)$ une base de $G$.

  \item Justifier que $(f_1, f_2, \hdots, f_p, \ g_1,g_2,\hdots,g_q)$
    est une base de $E$.
    
    \begin{proof}~%
      \begin{noliste}{$\sbullet$}
      \item On suppose ici que $\dim(\kr(u + \id)) = p \geq 1$. En
        particulier : $\kr(u + \id) \neq \{ 0_E \}$.%
        \conc{Ainsi, $-1$ est valeur propre de $u$ et $F = \kr(u +
          \id) = E_{-1}(u)$.}

      \item De même, $\dim(\kr(u - \id)) = q > 1$. En particulier :
        $\kr(u - \id) \neq \{ 0_E \}$.%
        \conc{Ainsi, $1$ est valeur propre de $u$ et $G = \kr(u -
          \id) = E_{1}(u)$.}


        \newpage

        
      \item On a alors :
        \begin{noliste}{$\stimes$}
        \item la famille $(f_1, f_2, \ldots, f_p)$ est une base de
          $E_{-1}(u)$.\\
          En particulier, c'est donc une famille libre de $E_{-1}(u)$.
        \item la famille $(g_1, g_2, \ldots, g_q)$ est une base de
          $E_{1}(u)$.\\
          En particulier, c'est donc une famille libre de $E_{1}(u)$.
        \end{noliste}
        La famille ${\cal F} = (f_1, f_2, \ldots, f_p, \ g_1, g_2,
        \ldots, g_q)$ est la concaténation de deux familles libres de
        sous-espaces propres associés à des valeurs propres
        distinctes.%
        \conc{Ainsi, ${\cal F}$ est une famille libre de $E$.}
      \end{noliste}
      \begin{remark}
        L'énoncé demande de \og Justifier \fg{} que ${\cal F}$ est une
        base de $G$. Cette terminologie est souvent associée à des
        démonstrations courtes. Ici, il est aussi possible de
        démontrer que ${\cal F} = (f_1, f_2, \ldots, f_p, g_1, g_2,
        \ldots, g_q)$ est libre en revenant à la définition. Ce
        raisonnement est tout aussi rigoureux mais plus long à mettre
        en place, ce qui provoque une perte de temps pénalisante pour
        la suite. Détaillons cette rédaction.
        \begin{noliste}{$\sbullet$}
        \item Soit $(\lambda_1, \ldots, \lambda_p, \mu_1, \ldots,\mu_q)
          \in \R^{p+q}$. On suppose :
          \[
          \begin{array}{rcccccccccccl}
            \lambda_1 \cdot f_1 & + & \ldots & + & \lambda_p \cdot f_p &
            + & \mu_1 \cdot g_1 & + & \ldots & + & \mu_q \cdot g_q & = &
            0_E \quad (*)
          \end{array}
          \]
          En appliquant l'endomorphisme $u$ de part et d'autre, on
          obtient par linéarité :
          \[
          \begin{array}{C{1cm}rcl}
            & u\big(  \lambda_1 \cdot f_1  +  \ldots  +  \lambda_p \cdot
            f_p  +   \mu_1 \cdot g_1  +  \ldots  +  \mu_q \cdot
            g_q  \big) & = & u(0_E)
            \\[.4cm]
            puis & \lambda_1 \cdot u(f_1)  +  \ldots  +  \lambda_p \cdot
            u(f_p)  +  \mu_1 \cdot u(g_1)  +  \ldots  +  \mu_q
            \cdot u(g_q)  & = & 0_E 
            \\[.4cm]
            et & -\lambda_1 \cdot f_1  -  \ldots  -  \lambda_p \cdot
            f_p  +  \mu_1 \cdot g_1  +  \ldots  +  \mu_q \cdot
            g_q & = & 0_E \quad (**)
          \end{array}
          \]
%           \[
%           \begin{array}{C{.4cm}rcccccccccccccl}
%             & u\big( & \lambda_1 \cdot f_1 & + & \ldots & + & \lambda_p \cdot
%             f_p & + &  \mu_1 \cdot g_1 & + & \ldots & + & \mu_q \cdot
%             g_q & \big) & = & u(0_E)
%             \\[.4cm]
%             puis & & \lambda_1 \cdot u(f_1) & + & \ldots & + & \lambda_p \cdot
%             u(f_p) & + & \mu_1 \cdot u(g_1) & + & \ldots & + & \mu_q
%             \cdot u(g_q) & & = & 0_E 
%             \\[.4cm]
%             et & & -\lambda_1 \cdot f_1 & - & \ldots & - & \lambda_p \cdot
%             f_p & + & \mu_1 \cdot g_1 & + & \ldots & + & \mu_q \cdot
%             g_q & & = & 0_E \quad (**)
%           \end{array}
%           \]
          En effet :
          \begin{noliste}{$\stimes$}
          \item pour tout $i \in \llb 1, p \rrb$, $f_i\in F = \kr(u +
            \id)$. Et ainsi :
            \[
            \begin{array}{ccl}
              (u + \id) (f_i) & = & 0_E
              \\[.2cm]
              \shortparallel 
              \\[.2cm]
              u(f_i) + \id(f_i) & = & u(f_i) + f_i
            \end{array}
            % \ie $u(f_i)+f_i=0$, c'est-à-dire $u(f_i)=-f_i$
            \]
            On en déduit : $\forall i \in \llb 1, p \rrb$, $u(f_i) =
            -f_i$.
            
          \item pour tout $j \in \llb 1, q \rrb$, $g_j \in G = \kr(u -
            \id)$. Et ainsi : $(u - \id) (g_j) = 0_E$.\\[.1cm]%
            On obtient de même : $\forall j \in \llb 1, q \rrb$,
            $u(g_j) = g_j$.
          \end{noliste}
          
        \item En sommant membre à membre les égalités $(*)$ et $(**)$,
          on obtient :
          \[
          \mu_1 \cdot g_1 + \ldots + \mu_q \cdot g_q = 0_E
          \]
          Or $(g_1,\hdots, g_q)$ est une base de $G$. En particulier,
          c'est une famille libre.\\
          On en déduit : $\mu_1 = \ldots = \mu_q = 0_{\R}$.
          
        \item On reporte dans $(*)$ et on obtient : $\lambda_1 \cdot
          f_1 + \ldots + \lambda_p \cdot f_p = 0_E$.\\
          Or $(f_1,\hdots, f_q)$ est une base de $F$. En particulier,
          c'est une famille libre.\\
          On en déduit : $\lambda_1 = \ldots = \lambda_p = 0_{\R}$.
          
        \item Finalement : $\lambda_1 = \ldots = \lambda_p = \mu_1 =
          \ldots = \mu_q = 0$.%
          \conc{La famille $(f_1,f_2,\hdots,f_p,g_1,g_2,\hdots,g_q)$ est
            libre.}
        \end{noliste}
      \end{remark}


    \newpage


    \begin{noliste}{$\sbullet$}
    \item La famille ${\cal F}$ est une famille libre de $E$, espace
      vectoriel de dimension $n$. Ainsi :
      \[
      \Card({\cal F}) = p+q \ \leq \ n = \dim(E)
      \]
      Or, d'après la question précédente : $p+q \geq n$.%
      \conc{On en déduit : $p + q = n$.}
      
    \item En résumé :
      \begin{noliste}{$\stimes$}
      \item la famille ${\cal F}$ est libre.
      \item $\Card({\cal F}) = n = \dim(E)$.
      \end{noliste}
      \conc{La famille ${\cal F} = (f_1, f_2, \ldots, f_p, g_1, g_2,
        \ldots, g_q)$ est donc une base de $E$.}~\\[-1.6cm]
    \end{noliste}    
    \end{proof}
  
  \item Calculer $u(g_1-f_1)$ et $u(g_1+f_1)$.
    
    \begin{proof}~
      \begin{noliste}{$\sbullet$}
      \item Tout d'abord : 
        \[
        \begin{array}{rcccc@{\quad}>{\it}R{4cm}}
          u(g_1 - f_1) & = & u(g_1) & - & u(f_1) & (par linéarité de $u$) 
          \nl
          \nl[-.2cm]
          & = & 1 \cdot g_1 & - & (-1) \cdot f_1 & (car $g_1 \in
          E_1(u)$ \\ et $f_1 \in E_{-1}(u)$) 
          \nl
          \nl[-.2cm]
          & = & g_1 & + & f_1
        \end{array}
        \]

      \item On démontre de même : $u(g_1+f_1) = u(g_1) + u(f_1) =
        g_1-f_1$.%
      \end{noliste}
      \conc{$u(g_1-f_1) = g_1+f_1$ \quad et \quad $u(g_1+f_1) =
        g_1-f_1$}~\\[-1.2cm]
    \end{proof}
    
  \item Trouver une base de $E$ dans laquelle la matrice de $u$
    appartient à $\Bc{n}$.
	
    \begin{proof}~\\
      On considère $\B'$ la famille :
      \[
      \B' = (g_1 - f_1, \ g_1 + f_1, \ g_2 - f_2, \ g_2 + f_2, \ldots,
      \ g_p - f_p, \ g_p + f_p, \ g_{p+1} , \ g_{p+2}, \ldots, \ g_q)
      \]
      (qui est bien définie car $p < q$)	
      \begin{noliste}{$\sbullet$}
      \item Montrons tout d'abord que $\B'$ est une famille libre.\\
        Soit $(\lambda_1, \hdots, \lambda_p, \mu_1, \hdots, \mu_p,
        \gamma_{p+1},\hdots,\gamma_q)\in\R^n$. Supposons :
        \[
        \lambda_1 \cdot (g_1-f_1) + \mu_1 \cdot (g_1+f_1) + \ldots +
        \lambda_p \cdot (g_p-f_p) + \mu_p \cdot (g_p+f_p) +
        \gamma_{p+1} \cdot g_{p+1} + \ldots + \gamma_q \cdot g_q = 0_E
        \]
        En réordonnant :
        \[
        (\mu_1-\lambda_1) \cdot f_1 + \ldots + (\mu_p-\lambda_p) \cdot
        f_p + (\mu_1+\lambda_1) \cdot g_1 + \ldots + (\mu_p+\lambda_p)
        \cdot g_p + \gamma_{p+1} \cdot g_{p+1} + \ldots + \gamma_q
        \cdot g_q = 0_E
        \]
        Or, d'après la question \itbf{4.c)},
        $(f_1,\hdots,f_p,g_1,\hdots,g_q)$ est une base de $E$.\\
        C'est donc une famille libre. Ainsi :
        \[
        \left\{
          \begin{array}{l}
            \forall i\in\llb 1,p\rrb, \quad \mu_i-\lambda_i=0 \\[.2cm]
            \forall i\in\llb 1,p\rrb, \quad \mu_i+\lambda_i=0 \\[.2cm]
            \forall j\in\llb p+1,q\rrb, \quad \gamma_j=0 
          \end{array}
        \right.
        \]
        Or, pour tout $i\in\llb 1,p\rrb$ :
        \[
        \begin{array}{ccccl}
        \left\{
          \begin{matrix}
            \mu_i & - & \lambda_i & = & 0 \\
            \mu_i & + & \lambda_i & = & 0
          \end{matrix}
        \right. 
        &
        \begin{arrayEq}
          L_2 \leftarrow L_2 - L_1
        \end{arrayEq}
        & 
        \left\{
          \begin{matrix}
            \mu_i & - & \lambda_i & = & 0 \\
            & & 2 \ \lambda_i & = & 0
          \end{matrix}
        \right. 
        &
        \Longleftrightarrow
        &
        \left\{
          \begin{matrix}
            \mu_i & & & = & 0 \\
            & & \lambda_i & = & 0
          \end{matrix}
        \right.
        \\[.4cm]
        & & & & \text{\it (par remontées successives)}
      \end{array}
      \]
      Ainsi : $\lambda_1 = \ldots = \lambda_p = \mu_1 = \ldots = \mu_p
      = \gamma_{p+1} = \ldots = \gamma_q = 0$. %
      \conc{La famille $\B'$ est libre.}


      \newpage


    \item On a alors : 
      \begin{noliste}{$\stimes$}
      \item la famille $\B'$ est libre.
      \item $\Card(\B') = n = \dim(E)$.
      \end{noliste}~\\[-1cm]
      \conc{Ainsi, $\B'$ est une base de $E$.}
    
    \item Déterminons la matrice de $u$ dans cette base.\\
      Pour plus de lisibilité, notons : $\forall i \in \llb 1,p\rrb$,
      $r_i = g_i - f_i$ \ et \ $s_i = g_i + f_i$.\\
      On démontre, par le même raisonnement que dans la question
      précédente : 
      \[
      \forall i \in \llb 1, p \rrb, \ u(g_i - f_i) = g_i + f_i \quad
      \text{ et } \quad u(g_i + f_i) = g_i - f_i
      \]
      La base $\B'$ s'écrit alors : $\B' = (r_1, \ s_1, \ \ldots, \
      r_i, \ s_i, \ldots, \ r_p, \ s_p, \ g_{p+1}, \ \ldots, \ g_q)$
      et :
      \[
      \left\{
        \begin{array}{l}
          \forall i\in\llb 1, p \rrb, \quad u(r_i) = s_i
          \\[.2cm]
          \forall i \in\llb 1, p \rrb, \quad u(s_i) = r_i 
          \\[.2cm]
          \forall j \in\llb p+1, q \rrb, \quad u(g_j) = g_j
        \end{array}
      \right.      
      \]
      On en déduit la matrice représentative de $u$ dans la base $\B'$
      :
      \[
      \Mat_{\B'}(u) = %
      \scalebox{.8}{$
        \begin{array}{l}
          \begin{array}{C{.cm}*{8}{>{$}C{.8cm}<{$}}>{$}C{1cm}<{$}*{2}{>{$}C{.8cm}<{$}}}
            & u(r_1) & u(s_1) & \ldots & u(r_i) & u(s_i) & \ldots & u(r_p)
            & u(s_p) & u(g_{p+1}) & \ldots & u(g_q)
          \end{array}
          \\[.6cm]
          \left(
            \begin{array}{*{8}{>{$}C{.8cm}<{$}}>{$}C{1cm}<{$}*{2}{>{$}C{.8cm}<{$}}}
              0 & 1 & & & & & & & 0 & \cdots & 0
              \nl
              \nl[-.2cm]
              1 & 0 & & & & & & & \vdots & & \vdots
              \nl
              \nl[-.2cm]
              & & \ddots & & & & & & \vdots & & \vdots
              \nl
              \nl[-.2cm]
              & & & 0 & 1 & & & & \vdots & & \vdots
              \nl
              \nl[-.2cm]
              & & & 1 & 0 & & & & \vdots & & \vdots
              \nl
              \nl[-.2cm]
              & & & & & \ddots & & & \vdots & & \vdots
              \nl
              \nl[-.2cm]
              & & & & & & 0 & 1 & \vdots & & \vdots
              \nl
              \nl[-.2cm]
              & & & & & & 1 & 0 & 0 & \cdots & 0
              \nl
              \nl[-.2cm]
              0 & \cdots & \cdots & \cdots & \cdots & \cdots & \cdots
              & 0 & 1 & &  
              \nl
              \nl[-.2cm]
              \vdots & & & & & & & \vdots & & \ddots & 
              \nl
              \nl[-.2cm]
              0 & \cdots & \cdots & \cdots & \cdots & \cdots & \cdots
              & 0 & & & 1 
            \end{array}
          \right)
          \begin{array}{>{$}C{.8cm}<{$}}
            r_1 \phantom{0}
            \nl
            \nl[-.2cm]
            s_1 \phantom{\vdots}
            \nl
            \nl[-.2cm]
            \vdots 
            \nl
            \nl[-.2cm]
            r_i \phantom{\vdots}
            \nl
            \nl[-.2cm]
            s_i \phantom{\vdots}
            \nl
            \nl[-.2cm]
            \vdots
            \nl
            \nl[-.2cm]
            r_p \phantom{\vdots}
            \nl
            \nl[-.2cm]
            s_p \phantom{0}
            \nl
            \nl[-.2cm]
            g_{p+1} \phantom{0}
            \nl
            \nl[-.2cm]
            \vdots 
            \nl
            \nl[-.22cm]
            g_q \phantom{0}
          \end{array}
        \end{array}
        $}
    \]%~\\[-.4cm]
    \conc{Dans la base $\B'$, la matrice de $u$ appartient à
      $\Bc{n}$.}
  \end{noliste}
  \begin{remark}%~\\
      Cette question est difficile car elle demande de prendre
      beaucoup d'initiatives. 
      \begin{noliste}{$\sbullet$}
      \item Il faut tout d'abord se rendre compte que la base ${\cal
          F} = (f_1, \ldots, f_p, g_1, \ldots, g_q)$ ne convient
        pas. En effet, comme : 
        \[
        \left\{
          \begin{array}{l}
            \forall i\in\llb 1, p \rrb, \quad u(f_i) = - f_i
            \\[.2cm]
            \forall i \in\llb 1, q \rrb, \quad u(g_i) = g_i 
          \end{array}
        \right.      
      \]        
      la matrice représentative de $u$ dans ${\cal F}$ contient des
      $-1$ (en plus des $0$ et des $1$).
    \item L'idée est alors de créer une base sur laquelle $u$ a pour
      effet d'échanger les éléments de cette base. Cette idée est
      guidée par la question précédente dans laquelle on démontre :
      $u(g_1 - f_1) = g_1 + f_1$ \ et \ $u(g_1 + f_1) = g_1 - f_1$.
    \end{noliste}
    \end{remark}~\\[-1.4cm]
\end{proof}
\end{noliste}
%       \[
%       \Mat_{\B'}(u) = %
%       \begin{array}{l}
%         \left(
%         \begin{array}{cccccccccccccc}        
%           0 & 1 & 0 & 0 & \cdots & 0 & 0
%           & 0 & 0 & \cdots &
%           0 &  0 &  \cdots &  0 \\
%           1 & 0 & 0 & 0 & \cdots & 0 & 0 & 0 & 0 & \cdots & 0 & 0 &
%           \cdots & 0 \\
%           0 & 0 & 0 & 1 & \cdots & 0 & 0 & 0 & 0 & \cdots &
%           0 &  0 &  \cdots &  0 \\
%           0 & 0 & 1 & 0 & \cdots & 0 & 0 & 0 & 0 & \cdots & 0 & 0 &
%           \cdots & 0 \\
%           \vdots & \vdots & \vdots & \vdots & \ddots & \vdots & \vdots &
%           \vdots & \vdots & \vdots & & \vdots & &
%           \vdots \\
%           0 & 0 & 0 & 0 & \cdots & 0 & 1 & 0 & 0 & \cdots &
%           0 &  0 &  \cdots &  0 \\
%           0 & 0 & 0 & 0 & \cdots & 1 & 0 & 0 & 0 & \cdots &
%           0 &  0 &  \cdots &  0 \\
%           0 & 0 & 0 & 0 & \cdots & 0 & 0 & 0 & 1 & \cdots &
%           0 &  0 &  \cdots &  0 \\
%           0 & 0 & 0 & 0 & \cdots & 0 & 0 & 1 & 0 & \cdots &
%           0 &  0 &  \cdots &  0 \\
%           \vdots & \vdots & \vdots & \vdots & & \vdots & \vdots & \vdots
%           & \vdots & \ddots & \vdots & \vdots &
%           &  \vdots \\
%           0 & 0 & 0 & 0 & \cdots & 0 & 0 & 0 & 0 & \cdots &
%           1 &  0 &  \cdots &  0 \\
%           0 & 0 & 0 & 0 & \cdots & 0 & 0 & 0 & 0 & \cdots &
%           0 &  1 &  \cdots &  0 \\
%           \vdots & \vdots & \vdots & \vdots & & \vdots & \vdots & \vdots
%           & \vdots & & \vdots & \vdots &
%           \ddots &  \vdots \\
%           0 & 0 & 0 & 0 & \cdots & 0 & 0 & 0 & 0 & \cdots &
%           0 &  0 &  \cdots &  1 \\
%         \end{array}
%       \right)
%       \begin{array}{c}
%         g_1-f_1 \\ 
%         g_1+f_1 \\ 
%         g_2-f_2 \\
%         g_2+f_2 \\ 
%         \vdots \\ 
%         g_i-f_i \\
%         g_i+f_i \\ 
%         g_{i+1}-f_{i+1} \\
%         g_{i+1}+f_{i+1} \\ 
%         \vdots \\ 
%         g_{p+1}\\ 
%         g_{p+2}\\ 
%         \vdots \\ 
%         g_q
%       \end{array}
%     \end{array}
%     \]


\newpage


\begin{remark}%L}{.95}%~
  Prenons un peu de recul sur le thème développé dans cet exercice.
  \begin{noliste}{$\sbullet$}
  \item Dans cet exercice, on s'intéresse aux endomorphismes
    involutifs de $E$, c'est à dire aux applications $s \in \LL{E}$
    qui vérifient :
    \[
    s \circ s = \id
    \]
    (cette égalité démontre que $s$ est bijectif de réciproque $s$)\\
    Les cas les plus simples de telles applications sont : $s = \id$
    et $s = -\id$.
    % Un tel endomorphisme est appelé une symétrie.

  \item Lors de l'étude de $s$, on considère généralement les espaces
    vectoriels suivants :
    \begin{noliste}{$\stimes$}
    \item $F = \kr(s - \id)$ : c'est l'ensemble des vecteurs
      invariants par $s$.\\
      En effet, si $x \in F$, alors : $(s - \id)(x) = 0$ et donc $s(x)
      = x$.
    \item $G = \kr(s + \id)$ : c'est l'ensemble des vecteurs changés
      en leur opposé par $s$.\\
      En effet, si $x \in G$, alors : $(s + \id)(x) = 0$ et donc $s(x)
      = -x$.      
    \end{noliste}
    {\it (on prend ici les notations classiques - dans l'énoncé, les
      rôles de $F$ et $G$ sont échangés)}\\[.2cm]
    Dès lors, on comprend pourquoi un endomorphisme involutif $s \in
    \LL{E}$ est appelé {\bf symétrie} par rapport à $F$ parallèlement
    à $G$.
  \item On peut aussi étudier les symétries dans le cadre de la
    réduction.\\
    On suppose que $F \neq \{ 0_E \}$ et $G \neq \{ 0_E \}$. Alors : 
    \[
    F = \kr(s - \id) = E_{1}(s) \quad \text{ et } \quad G = \kr(s +
    \id) = E_{-1}(s)
    \]
    On peut démontrer : 
    \[
    E \ = \ \kr(s - \id) \ \oplus \ \kr(s + \id)
    \]
    Cette égalité signifie que tout vecteur de $E$ s'écrit de manière
    unique sous la forme d'une somme d'un vecteur de $\kr(s - \id)$ et
    d'un vecteur de $\kr(s + \id)$.\\
    Autrement dit :
    \[
    \forall x \in E, \exists ! (y, z) \in \kr(s - \id) \times \kr(s +
    \id), \ x = y + z
    \]
%     Il faut expliciter la notation $\oplus$ qui n'est pas un attendu
%     du programme.\\
%     L'égalité précédente signifie :
%     \begin{noliste}{$\stimes$}
%     \item $E \ = \ \kr(s - \id) \ + \ \kr(s + \id)$\\
%       Ceci signifie que tout vecteur de $E$ peut s'écrire sous la
%       forme d'une somme d'un vecteur de $\kr(s - \id)$ et d'un vecteur
%       de $\kr(s + \id)$.\\
%       ($\forall x \in E$, $\exists (y, z) \in \kr(s - \id) \times
%       \kr(s + \id)$, $x = y+z$)
%     \item $\kr(s - \id) \ \cap \ \kr(s + \id) \ = \ \{ 0_E \}$\\
%       Ceci permet de démontrer que la décomposition sous forme de
%       somme est unique.% pour chaque vecteur de $E$.
%     \end{noliste}
    Considérons alors la famille $\B = \B_1 \cup \B_{-1}$ obtenue en
    concaténant une base $\B_1$ de $E_1(s) = \kr(s-\id)$ et une base
    $\B_{-1}$ de $E_{-1}(s) = \kr(s+\id)$.\\
    Cette famille est libre par construction.\\
    De plus, elle est génératrice de $E$ par la décomposition
    précédente $E = F + G$ (tout vecteur de $E$ peut s'écrire comme
    somme d'un vecteur de $F$ et d'un vecteur de $G$).\\[.2cm]
    Finalement $\B$ est une base de vecteurs propres et $s$ s'écrit
    dans cette base comme matrice diagonale dont la diagonale ne
    contient que des $1$ et des $-1$.

  \item La notion de symétrie n'est pas au programme de la voie ECE.\\
    Toutefois, on peut faire l'étude de tels endomorphismes avec des
    outils au programme. C'est donc un candidat idéal pour faire un
    sujet de concours.

  \item Cette notion est très liée à la notion de {\bf projecteurs}
    qui ne sont autres que les endomorphismes idempotents de $E$,
    c'est à dire les applications $p \in \LL{E}$ qui vérifient :
    \[
    p \circ p = p
    \]
    Ceci n'est pas non plus au programme mais revient
    régulièrement pour les raisons citées au-dessus.
    %   Un tel endomorphisme est appelé un {\bf projecteur}. L'étude de
    %   tels endomorphismes est très similaire à la précédente. On
    %   démontre notamment que :
    %   \[
    %   E \ = \ \kr(p - \id) \ \oplus \ \kr(p)
    %   \]
  \end{noliste}
  % Faire un topo de contenu sur les symétries.
\end{remark}%L}
\end{noliste}


\newpage


\section*{PROBLÈME}

\noindent %
{\it Les tables de mortalité sont utilisées en démographie et en
  actuariat pour prévoir l'espérance de vie des individus d'une
  population. On s'intéresse dans ce problème à un modèle qui permet
  d'ajuster la durée de vie à des statistiques portant sur les décès
  observés au sein d'une génération.}\\
{\bf Dans tout le problème}, on note :
\begin{noliste}{$\sbullet$}
\item $a$ et $b$ deux réels strictement positifs ;
\item $(\Omega,\mathcal{A},\Prob)$ un espace probabilisé sur lequel sont
  définies toutes les variables aléatoires du problème ;
\item $G_{a,b}$ la fonction définie sur $\R_+$ par :
  $G_{a,b}(x)=\exp\left(-ax-\dfrac{b}{2}x^2\right)$.
\end{noliste}

\subsection*{Partie I. Loi exponentielle linéaire}

\begin{noliste}{1.}
\item
  \begin{noliste}{a)}
    \setlength{\itemsep}{2mm}
  \item Montrer que la fonction $G_{a,b}$ réalise une bijection de
    $\R_+$ sur l'intervalle $]0,1]$.
    
    \begin{proof}~
      \begin{noliste}{$\sbullet$}
      \item La fonction $G_{a, b}$ est de classe $\Cont{\infty}$ sur $[0,
        +\infty[$ car elle est la composée $G_{a, b} = g_2 \circ g_1$ 
	où :
      \end{noliste}
      \begin{liste}{$\stimes$}
      \item $g_1 : x \mapsto -a x - \dfrac{b}{2} x^2$ est :
        \begin{noliste}{-}
        \item de classe $\Cont{\infty}$ sur $[0, +\infty[$ car
          polynomiale,
        \item telle que : $g_1([0, +\infty[) \subset \R$.
        \end{noliste}
      \item $g_2 : x \mapsto \exp(x)$, de classe $\Cont{\infty}$ sur $\R$.
      \end{liste}

      \begin{noliste}{$\sbullet$}
      \item Pour tout $x \in [0, +\infty[$ :
        \[
        \begin{array}{cccccc}
          G'_{a, b}(x) = (- a - b x) \ \exp\left( -a x - \dfrac{b}{2}
            x^2 \right) & = & - & (a+bx) & \exp\left( -a x - \dfrac{b}{2} x^2
          \right) & < \ 0 \\[-.5cm]
          & & & \bbacc{1.4cm} & \bbacc{2.4cm} & \\[.4cm]
          & & & > 0 & > 0 
        \end{array}
        \]
        La fonction $G_{a, b}$ est donc strictement décroissante sur
        $[0, +\infty[$.

      \item La fonction $G_{a,b}$ est :
        \begin{noliste}{$\stimes$}
        \item continue sur $[0, +\infty[$,
        \item strictement décroissante sur $[0, +\infty[$.        
        \end{noliste}
        Ainsi, $G_{a,b}$ réalise une bijection de $[0, +\infty[$ sur
        $G_{a,b}([0,+\infty[) = \ ]\dlim{x \tend +\infty} G_{a,b}(x),
        G_{a,b}(0)]$.\\
        Enfin : $G_{a, b}(0) = \exp(0) = 1$.\\
        Et : $\dlim{x \tend +\infty} G_{a,b}(x) = 0$ car $\dlim{x
          \tend +\infty} \left(-a x - \dfrac{b}{2} x^2\right) = - 
	\infty$.
      \end{noliste}
      \conc{Ainsi, $G_{a, b}$ réalise une bijection de $[0, +\infty[$
        sur $]0, 1]$.}~\\[-1.2cm]
    \end{proof}
    

    \newpage


  \item Pour tout réel $y>0$, résoudre l'équation d'inconnue $x\in\R$
    : $ax + \dfrac{b}{2} x^2 = y$.

    \begin{proof}~\\
      Soit $y > 0$. Notons : $P(x) = \dfrac{b}{2} \ x^2 + a x - y$.
      \begin{noliste}{$\sbullet$}
      \item Calculons le discriminant du polynôme $P$ :
        \[
        \Delta = a^2 - 4 \times \dfrac{b}{2} \times (-y) = a^2 + 2by \
        > \ 0
        \]

      \item On en déduit que $P$ admet exactement deux racines notées
        $r_+$ et $r_-$ :
        \[
        r_+ = \dfrac{-a + \sqrt{\Delta}}{2 \frac{b}{2}} = \dfrac{-a +
          \sqrt{a^2 + 2by}}{b} \qquad \mbox{ et } \qquad r_- =
        \dfrac{-a - \sqrt{\Delta}}{2 \frac{b}{2}} = - \dfrac{a +
          \sqrt{a^2 + 2by}}{b}
        \]

      \item Or $b > 0$ et $y > 0$ donc $a^2 + 2by > a^2$ et $\sqrt{a^2
          + 2by} > \sqrt{a^2} = \vert a \vert = a$. \\
          On en déduit que $r_+ > 0$.\\[.2cm]
        D'autre part, $r_- < 0$ car $a > 0$, $\sqrt{a^2 + 2by} > 0$ et
        $b > 0$.
      \end{noliste}
      \conc{L'équation $ax + \dfrac{b}{2} x^2 = y$ admet deux
        solutions : $r_+ > 0$ et $r_- < 0$.}~\\[-1cm]
    \end{proof}

  \item On note $G_{a,b}^{-1}$ la bijection réciproque de
    $G_{a,b}$. \\
    Quelle est, pour tout $u\in[0,1[$, l'expression de
    $G_{a,b}^{-1}(1-u)$ ?

    \begin{proof}~\\
      Soit $u \in [0, 1[$. 
      \begin{noliste}{$\sbullet$}
      \item On remarque tout d'abord que $1 - u \in \ ]0, 1]$. Notons
        alors : $v = G_{a,b}^{-1}(1-u)$. \\
        Par définition de $G_{a,b}$ et $G^{-1}_{a,b}$ :
        \[
        \begin{array}{crcl}
          & v & = & G^{-1}_{a,b}(1-u) \\[.4cm]
          \Leftrightarrow & G_{a,b}(v) & = & 1 - u \\[.4cm]
          \Leftrightarrow & \exp\left(- a v - \dfrac{b}{2} v^2 \right) &
          = & 1 - u \\[.4cm] 
          \Leftrightarrow & - a v - \dfrac{b}{2} v^2 & = & \ln(1 - u) \\[.4cm]
          \Leftrightarrow & a v + \dfrac{b}{2} v^2 & = & -\ln(1 - u) 
        \end{array}
        \]

      \item Notons alors : $y = -\ln(1-u)$. Comme $1-u \in \ ]0,1]$,
        $\ln(1-u) \in \ ]-\infty, 0]$ et donc $y \geq 0$.\\
        On retombe alors sur l'équation de la question précédente dont
        la résolution est valable pour $y = 0$ (car on a toujours dans
        ce cas $\Delta > 0$).\\
        Cette équation admet pour solution :
        \[
        r_+ = \dfrac{-a + \sqrt{a^2 + 2by}}{b} = \dfrac{-a + \sqrt{a^2
            - 2b \ \ln(1-u)}}{b} \ \geq 0 \quad \mbox{ et } \quad r_- < 0
        \]

      \item Or, comme $G^{-1}_{a,b}$ est à valeurs dans $[0, +\infty[$,
        on en déduit :
        \[
        v = G^{-1}_{a,b}(1-u) \ \Leftrightarrow \ v = \dfrac{-a +
          \sqrt{a^2 - 2b \ \ln(1-u)}}{b}
        \]
      \end{noliste}
      \conc{Pour tout $u \in [0,1[$, $G^{-1}_{a,b}(1-u) = \dfrac{-a +
          \sqrt{a^2 - 2b \ \ln(1-u)}}{b}$.}~\\[-1cm]
    \end{proof}
  \end{noliste}


  \newpage


\item
  \begin{noliste}{a)}
    \setlength{\itemsep}{2mm}
  \item Justifier la convergence de l'intégrale $\dint{0}{+\infty}
    G_{a,b}(x)\dx$.

    \begin{proof}~
      \begin{noliste}{$\sbullet$}
      \item L'intégrale $\dint{0}{1} G_{a,b}(x) \dx$ est bien définie
        comme intégrale sur le segment $[0,1]$ de la fonction
        $G_{a,b}$ continue sur $[0,1]$.

      \item
        \begin{noliste}{$\stimes$}
        \item Or : $G_{a,b}(x) = \exp\left(-ax - \dfrac{b}{2} \
            x^2\right) = \oox{+\infty} \left( \dfrac{1}{x^2}
          \right)$.\\
          En effet :
          \[
          x^2 \ G_{a,b}(x) = x^2 \ \exp\left(-ax - \dfrac{b}{2} \
            x^2\right) = \dfrac{x^2}{(\ee^a)^x} \times
          \dfrac{1}{\ee^{\frac{b}{2} x^2}} \tendx{+\infty} 0 \times 0
          = 0
          \]
          par croissances comparées.

        \item $\forall x \in [1, +\infty[$, $G_{a,b}(x) \geq 0$ \quad
          et \quad $\dfrac{1}{x^2} \geq 0$.

        \item L'intégrale impropre $\dint{1}{+\infty} \dfrac{1}{x^2}
          \dx$ est convergente en tant qu'intégrale de Riemann
          impropre en $+\infty$, d'exposant $2$ ($2 > 1$).
        \end{noliste}%~\\
        Par critère d'équivalence des intégrales généralisées de
        fonctions continues positives, l'intégrale impropre
        $\dint{1}{+\infty} G_{a,b}(x) \dx$ est elle aussi convergente.
      \end{noliste}
      \conc{L'intégrale $\dint{0}{1} G_{a,b}(x) \dx$ est convergente.}~\\[-1cm]
    \end{proof}

  \item Soit $f$ la fonction définie sur $\R$ par : $f(x) =
    \sqrt{\dfrac{b}{2\pi}}\times \exp \left(-\dfrac{1}{2} b
      \left(x+\dfrac{a}{b}\right)^2 \right)$.\\
    Montrer que $f$ est une densité d'une variable aléatoire suivant
    une loi normale dont on précisera les paramètres (espérance et
    variance).

    \begin{proof}~
      \begin{noliste}{$\sbullet$}
      \item On rappelle qu'une \var $X$ suit la loi normale de
        paramètres $(m, \sigma^2)$ si :
        \begin{noliste}{a)}
    \setlength{\itemsep}{2mm}
        \item $X(\Omega) = \R$,
        \item $X$ admet pour densité la fonction $x \mapsto
          \dfrac{1}{\sigma \ \sqrt{2 \pi}} \ \ee^{-\frac{1}{2}
            (\frac{x-m}{\sigma})^2}$ (définie sur $\R$).
        \end{noliste}

      \item Soit $x \in \R$.
        \[
        \begin{array}{rcl}
          f(x) & = & \sqrt{\dfrac{b}{2\pi}} \times \exp \left(-\dfrac{1}{2} b
            \left(x+\dfrac{a}{b}\right)^2 \right) \\[.6cm]
          & = & \dfrac{\sqrt{b}}{\sqrt{2\pi}} \times \exp
          \left(-\dfrac{1}{2} b \left(\dfrac{bx + a}{b}\right)^2
          \right) \\[.6cm]  
          & = & \dfrac{1}{\frac{1}{\sqrt{b}} \ \sqrt{2\pi}} \times \exp
          \left(-\dfrac{1}{2} \left(\sqrt{b} \ \dfrac{\bcancel{b}(x +
                \frac{a}{b})}{\bcancel{b}}\right)^2 \right) \\[.6cm]
          & = & \dfrac{1}{\frac{1}{\sqrt{b}} \ \sqrt{2\pi}}
          \times \exp \left(-\dfrac{1}{2} \left(\dfrac{x +
                \frac{a}{b}}{\frac{1}{\sqrt{b}}}\right)^2 \right)  
        \end{array}        
        \]
        \conc{On en déduit que $f$ est la densité d'une \var $X$ qui
          suit la loi $\Norm{- \frac{a}{b} }{ \frac{1}{b} }$.\\
          Ainsi, $\E(X) = - \dfrac{a}{b}$ et $\V(X) =
          \dfrac{1}{b}$.}~\\[-1.2cm]
      \end{noliste}
    \end{proof}


    \newpage


  \item Soit $\Phi$ la fonction de répartition de la loi normale
    centrée réduite. \\
    Déduire de la question \itbf{2.b)}, l'égalité :
    \[
    \dint{0}{+\infty} G_{a,b}(x)\dx = \sqrt{\frac{2\pi}{b}}\times
    \exp\left(\frac{a^2}{2b}\right) \times
    \Phi\left(-\frac{a}{\sqrt{b}}\right)
    \]

    \begin{proof}~
      \begin{noliste}{$\sbullet$}
      \item Soit $x \in [0, +\infty[$. Remarquons tout d'abord :
        \[
        \begin{array}{rcl}
          -a x - \dfrac{b}{2} \ x^2 & = & - \dfrac{1}{2} \ b \ \left(
            2 \ \dfrac{a}{b} \ x + x^2 \right) \\[.4cm]
          & = & - \dfrac{1}{2} \ b \ \left( \left( x + \dfrac{a}{b}
            \right)^2 - \dfrac{a^2}{b^2} \right) % \\[.4cm]
          \ = \ - \dfrac{1}{2} \ b \ \left( x + \dfrac{a}{b}
          \right)^2 \ + \ \dfrac{1}{2} \dfrac{a^2}{b}
        \end{array}
        \]

      \item On en déduit : 
        \[
        \begin{array}{rcl}
          G_{a, b}(x) & = & \exp\left( -ax - \dfrac{b}{2} x^2 \right)
          \\[.4cm]
          & = & \exp\left( - \dfrac{1}{2} \ b \ \left( x + \dfrac{a}{b}
            \right)^2 \ + \ \dfrac{1}{2} \dfrac{a^2}{b} \right)
          \\[.4cm]
          & = & \exp\left( \dfrac{a^2}{2b} \right) \times \exp\left( -
            \dfrac{1}{2} \ b \ \left( x + \dfrac{a}{b} \right)^2 \right)
          \\[.4cm]
          & = & \sqrt{\dfrac{2 \pi}{b}} \times \exp\left(
            \dfrac{a^2}{2b} \right) \times \sqrt{\dfrac{b}{2 \pi}}
          \times \exp\left( - \dfrac{1}{2} \ b \ \left( x +
              \dfrac{a}{b} \right)^2 \right) 
          \\[.4cm]
          & = & \sqrt{\dfrac{2 \pi}{b}} \times \exp\left(
            \dfrac{a^2}{2b} \right) \times f(x)
        \end{array}
        \]

      \item Et ainsi : 
        \[
        \dint{0}{+\infty} G_{a,b}(x) \dx = \sqrt{\dfrac{2 \pi}{b}}
        \times \exp\left( \dfrac{a^2}{2b} \right) \times
        \dint{0}{+\infty} f(x) \dx
        \]

      \item On rappelle : $X \suit \Norm{-\frac{a}{b}}{\frac{1}{b}} \
        \Leftrightarrow \ X^* = \dfrac{X -
          (-\frac{a}{b})}{\frac{1}{\sqrt{b}}} \suit
        \Norm{0}{1}$.\\[.2cm]
        On effectue alors le changement de variable $u =
        \dfrac{x+\frac{a}{b}}{\frac{1}{\sqrt{b}}}$, autrement dit
        $\Boxed{u = \sqrt{b} \ \left(x + \dfrac{a}{b} \right)}$ :
        \[
        \left|
          \begin{array}{P{11cm}}
            $u = \sqrt{b} \ \left(x + \dfrac{a}{b} \right)$ \quad (et donc
            $x =
            \dfrac{1}{\sqrt{b}} \ u - \dfrac{a}{b}$) \nl    
            $\hookrightarrow$ $du = \sqrt{b} \, \dx$ \quad et \quad $dx
            = \dfrac{1}{\sqrt{b}} \ du$ \nl
            \vspace{-.4cm}
            \begin{noliste}{$\sbullet$}
            \item $x = 0 \ \Rightarrow \ u = \dfrac{a}{\sqrt{b}}$
            \item $x = +\infty \ \Rightarrow \ u = +\infty$ %
              \vspace{-.4cm}
            \end{noliste}
          \end{array}
        \right.
        \]
        Ce changement de variable est valide car la fonction $\varphi
        : u \mapsto \dfrac{1}{\sqrt{b}} u - \dfrac{a}{b}$ est
        $\Cont{1}$ sur $[\frac{a}{\sqrt{b}}, +\infty[$.


        \newpage


        \noindent
        Et donc : 
        \[
        \begin{array}{rcl}
          \dint{0}{+\infty} f(x) \dx & = & \sqrt{\dfrac{b}{2 \pi}} \ 
          \dint{0}{+\infty} \exp\left( -\dfrac{1}{2} \ b \ \left( x +
              \dfrac{a}{b} \right)^2 \right) \dx \\[.6cm]
          & = & \sqrt{\dfrac{b}{2 \pi}} \
          \dint{\frac{a}{\sqrt{b}}}{+\infty} \exp\left( - \dfrac{1}{2}
            \ u^2 \right) \ \dfrac{1}{\sqrt{b}} \ du \ = \
          \dint{\frac{a}{\sqrt{b}}}{+\infty} \dfrac{1}{\sqrt{2 \pi}} \
          \ee^{-\frac{u^2}{2}} \ du
          \\[.6cm]
          & = & \dint{\frac{a}{\sqrt{b}}}{+\infty} \varphi(u) \ du \ =
          \ 1 - \dint{+\infty}{\frac{a}{\sqrt{b}}} \varphi(u) \ du \ =
          \ 1 - \Phi\left( \dfrac{a}{\sqrt{b}} \right) \ = \
          \Phi\left( -\dfrac{a}{\sqrt{b}} \right)
        \end{array}      
        \]
      \end{noliste}
      \conc{$\dint{0}{+\infty} G_{a,b}(x)\dx =
        \sqrt{\frac{2\pi}{b}}\times \exp\left(\frac{a^2}{2b}\right)
        \times \Phi\left(-\frac{a}{\sqrt{b}}\right)$}
      \begin{remark}%~\\
        On utilise ici la propriété :
        %\[
        $\Boxed{X \suit \Norm{m}{\sigma^2} \ \Leftrightarrow \ X^* = \dfrac{X
          - m}{\sigma} \suit \Norm{0}{1}}$\\
        %\]
        dans le cas particulier où $m = -\dfrac{a}{b}$ et $\sigma^2 =
        \dfrac{1}{b}$ (propriété à connaître !).
      \end{remark}~\\[-1.4cm]
    \end{proof}
  \end{noliste}

\item Pour tout $a>0$ et pour tout $b>0$, on pose : $f_{a,b}(x) =
  \left\{
    \begin{array}{cR{1.4cm}}
      (a+bx) \ \exp\left(-ax-\dfrac{b}{2}x^2\right) & si $x\geq 0$ \nl
      0 & si $x<0$
    \end{array}
  \right.$.
  \begin{noliste}{a)}
    \setlength{\itemsep}{2mm}
  \item Justifier que la fonction $f_{a,b}$ est une densité de
    probabilité.\\
    {\it On dit qu'une variable aléatoire suit la loi exponentielle
      linéaire de paramètres $a$ et $b$, notée
      $\mathcal{E}_\ell(a,b)$, si elle admet $f_{a,b}$ pour densité.}

    \begin{proof}~\\
      On vérifie les trois propriétés des densités de probabilité.
      \begin{nonoliste}{(i)}
      \item La fonction $f_{a,b}$ est :
        \begin{noliste}{$\stimes$}
        \item continue sur $]-\infty, 0[$ car constante sur cet intervalle,
        \item continue sur $]0, +\infty[$ comme composée et produit de
          fonctions continues sur $]0,+\infty[$.
        \end{noliste}

      \item D'autre part, pour tout $x \in \R$, $f_{a, b}(x) \geq 0$
        car :
        \begin{noliste}{$\stimes$}
        \item \dashuline{si $x \geq 0$} : $f_{a, b}(x) =
          (a+bx)\exp\left(-ax-\dfrac{b}{2}x^2\right) > 0$ (car $a>0$ et 
	  $b>0$),
        \item \dashuline{si $x < 0$} : $f_{a, b}(x) = 0 \geq 0$.
        \end{noliste}
        
      \item $\dint{-\infty}{+\infty} f_{a, b}(x) \dx =
        \dint{0}{+\infty} f_{a, b}(x) \dx$ car $f_{a, b}$ est nulle en
        dehors de $[0, +\infty[$.\\
        La fonction $f_{a, b}$ est continue par morceaux sur $[0, +\infty[$.\\
        Soit $A \in [0, +\infty[$.
        \[
        \begin{array}{rcl}
          \dint{0}{A} f_{a, b}(x) \dx & = & \dint{0}{A} (a+bx) \
          \exp\left(-ax-\dfrac{b}{2}x^2\right) \dx \\[.6cm]
          & = & - \Prim{ \exp\left(-ax-\dfrac{b}{2}x^2\right) }{0}{A}
          \\[.6cm]
          & = & - \left( \exp\left(-a A - \dfrac{b}{2} A^2\right) -
            \exp(0) \right) \\[.6cm]
          & = & 1 - \ee^{-a A} \times \ee^{-\frac{b}{2} A^2} \
          \tendd{A}{+\infty} 1
        \end{array}
        \]


        \newpage


        \noindent
        Ainsi, l'intégrale impropre $\dint{-\infty}{+\infty} f_{a,
          b}(x) \dx$ est convergente et vaut $1$.
      \end{nonoliste}
      \conc{On en conclut que $f_{a, b}$ est une densité de
        probabilité.}~\\[-1cm] 
    \end{proof}

  \item Soit $X$ une variable aléatoire suivant la loi
    $\mathcal{E}_\ell(a,b)$. À l'aide d'une intégration par parties,
    justifier que $X$ admet une espérance $\E(X)$ telle que : $\E(X) =
    \dint{0}{+\infty} G_{a,b}(x)\dx$.

    \begin{proof}~
      \begin{noliste}{$\sbullet$}
      \item La \var $X$ admet une espérance si et seulement si
        l'intégrale $\dint{-\infty}{+\infty} x \ f_{a, b}(x) \dx$
        converge absolument, ce qui équivaut à démontrer la
        convergence pour des calculs de moment du type $\dint{-\infty}
        {+\infty} x^n f_{a,b}(x) \dx$.

      \item La fonction $f_{a, b}$ est nulle en dehors de $[0,
        +\infty[$. Donc :
        \[
        \dint{-\infty}{+\infty} x f_{a, b}(x) \dx = \dint{0}{+\infty}
        x f_{a, b}(x) \dx
        \]

      \item Soit $A \in [0, +\infty[$.\\
        La fonction $x \mapsto x \ f_{a, b}(x)$ est continue par
        morceaux sur $[0, A]$.\\
        On procède par intégration par parties (IPP).
        \[
        \renewcommand{\arraystretch}{2}
        \begin{array}{|rcl@{\qquad}rcl}
          u(x) & = & x & u'(x) & = & 1 \\
          v'(x) & = & f_{a, b}(x) & v(x) & = & - G_{a, b}(x)
        \end{array}
        \]
        Cette IPP est valide car les fonctions $u$ et $v$ sont 
	de classe $\Cont{1}$
        sur $[0, A]$. On obtient :
        \[
        \begin{array}{rcl}
          \dint{0}{A} x \ f_{a, b}(x) \dx & = & \Prim{-x \ G_{a,
              b}(x)}{0}{A} + \dint{0}{A} G_{a, b}(x) \dx \\[.6cm]
          & = & - (A \ G_{a, b}(A) - 0) + \dint{0}{A} G_{a, b}(x) \dx 
%           \\[.6cm]
%           & \tendd{A}{+\infty} & 0 \ + \ \dint{0}{+\infty} G_{a, b}(x) 
% 	    \dx
        \end{array}
        \]
        De plus, par croissances comparées :
        \[
        A \ G_{a,b}(A) = A \ \exp\left(-a A - \dfrac{b}{2} \
          A^2\right) = \dfrac{A}{(\ee^a)^A} \times
        \dfrac{1}{\ee^{\frac{b}{2} A^2}} \tendd{A}{+\infty} 0 \times 0
        = 0
        \]
        Et, comme l'intégrale $\dint{0}{+\infty} G_{a, b}(x) \dx$ est
        convergente d'après la question \itbf{2.a)} : 
        \[
        \dint{0}{A} x \ f_{a, b}(x) \dx \tendd{A}{+\infty} 0+
        \dint{0}{+\infty} G_{a, b}(x) \dx
        \]
      \end{noliste}
      \conc{On en déduit que $X$ admet une espérance et que $\E(X) =
        \dint{0}{+\infty} G_{a, b}(x) \dx$.}~\\[-1.2cm]
    \end{proof}
  \end{noliste}


  \newpage


\item Soit $Y$ une variable aléatoire suivant la loi exponentielle de
  paramètre $1$. On pose : $X = \dfrac{-a+\sqrt{a^2+2bY}}{b}$.
  \begin{noliste}{a)}
    \setlength{\itemsep}{2mm}
  \item Justifier que pour tout réel $x\in\R_+$, on a : $\Prob(\Ev{X\geq
      x}) = G_{a,b}(x)$.

    \begin{proof}~
      \begin{noliste}{$\sbullet$}
      \item D'après la question \itbf{1.c)} :
        \[
        \forall u \in [0,1[, \ G^{-1}_{a,b}(1-u) = \dfrac{-a +
          \sqrt{a^2 - 2b \ \ln(1-u)}}{b}
        \]

%       \item Considérons $u\in [0,1[$ et $y \geq 0$. et notons $u = 1 - 
% \ee^{-y}$. Comme
%         $y \geq 0$, $\ee^{-y} \in \ ]0, 1]$ et donc $u \in [0, 1[$.\\
%         On peut donc appliquer la formule précédente.\\
        On remarque :
        \[
         -\ln(1-u) = y \ \Leftrightarrow \ \ln(1-u) = -y \ 
	 \Leftrightarrow \ 1 - u = \ee^{-y} \ \Leftrightarrow \
	 u = 1 - \ee^{-y}
        \]
        Si $y \geq 0$, $\ee^{-y} \in \ ]0, 1]$ et donc $u \in [0, 1[$.\\
        On peut donc appliquer la formule précédente et
        on obtient :
        \[
        G^{-1}_{a,b}(\ee^{-y}) = \dfrac{-a + \sqrt{a^2 + 2b \ y}}{b}
        \]

      \item Comme $Y \suit \Exp{1}$, $Y(\Omega) = [0,
        +\infty[$. Ainsi, d'après ce qui précède : $X =
        G^{-1}_{a,b}(\ee^{-Y})$.\\[.2cm]
        On a alors, pour tout $x \in [0, +\infty[$ :
        \[
        \begin{array}{rcl@{\qquad}>{\it}R{5cm}}
          \Prob(\Ev{X \geq x}) & = &
          \Prob\left(\Ev{G^{-1}_{a,b}(\ee^{-Y}) \ \geq \ x} \right) \\[.2cm]
          & = & \Prob\left(\Ev{\ee^{-Y} \ \leq \ G_{a,b}(x)}\right) & (par
          stricte décroissance \\ de $G_{a, b}$ sur $[0, +\infty[$)
          \nl
          \nl[-.2cm]
          & = & \Prob\left(\Ev{-Y \ \leq \ \ln(G_{a,b}(x))}\right) & (par
          stricte croissance \\ de $\ln$ sur $]0, 1]$)
          \nl
          \nl[-.2cm]
          & = & \Prob\left(\Ev{Y \ \geq \ -\ln(G_{a,b}(x))}\right) 
          \\[.4cm]
          & = & 1 - \Prob\left(\Ev{Y \ < \ -\ln(G_{a,b}(x))}\right)
          \\[.2cm]          
          & = & 1 - F_{y}(-\ln(G_{a,b}(x))) & (car $Y$ est une \\ \var à
          densité) \nl
          \nl[-.2cm]
          & = & \bcancel{1} - \left(\bcancel{1} -
            \ee^{-(-\ln(G_{a,b}(x)))} \right) & (car $Y \suit
          \Exp{1}$ \\ et $-\ln(G_{a,b}(x)) \geq 0$) \nl 
          \nl[-.2cm]
          & = & \ee^{\ln(G_{a,b}(x))} \ = \ G_{a,b}(x)
        \end{array}
        \]
      \end{noliste}
      \conc{$\forall x \in [0, +\infty[$, $\Prob(\Ev{ X \geq x }) =
        G_{a, b}(x)$}


      \newpage


      \begin{remarkL}{0.96}%~
        \begin{noliste}{$\sbullet$}
        \item Il est aussi possible de traiter cette question même
          sans avoir traité la question \itbf{1.c)}. \\
          Précisons ci-dessous cette rédaction :
          \[
          \begin{array}{rcl@{\qquad}>{\it}R{5.9cm}}
            \Prob(\Ev{X \geq x}) & = & \Prob\left(\Ev{
                \dfrac{-a+\sqrt{a^2+2bY}}{b} \geq x } \right) \\[.6cm] 
            & = & \Prob\left(\Ev{ -a+\sqrt{a^2+2bY} \geq b x } \right)
            & (car $b > 0$) \nl
            \nl[-.2cm]
            & = & \Prob\left(\Ev{ \sqrt{a^2+2bY} \geq a + b x }
            \right) \\[.2cm]
            & = & \Prob\left(\Ev{ a^2 + 2bY \geq (a + b x)^2 }
            \right) & (par stricte croissance de la fonction élévation
            au carré sur $\R_+$) \nl
            \nl[-.2cm]
            & = & \Prob\left(\Ev{ 2bY \geq (a + b x)^2 - a^2 } \right)
            \\[.2cm]
            & = & \Prob\left(\Ev{ Y \geq \dfrac{1}{2b} \left((a + b
                  x)^2 - a^2 \right) } \right) & (car $2b > 0$)
          \end{array}         
          \]
          On remarque alors : 
          \[
          \begin{array}{rcl}
            \dfrac{1}{2b} \left((a + b x)^2 - a^2 \right) & = &
            \dfrac{1}{2b} \left((\bcancel{a^2} + 2 ab \ x + b^2 x^2) -
              \bcancel{a^2} \right) \\[.2cm]
            & = & a \ x + \dfrac{b}{2} \ x^2
          \end{array}          
          \]
          Et, en reprenant le calcul précédent :
          \[
          \begin{array}{rcl@{\qquad}>{\it}R{3.5cm}}
            \Prob(\Ev{X \geq x}) & = & \Prob\left(\Ev{ Y \geq
                a \ x + \dfrac{b}{2} \ x^2 } \right) \\[.6cm]
            & = & 1 - \Prob\left(\Ev{ Y < a \ x + \dfrac{b}{2} \
                x^2 } \right) \\[.4cm]
            & = & \bcancel{1} - \left(\bcancel{1} - \ee^{-(a \ x +
                \frac{b}{2} \ x^2)} \right) & (car $Y \suit \Exp{1}$
            et $a \ x + \dfrac{b}{2} \ x^2 \geq 0$) \nl
            \nl[-.2cm]
            & = & \exp\left(-a \ x - \dfrac{b}{2} \ x^2 \right) \ = \
            G_{a, b}(x)
          \end{array}         
          \]

        \item En réalité, il s'agit de deux présentations différentes
          d'une seule et même démonstration. Il s'agit simplement \og
          d'inverser \fg{} l'inégalité : $X \geq x$, 
	  c'est-à-dire d'isoler $Y$.
          \begin{noliste}{$\stimes$}
          \item dans la première rédaction, on sait que $X =
            G^{-1}_{a, b}(Y)$ et il suffit donc d'appliquer $G_{a, b}$
            de part et d'autre. On aboutit tout de suite à l'inégalité 
	    : $Y \geq a \ x + \dfrac{b}{2} \ x^2$.\\
            {\it (notée $Y \geq - \ln(G_{a,b}(x))$ dans la
              démonstration)}

          \item dans la deuxième rédaction, on prend moins de recul :
            on part de la définition de $X$ donnée par l'énoncé et,
            par opérations successives, on tombe encore
            une fois sur l'inégalité : $Y \geq a \ x + \dfrac{b}{2} \
            x^2$.
          \end{noliste}
        \end{noliste}
      \end{remarkL}~\\[-1.4cm]
    \end{proof}


    \newpage


  \item En déduire que $X$ suit la loi $\mathcal{E}_\ell(a,b)$.

    \begin{proof}~
      \begin{noliste}{$\sbullet$}
      \item Notons $g : x \mapsto \dfrac{-a + \sqrt{a^2 + 2b \
            x}}{b}$. Tout d'abord :
        \[
        \begin{array}{rcl@{\quad}>{\it}R{5cm}}
          X(\Omega) & = & \big( g(Y) \big)
          \hspace{.1cm} (\Omega)
          \\[.2cm]
          & = & g \hspace{.1cm} \big(Y(\Omega) \big)
          \\[.2cm]
          & = & g \hspace{.1cm} \big( [0, +\infty[ \big) 
          % \\[.2cm]
          % & \subset & ]0,1[ & (d'après les implications précédentes)
        \end{array}
        \]
        Or, comme $g$ est continue et strictement croissante sur $[0,
        +\infty[$ :
        \[
        g \hspace{.1cm} \big( [0, +\infty[ \big) = \ [g(0), \dlim{x
          \tend +\infty} g(x)[ \ = \ [0, +\infty[
        \]
        \conc{Ainsi : $X(\Omega) = [0, +\infty[$.}
        
      \item Déterminons alors la fonction de répartition de $X$.\\
        Soit $x \in \R$. Deux cas se présentent.
        \begin{noliste}{$\stimes$}
        \item \dashuline{Si $x < 0$} alors $\Ev{X \leq x} = \emptyset$
          car $X(\Omega) = [0, +\infty[$. Ainsi :
          \[
          F_X(x) = \Prob(\Ev{X \leq x}) = 0
          \]

        \item \dashuline{Si $x \geq 0$} alors :
          \[
          \begin{array}{rcl@{\qquad}>{\it}R{7.4cm}}
            F_X(x) & = & \Prob(\Ev{X \leq x}) \\[.2cm]
            & = & 1 - \Prob(\Ev{X > x}) \\[.2cm]
            & = & 1 - G_{a, b}(x) & (en reprenant la démonstration 
            précédente \\ en remplaçant $\Ev{X \geq x}$ par $\Ev{X > x}$)
          \end{array}
          \]          
        \end{noliste}
        En résumé :
        \[
          F_X \ : \ x  \mapsto 
          \left\{
          \begin{array}{cR{3cm}}
            0 & si $x < 0$ \nl
            1 - G_{a, b}(x) & si $x \geq 0$
          \end{array}
          \right.
        \]

      \item La fonction $F_X$ est :
      \end{noliste}
      \begin{liste}{1)}
      \item continue sur $\R$ puisque :
        \begin{noliste}{$\stimes$}
        \item $x \mapsto 1 - G_{a, b}(x)$ est continue sur $]0,
          +\infty[$ car $G_{a, b}$ l'est.
        \item $x \mapsto 0$ est continue sur $]-\infty, 0[$.
        \item $\dlim{x \tend 0} \left(1 - G_{a,b}(x)\right) = 1 - 1 = 0 
	= F_X(0) = \dlim{x \tend 0} 0$.
        \end{noliste}

      \item de classe $\Cont{1}$ sur $]-\infty, 0[$ et sur $]0, 
      +\infty[$ car :
        \begin{noliste}{$\stimes$}
        \item $x \mapsto 0$ est de classe $\Cont{1}$ sur $]-\infty, 0[$,
        \item $x \mapsto 1 - G_{a, b}(x)$ est de classe $\Cont{1}$ sur 
	$]0, +\infty[$ car $G_{a, b}$ l'est.
        \end{noliste}        
      \end{liste}
      \conc{On en déduit que $X$ est une \var à densité.}


      \newpage


      \begin{noliste}{$\sbullet$}
      \item On obtient une densité $f_X$ de $X$ en dérivant sur les
        intervalles ouverts.\\
        On {\bf pose} de plus : $f_X(0) = (a + b \times 0) \ \exp(0) =
        a$.
        \[
	  f_X \ : \ x \mapsto
          \left\{
          \begin{array}{cR{3cm}}
            0 & si $x < 0$ \nl
            (a + bx) \ \exp\left( -ax - \dfrac{b}{2} x^2 \right) & si 
	    $x \geq 0$
          \end{array}
          \right.
        \]
      \end{noliste}
      \conc{Ainsi, $f_X$ coïncide avec $f_{a, b}$. On en déduit : $X
        \suit \mathcal{E}_\ell(a,b)$.}
      \begin{remark}%~
        \begin{noliste}{$\sbullet$}
        \item L'énoncé demande de déterminer, en question \itbf{4.a)}
          : $\Prob(\Ev{X \geq x})$. Le caractère large de l'inégalité
          est étonnant puisque pour déterminer la fonction de
          répartition de $X$ on se sert de l'égalité :
          \[
          \Prob(\Ev{X \leq x}) = 1 - \Prob(\Ev{X > x})
          \]

        \item Ce choix est validé après coup puisqu'on démontre, {\bf
            après avoir déterminé $F_X$}, que $X$ est une \var à
          densité et donc : $\Prob(\Ev{X \geq x}) = \Prob(\Ev{X >
            x})$.

        \item La même remarque peut être faite en question
          \itbf{7.}. Par contre, la question \itbf{8.a)} ne détaillant
          pas précisément la méthode à suivre, on en profitera pour
          déterminer $\Prob(\Ev{U_n > x})$, ce qui est bien plus
          judicieux (d'autant plus que la \var $U_n$ étudiée n'est pas
          à densité !).
          
        \end{noliste}
      \end{remark}%~\\[-1.2cm]
    \end{proof}

  \item On note $U$ une variable aléatoire suivant la loi uniforme sur
    $[0,1[$.\\
    Déterminer la loi de la variable aléatoire $G_{a,b}^{-1}(1-U)$.

    \begin{proof}~
      \begin{noliste}{$\sbullet$}
      \item Si $U \suit \Ucfo{0}{1}$ et $\lambda > 0$, alors $-
        \dfrac{1}{\lambda} \ln(1-U) \suit \Exp{\lambda}$.\\[.2cm]
        On note alors : $Y = - \ln(1-U)$. On obtient ainsi $Y \suit
        \Exp{1}$.

      \item Or, comme vu en question \itbf{4.a)} et \itbf{4.b)} :
        \[
        G^{-1}_{a, b}(\ee^{-Y}) \suit \mathcal{E}_\ell(a, b)
        \]

      \item On remarque enfin :
        \[
        \ee^{-Y} = \ee^{-(- \ln(1-U))} = \ee^{\ln(1-U)} = 1 - U
        \]       
      \end{noliste}
      \conc{On en déduit que $G^{-1}_{a, b}(1-U) \suit
        \mathcal{E}_\ell(a, b)$.}
      
      
      \newpage
      
      
      \begin{remarkL}{.97}%~
        \begin{noliste}{$\sbullet$}
        \item Rappelons que si $U \suit \Ucfo{0}{1}$ alors $V =
          -\dfrac{1}{\lambda} \ln(1-U) \suit \Exp{\lambda}$.\\
          C'est un attendu du programme qu'on demande souvent de
          démontrer dans les énoncés. Rappelons ici la démonstration.
        \item Notons $g : x \mapsto - \frac{1}{\lambda} \
          \ln(1-x)$. Tout d'abord :
          \[
          V(\Omega) \ = \ \big( g(U) \big) \hspace{.1cm} (\Omega)
          \ = \ g \hspace{.1cm} \big(U(\Omega) \big) \ = \ g
          \hspace{.1cm} \big( [0, 1[ \big) \ = \ [0, +\infty[
          \]
          En effet, comme $g$ est continue et strictement croissante
          sur $[0, 1[$ :
          \[
          g \hspace{.1cm} \big( [0, 1[ \big) = \ [g(0), \dlim{x \tend 1}
          g(x)[ \ = \ [0, +\infty[
          \]
%           En effet :
%           \begin{noliste}{$\stimes$}
%           \item $g(0) = - \frac{1}{\lambda} \ \ln(1) = 0$.
%           \item $\dlim{x \tend 1} g(x) = \dlim{x \tend 1} -
%             \frac{1}{\lambda} \ \ln(1-x) = +\infty$ \ car \ $\dlim{x
%               \tend 1} \ln(1-x) = -\infty$.
%           \end{noliste}
%           \conc{$V(\Omega) \ = \ [0, +\infty[$}
          
        \item Déterminons la fonction de répartition de $V$.\\
          Soit $x\in\R$. Deux cas se présentent.
          \begin{noliste}{$\stimes$}
          \item \dashuline{Si $x<0$}, alors 
            $\Ev{V\leq x}=\varnothing$ car $V(\Omega)=[0,+\infty[$. Donc :
            \[
            F_V(x) = \Prob(\Ev{V\leq X}) = \Prob(\varnothing) 
            = 0
            \]~\\[-1.2cm]
            
          \item \dashuline{Si $x \geq 0$}.
            \[
            \begin{array}{rcl@{\quad}>{\it}R{5.5cm}}
              F_V(x) & = & \multicolumn{2}{l}{\Prob(\Ev{V\leq x}) 
                \ = \ \Prob\left(\Ev{ -\dfrac{1}{\lambda} \ln(1-U) 
                  \leq x}\right)}
              \\[.6cm]
              & = & \Prob(\Ev{\ln(1-U) \geq -\lambda x}) & (car $-\lambda<0$)
              \nl
              \nl[-.2cm]
              & = & \Prob\left(\Ev{1-U \geq \ee^{-\lambda x}}\right) & (car la 
              fonction $\exp$ est strictement croissante sur $\R$)
              \nl
              \nl[-.4cm]
              & = & \Prob\left(\Ev{ U \leq  1 - \ee^{-\lambda x}}\right)
              \\[.2cm]
              & = & F_U\left( 1-\ee^{-\lambda x} \right)
              \\[.2cm]
              & = & 1 - \ee^{-\lambda x} & (car $1 - \ee^{-\lambda x} \in [0,1[$)
            \end{array}
            \]
%             La dernière égalité est justifié par les équivalences suivantes :
%             \[
%             \begin{array}{rcl@{\quad}>{\it}R{5.5cm}}
%               0\leq x < +\infty & \Leftrightarrow & 0 \geq -\lambda x > 
%               -\infty & (car $-\lambda <0$)
%               \nl
%               \nl[-.2cm]
%               & \Leftrightarrow & 1=\ee^0 \geq \ee^{-\lambda x} > 0 & (car 
%               la fonction $\exp$ est strictement croissante sur $\R$)
%               \nl[-.4cm]
%               \nl
%               & \Leftrightarrow & 0 \leq 1-\ee^{-\lambda x} < 1
%             \end{array}
%             \]
          \end{noliste}

        \item Il est possible de faire une 
	  démonstration identique à celle de la question \itbf{4.a)}. En 
	  reprenant la $\eme{2}$
          rédaction et en posant : $Y = \ln(1-U)$, on obtient, pour
          tout $x \geq 0$ :
          \[
          \begin{array}{rcl@{\quad}>{\it}R{4.4cm}}
            \Prob(\Ev{X \geq x}) & = & \multicolumn{2}{l}{1 -
              \Prob\left(\Ev{ Y < a \ x + \dfrac{b}{2} \ x^2} \right) 
            \ = \ 1 - \Prob\left(\Ev{ -\ln(1-U) < a \ x + \dfrac{b}{2} \ x^2}
            \right)} %& (par définition de $Y$) \nl
            \\[.6cm]
            & = & 1 - \Prob\left(\Ev{ \ln(1-U) > - a \ x - \dfrac{b}{2} \ x^2}
            \right) \\[.6cm] 
            & = & 1 - \Prob\left(\Ev{ 1-U > \exp\left(- a \ x -
                  \dfrac{b}{2} \ x^2 \right)} \right) & (par stricte
            croissance \\ de la fonction $\exp$ sur $\R$) \nl
            \nl[-.2cm] 
            & = & 1 - \Prob\left(\Ev{ U < 1 - \exp\left(- a \ x -
                  \dfrac{b}{2} \ x^2 \right)} \right) \\[.6cm] 
            & = & \bcancel{1} - \left(\bcancel{1} - \exp\left(- a \ x -
                  \dfrac{b}{2} \ x^2 \right) \right) \ = \ G_{a, b}(x)
          \end{array}
          \]
          et on conclut en utilisant la question \itbf{4.b)}.
        \end{noliste}
%         \begin{liste}{$-$}
%         \item On note $Y = - \dfrac{1}{\lambda} \
%           \ln(1-U)$. Commençons par déterminer $Y(\Omega)$.\\[.2cm]
%           Comme $U(\Omega) = [0, 1[$, $(1-U)(\Omega) = \ ]0,1]$.\\[.2cm]
%           Et ainsi : $(\ln(1-U))(\Omega) = \ln\left( (1-U)(\Omega)
%           \right) = \ ]\dlim{x \tend 0} \ln(x), \ln(1)] = \ ]-\infty,
%           0]$.\\
%           On en déduit : $Y(\Omega) = [0, +\infty[$.

%         \item Soit $x \in \R$. On déterminer $F_Y$ en procédant par
%           disjonction de cas :
%           \begin{noliste}{$\stimes$}
%           \item \dashuline{si $x < 0$} \ alors $\Ev{Y \leq x} = \emptyset$.\\
%             Ainsi $F_Y(x) = \Prob(\Ev{Y \leq x}) = \Prob(\emptyset) = 0$.
            
%           \item \dashuline{si $x \geq 0$} \ alors :
%             \[
%             \begin{array}{rcl@{\qquad}>{\it}R{5cm}}
%               F_Y(x) & = & \Prob(\Ev{Y \leq x}) \\[.2cm]
%               & = & \Prob\left( \Ev{- \dfrac{1}{\lambda} \
%                   \ln\left(1 - U\right) \leq x} \right) \\[.6cm]
%               & = & \Prob\left( \Ev{ \ln\left(1 - U\right) \geq
%                   -\lambda \ x} \right) & (car $-\lambda < 0$) \nl
%               \nl[-.2cm]
%               & = & \Prob\left( \Ev{ 1-U \geq \ee^{-\lambda \ x}} \right) &
%               (par stricte croissance \\ de la fonction $\exp$) \nl
%               \nl[-.2cm]
%               & = & \Prob\left( \Ev{ U \leq 1 - \ee^{-\lambda \ x}}
%               \right) \\[.2cm] 
%               & = & F_U(1 - \ee^{-\lambda \ x}) \\[.2cm]
%               & = & 1 - \ee^{-\lambda \ x} & (car $1 - \ee^{-\lambda \
%                 x} \in [0,1]$)
%             \end{array}
%             \]            
%           \end{noliste}
%           On reconnaît la fonction de répartition d'une \var qui suit
%           une loi exponentielle de paramètre $\lambda$. Donc $Y \suit
%           \Exp{\lambda}$.
%         \end{liste}
%         \begin{noliste}{$\sbullet$}
%         \end{noliste}
      \end{remarkL}~\\[-1.4cm]
    \end{proof}
  \end{noliste}


  \newpage


\item La fonction \Scilab{} suivante génère des simulations de la loi
  exponentielle linéaire.
  \begin{scilab}
    & \tcFun{function} \tcVar{x} =
    grandlinexp(\tcVar{a},\tcVar{b},\tcVar{n}) \nl %
    & \qquad u = rand(\tcVar{n},1) \nl %
    & \qquad y = ............ \nl %
    & \qquad \tcVar{x} = (-\tcVar{a} + sqrt(\tcVar{a}\puis 2 + 2
    \Sfois{} \tcVar{b} \Sfois{} y)) / \tcVar{b} \nl %
    & \tcFun{endfunction} \nl %
  \end{scilab}

  \begin{noliste}{a)}
    \setlength{\itemsep}{2mm}
  \item Quelle est la signification de la ligne de code \ligne{2} ?

    \begin{proof}~\\[-1cm]%
      \concL{L'instruction {\tt rand(n,1)} renvoie un vecteur colonne
        de taille $\mathtt{n} \times 1$ contenant le résultat de la
        simulation de {\tt n} \var aléatoires indépendantes qui
        suivent toutes la même loi $\Uc{0}{1}$.}{15.6}~\\[-.8cm]
    \end{proof}

  \item Compléter la ligne de code \ligne{3} pour que la fonction {\tt
      grandlinexp} génère les simulations désirées.

    \begin{proof}~\\
      D'après la question \itbf{4.c)}, il suffit d'écrire :
      \begin{scilabC}{2}
        & \qquad y = - log(1 - u)
      \end{scilabC}~\\[-1cm]
    \end{proof}

  \end{noliste}

\item De quel nombre réel peut-on penser que les six valeurs générées
  par la boucle \Scilab{} suivante fourniront des valeurs approchées
  de plus en plus précises et pourquoi ?
  \begin{scilab}
    & \tcFor{for} k = 1:6 \nl %
    & \qquad mean(grandlinexp(0, 1, 10\puis{}k) \nl %
    & \tcFor{end} \nl %
  \end{scilab}

  \begin{proof}~
    \begin{noliste}{$\sbullet$}
    \item Soit $X$ une \var telle que $X \suit \mathcal{E}_\ell(0,
      1)$.\\
      Pour $m \in \N^*$, on considère $(X_1, \ldots, X_m)$ un
      $m$-échantillon de la \var $X$.\\
      Autrement dit, on considère $(X_i)_{i \in \N^*}$ une suite de \var
      indépendantes et toutes de même loi $\mathcal{E}_\ell(0, 1)$. On
      note alors :
      \[
      \overline{X_{m}} \ = \ \dfrac{X_1 + \ldots + X_{m}}{m}
      \]
      la \var donnant la moyenne empirique associée à \var $X$.

    \item Pour chaque {\tt k}, l'instruction {\tt mean(grandlinexp(0,
        1, 10\puis{}k))} permet de simuler la \var
      $\overline{X_{10^{\mathtt{k}}}}$.

    \item En vertu de la loi faible des grands nombres, la \var
      $\overline{X_{10^{\mathtt{k}}}}$ converge en probabilité vers la
      variable aléatoire constante égale à $\E(X)$.

    \item L'entier {\tt k} prenant des valeurs de plus en plus grandes 
      (on
      considère une simulation de $\overline{X_{10}}$, puis
      $\overline{X_{100}}$, \ldots, puis $\overline{X_{1000000}}$), on
      peut penser que le résultat sera de plus en plus proche de
      $\E(X)$.
    \end{noliste}
    \concL{Les six valeurs générées par la boucle \Scilab{} fourniront
      des valeurs de plus en plus précises de $\E(X)$.}{15}~\\[-1cm]
  \end{proof}
\end{noliste}


\newpage


\noindent%
{\it Dans la suite du problème, on note $(X_n)_{n\in\N^*}$ une suite
  de variables aléatoires indépendantes suivant chacune la loi
  exponentielle linéaire $\mathcal{E}_\ell(a,b)$ dont les paramètres
  $a>0$ et $b>0$ sont inconnus.\\
  Soit $h$ un entier supérieur ou égal à $2$. On suit pendant une
  période de $h$ années, une \og cohorte \fg{} de $n$ individus de
  même âge au début de l'étude et on modélise leurs durées de vie
  respectives à partir de cette date par les variables aléatoires
  $X_1$, $X_2$, $\hdots$, $X_n$.}


\subsection*{Partie II. Premier décès et intervalle de confiance de $a$}

\noindent
Pour tout $n\in\N^*$, on définit les variables aléatoires $M_n$, $H_n$
et $U_n$ par :
\[
M_n = \min(X_1,X_2,\hdots,X_n), \quad H_n=\min(h,X_1,X_2,\hdots,X_n)
\quad \mbox{et} \quad U_n=nH_n.
\]
\begin{noliste}{1.}
  \setcounter{enumi}{6}
\item Calculer pour tout $x \in \R_+$, la probabilité 
$\Prob(\Ev{M_n\geq x})$.\\
  Reconnaître la loi de la variable aléatoire $M_n$.

  \begin{proof}~\\
    Soit $n \in \N^*$.
    \begin{noliste}{$\sbullet$}
    \item Tout d'abord : $M_n(\Omega) \subset [0, +\infty[$. En effet :
    $\forall i \in \llb 1,n \rrb$, $X_i(\Omega) = [0,+\infty[$.

    \item Soit $x \in \R$. Deux cas se présentent :
      \begin{noliste}{$\stimes$}
      \item \dashuline{si $x < 0$} alors $\Ev{M_n \geq x} =
        \Omega$ car $M_n(\Omega) \subset [0,+\infty[$.\\[.1cm]
        Ainsi : $\Prob(\Ev{M_n \geq x}) = \Prob(\Omega) = 1$.

      \item \dashuline{si $x \geq 0$} alors :
        \[
        \begin{array}{rcl@{\qquad}>{\it}R{5cm}}
          \Prob(\Ev{M_n \geq x}) & = & \Prob(\Ev{\min(X_1, \ldots,
            X_n) \geq x}) \\[.2cm]
          & = & \Prob(\Ev{X_1 \geq x} \ \cap \ \ldots \ \cap \ \Ev{X_n
            \geq x} ) \\[.2cm]
          & = & \Prob(\Ev{X_1 \geq x}) \ \times \ \ldots \ \times \
          \Prob(\Ev{X_n \geq x} ) & (car les \var $X_i$ \\ sont
          indépendantes) \nl
          \nl[-.2cm]
          & = & G_{a, b}(x) \ \times \ \ldots \ \times \ G_{a, b}(x) &
          (d'après la question \itbf{4.a)}) \nl
          \nl[-.2cm]
          & = & (G_{a, b}(x))^n
        \end{array}
        \]        
      \end{noliste}
      Enfin, comme $\Prob(\Ev{M_n \leq x}) = 1 - \Prob(\Ev{M_n > x})$
      (on peut remplacer, sans modification du résultat, $\Ev{M_n \geq
        x}$ par $\Ev{M_n > x}$ dans la démonstration ci-dessus), on
      obtient :
      \[
        F_{M_n} \ : \ x \mapsto
        \left\{
          \begin{array}{cR{3cm}}
            0 & si $x < 0$ \nl
            1 - (G_{a, b}(x))^n & si $x \geq 0$
          \end{array}
        \right.
      \]

    \item La fonction $F_{M_n}$ (\cf \itbf{4.b)}) est :
      \begin{noliste}{1)}
      \item continue sur $\R$,
      \item de classe $\Cont{1}$ sur $]-\infty, 0[$ et sur $]0, 
      +\infty[$.
      \end{noliste}
      On en déduit que $M_n$ est une \var à densité.

    \item On obtient une densité $f_{M_n}$ de $M_n$ en dérivant 
    $F_{M_n}$ sur
      les intervalles ouverts.\\
      On choisit de plus : $f_{M_n}(0) = -n \ (G_{a, b}(0))^{n-1} \ 
      G'_{a, b}(0)$. 
      On obtient :\\
      \[
	f_{M_n} \ : \ x \mapsto
        \left\{
          \begin{array}{cR{3cm}}
            0 & si $x < 0$ \nl
            - n \ (G_{a, b}(x))^{n-1} \ G'_{a, b}(x) & si $x \geq 0$
          \end{array}
        \right.
      \]


      \newpage


      \noindent
      Or : 
      \[
      \begin{array}{rcl}
        - n \ (G_{a, b}(x))^{n-1} \ G'_{a, b}(x) & = & - n \ \left(\exp\left(-ax
            - \frac{b}{2} \ x^2\right)\right)^{n-1} \ (-a - bx) \
        \exp\left(-ax - \frac{b}{2} \ x^2\right)  \\[.2cm]
        & = & n \ (a + bx) \ \left(\exp\left(-ax - \frac{b}{2} \
            x^2 \right) \right)^{n} \\[.2cm]
        & = & ((na) + (nb) x) \ \exp\left(-(na) x - \frac{(nb)}{2} \
            x^2 \right) \ = \ f_{na, nb}(x)
      \end{array}
      \]
    \end{noliste}
    \conc{On en déduit que $M_n \suit \mathcal{E}_{\ell}(na, nb)$.}~\\[-1cm]
  \end{proof}

\item Pour tout $n\in\N^*$, on note $F_{U_n}$ la fonction de
  répartition de la variable aléatoire $U_n$.
  
  \begin{noliste}{a)}
    \setlength{\itemsep}{2mm}
  \item Montrer que pour tout $x\in\R$, on a : $F_{U_n}(x) = %
    \left\{
      \begin{array}{lR{2.3cm}}
	0 & si $x < 0$ \nl
	1 - \exp\left(- ax - \dfrac{b}{2n}x^2\right) & si $0\leq x < nh$ \nl
	1 & si $x \geq nh$
      \end{array}
    \right.$.

    \begin{proof}~\\
      Soit $n \in \N^*$.
      \begin{noliste}{$\sbullet$}
      \item Commençons par déterminer $U_n(\Omega)$.\\
        Remarquons tout d'abord :
        \[
        \begin{array}{rcl}
          H_n & = & \min(h, X_1, X_2, \ldots, X_n) \\[.2cm]
          & = & \min(h, \min(X_1, X_2, \ldots, X_n)) \\[.2cm]
          & = & \min(h, M_n) \\[.2cm]
        \end{array}
        \]
        On en déduit : $H_n(\Omega) \subset [0, +\infty[$.%
        \conc{$U_n(\Omega) \subset [0, +\infty[$}

      \item Soit $x \in \R$. Trois cas se présentent :
        \begin{noliste}{$\stimes$}
        \item \dashuline{si $x < 0$} alors $\Ev{U_n > x} =
          \Omega$ car $U_n(\Omega) \subset [0,+\infty[$.\\[.1cm]
          Ainsi, $\Prob(\Ev{U_n > x}) = \Prob(\Omega) = 1$.
          
        \item \dashuline{si $x \in [0, nh[$} alors :
          \[
          \begin{array}{rcl@{\qquad}>{\it}R{5.4cm}}
            \Prob(\Ev{U_n > x}) & = & \Prob(\Ev{n H_n > x}) \\[.2cm]
            & = & \Prob(\Ev{H_n > \frac{x}{n}}) & (puisque $n > 0$)
            \nl
            \nl[-.2cm]
            & = & \Prob(\Ev{\min(h, M_n) > \frac{x}{n}}) \\[.4cm]
            & = & \Prob(\Ev{h > \frac{x}{n}} \cap \Ev{M_n >
              \frac{x}{n}}) \\[.4cm] 
            & = & \Prob(\Ev{h > \frac{x}{n}}) \ \times \ 
            \Prob(\Ev{M_n > \frac{x}{n}}) & (car la \var constante
            $h$ et la \var $M_n$ sont indépendantes) \nl
            \nl[-.2cm]
            & = & 1 \ \times \ \Prob(\Ev{M_n > \frac{x}{n}}) & (car
            $\Ev{x < nh} = \Omega$ \\ puisque $x \in [0, nh[$) \nl
            \nl[-.2cm]
            & = & \left(G_{a, b}(\frac{x}{n}) \right)^n & (d'après la
            question \itbf{7}) 
          \end{array}
          \]        

        \item \dashuline{si $x \geq nh$} alors (en reprenant la
          démonstration ci-dessus) :
          \[
          \begin{array}{rcl@{\qquad}>{\it}R{5.4cm}}
            \Prob(\Ev{U_n > x}) & = & \Prob(\Ev{h > \frac{x}{n}}) \
            \times \ \Prob(\Ev{M_n > \frac{x}{n}}) \\[.2cm]
            & = & 0 \ \times \ \Prob(\Ev{M_n > \frac{x}{n}}) \ = \ 0 & (car
            $\Ev{x < nh} = \emptyset$ \\ puisque $x \geq nh$) \nl
          \end{array}
          \]        
        \end{noliste}
        
        
        \newpage


      \item Enfin, comme $\Prob(\Ev{U_n \leq x}) = 1 - \Prob(\Ev{U_n >
          x})$, on obtient : %
        \conc{$
          F_{U_n} \ : \ x \mapsto
          \left\{
            \begin{array}{cR{3cm}}
              0 & si $x < 0$ \nl
              \nl[-.2cm]
              1 - \left(G_{a, b}\left( \frac{x}{n} \right) \right)^n &
              si $x \in [0, nh[$ \nl 
              \nl[-.2cm]
              1 & si $x \geq nh$ 
            \end{array}
          \right.
        $}%
      Il suffit alors de remarquer : 
      \[
      \left(G_{a, b}\left(\dfrac{x}{n} \right)\right)^n = \exp\left(
        -a \dfrac{x}{n} - \dfrac{b}{2} \left(\dfrac{x}{n}\right)^2
      \right)^n = \exp\left( -a\bcancel{n} \dfrac{x}{\bcancel{n}} -
        \dfrac{b}{2} \bcancel{n} \ \dfrac{x^2}{n^{\bcancel{2}}}\right)
      = \exp\left( -a x - \dfrac{b}{2n} x^2 \right)
      \]
      \end{noliste}~\\[-1cm]
    \end{proof}

  \item Étudier la continuité de la fonction $F_{U_n}$.

    \begin{proof}~\\
      La fonction $F_{U_n}$ est :
      \begin{noliste}{$\stimes$}
      \item continue sur $]-\infty, 0[$ et sur $]nh, +\infty[$ car
        constante sur chacun de ces intervalles.

      \item continue sur $]0, nh[$ car $G_{a, b}$ l'est sur $[0,
        +\infty[$.

      \item continue en $0$ puisque : 
        \begin{noliste}{1)}
        \item $\dlim{x \tend 0^-} F_{U_n}(x) = \dlim{x \tend 0} 0 = 0$,
        \item $\dlim{x \tend 0^+} F_{U_n}(x) = \dlim{x \tend 0} 1 -
          \left(G_{a, b}\left( \frac{x}{n} \right) \right)^n = 1 -
          \left(G_{a, b}( 0 ) \right)^n = 1 - 1^n = 0$,
        \item $F_{U_n}(0) = 0$.
        \end{noliste}

      \item non continue en $nh$. En effet :
        \[
        \dlim{x \tend (nh)^-} F_{U_n}(x) = \dlim{x \tend nh} 1 -
        \left(G_{a, b}\left( \frac{x}{n} \right) \right)^n = 1 -
        \left(G_{a, b}\left( \frac{nh}{n} \right) \right)^n = 1 -
        \exp\left(-a nh - \dfrac{b}{2\bcancel{n}} n^{\bcancel{2}} \
          h^2 \right) < 1
        \]
        et $F_{U_n}(nh) = 1$ 
      \end{noliste}
      \conc{Ainsi, $F_{U_n}$ est continue uniquement sur $]-\infty,
        nh[$ et sur $]nh, +\infty[$.}~\\[-1.2cm]
    \end{proof}

  \item La variable aléatoire $U_n$ admet-elle une densité ?

    \begin{proof}~%
      \concL{D'après la question précédente, $F_{U_n}$ n'est pas
        continue sur $\R$. Ainsi, $U_n$ n'est pas une variable à
        densité.}{15}~\\[-1cm]
    \end{proof}

  \item Montrer que la suite de variables aléatoires
    $(U_n)_{n\in\N^*}$ converge en loi vers une variable aléatoire
    dont on précisera la loi.

    \begin{proof}~\\
      Deux cas se présentent.
      \begin{noliste}{$\sbullet$}
      \item \dashuline{Si $x < 0$} alors :
        \[
        \dlim{n \tend +\infty} F_{U_n}(x) = \dlim{n \tend +\infty} 0 = 0
        \]

      \item \dashuline{Si $x \geq 0$} alors, pour $n$ suffisamment
        grand (plus précisément pour tout $n > \left\lceil
          \dfrac{x}{h} \right\rceil$) :
        \[
        \dlim{n \tend +\infty} F_{U_n}(x) = \dlim{n \tend +\infty} 
        \left(1 -
        \exp\left( -ax - \dfrac{b}{2n} x^2 \right)\right)  = 1
        - \exp( -ax )
        \]
        
        
        \newpage


      \item On reconnaît la fonction de répartition d'une variable
        aléatoire $Z$ telle que $Z \suit \Exp{a}$.
      \end{noliste}
      \conc{Ainsi, $U_n \tendL Z$ où $Z \suit \Exp{a}$.}~\\[-1cm]
    \end{proof}
  \end{noliste}

\item Soit $\alpha \in \ ]0,1[$.
  \begin{noliste}{a)}
    \setlength{\itemsep}{2mm}
  \item Soit $Y$ une variable aléatoire qui suit la loi exponentielle
    de paramètre $1$.\\
    Trouver deux réels $c$ et $d$ strictement positifs tels que :
    \[
    \Prob(\Ev{c\leq Y\leq d}) = 1-\alpha \quad \mbox{et} \quad 
    \Prob(\Ev{Y\leq
      c}) = \dfrac{\alpha}{2}
    \]
    
    \begin{proof}~
      \begin{noliste}{$\sbullet$}
      \item Comme $Y \suit \Exp{1}$ et que les réels $c$ et $d$ doivent
      être strictement positifs :
        \[
        \begin{array}{rcl}
          \Prob(\Ev{c\leq Y\leq d}) & = & F_Y(d) - F_Y(c) \ = \ (\bcancel{1} -
          \ee^{-d}) - (\bcancel{1} - \ee^{-c}) \ = \ \ee^{-c} - \ee^{-d}
          \\[.6cm]
          \Prob(\Ev{Y \leq c}) & = & F_Y(c) \ = \ 1 - \ee^{-c}
        \end{array}
        \]
        
      \item On obtient ainsi :
        \[
        \begin{array}{rcl}
          \left\{
            \begin{array}{ccccc}
              \ee^{-c} & - & \ee^{-d} & = & 1 - \alpha \\[.4cm]
              -\ee^{-c} & & & = & -1 + \dfrac{\alpha}{2}
            \end{array}
          \right. %
          & \Leftrightarrow &
          \left\{
            \begin{array}{ccccc}
              & - & \ee^{-d} & = & - \dfrac{\alpha}{2} \\[.4cm]
              - \ee^{-c} & & & = & -1 + \dfrac{\alpha}{2}
            \end{array}
          \right. \\[1cm]
          & \Leftrightarrow &
          \left\{
            \begin{array}{l}
              \ee^{-d} = \dfrac{\alpha}{2} \\[.4cm]
              \ee^{-c} = 1 - \dfrac{\alpha}{2}
            \end{array}
          \right. \\[1cm]
          & \Leftrightarrow &
          \left\{
            \begin{array}{l}
              -d = \ln\left( \dfrac{\alpha}{2} \right) \\[.4cm]
              -c = \ln\left(1 - \dfrac{\alpha}{2} \right)
            \end{array}
          \right.
        \end{array}
        \]
        Ainsi, $c = -\ln\left(1 - \dfrac{\alpha}{2} \right) > 0$ car
        $1 - \dfrac{\alpha}{2} \in \ ]0,1[$ puisque $\alpha \in \
        ]0,1[$.\\[.2cm]
        De même, $d = - \ln\left( \dfrac{\alpha}{2} \right) > 0$ car
        $\dfrac{\alpha}{2} \in \ ]0,1[$.
      \end{noliste}
      \conc{$c = -\ln\left(1 - \dfrac{\alpha}{2} \right)$ \quad et
        \quad $d = - \ln\left( \dfrac{\alpha}{2} \right)$}
      \begin{remarkL}{0.99}%~\\
        On pouvait bien évidemment faire les calculs de probabilité à
        l'aide d'une densité de probabilité :
        \[
        \begin{array}{rcl}
          \Prob(\Ev{c\leq Y\leq d}) & = & \dint{c}{d} f_Y(x) \dx =
          \dint{c}{d} 1 \times \ee^{-x} \dx = \Prim{-\ee^{-x}}{c}{d} = -
          (\ee^{-d} - \ee^{-c}) = \ee^{-c} - \ee^{-d}
          \\[.6cm]
          \Prob(\Ev{Y \leq c}) & = & \dint{-\infty}{c} f_Y(x) \dx =
          \dint{0}{c} \ee^{-x} \dx = \Prim{-\ee^{-x}}{0}{c} = -(\ee^{-c}
          - \ee^{0}) = 1 - \ee^{-c}
        \end{array}
        \]
      \end{remarkL}~\\[-1.4cm]
    \end{proof}


    \newpage


  \item Montrer que $\left[ \dfrac{c}{U_n} \ , \ \dfrac{d}{U_n} \right]$
    est un intervalle de confiance asymptotique de $a$, de niveau de
    confiance $1-\alpha$.

    \begin{proof}~
      \[
      \begin{array}{rcl@{\qquad}>{\it}R{5cm}}
        \Prob\left( a \in \left[ \dfrac{c}{U_n}, \dfrac{d}{U_n} \right]
        \right) & = & \Prob\left(\Ev{ \dfrac{c}{U_n} \leq a \leq \dfrac{d}{U_n}}
        \right) \\[.6cm]
        & = & \Prob\left( \Ev{c \leq a \ U_n \leq d} \right) & (car $U_n >
        0$) \nl
        \nl[-.2cm]
        & = & \Prob\left( \Ev{\dfrac{c}{a} \leq U_n \leq \dfrac{d}{a}}
        \right) & (car $a > 0$) \nl
        \nl[-.2cm]
        & \tendn & \Prob\left( \Ev{\dfrac{c}{a} \leq Z \leq \dfrac{d}{a}}
        \right) & (d'après la question \itbf{8.d)}) \nl
        \nl[-.2cm]         
      \end{array}
      \]
      On constate enfin :
      \[
      \begin{array}{rcl@{\qquad}>{\it}R{5cm}}
        \Prob\left(\Ev{ \dfrac{c}{a} \leq Z \leq \dfrac{d}{a} } \right) & = &
        F_Z\left( \dfrac{d}{a} \right) - F_Z\left( \dfrac{c}{a}
        \right) \\[.4cm]
        & = & (\bcancel{1}- \ee^{-a \frac{d}{a}}) - (\bcancel{1}-
        \ee^{-a \frac{c}{a}}) & (car $Z \suit \Exp{a}$) \nl 
        \nl[-.2cm]
        & = & \ee^{-c} - \ee^{-d} \\[.2cm]
        & = & 1 - \alpha & (d'après la question précédente)
      \end{array}
      \]
      \concL{Ainsi, $\Prob\left( a \in \left[ \dfrac{c}{U_n},
            \dfrac{d}{U_n} \right] \right) \tendn 1 - \alpha$ ce qui
        démontre que $\left[ \dfrac{c}{U_n} \ , \ \dfrac{d}{U_n} 
	\right]$
        est un intervalle \\[.2cm] de confiance asymptotique de $a$, de niveau
        de confiance $1 - \alpha$.}{16}
      \begin{remark}%~
        \begin{noliste}{$\sbullet$}
        \item Avec une telle rédaction, il est difficile de comprendre
          pourquoi on a introduit la loi $\Exp{1}$ dans la question
          précédente. Pour bien comprendre ce point, on peut utiliser
          la propriété :
          \[
          Z \suit \Exp{a} \ \Leftrightarrow \ a Z \suit \Exp{1}
          \]
          Ainsi, on peut écrire dans la rédaction précédente :
          \[
          \Prob\left( \Ev{\dfrac{c}{a} \leq Z \leq \dfrac{d}{a}} \right) =
          \Prob\left( \Ev{c \leq a Z \leq d} \right) = \Prob(\Ev{c
            \leq Y \leq d}) = \ee^{-c} - \ee^{-d}
          \]
          où $Y \suit \Exp{1}$.
          
        \item Attention toutefois : si la transformée affine d'une
          \var est bien au programme, la transformée affine d'une \var
          suivant une loi exponentielle n'est pas explicitement
          mentionnée. Il faudrait donc démontrer la propriété
          précédente ! C'est assez simple :
          \begin{noliste}{$\stimes$}
          \item $(aZ) (\Omega) = [0, +\infty[$ car $Z(\Omega) = [0,
            +\infty[$ et $a > 0$.
          \item Ainsi, si $x < 0$, $\Prob(\Ev{aZ \leq x}) =
            \Prob(\emptyset) = 0$.\\
            Et si $x \geq 0$, $\Prob(\Ev{aZ \leq x}) = \Prob(\Ev{Z
              \leq \dfrac{x}{a}}) = F_Z\left( \dfrac{x}{a} \right) = 1
            - \ee^{- a \ \frac{x}{a}} = 1 - \ee^{-x}$.\\
            On reconnaît la fonction de répartition d'une \var qui
            suit la loi $\Exp{1}$.            
          \end{noliste}
        \end{noliste}
      \end{remark}~\\[-1.4cm]
    \end{proof}
  \end{noliste}
\end{noliste}
  

\newpage


\subsection*{Partie III. Nombre de survivants et estimateur convergent de $b$}

\noindent
Pour tout $i \in \N^*$, soit $S_i$ et $D_i$ les variables aléatoires
telles que :
\[
S_i = \left\{
  \begin{array}{lR{1.6cm}}
    1 & si $X_i\geq h$ \nl
    0 & sinon
  \end{array}
\right. %
\qquad \mbox{ et } \qquad %
D_i = \left\{
  \begin{array}{lR{1.6cm}}
    1 & si $X_i\leq 1$ \nl
    0 & sinon
  \end{array}
\right.
\]
Pour tout $n\in\N^*$, on pose : $\overline{S}_n = \dfrac{1}{n} \
\Sum{i=1}{n} S_i$ et $\overline{D}_n = \dfrac{1}{n} \ \Sum{i=1}{n}
D_i$.
\begin{noliste}{1.}
  \setcounter{enumi}{9}
\item 
  \begin{noliste}{a)}
    \setlength{\itemsep}{2mm}
  \item Justifier que pour tout $i\in\llb 1,n\rrb$, on a $\E(S_i) =
    G_{a,b}(h)$ et calculer $\E(S_iD_i)$.

    \begin{proof}~\\
      Soit $i \in \llb 1, n \rrb$.
      \begin{noliste}{$\sbullet$}
      \item Comme $S_i(\Omega) = \{0, 1\}$, la \var $S_i$ suit une loi
      de Bernoulli de paramètre : 
      \[
	p=\Prob(\Ev{X_i \geq h}) = G_{a,b}(h)
      \]
      \conc{Ainsi, la \var $S_i$ admet une espérance et : $\E(S_i) = 
      G_{a, b}(h)$.}
% 
%       \item Par définition, on a alors :
%         \[
%         \begin{array}{rcl@{\qquad}>{\it}R{5cm}}
%           \E(S_i) & = & 1 \times \Prob(\Ev{X_i \geq h}) + 0 \times
%           \Prob(\Ev{X_i < h}) \\[.2cm]
%           & = & \Prob(\Ev{X_i \geq h}) \\[.2cm]
%           & = & G_{a, b}(h) & (d'après la question \itbf{4.a)})
%         \end{array}
%         \]
%         \conc{$\E(S_i) = G_{a, b}(h)$}~

      \item De même, $(S_i D_i)(\Omega) = \{0, 1\}$. Plus précisément :
        \[
        S_i D_i = \left\{
          \begin{array}{lR{3.6cm}}
            1 & si $X_i\geq h$ et $X_i \leq 1$ \nl
            0 & sinon
          \end{array}
        \right.
        \]
        Donc la \var $S_i \, D_i$ suit une loi de Bernoulli de 
        paramètre :
        \[
	  p= \Prob(\Ev{X_i \geq h} \cap \Ev{X_i \leq 1})
	  = \Prob(\Ev{h \leq X_i \leq 1}) = 0
	\]
        En effet, $\Ev{h \leq X_i \leq 1} = \emptyset$ puisque $h \geq
        2$.%
        \conc{On en déduit : $\E(S_i D_i) = 0$.}~\\[-1.4cm]
      \end{noliste}
    \end{proof}

  \item Pour quels couples $(i,j) \in \llb 1,n\rrb^2$, les variables
    aléatoires $S_i$ et $D_j$ sont-elles indépendantes ?

    \begin{proof}~\\
      Soit $i \in \llb 1,n\rrb$. On raisonne comme dans la question
      précédente.
      \begin{noliste}{$\sbullet$}
      \item Comme $D_i(\Omega) = \{0, 1\}$, , la \var $S_i$ suit une loi
      de Bernoulli de paramètre : 
      \[
	p= \Prob(\Ev{X_i \leq 1}) = 1-G_{a,b}(1)
      \]
      \conc{Ainsi la \var $D_i$ admet une espérance et : $\E(D_i) =1- 
      G_{a, b}(1)$.}
%         \[
%         \begin{array}{rcl@{\qquad}>{\it}R{5cm}}
%           \E(D_i) & = & 1 \times \Prob(\Ev{X_i \leq 1}) + 0 \times
%           \Prob(\Ev{X_i > 1}) \\[.2cm]
%           & = & \Prob(\Ev{X_i \leq 1}) \\[.2cm]
%           & = & 1 - G_{a, b}(1) & (d'après la question \itbf{4.b)})
%         \end{array}
%         \]
%         \conc{$\E(D_i) = 1 - G_{a, b}(1)$} %
        De plus, $\E(D_i) \neq 0$ puisque $G_{a, b}(1) \in \ ]0,1[$
        (d'après la question \itbf{1.a)}).

      \item On en déduit que $D_i$ et $S_i$ ne sont pas indépendantes
        puisque, d'après ce qui précède :
        \[
        \E(S_i D_i) \ = \ 0 \ \neq \ \E(S_i) \ \E(D_i)
        \]

      \item Considérons maintenant $j \in \llb 1, n\rrb$ tel que $j
        \neq i$. Comme $X_i$ et $X_j$ sont indépendantes :
        \[
        \begin{array}{rcl}
          \Prob(\Ev{S_i = 0} \cap \Ev{D_j = 0}) & = & \Prob(\Ev{X_i < h}
          \cap \Ev{X_j > 1}) \ = \ \Prob(\Ev{X_i < h}) \times 
	  \Prob(\Ev{X_j > 1}) 
	  \\[.2cm]
	  &=& \Prob(\Ev{S_i=0}) \times \Prob(\Ev{D_j=0})
        \end{array}
        \]
        \[
          \begin{array}{rcl}
            \Prob(\Ev{S_i = 0} \cap \Ev{D_j = 1}) & = & \Prob(\Ev{X_i < 
	    h} \cap \Ev{X_j \leq 1}) \ = \ \Prob(\Ev{X_i < h}) \times
           \Prob(\Ev{X_j \leq 1}) 
           \\[.2cm]
           &=& \Prob(\Ev{S_i=0}) \times \Prob(\Ev{D_j=1})
          \end{array}
        \]
        
        
        \newpage
        
        
        \[
          \begin{array}{rcl}
            \Prob(\Ev{S_i = 1} \cap \Ev{D_j = 0}) & = & \Prob(\Ev{X_i 
	    \geq h} \cap \Ev{X_j > 1}) \ = \ \Prob(\Ev{X_i \geq h}) 
	    \times \Prob(\Ev{X_j > 1}) 
	    \\[.2cm]
	    &=& \Prob(\Ev{S_i=1}) \times \Prob(\Ev{D_j=0})
          \end{array}
        \]
        
        \[
          \begin{array}{rcl}
            \Prob(\Ev{S_i = 1} \cap \Ev{D_j = 1}) & = & \Prob(\Ev{X_i 
	    \geq h} \cap \Ev{X_j \leq 1}) \ = \ \Prob(\Ev{X_i \geq h}) 
	    \times \Prob(\Ev{X_j \leq 1})
	    \\[.2cm]
	    &=& \Prob(\Ev{S_i=1}) \times \Prob(\Ev{D_j=1})
          \end{array}
        \]
        On en déduit que $S_i$ et $D_j$ sont indépendantes.
      \end{noliste}
      \conc{$S_i$ et $D_j$ sont indépendantes si et seulement si $i
        \neq j$.}%~\\[-1.2cm]
        
      \begin{remark}%~
        \begin{noliste}{$\sbullet$}
        \item Rappelons :
          $
          \mbox{$X$ et $Y$ indépendantes} \ \Rightarrow \ \E(XY) =
          \E(X) \ \E(Y)
          $\\
          On se sert dans la démonstration de la contraposée de cet
          énoncé à savoir : 
          \[
          \E(XY) \neq \E(X) \ \E(Y) \Rightarrow \mbox{$X$ et $Y$ ne
            sont pas indépendantes}
          \]
        \item Ce résultat {\tt N'EST PAS} une équivalence. Autrement
          dit :
          \[
          \E(XY) = \E(X) \ \E(Y) \quad \xcancel{\rule[-.15cm]{0cm}{.5cm}
            \Rightarrow \rule[-.15cm]{0cm}{.5cm}} \quad \mbox{$X$ et
            $Y$ sont indépendantes}
          \]
        \end{noliste}
      \end{remark}~\\[-1.4cm]
    \end{proof}

  \item Déduire des questions précédentes l'expression de la
    covariance $\cov(\overline{S}_n, \overline{D}_n)$ de
    $\overline{S}_n$ et $\overline{D}_n$ en fonction de $n$,
    $G_{a,b}(h)$ et $G_{a,b}(1)$. Le signe de cette covariance
    était-il prévisible ?

    \begin{proof}~
      \begin{noliste}{$\sbullet$}
      \item D'après ce qui précède, pour tout $(i, j) \in \llb 1, n
        \rrb^2$ tel que $i \neq j$ :
        \[
        \begin{array}{rcl@{\qquad}>{\it}R{5cm}}
          \Cov(S_i, D_j) & = & \E(S_i D_j) - \E(S_i) \ \E(D_j) \ = \ 0
          & (car $S_i$ et $D_j$ sont indépendantes) \nl
          \nl[-.2cm]
          \Cov(S_i, D_i) & = & \E(S_i D_i) - \E(S_i) \ \E(D_i)
          \\[.2cm]
          & = & 0 - G_{a,b}(h) \ (1 - G_{a, b}(1)) \\[.2cm]
          & = & - G_{a,b}(h) \ (1 - G_{a, b}(1))
        \end{array}
        \]

      \item On a alors :
        \[
        \begin{array}{rcl@{\qquad}>{\it}R{5cm}}
          \Cov\left( \overline{S_n}, \overline{D_n} \right) & = & 
          \Cov\left( \dfrac{1}{n} \ \Sum{i=1}{n} S_i, \overline{D_n}
          \right) & (par définition de $\overline{S_n}$) \nl
          \nl[-.2cm]
          & = & \dfrac{1}{n} \ \Cov\left( \Sum{i=1}{n} S_i, \overline{D_n}
          \right) & (par linéarité à gauche) \nl
          \nl[-.2cm]
          & = & \dfrac{1}{n} \ \Sum{i=1}{n} \Cov\left( S_i, \overline{D_n}
          \right) & (par linéarité à gauche) \nl
          \nl[-.2cm]
          & = & \dfrac{1}{n} \ \Sum{i=1}{n} \Cov\left( S_i, \dfrac{1}{n} \
            \Sum{j=1}{n} D_j \right) & (par définition de 
	    $\overline{D_n}$) \nl
          \nl[-.2cm]
           & = & \dfrac{1}{n^2} \ \Sum{i=1}{n} \Sum{j=1}{n} \Cov(S_i,
           D_j) & (par linéarité à droite) \nl
          \nl[-.2cm]
        \end{array}
        \]
        

        \newpage


        \noindent
        Ainsi :
        \[
        \begin{array}{rcl@{\qquad}>{\it}R{5cm}}
          \Cov\left( \overline{S_n}, \overline{D_n} \right) 
          & = & \dfrac{1}{n^2} \ \Sum{1 \leq i, j \leq n}{} \Cov(S_i,
          D_j) \\[.6cm] 
          & = & \dfrac{1}{n^2} \ \Sum{ %
              \scriptsize
              \begin{array}{c}
                1 \leq i, j \leq n \\
                i = j  
              \end{array}
            }{} \Cov(S_i, D_j) %
            + 
            \bcancel{%
              \dfrac{1}{n^2} \ 
              \Sum{ %
                \scriptsize
                \begin{array}{c}
                  1 \leq i, j \leq n \\
                  i \neq j  
                \end{array}
              }{} \Cov(S_i, D_j) %
            }
            \\[.8cm]
            & = & \dfrac{1}{n^2} \ \Sum{i=1}{n} \Cov(S_i, D_i) \\[.6cm]
            & = & \dfrac{1}{n^2} \ \Sum{i=1}{n} (-G_{a,b}(h) \ (1 -
            G_{a,b}(1))) \\[.6cm]
            & = & \dfrac{1}{n^2} \ n \ (-G_{a,b}(h) \ (1 -
            G_{a,b}(1))) \ = \ - \dfrac{G_{a,b}(h) \ (1 -
              G_{a,b}(1))}{n}
          \end{array}
          \]
          \conc{$\Cov\left( \overline{S_n}, \overline{D_n} \right) = -
            \dfrac{G_{a,b}(h) \ (1 - G_{a,b}(1))}{n}$}~

        \item Comme $G_{a, b}(h) > 0$ et $1 - G_{a, b}(1) > 0$,
          $\Cov(\overline{S_n}, \overline{D_n}) < 0$.\\
          Revenons à la définition de $S_i$ et $D_i$ pour comprendre
          ce signe.
          \begin{noliste}{$\stimes$}
          \item $S_i = 1$ ($0$ sinon) si le $\eme{i}$ individu de la
            cohorte est encore en vie après $h$ années,
          \item $D_i = 1$ ($0$ sinon) si le $\eme{i}$ individu de la
            cohorte est mort au cours de la première année.
          \end{noliste}
          Ainsi, $\overline{S_n}$ représente la proportion d'individus
          encore en vie après $h$ années et $\overline{D_n}$
          représente la proportion d'individus morts au cours de la
          première année. %
          \concL{Lorsqu'une de ces deux proportions augmente, l'autre
            à tendance à diminuer.\\
            Le signe négatif de la quantité était donc
            prévisible.}{15}~\\[-1.2cm]
      \end{noliste}
    \end{proof}
  \end{noliste}

\item
  \begin{noliste}{a)}
    \setlength{\itemsep}{2mm}
  \item Montrer que $\overline{S}_n$ est un estimateur sans biais et
    convergent du paramètre $G_{a,b}(h)$.

    \begin{proof}~
      \begin{noliste}{$\sbullet$}
      \item La \var $\overline{S_n}$ admet une espérance en tant que
        combinaison linéaire des $\var$ $S_1$, \ldots, $S_n$ qui
        admettent toutes une espérance. De plus : 
        \[
        \begin{array}{rcl@{\qquad}>{\it}R{5cm}}
          \E(\overline{S_n}) & = & \E\left(\dfrac{1}{n} \ \Sum{i=1}{n}
            S_i \right) \\[.6cm]
          & = & \dfrac{1}{n} \ \Sum{i=1}{n} \ \E\left( S_i \right) &
          (par linéarité de l'espérance) \nl
          \nl[-.2cm]
          & = & \dfrac{1}{n} \ \Sum{i=1}{n} \ G_{a, b}(h) & (d'après
          la question \itbf{10.a)}) \nl
          \nl[-.2cm]
          & = & \dfrac{1}{n} \ n \ G_{a, b}(h) \ = \ G_{a, b}(h)
        \end{array}
        \]
        Donc : $b(\overline{S_n}) = \E(\overline{S_n}) - G_{a,b}(h)
        =0$.
        \conc{Ainsi, $\overline{S_n}$ est un estimateur sans biais de
          $G_{a, b}(h)$.}


        \newpage


      \item La \var $\overline{S_n}$ admet un moment d'ordre $2$ en
        tant que combinaison linéaire des $\var$ $S_1$, \ldots, $S_n$
        qui sont {\bf indépendantes} (car les \var $X_1$, \ldots,
        $X_n$ le sont) et admettent toutes un moment d'ordre $2$
        puisqu'elles sont finies.\\
        Ainsi, $\overline{S_n}$ admet un risque quadratique et d'après
        la décomposition biais-variance :
        \[
        \begin{array}{rcl@{\qquad}>{\it}R{6.6cm}}
          r(\overline{S_n}) & = & \V(\overline{S_n}) +
          \left(\bcancel{b(\overline{S_n}}) \right)^2 \\[.2cm]
          & = & \V\left( \dfrac{1}{n} \ \Sum{i=1}{n} S_i \right)
          \\[.6cm] 
          & = & \dfrac{1}{n^2} \ \V\left( \Sum{i=1}{n} S_i \right) &
          (par propriété de la variance) \nl
          \nl[-.2cm]
          & = & \dfrac{1}{n^2} \ \Sum{i=1}{n} \V(S_i) &
          (par indépendance \\ des \var $S_1$, \ldots, $S_n$) \nl
          \nl[-.2cm]
          & = & \dfrac{1}{n^2} \ n \ \V(S_1) & (les \var $X_i$ étant
          toutes de même loi, \\ il en est de même des \var $S_i$) \nl
          \nl[-.2cm]
          & = & \dfrac{1}{n} \ \V(S_1) \tendn 0
          & (car $\V(S_1)$ est une constante)
        \end{array}
        \]
        \conc{$\overline{S_n}$ est un estimateur convergent de 
	$G_{a,b}(h)$.}~\\[-1.6cm]
      \end{noliste}
    \end{proof}

  \item De quel paramètre, $\overline{D}_n$ est-il un estimateur sans
    biais et convergent ?
    
    \begin{proof}~
    \begin{noliste}{$\sbullet$}
    \item Montrons que $\overline{D_n}$ est un estimateur sans biais
      de $1-G_{a,b}(1)$.\\
      La \var $\overline{D_n}$ admet une espérance en tant que
      combinaison linéaire des $\var$ $D_1$, \ldots, $D_n$ qui
      admettent toutes une espérance. De plus :
        \[
        \begin{array}{rcl@{\qquad}>{\it}R{5cm}}
          \E(\overline{D_n}) & = & \E\left(\dfrac{1}{n} \ \Sum{i=1}{n}
            D_i \right) \\[.6cm]
          & = & \dfrac{1}{n} \ \Sum{i=1}{n} \ \E\left( D_i \right) &
          (par linéarité de l'espérance) \nl
          \nl[-.2cm]
          & = & \dfrac{1}{n} \ \Sum{i=1}{n} \ (1-G_{a, b}(1)) & (d'après
          la question \itbf{10.b)}) \nl
          \nl[-.2cm]
          & = & \dfrac{1}{n} \ n \ (1-G_{a, b}(1)) \ = \ 1-G_{a, b}(1)
        \end{array}
        \]
        Donc : $b(\overline{D_n}) = \E(\overline{D_n}) - \Big(
        1- G_{a,b}(1)\Big)=0$.
    \conc{$\overline{D_n}$ est un estimateur sans biais de 
$1-G_{a,b}(1)$.}



\newpage



  \item Montrons que $\overline{D_n}$ est un estimateur convergent de
    $1-G_{a,b}(1)$.\\
    La \var $\overline{D_n}$ admet un moment d'ordre $2$ comme
    combinaison linéaire des \var $D_1$, \ldots, $D_n$ qui sont {\bf
      indépendantes} (car les \var $X_1$, \ldots, $X_n$ le sont) et
    admettent toutes un moment d'ordre $2$
    puisqu'elles sont finies.\\
    Ainsi, $\overline{D_n}$ admet un risque quadratique et d'après la
    décomposition biais-variance :
        \[
        \begin{array}{rcl@{\qquad}>{\it}R{6.6cm}}
          r(\overline{D_n}) & = & \V(\overline{D_n}) +
          \left(\bcancel{b(\overline{D_n}}) \right)^2 
          \ = \ \V\left( \dfrac{1}{n} \ \Sum{i=1}{n} D_i \right)
          \\[.6cm] 
          & = & \dfrac{1}{n^2} \ \V\left( \Sum{i=1}{n} D_i \right) &
          (par propriété de la variance) \nl
          \nl[-.2cm]
          & = & \dfrac{1}{n^2} \ \Sum{i=1}{n} \V(D_i) &
          (par indépendance \\ des \var $D_1$, \ldots, $D_n$) \nl
          \nl[-.2cm]
          & = & \dfrac{1}{n^2} \ n \ \V(D_1) & (les \var $X_i$ étant
          toutes de même loi, \\ il en est de même des \var $D_i$) \nl
          \nl[-.2cm]
          & = & \dfrac{1}{n} \ \V(D_1) \tendn 0
          & (car $\V(D_1)$ est une constante)
        \end{array}
        \]
        \conc{$\overline{D_n}$ est un estimateur convergent de $1-G_{a,b}(1)$.}
    \end{noliste}
    
    \begin{remarkL}{.97}%~\\
    Le caractère \og convergent \fg{} est 
    une qualité recherchée pour un estimateur. Il permet 
    notamment de classer entre eux les estimateurs d'un même 
    paramètre.\\
    Considérons par exemple un $n$-échantillon 
    $(X_1,\hdots,X_n)$ de loi $\Bern{\theta}$, où $\theta\in]0,1[$ est 
    inconnu, et les estimateurs $T_n=X_n$ et 
    $\overline{X_n}=\dfrac{1}{n}\Sum{i=1}{n} X_i$.\\[.1cm]
    On cherche en fait à savoir si on obtient une estimation plus
    précise de $\theta$ en augmentant la taille de notre
    échantillon. C'est ce qu'indique le caractère convergent.
    \begin{noliste}{$\sbullet$}
    \item Pour $T_n$, on sait :
      $T_n(\Omega)=X_n(\Omega)=\{0,1\}$. Donc $\vert
      T_n-\theta\vert(\Omega)=\{\theta,1-\theta\}$. Donc
      \[
      \Prob(\vert T_n-\theta\vert \geq \min(\theta,1-\theta))=1 
      \]
      Donc : $\dlim{n\to + \infty} \Prob(\Ev{\vert T_n - \theta \vert 
      \geq \min(\theta, 1-\theta)}) \neq 0$.\\
      Donc $T_n$ n'est pas un estimateur convergent de $\theta$.
    \item Pour $\overline{X_n}$, comme cet estimateur est sans biais 
    (se démontre grâce à la linéarité de l'espérance), on obtient :
      \[
      \begin{array}{rcl@{\qquad}>{\it}R{5cm}}
        r_\theta(\overline{X_n}) = \V(\overline{X_n}) & = &
        \dfrac{1}{n^2} \ \Sum{i=1}{n} \V(X_i) 
        & (par indépendance de $X_1,\hdots,X_n$) 
        \nl
        \nl[-.2cm]
        & = &\dfrac{1}{n^2} \ \Sum{i=1}{n} (\theta(1-\theta)) \\[.4cm]
        & = &\dfrac{\theta(1-\theta)}{n} \tendn 0
      \end{array}
    \]
    Donc $\overline{X_n}$ est un estimateur convergent de $\theta$.
  \end{noliste}
  L'estimation de $\theta$ donnée par $\overline{X_n}$ va donc être de
  plus en plus précise, contrairement à celle de $T_n$, qui n'est pas un
  estimateur convergent.\\
  Dans cet exemple, le meilleur estimateur est donc $\overline{X_n}$.
    \end{remarkL}~\\[-1.4cm]
    \end{proof}
  \end{noliste}
  

  \newpage
  
  
  
  \begin{remark}%~\\
    On rappelle qu'il existe deux manières de montrer qu'un estimateur 
    $T_n$ est convergent pour un paramètre $\theta$.
    \begin{noliste}{1)}
    \item \underline{La définition} : $T_n$ est un estimateur convergent 
    de $\theta$ si et seulement si
    \[
    \forall \eps>0, \ \dlim{n\to+\infty} \Prob(\vert 
    T_n-\theta\vert \geq \eps)=0.
    \]
    \item \underline{L'utilisation du risque quadratique} : 
    \[
    \dlim{n\to+\infty} r_\theta(T_n)=0 \quad \Rightarrow \quad
    \mbox{$T_n$ est un estimateur convergent de $\theta$.}
    \]
    C'est la méthode employée pour la question \itbf{11.}.\\
    Cette méthode est à privilégier lorsque l'estimateur $T_n$ est sans 
    biais.
    \end{noliste}
    \end{remark}


\item On pose : $z(a,b)=\ln(G_{a,b}(1))$ et $r(a,b)=\ln(G_{a,b}(h))$.\\
  Pour tout $n\in\N^*$, on pose : $Z_n = \ln\left(1-\overline{D}_n
    +\dfrac{1}{n}\right)$ et $R_n =
  \ln\left(\overline{S}_n+\dfrac{1}{n}\right)$.\\
  {\it On admet} que $Z_n$ et $R_n$ sont des estimateurs convergents
  de $z(a,b)$ et $r(a,b)$ respectivement.
  \begin{noliste}{a)}
    \setlength{\itemsep}{2mm}
  \item Soit $\eps$, $\lambda$ et $\mu$ des réels strictement
    positifs.
    \begin{noliste}{(i)}
    \item Justifier l'inclusion suivante :
      \[
      \Ev{ \left\vert (\lambda Z_n-\mu R_n) - (\lambda z(a,b) -\mu
          r(a,b))\right\vert \geq \eps} \subset \Ev{\lambda
        \vert Z_n - z(a,b)\vert + \mu \vert R_n-r(a,b)\vert \geq
        \eps}.
      \]
      
    \begin{proof}~\\
    Soit $\omega \in \Omega$.\\
    Supposons : $\omega \ \in \ \Ev{ \left\vert (\lambda Z_n-\mu R_n) - 
    (\lambda \, z(a,b) -\mu \, r(a,b))\right\vert \geq \eps}$.\\[.1cm]
    Autrement dit : $\left\vert (\lambda Z_n(\omega)-\mu R_n(\omega)) - 
    (\lambda \, z(a,b) -\mu \, r(a,b))\right\vert \geq \eps$.\\
    Or :
    \[
    \begin{array}{rcl@{\qquad}>{\it}R{6.6cm}}
      & &\left\vert \ (\lambda Z_n(\omega)-\mu R_n(\omega)) \ - \ 
      (\lambda \, z(a,b) -\mu \,
        r(a,b)) \ \right\vert \\[.2cm]
      & = & \left\vert \ \lambda Z_n(\omega) - \lambda \, z(a,b)- \mu 
      R_n(\omega) +\mu \,
        r(a,b)) \ \right\vert\\[.2cm]
      & = & \left\vert \ \lambda (Z_n(\omega) - z(a,b)) \ - \ \mu 
      (R_n(\omega) -r(a,b)) \ \right\vert\\[.2cm]
      &\leq&  \left\vert \ \lambda (Z_n(\omega) - z(a,b)) \ \right\vert 
      \ + \ \left\vert \ \mu (R_n(\omega) -r(a,b)) \ \right\vert & (par 
      inégalité triangulaire)
      \nl
      \nl[-.2cm]
      & = & \lambda \left\vert \ (Z_n(\omega) - z(a,b)) \ \right\vert \ 
      + \ \mu \left\vert \ (R_n(\omega) - r(a,b)) \ \right\vert & (car 
      $\lambda>0$ et $\mu>0$)
    \end{array}
    \]
    Donc : $\lambda \left\vert \ (Z_n(\omega) - z(a,b)) \ \right\vert \ 
      + \ \mu \left\vert \ (R_n(\omega) - r(a,b)) \ \right\vert
      \geq \eps$.\\[.1cm]
    Autrement dit : $\omega \in \Ev{\lambda \vert Z_n - z(a,b)\vert + 
      \mu \vert R_n-r(a,b)\vert \geq \eps}$.~\\[-.8cm]
      
    \conc{D'où : $\Ev{ \left\vert (\lambda Z_n-\mu R_n) - (\lambda
          z(a,b) -\mu r(a,b))\right\vert \geq \eps} \subset
      \Ev{\lambda \vert Z_n - z(a,b)\vert + \mu \vert R_n-r(a,b)\vert
        \geq \eps}$.}
        
    \begin{remarkL}{.98}%~%
      \begin{noliste}{$\sbullet$}
      \item Par définition, un événement est un ensemble. Démontrer
        l'inclusion de deux ensembles c'est démontrer que tout élément
        du premier ensemble est dans le second.
        % Ou encore qu'un élément est dans le premier ensemble si et
        % seulement si il est aussi dans le second.
      \item En terme d'événement, cela signifie que si le premier
        événement est réalisé (il existe $\omega$ réalisant cet
        événement \ie il existe $\omega$ appartenant à cet événement)
        alors le second événement est réalisé (l'élément $\omega$
        précédent est aussi élément de cet événement).
      \end{noliste}
    \end{remarkL}~\\[-1.4cm]
  \end{proof}
	
	
	
	\newpage
	
      
    \item En déduire l'inégalité suivante :
    \end{noliste}
    \[
    \Prob([\vert(\lambda Z_n-\mu R_n)-\left(\lambda z(a,b) -\mu
      r(a,b)\right)\vert \geq \eps ]) \leq \Prob\left(\left[ \vert
        Z_n-z(a,b)\vert \geq \dfrac{\eps}{2\lambda}\right]\right) +
    \Prob\left(\left[\vert R_n-r(a,b)\vert \geq
        \dfrac{\eps}{2\mu}\right]\right).
    \]
    
    \begin{proof}~
    \begin{noliste}{$\sbullet$}
    \item D'après la question précédente, on a déjà :
    \[
    \Prob(\Ev{\left\vert (\lambda Z_n-\mu R_n) - (\lambda z(a,b) -\mu
          r(a,b))\right\vert \geq \eps}) \ \leq \ \Prob(\Ev{\lambda
        \vert Z_n - z(a,b)\vert + \mu \vert R_n-r(a,b)\vert \geq
        \eps}) \quad (\star)
    \]
    \item On note :
    \[
      \begin{array}{l}
        A \ = \ \Ev{\lambda \vert Z_n - z(a,b)\vert + \mu \vert 
	R_n-r(a,b)\vert \geq \eps}
	\\[.2cm]
	B \ = \ \Ev{\lambda \vert Z_n - 
      z(a,b)\vert \geq \dfrac{\eps}{2}}
      \\[.4cm]
      C \ = \ \Ev{\mu \vert 
      R_n - r(a,b)\vert \geq \dfrac{\eps}{2}}
      \end{array}
    \]
    Si on parvient à démontrer : $A \ \subset \ B \cup C$,
    alors par croissance de $\Prob$ :
    \[
      \Prob(A) \ \leq \ \Prob( B\cup C)
    \]
    Et comme :
    \[
    \begin{array}{rcll}
    \Prob(B\cup C) & = & \Prob(B)+\Prob(C)-\Prob(B\cap C)
    \\[.2cm]
    & \leq & \Prob(B)+\Prob(C)
    \end{array}
    \]
    En utilisant la propriété $(\star)$, on obtient 
    alors :
    \[
    \begin{array}{rcl@{\qquad}>{\it}R{6.6cm}}
    \Prob(\Ev{\left\vert (\lambda Z_n-\mu R_n) - (\lambda z(a,b) -\mu
    r(a,b))\right\vert \geq \eps})
    &\leq& \Prob(A) & (d'après $(\star)$)
    \nl
    \nl[-.2cm]
    &\leq& \Prob\left(B \cup C\right) 
    \\[.2cm]
    &\leq& \Prob\left(B\right) + \Prob\left(C \right)
    \end{array}
    \]
    ce qui permet de conclure la question.

    \item Il reste alors à montrer :
    $A \ \subset \ B \cup C$. Autrement dit :
    \[
      \forall \omega \in \Omega, \ \omega \in A \ \Rightarrow \ \omega 
      \in B \cup C
    \]
    ce qui équivaut par contraposée à :
    \[
      \forall \omega \in \Omega, \ \NON{\omega \in B \cup C} \ 
      \Rightarrow \ \NON{\omega \in A}
    \]
    Soit $\omega \in \Omega$.
    Supposons donc : 
    $
      \NON{\omega \in B \cup C}
    $.\\[.1cm]
    Ainsi : $\omega \in \overline{B \cup C} = \overline{B} \cap 
    \overline{C}$, ou encore :
    \[
      \lambda \vert Z_n(\omega) - z(a,b) \vert < \dfrac{\eps}{2}
      \quad \text{et} \quad \mu \vert R_n(\omega) -r(a,b) \vert 
      < \dfrac{\eps}{2}
    \]
    Alors, en sommant membre à membre :
    \[
      \lambda \vert Z_n(\omega) - z(a,b) \vert + \mu \vert R_n(\omega) 
      -r(a,b) \vert < \eps
    \]
    D'où :
    \[
      \omega \ \in \ \Ev{\lambda \vert Z_n(\omega) - z(a,b) \vert + \mu 
      \vert R_n(\omega) -r(a,b) \vert < \eps} \ = \
      \overline{A}
    \]
    De plus : $\omega \in \overline{A} \ \Leftrightarrow \
    \NON{\omega \in A}$.\\[.1cm]
    On a donc bien démontré : $A \ \subset \ B \cup C$.
    
    {\small
    \hspace*{-1.4cm}\fbox{
      \begin{tabular}{c}
    	$\Prob([\vert(\lambda Z_n-\mu R_n)-\left(\lambda z(a,b) -\mu
      r(a,b)\right)\vert \geq \eps ]) \leq \Prob\left(\left[ \vert
        Z_n-z(a,b)\vert \geq \dfrac{\eps}{2\lambda}\right]\right) +
    \Prob\left(\left[\vert R_n-r(a,b)\vert \geq
        \dfrac{\eps}{2\mu}\right]\right).$
      \end{tabular}
      }}
      \end{noliste}
      
      
      
      \newpage
      
      
    
      \begin{remark}%~\\
        On utilise dans cette question la formule du crible : pour
        tout $(A,B)\in\A$,
        \[
        \Prob(A\cup B) = \Prob(A) + \Prob(B) - \Prob(A\cap B).
        \]
        On peut distinguer $3$ corollaires usuels de cette formule.
        \begin{noliste}{1)}
        \item \underline{En toute généralité} :
          \[
          \Prob(A\cup B) \leq \Prob(A) + \Prob(B),
          \]
          car une probabilité est toujours positive. \\
          C'est le
          corollaire utilisé dans cette question.\\
          On remarquera bien que celui-ci est valable pour tout
          type d'événements $A$ et $B$.
        \item \underline{Si $A$ et $B$ sont incompatibles} :
          \[
          \Prob(A\cup B) = \Prob(A) + \Prob(B).
          \]
          En effet, si $A$ et $B$ sont incompatibles, alors $A\cap
          B = \varnothing$ et donc $\Prob(A\cap B)=0$.
        \item \underline{Si $A$ et $B$ sont indépendants} :
          \[
          \Prob(A\cup B)=\Prob(A)+\Prob(B)-\Prob(A) \, \Prob(B).
          \]
          En effet, si $A$ et $B$ sont indépendants, alors :
          $\Prob(A\cap B) = \Prob(A) \, \Prob(B)$.
        \end{noliste}
      \end{remark}~\\[-1.4cm]
    \end{proof}
    
  \item Pour tout $n\in\N^*$, on pose : $B_n=\dfrac{2}{h-1}Z_n -
    \dfrac{2}{h(h-1)}R_n$.\\
    Montrer que $B_n$ est un estimateur convergent du paramètre $b$.
    
    \begin{proof}~\\
    On pose $\lambda=\dfrac{2}{h-1}$ et $\mu=\dfrac{2}{h(h-1)}$. \\
    On remarque que $\lambda>0$ et $\mu>0$, car $h\geq 2$, et 
    $B_n=\lambda Z_n-\mu R_n$.\\
    D'après la question précédente et puisqu'une probabilité est 
    toujours positive, on obtient :
    \[
    0 \ \leq \ \Prob([\vert B_n-\left(\lambda z(a,b) -\mu
      r(a,b)\right)\vert \geq \eps ]) \ \leq \ \Prob\left(\left[ \vert
        Z_n-z(a,b)\vert \geq \dfrac{\eps}{2\lambda}\right]\right) +
    \Prob\left(\left[\vert R_n-r(a,b)\vert \geq
        \dfrac{\eps}{2\mu}\right]\right)
    \]
    Or :
    \begin{noliste}{$\stimes$}
    \item par hypothèse, $Z_n$ est un estimateur convergent de 
    $z(a,b)$. \\[.1cm]
    D'où : $\dlim{n\to+\infty} \Prob\left( \left[\vert 
    Z_n-z(a,b)\vert \geq \dfrac{\eps}{2\lambda}\right]\right)=0$.
    \item de plus, $R_n$ est un estimateur convergent de 
    $r(a,b)$.\\[.1cm] 
    D'où :
    $\dlim{n\to+\infty} \Prob\left( \left[\vert 
    R_n-r(a,b)\vert \geq \dfrac{\eps}{2\mu}\right]\right)=0$.
    \end{noliste}
    D'après le théorème d'encadrement, on a donc :
    \[
    \dlim{n\to+\infty} \Prob([\vert B_n-\left(\lambda z(a,b) -\mu
      r(a,b)\right)\vert \geq \eps ])=0
    \]
    c'est-à-dire que $B_n$ est un estimateur convergent de $\lambda 
    z(a,b)-\mu r(a,b)$.\\
    Or :
    \[
    \begin{array}{rcl@{\qquad}>{\it}R{6.6cm}}
      \lambda z(a,b)-\mu r(a,b) & = & \dfrac{2}{h-1}\ln\left(G_{a,b}(1)\right) - \dfrac{2}{h(h-1)} \ln\left(G_{a,b}(h)\right)\\[.4cm]
      & = & \dfrac{2}{h-1}\left[\ln\left(\exp\left(-a-\dfrac{b}{2}\right)\right)-\dfrac{1}{h}\ln\left(\exp\left(-ah-\dfrac{b}{2}h^2\right)\right)\right]\\[.4cm]
      & = & \dfrac{2}{h-1}\left[\left(-a-\dfrac{b}{2}\right)-\dfrac{1}{h}\left(-ah-\dfrac{b}{2}h^2\right)\right]\\[.4cm]
      & = & \dfrac{2}{h-1}\left[-\bcancel{a}-\dfrac{b}{2}+\bcancel{a}+\dfrac{b}{2}h\right]\\[.4cm]
      & = & \dfrac{\bcancel{2}}{\bcancel{h-1}}\left[\dfrac{b}{\bcancel{2}}\bcancel{(h-1)}\right]\\[.4cm]
      & = & b
    \end{array}
    \]
    \conc{$B_n$ est un estimateur convergent de $b$.}
    
    \begin{remark}%~\\
    On utilise ici la définition d'un estimateur convergent et non la 
    propriété nécessitant le risque quadratique.
    \end{remark}~\\[-1.4cm]
    \end{proof}
  \end{noliste}
\end{noliste}

% \begin{remark}[Commentaires sur le problème]~
%   \begin{noliste}{$\sbullet$}
%   \item On peut considérer que ce problème est classique dans sa
%     conception. En effet, il s'agit d'étudier une loi (la loi
%     exponentielle linéaire) qui n'est pas abordée dans le programme
%     ECE.
%   \item De tels sujets sont excellents pour juger du niveau d'un
%     candidat. Si l'objet à étudier est nouveau, les méthodes d'étude,
%     elles, sont classiques. Le niveau d'exigence se situe alors bien
%     au-delà du simple bachotage : il s'agit de mesurer la capacité
%     d'un candidat à réinvestir ses connaissances.
%   \end{noliste}
%   % Faire un topo de contenu sur les symétries.
% \end{remark}


\end{document}



