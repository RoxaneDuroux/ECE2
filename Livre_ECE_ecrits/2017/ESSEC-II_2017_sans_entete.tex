\chapter*{ESSEC-II 2017 : le sujet}
  
%

\noindent
Étudier l'évolution des inégalités dans la répartition des richesses, 
matérielles ou symboliques, dans une société est un des thèmes majeurs 
des sciences humaines. Considérons un exemple élémentaire. Le tableau 
ci-dessous présente le pourcentage d'accès à l'enseignement secondaire 
en Grande-Bretagne lors de deux périodes pour deux catégories sociales: 
\\
\begin{center} 
 \begin{tabular}{c|c|c|} 
  & avant 1910 & entre 1935 et 1940\\ \hline
  Profession libérale & $37\%$ & $62\%$ \\ \hline
  Ouvriers & $1\%$ & $10\%$ \\ \hline
 \end{tabular}
\end{center}
On propose trois modes de comparaison des inégalités entre les deux 
classes sociales.
\begin{noliste}{1.}
 \item En regardant l'augmentation des pourcentages pour les deux 
 classes entre les deux périodes on conclut que l'inégalité a augmenté 
 entre la classe la plus aisée (Profession libérale) et la plus 
 défavorisée (Ouvriers). 
 
 \item Si on regarde le taux d'accroissement des pourcentages, comme 
 $\frac{10}{1} > \frac{62}{37}$, on déduit que l'inégalité a diminué.
 
 \item Si on regarde le taux d'accroissement des pourcentages \emph{de 
 ceux qui n'accèdent pas à l'enseignement secondaire}, comme 
 $\frac{90}{99} > \frac{38}{63}$, on déduit que l'inégalité a 
 augmenté puisque le nombre de ceux qui n'ont pas accès à 
 l'enseignement supérieur a proportionnellement plus diminué que celui 
 de ceux qui y ont accès.
\end{noliste}
Comme on le voit chacune des façons de voir est légitime à sa manière. 
L'objet du problème est d'introduire des outils afin d'étudier la 
\emph{concentration} d'une loi de probabilité pour contourner des 
paradoxes auxquels une analyse trop rapide peut conduire, ou du moins 
d'en être conscient.




\section*{Partie I - Indice de Gini}

\noindent
On rappelle qu'une fonction numérique définie sur l'intervalle $J$ de 
$\R$ est \emph{convexe} sur $J$ si elle vérifie la propriété suivante: 
$\forall (t_1,t_2) \in J^2, \ \forall \lambda \in [0,1], \ f( \lambda 
\, t_1 + (1- \lambda) \, t_2) \leq \lambda \, f(t_1) + (1-\lambda ) \, 
f(t_2)$. \\
On rappelle en outre qu'une fonction $f$ est \emph{concave} si $-f$ est 
convexe.\\
On désigne par $E$ l'ensemble des applications $f$ définies sur $[0,1]$ 
à valeurs dans $[0,1]$, continues et convexes sur $[0,1]$, et telles 
que 
$f(0)=0$ et $f(1)=1$. Pour toute application $f$ de $E$, on note 
$\tilde{f}$ l'application associée à $f$, définie sur $[0,1]$ par 
$\tilde{f}(t)=t-f(t)$. \\[.1cm]
On pose $I(f)= 2\dint{0}{1} \tilde{f}(t) \dt= 
2\dint{0}{1}(t-f(t)) \dt$. $I(f)$ s'appelle l'\textbf{indice de Gini} 
de l'application $f$. 

\begin{noliste}{1.}
 \setlength{\itemsep}{4mm}
 \item 
 \begin{noliste}{a)}
  \setlength{\itemsep}{2mm}
  \item Donnez une interprétation géométrique de la propriété de 
  convexité.
  
  
  
  
  

  
  \item Lorsque $f$ est une fonction de classe $\Cont{1}$ sur $[0,1]$, 
  rappeler la caractérisation de la convexité de $f$ sur $[0,1]$ à 
  l'aide de la dérivée $f'$.
  
  

 \end{noliste}
 
 \item 
 \begin{noliste}{a)}
  \setlength{\itemsep}{2mm}
  \item Justifier que $\tilde{f}$ est concave sur $[0,1]$.
  
  
  
  \item Montrer que $I(f)= 1-2 \ \dint{0}{1}f(t) \dt$. 
  
  

  
  \item Représenter dans un même repère orthonormé les fonctions $f$ et 
  $t \mapsto t $ et donner une interprétation géométrique de $I(f)$.
  
  
 \end{noliste}
 
 
 
 %\newpage
 
 
 
 \item \textbf{Un premier exemple}.\\
 Soit $f : [0,1] \to \R$ telle que $f(t)= t^2$ pour tout $t \in [0,1]$. 
 \begin{noliste}{a)}
  \setlength{\itemsep}{2mm}
  \item Montrer que $f$ est un élément de $E$.
  
  

  
  \item Calculer $I(f)$. 
  
  
 \end{noliste}
 

\newpage


 \item \textbf{Propriétés de l'indice de Gini.}
 \begin{noliste}{a)}
  \item Pour $f$ élément de $E$, établir que $I(f) \geq 0$. 
  
  

  
  \item Montrer que $I(f) =0$ si et seulement si $f(t)=t$ pour tout $t 
  \in [0,1]$. 
  
  
  
  
  
  %\newpage
  

  
  \item Montrer que pour tout $f$ élément de $E$, $I(f) < 1$.
  
  

  
  \item Pour tout entier $n >0$, on définit $f_n$ sur $[0,1]$ par 
  $f_n(t)=t^n$.
  \begin{nonoliste}{(i)}
    \item Pour tout entier $n$ strictement positif, calculer $I(f_n)$.
    
    

    
    \item En déduire que pour tout réel $A$ vérifiant $0 \leq A <1$, il 
    existe $f$ appartenant à $E$ telle que $I(f)>A$. 
    
    

  \end{nonoliste}
 \end{noliste}
 
 \item \textbf{Minoration de l'indice de Gini} 
 \begin{noliste}{a)}
  \item Soit $f$ élément de $E$. Montrer qu'il existe $t_0$ dans 
  $]0,1[$ tel que $\tilde{f}(t_0)= \dmax{t \in [0,1]} \tilde{f}(t)$. 
  
  

 
  \item Montrer que pour tout $t$ de $[0,t_0]$, $\tilde{f}(t) \geq 
  \tilde{f}(t_0) \cdot \dfrac{t}{t_0}$. 
  
  

  
  \item Montrer que pour tout $t$ de $[t_0,1]$, $\tilde{f}(t) \geq 
  \tilde{f}(t_0) \cdot \dfrac{t-1}{t_0-1}$. 
  
  
  
  \item En déduire que $I(f) \geq \tilde{f}(t_0)$. 
  
  

 \end{noliste}
 L'indice de Gini donne une indication sur la concentration des 
 richesses d'un pays si l'on suppose que la fonction $f$ rend compte de 
 cette concentration. Par exemple, $f(0,3)=0,09$ s'interprète par le 
 fait que dans la population classée par ordre de richesse croissante, 
 les premiers $30 \%$ de la population possèdent $9\%$ de la richesse 
 totale du pays. Plus l'indice $I(f)$ est grand, plus la répartition 
 des richesses est inégalitaire.
\end{noliste}



\section*{Partie II - Le cas à densité}

\noindent
Soit $g$ une densité de probabilité sur $\R$, nulle sur $]-\infty, 0]$, 
continue et strictement positive sur $]0,+\infty[$. On définit une 
fonction $G$ sur $\R_+$ par $G(x)=\dint{0}{x} g(v) \ dv$ pour $x \in 
\R_+$. Si $g$ représente la densité de population classée suivant son 
revenu croissant, $G(x)$ représente la proportion de la population dont 
le revenu est inférieur à $x$. On suppose de plus que 
$\dint{0}{+\infty} v \, g(v) \ dv$ est convergente et on note $m$ sa 
valeur qui représente donc la richesse moyenne de la population. 

\begin{noliste}{1.}
 \setlength{\itemsep}{4mm}
 \setcounter{enumi}{5}
 \item 
 \begin{noliste}{a)}
  \setlength{\itemsep}{2mm}
  \item Montrer que $m >0$.
  
  

  
  \item Montrer que $G$ est une bijection de $[0,+\infty[$ sur $[0,1[$. 
  On notera $G^{-1}$ son application réciproque.
  
  

  
  \item Quel est le sens de variation de $G^{-1}$ sur $[0,1[$ ?
  
  
 \end{noliste}
 
 \item
 \begin{noliste}{a)}
  \setlength{\itemsep}{2mm}
  \item À l'aide du changement de variable $u=G(v)$, établir que pour 
  tout $t \in [0,1[$, 
  \[
   \dint{0}{t} G^{-1}(u) \ du= \dint{0}{G^{-1}(t)} v \, g(v) \ dv
  \]
  
  
  
  \item En déduire que $\dint{0}{1} G^{-1}(u) \ du$ converge et donner 
  sa valeur. 
  
  

 \end{noliste}
 

\newpage


 \item Soit $f$ la fonction définie sur $[0,1]$ par : $f(t)= 
 \dfrac{1}{m} \dint{0}{t} G^{-1}(u) \ du$ pour tout $t \in [0,1[$ et 
 $f(1)=1$.
 \begin{noliste}{a)}
  \setlength{\itemsep}{2mm}
  \item 
  \begin{nonoliste}{(i)}
   \item Montrer que $f$ est continue sur $[0,1]$. 
   
   

   
   
   
   %\newpage
   
   
   
   \item Montrer que $f$ est convexe sur $[0,1[$. \textbf{On admettra 
   qu'en fait $f$ est convexe sur $[0,1]$}. 
   
   
   
   \item En déduire que $f$ est un élément de $E$.
   
   
  \end{nonoliste}
  
  \item Montrer, à l'aide d'une intégration par parties, l'égalité
  \[
   I(f)= -1+\dfrac{2}{m} \dint{0}{\infty} v \, g(v) \, G(v) \ dv
  \]
  
  

 \end{noliste}
 
 
 %\newpage
 
 
 \item Soit $\lambda$ un réel strictement positif. On suppose dans 
 cette question que $g$ est une densité de la loi exponentielle de 
 paramètre $\lambda$.
 \begin{noliste}{a)}
  \setlength{\itemsep}{2mm}
  \item Expliciter $G(x)$ pour $x >0$. 
  
  
  
  \item Expliciter $G^{-1}(u)$ pour $u \in [0,1[$. 
  
  

  
  \item Donner la valeur de $m$. 
  
  

  
  \item Soit $t \in[0,1[$. Montrer que $f(t)= - \dint{0}{t} \ln(1-u) 
  \ du.$
  
  

  
  
  %\newpage
  
  
  \item En déduire que pour tout $t$ élément de $[0,1[$, on a $f(t)= 
  (1-t)\ln(1-t)+t$. 
  
  
  
  
  
  %\newpage
  

  
  \item Justifier la convergence de l'intégrale $\dint{0}{1} 
  (1-t)\ln(1-t) \dt$ et la calculer. 
  
  
  
  \item En déduire la valeur de $I(f)$.
  
  
 \end{noliste}
\end{noliste}


%\newpage


\section*{Partie III - Application à une population}

\noindent
Une population de $N$ personnes est divisée en deux classes 
(typiquement hommes et femmes) et en $n$ catégories (par exemple 
socio-professionnelles), suivant le tableau à double entrée suivant où 
tous les $x_i$ et $y_i$ pour $i$ dans $\llb 1,n \rrb$ sont des entiers 
naturels. \\
On suppose en outre que pour tout $i$ dans $\llb 1,n \rrb$, 
$x_i \neq 0$. \\ 
\begin{center}
 \begin{tabular}{|c|c|c|c|c|c|c|c|c|}
  \hline
  \backslashbox{Classes}{Catégories} & $c_1$ & $c_2$ & $c_3$ & $\cdots$ 
  & $c_i$ & $\cdots$ & $c_n$ & Total \\ 
  \hline
  I & $x_1$ & $x_2$ & $x_3$ & $\cdots$ & $x_i$ & $\cdots$ & $x_n$ & $X$ 
  \\ 
  \hline
  II & $y_1$ & $y_2$ & $y_3$ & $\cdots$ & $y_i$ & $\cdots$ & $y_n$ & 
  $Y$ \\ 
  \hline
  Total & $n_1$ & $n_2$ & $n_3$ & $\cdots$ & $n_i$ & $\cdots$ & $n_n$ & 
  $N$ \\ 
  \hline
 \end{tabular}
\end{center}

\noindent
où on a donc posé $X= \Sum{i=1}{n} x_i$, $Y= \Sum{i=1}{n} y_i$ et 
$X+Y=N$. On suppose en outre que $Y>0$. \\
Pour $i$ appartenant à $\llb 1,n\rrb$, on adopte les notations 
suivantes: 
\[
 p_i=\dfrac{n_i}{N}, \ q_i=\dfrac{x_i}{X},  \ r_i=\dfrac{y_i}{Y}
\]
On note aussi $\varepsilon_i= \dfrac{x_i}{n_i}$, et 
$\varepsilon=\dfrac{X}{N}$, et on suppose que les catégories sont 
numérotées de telle sorte que : 
\[
 \varepsilon_1 \leq \varepsilon_2 \leq \cdots \leq \varepsilon_n
\]


\newpage


\begin{noliste}{1.}
 \setlength{\itemsep}{4mm}
 \setcounter{enumi}{9}
 \item On pose $\Omega=\{c_1,c_2, \ldots ,c_n\}$, ensemble des 
 catégories dans la population.
 \begin{noliste}{a)}
  \setlength{\itemsep}{2mm}
  \item Montrer que $P=(p_i)_{1 \leq i \leq n}$, $Q= (q_i)_{1 \leq i 
  \leq n}$ et $R= (r_i)_{1 \leq i \leq n}$ sont des distributions de 
  probabilités.
  
  
  
  
  %\newpage
  

  
  \item Montrer que : $\dfrac{q_1}{p_1} \leq \cdots \leq 
  \dfrac{q_n}{p_n}$ $(\ast)$
  
  
  
  \item Montrer que : $\dfrac{r_1}{p_1} \geq \cdots \geq 
  \dfrac{r_n}{p_n}$.
  
  

  
  \item Montrer que pour $i$ appartenant à $\llb 1,n \rrb$, $r_i= 
  \dfrac{n_i-x_i}{N-X}= \dfrac{p_i-\varepsilon q_i}{1- \varepsilon}$. 
  
  
 \end{noliste}
 
 
 
 %\newpage
 
 
 
 \item Dans un premier temps, nous allons construire une application 
 appartenant à $E$, qui permet de mesurer les inégalités à l'intérieur 
 de la classe I.\\
 On pose $P_0=Q_0=0$, et pour $i \in \llb 1,n \rrb$, $P_i= \Sum{h=1}{i} 
 p_h$ et $Q_i=\Sum{h=1}{i} q_h$. On définit alors l'application 
 $\varphi$ de $[0,1]$ dans $[0,1]$ telle que, pour tout entier $i \in 
 \llb 0,n \rrb$, $\varphi(P_i)=Q_i$ et pour tout entier $i \in \llb 
 0,n-1 \rrb$, $\varphi$ est affine sur le segment $[P_i, P_{i+1}]$. 
 \begin{noliste}{a)}
  \setlength{\itemsep}{2mm}
  \item On suppose \textbf{dans cette question} $n=3$.\\
  Représenter dans un repère orthonormé $\varphi$ lorsque $P= \left( 
  \frac{1}{2}, \frac{1}{4}, \frac{1}{4} \right)$ et $Q=\left( 
  \frac{1}{3}, \frac{1}{6}, \frac{1}{2}  \right)$. 
  
  

  
  \item Montrer que, dans le plan rapporté à un repère orthonormé, la 
  pente de la droite passant par les points de coordonnées $(P_{i-1}, 
  Q_{i-1})$ et $(P_i, Q_i)$ est $u_i=\frac{q_i}{p_i}$ pour $i$ 
  appartenant à $\llb 1,n \rrb$. 
  
  
  
  
  
  %\newpage
  

  
  \item Montrer que si $i \in \llb 0,n-1 \rrb$ et $t \in [P_i, 
  P_{i+1}]$, on a $\varphi(t) =u_{i+1} (t-P_i)+Q_i$. 
  
  
  
  \item \textbf{En admettant} que les inégalités $(\ast)$ de la 
  question \itbf{10.b)} permettent d'affirmer que $\varphi$ est 
  convexe, justifier que $\varphi$ appartient à $E$. 
  
  

  
  \item Pour $i \in \llb 0,n-1 \rrb$, calculer $\dint{P_i}{P_{i+1}} 
  \varphi(t) \dt$. 
  
  
  
  \item Exprimer $I(\varphi)$ sous la forme d'une somme en fonction de 
  $P_0, P_1,...,P_n, Q_0,...Q_n$. 
  
  
 \end{noliste}
 
 \item Nous allons maintenant étudier l'application correspondante pour 
 la classe II.\\
 On pose $P_0=R_0=0$ et pour $i \in \llb 1,n \rrb$ , $P_i = 
 \Sum{h=1}{i} p_h$ et $R_i=\Sum{h=1}{i} r_h$. De même, on définit pour 
 $i$ élément de $\llb 0,n \rrb$, $\Pi_i=1-P_{n-i}$. On considère 
 l'application $\psi$ de $[0,1]$ dans $[0,1]$ telle que pour tout $i 
 \in \llb 0,n \rrb$, $\psi(P_i)=R_i$ et pour tout entier $i \in \llb 0, 
 n-1 \rrb$, $\psi$ est affine sur le segment $[P_i, P_{i+1}]$. 
 \begin{noliste}{a)}
  \setlength{\itemsep}{2mm}
  \item Montrer que la pente de la droite passant par les points de 
  coordonnées $(P_{i-1}, R_{i-1})$ et $(P_i, R_i)$ est $v_i 
  = \frac{r_i}{p_i}$ pour $i \in \llb 1,n \rrb$. 
  
  
  
  
  
  %\newpage
  

  
  \item On considère l'application $\psi^*$ définie pour tout $t \in 
  [0,1]$, par $\psi^*(t)= 1-\psi(1-t)$. 
  \begin{nonoliste}{(i)}
   \item On suppose \textbf{dans cette question} $n=3$.\\
   Représenter dans un même repère orthonormé les courbes 
   représentatives de $\psi$ et $\psi^*$ lorsque $P= \left( 
   \frac{1}{2}, \frac{1}{4}, \frac{1}{4} \right)$ et $R=\left( 
   \frac{2}{3}, \frac{1}{6}, \frac{1}{6}  \right)$. 
   
   
   
   \item Montrer que $\psi^*$ est convexe sur $[0,1]$. 
   
   

   
   \item Montrer que $\psi^*$ est affine sur $[\Pi_{i-1}, \Pi_i]$ pour 
   $i \in \llb 0,n-1 \rrb$. 
   
   

   
   \item Montrer que la pente de $\psi^*$ sur $[\Pi_{i-1}, \Pi_i]$ est 
   $v_{n-i+1}$ pour $i \in \llb 1,n \rrb$. 
   
   
  \end{nonoliste}

  \noindent
  On dit dans cette situation que les fonctions $\varphi$ et $\psi^*$ 
  sont \textbf{adjointes} l'une de l'autre. C'est leur comparaison que 
  Gini a proposé de considérer pour \og mesurer les inégalités \fg{} 
  entre la population de catégorie I et celle de catégorie II. \\
  Une égalité entre les fonctions adjointes signale notamment l'absence 
  totale d'inégalité sociale. La dernière question précise quelque peu 
  ce point. 
 \end{noliste}
 
 \item 
 \begin{noliste}{a)}
  \setlength{\itemsep}{2mm}
  \item Montrer que si $\varphi= \psi^*$ alors pour tout $i$ 
  appartenant à $\llb 1,n \rrb$ :
  \[
   \dfrac{\varepsilon_i}{\varepsilon} = 
   \dfrac{1-\varepsilon_{n-i+1}}{1-\varepsilon}
  \]
  
  
  
  \item Montrer que si $\varphi= \psi^*$, alors pour tout $i$ 
  appartenant à $\llb 1,n \rrb$, $\varepsilon_i+ \varepsilon_{n-i+1}=2 
  \varepsilon$. 
  
  

  
  \item Déduire que si $\varphi= \psi^*$, on a pour tout $i$ 
  appartenant à $\llb 1,n \rrb$, $\varepsilon_i(1-2\varepsilon)= 
  \varepsilon(1-2\varepsilon)$.
  
  
  
  
  %\newpage

  
  \item On suppose que $\varepsilon \neq \dfrac{1}{2}$.  Montrer que si 
  $\varphi= \psi^*$, alors pour tout $i$ appartenant à $\llb 1,n \rrb$, 
  $\varepsilon_i=\varepsilon$. \\[.1cm]
  Interpréter ce résultat.
  
  

 \end{noliste}
\end{noliste}






