\documentclass[11pt]{article}%
\usepackage{geometry}%
\geometry{a4paper,
  lmargin=2cm,rmargin=2cm,tmargin=2.5cm,bmargin=2.5cm}

\usepackage{array}
\usepackage{paralist}

\usepackage[svgnames, usenames, dvipsnames]{xcolor}
\xdefinecolor{RecColor}{named}{Aqua}
\xdefinecolor{IncColor}{named}{Aqua}
\xdefinecolor{ImpColor}{named}{PaleGreen}

% \usepackage{frcursive}

\usepackage{adjustbox}

%%%%%%%%%%%
\newcommand{\cRB}[1]{{\color{Red} \pmb{#1}}} %
\newcommand{\cR}[1]{{\color{Red} {#1}}} %
\newcommand{\cBB}[1]{{\color{Blue} \pmb{#1}}}
\newcommand{\cB}[1]{{\color{Blue} {#1}}}
\newcommand{\cGB}[1]{{\color{LimeGreen} \pmb{#1}}}
\newcommand{\cG}[1]{{\color{LimeGreen} {#1}}}

%%%%%%%%%%

\usepackage{diagbox} %
\usepackage{colortbl} %
\usepackage{multirow} %
\usepackage{pgf} %
\usepackage{environ} %
\usepackage{fancybox} %
\usepackage{textcomp} %
\usepackage{marvosym} %

%%%%%%%%%% pour qu'une cellcolor ne recouvre pas le trait du tableau
\usepackage{hhline}%

\usepackage{pgfplots}
\pgfplotsset{compat=1.10}
\usepgfplotslibrary{patchplots}
\usepgfplotslibrary{fillbetween}
\usepackage{tikz,tkz-tab}
\usepackage{ifthen}
\usepackage{calc}
\usetikzlibrary{calc,decorations.pathreplacing,arrows,positioning} 
\usetikzlibrary{fit,shapes,backgrounds}
\usepackage[nomessages]{fp}% http://ctan.org/pkg/fp

\usetikzlibrary{matrix,arrows,decorations.pathmorphing,
  decorations.pathreplacing} 

\newcommand{\myunit}{1 cm}
\tikzset{
    node style sp/.style={draw,circle,minimum size=\myunit},
    node style ge/.style={circle,minimum size=\myunit},
    arrow style mul/.style={draw,sloped,midway,fill=white},
    arrow style plus/.style={midway,sloped,fill=white},
}

%%%%%%%%%%%%%%
%%%%% écrire des inférieur égal ou supérieur égal avec typographie
%%%%% francaise
%%%%%%%%%%%%%

\renewcommand{\geq}{\geqslant}
\renewcommand{\leq}{\leqslant}
\renewcommand{\emptyset}{\varnothing}

\newcommand{\Leq}{\leqslant}
\newcommand{\Geq}{\geqslant}

%%%%%%%%%%%%%%
%%%%% Macro Celia
%%%%%%%%%%%%%

\newcommand{\ff}[2]{\left[#1, #2\right]} %
\newcommand{\fo}[2]{\left[#1, #2\right[} %
\newcommand{\of}[2]{\left]#1, #2\right]} %
\newcommand{\soo}[2]{\left]#1, #2\right[} %
\newcommand{\abs}[1]{\left|#1\right|} %
\newcommand{\Ent}[1]{\left\lfloor #1 \right\rfloor} %


%%%%%%%%%%%%%%
%%%%% tikz : comment dessiner un "oeil"
%%%%%%%%%%%%%

\newcommand{\eye}[4]% size, x, y, rotation
{ \draw[rotate around={#4:(#2,#3)}] (#2,#3) -- ++(-.5*55:#1) (#2,#3)
  -- ++(.5*55:#1); \draw (#2,#3) ++(#4+55:.75*#1) arc
  (#4+55:#4-55:.75*#1);
  % IRIS
  \draw[fill=gray] (#2,#3) ++(#4+55/3:.75*#1) arc
  (#4+180-55:#4+180+55:.28*#1);
  % PUPIL, a filled arc
  \draw[fill=black] (#2,#3) ++(#4+55/3:.75*#1) arc
  (#4+55/3:#4-55/3:.75*#1);%
}


%%%%%%%%%%
%% discontinuité fonction
\newcommand\pointg[2]{%
  \draw[color = red, very thick] (#1+0.15, #2-.04)--(#1, #2-.04)--(#1,
  #2+.04)--(#1+0.15, #2+.04);%
}%

\newcommand\pointd[2]{%
  \draw[color = red, very thick] (#1-0.15, #2+.04)--(#1, #2+.04)--(#1,
  #2-.04)--(#1-0.15, #2-.04);%
}%

%%%%%%%%%%
%%% 1 : position abscisse, 2 : position ordonnée, 3 : taille, 4 : couleur
%%%%%%%%%%
% \newcommand\pointG[4]{%
%   \draw[color = #4, very thick] (#1+#3, #2-(#3/3.75))--(#1,
%   #2-(#3/3.75))--(#1, #2+(#3/3.75))--(#1+#3, #2+(#3/3.75)) %
% }%

\newcommand\pointG[4]{%
  \draw[color = #4, very thick] ({#1+#3/3.75}, {#2-#3})--(#1,
  {#2-#3})--(#1, {#2+#3})--({#1+#3/3.75}, {#2+#3}) %
}%

\newcommand\pointD[4]{%
  \draw[color = #4, very thick] ({#1-#3/3.75}, {#2+#3})--(#1,
  {#2+#3})--(#1, {#2-#3})--({#1-#3/3.75}, {#2-#3}) %
}%

\newcommand\spointG[4]{%
  \draw[color = #4, very thick] ({#1+#3/1.75}, {#2-#3})--(#1,
  {#2-#3})--(#1, {#2+#3})--({#1+#3/1.75}, {#2+#3}) %
}%

\newcommand\spointD[4]{%
  \draw[color = #4, very thick] ({#1-#3/2}, {#2+#3})--(#1,
  {#2+#3})--(#1, {#2-#3})--({#1-#3/2}, {#2-#3}) %
}%

%%%%%%%%%%

\newcommand{\Pb}{\mathtt{P}}

%%%%%%%%%%%%%%%
%%% Pour citer un précédent item
%%%%%%%%%%%%%%%
\newcommand{\itbf}[1]{{\small \bf \textit{#1}}}


%%%%%%%%%%%%%%%
%%% Quelques couleurs
%%%%%%%%%%%%%%%

\xdefinecolor{cancelcolor}{named}{Red}
\xdefinecolor{intI}{named}{ProcessBlue}
\xdefinecolor{intJ}{named}{ForestGreen}

%%%%%%%%%%%%%%%
%%%%%%%%%%%%%%%
% barrer du texte
\usetikzlibrary{shapes.misc}

\makeatletter
% \definecolor{cancelcolor}{rgb}{0.127,0.372,0.987}
\newcommand{\tikz@bcancel}[1]{%
  \begin{tikzpicture}[baseline=(textbox.base), inner sep=0pt]
    \node[strike out, draw] (textbox) {#1}[thick, color=cancelcolor];
    \useasboundingbox (textbox);
  \end{tikzpicture}%
}
\newcommand{\bcancel}[1]{%
  \relax\ifmmode
    \mathchoice{\tikz@bcancel{$\displaystyle#1$}}
               {\tikz@bcancel{$\textstyle#1$}}
               {\tikz@bcancel{$\scriptstyle#1$}}
               {\tikz@bcancel{$\scriptscriptstyle#1$}}
  \else
    \tikz@bcancel{\strut#1}%
  \fi
}
\newcommand{\tikz@xcancel}[1]{%
  \begin{tikzpicture}[baseline=(textbox.base),inner sep=0pt]
  \node[cross out,draw] (textbox) {#1}[thick, color=cancelcolor];
  \useasboundingbox (textbox);
  \end{tikzpicture}%
}
\newcommand{\xcancel}[1]{%
  \relax\ifmmode
    \mathchoice{\tikz@xcancel{$\displaystyle#1$}}
               {\tikz@xcancel{$\textstyle#1$}}
               {\tikz@xcancel{$\scriptstyle#1$}}
               {\tikz@xcancel{$\scriptscriptstyle#1$}}
  \else
    \tikz@xcancel{\strut#1}%
  \fi
}
\makeatother

\newcommand{\xcancelRA}{\xcancel{\rule[-.15cm]{0cm}{.5cm} \Rightarrow
    \rule[-.15cm]{0cm}{.5cm}}}

%%%%%%%%%%%%%%%%%%%%%%%%%%%%%%%%%%%
%%%%%%%%%%%%%%%%%%%%%%%%%%%%%%%%%%%

\newcommand{\vide}{\multicolumn{1}{c}{}}

%%%%%%%%%%%%%%%%%%%%%%%%%%%%%%%%%%%
%%%%%%%%%%%%%%%%%%%%%%%%%%%%%%%%%%%


\usepackage{multicol}
% \usepackage[latin1]{inputenc}
% \usepackage[T1]{fontenc}
\usepackage[utf8]{inputenc}
\usepackage[T1]{fontenc}
\usepackage[normalem]{ulem}
\usepackage[french]{babel}

\usepackage{url}    
\usepackage{hyperref}
\hypersetup{
  backref=true,
  pagebackref=true,
  hyperindex=true,
  colorlinks=true,
  breaklinks=true,
  urlcolor=blue,
  linkcolor=black,
  %%%%%%%%
  % ATTENTION : red changé en black pour le Livre !
  %%%%%%%%
  bookmarks=true,
  bookmarksopen=true
}

%%%%%%%%%%%%%%%%%%%%%%%%%%%%%%%%%%%%%%%%%%%
%% Pour faire des traits diagonaux dans les tableaux
%% Nécessite slashbox.sty
%\usepackage{slashbox}

\usepackage{tipa}
\usepackage{verbatim,listings}
\usepackage{graphicx}
\usepackage{fancyhdr}
\usepackage{mathrsfs}
\usepackage{pifont}
\usepackage{tablists}
\usepackage{dsfont,amsfonts,amssymb,amsmath,amsthm,stmaryrd,upgreek,manfnt}
\usepackage{enumerate}

%\newcolumntype{M}[1]{p{#1}}
\newcolumntype{C}[1]{>{\centering}m{#1}}
\newcolumntype{R}[1]{>{\raggedright}m{#1}}
\newcolumntype{L}[1]{>{\raggedleft}m{#1}}
\newcolumntype{P}[1]{>{\raggedright}p{#1}}
\newcolumntype{B}[1]{>{\raggedright}b{#1}}
\newcolumntype{Q}[1]{>{\raggedright}t{#1}}

\newcommand{\alias}[2]{
\providecommand{#1}{}
\renewcommand{#1}{#2}
}
\alias{\R}{\mathbb{R}}
\alias{\N}{\mathbb{N}}
\alias{\Z}{\mathbb{Z}}
\alias{\Q}{\mathbb{Q}}
\alias{\C}{\mathbb{C}}
\alias{\K}{\mathbb{K}}

%%%%%%%%%%%%
%% rendre +infty et -infty plus petits
%%%%%%%%%%%%
\newcommand{\sinfty}{{\scriptstyle \infty}}

%%%%%%%%%%%%%%%%%%%%%%%%%%%%%%%
%%%%% macros TP Scilab %%%%%%%%
\newcommand{\Scilab}{\textbf{Scilab}} %
\newcommand{\Scinotes}{\textbf{SciNotes}} %
\newcommand{\faire}{\noindent $\blacktriangleright$ } %
\newcommand{\fitem}{\scalebox{.8}{$\blacktriangleright$}} %
\newcommand{\entree}{{\small\texttt{ENTRÉE}}} %
\newcommand{\tab}{{\small\texttt{TAB}}} %
\newcommand{\mt}[1]{\mathtt{#1}} %
% guillemets droits

\newcommand{\ttq}{\textquotesingle} %

\newcommand{\reponse}[1]{\longboxed{
    \begin{array}{C{0.9\textwidth}}
      \nl[#1]
    \end{array}
  }} %

\newcommand{\reponseR}[1]{\longboxed{
    \begin{array}{R{0.9\textwidth}}
      #1
    \end{array}
  }} %

\newcommand{\reponseC}[1]{\longboxed{
    \begin{array}{C{0.9\textwidth}}
      #1
    \end{array}
  }} %

\colorlet{pyfunction}{Blue}
\colorlet{pyCle}{Magenta}
\colorlet{pycomment}{LimeGreen}
\colorlet{pydoc}{Cyan}
% \colorlet{SansCo}{white}
% \colorlet{AvecCo}{black}

\newcommand{\visible}[1]{{\color{ASCo}\colorlet{pydoc}{pyDo}\colorlet{pycomment}{pyCo}\colorlet{pyfunction}{pyF}\colorlet{pyCle}{pyC}\colorlet{function}{sciFun}\colorlet{var}{sciVar}\colorlet{if}{sciIf}\colorlet{comment}{sciComment}#1}} %

%%%% à changer ????
\newcommand{\invisible}[1]{{\color{ASCo}\colorlet{pydoc}{pyDo}\colorlet{pycomment}{pyCo}\colorlet{pyfunction}{pyF}\colorlet{pyCle}{pyC}\colorlet{function}{sciFun}\colorlet{var}{sciVar}\colorlet{if}{sciIf}\colorlet{comment}{sciComment}#1}} %

\newcommand{\invisibleCol}[2]{{\color{#1}#2}} %

\NewEnviron{solution} %
{ %
  \Boxed{
    \begin{array}{>{\color{ASCo}} R{0.9\textwidth}}
      \colorlet{pycomment}{pyCo}
      \colorlet{pydoc}{pyDo}
      \colorlet{pyfunction}{pyF}
      \colorlet{pyCle}{pyC}
      \colorlet{function}{sciFun}
      \colorlet{var}{sciVar}
      \colorlet{if}{sciIf}
      \colorlet{comment}{sciComment}
      \BODY
    \end{array}
  } %
} %

\NewEnviron{solutionC} %
{ %
  \Boxed{
    \begin{array}{>{\color{ASCo}} C{0.9\textwidth}}
      \colorlet{pycomment}{pyCo}
      \colorlet{pydoc}{pyDo}
      \colorlet{pyfunction}{pyF}
      \colorlet{pyCle}{pyC}
      \colorlet{function}{sciFun}
      \colorlet{var}{sciVar}
      \colorlet{if}{sciIf}
      \colorlet{comment}{sciComment}
      \BODY
    \end{array}
  } %
} %

\newcommand{\invite}{--\!\!>} %

%%%%% nouvel environnement tabular pour retour console %%%%
\colorlet{ConsoleColor}{Black!12}
\colorlet{function}{Red}
\colorlet{var}{Maroon}
\colorlet{if}{Magenta}
\colorlet{comment}{LimeGreen}

\newcommand{\tcVar}[1]{\textcolor{var}{\bf \small #1}} %
\newcommand{\tcFun}[1]{\textcolor{function}{#1}} %
\newcommand{\tcIf}[1]{\textcolor{if}{#1}} %
\newcommand{\tcFor}[1]{\textcolor{if}{#1}} %

\newcommand{\moins}{\!\!\!\!\!\!- }
\newcommand{\espn}{\!\!\!\!\!\!}

\usepackage{booktabs,varwidth} \newsavebox\TBox
\newenvironment{console}
{\begin{lrbox}{\TBox}\varwidth{\linewidth}
    \tabular{>{\tt\small}R{0.84\textwidth}}
    \nl[-.4cm]} {\endtabular\endvarwidth\end{lrbox}%
  \fboxsep=1pt\colorbox{ConsoleColor}{\usebox\TBox}}

\newcommand{\lInv}[1]{%
  $\invite$ #1} %

\newcommand{\lAns}[1]{%
  \qquad ans \ = \nl %
  \qquad \qquad #1} %

\newcommand{\lVar}[2]{%
  \qquad #1 \ = \nl %
  \qquad \qquad #2} %

\newcommand{\lDisp}[1]{%
  #1 %
} %

\newcommand{\ligne}[1]{\underline{\small \tt #1}} %

\newcommand{\ligneAns}[2]{%
  $\invite$ #1 \nl %
  \qquad ans \ = \nl %
  \qquad \qquad #2} %

\newcommand{\ligneVar}[3]{%
  $\invite$ #1 \nl %
  \qquad #2 \ = \nl %
  \qquad \qquad #3} %

\newcommand{\ligneErr}[3]{%
  $\invite$ #1 \nl %
  \quad !-{-}error #2 \nl %
  #3} %
%%%%%%%%%%%%%%%%%%%%%% 

\newcommand{\bs}[1]{\boldsymbol{#1}} %
\newcommand{\nll}{\nl[.4cm]} %
\newcommand{\nle}{\nl[.2cm]} %
%% opérateur puissance copiant l'affichage Scilab
%\newcommand{\puis}{\!\!\!~^{\scriptscriptstyle\pmb{\wedge}}}
\newcommand{\puis}{\mbox{$\hspace{-.1cm}~^{\scriptscriptstyle\pmb{\wedge}}
    \hspace{0.05cm}$}} %
\newcommand{\pointpuis}{.\mbox{$\hspace{-.15cm}~^{\scriptscriptstyle\pmb{\wedge}}$}} %
\newcommand{\Sfois}{\mbox{$\mt{\star}$}} %

%%%%% nouvel environnement tabular pour les encadrés Scilab %%%%
\newenvironment{encadre}
{\begin{lrbox}{\TBox}\varwidth{\linewidth}
    \tabular{>{\tt\small}C{0.1\textwidth}>{\small}R{0.7\textwidth}}}
  {\endtabular\endvarwidth\end{lrbox}%
  \fboxsep=1pt\longboxed{\usebox\TBox}}

\newenvironment{encadreL}
{\begin{lrbox}{\TBox}\varwidth{\linewidth}
    \tabular{>{\tt\small}C{0.25\textwidth}>{\small}R{0.6\textwidth}}}
  {\endtabular\endvarwidth\end{lrbox}%
  \fboxsep=1pt\longboxed{\usebox\TBox}}

\newenvironment{encadreF}
{\begin{lrbox}{\TBox}\varwidth{\linewidth}
    \tabular{>{\tt\small}C{0.2\textwidth}>{\small}R{0.70\textwidth}}}
  {\endtabular\endvarwidth\end{lrbox}%
  \fboxsep=1pt\longboxed{\usebox\TBox}}

\newenvironment{encadreLL}[2]
{\begin{lrbox}{\TBox}\varwidth{\linewidth}
    \tabular{>{\tt\small}C{#1\textwidth}>{\small}R{#2\textwidth}}}
  {\endtabular\endvarwidth\end{lrbox}%
  \fboxsep=1pt\longboxed{\usebox\TBox}}

%%%%% nouvel environnement tabular pour les script et fonctions %%%%
\newcommand{\commentaireDL}[1]{\multicolumn{1}{l}{\it
    \textcolor{comment}{$\slash\slash$ #1}}}

\newcommand{\commentaire}[1]{{\textcolor{comment}{$\slash\slash$ #1}}}

\newcounter{cptcol}

\newcommand{\nocount}{\multicolumn{1}{c}{}}

\newcommand{\sciNo}[1]{{\small \underbar #1}}

\NewEnviron{scilab}{ %
  \setcounter{cptcol}{0}
  \begin{center}
    \longboxed{
      \begin{tabular}{>{\stepcounter{cptcol}{\tiny \underbar
              \thecptcol}}c>{\tt}l}
        \BODY
      \end{tabular}
    }
  \end{center}
}

\NewEnviron{scilabNC}{ %
  \begin{center}
    \longboxed{
      \begin{tabular}{>{\tt}l} %
          \BODY
      \end{tabular}
    }
  \end{center}
}

\NewEnviron{scilabC}[1]{ %
  \setcounter{cptcol}{#1}
  \begin{center}
    \longboxed{
      \begin{tabular}{>{\stepcounter{cptcol}{\tiny \underbar
              \thecptcol}}c>{\tt}l}
        \BODY
      \end{tabular}
    }
  \end{center}
}

\newcommand{\scisol}[1]{ %
  \setcounter{cptcol}{0}
  \longboxed{
    \begin{tabular}{>{\stepcounter{cptcol}{\tiny \underbar
            \thecptcol}}c>{\tt}l}
      #1
    \end{tabular}
  }
}

\newcommand{\scisolNC}[1]{ %
  \longboxed{
    \begin{tabular}{>{\tt}l}
      #1
    \end{tabular}
  }
}

\newcommand{\scisolC}[2]{ %
  \setcounter{cptcol}{#1}
  \longboxed{
    \begin{tabular}{>{\stepcounter{cptcol}{\tiny \underbar
            \thecptcol}}c>{\tt}l}
      #2
    \end{tabular}
  }
}

\NewEnviron{syntaxe}{ %
  % \fcolorbox{black}{Yellow!20}{\setlength{\fboxsep}{3mm}
  \shadowbox{
    \setlength{\fboxsep}{3mm}
    \begin{tabular}{>{\tt}l}
      \BODY
    \end{tabular}
  }
}

%%%%% fin macros TP Scilab %%%%%%%%
%%%%%%%%%%%%%%%%%%%%%%%%%%%%%%%%%%%

%%%%%%%%%%%%%%%%%%%%%%%%%%%%%%%%%%%
%%%%% TP Python - listings %%%%%%%%
%%%%%%%%%%%%%%%%%%%%%%%%%%%%%%%%%%%
\newcommand{\Python}{\textbf{Python}} %

\lstset{% general command to set parameter(s)
basicstyle=\ttfamily\small, % print whole listing small
keywordstyle=\color{blue}\bfseries\underbar,
%% underlined bold black keywords
frame=lines,
xleftmargin=10mm,
numbers=left,
numberstyle=\tiny\underbar,
numbersep=10pt,
%identifierstyle=, % nothing happens
commentstyle=\color{green}, % white comments
%%stringstyle=\ttfamily, % typewriter type for strings
showstringspaces=false}

\newcommand{\pysolCpt}[2]{
  \setcounter{cptcol}{#1}
  \longboxed{
    \begin{tabular}{>{\stepcounter{cptcol}{\tiny \underbar
            \thecptcol}}c>{\tt}l}
        #2
      \end{tabular}
    }
} %

\newcommand{\pysol}[1]{
  \setcounter{cptcol}{0}
  \longboxed{
    \begin{tabular}{>{\stepcounter{cptcol}{\tiny \underbar
            \thecptcol}}c>{\tt}l}
        #1
      \end{tabular}
    }
} %

% \usepackage[labelsep=endash]{caption}

% avec un caption
\NewEnviron{pythonCap}[1]{ %
  \renewcommand{\tablename}{Programme}
  \setcounter{cptcol}{0}
  \begin{center}
    \longboxed{
      \begin{tabular}{>{\stepcounter{cptcol}{\tiny \underbar
              \thecptcol}}c>{\tt}l}
        \BODY
      \end{tabular}
    }
    \captionof{table}{#1}
  \end{center}
}

\NewEnviron{python}{ %
  \setcounter{cptcol}{0}
  \begin{center}
    \longboxed{
      \begin{tabular}{>{\stepcounter{cptcol}{\tiny \underbar
              \thecptcol}}c>{\tt}l}
        \BODY
      \end{tabular}
    }
  \end{center}
}

\newcommand{\pyVar}[1]{\textcolor{var}{\bf \small #1}} %
\newcommand{\pyFun}[1]{\textcolor{pyfunction}{#1}} %
\newcommand{\pyCle}[1]{\textcolor{pyCle}{#1}} %
\newcommand{\pyImp}[1]{{\bf #1}} %

%%%%% commentaire python %%%%
\newcommand{\pyComDL}[1]{\multicolumn{1}{l}{\textcolor{pycomment}{\#
      #1}}}

\newcommand{\pyCom}[1]{{\textcolor{pycomment}{\# #1}}}
\newcommand{\pyDoc}[1]{{\textcolor{pydoc}{#1}}}

\newcommand{\pyNo}[1]{{\small \underbar #1}}

%%%%%%%%%%%%%%%%%%%%%%%%%%%%%%%%%%%
%%%%%% Système linéaire paramétré : écrire les opérations au-dessus
%%%%%% d'un symbole équivalent
%%%%%%%%%%%%%%%%%%%%%%%%%%%%%%%%%%%

\usepackage{systeme}

\NewEnviron{arrayEq}{ %
  \stackrel{\scalebox{.6}{$
      \begin{array}{l} 
        \BODY \\[.1cm]
      \end{array}$}
  }{\Longleftrightarrow}
}

\NewEnviron{arrayEg}{ %
  \stackrel{\scalebox{.6}{$
      \begin{array}{l} 
        \BODY \\[.1cm]
      \end{array}$}
  }{=}
}

\NewEnviron{operationEq}{ %
  \scalebox{.6}{$
    \begin{array}{l} 
      \scalebox{1.6}{$\mbox{Opérations :}$} \\[.2cm]
      \BODY \\[.1cm]
    \end{array}$}
}

% \NewEnviron{arraySys}[1]{ %
%   \sysdelim\{.\systeme[#1]{ %
%     \BODY %
%   } %
% }

%%%%%

%%%%%%%%%%
%%%%%%%%%% ESSAI
\newlength\fboxseph
\newlength\fboxsepva
\newlength\fboxsepvb

\setlength\fboxsepva{0.2cm}
\setlength\fboxsepvb{0.2cm}
\setlength\fboxseph{0.2cm}

\makeatletter

\def\longboxed#1{\leavevmode\setbox\@tempboxa\hbox{\color@begingroup%
\kern\fboxseph{\m@th$\displaystyle #1 $}\kern\fboxseph%
\color@endgroup }\my@frameb@x\relax}

\def\my@frameb@x#1{%
  \@tempdima\fboxrule \advance\@tempdima \fboxsepva \advance\@tempdima
  \dp\@tempboxa\hbox {%
    \lower \@tempdima \hbox {%
      \vbox {\hrule\@height\fboxrule \hbox{\vrule\@width\fboxrule #1
          \vbox{%
            \vskip\fboxsepva \box\@tempboxa \vskip\fboxsepvb}#1
          \vrule\@width\fboxrule }%
        \hrule \@height \fboxrule }}}}

\newcommand{\boxedhv}[3]{\setlength\fboxseph{#1cm}
  \setlength\fboxsepva{#2cm}\setlength\fboxsepvb{#2cm}\longboxed{#3}}

\newcommand{\boxedhvv}[4]{\setlength\fboxseph{#1cm}
  \setlength\fboxsepva{#2cm}\setlength\fboxsepvb{#3cm}\longboxed{#4}}

\newcommand{\Boxed}[1]{{\setlength\fboxseph{0.2cm}
  \setlength\fboxsepva{0.2cm}\setlength\fboxsepvb{0.2cm}\longboxed{#1}}}

\newcommand{\mBoxed}[1]{{\setlength\fboxseph{0.2cm}
  \setlength\fboxsepva{0.2cm}\setlength\fboxsepvb{0.2cm}\longboxed{\mbox{#1}}}}

\newcommand{\mboxed}[1]{{\setlength\fboxseph{0.2cm}
  \setlength\fboxsepva{0.2cm}\setlength\fboxsepvb{0.2cm}\boxed{\mbox{#1}}}}

\newsavebox{\fmbox}
\newenvironment{fmpage}[1]
     {\begin{lrbox}{\fmbox}\begin{minipage}{#1}}
     {\end{minipage}\end{lrbox}\fbox{\usebox{\fmbox}}}

%%%%%%%%%%
%%%%%%%%%%

\DeclareMathOperator{\ch}{ch}
\DeclareMathOperator{\sh}{sh}

%%%%%%%%%%
%%%%%%%%%%

\newcommand{\norme}[1]{\Vert #1 \Vert}

%\newcommand*\widefbox[1]{\fbox{\hspace{2em}#1\hspace{2em}}}

\newcommand{\nl}{\tabularnewline}

\newcommand{\hand}{\noindent\ding{43}\ }
\newcommand{\ie}{\textit{i.e. }}
\newcommand{\cf}{\textit{cf }}

\newcommand{\Card}{\operatorname{Card}}

\newcommand{\aire}{\mathcal{A}}

\newcommand{\LL}[1]{\mathscr{L}(#1)} %
\newcommand{\B}{\mathscr{B}} %
\newcommand{\Bc}[1]{B_{#1}} %
\newcommand{\M}[1]{\mathscr{M}_{#1}(\mathbb{R})}

\DeclareMathOperator{\im}{Im}
\DeclareMathOperator{\kr}{Ker}
\DeclareMathOperator{\rg}{rg}
\DeclareMathOperator{\spc}{Sp}
\DeclareMathOperator{\sgn}{sgn}
\DeclareMathOperator{\supp}{Supp}

\newcommand{\Mat}{{\rm{Mat}}}
\newcommand{\Vect}[1]{{\rm{Vect}}\left(#1\right)}

\newenvironment{smatrix}{%
  \begin{adjustbox}{width=.9\width}
    $
    \begin{pmatrix}
    }{%      
    \end{pmatrix}
    $
  \end{adjustbox}
}

\newenvironment{sarray}[1]{%
  \begin{adjustbox}{width=.9\width}
    $
    \begin{array}{#1}
    }{%      
    \end{array}
    $
  \end{adjustbox}
}

\newcommand{\vd}[2]{
  \scalebox{.8}{
    $\left(\!
      \begin{array}{c}
        #1 \\
        #2
      \end{array}
    \!\right)$
    }}

\newcommand{\vt}[3]{
  \scalebox{.8}{
    $\left(\!
      \begin{array}{c}
        #1 \\
        #2 \\
        #3 
      \end{array}
    \!\right)$
    }}

\newcommand{\vq}[4]{
  \scalebox{.8}{
    $\left(\!
      \begin{array}{c}
        #1 \\
        #2 \\
        #3 \\
        #4 
      \end{array}
    \!\right)$
    }}

\newcommand{\vc}[5]{
  \scalebox{.8}{
    $\left(\!
      \begin{array}{c}
        #1 \\
        #2 \\
        #3 \\
        #4 \\
        #5 
      \end{array}
    \!\right)$
    }}

\newcommand{\ee}{\text{e}}

\newcommand{\dd}{\text{d}}

%%% Ensemble de définition
\newcommand{\Df}{\mathscr{D}}
\newcommand{\Cf}{\mathscr{C}}
\newcommand{\Ef}{\mathscr{C}}

\newcommand{\rond}[1]{\,\overset{\scriptscriptstyle \circ}{\!#1}}

\newcommand{\df}[2]{\dfrac{\partial #1}{\partial #2}} %
\newcommand{\dfn}[2]{\partial_{#2}(#1)} %
\newcommand{\ddfn}[2]{\partial^2_{#2}(#1)} %
\newcommand{\ddf}[2]{\dfrac{\partial^2 #1}{\partial #2^2}} %
\newcommand{\ddfr}[3]{\dfrac{\partial^2 #1}{\partial #2 \partial
    #3}} %


\newcommand{\dlim}[1]{{\displaystyle \lim_{#1} \ }}
\newcommand{\dlimPlus}[2]{
  \dlim{
    \scalebox{.6}{
      $
      \begin{array}{l}
        #1 \rightarrow #2\\
        #1 > #2
      \end{array}
      $}}}
\newcommand{\dlimMoins}[2]{
  \dlim{
    \scalebox{.6}{
      $
      \begin{array}{l}
        #1 \rightarrow #2\\
        #1 < #2
      \end{array}
      $}}}

%%%%%%%%%%%%%%
%% petit o, développement limité
%%%%%%%%%%%%%%

\newcommand{\oo}[2]{{\underset {{\overset {#1\rightarrow #2}{}}}{o}}} %
\newcommand{\oox}[1]{{\underset {{\overset {x\rightarrow #1}{}}}{o}}} %
\newcommand{\oon}{{\underset {{\overset {n\rightarrow +\infty}{}}}{o}}} %
\newcommand{\po}[1]{{\underset {{\overset {#1}{}}}{o}}} %
\newcommand{\neqx}[1]{{\ \underset {{\overset {x \to #1}{}}}{\not\sim}\ }} %
\newcommand{\eqx}[1]{{\ \underset {{\overset {x \to #1}{}}}{\sim}\ }} %
\newcommand{\eqn}{{\ \underset {{\overset {n \to +\infty}{}}}{\sim}\ }} %
\newcommand{\eq}[2]{{\ \underset {{\overset {#1 \to #2}{}}}{\sim}\ }} %
\newcommand{\DL}[1]{{\rm{DL}}_1 (#1)} %
\newcommand{\DLL}[1]{{\rm{DL}}_2 (#1)} %

\newcommand{\negl}{<<}

\newcommand{\neglP}[1]{\begin{array}{c}
    \vspace{-.2cm}\\
    << \\
    \vspace{-.7cm}\\
    {\scriptstyle #1}
  \end{array}}

%%%%%%%%%%%%%%
%% borne sup, inf, max, min
%%%%%%%%%%%%%%
\newcommand{\dsup}[1]{\displaystyle \sup_{#1} \ }
\newcommand{\dinf}[1]{\displaystyle \inf_{#1} \ }
\newcommand{\dmax}[1]{\max\limits_{#1} \ }
\newcommand{\dmin}[1]{\min\limits_{#1} \ }

\newcommand{\dcup}[2]{{\textstyle\bigcup\limits_{#1}^{#2}}\hspace{.1cm}}
%\displaystyle \bigcup_{#1}^{#2}}
\newcommand{\dcap}[2]{{\textstyle\bigcap\limits_{#1}^{#2}}\hspace{.1cm}}
% \displaystyle \bigcap_{#1}^{#2}
%%%%%%%%%%%%%%
%% opérateurs logiques
%%%%%%%%%%%%%%
\newcommand{\NON}[1]{\mathop{\small \tt{NON}} (#1)}
\newcommand{\ET}{\mathrel{\mathop{\small \mathtt{ET}}}}
\newcommand{\OU}{\mathrel{\mathop{\small \tt{OU}}}}
\newcommand{\XOR}{\mathrel{\mathop{\small \tt{XOR}}}}

\newcommand{\id}{{\rm{id}}}

\newcommand{\sbullet}{\scriptstyle \bullet}
\newcommand{\stimes}{\scriptstyle \times}

%%%%%%%%%%%%%%%%%%
%% Probabilités
%%%%%%%%%%%%%%%%%%
\newcommand{\Prob}{\mathbb{P}}
\newcommand{\Ev}[1]{\left[ {#1} \right]}
\newcommand{\Evmb}[1]{[ {#1} ]}
\newcommand{\E}{\mathbb{E}}
\newcommand{\V}{\mathbb{V}}
\newcommand{\Cov}{{\rm{Cov}}}
\newcommand{\U}[2]{\mathcal{U}(\llb #1, #2\rrb)}
\newcommand{\Uc}[2]{\mathcal{U}([#1, #2])}
\newcommand{\Ucof}[2]{\mathcal{U}(]#1, #2])}
\newcommand{\Ucoo}[2]{\mathcal{U}(]#1, #2[)}
\newcommand{\Ucfo}[2]{\mathcal{U}([#1, #2[)}
\newcommand{\Bern}[1]{\mathcal{B}\left(#1\right)}
\newcommand{\Bin}[2]{\mathcal{B}\left(#1, #2\right)}
\newcommand{\G}[1]{\mathcal{G}\left(#1\right)}
\newcommand{\Pois}[1]{\mathcal{P}\left(#1\right)}
\newcommand{\HG}[3]{\mathcal{H}\left(#1, #2, #3\right)}
\newcommand{\Exp}[1]{\mathcal{E}\left(#1\right)}
\newcommand{\Norm}[2]{\mathcal{N}\left(#1, #2\right)}

\DeclareMathOperator{\cov}{Cov}

\newcommand{\var}{v.a.r. }
\newcommand{\suit}{\hookrightarrow}

\newcommand{\flecheR}[1]{\rotatebox{90}{\scalebox{#1}{\color{red}
      $\curvearrowleft$}}}


\newcommand{\partie}[1]{\mathcal{P}(#1)}
\newcommand{\Cont}[1]{\mathcal{C}^{#1}}
\newcommand{\Contm}[1]{\mathcal{C}^{#1}_m}

\newcommand{\llb}{\llbracket}
\newcommand{\rrb}{\rrbracket}

%\newcommand{\im}[1]{{\rm{Im}}(#1)}
\newcommand{\imrec}[1]{#1^{- \mathds{1}}}

\newcommand{\unq}{\mathds{1}}

\newcommand{\Hyp}{\mathtt{H}}

\newcommand{\eme}[1]{#1^{\scriptsize \mbox{ème}}}
\newcommand{\er}[1]{#1^{\scriptsize \mbox{er}}}
\newcommand{\ere}[1]{#1^{\scriptsize \mbox{ère}}}
\newcommand{\nd}[1]{#1^{\scriptsize \mbox{nd}}}
\newcommand{\nde}[1]{#1^{\scriptsize \mbox{nde}}}

\newcommand{\truc}{\mathop{\top}}
\newcommand{\fois}{\mathop{\ast}}

\newcommand{\f}[1]{\overrightarrow{#1}}

\newcommand{\checked}{\textcolor{green}{\checkmark}}

\def\restriction#1#2{\mathchoice
              {\setbox1\hbox{${\displaystyle #1}_{\scriptstyle #2}$}
              \restrictionaux{#1}{#2}}
              {\setbox1\hbox{${\textstyle #1}_{\scriptstyle #2}$}
              \restrictionaux{#1}{#2}}
              {\setbox1\hbox{${\scriptstyle #1}_{\scriptscriptstyle #2}$}
              \restrictionaux{#1}{#2}}
              {\setbox1\hbox{${\scriptscriptstyle #1}_{\scriptscriptstyle #2}$}
              \restrictionaux{#1}{#2}}}
\def\restrictionaux#1#2{{#1\,\smash{\vrule height .8\ht1 depth .85\dp1}}_{\,#2}}

\makeatletter
\newcommand*{\da@rightarrow}{\mathchar"0\hexnumber@\symAMSa 4B }
\newcommand*{\da@leftarrow}{\mathchar"0\hexnumber@\symAMSa 4C }
\newcommand*{\xdashrightarrow}[2][]{%
  \mathrel{%
    \mathpalette{\da@xarrow{#1}{#2}{}\da@rightarrow{\,}{}}{}%
  }%
}
\newcommand{\xdashleftarrow}[2][]{%
  \mathrel{%
    \mathpalette{\da@xarrow{#1}{#2}\da@leftarrow{}{}{\,}}{}%
  }%
}
\newcommand*{\da@xarrow}[7]{%
  % #1: below
  % #2: above
  % #3: arrow left
  % #4: arrow right
  % #5: space left 
  % #6: space right
  % #7: math style 
  \sbox0{$\ifx#7\scriptstyle\scriptscriptstyle\else\scriptstyle\fi#5#1#6\m@th$}%
  \sbox2{$\ifx#7\scriptstyle\scriptscriptstyle\else\scriptstyle\fi#5#2#6\m@th$}%
  \sbox4{$#7\dabar@\m@th$}%
  \dimen@=\wd0 %
  \ifdim\wd2 >\dimen@
    \dimen@=\wd2 %   
  \fi
  \count@=2 %
  \def\da@bars{\dabar@\dabar@}%
  \@whiledim\count@\wd4<\dimen@\do{%
    \advance\count@\@ne
    \expandafter\def\expandafter\da@bars\expandafter{%
      \da@bars
      \dabar@ 
    }%
  }%  
  \mathrel{#3}%
  \mathrel{%   
    \mathop{\da@bars}\limits
    \ifx\\#1\\%
    \else
      _{\copy0}%
    \fi
    \ifx\\#2\\%
    \else
      ^{\copy2}%
    \fi
  }%   
  \mathrel{#4}%
}
\makeatother



\newcount\depth

\newcount\depth
\newcount\totaldepth

\makeatletter
\newcommand{\labelsymbol}{%
      \ifnum\depth=0
        %
      \else
        \rlap{\,$\bullet$}%
      \fi
}

\newcommand*\bernoulliTree[1]{%
    \depth=#1\relax            
    \totaldepth=#1\relax
    \draw node(root)[bernoulli/root] {\labelsymbol}[grow=right] \draw@bernoulli@tree;
    \draw \label@bernoulli@tree{root};                                   
}                                                                        

\def\draw@bernoulli@tree{%
    \ifnum\depth>0 
      child[parent anchor=east] foreach \type/\label in {left child/$E$,right child/$S$} {%
          node[bernoulli/\type] {\label\strut\labelsymbol} \draw@bernoulli@tree
      }
      coordinate[bernoulli/increment] (dummy)
   \fi%
}

\def\label@bernoulli@tree#1{%
    \ifnum\depth>0
      ($(#1)!0.5!(#1-1)$) node[fill=white,bernoulli/decrement] {\tiny$p$}
      \label@bernoulli@tree{#1-1}
      ($(#1)!0.5!(#1-2)$) node[fill=white] {\tiny$q$}
      \label@bernoulli@tree{#1-2}
      coordinate[bernoulli/increment] (dummy)
   \fi%
}

\makeatother

\tikzset{bernoulli/.cd,
         root/.style={},
         decrement/.code=\global\advance\depth by-1\relax,
         increment/.code=\global\advance\depth by 1\relax,
         left child/.style={bernoulli/decrement},
         right child/.style={}}


\newcommand{\eps}{\varepsilon}

% \newcommand{\tendi}[1]{\xrightarrow[\footnotesize #1 \rightarrow
%   +\infty]{}}%

\newcommand{\tend}{\rightarrow}%
\newcommand{\tendn}{\underset{n\to +\infty}{\longrightarrow}} %
\newcommand{\ntendn}{\underset{n\to
    +\infty}{\not\hspace{-.15cm}\longrightarrow}} %
% \newcommand{\tendn}{\xrightarrow[\footnotesize n \rightarrow
%   +\infty]{}}%
\newcommand{\Tendx}[1]{\xrightarrow[\footnotesize x \rightarrow
  #1]{}}%
\newcommand{\tendx}[1]{\underset{x\to #1}{\longrightarrow}}%
\newcommand{\ntendx}[1]{\underset{x\to #1}{\not\!\!\longrightarrow}}%
\newcommand{\tendd}[2]{\underset{#1\to #2}{\longrightarrow}}%
% \newcommand{\tendd}[2]{\xrightarrow[\footnotesize #1 \rightarrow
%   #2]{}}%
\newcommand{\tendash}[1]{\xdashrightarrow[\footnotesize #1 \rightarrow
  +\infty]{}}%
\newcommand{\tendashx}[1]{\xdashrightarrow[\footnotesize x \rightarrow
  #1]{}}%
\newcommand{\tendb}[1]{\underset{#1 \to +\infty}{\longrightarrow}}%
\newcommand{\tendL}{\overset{\mathscr L}{\underset{n \to
      +\infty}{\longrightarrow}}}%
\newcommand{\tendP}{\overset{\Prob}{\underset{n \to
      +\infty}{\longrightarrow}}}%
\newcommand{\tenddL}[1]{\overset{\mathscr L}{\underset{#1 \to
      +\infty}{\longrightarrow}}}%

\NewEnviron{attention}{ %
  ~\\[-.2cm]\noindent
  \begin{minipage}{\linewidth}
  \setlength{\fboxsep}{3mm}%
  \ \ \dbend \ \ %
  \fbox{\parbox[t]{.88\linewidth}{\BODY}} %
  \end{minipage}\\
}

\NewEnviron{sattention}[1]{ %
  ~\\[-.2cm]\noindent
  \begin{minipage}{#1\linewidth}
  \setlength{\fboxsep}{3mm}%
  \ \ \dbend \ \ %
  \fbox{\parbox[t]{.88\linewidth}{\BODY}} %
  \end{minipage}\\
}

%%%%% OBSOLETE %%%%%%

% \newcommand{\attention}[1]{
%   \noindent
%   \begin{tabular}{@{}l|p{11.5cm}|}
%     \cline{2-2}
%     \vspace{-.2cm} 
%     & \nl
%     \dbend & #1 \nl
%     \cline{2-2}
%   \end{tabular}
% }

% \newcommand{\attentionv}[2]{
%   \noindent
%   \begin{tabular}{@{}l|p{11.5cm}|}
%     \cline{2-2}
%     \vspace{-.2cm} 
%     & \nl
%     \dbend & #2 \nl[#1 cm]
%     \cline{2-2}
%   \end{tabular}
% }

\newcommand{\explainvb}[2]{
  \noindent
  \begin{tabular}{@{}l|p{11.5cm}|}
    \cline{2-2}
    \vspace{-.2cm} 
    & \nl
    \hand & #2 \nl [#1 cm]
    \cline{2-2}
  \end{tabular}
}


% \noindent
% \begin{tabular}{@{}l|lp{11cm}|}
%   \cline{3-3} 
%   \multicolumn{1}{@{}l@{\dbend}}{} & & #1 \nl
%   \multicolumn{1}{l}{} & & \nl [-.8cm]
%   & & #2 \nl
%   \cline{2-3}
% \end{tabular}

% \newcommand{\attention}[1]{
%   \noindent
%   \begin{tabular}{@{}@{}cp{11cm}}
%     \dbend & #1 \nl
%   \end{tabular}
% }

\newcommand{\PP}[1]{\mathcal{P}(#1)}
\newcommand{\HH}[1]{\mathcal{H}(#1)}
\newcommand{\FF}[1]{\mathcal{F}(#1)}

\newcommand{\DSum}[2]{\displaystyle\sum\limits_{#1}^{#2}\hspace{.1cm}}
\newcommand{\Sum}[2]{{\textstyle\sum\limits_{#1}^{#2}}\hspace{.1cm}}
\newcommand{\Serie}{\textstyle\sum\hspace{.1cm}}
\newcommand{\Prod}[2]{\textstyle\prod\limits_{#1}^{#2}}

\newcommand{\Prim}[3]{\left[\ {#1} \ \right]_{\scriptscriptstyle
   \hspace{-.15cm} ~_{#2}\, }^ {\scriptscriptstyle \hspace{-.15cm} ~^{#3}\, }}

% \newcommand{\Prim}[3]{\left[\ {#1} \ \right]_{\scriptscriptstyle
%     \!\!~_{#2}}^ {\scriptscriptstyle \!\!~^{#3}}}

\newcommand{\dint}[2]{\displaystyle \int_{#1}^{#2}\ }
\newcommand{\Int}[2]{{\rm{Int}}_{\scriptscriptstyle #1, #2}}
\newcommand{\dt}{\ dt}
\newcommand{\dx}{\ dx}

\newcommand{\llpar}[1]{\left(\!\!\!
    \begin{array}{c}
      \rule{0pt}{#1}
    \end{array}
  \!\!\!\right.}

\newcommand{\rrpar}[1]{\left.\!\!\!
    \begin{array}{c}
      \rule{0pt}{#1}
    \end{array}
  \!\!\!\right)}

\newcommand{\llacc}[1]{\left\{\!\!\!
    \begin{array}{c}
      \rule{0pt}{#1}
    \end{array}
  \!\!\!\right.}

\newcommand{\rracc}[1]{\left.\!\!\!
    \begin{array}{c}
      \rule{0pt}{#1}
    \end{array}
  \!\!\!\right\}}

\newcommand{\ttacc}[1]{\mbox{\rotatebox{-90}{\hspace{-.7cm}$\llacc{#1}$}}}
\newcommand{\bbacc}[1]{\mbox{\rotatebox{90}{\hspace{-.5cm}$\llacc{#1}$}}}

\newcommand{\comp}[1]{\overline{#1}}%

\newcommand{\dcomp}[2]{\stackrel{\mbox{\ \ \----}{\scriptscriptstyle
      #2}}{#1}}%

% \newcommand{\Comp}[2]{\stackrel{\mbox{\ \
%       \-------}{\scriptscriptstyle #2}}{#1}}

% \newcommand{\dcomp}[2]{\stackrel{\mbox{\ \
%       \-------}{\scriptscriptstyle #2}}{#1}}

\newcommand{\A}{\mathscr{A}}

\newcommand{\conc}[1]{
  \begin{center}
    \fbox{
      \begin{tabular}{c}
        #1
      \end{tabular}
    }
  \end{center}
}

\newcommand{\concC}[1]{
  \begin{center}
    \fbox{
    \begin{tabular}{C{10cm}}
      \quad #1 \quad
    \end{tabular}
    }
  \end{center}
}

\newcommand{\concL}[2]{
  \begin{center}
    \fbox{
    \begin{tabular}{C{#2cm}}
      \quad #1 \quad
    \end{tabular}
    }
  \end{center}
}


% \newcommand{\lims}[2]{\prod\limits_{#1}^{#2}}

\newtheorem{theorem}{Théorème}[]
\newtheorem{lemma}{Lemme}[]
\newtheorem{proposition}{Proposition}[]
\newtheorem{corollary}{Corollaire}[]

% \newenvironment{proof}[1][Démonstration]{\begin{trivlist}
% \item[\hskip \labelsep {\bfseries #1}]}{\end{trivlist}}
\newenvironment{definition}[1][Définition]{\begin{trivlist}
\item[\hskip \labelsep {\bfseries #1}]}{\end{trivlist}}
\newenvironment{example}[1][Exemple]{\begin{trivlist}
\item[\hskip \labelsep {\bfseries #1}]}{\end{trivlist}}
\newenvironment{examples}[1][Exemples]{\begin{trivlist}
\item[\hskip \labelsep {\bfseries #1}]}{\end{trivlist}}
\newenvironment{notation}[1][Notation]{\begin{trivlist}
\item[\hskip \labelsep {\bfseries #1}]}{\end{trivlist}}
\newenvironment{propriete}[1][Propriété]{\begin{trivlist}
\item[\hskip \labelsep {\bfseries #1}]}{\end{trivlist}}
\newenvironment{proprietes}[1][Propriétés]{\begin{trivlist}
\item[\hskip \labelsep {\bfseries #1}]}{\end{trivlist}}
% \newenvironment{remark}[1][Remarque]{\begin{trivlist}
% \item[\hskip \labelsep {\bfseries #1}]}{\end{trivlist}}
\newenvironment{application}[1][Application]{\begin{trivlist}
\item[\hskip \labelsep {\bfseries #1}]}{\end{trivlist}}

% Environnement pour les réponses des DS
\newenvironment{answer}{\par\emph{Réponse :}\par{}}
{\vspace{-.6cm}\hspace{\stretch{1}}\rule{1ex}{1ex}\vspace{.3cm}}

\newenvironment{answerTD}{\vspace{.2cm}\par\emph{Réponse :}\par{}}
{\hspace{\stretch{1}}\rule{1ex}{1ex}\vspace{.3cm}}

\newenvironment{answerCours}{\noindent\emph{Réponse :}}
{\rule{1ex}{1ex}}%\vspace{.3cm}}


% footnote in footer
\newcommand{\fancyfootnotetext}[2]{%
  \fancypagestyle{dingens}{%
    \fancyfoot[LO,RE]{\parbox{0.95\textwidth}{\footnotemark[#1]\footnotesize
        #2}}%
  }%
  \thispagestyle{dingens}%
}

%%% définit le style (arabic : 1,2,3...) et place des parenthèses
%%% autour de la numérotation
\renewcommand*{\thefootnote}{(\arabic{footnote})}
% http://www.tuteurs.ens.fr/logiciels/latex/footnote.html

%%%%%%%% tikz axis
% \pgfplotsset{every axis/.append style={
%                     axis x line=middle,    % put the x axis in the middle
%                     axis y line=middle,    % put the y axis in the middle
%                     axis line style={<->,color=blue}, % arrows on the axis
%                     xlabel={$x$},          % default put x on x-axis
%                     ylabel={$y$},          % default put y on y-axis
%             }}

%%%% s'utilise comme suit

% \begin{axis}[
%   xmin=-8,xmax=4,
%   ymin=-8,ymax=4,
%   grid=both,
%   ]
%   \addplot [domain=-3:3,samples=50]({x^3-3*x},{3*x^2-9}); 
% \end{axis}

%%%%%%%%



%%%%%%%%%%%% Pour avoir des numéros de section qui correspondent à
%%%%%%%%%%%% ceux du tableau
\renewcommand{\thesection}{\Roman{section}.\hspace{-.3cm}}
\renewcommand{\thesubsection}{\Roman{section}.\arabic{subsection}.\hspace{-.2cm}}
\renewcommand{\thesubsubsection}{\Roman{section}.\arabic{subsection}.\alph{subsubsection})\hspace{-.2cm}}
%%%%%%%%%%%% 

%%% Changer le nom des figures : Fig. au lieu de Figure
\usepackage[font=small,labelfont=bf,labelsep=space]{caption}
\captionsetup{%
  figurename=Fig.,
  tablename=Tab.
}
% \renewcommand{\thesection}{\Roman{section}.\hspace{-.2cm}}
% \renewcommand{\thesubsection}{\Roman{section}
%   .\hspace{.2cm}\arabic{subsection}\ .\hspace{-.3cm}}
% \renewcommand{\thesubsubsection}{\alph{subsection})}

\newenvironment{tabliste}[1]
{\begin{tabenum}[\bfseries\small\itshape #1]}{\end{tabenum}} 

%%%% ESSAI contre le too deeply nested %%%%
%%%% ATTENTION au package enumitem qui se comporte mal avec les
%%%% noliste, à redéfinir !
% \usepackage{enumitem}

% \setlistdepth{9}

% \newlist{myEnumerate}{enumerate}{9}
% \setlist[myEnumerate,1]{label=(\arabic*)}
% \setlist[myEnumerate,2]{label=(\Roman*)}
% \setlist[myEnumerate,3]{label=(\Alph*)}
% \setlist[myEnumerate,4]{label=(\roman*)}
% \setlist[myEnumerate,5]{label=(\alph*)}
% \setlist[myEnumerate,6]{label=(\arabic*)}
% \setlist[myEnumerate,7]{label=(\Roman*)}
% \setlist[myEnumerate,8]{label=(\Alph*)}
% \setlist[myEnumerate,9]{label=(\roman*)}

%%%%%

\newenvironment{noliste}[1] %
{\begin{enumerate}[\bfseries\small\itshape #1]} %
  {\end{enumerate}}

\newenvironment{nonoliste}[1] %
{\begin{enumerate}[\hspace{-12pt}\bfseries\small\itshape #1]} %
  {\end{enumerate}}

\newenvironment{arrayliste}[1]{ 
  % List with minimal white space to fit in small areas, e.g. table
  % cell
  \begin{minipage}[t]{\linewidth} %
    \begin{enumerate}[\bfseries\small\itshape #1] %
      {\leftmargin=0.5em \rightmargin=0em
        \topsep=0em \parskip=0em \parsep=0em
        \listparindent=0em \partopsep=0em \itemsep=0pt
        \itemindent=0em \labelwidth=\leftmargin\labelsep+0.25em}
    }{
    \end{enumerate}\end{minipage}
}

\newenvironment{nolistes}[2]
{\begin{enumerate}[\bfseries\small\itshape
    #1]\setlength{\itemsep}{#2 mm}}{\end{enumerate}}

\newenvironment{liste}[1]
{\begin{enumerate}[\hspace{12pt}\bfseries\small\itshape
    #1]}{\end{enumerate}}   


%%%%%%%% Pour les programmes de colle %%%%%%%

\newcommand{\cours}{{\small \tt (COURS)}} %
\newcommand{\poly}{{\small \tt (POLY)}} %
\newcommand{\exo}{{\small \tt (EXO)}} %
\newcommand{\culture}{{\small \tt (CULTURE)}} %
\newcommand{\methodo}{{\small \tt (MÉTHODO)}} %
\newcommand{\methodob}{\Boxed{\mbox{\tt MÉTHODO}}} %

%%%%%%%% Pour les TD %%%%%%%
\newtheoremstyle{exostyle} {\topsep} % espace avant
{.6cm} % espace apres
{} % Police utilisee par le style de thm
{} % Indentation (vide = aucune, \parindent = indentation paragraphe)
{\bfseries} % Police du titre de thm
{} % Signe de ponctuation apres le titre du thm
{ } % Espace apres le titre du thm (\newline = linebreak)
{\thmname{#1}\thmnumber{ #2}\thmnote{.
    \normalfont{\textit{#3}}}} % composants du titre du thm : \thmname
                               % = nom du thm, \thmnumber = numéro du
                               % thm, \thmnote = sous-titre du thm
 
\theoremstyle{exostyle}
\newtheorem{exercice}{Exercice}
\newtheorem*{exoCours}{Exercice}

%%%%%%%% Pour des théorèmes Sans Espaces APRÈS %%%%%%%
\newtheoremstyle{exostyleSE} {\topsep} % espace avant
{} % espace apres
{} % Police utilisee par le style de thm
{} % Indentation (vide = aucune, \parindent = indentation paragraphe)
{\bfseries} % Police du titre de thm
{} % Signe de ponctuation apres le titre du thm
{ } % Espace apres le titre du thm (\newline = linebreak)
{\thmname{#1}\thmnumber{ #2}\thmnote{.
    \normalfont{\textit{#3}}}} % composants du titre du thm : \thmname
                               % = nom du thm, \thmnumber = numéro du
                               % thm, \thmnote = sous-titre du thm
 
\theoremstyle{exostyleSE}
\newtheorem{exerciceSE}{Exercice}
\newtheorem*{exoCoursSE}{Exercice}

% \newcommand{\lims}[2]{\prod\limits_{#1}^{#2}}

\newtheorem{theoremSE}{Théorème}[]
\newtheorem{lemmaSE}{Lemme}[]
\newtheorem{propositionSE}{Proposition}[]
\newtheorem{corollarySE}{Corollaire}[]

% \newenvironment{proofSE}[1][Démonstration]{\begin{trivlist}
% \item[\hskip \labelsep {\bfseries #1}]}{\end{trivlist}}
\newenvironment{definitionSE}[1][Définition]{\begin{trivlist}
  \item[\hskip \labelsep {\bfseries #1}]}{\end{trivlist}}
\newenvironment{exampleSE}[1][Exemple]{\begin{trivlist} 
  \item[\hskip \labelsep {\bfseries #1}]}{\end{trivlist}}
\newenvironment{examplesSE}[1][Exemples]{\begin{trivlist}
\item[\hskip \labelsep {\bfseries #1}]}{\end{trivlist}}
\newenvironment{notationSE}[1][Notation]{\begin{trivlist}
\item[\hskip \labelsep {\bfseries #1}]}{\end{trivlist}}
\newenvironment{proprieteSE}[1][Propriété]{\begin{trivlist}
\item[\hskip \labelsep {\bfseries #1}]}{\end{trivlist}}
\newenvironment{proprietesSE}[1][Propriétés]{\begin{trivlist}
\item[\hskip \labelsep {\bfseries #1}]}{\end{trivlist}}
\newenvironment{remarkSE}[1][Remarque]{\begin{trivlist}
\item[\hskip \labelsep {\bfseries #1}]}{\end{trivlist}}
\newenvironment{applicationSE}[1][Application]{\begin{trivlist}
\item[\hskip \labelsep {\bfseries #1}]}{\end{trivlist}}

%%%%%%%%%%% Obtenir les étoiles sans charger le package MnSymbol
%%%%%%%%%%%
\DeclareFontFamily{U} {MnSymbolC}{}
\DeclareFontShape{U}{MnSymbolC}{m}{n}{
  <-6> MnSymbolC5
  <6-7> MnSymbolC6
  <7-8> MnSymbolC7
  <8-9> MnSymbolC8
  <9-10> MnSymbolC9
  <10-12> MnSymbolC10
  <12-> MnSymbolC12}{}
\DeclareFontShape{U}{MnSymbolC}{b}{n}{
  <-6> MnSymbolC-Bold5
  <6-7> MnSymbolC-Bold6
  <7-8> MnSymbolC-Bold7
  <8-9> MnSymbolC-Bold8
  <9-10> MnSymbolC-Bold9
  <10-12> MnSymbolC-Bold10
  <12-> MnSymbolC-Bold12}{}

\DeclareSymbolFont{MnSyC} {U} {MnSymbolC}{m}{n}

\DeclareMathSymbol{\filledlargestar}{\mathrel}{MnSyC}{205}
\DeclareMathSymbol{\largestar}{\mathrel}{MnSyC}{131}

\newcommand{\facile}{\rm{(}$\scriptstyle\largestar$\rm{)}} %
\newcommand{\moyen}{\rm{(}$\scriptstyle\filledlargestar$\rm{)}} %
\newcommand{\dur}{\rm{(}$\scriptstyle\filledlargestar\filledlargestar$\rm{)}} %
\newcommand{\costaud}{\rm{(}$\scriptstyle\filledlargestar\filledlargestar\filledlargestar$\rm{)}}

%%%%%%%%%%%%%%%%%%%%%%%%%

%%%%%%%%%%%%%%%%%%%%%%%%%
%%%%%%%% Fin de la partie TD

%%%%%%%%%%%%%%%%
%%%%%%%%%%%%%%%%
\makeatletter %
\newenvironment{myitemize}{%
  \setlength{\topsep}{0pt} %
  \setlength{\partopsep}{0pt} %
  \renewcommand*{\@listi}{\leftmargin\leftmargini \parsep\z@
    \topsep\z@ \itemsep\z@} \let\@listI\@listi %
  \itemize %
}{\enditemize} %
\makeatother
%%%%%%%%%%%%%%%%
%%%%%%%%%%%%%%%%

%% Commentaires dans la correction du livre

\newcommand{\Com}[1]{
% Define box and box title style
\tikzstyle{mybox} = [draw=black!50,
very thick,
    rectangle, rounded corners, inner sep=10pt, inner ysep=8pt]
\tikzstyle{fancytitle} =[rounded corners, fill=black!80, text=white]
\tikzstyle{fancylogo} =[ text=white]
\begin{center}

\begin{tikzpicture}
\node [mybox] (box){%

    \begin{minipage}{0.90\linewidth}
\vspace{6pt}  #1
    \end{minipage}
};
\node[fancytitle, right=10pt] at (box.north west) 
{\bfseries\normalsize{Commentaire}};

\end{tikzpicture}%

\end{center}
%
}

\NewEnviron{remark}{%
  % Define box and box title style
  \tikzstyle{mybox} = [draw=black!50, very thick, rectangle, rounded
  corners, inner sep=10pt, inner ysep=8pt] %
  \tikzstyle{fancytitle} = [rounded corners , fill=black!80,
  text=white] %
  \tikzstyle{fancylogo} =[ text=white]
  \begin{center}
    \begin{tikzpicture}
      \node [mybox] (box){%
        \begin{minipage}{0.90\linewidth}
          \vspace{6pt} \BODY
        \end{minipage}
      }; %
      \node[fancytitle, right=10pt] at (box.north west) %
      {\bfseries\normalsize{Commentaire}}; %
    \end{tikzpicture}%
  \end{center}
}

\NewEnviron{remarkST}{%
  % Define box and box title style
  \tikzstyle{mybox} = [draw=black!50, very thick, rectangle, rounded
  corners, inner sep=10pt, inner ysep=8pt] %
  \tikzstyle{fancytitle} = [rounded corners , fill=black!80,
  text=white] %
  \tikzstyle{fancylogo} =[ text=white]
  \begin{center}
    \begin{tikzpicture}
      \node [mybox] (box){%
        \begin{minipage}{0.90\linewidth}
          \vspace{6pt} \BODY
        \end{minipage}
      }; %
      % \node[fancytitle, right=10pt] at (box.north west) %
%       {\bfseries\normalsize{Commentaire}}; %
    \end{tikzpicture}%
  \end{center}
}

\NewEnviron{remarkL}[1]{%
  % Define box and box title style
  \tikzstyle{mybox} = [draw=black!50, very thick, rectangle, rounded
  corners, inner sep=10pt, inner ysep=8pt] %
  \tikzstyle{fancytitle} =[rounded corners, fill=black!80,
  text=white] %
  \tikzstyle{fancylogo} =[ text=white]
  \begin{center}
    \begin{tikzpicture}
      \node [mybox] (box){%
        \begin{minipage}{#1\linewidth}
          \vspace{6pt} \BODY
        \end{minipage}
      }; %
      \node[fancytitle, right=10pt] at (box.north west) %
      {\bfseries\normalsize{Commentaire}}; %
    \end{tikzpicture}%
  \end{center}
}

\NewEnviron{remarkSTL}[1]{%
  % Define box and box title style
  \tikzstyle{mybox} = [draw=black!50, very thick, rectangle, rounded
  corners, inner sep=10pt, inner ysep=8pt] %
  \tikzstyle{fancytitle} =[rounded corners, fill=black!80,
  text=white] %
  \tikzstyle{fancylogo} =[ text=white]
  \begin{center}
    \begin{tikzpicture}
      \node [mybox] (box){%
        \begin{minipage}{#1\linewidth}
          \vspace{6pt} \BODY
        \end{minipage}
      }; %
%       \node[fancytitle, right=10pt] at (box.north west) %
%       {\bfseries\normalsize{Commentaire}}; %
    \end{tikzpicture}%
  \end{center}
}

\NewEnviron{titre} %
{ %
  ~\\[-1.8cm]
  \begin{center}
    \bf \LARGE \BODY
  \end{center}
  ~\\[-.6cm]
  \hrule %
  \vspace*{.2cm}
} %

\NewEnviron{titreL}[2] %
{ %
  ~\\[-#1cm]
  \begin{center}
    \bf \LARGE \BODY
  \end{center}
  ~\\[-#2cm]
  \hrule %
  \vspace*{.2cm}
} %



%%%%%%%%%%% Redefinition \chapter



\usepackage[explicit]{titlesec}
\usepackage{color}
\titleformat{\chapter}
{\gdef\chapterlabel{}
\selectfont\huge\bf}
%\normalfont\sffamily\Huge\bfseries\scshape}
{\gdef\chapterlabel{\thechapter)\ }}{0pt}
{\begin{tikzpicture}[remember picture,overlay]
\node[yshift=-3cm] at (current page.north west)
{\begin{tikzpicture}[remember picture, overlay]
\draw (.1\paperwidth,0) -- (.9\paperwidth,0);
\draw (.1\paperwidth,2) -- (.9\paperwidth,2);
%(\paperwidth,3cm);
\node[anchor=center,xshift=.5\paperwidth,yshift=1cm, rectangle,
rounded corners=20pt,inner sep=11pt]
{\color{black}\chapterlabel#1};
\end{tikzpicture}
};
\end{tikzpicture}
}
\titlespacing*{\chapter}{0pt}{50pt}{-75pt}




%%%%%%%%%%%%%% Affichage chapter dans Table des matieres

\makeatletter
\renewcommand*\l@chapter[2]{%
  \ifnum \c@tocdepth >\m@ne
    \addpenalty{-\@highpenalty}%
    \vskip 1.0em \@plus\p@
    \setlength\@tempdima{1.5em}%
    \begingroup
      \parindent \z@ \rightskip \@pnumwidth
      \parfillskip -\@pnumwidth
      \leavevmode %\bfseries
      \advance\leftskip\@tempdima
      \hskip -\leftskip
      #1\nobreak\ 
       \leaders\hbox{$\m@th
        \mkern \@dotsep mu\hbox{.}\mkern \@dotsep
        mu$}\hfil\nobreak\hb@xt@\@pnumwidth{\hss #2}\par
      \penalty\@highpenalty
    \endgroup
  \fi}
\makeatother




%%%%%%%%%%%%%%%%% Redefinition part




% \renewcommand{\thesection}{\Roman{section}.\hspace{-.3cm}}
% \renewcommand{\thesubsection}{\Alph{subsection}.\hspace{-.2cm}}

\pagestyle{fancy} %
\lhead{ECE2 \hfill Mathématiques \\} %
\chead{\hrule} %
\rhead{} %
\lfoot{} %
\cfoot{} %
\rfoot{\thepage} %

% \widowpenalty=10000
% \clubpenalty=10000

\renewcommand{\headrulewidth}{0pt}% : Trace un trait de séparation
                                    % de largeur 0,4 point. Mettre 0pt
                                    % pour supprimer le trait.

\renewcommand{\footrulewidth}{0.4pt}% : Trace un trait de séparation
                                    % de largeur 0,4 point. Mettre 0pt
                                    % pour supprimer le trait.

\setlength{\headheight}{14pt}

\title{\bf \vspace{-1.6cm} ESSEC II 2018} %
\author{} %
\date{} %
\begin{document}

\maketitle %
\vspace{-1.2cm}\hrule %
\thispagestyle{fancy}

\vspace*{.4cm}

%%DEBUT

\noindent
On s'intéresse à l'évolution d'une population de petits organismes 
(typiquement des insectes) pendant une \og saison \fg{} reproductrice 
de durée maximale $T$ où $T \in \N^*$. Les insectes sont supposés vivre 
une unité de temps, au bout de laquelle ils meurent en pondant un 
certain nombre d'{\oe}ufs. Au moment du dépôt d'un {\oe}uf, un 
processus chimique, la diapause, est susceptible de se mettre en marche 
qui entraîne l'arrêt de maturation de l'{\oe}uf jusqu'à la saison 
suivante. Ainsi, à chaque $t$ de la saison, une génération d'insectes 
s'éteint, en déposant des {\oe}ufs. Immédiatement, une proportion 
$p(t)$ de ces {\oe}ufs se mettent en diapause. Les {\oe}ufs qui ne sont 
pas entrés en diapause éclosent avant la date $t+1$, donnant naissance 
à une nouvelle génération d'insectes, qui s'éteindra à la date $t+1$ en 
déposant des {\oe}ufs, etc. Comme, à la fin de la saison, tous les 
organismes vivants de la population meurent, hormis les {\oe}ufs qui 
sont en diapause, ce sont ces derniers qui seront à l'origine d'une 
nouvelle population qui éclora à la saison suivante. Il est donc 
fondamental pour la survie de la lignée que les organismes adoptent une 
stratégie maximisant le nombre d'{\oe}ufs en diapause accumulés jusqu'à 
la date où la saison s'achève.\\[.1cm]
Au cours du problème, on s'intéressera plus particulièrement au cas où 
la durée de la saison est une variable aléatoire $\tau$ pouvant prendre 
des valeurs entières entre $1$ et $T$. Pour $t\in \{1,2, \ldots, T\}$, 
l'événement $\Ev{\tau=t}$ signifiera donc que la saison s'arrête à la 
date $t$.\\[.1cm]
Toutes les variables aléatoires intervenant dans le problème sont 
définies sur un espace probabilisé $(\Omega, \A, \Prob)$. Pour toute 
variable aléatoire $Y$, on notera $\E(Y)$ son espérance lorsqu'elle 
existe.



\section*{Partie I - Modèle de population saisonnière}

\noindent
\begin{minipage}{20cm}
  Dans cette question, on définit l'évolution formelle du nombre 
  d'{\oe}ufs en diapause entre les dates $0$ et $T$.
\end{minipage}

\noindent
On note $D(t)=$ nombre d'{\oe}ufs en diapause à la date $t$. Les 
{\oe}ufs pondus à la date $t$ qui entrent en diapause sont 
comptabilisés à la date $t+1$.
\[
  \begin{array}{rcR{12cm}}
    N(t) &=& nombre moyen d'{\oe}ufs produits à la date $t$
    \nl
    p(t) &=& proportion des {\oe}ufs produits à la date $t$ qui entrent 
    en diapause
  \end{array}
\]
Par convention, la date $0$ d'une saison est celle où les insectes nés 
des {\oe}ufs en diapause de la saison précédente pondent $N(0)$ 
{\oe}ufs. On suppose pour simplifier :
\begin{noliste}{$-$}
  \item que $N(0)$ est un entier naturel non nul.
  \item que tous les {\oe}ufs issus de la saison précédente ont éclos 
  et donc que $D(0)=0$.
  \item que pour tout $t\in \{0,1, \ldots, T-1\}$, $0<p(t)\leq 1$
\end{noliste}
Enfin, on suppose qu'à chaque date $t$ de la saison, un individu 
produit en moyenne $\alpha$ {\oe}ufs ($\alpha$ étant un réel 
strictement positif). {\bf Par simplicité, on supposera que $\alpha$ 
reste constant pendant toute la saison.}

\begin{noliste}{1.}
  \setlength{\itemsep}{4mm}
  \item 
  \begin{noliste}{a)}
    \setlength{\itemsep}{2mm}
    \item Montrer que $D(t+1)=D(t) + p(t) \, N(t)$ pour tout entier $t$ 
    tel que $0 \leq t \leq T-1$.
    
    \begin{proof}~\\
      Soit $t\in \{0,\ldots, T-1\}$.\\
      Le nombre d'{\oe}ufs en diapause à l'instant $(t+1)$, $D(t+1)$, 
      est la somme :
      \begin{noliste}{$\stimes$}
	\item du nombre d'{\oe}ufs qui étaient déjà en diapause à 
	l'instant $t$ : $D(t)$,
	\item et du nombre d'{\oe}ufs pondus à l'instant $t$ entrant 
	tout de suite en diapause.\\
	Or le nombre d'{\oe}ufs pondus à l'instant $t$ est $N(t)$, et
	la proportion d'entre eux entrant en diapause est $p(t)$.\\
	Ainsi le nombre d'{\oe}ufs pondus à l'instant $t$ entrant 
	immédiatement en diapause est : $p(t) \, N(t)$.
      \end{noliste}
      \conc{On en déduit : $\forall t \in \{0,\ldots, T-1\}$, $D(t+1) = 
      D(t) + p(t) \, N(t)$.}~\\[-1cm]
    \end{proof}
    
    
    %\newpage

    
    \item Montrer que $N(t+1)=\alpha (1-p(t)) \, N(t)$ pour tout entier 
    $t$ tel que $0 \leq t \leq T-1$.
    
    \begin{proof}~\\
      Soit $t\in \{0,\ldots, T-1\}$.\\
      À l'instant $t$, $N(t)$ {\oe}ufs sont pondus. Parmi ceux-ci :
      \begin{noliste}{$\stimes$}
	\item $p(t) \, N(t)$ entrent immédiatement en diapause,
	
	\item $(1-p(t)) \, N(t)$ n'entrent pas en diapause.
      \end{noliste}
      Seuls les {\oe}ufs qui ne sont pas entrés en diapause à l'instant 
      $t$ produisent des {\oe}ufs à l'instant $(t+1)$. Donc 
      $(1-p(t)) \, N(t)$ individus produisent des {\oe}ufs à l'instant 
      $(t+1)$.\\
      D'après l'énoncé, chaque individu produit $\alpha$ {\oe}ufs.\\
      Ainsi, le nombre d'{\oe}ufs produits à l'instant $(t+1)$ est 
      $\alpha \, (1-p(t)) \, N(t)$.
      \conc{On en déduit : $\forall t \in \{0, \ldots, T-1\}$, 
      $N(t+1) = \alpha \, (1-p(t)) \, N(t)$.}~\\[-1cm]
    \end{proof}
  \end{noliste}
  
  \item On suppose dans cette question que $\alpha \leq 1$.
  \begin{noliste}{a)}
    \setlength{\itemsep}{2mm}
    \item Montrer que pour tout entier $t$ tel que $0 \leq t \leq T-1$,
    $N(t+1) \leq N(t)$.
    
    \begin{proof}~\\
    Soit $t \in \{0, \ldots, T-1\}$.
      \begin{noliste}{$\sbullet$}
	\item D'après la question précédente : $N(t+1) = \alpha \, (1-
	p(t)) \, N(t)$.
	
	\item Or, dans cette question : $\alpha \leq 1$.
	
	\item De plus, d'après l'énoncé : $0 < p(t) \leq 1$. D'où : 
	$0 \leq 1-p(t) <1$.
      \end{noliste}
      On en déduit : $\alpha \, (1-p(t)) \leq 1$. Ainsi :
      \[
        \begin{array}{ccc}
          \alpha \, (1-p(t)) \, N(t) & \leq & N(t)
          \\
          \shortparallel
          \\
          N(t+1)
        \end{array}
      \]
      \conc{$\forall t \in \{0, \ldots, T-1\}$, $N(t+1) \leq 
      N(t)$}~\\[-1cm]
    \end{proof}

    
    \item Montrer que pour tout entier $t$ tel que $0 \leq t \leq T-1$ :
    \[
      D(t+1) + N(t+1) \ \leq \ D(t)+N(t)
    \]
    
    \begin{proof}~\\
      Soit $t\in \{0, \ldots, T-1\}$.\\
      D'après les questions \itbf{1.a)} et \itbf{1.b)} :
      \[
        D(t+1) + N(t+1) \ = \ \big(D(t) + p(t) \, N(t)\big) + 
        \alpha \, (1-p(t)) \, N(t)
      \]
      Or $\alpha \leq 1$. On obtient donc : $ \alpha \, (1-p(t)) \, 
      N(t) \leq (1-p(t)) \, N(t)$.\\
      Ainsi :
      \[
       \begin{array}{rcl}
        D(t+1) + N(t+1) & \leq & D(t) + p(t) \, N(t) + (1-p(t)) \, N(t)
        \\[.2cm]
        &=& D(t) + (\bcancel{p(t)} + 1 - \bcancel{p(t)}) \, N(t)
        \\[.2cm]
        &=& D(t) + N(t)
       \end{array}
      \]
      \conc{$\forall t \in \{0, \ldots, T-1\}$, $D(t+1) + N(t+1) \leq 
      D(t) + N(t)$.}~\\[-1cm]
    \end{proof}
    
    
    %\newpage
   
    
    \item Montrer que pour tout entier $t$ tel que $0 \leq t \leq T$ :
    \[
      D(t) + N(t) \leq N(0)
    \]
    
    \begin{proof}~\\
      Démontrons par récurrence que pour tout $t \in \{0, \ldots, T\}$, 
      $\PP{t}$ \quad où \quad $\PP{t}$ : $ D(t) + N(t) \leq N(0)$.
      \begin{noliste}{\fitem}
	\item {\bf Initialisation} :\\
	$D(0) + N(0) \ = \ 0 + N(0) \ = \ N(0) \ \leq \ N(0)$.\\
	D'où $\PP{0}$.
	
	\item {\bf Hérédité} : Soit $t\in \{0, \ldots, T-1\}$.\\
	Supposons $\PP{t}$ et démontrons $\PP{t+1}$ (\ie $D(t+1) + 
	N(t+1) \leq N(0)$)
	\[
	  \begin{array}{rcl@{\qquad}>{\it}R{5cm}}
	    D(t+1) + N(t+1) & \leq & D(t) + N(t) & (d'après la question 
	    \itbf{2.b)})
	    \nl
	    \nl[-.2cm]
	    & \leq & N(0) & (par hypothèse de récurrence)
	  \end{array}
	\]
	D'où $\PP{t+1}$.
      \end{noliste}
      \conc{Par principe de récurrence : $\forall t \in \{0, \ldots, 
      T\}$, $D(t) + N(t) \leq N(0)$.}
      
      ~\\[-1.4cm]
    \end{proof}

    
    \item Montrer que pour tout entier $t$ tel que $0 \leq t \leq T$, 
    $D(t) \leq N(0)$.
    
    \begin{proof}~\\
      Soit $t \in \{0, \ldots, T\}$.\\
      D'après la question précédente : $D(t) + N(t) \leq N(0)$.\\
      Or $N(t)$ est un entier naturel. En particulier : $N(t) \geq 0$.\\
      Ainsi : $D(t) \leq D(t) + N(t) \leq N(0)$.
      \conc{$\forall t \in \{0, \ldots, T\}$, $D(t) \leq N(0)$}~\\[-1cm]
    \end{proof}

    
    \item On suppose que $p(0)=1$.
    \begin{nonoliste}{(i)}
      \item Montrer que pour tout entier $t$ tel que $1 \leq t \leq T$, 
      $N(t)=0$.
      
      \begin{proof}~\\
        Démontrons pas récurrence que pour tout $t \in \{1, \ldots, 
	T\}$, $\PP{t}$ \quad où \quad $\PP{t}$ : $N(t)=0$.
        \begin{noliste}{\fitem}
	  \item {\bf Initialisation} :\\
	  D'après la question \itbf{1.b)} : $N(1) \ = \ \alpha \, 
	  (1-p(0)) \, N(0) \ = \ \alpha \, (\bcancel{1} - \bcancel{1})
	  \, N(0) \ = \ 0$.\\
	  D'où $\PP{1}$.
	  
	  
	  %\newpage
	  
	  
	  \item {\bf Hérédité} : Soit $t \in \{1, \ldots, T-1\}$.\\
	  Supposons $\PP{t}$ et démontrons $\PP{t+1}$ (\ie $N(t+1)=0$).
	  \[
	    \begin{array}{rcl@{\qquad}>{\it}R{5cm}}
	      N(t+1) &=& \alpha \, (1-p(t)) \, N(t) 
	      & (d'après la question \itbf{1.b)})
	      \nl
	      \nl[-.2cm]
	      &=& \alpha \, (1-p(t)) \times 0 
	      & (par hypothèse de récurrence)
	      \nl
	      \nl[-.2cm]
	      &=& 0
	    \end{array}
	  \]
	  D'où $\PP{t+1}$.
        \end{noliste}
        \conc{Par principe de récurrence : $\forall t \in \{1, \ldots,
        T\}$, $N(t)=0$.}
        ~\\[-1.4cm]
      \end{proof}

      
      \item Montrer que pour tout entier $t$ tel que $1 \leq t \leq T$,
      $D(t)=N(0)$.
      
      \begin{proof}~\\
        Démontrons par récurrence que pour tout $t \in \{1, \ldots, 
	T\}$, $\PP{t}$ \quad où \quad $\PP{t}$ : $D(t)=N(0)$.
        \begin{noliste}{\fitem}
	  \item {\bf Initialisation} :\\
	  D'après la question \itbf{1.a)} : $D(1) \ = \ D(0) + p(0) \, 
	  N(0) \ = \ 0 + 1 \times N(0) \ = \ N(0)$.\\
	  D'où $\PP{1}$.
	  
	  \item {\bf Hérédité} : Soit $t\in \{1, \ldots, T-1\}$.\\
	  Supposons $\PP{t}$ et démontrons $\PP{t+1}$ (\ie $D(t+1)
	  =N(0)$).
	  \[
	    \begin{array}{rcl@{\qquad}>{\it}R{5cm}}
	      D(t+1) &=& D(t) + p(t) \, N(t) 
	      & (d'après la question \itbf{1.a)})
	      \nl
	      \nl[-.2cm]
	      &=& D(t) + \bcancel{p(t) \times 0}
	      & (d'après la question \itbf{2.e)(ii)}, car $t\geq 1$)
	      \nl
	      \nl[-.2cm]
	      &=& N(0) & (par hypothèse de récurrence)
	    \end{array}
	  \]
	  D'où $\PP{t+1}$.
        \end{noliste}
        \conc{Par principe de récurrence : $\forall t \in \{1, \ldots, 
        T\}$, $D(t)=N(0)$.}~\\[-1cm]
      \end{proof}
      
      
      %\newpage

      
      \item En déduire que si $\alpha \leq 1$, la meilleure stratégie 
      adaptée à la saison est que les $N(0)$ {\oe}ufs produits à la date
      $0$ entrent en diapause immédiatement.
      
      \begin{proof}~
        \begin{noliste}{$\sbullet$}
	  \item D'après l'énoncé, la stratégie optimale est la stratégie
	  qui maximise le nombre d'{\oe}ufs en diapause accumulés 
	  jusqu'à la date $T$ (date de fin de saison).
	  
	  \item Si $\alpha \leq 1$, d'après la question \itbf{2.d)} :
	  $D(t) \leq N(0)$.
	  
	  \item Or, d'après la question \itbf{2.e)(ii)}, si $p(0)=1$, 
	  alors $D(t)=N(0)$.\\
	  Le nombre d'{\oe}ufs en diapause à chaque instant $t\in \{1, 
	  \ldots, T\}$ est donc maximal si $p(0)=1$, c'est-à-dire si la 
	  proportion d'{\oe}ufs entrant en diapause à l'instant $0$
	  parmi les $N(0)$ vaut $1$.
        \end{noliste}
        \conc{Autrement dit, la stratégie optimale est que les $N(0)$
        {\oe}ufs produits à la date $0$\\
        entrent en diapause 
        immédiatement.}~\\[-1cm]
      \end{proof}

    \end{nonoliste}
  \end{noliste}
  
  \item {\bf On suppose désormais $\alpha >1$ jusqu'à la fin du 
  problème.}\\
  On introduit maintenant $\tau$ une variable aléatoire à valeurs dans 
  $\{1,2, \ldots, T\}$ qui représente la date où s'achève la saison. On 
  suppose que pour tout $t \in \{1,2, \ldots, T\}$, $\Prob(
  \Ev{\tau=t}) >0$.
  \begin{noliste}{a)}
    \setlength{\itemsep}{2mm}
    \item Montrer que pour tout $t\in \{1,2, \ldots, T\}$, $\Prob(
    \Ev{\tau \geq t}) >0$. On définit alors $H(t)= \Prob_{\Ev{\tau \geq 
    t}}(\Ev{\tau =t})$.
    
    \begin{proof}~\\
      Soit $t\in \{1, \ldots, T\}$.\\
      Tout d'abord : $\Ev{\tau =t} \subset \Ev{\tau \geq t}$.
      Donc : $\Prob(\Ev{\tau = t}) \leq \Prob(\Ev{\tau \geq t})$.\\
      Or, d'après l'énoncé : $\Prob(\Ev{\tau =t}) >0$. D'où :
      \[
        \Prob(\Ev{\tau \geq t}) \ \geq \ \Prob(\Ev{\tau =t}) \ > \ 0
      \]
      \conc{Ainsi : $\forall t \in \{1, \ldots, T\}$, $\Prob(\Ev{\tau 
      \geq t}) >0$.}~\\[-1cm]
    \end{proof}

    
    \item Montrer que : $H(t)= \dfrac{\Prob(\Ev{\tau=t})}{\Prob(\Ev{\tau
    \geq t})}$.
    
    \begin{proof}~\\
      Soit $t \in \{1, \ldots, T\}$.
      \[
        \begin{array}{rcl@{\qquad}>{\it}R{5cm}}
          H(t) &=& \Prob_{\Ev{\tau \geq t}}(\Ev{\tau = t})
          \\[.4cm]
          &=&
          \dfrac{\Prob(\Ev{\tau \geq t} \cap \Ev{\tau = t})}
          {\Prob(\Ev{\tau \geq t})}
          \\[.6cm]
          &=& \dfrac{\Prob(\Ev{\tau = t})}{\Prob(\Ev{\tau \geq t})}
          & (car $\Ev{\tau =t} \subset \Ev{\tau \geq t}$)
        \end{array}
      \]
      \conc{$\forall t \in \{1, \ldots, T\}$, $H(t) =
      \dfrac{\Prob(\Ev{\tau=t})}{\Prob(\Ev{\tau \geq t})}$}~\\[-1cm]
    \end{proof}

    
    \item Montrer que : $H(T)=1$.
    
    \begin{proof}~\\
      La \var $\tau$ est à valeurs dans $\{1, \ldots, T\}$. En 
      particulier, $\tau$ ne prend pas de valeurs strictement 
      supérieures à $T$.
      Donc : $\Ev{\tau \geq T} = \Ev{\tau = T}$.\\[.1cm]
      D'où : $H(t) \ = \ \dfrac{\Prob(\Ev{\tau = T})}
      {\Prob(\Ev{\tau \geq T})} \ = \ \dfrac{\bcancel{\Prob(
      \Ev{\tau =T})}}{\bcancel{\Prob(\Ev{\tau =T})}} \ = \ 1$.
      \conc{Ainsi : $H(T)=1$.}~\\[-1cm]
    \end{proof}
    
    
    %\newpage

    
    \item Calculer $H(t)$ pour $t\in \{1,2, \ldots, T\}$ si $\tau$ 
    suit une loi uniforme sur $\{1,2, \ldots, T\}$.
    
    \begin{proof}~
      \begin{noliste}{$\sbullet$}
	\item Comme $\tau \suit \U{1}{T}$, alors :
	\begin{noliste}{$\stimes$}
	  \item $\tau(\Omega) = \llb 1, T\rrb$.
	  \item $\forall t \in \llb 1, T \rrb$, $\Prob(\Ev{\tau =t})
	  =\dfrac{1}{T}$.
	\end{noliste}
	
	\item Soit $t\in \llb 1, T\rrb$.
	\[
	  \Ev{\tau \geq t} \ = \ \dcup{k=t}{T} \Ev{\tau =k}
	\]
	De plus, les événements $\Ev{\tau=t}$, $\ldots$, $\Ev{\tau =
	T}$ sont incompatibles. Donc :
	\[
	  \Prob(\Ev{\tau \geq t}) \ = \ \Sum{k=t}{T} 
	  \Prob(\Ev{\tau =k}) \ = \ \Sum{k=t}{T} \dfrac{1}{T}
	  \ = \ \dfrac{T-t+1}{T}
	\]
	
	\item On en déduit :
	\[
	  H(t) \ = \ \dfrac{\Prob(\Ev{\tau =t})}{\Prob(\Ev{\tau \geq 
	  t})} \ = \ \dfrac{\frac{1}{\bcancel{T}}}{\frac{T-t+1}
	  {\bcancel{T}}} \ = \ \dfrac{1}{T-t+1}
	\]
      \end{noliste}
      \conc{$\forall t \in \{1, \ldots, T\}$, $H(t) = 
      \dfrac{1}{T-t+1}$}~\\[-1cm]
    \end{proof}

    
    \item 
    \begin{nonoliste}{(i)}
      \item Soient $T$ réels $\lambda_1$, $\lambda_2$, $\ldots$, 
      $\lambda_T$ tels que $0 < \lambda_1 < \lambda_2 < \cdots < 
      \lambda_T=1$. Par convention, on pose $\lambda_0=0$. Soient 
      $q_1=\lambda_1$, $q_2=\lambda_2-\lambda_1$, $\ldots$, 
      $q_T=\lambda_T - \lambda_{T-1}$.\\
      Montrer que $(q_i)_{1\leq i \leq T}$ définit une loi de 
      probabilité sur $\{1,2, \ldots, T\}$.
      
      \begin{proof}~
        \begin{noliste}{$\sbullet$}
	  \item Soit $i\in \llb 1, T\rrb$.\\
	  D'après l'énoncé : $\lambda_i > \lambda_{i-1}$. Donc 
	  $q_i = \lambda_i - \lambda_{i-1} >0$.
	  
	  \item De plus :
	  \[
	    \Sum{i=1}{T} q_i \ = \ \Sum{i=1}{T} (\lambda_i - 
	    \lambda_{i-1}) \ = \ \lambda_T - \lambda_0 \ = \ 1-0
	    \ = \ 1
	  \]
        \end{noliste}
        \conc{On en déduit que $(q_i)_{i \in \llb 1,T \rrb}$ 
        définit une loi de probabilité.}~\\[-1cm]
      \end{proof}

      
      \item Calculer $H(t)$ si $\tau$ suit la loi précédente.
      
      \begin{proof}~
        \begin{noliste}{$\sbullet$}
	  \item Si la \var $\tau$ suit la loi $(q_i)_{i \in \llb 1,T 
	  \rrb}$, alors :
	  \[
	    \forall t \in \llb 1,T \rrb, \ \Prob(\Ev{\tau =t}) = q_t
	  \]
	  
	  \item Soit $t\in \llb 1,T \rrb$.
	  Avec le même raisonnement qu'en question \itbf{3.d)} :
	  \[
	    \Prob(\Ev{\tau \geq t}) \ = \ \Sum{i=t}{T} 
	    \Prob(\Ev{\tau =i}) \ = \ \Sum{i=t}{T} q_i \ = \
	    \Sum{i=t}{T} (\lambda_i - \lambda_{i-1}) \ = \ 
	    \lambda_T - \lambda_{t-1}
	  \]
	  
	  \item Ainsi : $H(t) \ = \ \dfrac{\Prob(\Ev{\tau =t})}
	  {\Prob(\Ev{\tau \geq t})} \ = \ \dfrac{q_t}{\lambda_T - 
	  \lambda_{t-1}} \ = \ \dfrac{\lambda_t - \lambda_{t-1}}
	  {\lambda_T - \lambda_{t-1}}$.
        \end{noliste}
        \conc{$\forall t \in \llb 1,T \rrb$, $H(t) \ = \ \dfrac{q_t}
        {\lambda_T - \lambda_{t-1}} \ = \ \dfrac{\lambda_t - 
        \lambda_{t-1}}{\lambda_T - \lambda_{t-1}}$}~\\[-1cm]
      \end{proof}
      
      
     %\newpage

      
      \item On suppose que $T\geq 2$ et de plus que pour tout entier 
      $n$ tel que $1 \leq n \leq T-1$, on a $\lambda_{n+1} - 
      \lambda_n \geq \lambda_n - \lambda_{n-1}$. Montrer que 
      $t \mapsto H(t)$ est croissante sur $\{1,2, \ldots, T\}$.
      
      \begin{proof}~\\
        Soit $t \in \llb 1, T-1 \rrb$. D'après la question 
	\itbf{3.e)(ii)} :
	\[
	  H(t+1) \geq H(t) \ \ \Leftrightarrow \ \ \dfrac{\lambda_{t+1}
	  -\lambda_t}{\lambda_T - \lambda_t} \geq 
	  \dfrac{\lambda_t - \lambda_{t-1}}{\lambda_T - \lambda_{t-1}}
	\]
	\begin{noliste}{$\sbullet$}
	  \item On sait déjà, d'après l'énoncé : $\lambda_{t+1} - 
	  \lambda_t \ \geq \ \lambda_t - \lambda_{t-1} \ \geq \ 0
	  \qquad (\star) $.
	  
	  \item De plus :
	  \[
	    \begin{array}{rcl@{\qquad}>{\it}R{5cm}}
	      \dfrac{1}{\lambda_T - \lambda_t} \geq 
	      \dfrac{1}{\lambda_T - \lambda_{t-1}}
	      & \Leftrightarrow & \lambda_T - \lambda_t \leq 
	      \lambda_T - \lambda_{t-1}
	      & (par stricte décroissance de la fonction inverse sur 
	      $]0,+\infty[$)
	      \nl
	      \nl[-.2cm]
	      & \Leftrightarrow & \lambda_{t-1} \leq \lambda_t
	    \end{array}
	  \]
	  La dernière inégalité est vraie. Donc, par équivalence, la 
	  première également. On a donc :
	  \[
	    \dfrac{1}{\lambda_T - \lambda_t} \ \geq \ 
	    \dfrac{1}{\lambda_T - \lambda_{t-1}} \ \geq \ 0
	  \]
	  
	  \item On multiplie cette inégalité et l'inégalité $(\star)$
	  membre à membre
	  (ce qui ne change pas leurs sens car les termes en présence
	  sont positifs). On obtient alors :
	  \[
	   \begin{array}{ccc}
	    \dfrac{\lambda_{t+1} - \lambda_t}{\lambda_T - \lambda_t} 
	    & \geq & 
	    \dfrac{\lambda_t - \lambda_{t-1}}{\lambda_T - \lambda_{t-1}}
	    \\
	    \shortparallel & & \shortparallel
	    \\
	    H(t+1) & & H(t)
	   \end{array}
	  \]
	\end{noliste}
	\conc{On en déduit que $t \mapsto H(t)$ est croissante sur 
	$\{1, \ldots, T\}$.}~\\[-1.2cm]
      \end{proof}
    \end{nonoliste}
  \end{noliste}
  
  \noindent
  {\bf On suppose désormais que $t \mapsto H(t)$ est croissante.} Le 
  but est maintenant de trouver une stratégie adéquate pour maximiser 
  la quantité $\E\big(\ln(D(\tau))\big)$. On va commencer par regarder 
  un exemple simple.
  
  \item On suppose ici que $T=2$, que $H$ est donnée par $H(1) =
  \dfrac{1}{2}$ et $H(2)=1$ et que $\alpha=4$.
  \begin{noliste}{a)}
    \setlength{\itemsep}{2mm}
    \item 
    \begin{nonoliste}{(i)}
      \item Déterminer $\Prob(\Ev{\tau=1})$.
      
      \begin{proof}~
        \begin{noliste}{$\sbullet$}
	  \item D'après la question \itbf{3.b)} : $H(1) = 
	  \dfrac{\Prob(\Ev{\tau =1})}{\Prob(\Ev{\tau \geq 1})}$. Donc :
	  \[
	    \Prob(\Ev{\tau =1}) \ = \ H(1) \, \Prob(\Ev{\tau \geq 1})
	    \ = \ \dfrac{1}{2} \, \Prob(\Ev{\tau \geq 1})
	  \]
	  
	  \item De plus, comme $T=2$ : $\tau (\Omega) = \{1,2\}$.\\
	  On en déduit : $\Ev{\tau \geq 1} = \Omega$. Donc :
	  \[
	    \Prob(\Ev{\tau \geq 1}) \ = \ \Prob(\Omega) \ = \ 1
	  \]
	  
	  \item Ainsi : $\Prob(\Ev{\tau =1}) \ = \ \dfrac{1}{2} \,
	  \Prob(\Ev{\tau \geq 1}) \ = \  \dfrac{1}{2} \times 1 \ = \
	  \dfrac{1}{2}$.
        \end{noliste}
        \conc{$\Prob(\Ev{\tau=1}) = \dfrac{1}{2}$}~\\[-1cm]
      \end{proof}
      
      
      %\newpage

      
      \item Quelle est la loi de $\tau$ ?
      
      \begin{proof}~
        \begin{noliste}{$\sbullet$}
	  \item D'après les résultats de la question précédente : 
	  $\tau(\Omega) = \{1,2\}$ et $\Prob(\Ev{\tau=1}) = 
	  \dfrac{1}{2}$.
	  
	  \item La famille $(\Ev{\tau=1}, \Ev{\tau=2})$ est un 
	  système complet d'événements. Donc :
	  \[
	    \Prob(\Ev{\tau=2}) \ = \ 1-\Prob(\Ev{\tau =1}) \ = \ 1-
	    \dfrac{1}{2} \ = \ \dfrac{1}{2}
	  \]
	  
	  \item On reconnaît les caractéristiques d'une \var de loi 
	  uniforme sur $\{1,2\}$.
        \end{noliste}
        \conc{$\tau \suit \U{1}{2}$}~\\[-1.2cm]
      \end{proof}
    \end{nonoliste}
    
    \item Montrer que pour $D(1)$ et $N(1)$ donnés, $D(2)$ est maximum
    pour $p(1)=1$.
    
    \begin{proof}~
     \begin{noliste}{$\sbullet$}
      \item D'après la question \itbf{1.a)} : $D(2) = D(1) + p(1) \, 
      N(1)$.\\
      Donc l'entier $D(2)$ est maximal si $D(1)+p(1) \, N(1)$ l'est.
      
      \item Or, si $D(1)$ et $N(1)$ sont fixés, alors $D(1)+p(1) \, 
      N(1)$ est maximal si $p(1)$ l'est.
      
      \item De plus : $0< p(1) \leq 1$. Donc $p(1)$ est maximal 
      lorsque $p(1)=1$.
     \end{noliste}
     \conc{On en déduit que $D(2)$ est maximal lorsque 
     $p(1)=1$.}~\\[-1cm]
    \end{proof}

    
    \item On suppose $p(1)=1$. Montrer que :
    \[
      \E\big(\ln(D(\tau)\big) \ = \ \dfrac{1}{2} \, \ln\big((4-3 \, 
      p(0)) \, N(0)\big) + \dfrac{1}{2} \, \ln\big(p(0) \, N(0)\big)
    \]
    
    \begin{proof}~
      \begin{noliste}{$\sbullet$}
	\item La \var $\tau$ est une \var finie, donc la \var $\ln(
	D(\tau))$ l'est aussi.
	\conc{Ainsi, la \var $\ln(D(\tau))$ admet une espérance.}
	
	\item D'après le théorème de transfert :
	\[
	  \begin{array}{rcl}
	    \E\big(\ln(D(\tau))\big) &=& \ln(D(1)) \, \Prob(\Ev{\tau
	    =1}) + \ln(D(2)) \, \Prob(\Ev{\tau =2})
	    \\[.2cm]
	    &=& \dfrac{1}{2} \, \ln(D(1)) + \dfrac{1}{2} \, \ln(D(2))
	  \end{array}
	\]
	
	\item Or : $D(1) \ = \ D(0) + p(0) \, N(0) \ = \ p(0) \, N(0)$.
	
	\item De plus : 
	\[
	  \begin{array}{rcl}
	    D(2) &=& D(1) + p(1) \, N(1) \ = \ D(1) + N(1)
	    \\[.2cm]
	    &=& D(0) + p(0) \, N(0) + \alpha \, (1-p(0)) \, N(0)
	    \\[.2cm]
	    &=& p(0) \, N(0) + 4(1-p(0)) \, N(0)
	    \\[.2cm]
	    &=& (4-3 \, p(0)) \, N(0)
	  \end{array}
	\]
      \end{noliste}
      \conc{Finalement : $\E\big(\ln(D(\tau))\big) = \dfrac{1}{2} \,
      \ln\big(p(0) \, N(0)\big) + \dfrac{1}{2} \, \ln\big( (4-3 \, 
      p(0)) \, N(0) \big)$.}~\\[-1cm]
    \end{proof}
    
    
    %\newpage

    
    \item Construire le tableau de variations sur $]0,1]$ de la 
    fonction $\varphi$ définie par :
    \[
      \varphi(x) \ = \ \dfrac{1}{2} \, \ln\big((4-3x) \, N(0)\big) 
      + \dfrac{1}{2} \, \ln\big(N(0) \, x\big)
    \]
    
    \begin{proof}~
      \begin{noliste}{$\sbullet$}
	\item La fonction $\varphi$ est dérivable sur $]0,1]$ en tant 
	que composée et somme de fonctions dérivables sur $]0,1]$. 
	En effet, $N(0) >0$, donc : $N(0) \, x \in \ ]0,N(0)]$ et 
	$(4-3x) \, N(0) \in [N(0), 4 \, N(0)[$.

	
	\item Soit $x \in \ ]0,1]$.
	\[
	  \begin{array}{rcl}
	    \varphi'(x) &=& \dfrac{1}{2} \, \dfrac{-3 \, \bcancel{N(0)}}
	    {(4-3x) \, \bcancel{N(0)}} + \dfrac{1}{2} \, 
	    \dfrac{\bcancel{N(0)}}{\bcancel{N(0)} \, x}
	    \\[.6cm]
	    &=& \dfrac{1}{2} \left( - \dfrac{3}{4-3x} + \dfrac{1}{x}
	    \right) \ = \ \dfrac{1}{2} \, \dfrac{-3x+4-3x}{x \, 
	    (4-3x)}
	    \\[.6cm]
	    &=& \dfrac{1}{2} \, \dfrac{4-6x}{x(4-3x)} \ = \ 
	    \dfrac{2-3x}{x(4-3x)}
	  \end{array}
	\]
	
	\item Comme $x \in \ ]0,1]$ : $x >0$ et $4-3x>0$. Donc :
	\[
	  \varphi'(x) \geq 0 \ \Leftrightarrow \ 2-3x \geq 0 \
	  \Leftrightarrow \ 2 \geq 3x \ \Leftrightarrow \ 
	  \dfrac{2}{3} \geq x
	\]
	On obtient le tableau de variations suivant :
	\begin{center}
        \begin{tikzpicture}[scale=0.8, transform shape]
          \tkzTabInit[lgt=4,espcl=3] 
          {$x$ /1, Signe de $\varphi'(x)$ /1, Variations de $\varphi$ 
	  /2} 
          {$0$, $\frac{2}{3}$, $1$}%
          \tkzTabLine{ , + ,z, - , } 
          \tkzTabVar{-/$-\infty$, +/$\varphi\big( \frac{2}{3}\big)$, 
	  -/$\ln(N(0))$}
        \end{tikzpicture}
      \end{center}
      
      Détaillons les éléments de ce tableau :
      \begin{noliste}{$\stimes$}
	\item Tout d'abord : $\dlim{x\to 0} \ln \big( (4-3x) \, N(0)
	\big) = \ln\big(4 \, N(0)\big)$. De plus : $\dlim{x\to 0}
	\ln\big( N(0) \, x \big) = -\infty$.\\[.1cm]
	D'où : $\dlim{x\to 0} \varphi(x)=-\infty$.
	
	\item Ensuite :
	\[
	  \varphi(1) \ = \ \dfrac{1}{2} \, \ln\big( (4-3\times 1) \,
	  N(0)\big) + \dfrac{1}{2} \, \ln(N(0) \times 1) \ = \
	  \dfrac{1}{2} \, \ln(N(0)) + \dfrac{1}{2} \, \ln(N(0))
	  \ = \ \ln(N(0))
	\]
      \end{noliste}
      \end{noliste}
      
      ~\\[-1.4cm]
    \end{proof}
    
    
    %\newpage

    
    \item Déterminer $p^*(0)$ qui maximise $\E\big(\ln(D(\tau))\big)$.
    
    \begin{proof}~\\
      D'après les questions \itbf{4.c)} et \itbf{4.d)} : $\E\big(
      \ln(D(\tau))\big) = \varphi(p(0))$.\\
      Or, d'après la question \itbf{4.d)}, la fonction $\varphi$
      admet un unique maximum en $\dfrac{2}{3}$.
      \conc{Donc $\E\big(\ln(D(\tau))\big)$ est maximal pour 
      $p^*(0)= \dfrac{2}{3}$.}~\\[-1cm]
    \end{proof}
  \end{noliste}
\end{noliste}




\section*{Partie II - Transformation du problème}

\noindent
{\bf Par convention, on conviendra que si $h$ est une fonction 
numérique définie sur $\{0,1,2, \ldots, T\}$}, on a
\[
  \Sum{t=1}{0} h(t)=0
\]

\begin{noliste}{1.}
  \setlength{\itemsep}{4mm}
  \setcounter{enumi}{4}
  \item Montrer que pour tout $t \in \{0,1,2, \ldots, T\}$, $D(t) 
  +N(t) >0$.
  
  \begin{proof}~\\
    Démontrons par récurrence que pour tout $t\in \{0, \ldots, T\}$,
    $\PP{t}$ \quad où \quad $\PP{t}$ : $D(t)+N(t) >0$.
    \begin{noliste}{\fitem}
      \item {\bf Initialisation} :\\
      D'après l'énoncé, $N(0)$ est un entier naturel non nul. Donc :
      $D(0) + N(0) \ = \ 0 + N(0) \ > \ 0$.\\
      D'où $\PP{0}$.
      
      \item {\bf Hérédité} : Soit $t\in \{0, \ldots, T-1\}$.\\
      Supposons $\PP{t}$ et démontrons $\PP{t+1}$ (\ie $D(t+1) + 
      N(t+1) >0$).
      \[
        \begin{array}{rcl}
          D(t+1) + N(t+1) &=& D(t) + p(t) \, N(t) + \alpha \, 
          (1-p(t)) \, N(t)
          \\[.2cm]
          &=& D(t) + \big(p(t) + \alpha(1-p(t))\big) \, N(t)
          \\[.2cm]
          &=& D(t) + N(t) + \big(p(t) + \alpha (1-p(t)) -1\big) N(t)
          \\[.2cm]
          &=& D(t) + N(t) + (1-p(t))(\alpha -1) N(t)
        \end{array}
      \]
      Or :
      \begin{noliste}{$\stimes$}
	\item par hypothèse de récurrence : $D(t) + N(t) >0$,
	\item $1-p(t)\geq 0$, car $0<p(t) \leq 1$,
	\item $\alpha -1>0$, car $\alpha >1$,
	\item $N(t) \geq 0$, car $N(t)$ est un entier naturel.
      \end{noliste}
      On en déduit : $D(t+1) + N(t+1) >0$.\\
      D'où $\PP{t+1}$.
    \end{noliste}
    \conc{Par principe de récurrence : $\forall t \in \{0, \ldots, T\}$,
    $D(t)+N(t) >0$.}~\\[-1cm]
  \end{proof}
  
  
  %\newpage

  
  On pose
  \[
    X(t) \ = \ \dfrac{D(t)}{D(t) + N(t)}
  \]  
  
  \item Montrer que pour tout $t\in \{0,1,2, \ldots, T-1\}$ :
  \[
    X(t+1) \ = \ \dfrac{p(t) + (1-p(t)) \, X(t)}{p(t) + \alpha \, 
    (1-p(t)) + (1-\alpha)(1-p(t)) \, X(t)}
  \]
  
  \begin{proof}~\\
    Soit $t\in \{0, \ldots, T-1\}$.
    \begin{noliste}{$\sbullet$}
      \item D'une part : $X(t+1) \ = \ \dfrac{D(t+1)}{D(t+1) + 
      N(t+1)} \ = \ \dfrac{D(t) + p(t) \, N(t)}{D(t) + p(t) \, N(t) 
      + \alpha \, (1-p(t)) \, N(t)}$.
      
      \item D'autre part :
      \[
        \begin{array}{cl}
          & \dfrac{p(t) + (1-p(t)) \, X(t)}{p(t) + \alpha \, 
          (1-p(t)) + (1-\alpha)(1-p(t)) X(t)}
          \\[.6cm]
          =& \dfrac{p(t) + (1-p(t)) \, \frac{D(t)}{D(t)+N(t)}}
          {p(t) + \alpha \, (1-p(t)) + (1-\alpha)(1-p(t)) 
          \frac{D(t)}{D(t)+N(t)}}
          \\[.8cm]
          =& \dfrac{\frac{p(t)(D(t)+N(t)) + (1-p(t))D(t)} 
	  {\bcancel{D(t)+N(t)}}}
          {\frac{(p(t) + \alpha(1-p(t))(D(t) +N(t)) + (1-\alpha) 
          (1-p(t))D(t)} 
	  {\bcancel{D(t)+N(t)}}}
	  \\[.8cm]
	  =& \dfrac{\bcancel{p(t) \, D(t)} + p(t) \, N(t) + D(t) 
	  - \bcancel{p(t) \, D(t)}}
	  {\big(p(t) + \alpha(1-p(t)) + (1-\alpha)(1-p(t))\big) D(t)
	  + p(t) \, N(t) + \alpha(1-p(t))N(t)}
        \end{array}
      \]
      Or :
      \[
        p(t) + \alpha(1-p(t)) + (1-\alpha)(1-p(t)) \ = \
        p(t)\big(\bcancel{1-\alpha} - \bcancel{(1-\alpha)}\big) +
        \bcancel{\alpha} + 1 - \bcancel{\alpha} \ = \ 1
      \]
      D'où :
      \[
        \dfrac{p(t) + (1-p(t)) \, X(t)}{p(t) + \alpha \, 
        (1-p(t)) + (1-\alpha)(1-p(t)) X(t)} \ = \
        \dfrac{D(t) + p(t) \, N(t)}{D(t) + p(t) \, N(t) 
        + \alpha \, (1-p(t)) \, N(t)}
        \ = \ X(t+1)
      \]
    \end{noliste}
    \conc{$\forall t \in \{0, \ldots, T-1\}$, $X(t+1) \ = \ \dfrac{p(t) 
    + (1-p(t)) \, X(t)}{p(t) + \alpha \, 
    (1-p(t)) + (1-\alpha)(1-p(t)) \, X(t)}$}~\\[-1cm]
  \end{proof}

  
  \item Soit $\xi \in [0,1]$ fixé. Pour $x \in [0,1]$, on pose :
  \[
    \psi_\xi(x) \ = \ \dfrac{x+ (1-x) \xi}{x + \alpha \, (1-x) + 
    (1- \alpha)(1-x)\xi}
  \]
  \begin{noliste}{a)}
    \setlength{\itemsep}{2mm}
    \item Montrer que $\psi_\xi$ est croissante sur $[0,1]$.
    
    \begin{proof}~
      \begin{noliste}{$\sbullet$}
	\item La fonction $\psi_\xi$ est dérivable sur $[0,1]$ en tant
	que quotient de fonctions dérivables sur $[0,1]$ dont le 
	dénominateur ne s'annule pas.\\
	Démontrons que le dénominateur ne s'annule effectivement pas.\\
	Soit $x \in [0,1]$.
	\[
	 \begin{array}{rcl}
	  x + \alpha(1-x) + (1-\alpha)(1-x) \, \xi &=&
	  \big(1-\alpha - (1-\alpha) \, \xi\big) \, x + \alpha 
	  +(1-\alpha) \, \xi 
	  \\[.2cm]
	  &=& 
	  (1-\alpha)(1-\xi) \, x + \alpha + (1-\alpha) \, \xi
	 \end{array}
	\]
	
	
	%\newpage
	
	
	On en déduit :
	\[
	  \begin{array}{rcl@{\qquad}>{\it}R{4cm}}
	    x + \alpha(1-x) + (1-\alpha)(1-x) \, \xi = 0 
	    & \Leftrightarrow & 
	    (1-\alpha)(1-\xi) \, x + \alpha + (1-\alpha) \, \xi = 0
	    \\[.2cm]
	    & \Leftrightarrow & 
	    (1-\alpha)(1-\xi) \, x = -\big(\alpha + (1-\alpha) \, 
	    \xi\big)
	    \\[.2cm]
	    & \Leftrightarrow & 
	    x = - \dfrac{\alpha + (1-\alpha) \, \xi}{(1-\alpha)(1-\xi)}
	    = \dfrac{\alpha + (1-\alpha) \, \xi}{(\alpha-1)(1-\xi)}
	    & (si $\xi \neq 1$)
	  \end{array}
	\]
	Or :
	\[
	  \dfrac{\alpha + (1-\alpha) \, \xi}{(\alpha-1)(1-\xi)}
	  \ = \ \dfrac{1+(\alpha -1) + (1-\alpha) \, \xi}
	  {(\alpha-1)(1-\xi)} \ = \ 
	  \dfrac{1+(\alpha-1)(1-\xi)}{(\alpha-1)(1-\xi)}
	  \ > \ 1
	\]
	De plus $x\in [0,1]$. Donc $x \neq \dfrac{\alpha + (1-\alpha) \, 
	\xi}{(\alpha-1)(1-\xi)}$.\\
	Par équivalence, on en déduit bien :
	$x + \alpha(1-x) + (1-\alpha)(1-x) \, \xi \neq 0$, c'est-à-dire
	que le dénominateur de $\psi_\xi$ ne s'annule pas sur $[0,1]$
	si $\xi \neq 1$.\\[.1cm]
	Soit $x \in [0,1]$. Si $\xi = 1$ :
	\[
	  x + \alpha (1-x) + (1-\alpha)(1-x) \ = \
	  x + (1-x)\big(\bcancel{\alpha} +(1-\bcancel{\alpha})\big)
	  \ = \ \bcancel{x} +1- \bcancel{x} \ = \ 1
	\]
	Donc le dénominateur de $\psi_1$ est toujours non nul
	sur $[0,1]$.
	
	

	
	\item Soit $x \in [0,1]$.
	\[
	  \psi_\xi(x) \ = \ 
	  \dfrac{x+ (1-x) \xi}{x + \alpha \, (1-x) + 
	  (1- \alpha)(1-x)\xi}
	  \ = \
	  \dfrac{(1-\xi) \, x + \xi}
	  {(1-\alpha)(1-\xi) \, x + \alpha + (1-\alpha) \, \xi}
	\]
	Donc :
	\[
	  \begin{array}{rcl}
	    \psi_\xi'(x) &=& 
	    \dfrac{(1-\xi)\big((1-\alpha)(1-\xi) \, x + \alpha 
	    + (1-\alpha) \, \xi\big) - \big((1-\xi) \, x + \xi\big)
	    (1-\alpha)(1-\xi)}
	    {\big((1-\alpha)(1-\xi) \, x + \alpha + (1-\alpha) \,
	    \xi\big)^2}
	    \\[.6cm]
	    &=& \dfrac{\bcancel{(1-\alpha)(1-\xi)^2 \, x} + 
	    (1-\xi)\big(\alpha + (1-\alpha) \, \xi\big) - 
	    \bcancel{(1-\alpha)(1-\xi)^2 \, x} - (1-\alpha)(1-\xi)
	    \, \xi}
	    {\big((1-\alpha)(1-\xi) \, x + \alpha + (1-\alpha) \,
	    \xi\big)^2}
	    \\[.6cm]
	    &=& \dfrac{(1-\xi)\big(\alpha + \bcancel{(1-\alpha) \, \xi}
	    - \bcancel{(1-\alpha) \, \xi}\big)}
	    {\big((1-\alpha)(1-\xi) \, x + \alpha + (1-\alpha) \,
	    \xi\big)^2}
	    \\[.6cm]
	    &=& \dfrac{\alpha (1-\xi)}
	    {\big((1-\alpha)(1-\xi) \, x + \alpha + (1-\alpha) \,
	    \xi\big)^2}
	  \end{array}
	\]
	Or $\alpha >1>0$ et $1-\xi \geq 0$ car $\xi \in [0,1]$. Donc :
	$\psi_\xi(x) \geq 0$.
	\conc{On en déduit que $\psi_\xi$ est croissante sur $[0,1]$.}
      \end{noliste}
      
      ~\\[-1.4cm]
    \end{proof}

    
    \item Calculer $\psi_\xi(1)$.
    
    \begin{proof}~\\
      On applique la définition de $\psi_\xi$ :
      \[
        \psi_\xi(1) \ = \ \dfrac{1+(\bcancel{1} - \bcancel{1}) \, \xi}
        {1+ \alpha(\bcancel{1}-\bcancel{1}) + (1-\alpha)
        (\bcancel{1}-\bcancel{1}) \, \xi} \ = \ \dfrac{1}{1} \ = \ 1
      \]
      \conc{$\psi_\xi(1) = 1$}~\\[-1cm]
    \end{proof}

    
    \item 
    \begin{nonoliste}{(i)}
      \item Calculer $\psi_\xi(0)$. On pose désormais
      $
        A(\xi) \ = \ \psi_\xi(0)
      $
      
      \begin{proof}~\\
        On calcule :
        \[
          \psi_\xi(0) \ = \ \dfrac{0 + (1-0) \, \xi}{0 + \alpha
          (1-0) + (1-\alpha)(1-0) \, \xi} \ = \ 
          \dfrac{\xi}{\alpha + (1-\alpha) \, \xi}
        \]
        \conc{$A(\xi) \ = \ \psi_\xi(0) \ = \ 
        \dfrac{\xi}{\alpha + (1-\alpha) \, \xi}$}~\\[-1cm]
      \end{proof}
      
      
      %\newpage

      
      \item Montrer que pour tout $t\in \{0,1,2, \ldots, T-1\}$, 
      $A(X(t)) \leq X(t+1) \leq 1$.
      
      \begin{proof}~\\
      Soit $t \in \{0, \ldots, T-1\}$.
        \begin{noliste}{$\sbullet$}
	  \item D'après la question \itbf{6.} : $X(t+1) = 
	  \psi_{X(t)}(p(t))$.
	  
	  \item D'après la question \itbf{7.a)}, la fonction 
	  $\psi_\xi$ est croissante sur $[0,1]$. Donc :
	  \[
	    \forall x \in [0,1], \ \psi_\xi(0) \leq \psi_\xi(x) 
	    \leq \psi_\xi(1)
	  \]
	  
	  \item Or d'après la question \itbf{7.b)}, $\psi_\xi(1)=1$, et
	  d'après la question \itbf{7.c)(i)}, $\psi_\xi(0)=A(\xi)$.
	  D'où :
	  \[
	    \forall x \in [0,1], \ A(\xi) \leq \psi_\xi(x) \leq 1
	  \]
	  
	  \item L'encadrement précédent est valable pour tout $x \in
	  [0,1]$ et pour tout $\xi \in [0,1]$.\\
	  On l'applique alors à $x=p(t) \in \ ]0,1]$ et $\xi = 
	  X(t) \in [0,1]$ (par définition de $X(t)$).\\
	  On obtient alors :
	  \[
	    A(X(t)) \ \leq \ X(t+1) \ \leq \ 1
	  \]
        \end{noliste}
        \conc{$\forall t \in \{0, \ldots, T-1\}$, $A(X(t)) \leq 
        X(t+1) \leq 1$}
        
        ~\\[-1.4cm]
      \end{proof}

      
      \item Montrer que $\xi \mapsto A(\xi)$ est croissante sur $[0,1]$.
      
      \begin{proof}~
        \begin{noliste}{$\sbullet$}
	  \item On rappelle que, d'après la question \itbf{7.c)(i)} :
	  $\forall \xi \in [0,1]$, $A(\xi) = \dfrac{\xi}{\alpha +
	  (1-\alpha) \, \xi}$.
	  
	  \item La fonction $A$ est dérivable sur $[0,1]$ en tant que 
	  quotient de fonctions dérivables sur $[0,1]$ dont le 
	  dénominateur ne s'annule pas (même raisonnement qu'en question 
	  \itbf{7.a)}).
	  
	  \item Soit $\xi \in [0,1]$.
	  \[
	    A'(\xi) \ = \ \dfrac{\alpha + \bcancel{(1-\alpha) \, \xi}
	    - \bcancel{\xi(1-\alpha)}}{\big(\alpha + (1-\alpha) \,
	    \xi \big)^2} \ > \ 0
	  \]
	  \conc{On en déduit que la fonction $A$ est strictement 
	  croissante sur $[0,1]$.}~\\[-1.4cm]
        \end{noliste}
      \end{proof}
    \end{nonoliste}
  \end{noliste}
    
  \item Justifier l'égalité de variables aléatoires :
  \[
    D(\tau) \ = \ \dfrac{D(\tau)}{D(\tau-1)} \cdot \dfrac{D(\tau-1)}
    {D(\tau-2)} \cdot \cdots \cdot \dfrac{D(2)}{D(1)} \cdot 
    \dfrac{D(1)}{N(0)} \cdot N(0)
  \]
  
  \begin{proof}~
    \begin{noliste}{$\sbullet$}
      \item Par récurrence : $\forall t \in \{1, \ldots, T\}$, 
      $D(t) >0$.\\
      De plus : $N(0)>0$.
      
      
      %\newpage
      
      
      \item Par simplification de fractions :
      \[
	\dfrac{D(\tau)}{\bcancel{D(\tau-1)}}
        \dfrac{\bcancel{D(\tau-1)}}{\bcancel{D(\tau-2)}} 
        \cdot \cdots \cdot 
	\dfrac{\bcancel{D(2)}}{\bcancel{D(1)}}
	\cdot \dfrac{\bcancel{D(1)}}{\bcancel{N(0)}} 
	\cdot \bcancel{N(0)}
	\ = \ D(\tau)
      \]
    \end{noliste}
    \conc{$D(\tau) = \dfrac{D(\tau)}{D(\tau-1)} \cdot 
    \dfrac{D(\tau-1)}{D(\tau-2)} \cdot \cdots \cdot \dfrac{D(2)}{D(1)}
    \cdot \dfrac{D(1)}{N(0)} \cdot N(0)$}
    
    ~\\[-1.4cm]
  \end{proof}
  
  
  
  \noindent
  On pose $\hat{R}(0) = \ln \left( \dfrac{D(1)}{N(0)}\right)$ et 
  $\hat{R}(t) = \ln \left( \dfrac{D(t+1)}{D(t)}\right)$ pour $1 \leq t 
  \leq T-1$.
  
  \item 
  \begin{noliste}{a)}
    \setlength{\itemsep}{2mm}
    \item Montrer que :
    \[
      \E\big(\ln(D(\tau))\big) \ = \ \ln(N(0)) + \E\left( \hat{R}(0)
      + \Sum{t=1}{\tau -1} \hat{R}(t)\right)
    \]
    
    \begin{proof}~
      \begin{noliste}{$\sbullet$}
	\item D'après la question précédente : 
	\[
	  \begin{array}{rcl}
	    \ln(D(\tau)) &=& \ln \left( \dfrac{D(\tau)}{D(\tau -1)}
	    \cdot \dfrac{D(\tau-1)}{D(\tau -2)} \cdot \cdots \cdot
	    \dfrac{D(2)}{D(1)} \cdot \dfrac{D(1)}{N(0)} \cdot N(0)
	    \right)
	    \\[.6cm]
	    &=& \ln\left( \dfrac{D(\tau)}{D(\tau -1)}\right) + 
	    \ln\left(\dfrac{D(\tau-1)}{D(\tau-2)}\right) +
	    \cdots + \ln\left(\dfrac{D(2)}{D(1)}\right) + 
	    \ln\left(\dfrac{D(1)}{N(0)}\right) + \ln(N(0))
	    \\[.6cm]
	    &=& \Sum{t=1}{\tau-1} \ln \left(\dfrac{D(t+1)}{D(t)}\right)
	    + \hat{R}(0) + \ln(N(0))
	    \\[.6cm]
	    &=& \Sum{t=1}{\tau-1} \hat{R}(t) + \hat{R}(0) + \ln(N(0))
	  \end{array}
	\]
	
	
	%\newpage
	
	
	\item La \var $\ln(D(\tau))$ admet une espérance en tant que 
	variable aléatoire finie.\\
	Par linéarité de l'espérance :
	\[
	  \E\big(\ln(D(\tau))\big) \ = \ \E\left(\ln(N(0))\big) +
	  \hat{R}(0) + \Sum{t=1}{\tau-1} \hat{R}(t)\right) \ = \
	  \ln(N(0)) + \E\left(\hat{R}[0) + \Sum{t=1}{\tau-1}
	  \hat{R}(t)\right)
	\]
      \end{noliste}
      \conc{$\E\big(\ln(D(\tau))\big) \ = \ \ln(N(0)) + 
      \E\left(\hat{R}(0) + \Sum{t=1}{\tau-1} 
      \hat{R}(t)\right)$}~\\[-1cm]
    \end{proof}
    
    \item Montrer que : $\dfrac{D(1)}{N(0)} = \dfrac{\alpha \, X(1)}
    {1+(\alpha-1) \, X(1)}$.
    
    \begin{proof}~\\
      On calcule :
      \[
        \begin{array}{rcl@{\qquad}>{\it}R{2.5cm}}
          \dfrac{\alpha \, X(1)}{1+ (\alpha-1) X(1)} &=& 
          \dfrac{\alpha \, \frac{D(1)}{D(1)+N(1)}}{1+ (\alpha -1) \,
          \frac{D(1)}{D(1)+N(1)}}
          & (par définition de $X(1)$)
          \nl
          \nl[-.2cm]
          &=& \dfrac{\frac{\alpha \, D(1)}{\bcancel{D(1)+N(1)}}}
          {\frac{D(1)+N(1) + (\alpha-1) \, D(1)}
          {\bcancel{D(1)+N(1)}}}
          \ = \ \dfrac{\alpha \, D(1)}{\bcancel{D(1)} + N(1) + 
          (\alpha -\bcancel{1}) \, D(1)}
          \\[.6cm] 
          &=& \dfrac{\alpha \, D(1)}{N(1) + \alpha \, D(1)}
          \ = \ \dfrac{\bcancel{\alpha} \, D(1)}
          {\bcancel{\alpha}(1-p(0)) \, N(0) + \bcancel{\alpha} \,
          p(0) \, N(0)}
          & (d'après les questions \itbf{1.a)} et \itbf{1.b)})
          \nl
          \nl[-.2cm]
          &=& \dfrac{D(1)}{(1-\bcancel{p(0)} + \bcancel{p(0)}) \,
          N(0)}
          \ = \ \dfrac{D(1)}{N(0)}
        \end{array}
      \]
      \conc{$\dfrac{D(1)}{N(0)} \ = \ \dfrac{\alpha \, X(1)}
      {1+(\alpha -1) \, X(1)}$}~\\[-1cm]
    \end{proof}
        
    \item Montrer que : $\dfrac{D(t+1)}{D(t)} = \dfrac{\alpha \, X(t+1)}
    {X(t) \big(1+ (\alpha -1) \, X(t+1)\big)}$ pour $1 \leq t \leq 
    T-1$.
    
    \begin{proof}~\\
      Soit $t \in \{1, \ldots, T-1\}$.
      \[
        \begin{array}{rcl@{\qquad}>{\it}R{5cm}}
          \dfrac{\alpha \, X(t+1)}{X(t)\big(1+ (\alpha-1) \, X(t+1)
          \big)}
          &=& \dfrac{\alpha \, \frac{D(t+1)}{D(t+1) + N(t+1)}}
          {\frac{D(t)}{D(t) + N(t)} \, \Big( 1+ (\alpha-1)
          \frac{D(t+1)}{D(t+1) + N(t+1)}\Big)}
          & (par définition\\ de $X(t+1)$)
          \nl
          \nl[-.2cm]
          &=& \dfrac{\alpha \, D(t+1)(D(t)+N(t))}
          {D(t) \big(\bcancel{D(t+1)} + N(t+1) + (\alpha - 
          \bcancel{1}) \, D(t+1)\big)}
          \\[.6cm]
          &=& \dfrac{\alpha \, D(t+1)(D(t) + N(t))}
          {D(t) \big(\alpha \, (1-p(t)) \, N(t) + \alpha (D(t) + 
          p(t) \, N(t))\big)}
          \\[.6cm]
          &=& \dfrac{\bcancel{\alpha} \, D(t+1)(D(t) + N(t))}
          {\bcancel{\alpha} \, D(t) \big((1-\bcancel{p(t)} +
          \bcancel{p(t)}) \, N(t) + D(t)\big)}
          \\[.6cm]
          &=& \dfrac{D(t+1) \, \bcancel{(D(t) + N(t))}}
          {D(t) \bcancel{(N(t)+D(t))}}
          \ = \ \dfrac{D(t+1)}{D(t)}
        \end{array}
      \]
      \conc{$\forall t \in \{1, \ldots, T-1\}$, $\dfrac{D(t+1)}{D(t)}
      = \dfrac{\alpha \, X(t+1)}{X(t)(1+ (\alpha-1) \, 
      X(t+1))}$}~\\[-1cm]
    \end{proof}

    
    %\newpage
    
    
    Pour $x$ et $y$ deux réels strictement positifs, on pose $u(x,y)=
    \ln(\alpha) - \ln(x) + \ln(y) - \ln(1+(\alpha-1)y)$.
    
    \item Montrer que : $\hat{R}(0) = u(1,X(1))$.
    
    \begin{proof}~\\
      Par définition de la fonction $u$ :
      \[
        \begin{array}{rcl@{\qquad}>{\it}R{5cm}}
          u(1,X(1)) &=& \ln(\alpha) - \bcancel{\ln(1)} + \ln(X(1))
          - \ln\big(1+ (\alpha-1) \, X(1)\big)
          \\[.4cm]
          &=& \ln\left( \dfrac{\alpha \, X(1)}{1+ (\alpha-1) \,
          X(1)}\right)
          \ = \ \ln\left(\dfrac{D(1)}{N(0)}\right) 
          & (d'après la question \itbf{9.b)})
          \nl
          \nl[-.2cm]
          &=& \hat{R}(0)
        \end{array}
      \]
      \conc{$\hat{R}(0) = u(1,X(1))$}~\\[-1cm]
    \end{proof}

    
    \item Montrer que : $\hat{R}(t) = u(X(t),X(t+1))$ pour 
    $1 \leq t \leq T-1$.
    
    \begin{proof}~\\
      Soit $t \in \{1, \ldots, T-1\}$.
      \[
        \begin{array}{rcl@{\qquad}>{\it}R{2.5cm}}
          u(X(t),X(t+1)) &=& \ln(\alpha) - \ln(X(t)) + \ln(X(t+1))
          - \ln\big(1+ (\alpha-1) X(t+1)\big)
          \\[.4cm]
          &=& \ln\left( \dfrac{\alpha \, X(t+1)}
          {X(t)\big(1+ (\alpha-1) \, X(t+1)\big)}\right)
          \\[.6cm]
          &=& \ln\left( \dfrac{D(t+1)}{D(t)}\right) 
          & (d'après la question \itbf{9.c)})
          \nl
          \nl[-.2cm]
          &=& \hat{R}(t)
        \end{array}
      \]
      \conc{$\forall t \in \{1, \ldots, T-1\}$, $\hat{R}(t) = 
      u(X(t),X(t+1))$}~\\[-1cm]
    \end{proof}
        
    \item Conclure que :
    \[
      \E\big(\ln(D(\tau))\big) \ = \ \ln(N(0)) + \E\left( u(1,X(1)) + 
      \Sum{t=1}{\tau -1} u(X(t),X(t+1))\right)
    \]
    
    \begin{proof}~
      On utilise les questions précédentes :
      \[
	\begin{array}{rcl@{\qquad}>{\it}R{4cm}}
	  \E\big(\ln(D(\tau))\big) &=& \ln(N(0)) + 
	  \E\left(\hat{R}(0) + \Sum{t=1}{\tau-1} \hat{R}(t)\right)
	  & (d'après la\\ question \itbf{9.a)})
	  \nl
	  \nl[-.2cm]
	  &=& \ln(N(0)) + \E\left(u(1,X(1)) + \Sum{t=1}{\tau-1}
	  u(X(t),X(t+1))\right)
	  & (d'après les questions \itbf{9.d)} et \itbf{9.e)})
	\end{array}
      \]
      \conc{$\E\big(\ln(D(\tau))\big) = \ln(N(0)) + \E\left(u(1,X(1))
      + \Sum{t=1}{\tau-1} u(X(t),X(t+1))\right)$}~\\[-1cm]
    \end{proof}
  \end{noliste}
\end{noliste}

\noindent
On voit donc que maximiser $\E\big(\ln(D(\tau))\big)$ revient à choisir,
à chaque date $t$ telle que $1 \leq t \leq \tau -1$, la valeur $X(t+1)$ 
vérifiant la contrainte $A(X(t)) \leq X(t+1) \leq 1$ de façon à rendre 
maximale l'expression
\[
  \E\left(u(1,X(1)) + \Sum{t=1}{\tau-1} u(X(t), X(t+1))\right)
\]




%\newpage




\section*{Partie III - Programmation dynamique}

\noindent
On expose dans cette partie les deux premières étapes de la méthode de 
la programmation dynamique pour résoudre le problème.

\begin{noliste}{1.}
  \setlength{\itemsep}{4mm}
  \setcounter{enumi}{9}
  \item Soit $B$ un événement. On note $\unq{}_B$ la variable aléatoire
  telle que 
  \[
    \unq{}_B(\omega) \ = \ \left\{
    \begin{array}{cR{3cm}}
      1 & si $\omega \in B$
      \nl
      0 & sinon
    \end{array}
    \right.
  \]
  \begin{noliste}{a)}
    \setlength{\itemsep}{2mm}
    \item Déterminer la loi de $\unq{}_B$.
    
    \begin{proof}~
      \begin{noliste}{$\sbullet$}
	\item Par définition de $\unq{}_B$, cette \var ne prend comme
	valeur que $0$ ou $1$.
	\conc{$\unq{}_B(\Omega) = \{0,1\}$}
	
	\item Soit $\omega \in \Omega$.
	\[
	  \omega \in \Ev{\unq{}_B =1} \ \Leftrightarrow \
	  \unq{}_B (\omega)=1 \ \Leftrightarrow \ \omega \in B
	\]
	D'où : $\Ev{\unq{}_B =1} = B$.
	Ainsi : $\Prob(\Ev{\unq{}_B =1}) = \Prob(B)$.
	\conc{On en déduit : $\unq{}_B \suit 
	\Bern{\Prob(B)}$.}
      \end{noliste}
      
      ~\\[-1.4cm]
    \end{proof}

    
    \item Soient $B$ et $C$ deux événements. Montrer l'égalité de 
    variables aléatoires : $\unq{}_{B \cap C} = \unq{}_B \times 
    \unq{}_C$.
    
    \begin{proof}~\\
    Soit $\omega \in \Omega$. Deux cas se présentent.
      \begin{noliste}{$\sbullet$}
	\item \dashuline{Si $\omega \in B \cap C$}, alors :
	\begin{noliste}{$\stimes$}
	  \item par définition de $\unq{}_{B\cap C}$ : 
	  $\unq{}_{B \cap C}(\omega)=1$,
	  
	  \item comme $\omega \in B \cap C$, alors $\omega \in B$
	  $\ET$ $\omega \in C$.\\
	  Donc, par définition de $\unq{}_B$ et $\unq{}_C$ : 
	  $\unq{}_B(\omega) =1 $ $\ET$ $\unq{}_C(\omega)=1$.\\
	  D'où : $\unq{}_B(\omega) \times \unq{}_C(\omega) = 1 \times
	  1 = 1$.
	\end{noliste}
	On en déduit : $\unq{}_{B \cap C}(\omega) = 1 = 
	\unq{}_B(\omega) \times \unq{}_C(\omega)$.
	
	\item \dashuline{Si $\omega \in 
	\overline{B \cap C} = \overline{B} \cup \overline{C}$}, alors :
	\begin{noliste}{$\stimes$}
	  \item par définition de $\unq{}_{B \cap C}$ : 
	  $\unq{}_{B \cap C}(\omega)=0$.
	  
	  \item comme $\omega \in \overline{B} \cup \overline{C}$, 
	  alors :
	  $\omega \in \overline{B}$ $\OU$ $\omega \in \overline{C}$.\\
	  Donc, par définition de $\unq{}_B$ et $\unq{}_C$ : 
	  $\unq{}_B(\omega) =0$ $\OU$ $\unq{}_C(\omega)=0$.\\
	  D'où : $\unq{}_B(\omega) \times \unq{}_C(\omega) =0$.
	\end{noliste}
	On en déduit : $\unq{}_{B\cap C}(\omega) = 0 = \unq{}_B
	(\omega) \cdot \unq{}_C(\omega)$.
      \end{noliste}
      Finalement : $\forall \omega \in \Omega$, $\unq{}_{B\cap C} 
      (\omega) = \unq{}_B(\omega) \times \unq{}_C(\omega)$.
      \conc{$\unq{}_{B \cap C} = \unq{}_B \times \unq{}_C$}
      
      
      %\newpage
      
      
      ~\\[-1.4cm]
    \end{proof}
    
    \item On suppose que $0 < \Prob(B) < 1$. Si $Y$ est une variable
    aléatoire prenant un nombre fini de valeurs, on définit la {\bf 
    variable aléatoire} notée $\E_B(Y)$ par :
    \[
      \E_B(Y) \ = \ \dfrac{1}{\Prob(B)} \, \E(Y \, \unq{}_B) \, 
      \unq{}_B + \dfrac{1}{\Prob(\bar{B})} \, \E(Y \, \unq{}_{\bar{B}})
      \, \unq{}_{\bar{B}}
    \]
    où $\bar{B}$ désigne l'événement contraire de $B$.
    
    
    
    % Attention : la définition de $\E_B(Y)$ donnée ici est en 
    % fait l'espérance conditionnelle de $Y$ sachant la \var 
    % $\unq{}_B$, et non l'espérance conditionnelle sachant 
    % l'événement $B$
    
    
    %\newpage

    
    \begin{nonoliste}{(i)}
      \item Soient $Y$ et $Z$ deux variables aléatoires prenant un 
      nombre fini de valeurs. Montrer que :
      \[
        \E_B(Y+Z) \ = \ \E_B(Y) + \E_B(Z)
      \]
      
      \begin{proof}~
        \begin{noliste}{$\sbullet$}
	  \item Si $Y$ et $Z$ sont des \var finies, alors $Y+Z$ est
	  aussi une \var finie.\\ 
	  Donc $\E_B(Y+Z)$ est bien définie.
	  
	  \item Par définition de $\E_B(Y)$ :
	  \[
	    \begin{array}{cl}
	      & \E_B(Y+Z) 
	      \\[.2cm]
	      =& \dfrac{1}{\Prob(B)} \, \E\big((Y+Z) \, 
	      \unq{}_B\big) \, \unq{}_B + \dfrac{1}{\Prob(\bar{B})}
	      \, \E\big((Y+Z) \, \unq{}_{\bar{B}}\big) \, 
	      \unq{}_{\bar{B}}
	      \\[.4cm]
	      =& \dfrac{1}{\Prob(B)} \, \E\big(Y \, \unq{}_B +Z \, 
	      \unq{}_B\big) \, \unq{}_B + \dfrac{1}{\Prob(\bar{B})}
	      \, \E\big(Y \, \unq{}_{\bar{B}} +Z \, 
	      \unq{}_{\bar{B}}\big) \, \unq{}_{\bar{B}}
	      \\[.4cm]
	      =& \dfrac{1}{\Prob(B)} \big( \E(Y \, \unq{}_B) + \E(Z \, 
	      \unq{}_B)\big) \, \unq{}_B + \dfrac{1}{\Prob(\bar{B})}
	      \big( \E(Y \, \unq{}_{\bar{B}} + \E(Z \, 
	      \unq{}_{\bar{B}})\big) \, \unq{}_{\bar{B}}
	      \\[.4cm]
	      =& \textcolor{red}{\dfrac{1}{\Prob(B)} \, \E(Y \, 
	      \unq{}_B) \, \unq{}_B}
	      + \textcolor{blue}{\dfrac{1}{\Prob(B)} \, \E(Z \, 
	      \unq{}_B) \, \unq{}_B} 
	      + \textcolor{red}{\dfrac{1}{\Prob(\bar{B})}
	      \, \E(Y \, \unq{}_{\bar{B}}) \, \unq{}_{\bar{B}}} 
	      + \textcolor{blue}{\dfrac{1}{\Prob(\bar{B})} \, \E(Z \, 
	      \unq{}_{\bar{B}}) \, \unq{}_{\bar{B}}}
	      \\[.4cm]
	      =& \textcolor{red}{\E_B(Y)} + \textcolor{blue}{\E_B(Z)}
	    \end{array}
	  \]
        \end{noliste}
        \conc{$\E_B(Y+Z) = \E_B(Y) + \E_B(Z)$}~\\[-1.2cm]
      \end{proof}

      
      \item Montrer que : $\E(\E_B(Y)) = \E(Y)$.
      
      \begin{proof}~
        \begin{noliste}{$\sbullet$}
	  \item D'après la question \itbf{10.a)}, les \var $\unq{}_B$
	  et $\unq{}_{\bar{B}}$ sont des \var finies.\\
	  Donc $\E_B(Y)$ est une \var finie en tant que combinaison 
	  linéaire de \var finies.
	  \conc{Ainsi la \var $\E_B(Y)$ admet une espérance.}
	  
	  \item Par linéarité de l'espérance :
	  \[
	    \begin{array}{rcl}
	      \E\big(\E_B(Y)\big) &=& \E\left( \dfrac{1}{\Prob(B)}
	      \, \E(Y \, \unq{}_B) \, \unq{}_B + 
	      \dfrac{1}{\Prob(\bar{B})} \, \E(Y \, \unq{}_{\bar{B}})
	      \, \unq{}_{\bar{B}}\right)
	      \\[.4cm]
	      &=& \dfrac{1}{\Prob(B)} \, \E(Y \, \unq{}_B) \, 
	      \E(\unq{}_B) + \dfrac{1}{\Prob(\bar{B})} \, \E(Y \,
	      \unq{}_{\bar{B}}) \, \E(\unq{}_{\bar{B}})
	    \end{array}
	  \]
	  Or, d'après la question \itbf{10.a)}, $\unq{}_B \suit
	  \Bern{\Prob(B)}$ et $\unq{}_{\bar{B}} \suit \Bern{\Prob(
	  \bar{B})}$.\\
	  On en déduit : $\E(\unq{}_B) = \Prob(B)$ et $\E(\unq{}_{
	  \bar{B}}) = \Prob(\bar{B})$. D'où :
	  \[
	    \begin{array}{rcl}
	      \E\big(\E_b(Y)\big) &=& \dfrac{1}{\bcancel{\Prob(B)}}
	      \, \E(Y \, \unq{}_B) \, \bcancel{\Prob(B)} + 
	      \dfrac{1}{\bcancel{\Prob(\bar{B})}} \, \E(Y \,
	      \unq{}_{\bar{B}}) \, \bcancel{\Prob(\bar{B})}
	      \\[.4cm]
	      &=& \E(Y \, \unq{}_B) + \E(Y \, \unq{}_{\bar{B}})
	      \ = \ \E(Y \, \unq{}_B + Y \, \unq{}_{\bar{B}})
	      \\[.2cm]
	      &=& \E\big(Y(\unq{}_B + \unq{}_{\bar{B}})\big)
	    \end{array}
	  \]
	  
	  \item Montrons la relation : $\unq{}_B + \unq{}_{\bar{B}}
	  =1$.\\
	  Soit $\omega \in \Omega$. Deux cas se présentent :
	  \end{noliste}
	  \begin{liste}{$\stimes$}
	    \item \dashuline{si $\omega \in B$}, alors : $\unq{}_B
	    (\omega)=1$.\\
	    Comme de plus $B \cap \bar{B} = \emptyset$, alors 
	    $\omega \notin \bar{B}$. Donc : $\unq{}_{\bar{B}}
	    (\omega)=0$.\\
	    D'où : $\unq{}_B(\omega) + \unq{}_{\bar{B}}(\omega) =
	    1+0=1$.
	    
	    
	    %\newpage
	    
	    
	    \item \dashuline{si $\omega \in \bar{B}$}, alors : 
	    $\unq{}_{\bar{B}}(\omega) = 1$.\\
	    Avec le même raisonnement que précédemment : $\unq{}_B
	    (\omega) =0$.\\
	    D'où : $\unq{}_B(\omega) + \unq{}_{\bar{B}}(\omega) = 
	    0+1=1$.
	  \end{liste}
	  \begin{noliste}{}
	  \item Finalement : $\forall \omega \in \Omega$, 
	  $\unq{}_B(\omega)
	  + \unq{}_{\bar{B}}(\omega) = 1$. Donc : $\unq{}_B + 
	  \unq{}_{\bar{B}} = 1$ (où $1$ désigne la variable 
	  aléatoire certaine égalé à $1$).
	  
	  \item[$\sbullet$] On en déduit : $\E\big(\E_B(Y)\big) = 
	  \E\big(Y(\unq{}_B + \unq{}_{\bar{B}})\big) = \E(Y \times 1)
	  =\E(Y)$.
        \end{noliste}
        \conc{$\E\big(\E_B(Y)\big) = \E(Y)$}~\\[-1cm]
      \end{proof}

      
      \item Montrer que : $\E_B(Y \, \unq{}_B) = \E_B(Y) \, \unq{}_B$.
      
      \begin{proof}~
        \begin{noliste}{$\sbullet$}
	  \item D'une part :
	  \[
	    \begin{array}{rcl@{\qquad}>{\it}R{5cm}}
	      \E_B(Y \, \unq{}_B) &=& \dfrac{1}{\Prob(B)} \,
	      \E\big((Y \, \unq{}_B) \, \unq{}_B\big) \, \unq{}_B
	      + \dfrac{1}{\Prob(\bar{B})} \, \E\big((Y \, 
	      \unq{}_{\bar{B}}) \, \unq{}_{\bar{B}}\big) 
	      \unq{}_{\bar{B}}
	      \\[.4cm]
	      &=& \dfrac{1}{\Prob(B)} \, \E(Y \, \unq{}_{B \cap B})
	      \, \unq{}_B + \dfrac{1}{\Prob(\bar{B})} \, \E(Y \,
	      \unq{}_{B \cap \bar{B}}) \, \unq{}_{\bar{B}}
	      & (d'après la\\ question \itbf{10.b)})
	      \nl
	      \nl[-.2cm]
	      &=& \dfrac{1}{\Prob(B)} \, \E(Y \, \unq{}_B) \,
	      \unq{}_B + \dfrac{1}{\Prob(\bar{B})} \, \E(Y \,
	      \unq{}_{\emptyset}) \, \unq{}_{\bar{B}}
	    \end{array}
	  \]
	  Or, par définition d'une variable aléatoire 
	  indicatrice : $\unq{}_{\emptyset}=0$ (où $0$ 
	  désigne la variable aléatoire certaine égale à $0$).\\
	  Ainsi : $\E_B(Y \, \unq{}_B) = \dfrac{1}{\Prob(B)} \,
	  \E(Y \, \unq{}_B) \, \unq{}_B$.
	  
	  \item D'autre part :
	  \[
	    \begin{array}{rcl}
	      \E_B(Y) &=& \left(\dfrac{1}{\Prob(B)} \, 
	      \E(Y \, \unq{}_B) \, \unq{}_B + \dfrac{1}{\Prob(\bar{B})}
	      \, \E(Y \, \unq{}_{\bar{B}}) \, \unq{}_{\bar{B}}\right)
	      \, \unq{}_B
	      \\[.6cm]
	      &=& \dfrac{1}{\Prob(B)} \, \E(Y \, \unq{}_B) \, 
	      \unq{}_{B \cap B} + \dfrac{1}{\Prob(\bar{B})} \,
	      \E(Y \, \unq{}_{\bar{B}}) \, \unq{}_{B \cap \bar{B}}
	      \\[.6cm]
	      &=& \dfrac{1}{\Prob(B)} \, \E(Y \, \unq{}_B) \, 
	      \unq{}_{B} + \bcancel{\dfrac{1}{\Prob(\bar{B})} \,
	      \E(Y \, \unq{}_{\bar{B}}) \, \unq{}_{\emptyset}}
	      \\[.6cm]
	      &=& \E_B(Y \, \unq{}_B)
	    \end{array}
	  \]
        \end{noliste}
        \conc{$\E_B(Y \, \unq{}_B) = \E_B(Y) \, \unq{}_B$}~\\[-1.2cm]
      \end{proof}
    \end{nonoliste}
  \end{noliste}
    
  \item On suppose dans cette question que, quand l'événement $\Ev{\tau 
  =T}$ est réalisé, $X(1)$, $\ldots$, $X(T-1)$ sont connus. Comme on 
  l'a vu précédemment, si on pose $x=X(T-1)$, le meilleur choix à faire
  est alors de prendre pour $X(T)$ la valeur $y^*(x,T-1) \in [A(x),1]$
  qui maximise $u(x,y)$.
  \begin{noliste}{a)}
    \setlength{\itemsep}{2mm}
    \item Montrer que : $y^*(x,T-1)=1$.
    
    \begin{proof}~
      \begin{noliste}{$\sbullet$}
	\item On cherche à maximiser la fonction $f:y \mapsto 
	u(x,y)$ sur $]0,1]$.\\
	La fonction $f$ est dérivable sur $]0,1]$ en tant que composée
	et somme de fonctions dérivables sur $]0,1]$. En effet :
	$1+ (\alpha -1)y \in \ ]1, \alpha]$.
	
	
	%\newpage
	
	
	\item Soit $y \in \ ]0,1]$.
	\[
	  f'(y) \ = \ \dfrac{1}{y} - \dfrac{\alpha -1}{1+(\alpha-1)y}
	  \ = \ \dfrac{1+\bcancel{(\alpha-1)y} - \bcancel{(\alpha 
	  -1)y}}{y(1+(\alpha-1)y)} \ = \
	  \dfrac{1}{y (1+ (\alpha-1)y)}
	\]
	Or $\alpha-1 >0$, car $\alpha >1$. De plus $y>0$. Donc 
	$f'(y)>0$.\\
	On obtient le tableau de variations suivant :
	\begin{center}
        \begin{tikzpicture}[scale=0.8, transform shape]
          \tkzTabInit[lgt=4,espcl=3] 
          {$y$ /1, Signe de $f'(y)$ /1, Variations de $f$ 
	  /2} 
          {$0$, $1$}%
          \tkzTabLine{ , + , } 
          \tkzTabVar{-/$-\infty$, +/$f(1)$}
        \end{tikzpicture}
      \end{center}
	Donc $f$ atteint son unique maximum pour $y=1$.
      \end{noliste}
      \conc{Autrement dit : $y^*(x,T-1)=1$.}
      
      ~\\[-1.4cm]
    \end{proof}
    
    \item Montrer que : $u(x,1)=-\ln(x)$.
    
    \begin{proof}~\\
      On calcule :
      \[
        \begin{array}{rcl}
          u(x,1) &=& \ln(\alpha) - \ln(x) + \bcancel{\ln(1)} - 
          \ln(1+(\alpha -1) \times 1)
          \\[.2cm]
          &=& \ln(\alpha) - \ln(x) - \ln(\bcancel{1} + \alpha -
          \bcancel{1})
          \\[.2cm]
          &=& \bcancel{\ln(\alpha)} - \ln(x) - \bcancel{\ln(\alpha)}
          \\[.2cm]
          &=& - \ln(x)
        \end{array}
      \]
      \conc{$u(x,1) = -\ln(x)$}~\\[-1cm]
    \end{proof}
  \end{noliste}
  
  \item On suppose maintenant que, quand l'événement $\Ev{\tau \geq 
  T-1}$ est réalisé, les réels $X(1)$, $\ldots$, $X(T-2)$ sont connus. 
  \\[.1cm]
  La 
  stratégie reste donc de choisir $X(T-1)$ et $X(T)$ de façon à 
  maximiser $\E\left(\Sum{t=T-2}{\tau-1} u(X(t), 
  X(t+1))\right)$.\\[.1cm]
  La variable aléatoire $\tau$ prend les deux valeurs $T-1$ et $T$ 
  avec les probabilités respectives 
  \[
    \Prob_{\Ev{\tau \geq T-1}}(\Ev{\tau = T-1}) \quad \text{et} 
    \quad \Prob_{\Ev{\tau \geq T-1}}(\Ev{\tau = T})
  \]
  \end{noliste}
  
  
  %\newpage
  
  
  \begin{noliste}{a)}
    \setlength{\itemsep}{2mm}
    \item Montrer que :
    \[
      \unq{}_{\Ev{\tau \geq T-1}} \ \Sum{t=T-2}{\tau-1} 
      u(X(t),X(t+1))
      \ = \ \unq{}_{\Ev{\tau \geq T-1}} \, u(X(T-2),X(T-1)) + 
      \unq{}_{\Ev{\tau = T}} \, u(X(T-1),X(T))
    \]
    
    \begin{proof}~\\
      Soit $\omega \in \Omega$. Deux cas se présentent.
      \begin{noliste}{$\sbullet$}
        \item \dashuline{Soit $\omega \in \Ev{\tau \geq T-1}$}, alors
        $\unq{}_{\Ev{\tau \geq T-1}}(\omega)=1$.\\[.1cm]
        Comme la \var $\tau$ est à valeurs dans $\{1, \ldots , T\}$, 
        deux cas se présentent :
        \begin{noliste}{$\stimes$}
	  \item soit $\omega \in \Ev{\tau = T-1}$. Dans ce cas : 
	  $\unq{}_{\Ev{\tau = T}}(\omega)=0$.\\
	  De plus : $\omega \in \Ev{\tau = T-1} \ \Leftrightarrow \
	  \tau(\omega) = T-1$. Donc, d'une part :
	  \[
	    \begin{array}{rcl}
	      \left(\Sum{t=T-2}{\tau(\omega)-1} u(X(t),X(t+1))\right)
	      \, \unq{}_{\Ev{\tau \geq T-1}}(\omega) 
	      &=&
	      \left(\Sum{t=T-2}{(T-1)-1} u(X(t),X(t+1))\right)
	      \times 1
	      \\[.6cm]
	      &=& \Sum{t=T-2}{T-2} u(X(t),X(t+1))
	      \\[.6cm]
	      &=& u(X(T-2),X(T-1))
	    \end{array}
	  \]
	  D'autre part :
	  \[
	    \begin{array}{cl}
	      & u(X(T-2),X(T-1)) \, \unq{}_{\Ev{\tau \geq T-1}}
	      (\omega) + u(X(T-1),X(T)) \, \unq{}_{\Ev{\tau =T}}
	      (\omega)
	      \\[.4cm]
	      =& u(X(T-2),X(T-1)) \times 1 + \bcancel{u(X(T-1),X(T))
	      \times 0}
	      \\[.4cm]
	      =& u(X(T-2),X(T-1))
	    \end{array}
	  \]
	  On en déduit :
	  \[
	    \begin{array}{cl}
	      & \left( \Sum{t=T-2}{\tau(\omega)-1} 
	      u(X(t),X(t+1))\right)   \,
	      \unq{}_{\Ev{\tau \geq T-1}}(\omega)
	      \\[.8cm]
	      =& u(X(T-2),X(T-1)) \, \unq{}_{\Ev{\tau \geq T-1}}(\omega)
	      + u(X(T-1),X(T)) \, \unq{}_{\Ev{\tau =T}}(\omega)
	    \end{array}
	  \]
	  
	  \item soit $\omega \in \Ev{\tau = T}$. Dans ce cas : 
	  $\unq{}_{\Ev{\tau = T}}(\omega)=1$.\\
	  De plus : $\omega \in \Ev{\tau = T} \ \Leftrightarrow \
	  \tau(\omega) = T$. Donc, d'une part :
	  \[
	    \begin{array}{rcl}
	      \left(\Sum{t=T-2}{\tau(\omega)-1} u(X(t),X(t+1))\right)
	      \, \unq{}_{\Ev{\tau \geq T-1}}(\omega) 
	      &=&
	      \left(\Sum{t=T-2}{T-1} u(X(t),X(t+1))\right)
	      \times 1
	      \\[.6cm]
	      &=& u(X(T-2),X(T-1)) + u(X(T-1),X(T))
	    \end{array}
	  \]
	  D'autre part :
	  \[
	    \begin{array}{cl}
	      & u(X(T-2),X(T-1)) \, \unq{}_{\Ev{\tau \geq T-1}}
	      (\omega) + u(X(T-1),X(T)) \, \unq{}_{\Ev{\tau =T}}
	      (\omega)
	      \\[.4cm]
	      =& u(X(T-2),X(T-1)) \times 1 + u(X(T-1),X(T)) \times 1
	      \\[.4cm]
	      =& u(X(T-2),X(T-1)) + u(X(T-1),X(T))
	    \end{array}
	  \]
	  On en déduit :
	  \[
	    \begin{array}{cl}
	      & \left( \Sum{t=T-2}{\tau(\omega)-1} 
	      u(X(t),X(t+1))\right)   \,
	      \unq{}_{\Ev{\tau \geq T-1}}(\omega)
	      \\[.8cm]
	      =& u(X(T-2),X(T-1)) \, \unq{}_{\Ev{\tau \geq T-1}}(\omega)
	      + u(X(T-1),X(T)) \, \unq{}_{\Ev{\tau =T}}(\omega)
	    \end{array}
	  \]
        \end{noliste}
        
        
        %\newpage
        
        
        \item \dashuline{Soit $\omega \in \overline{\Ev{\tau \geq 
	T-1}}$}. Comme la \var $\tau$ est à valeurs entières :
	\[
	  \overline{\Ev{\tau \geq T-1}} \ = \ \Ev{\tau < T-1} \ = \
	  \Ev{\tau \leq T-2}
	\]
	Donc $\omega \notin \Ev{\tau \geq T-1}$ et $\omega \notin 
	\Ev{\tau =T}$.
	Ainsi : $\unq{}_{\Ev{\tau \geq T-1}}(\omega)=0$ et 
	$\unq{}_{\Ev{\tau=T}}(\omega)=0$.\\
	On a donc bien :
	\[
	  \begin{array}{c}
	    \left( \Sum{t=T-2}{\tau(\omega)-1} u(X(t),X(t+1))\right) 
	    \, \unq{}_{\Ev{\tau \geq T-1}}(\omega)
	    \\[.2cm]
	    \shortparallel
	    \\
	    0
	    \\[-.1cm]
	    \shortparallel
	    \\[.1cm]
	    u(X(T-2),X(T-1)) \, \unq{}_{\Ev{\tau \geq T-1}}(\omega)
	    + u(X(T-1),X(T)) \, \unq{}_{\Ev{\tau =T}}(\omega)
	  \end{array}
	\]
      \end{noliste}
      Finalement, pour tout $\omega \in \Omega$ :
      \[
        \begin{array}{cl}
          & \left( \Sum{t=T-2}{\tau(\omega)-1} u(X(t),X(t+1))\right) \,
          \unq{}_{\Ev{\tau \geq T-1}}(\omega)
          \\[.8cm]
          =& u(X(T-2),X(T-1)) \, \unq{}_{\Ev{\tau \geq T-1}}(\omega)
          + u(X(T-1),X(T)) \, \unq{}_{\Ev{\tau =T}}(\omega)
        \end{array}
      \]
      \conc{$\unq{}_{\Ev{\tau \geq T-1}} \ \Sum{t=T-2}{\tau-1} 
      u(X(t),X(t+1))
      \ = \ \unq{}_{\Ev{\tau \geq T-1}} \, u(X(T-2),X(T-1)) + 
      \unq{}_{\Ev{\tau = T}} \, u(X(T-1),X(T))$.}
      
      
        
        
        %\newpage
        
        
        ~\\[-1.4cm]
    \end{proof}
    
    \item Montrer que :
    \[
     \begin{array}{cl}
      & \unq{}_{\Ev{\tau \geq T-1}} \ \E_{\Ev{\tau \geq T-1}} 
      \left( \Sum{t=T-2}{\tau -1} u(X(t),X(t+1)\right) 
      \\[.8cm]
      = &
      \E_{\Ev{\tau \geq T-1}} \Big( u(X(T-2),X(T-1)) \cdot
      \unq{}_{\Ev{\tau \geq T-1}} + u(X(T-1),X(T)) \cdot 
      \unq{}_{\Ev{\tau = T}}\Big)
     \end{array}
    \]
    
    \begin{proof}~\\
      On note $Y = \Sum{t=T-2}{\tau -1} u(X(t),X(t+1))$ et $B=
      \Ev{\tau \geq T-1}$.\\[.1cm]
      D'après la question \itbf{10.c)(iii)} :
      \[
        \begin{array}{rcl}
          \E_{\Ev{\tau \geq T-1}} \left( \Sum{t=T-2}{\tau -1}
          u(X(t),X(t+1))\right) \ \unq{}_{\Ev{\tau \geq T-1}}
          &=& \E_B(Y) \, \unq{}_B
          \ = \ \E_B(Y \, \unq{}_B) 
          \ = \ \E_B(\unq{}_B \ Y)
          \\[.4cm]
          &=& \E_{\Ev{\tau \geq T-1}} \left( 
          \unq{}_{\Ev{\tau \geq T-1}} \ \Sum{t=T-2}{\tau -1}
          u(X(t),X(t+1))\right)
        \end{array}
      \]
      \conc{Ainsi, d'après la question \itbf{12.a)} :\\
      $\begin{array}[t]{cl}
        & \E_{\Ev{\tau \geq T-1}} \left( \Sum{t=T-2}{\tau -1}
        u(X(t),X(t+1))\right) \ \unq{}_{\Ev{\tau \geq T-1}}
        \\[.8cm]
        = & 
        \E_{\Ev{\tau \geq T-1}} \Big( u(X(T-2),X(T-1)) \cdot
	\unq{}_{\Ev{\tau \geq T-1}} + u(X(T-1),X(T)) \cdot 
	\unq{}_{\Ev{\tau = T}}\Big)
      \end{array}$.}~\\[-1cm]
    \end{proof}
    
    
    %\newpage
    
    
    \item Montrer que :
    \[
     \begin{array}{cl}
      & \E_{\Ev{\tau \geq T-1}}\left( \unq{}_{\Ev{\tau \geq T-1}} \, 
      u(X(T-2),X(T-1)) + \unq{}_{\Ev{\tau = T}} \, u(X(T-1), 
      X(T))\right)
      \\[.4cm]
      = &
      u(X(T-2), X(T-1)) \, \unq{}_{\Ev{\tau \geq T-1}} + 
      \dfrac{\Prob(\Ev{\tau =T})}{\Prob(\Ev{\tau \geq T-1})} \, 
      u(X(T-1), X(T)) \, \unq{}_{\Ev{\tau \geq T-1}}
     \end{array}
    \]
    
    \begin{proof}~
      \begin{noliste}{$\sbullet$}
	\item D'après la question \itbf{10.c)(i)} :
	\[
	  \begin{array}{cl}
	    & \E_{\Ev{\tau \geq T-1}}\left(u(X(T-2),X(T-1)) \, 
	    \, \unq{}_{\Ev{\tau \geq T-1}} + u(X(T-1),X(T)) \,
	    \unq{}_{\Ev{\tau =T}}\right)
	    \\[.4cm]
	    =& \E_{\Ev{\tau \geq T-1}}\left(u(X(T-2),X(T-1)) \, 
	    \, \unq{}_{\Ev{\tau \geq T-1}}\right)
	    + 
	    \E_{\Ev{\tau \geq T-1}}\left(u(X(T-1),X(T)) \,
	    \unq{}_{\Ev{\tau =T}}\right)
	  \end{array}
	\]
	
	\item Déterminons $\E_{\Ev{\tau \geq T-1}}\left(u(X(T-2),X(T-1)) 
	\, \unq{}_{\Ev{\tau \geq T-1}}\right)$.\\[.1cm]
	Tout d'abord, d'après la question \itbf{10.c)(iii)} :
	\[
	  \E_{\Ev{\tau \geq T-1}}\left(u(X(T-2),X(T-1)) \, 
	    \, \unq{}_{\Ev{\tau \geq T-1}}\right)
	  \ = \
	  \E_{\Ev{\tau \geq T-1}}\big(u(X(T-2),X(T-1)) \big)
	    \, \unq{}_{\Ev{\tau \geq T-1}}
	\]
	Notons que $\lambda = u(X(T-2),X(T-1))$ est un réel (et non une 
	\var) et rappelons que\\ 
	$\unq{}_{\Ev{\tau \geq T-1}} \suit \Bern{
	\Prob(\Ev{\tau \geq T-1})}$. On en déduit, d'après la 
	définition donnée au \itbf{10.c)}:
	\[
	  \begin{array}{cl}
	    & \E_{\Ev{\tau \geq T-1}}\big(u(X(T-2),X(T-1)) \big)
	    \ = \ \E_{\Ev{\tau \geq T-1}}(\lambda)
	    \\[.4cm]
	    =& \dfrac{1}{\Prob(\Ev{\tau \geq T-1})} \,
	    \E\big(\lambda \, \unq{}_{\Ev{\tau \geq T-1}}\big)
	    \, \unq{}_{\Ev{\tau \geq T-1}}
	    +
	    \dfrac{1}{\Prob(\overline{\Ev{\tau \geq T-1}})} \,
	    \E\big(\lambda \, \unq{}_{\overline{\Ev{\tau \geq 
	    T-1}}}\big)
	    \, \unq{}_{\overline{\Ev{\tau \geq T-1}}}
	    \\[.6cm]
	    =& \dfrac{\lambda}{\Prob(\Ev{\tau \geq T-1})} \,
	    \E\big(\unq{}_{\Ev{\tau \geq T-1}}\big)
	    \, \unq{}_{\Ev{\tau \geq T-1}}
	    +
	    \dfrac{\lambda}{\Prob(\overline{\Ev{\tau \geq T-1}})} \,
	    \E\big(\unq{}_{\overline{\Ev{\tau \geq T-1}}}\big)
	    \, \unq{}_{\overline{\Ev{\tau \geq T-1}}}
	    \\[.6cm]
	    =& \dfrac{\lambda}{\bcancel{\Prob(\Ev{\tau \geq T-1})}} \,
	    \bcancel{\Prob(\Ev{\tau \geq T-1})}
	    \, \unq{}_{\Ev{\tau \geq T-1}}
	    +
	    \dfrac{\lambda}{\bcancel{\Prob(\overline{ \Ev{\tau \geq 
	    T-1}})}} \,
	    \bcancel{\Prob(\overline{\Ev{\tau \geq T-1}})}
	    \, \unq{}_{\overline{\Ev{\tau \geq T-1}}}
	    \\[.6cm]
	    =& \lambda \left(\unq{}_{\Ev{\tau \geq T-1}} + 
	    \unq{}_{\overline{\Ev{\tau \geq T-1}}}\right)
	    \ = \ \lambda \times 1
	    \\[.4cm]
	    =&  u(X(T-2),X(T-1))
	  \end{array}
	\]
	D'où :
	\[
	  \E_{\Ev{\tau \geq T-1}}\left(u(X(T-2),X(T-1)) 
	  \, \unq{}_{\Ev{\tau \geq T-1}}\right)
	  \ = \ u(X(T-2),X(T-1)) \, \unq{}_{\Ev{\tau \geq T-1}}
	\]
	
      \item Déterminons $\E_{\Ev{\tau \geq T-1}}\left(
          u(X(T-1),X(T)) \, \unq{}_{\Ev{\tau =T}}\right)$.\\[.1cm]
	On note $\mu = u(X(T-1),X(T)) \in \R$. \\
        Par définition de $\E_{\Ev{\tau \geq T-1}}\left(\mu \,
          \unq{}_{\Ev{\tau = T}}\right)$ et toujours d'après
        \itbf{10.c)} :
	\[
	  \begin{array}{cl}
	    & \E_{\Ev{\tau \geq T-1}}\left(\mu \, \unq{}_{\Ev{\tau 
	    = T}}\right)
	    \\[.4cm]
	    =& \dfrac{1}{\Prob(\Ev{\tau \geq T-1})} \, \E\left(\mu \,
	    \unq{}_{\Ev{\tau =T}} \times \unq{}_{\Ev{\tau \geq T-1}}
	    \right) \, \unq{}_{\Ev{\tau \geq T-1}}
	    +
	    \dfrac{1}{\Prob(\overline{\Ev{\tau \geq T-1}})} \, 
	    \E\left(\mu \, \unq{}_{\Ev{\tau =T}} \times 
	    \unq{}_{\overline{\Ev{\tau \geq T-1}}}
	    \right) \, \unq{}_{\overline{\Ev{\tau \geq T-1}}}
	  \end{array}
	\]
	Or, d'après la question \itbf{10.b)} :
	\begin{noliste}{$\stimes$}
	  \item $\unq{}_{\Ev{\tau =T}} \times \unq{}_{\Ev{\tau \geq 
	  T-1}} \ = \ \unq{}_{\Ev{\tau =T} \cap \Ev{\tau \geq T-1}}
	  \ = \ \unq{}_{\Ev{\tau =T}}$ car $\Ev{\tau =T} \subset
	  \Ev{\tau \geq T-1}$.
	  
	  \item $\unq{}_{\Ev{\tau =T}} \times \unq{}_{\overline{\Ev{
	  \tau \geq T-1}}} \ = \ \unq{}_{\Ev{\tau =T}} \times 
	  \unq{}_{\Ev{\tau < T-1}} \ = \ \unq{}_{\Ev{\tau =T} \cap 
	  \Ev{\tau < T-1}} \ = \ \unq{}_\emptyset \ = \ 0$.
	\end{noliste}
	
	
	%\newpage
	
	
	Donc : 
	\[
	  \begin{array}{rcl}
	    \E_{\Ev{\tau \geq T-1}}\left(\mu \, \unq{}_{\Ev{\tau 
	    = T}}\right)
	    &=& \dfrac{1}{\Prob(\Ev{\tau \geq T-1})} \, \E\left(\mu \,
	    \unq{}_{\Ev{\tau =T}} \right) \, \unq{}_{\Ev{\tau 
	    \geq T-1}}
	    \\[.6cm]
	    &=& \dfrac{\mu}{\Prob(\Ev{\tau \geq T-1})} \, \E\left(
	    \unq{}_{\Ev{\tau =T}} \right) \, \unq{}_{\Ev{\tau 
	    \geq T-1}}
	    \\[.6cm]
	    &=& \dfrac{\mu}{\Prob(\Ev{\tau \geq T-1})} \, 
	    \Prob(\Ev{\tau =T}) \, \unq{}_{\Ev{\tau 
	    \geq T-1}}
	    \\[.6cm]
	    &=& \dfrac{\Prob(\Ev{\tau =T})}{\Prob(\Ev{\tau \geq T-1})}
	    \, u(X(T-1),X(T)) \, \unq{}_{\Ev{\tau \geq T-1}}
	  \end{array}
	\]
      \end{noliste}
      
      \conc{On en déduit : $
      \begin{array}[t]{cl}
      & \E_{\Ev{\tau \geq T-1}}\left( \unq{}_{\Ev{\tau \geq T-1}} \, 
      u(X(T-2),X(T-1)) + \unq{}_{\Ev{\tau = T}} \, u(X(T-1), 
      X(T))\right)
      \\[.4cm]
      = &
      u(X(T-2), X(T-1)) \, \unq{}_{\Ev{\tau \geq T-1}} + 
      \dfrac{\Prob(\Ev{\tau =T})}{\Prob(\Ev{\tau \geq T-1})} \, 
      u(X(T-1), X(T)) \, \unq{}_{\Ev{\tau \geq T-1}}
     \end{array}$.}~\\[-.8cm]
    \end{proof}
    
    \item On suppose que $X(T-1)$ est donné.
    \begin{nonoliste}{(i)}
      \item Montrer que le meilleur choix pour $X(T)$ est $1$.
      
      \begin{proof}~\\
        Comme $\Ev{\tau \geq T-1}$ est réalisé, $X(1)$, $\ldots$, 
        $X(T-2)$ sont connus.\\
        On suppose de plus ici que $X(T-1)$ est connu.\\
        D'après la question \itbf{11.}, si $X(1)$, $\ldots$, $X(T-1)$
        sont connus, alors le meilleur choix pour $X(T)$ est 
        $y^*(X(T-1),T-1)$.
        \conc{D'après la question \itbf{11.a)}, on en déduit que le
        meilleur choix pour $X(T)$ est $1$.}
        
        ~\\[-1.4cm]
      \end{proof}

      
      \item Montrer que pour un tel choix $u(X(T-1), X(T)) = 
      - \ln (X(T-1))$.
      
      \begin{proof}~\\
        Si $X(T)=1$, alors, d'après la question \itbf{11.b)} :
        \[
          u(X(T-1),X(T)) \ = \ u(X(T-1),1) \ = \ -\ln(X(T-1))
        \]
        \conc{Pour $X(T)=1$, $u(X(T-1),X(T)) = - \ln(X(T-1))$}~\\[-1cm]
      \end{proof}
    \end{nonoliste}
    
    
    %\newpage
    
    
    \item Montrer que : $\Prob_{\Ev{\tau \geq T-1}}(\Ev{\tau = T})
    =1- H(T-1)$.
    
    \begin{proof}~
      \begin{noliste}{$\sbullet$}
	\item D'une part, comme $\Ev{\tau =T} \subset \Ev{\tau \geq 
	T-1}$ :
	\[
	  \Prob_{\Ev{\tau \geq T-1}}(\Ev{\tau =T}) \ = \
	  \dfrac{\Prob(\Ev{\tau \geq T-1} \cap \Ev{\tau =T})}
	  {\Prob(\Ev{\tau \geq T-1})} \ = \ 
	  \dfrac{\Prob(\Ev{\tau =T})}{\Prob(\Ev{\tau \geq T-1})}
	\]
	
	\item D'autre part, par définition de $H$ (question \itbf{3.a)} 
	:
	\[
	  \begin{array}{rcl}
	    1-H(T-1) &=& 1- \Prob_{\Ev{\tau \geq T-1}}(\Ev{\tau = T-1})
	    \ = \ \Prob_{\Ev{\tau \geq T-1}}(\overline{\Ev{\tau = T-1}})
	    \\[.4cm]
	    &=& \Prob_{\Ev{\tau \geq T-1}}(\Ev{\tau \neq T-1})
	    \ = \ \dfrac{\Prob(\Ev{\tau \geq T-1} \cap \Ev{\tau \neq 
	    T-1})}{\Prob(\Ev{\tau \geq T-1})}
	    \\[.4cm]
	    &=& \dfrac{\Prob(\Ev{\tau =T})}{\Prob(\Ev{\tau \geq T-1})}
	  \end{array}
	\]
      \end{noliste}
      \conc{Finalement : $\Prob_{\Ev{\tau \geq T-1}}(\Ev{\tau =T})
      = 1-H(T-1)$.}~\\[-1cm]
    \end{proof}
    
    On veut maintenant choisir la stratégie optimale à la date $T-2$.
    
    \item Montrer qu'on doit choisir pour $X(T-1)$ la valeur $y^*(
    X(T-2),T-2) \in [A(X(T-2)),1]$ de telle sorte que 
    \[
      \phi(y) \ = \ u(X(T-2),y) - (1-H(T-1)) \, \ln(y)
    \]
    soit maximal.
  \end{noliste}
    
    \begin{proof}~
      \begin{noliste}{$\sbullet$}
        \item D'après l'énoncé du début de la question \itbf{12.}, on 
        souhaite choisir le réel $X(T-1)$ qui rend maximal 
        $\E\left( \Sum{t=T-2}{\tau -1} u(X(t),X(t+1))\right)$.
        
        \item On note $Y= \Sum{t=T-2}{\tau -1} u(X(t),X(t+1))$ et 
        $B= \Ev{\tau \geq T-1}$. On obtient : 
        \[
          \begin{array}{rcl@{\qquad}>{\it}R{5cm}}
            \E\left( Y \right)
            &=&  \E\left( Y \big(
	    \unq{}_B + \unq{}_{\bar{B}}\big) \right)
	    & (car $\unq{}_B + \unq{}_{\bar{B}} = 1$)
	    \nl
	    \nl[-.2cm]
	    &=& \E\left(Y \,
	    \unq{}_B\right) 
	    + \E\left(Y \, \unq{}_{\bar{B}} \right)
	    & (par linéarité de l'espérance)
          \end{array}
        \]
	
	\item Or, comme $\tau$ est à valeurs entières :
	\[
	  \overline{\Ev{\tau \geq T-1}} \ = \ \Ev{\tau < T-1} \ = \
	  \Ev{\tau \leq T-2}
	\]
	On en déduit que, si l'événement $\overline{\Ev{\tau \geq
            T-1}}$ est réalisé, alors l'ensemble d'indices de la somme
        $\Sum{t=T-2}{\tau -1} u(X(t),X(t+1))$ est vide.\\
        Ainsi, d'après l'énoncé : $ Y \, \unq{}_{\bar{B}} \ = \ \Big(
        \Sum{t=T-2}{\tau -1} u(X(t),X(t+1))\Big) \,
        \unq{}_{\overline{\Ev{\tau \geq T-1}}} \ = \ 0$.
	
	
	%\newpage
	
	
	\item On en déduit :
	\end{noliste}
	\[
	  \begin{array}{cl@{\qquad}>{\it}R{5cm}}
	    & \E\left( \Sum{t=T-2}{\tau -1} u(X(t),X(t+1))\right)
	    \ = \ \E \left(Y \, \unq{}_B \right)
	    \ = \ \E\Big( \E_B(Y \, \unq{}_B)\Big)
	    & (d'après \itbf{10.c)(ii)})
	    \nl
	    \nl[-.2cm]
	    =& \E\Big( \E_B(Y) \, \unq{}_B\Big)
	    \ = \
	    \E\left( \E_{\Ev{\tau \geq T-1}}\Big( 
	    \Sum{t=T-2}{\tau -1} u(X(t),X(t+1)) \Big) \ 
	    \unq{}_{\Ev{\tau \geq T-1}}\right)
	    \\[.8cm]
	    =& \E\Big( \E_{\Ev{\tau \geq T-1}}\Big[ 
	    u(X(T-2),X(T-1)) \, \unq{}_{\Ev{\tau \geq T-1}} + 
	    u(X(T-1),X(T)) \, \unq{}_{\Ev{\tau =T}}
	    \Big] \Big)
	    & (d'après \itbf{12.b)})
	    \nl
	    \nl[-.2cm]
	    =& \E\left( u(X(T-2),X(T-1)) \, \unq{}_{\Ev{\tau \geq T-1}}
	    + \dfrac{\Prob(\Ev{\tau =T})}{\Prob(\Ev{\tau \geq T-1})}
	    \, u(X(T-1),X(T)) \, \unq{}_{\Ev{\tau \geq T-1}}\right)
	    & (d'après \itbf{12.c)})
	    \nl
	    \nl[-.2cm]
	    =& \E\Big( u(X(T-2),X(T-1)) \, \unq{}_{\Ev{\tau \geq T-1}}
	    + (1- H(T-1))
	    \, u(X(T-1),X(T)) \, \unq{}_{\Ev{\tau \geq T-1}}\Big)
	    & (d'après \itbf{12.e)})
	  \end{array}
	\]
	\begin{noliste}{}
	\item De là, par linéarité de l'espérance :
	\[
	  \begin{array}{cl}
	    & \E\left( \Sum{t=T-2}{\tau -1} u(X(t),X(t+1))\right)
	    \\[.6cm]
	    =& u(X(T-2),X(T-1)) \, \E\left( \unq{}_{\Ev{\tau \geq T-1}}
	    \right) + (1-H(T-1)) \, u(X(T-1),X(T)) \, \E\left( 
	    \unq{}_{\Ev{\tau \geq T-1}} \right)
	    \\[.4cm]
	    =& \Big(u(X(T-2),X(T-1)) 
	    + (1-H(T-1)) \, u(X(T-1),X(T)) \Big) \,
	    \E\left(\unq{}_{\Ev{\tau \geq T-1}}\right)
	  \end{array}
	\]
	
	\item[$\sbullet$] On rappelle que $\unq{}_{\Ev{\tau \geq T-1}} 
	\suit 
	\Bern{\Prob(\Ev{\tau \geq T-1})}$.\\[.1cm]
	Donc $\E\left(\unq{}_{\Ev{\tau \geq T-1}}\right) =  
	\Prob(\Ev{\tau \geq T-1})$ 
	est une constante strictement positive.\\[.1cm]
	Or, si $\lambda$ est un réel {\bf strictement positif}, 
	maximiser la 
	fonction $x \mapsto \lambda \, f(x)$ est équivalent à 
	maximiser $x \mapsto f(x)$.\\
	On en déduit que maximiser $\E\left( \Sum{t=T-2}{\tau 
	-1} u(X(t),X(t+1))\right)$ est équivalent à maximiser :
	\[
	  u(X(T-2),X(T-1))  + (1-H(T-1)) \, u(X(T-1),X(T))
	\]

	\item[$\sbullet$] On rappelle maintenant que, dans cette partie, 
	on 
	suppose connus $X(1)$, $\ldots$, $X(T-2)$ et que l'on 
	cherche à déterminer $X(T-1)$ et $X(T)$ maximisant 
	$\E\left(\Sum{t=T-2}{\tau -1} u(X(t),X(t+1))\right)$.\\
	Dans cette question, on cherche plus particulièrement à 
	déterminer $X(T-1)$.\\
	Or, d'après la question \itbf{12.d)(i)}, si $X(T-1)$ est connu,
	alors le meilleur choix pour $X(T)$ est $1$, {\bf peu 
	importe la valeur de $X(T-1)$}.\\
	Pour déterminer le meilleur choix pour $X(T-1)$, on fixe donc 
	dès à présent $X(T)=1$.\\
	On en déduit que l'on souhaite maximiser :
	\[
	  u(X(T-2),X(T-1)) + (1-H(T-1)) \, u(X(T-1),1)
	\]
	Or, d'après la question \itbf{12.d)(ii)} : $u(X(T-1),1) = 
	-\ln(X(T-1))$. On souhaite donc maximiser :
	\[
	  u(X(T-2),X(T-1)) - (1- H(T-1)) \, \ln(X(T-1))
	\]
	
	\item[$\sbullet$] Ainsi, choisir $X(T-1)$ pour maximiser 
	$\E\left( 
	\Sum{t=T-2}{\tau -1} u(X(t),X(t+1))\right)$ revient à maximiser 
	la fonction :
	\[
	  \phi : y \mapsto u(X(T-2),y) - (1-H(T-1)) \, \ln(y)
	\]
	\conc{Finalement, le meilleur choix pour $X(T-1)$ est le 
	réel maximisant\\
	la fonction $\phi : y \mapsto u(X(T-2),y) - 
	(1-H(T-1)) \, \ln(y)$.}
      \end{noliste}
      
      
      %\newpage
      
      
      ~\\[-1.4cm]
    \end{proof}

  \begin{noliste}{a)}
    \setcounter{enumi}{6}
    \setlength{\itemsep}{2mm}
    \item Calculer $\phi'(y)$.
    
    \begin{proof}~
      \begin{noliste}{$\sbullet$}
	\item Soit $y \in \ ]0,1]$.
	\[
	  \begin{array}{rcl}
	    \phi(y) &=& u(X(T-2),y) -(1- H(T-1)) \, \ln(y)
	    \\[.2cm]
	    &=& \ln(\alpha) - \ln(X(T-2)) + \ln(y) - 
	    \ln(1+(\alpha -1) \, y) - (1- H(T-1)) \, \ln(y)
	    \\[.2cm]
	    &=& \ln(\alpha) - \ln(X(T-2)) 
	    + H(T-1) \, \ln(y) - \ln(1+(\alpha-1) \,y)
	  \end{array}
	\]
	On en déduit que la fonction $\phi$ est dérivable sur $]0,1]$
	en tant que somme de fonctions dérivables sur $]0,1]$.
	
	\item Soit $y \in \ ]0,1]$.
	\[
	  \begin{array}{rcl}
	    \phi'(y) &=& \dfrac{H(T-1)}{y} - \dfrac{\alpha -1}
	    {1+(\alpha -1) \, y}
	    \ = \ \dfrac{H(T-1) \, \big(1+ (\alpha-1) \, y \big) - 
	    (\alpha -1) \, y}{y \, \big(1+ (\alpha -1) \, y\big)}
	    \\[.6cm]
	    &=& \dfrac{H(T-1) + (\alpha-1) (H(T-1)-1) \, y}
	    {y \, \big(1+ (\alpha -1) \, y\big)}
	  \end{array}
	\]
      \end{noliste}
      \conc{$\forall y \in \ ]0,1]$, $\phi'(y) =
      \dfrac{H(T-1) + (\alpha-1) (H(T-1)-1) \, y}
	    {y \, \big(1+ (\alpha -1) \, y\big)}$}~\\[-1cm]
    \end{proof}
    
    \item Construire le tableau de variation de $\phi$ dans le cas 
    $\dfrac{H(T-1)}{(\alpha-1)(1-H(T-1))} \leq 1$.
    
    \begin{proof}~
      \begin{noliste}{$\sbullet$}
	\item Avec le même raisonnement qu'à la question \itbf{11.a)} :
	$y \, \big(1+ (\alpha -1) \, y\big) >0$. Donc :
	\[
	  \begin{array}{rcl}
	    \phi'(y) \geq 0 & \Leftrightarrow & 
	    H(T-1) + (\alpha -1) (H(T-1) -1) \, y \geq 0
	    \\[.2cm]
	    & \Leftrightarrow & 
	    (\alpha-1) (H(T-1)-1) \, y \geq - H(T-1)
	    \\[.2cm]
	    & \Leftrightarrow & 
	    y \leq - \dfrac{H(T-1)}{(\alpha -1) (H(T-1) -1)}
	  \end{array}
	\]
	En effet :
	\begin{noliste}{$\stimes$}
	  \item $\alpha-1 >0$, car $\alpha >1$.
	  \item $1-H(T-1) \geq 0$, car $H(T-1)$ est une probabilité, 
	  donc en particulier $H(T-1) \leq 1$.~\\[-.2cm]
	\end{noliste}
	
	
	%\newpage
	
	
	De plus : $- \dfrac{H(T-1)}{(\alpha -1) (H(T-1) -1)}
	\ = \ \dfrac{H(T-1)}{(\alpha -1) (1- H(T-1))}$. Donc :
	\[
	  \phi'(y) \geq 0 \ \Leftrightarrow \ 
	  y \leq \dfrac{H(T-1)}{(\alpha -1) (1- H(T-1))}
	\]
	
	\item Comme $\beta = 
	\dfrac{H(T-1)}{(\alpha -1) (1- H(T-1))} \leq 1$, alors on 
	obtient le tableau de variations suivant :
	\begin{center}
	  \begin{tikzpicture}[scale=.8, transform shape]
	    \tkzTabInit[lgt=4,espcl=3]
	    {$y$ /1, Signe de $\phi'(y)$ /1, Variations de $\phi$ 
	    /2} 
	    {$0$, $\beta$, $1$}%
	    \tkzTabLine{ , + ,z, - , } 
	    \tkzTabVar{-/$-\infty$, +/$\phi( \beta)$, 
	    -/$\phi(1)$}
	  \end{tikzpicture}
	\end{center}
	
	\item Détaillons les éléments de ce tableau.
	\begin{noliste}{$\stimes$}
	  \item Tout d'abord : $\dlim{y \to 0} \ln\big(1+ (\alpha-1)
	  \,y\big) \ = \ \ln(1) \ = \ 0$.\\[.1cm]
	  De plus : $H(T-1) \geq 0$ et $\dlim{y \to 0} 
	  \ln(y)=-\infty$.\\
	  On en déduit : $\dlim{y \to 0} \phi(y) = -\infty$.
	  
	  \item Ensuite :
	  \[
	    \begin{array}{rcl}
	      \phi(1) &=& \ln(\alpha) - \ln(X(T-2)) + 
	      \bcancel{H(T-1) \, \ln(1)} - \ln(1+ (\alpha-1) \times 1)
	      \\[.2cm]
	      &=& \bcancel{\ln(\alpha)} - \ln(X(T-2)) - 
	      \bcancel{\ln(\alpha)}
	      \\[.2cm]
	      &=& - \ln(X(T-2))
	    \end{array}
	  \]~\\[-1.8cm]
	\end{noliste}
      \end{noliste}
    \end{proof}
    
    \item Construire le tableau de variation de $\phi$ dans le cas 
    $\dfrac{H(T-1)}{(\alpha-1)(1-H(T-1))} \geq 1$.
    
    \begin{proof}~\\
      Soit $y \in \ ]0,1]$.\\
      On a toujours : $\phi'(y) \geq 0 \ \Leftrightarrow \
      y \leq \dfrac{H(T-1)}{(\alpha -1) (1- H(T-1))} = \beta$.\\[.1cm]
      Comme $\beta \geq 1$, avec les mêmes calculs qu'en 
      question \itbf{12.h)}, on obtient le tableau de variations 
      suivant :
      \begin{center}
	\begin{tikzpicture}[scale=.8, transform shape]
	  \tkzTabInit[lgt=4,espcl=3]
	  {$y$ /1, Signe de $\phi'(y)$ /1, Variations de $\phi$/2} 
	  {$0$, $1$}%
	  \tkzTabLine{ , + , } 
	  \tkzTabVar{-/$-\infty$, +/$\phi(1)$}
	\end{tikzpicture}
      \end{center}~\\[-1.4cm]
    \end{proof}

    
    \item Donner la valeur de $y^*(X(T-2),T-2)$.
    
    \begin{proof}~\\
      Deux cas se présentent :
      \begin{noliste}{$\stimes$}
	\item \dashuline{si $\beta \leq 1$}, alors d'après la 
	question \itbf{12.h)}, la fonction $\phi$ atteint son 
	maximum en $\beta$.
	\conc{Si $\dfrac{H(T-1)}{(\alpha -1) (1- H(T-1))} \leq 1$, 
	alors $y^*(X(T-2),T-2)= 
	\dfrac{H(T-1)}{(\alpha -1) (1- H(T-1))}$.}
	
	\item \dashuline{si $\beta \geq 1$}, alors d'après la 
	question \itbf{12.i)}, la fonction $\phi$ atteint son 
	maximum en $1$.
	\conc{Si $\dfrac{H(T-1)}{(\alpha -1) (1- H(T-1))} \geq 1$, 
	alors $y^*(X(T-2),T-2)=1$.}~\\[-1.2cm]
      \end{noliste}
    \end{proof}
  \end{noliste}





\end{document}
