\documentclass[11pt]{article}%
\usepackage{geometry}%
\geometry{a4paper,
  lmargin=2cm,rmargin=2cm,tmargin=2.5cm,bmargin=2.5cm}

\input{../macros_Livre.tex}

% \renewcommand{\thesection}{\Roman{section}.\hspace{-.3cm}}
% \renewcommand{\thesubsection}{\Alph{subsection}.\hspace{-.2cm}}

\pagestyle{fancy} %
\lhead{ECE2 \hfill Mathématiques \\} %
\chead{\hrule} %
\rhead{} %
\lfoot{} %
\cfoot{} %
\rfoot{\thepage} %

% \widowpenalty=10000
% \clubpenalty=10000

\renewcommand{\headrulewidth}{0pt}% : Trace un trait de séparation
                                    % de largeur 0,4 point. Mettre 0pt
                                    % pour supprimer le trait.

\renewcommand{\footrulewidth}{0.4pt}% : Trace un trait de séparation
                                    % de largeur 0,4 point. Mettre 0pt
                                    % pour supprimer le trait.

\setlength{\headheight}{14pt}

\title{\bf \vspace{-1.6cm} ESSEC II 2018} %
\author{} %
\date{} %
\begin{document}

\maketitle %
\vspace{-1.2cm}\hrule %
\thispagestyle{fancy}

\vspace*{.4cm}

%%DEBUT

\noindent
On s'intéresse à l'évolution d'une population de petits organismes
(typiquement des insectes) pendant une \og saison \fg{} reproductrice
de durée maximale $T$ où $T \in \N^*$. Les insectes sont supposés
vivre une unité de temps, au bout de laquelle ils meurent en pondant
un certain nombre d'{\oe}ufs. Au moment du dépôt d'un {\oe}uf, un
processus chimique, la diapause, est susceptible de se mettre en
marche qui entraîne l'arrêt de maturation de l'{\oe}uf jusqu'à la
saison suivante. Ainsi, à chaque $t$ de la saison, une génération
d'insectes s'éteint, en déposant des {\oe}ufs. Immédiatement, une
proportion $p(t)$ de ces {\oe}ufs se mettent en diapause. Les {\oe}ufs
qui ne sont pas entrés en diapause éclosent avant la date $t+1$,
donnant naissance à une nouvelle génération d'insectes, qui s'éteindra
à la date $t+1$ en déposant des {\oe}ufs, etc. Comme, à la fin de la
saison, tous les organismes vivants de la population meurent, hormis
les {\oe}ufs qui sont en diapause, ce sont ces derniers qui seront à
l'origine d'une nouvelle population qui éclora à la saison
suivante. Il est donc fondamental pour la survie de la lignée que les
organismes adoptent une stratégie maximisant le nombre d'{\oe}ufs en
diapause accumulés jusqu'à
la date où la saison s'achève.\\[.1cm]
Au cours du problème, on s'intéressera plus particulièrement au cas où
la durée de la saison est une variable aléatoire $\tau$ pouvant
prendre des valeurs entières entre $1$ et $T$. Pour $t\in \{1,2,
\ldots, T\}$, l'événement $\Ev{\tau=t}$ signifiera donc que la saison
s'arrête à la date $t$.\\[.1cm]
Toutes les variables aléatoires intervenant dans le problème sont
définies sur un espace probabilisé $(\Omega, \A, \Prob)$. Pour toute
variable aléatoire $Y$, on notera $\E(Y)$ son espérance lorsqu'elle
existe.



\section*{Partie I - Modèle de population saisonnière}

\noindent
\begin{minipage}{20cm}
  Dans cette question, on définit l'évolution formelle du nombre 
  d'{\oe}ufs en diapause entre les dates $0$ et $T$.
\end{minipage}

\noindent
On note $D(t)=$ nombre d'{\oe}ufs en diapause à la date $t$. Les 
{\oe}ufs pondus à la date $t$ qui entrent en diapause sont 
comptabilisés à la date $t+1$.
\[
  \begin{array}{rcR{12cm}}
    N(t) &=& nombre moyen d'{\oe}ufs produits à la date $t$
    \nl
    p(t) &=& proportion des {\oe}ufs produits à la date $t$ qui entrent 
    en diapause
  \end{array}
\]
Par convention, la date $0$ d'une saison est celle où les insectes nés 
des {\oe}ufs en diapause de la saison précédente pondent $N(0)$ 
{\oe}ufs. On suppose pour simplifier :
\begin{noliste}{$-$}
  \item que $N(0)$ est un entier naturel non nul.
  \item que tous les {\oe}ufs issus de la saison précédente ont éclos 
  et donc que $D(0)=0$.
  \item que pour tout $t\in \{0,1, \ldots, T-1\}$, $0<p(t)\leq 1$
\end{noliste}
Enfin, on suppose qu'à chaque date $t$ de la saison, un individu 
produit en moyenne $\alpha$ {\oe}ufs ($\alpha$ étant un réel 
strictement positif). {\bf Par simplicité, on supposera que $\alpha$ 
reste constant pendant toute la saison.}

\begin{noliste}{1.}
  \setlength{\itemsep}{4mm}
  \item 
  \begin{noliste}{a)}
    \setlength{\itemsep}{2mm}
    \item Montrer que $D(t+1)=D(t) + p(t) \, N(t)$ pour tout entier $t$ 
    tel que $0 \leq t \leq T-1$.
    
    
    
    
    %\newpage

    
    \item Montrer que $N(t+1)=\alpha (1-p(t)) \, N(t)$ pour tout entier 
    $t$ tel que $0 \leq t \leq T-1$.
    
    
  \end{noliste}
  
  \item On suppose dans cette question que $\alpha \leq 1$.
  \begin{noliste}{a)}
    \setlength{\itemsep}{2mm}
    \item Montrer que pour tout entier $t$ tel que $0 \leq t \leq T-1$,
    $N(t+1) \leq N(t)$.
    
    

    
    \item Montrer que pour tout entier $t$ tel que $0 \leq t \leq T-1$ :
    \[
      D(t+1) + N(t+1) \ \leq \ D(t)+N(t)
    \]
    
    
    
    
    %\newpage
   
    
    \item Montrer que pour tout entier $t$ tel que $0 \leq t \leq T$ :
    \[
      D(t) + N(t) \leq N(0)
    \]
    
    

    
    \item Montrer que pour tout entier $t$ tel que $0 \leq t \leq T$, 
    $D(t) \leq N(0)$.
    
    

    
    \item On suppose que $p(0)=1$.
    \begin{nonoliste}{(i)}
      \item Montrer que pour tout entier $t$ tel que $1 \leq t \leq T$, 
      $N(t)=0$.
      
      

      
      \item Montrer que pour tout entier $t$ tel que $1 \leq t \leq T$,
      $D(t)=N(0)$.
      
      
      
      
      %\newpage

      
      \item En déduire que si $\alpha \leq 1$, la meilleure stratégie 
      adaptée à la saison est que les $N(0)$ {\oe}ufs produits à la date
      $0$ entrent en diapause immédiatement.
      
      

    \end{nonoliste}
  \end{noliste}
  
  \item {\bf On suppose désormais $\alpha >1$ jusqu'à la fin du 
  problème.}\\
  On introduit maintenant $\tau$ une variable aléatoire à valeurs dans 
  $\{1,2, \ldots, T\}$ qui représente la date où s'achève la saison. On 
  suppose que pour tout $t \in \{1,2, \ldots, T\}$, $\Prob(
  \Ev{\tau=t}) >0$.
  \begin{noliste}{a)}
    \setlength{\itemsep}{2mm}
    \item Montrer que pour tout $t\in \{1,2, \ldots, T\}$, $\Prob(
    \Ev{\tau \geq t}) >0$. On définit alors $H(t)= \Prob_{\Ev{\tau \geq 
    t}}(\Ev{\tau =t})$.
    
    

    
    \item Montrer que : $H(t)= \dfrac{\Prob(\Ev{\tau=t})}{\Prob(\Ev{\tau
    \geq t})}$.
    
    

    
    \item Montrer que : $H(T)=1$.
    
    
    
    
    %\newpage

    
    \item Calculer $H(t)$ pour $t\in \{1,2, \ldots, T\}$ si $\tau$ 
    suit une loi uniforme sur $\{1,2, \ldots, T\}$.
    
    

    
    \item 
    \begin{nonoliste}{(i)}
      \item Soient $T$ réels $\lambda_1$, $\lambda_2$, $\ldots$, 
      $\lambda_T$ tels que $0 < \lambda_1 < \lambda_2 < \cdots < 
      \lambda_T=1$. Par convention, on pose $\lambda_0=0$. Soient 
      $q_1=\lambda_1$, $q_2=\lambda_2-\lambda_1$, $\ldots$, 
      $q_T=\lambda_T - \lambda_{T-1}$.\\
      Montrer que $(q_i)_{1\leq i \leq T}$ définit une loi de 
      probabilité sur $\{1,2, \ldots, T\}$.
      
      

      
      \item Calculer $H(t)$ si $\tau$ suit la loi précédente.
      
      
      
      
     %\newpage

      
      \item On suppose que $T\geq 2$ et de plus que pour tout entier 
      $n$ tel que $1 \leq n \leq T-1$, on a $\lambda_{n+1} - 
      \lambda_n \geq \lambda_n - \lambda_{n-1}$. Montrer que 
      $t \mapsto H(t)$ est croissante sur $\{1,2, \ldots, T\}$.
      
      
    \end{nonoliste}
  \end{noliste}
  
  \noindent
  {\bf On suppose désormais que $t \mapsto H(t)$ est croissante.} Le 
  but est maintenant de trouver une stratégie adéquate pour maximiser 
  la quantité $\E\big(\ln(D(\tau))\big)$. On va commencer par regarder 
  un exemple simple.
  
  \item On suppose ici que $T=2$, que $H$ est donnée par $H(1) =
  \dfrac{1}{2}$ et $H(2)=1$ et que $\alpha=4$.
  \begin{noliste}{a)}
    \setlength{\itemsep}{2mm}
    \item 
    \begin{nonoliste}{(i)}
      \item Déterminer $\Prob(\Ev{\tau=1})$.
      
      
      
      
      %\newpage

      
      \item Quelle est la loi de $\tau$ ?
      
      
    \end{nonoliste}
    
    \item Montrer que pour $D(1)$ et $N(1)$ donnés, $D(2)$ est maximum
    pour $p(1)=1$.
    
    

    
    \item On suppose $p(1)=1$. Montrer que :
    \[
      \E\big(\ln(D(\tau)\big) \ = \ \dfrac{1}{2} \, \ln\big((4-3 \, 
      p(0)) \, N(0)\big) + \dfrac{1}{2} \, \ln\big(p(0) \, N(0)\big)
    \]
    
    
    
    
    %\newpage

    
    \item Construire le tableau de variations sur $]0,1]$ de la 
    fonction $\varphi$ définie par :
    \[
      \varphi(x) \ = \ \dfrac{1}{2} \, \ln\big((4-3x) \, N(0)\big) 
      + \dfrac{1}{2} \, \ln\big(N(0) \, x\big)
    \]
    
    
    
    
    %\newpage

    
    \item Déterminer $p^*(0)$ qui maximise $\E\big(\ln(D(\tau))\big)$.
    
    
  \end{noliste}
\end{noliste}




\section*{Partie II - Transformation du problème}

\noindent
{\bf Par convention, on conviendra que si $h$ est une fonction 
numérique définie sur $\{0,1,2, \ldots, T\}$}, on a
\[
  \Sum{t=1}{0} h(t)=0
\]

\begin{noliste}{1.}
  \setlength{\itemsep}{4mm}
  \setcounter{enumi}{4}
  \item Montrer que pour tout $t \in \{0,1,2, \ldots, T\}$, $D(t) 
  +N(t) >0$.
  
  
  
  
  %\newpage

  
  On pose
  \[
    X(t) \ = \ \dfrac{D(t)}{D(t) + N(t)}
  \]  
  
  \item Montrer que pour tout $t\in \{0,1,2, \ldots, T-1\}$ :
  \[
    X(t+1) \ = \ \dfrac{p(t) + (1-p(t)) \, X(t)}{p(t) + \alpha \, 
    (1-p(t)) + (1-\alpha)(1-p(t)) \, X(t)}
  \]
  
  

  
  \item Soit $\xi \in [0,1]$ fixé. Pour $x \in [0,1]$, on pose :
  \[
    \psi_\xi(x) \ = \ \dfrac{x+ (1-x) \xi}{x + \alpha \, (1-x) + 
    (1- \alpha)(1-x)\xi}
  \]
  \begin{noliste}{a)}
    \setlength{\itemsep}{2mm}
    \item Montrer que $\psi_\xi$ est croissante sur $[0,1]$.
    
    

    
    \item Calculer $\psi_\xi(1)$.
    
    

    
    \item 
    \begin{nonoliste}{(i)}
      \item Calculer $\psi_\xi(0)$. On pose désormais
      $
        A(\xi) \ = \ \psi_\xi(0)
      $
      
      
      
      
      %\newpage

      
      \item Montrer que pour tout $t\in \{0,1,2, \ldots, T-1\}$, 
      $A(X(t)) \leq X(t+1) \leq 1$.
      
      

      
      \item Montrer que $\xi \mapsto A(\xi)$ est croissante sur $[0,1]$.
      
      
    \end{nonoliste}
  \end{noliste}
    
  \item Justifier l'égalité de variables aléatoires :
  \[
    D(\tau) \ = \ \dfrac{D(\tau)}{D(\tau-1)} \cdot \dfrac{D(\tau-1)}
    {D(\tau-2)} \cdot \cdots \cdot \dfrac{D(2)}{D(1)} \cdot 
    \dfrac{D(1)}{N(0)} \cdot N(0)
  \]
  
  
  
  
  
  \noindent
  On pose $\hat{R}(0) = \ln \left( \dfrac{D(1)}{N(0)}\right)$ et 
  $\hat{R}(t) = \ln \left( \dfrac{D(t+1)}{D(t)}\right)$ pour $1 \leq t 
  \leq T-1$.
  
  \item 
  \begin{noliste}{a)}
    \setlength{\itemsep}{2mm}
    \item Montrer que :
    \[
      \E\big(\ln(D(\tau))\big) \ = \ \ln(N(0)) + \E\left( \hat{R}(0)
      + \Sum{t=1}{\tau -1} \hat{R}(t)\right)
    \]
    
    
    
    \item Montrer que : $\dfrac{D(1)}{N(0)} = \dfrac{\alpha \, X(1)}
    {1+(\alpha-1) \, X(1)}$.
    
    
        
    \item Montrer que : $\dfrac{D(t+1)}{D(t)} = \dfrac{\alpha \, X(t+1)}
    {X(t) \big(1+ (\alpha -1) \, X(t+1)\big)}$ pour $1 \leq t \leq 
    T-1$.
    
    

    
    %\newpage
    
    
    Pour $x$ et $y$ deux réels strictement positifs, on pose $u(x,y)=
    \ln(\alpha) - \ln(x) + \ln(y) - \ln(1+(\alpha-1)y)$.
    
    \item Montrer que : $\hat{R}(0) = u(1,X(1))$.
    
    

    
    \item Montrer que : $\hat{R}(t) = u(X(t),X(t+1))$ pour 
    $1 \leq t \leq T-1$.
    
    
        
    \item Conclure que :
    \[
      \E\big(\ln(D(\tau))\big) \ = \ \ln(N(0)) + \E\left( u(1,X(1)) + 
      \Sum{t=1}{\tau -1} u(X(t),X(t+1))\right)
    \]
    
    
  \end{noliste}
\end{noliste}

\noindent
On voit donc que maximiser $\E\big(\ln(D(\tau))\big)$ revient à choisir,
à chaque date $t$ telle que $1 \leq t \leq \tau -1$, la valeur $X(t+1)$ 
vérifiant la contrainte $A(X(t)) \leq X(t+1) \leq 1$ de façon à rendre 
maximale l'expression
\[
  \E\left(u(1,X(1)) + \Sum{t=1}{\tau-1} u(X(t), X(t+1))\right)
\]




%\newpage




\section*{Partie III - Programmation dynamique}

\noindent
On expose dans cette partie les deux premières étapes de la méthode de 
la programmation dynamique pour résoudre le problème.

\begin{noliste}{1.}
  \setlength{\itemsep}{4mm}
  \setcounter{enumi}{9}
  \item Soit $B$ un événement. On note $\unq{}_B$ la variable aléatoire
  telle que 
  \[
    \unq{}_B(\omega) \ = \ \left\{
    \begin{array}{cR{3cm}}
      1 & si $\omega \in B$
      \nl
      0 & sinon
    \end{array}
    \right.
  \]
  \begin{noliste}{a)}
    \setlength{\itemsep}{2mm}
    \item Déterminer la loi de $\unq{}_B$.
    
    

    
    \item Soient $B$ et $C$ deux événements. Montrer l'égalité de 
    variables aléatoires : $\unq{}_{B \cap C} = \unq{}_B \times 
    \unq{}_C$.
    
    
    
    \item On suppose que $0 < \Prob(B) < 1$. Si $Y$ est une variable
    aléatoire prenant un nombre fini de valeurs, on définit la {\bf 
    variable aléatoire} notée $\E_B(Y)$ par :
    \[
      \E_B(Y) \ = \ \dfrac{1}{\Prob(B)} \, \E(Y \, \unq{}_B) \, 
      \unq{}_B + \dfrac{1}{\Prob(\bar{B})} \, \E(Y \, \unq{}_{\bar{B}})
      \, \unq{}_{\bar{B}}
    \]
    où $\bar{B}$ désigne l'événement contraire de $B$.
    
    
    
    % Attention : la définition de $\E_B(Y)$ donnée ici est en 
    % fait l'espérance conditionnelle de $Y$ sachant la \var 
    % $\unq{}_B$, et non l'espérance conditionnelle sachant 
    % l'événement $B$
    
    
    %\newpage

    
    \begin{nonoliste}{(i)}
      \item Soient $Y$ et $Z$ deux variables aléatoires prenant un 
      nombre fini de valeurs. Montrer que :
      \[
        \E_B(Y+Z) \ = \ \E_B(Y) + \E_B(Z)
      \]
      
      

      
      \item Montrer que : $\E(\E_B(Y)) = \E(Y)$.
      
      

      
      \item Montrer que : $\E_B(Y \, \unq{}_B) = \E_B(Y) \, \unq{}_B$.
      
      
    \end{nonoliste}
  \end{noliste}
    
  \item On suppose dans cette question que, quand l'événement $\Ev{\tau 
  =T}$ est réalisé, $X(1)$, $\ldots$, $X(T-1)$ sont connus. Comme on 
  l'a vu précédemment, si on pose $x=X(T-1)$, le meilleur choix à faire
  est alors de prendre pour $X(T)$ la valeur $y^*(x,T-1) \in [A(x),1]$
  qui maximise $u(x,y)$.
  \begin{noliste}{a)}
    \setlength{\itemsep}{2mm}
    \item Montrer que : $y^*(x,T-1)=1$.
    
    
    
    \item Montrer que : $u(x,1)=-\ln(x)$.
    
    
  \end{noliste}
  
  \item On suppose maintenant que, quand l'événement $\Ev{\tau \geq 
  T-1}$ est réalisé, les réels $X(1)$, $\ldots$, $X(T-2)$ sont connus. 
  \\[.1cm]
  La 
  stratégie reste donc de choisir $X(T-1)$ et $X(T)$ de façon à 
  maximiser $\E\left(\Sum{t=T-2}{\tau-1} u(X(t), 
  X(t+1))\right)$.\\[.1cm]
  La variable aléatoire $\tau$ prend les deux valeurs $T-1$ et $T$ 
  avec les probabilités respectives 
  \[
    \Prob_{\Ev{\tau \geq T-1}}(\Ev{\tau = T-1}) \quad \text{et} 
    \quad \Prob_{\Ev{\tau \geq T-1}}(\Ev{\tau = T})
  \]
  \end{noliste}
  
  
  %\newpage
  
  
  \begin{noliste}{a)}
    \setlength{\itemsep}{2mm}
    \item Montrer que :
    \[
      \unq{}_{\Ev{\tau \geq T-1}} \ \Sum{t=T-2}{\tau-1} 
      u(X(t),X(t+1))
      \ = \ \unq{}_{\Ev{\tau \geq T-1}} \, u(X(T-2),X(T-1)) + 
      \unq{}_{\Ev{\tau = T}} \, u(X(T-1),X(T))
    \]
    
    
    
    \item Montrer que :
    \[
     \begin{array}{cl}
      & \unq{}_{\Ev{\tau \geq T-1}} \ \E_{\Ev{\tau \geq T-1}} 
      \left( \Sum{t=T-2}{\tau -1} u(X(t),X(t+1)\right) 
      \\[.8cm]
      = &
      \E_{\Ev{\tau \geq T-1}} \Big( u(X(T-2),X(T-1)) \cdot
      \unq{}_{\Ev{\tau \geq T-1}} + u(X(T-1),X(T)) \cdot 
      \unq{}_{\Ev{\tau = T}}\Big)
     \end{array}
    \]
    
    
    
    
    %\newpage
    
    
    \item Montrer que :
    \[
     \begin{array}{cl}
      & \E_{\Ev{\tau \geq T-1}}\left( \unq{}_{\Ev{\tau \geq T-1}} \, 
      u(X(T-2),X(T-1)) + \unq{}_{\Ev{\tau = T}} \, u(X(T-1), 
      X(T))\right)
      \\[.4cm]
      = &
      u(X(T-2), X(T-1)) \, \unq{}_{\Ev{\tau \geq T-1}} + 
      \dfrac{\Prob(\Ev{\tau =T})}{\Prob(\Ev{\tau \geq T-1})} \, 
      u(X(T-1), X(T)) \, \unq{}_{\Ev{\tau \geq T-1}}
     \end{array}
    \]
    
    
    
    \item On suppose que $X(T-1)$ est donné.
    \begin{nonoliste}{(i)}
      \item Montrer que le meilleur choix pour $X(T)$ est $1$.
      
      

      
      \item Montrer que pour un tel choix $u(X(T-1), X(T)) = 
      - \ln (X(T-1))$.
      
      
    \end{nonoliste}
    
    
    %\newpage
    
    
    \item Montrer que : $\Prob_{\Ev{\tau \geq T-1}}(\Ev{\tau = T})
    =1- H(T-1)$.
    
    
    
    On veut maintenant choisir la stratégie optimale à la date $T-2$.
    
    \item Montrer qu'on doit choisir pour $X(T-1)$ la valeur $y^*(
    X(T-2),T-2) \in [A(X(T-2)),1]$ de telle sorte que 
    \[
      \phi(y) \ = \ u(X(T-2),y) - (1-H(T-1)) \, \ln(y)
    \]
    soit maximal.
  \end{noliste}
    
    

  \begin{noliste}{a)}
    \setcounter{enumi}{6}
    \setlength{\itemsep}{2mm}
    \item Calculer $\phi'(y)$.
    
    
    
    \item Construire le tableau de variation de $\phi$ dans le cas 
    $\dfrac{H(T-1)}{(\alpha-1)(1-H(T-1))} \leq 1$.
    
    
    
    \item Construire le tableau de variation de $\phi$ dans le cas 
    $\dfrac{H(T-1)}{(\alpha-1)(1-H(T-1))} \geq 1$.
    
    

    
    \item Donner la valeur de $y^*(X(T-2),T-2)$.
    
    
  \end{noliste}





\end{document}
