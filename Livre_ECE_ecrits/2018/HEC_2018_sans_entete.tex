\chapter*{HEC 2018 : le sujet}
  
%

\section*{Exercice}

\noindent
Soit $n$ un entier supérieur ou égal à $2$ et $f$ un endomorphisme de
$\R^n$.
\begin{noliste}{$\sbullet$}
\item On note $\id_{\R^n}$ l'endomorphisme identité de $\R^n$ et
  $0_{\LL{\R^n}}$ l'endomorphisme nul de $\R^n$.

\item On pose $f^0=\id_{\R^n}$ et : $\forall j \in \N, \ f^{j+1}=f
  \circ f^j$.
  
\item On suppose que $f^n$ est l'endomorphisme nul de $\R^n$ : \
  \fbox{$f^n=0_{\LL{\R^n}}$ }
\end{noliste}

\begin{noliste}{1.}
  \setlength{\itemsep}{4mm}
  \item Soit $M$ la matrice définie par : 
  $M = \begin{smatrix}
    0 & 0 & 0 & 0\\
    0 & 0 & 1 & 0\\
    0 & 0 & 0 & 1\\
    0 & 0 & 0 & 0
  \end{smatrix}$.
  \begin{noliste}{a)}
    \setlength{\itemsep}{2mm}
    \item Déterminer le spectre de $M$. La matrice $M$ est-elle 
    diagonalisable ?
    
    

    
    \item Préciser le rang des matrices $M$ et $M^2$ respectivement.
    
    

    
    \item Quels sont les polynômes annulateurs de $M$ dont le degré 
    est égal à $3$ ? 
    
    
  \end{noliste}
  
  
  %\newpage
  
  
\item Pour tout $j \in \llb 0,n \rrb$, on note $F_j$ l'image de
  l'endomorphisme $f^j$ et $r_j$ son rang : 
  \[
  F_j = \im(f^j) \quad \text{ et } \quad r_j = \dim(F_j)
  \]
  Pour tout $j \in \llb 0, n-1 \rrb$, on note $g_j$ la restriction de
  $f$ à $F_j$, c'est à dire l'application linéaire de $F_j$ dans
  $\R^n$ définie par : $\forall x \in F_j, \ g_j(x)=f(x)$.
  \begin{noliste}{a)}
    \setlength{\itemsep}{2mm}
    \item Calculer $r_0$ et $r_n$.
    
    

    
    \item Soit $j \in \llb 0,n-1\rrb$.
    \begin{nonoliste}{(i)}
      \item Déterminer le rang de $g_j$.
    \end{nonoliste}

    
    
%       

    \begin{nonoliste}{(i)}
      \setcounter{enumiii}{1}
    \item Justifier l'égalité : $r_j-r_{j+1}=\dim(\kr(f) \cap F_j)$.
      
      
    \end{nonoliste}
    

    %\newpage


  \item Établir les inégalités : $n \geq r_0-r_1 \geq r_1-r_2 \geq
    \ldots \geq r_{n-1}-r_n \geq 0$.
    
    
  \end{noliste}
\end{noliste}

\noindent {\it On rappelle que le cardinal d'un ensemble fini $H$,
  noté $\Card(H)$, est le nombre de ses éléments.\\
  Pour $k \in \N^*$, on note $P(k)$ l'ensemble des $k$-uplets
  $(x_1,x_2,\ldots,x_k)$ d'entiers naturels tels que :
  \[
  \Sum{i=1}{k} i \ x_i = k
  \]
  c'est à dire : $P(k)=\{(x_1,x_2,\ldots,x_k) \in \N^k \ | \ x_1 +
  2x_2 + \ldots + k x_k = k\}$.\\
  On pose $p(k) = \Card\big(P(k) \big)$.}




%\newpage


\begin{noliste}{1.}
  \setlength{\itemsep}{4mm}
  \setcounter{enumi}{2}
\item Pour tout $i \in \llb 1,n \rrb$, on pose : $x_i= \Card\left(\{j
    \in \llb 0, n-1 \rrb \ | \ r_j-r_{j+1}=i\} \right)$ \quad $(*)$.
  \begin{noliste}{a)}
    \setlength{\itemsep}{2mm}
    \item Montrer que $(x_1,x_2, \ldots,x_n)$ est un élément de 
    $P(n)$.

    


    %\newpage


  \item Dans cette question, on suppose que $n$ est égal à $4$.
  \end{noliste}
    \begin{noliste}{(i)}
    \item Déterminer $(x_1,x_2,x_3,x_4)$ lorsque $f$ est
      l'endomorphisme de matrice $M$ dans la base canonique
      \nolinebreak de $\R^4$.

      
      
    \item Trouver l'ensemble $P(4)$ et vérifier que $p(4)=5$.
    \end{noliste}

    


    %\newpage


    \begin{noliste}{(i)}
      \setcounter{enumii}{2}
    \item Montrer que pour tout $(x_1,x_2,x_3,x_4) \in P(4)$, il
      existe un endomorphisme $f$ de $\R^4$ vérifiant \nolinebreak
      $(*)$.

      
    \end{noliste}
  \end{noliste}


  \newpage


  \begin{noliste}{1.}
    \setcounter{enumi}{3}
  \item Pour tout couple $(\ell,k) \in (\N^*)^2$, on pose : \
    $Q(\ell,k)=\{(x_1,x_2, \ldots, x_k)\in P(k) \ | \ x_1+x_2
    +\ldots +x_k \leq \ell\}$\\ et $q(\ell, k)=\Card(Q(\ell,k))$.
  \begin{noliste}{a)}
    \setlength{\itemsep}{2mm}
    \item Soit $k \in \N^*$.
    \begin{nonoliste}{(i)}
      \item Trouver l'ensemble $Q(1,k)$.

        
      
      \item Pour tout entier $\ell \geq k$, justifier l'égalité :
        $Q(\ell, k) = P(k)$.
        % \end{nonoliste}
        
        
        \end{nonoliste}

    \item Pour tout couple $(\ell,k)$ d'entiers tels que $k > \ell 
    \geq 2$, établir la relation :
    \[
    q(\ell,k-\ell) = \Card\big(\{(x_1,x_2,\ldots,x_k) \in P(k) \ | \
      x_1+x_2 +\ldots + x_k=\ell \}\big)
    \]

    


    %\newpage
    
    
  \item Soit $\ell$ un entier supérieur ou égal à $2$.
    \begin{nonoliste}{(i)}
    \item Pour tout entier $k > \ell$, montrer l'égalité : \
      $q(\ell,k) = q(\ell-1,k) + q(\ell, k-\ell)$.
      
      

    \item Que vaut $q(\ell,\ell) - q(\ell-1,\ell)$ ?

      
    \end{nonoliste}
  \end{noliste}
  
\item La fonction \Scilab{} suivante dont le script est incomplet
  (lignes \ligne{5} et \ligne{6}), calcule une matrice {\tt
    qmatrix(n)} telle que pour chaque couple $(\ell,k) \in\llb 1,n
  \rrb^2$, le coefficient situé à l'intersection de la ligne $\ell$ et
  de la colonne $k$ est égal à $q(\ell,k)$.
  \begin{scilab}
    & \tcFun{function} \tcVar{q} = qmatrix(\tcVar{n}) \nl %
    & \qquad \tcVar{q} = ones(\tcVar{n}, \tcVar{n}) \nl %
    & \qquad \tcFor{for} L = 2:\tcVar{n} \nl %
    & \qquad \qquad \tcFor{for} K = 2:\tcVar{n} \nl %
    & \qquad \qquad \qquad \tcIf{if} (K<L) \tcIf{then} \nl %
    & \qquad \qquad \qquad \qquad \tcVar{q}(L,K) = ................
    \nl %
    & \qquad \qquad \qquad \tcIf{elseif} (K==L) \tcIf{then} \nl %
    & \qquad \qquad \qquad \qquad \tcVar{q}(L,K) = ..............
    \nl %
    & \qquad \qquad \qquad \tcIf{else} \nl %
    & \qquad \qquad \qquad \qquad \tcVar{q}(L,K) = \tcVar{q}(L-1,K) +
    \tcVar{q}(L,K-L) \nl %
    & \qquad \qquad \qquad \tcIf{end} \nl %
    & \qquad \qquad \tcFor{end} \nl %
    & \qquad \tcFor{end} \nl %
    & \tcFun{endfunction}
  \end{scilab} 
  
  \noindent
  L'application de la fonction {\tt qmatrix} à l'entier $n=9$ fournit 
  la sortie suivante :
  \[
    \begin{console}
      \lDisp{----> qmatrix(9)} \nl %
      \lDisp{\qquad 1. \qquad 1. \qquad 1. \qquad 1. \qquad 1.  \qquad
        1. \qquad 1. \qquad 1. \qquad 1.} \nl %
      \lDisp{\qquad 1. \qquad 2. \qquad 2. \qquad 3. \qquad 3.  \qquad
        4. \qquad 4. \qquad 5. \qquad 5.} \nl %
      \lDisp{\qquad 1. \qquad 2. \qquad 3. \qquad 4. \qquad 5.  \qquad
        7. \qquad 8. \qquad\espn{} 10. \qquad\espn{} 12.} \nl %
      \lDisp{\qquad 1. \qquad 2. \qquad 3. \qquad 5. \qquad 6.  \qquad
        9. \qquad\espn{} 11. \qquad\espn{} 15. \qquad\espn{} 18.}
      \nl %
      \lDisp{\qquad 1. \qquad 2. \qquad 3. \qquad 5. \qquad
        7. \qquad\espn{} 10. \qquad\espn{} 13. \qquad\espn{}
        18. \qquad\espn{} 23.} \nl %
      \lDisp{\qquad 1. \qquad 2. \qquad 3. \qquad 5. \qquad 7.
        \qquad\espn{} 11. \qquad\espn{} 14. \qquad\espn{}
        20. \qquad\espn{} 26.} \nl %
      \lDisp{\qquad 1. \qquad 2. \qquad 3. \qquad 5. \qquad 7.
        \qquad\espn{} 11. \qquad\espn{} 15. \qquad\espn{}
        21. \qquad\espn{} 28.} \nl %
      \lDisp{\qquad 1. \qquad 2. \qquad 3. \qquad 5. \qquad 7.
        \qquad\espn{} 11. \qquad\espn{} 15. \qquad\espn{}
        22. \qquad\espn{} 29.} \nl %
      \lDisp{\qquad 1. \qquad 2. \qquad 3. \qquad 5. \qquad 7.
        \qquad\espn{} 11. \qquad\espn{} 15. \qquad\espn{}
        22. \qquad\espn{} 30.}
    \end{console}
    \]


  %\newpage


  \begin{noliste}{a)}
    \setlength{\itemsep}{2mm}
  \item Compléter les lignes \ligne{5} et \ligne{6} du script de la
    fonction {\tt qmatrix}.

    

  \item Donner un script \Scilab{} permettant de calculer $p(n)$ à
    partir d'une valeur de $n$ entrée au clavier.
    
    

  \item Conjecturer une formule générale pour $q(2,k)$ applicable à
    tout entier $k \geq 1$, puis la démontrer.

        
  \end{noliste}
\end{noliste}



\newpage


\section*{Problème}

\noindent
{\bfseries Dans tout le problème :}
\begin{noliste}{$\sbullet$}
  \item toutes les variables aléatoires introduites sont 
  supposées définies sur un même espace probabilisé $(\Omega, \A, 
  \Prob)$ ;
  
  \item on note $n$ un entier supérieur ou égal à $2$.
\end{noliste}
{\it L'objet du problème est l'étude de sommes de variables 
aléatoires suivant une loi de Bernoulli de même paramètre, mais qui ne 
sont pas nécessairement indépendantes.

\noindent
Les parties II et III sont indépendantes de la partie I.}


\subsection*{Partie I. Valeurs possibles du coefficient de 
corrélation linéaire dans divers schémas de Bernoulli}

\noindent
{\it Dans cette partie, on considère des variables aléatoires 
$X_1, X_2, \ldots, X_n$ suivant chacune la même loi de Bernoulli de 
paramètre $p$ avec $0 < p < 1$, c'est à dire : 
\[
  \forall k \in \llb 1, n \rrb, \ \Prob(\Ev{X_k=1}) = p \quad \text{et} 
  \quad \Prob(\Ev{X_k=0}) = 1-p
\]

\noindent
On suppose que pour tout couple $(k,\ell) \in \llb 1,n \rrb^2$ avec $k 
\neq \ell$, le coefficient de corrélation linéaire des variables $X_k$ 
et $X_\ell$ est le même ; on note $r$ ce coefficient. On a donc :
\[
  \forall (k,\ell) \in \llb 1,n \rrb^2, \ 
  \dfrac{\Cov(X_k,X_\ell)}{\sqrt{\V(X_k) 
  \V(X_\ell)}} = \left\{
  \begin{array}{cR{3cm}}
    1 & si $k=\ell$
    \nl
    \nl[-.4cm]
    r & si $k \neq \ell$
  \end{array}
  \right.
\]
}

\begin{noliste}{1.}
  \setlength{\itemsep}{4mm}
  \item
  \begin{noliste}{a)}
    \setlength{\itemsep}{2mm}
    \item Dans les cas $(i)$ et $(ii)$ suivants, calculer la valeur 
    de $r$ et exprimer la variance de la variable aléatoire $ 
    \Sum{k=1}{n} X_k$ en fonction de $n$ et $p$.
    \begin{nonoliste}{(i)}
      \item Les variables aléatoires $X_1, X_2, \ldots, X_n$ sont 
      mutuellement indépendantes.
      
      \item Les variables aléatoires $X_1, X_2, \ldots, X_n$ sont 
      toutes égales.
    \end{nonoliste}
    
    De plus, préciser la loi de $\Sum{k=1}{n} X_k$ dans chacun 
    des deux cas précédents.
    
    


    %\newpage

    
  \item Montrer que pour tout $k \in \llb 1,n \rrb$, la variance de la
    variable aléatoire $\Sum{i=1}{k} X_i$ est donnée par la formule :
    \[
    \V \Big(\Sum{i=1}{k} X_i\Big) \ = \ k \, p \, (1-p) \, (1+(k-1)r)
    \]
    
    
    
  \item En déduire que le coefficient $r$ est au moins égal à
    $-\dfrac{1}{n-1}$.

    
  \end{noliste}
  
  
  %\newpage
  
  
\item On suppose dans cette question que $n$ est au moins égal à $2$.
  \begin{noliste}{a)}
    \setlength{\itemsep}{2mm}
  \item Montrer que $r$ est égal à $-1$ si et seulement si on a :
    $\Prob (\Ev{X_1=1} \cap \Ev{X_2=1}) = p(2p-1)$.
    
    

  \item Que vaut alors $\Prob\big(\Ev{X_1=0} \cap \Ev{X_2=0} \big)$ ?

    
    
  \item En déduire que le coefficient $r$ ne peut-être égal à $-1$ que
    lorsque $p = \dfrac{1}{2}$ et $\Prob(\Ev{X_1 + X_2 = 1}) = 1$.

    
  \end{noliste}

\item On suppose dans cette question que $n$ est supérieur ou égal à
  $3$ et que $\Prob\left(\Ev{\Sum{k=1}{n} X_k = 1}\right)=1$.
  \begin{noliste}{a)}
    \setlength{\itemsep}{2mm}
  \item Exprimer les valeurs de $p$ et $r$ en fonction de $n$.

    

  \item Déterminer les $n$-uplets $(x_1, x_2, \ldots, x_n) \in
    \{0,1\}^n$ pour lesquels la probabilité $\Prob\left( \dcap{k=1}{n}
      \Ev{X_k = x_k}\right)$ est strictement positive et la calculer.

    
  \end{noliste}
\end{noliste}



\newpage


\subsection*{Partie II. Lois bêta-binomiales}

\begin{noliste}{1.}
  \setlength{\itemsep}{4mm} %
  \setcounter{enumi}{3}
\item Soit $(x,y) \in \R^2$.
  \begin{noliste}{a)}
    \setlength{\itemsep}{2mm}
  \item Justifier que l'intégrale $\dint{0}{\frac{1}{2}} t^{x-1} \
    (1-t)^{y-1} \dt$ est convergente si seulement si $x>0$.

    
    
  \item Pour tout réel $\eps$ tel que $0 < \eps < \dfrac{1}{2}$,
    établir à l'aide d'un changement de variable affine, l'égalité :
    \[
    \dint{\frac{1}{2}}{1-\eps} t^{x-1} \ (1-t)^{y-1} \dt \ = \
    \dint{\eps}{\frac{1}{2}} t^{y-1} \ (1-t)^{x-1} \dt
    \]

    
    

    %\newpage


  \item En déduire que l'intégrale $\dint{0}{1} t^{x-1} \ (1-t)^{y-1}
    \dt$ est convergente si et seulement si $x>0$ et $y>0$.

    
  \end{noliste}
\end{noliste}

\noindent {\it Dans toute la suite du problème, on pose : \ $\forall
  (x,y) \in (\R_+^*)^2 , \ B(x,y) = \dint{0}{1} t^{x-1} \, (1-t)^{y-1}
  \dt$.}


%\newpage


\begin{noliste}{1.}
  \setlength{\itemsep}{4mm} %
  \setcounter{enumi}{4}
\item Soit $x$ et $y$ des réels strictement positifs.
  \begin{noliste}{a)}
    \setlength{\itemsep}{2mm}
  \item À l'aide d'une intégration par parties, établir la relation :
    $B(x+1,y) = \dfrac{x}{y} \times B(x,y+1)$.
    
    


    %\newpage


  \item En déduire l'égalité : $B(x,y+1) = \dfrac{y}{x+y} \times
    B(x,y)$.

    

  \end{noliste}
  
\item Pour tout réel $z$, soit $\left( (z)^{[m]} \right)_{m \in \N}$
  la suite définie par :
  \[
  (z)^{[0]}=1 \quad \text{et} \quad \forall m \in \N, \ 
  (z)^{[m+1]}=(z+m)   \times (z)^{[m]}
  \]
  (par exemple, pour tout $m \in \N$, on a $(1)^{[m]}=m!$)\\[.1cm]
  Établir pour tout $(x,y) \in (\R_+^*)^2$ et pour tout 
  couple $(k, \ell)$ d'entiers tels que $0 \leq k \leq \ell$, la 
  relation :
  \[
  B(x+k, y+\ell-k) \ = \ \dfrac{(x)^{[k]} \times (y)^{[\ell-k]}}{(x +
    y)^{[\ell]}} \times B(x,y)
  \]
  
  
  
\item Soit $a$ et $b$ des réels strictement positifs.\\[.2cm]
  Pour $k \in \llb 0,n \rrb$, on pose : $p_k = \dbinom{n}{k} \
  \dfrac{(a)^{[k]} \times (b)^{[n-k]}}{(a+b)^{[n]}}$.


  %\newpage


  \begin{noliste}{a)}
    \setlength{\itemsep}{2mm}
  \item À l'aide de la relation obtenue dans la question \itbf{6},
    montrer que $\Sum{k=0}{n} p_k = 1$.
    
    
    
  \end{noliste}
  
  \noindent
  {\itshape On dit qu'une variable aléatoire $S$ suit une loi
    bêta-binomiale ${\bf B}(n;a,b)$ si $S(\Omega)=\llb 0,n\rrb$ et si
    :
    \[
    \forall k \in \llb 0,n \rrb, \ \Prob(\Ev{S=k}) = \dbinom{n}{k} \
    \dfrac{(a)^{[k]} \times (b)^{[n-k]}}{(a+b)^{[n]}}
    \]
  }


  %\newpage


  \begin{noliste}{a)}
    \setlength{\itemsep}{2mm}
    \setcounter{enumii}{1}
    \item Reconnaître la loi ${\bf B}(n;1,1)$. 

      

    \item Montrer que l'espérance d'une variable aléatoire $S$ qui
      suit la loi ${\bf B}(n;a,b)$ est égale à $\dfrac{na}{a+b}$.

      
  \end{noliste}
\end{noliste}


\newpage


\subsection*{Partie III. Un possible dans le cas où $n=2$}

\noindent
Soit $a$ et $b$ des réels strictement positifs et $X_1$ et $X_2$ deux 
variables aléatoires à valeurs dans $\{0,1\}$ telles que :
\[
\forall (x_1,x_2) \in \{0,1\}^2, \ \Prob\left(\Ev{X_1=x_1}\cap
  \Ev{X_2=x_2}\right) \ = \ \dfrac{B(a+x_1+x_2,b+2-x_1-x_2)}{B(a,b)}
\]

\begin{noliste}{1.}
  \setlength{\itemsep}{4mm} %
  \setcounter{enumi}{7}
\item
  \begin{noliste}{a)}
    \setlength{\itemsep}{2mm}
  \item Montrer que les deux variables $X_1$ et $X_2$ suivent la même
    loi de Bernoulli.

    


    %\newpage


  \item Montrer que la variable aléatoire $X_1 + X_2$ suit la loi
    bêta-binomiale ${\bf B}(2;a,b)$.

    


    %\newpage
    

  \item Établir la relation : $\Prob_{\Ev{X_1=1}}(\Ev{X_2=1}) =
    \dfrac{a+1}{a+b+1}$.

    
  \end{noliste}
  
\item La fonction \Scilab{} suivante dont le script est incomplet
  (lignes \ligne{7} et \ligne{11}), effectue une simulation des deux
  variables $X_1$ et $X_2$ qu'elle place dans un vecteur ligne à deux
  composantes.
\end{noliste}

\begin{scilab}
  & \tcFun{function} \tcVar{x} = randbetabin(\tcVar{a}, \tcVar{b})
  \nl %
  & \qquad \tcVar{x} = zeros(1,2) \nl %
  & \qquad u = (\tcVar{a} + \tcVar{b}) \Sfois{} rand() \nl %
  & \qquad v = (\tcVar{a} + \tcVar{b} + 1) \Sfois{} rand() \nl %
  & \qquad \tcIf{if} (u < \tcVar{a}) \tcIf{then} \nl %
  & \qquad \qquad \tcVar{x}(1,1) = 1 \nl %
  & \qquad \qquad \tcIf{if} .............. \tcIf{then} \nl %
  & \qquad \qquad \qquad \tcVar{x}(1,2) = 1 \nl %
  & \qquad \qquad \tcIf{end} \nl %
  & \qquad \tcIf{else} \nl %
  & \qquad \qquad \tcIf{if} ............. \tcIf{then} \nl %
  & \qquad \qquad \qquad \tcVar{x}(1,2) = 1 \nl %
  & \qquad \qquad \tcIf{end} \nl %
  & \qquad \tcIf{end} \nl %
  & \tcFun{endfunction}
\end{scilab}
  
\begin{liste}{a)}
  \setlength{\itemsep}{2mm}
\item Préciser la loi simulée par la variable {\tt u} de la ligne
  \ligne{3}.

  
    
  \item Compléter les lignes \ligne{7} et \ligne{11}.
    
    
\end{liste}


%\newpage

  
\begin{noliste}{1.}
  \setcounter{enumi}{9} %
  \setlength{\itemsep}{4mm}
\item
  \begin{noliste}{a)}
    \setlength{\itemsep}{2mm}
  \item Calculer le coefficient de corrélation linéaire de $X_1$ et
    $X_2$.
      
    


    %\newpage


  \item Soit $(p,r)$ un couple de réels vérifiant $0 < p < 1$ et $0 <
    r < 1$.\\
    Expliquer comment utiliser la fonction {\tt randbetabin} pour
    simuler deux variables aléatoires suivant une même loi de
    Bernoulli de paramètre $p$ et dont le coefficient de corrélation
    linéaire est égal à $r$.

    

  \end{noliste}
\end{noliste}

