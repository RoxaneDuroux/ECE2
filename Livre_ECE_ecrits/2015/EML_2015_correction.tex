\documentclass[11pt]{article}%
\usepackage{geometry}%
\geometry{a4paper,
  lmargin=2cm,rmargin=2cm,tmargin=2.5cm,bmargin=2.5cm}

\usepackage{array}
\usepackage{paralist}

\usepackage[svgnames, usenames, dvipsnames]{xcolor}
\xdefinecolor{RecColor}{named}{Aqua}
\xdefinecolor{IncColor}{named}{Aqua}
\xdefinecolor{ImpColor}{named}{PaleGreen}

% \usepackage{frcursive}

\usepackage{adjustbox}

%%%%%%%%%%%
\newcommand{\cRB}[1]{{\color{Red} \pmb{#1}}} %
\newcommand{\cR}[1]{{\color{Red} {#1}}} %
\newcommand{\cBB}[1]{{\color{Blue} \pmb{#1}}}
\newcommand{\cB}[1]{{\color{Blue} {#1}}}
\newcommand{\cGB}[1]{{\color{LimeGreen} \pmb{#1}}}
\newcommand{\cG}[1]{{\color{LimeGreen} {#1}}}

%%%%%%%%%%

\usepackage{diagbox} %
\usepackage{colortbl} %
\usepackage{multirow} %
\usepackage{pgf} %
\usepackage{environ} %
\usepackage{fancybox} %
\usepackage{textcomp} %
\usepackage{marvosym} %

%%%%%%%%%% pour qu'une cellcolor ne recouvre pas le trait du tableau
\usepackage{hhline}%

\usepackage{pgfplots}
\pgfplotsset{compat=1.10}
\usepgfplotslibrary{patchplots}
\usepgfplotslibrary{fillbetween}
\usepackage{tikz,tkz-tab}
\usepackage{ifthen}
\usepackage{calc}
\usetikzlibrary{calc,decorations.pathreplacing,arrows,positioning} 
\usetikzlibrary{fit,shapes,backgrounds}
\usepackage[nomessages]{fp}% http://ctan.org/pkg/fp

\usetikzlibrary{matrix,arrows,decorations.pathmorphing,
  decorations.pathreplacing} 

\newcommand{\myunit}{1 cm}
\tikzset{
    node style sp/.style={draw,circle,minimum size=\myunit},
    node style ge/.style={circle,minimum size=\myunit},
    arrow style mul/.style={draw,sloped,midway,fill=white},
    arrow style plus/.style={midway,sloped,fill=white},
}

%%%%%%%%%%%%%%
%%%%% écrire des inférieur égal ou supérieur égal avec typographie
%%%%% francaise
%%%%%%%%%%%%%

\renewcommand{\geq}{\geqslant}
\renewcommand{\leq}{\leqslant}
\renewcommand{\emptyset}{\varnothing}

\newcommand{\Leq}{\leqslant}
\newcommand{\Geq}{\geqslant}

%%%%%%%%%%%%%%
%%%%% Macro Celia
%%%%%%%%%%%%%

\newcommand{\ff}[2]{\left[#1, #2\right]} %
\newcommand{\fo}[2]{\left[#1, #2\right[} %
\newcommand{\of}[2]{\left]#1, #2\right]} %
\newcommand{\soo}[2]{\left]#1, #2\right[} %
\newcommand{\abs}[1]{\left|#1\right|} %
\newcommand{\Ent}[1]{\left\lfloor #1 \right\rfloor} %


%%%%%%%%%%%%%%
%%%%% tikz : comment dessiner un "oeil"
%%%%%%%%%%%%%

\newcommand{\eye}[4]% size, x, y, rotation
{ \draw[rotate around={#4:(#2,#3)}] (#2,#3) -- ++(-.5*55:#1) (#2,#3)
  -- ++(.5*55:#1); \draw (#2,#3) ++(#4+55:.75*#1) arc
  (#4+55:#4-55:.75*#1);
  % IRIS
  \draw[fill=gray] (#2,#3) ++(#4+55/3:.75*#1) arc
  (#4+180-55:#4+180+55:.28*#1);
  % PUPIL, a filled arc
  \draw[fill=black] (#2,#3) ++(#4+55/3:.75*#1) arc
  (#4+55/3:#4-55/3:.75*#1);%
}


%%%%%%%%%%
%% discontinuité fonction
\newcommand\pointg[2]{%
  \draw[color = red, very thick] (#1+0.15, #2-.04)--(#1, #2-.04)--(#1,
  #2+.04)--(#1+0.15, #2+.04);%
}%

\newcommand\pointd[2]{%
  \draw[color = red, very thick] (#1-0.15, #2+.04)--(#1, #2+.04)--(#1,
  #2-.04)--(#1-0.15, #2-.04);%
}%

%%%%%%%%%%
%%% 1 : position abscisse, 2 : position ordonnée, 3 : taille, 4 : couleur
%%%%%%%%%%
% \newcommand\pointG[4]{%
%   \draw[color = #4, very thick] (#1+#3, #2-(#3/3.75))--(#1,
%   #2-(#3/3.75))--(#1, #2+(#3/3.75))--(#1+#3, #2+(#3/3.75)) %
% }%

\newcommand\pointG[4]{%
  \draw[color = #4, very thick] ({#1+#3/3.75}, {#2-#3})--(#1,
  {#2-#3})--(#1, {#2+#3})--({#1+#3/3.75}, {#2+#3}) %
}%

\newcommand\pointD[4]{%
  \draw[color = #4, very thick] ({#1-#3/3.75}, {#2+#3})--(#1,
  {#2+#3})--(#1, {#2-#3})--({#1-#3/3.75}, {#2-#3}) %
}%

\newcommand\spointG[4]{%
  \draw[color = #4, very thick] ({#1+#3/1.75}, {#2-#3})--(#1,
  {#2-#3})--(#1, {#2+#3})--({#1+#3/1.75}, {#2+#3}) %
}%

\newcommand\spointD[4]{%
  \draw[color = #4, very thick] ({#1-#3/2}, {#2+#3})--(#1,
  {#2+#3})--(#1, {#2-#3})--({#1-#3/2}, {#2-#3}) %
}%

%%%%%%%%%%

\newcommand{\Pb}{\mathtt{P}}

%%%%%%%%%%%%%%%
%%% Pour citer un précédent item
%%%%%%%%%%%%%%%
\newcommand{\itbf}[1]{{\small \bf \textit{#1}}}


%%%%%%%%%%%%%%%
%%% Quelques couleurs
%%%%%%%%%%%%%%%

\xdefinecolor{cancelcolor}{named}{Red}
\xdefinecolor{intI}{named}{ProcessBlue}
\xdefinecolor{intJ}{named}{ForestGreen}

%%%%%%%%%%%%%%%
%%%%%%%%%%%%%%%
% barrer du texte
\usetikzlibrary{shapes.misc}

\makeatletter
% \definecolor{cancelcolor}{rgb}{0.127,0.372,0.987}
\newcommand{\tikz@bcancel}[1]{%
  \begin{tikzpicture}[baseline=(textbox.base), inner sep=0pt]
    \node[strike out, draw] (textbox) {#1}[thick, color=cancelcolor];
    \useasboundingbox (textbox);
  \end{tikzpicture}%
}
\newcommand{\bcancel}[1]{%
  \relax\ifmmode
    \mathchoice{\tikz@bcancel{$\displaystyle#1$}}
               {\tikz@bcancel{$\textstyle#1$}}
               {\tikz@bcancel{$\scriptstyle#1$}}
               {\tikz@bcancel{$\scriptscriptstyle#1$}}
  \else
    \tikz@bcancel{\strut#1}%
  \fi
}
\newcommand{\tikz@xcancel}[1]{%
  \begin{tikzpicture}[baseline=(textbox.base),inner sep=0pt]
  \node[cross out,draw] (textbox) {#1}[thick, color=cancelcolor];
  \useasboundingbox (textbox);
  \end{tikzpicture}%
}
\newcommand{\xcancel}[1]{%
  \relax\ifmmode
    \mathchoice{\tikz@xcancel{$\displaystyle#1$}}
               {\tikz@xcancel{$\textstyle#1$}}
               {\tikz@xcancel{$\scriptstyle#1$}}
               {\tikz@xcancel{$\scriptscriptstyle#1$}}
  \else
    \tikz@xcancel{\strut#1}%
  \fi
}
\makeatother

\newcommand{\xcancelRA}{\xcancel{\rule[-.15cm]{0cm}{.5cm} \Rightarrow
    \rule[-.15cm]{0cm}{.5cm}}}

%%%%%%%%%%%%%%%%%%%%%%%%%%%%%%%%%%%
%%%%%%%%%%%%%%%%%%%%%%%%%%%%%%%%%%%

\newcommand{\vide}{\multicolumn{1}{c}{}}

%%%%%%%%%%%%%%%%%%%%%%%%%%%%%%%%%%%
%%%%%%%%%%%%%%%%%%%%%%%%%%%%%%%%%%%


\usepackage{multicol}
% \usepackage[latin1]{inputenc}
% \usepackage[T1]{fontenc}
\usepackage[utf8]{inputenc}
\usepackage[T1]{fontenc}
\usepackage[normalem]{ulem}
\usepackage[french]{babel}

\usepackage{url}    
\usepackage{hyperref}
\hypersetup{
  backref=true,
  pagebackref=true,
  hyperindex=true,
  colorlinks=true,
  breaklinks=true,
  urlcolor=blue,
  linkcolor=black,
  %%%%%%%%
  % ATTENTION : red changé en black pour le Livre !
  %%%%%%%%
  bookmarks=true,
  bookmarksopen=true
}

%%%%%%%%%%%%%%%%%%%%%%%%%%%%%%%%%%%%%%%%%%%
%% Pour faire des traits diagonaux dans les tableaux
%% Nécessite slashbox.sty
%\usepackage{slashbox}

\usepackage{tipa}
\usepackage{verbatim,listings}
\usepackage{graphicx}
\usepackage{fancyhdr}
\usepackage{mathrsfs}
\usepackage{pifont}
\usepackage{tablists}
\usepackage{dsfont,amsfonts,amssymb,amsmath,amsthm,stmaryrd,upgreek,manfnt}
\usepackage{enumerate}

%\newcolumntype{M}[1]{p{#1}}
\newcolumntype{C}[1]{>{\centering}m{#1}}
\newcolumntype{R}[1]{>{\raggedright}m{#1}}
\newcolumntype{L}[1]{>{\raggedleft}m{#1}}
\newcolumntype{P}[1]{>{\raggedright}p{#1}}
\newcolumntype{B}[1]{>{\raggedright}b{#1}}
\newcolumntype{Q}[1]{>{\raggedright}t{#1}}

\newcommand{\alias}[2]{
\providecommand{#1}{}
\renewcommand{#1}{#2}
}
\alias{\R}{\mathbb{R}}
\alias{\N}{\mathbb{N}}
\alias{\Z}{\mathbb{Z}}
\alias{\Q}{\mathbb{Q}}
\alias{\C}{\mathbb{C}}
\alias{\K}{\mathbb{K}}

%%%%%%%%%%%%
%% rendre +infty et -infty plus petits
%%%%%%%%%%%%
\newcommand{\sinfty}{{\scriptstyle \infty}}

%%%%%%%%%%%%%%%%%%%%%%%%%%%%%%%
%%%%% macros TP Scilab %%%%%%%%
\newcommand{\Scilab}{\textbf{Scilab}} %
\newcommand{\Scinotes}{\textbf{SciNotes}} %
\newcommand{\faire}{\noindent $\blacktriangleright$ } %
\newcommand{\fitem}{\scalebox{.8}{$\blacktriangleright$}} %
\newcommand{\entree}{{\small\texttt{ENTRÉE}}} %
\newcommand{\tab}{{\small\texttt{TAB}}} %
\newcommand{\mt}[1]{\mathtt{#1}} %
% guillemets droits

\newcommand{\ttq}{\textquotesingle} %

\newcommand{\reponse}[1]{\longboxed{
    \begin{array}{C{0.9\textwidth}}
      \nl[#1]
    \end{array}
  }} %

\newcommand{\reponseR}[1]{\longboxed{
    \begin{array}{R{0.9\textwidth}}
      #1
    \end{array}
  }} %

\newcommand{\reponseC}[1]{\longboxed{
    \begin{array}{C{0.9\textwidth}}
      #1
    \end{array}
  }} %

\colorlet{pyfunction}{Blue}
\colorlet{pyCle}{Magenta}
\colorlet{pycomment}{LimeGreen}
\colorlet{pydoc}{Cyan}
% \colorlet{SansCo}{white}
% \colorlet{AvecCo}{black}

\newcommand{\visible}[1]{{\color{ASCo}\colorlet{pydoc}{pyDo}\colorlet{pycomment}{pyCo}\colorlet{pyfunction}{pyF}\colorlet{pyCle}{pyC}\colorlet{function}{sciFun}\colorlet{var}{sciVar}\colorlet{if}{sciIf}\colorlet{comment}{sciComment}#1}} %

%%%% à changer ????
\newcommand{\invisible}[1]{{\color{ASCo}\colorlet{pydoc}{pyDo}\colorlet{pycomment}{pyCo}\colorlet{pyfunction}{pyF}\colorlet{pyCle}{pyC}\colorlet{function}{sciFun}\colorlet{var}{sciVar}\colorlet{if}{sciIf}\colorlet{comment}{sciComment}#1}} %

\newcommand{\invisibleCol}[2]{{\color{#1}#2}} %

\NewEnviron{solution} %
{ %
  \Boxed{
    \begin{array}{>{\color{ASCo}} R{0.9\textwidth}}
      \colorlet{pycomment}{pyCo}
      \colorlet{pydoc}{pyDo}
      \colorlet{pyfunction}{pyF}
      \colorlet{pyCle}{pyC}
      \colorlet{function}{sciFun}
      \colorlet{var}{sciVar}
      \colorlet{if}{sciIf}
      \colorlet{comment}{sciComment}
      \BODY
    \end{array}
  } %
} %

\NewEnviron{solutionC} %
{ %
  \Boxed{
    \begin{array}{>{\color{ASCo}} C{0.9\textwidth}}
      \colorlet{pycomment}{pyCo}
      \colorlet{pydoc}{pyDo}
      \colorlet{pyfunction}{pyF}
      \colorlet{pyCle}{pyC}
      \colorlet{function}{sciFun}
      \colorlet{var}{sciVar}
      \colorlet{if}{sciIf}
      \colorlet{comment}{sciComment}
      \BODY
    \end{array}
  } %
} %

\newcommand{\invite}{--\!\!>} %

%%%%% nouvel environnement tabular pour retour console %%%%
\colorlet{ConsoleColor}{Black!12}
\colorlet{function}{Red}
\colorlet{var}{Maroon}
\colorlet{if}{Magenta}
\colorlet{comment}{LimeGreen}

\newcommand{\tcVar}[1]{\textcolor{var}{\bf \small #1}} %
\newcommand{\tcFun}[1]{\textcolor{function}{#1}} %
\newcommand{\tcIf}[1]{\textcolor{if}{#1}} %
\newcommand{\tcFor}[1]{\textcolor{if}{#1}} %

\newcommand{\moins}{\!\!\!\!\!\!- }
\newcommand{\espn}{\!\!\!\!\!\!}

\usepackage{booktabs,varwidth} \newsavebox\TBox
\newenvironment{console}
{\begin{lrbox}{\TBox}\varwidth{\linewidth}
    \tabular{>{\tt\small}R{0.84\textwidth}}
    \nl[-.4cm]} {\endtabular\endvarwidth\end{lrbox}%
  \fboxsep=1pt\colorbox{ConsoleColor}{\usebox\TBox}}

\newcommand{\lInv}[1]{%
  $\invite$ #1} %

\newcommand{\lAns}[1]{%
  \qquad ans \ = \nl %
  \qquad \qquad #1} %

\newcommand{\lVar}[2]{%
  \qquad #1 \ = \nl %
  \qquad \qquad #2} %

\newcommand{\lDisp}[1]{%
  #1 %
} %

\newcommand{\ligne}[1]{\underline{\small \tt #1}} %

\newcommand{\ligneAns}[2]{%
  $\invite$ #1 \nl %
  \qquad ans \ = \nl %
  \qquad \qquad #2} %

\newcommand{\ligneVar}[3]{%
  $\invite$ #1 \nl %
  \qquad #2 \ = \nl %
  \qquad \qquad #3} %

\newcommand{\ligneErr}[3]{%
  $\invite$ #1 \nl %
  \quad !-{-}error #2 \nl %
  #3} %
%%%%%%%%%%%%%%%%%%%%%% 

\newcommand{\bs}[1]{\boldsymbol{#1}} %
\newcommand{\nll}{\nl[.4cm]} %
\newcommand{\nle}{\nl[.2cm]} %
%% opérateur puissance copiant l'affichage Scilab
%\newcommand{\puis}{\!\!\!~^{\scriptscriptstyle\pmb{\wedge}}}
\newcommand{\puis}{\mbox{$\hspace{-.1cm}~^{\scriptscriptstyle\pmb{\wedge}}
    \hspace{0.05cm}$}} %
\newcommand{\pointpuis}{.\mbox{$\hspace{-.15cm}~^{\scriptscriptstyle\pmb{\wedge}}$}} %
\newcommand{\Sfois}{\mbox{$\mt{\star}$}} %

%%%%% nouvel environnement tabular pour les encadrés Scilab %%%%
\newenvironment{encadre}
{\begin{lrbox}{\TBox}\varwidth{\linewidth}
    \tabular{>{\tt\small}C{0.1\textwidth}>{\small}R{0.7\textwidth}}}
  {\endtabular\endvarwidth\end{lrbox}%
  \fboxsep=1pt\longboxed{\usebox\TBox}}

\newenvironment{encadreL}
{\begin{lrbox}{\TBox}\varwidth{\linewidth}
    \tabular{>{\tt\small}C{0.25\textwidth}>{\small}R{0.6\textwidth}}}
  {\endtabular\endvarwidth\end{lrbox}%
  \fboxsep=1pt\longboxed{\usebox\TBox}}

\newenvironment{encadreF}
{\begin{lrbox}{\TBox}\varwidth{\linewidth}
    \tabular{>{\tt\small}C{0.2\textwidth}>{\small}R{0.70\textwidth}}}
  {\endtabular\endvarwidth\end{lrbox}%
  \fboxsep=1pt\longboxed{\usebox\TBox}}

\newenvironment{encadreLL}[2]
{\begin{lrbox}{\TBox}\varwidth{\linewidth}
    \tabular{>{\tt\small}C{#1\textwidth}>{\small}R{#2\textwidth}}}
  {\endtabular\endvarwidth\end{lrbox}%
  \fboxsep=1pt\longboxed{\usebox\TBox}}

%%%%% nouvel environnement tabular pour les script et fonctions %%%%
\newcommand{\commentaireDL}[1]{\multicolumn{1}{l}{\it
    \textcolor{comment}{$\slash\slash$ #1}}}

\newcommand{\commentaire}[1]{{\textcolor{comment}{$\slash\slash$ #1}}}

\newcounter{cptcol}

\newcommand{\nocount}{\multicolumn{1}{c}{}}

\newcommand{\sciNo}[1]{{\small \underbar #1}}

\NewEnviron{scilab}{ %
  \setcounter{cptcol}{0}
  \begin{center}
    \longboxed{
      \begin{tabular}{>{\stepcounter{cptcol}{\tiny \underbar
              \thecptcol}}c>{\tt}l}
        \BODY
      \end{tabular}
    }
  \end{center}
}

\NewEnviron{scilabNC}{ %
  \begin{center}
    \longboxed{
      \begin{tabular}{>{\tt}l} %
          \BODY
      \end{tabular}
    }
  \end{center}
}

\NewEnviron{scilabC}[1]{ %
  \setcounter{cptcol}{#1}
  \begin{center}
    \longboxed{
      \begin{tabular}{>{\stepcounter{cptcol}{\tiny \underbar
              \thecptcol}}c>{\tt}l}
        \BODY
      \end{tabular}
    }
  \end{center}
}

\newcommand{\scisol}[1]{ %
  \setcounter{cptcol}{0}
  \longboxed{
    \begin{tabular}{>{\stepcounter{cptcol}{\tiny \underbar
            \thecptcol}}c>{\tt}l}
      #1
    \end{tabular}
  }
}

\newcommand{\scisolNC}[1]{ %
  \longboxed{
    \begin{tabular}{>{\tt}l}
      #1
    \end{tabular}
  }
}

\newcommand{\scisolC}[2]{ %
  \setcounter{cptcol}{#1}
  \longboxed{
    \begin{tabular}{>{\stepcounter{cptcol}{\tiny \underbar
            \thecptcol}}c>{\tt}l}
      #2
    \end{tabular}
  }
}

\NewEnviron{syntaxe}{ %
  % \fcolorbox{black}{Yellow!20}{\setlength{\fboxsep}{3mm}
  \shadowbox{
    \setlength{\fboxsep}{3mm}
    \begin{tabular}{>{\tt}l}
      \BODY
    \end{tabular}
  }
}

%%%%% fin macros TP Scilab %%%%%%%%
%%%%%%%%%%%%%%%%%%%%%%%%%%%%%%%%%%%

%%%%%%%%%%%%%%%%%%%%%%%%%%%%%%%%%%%
%%%%% TP Python - listings %%%%%%%%
%%%%%%%%%%%%%%%%%%%%%%%%%%%%%%%%%%%
\newcommand{\Python}{\textbf{Python}} %

\lstset{% general command to set parameter(s)
basicstyle=\ttfamily\small, % print whole listing small
keywordstyle=\color{blue}\bfseries\underbar,
%% underlined bold black keywords
frame=lines,
xleftmargin=10mm,
numbers=left,
numberstyle=\tiny\underbar,
numbersep=10pt,
%identifierstyle=, % nothing happens
commentstyle=\color{green}, % white comments
%%stringstyle=\ttfamily, % typewriter type for strings
showstringspaces=false}

\newcommand{\pysolCpt}[2]{
  \setcounter{cptcol}{#1}
  \longboxed{
    \begin{tabular}{>{\stepcounter{cptcol}{\tiny \underbar
            \thecptcol}}c>{\tt}l}
        #2
      \end{tabular}
    }
} %

\newcommand{\pysol}[1]{
  \setcounter{cptcol}{0}
  \longboxed{
    \begin{tabular}{>{\stepcounter{cptcol}{\tiny \underbar
            \thecptcol}}c>{\tt}l}
        #1
      \end{tabular}
    }
} %

% \usepackage[labelsep=endash]{caption}

% avec un caption
\NewEnviron{pythonCap}[1]{ %
  \renewcommand{\tablename}{Programme}
  \setcounter{cptcol}{0}
  \begin{center}
    \longboxed{
      \begin{tabular}{>{\stepcounter{cptcol}{\tiny \underbar
              \thecptcol}}c>{\tt}l}
        \BODY
      \end{tabular}
    }
    \captionof{table}{#1}
  \end{center}
}

\NewEnviron{python}{ %
  \setcounter{cptcol}{0}
  \begin{center}
    \longboxed{
      \begin{tabular}{>{\stepcounter{cptcol}{\tiny \underbar
              \thecptcol}}c>{\tt}l}
        \BODY
      \end{tabular}
    }
  \end{center}
}

\newcommand{\pyVar}[1]{\textcolor{var}{\bf \small #1}} %
\newcommand{\pyFun}[1]{\textcolor{pyfunction}{#1}} %
\newcommand{\pyCle}[1]{\textcolor{pyCle}{#1}} %
\newcommand{\pyImp}[1]{{\bf #1}} %

%%%%% commentaire python %%%%
\newcommand{\pyComDL}[1]{\multicolumn{1}{l}{\textcolor{pycomment}{\#
      #1}}}

\newcommand{\pyCom}[1]{{\textcolor{pycomment}{\# #1}}}
\newcommand{\pyDoc}[1]{{\textcolor{pydoc}{#1}}}

\newcommand{\pyNo}[1]{{\small \underbar #1}}

%%%%%%%%%%%%%%%%%%%%%%%%%%%%%%%%%%%
%%%%%% Système linéaire paramétré : écrire les opérations au-dessus
%%%%%% d'un symbole équivalent
%%%%%%%%%%%%%%%%%%%%%%%%%%%%%%%%%%%

\usepackage{systeme}

\NewEnviron{arrayEq}{ %
  \stackrel{\scalebox{.6}{$
      \begin{array}{l} 
        \BODY \\[.1cm]
      \end{array}$}
  }{\Longleftrightarrow}
}

\NewEnviron{arrayEg}{ %
  \stackrel{\scalebox{.6}{$
      \begin{array}{l} 
        \BODY \\[.1cm]
      \end{array}$}
  }{=}
}

\NewEnviron{operationEq}{ %
  \scalebox{.6}{$
    \begin{array}{l} 
      \scalebox{1.6}{$\mbox{Opérations :}$} \\[.2cm]
      \BODY \\[.1cm]
    \end{array}$}
}

% \NewEnviron{arraySys}[1]{ %
%   \sysdelim\{.\systeme[#1]{ %
%     \BODY %
%   } %
% }

%%%%%

%%%%%%%%%%
%%%%%%%%%% ESSAI
\newlength\fboxseph
\newlength\fboxsepva
\newlength\fboxsepvb

\setlength\fboxsepva{0.2cm}
\setlength\fboxsepvb{0.2cm}
\setlength\fboxseph{0.2cm}

\makeatletter

\def\longboxed#1{\leavevmode\setbox\@tempboxa\hbox{\color@begingroup%
\kern\fboxseph{\m@th$\displaystyle #1 $}\kern\fboxseph%
\color@endgroup }\my@frameb@x\relax}

\def\my@frameb@x#1{%
  \@tempdima\fboxrule \advance\@tempdima \fboxsepva \advance\@tempdima
  \dp\@tempboxa\hbox {%
    \lower \@tempdima \hbox {%
      \vbox {\hrule\@height\fboxrule \hbox{\vrule\@width\fboxrule #1
          \vbox{%
            \vskip\fboxsepva \box\@tempboxa \vskip\fboxsepvb}#1
          \vrule\@width\fboxrule }%
        \hrule \@height \fboxrule }}}}

\newcommand{\boxedhv}[3]{\setlength\fboxseph{#1cm}
  \setlength\fboxsepva{#2cm}\setlength\fboxsepvb{#2cm}\longboxed{#3}}

\newcommand{\boxedhvv}[4]{\setlength\fboxseph{#1cm}
  \setlength\fboxsepva{#2cm}\setlength\fboxsepvb{#3cm}\longboxed{#4}}

\newcommand{\Boxed}[1]{{\setlength\fboxseph{0.2cm}
  \setlength\fboxsepva{0.2cm}\setlength\fboxsepvb{0.2cm}\longboxed{#1}}}

\newcommand{\mBoxed}[1]{{\setlength\fboxseph{0.2cm}
  \setlength\fboxsepva{0.2cm}\setlength\fboxsepvb{0.2cm}\longboxed{\mbox{#1}}}}

\newcommand{\mboxed}[1]{{\setlength\fboxseph{0.2cm}
  \setlength\fboxsepva{0.2cm}\setlength\fboxsepvb{0.2cm}\boxed{\mbox{#1}}}}

\newsavebox{\fmbox}
\newenvironment{fmpage}[1]
     {\begin{lrbox}{\fmbox}\begin{minipage}{#1}}
     {\end{minipage}\end{lrbox}\fbox{\usebox{\fmbox}}}

%%%%%%%%%%
%%%%%%%%%%

\DeclareMathOperator{\ch}{ch}
\DeclareMathOperator{\sh}{sh}

%%%%%%%%%%
%%%%%%%%%%

\newcommand{\norme}[1]{\Vert #1 \Vert}

%\newcommand*\widefbox[1]{\fbox{\hspace{2em}#1\hspace{2em}}}

\newcommand{\nl}{\tabularnewline}

\newcommand{\hand}{\noindent\ding{43}\ }
\newcommand{\ie}{\textit{i.e. }}
\newcommand{\cf}{\textit{cf }}

\newcommand{\Card}{\operatorname{Card}}

\newcommand{\aire}{\mathcal{A}}

\newcommand{\LL}[1]{\mathscr{L}(#1)} %
\newcommand{\B}{\mathscr{B}} %
\newcommand{\Bc}[1]{B_{#1}} %
\newcommand{\M}[1]{\mathscr{M}_{#1}(\mathbb{R})}

\DeclareMathOperator{\im}{Im}
\DeclareMathOperator{\kr}{Ker}
\DeclareMathOperator{\rg}{rg}
\DeclareMathOperator{\spc}{Sp}
\DeclareMathOperator{\sgn}{sgn}
\DeclareMathOperator{\supp}{Supp}

\newcommand{\Mat}{{\rm{Mat}}}
\newcommand{\Vect}[1]{{\rm{Vect}}\left(#1\right)}

\newenvironment{smatrix}{%
  \begin{adjustbox}{width=.9\width}
    $
    \begin{pmatrix}
    }{%      
    \end{pmatrix}
    $
  \end{adjustbox}
}

\newenvironment{sarray}[1]{%
  \begin{adjustbox}{width=.9\width}
    $
    \begin{array}{#1}
    }{%      
    \end{array}
    $
  \end{adjustbox}
}

\newcommand{\vd}[2]{
  \scalebox{.8}{
    $\left(\!
      \begin{array}{c}
        #1 \\
        #2
      \end{array}
    \!\right)$
    }}

\newcommand{\vt}[3]{
  \scalebox{.8}{
    $\left(\!
      \begin{array}{c}
        #1 \\
        #2 \\
        #3 
      \end{array}
    \!\right)$
    }}

\newcommand{\vq}[4]{
  \scalebox{.8}{
    $\left(\!
      \begin{array}{c}
        #1 \\
        #2 \\
        #3 \\
        #4 
      \end{array}
    \!\right)$
    }}

\newcommand{\vc}[5]{
  \scalebox{.8}{
    $\left(\!
      \begin{array}{c}
        #1 \\
        #2 \\
        #3 \\
        #4 \\
        #5 
      \end{array}
    \!\right)$
    }}

\newcommand{\ee}{\text{e}}

\newcommand{\dd}{\text{d}}

%%% Ensemble de définition
\newcommand{\Df}{\mathscr{D}}
\newcommand{\Cf}{\mathscr{C}}
\newcommand{\Ef}{\mathscr{C}}

\newcommand{\rond}[1]{\,\overset{\scriptscriptstyle \circ}{\!#1}}

\newcommand{\df}[2]{\dfrac{\partial #1}{\partial #2}} %
\newcommand{\dfn}[2]{\partial_{#2}(#1)} %
\newcommand{\ddfn}[2]{\partial^2_{#2}(#1)} %
\newcommand{\ddf}[2]{\dfrac{\partial^2 #1}{\partial #2^2}} %
\newcommand{\ddfr}[3]{\dfrac{\partial^2 #1}{\partial #2 \partial
    #3}} %


\newcommand{\dlim}[1]{{\displaystyle \lim_{#1} \ }}
\newcommand{\dlimPlus}[2]{
  \dlim{
    \scalebox{.6}{
      $
      \begin{array}{l}
        #1 \rightarrow #2\\
        #1 > #2
      \end{array}
      $}}}
\newcommand{\dlimMoins}[2]{
  \dlim{
    \scalebox{.6}{
      $
      \begin{array}{l}
        #1 \rightarrow #2\\
        #1 < #2
      \end{array}
      $}}}

%%%%%%%%%%%%%%
%% petit o, développement limité
%%%%%%%%%%%%%%

\newcommand{\oo}[2]{{\underset {{\overset {#1\rightarrow #2}{}}}{o}}} %
\newcommand{\oox}[1]{{\underset {{\overset {x\rightarrow #1}{}}}{o}}} %
\newcommand{\oon}{{\underset {{\overset {n\rightarrow +\infty}{}}}{o}}} %
\newcommand{\po}[1]{{\underset {{\overset {#1}{}}}{o}}} %
\newcommand{\neqx}[1]{{\ \underset {{\overset {x \to #1}{}}}{\not\sim}\ }} %
\newcommand{\eqx}[1]{{\ \underset {{\overset {x \to #1}{}}}{\sim}\ }} %
\newcommand{\eqn}{{\ \underset {{\overset {n \to +\infty}{}}}{\sim}\ }} %
\newcommand{\eq}[2]{{\ \underset {{\overset {#1 \to #2}{}}}{\sim}\ }} %
\newcommand{\DL}[1]{{\rm{DL}}_1 (#1)} %
\newcommand{\DLL}[1]{{\rm{DL}}_2 (#1)} %

\newcommand{\negl}{<<}

\newcommand{\neglP}[1]{\begin{array}{c}
    \vspace{-.2cm}\\
    << \\
    \vspace{-.7cm}\\
    {\scriptstyle #1}
  \end{array}}

%%%%%%%%%%%%%%
%% borne sup, inf, max, min
%%%%%%%%%%%%%%
\newcommand{\dsup}[1]{\displaystyle \sup_{#1} \ }
\newcommand{\dinf}[1]{\displaystyle \inf_{#1} \ }
\newcommand{\dmax}[1]{\max\limits_{#1} \ }
\newcommand{\dmin}[1]{\min\limits_{#1} \ }

\newcommand{\dcup}[2]{{\textstyle\bigcup\limits_{#1}^{#2}}\hspace{.1cm}}
%\displaystyle \bigcup_{#1}^{#2}}
\newcommand{\dcap}[2]{{\textstyle\bigcap\limits_{#1}^{#2}}\hspace{.1cm}}
% \displaystyle \bigcap_{#1}^{#2}
%%%%%%%%%%%%%%
%% opérateurs logiques
%%%%%%%%%%%%%%
\newcommand{\NON}[1]{\mathop{\small \tt{NON}} (#1)}
\newcommand{\ET}{\mathrel{\mathop{\small \mathtt{ET}}}}
\newcommand{\OU}{\mathrel{\mathop{\small \tt{OU}}}}
\newcommand{\XOR}{\mathrel{\mathop{\small \tt{XOR}}}}

\newcommand{\id}{{\rm{id}}}

\newcommand{\sbullet}{\scriptstyle \bullet}
\newcommand{\stimes}{\scriptstyle \times}

%%%%%%%%%%%%%%%%%%
%% Probabilités
%%%%%%%%%%%%%%%%%%
\newcommand{\Prob}{\mathbb{P}}
\newcommand{\Ev}[1]{\left[ {#1} \right]}
\newcommand{\Evmb}[1]{[ {#1} ]}
\newcommand{\E}{\mathbb{E}}
\newcommand{\V}{\mathbb{V}}
\newcommand{\Cov}{{\rm{Cov}}}
\newcommand{\U}[2]{\mathcal{U}(\llb #1, #2\rrb)}
\newcommand{\Uc}[2]{\mathcal{U}([#1, #2])}
\newcommand{\Ucof}[2]{\mathcal{U}(]#1, #2])}
\newcommand{\Ucoo}[2]{\mathcal{U}(]#1, #2[)}
\newcommand{\Ucfo}[2]{\mathcal{U}([#1, #2[)}
\newcommand{\Bern}[1]{\mathcal{B}\left(#1\right)}
\newcommand{\Bin}[2]{\mathcal{B}\left(#1, #2\right)}
\newcommand{\G}[1]{\mathcal{G}\left(#1\right)}
\newcommand{\Pois}[1]{\mathcal{P}\left(#1\right)}
\newcommand{\HG}[3]{\mathcal{H}\left(#1, #2, #3\right)}
\newcommand{\Exp}[1]{\mathcal{E}\left(#1\right)}
\newcommand{\Norm}[2]{\mathcal{N}\left(#1, #2\right)}

\DeclareMathOperator{\cov}{Cov}

\newcommand{\var}{v.a.r. }
\newcommand{\suit}{\hookrightarrow}

\newcommand{\flecheR}[1]{\rotatebox{90}{\scalebox{#1}{\color{red}
      $\curvearrowleft$}}}


\newcommand{\partie}[1]{\mathcal{P}(#1)}
\newcommand{\Cont}[1]{\mathcal{C}^{#1}}
\newcommand{\Contm}[1]{\mathcal{C}^{#1}_m}

\newcommand{\llb}{\llbracket}
\newcommand{\rrb}{\rrbracket}

%\newcommand{\im}[1]{{\rm{Im}}(#1)}
\newcommand{\imrec}[1]{#1^{- \mathds{1}}}

\newcommand{\unq}{\mathds{1}}

\newcommand{\Hyp}{\mathtt{H}}

\newcommand{\eme}[1]{#1^{\scriptsize \mbox{ème}}}
\newcommand{\er}[1]{#1^{\scriptsize \mbox{er}}}
\newcommand{\ere}[1]{#1^{\scriptsize \mbox{ère}}}
\newcommand{\nd}[1]{#1^{\scriptsize \mbox{nd}}}
\newcommand{\nde}[1]{#1^{\scriptsize \mbox{nde}}}

\newcommand{\truc}{\mathop{\top}}
\newcommand{\fois}{\mathop{\ast}}

\newcommand{\f}[1]{\overrightarrow{#1}}

\newcommand{\checked}{\textcolor{green}{\checkmark}}

\def\restriction#1#2{\mathchoice
              {\setbox1\hbox{${\displaystyle #1}_{\scriptstyle #2}$}
              \restrictionaux{#1}{#2}}
              {\setbox1\hbox{${\textstyle #1}_{\scriptstyle #2}$}
              \restrictionaux{#1}{#2}}
              {\setbox1\hbox{${\scriptstyle #1}_{\scriptscriptstyle #2}$}
              \restrictionaux{#1}{#2}}
              {\setbox1\hbox{${\scriptscriptstyle #1}_{\scriptscriptstyle #2}$}
              \restrictionaux{#1}{#2}}}
\def\restrictionaux#1#2{{#1\,\smash{\vrule height .8\ht1 depth .85\dp1}}_{\,#2}}

\makeatletter
\newcommand*{\da@rightarrow}{\mathchar"0\hexnumber@\symAMSa 4B }
\newcommand*{\da@leftarrow}{\mathchar"0\hexnumber@\symAMSa 4C }
\newcommand*{\xdashrightarrow}[2][]{%
  \mathrel{%
    \mathpalette{\da@xarrow{#1}{#2}{}\da@rightarrow{\,}{}}{}%
  }%
}
\newcommand{\xdashleftarrow}[2][]{%
  \mathrel{%
    \mathpalette{\da@xarrow{#1}{#2}\da@leftarrow{}{}{\,}}{}%
  }%
}
\newcommand*{\da@xarrow}[7]{%
  % #1: below
  % #2: above
  % #3: arrow left
  % #4: arrow right
  % #5: space left 
  % #6: space right
  % #7: math style 
  \sbox0{$\ifx#7\scriptstyle\scriptscriptstyle\else\scriptstyle\fi#5#1#6\m@th$}%
  \sbox2{$\ifx#7\scriptstyle\scriptscriptstyle\else\scriptstyle\fi#5#2#6\m@th$}%
  \sbox4{$#7\dabar@\m@th$}%
  \dimen@=\wd0 %
  \ifdim\wd2 >\dimen@
    \dimen@=\wd2 %   
  \fi
  \count@=2 %
  \def\da@bars{\dabar@\dabar@}%
  \@whiledim\count@\wd4<\dimen@\do{%
    \advance\count@\@ne
    \expandafter\def\expandafter\da@bars\expandafter{%
      \da@bars
      \dabar@ 
    }%
  }%  
  \mathrel{#3}%
  \mathrel{%   
    \mathop{\da@bars}\limits
    \ifx\\#1\\%
    \else
      _{\copy0}%
    \fi
    \ifx\\#2\\%
    \else
      ^{\copy2}%
    \fi
  }%   
  \mathrel{#4}%
}
\makeatother



\newcount\depth

\newcount\depth
\newcount\totaldepth

\makeatletter
\newcommand{\labelsymbol}{%
      \ifnum\depth=0
        %
      \else
        \rlap{\,$\bullet$}%
      \fi
}

\newcommand*\bernoulliTree[1]{%
    \depth=#1\relax            
    \totaldepth=#1\relax
    \draw node(root)[bernoulli/root] {\labelsymbol}[grow=right] \draw@bernoulli@tree;
    \draw \label@bernoulli@tree{root};                                   
}                                                                        

\def\draw@bernoulli@tree{%
    \ifnum\depth>0 
      child[parent anchor=east] foreach \type/\label in {left child/$E$,right child/$S$} {%
          node[bernoulli/\type] {\label\strut\labelsymbol} \draw@bernoulli@tree
      }
      coordinate[bernoulli/increment] (dummy)
   \fi%
}

\def\label@bernoulli@tree#1{%
    \ifnum\depth>0
      ($(#1)!0.5!(#1-1)$) node[fill=white,bernoulli/decrement] {\tiny$p$}
      \label@bernoulli@tree{#1-1}
      ($(#1)!0.5!(#1-2)$) node[fill=white] {\tiny$q$}
      \label@bernoulli@tree{#1-2}
      coordinate[bernoulli/increment] (dummy)
   \fi%
}

\makeatother

\tikzset{bernoulli/.cd,
         root/.style={},
         decrement/.code=\global\advance\depth by-1\relax,
         increment/.code=\global\advance\depth by 1\relax,
         left child/.style={bernoulli/decrement},
         right child/.style={}}


\newcommand{\eps}{\varepsilon}

% \newcommand{\tendi}[1]{\xrightarrow[\footnotesize #1 \rightarrow
%   +\infty]{}}%

\newcommand{\tend}{\rightarrow}%
\newcommand{\tendn}{\underset{n\to +\infty}{\longrightarrow}} %
\newcommand{\ntendn}{\underset{n\to
    +\infty}{\not\hspace{-.15cm}\longrightarrow}} %
% \newcommand{\tendn}{\xrightarrow[\footnotesize n \rightarrow
%   +\infty]{}}%
\newcommand{\Tendx}[1]{\xrightarrow[\footnotesize x \rightarrow
  #1]{}}%
\newcommand{\tendx}[1]{\underset{x\to #1}{\longrightarrow}}%
\newcommand{\ntendx}[1]{\underset{x\to #1}{\not\!\!\longrightarrow}}%
\newcommand{\tendd}[2]{\underset{#1\to #2}{\longrightarrow}}%
% \newcommand{\tendd}[2]{\xrightarrow[\footnotesize #1 \rightarrow
%   #2]{}}%
\newcommand{\tendash}[1]{\xdashrightarrow[\footnotesize #1 \rightarrow
  +\infty]{}}%
\newcommand{\tendashx}[1]{\xdashrightarrow[\footnotesize x \rightarrow
  #1]{}}%
\newcommand{\tendb}[1]{\underset{#1 \to +\infty}{\longrightarrow}}%
\newcommand{\tendL}{\overset{\mathscr L}{\underset{n \to
      +\infty}{\longrightarrow}}}%
\newcommand{\tendP}{\overset{\Prob}{\underset{n \to
      +\infty}{\longrightarrow}}}%
\newcommand{\tenddL}[1]{\overset{\mathscr L}{\underset{#1 \to
      +\infty}{\longrightarrow}}}%

\NewEnviron{attention}{ %
  ~\\[-.2cm]\noindent
  \begin{minipage}{\linewidth}
  \setlength{\fboxsep}{3mm}%
  \ \ \dbend \ \ %
  \fbox{\parbox[t]{.88\linewidth}{\BODY}} %
  \end{minipage}\\
}

\NewEnviron{sattention}[1]{ %
  ~\\[-.2cm]\noindent
  \begin{minipage}{#1\linewidth}
  \setlength{\fboxsep}{3mm}%
  \ \ \dbend \ \ %
  \fbox{\parbox[t]{.88\linewidth}{\BODY}} %
  \end{minipage}\\
}

%%%%% OBSOLETE %%%%%%

% \newcommand{\attention}[1]{
%   \noindent
%   \begin{tabular}{@{}l|p{11.5cm}|}
%     \cline{2-2}
%     \vspace{-.2cm} 
%     & \nl
%     \dbend & #1 \nl
%     \cline{2-2}
%   \end{tabular}
% }

% \newcommand{\attentionv}[2]{
%   \noindent
%   \begin{tabular}{@{}l|p{11.5cm}|}
%     \cline{2-2}
%     \vspace{-.2cm} 
%     & \nl
%     \dbend & #2 \nl[#1 cm]
%     \cline{2-2}
%   \end{tabular}
% }

\newcommand{\explainvb}[2]{
  \noindent
  \begin{tabular}{@{}l|p{11.5cm}|}
    \cline{2-2}
    \vspace{-.2cm} 
    & \nl
    \hand & #2 \nl [#1 cm]
    \cline{2-2}
  \end{tabular}
}


% \noindent
% \begin{tabular}{@{}l|lp{11cm}|}
%   \cline{3-3} 
%   \multicolumn{1}{@{}l@{\dbend}}{} & & #1 \nl
%   \multicolumn{1}{l}{} & & \nl [-.8cm]
%   & & #2 \nl
%   \cline{2-3}
% \end{tabular}

% \newcommand{\attention}[1]{
%   \noindent
%   \begin{tabular}{@{}@{}cp{11cm}}
%     \dbend & #1 \nl
%   \end{tabular}
% }

\newcommand{\PP}[1]{\mathcal{P}(#1)}
\newcommand{\HH}[1]{\mathcal{H}(#1)}
\newcommand{\FF}[1]{\mathcal{F}(#1)}

\newcommand{\DSum}[2]{\displaystyle\sum\limits_{#1}^{#2}\hspace{.1cm}}
\newcommand{\Sum}[2]{{\textstyle\sum\limits_{#1}^{#2}}\hspace{.1cm}}
\newcommand{\Serie}{\textstyle\sum\hspace{.1cm}}
\newcommand{\Prod}[2]{\textstyle\prod\limits_{#1}^{#2}}

\newcommand{\Prim}[3]{\left[\ {#1} \ \right]_{\scriptscriptstyle
   \hspace{-.15cm} ~_{#2}\, }^ {\scriptscriptstyle \hspace{-.15cm} ~^{#3}\, }}

% \newcommand{\Prim}[3]{\left[\ {#1} \ \right]_{\scriptscriptstyle
%     \!\!~_{#2}}^ {\scriptscriptstyle \!\!~^{#3}}}

\newcommand{\dint}[2]{\displaystyle \int_{#1}^{#2}\ }
\newcommand{\Int}[2]{{\rm{Int}}_{\scriptscriptstyle #1, #2}}
\newcommand{\dt}{\ dt}
\newcommand{\dx}{\ dx}

\newcommand{\llpar}[1]{\left(\!\!\!
    \begin{array}{c}
      \rule{0pt}{#1}
    \end{array}
  \!\!\!\right.}

\newcommand{\rrpar}[1]{\left.\!\!\!
    \begin{array}{c}
      \rule{0pt}{#1}
    \end{array}
  \!\!\!\right)}

\newcommand{\llacc}[1]{\left\{\!\!\!
    \begin{array}{c}
      \rule{0pt}{#1}
    \end{array}
  \!\!\!\right.}

\newcommand{\rracc}[1]{\left.\!\!\!
    \begin{array}{c}
      \rule{0pt}{#1}
    \end{array}
  \!\!\!\right\}}

\newcommand{\ttacc}[1]{\mbox{\rotatebox{-90}{\hspace{-.7cm}$\llacc{#1}$}}}
\newcommand{\bbacc}[1]{\mbox{\rotatebox{90}{\hspace{-.5cm}$\llacc{#1}$}}}

\newcommand{\comp}[1]{\overline{#1}}%

\newcommand{\dcomp}[2]{\stackrel{\mbox{\ \ \----}{\scriptscriptstyle
      #2}}{#1}}%

% \newcommand{\Comp}[2]{\stackrel{\mbox{\ \
%       \-------}{\scriptscriptstyle #2}}{#1}}

% \newcommand{\dcomp}[2]{\stackrel{\mbox{\ \
%       \-------}{\scriptscriptstyle #2}}{#1}}

\newcommand{\A}{\mathscr{A}}

\newcommand{\conc}[1]{
  \begin{center}
    \fbox{
      \begin{tabular}{c}
        #1
      \end{tabular}
    }
  \end{center}
}

\newcommand{\concC}[1]{
  \begin{center}
    \fbox{
    \begin{tabular}{C{10cm}}
      \quad #1 \quad
    \end{tabular}
    }
  \end{center}
}

\newcommand{\concL}[2]{
  \begin{center}
    \fbox{
    \begin{tabular}{C{#2cm}}
      \quad #1 \quad
    \end{tabular}
    }
  \end{center}
}


% \newcommand{\lims}[2]{\prod\limits_{#1}^{#2}}

\newtheorem{theorem}{Théorème}[]
\newtheorem{lemma}{Lemme}[]
\newtheorem{proposition}{Proposition}[]
\newtheorem{corollary}{Corollaire}[]

% \newenvironment{proof}[1][Démonstration]{\begin{trivlist}
% \item[\hskip \labelsep {\bfseries #1}]}{\end{trivlist}}
\newenvironment{definition}[1][Définition]{\begin{trivlist}
\item[\hskip \labelsep {\bfseries #1}]}{\end{trivlist}}
\newenvironment{example}[1][Exemple]{\begin{trivlist}
\item[\hskip \labelsep {\bfseries #1}]}{\end{trivlist}}
\newenvironment{examples}[1][Exemples]{\begin{trivlist}
\item[\hskip \labelsep {\bfseries #1}]}{\end{trivlist}}
\newenvironment{notation}[1][Notation]{\begin{trivlist}
\item[\hskip \labelsep {\bfseries #1}]}{\end{trivlist}}
\newenvironment{propriete}[1][Propriété]{\begin{trivlist}
\item[\hskip \labelsep {\bfseries #1}]}{\end{trivlist}}
\newenvironment{proprietes}[1][Propriétés]{\begin{trivlist}
\item[\hskip \labelsep {\bfseries #1}]}{\end{trivlist}}
% \newenvironment{remark}[1][Remarque]{\begin{trivlist}
% \item[\hskip \labelsep {\bfseries #1}]}{\end{trivlist}}
\newenvironment{application}[1][Application]{\begin{trivlist}
\item[\hskip \labelsep {\bfseries #1}]}{\end{trivlist}}

% Environnement pour les réponses des DS
\newenvironment{answer}{\par\emph{Réponse :}\par{}}
{\vspace{-.6cm}\hspace{\stretch{1}}\rule{1ex}{1ex}\vspace{.3cm}}

\newenvironment{answerTD}{\vspace{.2cm}\par\emph{Réponse :}\par{}}
{\hspace{\stretch{1}}\rule{1ex}{1ex}\vspace{.3cm}}

\newenvironment{answerCours}{\noindent\emph{Réponse :}}
{\rule{1ex}{1ex}}%\vspace{.3cm}}


% footnote in footer
\newcommand{\fancyfootnotetext}[2]{%
  \fancypagestyle{dingens}{%
    \fancyfoot[LO,RE]{\parbox{0.95\textwidth}{\footnotemark[#1]\footnotesize
        #2}}%
  }%
  \thispagestyle{dingens}%
}

%%% définit le style (arabic : 1,2,3...) et place des parenthèses
%%% autour de la numérotation
\renewcommand*{\thefootnote}{(\arabic{footnote})}
% http://www.tuteurs.ens.fr/logiciels/latex/footnote.html

%%%%%%%% tikz axis
% \pgfplotsset{every axis/.append style={
%                     axis x line=middle,    % put the x axis in the middle
%                     axis y line=middle,    % put the y axis in the middle
%                     axis line style={<->,color=blue}, % arrows on the axis
%                     xlabel={$x$},          % default put x on x-axis
%                     ylabel={$y$},          % default put y on y-axis
%             }}

%%%% s'utilise comme suit

% \begin{axis}[
%   xmin=-8,xmax=4,
%   ymin=-8,ymax=4,
%   grid=both,
%   ]
%   \addplot [domain=-3:3,samples=50]({x^3-3*x},{3*x^2-9}); 
% \end{axis}

%%%%%%%%



%%%%%%%%%%%% Pour avoir des numéros de section qui correspondent à
%%%%%%%%%%%% ceux du tableau
\renewcommand{\thesection}{\Roman{section}.\hspace{-.3cm}}
\renewcommand{\thesubsection}{\Roman{section}.\arabic{subsection}.\hspace{-.2cm}}
\renewcommand{\thesubsubsection}{\Roman{section}.\arabic{subsection}.\alph{subsubsection})\hspace{-.2cm}}
%%%%%%%%%%%% 

%%% Changer le nom des figures : Fig. au lieu de Figure
\usepackage[font=small,labelfont=bf,labelsep=space]{caption}
\captionsetup{%
  figurename=Fig.,
  tablename=Tab.
}
% \renewcommand{\thesection}{\Roman{section}.\hspace{-.2cm}}
% \renewcommand{\thesubsection}{\Roman{section}
%   .\hspace{.2cm}\arabic{subsection}\ .\hspace{-.3cm}}
% \renewcommand{\thesubsubsection}{\alph{subsection})}

\newenvironment{tabliste}[1]
{\begin{tabenum}[\bfseries\small\itshape #1]}{\end{tabenum}} 

%%%% ESSAI contre le too deeply nested %%%%
%%%% ATTENTION au package enumitem qui se comporte mal avec les
%%%% noliste, à redéfinir !
% \usepackage{enumitem}

% \setlistdepth{9}

% \newlist{myEnumerate}{enumerate}{9}
% \setlist[myEnumerate,1]{label=(\arabic*)}
% \setlist[myEnumerate,2]{label=(\Roman*)}
% \setlist[myEnumerate,3]{label=(\Alph*)}
% \setlist[myEnumerate,4]{label=(\roman*)}
% \setlist[myEnumerate,5]{label=(\alph*)}
% \setlist[myEnumerate,6]{label=(\arabic*)}
% \setlist[myEnumerate,7]{label=(\Roman*)}
% \setlist[myEnumerate,8]{label=(\Alph*)}
% \setlist[myEnumerate,9]{label=(\roman*)}

%%%%%

\newenvironment{noliste}[1] %
{\begin{enumerate}[\bfseries\small\itshape #1]} %
  {\end{enumerate}}

\newenvironment{nonoliste}[1] %
{\begin{enumerate}[\hspace{-12pt}\bfseries\small\itshape #1]} %
  {\end{enumerate}}

\newenvironment{arrayliste}[1]{ 
  % List with minimal white space to fit in small areas, e.g. table
  % cell
  \begin{minipage}[t]{\linewidth} %
    \begin{enumerate}[\bfseries\small\itshape #1] %
      {\leftmargin=0.5em \rightmargin=0em
        \topsep=0em \parskip=0em \parsep=0em
        \listparindent=0em \partopsep=0em \itemsep=0pt
        \itemindent=0em \labelwidth=\leftmargin\labelsep+0.25em}
    }{
    \end{enumerate}\end{minipage}
}

\newenvironment{nolistes}[2]
{\begin{enumerate}[\bfseries\small\itshape
    #1]\setlength{\itemsep}{#2 mm}}{\end{enumerate}}

\newenvironment{liste}[1]
{\begin{enumerate}[\hspace{12pt}\bfseries\small\itshape
    #1]}{\end{enumerate}}   


%%%%%%%% Pour les programmes de colle %%%%%%%

\newcommand{\cours}{{\small \tt (COURS)}} %
\newcommand{\poly}{{\small \tt (POLY)}} %
\newcommand{\exo}{{\small \tt (EXO)}} %
\newcommand{\culture}{{\small \tt (CULTURE)}} %
\newcommand{\methodo}{{\small \tt (MÉTHODO)}} %
\newcommand{\methodob}{\Boxed{\mbox{\tt MÉTHODO}}} %

%%%%%%%% Pour les TD %%%%%%%
\newtheoremstyle{exostyle} {\topsep} % espace avant
{.6cm} % espace apres
{} % Police utilisee par le style de thm
{} % Indentation (vide = aucune, \parindent = indentation paragraphe)
{\bfseries} % Police du titre de thm
{} % Signe de ponctuation apres le titre du thm
{ } % Espace apres le titre du thm (\newline = linebreak)
{\thmname{#1}\thmnumber{ #2}\thmnote{.
    \normalfont{\textit{#3}}}} % composants du titre du thm : \thmname
                               % = nom du thm, \thmnumber = numéro du
                               % thm, \thmnote = sous-titre du thm
 
\theoremstyle{exostyle}
\newtheorem{exercice}{Exercice}
\newtheorem*{exoCours}{Exercice}

%%%%%%%% Pour des théorèmes Sans Espaces APRÈS %%%%%%%
\newtheoremstyle{exostyleSE} {\topsep} % espace avant
{} % espace apres
{} % Police utilisee par le style de thm
{} % Indentation (vide = aucune, \parindent = indentation paragraphe)
{\bfseries} % Police du titre de thm
{} % Signe de ponctuation apres le titre du thm
{ } % Espace apres le titre du thm (\newline = linebreak)
{\thmname{#1}\thmnumber{ #2}\thmnote{.
    \normalfont{\textit{#3}}}} % composants du titre du thm : \thmname
                               % = nom du thm, \thmnumber = numéro du
                               % thm, \thmnote = sous-titre du thm
 
\theoremstyle{exostyleSE}
\newtheorem{exerciceSE}{Exercice}
\newtheorem*{exoCoursSE}{Exercice}

% \newcommand{\lims}[2]{\prod\limits_{#1}^{#2}}

\newtheorem{theoremSE}{Théorème}[]
\newtheorem{lemmaSE}{Lemme}[]
\newtheorem{propositionSE}{Proposition}[]
\newtheorem{corollarySE}{Corollaire}[]

% \newenvironment{proofSE}[1][Démonstration]{\begin{trivlist}
% \item[\hskip \labelsep {\bfseries #1}]}{\end{trivlist}}
\newenvironment{definitionSE}[1][Définition]{\begin{trivlist}
  \item[\hskip \labelsep {\bfseries #1}]}{\end{trivlist}}
\newenvironment{exampleSE}[1][Exemple]{\begin{trivlist} 
  \item[\hskip \labelsep {\bfseries #1}]}{\end{trivlist}}
\newenvironment{examplesSE}[1][Exemples]{\begin{trivlist}
\item[\hskip \labelsep {\bfseries #1}]}{\end{trivlist}}
\newenvironment{notationSE}[1][Notation]{\begin{trivlist}
\item[\hskip \labelsep {\bfseries #1}]}{\end{trivlist}}
\newenvironment{proprieteSE}[1][Propriété]{\begin{trivlist}
\item[\hskip \labelsep {\bfseries #1}]}{\end{trivlist}}
\newenvironment{proprietesSE}[1][Propriétés]{\begin{trivlist}
\item[\hskip \labelsep {\bfseries #1}]}{\end{trivlist}}
\newenvironment{remarkSE}[1][Remarque]{\begin{trivlist}
\item[\hskip \labelsep {\bfseries #1}]}{\end{trivlist}}
\newenvironment{applicationSE}[1][Application]{\begin{trivlist}
\item[\hskip \labelsep {\bfseries #1}]}{\end{trivlist}}

%%%%%%%%%%% Obtenir les étoiles sans charger le package MnSymbol
%%%%%%%%%%%
\DeclareFontFamily{U} {MnSymbolC}{}
\DeclareFontShape{U}{MnSymbolC}{m}{n}{
  <-6> MnSymbolC5
  <6-7> MnSymbolC6
  <7-8> MnSymbolC7
  <8-9> MnSymbolC8
  <9-10> MnSymbolC9
  <10-12> MnSymbolC10
  <12-> MnSymbolC12}{}
\DeclareFontShape{U}{MnSymbolC}{b}{n}{
  <-6> MnSymbolC-Bold5
  <6-7> MnSymbolC-Bold6
  <7-8> MnSymbolC-Bold7
  <8-9> MnSymbolC-Bold8
  <9-10> MnSymbolC-Bold9
  <10-12> MnSymbolC-Bold10
  <12-> MnSymbolC-Bold12}{}

\DeclareSymbolFont{MnSyC} {U} {MnSymbolC}{m}{n}

\DeclareMathSymbol{\filledlargestar}{\mathrel}{MnSyC}{205}
\DeclareMathSymbol{\largestar}{\mathrel}{MnSyC}{131}

\newcommand{\facile}{\rm{(}$\scriptstyle\largestar$\rm{)}} %
\newcommand{\moyen}{\rm{(}$\scriptstyle\filledlargestar$\rm{)}} %
\newcommand{\dur}{\rm{(}$\scriptstyle\filledlargestar\filledlargestar$\rm{)}} %
\newcommand{\costaud}{\rm{(}$\scriptstyle\filledlargestar\filledlargestar\filledlargestar$\rm{)}}

%%%%%%%%%%%%%%%%%%%%%%%%%

%%%%%%%%%%%%%%%%%%%%%%%%%
%%%%%%%% Fin de la partie TD

%%%%%%%%%%%%%%%%
%%%%%%%%%%%%%%%%
\makeatletter %
\newenvironment{myitemize}{%
  \setlength{\topsep}{0pt} %
  \setlength{\partopsep}{0pt} %
  \renewcommand*{\@listi}{\leftmargin\leftmargini \parsep\z@
    \topsep\z@ \itemsep\z@} \let\@listI\@listi %
  \itemize %
}{\enditemize} %
\makeatother
%%%%%%%%%%%%%%%%
%%%%%%%%%%%%%%%%

%% Commentaires dans la correction du livre

\newcommand{\Com}[1]{
% Define box and box title style
\tikzstyle{mybox} = [draw=black!50,
very thick,
    rectangle, rounded corners, inner sep=10pt, inner ysep=8pt]
\tikzstyle{fancytitle} =[rounded corners, fill=black!80, text=white]
\tikzstyle{fancylogo} =[ text=white]
\begin{center}

\begin{tikzpicture}
\node [mybox] (box){%

    \begin{minipage}{0.90\linewidth}
\vspace{6pt}  #1
    \end{minipage}
};
\node[fancytitle, right=10pt] at (box.north west) 
{\bfseries\normalsize{Commentaire}};

\end{tikzpicture}%

\end{center}
%
}

\NewEnviron{remark}{%
  % Define box and box title style
  \tikzstyle{mybox} = [draw=black!50, very thick, rectangle, rounded
  corners, inner sep=10pt, inner ysep=8pt] %
  \tikzstyle{fancytitle} = [rounded corners , fill=black!80,
  text=white] %
  \tikzstyle{fancylogo} =[ text=white]
  \begin{center}
    \begin{tikzpicture}
      \node [mybox] (box){%
        \begin{minipage}{0.90\linewidth}
          \vspace{6pt} \BODY
        \end{minipage}
      }; %
      \node[fancytitle, right=10pt] at (box.north west) %
      {\bfseries\normalsize{Commentaire}}; %
    \end{tikzpicture}%
  \end{center}
}

\NewEnviron{remarkST}{%
  % Define box and box title style
  \tikzstyle{mybox} = [draw=black!50, very thick, rectangle, rounded
  corners, inner sep=10pt, inner ysep=8pt] %
  \tikzstyle{fancytitle} = [rounded corners , fill=black!80,
  text=white] %
  \tikzstyle{fancylogo} =[ text=white]
  \begin{center}
    \begin{tikzpicture}
      \node [mybox] (box){%
        \begin{minipage}{0.90\linewidth}
          \vspace{6pt} \BODY
        \end{minipage}
      }; %
      % \node[fancytitle, right=10pt] at (box.north west) %
%       {\bfseries\normalsize{Commentaire}}; %
    \end{tikzpicture}%
  \end{center}
}

\NewEnviron{remarkL}[1]{%
  % Define box and box title style
  \tikzstyle{mybox} = [draw=black!50, very thick, rectangle, rounded
  corners, inner sep=10pt, inner ysep=8pt] %
  \tikzstyle{fancytitle} =[rounded corners, fill=black!80,
  text=white] %
  \tikzstyle{fancylogo} =[ text=white]
  \begin{center}
    \begin{tikzpicture}
      \node [mybox] (box){%
        \begin{minipage}{#1\linewidth}
          \vspace{6pt} \BODY
        \end{minipage}
      }; %
      \node[fancytitle, right=10pt] at (box.north west) %
      {\bfseries\normalsize{Commentaire}}; %
    \end{tikzpicture}%
  \end{center}
}

\NewEnviron{remarkSTL}[1]{%
  % Define box and box title style
  \tikzstyle{mybox} = [draw=black!50, very thick, rectangle, rounded
  corners, inner sep=10pt, inner ysep=8pt] %
  \tikzstyle{fancytitle} =[rounded corners, fill=black!80,
  text=white] %
  \tikzstyle{fancylogo} =[ text=white]
  \begin{center}
    \begin{tikzpicture}
      \node [mybox] (box){%
        \begin{minipage}{#1\linewidth}
          \vspace{6pt} \BODY
        \end{minipage}
      }; %
%       \node[fancytitle, right=10pt] at (box.north west) %
%       {\bfseries\normalsize{Commentaire}}; %
    \end{tikzpicture}%
  \end{center}
}

\NewEnviron{titre} %
{ %
  ~\\[-1.8cm]
  \begin{center}
    \bf \LARGE \BODY
  \end{center}
  ~\\[-.6cm]
  \hrule %
  \vspace*{.2cm}
} %

\NewEnviron{titreL}[2] %
{ %
  ~\\[-#1cm]
  \begin{center}
    \bf \LARGE \BODY
  \end{center}
  ~\\[-#2cm]
  \hrule %
  \vspace*{.2cm}
} %



%%%%%%%%%%% Redefinition \chapter



\usepackage[explicit]{titlesec}
\usepackage{color}
\titleformat{\chapter}
{\gdef\chapterlabel{}
\selectfont\huge\bf}
%\normalfont\sffamily\Huge\bfseries\scshape}
{\gdef\chapterlabel{\thechapter)\ }}{0pt}
{\begin{tikzpicture}[remember picture,overlay]
\node[yshift=-3cm] at (current page.north west)
{\begin{tikzpicture}[remember picture, overlay]
\draw (.1\paperwidth,0) -- (.9\paperwidth,0);
\draw (.1\paperwidth,2) -- (.9\paperwidth,2);
%(\paperwidth,3cm);
\node[anchor=center,xshift=.5\paperwidth,yshift=1cm, rectangle,
rounded corners=20pt,inner sep=11pt]
{\color{black}\chapterlabel#1};
\end{tikzpicture}
};
\end{tikzpicture}
}
\titlespacing*{\chapter}{0pt}{50pt}{-75pt}




%%%%%%%%%%%%%% Affichage chapter dans Table des matieres

\makeatletter
\renewcommand*\l@chapter[2]{%
  \ifnum \c@tocdepth >\m@ne
    \addpenalty{-\@highpenalty}%
    \vskip 1.0em \@plus\p@
    \setlength\@tempdima{1.5em}%
    \begingroup
      \parindent \z@ \rightskip \@pnumwidth
      \parfillskip -\@pnumwidth
      \leavevmode %\bfseries
      \advance\leftskip\@tempdima
      \hskip -\leftskip
      #1\nobreak\ 
       \leaders\hbox{$\m@th
        \mkern \@dotsep mu\hbox{.}\mkern \@dotsep
        mu$}\hfil\nobreak\hb@xt@\@pnumwidth{\hss #2}\par
      \penalty\@highpenalty
    \endgroup
  \fi}
\makeatother




%%%%%%%%%%%%%%%%% Redefinition part




% \renewcommand{\thesection}{\Roman{section}.\hspace{-.3cm}}
% \renewcommand{\thesubsection}{\Alph{subsection}.\hspace{-.2cm}}

\pagestyle{fancy} %
\pagestyle{fancy} %
 \lhead{ECE2 \hfill Mathématiques \\} %
\chead{\hrule} %
\rhead{} %
\lfoot{} %
\cfoot{} %
\rfoot{\thepage} %

\renewcommand{\headrulewidth}{0pt}% : Trace un trait de séparation
                                    % de largeur 0,4 point. Mettre 0pt
                                    % pour supprimer le trait.

\renewcommand{\footrulewidth}{0.4pt}% : Trace un trait de séparation
                                    % de largeur 0,4 point. Mettre 0pt
                                    % pour supprimer le trait.

\setlength{\headheight}{14pt}

\title{\bf \vspace{-1.6cm} EML 2015} %
\author{} %
\date{} %
\begin{document}

\maketitle %
\vspace{-1.2cm}\hrule %
\thispagestyle{fancy}

\vspace*{.4cm}

%%DEBUT

\section*{Exercice 1}

\noindent
Dans tout l'exercice, $(\Omega, \A,\Prob)$ désigne un espace
probabilisé et toutes les variables aléatoires considérées seront
supposées définies sur cet espace.

\subsection*{Partie I : Loi exponentielle}

\noindent
Dans toute cette partie, $\lambda$ désigne un réel strictement positif.
\begin{noliste}{1.}
 \setlength{\itemsep}{4mm}
 \item Donner une densité, la fonction de répartition, l'espérance et 
 la variance d'une variable aléatoire suivant la loi exponentielle de 
 paramètre $\lambda$.
 
 \begin{proof}~\\
   Soit $X$ une \var de loi $\Exp{\lambda}$.%
   \conc{Alors la fonction $f_X : x \mapsto \left\{
       \begin{array}{cl}
         \lambda \, \ee^{-\lambda \, x} & \mbox{ si $x\geq 0$}\\
         0 & \mbox{ si $x<0$}
       \end{array}
     \right.$ est une densité de $X$.}%
   \conc{La fonction de répartition de $X$ est la fonction $F_X : x
     \mapsto \left\{
       \begin{array}{cl}
         1-\ee^{-\lambda \, x} & \mbox{ si $x\geq 0$}\\
         0 & \mbox{ si $x<0$}
       \end{array}
     \right.$.}%
   \conc{Enfin, $X$ admet une espérance et une variance données par :
     $\E(X)=\dfrac{1}{\lambda}$ et $\V(X)=\dfrac{1}{\lambda^2}$.}
   ~\\[-1cm]
 \end{proof}
 
 \item Justifier que les intégrales suivantes convergent et donner 
 leurs valeurs : 
 \[
  \dint{0}{+\infty} \ee^{-\lambda x} \dx, \quad \dint{0}{+\infty} x \, 
  \ee^{-\lambda x} \dx
 \]
 
 \begin{proof}~%\\
   \begin{noliste}{$\sbullet$}
   \item Tout d'abord : 
     \[
     \dint{-\infty}{+\infty} f_X(x) \dx \ = \ \dint{0}{+\infty} f_X(x)
     \dx
     \]
     car $f_X$ est nulle en dehors de $[0, +\infty[$.\\
     La fonction $f_X$ étant une densité de probabilité, l'intégrale
     impropre $\dint{0}{+\infty} f_X(x) \dx$ est convergente. Il en
     est de même de $\dint{0}{+\infty} \ee^{-\lambda x} \dx$ car on ne
     change pas la nature d'une intégrale impropre en multipliant son
     intégrande par un réel non nul.
   \item On en déduit alors : 
     \[
     \begin{array}{rcl@{\quad}>{\it}R{4.5cm}}
       \dint{0}{+\infty} \ee^{-\lambda \, x} \dx & = & \dfrac{1}{\lambda} 
       \dint{0}{+\infty} \lambda \, \ee^{-\lambda \, x} \dx 
       \\[.6cm]
       & = & \dfrac{1}{\lambda} \dint{-\infty}{+\infty} f_X(x) \dx 
       & (par définition de $f_X$)
       \nl
       \nl[-.2cm]
       & = & \dfrac{1}{\lambda} \times 1
       & (car $f_X$ est une densité)
     \end{array}
     \] %
     \conc{$\dint{0}{+\infty} \ee^{-\lambda \, x} \dx =
       \dfrac{1}{\lambda}$}


     \newpage


   \item En raisonnant de même, on démontre que l'intégrale impropre
     $\dint{0}{+\infty} \ee^{- \lambda x} \dx$ est convergente car $X$
     admet une espérance.
   \item De plus :
     \[
     \begin{array}{rcl@{\quad}>{\it}R{4.5cm}}
       \dint{0}{+\infty} x \, \ee^{-\lambda \, x} \dx 
       & = & \dfrac{1}{\lambda} \dint{0}{+\infty} x \, \lambda \, 
       \ee^{-\lambda \, x} \dx 
       \\[.6cm]
       & = & \dfrac{1}{\lambda} \dint{-\infty}{+\infty} x \, f_X(x) \dx 
       & (par définition de $f_X$)
       \nl
       \nl[-.2cm]
       & = & \dfrac{1}{\lambda} \times \E(X) \ = \ \dfrac{1}{\lambda}
       \times \dfrac{1}{\lambda} \ = \ \dfrac{1}{\lambda^2}
     \end{array}
     \] %
     \conc{$\dint{0}{+\infty} x \, \ee^{-\lambda \, x} \dx =
       \dfrac{1}{\lambda^2}$}~\\[-1cm]
   \end{noliste}
 \end{proof}
 
\item
  \begin{noliste}{a)}
    \setlength{\itemsep}{2mm}
  \item Soit $U$ une variable aléatoire suivant la loi uniforme sur
    $[0,1[$.\\
    Quelle est la loi de la variable aléatoire $V= - \frac{1}{\lambda}
    \, \ln(1-U)$ ?
    
    \begin{proof}~
      \begin{noliste}{$\sbullet$}
      \item Notons $g : x \mapsto - \frac{1}{\lambda} \
        \ln(1-x)$. Tout d'abord :
        \[
        \begin{array}{rcl@{\quad}>{\it}R{5cm}}
          V(\Omega) & = & \big( g(U) \big)
          \hspace{.1cm} (\Omega)
          \\[.2cm]
          & = & g \hspace{.1cm} \big(U(\Omega) \big)
          \\[.2cm]
          & = & g \hspace{.1cm} \big( [0, 1[ \big) 
          % \\[.2cm]
          % & \subset & ]0,1[ & (d'après les implications précédentes)
        \end{array}
        \]
        Or, comme $g$ est continue et strictement croissante sur $[0,
        1[$ :
        \[
        g \hspace{.1cm} \big( [0, 1[ \big) = \ [g(0), \dlim{x \tend 1}
        g(x)[ \ = \ [0, +\infty[
        \]
        En effet : 
        \begin{noliste}{$\stimes$}
        \item $g(0) = - \frac{1}{\lambda} \ \ln(1) = 0$.
        \item $\dlim{x \tend 1} g(x) = \dlim{x \tend 1} -
          \frac{1}{\lambda} \ \ln(1-x) = +\infty$ \ car \ $\dlim{x
            \tend 1} \ln(1-x) = -\infty$.
        \end{noliste}
        \conc{$V(\Omega) \ = \ [0, +\infty[$}

%     \item Déterminons le support de $V=f(U)$, où $f:x\mapsto 
%     -\dfrac{1}{\lambda} \ln(1-x)$.\\
%     On a les équivalences suivantes :
%     \[
%       \begin{array}{rcl@{\quad}>{\it}R{4cm}}
%       x\in U(\Omega) & \Leftrightarrow & x\in[0,1]\\[.2cm] 
%       & \Leftrightarrow & 0 \leq x \leq 1\\[.2cm]
%       & \Leftrightarrow & 0 \leq 1-x \leq 1\\
%       & \Leftrightarrow & - \infty < \ln(1-x) \leq 0 & (car la 
%       fonction $\ln$ est strictement croissante sur $]0,+\infty[$)
%       \nl[-.4cm]
%       \nl
%       & \Leftrightarrow & 0 \leq -\dfrac{1}{\lambda} \ln(1-x) < + 
%     \infty & (car $-\dfrac{1}{\lambda} <0$)
%       \nl[-.2cm]
%       \nl
%       & \Leftrightarrow & 0 \leq f(x) < +\infty\\[.2cm]
%       & \Leftrightarrow & f(x)\in[0,+\infty[
%       \end{array}
%     \]
%     Or on sait que : 
%     \[
%     \begin{array}{rcl@{\quad}>{\it}R{5cm}}
%       V(\Omega) &=& (f(U))(\Omega)\\[.2cm]
%       &=& \{f(x) \ | \ x\in U(\Omega)\}\\
%       &=& [0,+\infty[ & (d'après les équivalences précédentes)
%     \end{array}
%     \]
%     Finalement $V(\Omega)=[0,+\infty[$.

    \item Déterminons la fonction de répartition de $V$.\\
    Soit $x\in\R$. Deux cas se présentent.
    \begin{noliste}{$\stimes$}
    \item \dashuline{Si $x<0$}, alors 
    $\Ev{V\leq x}=\varnothing$ car $V(\Omega)=[0,+\infty[$. Donc :
    \[
      F_V(x) = \Prob(\Ev{V\leq X}) = \Prob(\varnothing) 
      = 0
    \]

  \item \dashuline{Si $x \geq 0$}.
    \[
    \begin{array}{rcl@{\quad}>{\it}R{5.5cm}}
      F_V(x) & = & \Prob(\Ev{V\leq x})\\[.2cm]
      & = & \Prob\left(\Ev{ -\dfrac{1}{\lambda} \ln(1-U) 
          \leq x}\right)
      \\[.6cm]
      & = & \Prob(\Ev{\ln(1-U) \geq -\lambda x}) & (car $-\lambda<0$)
      \nl
      \nl[-.2cm]
      & = & \Prob\left(\Ev{1-U \geq \ee^{-\lambda x}}\right) & (car la 
      fonction $\exp$ est strictement croissante sur $\R$)
      \nl
      \nl[-.4cm]
      & = & \Prob\left(\Ev{ U \leq  1 - \ee^{-\lambda x}}\right)
      \\[.2cm]
      & = & F_U\left( 1-\ee^{-\lambda x} \right)
      \\[.2cm]
      & = & 1 - \ee^{-\lambda x} & (car $1 - \ee^{-\lambda x} \in [0,1[$)
    \end{array}
    \]


    \newpage


    \noindent
    La dernière égalité est justifié par les équivalences suivantes :
    \[
    \begin{array}{rcl@{\quad}>{\it}R{5.5cm}}
      0\leq x < +\infty & \Leftrightarrow & 0 \geq -\lambda x > 
      -\infty & (car $-\lambda <0$)
      \nl
      \nl[-.2cm]
      & \Leftrightarrow & 1=\ee^0 \geq \ee^{-\lambda x} > 0 & (car 
      la fonction $\exp$ est strictement croissante sur $\R$)
      \nl[-.4cm]
      \nl
      & \Leftrightarrow & 0 \leq 1-\ee^{-\lambda x} < 1
    \end{array}
    \]
%     De plus, comme $U \suit \mathcal{U}([0,1[)$, pour tout $x\in\R$, 
%     $F_U(x)=\left\{
%     \begin{array}{l@{\quad}>{}R{2cm}}
%     0 & si $x<0$\nl
%     x & si $x\in[0,1[$\nl
%     1 & si $x\geq 1$
%     \end{array}
%     \right.$,\\ 
%     donc $F_U\left( 1-\ee^{-\lambda x} \right)=
%     1-\ee^{-\lambda x}$.
  \end{noliste}
\item Finalement : 
  \[
  F_V : x \mapsto \left\{
    \begin{array}{lR{1.5cm}}
      1-\ee^{-\lambda x} & si $x\geq 0$ \nl
      0 & si $x<0$ 
    \end{array}
  \right.
  \]
  On reconnaît la fonction de répartition de la loi exponentielle de
  paramètre $\lambda$. %
  \end{noliste}
  \conc{Or la fonction de répartition caractérise la loi d'une \var,
    donc $X\suit \Exp{\lambda}$.}
   
   \begin{remark}~%\\
     \begin{noliste}{$\sbullet$}
     \item Cette question est un cas particulier de la question
       classique consistant à déterminer la loi de la transformée
       d'une \var à densité. Il est essentiel de maîtriser la
       méthodologie de résolution.
     \item Ici, on est dans un cas encore plus classique puisque la
       \var à densité de départ $U$ suit une loi uniforme sur
       $[0,1[$. On illustre ici la méthode d'inversion : on obtient
       une \var $V$ qui suit une loi particulière ($V \suit \Exp{1}$)
       en écrivant $V$ comme transformée d'une \var $U$ qui suit la
       loi uniforme sur $[0, 1[$. Ce type de question est fréquent
       dans les sujets. Et est généralement suivi, comme c'est le cas
       dans cet énoncé, d'une question de simulation informatique.\\
       Finalement, la Partie I de cet exercice est constituée en
       intégralité de questions de cours.
     \end{noliste}

   \end{remark}~\\[-1.2cm]
  \end{proof}
  
  \item Écrire une fonction en \Scilab{} qui, étant donné un réel 
  $\lambda$ strictement positif, simule la loi exponentielle de 
  paramètre $\lambda$.
  
  \begin{proof}~
   \begin{scilab}
     & \tcFun{function} \tcVar{v} = simuExp(\tcVar{lambda}) \nl %
     & \qquad u = rand() \nl %
     & \qquad \tcVar{v} = -(1/\tcVar{lambda}) \Sfois{} log(1-u) \nl %
     & \tcFun{endfunction}
   \end{scilab}
  \end{proof}
 \end{noliste}
 
 \noindent
 On considère une suite $(X_n)_{n\in\N^*}$ de variables aléatoires 
 indépendantes suivant toutes la loi exponentielle de paramètre 1.\\
 Pour tout $n$ de $\N^*$, on définit la variable aléatoire 
 $T_n=\max(X_1,...,X_n)$ qui, à tout $\omega$ de $\Omega$, associe le 
 plus grand des réels $X_1(\omega), \ldots, X_n(\omega)$ et on note 
 $f_n$ la fonction définie sur $\R$ par :
 \[
 \forall x\in\R, \ f_n(x)= \left\{
 \begin{array}{cl}
  n\ee^{-x}(1-\ee^{-x})^{n-1} & \text{si } x\geq 0,\\
  0 & \text{si } x<0.
 \end{array}
 \right.
 \]
\end{noliste}
 

\subsection*{Partie II : Loi de la variable aléatoire $T_n$}

\begin{noliste}{1.}
 \setlength{\itemsep}{4mm}
 \setcounter{enumi}{3}
 \item 
 \begin{noliste}{a)}
  \setlength{\itemsep}{2mm}
  \item Calculer, pour tout $n$ de $\N^*$ et pour tout $x$ de $\R_+^*$, 
  la probabilité $\Prob(\Ev{T_n\leq x})$.
  
  \begin{proof}~\\
    Soit $n\in\N^*$ et soit $x\in \R_+^*$.
    \begin{noliste}{$\sbullet$}
    \item Tout d'abord :
      \[
      \Ev{T_n \leq x} = \Ev{ \max(X_1, \ldots, X_n) \leq x} = \Ev{X_1
        \leq x} \cap \cdots \cap \Ev{X_n \leq x} = \dcap{i=1}{n}
      \Ev{X_i \leq x}
      \]
   
   
      \newpage


    \item On en déduit :
      \[
      \begin{array}{rcl@{\quad}>{\it}R{5cm}}
        \Prob(\Ev{T_n \leq x}) & = & \Prob\left( \dcap{i=1}{n} \Ev{X_i \leq 
            x}\right)
        \\[.4cm]
        & = & \Prod{i=1}{n} \Prob(\Ev{X_i \leq x}) & (car les \var
        $X_1$, $\ldots$, $X_n$ \\ sont indépendantes)
        \nl
        \nl[-.2cm]
        & = & \Prod{i=1}{n} \Prob(\Ev{X_1 \leq x}) & (car les \var
        $X_1$, $\ldots$, $X_n$ \\ ont même loi)
        \nl
        \nl[-.2cm]
        & = & \big(\Prob(\Ev{X_1\leq x}) \big)^n
        \\[.4cm]
        & = & \big(1-\ee^{-x} \big)^n & (car $X_1 \suit \Exp{1}$ et $x
        \geq 0$)
      \end{array}
      \] %
    \end{noliste}
   \conc{$\forall n\in\N^*$, $\forall x\in \R_+^*$, 
   $\Prob(\Ev{T_n \leq x}) = \left(1-\ee^{-x}\right)^n$}~\\[-1cm]
  \end{proof}
  
\item En déduire que, pour tout $n$ de $\N^*$, $T_n$ est une variable
  aléatoire à densité, admettant pour densité la fonction $f_n$.
  
  \begin{proof}~
   \begin{noliste}{$\sbullet$}
   \item Par définition : $T_n = \max(X_1, \ldots, X_n)$. \\
     Or, pour tout $i\in \llb 1, n \rrb$, $X_i(\Omega)=[0,+\infty[$
     car $X_i \suit \Exp{1}$.%
     \conc{Ainsi, $T_n(\Omega) \subset [0,+\infty[$.}
    
    \item Soit $x\in\R$.
    \begin{noliste}{$\stimes$}
      \item \dashuline{Si $x\leq 0$}, alors $\Ev{T_n \leq x} =
      \varnothing$ car $T_n(\Omega) \subset [0,+\infty[$. Ainsi :
      \[
       F_{T_n}(x)=\Prob(\Ev{T_n \leq x}) = \Prob(\varnothing) = 0
      \]
      
      \item \dashuline{Si $x>0$}. D'après la question \itbf{4.a)} :
      \[
       F_{T_n}(x) = \Prob(\Ev{T_n \leq x}) = \left(1-\ee^{-x}\right)^n
      \]
    \end{noliste}
    \conc{Finalement : $F_{T_n} : x \mapsto \left\{
    \begin{array}{cl}
     \left(1-\ee^{-x}\right)^n & \mbox{ si $x>0$}\\
     0 & \mbox{ si $x\leq 0$}
    \end{array}
    \right.$.}
    
    \item Montrons que $F_{T_n}$ est continue sur $\R$.
    \begin{noliste}{-}
      \item La fonction $F_{T_n}$ est continue sur $]0,+\infty[$ en 
      tant que composée $g_2 \circ g_1$ de :
    \end{noliste}
      \begin{liste}{$\stimes$}
      \item $g_1 : x \mapsto 1-\ee^{-x}$ continue sur $]0,+\infty[$ \\
        et $g_1(]0,+\infty[) \subset \R$,
      \item $g_2 : y \mapsto y^n$ continue sur $\R$.
      \end{liste}
    \begin{noliste}{-}
      \item La fonction $F_{T_n}$ est continue sur $]-\infty,0[$ en 
      tant que fonction constante.
      
      \item On remarque :
      \[
       \dlim{x\to 0^+} F_{T_n}(x)=\left(1-\ee^{-0}\right)^n = 0
       =F_{T_n}(0)=\dlim{x\to 0^-} F_{T_n}(x)
      \]
      Donc $F_{T_n}$ est continue en $0$.
    \end{noliste}
    \conc{La fonction $F_{T_n}$ est continue sur $\R$.}
    
  \item Avec le même raisonnement que précédemment, on montre que
    $F_{T_n}$ est de classe $\Cont{1}$ sur $\R$ sauf éventuellement en
    $0$.%
    \conc{La \var $T_n$ est une variable à densité.}
    
    
    \newpage
    
    
  \item Pour déteminer une densité de $T_n$, on dérive $F_{T_n}$ sur
    des intervalles ouverts.\\
    Soit $x \in \R$.
    \begin{noliste}{$\stimes$}
      \item \dashuline{Si $x\in \ ]-\infty,0[$}.
      \[
       f_{T_n}(x) = F_n'(x)=0=f_n(x)
      \]
      
    \item \dashuline{Si $x\in \ ]0,+\infty[$}.
      \[
       f_{T_n}(x)= F_n'(x) = n \, \ee^{-x} (1-\ee^{-x})^{n-1}
       =f_n(x)
      \]
      
      \item On choisit $f_{T_n}(0)=0=f_n(0)$.
      \end{noliste}
      \conc{La fonction $f_n$ est bien une densité de $T_n$.}~\\[-1.6cm]
   \end{noliste}
 \end{proof}
\end{noliste}
 
\item
  \begin{noliste}{a)}
    \setlength{\itemsep}{2mm}
  \item Montrer que, pour tout $n$ de $\N^*$, la variable aléatoire
    $T_n$ admet une espérance.
    
    \begin{proof}~\\
      Soit $n\in\N^*$.
      \begin{noliste}{$\sbullet$}
      \item La \var $T_n$ admet une espérance si et seulement si 
        l'intégrale $\dint{-\infty}{+\infty} x \, f_n(x) \dx$ converge 
        absolument, ce qui équivaut à démontrer sa convergence pour des 
        calculs de moments.
        
      \item Tout d'abord : 
        \[
        \dint{-\infty}{+\infty} x \, f_n(x) \dx = \dint{0}{+\infty} x
        \, f_n(x) \dx
        \]
        car $f_n$ est nulle en dehors de $[0, +\infty[$.
    
      \item La fonction $x \mapsto x \, f_n(x)$ est $\Cont{0}$ sur
        $[0,+\infty[$.
        % en tant que produit de fonctions continues sur
        % $[0,+\infty[$.
    
        % \item Montrons que : $x \, f_n(x) = \oox{+\infty} \left(
        %     \dfrac{1}{x^2}\right)$.\\
                
      \item 
        \begin{noliste}{$\stimes$}
        \item Tout d'abord : $x \, f_n(x) = \oox{+\infty} \left( \dfrac{1}{x^2}
          \right)$\\[.2cm]
          En effet, pour tout $x>0$ :
          \[
          \dfrac{x \, f_n(x)}{\frac{1}{x^2}} = x^2 \, x \, n \, 
          \ee^{-x}(1-\ee^{-x})^{n-1} = n \, x^3 \, \ee^{-x}(1-\ee^{-x})^{n-1}
          \]
          Or : $\dlim{x\to+\infty} \ee^{-x}=0$. Donc : $\dlim{x\to+\infty}
          (1-\ee^{-x})^{n-1} = 1$.\\[.2cm]
          De plus, par croissances comparées : $\dlim{x\to +\infty} x^3 
          \ee^{-x}=0$.\\
          Finalement : $\dlim{x\to+\infty} \dfrac{x \, f_n(x)}{\frac{1}{x^2}} 
          = 0$, \ie : $x \, f_n(x) = \oox{+\infty} 
          \left(\dfrac{1}{x^2}\right)$
          
        \item $\forall x \in [1,+\infty[$, $x\, f_n(x) \geq 0$ et 
          $\dfrac{1}{x^2} \geq 0$
          
        \item L'intégrale $\dint{1}{+\infty} \dfrac{1}{x^2} \dx$ est une 
          intégrale de Riemann, impropre en $+\infty$, d'exposant $2$ 
          ($2>1$), donc elle converge.
        \end{noliste}
        D'après le critère de négligeabilité des intégrales généralisées de 
        fonctions continues positives, l'intégrale $\dint{1}{+\infty}
        x\, f_n(x) \dx$ converge.
        
      \item De plus, la fonction $x\mapsto x \, f_n(x)$ est continue
        sur le segment $[0,1]$, donc l'intégrale $\dint{0}{1} x\,
        f_n(x) \dx$ est bien définie.
      \end{noliste}
      Finalement, $\dint{0}{+\infty} x\, f_n(x) \dx$ converge pour tout 
      $n\in\N^*$. %
      \conc{Pour tout $n\in\N^*$, la \var $T_n$ admet une 
        espérance.}~\\[-1cm]
    \end{proof}
  
  
  \newpage
  

\item Déterminer l'espérance $\E(T_1)$ de $T_1$ et l'espérance
  $\E(T_2)$ de $T_2$.
  
  \begin{proof}~
    \begin{noliste}{$\sbullet$}
    \item On remarque que : $T_1 = \max(X_1)=X_1$.  \conc{Donc,
        d'après la question \itbf{1.} : $\E(T_1)=1$.}
    
    \item D'après la question \itbf{2.} les intégrales impropres
      $\dint{0}{+\infty} \ee^{-x} \dx$ et $\dint{0}{+\infty} x
      \ee^{-x} \dx$ converge.\\
      On peut donc effectuer le calcul suivant :
      \[
      \begin{array}{rcl@{\quad}>{\it}R{4cm}}
        \E(T_2) & = & \dint{-\infty}{+\infty} x \, f_2(x) \dx \ = \
        \dint{0}{+\infty} 2x \, \ee^{-x} (1-\ee^{-x}) \dx
        \\[.6cm]
        & = & 2 \dint{0}{+\infty} x\, \ee^{-x} \dx - 2 
        \dint{0}{+\infty} x \, \ee^{-2x} \dx
        & (par linéarité de l'intégration)
        \nl
        \nl[-.2cm]
        & = & 2 \times \dfrac{1}{1^2} - 2 \times \dfrac{1}{2^2}
        & (d'après le résultat de la question \itbf{2.})
        \nl
        \nl[-.2cm]
        & = & \dfrac{3}{2}
      \end{array}
      \]
      \conc{$\E(T_2)=\dfrac{3}{2}$}~\\[-1.4cm]
    \end{noliste}
  \end{proof}

 \end{noliste}

 \item 
 \begin{noliste}{a)}
  \setlength{\itemsep}{2mm}
  \item Vérifier : $\forall n\in\N^*$, $\forall x\in\R_+$, 
  $f_{n+1}(x)-f_n(x)=-\dfrac{1}{n+1}f'_{n+1}(x)$.
  
  \begin{proof}~\\
   Soit $n\in\N^*$.
   \begin{noliste}{$\sbullet$}
    \item La fonction $f_n$ est dérivable sur $]0,+\infty[$ en tant 
    que produit de fonctions dérivables sur $]0,+\infty[$.\\
    On obtient alors, pour tout $x>0$ :
    \[
     \begin{array}{rcl}
      f_{n}'(x) & = & -n\ee^{-x}(1-\ee^{-x})^{n-1} + 
      n(n-1)\ee^{-2x}(1-\ee^{-x})^{n-2} 
      \\[.2cm]
      & = & n\ee^{-x} (1-\ee^{-x})^{n-2}
      (-(1-\ee^{-x})+(n-1)\ee^{-x})
      \\[.2cm]
      & = & n\ee^{-x} (1-\ee^{-x})^{n-2} (n\ee^{-x} -1)
     \end{array}
    \]    
    \conc{$\forall x>0, \ f_{n+1}'(x) = (n+1)\ee^{-x}
      (1-\ee^{-x})^{n-1}((n+1)\ee^{-x}-1)$}
    
    \item Soit $x>0$.
    \[
      \begin{array}{rcl}
       f_{n+1}(x)-f_n(x) & = & (n+1)\ee^{-x}(1-\ee^{-x})^n - 
       n\ee^{-x}(1-\ee^{-x})^{n-1}
       \\[.2cm]
       & = & \ee^{-x}(1-\ee^{-x})^{n-1}((n+1)(1-\ee^{-x}) - n)
       \\[.2cm]
       & = & \ee^{-x}(1-\ee^{-x})^{n-1}(1-(n+1)\ee^{-x})
       \\[.2cm]
       & = & -\dfrac{1}{n+1} f_{n+1}'(x)
      \end{array} 
    \]
   \end{noliste}
   \conc{$\forall n\in\N^*$, $\forall x>0$ : $f_{n+1}(x) - f_n(x) =
   -\dfrac{1}{n+1} f_{n+1}'(x)$}
   
 \begin{remark}~\\
   On a ici considéré $x$ dans $\R_+^*$ et non dans $\R_+$. En effet,
   $f_n$ n'est pas toujours dérivable en $0$.\\
   Par exemple, $f_1$ et $f_2$ ne le sont pas.
 \end{remark}~\\[-1.4cm]
\end{proof}
  

\newpage
  

\item Montrer ensuite, à l'aide d'une intégration par parties :
  \[
  \forall n\in\N^*, \ \dint{0}{+\infty} x \, (f_{n+1}(x)-f_n(x)) \dx 
  = \dfrac{1}{n+1} \dint{0}{+\infty} f_{n+1}(x) \dx
  \]
  
  \begin{proof}~\\
    Soit $n\in\N^*$.
    \begin{noliste}{$\sbullet$}
    \item La fonction $x \mapsto x(f_{n+1}(x) - f_n(x))$ est
      $\Cont{0}$ sur $[0, +\infty[$.
    \item Soit $A\geq 0$. D'après la question précédente :
      \[
      \dint{0}{A} x(f_{n+1}(x)-f_n(x))\dx = -\dfrac{1}{n+1}
      \dint{0}{A} x \, f_{n+1}'(x)\dx
      \]
      On effectue alors une intégration par parties (IPP).
      \[
      \renewcommand{\arraystretch}{2.2}
      \begin{array}{|rcl@{\qquad}rcl}
        u(x) & = & x & u'(x) & = & 1\\
        v'(x) & = & f_{n+1}'(x) & v(x) & = & f_{n+1}(x)
      \end{array}
      \]
      Cette IPP est valide car les fonctions $u$ et $v$ sont de classe
      $\Cont{1}$ sur $[0,A]$. On obtient :
      \[
      \begin{array}{rcl}
        -\dfrac{1}{n+1} \dint{0}{A} x \, f_{n+1}'(x)\dx
        & = & -\dfrac{1}{n+1} \left( \Prim{x \, f_{n+1}(x)}{0}{A} - 
          \dint{0}{A} f_{n+1}(x) \dx \right)
        \\[.6cm]
        & = & -\dfrac{1}{n+1} \, A \, f_{n+1}(A) + \dfrac{1}{n+1}
        \dint{0}{A} f_{n+1}(x) \dx
      \end{array}
      \]
    \item De plus :
      \begin{noliste}{$\stimes$}
      \item la fonction $f_{n+1}$ est une densité de probabilité nulle
        en dehors de $[0, +\infty[$.\\[.2cm]
        Ainsi, l'intégrale impropre $\dint{-\infty}{+\infty}
        f_{n+1}(x) \dx = \dint{0}{+\infty} f_{n+1}(x) \dx$ converge
        (et vaut $1$).\\[.2cm]
        D'où : $\dint{0}{A} f_{n+1}(x)\dx \ \tendd{A}{+\infty} \
        \dint{0}{+\infty} f_{n+1}(x) \dx$.
      
      \item comme $A\geq 0$ :
        \[
        A \, f_{n+1}(A) = A \times n \ee^{-A}(1-\ee^{-A})^n = n \times
        \dfrac{A}{\ee^{A}} \times (1-\ee^{-A})^n \ \tendd{A}{+\infty}
        \ 0 \quad \mbox{\it (par croissances comparées)}
        \]
      \end{noliste}
      On en déduit que l'intégrale $\dint{0}{+\infty} x \,
      (f_{n+1}(x)-f_n(x)) \dx$ converge.
    \end{noliste}
    \conc{Ainsi : $\forall n \in\N^*$, 
    $\dint{0}{+\infty} x \, (f_{n+1}(x)-f_n(x)) \dx = 
    \dfrac{1}{n+1} \dint{0}{+\infty} f_{n+1}(x) \dx$.}~\\[-1cm]
  \end{proof}

\item En déduire, pour tout $n$ de $\N^*$, une relation entre
  $\E(T_{n+1})$ et $\E(T_n)$, puis une expression de $\E(T_n)$ sous
  forme d'une somme.
  
  \begin{proof}~\\
    Soit $n\in\N^*$.
    \begin{noliste}{$\sbullet$}
    \item D'après la question \itbf{5.a)}, les \var $T_{n+1}$ et $T_n$
      admettent une espérance.\\[.2cm]
      On en conclut que les intégrales impropres
      $\dint{-\infty}{+\infty} x \, f_{n+1}(x) \dx$ et
      $\dint{-\infty}{+\infty} x \, f_n(x) \dx$ sont (absolument)
      convergentes.

    
    \newpage
        
    
    \item On calcule alors :
    \[
     \begin{array}{rcl@{\quad}>{\it}R{5cm}}
       \E(T_{n+1}) - \E(T_n) & = & \dint{-\infty}{+\infty} x\, 
       f_{n+1}(x) \dx - \dint{-\infty}{+\infty} x\, f_n(x) \dx
       \\[.6cm]
       & = & \dint{0}{+\infty} x \, f_{n+1}(x) \dx - \dint{0}{+\infty}
       x\, f_n(x) \dx & (car $f_n$ et $f_{n+1}$ sont nulles \\
       en dehors de $[0, +\infty[$)
       \nl
       \nl[-.2cm]
       & = & \dint{0}{+\infty} x\, (f_{n+1}(x)-f_n(x)) \dx
       \\[.6cm]
       & = & \dfrac{1}{n+1} \dint{0}{+\infty} f_{n+1}(x) \dx 
       & (d'après la question \itbf{6.b)})
       \nl
       \nl[-.2cm]
       & = & \dfrac{1}{n+1} \dint{-\infty}{+\infty} f_{n+1}(x) \dx
       & (car $f_{n+1}$ est nulle \\
       en dehors de $[0, +\infty[$)
       \nl
       \nl[-.2cm]
       & = & \dfrac{1}{n+1} \times 1 & (car $f_{n+1}$ est une densité)
     \end{array}
    \]
    \conc{$\forall n\in \N^*$, $\E(T_{n+1})-\E(T_n) = \dfrac{1}{n+1}$}
    
  \item On a donc : $\forall k\in \N^*$, $\E(T_{k+1}) - \E(T_k) =
    \dfrac{1}{k+1}$.\\[.2cm]
    En sommant ces égalités pour $k$ variant de $1$ à $n-1$, on
    obtient :
    \[
     \Sum{k=1}{n-1} \left(\E(T_{k+1})-\E(T_k)\right) = 
     \Sum{k=1}{n-1} \dfrac{1}{k+1} = \Sum{k=2}{n} \dfrac{1}{k}
    \]
    Or, oar télescopage :
    \[
     \E(T_n)-\E(T_1) = \Sum{k=2}{n} \dfrac{1}{k}
    \]
    Or, d'après la question \itbf{}, $\E(T_1)=1$.
    \conc{On en déduit que : $\forall n\in \N^*$, 
    $\E(T_n)=1+ \Sum{k=2}{n} \dfrac{1}{k} = 
    \Sum{k=1}{n} \dfrac{1}{k}$}~\\[-1.4cm]
   \end{noliste}
  \end{proof}
 \end{noliste}
\end{noliste}
 
 
\subsection*{Partie III : Loi du premier dépassement}

\noindent
Dans toute cette partie, $a$ désigne un réel strictement positif.\\
On définit la variable aléatoire $N$ égale au plus petit entier $n$ de 
$\N^*$ tel que $X_n>a$ si un tel entier existe, et égale à 0 
sinon.~\\[-.6cm]

\begin{noliste}{1.}
 \setlength{\itemsep}{4mm}
 \setcounter{enumi}{6}
 \item Justifier l'égalité d'événements : $\Ev{N=0} = 
 \dcap{k=1}{+\infty} \Ev{X_k\leq a}$. En déduire la probabilité 
 $\Prob(\Ev{N=0})$.
 
 \begin{proof}~
  \begin{noliste}{$\sbullet$}
   \item Soit $\omega \in \Omega$.
   $N(\omega)=0$ si et seulement s'il n'existe pas d'entier $n\in\N^*$
   tel que $X_n(\omega)>a$.\\
   Autrement dit, si la proposition suivante est vérifiée :
   \[
    \texttt{NON}\left( \exists n\in \N^*, \ X_n(\omega)>a\right)
   \]
   Ce qui équivaut à : $\forall n\in \N^*$, $X_n(\omega) \leq a$.
   Puis à : $\omega \in \dcap{n=1}{+\infty} \Ev{X_n \leq a}$.
   \conc{$\Ev{N=0}= \dcap{k=1}{+\infty} \Ev{X_k \leq a}$}
  

  \newpage

  
  \begin{remark}~\\
    On aurait sans doute obtenu tous les points de cette question sans
    l'introduction propre de $\omega$.\\
    En effet, l'énoncé prend le parti de ne pas le faire lors de la
    définition de la \var $N$.\\
    Cela se fait cependant au prix d'une confusion d'objets entre
    variables aléatoires / réalisations / événements. Détaillons la
    démonstration qui semble plus proche de l'esprit du concepteur.\\[.2cm]
    $N=0$ si et seulement s'il n'existe pas d'entier $n\in\N^*$
    tel que $X_n>a$.\\
    Autrement dit, si : $\NON{ \exists n\in \N^*, \ X_n>a}$.\\
    Ce qui équivaut à : $\forall n\in \N^*$, $X_n \leq a$.\\
    Finalement, on a : $\Ev{N=0}= \dcap{k=1}{+\infty} \Ev{X_k \leq a}$
  \end{remark}
  
\item Ainsi :
   \[
    \Prob(\Ev{N=0}) = \Prob\left(\dcap{k=1}{+\infty} \Ev{X_k\leq 
    a}\right)
   \]
   D'après le théorème de la limite monotone :
   \[
    \Prob\left(\dcap{k=1}{+\infty} \Ev{X_k \leq a}\right) = 
    \dlim{n\to+\infty} \Prob\left(\dcap{k=1}{n} \Ev{X_k \leq a}\right)
   \]

 \item Soit $n\in \N^*$.
   \[
    \begin{array}{rcl@{\quad}>{\it}R{5cm}}
     \Prob\left(\dcap{k=1}{n} \Ev{X_k \leq a}\right) & = & 
     \Prod{k=1}{n} \Prob(\Ev{X_k \leq a}) 
     & (car les \var $X_1$, $\cdots$, $X_n$ \\ sont indépendantes)
     \nl
     \nl[-.2cm]
     & = & \Prod{k=1}{n} \Prob(\Ev{X_1 \leq a}) &
     (car les \var $X_1$, $\cdots$, $X_n$ \\ ont même loi)
     \nl
     \nl[-.2cm]
     & = & \left(\Prob(\Ev{X_1 \leq a})\right)^n
     \ = \ \left(F_{X_1}(a)\right)^n
     \\[.4cm]
     & = & \left(1-\ee^{-a}\right)^n & (car $X_1 \suit \Exp{1}$ et $a
     \geq 0$)
    \end{array}
   \]
   Or : $0 < 1-\ee^{-a} <1$. En effet :
   \[
   \begin{array}{rcl@{\quad}>{\it}R{5cm}}
     +\infty > a > 0 & \Leftrightarrow & -\infty < -a < 0 
     \\[.2cm]
     & \Leftrightarrow & 0 < \ee^{-a} <1 & (par stricte croissance \\ de
     $\exp$ sur $\R$)
     \nl
     \nl[-.2cm]
     & \Leftrightarrow & -1 < -\ee^{-a} <0
   \end{array}
   \]
   On en déduit : $\dlim{n\to +\infty} \left(1-\ee^{-a}\right)^n =0$.\\[.2cm]
   Et : $\Prob\left(\dcap{k=1}{+\infty} \Ev{X_k\leq a}\right) =0$.%
   \conc{$\Prob(\Ev{N=0})=0$.}~\\[-1.4cm]
  \end{noliste}
 \end{proof}

 \item Montrer : $\forall n\in\N^*$, $\Prob(\Ev{N=n}) = 
 (1-\ee^{-a})^{n-1} \, \ee^{-a}$.
 
 \begin{proof}~\\
  Soit $n\in\N^*$.
  \begin{noliste}{$\sbullet$}
  \item Par définition, l'événement $\Ev{N=n}$ est réalisé si et
    seulement si $n$ est le plus petit entier tel que $\Ev{X_n > a}$
    est réalisé. Autrement dit, si on a à la fois :
    \begin{noliste}{$\stimes$}
    \item pour tout $k \in \llb 1, n-1\rrb$, l'événement $\Ev{X_k \leq
        a}$ est réalisé,
    \item l'événement $\Ev{X_n >a}$ est réalisé.
    \end{noliste}
    \conc{Ainsi : $\Ev{N=n} = \Ev{X_1 \leq a} \cap \cdots \cap
      \Ev{X_{n-1} \leq a} \cap \Ev{X_n >a} = \left( \dcap{k=1}{n-1}
        \Ev{X_k \leq a}\right) \cap \Ev{X_n >a}$}
  
  
    \newpage

    
    \noindent
    On en déduit :
    \[
    \begin{array}{rcl@{\quad}>{\it}R{5cm}}
      \Prob(\Ev{N=n}) & = & \Prob\left(\left( \dcap{k=1}{n-1} \Ev{X_k \leq 
            a}\right) \cap \Ev{X_n >a}\right)
      \\[.6cm]
      & = & \left(\Prod{k=1}{n-1} \Prob(\Ev{X_k \leq a})\right) \times
      \Prob(\Ev{X_n >a})
      & (car les \var $X_1$, $\cdots$, $X_n$ \\ sont indépendantes)
      \nl
      \nl[-.2cm]
      & = & \left(\Prob(\Ev{X_1\leq a})\right)^n \times 
      \Prob(\Ev{X_1 >a}) & (car les \var $X_1$, $\cdots$, $X_n$ \\ ont même loi)
      \nl
      \nl[-.2cm]
      & = & \left(F_{X_1}(a)\right)^{n-1} \times (1-F_{X_1}(a))
      \\[.4cm]
      & = & \left(1-\ee^{-a}\right)^{n-1} \times \ee^{-a}
      & (car $X_1 \suit \Exp{1}$ et $a \geq 0$)
    \end{array}
    \]
  \end{noliste}
  \conc{$\forall n\in\N^*$, \ $\Prob(\Ev{N=n}) =
    \left(1-\ee^{-a}\right)^{n-1} \, \ee^{-a}$}~\\[-1cm]
\end{proof}
 
\item Déterminer l'espérance $\E(N)$ et la variance $\V(N)$ de $N$.
 
 \begin{proof}~
  \begin{noliste}{$\sbullet$}
  \item D'après l'énoncé : $N(\Omega) = \N$.
  \item De plus, on a démontré :
    \begin{noliste}{$\stimes$}
    \item $\Prob(\Ev{N = 0}) = 0$.
    \item $\forall n \in\N^*$, \ $\Prob(\Ev{N=n}) = \left(1-\ee^{-a}
      \right)^{n-1} \, \ee^{-a}$.
    \end{noliste}
    \conc{On en déduit : $N \suit \G{\ee^{-a}}$.}
    
  \item Ainsi $N$ admet une espérance et une variance. %
    \conc{De plus : $\E(N) = \dfrac{1}{\ee^{-a}}=\ee^a$ \quad et \quad
      $\V(N) = \dfrac{1-\ee^{-a}}{(\ee^{-a})^2} = \ee^{2a}(1-\ee^{-a})
      = \ee^a \, (\ee^a-1)$.}
  \end{noliste}
  \begin{remark}~
    \begin{noliste}{$\sbullet$}
    \item Dans le cours, il est précisé qu'une \var $X$ qui suit une
      loi géométrique (de paramètre $\ee^{-a}$) admet pour support
      $\N^*$. Ici, $N$ a pour support $\N$ mais prend la valeur $0$
      avec probabilité nulle.\\
      On a alors :
      \[
      \forall x \in \R, \ \Prob(\Ev{X = x}) = \Prob(\Ev{N = x})
      \]
      {\it (et ces probabilités sont nulles si $x \notin \N^*$)}\\
      Dans ce cas, on considère que $X$ et $N$ ont même loi (qui est
      $\G{\ee^{-a}}$).

    \item Évidemment, il est tout à fait possible d'effectuer un
      calcul direct avec les probabilités calculées en questions
      \itbf{7.} et \itbf{8.}. Cependant, cela démontre une manque de
      prise de recul et finit par coûter des points car demande
      beaucoup plus de temps.
    \end{noliste}
  \end{remark}~\\[-1.4cm]
 \end{proof}
\noindent 
 On s'intéresse maintenant à la variable aléatoire $Z$, définie pour 
 tout $\omega$ de $\Omega$ par :\\ 
 \[
  Z(\omega)=\left\{
  \begin{array}{cl}
   X_{N(\omega)}(\omega) & \text{si } N(\omega)\neq 0\\[.2cm]
   0 & \text{si } N(\omega)=0
  \end{array}
  \right.
 \]


\newpage

 
\item Justifier : $\Prob(\Ev{Z\leq a})=0$.
 
  \begin{proof}~
    \begin{noliste}{$\sbullet$}
    \item La famille $(\Ev{N=0}, \Ev{N\neq 0})$ est un système complet
      d'événements.\\
      D'après la formule des probabilités totales :
      \[
      % \begin{array}{rcl@{\quad}>{\it}R{4cm}}
      \Prob(\Ev{Z\leq a}) \ = \ \Prob(\Ev{Z\leq a} \cap \Ev{N=0}) +
      \Prob(\Ev{Z \leq a} \cap \Ev{N \neq 0})
      % \\[.2cm]
      % & = & \Prob(\Ev{0\leq a} \cap \Ev{N=0}) + 
      % \Prob(\Ev{X_N \leq a} \cap \Ev{N \neq 0})
      % & (par définition de $Z$)
      % \end{array}
      \]

    \item Or, par définition de $Z$ :
      \[
      \Ev{Z\leq a} \cap \Ev{N=0} = \Ev{0\leq a} \cap \Ev{N=0} = \Omega
      \cap \Ev{N=0} = \Omega \cap \Ev{N=0} = \Ev{N = 0}
      \]
      (on rappelle que d'après l'énoncé : $a > 0 \geq 0$)

    \item D'autre part, par définition de $N$ :
      \[
      \Ev{Z \leq a} \cap \Ev{N \neq 0} = \emptyset
      \]
      Démontrons-le en supposant par l'absurde qu'il existe $\omega$
      qui réalise $\Ev{Z \leq a} \cap \Ev{N \neq 0}$.\\
      Cela signifie que $N(\omega) \neq 0$ et $Z(\omega) \leq a$.\\
      Comme $N(\omega) \neq 0$ alors $Z(\omega) =
      X_{N(\omega)}(\omega)$. On en déduit :
      \[
      Z(\omega) = X_{N(\omega)}(\omega) \leq a
      \]
      Or, comme $N(\omega) \neq 0$, alors $N(\omega)$ est par
      définition le plus petit entier $n \in \N^*$ tel que
      $X_{n}(\omega) > a$.\\
      On a alors : $X_{N(\omega)}(\omega) > a$ ce qui contredit
      $X_{N(\omega)}(\omega) \leq a$.
      
    \item On revient à la première égalité :
      \[
      \begin{array}{rcl@{\quad}>{\it}R{4cm}}
        \Prob(\Ev{Z\leq a}) 
        & = & \Prob(\Ev{N=0}) + \Prob(\varnothing)
        \\[.2cm]
        & = & 0 + 0 & (d'après la question \itbf{7.}) 
      \end{array}
      \]
    \end{noliste}
    \conc{Finalement, on a bien : $\Prob(\Ev{Z \leq a})=0$}
    \begin{remark}~\\
      L'utilisation de la formule des probabilités totales devrait
      relever ici de l'automatisme. En effet, on traite d'une \var qui
      est définie par cas. Pour le calcul de $\Prob(\Ev{Z \leq a})$,
      on est donc naturellement amené à vouloir traiter à part le cas
      où $N = 0$ et celui où $N \neq 0$. La formule des probabilités
      totales n'est autre qu'une formalisation correcte de cette idée.
    \end{remark}~\\[-1.2cm]
 \end{proof}
 
 \item Soit $x\in \ ]a,+\infty[$.
 \begin{noliste}{a)}
  \item Soit $n\in\N^*$, justifier l'égalité d'événements : 
  \[
   \Ev{N=n}\cap \Ev{Z\leq x}=\left\{
   \begin{array}{cl}
    \Ev{a<X_1\leq x} & \text{si } n=1\\[.2cm]
    \Ev{T_{n-1}\leq a}\cap \Ev{a<X_n\leq x} & \text{si } n\geq 2
   \end{array}
   \right.
  \]
  En déduire la probabilité $\Prob(\Ev{N=n}\cap \Ev{Z\leq x})$.
  
  \begin{proof}~\\
    Deux cas se présentent.
    \begin{noliste}{$\sbullet$}
    \item \dashuline{Si $n=1$}.
      \[
       \begin{array}{rcl@{\quad}>{\it}R{4cm}}
         \Ev{N=1} \cap \Ev{Z \leq x} & = & \Ev{N=1} \cap \Ev{X_N \leq
           x} 
         & (par définition de $Z$)
         \nl
         \nl[-.2cm]
         & = & \Ev{N=1} \cap \Ev{X_1 \leq x}
         & (par définition de $Z$)
         \nl
         \nl[-.2cm]
         & = & \Ev{X_1 >a} \cap \Ev{X_1 \leq x}
         & (par définition de $N$)
         \nl
         \nl[-.2cm]
         & = & \Ev{a< X_1 \leq x}
       \end{array}
      \]
      \conc{On a bien : $\Ev{N=1}\cap \Ev{Z\leq x} = \Ev{a<X_1\leq x}$.}


      \newpage


      \noindent
      On en déduit alors : 
      \[
       \begin{array}{rcl}
        \Prob(\Ev{N=1}\cap \Ev{Z\leq x}) & = & \Prob(\Ev{a < X_1 \leq x})
        \ = \ F_{X_1}(x)-F_{X_1}(a) 
        \\[.2cm]
        & = & (\bcancel{1}-\ee^{-x}) 
        - (\bcancel{1}-\ee^{-a}) \ = \ \ee^{-a} - \ee^{-x}
%         \\[.2cm]
%         & = & \ee^{-a} \ (1-\ee^{a-x})
       \end{array}
      \]
      \conc{$\Prob(\Ev{N=1}\cap \Ev{Z\leq x}) = \ee^{-a} - \ee^{-x}$}
      
    \item \dashuline{Si $n\geq 2$}.
      \[
       \begin{array}{rcl@{\quad}>{\it}R{4cm}}
         \Ev{N=n} \cap \Ev{Z \leq x} & = & \Ev{N=n} \cap \Ev{X_N \leq x}
         & (par définition de $Z$)
         \nl
         \nl[-.2cm]
         & = & \Ev{N=n} \cap \Ev{X_n \leq x}
         & (par définition de $Z$)
         \nl
         \nl[-.2cm]
         & = & \left(\dcap{k=1}{n-1} \Ev{X_k \leq a}\right) \cap 
         \Ev{X_n >a} \ \cap \ \Ev{X_n \leq x}
         & (d'après la question \itbf{8.})
         \nl
         \nl[-.2cm]
         & = & \left(\dcap{k=1}{n-1} \Ev{X_k \leq a}\right) \cap
         \Ev{a< X_n \leq x}
         \\[.6cm]
         & = & \Ev{\max(X_1, \ldots, X_n) \leq a} \cap 
         \Ev{a < X_n \leq x}
         \\[.4cm]
         & = & \Ev{T_{n-1} \leq a} \cap \Ev{a < X_n \leq x}
         & (par définition de $T_{n-1}$)
       \end{array}
      \]
      \conc{Si $n \geq 2$, on a bien : $\Ev{N=n} \cap \Ev{Z\leq x} =
        \Ev{T_{n-1}\leq a}\cap \Ev{a<X_n\leq x}$.}%
      On en déduit alors :
      \[
      \begin{array}{rcl@{\quad}>{\it}R{6cm}}
        \Prob(\Ev{N=n} \cap \Ev{T \leq x}) & = & 
        \Prob(\Ev{T_{n-1} \leq a} \cap \Ev{a< X_n \leq x})
        \\[.2cm]
        & = & \Prob(\Ev{T_{n-1} \leq a}) \, \Prob(\Ev{a < X_n \leq x})
        & (car $T_{n-1}$ et $X_n$ sont indépendantes d'après le lemme
        des coalitions)
        \nl
        \nl[-.2cm]
        & = & \left(1-\ee^{-a}\right)^{n-1} \, \Prob(\Ev{a < X_n \leq x})
        & (d'après la question \itbf{4.a)})
      \end{array}
      \]
      De plus $X_n$ suit la même loi que $X_1$, donc :
      \[
      \Prob(\Ev{a< X_n \leq x}) = \Prob(\Ev{a< X_1 \leq x}) =
      \ee^{-a} - \ee^{-x} % \ee^{-a}(1 - \ee^{a-x})
      \]
      \conc{Si $n \geq 2$, $\Prob(\Ev{N=n}\cap \Ev{Z\leq x}) =
        \left(1-\ee^{-a}\right)^{n-1} \ (\ee^{-a} - \ee^{-x})$.}
      %\ee^{-a}(1 - \ee^{a-x})$.}
      % \end{noliste}
   \end{noliste}
   On remarque que l'expression trouvée dans le cas $n\geq 2$ est
   valide pour $n=1$.%
   \conc{On en déduit : $\forall n\in \N^*$, $\Prob(\Ev{N=n} \cap
     \Ev{Z \leq x}) = (\ee^{-a} - \ee^{-x}) \
     \left(1-\ee^{-a}\right)^{n-1}$}%~\\[-1cm]
   \begin{remark}~\\
     On fait remarquer que la formule obtenue est valide dans le cas
     $n =1$ et $n \geq 2$.\\
     Ce n'est pas un objectif annoncé de la question. L'avantage est
     que cela rend la question suivante plus simple à rédiger : on
     n'est pas obligé de distinguer les cas $n =1$ et $n \geq 2$
     puisque l'expression est valide dans ces deux cas.
   \end{remark}~\\[-1cm]
  \end{proof}


  \newpage
 

  \item Montrer alors : $\Prob(\Ev{Z\leq x})=1-\ee^{a-x}$.
  
    \begin{proof}~\\
      La famille $\big( \Ev{N=n} \big)_{n\geq 1}$ est un système
      complet d'événements.\\
      Ainsi, d'après la formule des probabilités totales :
      \[
      \begin{array}{rcl@{\quad}>{\it}R{5cm}}
        \Prob(\Ev{Z\leq x}) & = & \Sum{n=1}{+\infty} 
        \Prob(\Ev{N=n} \cap \Ev{Z\leq x})
        \\[.4cm]
        & = & \Sum{n=1}{+\infty} (\ee^{-a} - \ee^{-x}) \ 
        \left(1-\ee^{-a}\right)^{n-1}
        & (d'après la question \itbf{11.a)})
        \nl
        \nl[-.2cm]
        & = & (\ee^{-a} - \ee^{-x}) \ \Sum{n=1}{+\infty} (1-\ee^{-a})^{n-1}
        \\[.4cm]
        & = & (\ee^{-a} - \ee^{-x}) \ \Sum{n=0}{+\infty} (1-\ee^{-a})^n
        & (par décalage d'indice)
        \nl
        \nl[-.2cm]
        & = & (\ee^{-a} - \ee^{-x}) \ \dfrac{1}{\bcancel{1} - (
          \bcancel{1} - \ee^{-a})}
        \\[.6cm]
        & = & \bcancel{\ee^{-a}} \ (1-\ee^{a-x}) \ \dfrac{1}{\bcancel{\ee^{-a}}}
        \\[.6cm]
        & = & 1 - \ee^{a-x}
      \end{array}
      \]
      \conc{Ainsi : $\Prob(\Ev{Z\leq x}) = 1-\ee^{a-x}$}~\\[-1cm]
    \end{proof}
 \end{noliste}
  
 \item 
 \begin{noliste}{a)}
  \item Montrer que la variable aléatoire $Z-a$ suit une loi 
  exponentielle dont on précisera le paramètre.
  
  \begin{proof}~
   \begin{noliste}{$\sbullet$}
    \item Soit $x\in \R$.
    \begin{noliste}{$\stimes$}
      \item \dashuline{si $x< 0$}, alors : 
      \[
       \Ev{Z-a \leq x} = \Ev{Z \leq x+a} \subset \Ev{Z \leq a}
      \]
      Donc, par croissance de $\Prob$ et d'après la question
      \itbf{10.} :
      \[
       0 \leq \Prob(\Ev{Z-a\leq x}) \leq \Prob(\Ev{Z\leq a}) = 0
      \]
      D'où : $F_{Z-a}(x) = \Prob(\Ev{Z-a\leq x}) = 0$.
      
      \item \dashuline{si $x \geq 0$} :
      \[
       \begin{array}{rcl@{\quad}>{\it}R{5cm}}
        F_{Z-a}(x) & = & \Prob(\Ev{Z-a \leq x}) \ = \ 
        \Prob(\Ev{Z \leq x+a})
        \\[.2cm]
        & = & 1-\ee^{\bcancel{a}-(x+\bcancel{a})} & 
        (d'après la question \itbf{11.b)}, car $x+a \geq a$)
        \nl
        \nl[-.2cm]
        & = & 1-\ee^{-x}
       \end{array}
      \]
    \end{noliste}
    Finalement : 
    \[
     \forall x\in \R, \ F_{Z-a}(x) = \left\{
     \begin{array}{cl}
      1-\ee^{-x} & \mbox{ si $x\geq 0$}\\
      0 & \mbox{ si $x<0$}
     \end{array}
     \right.
    \]
    On reconnaît la fonction de répartition de la loi
    exponentielle de paramètre $1$.\\
    Or, la fonction de répartition caractérise la loi.
    \conc{Donc la \var $Z-a$ suit une loi exponentielle de 
    paramètre $1$.}
   \end{noliste}
   
   \begin{remark}~
    \begin{noliste}{$\sbullet$}
     \item On reconnaît ici une question du type \og déterminer la 
     transformée affine d'une \var $Z$ \fg{}.\\
     Ce type de question est à savoir faire sans hésitation.
     
   \item L'énoncé nous guide ici dans la disjonction de cas à
     considérer : il précise que $Z-a$ suit une loi exponentielle. Or
     le support d'une \var suivant une loi exponentielle est
     $[0,+\infty[$.\\
     La disjonction de cas attendue pour déterminer la fonction de
     répartition de $Z-a$ est donc :
     \begin{noliste}{$\stimes$}
      \item le cas $x \geq 0$,
      \item le cas $x<0$.
     \end{noliste}
    \end{noliste}
   \end{remark}~\\[-1.4cm]
  \end{proof}

  
  \item En déduire l'existence et la valeur de $\E(Z)$, ainsi que 
  l'existence et la valeur de $\V(Z)$.
  
  \begin{proof}~
   \begin{noliste}{$\sbullet$}
    \item On remarque que : $Z = (Z-a) +a$.
    \conc{La \var $Z$ admet une espérance et une variance en tant que\\ 
    somme de \var qui en admettent.}
    
    \item Par linéarité de l'espérance :
    \[
     \E(Z) = \E(Z-a) + \E(a) = \dfrac{1}{1} + a = 1+a
    \]
    \item Par propriété de la variance :
    \[
     \V(Z) = \V(Z-a)=\dfrac{1}{1^2} = 1
    \]
   \end{noliste}
   \conc{Finalement : $\E(Z)=1+a$ et $\V(Z)=1$.}~\\[-1.2cm]
  \end{proof}

 \end{noliste}
\end{noliste}

\section*{Exercice 2}

\noindent
Dans cet exercice on pourra utiliser l'encadrement suivant : $2 < \ee
< 3$.

\subsection*{Partie I : Etude d'une fonction}

\noindent
On considère l'application $\varphi : \R \rightarrow \R, x \mapsto
\varphi(x) = x^2 \, \ee^x - 1$.
\begin{enumerate}[\bfseries 1.]
\item Dresser le tableau de variations de $\varphi$, en précisant la
  limite de $\varphi$ en $-\infty$, sa valeur en 0 et sa limite en
  $+\infty$.

  \begin{proof}~
    \begin{noliste}{$\sbullet$}
    \item La fonction $\varphi$ est $C^\infty$ sur $\R$ comme somme et
      produit de fonctions $C^\infty$ sur $\R$.

    \item Soit $x \in \R$. 
      \[
      \varphi'(x) = 2x \, \ee^x + x^2 \, \ee^x = x \, (2 + x) \, \ee^x
      \]
      Comme $\ee^x > 0$, $\varphi'(x)$ est du signe de $x \, (2+x)$
      {\it (polynôme de degré $2$)}.

    \item On obtient le tableau de variations suivant.\\
      \begin{center}
        \begin{tikzpicture}[scale=.8, transform shape] %
          \tkzTabInit[lgt=4,espcl=4] %
          {$x$ /1, %
            Signe de $f'(x)$ /1, %
            Variations de $f$ /3} %
          { $-\infty$, $-2$, $0$, $+\infty$ } %
          \tkzTabLine{ , + , z, - , z, + , }%
          \tkzTabVar{-/$-1$, +/$4 \, \ee^{-2} - 1$, -/$-1$, +/$+\infty$}%
        \end{tikzpicture}
      \end{center}

    \item Détaillons les différents éléments de ce tableau.

      \begin{noliste}{$\stimes$}
      \item En posant $X = -x$, on obtient :
        \[
        \dlim{x \tend -\infty} x^2 \, \ee^x - 1 = \dlim{X \tend
          +\infty} (-X)^2 \, \ee^{-X} - 1 = \dlim{X \tend +\infty}
        \dfrac{X^2}{\ee^{X}} - 1 = -1
        \]
        par croissances comparées.
      \item $\varphi(0) = -1$
      \item $\dlim{x \tend +\infty} x^2 \, \ee^x - 1 = +\infty$ \\[-1cm]
      \end{noliste}
    \end{noliste}
  \end{proof}

\item Établir que l'équation $\ee^x = \dfrac{1}{x^2}$, d'inconnue $x
  \in \ ]0;+\infty[$, admet une solution et une seule, notée $\alpha$,
  et que $\alpha$ appartient à l'intervalle $\left] \dfrac{1}{2}, 1
  \right[$.

  \begin{proof}~
    \begin{noliste}{$\sbullet$}
    \item Soit $x > 0$. Remarquons que :
      \[
      \ee^x = \dfrac{1}{x^2} \ \Leftrightarrow \ x^2 \, \ee^x = 1 \
      \Leftrightarrow \ x^2 \, \ee^x - 1 = 0 \ \Leftrightarrow \
      \varphi(x) = 0
      \]

    \item La fonction $\varphi$ est :
      \begin{noliste}{$\stimes$}
      \item continue sur l'intervalle $]0,+\infty[$,
      \item strictement croissante sur $]0,+\infty[$.
      \end{noliste}
      Ainsi, $\varphi$ réalise une bijection de $]0,+\infty[$ dans
      $\varphi(]0,+\infty[) \ = \ ]\varphi(0), \dlim{x \tend +\infty}
      \varphi(x)[ \ = \ ]-1, +\infty[$.\\[.2cm]
      Comme $0 \in \ ]-1, +\infty[$, on en déduit que $0$ admet un
      unique antécédent $\alpha \in \ ]0, +\infty[$ par la fonction
      $\varphi$ {\it (ou encore que l'équation $\varphi(x) = 0$ admet
        une unique solution $\alpha \in \ ]0, +\infty[$)}.%
      \conc{L'équation $\ee^x = \dfrac{1}{x^2}$ admet une unique
        solution $\alpha \in \ ]0, +\infty[$.}


      \newpage


    \item De plus :
      \begin{noliste}{$\stimes$}
      \item $\varphi(\frac{1}{2}) = \dfrac{1}{4} \, \ee^{\frac{1}{2}}
        - 1 < \dfrac{1}{4} \, \ee - 1 < \dfrac{3}{4} - 1 =
        -\dfrac{1}{4} < 0$.
      \item $\varphi(\alpha) = 0$.
      \item $\varphi(1) = \ee^1-1 > 0$ car $\ee > 2$.        
      \end{noliste}
      Ainsi : $\varphi(\frac{1}{2}) < 0 < \varphi(1)$.\\
      Or, d'après le théorème de la bijection, $f^{-1} : \ ]0, +\infty[
      \ \to \ ]0, +\infty[$ est strictement croissante.%
      \conc{En appliquant $f^{-1}$ à l'inégalité précédente, on
        obtient : $\dfrac{1}{2} < \alpha < 1$.}~\\[-1.3cm]
    \end{noliste}
  \end{proof}
\end{enumerate}

\subsection*{Partie II : Étude d'une suite}

\noindent
On considère l'application $f:\R\rightarrow \R, x\mapsto
f(x) = x^3 \, \ee^x$,\\
et la suite réelle $(u_n)_{n\in\N}$ définie par : $u_0=1$ et $\forall
n\in\N$, $u_{n+1}=f(u_n)$.

\begin{enumerate}[\bfseries 1.]
  \setcounter{enumi}{2}
\item Montrer : $\forall n\in\N$, $u_n\geq 1$.

  \begin{proof}~\\
    Démontrons par récurrence que : $\forall n \in \N, \ \PP{n}$ \quad
    où \quad $\PP{n}$ : $u_n \geq 1$.
    \begin{noliste}{1)}
    \item {\bf Initialisation} :\\
      $u_0 = 1 \geq 1$ donc $\PP{0}$ est vérifiée.

    \item {\bf Hérédité} : soit $n \in \N$.\\
      Supposons $\PP{n}$ et démontrons $\PP{n+1}$ (\ie $u_{n+1} \geq
      1$).\\
      Par hypothèse de récurrence : $u_n \geq 1$.\\
      Par croissance des fonctions élévation au cube et exponentielle
      : $u_n^3 \geq 1$ et $\ee^{u_n} \geq \ee^{1} \geq 1$.\\
      On en déduit que : $u_{n+1} = u_n^3 \, \ee^{u_n} \geq 1$.\\
      D'où $\PP{n+1}$.
    \end{noliste}
    \conc{Par principe de récurrence : $\forall n \in \N, \ u_n \geq
      1$.}~\\[-1cm]
  \end{proof}

\item Établir que la suite $(u_n)_{n\in\N}$ est croissante.

  \begin{proof}~\\
    Soit $n \in \N$.
    \begin{noliste}{$\sbullet$}
    \item Tout d'abord : % On raisonnne par équivalence :
      \[
      u_{n+1} - u_n = u_n^3 \, \ee^{u_n} - u_n = (u_n^2 \, \ee^{u_n} -
      1) \ u_n 
      \]

    \item Or, comme $u_n \geq 1$ (d'après la question précédente),
      $u_n^2 \geq 1$ et $\ee^{u_n} \geq \ee^1 \geq 1$.\\
      Ainsi : $u_n^2 \, \ee^{u_n} \geq 1$ et donc $u_n^2 \, \ee^{u_n}
      - 1 \geq 0$. 
    \end{noliste}
    \conc{Comme $u_n \geq 1$, on en déduit que : $\forall n \in \N, \
      u_{n+1} \geq u_n$. La suite $(u_n)$ est donc croissante.}
    \begin{remark}~
      \begin{noliste}{$\sbullet$}
      \item On pouvait aussi démontrer par récurrence que : $\forall n
        \in \N, \ u_{n+1} \geq u_n$ :
        \begin{noliste}{$\stimes$}
        \item {\bf Initialisation} : $u_1 \geq u_0$.
        \item {\bf Hérédité} : si $u_{n+1} \geq u_n$ alors $f(u_{n+1})
          \geq f(u_n)$ par croissance de la fonction $f$ (à
          démontrer).  
        \end{noliste}
        Il faut faire attention : le caractère croissant de $f$ ne
        suffit pas à démontrer que $(u_n)$ est croissant ! (si $u_0
        \geq u_1$, la suite obtenue est décroissante).


        \newpage


      \item On pouvait aussi démontrer que : $\forall x \geq 1, \ f(x)
        \geq x$ puis en déduire, en remplaçant $x$ par $u_n \geq 1$
        que : $f(u_n) \geq u_n$. Attention cependant : 
        \[
        \mbox{$f$ croissante} \ \not \Rightarrow \ f(x) \geq x
        \]
        On peut penser à la fonction racine qui est croissante sur
        $\R^+$ et vérifie : $\forall x \geq 1, \ f(x) \leq x$.
      \end{noliste}
    \end{remark}
    ~\\[-1cm]
  \end{proof}

\item Quelle est la limite de $u_n$ lorsque l'entier $n$ tend vers
  l'infini ?

  \begin{proof}~
    \begin{noliste}{$\sbullet$}
    \item D'après la question précédente, la suite $(u_n)$ est
      croissante.
    \item Supposons par l'absurde que $(u_n)$ est majorée.\\
      Étant croissante, $(u_n)$ est convergente vers un réel $\ell$.\\
      D'après la question \itbf{3.}, on sait que : $\forall n \in \N,
      \ u_n \geq 1$. Ainsi, par passage à la limite : $\ell \geq 1$.

    \item D'autre part, par définition : $\forall n \in \N, \ u_{n+1}
      = f(u_n)$.\\
      Par passage à la limite, on obtient : $\ell = f(\ell)$. Or,
      comme $\ell \neq 0$ :
      \[
      \ell = f(\ell) \ \Leftrightarrow \ \ell = \ell^3 \, \ee^\ell \
      \Leftrightarrow \ 1 = \ell^2 \, \ee^\ell \ \Leftrightarrow \
      \ell^2 \, \ee^\ell - 1 = 0
      \]
      Or, d'après la question \itbf{2.}, cette équation admet $\alpha$
      pour unique solution sur $]0, +\infty[$.\\
      On en déduit que $\ell = \alpha$. Enfin, comme $\alpha < 1$, on
      en déduit que $\ell < 1$.%
      \concL{Impossible car $\alpha \geq 1$. Ainsi, la suite $(u_n)$
        n'est pas majorée. Étant croissante, elle divrege vers
        $+\infty$.}{15}
    \end{noliste}
    \begin{remark}~
      \begin{noliste}{$\sbullet$}
      \item La suite $(u_n)$ étant croissante, soit elle est majorée
        et dans ce cas elle converge; soit elles n'est pas majorée et
        dans ce cas elle diverge vers $+\infty$.
      \item Dans cette question, on ne demande rien d'autre que
        d'établir cette disjonction de cas.       
      \end{noliste}
    \end{remark}
    ~\\[-1.2cm]
  \end{proof}

\end{enumerate}

\subsection*{Partie III : Étude d'une série}

\begin{enumerate}[\bfseries 1.]
  \setcounter{enumi}{5}
\item Montrer que la série $\Sum{n\geq 1}{} \dfrac{1}{f(n)}$
  converge. On note $S = \Sum{n=1}{+\infty} \dfrac{1}{f(n)}$.

  \begin{proof}~
    \begin{noliste}{$\sbullet$}
    \item La série $\Sum{n\geq 1}{} \dfrac{1}{f(n)}$ est à termes
      positifs.

    \item De plus : $\dfrac{1}{f(n)} = \dfrac{1}{n^3 \, \ee^{n}} =
      \oon\left(\dfrac{1}{n^2}\right)$. En effet, $\dfrac{\frac{1}{n^3
          \, \ee^{n}}}{\frac{1}{n^2}} = \dfrac{n^2}{n^3 \, \ee^{n}} =
      \dfrac{1}{n \, \ee^{n}} \tendn 0$.\\[.2cm]
      Or, d'après le critère de Riemann ($2 > 1$), la série
      $\Sum{n\geq 1}{} \dfrac{1}{n^2}$ est convergente.%
      \concL{Ainsi, d'après le théorème de négligeabilité des séries à
        termes positifs, la série $\Sum{n\geq 1}{} \dfrac{1}{f(n)}$
        est convergente.}{15}~\\[-1cm]
    \end{noliste}
  \end{proof}


  \newpage


\item Montrer : $\forall n\in\N^*$, $\left| \ S - \Sum{k=1}{n}
    \dfrac{1}{f(k)} \ \right| \leq \dfrac{1}{(\ee-1) \, \ee^n}$.

  \begin{proof}~\\
    Soit $n \in \N^*$.
    \begin{noliste}{$\sbullet$}
    \item Notons tout d'abord que : 
      \[
      S - \Sum{k=1}{n} \dfrac{1}{f(k)} \ = \ \Sum{k=1}{+\infty}
      \dfrac{1}{f(k)} - \Sum{k=1}{n} \dfrac{1}{f(k)} \ = \
      \Sum{k=n+1}{+\infty} \dfrac{1}{f(k)} \ = \ \dlim{m \tend
        +\infty} \Sum{k=n+1}{m} \dfrac{1}{f(k)}
      \]
    \item Or, comme pour tout $k \geq 1, \ k^3 \, \ee^k \geq \ee^k$ :
      \[
      \Sum{k=n+1}{m} \dfrac{1}{f(k)} \ = \ \Sum{k=n+1}{m}
      \dfrac{1}{k^3 \, \ee^{k}} \ \leq \ \Sum{k=n+1}{m}
      \dfrac{1}{\ee^{k}}
      \]
    \item Enfin :
      \[
      \begin{array}{rcl}
        \Sum{k=n+1}{m} \left(\dfrac{1}{\ee}\right)^k & = &
        \dfrac{(\frac{1}{\ee})^{n+1} - (\frac{1}{\ee})^{m+1}}{1-
          \frac{1}{\ee}} \ = \ \dfrac{(\frac{1}{\ee})^{n+1} -
          (\frac{1}{\ee})^{m+1}}{\frac{\ee-1}{\ee}} \ = \ \dfrac{\ee \,
          (\frac{1}{\ee})^{n+1} - \ee \,
          (\frac{1}{\ee})^{m+1}}{\ee-1} \\[.6cm]
        & = & \dfrac{\frac{1}{\ee^n} - \frac{1}{\ee^m}}{\ee-1} \ 
        \tendd{m}{+\infty} \ \dfrac{\frac{1}{\ee^n}}{\ee-1} \ = \ 
        \dfrac{1}{(\ee-1) \, \ee^n}
      \end{array}
      \]      
    \end{noliste}
    \conc{$\forall n\in\N^*$, $\left| \ S - \Sum{k=1}{n}
        \dfrac{1}{f(k)} \ \right| \leq \dfrac{1}{(\ee-1) \, \ee^n}$}~\\[-1cm]
  \end{proof}

\item En déduire une fonction en {\bf Scilab} qui calcule une valeur
  approchée de $S$ à $10^{-4}$ près.

  \begin{proof}~
    \begin{noliste}{$\sbullet$}
    \item Afin de calculer une valeur approchée de $S$ à $10^{-4}$
      près, il suffit de trouver $n \in \N^*$ tel que :
      \[
      \dfrac{1}{(\ee-1) \, \ee^n} \ \leq \  10^{-4}
      \]
    \item En effet, d'après la question précédente, on a alors :
      \[
      \left| \ S - \Sum{k=1}{n} \dfrac{1}{f(k)} \ \right| \ \leq \
      \dfrac{1}{(\ee-1) \, \ee^n} \ \leq \ 10^{-4}
      \]
      et $\Sum{k=1}{n} \dfrac{1}{f(k)}$ constitue dans ce cas
      l'approximation recherchée.
    \item On en déduit le programme suivant.

      \begin{scilab}
        & n = 1 \nl %
        & S = 1 / exp(1) \nl %
        & \tcFor{while} 1 / ((exp(1)-1) \Sfois{} exp(n)) >
        10\puis{}(-4) \nl %
        & \qquad n = n + 1 \nl %
        & \qquad S = S + 1 / (n\puis{}3 \Sfois{} exp(n)) \nl %
        & \tcFor{end} \nl %
      \end{scilab}      
    \end{noliste}


    \newpage


    \begin{remark}~\\
      Le programme précédent propose de déterminer la valeur de $n$ et
      de $S_n$ en procédant par itération. On peut aussi remarquer que :
      \[
      \dfrac{1}{(\ee-1) \, \ee^n} \ \leq \ 10^{-4} \ \Leftrightarrow \
      \ee^n \ \geq \ \dfrac{10^{4}}{(\ee-1)} \ \Leftrightarrow \ n \
      \geq \ 4 \, \ln(10) - \ln(\ee-1)
      \]
      {\it (par stricte croissance de la fonction $\ln$)}\\[.2cm]
      Ainsi, $S_n$ est une approximation de $S$ pour tout $n \geq m =
      \lceil 4 \, \ln(10) - \ln(\ee-1) \rceil$.\\
      On en déduit le programme \Scilab{} suivant.

      \begin{scilab}
        & m = ceil(4 \Sfois{} log(10) - log(exp(1)-1)) \nl %
        & S = 0 \nl %
        & \tcFor{for} i = 1:m \nl %
        & \qquad S = S + 1 / (i\puis{}3 \Sfois{} exp(i)) \nl %
        & \tcFor{end} \nl %
      \end{scilab}
    \end{remark}
  \end{proof}

  % \subsection*{Partie IV : Etude d'une fonction de deux variables}
  % On considère l'ouvert $U=]0,+\infty[\times \R$ de $\R^2$ et
  % l'application de classe $\Cont{2}$ suivante : \\
  % \[
  % g=U\rightarrow \R, (x,y)\mapsto g(x,y)=\frac{1}{x}+\ee^x-y^2\ee^y
  % \]
  % \item Représenter graphiquement l'ensemble $U$.
  % \item Calculer, pour tout $(x,y)$ de $U$, les dérivées partielles
  %   premières de $g$ en $(x,y)$.
  % \item Montrer que $g$ admet deux points critiques et deux
  %   seulement, et que ceux-ci sont $(\alpha, 0)$ et $(\alpha, -2)$,
  %   où $\alpha$ est le réel défini à la question 2.
  % \item Est-ce que $g$ admet un extremum local en $(\alpha, 0)$?
  % \item Est-ce que $g$ admet un extremum local en $(\alpha, -2)$?
  % \item Est-ce que $g$ admet un extremum global sur $U$?
\end{enumerate}

\section*{Exercice 3}

\noindent
Soit $E$ un espace vectoriel de dimension 3. On note $0_E$ le vecteur
nul de $E$.\\
On note $i$ l'application identité de $E$, et $\theta$ l'application
constante nulle de $E$ dans $E$ :
\[
 i : \left\{
 \begin{array}{rcl}
  E & \to & E\\
  x & \mapsto & x
 \end{array}
 \right.
 \quad \mbox{ et } \quad 
 \theta : \left\{
 \begin{array}{rcl}
  E & \to & E\\
  x & \mapsto & 0_E
 \end{array}
 \right.
\]
On considère un endomorphisme $f$ de $E$ tel que : 
\[
 f \neq \theta, \quad f^2 + i \neq \theta, \quad f \circ (f^2+i)=\theta
\]
où $f^2$ désigne $f \circ f$.

\begin{noliste}{1.}
 \setlength{\itemsep}{4mm}
 \item 
 \begin{noliste}{a)}
  \item Montrer que $f$ n'est pas bijectif.
  
    \begin{proof}~\\
      Supposons par l'absurde que l'endomorphisme $f$ est bijectif.\\
      Alors l'endomorphisme $f$ admet une bijection réciproque $f^{-1}
      : E \to E$. Or, d'après l'énoncé :
      \[
      f \circ (f^2 +i) = \theta
      \]
      On en déduit, en composant de part et d'autre à gauche par
      $f^{-1}$ :
      \[
      \begin{array}{ccl}
        f^{-1} \circ \big( f \circ (f^2 +i) \big) & = & f^{-1} \circ
        \theta \ = \ \theta 
        \\[.2cm]
        \shortparallel & 
        \\[.2cm]
        \multicolumn{3}{l}{( f^{-1} \circ f ) \circ (f^2 +i) \ = \ 
          i \circ (f^2+i) = f^2+i}
    \end{array} 
   \]
   Ce qui est absurde, puisque d'après l'énoncé : $f^2+i \neq \theta$.%
   \conc{On en déduit que $f$ n'est pas bijectif.}%~\\[-1cm]
   \begin{remark}~\\
     On peut aussi raisonner de manière directe. Comme :
     \[
     f \circ (f^2 + i) = \theta
     \]
     alors, pour tout $x \in E$ : $\big( f \circ (f^2 + i) \big)(x) =
     f \big (f^2 + i) (x) \big) = 0_E$.\\[.1cm]
     Autrement dit : $\forall x \in E$, $(f^2 + i) (x) \in \kr(f)$.\\
     Or, d'après l'énoncé : $f^2 + i \neq 0_{\LL{E}}$. Ainsi, il
     existe $x \in E$ tel que : $(f^2 + i)(x) \neq 0_E$.\\
     On en déduit :
     \[
     \kr(f) \supset \{0_E, (f^2 + i)(x)\} \neq \{ 0_E \}
     \]
     Ainsi, $f$ n'est pas injective. Elle n'est donc pas bijective.
   \end{remark}~\\[-1.4cm]
  \end{proof}
  
\item En déduire que $0$ est valeur propre de $f$, puis montrer qu'il
  existe $u$ appartenant à $E$ tel que : $u\neq 0_E$ et $f(u)=0_E$.
  
  \begin{proof}~
   \begin{noliste}{$\sbullet$}
   \item D'après la question précédente, $f$ n'est pas bijectif,
     c'est-à-dire $f- 0 \cdot i$ n'est pas bijectif.\\ %
     Comme $E$ est de dimension finie, ceci équivaut à $f$ non injectif.%
     \conc{Donc $0$ est valeur propre de $f$.}
    
  \item $0$ est valeur propre de $f$, donc : $\kr(f) = \kr(f - 0 \cdot
    i) \neq \{0_E\}$.\\[.1cm]
    Il existe donc $u \neq 0_E$ tel que $u \in \kr(f)$.%
    \conc{Autrement dit, il existe $u\in E$ tel que : $u \neq 0_E$ et
      $f(u) = 0_E$.}~\\[-1.4cm]
   \end{noliste}
  \end{proof}
 \end{noliste}


\newpage


 \noindent
 Soit $v_1$ appartenant à $E$ tel que : $v_1\neq 0_E$ et $f(v_1)=0_E$.\\[-.8cm]
 \item Montrer : $\spc(f)=\{0\}$.
 
 \begin{proof}~
  \begin{noliste}{$\sbullet$}
  \item D'après l'énoncé : $f \circ (f^2+i) = \theta$.\\[.1cm]
    On en déduit que le polynôme $Q(X)=X(X^2+1)$ est un polynôme
    annulateur de $f$.\\
    De plus l'unique racine de $Q$ est $0$ (le polynôme $X^2+1$
    n'admet pas de racine réelle).%
    \conc{D'où : $\spc(f) \subset \{ \text{racines de $Q$} \} =
      \{0 \}$.}
   
  \item De plus, d'après la question \itbf{1.b)}, $0 \in \spc(f)$.
  \end{noliste}
  \conc{On en déduit : $\spc(f) = \{0\}$.}~\\[-1cm]
 \end{proof}
 
\item Est-ce que $f$ est diagonalisable ?
 
  \begin{proof}~\\
    D'après l'énoncé, $E$ est un espace vectoriel de dimension $3$. On
    note $\B$ l'une de ses bases.\\[.2cm]
    Supposons par l'absurde que $f$ est diagonalisable. \\
    Il existe alors une base $\B' = (e_1', e_2', e_3')$ constituée de
    vecteurs propres de $f$ dans laquelle la matrice représentative de
    $f$ est la matrice diagonale :
    \[
    D = 
    \begin{smatrix}
      0 & 0 & 0 \\
      0 & 0 & 0 \\
      0 & 0 & 0
    \end{smatrix} = 0_{\M{3,1}}
    \]
    D'après la formule de changement de base :
    \[
    \begin{array}{ccccc}
      \Mat_{\B}(f) & = & P_{\B, \B'} & \Mat_{\B'}(f) & P_{\B', \B}
      \\[.2cm]
      & & \shortparallel & \shortparallel &
      \shortparallel
      \\[.2cm]
      & & P & D & P^{-1}
    \end{array}
    \]
    Ainsi :
    \[
    \Mat_{\B}(f) = P \ 0_{\M{3,1}} \ P^{-1} = 0_{\M{3,1}} =
    \Mat_{\B}(\theta)
    \]
    L'application $\Mat_{\B}(.)$ étant bijective, on en déduit : $f =
    \theta$.\\
    Absurde ! ~\\[-.6cm]%
    \conc{Ainsi, $f$ n'est pas diagonalisable.}
    \begin{remark}~
      \begin{noliste}{$\sbullet$}
      \item Il était possible de rédiger différemment en prenant le
        parti de diagonaliser la matrice représentative de
        $f$. Détaillons cette rédaction.
      \item On commence par noter $M = \Mat_{\B}(f)$ la matrice
        représentative de $f$ dans une base $\B$ de $E$.\\
        Supposons par l'abusrde que $f$ est diagonalisable. Alors $M$
        est diagonalisable. \\
        Il existe alors une matrice $P \in \M{n}$ inversible telle que
        :
        \[
        M = P D P^{-1}
        \]
        où $D$ est une matrice diagonale dont les c\oe{}fficients
        diagonaux sont les valeurs propres de $M$. \\
        Or $\spc(M) = \spc(f) = \{0\}$. Donc : $D = 0_{\M{3,1}}$.\\
        Et ainsi :
        \[
        \Mat_{\B}(f) = P \ 0_{\M{3,1}} \ P^{-1} = 0_{\M{3,1}} =
        \Mat_{\B}(\theta)
        \]
        Et on peut donc conclure comme ci-dessus.
      \item Cette question est un grand classique des sujets.\\
        Il faut donc savoir la traiter correctement, en adoptant l'un
        ou l'autre des rédactions ci-dessus.
      \end{noliste}
    \end{remark}~\\[-1.2cm]
 \end{proof}
 

\newpage


\item Montrer que $f^2+i$ n'est pas bijectif, puis en déduire qu'il
  existe $v$ appartenant à $E$ tel que : \\
  $v\neq 0_E$ et $f^2(v)=-v$.
 
  \begin{proof}~
    \begin{noliste}{$\sbullet$}
    \item Supposons par l'absurde que l'endomorphisme $f^2 + i$ est
      bijectif.\\
      Alors l'endomorphisme $g = f^2 + i$ admet une bijection
      réciproque $g^{-1} : E \to E$. Or, d'après l'énoncé :
      \[
      f \circ (f^2 +i) = \theta
      \]
      On en déduit, en composant de part et d'autre à droite par $g$ :
      \[
      \begin{array}{ccl}
        \big(f \circ (f^2 +i) \big) \circ g^{-1} & = & \theta \circ g^{-1} \ = \
        \theta
        \\[.2cm]
        \shortparallel & 
        \\[.2cm]
        \multicolumn{3}{l}{ \big(f \circ (f^2 +i) \big) \circ g^{-1} \ = \
          f \circ \big( (f^2 +i) \circ g^{-1} \big) = f \circ i = f}
      \end{array} 
      \]
      Ce qui est absurde, puisque d'après l'énoncé : $f \neq \theta$.%
      \conc{On en déduit que $f^2 + i$ n'est pas bijectif.}~\\[-1cm]
   
    \item L'endomorphisme $f^2+i$ n'est pas bijectif. Donc :
      $\kr(f^2+i) \neq \{0_E\}$.\\[.1cm]
      Il existe donc $v \neq 0_E$ tel que $v \in \kr(f^2+i)$. Or :
   \[
    v\in \kr(f^2+i) \ \Leftrightarrow \ (f^2+i)(v)=0_E \ 
    \Leftrightarrow \ f^2(v) +v=0_E \ \Leftrightarrow \
    f^2(v)=-v
   \]
   \conc{Ainsi, il existe $v\in E$ tel que : $v \neq 0_E$ et
     $f^2(v)=-v$.}~\\[-1.4cm]
  \end{noliste}
 \end{proof}

\noindent 
Soit $v_2$ appartenant à $E$ tel que : $v_2\neq 0_E$ et
$f^2(v_2)=-v_2$. On note $v_3=f(v_2)$.
 
\item Montrer : $f(v_3)=-v_2$.
 
  \begin{proof}~\\
    On calcule :
    \[
    f(v_3) = f(f(v_2)) = f^2(v_2) = -v_2
    \]
    \conc{On a bien : $f(v_3) = -v_2$.}~\\[-1cm]
  \end{proof}
 
 \item 
 \begin{noliste}{a)}
  \item Montrer que la famille $\B=(v_1,v_2,v_3)$ est une base 
  de $E$.
  
  \begin{proof}~
    \begin{noliste}{$\sbullet$}
    \item Montrons que la famille $\B$ est libre.\\
      Soit $(\lambda_1, \lambda_2, \lambda_3) \in \R^3$. Supposons que
      :
      \[
      \lambda_1 \cdot v_1 + \lambda_2 \cdot v_2 + \lambda_3 \cdot v_3
      = 0_E
      \]
      \begin{noliste}{-}
      \item On applique $f$ de part et d'autre. On obtient, par
        linéarité de $f$ :
	\[
        \lambda_1 \cdot f(v_1) + \lambda_2 \cdot f(v_2) + 
        \lambda_3 \cdot f(v_3) = f(0_E) = 0_E
	\]
	Or, on a les relations suivantes :
	\[
        f(v_1)=0_E, \quad f(v_2)=v_3 \quad \mbox{et} \quad 
        f(v_3)=-v_2
	\]
	On obtient alors :
	\[
        \lambda_2 \cdot v_3 - \lambda_3 \cdot v_2 = 0_E \quad (L_1)
	\]


        \newpage

	
      \item On applique de nouveau $f$ de part et d'autre. On obtient
        alors :
	\[
        \lambda_2 \cdot f(v_3) - \lambda_3 \cdot f(v_2) = f(0_E) = 0_E
	\]
	Ainsi, par les mêmes relations que précédemment :
	\[
        -\lambda_2 \cdot v_2 - \lambda_3 \cdot v_3 = 0_E \quad (L_2)
	\]
	
      \item Par combinaison linéaire des égalités précédentes
        ($\lambda_3 \, L_1 + \lambda_2 \, L_2$), on obtient :
	\[
        -\lambda_3^2 \cdot v_2 - \lambda_2^2 \cdot v_2 = 0_E
	\]
	Autrement dit : $(\lambda_2^2 + \lambda_3^2) \cdot 
	v_2=0_E$.\\[.1cm]
	Or : $v_2 \neq 0_E$. Donc : $\lambda_2^2 + \lambda_3^2 = 
	0_{\R}$. D'où : $\lambda_2=0$ et $\lambda_3=0$.
	
      \item L'équation initiale devient alors :
	\[
        \lambda_1 \cdot v_1 = 0_E
	\]
	Or : $v_1 \neq 0_E$. Donc : $\lambda_1 = 0_{\R}$.
      \end{noliste}
      Finalement : $\lambda_1 = \lambda_2 = \lambda_3 = 0$. %
      \conc{La famille $\B$ est donc libre.}
      
    \item De plus : $\Card(\B)=\Card((v_1,v_2,v_3))=3=\dim(E)$.
    \end{noliste}
    \conc{Donc $\B$ est une base de $E$.}~\\[-1cm]
  \end{proof}
  
  \item Déterminer la matrice $C$ de $f$ dans la base $\B$.
  
  \begin{proof}~
   \begin{noliste}{$\sbullet$}
    \item On a : $f(v_1) = 0_E = 0 \cdot v_1 + 0 \cdot v_2 + 0 \cdot 
    v_3$.\\[.2cm]
    On en déduit que : $\Mat_{\B}(f(v_1))=
    \begin{smatrix}
     0\\
     0\\
     0
    \end{smatrix}$.
    
    \item On a : $f(v_2) = v_3 = 0 \cdot v_1 + 0 \cdot v_2 + 1 \cdot 
    v_3$.\\[.2cm]
    On en déduit que : $\Mat_{\B}(f(v_2))=
    \begin{smatrix}
     0\\
     0\\
     1
    \end{smatrix}$.
    
    \item On a : $f(v_3) = -v_2 = 0 \cdot v_1 - 1 \cdot v_2 + 0 \cdot 
    v_3$.\\[.2cm]
    On en déduit que : $\Mat_{\B}(f(v_3))=
    \begin{smatrix}
     0\\
     -1\\
     0
    \end{smatrix}$.
   \end{noliste}
   \conc{Ainsi : $C=\Mat_{\B}(f) = 
   \begin{smatrix}
    0 & 0 & 0\\
    0 & 0 & -1\\
    0 & 1 & 0
   \end{smatrix}$.}~\\[-1cm]
  \end{proof}

 \end{noliste}
 
 \noindent
 On considère les matrices suivantes : $A=
 \begin{smatrix}
  1 & 0 & 0\\
  0 & 0 & 0\\
  0 & 0 & 0
 \end{smatrix}$ et $B= 
 \begin{smatrix}
  0 & 0 & 0\\
  0 & 1 & 0\\
  0 & 0 & 1
 \end{smatrix}$,\\[.2cm]
 et le sous-espace vectoriel ${\cal F}$ de $\M{3}$ engendré par 
 $(A,B,C)$, c'est-à-dire :
 \[
  {\cal F} = \{ a \, A + b \, B + c \, C \ | \ (a,b,c)\in\R^3\}
 \]
 

 \newpage


 \item Déterminer la dimension de ${\cal F}$.
 
 \begin{proof}~
  \begin{noliste}{$\sbullet$}
  \item Montrons que $(A,B,C)$ est une famille libre de ${\cal F}$.\\
    Soit $(\lambda_1, \lambda_2, \lambda_3) \in \R^3$. Supposons que :
   \[
    \lambda_1 \cdot A + \lambda_2 \cdot B + \lambda_3 \cdot C = 0_{\M{3}}
   \]
   Ceci équivaut à : 
   \[
    \begin{smatrix}
     \lambda_1 & 0 & 0\\
     0 & \lambda_2 & -\lambda_3\\
     0 & \lambda_3 & \lambda_2
    \end{smatrix}
    =
    \begin{smatrix}
     0 & 0 & 0\\
     0 & 0 & 0\\
     0 & 0 & 0
    \end{smatrix}
   \]
   Par identification, on en déduit : $\lambda_1 = \lambda_2 =
   \lambda_3 = 0$.%
   \conc{La famille $(A,B,C)$ est libre.}
   
 \item De plus, par définition de ${\cal F}$, %($= \Vect{A, B, C}$)
   la famille $(A,B,C)$ engendre ${\cal F}$.%
   \conc{Ainsi, $(A,B,C)$ est une base de ${\cal F}$.}
 \end{noliste}
 \conc{On en déduit : $\dim({\cal F})=\Card((A,B,C))=3$.}~\\[-1.2cm]
 \end{proof}

\item Montrer : $\{M\in \M{3} \ | \ CM=MC\} = {\cal F}$
 
 \begin{proof}~\\
  Soit $M\in \M{3}$. Alors il existe $(a_{1,1},a_{1,2}, a_{1,3},
  a_{2,1}, a_{2,2}, a_{2,3}, a_{3,1}, a_{3,2}, a_{3,3}) \in \R^9$ 
  tels que : 
  \[
   M=
   \begin{smatrix}
    a_{1,1} & a_{1,2} & a_{1,3}\\
    a_{2,1} & a_{2,2} & a_{3,2}\\
    a_{3,1} & a_{3,2} & a_{3,3}
   \end{smatrix}
  \]
  On calcule :
  \[
   CM = 
   \begin{smatrix}
    0 & 0 & 0\\
    0 & 0 & -1\\
    0 & 1 & 0
   \end{smatrix}
   \begin{smatrix}
    a_{1,1} & a_{1,2} & a_{1,3}\\
    a_{2,1} & a_{2,2} & a_{3,2}\\
    a_{3,1} & a_{3,2} & a_{3,3}
   \end{smatrix}
   =
   \begin{smatrix}
    0 & 0 & 0\\
    -a_{3,1} & -a_{3,2} & -a_{3,3}\\
    a_{2,1} & a_{2,2} & a_{2,3}
   \end{smatrix}
  \]
  et :
  \[
   MC =
   \begin{smatrix}
    a_{1,1} & a_{1,2} & a_{1,3}\\
    a_{2,1} & a_{2,2} & a_{3,2}\\
    a_{3,1} & a_{3,2} & a_{3,3}
   \end{smatrix}
   \begin{smatrix}
    0 & 0 & 0\\
    0 & 0 & -1\\
    0 & 1 & 0
   \end{smatrix}
   =
   \begin{smatrix}
    0 & a_{1,3} & -a_{1,2}\\
    0 & a_{2,3} & -a_{2,2}\\
    0 & a_{3,3} & -a_{3,2}
   \end{smatrix}
  \]  
  On obtient alors les équivalences suivantes :
  \[
   \begin{array}{rcl}
    CM=MC & \Leftrightarrow & 
    \begin{smatrix}
    0 & 0 & 0\\
    -a_{3,1} & -a_{3,2} & -a_{3,3}\\
    a_{2,1} & a_{2,2} & a_{2,3}
   \end{smatrix}
   =
   \begin{smatrix}
    0 & a_{1,3} & -a_{1,2}\\
    0 & a_{2,3} & -a_{2,2}\\
    0 & a_{3,3} & -a_{3,2}
   \end{smatrix}
   \\[.8cm]
   & \Leftrightarrow & \left\{
   \begin{array}{rcl}
    a_{1,3} & = & 0\\
    -a_{1,2} & = & 0\\
    -a_{3,1} & = & 0\\
    -a_{3,2} & = & a_{2,3}\\
    -a_{3,3} & = & -a_{2,2}\\
    a_{2,1} & = & 0\\
    a_{2,2} & = & a_{3,3}\\
    a_{2,3} & = & -a_{3,2}
   \end{array}
   \right.
   \ \Leftrightarrow \ 
   M=
   \begin{smatrix}
    a_{1,1} & 0 & 0\\
    0 & a_{2,2} & -a_{3,2}\\
    0 & a_{3,2} & a_{2,2}
   \end{smatrix}
   \\[2cm]
   & \Leftrightarrow & M = a_{1,1} \, A + a_{2,2} \, B + a_{3,2} \, C
   \\[.2cm]
   & \Leftrightarrow & M \in {\cal F}
   \end{array}
  \]
  \conc{On en déduit : $\{M\in\M{3} \ | \ CM=MC\} = {\cal F}$}~\\[-1.2cm]
 \end{proof}

 
 \newpage


 \item 
 \begin{noliste}{a)}
  \item Pour tout $(a,b,c)\in\R^3$, calculer la matrice $(a 
  \, A+b \, B+c \, C)^2$.
  
  \begin{proof}~\\
   Soit $(a,b,c)\in \R^3$.
   \[
    (a\, A + b\, B + c\, C)^2 = 
    \begin{smatrix}
     a & 0 & 0\\
     0 & b & -c\\
     0 & c & b
    \end{smatrix}
    \begin{smatrix}
     a & 0 & 0\\
     0 & b & -c\\
     0 & c & b
    \end{smatrix}
    =
    \begin{smatrix}
     a^2 & 0 & 0\\
     0 & b^2-c^2 & -2bc\\
     0 & 2bc & b^2-c^2
    \end{smatrix}
    = a^2 \, A + (b^2-c^2)\, B + 2bc \, C
   \]
   \conc{$(a\, A + b\, B + c\, C)^2 = 
   a^2 \, A + (b^2-c^2)\, B + 2bc \, C$}~\\[-1.2cm]
  \end{proof}
  
\item En déduire une matrice $M$ de $\M{3}$ telle que : $M^2=
  \begin{smatrix}
   4 & 0 & 0\\
   0 & 5 & -12\\
   0 & 12 & 5
  \end{smatrix}$.
  
  \begin{proof}~
   \begin{noliste}{$\sbullet$}
    \item On remarque que :
    \[
     \begin{smatrix}
      4 & 0 & 0\\
      0 & 5 & -12\\
      0 & 12 & 5
     \end{smatrix}
     = 4 \, A + 5 \, B + 12 \, C
    \]
    
    \item D'après la question \itbf{9.a)}, si on trouve $(a,b,c)\in 
    \R^3$ tels que :
    \[
     4 \, A + 5 \, B + 12 \, C = 
     a^2 \, A + (b^2-c^2)\, B + 2bc \, C
    \]
     alors, en posant
    $M = a \, A + b\, B + c \, C$, on a : 
    \[
     M^2 = a^2 \, A + (b^2-c^2)\, B + 2bc \, C = 4 \, A + 5 \, B + 12 
     \, C =
     \begin{smatrix}
      4 & 0 & 0\\
      0 & 5 & -12\\
      0 & 12 & 5
     \end{smatrix}
    \]    
    
    \item Cherchons donc $(a,b,c)\in\R^3$ tels que :
    \[
     a^2 \, A + (b^2-c^2)\, B + 2bc \, C = 4 \, A + 5 \, B + 12\, C
    \]
    On obtient le système d'équations suivant :
    \[
     \left\{
     \begin{array}{rcl}
      a^2 & = & 4\\
      b^2-c^2 & = & 5\\
      2bc & = & 12
     \end{array}
     \right. 
     \quad \Leftrightarrow \quad
     \left\{
     \begin{array}{rcl}
      a & = & 2\\
      b^2-c^2 & = & 5\\
      bc & = & 6
     \end{array}
     \right. 
     \quad \OU{} \quad
     \left\{
     \begin{array}{rcl}
      a & = & -2\\
      b^2 - c^2 & = & 5\\
      bc & = & 6
     \end{array}
     \right.
    \]
    Remarquons alors que $b = 0$ ne peut convenir (si $b = 0$ alors
    $bc = 0 \neq 6$). On suppose donc : $b \neq 0$. L'égalité $bc = 6$
    permet d'écrire : $c = \dfrac{6}{b}$.\\
    En réinjectant dans la deuxième ligne, on obtient :
    \[
    \begin{array}{rcl}
      b^2 - c^2 = 5 \ \Leftrightarrow \ b^2 - \frac{36}{b^2} = 5 \
      \Leftrightarrow \ b^4 - 36 = 5 \ b^2 & \Leftrightarrow & b^4 - 5
      b^2 - 36 = 0
      \\[.2cm]
      & \Leftrightarrow & (b^2 + 4) \ (b^2 - 9) \ = \ 0
      \\[.2cm]
      & \Leftrightarrow & b^2 = -4 \ \OU{} \ b^2 = 9
      \\[.2cm]
      & \Leftrightarrow & b^2 = 9
      \\[.2cm]
      & \Leftrightarrow & b = -3 \ \OU{} \ b = 3
    \end{array}
    \]
    On obtient alors : $c^2 = b^2 - 5 = 9 - 5 = 4$. Ainsi : $c = -2 \
    \OU{} \ c = 2$.\\
    Le triplet $(a,b,c)$ suivant convient : $\left\{
     \begin{array}{rcl}
      a & = & 2\\
      b & = & 3\\
      c & = & 2
     \end{array}
     \right.$.
     

     \newpage

 
     \item On a alors :
     \[
      \begin{array}{rcl@{\quad}>{\it}R{5cm}}
       \begin{smatrix}
	4 & 0 & 0 \\
	0 & 5 & -12 \\
	0 & 12 & 5
       \end{smatrix}
       & = & 2^2 \, A + (3^2-2^2)\, B + 2\times 3 \times 2 \, C
       \\[.2cm]
       & = & (2 \, A + 3 \, B + 2 \, C)^2
       & (d'après la question \itbf{9.a)})
      \end{array}
     \]
   \end{noliste}
   \conc{Donc, en posant $M=2\, A + 3\, B + 2\, C$, on obtient : $M^2
     =
     \begin{smatrix}
       4 & 0 & 0\\
       0 & 5 & -12\\
       0 & 12 & 5
     \end{smatrix}$.}~\\[-1cm]
 \end{proof}
\end{noliste}

\item On note $g=f^2-i$.\\
  Montrer que $g$ est bijectif et exprimer $g^{-1}$ à l'aide de $f$ et
  de $i$.
 
 \begin{proof}~
  \begin{noliste}{$\sbullet$}
    \item Tout d'abord :
    \[
    \begin{array}{rcl@{\quad}>{\it}R{5cm}}
      \Mat_{\B}(g) & = & \Mat_{\B}(f^2-i) 
      \\[.2cm]
      & = & \Mat_{\B}(f^2) - \Mat_{\B}(i) 
      & (par linéarité de $\Mat_{\B}(.)$)
      \nl
      \nl[-.2cm]
      & = & \left(\Mat_{\B}(f)\right)^2 - I_3 
      \\[.2cm]
      & = & C^2-I_3 
      \\[.2cm]
      & = & 
     \begin{smatrix}
      0 & 0 & 0\\
      0 & -1 & 0\\
      0 & 0 & -1
     \end{smatrix}
     -
     \begin{smatrix}
      1 & 0 & 0\\
      0 & 1 & 0\\
      0 & 0 & 1
     \end{smatrix}
     \\[.6cm]
     & = & 
     \begin{smatrix}
      -1 & 0 & 0\\
      0 & -2 & 0\\
      0 & 0 & -2
     \end{smatrix}
    \end{array}
    \]
    La matrice représentative de $g$ dans la base $\B$ est une matrice
    diagonale à c\oe{}fficients tous non nuls. Elle est donc
    inversible.%
    \conc{On en déduit que l'endomorphisme $g$ est bijectif.}
    
  \item Par propriété de l'application $\Mat_\B(.)$ :
    \[
    \begin{array}{rcl}
      \Mat_{\B}(g^{-1}) & = & \left(\Mat_{\B}(g)\right)^{-1} 
      \\[.4cm]
      & = & 
      \begin{smatrix}
        -1 & 0 & 0 \\
        0 & -\frac{1}{2} & 0\\
        0 & 0 & -\frac{1}{2}
      \end{smatrix}
      \\[.8cm]
      & = & -I_3 - \dfrac{1}{2} \, C^2
      \\[.4cm]
      & = & \Mat_{\B}\left( -i-\dfrac{1}{2}\, f^2 \right)
    \end{array}
    \]
    \conc{L'application $\Mat_\B(f)$ étant bijective, on en déduit :
      $g^{-1} = -i-\dfrac{1}{2}\, f^2$.}~\\[-1.2cm]
  \end{noliste}
 \end{proof}
\end{noliste}

\end{document}
