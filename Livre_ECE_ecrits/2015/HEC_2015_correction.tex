\documentclass[11pt]{article}%
\usepackage{geometry}%
\geometry{a4paper,
  lmargin=2cm,rmargin=2cm,tmargin=2.5cm,bmargin=2.5cm}

\input{../macros_Livre.tex}

% \renewcommand{\thesection}{\Roman{section}.\hspace{-.3cm}}
% \renewcommand{\thesubsection}{\Alph{subsection}.\hspace{-.2cm}}

\pagestyle{fancy} %
\pagestyle{fancy} %
 \lhead{ECE2 \hfill Mathématiques \\} %
\chead{\hrule} %
\rhead{} %
\lfoot{} %
\cfoot{} %
\rfoot{\thepage} %

\renewcommand{\headrulewidth}{0pt}% : Trace un trait de séparation
                                    % de largeur 0,4 point. Mettre 0pt
                                    % pour supprimer le trait.

\renewcommand{\footrulewidth}{0.4pt}% : Trace un trait de séparation
                                    % de largeur 0,4 point. Mettre 0pt
                                    % pour supprimer le trait.

\setlength{\headheight}{14pt}

\title{\bf \vspace{-1.6cm} HEC 2015} %
\author{} %
\date{} %
\begin{document}

\maketitle %
\vspace{-1.2cm}\hrule %
\thispagestyle{fancy}

\vspace*{.4cm}

%%DEBUT

\section*{Exercice}

\noindent
Soit $n$ un entier supérieur ou égal à 2 et $\B=(e_1,e_2,\ldots,e_n)$
la base canonique de $\R^n$.\\
Soit $v$ un vecteur donné de $\R^n$ de coordonnées
$v_1,v_2,\ldots,v_n$ dans la base $\B$ et qui vérifie $\Sum{i=1}{n}
v_i=1$.\\[.2cm]
Soit $f$ l'application définie sur $\R^n$ qui à tout vecteur
$x=(x_1,x_2,\ldots,x_n) \in \R^n$, associe le vecteur $f(x)$ défini
par: $f(x) = x-\Big(\Sum{i=1}{n} x_i\Big) \cdot v$.

\begin{noliste}{1.}
  \setlength{\itemsep}{4mm}
\item
  \begin{noliste}{a)}
    \setlength{\itemsep}{2mm}
  \item Montrer que $f$ est un endomorphisme de $\R^n$.
    
    \begin{proof}~
      \begin{noliste}{$\sbullet$}
      \item Montrons que $f$ est une application linéaire.\\
        Soit $(\lambda, \mu)\in \R^2$. Soit $(x,y)\in (\R^n)^2$.
        Alors il existe $(x_1, \ldots, x_n, y_1, \ldots, y_n) \in
        \R^{2n}$ tels que :
        \[
        x=(x_1,x_2, \ldots, x_n) \quad \mbox{et} \quad
        y=(y_1,y_2, \ldots, y_n)
        \]
        Par définition de $f$ :
        \[
        \begin{array}{rcl}
          f(\lambda \cdot x + \mu \cdot y) & = & (\lambda \cdot x + 
          \mu \cdot y) - \left(\Sum{i=1}{n}(\lambda \, x_i + \mu \, y_i)
          \right) \cdot v
          \\[.6cm]
          & = & \lambda \cdot x + \mu \cdot y - \left(\lambda 
            \Sum{i=1}{n} x_i + \mu \Sum{i=1}{n} y_i \right) \cdot v
          \\[.6cm]
          & = & \lambda \cdot x + \mu \cdot y - \lambda \left( 
            \Sum{i=1}{n} x_i \right) \cdot v - \mu \left( \Sum{i=1}{n} 
            y_i \right) \cdot v
          \\[.6cm]
          & = & \lambda \cdot \left(x - \Big( \Sum{i=1}{n} x_i \Big) 
\cdot 
            v\right) + \mu \cdot \left(y - \Big( \Sum{i=1}{n} y_i \Big) 
\cdot 
            v\right)
          \\[.6cm]
          & = & \lambda \cdot f(x) + \mu \cdot f(y)
        \end{array}
        \]
        \conc{L'application $f$ est linéaire.}
        
      \item Montrons que $f(\R^n) \subset \R^n$, \ie : $\forall x\in
        \R^n$, $f(x) \in \R^n$.\\
        Soit $x = (x_1, \ldots, x_n) \in \R^n$. Alors : $\Sum{i=1}{n}
        x_i \in \R$.\\
        On rappelle que : $f(x) = x-\Big(\Sum{i=1}{n} x_i\Big) \cdot
        v$.\\
        Ainsi, $f(x)$ appraît comme combinaison linéaire des vecteurs
        de $\R^n$, $x$ et $v$.\\
        Comme $\R^n$ est un espace vectoriel, on a bien : $f(x) \in
        \R^n$.  \conc{L'application $f$ est à valeurs dans $\R^n$.}
      \end{noliste}
      \conc{L'application $f$ est un endomorphisme de $\R^n$.}~\\[-1cm]
    \end{proof}

  
    \newpage
    
    
  \item Montrer que $f\circ f=f$.
  
    \begin{proof}~\\
      Soit $x=(x_1, \ldots, x_n) \in \R^n$.
%       \[
%       (f\circ f)(x) \ = \ f(f(x)) \ = \ f\left(x - \Big(\Sum{i=1}{n}
%         x_i \Big) \cdot v\right)
%       \]
%       Notons $u = f(x)$. On obtient alors :
%       \[
%       \begin{array}{rcl}
%         u & = & x- \Big( \Sum{i=1}{n} x_i \Big) \cdot v \ = \ 
%         \left(x_1 - \Big( \Sum{i=1}{n} x_i \Big) \ v_1, \ x_2 - 
%           \Big( \Sum{i=1}{n} x_i \Big) \ v_2, \ldots,
%           \ x_n - \Big( \Sum{i=1}{n} x_i \Big) \ v_n\right)
%         \\[.6cm]
%         & = & (u_1, u_2, \ldots, u_n)
%       \end{array}
%       \]
%       En particulier :
      \[
      \begin{array}{rcl@{\quad}>{\it}R{4cm}}
        (f\circ f)(x) & = & f(f(x)) \ = \ f\left(x - \Big(\Sum{i=1}{n}
          x_i \Big) \cdot v\right)
        \\[.2cm]
        & = & f(x) - \Big(\Sum{i=1}{n} x_i \Big) \cdot f(v)
        & (par linéarité de $f$)
        \nl
        \nl[-.2cm]
        & = & f(x) - \Big(\Sum{i=1}{n} x_i \Big) \cdot \left( v -
          \Big(\Sum{i=1}{n} v_i \Big) \cdot v \right)
        \nl
        \nl[-.2cm]
        & = & f(x) - \Big(\Sum{i=1}{n} x_i \Big) \cdot \left( v -
          v \right)
        & (car $\Sum{i=1}{n} v_i = 1$)
        \nl
        \nl[-.2cm]
        & = & f(x)
%         \Sum{i=1}{n} u_i & = & \Sum{i=1}{n} \left(
%           x_i - \Big( \Sum{j=1}{n} x_j \Big) \ v_i \right)
%         \\[.6cm]
%         & = & \Sum{i=1}{n} x_i - \Sum{i=1}{n} \Big(\Sum{j=1}{n} x_j 
%         \Big) 
%         \ v_i & (par linéarité de \\ la sommation)
%         \nl
%         \nl[-.2cm]
%         & = & \Sum{i=1}{n} x_i - \Big(\Sum{j=1}{n} x_j \Big) 
%         \Big( \Sum{i=1}{n} v_i \Big)
%         \\[.6cm]
%         & = & \bcancel{\Sum{i=1}{n} x_i} - \bcancel{\Big( \Sum{j=1}{n} 
%           x_j \Big)} \times 1
%         & (car, d'après l'énoncé, $\Sum{i=1}{n} v_i=1$)
%         \nl
%         \nl[-.2cm]
%         & = & 0
      \end{array}
      \]
%       Enfin :
%       \[
%       (f\circ f)(x) \ = \ f(f(x)) \ = \ f(u) \ = \ u -
%       \Big(\Sum{i=1}{n} u_i \Big) \cdot v \ = \ u \ = \ f(x)
%       \]
      \conc{On en déduit : $f\circ f=f$.}      
      \begin{remark}~\\
        Si $E$ est un espace vectoriel, un endomorphisme $f \in
        \LL{E}$ qui vérifie $f \circ f = f$ est appelé projecteur de
        $E$. Dans cet exercice, on va étudier quelques propriétés
        classiques (mais hors programme) des projecteurs.
      \end{remark}~\\[-1.4cm]
    \end{proof}
  \end{noliste}
 
 \item Déterminer le spectre de $f$.
 
   \begin{proof}~
     \begin{noliste}{$\sbullet$}
     \item D'après la question précédente : $f\circ
       f-f=0_{\LL{\R^n}}$.\\[.1cm]
       Donc $Q(X) = X^2-X = X(X-1)$ est un polynôme annulateur de
       $f$.%
       \conc{Ainsi : $\spc(f) \subset \{\mbox{racines de $Q$}\} =
         \{0,1\}$.}

     \item On remarque alors :
       \[
       f(v) = v - \Big(\Sum{i=1}{n} v_i \Big) \cdot v = v - 1 \cdot v
       = 0
       \]
       Or $v\neq 0_{\R^n}$ car $\Sum{i=1}{n} v_i =1$. Donc :
       $\left\{
         \begin{array}{l}
           v \neq 0_{\R^n}\\[.1cm]
           f(v)=0_{\R^n}
         \end{array}
       \right.$.\\[.2cm]
       On en déduit que $v$ est un vecteur propre de $f$ associé à la
       valeur propre $0$. %
       \conc{Ainsi, $0$ est valeur propre de $f$.}

     \item D'après ce qui précède, pour tout $x \in \R^n$, $f\big(
       f(x) \big) = f(x) = 1 \cdot f(x)$.\\
       Ainsi, tout vecteur $f(x)$ {\bf non nul} est vecteur propre
       associé à la valeur propre $1$.\\
       Un tel élément existe forcément. Pour le démontrer, remarquons
       tout d'abord que comme l'endomorphisme $f$ n'est pas
       l'application nulle alors : $\im(f) \neq \{0_{\R^n}\}$.\\
       Ainsi il existe $u \in \im(f)$ tel que $u \neq
       0_{\R^n}$. Et d'après ce qui précède : $f(u) = 1 \cdot u$.\\[.1cm]
       {\it (on peut de nouveau le détailler. Comme : $u\in \im(f)$,
         il existe $x \in \R^n$ tel que $u = f(x)$ et $f(u) = f(f(x))
         = (f\circ f)(x) = f(x) =u$)}%
       \conc{Ainsi, $1$ est valeur propre de $f$.}
     \end{noliste}
     \conc{Finalement : $\spc(f) = \{0,1\}$.}


     \newpage


     \begin{remark}~
       \begin{noliste}{$\sbullet$}
       \item L'application $f$ est définie à l'aide du vecteur
         $v$. Penser à déterminer $f(v)$, dès la lecture de cette
         définition, est un bon réflexe. Il faut alors penser à
         utiliser ce calcul au bon moment : pour démontrer $0$ est
         bien une valeur propre de $f$ puisque $v$ en est un vecteur
         propre.
         
       \item Il est plus difficile de trouver un vecteur propre
         associé à la valeur propre $1$.\\
         On se sert pour cette question du résultat suivant :
         \[
         \forall u \in \im(f), \ f(u) = u
         \]
         Cette relation est vraie pour tout projecteur $f$ et se
         démontre grâce à la relation : $f \circ f=f$ (c'est ce qui a
         été fait dans le corrigé de cette question et dans la
         suivante).
         
       \item Il est important de penser à la notion de polynôme
         annulateur dès que l'énoncé met en jeu des puissances de
         matrices ou des itérées d'applications linéaires.\\
         Ce réflexe permet de démontrer l'étape :
         \[
         \spc(f) \subset \{0,1\}
         \]
         S'il est évidemment préférable d'écrire toutes les étapes de
         démonstration d'une question, chacune d'entre elles rapporte
         des points. Il est donc vivement conseillé d'écrire la
         première étape de démonstration quitte à admettre celles qui
         suivent.
       \end{noliste}
     \end{remark}~\\[-1.4cm]
   \end{proof}
   
 \item 
 \begin{noliste}{a)}
  \setlength{\itemsep}{1mm}
  \item Montrer que le vecteur $y$ appartient à l'image de $f$, 
  notée $\im(f)$, si et seulement si $f(y)=y$.
  
  \begin{proof}~\\
   Soit $y\in\R^n$.
   \begin{liste}{$\quad$}
   \item[$(\Rightarrow)$] 
     % Montrons l'implication : $(y \in \im(f)) \ \Rightarrow
     % \ (f(y)=y)$.\\[.1cm]
     Supposons que $y\in \im(f)$. Alors il existe $x\in \R^n$ tel que
     : $y=f(x)$. \\
     On obtient alors :
     \[
     f(y) = f(f(x)) = (f\circ f)(x) = f(x) =y
     \]
   \item[$(\Leftarrow)$] % Montrons l'implication : $(f(y)=y) \ 
     % \Rightarrow
     % \ (y\in\im(f))$.\\[.1cm]
     Supposons que : $f(y) = y$.\\
     Alors, en posant : $x = y$, on exhibe bien $x\in \R^n$ tel
     que $y = f(x)$.\\
     Donc $y\in \im(f)$.~\\[-.8cm]
   \end{liste}
   \conc{$\forall y\in \R^n$, \ $(y\in \im(f)) \ \Leftrightarrow \
     (f(y)=y)$}%~\\[-1.4cm]
   \begin{remark}~\\
     Cette démonstration n'utilise pas la définition de $f$ mais
     seulement la propriété : $f\circ f = f$.\\
     C'est donc un résultat général sur les projecteurs que l'on
     montre ici.
   \end{remark}~\\[-1.4cm]
  \end{proof}

\item Montrer que la dimension de $\im(f)$ est inférieure ou égale à
  $n-1$.
  
  \begin{proof}~
    \begin{noliste}{$\sbullet$}
    \item D'après la question \itbf{2.}, $0$ est valeur propre de $f$.
      Donc : $\kr(f) \neq \{0_{\R^n}\}$.\\
      On en déduit : $\dim(\kr(f)) \geq 1$.
      
    \item Ainsi, d'après le théorème du rang :
      \[
      \begin{array}{ccl}
        \dim(\R^n) & = & \dim(\kr(f)) + \dim(\im(f)) \ \geq \ 1 + 
        \dim(\im(f))
        \\[.1cm]
        \shortparallel 
        \\%[.2cm]
        n
      \end{array}      
      \]
    \end{noliste}
    \conc{Comme $n \geq 1 + \dim(\im(f))$, on a bien : $\dim(\im(f))
      \leq n-1$.}~\\[-1.2cm]
  \end{proof}
    

\newpage


\item Montrer que pour tout $i \in \llb 1,n-1 \rrb$, on a :
  $(e_i-e_{i+1})\in \im(f)$.
  
  \begin{proof}~\\
   Soit $i\in \llb 1, n-1 \rrb$.
   \begin{noliste}{$\sbullet$}
   \item Pour tout $j \in \llb 1, n \rrb$, notons : $e_j = (e_j^1,
     e_j^2, \ldots, e_j^n)$. Alors, par définition de $e_j$ :
     \[
     \forall k \in \llb 1, n \rrb, \ e_j^k = \left\{
       \begin{array}{cl}
         1 & \mbox{ si $k=j$}\\
         0 & \mbox{ sinon}
       \end{array}
     \right.
     \]
     Ainsi : $\Sum{k=1}{n} e_j^k = 1$.
     
   \item On calcule alors :
     \[
     f(e_i) = e_i - 1 \cdot v = e_i - v
     \]
     De même : $f(e_{i+1}) = e_{i+1}-v$.\\
     On en déduit, par linéarité de $f$ :
     \[
     f(e_i-e_{i+1}) = f(e_i)-f(e_{i+1}) = (e_i - \bcancel{v}) - (e_{i+1}
     - \bcancel{v}) = e_i - e_{i+1}
     \]
     D'après la question \itbf{3.a)}, on en déduit : 
     $(e_i-e_{i+1}) \in \im(f)$.
   \end{noliste}
   \conc{$\forall i \in \llb 1, n-1 \rrb$, $(e_i-e_{i+1}) \in 
     \im(f)$}~\\[-1.2cm]
  \end{proof}

  \item En déduire une base et la dimension  de $\im(f)$. Quel est le 
  rang de $f$ ?
  
  \begin{proof}~
    \begin{noliste}{$\sbullet$}
    \item Montrons que la famille $(e_1-e_2, e_2-e_3, \ldots, 
      e_{n-1} - e_n)$ est une famille libre de $\im(f)$.
      \begin{noliste}{$\stimes$}
      \item D'après la question précédente :
        \[
        \forall i \in \llb 1, n-1 \rrb, \ (e_i - e_{i+1}) \in \im(f)
        \]
      
      \item Démoontrons maintenant que cette famille est libre. \\
        Soit $(\lambda_1, \lambda_2, \ldots, \lambda_{n-1}) \in
        \R^{n-1}$. Supposons :
        \[
        \lambda_1 \cdot (e_1 - e_2) + \lambda_2 \cdot (e_2 - e_3)
        + \cdots + \lambda_{n-1} \cdot (e_{n-1}-e_n) = 0_{\R^n}
      \]
      On obtient alors :
      \[
       \lambda_1 \cdot e_1 + (\lambda_2 - \lambda_1) \cdot e_2 + 
       \cdots + (\lambda_{n-1} - \lambda_{n-2}) \cdot e_{n-1} +
       \lambda_{n-1} \cdot e_n = 0_{\R^n}
      \]


      % \newpage


      \noindent
      Or la famille $(e_1, \ldots, e_n)$ est une base de $\R^n$. La
      proposition précédente équivaut donc à :
      \[
      \begin{array}{rcl}
       & & \left\{
       \begin{array}{rrrrrrrrrcl}
        \lambda_1 & & & & & & & & & = & 0\\
        -\lambda_1 & + & \lambda_2 & & & & & & & = & 0\\
         & & & & \vdots & & & & & & \\
         & & & & & - & \lambda_{n-2} & + & \lambda_{n-1} & = & 0\\
         & & & & & & & & \lambda_{n-1} & = & 0
       \end{array}
       \right.
       \\[2cm]
       & \Leftrightarrow & 
        \left\{ \lambda_1 = \lambda_2 = \cdots = \lambda_{n-2} = 
        \lambda_{n-1} = 0 \right.
       \\[.2cm]
       & & \mbox{\it (par remontées successives)}
      \end{array}
      \]
      \conc{La famille $(e_1-e_2, e_2-e_3, \ldots, e_{n-1} - e_n)$ est
        une famille libre de $\im(f)$.}
     \end{noliste}
         
   \item On en déduit :
      \[
       \dim(\im(f)) \geq \Card((e_1-e_2, e_2-e_3, \ldots, 
       e_{n-1} - e_n)) = n-1
      \]
      Or, d'après la question \itbf{3.b)} : $\dim(\im(f)) \leq n-1$.
      \conc{On en déduit : $\dim(\im(f))=n-1$}
      
      \item On sait que :
      \begin{noliste}{$\stimes$}
	\item $(e_1-e_2, e_2-e_3, \ldots, 
	e_{n-1} - e_n)$ est une famille libre de $\im(f)$,
	\item $\Card((e_1-e_2, e_2-e_3, \ldots, 
	e_{n-1} - e_n))=n-1 = \dim(\im(f))$
      \end{noliste}
      \conc{On en déduit que $(e_1- e_2, e_2-e_3, \ldots, e_{n-1} - 
      e_n)$ est une base de $\im(f)$.\\[.1cm]
      De plus : $\rg(f) = \dim(\im(f))=n-1$.}
   \end{noliste}
   \begin{remark}~\\
     Les énoncés de type HEC / ESSEC se distinguent des énoncés EML /
     EDHEC par un découpage plus faible des questions qui oblige à
     prendre plus d'initiatives. Ici, la formulation de la question
     \og En déduire que \ldots \fg{} doit aider à comprendre qu'il
     s'agit de se servir du résultat précédent. En question
     précédente, on exhibe $(n-1)$ vecteurs de $\im(f)$. Il s'agit
     alors de tester si la famille constituée de ces vecteurs est une
     base de $\im(f)$.
   \end{remark}~\\[-1.4cm]
  \end{proof}

 \end{noliste}
 
 \item 
 \begin{noliste}{a)}
  \setlength{\itemsep}{1mm}
  \item Déterminer une base du noyau de $f$.
  
  \begin{proof}~
   \begin{noliste}{$\sbullet$}
    \item D'après le théorème du rang :
    \[
     \begin{array}{ccccc}
      \dim(\kr(f)) & + & \dim(\im(f)) & = & \dim(\R^n)
      \\[.2cm]
       & & \rotatebox{90}{$=$} & & \rotatebox{90}{$=$}
      \\
      & & n-1 & & n
     \end{array}
    \]
    Donc : $\dim(\kr(f))=n-(n-1)=1$.
    
    \item D'après la question \itbf{2.} :
    \begin{noliste}{$\stimes$}
      \item $v\in \kr(f)$,
      \item $v \neq 0_{\R^n}$.
    \end{noliste}
    Donc $(v)$ forme une famille libre de $\kr(f)$.

    
    % \newpage

    
    \item On obtient alors :
    \begin{noliste}{$\stimes$}
      \item $(v)$ est une famille libre de $\kr(f)$,
      \item $\Card((v)) = 1 = \dim(\kr(f))$.
    \end{noliste}
    \conc{On en déduit que $(v)$ est une base de $\kr(f)$.}~\\[-1.4cm]
   \end{noliste}
  \end{proof}

  
  \item Quels sont les sous-espaces propres de $f$ ?
  
  \begin{proof}~
   \begin{noliste}{$\sbullet$}
    \item On a déjà, d'après la question \itbf{4.a)} :
    \[
     E_0(f)=\kr(f- 0_{\R} \cdot \id) = \kr(f) 
     =\Vect{v}
    \]
    \conc{On en déduit que : $E_0(f)=\Vect{v}$.}
        
    \item Soit $y\in \R^n$. D'après la question \itbf{3.a)} :
    \[
     y \in \im(f) \ \Leftrightarrow \ f(y)=y \ \Leftrightarrow \
     y \in E_1(f)
    \]
    On en déduit que : $\im(f)=E_1(f)$.
    \conc{D'après la question \itbf{3.d)} :\\[.1cm]
    $E_1(f)= \im(f) =
    \Vect{e_1- e_2, e_2-e_3, \ldots, e_{n-1} -e_n}$}~\\[-1.2cm]
   \end{noliste}
  \end{proof}

  \item L'endomorphisme $f$ est-il diagonalisable ?
  
    \begin{proof}~\\
      D'après les questions \itbf{3.d)}, \itbf{4.a)} et \itbf{4.b)} :
   \[
    \dim(E_0(f)) + \dim(E_1(f)) = 1 + (n-1) = n = \dim(\R^n)
   \]
   \conc{On en déduit que $f$ est diagonalisable.}~\\[-1.2cm]
  \end{proof}
 \end{noliste}
 
 \item Écrire la matrice $M$ de l'endomorphisme $f$ dans la base 
  canonique de $\R^n$ et la matrice $M'$ de $f$ dans une base de 
  vecteurs propres.
  
  \begin{proof}~
   \begin{noliste}{$\sbullet$}
    \item Soit $i\in \llb 1,n \rrb$. On a déjà montré en question 
    \itbf{3.c)} :
    \[
     \begin{array}{rcl}
       f(e_i) & = & e_i - v \ = \ e_i - \Sum{j=1}{n} v_j \cdot e_j
       \\[.6cm]
       & = & (-v_1) \cdot e_1 + \cdots + (1-v_i) \cdot e_i + (- v_{i+1})
       \cdot e_{i+1} + \cdots + (- v_n) \cdot e_n
     \end{array}
    \]
    \conc{On obtient alors : $M=
    \begin{smatrix}
     1-v_1 & -v_1 & \cdots & -v_1 & -v_1\\
     -v_2 & 1-v_2 & \cdots & -v_2 & -v_2\\
     \vdots & \vdots & \ddots & \vdots & \vdots\\
     -v_{n-1} & -v_{n-1} & \cdots & 1-v_{n-1} & -v_{n-1}\\
     -v_n & -v_n & \cdots & -v_n & 1-v_n
    \end{smatrix}$}
    

  % \newpage


    \item La famille $(v,e_1- e_2, e_2-e_3, \ldots, e_{n-1} -e_n)$ est 
    une base de vecteurs propres de $f$.
    \begin{noliste}{$\stimes$}
     \item Comme $v\in E_0(f)$ : 
     \[
      f(v)=0_{\R^n} = 0 \cdot v + 0 \cdot 
      (e_1-e_2) + \cdots + 0 \cdot (e_{n-1}-e_n)
     \]
     
     \item Comme $(e_1-e_2)\in E_1(f)$ : 
     \[
      f(e_1-e_2)=e_1-e_2 = 0 \cdot 
      v + 1 \cdot (e_1-e_2) + 0 \cdot (e_2-e_3)+ \cdots + 0 \cdot 
      (e_{n-1}-e_n)
     \]
     
     \item $\cdots$
     
     \item Comme $(e_{n-1}-e_n)\in E_1(f)$ : 
     \[
      f(e_{n-1}-e_n)= 
      e_{n-1}-e_n = 0 \cdot v + 0 \cdot (e_1-e_2) + \cdots + 0 \cdot 
      (e_{n-2}-e_{n-1}) + 1 \cdot (e_{n-1}-e_n)
     \]
    \end{noliste}    
    \conc{On obtient alors :
    $
     M' = 
     \begin{smatrix}
      0 & 0 & \cdots & 0
      \\
      0 & 1 & \cdots & 0
      \\
      \vdots & \vdots & \ddots & \vdots 
      \\
      0 & 0 & \cdots & 1 
     \end{smatrix}
    $}~\\[-1.6cm]
%     \conc{On en déduit : $M' =
%     \left(
%     \begin{sarray}{c|c}
%       0 & 0_{1,n-1}
%       \\[.2cm]
%       \cline{1-2}
%       \\[-.2cm]
%       0_{n-1,1} & I_{n-1,n-1}
%     \end{sarray}
%     \right)$}~\\[-1.2cm]
   \end{noliste}
  \end{proof}

\end{noliste}




\end{document}
