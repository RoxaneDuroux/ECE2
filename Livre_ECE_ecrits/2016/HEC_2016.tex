\documentclass[11pt]{article}%
\usepackage{geometry}%
\geometry{a4paper,
  lmargin=2cm,rmargin=2cm,tmargin=2.5cm,bmargin=2.5cm}

\usepackage{array}
\usepackage{paralist}

\usepackage[svgnames, usenames, dvipsnames]{xcolor}
\xdefinecolor{RecColor}{named}{Aqua}
\xdefinecolor{IncColor}{named}{Aqua}
\xdefinecolor{ImpColor}{named}{PaleGreen}

% \usepackage{frcursive}

\usepackage{adjustbox}

%%%%%%%%%%%
\newcommand{\cRB}[1]{{\color{Red} \pmb{#1}}} %
\newcommand{\cR}[1]{{\color{Red} {#1}}} %
\newcommand{\cBB}[1]{{\color{Blue} \pmb{#1}}}
\newcommand{\cB}[1]{{\color{Blue} {#1}}}
\newcommand{\cGB}[1]{{\color{LimeGreen} \pmb{#1}}}
\newcommand{\cG}[1]{{\color{LimeGreen} {#1}}}

%%%%%%%%%%

\usepackage{diagbox} %
\usepackage{colortbl} %
\usepackage{multirow} %
\usepackage{pgf} %
\usepackage{environ} %
\usepackage{fancybox} %
\usepackage{textcomp} %
\usepackage{marvosym} %

%%%%%%%%%% pour qu'une cellcolor ne recouvre pas le trait du tableau
\usepackage{hhline}%

\usepackage{pgfplots}
\pgfplotsset{compat=1.10}
\usepgfplotslibrary{patchplots}
\usepgfplotslibrary{fillbetween}
\usepackage{tikz,tkz-tab}
\usepackage{ifthen}
\usepackage{calc}
\usetikzlibrary{calc,decorations.pathreplacing,arrows,positioning} 
\usetikzlibrary{fit,shapes,backgrounds}
\usepackage[nomessages]{fp}% http://ctan.org/pkg/fp

\usetikzlibrary{matrix,arrows,decorations.pathmorphing,
  decorations.pathreplacing} 

\newcommand{\myunit}{1 cm}
\tikzset{
    node style sp/.style={draw,circle,minimum size=\myunit},
    node style ge/.style={circle,minimum size=\myunit},
    arrow style mul/.style={draw,sloped,midway,fill=white},
    arrow style plus/.style={midway,sloped,fill=white},
}

%%%%%%%%%%%%%%
%%%%% écrire des inférieur égal ou supérieur égal avec typographie
%%%%% francaise
%%%%%%%%%%%%%

\renewcommand{\geq}{\geqslant}
\renewcommand{\leq}{\leqslant}
\renewcommand{\emptyset}{\varnothing}

\newcommand{\Leq}{\leqslant}
\newcommand{\Geq}{\geqslant}

%%%%%%%%%%%%%%
%%%%% Macro Celia
%%%%%%%%%%%%%

\newcommand{\ff}[2]{\left[#1, #2\right]} %
\newcommand{\fo}[2]{\left[#1, #2\right[} %
\newcommand{\of}[2]{\left]#1, #2\right]} %
\newcommand{\soo}[2]{\left]#1, #2\right[} %
\newcommand{\abs}[1]{\left|#1\right|} %
\newcommand{\Ent}[1]{\left\lfloor #1 \right\rfloor} %


%%%%%%%%%%%%%%
%%%%% tikz : comment dessiner un "oeil"
%%%%%%%%%%%%%

\newcommand{\eye}[4]% size, x, y, rotation
{ \draw[rotate around={#4:(#2,#3)}] (#2,#3) -- ++(-.5*55:#1) (#2,#3)
  -- ++(.5*55:#1); \draw (#2,#3) ++(#4+55:.75*#1) arc
  (#4+55:#4-55:.75*#1);
  % IRIS
  \draw[fill=gray] (#2,#3) ++(#4+55/3:.75*#1) arc
  (#4+180-55:#4+180+55:.28*#1);
  % PUPIL, a filled arc
  \draw[fill=black] (#2,#3) ++(#4+55/3:.75*#1) arc
  (#4+55/3:#4-55/3:.75*#1);%
}


%%%%%%%%%%
%% discontinuité fonction
\newcommand\pointg[2]{%
  \draw[color = red, very thick] (#1+0.15, #2-.04)--(#1, #2-.04)--(#1,
  #2+.04)--(#1+0.15, #2+.04);%
}%

\newcommand\pointd[2]{%
  \draw[color = red, very thick] (#1-0.15, #2+.04)--(#1, #2+.04)--(#1,
  #2-.04)--(#1-0.15, #2-.04);%
}%

%%%%%%%%%%
%%% 1 : position abscisse, 2 : position ordonnée, 3 : taille, 4 : couleur
%%%%%%%%%%
% \newcommand\pointG[4]{%
%   \draw[color = #4, very thick] (#1+#3, #2-(#3/3.75))--(#1,
%   #2-(#3/3.75))--(#1, #2+(#3/3.75))--(#1+#3, #2+(#3/3.75)) %
% }%

\newcommand\pointG[4]{%
  \draw[color = #4, very thick] ({#1+#3/3.75}, {#2-#3})--(#1,
  {#2-#3})--(#1, {#2+#3})--({#1+#3/3.75}, {#2+#3}) %
}%

\newcommand\pointD[4]{%
  \draw[color = #4, very thick] ({#1-#3/3.75}, {#2+#3})--(#1,
  {#2+#3})--(#1, {#2-#3})--({#1-#3/3.75}, {#2-#3}) %
}%

\newcommand\spointG[4]{%
  \draw[color = #4, very thick] ({#1+#3/1.75}, {#2-#3})--(#1,
  {#2-#3})--(#1, {#2+#3})--({#1+#3/1.75}, {#2+#3}) %
}%

\newcommand\spointD[4]{%
  \draw[color = #4, very thick] ({#1-#3/2}, {#2+#3})--(#1,
  {#2+#3})--(#1, {#2-#3})--({#1-#3/2}, {#2-#3}) %
}%

%%%%%%%%%%

\newcommand{\Pb}{\mathtt{P}}

%%%%%%%%%%%%%%%
%%% Pour citer un précédent item
%%%%%%%%%%%%%%%
\newcommand{\itbf}[1]{{\small \bf \textit{#1}}}


%%%%%%%%%%%%%%%
%%% Quelques couleurs
%%%%%%%%%%%%%%%

\xdefinecolor{cancelcolor}{named}{Red}
\xdefinecolor{intI}{named}{ProcessBlue}
\xdefinecolor{intJ}{named}{ForestGreen}

%%%%%%%%%%%%%%%
%%%%%%%%%%%%%%%
% barrer du texte
\usetikzlibrary{shapes.misc}

\makeatletter
% \definecolor{cancelcolor}{rgb}{0.127,0.372,0.987}
\newcommand{\tikz@bcancel}[1]{%
  \begin{tikzpicture}[baseline=(textbox.base), inner sep=0pt]
    \node[strike out, draw] (textbox) {#1}[thick, color=cancelcolor];
    \useasboundingbox (textbox);
  \end{tikzpicture}%
}
\newcommand{\bcancel}[1]{%
  \relax\ifmmode
    \mathchoice{\tikz@bcancel{$\displaystyle#1$}}
               {\tikz@bcancel{$\textstyle#1$}}
               {\tikz@bcancel{$\scriptstyle#1$}}
               {\tikz@bcancel{$\scriptscriptstyle#1$}}
  \else
    \tikz@bcancel{\strut#1}%
  \fi
}
\newcommand{\tikz@xcancel}[1]{%
  \begin{tikzpicture}[baseline=(textbox.base),inner sep=0pt]
  \node[cross out,draw] (textbox) {#1}[thick, color=cancelcolor];
  \useasboundingbox (textbox);
  \end{tikzpicture}%
}
\newcommand{\xcancel}[1]{%
  \relax\ifmmode
    \mathchoice{\tikz@xcancel{$\displaystyle#1$}}
               {\tikz@xcancel{$\textstyle#1$}}
               {\tikz@xcancel{$\scriptstyle#1$}}
               {\tikz@xcancel{$\scriptscriptstyle#1$}}
  \else
    \tikz@xcancel{\strut#1}%
  \fi
}
\makeatother

\newcommand{\xcancelRA}{\xcancel{\rule[-.15cm]{0cm}{.5cm} \Rightarrow
    \rule[-.15cm]{0cm}{.5cm}}}

%%%%%%%%%%%%%%%%%%%%%%%%%%%%%%%%%%%
%%%%%%%%%%%%%%%%%%%%%%%%%%%%%%%%%%%

\newcommand{\vide}{\multicolumn{1}{c}{}}

%%%%%%%%%%%%%%%%%%%%%%%%%%%%%%%%%%%
%%%%%%%%%%%%%%%%%%%%%%%%%%%%%%%%%%%


\usepackage{multicol}
% \usepackage[latin1]{inputenc}
% \usepackage[T1]{fontenc}
\usepackage[utf8]{inputenc}
\usepackage[T1]{fontenc}
\usepackage[normalem]{ulem}
\usepackage[french]{babel}

\usepackage{url}    
\usepackage{hyperref}
\hypersetup{
  backref=true,
  pagebackref=true,
  hyperindex=true,
  colorlinks=true,
  breaklinks=true,
  urlcolor=blue,
  linkcolor=black,
  %%%%%%%%
  % ATTENTION : red changé en black pour le Livre !
  %%%%%%%%
  bookmarks=true,
  bookmarksopen=true
}

%%%%%%%%%%%%%%%%%%%%%%%%%%%%%%%%%%%%%%%%%%%
%% Pour faire des traits diagonaux dans les tableaux
%% Nécessite slashbox.sty
%\usepackage{slashbox}

\usepackage{tipa}
\usepackage{verbatim,listings}
\usepackage{graphicx}
\usepackage{fancyhdr}
\usepackage{mathrsfs}
\usepackage{pifont}
\usepackage{tablists}
\usepackage{dsfont,amsfonts,amssymb,amsmath,amsthm,stmaryrd,upgreek,manfnt}
\usepackage{enumerate}

%\newcolumntype{M}[1]{p{#1}}
\newcolumntype{C}[1]{>{\centering}m{#1}}
\newcolumntype{R}[1]{>{\raggedright}m{#1}}
\newcolumntype{L}[1]{>{\raggedleft}m{#1}}
\newcolumntype{P}[1]{>{\raggedright}p{#1}}
\newcolumntype{B}[1]{>{\raggedright}b{#1}}
\newcolumntype{Q}[1]{>{\raggedright}t{#1}}

\newcommand{\alias}[2]{
\providecommand{#1}{}
\renewcommand{#1}{#2}
}
\alias{\R}{\mathbb{R}}
\alias{\N}{\mathbb{N}}
\alias{\Z}{\mathbb{Z}}
\alias{\Q}{\mathbb{Q}}
\alias{\C}{\mathbb{C}}
\alias{\K}{\mathbb{K}}

%%%%%%%%%%%%
%% rendre +infty et -infty plus petits
%%%%%%%%%%%%
\newcommand{\sinfty}{{\scriptstyle \infty}}

%%%%%%%%%%%%%%%%%%%%%%%%%%%%%%%
%%%%% macros TP Scilab %%%%%%%%
\newcommand{\Scilab}{\textbf{Scilab}} %
\newcommand{\Scinotes}{\textbf{SciNotes}} %
\newcommand{\faire}{\noindent $\blacktriangleright$ } %
\newcommand{\fitem}{\scalebox{.8}{$\blacktriangleright$}} %
\newcommand{\entree}{{\small\texttt{ENTRÉE}}} %
\newcommand{\tab}{{\small\texttt{TAB}}} %
\newcommand{\mt}[1]{\mathtt{#1}} %
% guillemets droits

\newcommand{\ttq}{\textquotesingle} %

\newcommand{\reponse}[1]{\longboxed{
    \begin{array}{C{0.9\textwidth}}
      \nl[#1]
    \end{array}
  }} %

\newcommand{\reponseR}[1]{\longboxed{
    \begin{array}{R{0.9\textwidth}}
      #1
    \end{array}
  }} %

\newcommand{\reponseC}[1]{\longboxed{
    \begin{array}{C{0.9\textwidth}}
      #1
    \end{array}
  }} %

\colorlet{pyfunction}{Blue}
\colorlet{pyCle}{Magenta}
\colorlet{pycomment}{LimeGreen}
\colorlet{pydoc}{Cyan}
% \colorlet{SansCo}{white}
% \colorlet{AvecCo}{black}

\newcommand{\visible}[1]{{\color{ASCo}\colorlet{pydoc}{pyDo}\colorlet{pycomment}{pyCo}\colorlet{pyfunction}{pyF}\colorlet{pyCle}{pyC}\colorlet{function}{sciFun}\colorlet{var}{sciVar}\colorlet{if}{sciIf}\colorlet{comment}{sciComment}#1}} %

%%%% à changer ????
\newcommand{\invisible}[1]{{\color{ASCo}\colorlet{pydoc}{pyDo}\colorlet{pycomment}{pyCo}\colorlet{pyfunction}{pyF}\colorlet{pyCle}{pyC}\colorlet{function}{sciFun}\colorlet{var}{sciVar}\colorlet{if}{sciIf}\colorlet{comment}{sciComment}#1}} %

\newcommand{\invisibleCol}[2]{{\color{#1}#2}} %

\NewEnviron{solution} %
{ %
  \Boxed{
    \begin{array}{>{\color{ASCo}} R{0.9\textwidth}}
      \colorlet{pycomment}{pyCo}
      \colorlet{pydoc}{pyDo}
      \colorlet{pyfunction}{pyF}
      \colorlet{pyCle}{pyC}
      \colorlet{function}{sciFun}
      \colorlet{var}{sciVar}
      \colorlet{if}{sciIf}
      \colorlet{comment}{sciComment}
      \BODY
    \end{array}
  } %
} %

\NewEnviron{solutionC} %
{ %
  \Boxed{
    \begin{array}{>{\color{ASCo}} C{0.9\textwidth}}
      \colorlet{pycomment}{pyCo}
      \colorlet{pydoc}{pyDo}
      \colorlet{pyfunction}{pyF}
      \colorlet{pyCle}{pyC}
      \colorlet{function}{sciFun}
      \colorlet{var}{sciVar}
      \colorlet{if}{sciIf}
      \colorlet{comment}{sciComment}
      \BODY
    \end{array}
  } %
} %

\newcommand{\invite}{--\!\!>} %

%%%%% nouvel environnement tabular pour retour console %%%%
\colorlet{ConsoleColor}{Black!12}
\colorlet{function}{Red}
\colorlet{var}{Maroon}
\colorlet{if}{Magenta}
\colorlet{comment}{LimeGreen}

\newcommand{\tcVar}[1]{\textcolor{var}{\bf \small #1}} %
\newcommand{\tcFun}[1]{\textcolor{function}{#1}} %
\newcommand{\tcIf}[1]{\textcolor{if}{#1}} %
\newcommand{\tcFor}[1]{\textcolor{if}{#1}} %

\newcommand{\moins}{\!\!\!\!\!\!- }
\newcommand{\espn}{\!\!\!\!\!\!}

\usepackage{booktabs,varwidth} \newsavebox\TBox
\newenvironment{console}
{\begin{lrbox}{\TBox}\varwidth{\linewidth}
    \tabular{>{\tt\small}R{0.84\textwidth}}
    \nl[-.4cm]} {\endtabular\endvarwidth\end{lrbox}%
  \fboxsep=1pt\colorbox{ConsoleColor}{\usebox\TBox}}

\newcommand{\lInv}[1]{%
  $\invite$ #1} %

\newcommand{\lAns}[1]{%
  \qquad ans \ = \nl %
  \qquad \qquad #1} %

\newcommand{\lVar}[2]{%
  \qquad #1 \ = \nl %
  \qquad \qquad #2} %

\newcommand{\lDisp}[1]{%
  #1 %
} %

\newcommand{\ligne}[1]{\underline{\small \tt #1}} %

\newcommand{\ligneAns}[2]{%
  $\invite$ #1 \nl %
  \qquad ans \ = \nl %
  \qquad \qquad #2} %

\newcommand{\ligneVar}[3]{%
  $\invite$ #1 \nl %
  \qquad #2 \ = \nl %
  \qquad \qquad #3} %

\newcommand{\ligneErr}[3]{%
  $\invite$ #1 \nl %
  \quad !-{-}error #2 \nl %
  #3} %
%%%%%%%%%%%%%%%%%%%%%% 

\newcommand{\bs}[1]{\boldsymbol{#1}} %
\newcommand{\nll}{\nl[.4cm]} %
\newcommand{\nle}{\nl[.2cm]} %
%% opérateur puissance copiant l'affichage Scilab
%\newcommand{\puis}{\!\!\!~^{\scriptscriptstyle\pmb{\wedge}}}
\newcommand{\puis}{\mbox{$\hspace{-.1cm}~^{\scriptscriptstyle\pmb{\wedge}}
    \hspace{0.05cm}$}} %
\newcommand{\pointpuis}{.\mbox{$\hspace{-.15cm}~^{\scriptscriptstyle\pmb{\wedge}}$}} %
\newcommand{\Sfois}{\mbox{$\mt{\star}$}} %

%%%%% nouvel environnement tabular pour les encadrés Scilab %%%%
\newenvironment{encadre}
{\begin{lrbox}{\TBox}\varwidth{\linewidth}
    \tabular{>{\tt\small}C{0.1\textwidth}>{\small}R{0.7\textwidth}}}
  {\endtabular\endvarwidth\end{lrbox}%
  \fboxsep=1pt\longboxed{\usebox\TBox}}

\newenvironment{encadreL}
{\begin{lrbox}{\TBox}\varwidth{\linewidth}
    \tabular{>{\tt\small}C{0.25\textwidth}>{\small}R{0.6\textwidth}}}
  {\endtabular\endvarwidth\end{lrbox}%
  \fboxsep=1pt\longboxed{\usebox\TBox}}

\newenvironment{encadreF}
{\begin{lrbox}{\TBox}\varwidth{\linewidth}
    \tabular{>{\tt\small}C{0.2\textwidth}>{\small}R{0.70\textwidth}}}
  {\endtabular\endvarwidth\end{lrbox}%
  \fboxsep=1pt\longboxed{\usebox\TBox}}

\newenvironment{encadreLL}[2]
{\begin{lrbox}{\TBox}\varwidth{\linewidth}
    \tabular{>{\tt\small}C{#1\textwidth}>{\small}R{#2\textwidth}}}
  {\endtabular\endvarwidth\end{lrbox}%
  \fboxsep=1pt\longboxed{\usebox\TBox}}

%%%%% nouvel environnement tabular pour les script et fonctions %%%%
\newcommand{\commentaireDL}[1]{\multicolumn{1}{l}{\it
    \textcolor{comment}{$\slash\slash$ #1}}}

\newcommand{\commentaire}[1]{{\textcolor{comment}{$\slash\slash$ #1}}}

\newcounter{cptcol}

\newcommand{\nocount}{\multicolumn{1}{c}{}}

\newcommand{\sciNo}[1]{{\small \underbar #1}}

\NewEnviron{scilab}{ %
  \setcounter{cptcol}{0}
  \begin{center}
    \longboxed{
      \begin{tabular}{>{\stepcounter{cptcol}{\tiny \underbar
              \thecptcol}}c>{\tt}l}
        \BODY
      \end{tabular}
    }
  \end{center}
}

\NewEnviron{scilabNC}{ %
  \begin{center}
    \longboxed{
      \begin{tabular}{>{\tt}l} %
          \BODY
      \end{tabular}
    }
  \end{center}
}

\NewEnviron{scilabC}[1]{ %
  \setcounter{cptcol}{#1}
  \begin{center}
    \longboxed{
      \begin{tabular}{>{\stepcounter{cptcol}{\tiny \underbar
              \thecptcol}}c>{\tt}l}
        \BODY
      \end{tabular}
    }
  \end{center}
}

\newcommand{\scisol}[1]{ %
  \setcounter{cptcol}{0}
  \longboxed{
    \begin{tabular}{>{\stepcounter{cptcol}{\tiny \underbar
            \thecptcol}}c>{\tt}l}
      #1
    \end{tabular}
  }
}

\newcommand{\scisolNC}[1]{ %
  \longboxed{
    \begin{tabular}{>{\tt}l}
      #1
    \end{tabular}
  }
}

\newcommand{\scisolC}[2]{ %
  \setcounter{cptcol}{#1}
  \longboxed{
    \begin{tabular}{>{\stepcounter{cptcol}{\tiny \underbar
            \thecptcol}}c>{\tt}l}
      #2
    \end{tabular}
  }
}

\NewEnviron{syntaxe}{ %
  % \fcolorbox{black}{Yellow!20}{\setlength{\fboxsep}{3mm}
  \shadowbox{
    \setlength{\fboxsep}{3mm}
    \begin{tabular}{>{\tt}l}
      \BODY
    \end{tabular}
  }
}

%%%%% fin macros TP Scilab %%%%%%%%
%%%%%%%%%%%%%%%%%%%%%%%%%%%%%%%%%%%

%%%%%%%%%%%%%%%%%%%%%%%%%%%%%%%%%%%
%%%%% TP Python - listings %%%%%%%%
%%%%%%%%%%%%%%%%%%%%%%%%%%%%%%%%%%%
\newcommand{\Python}{\textbf{Python}} %

\lstset{% general command to set parameter(s)
basicstyle=\ttfamily\small, % print whole listing small
keywordstyle=\color{blue}\bfseries\underbar,
%% underlined bold black keywords
frame=lines,
xleftmargin=10mm,
numbers=left,
numberstyle=\tiny\underbar,
numbersep=10pt,
%identifierstyle=, % nothing happens
commentstyle=\color{green}, % white comments
%%stringstyle=\ttfamily, % typewriter type for strings
showstringspaces=false}

\newcommand{\pysolCpt}[2]{
  \setcounter{cptcol}{#1}
  \longboxed{
    \begin{tabular}{>{\stepcounter{cptcol}{\tiny \underbar
            \thecptcol}}c>{\tt}l}
        #2
      \end{tabular}
    }
} %

\newcommand{\pysol}[1]{
  \setcounter{cptcol}{0}
  \longboxed{
    \begin{tabular}{>{\stepcounter{cptcol}{\tiny \underbar
            \thecptcol}}c>{\tt}l}
        #1
      \end{tabular}
    }
} %

% \usepackage[labelsep=endash]{caption}

% avec un caption
\NewEnviron{pythonCap}[1]{ %
  \renewcommand{\tablename}{Programme}
  \setcounter{cptcol}{0}
  \begin{center}
    \longboxed{
      \begin{tabular}{>{\stepcounter{cptcol}{\tiny \underbar
              \thecptcol}}c>{\tt}l}
        \BODY
      \end{tabular}
    }
    \captionof{table}{#1}
  \end{center}
}

\NewEnviron{python}{ %
  \setcounter{cptcol}{0}
  \begin{center}
    \longboxed{
      \begin{tabular}{>{\stepcounter{cptcol}{\tiny \underbar
              \thecptcol}}c>{\tt}l}
        \BODY
      \end{tabular}
    }
  \end{center}
}

\newcommand{\pyVar}[1]{\textcolor{var}{\bf \small #1}} %
\newcommand{\pyFun}[1]{\textcolor{pyfunction}{#1}} %
\newcommand{\pyCle}[1]{\textcolor{pyCle}{#1}} %
\newcommand{\pyImp}[1]{{\bf #1}} %

%%%%% commentaire python %%%%
\newcommand{\pyComDL}[1]{\multicolumn{1}{l}{\textcolor{pycomment}{\#
      #1}}}

\newcommand{\pyCom}[1]{{\textcolor{pycomment}{\# #1}}}
\newcommand{\pyDoc}[1]{{\textcolor{pydoc}{#1}}}

\newcommand{\pyNo}[1]{{\small \underbar #1}}

%%%%%%%%%%%%%%%%%%%%%%%%%%%%%%%%%%%
%%%%%% Système linéaire paramétré : écrire les opérations au-dessus
%%%%%% d'un symbole équivalent
%%%%%%%%%%%%%%%%%%%%%%%%%%%%%%%%%%%

\usepackage{systeme}

\NewEnviron{arrayEq}{ %
  \stackrel{\scalebox{.6}{$
      \begin{array}{l} 
        \BODY \\[.1cm]
      \end{array}$}
  }{\Longleftrightarrow}
}

\NewEnviron{arrayEg}{ %
  \stackrel{\scalebox{.6}{$
      \begin{array}{l} 
        \BODY \\[.1cm]
      \end{array}$}
  }{=}
}

\NewEnviron{operationEq}{ %
  \scalebox{.6}{$
    \begin{array}{l} 
      \scalebox{1.6}{$\mbox{Opérations :}$} \\[.2cm]
      \BODY \\[.1cm]
    \end{array}$}
}

% \NewEnviron{arraySys}[1]{ %
%   \sysdelim\{.\systeme[#1]{ %
%     \BODY %
%   } %
% }

%%%%%

%%%%%%%%%%
%%%%%%%%%% ESSAI
\newlength\fboxseph
\newlength\fboxsepva
\newlength\fboxsepvb

\setlength\fboxsepva{0.2cm}
\setlength\fboxsepvb{0.2cm}
\setlength\fboxseph{0.2cm}

\makeatletter

\def\longboxed#1{\leavevmode\setbox\@tempboxa\hbox{\color@begingroup%
\kern\fboxseph{\m@th$\displaystyle #1 $}\kern\fboxseph%
\color@endgroup }\my@frameb@x\relax}

\def\my@frameb@x#1{%
  \@tempdima\fboxrule \advance\@tempdima \fboxsepva \advance\@tempdima
  \dp\@tempboxa\hbox {%
    \lower \@tempdima \hbox {%
      \vbox {\hrule\@height\fboxrule \hbox{\vrule\@width\fboxrule #1
          \vbox{%
            \vskip\fboxsepva \box\@tempboxa \vskip\fboxsepvb}#1
          \vrule\@width\fboxrule }%
        \hrule \@height \fboxrule }}}}

\newcommand{\boxedhv}[3]{\setlength\fboxseph{#1cm}
  \setlength\fboxsepva{#2cm}\setlength\fboxsepvb{#2cm}\longboxed{#3}}

\newcommand{\boxedhvv}[4]{\setlength\fboxseph{#1cm}
  \setlength\fboxsepva{#2cm}\setlength\fboxsepvb{#3cm}\longboxed{#4}}

\newcommand{\Boxed}[1]{{\setlength\fboxseph{0.2cm}
  \setlength\fboxsepva{0.2cm}\setlength\fboxsepvb{0.2cm}\longboxed{#1}}}

\newcommand{\mBoxed}[1]{{\setlength\fboxseph{0.2cm}
  \setlength\fboxsepva{0.2cm}\setlength\fboxsepvb{0.2cm}\longboxed{\mbox{#1}}}}

\newcommand{\mboxed}[1]{{\setlength\fboxseph{0.2cm}
  \setlength\fboxsepva{0.2cm}\setlength\fboxsepvb{0.2cm}\boxed{\mbox{#1}}}}

\newsavebox{\fmbox}
\newenvironment{fmpage}[1]
     {\begin{lrbox}{\fmbox}\begin{minipage}{#1}}
     {\end{minipage}\end{lrbox}\fbox{\usebox{\fmbox}}}

%%%%%%%%%%
%%%%%%%%%%

\DeclareMathOperator{\ch}{ch}
\DeclareMathOperator{\sh}{sh}

%%%%%%%%%%
%%%%%%%%%%

\newcommand{\norme}[1]{\Vert #1 \Vert}

%\newcommand*\widefbox[1]{\fbox{\hspace{2em}#1\hspace{2em}}}

\newcommand{\nl}{\tabularnewline}

\newcommand{\hand}{\noindent\ding{43}\ }
\newcommand{\ie}{\textit{i.e. }}
\newcommand{\cf}{\textit{cf }}

\newcommand{\Card}{\operatorname{Card}}

\newcommand{\aire}{\mathcal{A}}

\newcommand{\LL}[1]{\mathscr{L}(#1)} %
\newcommand{\B}{\mathscr{B}} %
\newcommand{\Bc}[1]{B_{#1}} %
\newcommand{\M}[1]{\mathscr{M}_{#1}(\mathbb{R})}

\DeclareMathOperator{\im}{Im}
\DeclareMathOperator{\kr}{Ker}
\DeclareMathOperator{\rg}{rg}
\DeclareMathOperator{\spc}{Sp}
\DeclareMathOperator{\sgn}{sgn}
\DeclareMathOperator{\supp}{Supp}

\newcommand{\Mat}{{\rm{Mat}}}
\newcommand{\Vect}[1]{{\rm{Vect}}\left(#1\right)}

\newenvironment{smatrix}{%
  \begin{adjustbox}{width=.9\width}
    $
    \begin{pmatrix}
    }{%      
    \end{pmatrix}
    $
  \end{adjustbox}
}

\newenvironment{sarray}[1]{%
  \begin{adjustbox}{width=.9\width}
    $
    \begin{array}{#1}
    }{%      
    \end{array}
    $
  \end{adjustbox}
}

\newcommand{\vd}[2]{
  \scalebox{.8}{
    $\left(\!
      \begin{array}{c}
        #1 \\
        #2
      \end{array}
    \!\right)$
    }}

\newcommand{\vt}[3]{
  \scalebox{.8}{
    $\left(\!
      \begin{array}{c}
        #1 \\
        #2 \\
        #3 
      \end{array}
    \!\right)$
    }}

\newcommand{\vq}[4]{
  \scalebox{.8}{
    $\left(\!
      \begin{array}{c}
        #1 \\
        #2 \\
        #3 \\
        #4 
      \end{array}
    \!\right)$
    }}

\newcommand{\vc}[5]{
  \scalebox{.8}{
    $\left(\!
      \begin{array}{c}
        #1 \\
        #2 \\
        #3 \\
        #4 \\
        #5 
      \end{array}
    \!\right)$
    }}

\newcommand{\ee}{\text{e}}

\newcommand{\dd}{\text{d}}

%%% Ensemble de définition
\newcommand{\Df}{\mathscr{D}}
\newcommand{\Cf}{\mathscr{C}}
\newcommand{\Ef}{\mathscr{C}}

\newcommand{\rond}[1]{\,\overset{\scriptscriptstyle \circ}{\!#1}}

\newcommand{\df}[2]{\dfrac{\partial #1}{\partial #2}} %
\newcommand{\dfn}[2]{\partial_{#2}(#1)} %
\newcommand{\ddfn}[2]{\partial^2_{#2}(#1)} %
\newcommand{\ddf}[2]{\dfrac{\partial^2 #1}{\partial #2^2}} %
\newcommand{\ddfr}[3]{\dfrac{\partial^2 #1}{\partial #2 \partial
    #3}} %


\newcommand{\dlim}[1]{{\displaystyle \lim_{#1} \ }}
\newcommand{\dlimPlus}[2]{
  \dlim{
    \scalebox{.6}{
      $
      \begin{array}{l}
        #1 \rightarrow #2\\
        #1 > #2
      \end{array}
      $}}}
\newcommand{\dlimMoins}[2]{
  \dlim{
    \scalebox{.6}{
      $
      \begin{array}{l}
        #1 \rightarrow #2\\
        #1 < #2
      \end{array}
      $}}}

%%%%%%%%%%%%%%
%% petit o, développement limité
%%%%%%%%%%%%%%

\newcommand{\oo}[2]{{\underset {{\overset {#1\rightarrow #2}{}}}{o}}} %
\newcommand{\oox}[1]{{\underset {{\overset {x\rightarrow #1}{}}}{o}}} %
\newcommand{\oon}{{\underset {{\overset {n\rightarrow +\infty}{}}}{o}}} %
\newcommand{\po}[1]{{\underset {{\overset {#1}{}}}{o}}} %
\newcommand{\neqx}[1]{{\ \underset {{\overset {x \to #1}{}}}{\not\sim}\ }} %
\newcommand{\eqx}[1]{{\ \underset {{\overset {x \to #1}{}}}{\sim}\ }} %
\newcommand{\eqn}{{\ \underset {{\overset {n \to +\infty}{}}}{\sim}\ }} %
\newcommand{\eq}[2]{{\ \underset {{\overset {#1 \to #2}{}}}{\sim}\ }} %
\newcommand{\DL}[1]{{\rm{DL}}_1 (#1)} %
\newcommand{\DLL}[1]{{\rm{DL}}_2 (#1)} %

\newcommand{\negl}{<<}

\newcommand{\neglP}[1]{\begin{array}{c}
    \vspace{-.2cm}\\
    << \\
    \vspace{-.7cm}\\
    {\scriptstyle #1}
  \end{array}}

%%%%%%%%%%%%%%
%% borne sup, inf, max, min
%%%%%%%%%%%%%%
\newcommand{\dsup}[1]{\displaystyle \sup_{#1} \ }
\newcommand{\dinf}[1]{\displaystyle \inf_{#1} \ }
\newcommand{\dmax}[1]{\max\limits_{#1} \ }
\newcommand{\dmin}[1]{\min\limits_{#1} \ }

\newcommand{\dcup}[2]{{\textstyle\bigcup\limits_{#1}^{#2}}\hspace{.1cm}}
%\displaystyle \bigcup_{#1}^{#2}}
\newcommand{\dcap}[2]{{\textstyle\bigcap\limits_{#1}^{#2}}\hspace{.1cm}}
% \displaystyle \bigcap_{#1}^{#2}
%%%%%%%%%%%%%%
%% opérateurs logiques
%%%%%%%%%%%%%%
\newcommand{\NON}[1]{\mathop{\small \tt{NON}} (#1)}
\newcommand{\ET}{\mathrel{\mathop{\small \mathtt{ET}}}}
\newcommand{\OU}{\mathrel{\mathop{\small \tt{OU}}}}
\newcommand{\XOR}{\mathrel{\mathop{\small \tt{XOR}}}}

\newcommand{\id}{{\rm{id}}}

\newcommand{\sbullet}{\scriptstyle \bullet}
\newcommand{\stimes}{\scriptstyle \times}

%%%%%%%%%%%%%%%%%%
%% Probabilités
%%%%%%%%%%%%%%%%%%
\newcommand{\Prob}{\mathbb{P}}
\newcommand{\Ev}[1]{\left[ {#1} \right]}
\newcommand{\Evmb}[1]{[ {#1} ]}
\newcommand{\E}{\mathbb{E}}
\newcommand{\V}{\mathbb{V}}
\newcommand{\Cov}{{\rm{Cov}}}
\newcommand{\U}[2]{\mathcal{U}(\llb #1, #2\rrb)}
\newcommand{\Uc}[2]{\mathcal{U}([#1, #2])}
\newcommand{\Ucof}[2]{\mathcal{U}(]#1, #2])}
\newcommand{\Ucoo}[2]{\mathcal{U}(]#1, #2[)}
\newcommand{\Ucfo}[2]{\mathcal{U}([#1, #2[)}
\newcommand{\Bern}[1]{\mathcal{B}\left(#1\right)}
\newcommand{\Bin}[2]{\mathcal{B}\left(#1, #2\right)}
\newcommand{\G}[1]{\mathcal{G}\left(#1\right)}
\newcommand{\Pois}[1]{\mathcal{P}\left(#1\right)}
\newcommand{\HG}[3]{\mathcal{H}\left(#1, #2, #3\right)}
\newcommand{\Exp}[1]{\mathcal{E}\left(#1\right)}
\newcommand{\Norm}[2]{\mathcal{N}\left(#1, #2\right)}

\DeclareMathOperator{\cov}{Cov}

\newcommand{\var}{v.a.r. }
\newcommand{\suit}{\hookrightarrow}

\newcommand{\flecheR}[1]{\rotatebox{90}{\scalebox{#1}{\color{red}
      $\curvearrowleft$}}}


\newcommand{\partie}[1]{\mathcal{P}(#1)}
\newcommand{\Cont}[1]{\mathcal{C}^{#1}}
\newcommand{\Contm}[1]{\mathcal{C}^{#1}_m}

\newcommand{\llb}{\llbracket}
\newcommand{\rrb}{\rrbracket}

%\newcommand{\im}[1]{{\rm{Im}}(#1)}
\newcommand{\imrec}[1]{#1^{- \mathds{1}}}

\newcommand{\unq}{\mathds{1}}

\newcommand{\Hyp}{\mathtt{H}}

\newcommand{\eme}[1]{#1^{\scriptsize \mbox{ème}}}
\newcommand{\er}[1]{#1^{\scriptsize \mbox{er}}}
\newcommand{\ere}[1]{#1^{\scriptsize \mbox{ère}}}
\newcommand{\nd}[1]{#1^{\scriptsize \mbox{nd}}}
\newcommand{\nde}[1]{#1^{\scriptsize \mbox{nde}}}

\newcommand{\truc}{\mathop{\top}}
\newcommand{\fois}{\mathop{\ast}}

\newcommand{\f}[1]{\overrightarrow{#1}}

\newcommand{\checked}{\textcolor{green}{\checkmark}}

\def\restriction#1#2{\mathchoice
              {\setbox1\hbox{${\displaystyle #1}_{\scriptstyle #2}$}
              \restrictionaux{#1}{#2}}
              {\setbox1\hbox{${\textstyle #1}_{\scriptstyle #2}$}
              \restrictionaux{#1}{#2}}
              {\setbox1\hbox{${\scriptstyle #1}_{\scriptscriptstyle #2}$}
              \restrictionaux{#1}{#2}}
              {\setbox1\hbox{${\scriptscriptstyle #1}_{\scriptscriptstyle #2}$}
              \restrictionaux{#1}{#2}}}
\def\restrictionaux#1#2{{#1\,\smash{\vrule height .8\ht1 depth .85\dp1}}_{\,#2}}

\makeatletter
\newcommand*{\da@rightarrow}{\mathchar"0\hexnumber@\symAMSa 4B }
\newcommand*{\da@leftarrow}{\mathchar"0\hexnumber@\symAMSa 4C }
\newcommand*{\xdashrightarrow}[2][]{%
  \mathrel{%
    \mathpalette{\da@xarrow{#1}{#2}{}\da@rightarrow{\,}{}}{}%
  }%
}
\newcommand{\xdashleftarrow}[2][]{%
  \mathrel{%
    \mathpalette{\da@xarrow{#1}{#2}\da@leftarrow{}{}{\,}}{}%
  }%
}
\newcommand*{\da@xarrow}[7]{%
  % #1: below
  % #2: above
  % #3: arrow left
  % #4: arrow right
  % #5: space left 
  % #6: space right
  % #7: math style 
  \sbox0{$\ifx#7\scriptstyle\scriptscriptstyle\else\scriptstyle\fi#5#1#6\m@th$}%
  \sbox2{$\ifx#7\scriptstyle\scriptscriptstyle\else\scriptstyle\fi#5#2#6\m@th$}%
  \sbox4{$#7\dabar@\m@th$}%
  \dimen@=\wd0 %
  \ifdim\wd2 >\dimen@
    \dimen@=\wd2 %   
  \fi
  \count@=2 %
  \def\da@bars{\dabar@\dabar@}%
  \@whiledim\count@\wd4<\dimen@\do{%
    \advance\count@\@ne
    \expandafter\def\expandafter\da@bars\expandafter{%
      \da@bars
      \dabar@ 
    }%
  }%  
  \mathrel{#3}%
  \mathrel{%   
    \mathop{\da@bars}\limits
    \ifx\\#1\\%
    \else
      _{\copy0}%
    \fi
    \ifx\\#2\\%
    \else
      ^{\copy2}%
    \fi
  }%   
  \mathrel{#4}%
}
\makeatother



\newcount\depth

\newcount\depth
\newcount\totaldepth

\makeatletter
\newcommand{\labelsymbol}{%
      \ifnum\depth=0
        %
      \else
        \rlap{\,$\bullet$}%
      \fi
}

\newcommand*\bernoulliTree[1]{%
    \depth=#1\relax            
    \totaldepth=#1\relax
    \draw node(root)[bernoulli/root] {\labelsymbol}[grow=right] \draw@bernoulli@tree;
    \draw \label@bernoulli@tree{root};                                   
}                                                                        

\def\draw@bernoulli@tree{%
    \ifnum\depth>0 
      child[parent anchor=east] foreach \type/\label in {left child/$E$,right child/$S$} {%
          node[bernoulli/\type] {\label\strut\labelsymbol} \draw@bernoulli@tree
      }
      coordinate[bernoulli/increment] (dummy)
   \fi%
}

\def\label@bernoulli@tree#1{%
    \ifnum\depth>0
      ($(#1)!0.5!(#1-1)$) node[fill=white,bernoulli/decrement] {\tiny$p$}
      \label@bernoulli@tree{#1-1}
      ($(#1)!0.5!(#1-2)$) node[fill=white] {\tiny$q$}
      \label@bernoulli@tree{#1-2}
      coordinate[bernoulli/increment] (dummy)
   \fi%
}

\makeatother

\tikzset{bernoulli/.cd,
         root/.style={},
         decrement/.code=\global\advance\depth by-1\relax,
         increment/.code=\global\advance\depth by 1\relax,
         left child/.style={bernoulli/decrement},
         right child/.style={}}


\newcommand{\eps}{\varepsilon}

% \newcommand{\tendi}[1]{\xrightarrow[\footnotesize #1 \rightarrow
%   +\infty]{}}%

\newcommand{\tend}{\rightarrow}%
\newcommand{\tendn}{\underset{n\to +\infty}{\longrightarrow}} %
\newcommand{\ntendn}{\underset{n\to
    +\infty}{\not\hspace{-.15cm}\longrightarrow}} %
% \newcommand{\tendn}{\xrightarrow[\footnotesize n \rightarrow
%   +\infty]{}}%
\newcommand{\Tendx}[1]{\xrightarrow[\footnotesize x \rightarrow
  #1]{}}%
\newcommand{\tendx}[1]{\underset{x\to #1}{\longrightarrow}}%
\newcommand{\ntendx}[1]{\underset{x\to #1}{\not\!\!\longrightarrow}}%
\newcommand{\tendd}[2]{\underset{#1\to #2}{\longrightarrow}}%
% \newcommand{\tendd}[2]{\xrightarrow[\footnotesize #1 \rightarrow
%   #2]{}}%
\newcommand{\tendash}[1]{\xdashrightarrow[\footnotesize #1 \rightarrow
  +\infty]{}}%
\newcommand{\tendashx}[1]{\xdashrightarrow[\footnotesize x \rightarrow
  #1]{}}%
\newcommand{\tendb}[1]{\underset{#1 \to +\infty}{\longrightarrow}}%
\newcommand{\tendL}{\overset{\mathscr L}{\underset{n \to
      +\infty}{\longrightarrow}}}%
\newcommand{\tendP}{\overset{\Prob}{\underset{n \to
      +\infty}{\longrightarrow}}}%
\newcommand{\tenddL}[1]{\overset{\mathscr L}{\underset{#1 \to
      +\infty}{\longrightarrow}}}%

\NewEnviron{attention}{ %
  ~\\[-.2cm]\noindent
  \begin{minipage}{\linewidth}
  \setlength{\fboxsep}{3mm}%
  \ \ \dbend \ \ %
  \fbox{\parbox[t]{.88\linewidth}{\BODY}} %
  \end{minipage}\\
}

\NewEnviron{sattention}[1]{ %
  ~\\[-.2cm]\noindent
  \begin{minipage}{#1\linewidth}
  \setlength{\fboxsep}{3mm}%
  \ \ \dbend \ \ %
  \fbox{\parbox[t]{.88\linewidth}{\BODY}} %
  \end{minipage}\\
}

%%%%% OBSOLETE %%%%%%

% \newcommand{\attention}[1]{
%   \noindent
%   \begin{tabular}{@{}l|p{11.5cm}|}
%     \cline{2-2}
%     \vspace{-.2cm} 
%     & \nl
%     \dbend & #1 \nl
%     \cline{2-2}
%   \end{tabular}
% }

% \newcommand{\attentionv}[2]{
%   \noindent
%   \begin{tabular}{@{}l|p{11.5cm}|}
%     \cline{2-2}
%     \vspace{-.2cm} 
%     & \nl
%     \dbend & #2 \nl[#1 cm]
%     \cline{2-2}
%   \end{tabular}
% }

\newcommand{\explainvb}[2]{
  \noindent
  \begin{tabular}{@{}l|p{11.5cm}|}
    \cline{2-2}
    \vspace{-.2cm} 
    & \nl
    \hand & #2 \nl [#1 cm]
    \cline{2-2}
  \end{tabular}
}


% \noindent
% \begin{tabular}{@{}l|lp{11cm}|}
%   \cline{3-3} 
%   \multicolumn{1}{@{}l@{\dbend}}{} & & #1 \nl
%   \multicolumn{1}{l}{} & & \nl [-.8cm]
%   & & #2 \nl
%   \cline{2-3}
% \end{tabular}

% \newcommand{\attention}[1]{
%   \noindent
%   \begin{tabular}{@{}@{}cp{11cm}}
%     \dbend & #1 \nl
%   \end{tabular}
% }

\newcommand{\PP}[1]{\mathcal{P}(#1)}
\newcommand{\HH}[1]{\mathcal{H}(#1)}
\newcommand{\FF}[1]{\mathcal{F}(#1)}

\newcommand{\DSum}[2]{\displaystyle\sum\limits_{#1}^{#2}\hspace{.1cm}}
\newcommand{\Sum}[2]{{\textstyle\sum\limits_{#1}^{#2}}\hspace{.1cm}}
\newcommand{\Serie}{\textstyle\sum\hspace{.1cm}}
\newcommand{\Prod}[2]{\textstyle\prod\limits_{#1}^{#2}}

\newcommand{\Prim}[3]{\left[\ {#1} \ \right]_{\scriptscriptstyle
   \hspace{-.15cm} ~_{#2}\, }^ {\scriptscriptstyle \hspace{-.15cm} ~^{#3}\, }}

% \newcommand{\Prim}[3]{\left[\ {#1} \ \right]_{\scriptscriptstyle
%     \!\!~_{#2}}^ {\scriptscriptstyle \!\!~^{#3}}}

\newcommand{\dint}[2]{\displaystyle \int_{#1}^{#2}\ }
\newcommand{\Int}[2]{{\rm{Int}}_{\scriptscriptstyle #1, #2}}
\newcommand{\dt}{\ dt}
\newcommand{\dx}{\ dx}

\newcommand{\llpar}[1]{\left(\!\!\!
    \begin{array}{c}
      \rule{0pt}{#1}
    \end{array}
  \!\!\!\right.}

\newcommand{\rrpar}[1]{\left.\!\!\!
    \begin{array}{c}
      \rule{0pt}{#1}
    \end{array}
  \!\!\!\right)}

\newcommand{\llacc}[1]{\left\{\!\!\!
    \begin{array}{c}
      \rule{0pt}{#1}
    \end{array}
  \!\!\!\right.}

\newcommand{\rracc}[1]{\left.\!\!\!
    \begin{array}{c}
      \rule{0pt}{#1}
    \end{array}
  \!\!\!\right\}}

\newcommand{\ttacc}[1]{\mbox{\rotatebox{-90}{\hspace{-.7cm}$\llacc{#1}$}}}
\newcommand{\bbacc}[1]{\mbox{\rotatebox{90}{\hspace{-.5cm}$\llacc{#1}$}}}

\newcommand{\comp}[1]{\overline{#1}}%

\newcommand{\dcomp}[2]{\stackrel{\mbox{\ \ \----}{\scriptscriptstyle
      #2}}{#1}}%

% \newcommand{\Comp}[2]{\stackrel{\mbox{\ \
%       \-------}{\scriptscriptstyle #2}}{#1}}

% \newcommand{\dcomp}[2]{\stackrel{\mbox{\ \
%       \-------}{\scriptscriptstyle #2}}{#1}}

\newcommand{\A}{\mathscr{A}}

\newcommand{\conc}[1]{
  \begin{center}
    \fbox{
      \begin{tabular}{c}
        #1
      \end{tabular}
    }
  \end{center}
}

\newcommand{\concC}[1]{
  \begin{center}
    \fbox{
    \begin{tabular}{C{10cm}}
      \quad #1 \quad
    \end{tabular}
    }
  \end{center}
}

\newcommand{\concL}[2]{
  \begin{center}
    \fbox{
    \begin{tabular}{C{#2cm}}
      \quad #1 \quad
    \end{tabular}
    }
  \end{center}
}


% \newcommand{\lims}[2]{\prod\limits_{#1}^{#2}}

\newtheorem{theorem}{Théorème}[]
\newtheorem{lemma}{Lemme}[]
\newtheorem{proposition}{Proposition}[]
\newtheorem{corollary}{Corollaire}[]

% \newenvironment{proof}[1][Démonstration]{\begin{trivlist}
% \item[\hskip \labelsep {\bfseries #1}]}{\end{trivlist}}
\newenvironment{definition}[1][Définition]{\begin{trivlist}
\item[\hskip \labelsep {\bfseries #1}]}{\end{trivlist}}
\newenvironment{example}[1][Exemple]{\begin{trivlist}
\item[\hskip \labelsep {\bfseries #1}]}{\end{trivlist}}
\newenvironment{examples}[1][Exemples]{\begin{trivlist}
\item[\hskip \labelsep {\bfseries #1}]}{\end{trivlist}}
\newenvironment{notation}[1][Notation]{\begin{trivlist}
\item[\hskip \labelsep {\bfseries #1}]}{\end{trivlist}}
\newenvironment{propriete}[1][Propriété]{\begin{trivlist}
\item[\hskip \labelsep {\bfseries #1}]}{\end{trivlist}}
\newenvironment{proprietes}[1][Propriétés]{\begin{trivlist}
\item[\hskip \labelsep {\bfseries #1}]}{\end{trivlist}}
% \newenvironment{remark}[1][Remarque]{\begin{trivlist}
% \item[\hskip \labelsep {\bfseries #1}]}{\end{trivlist}}
\newenvironment{application}[1][Application]{\begin{trivlist}
\item[\hskip \labelsep {\bfseries #1}]}{\end{trivlist}}

% Environnement pour les réponses des DS
\newenvironment{answer}{\par\emph{Réponse :}\par{}}
{\vspace{-.6cm}\hspace{\stretch{1}}\rule{1ex}{1ex}\vspace{.3cm}}

\newenvironment{answerTD}{\vspace{.2cm}\par\emph{Réponse :}\par{}}
{\hspace{\stretch{1}}\rule{1ex}{1ex}\vspace{.3cm}}

\newenvironment{answerCours}{\noindent\emph{Réponse :}}
{\rule{1ex}{1ex}}%\vspace{.3cm}}


% footnote in footer
\newcommand{\fancyfootnotetext}[2]{%
  \fancypagestyle{dingens}{%
    \fancyfoot[LO,RE]{\parbox{0.95\textwidth}{\footnotemark[#1]\footnotesize
        #2}}%
  }%
  \thispagestyle{dingens}%
}

%%% définit le style (arabic : 1,2,3...) et place des parenthèses
%%% autour de la numérotation
\renewcommand*{\thefootnote}{(\arabic{footnote})}
% http://www.tuteurs.ens.fr/logiciels/latex/footnote.html

%%%%%%%% tikz axis
% \pgfplotsset{every axis/.append style={
%                     axis x line=middle,    % put the x axis in the middle
%                     axis y line=middle,    % put the y axis in the middle
%                     axis line style={<->,color=blue}, % arrows on the axis
%                     xlabel={$x$},          % default put x on x-axis
%                     ylabel={$y$},          % default put y on y-axis
%             }}

%%%% s'utilise comme suit

% \begin{axis}[
%   xmin=-8,xmax=4,
%   ymin=-8,ymax=4,
%   grid=both,
%   ]
%   \addplot [domain=-3:3,samples=50]({x^3-3*x},{3*x^2-9}); 
% \end{axis}

%%%%%%%%



%%%%%%%%%%%% Pour avoir des numéros de section qui correspondent à
%%%%%%%%%%%% ceux du tableau
\renewcommand{\thesection}{\Roman{section}.\hspace{-.3cm}}
\renewcommand{\thesubsection}{\Roman{section}.\arabic{subsection}.\hspace{-.2cm}}
\renewcommand{\thesubsubsection}{\Roman{section}.\arabic{subsection}.\alph{subsubsection})\hspace{-.2cm}}
%%%%%%%%%%%% 

%%% Changer le nom des figures : Fig. au lieu de Figure
\usepackage[font=small,labelfont=bf,labelsep=space]{caption}
\captionsetup{%
  figurename=Fig.,
  tablename=Tab.
}
% \renewcommand{\thesection}{\Roman{section}.\hspace{-.2cm}}
% \renewcommand{\thesubsection}{\Roman{section}
%   .\hspace{.2cm}\arabic{subsection}\ .\hspace{-.3cm}}
% \renewcommand{\thesubsubsection}{\alph{subsection})}

\newenvironment{tabliste}[1]
{\begin{tabenum}[\bfseries\small\itshape #1]}{\end{tabenum}} 

%%%% ESSAI contre le too deeply nested %%%%
%%%% ATTENTION au package enumitem qui se comporte mal avec les
%%%% noliste, à redéfinir !
% \usepackage{enumitem}

% \setlistdepth{9}

% \newlist{myEnumerate}{enumerate}{9}
% \setlist[myEnumerate,1]{label=(\arabic*)}
% \setlist[myEnumerate,2]{label=(\Roman*)}
% \setlist[myEnumerate,3]{label=(\Alph*)}
% \setlist[myEnumerate,4]{label=(\roman*)}
% \setlist[myEnumerate,5]{label=(\alph*)}
% \setlist[myEnumerate,6]{label=(\arabic*)}
% \setlist[myEnumerate,7]{label=(\Roman*)}
% \setlist[myEnumerate,8]{label=(\Alph*)}
% \setlist[myEnumerate,9]{label=(\roman*)}

%%%%%

\newenvironment{noliste}[1] %
{\begin{enumerate}[\bfseries\small\itshape #1]} %
  {\end{enumerate}}

\newenvironment{nonoliste}[1] %
{\begin{enumerate}[\hspace{-12pt}\bfseries\small\itshape #1]} %
  {\end{enumerate}}

\newenvironment{arrayliste}[1]{ 
  % List with minimal white space to fit in small areas, e.g. table
  % cell
  \begin{minipage}[t]{\linewidth} %
    \begin{enumerate}[\bfseries\small\itshape #1] %
      {\leftmargin=0.5em \rightmargin=0em
        \topsep=0em \parskip=0em \parsep=0em
        \listparindent=0em \partopsep=0em \itemsep=0pt
        \itemindent=0em \labelwidth=\leftmargin\labelsep+0.25em}
    }{
    \end{enumerate}\end{minipage}
}

\newenvironment{nolistes}[2]
{\begin{enumerate}[\bfseries\small\itshape
    #1]\setlength{\itemsep}{#2 mm}}{\end{enumerate}}

\newenvironment{liste}[1]
{\begin{enumerate}[\hspace{12pt}\bfseries\small\itshape
    #1]}{\end{enumerate}}   


%%%%%%%% Pour les programmes de colle %%%%%%%

\newcommand{\cours}{{\small \tt (COURS)}} %
\newcommand{\poly}{{\small \tt (POLY)}} %
\newcommand{\exo}{{\small \tt (EXO)}} %
\newcommand{\culture}{{\small \tt (CULTURE)}} %
\newcommand{\methodo}{{\small \tt (MÉTHODO)}} %
\newcommand{\methodob}{\Boxed{\mbox{\tt MÉTHODO}}} %

%%%%%%%% Pour les TD %%%%%%%
\newtheoremstyle{exostyle} {\topsep} % espace avant
{.6cm} % espace apres
{} % Police utilisee par le style de thm
{} % Indentation (vide = aucune, \parindent = indentation paragraphe)
{\bfseries} % Police du titre de thm
{} % Signe de ponctuation apres le titre du thm
{ } % Espace apres le titre du thm (\newline = linebreak)
{\thmname{#1}\thmnumber{ #2}\thmnote{.
    \normalfont{\textit{#3}}}} % composants du titre du thm : \thmname
                               % = nom du thm, \thmnumber = numéro du
                               % thm, \thmnote = sous-titre du thm
 
\theoremstyle{exostyle}
\newtheorem{exercice}{Exercice}
\newtheorem*{exoCours}{Exercice}

%%%%%%%% Pour des théorèmes Sans Espaces APRÈS %%%%%%%
\newtheoremstyle{exostyleSE} {\topsep} % espace avant
{} % espace apres
{} % Police utilisee par le style de thm
{} % Indentation (vide = aucune, \parindent = indentation paragraphe)
{\bfseries} % Police du titre de thm
{} % Signe de ponctuation apres le titre du thm
{ } % Espace apres le titre du thm (\newline = linebreak)
{\thmname{#1}\thmnumber{ #2}\thmnote{.
    \normalfont{\textit{#3}}}} % composants du titre du thm : \thmname
                               % = nom du thm, \thmnumber = numéro du
                               % thm, \thmnote = sous-titre du thm
 
\theoremstyle{exostyleSE}
\newtheorem{exerciceSE}{Exercice}
\newtheorem*{exoCoursSE}{Exercice}

% \newcommand{\lims}[2]{\prod\limits_{#1}^{#2}}

\newtheorem{theoremSE}{Théorème}[]
\newtheorem{lemmaSE}{Lemme}[]
\newtheorem{propositionSE}{Proposition}[]
\newtheorem{corollarySE}{Corollaire}[]

% \newenvironment{proofSE}[1][Démonstration]{\begin{trivlist}
% \item[\hskip \labelsep {\bfseries #1}]}{\end{trivlist}}
\newenvironment{definitionSE}[1][Définition]{\begin{trivlist}
  \item[\hskip \labelsep {\bfseries #1}]}{\end{trivlist}}
\newenvironment{exampleSE}[1][Exemple]{\begin{trivlist} 
  \item[\hskip \labelsep {\bfseries #1}]}{\end{trivlist}}
\newenvironment{examplesSE}[1][Exemples]{\begin{trivlist}
\item[\hskip \labelsep {\bfseries #1}]}{\end{trivlist}}
\newenvironment{notationSE}[1][Notation]{\begin{trivlist}
\item[\hskip \labelsep {\bfseries #1}]}{\end{trivlist}}
\newenvironment{proprieteSE}[1][Propriété]{\begin{trivlist}
\item[\hskip \labelsep {\bfseries #1}]}{\end{trivlist}}
\newenvironment{proprietesSE}[1][Propriétés]{\begin{trivlist}
\item[\hskip \labelsep {\bfseries #1}]}{\end{trivlist}}
\newenvironment{remarkSE}[1][Remarque]{\begin{trivlist}
\item[\hskip \labelsep {\bfseries #1}]}{\end{trivlist}}
\newenvironment{applicationSE}[1][Application]{\begin{trivlist}
\item[\hskip \labelsep {\bfseries #1}]}{\end{trivlist}}

%%%%%%%%%%% Obtenir les étoiles sans charger le package MnSymbol
%%%%%%%%%%%
\DeclareFontFamily{U} {MnSymbolC}{}
\DeclareFontShape{U}{MnSymbolC}{m}{n}{
  <-6> MnSymbolC5
  <6-7> MnSymbolC6
  <7-8> MnSymbolC7
  <8-9> MnSymbolC8
  <9-10> MnSymbolC9
  <10-12> MnSymbolC10
  <12-> MnSymbolC12}{}
\DeclareFontShape{U}{MnSymbolC}{b}{n}{
  <-6> MnSymbolC-Bold5
  <6-7> MnSymbolC-Bold6
  <7-8> MnSymbolC-Bold7
  <8-9> MnSymbolC-Bold8
  <9-10> MnSymbolC-Bold9
  <10-12> MnSymbolC-Bold10
  <12-> MnSymbolC-Bold12}{}

\DeclareSymbolFont{MnSyC} {U} {MnSymbolC}{m}{n}

\DeclareMathSymbol{\filledlargestar}{\mathrel}{MnSyC}{205}
\DeclareMathSymbol{\largestar}{\mathrel}{MnSyC}{131}

\newcommand{\facile}{\rm{(}$\scriptstyle\largestar$\rm{)}} %
\newcommand{\moyen}{\rm{(}$\scriptstyle\filledlargestar$\rm{)}} %
\newcommand{\dur}{\rm{(}$\scriptstyle\filledlargestar\filledlargestar$\rm{)}} %
\newcommand{\costaud}{\rm{(}$\scriptstyle\filledlargestar\filledlargestar\filledlargestar$\rm{)}}

%%%%%%%%%%%%%%%%%%%%%%%%%

%%%%%%%%%%%%%%%%%%%%%%%%%
%%%%%%%% Fin de la partie TD

%%%%%%%%%%%%%%%%
%%%%%%%%%%%%%%%%
\makeatletter %
\newenvironment{myitemize}{%
  \setlength{\topsep}{0pt} %
  \setlength{\partopsep}{0pt} %
  \renewcommand*{\@listi}{\leftmargin\leftmargini \parsep\z@
    \topsep\z@ \itemsep\z@} \let\@listI\@listi %
  \itemize %
}{\enditemize} %
\makeatother
%%%%%%%%%%%%%%%%
%%%%%%%%%%%%%%%%

%% Commentaires dans la correction du livre

\newcommand{\Com}[1]{
% Define box and box title style
\tikzstyle{mybox} = [draw=black!50,
very thick,
    rectangle, rounded corners, inner sep=10pt, inner ysep=8pt]
\tikzstyle{fancytitle} =[rounded corners, fill=black!80, text=white]
\tikzstyle{fancylogo} =[ text=white]
\begin{center}

\begin{tikzpicture}
\node [mybox] (box){%

    \begin{minipage}{0.90\linewidth}
\vspace{6pt}  #1
    \end{minipage}
};
\node[fancytitle, right=10pt] at (box.north west) 
{\bfseries\normalsize{Commentaire}};

\end{tikzpicture}%

\end{center}
%
}

\NewEnviron{remark}{%
  % Define box and box title style
  \tikzstyle{mybox} = [draw=black!50, very thick, rectangle, rounded
  corners, inner sep=10pt, inner ysep=8pt] %
  \tikzstyle{fancytitle} = [rounded corners , fill=black!80,
  text=white] %
  \tikzstyle{fancylogo} =[ text=white]
  \begin{center}
    \begin{tikzpicture}
      \node [mybox] (box){%
        \begin{minipage}{0.90\linewidth}
          \vspace{6pt} \BODY
        \end{minipage}
      }; %
      \node[fancytitle, right=10pt] at (box.north west) %
      {\bfseries\normalsize{Commentaire}}; %
    \end{tikzpicture}%
  \end{center}
}

\NewEnviron{remarkST}{%
  % Define box and box title style
  \tikzstyle{mybox} = [draw=black!50, very thick, rectangle, rounded
  corners, inner sep=10pt, inner ysep=8pt] %
  \tikzstyle{fancytitle} = [rounded corners , fill=black!80,
  text=white] %
  \tikzstyle{fancylogo} =[ text=white]
  \begin{center}
    \begin{tikzpicture}
      \node [mybox] (box){%
        \begin{minipage}{0.90\linewidth}
          \vspace{6pt} \BODY
        \end{minipage}
      }; %
      % \node[fancytitle, right=10pt] at (box.north west) %
%       {\bfseries\normalsize{Commentaire}}; %
    \end{tikzpicture}%
  \end{center}
}

\NewEnviron{remarkL}[1]{%
  % Define box and box title style
  \tikzstyle{mybox} = [draw=black!50, very thick, rectangle, rounded
  corners, inner sep=10pt, inner ysep=8pt] %
  \tikzstyle{fancytitle} =[rounded corners, fill=black!80,
  text=white] %
  \tikzstyle{fancylogo} =[ text=white]
  \begin{center}
    \begin{tikzpicture}
      \node [mybox] (box){%
        \begin{minipage}{#1\linewidth}
          \vspace{6pt} \BODY
        \end{minipage}
      }; %
      \node[fancytitle, right=10pt] at (box.north west) %
      {\bfseries\normalsize{Commentaire}}; %
    \end{tikzpicture}%
  \end{center}
}

\NewEnviron{remarkSTL}[1]{%
  % Define box and box title style
  \tikzstyle{mybox} = [draw=black!50, very thick, rectangle, rounded
  corners, inner sep=10pt, inner ysep=8pt] %
  \tikzstyle{fancytitle} =[rounded corners, fill=black!80,
  text=white] %
  \tikzstyle{fancylogo} =[ text=white]
  \begin{center}
    \begin{tikzpicture}
      \node [mybox] (box){%
        \begin{minipage}{#1\linewidth}
          \vspace{6pt} \BODY
        \end{minipage}
      }; %
%       \node[fancytitle, right=10pt] at (box.north west) %
%       {\bfseries\normalsize{Commentaire}}; %
    \end{tikzpicture}%
  \end{center}
}

\NewEnviron{titre} %
{ %
  ~\\[-1.8cm]
  \begin{center}
    \bf \LARGE \BODY
  \end{center}
  ~\\[-.6cm]
  \hrule %
  \vspace*{.2cm}
} %

\NewEnviron{titreL}[2] %
{ %
  ~\\[-#1cm]
  \begin{center}
    \bf \LARGE \BODY
  \end{center}
  ~\\[-#2cm]
  \hrule %
  \vspace*{.2cm}
} %



%%%%%%%%%%% Redefinition \chapter



\usepackage[explicit]{titlesec}
\usepackage{color}
\titleformat{\chapter}
{\gdef\chapterlabel{}
\selectfont\huge\bf}
%\normalfont\sffamily\Huge\bfseries\scshape}
{\gdef\chapterlabel{\thechapter)\ }}{0pt}
{\begin{tikzpicture}[remember picture,overlay]
\node[yshift=-3cm] at (current page.north west)
{\begin{tikzpicture}[remember picture, overlay]
\draw (.1\paperwidth,0) -- (.9\paperwidth,0);
\draw (.1\paperwidth,2) -- (.9\paperwidth,2);
%(\paperwidth,3cm);
\node[anchor=center,xshift=.5\paperwidth,yshift=1cm, rectangle,
rounded corners=20pt,inner sep=11pt]
{\color{black}\chapterlabel#1};
\end{tikzpicture}
};
\end{tikzpicture}
}
\titlespacing*{\chapter}{0pt}{50pt}{-75pt}




%%%%%%%%%%%%%% Affichage chapter dans Table des matieres

\makeatletter
\renewcommand*\l@chapter[2]{%
  \ifnum \c@tocdepth >\m@ne
    \addpenalty{-\@highpenalty}%
    \vskip 1.0em \@plus\p@
    \setlength\@tempdima{1.5em}%
    \begingroup
      \parindent \z@ \rightskip \@pnumwidth
      \parfillskip -\@pnumwidth
      \leavevmode %\bfseries
      \advance\leftskip\@tempdima
      \hskip -\leftskip
      #1\nobreak\ 
       \leaders\hbox{$\m@th
        \mkern \@dotsep mu\hbox{.}\mkern \@dotsep
        mu$}\hfil\nobreak\hb@xt@\@pnumwidth{\hss #2}\par
      \penalty\@highpenalty
    \endgroup
  \fi}
\makeatother




%%%%%%%%%%%%%%%%% Redefinition part




% \renewcommand{\thesection}{\Roman{section}.\hspace{-.3cm}}
% \renewcommand{\thesubsection}{\Alph{subsection}.\hspace{-.2cm}}

\pagestyle{fancy} %
\pagestyle{fancy} %
 \lhead{ECE2 \hfill Mathématiques \\} %
\chead{\hrule} %
\rhead{} %
\lfoot{} %
\cfoot{} %
\rfoot{\thepage} %

\renewcommand{\headrulewidth}{0pt}% : Trace un trait de séparation
                                    % de largeur 0,4 point. Mettre 0pt
                                    % pour supprimer le trait.

\renewcommand{\footrulewidth}{0.4pt}% : Trace un trait de séparation
                                    % de largeur 0,4 point. Mettre 0pt
                                    % pour supprimer le trait.

\setlength{\headheight}{14pt}

\title{\bf \vspace{-1.6cm} HEC 2016} %
\author{} %
\date{} %
\begin{document}

\maketitle %
\vspace{-1.2cm}\hrule %
\thispagestyle{fancy}

\vspace*{.4cm}

%%DEBUT

\section*{EXERCICE} % HEC 2016

\noindent 
Soit $n$ et $p$ deux entiers supérieurs ou égaux à 1.
Si $M$ est une matrice de $\M{n,p},$ la matrice ${}^t{}M$ de 
$\M{p,n}$ désigne la transposée de $M.$\\
On identifie les ensembles $\M{1,1}$ et $\R$ en assimilant 
une matrice de $\M{1,1}$ à son unique coefficient.\\
On note $\B_n$ la base canonique de $\M{n,1}$ et 
$\B_p$ la base canonique de $\M{p,1}.$\\
Si $M \in \M{n,p}$ et $N \in \M{p,q}$ ($q 
\in \N^*$), {\it on admet} que 
${}^t{}{(MN)}={}^{t}{}{N}{}^{t}{}{M}$.

\begin{noliste}{1.}
 \setlength{\itemsep}{4mm}
 \item Soit $X$ une matrice colonne non nulle de 
 $\M{n,1}$ de composantes $x_1,x_2,...,x_n$ dans la base 
 $\B_n.$\\
 On pose : $A=X{}^{t}{}{X}$ et $\alpha = {}^{t}{}{X}X.$
 \begin{noliste}{a)}
  \setlength{\itemsep}{2mm}
  \item Exprimer $A$ et $\alpha$ en fonction de $x_1,x_2,...,x_n$.
  Justifier que la matrice $A$ est diagonalisable.
  
  


  \item Soit $f$ l'endomorphisme de $\M{n,1}$ de matrice $A$ dans la 
  base $\B_n.$\\
  Déterminer $\im(f)$ et $\kr(f)$ ; donner une base de 
  $\im(f)$ et préciser la dimension de $\kr(f)$.
  
  

  
  \item Calculer la matrice $AX$. \\
  Déterminer les valeurs propres de $A$ 
  ainsi que les sous-espaces propres associés.
  
  
 \end{noliste}
 
 
 
 
 %\newpage
 
 
 
 
 \item On suppose que $n$ et $p$ vérifient $1 \leq p \leq n$.\\
 Soit $(V_1,V_2,...,V_p)$ une famille libre de $p$ vecteurs de 
 $\M{n,1}.$\\
 On note $V$ la matrice de $\M{n,p}$ dont les colonnes sont, dans cet 
 ordre, $V_1,V_2,...,V_p.$\\
 Soit $g$ l'application linéaire de $\M{p,1}$ dans $\M{n,1}$ de matrice 
 $V$ dans les bases $\B_p$ et $\B_n$.
 \begin{noliste}{a)}
  \setlength{\itemsep}{2mm}
  \item Justifier que le rang de $V$ est égal à $p$. Déterminer 
  $\kr(g)$.
  
  
  
  \item Soit $Y$ une matrice colonne de $\M{p,1}.$\\
  Montrer que l'on a $VY=0$ si et seulement si l'on a ${}^t{}VVY=0$.
  
  
  
  
  
  
  %\newpage
  
  
  
  \item En déduire que la matrice ${}^t{}VV$ est inversible.
  
  
 \end{noliste}
\end{noliste}


%\newpage


\section*{PROBLÈME} % HEC 2016
\noindent 
{\it On s'intéresse dans ce problème à quelques aspects 
mathématiques de la fonction de production d'une entreprise
qui produit un certain bien à une époque donnée, à partir de deux 
facteurs de production travail et capital.}\\
\textbf{Dans tout le problème :}
\begin{noliste}{$\sbullet$}
 \item {\it On note respectivement $x$ et $y$ les quantités de 
 travail et de capital requises pour produire une certaine quantité de 
 ce bien.}
 \item {\it On suppose que $x>0$ et $y>0.$ On pose 
 $\mathcal{D}=(\R_+^*)^2$ et pour tout $(x,y)\in \mathcal{D},$ 
 $z=\dfrac{x}{y}.$}
\end{noliste}
{\it La partie III est indépendante des parties I et II.}


\subsection*{Partie I : Fonction de production CES (Constant Elasticity 
of Substitution).}

\noindent
{\it Dans toute cette partie,} on note $c$ un réel vérifiant $0<c<1$ et 
$\theta$ un réel vérifiant $\theta < 1$ avec $\theta \neq 0.$\\
Soit $f$ la fonction définie sur $\mathcal{D},$ à valeurs dans 
$\R_+^*$, telle que :
\[
\forall (x,y)\in \mathcal{D},\quad 
f(x,y)=\left(c \, x^\theta+(1-c) \, y^\theta\right)^{\frac{1}{\theta}} 
\quad \text{({\it fonction de production CES})}
\]
\begin{noliste}{1.}
 \setlength{\itemsep}{4mm}
 \item {\it Exemple.} Dans cette question \textbf{uniquement,} on prend 
 $\theta=-1$ et $c=\dfrac{1}{2}$.
 \begin{noliste}{a)}
  \setlength{\itemsep}{2mm}
  \item Montrer que pour tout $(x,y) \in \mathcal{D},$ on a : $f(x,y) = 
  \dfrac{2xy}{x+y}$. Justifier que $f$ est de classe $\Cont{2}$ sur 
  $\mathcal{D}$ et calculer pour tout $(x,y) \in \mathcal{D},$ les 
  dérivées partielles $\dfn{f}{1}(x,y)$ et $\dfn{f}{2}(x,y)$.
  
  
  
  
  
  
  
  %\newpage
  
  

  
  \item Soit $w$ et $U$ les fonctions définies sur $\R_+^*$ par :
  $\forall t > 0$, $w(t)=\dfrac{2t}{1+t}$ et $U(t)=w(t)-t \, w'(t).$\\
  Dresser le tableau de variation de la fonction $U$ sur $\R_+^*$ et 
  étudier la convexité de $U$ sur $\R_+^*$.
  
  
  
  \item On rappelle que $z=\dfrac{x}{y}$. Montrer que pour tout $(x,y) 
  \in \mathcal{D}$, on a $f(x,y)=y \, w(z)$.
  
  

  
  \item Vérifier pour tout $(x,y) \in \mathcal{D}$, les relations : 
  $\dfn{f}{1}(x,y)=w'(z)$ et $\dfn{f}{2}(x,y)=U(z)$.
  
  
 \end{noliste}
 
 
 
 
 
 %\newpage
 
 
 
 

 \item 
 \begin{noliste}{a)}
  \setlength{\itemsep}{2mm}
  \item Montrer que pour tout $(x,y) \in \mathcal{D}$ et pour tout réel 
  $\lambda >0,$ on a : $f(\lambda x,\lambda y)=\lambda \, f(x,y)$.
  
  

  
  \item Justifier que $f$ est de classe $\Cont{2}$ sur $\mathcal{D}$ 
  et, pour tout $(x,y) \in \mathcal{D}$, calculer $\dfn{f}{1}(x,y)$ et 
  $\dfn{f}{2}(x,y)$.
  
  
  
  
  
  
  %\newpage
  
  

  
  \item Déterminer pour tout $y>0$ fixé, le signe et la monotonie de la 
  fonction $x \mapsto \partial_1(f)(x,y).$\\
  Déterminer pour tout $x>0$ fixé, le signe et la monotonie de 
  la fonction $y \mapsto \dfn{f}{2}(x,y)$.
  
  

 \end{noliste}
 
 \item Soit $G$ la fonction définie sur $\mathcal{D}$ par 
 $G(x,y)=\dfrac{\dfn{f}{1}(x,y)}{\dfn{f}{2}(x,y)}$ 
 ({\it taux marginal de substitution technique}) et $g$ la fonction 
 définie sur $\R_+^*$ par : $\forall t>0$, 
 $g(t)=\dfrac{c}{1-c}t^{-1+\theta}$.
 \begin{noliste}{a)}
  \setlength{\itemsep}{2mm}
  \item Pour tout $(x,y) \in \mathcal{D}$, exprimer $G(x,y)$ en 
  fonction de $g(z)$.
  
  

  
  \item\label{3b} Pour tout $t>0,$ on pose $s(t)=-\dfrac{g(t)}{t 
  \, g'(t)}$.
  Calculer $s(z)$ ({\it élasticité de substitution}). Conclusion.
  
  
 \end{noliste}
 
 
 
 
 %\newpage
 
 
 
 
 \item Soit $w$ et $U$ les fonctions définies sur $\R_+^*$ par :
 $\forall t>0,$ $w(t)=f(t,1)$ et $U(t)=w(t)-t \, w'(t)$.
 \begin{noliste}{a)}
  \setlength{\itemsep}{2mm}
  \item Montrer que pour tout $(x,y) \in \mathcal{D},$ on a : 
  $f(x,y)=y \, w(z)$.
  
  

  
  \item En distinguant les deux cas $0 < \theta < 1$ et $\theta < 0$, 
  dresser le tableau de variation de $U$ sur $\R_+^*.$\\
  Préciser $\dlim{t \to 0^+} U(t)$, $\dlim{t \to + \infty} U(t)$ ainsi 
  que la convexité de $U$ sur $\R_+^*$.
  
  
 \end{noliste}
\end{noliste}





%\newpage





\subsection*{Partie II : Caractérisation des fonctions de production à 
élasticité de substitution constante.}
\noindent 
{\it Dans toute cette partie,} on note $\Psi$ une fonction 
définie et de classe $\Cont{2}$ sur $\mathcal{D},$ à valeurs dans 
$\R_+^*$, vérifiant la condition $\Psi(1,1)=1$ et pour tout réel 
$\lambda > 0,$ la relation : $\Psi(\lambda x, \lambda y)=\lambda \,
\Psi(x,y)$.\\
De plus, on suppose que pour tout $y>0$ fixé, la fonction $x \mapsto 
\dfn{\Psi}{1}(x,y)$ est strictement positive et strictement 
décroissante et que pour tout $x>0$ fixé, la fonction $y \mapsto 
\dfn{\Psi}{2}(x,y)$ est également strictement positive et strictement 
décroissante.
\begin{noliste}{1.}
 \setlength{\itemsep}{4mm}
 \setcounter{enumi}{4}
 \item Soit $v$ la fonction définie sur $\R_+^*$ par : $\forall t > 0,$ 
 $v(t)=\Psi(t,1).$
 \begin{noliste}{a)}
  \setlength{\itemsep}{2mm}
  \item Justifier que la fonction $v$ est de classe $\Cont{2}$, 
  strictement croissante et concave sur $\R_+^*$.
  
  

  
  
  
  %\newpage
  
  
  
  
  \item Soit $\varphi$ la fonction définie sur $\R_+^*$ par : $\forall 
  t >0$, $\varphi(t)=v(t)-t \, v'(t)$. On suppose l'existence de la 
  limite de $\varphi(t)$ lorsque $t$ tend vers 0 par valeurs supérieures
  et que $\dlim{t \to 0^+} \varphi(t)=\mu$, avec $\mu \geq 0$.\\
  Déterminer pour tout $t>0,$ le signe de $\varphi(t)$ et montrer que 
  $\mu \leq 1$.
  
  

  
  \item Montrer que : $\forall (x,y) \in \mathcal{D}$, 
  $\Psi(x,y)=y \, v(z)$.
  
  
 \end{noliste}
 
 
 
 
 
 %\newpage
 
 
 
 
 \item\label{6} 
 \begin{noliste}{a)}
  \setlength{\itemsep}{2mm}
  \item Pour tout $t>0,$ on pose : $h(t)=\dfrac{v'(t)}{\varphi(t)}$.\\
  Montrer que pour tout $(x,y) \in \mathcal{D},$ on a : 
  $\dfrac{\dfn{\Psi}{1}(x,y)}{\dfn{\Psi}{2}(x,y)}=h(z)$.
  
  

  
  \item Pour tout $t>0,$ on pose : $\sigma(t)=-\dfrac{h(t)}{t \, 
  h'(t)}$. Déterminer pour tout $t>0,$ le signe de $\sigma(t)$.
  
  
 \end{noliste}
 
 
 
 
 %\newpage
 
 
 
 \item Les fonctions $\sigma$ et $h$ sont celles qui ont été définies 
 dans la question \itbf{6}. On suppose que la fonction $\sigma$ est 
 constante sur $\R_+^*$ ; on note $\sigma_0$ cette constante et on 
 suppose $\sigma_0 \neq 1$. On pose : $r=1-\dfrac{1}{\sigma_0}$.
 \begin{noliste}{a)}
  \setlength{\itemsep}{2mm}
  \item\label{7a} Pour tout $t>0,$ on pose $\ell(t)=t^{1-r}h(t)$.
  Calculer $\ell'(t)$ et en déduire que : $\forall t >0$, 
  $h(t)=h(1)t^{r-1}$.
  
  

  
  \item Par une méthode analogue à celle de la question \itbf{7a}, 
  établir la relation : 
  \[
  \forall t >0, \  
  v(t)=\left(\dfrac{1+h(1)t^r}{1+h(1)}\right)^{\frac{1}{r}}
  \]
  
  

  
  \item\label{7c} En déduire l'existence d'une constante $a \in ]0,1[$ 
  telle que : $\forall (x,y) \in \mathcal{D}$, 
  $\Psi(x,y)=\left(ax^r+(1-a)y^r\right)^{\frac{1}{r}}$.
  
  
  
  
  
  
  %\newpage
  
  
  

  
  \item Quelle conclusion peut-on tirer des résultats des questions 
  \itbf{3.b)} et \itbf{7.c)} ?
  
  
 \end{noliste}
 
 \item Soit $a \in \ ]0,1[$. Pour tout $t>0,$ soit $S_t$ la fonction 
 définie sur $]-\infty,1[ \setminus \{0\}$ par : 
 $S_t(r)=(at^r+1-a)^{\frac{1}{r}}$.
 \begin{noliste}{a)}
  \setlength{\itemsep}{2mm}
  \item On pose $H_t(r)=\ln S_t(r)$. Calculer la limite de $S_t(r)$ 
  lorsque $r$ tend vers 0.
  
  
  
  
  
  
  %\newpage
  
  
  

  
  \item Pour tout couple $(x,y) \in \mathcal{D}$ fixé, on pose : 
  $N_{(x,y)}(r) = y \, S_z(r)$ et $F(x,y)=\dlim{r \to 0} 
  N_{(x,y)}(r)$.\\
  Montrer que pour tout $(x,y) \in \mathcal{D},$ on a 
  $F(x,y)=x^a \, y^{1-a}$ ({\it fonction de production de 
  Cobb-Douglas}).
  
  
 \end{noliste}
\end{noliste}




\subsection*{Partie III : Estimation des paramètres d'une fonction de 
production de Cobb-Douglas.}
\noindent 
Soit $a$ un réel vérifiant $0 < a < 1$ et $B$ un réel 
strictement positif.\\
On suppose que la production totale $Q$ présente une composante 
déterministe et une composante aléatoire.
\begin{noliste}{$\sbullet$}
 \item La {\it composante déterministe} est une fonction de 
 production de type Cobb-Douglas, c'est-à-dire telle que :
 \[
  \forall (x,y) \in \mathcal{D},\quad f(x,y)=Bx^ay^{1-a}
 \]
 
 \item La {\it composante aléatoire} est une variable aléatoire de la 
 forme $\exp(R)$ où $R$ est une variable aléatoire suivant la loi 
 normale centrée, de variance $\sigma^2>0$.
 
 \item La {\it production totale} $Q$ est une variable aléatoire à 
 valeurs strictement positives telle que :
 \[
  Q=Bx^ay^{1-a} \exp(R)
 \]
\end{noliste}
On suppose que les variables aléatoires $Q$ et $R$ sont définies sur 
le même espace probabilisé $(\Omega,\A,\Prob)$.



\noindent
On pose : $b=\ln(B)$, $u=\ln(x) - \ln(y)$ et $T=\ln(Q)-\ln(y)$. On a 
donc : $T=au+b+R$.\\
On sélectionne $n$ entreprises ($n \geq 1$) qui produisent le bien 
considéré à l'époque donnée.\\
On mesure pour chaque entreprise $i$ ($i \in \llb 1,n \rrb$) la 
quantité de travail $x_i$ et la quantité de capital $y_i$ utilisées
ainsi que la quantité produite $Q_i^*$.\\
On suppose que pour tout $i \in \llb 1,n \rrb$, on a $x_i>0$, $y_i>0$ 
et $Q_i^*>0$.



%\newpage


\noindent
Pour tout $i \in \llb 1,n \rrb$, la production totale de l'entreprise 
$i$ est alors une variable aléatoire $Q_i$ telle que 
$Q_i=B \, x_i^a \, y_i^{1-a} \exp(R_i)$, où $R_1$, $R_2$, 
$\ldots$, $R_n$ sont des variables aléatoires supposées indépendantes 
et de même loi que $R$ et le réel strictement positif $Q_i^*$ est une 
réalisation de la variable aléatoire $Q_i.$\\
On pose pour tout $i \in \llb 1,n \rrb$ : $u_i=\ln(x_i) - \ln(y_i)$, 
$T_i=\ln(Q_i) - \ln(y_i)$ et $t_i=\ln(Q_i^*)-\ln(y_i)$.\\
Ainsi, pour chaque entreprise $i \in \llb 1,n \rrb$, on a 
$T_i=au_i+b+R_i$ et le réel $t_i$ est une réalisation de la variable 
aléatoire $T_i$.\\
{\it On rappelle les définitions et résultats suivants} :
\begin{noliste}{$\sbullet$}
 \item Si $(v_i)_{1 \leq i \leq n}$ est une série statistique, la 
 moyenne et la variance empiriques, notées respectivement 
 $\overline{v}$ et $s_v^2$, sont données par :
 $\overline{v}=\dfrac{1}{n}\Sum{i=1}{n} v_i$ et 
 $s_v^2=\dfrac{1}{n}\Sum{i=1}{n} (v_i-\overline{v})^2 = 
 \dfrac{1}{n}\Sum{i=1}{n} v_i^2-\overline{v}^2$.\\
 
 \item Si $(v_i)_{1 \leq i \leq n}$ et $(w_i)_{1 \leq i \leq n}$ sont 
 deux séries statistiques, la covariance empirique de la série double 
 $(v_i,w_i)_{1 \leq i \leq n},$ notée $\mathrm{cov}(v,w)$, est donnée 
 par :  
 \[
 \mathrm{cov}(v,w)=\dfrac{1}{n}\Sum{i=1}{n}(v_i - 
 \overline{v})(w_i-\overline{w}) = \dfrac{1}{n}\Sum{i=1}{n} v_i \, w_i 
 - \overline{v}\overline{w} = \dfrac{1}{n}\Sum{i=1}{n} 
 (v_i-\overline{v})w_i
 \]
\end{noliste}


\begin{noliste}{1.}
 \setlength{\itemsep}{4mm}
 \setcounter{enumi}{8}
 \item 
 \begin{noliste}{a)}
  \setlength{\itemsep}{2mm}
  \item Montrer que pour tout $i \in \llb 1,n \rrb$, la variable 
  aléatoire $T_i$ suit la loi normale $\Norm{au_i+b}{\sigma^2}$.
  
  

  
  \item Les variables aléatoires $T_1,T_2,...,T_n$ sont-elles 
  indépendantes ?
  
  
 \end{noliste}
\end{noliste}

\noindent
Pour tout $i \in \llb 1,n \rrb$, soit $\varphi_i$ la densité continue 
sur $\R$ de $T_i$ : 
\[
 \forall d \in \R, \ 
 \varphi_i(d)=\dfrac{1}{\sigma\sqrt{2\pi}}\exp\left(-\dfrac{1}{2\sigma^2
 } (d-(au_i+b))^2\right)
\]
Soit $\mathcal{F}$ l'ouvert défini par $\mathcal{F}= \ ]0,1[ \times \R$ 
et 
$M$ la fonction de $\mathcal{F}$ dans $\R$ définie par :
\[
 M(a,b)=\ln\left(\Prod{i=1}{n} \varphi_i(t_i)\right)
\]
On suppose que : $0 < \mathrm{cov}(u,t) < s_u^2$.

\begin{noliste}{1.}
 \setlength{\itemsep}{4mm}
 \setcounter{enumi}{9}
 \item 
 \begin{noliste}{a)}
  \setlength{\itemsep}{2mm}
  \item Calculer le gradient $\nabla(M)(a,b)$ de $M$ en tout point 
  $(a,b) \in \mathcal{F}$.
  
  
  
  
  
  %\newpage
  

  
  \item En déduire que $M$ admet sur $\mathcal{F}$ un unique point 
  critique, noté $(\hat{a},\hat{b})$.
  
  
  
  \item Exprimer $\hat{a}$ et $\hat{b}$ en fonction de 
  $\mathrm{cov}(u,t)$, $s_u^2$, $\overline{t}$ et $\overline{u}$.\\
  ({\it $\hat{a}$ et $\hat{b}$ sont les estimations de $a$ et $b$ par 
  la méthode dite du maximum de vraisemblance})
  
  

 \end{noliste}
 
 \item 
 \begin{noliste}{a)}
  \setlength{\itemsep}{2mm}
  \item Soit $\nabla^2(M)(a,b)$ la matrice hessienne de $M$ en $(a,b) 
  \in \mathcal{F}$. \\
  Montrer que $\nabla^2(M)(a,b) = -\dfrac{n}{\sigma^2}
  \begin{smatrix}
   s_u^2+\overline{u}^2 & \overline{u} \\
   \overline{u} & 1
  \end{smatrix}$
  
  
    
  \item En déduire que $M$ admet en $(\hat{a},\hat{b})$ un maximum 
  local.
  
  

 \end{noliste}
 
 
 
 %\newpage
 
 
 
 %\newpage
 
 
 
 \item Soit $(h,k)$ un couple de réels non nuls. Calculer 
 $M(\hat{a}+h,\hat{b}+k)-M(\hat{a},\hat{b}).$\\
 En déduire que $M$ admet en $(\hat{a},\hat{b})$ un maximum global.
 
 
 
 \item On rappelle qu'en \Scilab{}, les commandes \texttt{variance} et 
 \texttt{corr} permettent de calculer respectivement la variance d'une 
 série statistique et la covariance d'une série statistique double.\\
 Si $(v_i)_{1 \leq i \leq n}$ et $(w_i)_{1 \leq i \leq n}$ sont deux 
 séries statistiques, alors la variance de  $(v_i)_{1 \leq i \leq n}$ 
 est calculable par \texttt{variance(v)} et la covariance de  
 $(v_i,w_i)_{1 \leq i \leq n}$ est calculable par 
 \texttt{corr(v,w,1)}.\\
 On a relevé pour $n=16$ entreprises qui produisent le bien considéré à 
 l'époque donnée, les deux séries statistiques $(u_i)_{1 \leq i 
 \leq n}$ et  $(t_i)_{1 \leq i \leq n}$ reproduites dans les lignes 
 \ligne{1} à \ligne{4} du code 
 \Scilab{} suivant dont la ligne \ligne{9} est 
 incomplète :
 
 {\small 
 \begin{scilab}
   & u = [1.06, 0.44, 2.25, 3.88, 0.61, 1.97, 3.43, 2.10, \nl
   & \quad \quad 1.50, 1.68, 2.72, 1.35, 2.94, 2.78, 3.43, 3.58] \nl
   & t = [2.58, 2.25, 2.90, 3.36, 2.41, 2.79, 3.32, 2.81, \nl
   & \quad \quad 2.62, 2.70, 3.17, 2.65, 3.07, 3.13, 3.07, 3.34] \nl
   & plot2d(u, t, -4) \nl
   & \commentaire{-4 signifie que les points sont 
   représentés par des losanges.} \nl
   & plot2d(u, corr(u,t,1)/variance(u)\Sfois{}u + mean(t) - 
   corr(u,t,1)/variance(u)\Sfois{}mean(u)) \nl
   & \commentaire{équation de la 
   droite de régression de t en u.} \nl
   & plot2d(u,...................................................) \nl
   & \commentaire{équation de la droite de régression de u en t.}
 \end{scilab}
 }
 
 \noindent
 Le code précédent complété par la ligne \ligne{9} 
 donne alors la figure suivante :
 \begin{center}
 \resizebox{305pt}{230pt}{
 \begin{tikzpicture}[y=0.80pt, x=0.80pt, yscale=-1.000000, 
  xscale=1.000000, inner sep=0pt, outer sep=0pt,
  draw=black,fill=black,line join=miter,line cap=rect,miter 
  limit=10.00,line width=0.800pt]
  \begin{scope}[draw=white,fill=white]
    \path[fill,rounded corners=0.0000cm] (0.0000,0.0000) rectangle
      (610.0000,460.0000);
  \end{scope}
  \begin{scope}[draw=white,fill=white,line join=bevel,line 
cap=butt,line 
    width=0.000pt]
    \path[fill] (533.7500,57.5000) -- (533.7500,402.5000) -- 
    (76.2500,57.5000) --
      cycle;
    \path[fill] (533.7500,402.5000) -- (76.2500,402.5000) -- 
    (76.2500,57.5000) --
      cycle;
    \path[fill] (533.7500,57.5000) -- (533.7500,402.5000) -- 
    (533.7500,402.5000) --
      cycle;
    \path[fill] (76.2500,57.5000) -- (76.2500,402.5000) -- 
    (76.2500,402.5000) --
      cycle;
    \path[fill] (533.7500,57.5000) -- (533.7500,402.5000) -- 
    (76.2500,57.5000) --
      cycle;
    \path[fill] (533.7500,402.5000) -- (76.2500,402.5000) -- 
    (76.2500,57.5000) --
      cycle;
    \path[fill] (533.7500,57.5000) -- (533.7500,57.5000) -- 
    (533.7500,402.5000) --
      cycle;
    \path[fill] (76.2500,57.5000) -- (76.2500,57.5000) -- 
    (76.2500,402.5000) --
      cycle;
    \path[fill] (533.7500,57.5000) -- (76.2500,57.5000) -- 
    (533.7500,57.5000) --
      cycle;
    \path[fill] (76.2500,57.5000) -- (76.2500,57.5000) -- 
    (533.7500,57.5000) --
      cycle;
    \path[fill] (533.7500,402.5000) -- (76.2500,402.5000) -- 
    (533.7500,402.5000) --
      cycle;
    \path[fill] (76.2500,402.5000) -- (76.2500,402.5000) -- 
    (533.7500,402.5000) --
      cycle;
  \end{scope}
  \begin{scope}[shift={(72.75,411.5)},line join=bevel,line 
cap=butt,line 
    width=0.000pt]
    \path[fill] (-0.4219,10.0537) node[above right] (text3400) {0};
  \end{scope}
  \begin{scope}[shift={(301.5,411.5)},line join=bevel,line 
cap=butt,line 
    width=0.000pt]
    \path[fill] (-0.2969,10.0537) node[above right] (text3404) {2};
  \end{scope}
  \begin{scope}[shift={(530.25,411.5)},line join=bevel,line 
    cap=butt,line width=0.000pt]
    \path[fill] (-0.1250,10.0537) node[above right] (text3408) {4};
  \end{scope}
  \begin{scope}[shift={(188.125,411.5)},line join=bevel,line 
    cap=butt,line width=0.000pt]
    \path[fill] (-1.0938,10.0537) node[above right] (text3412) {1};
  \end{scope}
  \begin{scope}[shift={(415.875,411.5)},line join=bevel,line 
    cap=butt,line width=0.000pt]
    \path[fill] (-0.4219,10.0537) node[above right] (text3416) {3};
  \end{scope}
  \begin{scope}[shift={(125.4375,411.5)},line join=bevel,line 
    cap=butt,line width=0.000pt]
    \path[fill] (-0.4219,10.0537) node[above right] (text3420) {0.5};
  \end{scope}
  \begin{scope}[shift={(239.8125,411.5)},line join=bevel,line 
    cap=butt,line width=0.000pt]
    \path[fill] (-1.0938,10.0537) node[above right] (text3424) {1.5};
  \end{scope}
  \begin{scope}[shift={(354.1875,411.5)},line join=bevel,line 
    cap=butt,line width=0.000pt]
    \path[fill] (-0.2969,10.0537) node[above right] (text3428) {2.5};
  \end{scope}
  \begin{scope}[shift={(468.5625,411.5)},line join=bevel,line 
    cap=butt,line width=0.000pt]
    \path[fill] (-0.4219,10.0537) node[above right] (text3432) {3.5};
  \end{scope}
  \begin{scope}[line cap=butt]
    \path[draw] (76.2500,402.5000) -- (76.2500,408.5000);
    \path[draw] (133.4375,402.5000) -- (133.4375,408.5000);
    \path[draw] (190.6250,402.5000) -- (190.6250,408.5000);
    \path[draw] (247.8125,402.5000) -- (247.8125,408.5000);
    \path[draw] (305.0000,402.5000) -- (305.0000,408.5000);
    \path[draw] (362.1875,402.5000) -- (362.1875,408.5000);
    \path[draw] (419.3750,402.5000) -- (419.3750,408.5000);
    \path[draw] (476.5625,402.5000) -- (476.5625,408.5000);
    \path[draw] (533.7500,402.5000) -- (533.7500,408.5000);
    \path[draw] (76.2500,402.5000) -- (76.2500,405.5000);
    \path[draw] (87.6875,402.5000) -- (87.6875,405.5000);
    \path[draw] (99.1250,402.5000) -- (99.1250,405.5000);
    \path[draw] (110.5625,402.5000) -- (110.5625,405.5000);
    \path[draw] (122.0000,402.5000) -- (122.0000,405.5000);
    \path[draw] (133.4375,402.5000) -- (133.4375,405.5000);
    \path[draw] (144.8750,402.5000) -- (144.8750,405.5000);
    \path[draw] (156.3125,402.5000) -- (156.3125,405.5000);
    \path[draw] (167.7500,402.5000) -- (167.7500,405.5000);
    \path[draw] (179.1875,402.5000) -- (179.1875,405.5000);
    \path[draw] (190.6250,402.5000) -- (190.6250,405.5000);
    \path[draw] (202.0625,402.5000) -- (202.0625,405.5000);
    \path[draw] (213.5000,402.5000) -- (213.5000,405.5000);
    \path[draw] (224.9375,402.5000) -- (224.9375,405.5000);
    \path[draw] (236.3750,402.5000) -- (236.3750,405.5000);
    \path[draw] (247.8125,402.5000) -- (247.8125,405.5000);
    \path[draw] (259.2500,402.5000) -- (259.2500,405.5000);
    \path[draw] (270.6875,402.5000) -- (270.6875,405.5000);
    \path[draw] (282.1250,402.5000) -- (282.1250,405.5000);
    \path[draw] (293.5625,402.5000) -- (293.5625,405.5000);
    \path[draw] (305.0000,402.5000) -- (305.0000,405.5000);
    \path[draw] (316.4375,402.5000) -- (316.4375,405.5000);
    \path[draw] (327.8750,402.5000) -- (327.8750,405.5000);
    \path[draw] (339.3125,402.5000) -- (339.3125,405.5000);
    \path[draw] (350.7500,402.5000) -- (350.7500,405.5000);
    \path[draw] (362.1875,402.5000) -- (362.1875,405.5000);
    \path[draw] (373.6250,402.5000) -- (373.6250,405.5000);
    \path[draw] (385.0625,402.5000) -- (385.0625,405.5000);
    \path[draw] (396.5000,402.5000) -- (396.5000,405.5000);
    \path[draw] (407.9375,402.5000) -- (407.9375,405.5000);
    \path[draw] (419.3750,402.5000) -- (419.3750,405.5000);
    \path[draw] (430.8125,402.5000) -- (430.8125,405.5000);
    \path[draw] (442.2500,402.5000) -- (442.2500,405.5000);
    \path[draw] (453.6875,402.5000) -- (453.6875,405.5000);
    \path[draw] (465.1250,402.5000) -- (465.1250,405.5000);
    \path[draw] (476.5625,402.5000) -- (476.5625,405.5000);
    \path[draw] (488.0000,402.5000) -- (488.0000,405.5000);
    \path[draw] (499.4375,402.5000) -- (499.4375,405.5000);
    \path[draw] (510.8750,402.5000) -- (510.8750,405.5000);
    \path[draw] (522.3125,402.5000) -- (522.3125,405.5000);
    \path[draw] (533.7500,402.5000) -- (533.7500,405.5000);
    \path[draw] (533.7500,402.5000) -- (76.2500,402.5000);
    \path[shift={(60.25,198.3571)},fill,line join=bevel,line 
    width=0.000pt]
      (-0.4219,10.0537) node[above right] (text3538) {3};
  \end{scope}
  \begin{scope}[shift={(51.25,395.5)},line join=bevel,line 
cap=butt,line 
    width=0.000pt]
    \path[fill] (-0.2969,10.0537) node[above right] (text3542) {2.2};
  \end{scope}
  \begin{scope}[shift={(51.25,346.2143)},line join=bevel,line 
    cap=butt,line width=0.000pt]
    \path[fill] (-0.2969,10.0537) node[above right] (text3546) {2.4};
  \end{scope}
  \begin{scope}[shift={(51.25,296.9286)},line join=bevel,line 
    cap=butt,line width=0.000pt]
    \path[fill] (-0.2969,10.0537) node[above right] (text3550) {2.6};
  \end{scope}
  \begin{scope}[shift={(51.25,247.6429)},line join=bevel,line 
    cap=butt,line width=0.000pt]
    \path[fill] (-0.2969,10.0537) node[above right] (text3554) {2.8};
  \end{scope}
  \begin{scope}[shift={(51.25,149.0714)},line join=bevel,line 
    cap=butt,line width=0.000pt]
    \path[fill] (-0.4219,10.0537) node[above right] (text3558) {3.2};
  \end{scope}
  \begin{scope}[shift={(51.25,99.7857)},line join=bevel,line 
    cap=butt,line width=0.000pt]
    \path[fill] (-0.4219,10.0537) node[above right] (text3562) {3.4};
  \end{scope}
  \begin{scope}[shift={(51.25,50.5)},line join=bevel,line cap=butt,line 
    width=0.000pt]
    \path[fill] (-0.4219,10.0537) node[above right] (text3566) {3.6};
  \end{scope}
  \begin{scope}[line cap=butt]
    \path[draw] (76.2500,402.5000) -- (70.2500,402.5000);
    \path[draw] (76.2500,353.2143) -- (70.2500,353.2143);
    \path[draw] (76.2500,303.9286) -- (70.2500,303.9286);
    \path[draw] (76.2500,254.6429) -- (70.2500,254.6429);
    \path[draw] (76.2500,205.3571) -- (70.2500,205.3571);
    \path[draw] (76.2500,156.0714) -- (70.2500,156.0714);
    \path[draw] (76.2500,106.7857) -- (70.2500,106.7857);
    \path[draw] (76.2500,57.5000) -- (70.2500,57.5000);
    \path[draw] (76.2500,402.5000) -- (73.2500,402.5000);
    \path[draw] (76.2500,377.8571) -- (73.2500,377.8571);
    \path[draw] (76.2500,353.2143) -- (73.2500,353.2143);
    \path[draw] (76.2500,328.5714) -- (73.2500,328.5714);
    \path[draw] (76.2500,303.9286) -- (73.2500,303.9286);
    \path[draw] (76.2500,279.2857) -- (73.2500,279.2857);
    \path[draw] (76.2500,254.6429) -- (73.2500,254.6429);
    \path[draw] (76.2500,230.0000) -- (73.2500,230.0000);
    \path[draw] (76.2500,205.3571) -- (73.2500,205.3571);
    \path[draw] (76.2500,180.7143) -- (73.2500,180.7143);
    \path[draw] (76.2500,156.0714) -- (73.2500,156.0714);
    \path[draw] (76.2500,131.4286) -- (73.2500,131.4286);
    \path[draw] (76.2500,106.7857) -- (73.2500,106.7857);
    \path[draw] (76.2500,82.1429) -- (73.2500,82.1429);
    \path[draw] (76.2500,57.5000) -- (73.2500,57.5000);
    \path[draw] (76.2500,57.5000) -- (76.2500,402.5000);
    \path[shift={(197.4875,308.8571)},fill,line join=bevel,line 
    width=0.000pt]
      (-4.0000,0.0000) -- (0.0000,-4.0000) -- (4.0000,0.0000) -- 
      (0.0000,4.0000) --
      cycle;
  \end{scope}
  \begin{scope}[shift={(126.575,390.1786)},line join=bevel,line 
    cap=butt,line width=0.000pt]
    \path[fill] (-4.0000,0.0000) -- (0.0000,-4.0000) -- (4.0000,0.0000) 
    -- (0.0000,4.0000) -- cycle;
  \end{scope}
  \begin{scope}[shift={(333.5938,230.0)},line join=bevel,line 
    cap=butt,line width=0.000pt]
    \path[fill] (-4.0000,0.0000) -- (0.0000,-4.0000) -- (4.0000,0.0000) 
    -- (0.0000,4.0000) -- cycle;
  \end{scope}
  \begin{scope}[shift={(520.025,116.6429)},line join=bevel,line 
    cap=butt,line width=0.000pt]
    \path[fill] (-4.0000,0.0000) -- (0.0000,-4.0000) -- (4.0000,0.0000) 
    -- (0.0000,4.0000) -- cycle;
  \end{scope}
  \begin{scope}[shift={(146.0188,350.75)},line join=bevel,line 
    cap=butt,line width=0.000pt]
    \path[fill] (-4.0000,0.0000) -- (0.0000,-4.0000) -- (4.0000,0.0000) 
    -- (0.0000,4.0000) -- cycle;
  \end{scope}
  \begin{scope}[shift={(301.5688,257.1071)},line join=bevel,line 
    cap=butt,line width=0.000pt]
    \path[fill] (-4.0000,0.0000) -- (0.0000,-4.0000) -- (4.0000,0.0000) 
    -- (0.0000,4.0000) -- cycle;
  \end{scope}
  \begin{scope}[shift={(468.5562,126.5)},line join=bevel,line 
    cap=butt,line width=0.000pt]
    \path[fill] (-4.0000,0.0000) -- (0.0000,-4.0000) -- (4.0000,0.0000) 
    -- (0.0000,4.0000) -- cycle;
  \end{scope}
  \begin{scope}[shift={(316.4375,252.1786)},line join=bevel,line 
cap=butt,line width=0.000pt]
    \path[fill] (-4.0000,0.0000) -- (0.0000,-4.0000) -- (4.0000,0.0000) 
--
      (0.0000,4.0000) -- cycle;
  \end{scope}
  \begin{scope}[shift={(247.8125,299.0)},line join=bevel,line 
cap=butt,line width=0.000pt]
    \path[fill] (-4.0000,0.0000) -- (0.0000,-4.0000) -- (4.0000,0.0000) 
--
      (0.0000,4.0000) -- cycle;
  \end{scope}
  \begin{scope}[shift={(268.4,279.2857)},line join=bevel,line 
cap=butt,line width=0.000pt]
    \path[fill] (-4.0000,0.0000) -- (0.0000,-4.0000) -- (4.0000,0.0000) 
--
      (0.0000,4.0000) -- cycle;
  \end{scope}
  \begin{scope}[shift={(387.35,163.4643)},line join=bevel,line 
cap=butt,line width=0.000pt]
    \path[fill] (-4.0000,0.0000) -- (0.0000,-4.0000) -- (4.0000,0.0000) 
--
      (0.0000,4.0000) -- cycle;
  \end{scope}
  \begin{scope}[shift={(230.6563,291.6071)},line join=bevel,line 
cap=butt,line width=0.000pt]
    \path[fill] (-4.0000,0.0000) -- (0.0000,-4.0000) -- (4.0000,0.0000) 
--
      (0.0000,4.0000) -- cycle;
  \end{scope}
  \begin{scope}[shift={(412.5125,188.1071)},line join=bevel,line 
cap=butt,line width=0.000pt]
    \path[fill] (-4.0000,0.0000) -- (0.0000,-4.0000) -- (4.0000,0.0000) 
--
      (0.0000,4.0000) -- cycle;
  \end{scope}
  \begin{scope}[shift={(394.2125,173.3214)},line join=bevel,line 
cap=butt,line width=0.000pt]
    \path[fill] (-4.0000,0.0000) -- (0.0000,-4.0000) -- (4.0000,0.0000) 
--
      (0.0000,4.0000) -- cycle;
  \end{scope}
  \begin{scope}[shift={(468.5562,188.1071)},line join=bevel,line 
cap=butt,line width=0.000pt]
    \path[fill] (-4.0000,0.0000) -- (0.0000,-4.0000) -- (4.0000,0.0000) 
--
      (0.0000,4.0000) -- cycle;
  \end{scope}
  \begin{scope}[shift={(485.7125,121.5714)},line join=bevel,line 
cap=butt,line width=0.000pt]
    \path[fill] (-4.0000,0.0000) -- (0.0000,-4.0000) -- (4.0000,0.0000) 
--
      (0.0000,4.0000) -- cycle;
  \end{scope}
  \begin{scope}[line cap=butt]
    \path[draw] (197.4875,316.9646) -- (126.5750,361.0770) -- 
(333.5938,232.2973) --
      (520.0250,116.3244) -- (146.0188,348.9817) -- (301.5688,252.2190) 
--
      (468.5562,148.3415) -- (316.4375,242.9697) -- (247.8125,285.6591) 
--
      (268.4000,272.8522) -- (387.3500,198.8573) -- (230.6562,296.3314) 
--
      (412.5125,183.2045) -- (394.2125,194.5883) -- (468.5562,148.3415) 
--
      (485.7125,137.6691);
    \path[draw] (197.4875,333.1348) -- (126.5750,385.7977) -- 
(333.5938,232.0560) --
      (520.0250,93.6035) -- (146.0188,371.3578) -- (301.5688,255.8392) 
--
      (468.5562,131.8265) -- (316.4375,244.7970) -- (247.8125,295.7611) 
--
      (268.4000,280.4719) -- (387.3500,192.1341) -- (230.6562,308.5021) 
--
      (412.5125,173.4472) -- (394.2125,187.0377) -- (468.5562,131.8265) 
--
      (485.7125,119.0855);
  \end{scope}
 \end{tikzpicture}}
 \end{center}
 
 
 
 %\newpage
 
 
 
 \begin{noliste}{a)}
  \setlength{\itemsep}{2mm}
  \item Compléter la ligne \ligne{9} du code 
  permettant d'obtenir la figure précédente ({\it on reportera sur sa 
  copie, uniquement la ligne \ligne{9} complétée}).
  
  
  
  
  
  %\newpage
  
  
  
  \item Interpréter le point d'intersection des deux droites de 
  régression.
  
  

  
  \item Estimer graphiquement les moyennes empiriques $\overline{u}$ et 
  $\overline{t}$.
  
  
  
  \item Le coefficient de corrélation empirique de la série statistique 
  double $(u_i,t_i)_{1 \leq i \leq 16}$ est-il plus 
  proche de $-1$, de $1$ ou de $0$ ?
  
  
  
  
  
  
  %\newpage
  

  
  \item On reprend les lignes \ligne{1} à 
  \ligne{4} du code précédent que l'on complète par les 
  instructions \ligne{11} à \ligne{17} 
  qui suivent et on obtient le graphique ci-dessous :
  
  {\small
  \begin{scilabC}{10}
    & a0 = corr(u,t,1)/variance(u) \nl
    & b0 = mean(t) - corr(u,t,1)/variance(u)\Sfois{}mean(u) \nl
    & t0 = a0 \Sfois{} u + b0 \nl
    & e = t0 - t \nl
    & p = 1:16 \nl
    & plot2d(p,e,-1) \nl
    & \commentaire{-1 signifie que les points sont représentés par des 
    symboles d'addition.}
  \end{scilabC}
  }

 \begin{center}
\resizebox{244pt}{168pt}{						
 
 
\begin{tikzpicture}[y=0.80pt, x=0.80pt, yscale=-1.000000, 
xscale=1.000000, inner sep=0pt, outer sep=0pt,
draw=black,fill=black,line join=miter,line cap=rect,miter 
limit=10.00,line width=0.800pt]
  \begin{scope}[draw=white,fill=white]
    \path[fill,rounded corners=0.0000cm] (0.0000,0.0000) rectangle
      (610.0000,460.0000);
  \end{scope}
  \begin{scope}[draw=white,fill=white,line join=bevel,line 
cap=butt,line 
width=0.000pt]
    \path[fill] (533.7500,57.5000) -- (533.7500,402.5000) -- 
(76.2500,57.5000) --
      cycle;
    \path[fill] (533.7500,402.5000) -- (76.2500,402.5000) -- 
(76.2500,57.5000) --
      cycle;
    \path[fill] (533.7500,57.5000) -- (533.7500,402.5000) -- 
(533.7500,402.5000) --
      cycle;
    \path[fill] (76.2500,57.5000) -- (76.2500,402.5000) -- 
(76.2500,402.5000) --
      cycle;
    \path[fill] (533.7500,57.5000) -- (533.7500,402.5000) -- 
(76.2500,57.5000) --
      cycle;
    \path[fill] (533.7500,402.5000) -- (76.2500,402.5000) -- 
(76.2500,57.5000) --
      cycle;
    \path[fill] (533.7500,57.5000) -- (533.7500,57.5000) -- 
(533.7500,402.5000) --
      cycle;
    \path[fill] (76.2500,57.5000) -- (76.2500,57.5000) -- 
(76.2500,402.5000) --
      cycle;
    \path[fill] (533.7500,57.5000) -- (76.2500,57.5000) -- 
(533.7500,57.5000) --
      cycle;
    \path[fill] (76.2500,57.5000) -- (76.2500,57.5000) -- 
(533.7500,57.5000) --
      cycle;
    \path[fill] (533.7500,402.5000) -- (76.2500,402.5000) -- 
(533.7500,402.5000) --
      cycle;
    \path[fill] (76.2500,402.5000) -- (76.2500,402.5000) -- 
(533.7500,402.5000) --
      cycle;
  \end{scope}
  \begin{scope}[shift={(72.75,411.5)},line join=bevel,line 
cap=butt,line 
width=0.000pt]
    \path[fill] (-0.4219,10.0537) node[above right] (text4122) {0};
  \end{scope}
  \begin{scope}[shift={(356.1875,411.5)},line join=bevel,line 
cap=butt,line width=0.000pt]
    \path[fill] (-1.0938,10.0537) node[above right] (text4126) {10};
  \end{scope}
  \begin{scope}[shift={(129.9375,411.5)},line join=bevel,line 
cap=butt,line width=0.000pt]
    \path[fill] (-0.2969,10.0537) node[above right] (text4130) {2};
  \end{scope}
  \begin{scope}[shift={(187.125,411.5)},line join=bevel,line 
cap=butt,line width=0.000pt]
    \path[fill] (-0.1250,10.0537) node[above right] (text4134) {4};
  \end{scope}
  \begin{scope}[shift={(244.3125,411.5)},line join=bevel,line 
cap=butt,line width=0.000pt]
    \path[fill] (-0.3750,10.0537) node[above right] (text4138) {6};
  \end{scope}
  \begin{scope}[shift={(301.5,411.5)},line join=bevel,line 
cap=butt,line 
width=0.000pt]
    \path[fill] (-0.4062,10.0537) node[above right] (text4142) {8};
  \end{scope}
  \begin{scope}[shift={(413.375,411.5)},line join=bevel,line 
cap=butt,line width=0.000pt]
    \path[fill] (-1.0938,10.0537) node[above right] (text4146) {12};
  \end{scope}
  \begin{scope}[shift={(470.5625,411.5)},line join=bevel,line 
cap=butt,line width=0.000pt]
    \path[fill] (-1.0938,10.0537) node[above right] (text4150) {14};
  \end{scope}
  \begin{scope}[shift={(527.25,411.5)},line join=bevel,line 
cap=butt,line width=0.000pt]
    \path[fill] (-1.0938,10.0537) node[above right] (text4154) {16};
  \end{scope}
  \begin{scope}[shift={(102.3438,411.5)},line join=bevel,line 
cap=butt,line width=0.000pt]
    \path[fill] (-1.0938,10.0537) node[above right] (text4158) {1};
  \end{scope}
  \begin{scope}[shift={(158.5312,411.5)},line join=bevel,line 
cap=butt,line width=0.000pt]
    \path[fill] (-0.4219,10.0537) node[above right] (text4162) {3};
  \end{scope}
  \begin{scope}[shift={(215.7188,411.5)},line join=bevel,line 
cap=butt,line width=0.000pt]
    \path[fill] (-0.4219,10.0537) node[above right] (text4166) {5};
  \end{scope}
  \begin{scope}[shift={(272.9062,411.5)},line join=bevel,line 
cap=butt,line width=0.000pt]
    \path[fill] (-0.4688,10.0537) node[above right] (text4170) {7};
  \end{scope}
  \begin{scope}[shift={(330.0938,411.5)},line join=bevel,line 
cap=butt,line width=0.000pt]
    \path[fill] (-0.4219,10.0537) node[above right] (text4174) {9};
  \end{scope}
  \begin{scope}[shift={(385.2812,411.5)},line join=bevel,line 
cap=butt,line width=0.000pt]
    \path[fill] (-1.0938,10.0537) node[above right] (text4178) {11};
  \end{scope}
  \begin{scope}[shift={(441.4688,411.5)},line join=bevel,line 
cap=butt,line width=0.000pt]
    \path[fill] (-1.0938,10.0537) node[above right] (text4182) {13};
  \end{scope}
  \begin{scope}[shift={(498.6562,411.5)},line join=bevel,line 
cap=butt,line width=0.000pt]
    \path[fill] (-1.0938,10.0537) node[above right] (text4186) {15};
  \end{scope}
  \begin{scope}[line cap=butt]
    \path[draw] (76.2500,402.5000) -- (76.2500,408.5000);
    \path[draw] (104.8438,402.5000) -- (104.8438,408.5000);
    \path[draw] (133.4375,402.5000) -- (133.4375,408.5000);
    \path[draw] (162.0312,402.5000) -- (162.0312,408.5000);
    \path[draw] (190.6250,402.5000) -- (190.6250,408.5000);
    \path[draw] (219.2188,402.5000) -- (219.2188,408.5000);
    \path[draw] (247.8125,402.5000) -- (247.8125,408.5000);
    \path[draw] (276.4062,402.5000) -- (276.4062,408.5000);
    \path[draw] (305.0000,402.5000) -- (305.0000,408.5000);
    \path[draw] (333.5938,402.5000) -- (333.5938,408.5000);
    \path[draw] (362.1875,402.5000) -- (362.1875,408.5000);
    \path[draw] (390.7812,402.5000) -- (390.7812,408.5000);
    \path[draw] (419.3750,402.5000) -- (419.3750,408.5000);
    \path[draw] (447.9688,402.5000) -- (447.9688,408.5000);
    \path[draw] (476.5625,402.5000) -- (476.5625,408.5000);
    \path[draw] (505.1562,402.5000) -- (505.1562,408.5000);
    \path[draw] (533.7500,402.5000) -- (533.7500,408.5000);
    \path[draw] (76.2500,402.5000) -- (76.2500,405.5000);
    \path[draw] (90.5469,402.5000) -- (90.5469,405.5000);
    \path[draw] (104.8438,402.5000) -- (104.8438,405.5000);
    \path[draw] (119.1406,402.5000) -- (119.1406,405.5000);
    \path[draw] (133.4375,402.5000) -- (133.4375,405.5000);
    \path[draw] (147.7344,402.5000) -- (147.7344,405.5000);
    \path[draw] (162.0312,402.5000) -- (162.0312,405.5000);
    \path[draw] (176.3281,402.5000) -- (176.3281,405.5000);
    \path[draw] (190.6250,402.5000) -- (190.6250,405.5000);
    \path[draw] (204.9219,402.5000) -- (204.9219,405.5000);
    \path[draw] (219.2188,402.5000) -- (219.2188,405.5000);
    \path[draw] (233.5156,402.5000) -- (233.5156,405.5000);
    \path[draw] (247.8125,402.5000) -- (247.8125,405.5000);
    \path[draw] (262.1094,402.5000) -- (262.1094,405.5000);
    \path[draw] (276.4062,402.5000) -- (276.4062,405.5000);
    \path[draw] (290.7031,402.5000) -- (290.7031,405.5000);
    \path[draw] (305.0000,402.5000) -- (305.0000,405.5000);
    \path[draw] (319.2969,402.5000) -- (319.2969,405.5000);
    \path[draw] (333.5938,402.5000) -- (333.5938,405.5000);
    \path[draw] (347.8906,402.5000) -- (347.8906,405.5000);
    \path[draw] (362.1875,402.5000) -- (362.1875,405.5000);
    \path[draw] (376.4844,402.5000) -- (376.4844,405.5000);
    \path[draw] (390.7812,402.5000) -- (390.7812,405.5000);
    \path[draw] (405.0781,402.5000) -- (405.0781,405.5000);
    \path[draw] (419.3750,402.5000) -- (419.3750,405.5000);
    \path[draw] (433.6719,402.5000) -- (433.6719,405.5000);
    \path[draw] (447.9688,402.5000) -- (447.9688,405.5000);
    \path[draw] (462.2656,402.5000) -- (462.2656,405.5000);
    \path[draw] (476.5625,402.5000) -- (476.5625,405.5000);
    \path[draw] (490.8594,402.5000) -- (490.8594,405.5000);
    \path[draw] (505.1562,402.5000) -- (505.1562,405.5000);
    \path[draw] (519.4531,402.5000) -- (519.4531,405.5000);
    \path[draw] (533.7500,402.5000) -- (533.7500,405.5000);
    \path[draw] (533.7500,402.5000) -- (76.2500,402.5000);
    \path[shift={(60.25,247.6429)},fill,line join=bevel,line 
width=0.000pt]
      (-0.4219,10.0537) node[above right] (text4292) {0};
  \end{scope}
  \begin{scope}[shift={(51.25,50.5)},line join=bevel,line cap=butt,line 
width=0.000pt]
    \path[fill] (-0.4219,10.0537) node[above right] (text4296) {0.2};
  \end{scope}
  \begin{scope}[shift={(49.25,346.2143)},line join=bevel,line 
cap=butt,line width=0.000pt]
    \path[fill] (-0.3125,10.0537) node[above right] (text4300) {-0.1};
  \end{scope}
  \begin{scope}[shift={(52.25,149.0714)},line join=bevel,line 
cap=butt,line width=0.000pt]
    \path[fill] (-0.4219,10.0537) node[above right] (text4304) {0.1};
  \end{scope}
  \begin{scope}[shift={(42.25,395.5)},line join=bevel,line 
cap=butt,line 
width=0.000pt]
    \path[fill] (-0.3125,10.0537) node[above right] (text4308) {-0.15};
  \end{scope}
  \begin{scope}[shift={(42.25,296.9286)},line join=bevel,line 
cap=butt,line width=0.000pt]
    \path[fill] (-0.3125,10.0537) node[above right] (text4312) {-0.05};
  \end{scope}
  \begin{scope}[shift={(45.25,198.3571)},line join=bevel,line 
cap=butt,line width=0.000pt]
    \path[fill] (-0.4219,10.0537) node[above right] (text4316) {0.05};
  \end{scope}
  \begin{scope}[shift={(45.25,99.7857)},line join=bevel,line 
cap=butt,line width=0.000pt]
    \path[fill] (-0.4219,10.0537) node[above right] (text4320) {0.15};
  \end{scope}
  \begin{scope}[line cap=butt]
    \path[draw] (76.2500,402.5000) -- (70.2500,402.5000);
    \path[draw] (76.2500,353.2143) -- (70.2500,353.2143);
    \path[draw] (76.2500,303.9286) -- (70.2500,303.9286);
    \path[draw] (76.2500,254.6429) -- (70.2500,254.6429);
    \path[draw] (76.2500,205.3571) -- (70.2500,205.3571);
    \path[draw] (76.2500,156.0714) -- (70.2500,156.0714);
    \path[draw] (76.2500,106.7857) -- (70.2500,106.7857);
    \path[draw] (76.2500,57.5000) -- (70.2500,57.5000);
    \path[draw] (76.2500,402.5000) -- (73.2500,402.5000);
    \path[draw] (76.2500,392.6429) -- (73.2500,392.6429);
    \path[draw] (76.2500,382.7857) -- (73.2500,382.7857);
    \path[draw] (76.2500,372.9286) -- (73.2500,372.9286);
    \path[draw] (76.2500,363.0714) -- (73.2500,363.0714);
    \path[draw] (76.2500,353.2143) -- (73.2500,353.2143);
    \path[draw] (76.2500,343.3571) -- (73.2500,343.3571);
    \path[draw] (76.2500,333.5000) -- (73.2500,333.5000);
    \path[draw] (76.2500,323.6429) -- (73.2500,323.6429);
    \path[draw] (76.2500,313.7857) -- (73.2500,313.7857);
    \path[draw] (76.2500,303.9286) -- (73.2500,303.9286);
    \path[draw] (76.2500,294.0714) -- (73.2500,294.0714);
    \path[draw] (76.2500,284.2143) -- (73.2500,284.2143);
    \path[draw] (76.2500,274.3571) -- (73.2500,274.3571);
    \path[draw] (76.2500,264.5000) -- (73.2500,264.5000);
    \path[draw] (76.2500,254.6429) -- (73.2500,254.6429);
    \path[draw] (76.2500,244.7857) -- (73.2500,244.7857);
    \path[draw] (76.2500,234.9286) -- (73.2500,234.9286);
    \path[draw] (76.2500,225.0714) -- (73.2500,225.0714);
    \path[draw] (76.2500,215.2143) -- (73.2500,215.2143);
    \path[draw] (76.2500,205.3571) -- (73.2500,205.3571);
    \path[draw] (76.2500,195.5000) -- (73.2500,195.5000);
    \path[draw] (76.2500,185.6429) -- (73.2500,185.6429);
    \path[draw] (76.2500,175.7857) -- (73.2500,175.7857);
    \path[draw] (76.2500,165.9286) -- (73.2500,165.9286);
    \path[draw] (76.2500,156.0714) -- (73.2500,156.0714);
    \path[draw] (76.2500,146.2143) -- (73.2500,146.2143);
    \path[draw] (76.2500,136.3571) -- (73.2500,136.3571);
    \path[draw] (76.2500,126.5000) -- (73.2500,126.5000);
    \path[draw] (76.2500,116.6429) -- (73.2500,116.6429);
    \path[draw] (76.2500,106.7857) -- (73.2500,106.7857);
    \path[draw] (76.2500,96.9286) -- (73.2500,96.9286);
    \path[draw] (76.2500,87.0714) -- (73.2500,87.0714);
    \path[draw] (76.2500,77.2143) -- (73.2500,77.2143);
    \path[draw] (76.2500,67.3571) -- (73.2500,67.3571);
    \path[draw] (76.2500,57.5000) -- (73.2500,57.5000);
    \path[draw] (76.2500,57.5000) -- (76.2500,402.5000);
    \path[shift={(104.8438,287.0728)},draw] (-4.0000,0.0000) -- 
(4.0000,0.0000);
    \path[shift={(104.8438,287.0728)},draw] (0.0000,-4.0000) -- 
(0.0000,4.0000);
  \end{scope}
  \begin{scope}[shift={(133.4375,138.2366)},line cap=butt]
    \path[draw] (-4.0000,0.0000) -- (4.0000,0.0000);
    \path[draw] (0.0000,-4.0000) -- (0.0000,4.0000);
  \end{scope}
  \begin{scope}[shift={(162.0312,263.8321)},line cap=butt]
    \path[draw] (-4.0000,0.0000) -- (4.0000,0.0000);
    \path[draw] (0.0000,-4.0000) -- (0.0000,4.0000);
  \end{scope}
  \begin{scope}[shift={(190.625,253.3691)},line cap=butt]
    \path[draw] (-4.0000,0.0000) -- (4.0000,0.0000);
    \path[draw] (0.0000,-4.0000) -- (0.0000,4.0000);
  \end{scope}
  \begin{scope}[shift={(219.2188,247.5695)},line cap=butt]
    \path[draw] (-4.0000,0.0000) -- (4.0000,0.0000);
    \path[draw] (0.0000,-4.0000) -- (0.0000,4.0000);
  \end{scope}
  \begin{scope}[shift={(247.8125,235.0904)},line cap=butt]
    \path[draw] (-4.0000,0.0000) -- (4.0000,0.0000);
    \path[draw] (0.0000,-4.0000) -- (0.0000,4.0000);
  \end{scope}
  \begin{scope}[shift={(276.4062,342.0088)},line cap=butt]
    \path[draw] (-4.0000,0.0000) -- (4.0000,0.0000);
    \path[draw] (0.0000,-4.0000) -- (0.0000,4.0000);
  \end{scope}
  \begin{scope}[shift={(305.0,217.8072)},line cap=butt]
    \path[draw] (-4.0000,0.0000) -- (4.0000,0.0000);
    \path[draw] (0.0000,-4.0000) -- (0.0000,4.0000);
  \end{scope}
  \begin{scope}[shift={(333.5938,201.2791)},line cap=butt]
    \path[draw] (-4.0000,0.0000) -- (4.0000,0.0000);
    \path[draw] (0.0000,-4.0000) -- (0.0000,4.0000);
  \end{scope}
  \begin{scope}[shift={(362.1875,228.9089)},line cap=butt]
    \path[draw] (-4.0000,0.0000) -- (4.0000,0.0000);
    \path[draw] (0.0000,-4.0000) -- (0.0000,4.0000);
  \end{scope}
  \begin{scope}[shift={(390.7812,396.2148)},line cap=butt]
    \path[draw] (-4.0000,0.0000) -- (4.0000,0.0000);
    \path[draw] (0.0000,-4.0000) -- (0.0000,4.0000);
  \end{scope}
  \begin{scope}[shift={(419.375,273.5399)},line cap=butt]
    \path[draw] (-4.0000,0.0000) -- (4.0000,0.0000);
    \path[draw] (0.0000,-4.0000) -- (0.0000,4.0000);
  \end{scope}
  \begin{scope}[shift={(447.9688,235.0322)},line cap=butt]
    \path[draw] (-4.0000,0.0000) -- (4.0000,0.0000);
    \path[draw] (0.0000,-4.0000) -- (0.0000,4.0000);
  \end{scope}
  \begin{scope}[shift={(476.5625,339.7105)},line cap=butt]
    \path[draw] (-4.0000,0.0000) -- (4.0000,0.0000);
    \path[draw] (0.0000,-4.0000) -- (0.0000,4.0000);
  \end{scope}
  \begin{scope}[shift={(505.1562,95.5802)},line cap=butt]
    \path[draw] (-4.0000,0.0000) -- (4.0000,0.0000);
    \path[draw] (0.0000,-4.0000) -- (0.0000,4.0000);
  \end{scope}
  \begin{scope}[shift={(533.75,319.0337)},line cap=butt]
    \path[draw] (-4.0000,0.0000) -- (4.0000,0.0000);
    \path[draw] (0.0000,-4.0000) -- (0.0000,4.0000);
  \end{scope}
  \begin{scope}[shift={(533.75,319.0337)},line cap=butt]
    \path[draw] (-4.0000,0.0000) -- (4.0000,0.0000);
    \path[draw] (0.0000,-4.0000) -- (0.0000,4.0000);
  \end{scope}
  \begin{scope}[shift={(533.75,319.0337)},line cap=butt]
    \path[draw] (-4.0000,0.0000) -- (4.0000,0.0000);
    \path[draw] (0.0000,-4.0000) -- (0.0000,4.0000);
  \end{scope}

\end{tikzpicture}}
\end{center}

 Que représente ce graphique ?
 Quelle valeur peut-on conjecturer pour 
 la moyenne des ordonnées des 16 points obtenus sur le graphique ?\\
 Déterminer mathématiquement la valeur de cette moyenne.
 
 
 \end{noliste}
 
 \item Pour tout entier $n \geq 1,$ on pose 
 $A_n=\dfrac{1}{n \, s_u^2} \ \Sum{i=1}{n} (u_i-\overline{u})T_i$. On 
 suppose que le paramètre $\sigma^2$ est connu.
 \begin{noliste}{a)}
  \item Calculer l'espérance $\E(A_n)$ et la variance $\V(A_n)$ de la 
  variable aléatoire $A_n$. \\
  Préciser la loi de $A_n$.
  
  

  
  \item On suppose que $a$ est un paramètre inconnu. Soit $\alpha$ un 
  réel donné vérifiant $0 < \alpha < 1.$\\
  On note $\Phi$ la fonction de répartition de la loi normale centrée 
  réduite et $d_\alpha$ le réel tel que $\Phi(d_\alpha) = 
  1-\dfrac{\alpha}{2}.$\\
  Déterminer un intervalle de confiance du paramètre $a$ au niveau de 
  confiance $1-\alpha$.
  
  
 \end{noliste}
\end{noliste}








\end{document}
