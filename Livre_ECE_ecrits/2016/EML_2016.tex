\documentclass[11pt]{article}%
\usepackage{geometry}%
\geometry{a4paper,
  lmargin=2cm,rmargin=2cm,tmargin=2.5cm,bmargin=2.5cm}

\input{../macros_Livre.tex}

% \renewcommand{\thesection}{\Roman{section}.\hspace{-.3cm}}
% \renewcommand{\thesubsection}{\Alph{subsection}.\hspace{-.2cm}}

\pagestyle{fancy} %
\pagestyle{fancy} %
 \lhead{ECE2 \hfill Mathématiques \\} %
\chead{\hrule} %
\rhead{} %
\lfoot{} %
\cfoot{} %
\rfoot{\thepage} %

\renewcommand{\headrulewidth}{0pt}% : Trace un trait de séparation
                                    % de largeur 0,4 point. Mettre 0pt
                                    % pour supprimer le trait.

\renewcommand{\footrulewidth}{0.4pt}% : Trace un trait de séparation
                                    % de largeur 0,4 point. Mettre 0pt
                                    % pour supprimer le trait.

\setlength{\headheight}{14pt}

\title{\bf \vspace{-1.6cm} EML 2016} %
\author{} %
\date{} %
\begin{document}

\maketitle %
\vspace{-1.2cm}\hrule %
\thispagestyle{fancy}

\vspace*{.4cm}

%%DEBUT

\section*{EXERCICE I}

\noindent
On note $I$ et $A$ les matrices de $\M{3}$ définies 
par :
\[
I=
\begin{smatrix}
 1 & 0 & 0\\
 0 & 1 & 0\\
 0 & 0 & 1
\end{smatrix}
, \qquad 
A=
\begin{smatrix}
 0 & 1 & 0\\
 1 & 0 & 1\\
 0 & 1 & 0
\end{smatrix},
\]
et $\mathcal{E}$ l'ensemble des matrices de $\M{3}$ 
défini par :
\[
\mathcal{E}=\left\{
\begin{smatrix}
 a+c & b & c\\
 b & a+2c & b\\
 c & b & a+c
\end{smatrix} \; ; \; (a,b,c) \in \R^3 \right\}
\]

\subsection*{PARTIE I : Étude de la matrice $A$}

\begin{noliste}{1.}
\setlength{\itemsep}{2mm}
\item Calculer $A^2$.




\item Montrer que la famille $(I,A,A^2)$ est libre.



%\newpage

\item
\begin{noliste}{a)}
\item Justifier, sans calcul, que $A$ est diagonalisable.



\item Déterminer une matrice $P$ de $\M{3}$ 
inversible dont tous les coefficients de la première ligne sont égaux à 
1 et une matrice $D$ de $\M{3}$ diagonale dont tous 
les coefficients diagonaux sont dans l'ordre croissant telles que : 
$A=PDP^{-1}$.


\end{noliste}

\item Montrer : $A^3=2A$.


\end{noliste}

\subsection*{PARTIE II : Étude d'une application définie sur 
$\mathcal{E}$}
\begin{noliste}{1.}
\setcounter{enumi}{4}
\item Montrer que $\mathcal{E}$ est un sous-espace vectoriel de 
$\M{3}$ et que la famille $(I,A,A^2)$ est une base 
de $\mathcal{E}$. En déduire la dimension de $\mathcal{E}$.



%\newpage

\item Montrer que, pour toute matrice $M$ de $\mathcal{E}$, la matrice 
$AM$ appartient à $\mathcal{E}$.


\end{noliste}

\noindent
On note $f$ l'application de $\mathcal{E}$ dans $\mathcal{E}$ qui, à 
toute matrice $M$ de $\mathcal{E}$, associe $AM$.

\begin{noliste}{1.}
\setcounter{enumi}{6}
\item Vérifier que $f$ est un endomorphisme de l'espace vectoriel 
$\mathcal{E}$.



\item Former la matrice $F$ de $f$ dans la base $(I,A,A^2)$ de 
$\mathcal{E}$.



\item
\begin{noliste}{a)}
\item Montrer : $f\circ f\circ f=2f$.



%\newpage

\item En déduire que toute valeur propre $\lambda$ de $f$ vérifie : 
$\lambda^3=2\lambda$.



\item Déterminer les valeurs propres et les sous-espaces propres de $f$.


\end{noliste}

\item L'endomorphisme $f$ est-il bijectif ? diagonalisable ?



\item Déterminer une base de $\im(f)$ et une base de 
$\kr(f)$.



\item 
\begin{noliste}{a)}
\item Résoudre l'équation $f(M)=I+A^2$, d'inconnue $M\in \mathcal{E}$.




%\newpage


\item Résoudre l'équation $f(N)=A+A^2$, d'inconnue $N\in \mathcal{E}$.



\end{noliste}
\end{noliste}


%\newpage


\section*{EXERCICE II}

\noindent
On considère l'application $f: [0, +\infty[\to \R$ définie, pour tout
$t$ de $[0, +\infty[$, par :
\[
f(t) =
\left\{
  \begin{array}{cR{1.4cm}}
    t^2-t\ln(t) & si $t\neq 0$ \nl
    0 & si $t = 0$
  \end{array}
\right.
\]
On admet : $0, \ 69 < \ln(2)<0, \ 70$.

\subsection*{PARTIE I :  Étude de la fonction $f$}

\begin{noliste}{1.}
  \setlength{\itemsep}{2mm}
\item Montrer que $f$ est continue sur $[0, +\infty[$.
  
  


\item Justifier que $f$ est de classe $\Cont{2}$ sur $]0,+\infty[$ et
  calculer, pour tout $t$ de $]0,+\infty[$, $f'(t)$ et $f''(t)$.

  

\item Dresser le tableau des variations de $f$. On précisera la limite
  de $f$ en $+\infty$.



\item On note $C$ la courbe représentative de $f$ dans un repère 
orthonormal $(0, \vec{i} ,\vec{j})$.
\begin{noliste}{a)}
\item Montrer que $C$ admet une tangente en $0$ et préciser celle-ci.




\item Montrer que $C$ admet un point d'inflexion et un seul, noté $I$, 
et préciser les coordonnées de $I$.




%\newpage


\item Tracer l'allure de $C$.

  
  
\end{noliste}

\item Montrer que l'équation $f(t)=1$, d'inconnue $t\in[0,+\infty[$, 
admet une solution et une seule et que celle-ci est égale à $1$.



\end{noliste}


%\newpage


\subsection*{PARTIE II : Étude d'une fonction $F$ de deux 
variables réelles}

\noindent
On considère l'application $F: \ ]0,+\infty[^2 \to \R$ de 
classe $\Cont{2}$, définie, pour tout $(x,y)$ de $]0,+\infty[^2$ , par 
: 
\[
F(x,y)=x\ln(y)-y\ln(x)
\]
\begin{noliste}{1.}
  \setlength{\itemsep}{2mm} %
  \setcounter{enumi}{5}
\item Calculer les dérivées partielles premières de $F$ en tout
  $(x,y)$ de $]0,+\infty[^2$.

  
  

\item
\begin{noliste}{a)}
\item Soit $(x,y)\in \ ]0,+\infty[^2$. Montrer que $(x,y)$ est un point 
critique de $F$ si et seulement si :
\[
x > 1, \quad y=\dfrac{x}{\ln(x)} \quad \text{et} \quad 
f\left(\ln(x)\right)=1
\]




\item Établir que $F$ admet un point critique et un seul et qu'il
  s'agit de $(\ee,\ee)$.

  

\end{noliste}

\item La fonction $F$ admet-elle un extremum local en $(\ee,\ee)$?

  
  
\end{noliste}

\subsection*{PARTIE III : Étude d'une suite récurrente}

\noindent
On considère la suite $(u_n)_{n\in \N}$ définie par : $u_0 =
\dfrac{1}{2}$ \quad et \quad $\forall n\in \N$, $u_{n+1} = f(u_n)$.

\begin{noliste}{1.}
\setlength{\itemsep}{2mm}
\setcounter{enumi}{8}
\item Montrer : $\forall n\in \N$, $u_n \in \left[\frac{1}{2},
    1\right]$.

  

\item Montrer que la suite $(u_n)_{n\in \N}$ est croissante.

  

\item En déduire que la suite $(u_n)_{n\in \N}$ converge et déterminer
  sa limite.\\
  {\it (on pourra étudier les variations de la fonction $t\mapsto
    t-\ln(t)$)}



\item Écrire un programme en \Scilab{} qui calcule et affiche un entier 
naturel $N$ tel que $1-u_N<10^{-4}$.



\end{noliste}

%\newpage

\section*{EXERCICE III}


\subsection*{PARTIE I : Étude d'une variable aléatoire}

\noindent
On considère l'application $f:\R\rightarrow\R$ définie, 
pour tout $t$ de $\R$, par : 
$f(t)=\dfrac{\ee^{-t}}{(1+\ee^{-t})^2}$.

\begin{noliste}{1.}
\setlength{\itemsep}{2mm}
\item Vérifier que la fonction $f$ est paire.





%\newpage



\item Montrer que  $f$ est une densité d'une variable aléatoire réelle.



\end{noliste}

\noindent
Dans toute la suite de l'exercice, on considère une variable aléatoire 
réelle $X$ à densité, de densité $f$.
\begin{noliste}{1.}
\setlength{\itemsep}{2mm}
\setcounter{enumi}{2}
\item Déterminer la fonction de répartition de $X$.



%\newpage

\item
\begin{noliste}{a)}
\item Montrer que l'intégrale $\dint{0}{+\infty}t \, f(t)\dt$ 
converge.



\item En utilisant l'imparité de la fonction $\R\to \R$, $t\mapsto t
  \, f(t)$, montrer que $X$ admet une espérance et que l'on a :
  $\E(X)=0$.



\end{noliste}
\end{noliste}

\subsection*{PARTIE II. Étude d'une autre variable aléatoire}

\noindent
On considère l'application $\varphi: \R\rightarrow\R$ définie, pour
tout $x$ de $\R$, par : $\varphi(x) = \ln(1+\ee^x)$.
\begin{noliste}{1.}
  \setlength{\itemsep}{2mm}%
  \setcounter{enumi}{4}
\item Montrer que $\varphi$ est une bijection de $\R$ sur un 
  intervalle $I$ à préciser.



\item Exprimer, pour tout $y$ de $I$, $\varphi^{-1}(y)$.



\end{noliste}


%\newpage


\noindent
On considère la variable aléatoire réelle $Y$ définie par : 
$Y=\varphi(X)$.
\begin{noliste}{1.}
\setlength{\itemsep}{2mm}
\setcounter{enumi}{6}
\item Justifier : $\Prob(\Ev{Y\leq 0})=0$.



\item Déterminer la fonction de répartition de $Y$.




\item Reconnaître alors la loi de $Y$ et donner, sans calcul, son 
espérance et sa variance.



\end{noliste}


%\newpage


\subsection*{PARTIE III : Étude d'une convergence en loi}

\noindent
On considère une suite de variables aléatoires réelles $(X_n)_{n\in
  \N^*}$, mutuellement indépendantes, de même densité $f$, où $f$ a
été définie dans la partie I.\\
On pose, pour tout $n$ de $\N^*$ : $T_n=\max(X_1,\ldots,X_n)$ et
$U_n=T_n-\ln(n)$.

\begin{noliste}{1.}
  \setlength{\itemsep}{2mm}
  \setcounter{enumi}{9}
\item
  \begin{noliste}{a)}
  \item Déterminer, pour tout $n$ de $\N^*$, la fonction de
    répartition de $T_n$.

    
    
  \item En déduire : $\forall n\in \N^*, \ \forall x\in \R, \
    \Prob(\Ev{U_n\leq x}) = \left(1 + \dfrac{\ee^{-x}}{n}
    \right)^{-n}$.
    
    
  \end{noliste}

\item En déduire que la suite de variables aléatoires
  $(U_n)_{n\in\N^*}$ converge en loi vers une variable aléatoire
  réelle à densité dont on précisera la fonction de répartition et une
  densité.



\end{noliste} 

%%% VERSION ROXANE %%%
%%% la version au-dessus est une version "corrigé du DS8vA"
% \subsection*{PARTIE I : Étude d'une variable aléatoire}

% \noindent
% On considère l'application $f:\R\rightarrow\R$ définie, 
% pour tout $t$ de $\R$, par : 
% $f(t)=\dfrac{\ee^{-t}}{(1+\ee^{-t})^2}$.

% \begin{noliste}{1.}
% \setlength{\itemsep}{2mm}
% \item Vérifier que la fonction $f$ est paire.

% 



% %\newpage



% \item Montrer que  $f$ est une densité d'une variable aléatoire réelle.

% 

% \end{noliste}

% \noindent
% Dans toute la suite de l'exercice, on considère une variable aléatoire 
% réelle $X$ à densité, de densité $f$.
% \begin{noliste}{1.}
% \setlength{\itemsep}{2mm}
% \setcounter{enumi}{2}
% \item Déterminer la fonction de répartition de $X$.

% 

% %\newpage

% \item
% \begin{noliste}{a)}
% \item Montrer que l'intégrale $\dint{0}{+\infty}t \, f(t)\dt$ 
% converge.

% 



% \item En utilisant l'imparité de la fonction $\R\to 
% \R$, $t\mapsto t \, f(t)$, montrer que $X$ admet une espérance et 
% que l'on a : $\E(X)=0$.

% 

% \end{noliste}
% \end{noliste}


% \subsection*{PARTIE II. Étude d'une autre variable aléatoire}

% \noindent
% On considère l'application $\varphi: \R\rightarrow\R$ 
% définie, pour tout $x$ de $\R$, par : $\varphi(x)=\ln(1+\ee^x)$.
% \begin{noliste}{1.}
% \setlength{\itemsep}{2mm}
% \setcounter{enumi}{4}
% \item Montrer que $\varphi$ est une bijection de $\R$ sur un 
% intervalle $I$ à préciser.

% 



% %\newpage


% \item Exprimer, pour tout $y$ de $I$, $\varphi^{-1}(y)$.

% 

% \end{noliste}



% \noindent
% On considère la variable aléatoire réelle $Y$ définie par : 
% $Y=\varphi(X)$.
% \begin{noliste}{1.}
% \setlength{\itemsep}{2mm}
% \setcounter{enumi}{6}
% \item Justifier : $\Prob(\Ev{Y\leq 0})=0$.

% 


% \item Déterminer la fonction de répartition de $Y$.

% 


% \item Reconnaître alors la loi de $Y$ et donner, sans calcul, son 
% espérance et sa variance.

% 

% \end{noliste}

% \subsection*{PARTIE III : Étude d'une convergence en loi}

% \noindent
% On considère une suite de variables aléatoires réelles $(X_n)_{n\in 
% \N^*}$, mutuellement indépendantes, de même densité $f$, où $f$ 
% a été définie dans la partie I.\\
% On pose, pour tout $n$ de $\N^*$ : $T_n=\max(X_1,\ldots,X_n)$ et 
% $U_n=T_n-\ln(n)$.

% \begin{noliste}{1.}
% \setlength{\itemsep}{2mm}
% \setcounter{enumi}{9}
% \item
% \begin{noliste}{a)}
% \item Déterminer, pour tout $n$ de $\N^*$, la fonction de 
% répartition de $T_n$.

% 



% %\newpage


% \item En déduire : $\forall n\in \N^*, \quad \forall x\in 
% \R, \quad \Prob(\Ev{U_n\leq 
% x})=\left(1+\dfrac{\ee^{-x}}{n}\right)^{-n}$.

% 
% \end{noliste}



% \item En déduire que la suite de variables aléatoires 
% $(U_n)_{n\in\N^*}$ converge en loi vers une variable aléatoire 
% réelle à densité dont on précisera la fonction de répartition et une 
% densité.

% 

% \end{noliste} 







\end{document}
