\documentclass[11pt]{article}%
\usepackage{geometry}%
\geometry{a4paper,
  lmargin=2cm,rmargin=2cm,tmargin=2.5cm,bmargin=2.5cm}

\usepackage{array}
\usepackage{paralist}

\usepackage[svgnames, usenames, dvipsnames]{xcolor}
\xdefinecolor{RecColor}{named}{Aqua}
\xdefinecolor{IncColor}{named}{Aqua}
\xdefinecolor{ImpColor}{named}{PaleGreen}

% \usepackage{frcursive}

\usepackage{adjustbox}

%%%%%%%%%%%
\newcommand{\cRB}[1]{{\color{Red} \pmb{#1}}} %
\newcommand{\cR}[1]{{\color{Red} {#1}}} %
\newcommand{\cBB}[1]{{\color{Blue} \pmb{#1}}}
\newcommand{\cB}[1]{{\color{Blue} {#1}}}
\newcommand{\cGB}[1]{{\color{LimeGreen} \pmb{#1}}}
\newcommand{\cG}[1]{{\color{LimeGreen} {#1}}}

%%%%%%%%%%

\usepackage{diagbox} %
\usepackage{colortbl} %
\usepackage{multirow} %
\usepackage{pgf} %
\usepackage{environ} %
\usepackage{fancybox} %
\usepackage{textcomp} %
\usepackage{marvosym} %

%%%%%%%%%% pour qu'une cellcolor ne recouvre pas le trait du tableau
\usepackage{hhline}%

\usepackage{pgfplots}
\pgfplotsset{compat=1.10}
\usepgfplotslibrary{patchplots}
\usepgfplotslibrary{fillbetween}
\usepackage{tikz,tkz-tab}
\usepackage{ifthen}
\usepackage{calc}
\usetikzlibrary{calc,decorations.pathreplacing,arrows,positioning} 
\usetikzlibrary{fit,shapes,backgrounds}
\usepackage[nomessages]{fp}% http://ctan.org/pkg/fp

\usetikzlibrary{matrix,arrows,decorations.pathmorphing,
  decorations.pathreplacing} 

\newcommand{\myunit}{1 cm}
\tikzset{
    node style sp/.style={draw,circle,minimum size=\myunit},
    node style ge/.style={circle,minimum size=\myunit},
    arrow style mul/.style={draw,sloped,midway,fill=white},
    arrow style plus/.style={midway,sloped,fill=white},
}

%%%%%%%%%%%%%%
%%%%% écrire des inférieur égal ou supérieur égal avec typographie
%%%%% francaise
%%%%%%%%%%%%%

\renewcommand{\geq}{\geqslant}
\renewcommand{\leq}{\leqslant}
\renewcommand{\emptyset}{\varnothing}

\newcommand{\Leq}{\leqslant}
\newcommand{\Geq}{\geqslant}

%%%%%%%%%%%%%%
%%%%% Macro Celia
%%%%%%%%%%%%%

\newcommand{\ff}[2]{\left[#1, #2\right]} %
\newcommand{\fo}[2]{\left[#1, #2\right[} %
\newcommand{\of}[2]{\left]#1, #2\right]} %
\newcommand{\soo}[2]{\left]#1, #2\right[} %
\newcommand{\abs}[1]{\left|#1\right|} %
\newcommand{\Ent}[1]{\left\lfloor #1 \right\rfloor} %


%%%%%%%%%%%%%%
%%%%% tikz : comment dessiner un "oeil"
%%%%%%%%%%%%%

\newcommand{\eye}[4]% size, x, y, rotation
{ \draw[rotate around={#4:(#2,#3)}] (#2,#3) -- ++(-.5*55:#1) (#2,#3)
  -- ++(.5*55:#1); \draw (#2,#3) ++(#4+55:.75*#1) arc
  (#4+55:#4-55:.75*#1);
  % IRIS
  \draw[fill=gray] (#2,#3) ++(#4+55/3:.75*#1) arc
  (#4+180-55:#4+180+55:.28*#1);
  % PUPIL, a filled arc
  \draw[fill=black] (#2,#3) ++(#4+55/3:.75*#1) arc
  (#4+55/3:#4-55/3:.75*#1);%
}


%%%%%%%%%%
%% discontinuité fonction
\newcommand\pointg[2]{%
  \draw[color = red, very thick] (#1+0.15, #2-.04)--(#1, #2-.04)--(#1,
  #2+.04)--(#1+0.15, #2+.04);%
}%

\newcommand\pointd[2]{%
  \draw[color = red, very thick] (#1-0.15, #2+.04)--(#1, #2+.04)--(#1,
  #2-.04)--(#1-0.15, #2-.04);%
}%

%%%%%%%%%%
%%% 1 : position abscisse, 2 : position ordonnée, 3 : taille, 4 : couleur
%%%%%%%%%%
% \newcommand\pointG[4]{%
%   \draw[color = #4, very thick] (#1+#3, #2-(#3/3.75))--(#1,
%   #2-(#3/3.75))--(#1, #2+(#3/3.75))--(#1+#3, #2+(#3/3.75)) %
% }%

\newcommand\pointG[4]{%
  \draw[color = #4, very thick] ({#1+#3/3.75}, {#2-#3})--(#1,
  {#2-#3})--(#1, {#2+#3})--({#1+#3/3.75}, {#2+#3}) %
}%

\newcommand\pointD[4]{%
  \draw[color = #4, very thick] ({#1-#3/3.75}, {#2+#3})--(#1,
  {#2+#3})--(#1, {#2-#3})--({#1-#3/3.75}, {#2-#3}) %
}%

\newcommand\spointG[4]{%
  \draw[color = #4, very thick] ({#1+#3/1.75}, {#2-#3})--(#1,
  {#2-#3})--(#1, {#2+#3})--({#1+#3/1.75}, {#2+#3}) %
}%

\newcommand\spointD[4]{%
  \draw[color = #4, very thick] ({#1-#3/2}, {#2+#3})--(#1,
  {#2+#3})--(#1, {#2-#3})--({#1-#3/2}, {#2-#3}) %
}%

%%%%%%%%%%

\newcommand{\Pb}{\mathtt{P}}

%%%%%%%%%%%%%%%
%%% Pour citer un précédent item
%%%%%%%%%%%%%%%
\newcommand{\itbf}[1]{{\small \bf \textit{#1}}}


%%%%%%%%%%%%%%%
%%% Quelques couleurs
%%%%%%%%%%%%%%%

\xdefinecolor{cancelcolor}{named}{Red}
\xdefinecolor{intI}{named}{ProcessBlue}
\xdefinecolor{intJ}{named}{ForestGreen}

%%%%%%%%%%%%%%%
%%%%%%%%%%%%%%%
% barrer du texte
\usetikzlibrary{shapes.misc}

\makeatletter
% \definecolor{cancelcolor}{rgb}{0.127,0.372,0.987}
\newcommand{\tikz@bcancel}[1]{%
  \begin{tikzpicture}[baseline=(textbox.base), inner sep=0pt]
    \node[strike out, draw] (textbox) {#1}[thick, color=cancelcolor];
    \useasboundingbox (textbox);
  \end{tikzpicture}%
}
\newcommand{\bcancel}[1]{%
  \relax\ifmmode
    \mathchoice{\tikz@bcancel{$\displaystyle#1$}}
               {\tikz@bcancel{$\textstyle#1$}}
               {\tikz@bcancel{$\scriptstyle#1$}}
               {\tikz@bcancel{$\scriptscriptstyle#1$}}
  \else
    \tikz@bcancel{\strut#1}%
  \fi
}
\newcommand{\tikz@xcancel}[1]{%
  \begin{tikzpicture}[baseline=(textbox.base),inner sep=0pt]
  \node[cross out,draw] (textbox) {#1}[thick, color=cancelcolor];
  \useasboundingbox (textbox);
  \end{tikzpicture}%
}
\newcommand{\xcancel}[1]{%
  \relax\ifmmode
    \mathchoice{\tikz@xcancel{$\displaystyle#1$}}
               {\tikz@xcancel{$\textstyle#1$}}
               {\tikz@xcancel{$\scriptstyle#1$}}
               {\tikz@xcancel{$\scriptscriptstyle#1$}}
  \else
    \tikz@xcancel{\strut#1}%
  \fi
}
\makeatother

\newcommand{\xcancelRA}{\xcancel{\rule[-.15cm]{0cm}{.5cm} \Rightarrow
    \rule[-.15cm]{0cm}{.5cm}}}

%%%%%%%%%%%%%%%%%%%%%%%%%%%%%%%%%%%
%%%%%%%%%%%%%%%%%%%%%%%%%%%%%%%%%%%

\newcommand{\vide}{\multicolumn{1}{c}{}}

%%%%%%%%%%%%%%%%%%%%%%%%%%%%%%%%%%%
%%%%%%%%%%%%%%%%%%%%%%%%%%%%%%%%%%%


\usepackage{multicol}
% \usepackage[latin1]{inputenc}
% \usepackage[T1]{fontenc}
\usepackage[utf8]{inputenc}
\usepackage[T1]{fontenc}
\usepackage[normalem]{ulem}
\usepackage[french]{babel}

\usepackage{url}    
\usepackage{hyperref}
\hypersetup{
  backref=true,
  pagebackref=true,
  hyperindex=true,
  colorlinks=true,
  breaklinks=true,
  urlcolor=blue,
  linkcolor=black,
  %%%%%%%%
  % ATTENTION : red changé en black pour le Livre !
  %%%%%%%%
  bookmarks=true,
  bookmarksopen=true
}

%%%%%%%%%%%%%%%%%%%%%%%%%%%%%%%%%%%%%%%%%%%
%% Pour faire des traits diagonaux dans les tableaux
%% Nécessite slashbox.sty
%\usepackage{slashbox}

\usepackage{tipa}
\usepackage{verbatim,listings}
\usepackage{graphicx}
\usepackage{fancyhdr}
\usepackage{mathrsfs}
\usepackage{pifont}
\usepackage{tablists}
\usepackage{dsfont,amsfonts,amssymb,amsmath,amsthm,stmaryrd,upgreek,manfnt}
\usepackage{enumerate}

%\newcolumntype{M}[1]{p{#1}}
\newcolumntype{C}[1]{>{\centering}m{#1}}
\newcolumntype{R}[1]{>{\raggedright}m{#1}}
\newcolumntype{L}[1]{>{\raggedleft}m{#1}}
\newcolumntype{P}[1]{>{\raggedright}p{#1}}
\newcolumntype{B}[1]{>{\raggedright}b{#1}}
\newcolumntype{Q}[1]{>{\raggedright}t{#1}}

\newcommand{\alias}[2]{
\providecommand{#1}{}
\renewcommand{#1}{#2}
}
\alias{\R}{\mathbb{R}}
\alias{\N}{\mathbb{N}}
\alias{\Z}{\mathbb{Z}}
\alias{\Q}{\mathbb{Q}}
\alias{\C}{\mathbb{C}}
\alias{\K}{\mathbb{K}}

%%%%%%%%%%%%
%% rendre +infty et -infty plus petits
%%%%%%%%%%%%
\newcommand{\sinfty}{{\scriptstyle \infty}}

%%%%%%%%%%%%%%%%%%%%%%%%%%%%%%%
%%%%% macros TP Scilab %%%%%%%%
\newcommand{\Scilab}{\textbf{Scilab}} %
\newcommand{\Scinotes}{\textbf{SciNotes}} %
\newcommand{\faire}{\noindent $\blacktriangleright$ } %
\newcommand{\fitem}{\scalebox{.8}{$\blacktriangleright$}} %
\newcommand{\entree}{{\small\texttt{ENTRÉE}}} %
\newcommand{\tab}{{\small\texttt{TAB}}} %
\newcommand{\mt}[1]{\mathtt{#1}} %
% guillemets droits

\newcommand{\ttq}{\textquotesingle} %

\newcommand{\reponse}[1]{\longboxed{
    \begin{array}{C{0.9\textwidth}}
      \nl[#1]
    \end{array}
  }} %

\newcommand{\reponseR}[1]{\longboxed{
    \begin{array}{R{0.9\textwidth}}
      #1
    \end{array}
  }} %

\newcommand{\reponseC}[1]{\longboxed{
    \begin{array}{C{0.9\textwidth}}
      #1
    \end{array}
  }} %

\colorlet{pyfunction}{Blue}
\colorlet{pyCle}{Magenta}
\colorlet{pycomment}{LimeGreen}
\colorlet{pydoc}{Cyan}
% \colorlet{SansCo}{white}
% \colorlet{AvecCo}{black}

\newcommand{\visible}[1]{{\color{ASCo}\colorlet{pydoc}{pyDo}\colorlet{pycomment}{pyCo}\colorlet{pyfunction}{pyF}\colorlet{pyCle}{pyC}\colorlet{function}{sciFun}\colorlet{var}{sciVar}\colorlet{if}{sciIf}\colorlet{comment}{sciComment}#1}} %

%%%% à changer ????
\newcommand{\invisible}[1]{{\color{ASCo}\colorlet{pydoc}{pyDo}\colorlet{pycomment}{pyCo}\colorlet{pyfunction}{pyF}\colorlet{pyCle}{pyC}\colorlet{function}{sciFun}\colorlet{var}{sciVar}\colorlet{if}{sciIf}\colorlet{comment}{sciComment}#1}} %

\newcommand{\invisibleCol}[2]{{\color{#1}#2}} %

\NewEnviron{solution} %
{ %
  \Boxed{
    \begin{array}{>{\color{ASCo}} R{0.9\textwidth}}
      \colorlet{pycomment}{pyCo}
      \colorlet{pydoc}{pyDo}
      \colorlet{pyfunction}{pyF}
      \colorlet{pyCle}{pyC}
      \colorlet{function}{sciFun}
      \colorlet{var}{sciVar}
      \colorlet{if}{sciIf}
      \colorlet{comment}{sciComment}
      \BODY
    \end{array}
  } %
} %

\NewEnviron{solutionC} %
{ %
  \Boxed{
    \begin{array}{>{\color{ASCo}} C{0.9\textwidth}}
      \colorlet{pycomment}{pyCo}
      \colorlet{pydoc}{pyDo}
      \colorlet{pyfunction}{pyF}
      \colorlet{pyCle}{pyC}
      \colorlet{function}{sciFun}
      \colorlet{var}{sciVar}
      \colorlet{if}{sciIf}
      \colorlet{comment}{sciComment}
      \BODY
    \end{array}
  } %
} %

\newcommand{\invite}{--\!\!>} %

%%%%% nouvel environnement tabular pour retour console %%%%
\colorlet{ConsoleColor}{Black!12}
\colorlet{function}{Red}
\colorlet{var}{Maroon}
\colorlet{if}{Magenta}
\colorlet{comment}{LimeGreen}

\newcommand{\tcVar}[1]{\textcolor{var}{\bf \small #1}} %
\newcommand{\tcFun}[1]{\textcolor{function}{#1}} %
\newcommand{\tcIf}[1]{\textcolor{if}{#1}} %
\newcommand{\tcFor}[1]{\textcolor{if}{#1}} %

\newcommand{\moins}{\!\!\!\!\!\!- }
\newcommand{\espn}{\!\!\!\!\!\!}

\usepackage{booktabs,varwidth} \newsavebox\TBox
\newenvironment{console}
{\begin{lrbox}{\TBox}\varwidth{\linewidth}
    \tabular{>{\tt\small}R{0.84\textwidth}}
    \nl[-.4cm]} {\endtabular\endvarwidth\end{lrbox}%
  \fboxsep=1pt\colorbox{ConsoleColor}{\usebox\TBox}}

\newcommand{\lInv}[1]{%
  $\invite$ #1} %

\newcommand{\lAns}[1]{%
  \qquad ans \ = \nl %
  \qquad \qquad #1} %

\newcommand{\lVar}[2]{%
  \qquad #1 \ = \nl %
  \qquad \qquad #2} %

\newcommand{\lDisp}[1]{%
  #1 %
} %

\newcommand{\ligne}[1]{\underline{\small \tt #1}} %

\newcommand{\ligneAns}[2]{%
  $\invite$ #1 \nl %
  \qquad ans \ = \nl %
  \qquad \qquad #2} %

\newcommand{\ligneVar}[3]{%
  $\invite$ #1 \nl %
  \qquad #2 \ = \nl %
  \qquad \qquad #3} %

\newcommand{\ligneErr}[3]{%
  $\invite$ #1 \nl %
  \quad !-{-}error #2 \nl %
  #3} %
%%%%%%%%%%%%%%%%%%%%%% 

\newcommand{\bs}[1]{\boldsymbol{#1}} %
\newcommand{\nll}{\nl[.4cm]} %
\newcommand{\nle}{\nl[.2cm]} %
%% opérateur puissance copiant l'affichage Scilab
%\newcommand{\puis}{\!\!\!~^{\scriptscriptstyle\pmb{\wedge}}}
\newcommand{\puis}{\mbox{$\hspace{-.1cm}~^{\scriptscriptstyle\pmb{\wedge}}
    \hspace{0.05cm}$}} %
\newcommand{\pointpuis}{.\mbox{$\hspace{-.15cm}~^{\scriptscriptstyle\pmb{\wedge}}$}} %
\newcommand{\Sfois}{\mbox{$\mt{\star}$}} %

%%%%% nouvel environnement tabular pour les encadrés Scilab %%%%
\newenvironment{encadre}
{\begin{lrbox}{\TBox}\varwidth{\linewidth}
    \tabular{>{\tt\small}C{0.1\textwidth}>{\small}R{0.7\textwidth}}}
  {\endtabular\endvarwidth\end{lrbox}%
  \fboxsep=1pt\longboxed{\usebox\TBox}}

\newenvironment{encadreL}
{\begin{lrbox}{\TBox}\varwidth{\linewidth}
    \tabular{>{\tt\small}C{0.25\textwidth}>{\small}R{0.6\textwidth}}}
  {\endtabular\endvarwidth\end{lrbox}%
  \fboxsep=1pt\longboxed{\usebox\TBox}}

\newenvironment{encadreF}
{\begin{lrbox}{\TBox}\varwidth{\linewidth}
    \tabular{>{\tt\small}C{0.2\textwidth}>{\small}R{0.70\textwidth}}}
  {\endtabular\endvarwidth\end{lrbox}%
  \fboxsep=1pt\longboxed{\usebox\TBox}}

\newenvironment{encadreLL}[2]
{\begin{lrbox}{\TBox}\varwidth{\linewidth}
    \tabular{>{\tt\small}C{#1\textwidth}>{\small}R{#2\textwidth}}}
  {\endtabular\endvarwidth\end{lrbox}%
  \fboxsep=1pt\longboxed{\usebox\TBox}}

%%%%% nouvel environnement tabular pour les script et fonctions %%%%
\newcommand{\commentaireDL}[1]{\multicolumn{1}{l}{\it
    \textcolor{comment}{$\slash\slash$ #1}}}

\newcommand{\commentaire}[1]{{\textcolor{comment}{$\slash\slash$ #1}}}

\newcounter{cptcol}

\newcommand{\nocount}{\multicolumn{1}{c}{}}

\newcommand{\sciNo}[1]{{\small \underbar #1}}

\NewEnviron{scilab}{ %
  \setcounter{cptcol}{0}
  \begin{center}
    \longboxed{
      \begin{tabular}{>{\stepcounter{cptcol}{\tiny \underbar
              \thecptcol}}c>{\tt}l}
        \BODY
      \end{tabular}
    }
  \end{center}
}

\NewEnviron{scilabNC}{ %
  \begin{center}
    \longboxed{
      \begin{tabular}{>{\tt}l} %
          \BODY
      \end{tabular}
    }
  \end{center}
}

\NewEnviron{scilabC}[1]{ %
  \setcounter{cptcol}{#1}
  \begin{center}
    \longboxed{
      \begin{tabular}{>{\stepcounter{cptcol}{\tiny \underbar
              \thecptcol}}c>{\tt}l}
        \BODY
      \end{tabular}
    }
  \end{center}
}

\newcommand{\scisol}[1]{ %
  \setcounter{cptcol}{0}
  \longboxed{
    \begin{tabular}{>{\stepcounter{cptcol}{\tiny \underbar
            \thecptcol}}c>{\tt}l}
      #1
    \end{tabular}
  }
}

\newcommand{\scisolNC}[1]{ %
  \longboxed{
    \begin{tabular}{>{\tt}l}
      #1
    \end{tabular}
  }
}

\newcommand{\scisolC}[2]{ %
  \setcounter{cptcol}{#1}
  \longboxed{
    \begin{tabular}{>{\stepcounter{cptcol}{\tiny \underbar
            \thecptcol}}c>{\tt}l}
      #2
    \end{tabular}
  }
}

\NewEnviron{syntaxe}{ %
  % \fcolorbox{black}{Yellow!20}{\setlength{\fboxsep}{3mm}
  \shadowbox{
    \setlength{\fboxsep}{3mm}
    \begin{tabular}{>{\tt}l}
      \BODY
    \end{tabular}
  }
}

%%%%% fin macros TP Scilab %%%%%%%%
%%%%%%%%%%%%%%%%%%%%%%%%%%%%%%%%%%%

%%%%%%%%%%%%%%%%%%%%%%%%%%%%%%%%%%%
%%%%% TP Python - listings %%%%%%%%
%%%%%%%%%%%%%%%%%%%%%%%%%%%%%%%%%%%
\newcommand{\Python}{\textbf{Python}} %

\lstset{% general command to set parameter(s)
basicstyle=\ttfamily\small, % print whole listing small
keywordstyle=\color{blue}\bfseries\underbar,
%% underlined bold black keywords
frame=lines,
xleftmargin=10mm,
numbers=left,
numberstyle=\tiny\underbar,
numbersep=10pt,
%identifierstyle=, % nothing happens
commentstyle=\color{green}, % white comments
%%stringstyle=\ttfamily, % typewriter type for strings
showstringspaces=false}

\newcommand{\pysolCpt}[2]{
  \setcounter{cptcol}{#1}
  \longboxed{
    \begin{tabular}{>{\stepcounter{cptcol}{\tiny \underbar
            \thecptcol}}c>{\tt}l}
        #2
      \end{tabular}
    }
} %

\newcommand{\pysol}[1]{
  \setcounter{cptcol}{0}
  \longboxed{
    \begin{tabular}{>{\stepcounter{cptcol}{\tiny \underbar
            \thecptcol}}c>{\tt}l}
        #1
      \end{tabular}
    }
} %

% \usepackage[labelsep=endash]{caption}

% avec un caption
\NewEnviron{pythonCap}[1]{ %
  \renewcommand{\tablename}{Programme}
  \setcounter{cptcol}{0}
  \begin{center}
    \longboxed{
      \begin{tabular}{>{\stepcounter{cptcol}{\tiny \underbar
              \thecptcol}}c>{\tt}l}
        \BODY
      \end{tabular}
    }
    \captionof{table}{#1}
  \end{center}
}

\NewEnviron{python}{ %
  \setcounter{cptcol}{0}
  \begin{center}
    \longboxed{
      \begin{tabular}{>{\stepcounter{cptcol}{\tiny \underbar
              \thecptcol}}c>{\tt}l}
        \BODY
      \end{tabular}
    }
  \end{center}
}

\newcommand{\pyVar}[1]{\textcolor{var}{\bf \small #1}} %
\newcommand{\pyFun}[1]{\textcolor{pyfunction}{#1}} %
\newcommand{\pyCle}[1]{\textcolor{pyCle}{#1}} %
\newcommand{\pyImp}[1]{{\bf #1}} %

%%%%% commentaire python %%%%
\newcommand{\pyComDL}[1]{\multicolumn{1}{l}{\textcolor{pycomment}{\#
      #1}}}

\newcommand{\pyCom}[1]{{\textcolor{pycomment}{\# #1}}}
\newcommand{\pyDoc}[1]{{\textcolor{pydoc}{#1}}}

\newcommand{\pyNo}[1]{{\small \underbar #1}}

%%%%%%%%%%%%%%%%%%%%%%%%%%%%%%%%%%%
%%%%%% Système linéaire paramétré : écrire les opérations au-dessus
%%%%%% d'un symbole équivalent
%%%%%%%%%%%%%%%%%%%%%%%%%%%%%%%%%%%

\usepackage{systeme}

\NewEnviron{arrayEq}{ %
  \stackrel{\scalebox{.6}{$
      \begin{array}{l} 
        \BODY \\[.1cm]
      \end{array}$}
  }{\Longleftrightarrow}
}

\NewEnviron{arrayEg}{ %
  \stackrel{\scalebox{.6}{$
      \begin{array}{l} 
        \BODY \\[.1cm]
      \end{array}$}
  }{=}
}

\NewEnviron{operationEq}{ %
  \scalebox{.6}{$
    \begin{array}{l} 
      \scalebox{1.6}{$\mbox{Opérations :}$} \\[.2cm]
      \BODY \\[.1cm]
    \end{array}$}
}

% \NewEnviron{arraySys}[1]{ %
%   \sysdelim\{.\systeme[#1]{ %
%     \BODY %
%   } %
% }

%%%%%

%%%%%%%%%%
%%%%%%%%%% ESSAI
\newlength\fboxseph
\newlength\fboxsepva
\newlength\fboxsepvb

\setlength\fboxsepva{0.2cm}
\setlength\fboxsepvb{0.2cm}
\setlength\fboxseph{0.2cm}

\makeatletter

\def\longboxed#1{\leavevmode\setbox\@tempboxa\hbox{\color@begingroup%
\kern\fboxseph{\m@th$\displaystyle #1 $}\kern\fboxseph%
\color@endgroup }\my@frameb@x\relax}

\def\my@frameb@x#1{%
  \@tempdima\fboxrule \advance\@tempdima \fboxsepva \advance\@tempdima
  \dp\@tempboxa\hbox {%
    \lower \@tempdima \hbox {%
      \vbox {\hrule\@height\fboxrule \hbox{\vrule\@width\fboxrule #1
          \vbox{%
            \vskip\fboxsepva \box\@tempboxa \vskip\fboxsepvb}#1
          \vrule\@width\fboxrule }%
        \hrule \@height \fboxrule }}}}

\newcommand{\boxedhv}[3]{\setlength\fboxseph{#1cm}
  \setlength\fboxsepva{#2cm}\setlength\fboxsepvb{#2cm}\longboxed{#3}}

\newcommand{\boxedhvv}[4]{\setlength\fboxseph{#1cm}
  \setlength\fboxsepva{#2cm}\setlength\fboxsepvb{#3cm}\longboxed{#4}}

\newcommand{\Boxed}[1]{{\setlength\fboxseph{0.2cm}
  \setlength\fboxsepva{0.2cm}\setlength\fboxsepvb{0.2cm}\longboxed{#1}}}

\newcommand{\mBoxed}[1]{{\setlength\fboxseph{0.2cm}
  \setlength\fboxsepva{0.2cm}\setlength\fboxsepvb{0.2cm}\longboxed{\mbox{#1}}}}

\newcommand{\mboxed}[1]{{\setlength\fboxseph{0.2cm}
  \setlength\fboxsepva{0.2cm}\setlength\fboxsepvb{0.2cm}\boxed{\mbox{#1}}}}

\newsavebox{\fmbox}
\newenvironment{fmpage}[1]
     {\begin{lrbox}{\fmbox}\begin{minipage}{#1}}
     {\end{minipage}\end{lrbox}\fbox{\usebox{\fmbox}}}

%%%%%%%%%%
%%%%%%%%%%

\DeclareMathOperator{\ch}{ch}
\DeclareMathOperator{\sh}{sh}

%%%%%%%%%%
%%%%%%%%%%

\newcommand{\norme}[1]{\Vert #1 \Vert}

%\newcommand*\widefbox[1]{\fbox{\hspace{2em}#1\hspace{2em}}}

\newcommand{\nl}{\tabularnewline}

\newcommand{\hand}{\noindent\ding{43}\ }
\newcommand{\ie}{\textit{i.e. }}
\newcommand{\cf}{\textit{cf }}

\newcommand{\Card}{\operatorname{Card}}

\newcommand{\aire}{\mathcal{A}}

\newcommand{\LL}[1]{\mathscr{L}(#1)} %
\newcommand{\B}{\mathscr{B}} %
\newcommand{\Bc}[1]{B_{#1}} %
\newcommand{\M}[1]{\mathscr{M}_{#1}(\mathbb{R})}

\DeclareMathOperator{\im}{Im}
\DeclareMathOperator{\kr}{Ker}
\DeclareMathOperator{\rg}{rg}
\DeclareMathOperator{\spc}{Sp}
\DeclareMathOperator{\sgn}{sgn}
\DeclareMathOperator{\supp}{Supp}

\newcommand{\Mat}{{\rm{Mat}}}
\newcommand{\Vect}[1]{{\rm{Vect}}\left(#1\right)}

\newenvironment{smatrix}{%
  \begin{adjustbox}{width=.9\width}
    $
    \begin{pmatrix}
    }{%      
    \end{pmatrix}
    $
  \end{adjustbox}
}

\newenvironment{sarray}[1]{%
  \begin{adjustbox}{width=.9\width}
    $
    \begin{array}{#1}
    }{%      
    \end{array}
    $
  \end{adjustbox}
}

\newcommand{\vd}[2]{
  \scalebox{.8}{
    $\left(\!
      \begin{array}{c}
        #1 \\
        #2
      \end{array}
    \!\right)$
    }}

\newcommand{\vt}[3]{
  \scalebox{.8}{
    $\left(\!
      \begin{array}{c}
        #1 \\
        #2 \\
        #3 
      \end{array}
    \!\right)$
    }}

\newcommand{\vq}[4]{
  \scalebox{.8}{
    $\left(\!
      \begin{array}{c}
        #1 \\
        #2 \\
        #3 \\
        #4 
      \end{array}
    \!\right)$
    }}

\newcommand{\vc}[5]{
  \scalebox{.8}{
    $\left(\!
      \begin{array}{c}
        #1 \\
        #2 \\
        #3 \\
        #4 \\
        #5 
      \end{array}
    \!\right)$
    }}

\newcommand{\ee}{\text{e}}

\newcommand{\dd}{\text{d}}

%%% Ensemble de définition
\newcommand{\Df}{\mathscr{D}}
\newcommand{\Cf}{\mathscr{C}}
\newcommand{\Ef}{\mathscr{C}}

\newcommand{\rond}[1]{\,\overset{\scriptscriptstyle \circ}{\!#1}}

\newcommand{\df}[2]{\dfrac{\partial #1}{\partial #2}} %
\newcommand{\dfn}[2]{\partial_{#2}(#1)} %
\newcommand{\ddfn}[2]{\partial^2_{#2}(#1)} %
\newcommand{\ddf}[2]{\dfrac{\partial^2 #1}{\partial #2^2}} %
\newcommand{\ddfr}[3]{\dfrac{\partial^2 #1}{\partial #2 \partial
    #3}} %


\newcommand{\dlim}[1]{{\displaystyle \lim_{#1} \ }}
\newcommand{\dlimPlus}[2]{
  \dlim{
    \scalebox{.6}{
      $
      \begin{array}{l}
        #1 \rightarrow #2\\
        #1 > #2
      \end{array}
      $}}}
\newcommand{\dlimMoins}[2]{
  \dlim{
    \scalebox{.6}{
      $
      \begin{array}{l}
        #1 \rightarrow #2\\
        #1 < #2
      \end{array}
      $}}}

%%%%%%%%%%%%%%
%% petit o, développement limité
%%%%%%%%%%%%%%

\newcommand{\oo}[2]{{\underset {{\overset {#1\rightarrow #2}{}}}{o}}} %
\newcommand{\oox}[1]{{\underset {{\overset {x\rightarrow #1}{}}}{o}}} %
\newcommand{\oon}{{\underset {{\overset {n\rightarrow +\infty}{}}}{o}}} %
\newcommand{\po}[1]{{\underset {{\overset {#1}{}}}{o}}} %
\newcommand{\neqx}[1]{{\ \underset {{\overset {x \to #1}{}}}{\not\sim}\ }} %
\newcommand{\eqx}[1]{{\ \underset {{\overset {x \to #1}{}}}{\sim}\ }} %
\newcommand{\eqn}{{\ \underset {{\overset {n \to +\infty}{}}}{\sim}\ }} %
\newcommand{\eq}[2]{{\ \underset {{\overset {#1 \to #2}{}}}{\sim}\ }} %
\newcommand{\DL}[1]{{\rm{DL}}_1 (#1)} %
\newcommand{\DLL}[1]{{\rm{DL}}_2 (#1)} %

\newcommand{\negl}{<<}

\newcommand{\neglP}[1]{\begin{array}{c}
    \vspace{-.2cm}\\
    << \\
    \vspace{-.7cm}\\
    {\scriptstyle #1}
  \end{array}}

%%%%%%%%%%%%%%
%% borne sup, inf, max, min
%%%%%%%%%%%%%%
\newcommand{\dsup}[1]{\displaystyle \sup_{#1} \ }
\newcommand{\dinf}[1]{\displaystyle \inf_{#1} \ }
\newcommand{\dmax}[1]{\max\limits_{#1} \ }
\newcommand{\dmin}[1]{\min\limits_{#1} \ }

\newcommand{\dcup}[2]{{\textstyle\bigcup\limits_{#1}^{#2}}\hspace{.1cm}}
%\displaystyle \bigcup_{#1}^{#2}}
\newcommand{\dcap}[2]{{\textstyle\bigcap\limits_{#1}^{#2}}\hspace{.1cm}}
% \displaystyle \bigcap_{#1}^{#2}
%%%%%%%%%%%%%%
%% opérateurs logiques
%%%%%%%%%%%%%%
\newcommand{\NON}[1]{\mathop{\small \tt{NON}} (#1)}
\newcommand{\ET}{\mathrel{\mathop{\small \mathtt{ET}}}}
\newcommand{\OU}{\mathrel{\mathop{\small \tt{OU}}}}
\newcommand{\XOR}{\mathrel{\mathop{\small \tt{XOR}}}}

\newcommand{\id}{{\rm{id}}}

\newcommand{\sbullet}{\scriptstyle \bullet}
\newcommand{\stimes}{\scriptstyle \times}

%%%%%%%%%%%%%%%%%%
%% Probabilités
%%%%%%%%%%%%%%%%%%
\newcommand{\Prob}{\mathbb{P}}
\newcommand{\Ev}[1]{\left[ {#1} \right]}
\newcommand{\Evmb}[1]{[ {#1} ]}
\newcommand{\E}{\mathbb{E}}
\newcommand{\V}{\mathbb{V}}
\newcommand{\Cov}{{\rm{Cov}}}
\newcommand{\U}[2]{\mathcal{U}(\llb #1, #2\rrb)}
\newcommand{\Uc}[2]{\mathcal{U}([#1, #2])}
\newcommand{\Ucof}[2]{\mathcal{U}(]#1, #2])}
\newcommand{\Ucoo}[2]{\mathcal{U}(]#1, #2[)}
\newcommand{\Ucfo}[2]{\mathcal{U}([#1, #2[)}
\newcommand{\Bern}[1]{\mathcal{B}\left(#1\right)}
\newcommand{\Bin}[2]{\mathcal{B}\left(#1, #2\right)}
\newcommand{\G}[1]{\mathcal{G}\left(#1\right)}
\newcommand{\Pois}[1]{\mathcal{P}\left(#1\right)}
\newcommand{\HG}[3]{\mathcal{H}\left(#1, #2, #3\right)}
\newcommand{\Exp}[1]{\mathcal{E}\left(#1\right)}
\newcommand{\Norm}[2]{\mathcal{N}\left(#1, #2\right)}

\DeclareMathOperator{\cov}{Cov}

\newcommand{\var}{v.a.r. }
\newcommand{\suit}{\hookrightarrow}

\newcommand{\flecheR}[1]{\rotatebox{90}{\scalebox{#1}{\color{red}
      $\curvearrowleft$}}}


\newcommand{\partie}[1]{\mathcal{P}(#1)}
\newcommand{\Cont}[1]{\mathcal{C}^{#1}}
\newcommand{\Contm}[1]{\mathcal{C}^{#1}_m}

\newcommand{\llb}{\llbracket}
\newcommand{\rrb}{\rrbracket}

%\newcommand{\im}[1]{{\rm{Im}}(#1)}
\newcommand{\imrec}[1]{#1^{- \mathds{1}}}

\newcommand{\unq}{\mathds{1}}

\newcommand{\Hyp}{\mathtt{H}}

\newcommand{\eme}[1]{#1^{\scriptsize \mbox{ème}}}
\newcommand{\er}[1]{#1^{\scriptsize \mbox{er}}}
\newcommand{\ere}[1]{#1^{\scriptsize \mbox{ère}}}
\newcommand{\nd}[1]{#1^{\scriptsize \mbox{nd}}}
\newcommand{\nde}[1]{#1^{\scriptsize \mbox{nde}}}

\newcommand{\truc}{\mathop{\top}}
\newcommand{\fois}{\mathop{\ast}}

\newcommand{\f}[1]{\overrightarrow{#1}}

\newcommand{\checked}{\textcolor{green}{\checkmark}}

\def\restriction#1#2{\mathchoice
              {\setbox1\hbox{${\displaystyle #1}_{\scriptstyle #2}$}
              \restrictionaux{#1}{#2}}
              {\setbox1\hbox{${\textstyle #1}_{\scriptstyle #2}$}
              \restrictionaux{#1}{#2}}
              {\setbox1\hbox{${\scriptstyle #1}_{\scriptscriptstyle #2}$}
              \restrictionaux{#1}{#2}}
              {\setbox1\hbox{${\scriptscriptstyle #1}_{\scriptscriptstyle #2}$}
              \restrictionaux{#1}{#2}}}
\def\restrictionaux#1#2{{#1\,\smash{\vrule height .8\ht1 depth .85\dp1}}_{\,#2}}

\makeatletter
\newcommand*{\da@rightarrow}{\mathchar"0\hexnumber@\symAMSa 4B }
\newcommand*{\da@leftarrow}{\mathchar"0\hexnumber@\symAMSa 4C }
\newcommand*{\xdashrightarrow}[2][]{%
  \mathrel{%
    \mathpalette{\da@xarrow{#1}{#2}{}\da@rightarrow{\,}{}}{}%
  }%
}
\newcommand{\xdashleftarrow}[2][]{%
  \mathrel{%
    \mathpalette{\da@xarrow{#1}{#2}\da@leftarrow{}{}{\,}}{}%
  }%
}
\newcommand*{\da@xarrow}[7]{%
  % #1: below
  % #2: above
  % #3: arrow left
  % #4: arrow right
  % #5: space left 
  % #6: space right
  % #7: math style 
  \sbox0{$\ifx#7\scriptstyle\scriptscriptstyle\else\scriptstyle\fi#5#1#6\m@th$}%
  \sbox2{$\ifx#7\scriptstyle\scriptscriptstyle\else\scriptstyle\fi#5#2#6\m@th$}%
  \sbox4{$#7\dabar@\m@th$}%
  \dimen@=\wd0 %
  \ifdim\wd2 >\dimen@
    \dimen@=\wd2 %   
  \fi
  \count@=2 %
  \def\da@bars{\dabar@\dabar@}%
  \@whiledim\count@\wd4<\dimen@\do{%
    \advance\count@\@ne
    \expandafter\def\expandafter\da@bars\expandafter{%
      \da@bars
      \dabar@ 
    }%
  }%  
  \mathrel{#3}%
  \mathrel{%   
    \mathop{\da@bars}\limits
    \ifx\\#1\\%
    \else
      _{\copy0}%
    \fi
    \ifx\\#2\\%
    \else
      ^{\copy2}%
    \fi
  }%   
  \mathrel{#4}%
}
\makeatother



\newcount\depth

\newcount\depth
\newcount\totaldepth

\makeatletter
\newcommand{\labelsymbol}{%
      \ifnum\depth=0
        %
      \else
        \rlap{\,$\bullet$}%
      \fi
}

\newcommand*\bernoulliTree[1]{%
    \depth=#1\relax            
    \totaldepth=#1\relax
    \draw node(root)[bernoulli/root] {\labelsymbol}[grow=right] \draw@bernoulli@tree;
    \draw \label@bernoulli@tree{root};                                   
}                                                                        

\def\draw@bernoulli@tree{%
    \ifnum\depth>0 
      child[parent anchor=east] foreach \type/\label in {left child/$E$,right child/$S$} {%
          node[bernoulli/\type] {\label\strut\labelsymbol} \draw@bernoulli@tree
      }
      coordinate[bernoulli/increment] (dummy)
   \fi%
}

\def\label@bernoulli@tree#1{%
    \ifnum\depth>0
      ($(#1)!0.5!(#1-1)$) node[fill=white,bernoulli/decrement] {\tiny$p$}
      \label@bernoulli@tree{#1-1}
      ($(#1)!0.5!(#1-2)$) node[fill=white] {\tiny$q$}
      \label@bernoulli@tree{#1-2}
      coordinate[bernoulli/increment] (dummy)
   \fi%
}

\makeatother

\tikzset{bernoulli/.cd,
         root/.style={},
         decrement/.code=\global\advance\depth by-1\relax,
         increment/.code=\global\advance\depth by 1\relax,
         left child/.style={bernoulli/decrement},
         right child/.style={}}


\newcommand{\eps}{\varepsilon}

% \newcommand{\tendi}[1]{\xrightarrow[\footnotesize #1 \rightarrow
%   +\infty]{}}%

\newcommand{\tend}{\rightarrow}%
\newcommand{\tendn}{\underset{n\to +\infty}{\longrightarrow}} %
\newcommand{\ntendn}{\underset{n\to
    +\infty}{\not\hspace{-.15cm}\longrightarrow}} %
% \newcommand{\tendn}{\xrightarrow[\footnotesize n \rightarrow
%   +\infty]{}}%
\newcommand{\Tendx}[1]{\xrightarrow[\footnotesize x \rightarrow
  #1]{}}%
\newcommand{\tendx}[1]{\underset{x\to #1}{\longrightarrow}}%
\newcommand{\ntendx}[1]{\underset{x\to #1}{\not\!\!\longrightarrow}}%
\newcommand{\tendd}[2]{\underset{#1\to #2}{\longrightarrow}}%
% \newcommand{\tendd}[2]{\xrightarrow[\footnotesize #1 \rightarrow
%   #2]{}}%
\newcommand{\tendash}[1]{\xdashrightarrow[\footnotesize #1 \rightarrow
  +\infty]{}}%
\newcommand{\tendashx}[1]{\xdashrightarrow[\footnotesize x \rightarrow
  #1]{}}%
\newcommand{\tendb}[1]{\underset{#1 \to +\infty}{\longrightarrow}}%
\newcommand{\tendL}{\overset{\mathscr L}{\underset{n \to
      +\infty}{\longrightarrow}}}%
\newcommand{\tendP}{\overset{\Prob}{\underset{n \to
      +\infty}{\longrightarrow}}}%
\newcommand{\tenddL}[1]{\overset{\mathscr L}{\underset{#1 \to
      +\infty}{\longrightarrow}}}%

\NewEnviron{attention}{ %
  ~\\[-.2cm]\noindent
  \begin{minipage}{\linewidth}
  \setlength{\fboxsep}{3mm}%
  \ \ \dbend \ \ %
  \fbox{\parbox[t]{.88\linewidth}{\BODY}} %
  \end{minipage}\\
}

\NewEnviron{sattention}[1]{ %
  ~\\[-.2cm]\noindent
  \begin{minipage}{#1\linewidth}
  \setlength{\fboxsep}{3mm}%
  \ \ \dbend \ \ %
  \fbox{\parbox[t]{.88\linewidth}{\BODY}} %
  \end{minipage}\\
}

%%%%% OBSOLETE %%%%%%

% \newcommand{\attention}[1]{
%   \noindent
%   \begin{tabular}{@{}l|p{11.5cm}|}
%     \cline{2-2}
%     \vspace{-.2cm} 
%     & \nl
%     \dbend & #1 \nl
%     \cline{2-2}
%   \end{tabular}
% }

% \newcommand{\attentionv}[2]{
%   \noindent
%   \begin{tabular}{@{}l|p{11.5cm}|}
%     \cline{2-2}
%     \vspace{-.2cm} 
%     & \nl
%     \dbend & #2 \nl[#1 cm]
%     \cline{2-2}
%   \end{tabular}
% }

\newcommand{\explainvb}[2]{
  \noindent
  \begin{tabular}{@{}l|p{11.5cm}|}
    \cline{2-2}
    \vspace{-.2cm} 
    & \nl
    \hand & #2 \nl [#1 cm]
    \cline{2-2}
  \end{tabular}
}


% \noindent
% \begin{tabular}{@{}l|lp{11cm}|}
%   \cline{3-3} 
%   \multicolumn{1}{@{}l@{\dbend}}{} & & #1 \nl
%   \multicolumn{1}{l}{} & & \nl [-.8cm]
%   & & #2 \nl
%   \cline{2-3}
% \end{tabular}

% \newcommand{\attention}[1]{
%   \noindent
%   \begin{tabular}{@{}@{}cp{11cm}}
%     \dbend & #1 \nl
%   \end{tabular}
% }

\newcommand{\PP}[1]{\mathcal{P}(#1)}
\newcommand{\HH}[1]{\mathcal{H}(#1)}
\newcommand{\FF}[1]{\mathcal{F}(#1)}

\newcommand{\DSum}[2]{\displaystyle\sum\limits_{#1}^{#2}\hspace{.1cm}}
\newcommand{\Sum}[2]{{\textstyle\sum\limits_{#1}^{#2}}\hspace{.1cm}}
\newcommand{\Serie}{\textstyle\sum\hspace{.1cm}}
\newcommand{\Prod}[2]{\textstyle\prod\limits_{#1}^{#2}}

\newcommand{\Prim}[3]{\left[\ {#1} \ \right]_{\scriptscriptstyle
   \hspace{-.15cm} ~_{#2}\, }^ {\scriptscriptstyle \hspace{-.15cm} ~^{#3}\, }}

% \newcommand{\Prim}[3]{\left[\ {#1} \ \right]_{\scriptscriptstyle
%     \!\!~_{#2}}^ {\scriptscriptstyle \!\!~^{#3}}}

\newcommand{\dint}[2]{\displaystyle \int_{#1}^{#2}\ }
\newcommand{\Int}[2]{{\rm{Int}}_{\scriptscriptstyle #1, #2}}
\newcommand{\dt}{\ dt}
\newcommand{\dx}{\ dx}

\newcommand{\llpar}[1]{\left(\!\!\!
    \begin{array}{c}
      \rule{0pt}{#1}
    \end{array}
  \!\!\!\right.}

\newcommand{\rrpar}[1]{\left.\!\!\!
    \begin{array}{c}
      \rule{0pt}{#1}
    \end{array}
  \!\!\!\right)}

\newcommand{\llacc}[1]{\left\{\!\!\!
    \begin{array}{c}
      \rule{0pt}{#1}
    \end{array}
  \!\!\!\right.}

\newcommand{\rracc}[1]{\left.\!\!\!
    \begin{array}{c}
      \rule{0pt}{#1}
    \end{array}
  \!\!\!\right\}}

\newcommand{\ttacc}[1]{\mbox{\rotatebox{-90}{\hspace{-.7cm}$\llacc{#1}$}}}
\newcommand{\bbacc}[1]{\mbox{\rotatebox{90}{\hspace{-.5cm}$\llacc{#1}$}}}

\newcommand{\comp}[1]{\overline{#1}}%

\newcommand{\dcomp}[2]{\stackrel{\mbox{\ \ \----}{\scriptscriptstyle
      #2}}{#1}}%

% \newcommand{\Comp}[2]{\stackrel{\mbox{\ \
%       \-------}{\scriptscriptstyle #2}}{#1}}

% \newcommand{\dcomp}[2]{\stackrel{\mbox{\ \
%       \-------}{\scriptscriptstyle #2}}{#1}}

\newcommand{\A}{\mathscr{A}}

\newcommand{\conc}[1]{
  \begin{center}
    \fbox{
      \begin{tabular}{c}
        #1
      \end{tabular}
    }
  \end{center}
}

\newcommand{\concC}[1]{
  \begin{center}
    \fbox{
    \begin{tabular}{C{10cm}}
      \quad #1 \quad
    \end{tabular}
    }
  \end{center}
}

\newcommand{\concL}[2]{
  \begin{center}
    \fbox{
    \begin{tabular}{C{#2cm}}
      \quad #1 \quad
    \end{tabular}
    }
  \end{center}
}


% \newcommand{\lims}[2]{\prod\limits_{#1}^{#2}}

\newtheorem{theorem}{Théorème}[]
\newtheorem{lemma}{Lemme}[]
\newtheorem{proposition}{Proposition}[]
\newtheorem{corollary}{Corollaire}[]

% \newenvironment{proof}[1][Démonstration]{\begin{trivlist}
% \item[\hskip \labelsep {\bfseries #1}]}{\end{trivlist}}
\newenvironment{definition}[1][Définition]{\begin{trivlist}
\item[\hskip \labelsep {\bfseries #1}]}{\end{trivlist}}
\newenvironment{example}[1][Exemple]{\begin{trivlist}
\item[\hskip \labelsep {\bfseries #1}]}{\end{trivlist}}
\newenvironment{examples}[1][Exemples]{\begin{trivlist}
\item[\hskip \labelsep {\bfseries #1}]}{\end{trivlist}}
\newenvironment{notation}[1][Notation]{\begin{trivlist}
\item[\hskip \labelsep {\bfseries #1}]}{\end{trivlist}}
\newenvironment{propriete}[1][Propriété]{\begin{trivlist}
\item[\hskip \labelsep {\bfseries #1}]}{\end{trivlist}}
\newenvironment{proprietes}[1][Propriétés]{\begin{trivlist}
\item[\hskip \labelsep {\bfseries #1}]}{\end{trivlist}}
% \newenvironment{remark}[1][Remarque]{\begin{trivlist}
% \item[\hskip \labelsep {\bfseries #1}]}{\end{trivlist}}
\newenvironment{application}[1][Application]{\begin{trivlist}
\item[\hskip \labelsep {\bfseries #1}]}{\end{trivlist}}

% Environnement pour les réponses des DS
\newenvironment{answer}{\par\emph{Réponse :}\par{}}
{\vspace{-.6cm}\hspace{\stretch{1}}\rule{1ex}{1ex}\vspace{.3cm}}

\newenvironment{answerTD}{\vspace{.2cm}\par\emph{Réponse :}\par{}}
{\hspace{\stretch{1}}\rule{1ex}{1ex}\vspace{.3cm}}

\newenvironment{answerCours}{\noindent\emph{Réponse :}}
{\rule{1ex}{1ex}}%\vspace{.3cm}}


% footnote in footer
\newcommand{\fancyfootnotetext}[2]{%
  \fancypagestyle{dingens}{%
    \fancyfoot[LO,RE]{\parbox{0.95\textwidth}{\footnotemark[#1]\footnotesize
        #2}}%
  }%
  \thispagestyle{dingens}%
}

%%% définit le style (arabic : 1,2,3...) et place des parenthèses
%%% autour de la numérotation
\renewcommand*{\thefootnote}{(\arabic{footnote})}
% http://www.tuteurs.ens.fr/logiciels/latex/footnote.html

%%%%%%%% tikz axis
% \pgfplotsset{every axis/.append style={
%                     axis x line=middle,    % put the x axis in the middle
%                     axis y line=middle,    % put the y axis in the middle
%                     axis line style={<->,color=blue}, % arrows on the axis
%                     xlabel={$x$},          % default put x on x-axis
%                     ylabel={$y$},          % default put y on y-axis
%             }}

%%%% s'utilise comme suit

% \begin{axis}[
%   xmin=-8,xmax=4,
%   ymin=-8,ymax=4,
%   grid=both,
%   ]
%   \addplot [domain=-3:3,samples=50]({x^3-3*x},{3*x^2-9}); 
% \end{axis}

%%%%%%%%



%%%%%%%%%%%% Pour avoir des numéros de section qui correspondent à
%%%%%%%%%%%% ceux du tableau
\renewcommand{\thesection}{\Roman{section}.\hspace{-.3cm}}
\renewcommand{\thesubsection}{\Roman{section}.\arabic{subsection}.\hspace{-.2cm}}
\renewcommand{\thesubsubsection}{\Roman{section}.\arabic{subsection}.\alph{subsubsection})\hspace{-.2cm}}
%%%%%%%%%%%% 

%%% Changer le nom des figures : Fig. au lieu de Figure
\usepackage[font=small,labelfont=bf,labelsep=space]{caption}
\captionsetup{%
  figurename=Fig.,
  tablename=Tab.
}
% \renewcommand{\thesection}{\Roman{section}.\hspace{-.2cm}}
% \renewcommand{\thesubsection}{\Roman{section}
%   .\hspace{.2cm}\arabic{subsection}\ .\hspace{-.3cm}}
% \renewcommand{\thesubsubsection}{\alph{subsection})}

\newenvironment{tabliste}[1]
{\begin{tabenum}[\bfseries\small\itshape #1]}{\end{tabenum}} 

%%%% ESSAI contre le too deeply nested %%%%
%%%% ATTENTION au package enumitem qui se comporte mal avec les
%%%% noliste, à redéfinir !
% \usepackage{enumitem}

% \setlistdepth{9}

% \newlist{myEnumerate}{enumerate}{9}
% \setlist[myEnumerate,1]{label=(\arabic*)}
% \setlist[myEnumerate,2]{label=(\Roman*)}
% \setlist[myEnumerate,3]{label=(\Alph*)}
% \setlist[myEnumerate,4]{label=(\roman*)}
% \setlist[myEnumerate,5]{label=(\alph*)}
% \setlist[myEnumerate,6]{label=(\arabic*)}
% \setlist[myEnumerate,7]{label=(\Roman*)}
% \setlist[myEnumerate,8]{label=(\Alph*)}
% \setlist[myEnumerate,9]{label=(\roman*)}

%%%%%

\newenvironment{noliste}[1] %
{\begin{enumerate}[\bfseries\small\itshape #1]} %
  {\end{enumerate}}

\newenvironment{nonoliste}[1] %
{\begin{enumerate}[\hspace{-12pt}\bfseries\small\itshape #1]} %
  {\end{enumerate}}

\newenvironment{arrayliste}[1]{ 
  % List with minimal white space to fit in small areas, e.g. table
  % cell
  \begin{minipage}[t]{\linewidth} %
    \begin{enumerate}[\bfseries\small\itshape #1] %
      {\leftmargin=0.5em \rightmargin=0em
        \topsep=0em \parskip=0em \parsep=0em
        \listparindent=0em \partopsep=0em \itemsep=0pt
        \itemindent=0em \labelwidth=\leftmargin\labelsep+0.25em}
    }{
    \end{enumerate}\end{minipage}
}

\newenvironment{nolistes}[2]
{\begin{enumerate}[\bfseries\small\itshape
    #1]\setlength{\itemsep}{#2 mm}}{\end{enumerate}}

\newenvironment{liste}[1]
{\begin{enumerate}[\hspace{12pt}\bfseries\small\itshape
    #1]}{\end{enumerate}}   


%%%%%%%% Pour les programmes de colle %%%%%%%

\newcommand{\cours}{{\small \tt (COURS)}} %
\newcommand{\poly}{{\small \tt (POLY)}} %
\newcommand{\exo}{{\small \tt (EXO)}} %
\newcommand{\culture}{{\small \tt (CULTURE)}} %
\newcommand{\methodo}{{\small \tt (MÉTHODO)}} %
\newcommand{\methodob}{\Boxed{\mbox{\tt MÉTHODO}}} %

%%%%%%%% Pour les TD %%%%%%%
\newtheoremstyle{exostyle} {\topsep} % espace avant
{.6cm} % espace apres
{} % Police utilisee par le style de thm
{} % Indentation (vide = aucune, \parindent = indentation paragraphe)
{\bfseries} % Police du titre de thm
{} % Signe de ponctuation apres le titre du thm
{ } % Espace apres le titre du thm (\newline = linebreak)
{\thmname{#1}\thmnumber{ #2}\thmnote{.
    \normalfont{\textit{#3}}}} % composants du titre du thm : \thmname
                               % = nom du thm, \thmnumber = numéro du
                               % thm, \thmnote = sous-titre du thm
 
\theoremstyle{exostyle}
\newtheorem{exercice}{Exercice}
\newtheorem*{exoCours}{Exercice}

%%%%%%%% Pour des théorèmes Sans Espaces APRÈS %%%%%%%
\newtheoremstyle{exostyleSE} {\topsep} % espace avant
{} % espace apres
{} % Police utilisee par le style de thm
{} % Indentation (vide = aucune, \parindent = indentation paragraphe)
{\bfseries} % Police du titre de thm
{} % Signe de ponctuation apres le titre du thm
{ } % Espace apres le titre du thm (\newline = linebreak)
{\thmname{#1}\thmnumber{ #2}\thmnote{.
    \normalfont{\textit{#3}}}} % composants du titre du thm : \thmname
                               % = nom du thm, \thmnumber = numéro du
                               % thm, \thmnote = sous-titre du thm
 
\theoremstyle{exostyleSE}
\newtheorem{exerciceSE}{Exercice}
\newtheorem*{exoCoursSE}{Exercice}

% \newcommand{\lims}[2]{\prod\limits_{#1}^{#2}}

\newtheorem{theoremSE}{Théorème}[]
\newtheorem{lemmaSE}{Lemme}[]
\newtheorem{propositionSE}{Proposition}[]
\newtheorem{corollarySE}{Corollaire}[]

% \newenvironment{proofSE}[1][Démonstration]{\begin{trivlist}
% \item[\hskip \labelsep {\bfseries #1}]}{\end{trivlist}}
\newenvironment{definitionSE}[1][Définition]{\begin{trivlist}
  \item[\hskip \labelsep {\bfseries #1}]}{\end{trivlist}}
\newenvironment{exampleSE}[1][Exemple]{\begin{trivlist} 
  \item[\hskip \labelsep {\bfseries #1}]}{\end{trivlist}}
\newenvironment{examplesSE}[1][Exemples]{\begin{trivlist}
\item[\hskip \labelsep {\bfseries #1}]}{\end{trivlist}}
\newenvironment{notationSE}[1][Notation]{\begin{trivlist}
\item[\hskip \labelsep {\bfseries #1}]}{\end{trivlist}}
\newenvironment{proprieteSE}[1][Propriété]{\begin{trivlist}
\item[\hskip \labelsep {\bfseries #1}]}{\end{trivlist}}
\newenvironment{proprietesSE}[1][Propriétés]{\begin{trivlist}
\item[\hskip \labelsep {\bfseries #1}]}{\end{trivlist}}
\newenvironment{remarkSE}[1][Remarque]{\begin{trivlist}
\item[\hskip \labelsep {\bfseries #1}]}{\end{trivlist}}
\newenvironment{applicationSE}[1][Application]{\begin{trivlist}
\item[\hskip \labelsep {\bfseries #1}]}{\end{trivlist}}

%%%%%%%%%%% Obtenir les étoiles sans charger le package MnSymbol
%%%%%%%%%%%
\DeclareFontFamily{U} {MnSymbolC}{}
\DeclareFontShape{U}{MnSymbolC}{m}{n}{
  <-6> MnSymbolC5
  <6-7> MnSymbolC6
  <7-8> MnSymbolC7
  <8-9> MnSymbolC8
  <9-10> MnSymbolC9
  <10-12> MnSymbolC10
  <12-> MnSymbolC12}{}
\DeclareFontShape{U}{MnSymbolC}{b}{n}{
  <-6> MnSymbolC-Bold5
  <6-7> MnSymbolC-Bold6
  <7-8> MnSymbolC-Bold7
  <8-9> MnSymbolC-Bold8
  <9-10> MnSymbolC-Bold9
  <10-12> MnSymbolC-Bold10
  <12-> MnSymbolC-Bold12}{}

\DeclareSymbolFont{MnSyC} {U} {MnSymbolC}{m}{n}

\DeclareMathSymbol{\filledlargestar}{\mathrel}{MnSyC}{205}
\DeclareMathSymbol{\largestar}{\mathrel}{MnSyC}{131}

\newcommand{\facile}{\rm{(}$\scriptstyle\largestar$\rm{)}} %
\newcommand{\moyen}{\rm{(}$\scriptstyle\filledlargestar$\rm{)}} %
\newcommand{\dur}{\rm{(}$\scriptstyle\filledlargestar\filledlargestar$\rm{)}} %
\newcommand{\costaud}{\rm{(}$\scriptstyle\filledlargestar\filledlargestar\filledlargestar$\rm{)}}

%%%%%%%%%%%%%%%%%%%%%%%%%

%%%%%%%%%%%%%%%%%%%%%%%%%
%%%%%%%% Fin de la partie TD

%%%%%%%%%%%%%%%%
%%%%%%%%%%%%%%%%
\makeatletter %
\newenvironment{myitemize}{%
  \setlength{\topsep}{0pt} %
  \setlength{\partopsep}{0pt} %
  \renewcommand*{\@listi}{\leftmargin\leftmargini \parsep\z@
    \topsep\z@ \itemsep\z@} \let\@listI\@listi %
  \itemize %
}{\enditemize} %
\makeatother
%%%%%%%%%%%%%%%%
%%%%%%%%%%%%%%%%

%% Commentaires dans la correction du livre

\newcommand{\Com}[1]{
% Define box and box title style
\tikzstyle{mybox} = [draw=black!50,
very thick,
    rectangle, rounded corners, inner sep=10pt, inner ysep=8pt]
\tikzstyle{fancytitle} =[rounded corners, fill=black!80, text=white]
\tikzstyle{fancylogo} =[ text=white]
\begin{center}

\begin{tikzpicture}
\node [mybox] (box){%

    \begin{minipage}{0.90\linewidth}
\vspace{6pt}  #1
    \end{minipage}
};
\node[fancytitle, right=10pt] at (box.north west) 
{\bfseries\normalsize{Commentaire}};

\end{tikzpicture}%

\end{center}
%
}

\NewEnviron{remark}{%
  % Define box and box title style
  \tikzstyle{mybox} = [draw=black!50, very thick, rectangle, rounded
  corners, inner sep=10pt, inner ysep=8pt] %
  \tikzstyle{fancytitle} = [rounded corners , fill=black!80,
  text=white] %
  \tikzstyle{fancylogo} =[ text=white]
  \begin{center}
    \begin{tikzpicture}
      \node [mybox] (box){%
        \begin{minipage}{0.90\linewidth}
          \vspace{6pt} \BODY
        \end{minipage}
      }; %
      \node[fancytitle, right=10pt] at (box.north west) %
      {\bfseries\normalsize{Commentaire}}; %
    \end{tikzpicture}%
  \end{center}
}

\NewEnviron{remarkST}{%
  % Define box and box title style
  \tikzstyle{mybox} = [draw=black!50, very thick, rectangle, rounded
  corners, inner sep=10pt, inner ysep=8pt] %
  \tikzstyle{fancytitle} = [rounded corners , fill=black!80,
  text=white] %
  \tikzstyle{fancylogo} =[ text=white]
  \begin{center}
    \begin{tikzpicture}
      \node [mybox] (box){%
        \begin{minipage}{0.90\linewidth}
          \vspace{6pt} \BODY
        \end{minipage}
      }; %
      % \node[fancytitle, right=10pt] at (box.north west) %
%       {\bfseries\normalsize{Commentaire}}; %
    \end{tikzpicture}%
  \end{center}
}

\NewEnviron{remarkL}[1]{%
  % Define box and box title style
  \tikzstyle{mybox} = [draw=black!50, very thick, rectangle, rounded
  corners, inner sep=10pt, inner ysep=8pt] %
  \tikzstyle{fancytitle} =[rounded corners, fill=black!80,
  text=white] %
  \tikzstyle{fancylogo} =[ text=white]
  \begin{center}
    \begin{tikzpicture}
      \node [mybox] (box){%
        \begin{minipage}{#1\linewidth}
          \vspace{6pt} \BODY
        \end{minipage}
      }; %
      \node[fancytitle, right=10pt] at (box.north west) %
      {\bfseries\normalsize{Commentaire}}; %
    \end{tikzpicture}%
  \end{center}
}

\NewEnviron{remarkSTL}[1]{%
  % Define box and box title style
  \tikzstyle{mybox} = [draw=black!50, very thick, rectangle, rounded
  corners, inner sep=10pt, inner ysep=8pt] %
  \tikzstyle{fancytitle} =[rounded corners, fill=black!80,
  text=white] %
  \tikzstyle{fancylogo} =[ text=white]
  \begin{center}
    \begin{tikzpicture}
      \node [mybox] (box){%
        \begin{minipage}{#1\linewidth}
          \vspace{6pt} \BODY
        \end{minipage}
      }; %
%       \node[fancytitle, right=10pt] at (box.north west) %
%       {\bfseries\normalsize{Commentaire}}; %
    \end{tikzpicture}%
  \end{center}
}

\NewEnviron{titre} %
{ %
  ~\\[-1.8cm]
  \begin{center}
    \bf \LARGE \BODY
  \end{center}
  ~\\[-.6cm]
  \hrule %
  \vspace*{.2cm}
} %

\NewEnviron{titreL}[2] %
{ %
  ~\\[-#1cm]
  \begin{center}
    \bf \LARGE \BODY
  \end{center}
  ~\\[-#2cm]
  \hrule %
  \vspace*{.2cm}
} %



%%%%%%%%%%% Redefinition \chapter



\usepackage[explicit]{titlesec}
\usepackage{color}
\titleformat{\chapter}
{\gdef\chapterlabel{}
\selectfont\huge\bf}
%\normalfont\sffamily\Huge\bfseries\scshape}
{\gdef\chapterlabel{\thechapter)\ }}{0pt}
{\begin{tikzpicture}[remember picture,overlay]
\node[yshift=-3cm] at (current page.north west)
{\begin{tikzpicture}[remember picture, overlay]
\draw (.1\paperwidth,0) -- (.9\paperwidth,0);
\draw (.1\paperwidth,2) -- (.9\paperwidth,2);
%(\paperwidth,3cm);
\node[anchor=center,xshift=.5\paperwidth,yshift=1cm, rectangle,
rounded corners=20pt,inner sep=11pt]
{\color{black}\chapterlabel#1};
\end{tikzpicture}
};
\end{tikzpicture}
}
\titlespacing*{\chapter}{0pt}{50pt}{-75pt}




%%%%%%%%%%%%%% Affichage chapter dans Table des matieres

\makeatletter
\renewcommand*\l@chapter[2]{%
  \ifnum \c@tocdepth >\m@ne
    \addpenalty{-\@highpenalty}%
    \vskip 1.0em \@plus\p@
    \setlength\@tempdima{1.5em}%
    \begingroup
      \parindent \z@ \rightskip \@pnumwidth
      \parfillskip -\@pnumwidth
      \leavevmode %\bfseries
      \advance\leftskip\@tempdima
      \hskip -\leftskip
      #1\nobreak\ 
       \leaders\hbox{$\m@th
        \mkern \@dotsep mu\hbox{.}\mkern \@dotsep
        mu$}\hfil\nobreak\hb@xt@\@pnumwidth{\hss #2}\par
      \penalty\@highpenalty
    \endgroup
  \fi}
\makeatother




%%%%%%%%%%%%%%%%% Redefinition part




% \renewcommand{\thesection}{\Roman{section}.\hspace{-.3cm}}
% \renewcommand{\thesubsection}{\Alph{subsection}.\hspace{-.2cm}}

\pagestyle{fancy} %
\pagestyle{fancy} %
 \lhead{ECE2 \hfill Mathématiques \\} %
\chead{\hrule} %
\rhead{} %
\lfoot{} %
\cfoot{} %
\rfoot{\thepage} %

\renewcommand{\headrulewidth}{0pt}% : Trace un trait de séparation
                                    % de largeur 0,4 point. Mettre 0pt
                                    % pour supprimer le trait.

\renewcommand{\footrulewidth}{0.4pt}% : Trace un trait de séparation
                                    % de largeur 0,4 point. Mettre 0pt
                                    % pour supprimer le trait.

\setlength{\headheight}{14pt}

\title{\bf \vspace{-1.6cm} EML 2016} %
\author{} %
\date{} %
\begin{document}

\maketitle %
\vspace{-1.2cm}\hrule %
\thispagestyle{fancy}

\vspace*{.4cm}

%%DEBUT

\section*{EXERCICE I}

\noindent
On note $I$ et $A$ les matrices de $\M{3}$ définies 
par :
\[
I=
\begin{smatrix}
 1 & 0 & 0\\
 0 & 1 & 0\\
 0 & 0 & 1
\end{smatrix}
, \qquad 
A=
\begin{smatrix}
 0 & 1 & 0\\
 1 & 0 & 1\\
 0 & 1 & 0
\end{smatrix},
\]
et $\mathcal{E}$ l'ensemble des matrices de $\M{3}$ 
défini par :
\[
\mathcal{E}=\left\{
\begin{smatrix}
 a+c & b & c\\
 b & a+2c & b\\
 c & b & a+c
\end{smatrix} \; ; \; (a,b,c) \in \R^3 \right\}
\]

\subsection*{PARTIE I : Étude de la matrice $A$}

\begin{noliste}{1.}
\setlength{\itemsep}{2mm}
\item Calculer $A^2$.

\begin{proof}~
 \[
  A^2=
  \begin{smatrix}
   0 & 1 & 0\\
   1 & 0 & 1\\
   0 & 1 & 0
  \end{smatrix}
  \times
  \begin{smatrix}
   0 & 1 & 0\\
   1 & 0 & 1\\
   0 & 1 & 0
  \end{smatrix}
  =
  \begin{smatrix}
   1 & 0 & 1\\
   0 & 2 & 0\\
   1 & 0 & 1
  \end{smatrix}
 \]
 \conc{$A^2=
 \begin{smatrix}
   1 & 0 & 1\\
   0 & 2 & 0\\
   1 & 0 & 1
  \end{smatrix}$}~\\[-1cm]
\end{proof}


\item Montrer que la famille $(I,A,A^2)$ est libre.

\begin{proof}~\\
 Soit $(\lambda_1, \lambda_2,\lambda_3)\in\R^3$. On suppose :
 \[
  \lambda_1 \cdot I_3 + \lambda_2 \cdot A + \lambda_3 \cdot A^2 
  =0_{\M{3}} \qquad (1)
 \]
 Or :
 \[
  \begin{array}{rcl}
   \lambda_1 \cdot I_3 + \lambda_2 \cdot A + \lambda_3 \cdot A^2 &=&
   \lambda_1 \cdot
   \begin{smatrix}
    1 & 0 & 0\\
    0 & 1 & 0\\
    0 & 0 & 1
   \end{smatrix}
   + \lambda_2 \cdot
   \begin{smatrix}
    0 & 1 & 0\\
    1 & 0 & 1\\
    0 & 1 & 0
   \end{smatrix}
   + \lambda_3 \cdot
   \begin{smatrix}
    1 & 0 & 1\\
    0 & 2 & 0\\
    1 & 0 & 1
   \end{smatrix}
   \\[.8cm]
   &=& 
   \begin{smatrix}
    \lambda_1 + \lambda_3 & \lambda_2 & \lambda_3\\
    \lambda_2 & \lambda_1 + 2 \lambda_3 & \lambda_2\\
    \lambda_3 & \lambda_2 & \lambda_1 + \lambda_3
   \end{smatrix}
  \end{array}
 \]
 L'égalité $(1)$ entraîne alors :
 \[
  \begin{array}{rcl}
   \left\{
   \begin{array}{rrrrrcl}
    \lambda_1 & & & + & \lambda_3 & = & 0\\
    \lambda_1 & & & + & 2\, \lambda_3 & = & 0\\
    & & \lambda_2 & & & = & 0\\
    & & & & \lambda_3 & = & 0
   \end{array}
   \right.
   & \Leftrightarrow &
   \left\{
    \lambda_1 = 
    \lambda_2 = 
    \lambda_3 = 0
   \right.
   \\[-.6cm]
   & & \mbox{\it{(par remontées successives)}}
  \end{array}
 \]
 
\conc{La famille $(I,A,A^2)$ est libre.}~\\[-1.2cm] 
\end{proof}

\newpage

\item
\begin{noliste}{a)}
\item Justifier, sans calcul, que $A$ est diagonalisable.

\begin{proof}~
 \conc{$A$ est une matrice symétrique réelle, donc elle est 
 diagonalisable.}~\\[-1.2cm]
\end{proof}

\item Déterminer une matrice $P$ de $\M{3}$ 
inversible dont tous les coefficients de la première ligne sont égaux à 
1 et une matrice $D$ de $\M{3}$ diagonale dont tous 
les coefficients diagonaux sont dans l'ordre croissant telles que : 
$A=PDP^{-1}$.

\begin{proof}~
 \begin{noliste}{$\sbullet$}
  \item D'après la question \itbf{3.a)}, la matrice $A$ est 
  diagonalisable. Donc il existe une matrice $P$ inversible 
  ($P$ est la concaténation des vecteurs des 
  bases des sous-espaces propres de $A$) et une matrice $D$ diagonale 
  dont les coefficients diagonaux sont les valeurs propres de $A$ 
  telles que $A=PDP^{-1}$.
  \item Déterminons les valeurs propres de $A$.\\
  Les valeurs propres de $A$ sont les réels $\lambda$ tels que la 
  matrice $(A-\lambda I)$ n'est pas inversible, c'est-à-dire 
  $\rg(A-\lambda 
  I)<3$.\\
  Soit $\lambda\in\R$.
  \[
	 \begin{array}{rcl}
	  \rg(A-\lambda I) &=& \rg\left(
	  \begin{smatrix}
	   -\lambda & 1 & 0\\
	   1 & -\lambda & 1\\
	   0 & 1 & -\lambda
	  \end{smatrix}\right)
	  \\[.8cm]
	  &
	  \begin{arrayEg}
	   L_1 \leftrightarrow L_2
	  \end{arrayEg}
	  &
	  \rg\left(
	  \begin{smatrix}
	   1 & -\lambda & 1\\
	   -\lambda & 1 & 0\\
	   0 & 1 & -\lambda
	  \end{smatrix}\right)
	  \\[.8cm]
	  &
	  \begin{arrayEg}
	   L_2 \leftarrow L_2+\lambda L_1
	  \end{arrayEg}
	  &
	  \rg\left(
	  \begin{smatrix}
	   1 & -\lambda & 1\\
	   0 & 1-\lambda^2 & \lambda\\
	   0 & 1 & -\lambda
	  \end{smatrix}\right)
	  \\[.8cm]
	  &
	  \begin{arrayEg}
	   L_2 \leftrightarrow L_3
	  \end{arrayEg}
	  &
	  \rg\left(
	  \begin{smatrix}
	   1 & -\lambda & 1\\
	   0 & 1 & -\lambda\\
	   0 & 1-\lambda^2 & \lambda
	  \end{smatrix}\right)
	  \\[.8cm]
	  &
	  \begin{arrayEg}
	   L_3 \leftarrow L_3 - (1-\lambda^2)L_2
	  \end{arrayEg}
	  &
	  \rg\left(
	  \begin{smatrix}
	   1 & -\lambda & 1\\
	   0 & 1 & -\lambda\\
	   0 & 0 & \lambda+\lambda(1-\lambda^2)
	  \end{smatrix}\right)
	 \end{array}
	\]
	On obtient une réduite triangulaire supérieure.\\ 
	Elle est donc non inversible si et seulement si l'un de ses 
	coefficients diagonaux est nul.\\ 
	Ses $2$ premiers coefficients diagonaux sont $1$ et $1$ (et 
	$1\neq 0$), et on a :
	  \[
	    \lambda+\lambda(1+\lambda^2)=0 \ \Leftrightarrow \ \lambda 
	    (2-\lambda^2)=0 \ \Leftrightarrow \ 
	    \lambda(\sqrt{2}-\lambda)(\sqrt{2}+\lambda)=0
	    \ \Leftrightarrow \
	    \lambda\in\{0,-\sqrt{2},\sqrt{2}\}
	  \]
	  Ainsi : $\rg(A-\lambda I)<3 \ \Leftrightarrow \ 
	  \lambda\in\{0,-\sqrt{2},\sqrt{2}\}$.\\
	  \conc{Las valeurs propres de $A$ sont $-\sqrt{2}$, $0$ et 
	  $\sqrt{2}$.}
	  
	  \newpage
	  
  \item Déterminons une base de $E_{\sqrt{2}}(A)$ le sous-espace 
  propre de $A$ associé à la valeur propre $\sqrt{2}$.\\
  Soit $X=
	\begin{smatrix}
	 x\\ y\\ z
	\end{smatrix}\in\M{3,1}$.
	\[
	 \begin{array}{rcl}
	  X\in E_{\sqrt{2}}(A)
	  & \Longleftrightarrow & (A-\sqrt{2}\, I)X=0
	  \\[.2cm]
	  & \Longleftrightarrow & 
	  \begin{smatrix}
	   -\sqrt{2} & 1 & 0\\
	   1 & -\sqrt{2} & 1\\
	   0 & 1 & -\sqrt{2}
	  \end{smatrix}
	  \begin{smatrix}
	   x\\ y\\ z
	  \end{smatrix}
	  =
	  \begin{smatrix}
	   0\\ 0\\ 0
	  \end{smatrix}
	  \\[.8cm]
	  & \Longleftrightarrow & 
	  \left\{
	  \begin{array}{rrrrrcl}
	   -\sqrt{2}\, x & + & y & & & = & 0\\
	   x & - & \sqrt{2}\, y & + & z & = & 0\\
	    & & y & - & \sqrt{2}\, z & = & 0
	  \end{array}
	  \right.
	  \\[.8cm]
	  &
	  \begin{arrayEq}
	   L_1 \leftrightarrow L_2
	  \end{arrayEq}
	  &
	  \left\{
	  \begin{array}{rrrrrcl}
	   x & - & \sqrt{2}\, y & + & z & = & 0\\
	   -\sqrt{2}\, x & + & y & & & = & 0\\
	    & & y & - & \sqrt{2}\, z & = & 0
	  \end{array}
	  \right.
	  \\[.8cm]
	  &
	  \begin{arrayEq}
	   L_2 \leftarrow L_2+\sqrt{2}\, L_1
	  \end{arrayEq}
	  &
	  \left\{
	  \begin{array}{rrrrrcl}
	   x & - & \sqrt{2} \, y & + & z & = & 0\\
	    & - & y & + & \sqrt{2} \, z & = & 0\\
	    & & y & - & \sqrt{2} \, z & = & 0
	  \end{array}
	  \right.
	  \\[.8cm]
	  &
	  \Longleftrightarrow
	  &
	  \left\{
	  \begin{array}{rrrrrcl}
	   x & - & \sqrt{2} \, y & + & z & = & 0\\
	    & & y & - & \sqrt{2} \, z & = & 0
	  \end{array}
	  \right.
	  \\[.8cm]
	  &
	  \Longleftrightarrow
	  &
	  \left\{
	  \begin{array}{rrrcl}
	   x & - & \sqrt{2} \, y & = & -z\\
	    & & y & = & \sqrt{2} \, z
	  \end{array}
	  \right.
	  \\[.8cm]
	  &
	  \begin{arrayEq}
	   L_1 \leftarrow L_1+\sqrt{2} \, L_2
	  \end{arrayEq}
	  &
	  \left\{
	  \begin{array}{rcl}
	   x & = & -z+2z=z\\
	   y & = & \sqrt{2} \, z
	  \end{array}
	  \right.
	 \end{array}
	\]
	
	Finalement on obtient l'expression de $E_{\sqrt{2}}(A)$ 
	suivante :
	\[
	 \begin{array}{rcl}
	  E_{\sqrt{2}}(A) & = & \{X\in\M{3,1} \ | \ AX=\sqrt{2}\, X\}
	   = \{
	  \begin{smatrix}
	   x\\ y\\ z
	  \end{smatrix}
	  \ | \
	   x = z \mbox{ et }
	   y = \sqrt{2} \, z
	  \}
	  \\[.6cm]
	  & = & \left\{\right.
	  \begin{smatrix}
	   z\\ 
	   \sqrt{2}\, z\\ 
	   z
	  \end{smatrix}
	  \ \left/ \
	  z\in\R\right\}
	   = \left\{\right. z \cdot
	  \begin{smatrix}
	   1\\ 
	   \sqrt{2}\\ 
	   1
	  \end{smatrix}
	  \ \left/ \
	  z\in\R\right\}
	  \\[.6cm]
	  & = & \Vect{
	  \begin{smatrix}
	   1\\ \sqrt{2}\\ 1
	  \end{smatrix}
	  }
	 \end{array}
	\]
	On sait donc que la famille $\left( \begin{smatrix}
	   1\\ \sqrt{2}\\ 1
	  \end{smatrix}\right)$ :
	\begin{noliste}{$\stimes$}
	  \item engendre $E_{\sqrt{2}}(A)$,
	  \item est une famille libre de $\M{3,1}$ car elle est 
	  constituée d'un unique vecteur non nul.
	\end{noliste}
	
	\conc{$\left( \begin{smatrix}
	   1\\ \sqrt{2}\\ 1
	  \end{smatrix}\right)$ est une base de $E_{\sqrt{2}}(A)$.}
	  
  \item Déterminons une base de $E_{0}(A)$ le sous-espace 
  propre de $A$ associé à la valeur propre $0$.\\
  Soit $X=
	\begin{smatrix}
	 x\\ y\\ z
	\end{smatrix}\in\M{3,1}$.
	\[
	 \begin{array}{rcl}
	  X\in E_{0}(A) & \Longleftrightarrow & AX=0
	  \\[.2cm]
	  & \Longleftrightarrow & 
	  \begin{smatrix}
	   0 & 1 & 0\\
	   1 & 0 & 1\\
	   0 & 1 & 0
	  \end{smatrix}
	  \begin{smatrix}
	   x\\ y\\ z
	  \end{smatrix}
	  =
	  \begin{smatrix}
	   0\\ 0\\ 0
	  \end{smatrix}
	  \\[.8cm]
	  & \Longleftrightarrow & 
	  \left\{
	  \begin{array}{rrrrrcl}
	    & & y & & & = & 0\\
	   x & & & + & z & = & 0\\
	    & & y & & & = & 0
	  \end{array}
	  \right.
	  \\[.8cm]
	  &
	  \Longleftrightarrow
	  &
	  \left\{
	  \begin{array}{rrrrrcl}
	   x & & & + & z & = & 0\\
	    & & y & & & = & 0
	  \end{array}
	  \right.
	  \\[.8cm]
	  &
	  \Longleftrightarrow
	  &
	  \left\{
	  \begin{array}{rrrcl}
	   x & & & = & -z\\
	   & & y & = & 0
	  \end{array}
	  \right.
	 \end{array}
	\]
	Finalement on obtient l'expression de $E_{0}(A)$ suivante :
	\[
	 \begin{array}{rcl}
	  E_{0}(A) & = & \{X\in\M{3,1} \ | \ AX=0\}
	  = \{
	  \begin{smatrix}
	   x\\ y\\ z
	  \end{smatrix}
	  \ | \
	   x = -z \mbox{ et }
	   y = 0
	  \}
	  \\[.6cm]
	  & = & \left\{\right.
	  \begin{smatrix}
	   x\\ 0\\ -x
	  \end{smatrix}
	  \ \left/ \
	  x\in\R\right\}
	  = \left\{\right. x \cdot
	  \begin{smatrix}
	   1\\ 0\\ -1
	  \end{smatrix}
	  \ \left/ \
	  x\in\R\right\}
	  \\[.6cm]
	  & = & \Vect{
	  \begin{smatrix}
	   1\\ 0\\ -1
	  \end{smatrix}
	  }
	 \end{array}
	\]
	On sait donc que la famille $\left( \begin{smatrix}
	   1\\ 0\\ -1
	  \end{smatrix}\right)$ :
	\begin{noliste}{$\stimes$}
	  \item engendre $E_{0}(A)$,
	  \item est une famille libre de $\M{3,1}$ car elle est 
	  constituée d'un unique vecteur non nul.
	\end{noliste}
	
	\conc{$\left( \begin{smatrix}
	   1\\ 0\\ -1
	  \end{smatrix}\right)$ est une base de $E_{0}(A)$.}
	  
	  \newpage
  
  \item Déterminons une base de $E_{-\sqrt{2}}(A)$ le sous-espace 
  propre de $A$ associé à la valeur propre $-\sqrt{2}$.\\
  Soit $X=
	\begin{smatrix}
	 x\\ y\\ z
	\end{smatrix}\in\M{3,1}$.
	\[
	 \begin{array}{rcl}
	  X\in E_{-\sqrt{2}}(A) 
	  & \Longleftrightarrow & (A+\sqrt{2}\, I)X=0
	  \\[.2cm]
	  & \Longleftrightarrow & 
	  \begin{smatrix}
	   \sqrt{2} & 1 & 0\\
	   1 & \sqrt{2} & 1\\
	   0 & 1 & \sqrt{2}
	  \end{smatrix}
	  \begin{smatrix}
	   x\\ y\\ z
	  \end{smatrix}
	  =
	  \begin{smatrix}
	   0\\ 0\\ 0
	  \end{smatrix}
	  \\[.8cm]
	  & \Longleftrightarrow & 
	  \left\{
	  \begin{array}{rrrrrcl}
	   \sqrt{2}\, x & + & y & & & = & 0\\
	   x & + & \sqrt{2}\, y & + & z & = & 0\\
	    & & y & + & \sqrt{2}\, z & = & 0
	  \end{array}
	  \right.
	  \\[.8cm]
	  &
	  \begin{arrayEq}
	   L_1 \leftrightarrow L_2
	  \end{arrayEq}
	  &
	  \left\{
	  \begin{array}{rrrrrcl}
	   x & + & \sqrt{2}\, y & + & z & = & 0\\
	   \sqrt{2}\, x & + & y & & & = & 0\\
	    & & y & + & \sqrt{2}\, z & = & 0
	  \end{array}
	  \right.
	  \\[.8cm]
	  &
	  \begin{arrayEq}
	   L_2 \leftarrow L_2-\sqrt{2}\, L_1
	  \end{arrayEq}
	  &
	  \left\{
	  \begin{array}{rrrrrcl}
	   x & + & \sqrt{2}\, y & + & z & = & 0\\
	    & - & y & - & \sqrt{2}\, z & = & 0\\
	    & & y & + & \sqrt{2}\, z & = & 0
	  \end{array}
	  \right.
	  \\[.8cm]
	  &
	  \Longleftrightarrow
	  &
	  \left\{
	  \begin{array}{rrrrrcl}
	   x & + & \sqrt{2}\, y & + & z & = & 0\\
	    & & y & + & \sqrt{2}\, z & = & 0
	  \end{array}
	  \right.
	  \\[.8cm]
	  &
	  \Longleftrightarrow
	  &
	  \left\{
	  \begin{array}{rrrcl}
	   x & + & \sqrt{2} \, y & = & -z\\
	    & & y & = & -\sqrt{2} \, z
	  \end{array}
	  \right.
	  \\[.8cm]
	  &
	  \begin{arrayEq}
	   L_1 \leftarrow L_1-\sqrt{2} \, L_2
	  \end{arrayEq}
	  &
	  \left\{
	  \begin{array}{rcl}
	   x & = & -z+2z=z\\
	   y & = & -\sqrt{2} \, z
	  \end{array}
	  \right.
	 \end{array}
	\]
	Finalement on obtient l'expression de $E_{-\sqrt{2}}(A)$ 
	suivante :
	\[
	 \begin{array}{rcl}
	  E_{-\sqrt{2}}(A) & = & \{X\in\M{3,1} \ | \ AX=-\sqrt{2}\, X\}
	  = \{
	  \begin{smatrix}
	   x\\ y\\ z
	  \end{smatrix}
	  \ | \
	   x = z \mbox{ et }
	   y = -\sqrt{2} \, z
	  \}
	  \\[.6cm]
	  & = & \left\{\right.
	  \begin{smatrix}
	   z\\ -\sqrt{2}\, z\\ z
	  \end{smatrix}
	  \ \left/ \
	  z\in\R\right\}
	  = \left\{\right. z \cdot
	  \begin{smatrix}
	   1\\ -\sqrt{2}\\ 1
	  \end{smatrix}
	  \ \left/ \
	  z\in\R\right\}
	  \\[.6cm]
	  & = & \Vect{
	  \begin{smatrix}
	   1\\ -\sqrt{2}\\ 1
	  \end{smatrix}
	  }
	 \end{array}
	\]
	On sait donc que la famille $\left( \begin{smatrix}
	   1\\ -\sqrt{2}\\ 1
	  \end{smatrix}\right)$ :
	\begin{noliste}{$\stimes$}
	  \item engendre $E_{-\sqrt{2}}(A)$, d'après le point précédent,
	  \item est une famille libre de $\M{3,1}$ car elle est 
	  constituée d'un unique vecteur non nul.
	\end{noliste}
	
	\conc{$\left( \begin{smatrix}
	   1\\ -\sqrt{2}\\ 1
	  \end{smatrix}\right)$ est une base de $E_{-\sqrt{2}}(A)$.}
	  
	  \newpage
	  
  \item On rappelle que la matrice $P$ est la concaténation des
  vecteurs des
  bases des sous-espaces propres de $A$, et la matrice $D$ des 
  valeurs propres de $A$.
  
  \conc{On obtient donc la décomposition suivante :\\
  $A=PDP^{-1}$ où $P=
  \begin{smatrix}
   1 & 1 & 1\\
   -\sqrt{2} & 0 & \sqrt{2}\\
   1 & -1 & 1
  \end{smatrix}$ et $D=
  \begin{smatrix}
   -\sqrt{2} & 0 & 0\\
   0 & 0 & 0\\
   0 & 0 & \sqrt{2}
  \end{smatrix}$.}~\\[-1.4cm]
 \end{noliste}
\end{proof}
\end{noliste}

\item Montrer : $A^3=2A$.

\begin{proof}~
   \[
    A^3 = A^2 A = 
    \begin{smatrix}
     1 & 0 & 1\\
     0 & 2 & 0\\
     1 & 0 & 1
    \end{smatrix}
    \times
    \begin{smatrix}
     0 & 1 & 0\\
     1 & 0 & 1\\
     0 & 1 & 0
    \end{smatrix}
    =
    \begin{smatrix}
     0 & 2 & 0\\
     2 & 0 & 2\\
     0 & 2 & 0
    \end{smatrix}
    = 2A
   \]
   \conc{$A^3=2A$}~\\[-1cm]
  \end{proof}
\end{noliste}

\subsection*{PARTIE II : Étude d'une application définie sur 
$\mathcal{E}$}
\begin{noliste}{1.}
\setcounter{enumi}{4}
\item Montrer que $\mathcal{E}$ est un sous-espace vectoriel de 
$\M{3}$ et que la famille $(I,A,A^2)$ est une base 
de $\mathcal{E}$. En déduire la dimension de $\mathcal{E}$.

\begin{proof}~
 \begin{noliste}{$\sbullet$}
  \item Montrons que $\mathcal{E}$ est un sous-espace vectoriel de 
  $\M{3}$.
  \[
   \begin{array}{rcl}
    \mathcal{E} & = & \left\{
    \begin{smatrix}
     a+c & b & c\\
     b & a+2c & b\\
     c & b & a+c
    \end{smatrix}
    \ | \
    (a,b,c)\in\R^3
    \right\}
    \\[.8cm]
    & = &
    \left\{
    a\begin{smatrix}
      1 & 0 & 0\\
      0 & 1 & 0\\
      0 & 0 & 1
     \end{smatrix}
    +b\begin{smatrix}
       0 & 1 & 0\\
       1 & 0 & 1\\
       0 & 1 & 0
      \end{smatrix}
    +c\begin{smatrix}
       1 & 0 & 1\\
       0 & 2 & 0\\
       1 & 0 & 1
      \end{smatrix}
    \ | \
    (a,b,c)\in\R^3
    \right\}
    \\[.8cm]
    & = &
    \left\{
    aI+bA+cA^2 \ | \ (a,b,c)\in\R^3
    \right\}
    \\[.2cm]
    & = &
    \Vect{I,A,A^2}
   \end{array}
  \]
  De plus, $(I,A,A^2)\in(\M{3})^3$.
  \conc{$\mathcal{E}$ est un sous-espace vectoriel de $\M{3}$.}
  
  \item Montrons que $(I,A,A^2)$ est une base de $\mathcal{E}$.\\
  La famille $(I,A,A^2)$ :
  \begin{noliste}{$\stimes$}
    \item engendre $\mathcal{E}$ (d'après le point précédent),
    \item est libre dans $\M{3}$ d'après la question \itbf{2.}
  \end{noliste}
  \conc{$(I,A,A^2)$ est une base de $\mathcal{E}$.}
  
  \item Déterminons la dimension de $\mathcal{E}$.
  \conc{$\dim(\mathcal{E})=\Card((I,A,A^2))=3$}~\\[-1.4cm]
 \end{noliste}
\end{proof}

\newpage

\item Montrer que, pour toute matrice $M$ de $\mathcal{E}$, la matrice 
$AM$ appartient à $\mathcal{E}$.

\begin{proof}~\\
 Soit $M\in\mathcal{E}$.\\
 On sait que $\mathcal{E}=\Vect{I,A,A^2}$, donc il existe $(a,b,c) \in 
 \R^3$ tel que $M=a \cdot I+b \cdot A+c \cdot A^2$.
 \[
  AM = A(a\cdot I+b \cdot A+c \cdot A^2)=a \cdot A+b \cdot A^2+c 
  \cdot A^3 = a \cdot A+b \cdot A^2+2c \cdot A=(a+2c) \cdot A+b \cdot 
  A^2
  \in\Vect{I,A,A^2}
 \]
 \conc{$\forall M\in\mathcal{E}$, $AM\in\mathcal{E}$.}~\\[-1cm]
\end{proof}
\end{noliste}

\noindent
On note $f$ l'application de $\mathcal{E}$ dans $\mathcal{E}$ qui, à 
toute matrice $M$ de $\mathcal{E}$, associe $AM$.

\begin{noliste}{1.}
\setcounter{enumi}{6}
\item Vérifier que $f$ est un endomorphisme de l'espace vectoriel 
$\mathcal{E}$.

\begin{proof}~
 \begin{noliste}{$\sbullet$}
  \item D'après la question \itbf{6.}, $f(\mathcal{E}) \subset 
  \mathcal{E}$.
  
  \item Soit $(\lambda,\mu)\in\R^2$ et soit $(M,N)\in\mathcal{E}^2$.
 \[
  f(\lambda \cdot M+\mu \cdot N) = A(\lambda \cdot M+\mu \cdot 
  N)=\lambda \cdot AM + \mu \cdot AN 
  =\lambda \cdot f(M) + \mu \cdot f(N)
 \]
 Donc $f$ est une application linéaire.
 \end{noliste}
 \conc{$f$ est un endomorphisme de $\mathcal{E}$.}~\\[-1.2cm]
\end{proof}

\item Former la matrice $F$ de $f$ dans la base $(I,A,A^2)$ de 
$\mathcal{E}$.

\begin{proof}~
 \begin{noliste}{$\sbullet$}
 \item $f(I)=A=0\times I +1\cdot A + 0\cdot A^2$. Donc 
 $\Mat_{(I,A,A^2)}(f(I))=
 \begin{smatrix}
  0\\ 1\\ 0
 \end{smatrix}$.
 \item $f(A)=A^2=0\cdot I +0\cdot A + 1\cdot A^2$. Donc 
 $\Mat_{(I,A,A^2)}(f(A))=
 \begin{smatrix}
  0\\ 0\\ 1
 \end{smatrix}$.
 \item $f(A^2)=A^3=2A=0\cdot I +2\cdot A + 0\cdot A^2$. Donc 
  $\Mat_{(I,A,A^2)} 
  (f(A^2))=
  \begin{smatrix}
   0\\ 2\\ 0
  \end{smatrix}$.
 \end{noliste}
  \conc{$F=\Mat_{(I,A,A^2)}(f)=
  \begin{smatrix}
   0 & 0 & 0\\
   1 & 0 & 2\\
   0 & 1 & 0
  \end{smatrix}$.}~\\[-1cm]
\end{proof}

\item
\begin{noliste}{a)}
\item Montrer : $f\circ f\circ f=2f$.

\begin{proof}~\\
 Soit $M\in\mathcal{E}$.
 \[
  \begin{array}{rcl}
  (f\circ f \circ f)(M) & = & f(f(f(M))) = 
  f(f(AM))=f(A\times AM)=f(A^2M)
  \\[.2cm]
  & = & A\times A^2M=  A^3M =2AM=2f(M)
  \end{array}
 \]
 \conc{$f\circ f\circ f=2f$}~\\[-1.2cm]
\end{proof}

\newpage

\item En déduire que toute valeur propre $\lambda$ de $f$ vérifie : 
$\lambda^3=2\lambda$.

\begin{proof}~\\
 Soit $\lambda\in\R$ une valeur propre de $f$.\\
      Notons $M$ un vecteur propre associé à $\lambda$. Par définition
      : $f(M) = \lambda \cdot M$.
      \begin{noliste}{$\sbullet$}
      \item D'après la question précédente : 
        \[
        (f\circ f\circ f)(M) = 2 f(M) = 2 \lambda \cdot M
        \]
      \item En raisonnant comme à la question précédente :
        \[
        \begin{array}{rcl@{\qquad}>{\it}R{5cm}}
          (f\circ f \circ f)(M) & = & f(f(f(M))) \\[.2cm]
          & = & f(f(\lambda M)) & (par définition de $M$)
          \nl 
          \nl[-.2cm]
          & = & f(\lambda f(M)) & (car $f$ est linéaire)
          \nl 
          \nl[-.2cm]
          & = & f(\lambda \cdot (\lambda \cdot M)) & (par définition de 
	  $M$)
          \nl 
          \nl[-.2cm]
          & = & f(\lambda^2 \cdot M) \ = \ \lambda^2 \cdot f(M) & (car
          $f$ est linéaire)
          \nl 
          \nl[-.2cm]
          & = & \lambda^2 \cdot (\lambda \cdot M) \ = \ \lambda^3 \cdot
          M 
        \end{array}
        \]

      \item En combinant les deux résultats précédents, on obtient :
        \[
        \lambda^3 M= 2\lambda M \ \text{ ou encore } \ (\lambda^3 - 2
        \lambda) \ M = 0
        \]
        Or $M$ est un vecteur propre, donc $M\neq 0$. Ainsi,
        $\lambda^3 - 2 \lambda = 0$.
      \end{noliste}
      \conc{Toute valeur propre $\lambda$ de $f$ vérifie  
        $\lambda^3 = 2\lambda$.}~\\[-1.2cm]
\end{proof}

\item Déterminer les valeurs propres et les sous-espaces propres de $f$.

\begin{proof}~
\begin{noliste}{$\sbullet$}
  \item D'après la question précédente, si $\lambda$ est une valeur
        propre de $f$, alors $\lambda^3-2\lambda =0$.\\
        {\it (c'est une implication, pas une équivalence !)}\\
        Or : 
        \[
        \lambda^3-2\lambda =0 \ \Leftrightarrow \
        \lambda(\lambda-\sqrt{2})(\lambda+\sqrt{2})=0 \
        \Leftrightarrow \ \lambda \in \{-\sqrt{2},0,\sqrt{2}\}
        \]
        \conc{$\spc{(f)}\subset\{-\sqrt{2},0,\sqrt{2}\}$}%
        Ceci permet d'affirmer que $-\sqrt{2}$, $0$, et $\sqrt{2}$
        sont les seules valeurs propres possibles de $f$. Il reste à
        déterminer lesquelles sont réellement valeurs propres.
        
  \newpage
  
  \item Il reste à montrer que $\{-\sqrt{2},0,\sqrt{2}\} \subset 
  \spc(f)$.
  \begin{noliste}{$-$}
    \item Déterminons $E_{\sqrt{2}}(f)=\kr(f-\sqrt{2} \cdot \id)$.\\
    Soit $M\in\mathcal{E}$. On note $X=
    \begin{smatrix}
     x\\ y\\ z
    \end{smatrix}=\Mat_{(I,A,A^2)}(M)$.
    \[
     \begin{array}{rcl}
       M\in E_{\sqrt{2}}(f) & \Longleftrightarrow & 
       (f-\sqrt{2} \, \id)(M) = 0_{\mathcal{E}}
       \\[.2cm]
       & \Longleftrightarrow & (F-\sqrt{2} \, I)X=0_{\M{3,1}}
       \\[.2cm]
       & \Longleftrightarrow &
      \begin{smatrix}
       -\sqrt{2} & 0 & 0\\
       1 & -\sqrt{2} & 2\\
       0 & 1 & -\sqrt{2}
      \end{smatrix}
      \begin{smatrix}
       x\\ y\\ z
      \end{smatrix}
      =\begin{smatrix}
        0\\ 0\\ 0
       \end{smatrix}
      \\[.8cm]
      & \Longleftrightarrow & \left\{
      \begin{array}{rrrrrcl}
       -\sqrt{2} \, x & & & & & = & 0\\
       x & - & \sqrt{2} \, y & + & 2z & = & 0\\
       & & y & - & \sqrt{2} \, z & = & 0
      \end{array}
      \right.
      \\[.8cm]
      & \Longleftrightarrow & \left\{
      \begin{array}{rrrrrcl}
       x & & & & & = & 0\\
       & - & \sqrt{2} \, y & + & 2z & = & 0\\
       & & y & - & \sqrt{2} \, z & = & 0
      \end{array}
      \right.
      \\[.8cm]
      & \Longleftrightarrow & \left\{
      \begin{array}{rrrcl}
       x & & & = & 0\\
       & & y & = & \sqrt{2} \, z
      \end{array}
      \right.
     \end{array}
    \]
    On obtient donc :
    \[
     \begin{array}{rcl}
      E_{\sqrt{2}}(f) & = & \left\{ M\in\mathcal{E} \ | \ f(M)= 
      \sqrt{2}\, M \right\}
      \\[.2cm]
      & = & \{ x \cdot I+y \cdot A+z \cdot A^2 \ | \
       x = 0 \mbox{ et }
       y = \sqrt{2} \, z
      \}
      \\[.4cm]
      & = & \left\{ 0\cdot I + \sqrt{2} \, z \cdot A + z \cdot A^2 
      \ | \ z\in\R\right\}
      \\[.2cm]
      & = & \left\{ z(\sqrt{2} \, A + A^2) 
      \ | \ z\in\R\right\}
      \\[.2cm]
      & = & \Vect{\sqrt{2} \, A+A^2}
     \end{array}
    \]
    En particulier, $E_{\sqrt{2}}(f)\neq \{0_{\M{3}}\}$ donc $\sqrt{2}$ 
    est valeur propre de $f$.
    \conc{$\sqrt{2}$ est valeur propre de $f$ et le sous-espace propre
          associé est :\\[.2cm] 
          $E_{\sqrt{2}}(f) = \Vect{\sqrt{2} \, A+A^2}$}
    \end{noliste}
          
    \newpage
          
    \begin{remark}%~
          On profite de ce premier calcul pour faire un point sur les
          objets étudiés :
          \begin{noliste}{$\stimes$}
            \setlength{\itemsep}{2mm}
          \item ${\cal E} = \Vect{I, A, A^2}$ est un ev de dimension
            $3$.\\[.2cm]
            Ainsi, tout élément $M \in {\cal E}$ s'écrit sous la
            forme $M = x \cdot I + y \cdot A + z \cdot A^2$.\\[.2cm]
            Ce qui revient à dire que $M$ a pour coordonnées
            $(x, y, z)$ dans la base $(I, A, A^2)$.\\[.2cm]
            Il ne faut pas confondre la notion de coordonnées de $M$
            avec la matrice représentative de $M$ dans la base $(I, A,
            A^2)$ : $X = \Mat_{(I,A, A^2)}(M) =
            \begin{smatrix}
              x \\
              y \\
              z
            \end{smatrix}
            \neq (x, y, z)
            $.

          \item $f : {\cal E} \to {\cal E}$ est un endomorphisme de
            ${\cal E}$.\\[.2cm]
            L'espace ${\cal E}$ étant de dimension $3$, la matrice
            représentative de $f$ dans la base $(I, A, A^2)$ est une
            matrice de $\M{3}$ : $F = \Mat_{(I,A, A^2)}(f) =
            \begin{smatrix}
              0 & 0 & 0 \\
              1 & 0 & 2 \\
              0 & 1 & 0
            \end{smatrix}
            \in \M{3}$.

          \item $E_\lambda(f) = \{ M \in {\cal E} \ | \ f(M) =
            \lambda M\}$ est un sev de ${\cal E}$ donc un ensemble
            dont les éléments sont des sont des matrices de ${\cal E}
            \subset \M{3}$.
            
          \item $E_\lambda(F) = \{ X \in \M{3,1} \ | \ FX = \lambda
            X\}$ est un sev de $\M{3,1}$ donc un ensemble dont les
            éléments sont des vecteurs colonnes de $\M{3,1}$.
          \end{noliste}
          Il faut donc bien comprendre que, de manière générale :
          $E_\lambda(f) \neq E_\lambda(F)$.\\
          En revanche, $\spc(f) = \spc(F)$ : $f$ a les mêmes valeurs
          propres que toute matrice qui le représente.
    \end{remark}~\\[-1cm]
    
    
    \begin{noliste}{$-$}
    \item Déterminons $E_{0}(f)=\kr(f-0 \cdot \id)=\kr(f)$.\\
    Soit $M\in\mathcal{E}$. On note $X=
    \begin{smatrix}
     x\\ y\\ z
    \end{smatrix}=\Mat_{(I,A,A^2)}(M)$.
    \[
     \begin{array}{rcl}
      M\in \kr(f) & \Longleftrightarrow & 
      f(M)=0_{\mathcal{E}}
      \\[.2cm]
      & \Longleftrightarrow & FX=0_{\M{3,1}}
      \\[.2cm]
      & \Longleftrightarrow &
      \begin{smatrix}
       0 & 0 & 0\\
       1 & 0 & 2\\
       0 & 1 & 0
      \end{smatrix}
      \begin{smatrix}
       x\\ y\\ z
      \end{smatrix}
      =\begin{smatrix}
        0\\ 0\\ 0
       \end{smatrix}
      \\[.8cm]
      & \Longleftrightarrow & \left\{
      \begin{array}{rrrrrcl}
       x &  &  & + & 2z & = & 0\\
       & & y &  &  & = & 0
      \end{array}
      \right.
      \\[.8cm]
      & \Longleftrightarrow & \left\{
      \begin{array}{rrrcl}
       x & & & = & -2z\\
       & & y & = & 0
      \end{array}
      \right.
     \end{array}
    \]
    
    \newpage
    
    On obtient donc :
    \[
     \begin{array}{rcl}
      E_{0}(f) & = & \left\{ M\in\mathcal{E} \ | \ f(M)= 
      0 \right\}
      \\[.2cm]
      & = & \{ x \cdot I+y \cdot A+z \cdot A^2 \ | \
       x = -2z \mbox{ et }
       y = 0
      \}
      \\[.4cm]
      & = & \left\{ (-2z)\cdot I + 0 \cdot A + z \cdot A^2 
      \ | \ z\in\R\right\}
      \\[.2cm]
      & = & \left\{ z(-2I + A^2) 
      \ | \ z\in\R\right\}
      \\[.2cm]
      & = & \Vect{-2I+A^2}
     \end{array}
    \]
    En particulier, $E_{0}(f)\neq \{0_{\M{3}}\}$ donc $0$ est 
    valeur propre de $f$.
    \conc{$0$ est valeur propre de $f$ et le sous-espace propre
          associé est :\\[.2cm] 
          $E_{0}(f) = \Vect{-2I+A^2}$}
    
    \item Déterminons $E_{-\sqrt{2}}(f)=\kr(f+\sqrt{2} \cdot \id)$.\\
    Soit $M\in\mathcal{E}$. On note $X=
    \begin{smatrix}
     x\\ y\\ z
    \end{smatrix}=\Mat_{(I,A,A^2)}(M)$.
    \[
     \begin{array}{rcl}
      M\in E_{-\sqrt{2}}(f) & \Longleftrightarrow & 
      (f+\sqrt{2} \, \id)(M)=0_{\mathcal{E}}
      \\[.2cm]
      & \Longleftrightarrow & (F+\sqrt{2} \, I)X=0_{\M{3,1}}
      \\[.2cm]
      & \Longleftrightarrow &
      \begin{smatrix}
       \sqrt{2} & 0 & 0\\
       1 & \sqrt{2} & 2\\
       0 & 1 & \sqrt{2}
      \end{smatrix}
      \begin{smatrix}
       x\\ y\\ z
      \end{smatrix}
      =\begin{smatrix}
        0\\ 0\\ 0
       \end{smatrix}
      \\[.8cm]
      & \Longleftrightarrow & \left\{
      \begin{array}{rrrrrcl}
       \sqrt{2} \, x & & & & & = & 0\\
       x & + & \sqrt{2} \, y & + & 2z & = & 0\\
       & & y & + & \sqrt{2} \, z & = & 0
      \end{array}
      \right.
      \\[.8cm]
      & \Longleftrightarrow & \left\{
      \begin{array}{rrrrrcl}
       x & & & & & = & 0\\
       &  & \sqrt{2} \, y & + & 2z & = & 0\\
       & & y & + & \sqrt{2} \, z & = & 0
      \end{array}
      \right.
      \\[.8cm]
      & \Longleftrightarrow & \left\{
      \begin{array}{rrrcl}
       x & & & = & 0\\
       & & y & = & -\sqrt{2} \, z
      \end{array}
      \right.
     \end{array}
    \]
    On obtient donc :
    \[
     \begin{array}{rcl}
      E_{-\sqrt{2}}(f) & = & \left\{ M\in\mathcal{E} \ | \ f(M)= 
      -\sqrt{2}M \right\}
      \\[.2cm]
      & = & \{ x \cdot I+y \cdot A+z \cdot A^2 \ | \
       x = 0 \mbox{ et }
       y = -\sqrt{2} \, z
      \}
      \\[.4cm]
      & = & \left\{ 0\cdot I - \sqrt{2} \, z \cdot A + z \cdot A^2 
      \ | \ z\in\R\right\}
      \\[.2cm]
      & = & \left\{ z(-\sqrt{2}\, A + A^2) 
      \ | \ z\in\R\right\}
      \\[.2cm]
      & = & \Vect{-\sqrt{2} \, A+A^2}
     \end{array}
    \]
    En particulier, $E_{-\sqrt{2}}(f)\neq \{0_{\M{3}}\}$ donc 
   $-\sqrt{2}$ est valeur propre de $f$.
    \conc{$-\sqrt{2}$ est valeur propre de $f$ et l'espace propre
          associé est : \\[.2cm]
          $E_{-\sqrt{2}}(f) = \Vect{-\sqrt{2} \, A+A^2}$}
  \end{noliste}
\end{noliste}

\newpage

\begin{remark}
        \begin{noliste}{$\sbullet$}
        \item On peut vérifier que les matrices obtenues sont bien 
        des vecteurs propres de $f$. Par exemple :
        \[
        f(\sqrt{2}\, A +A^2)= A(\sqrt{2} \, A + A^2)
         = \sqrt{2} \, A^2+A^3
         \]
         D'après la question \itbf{4.}, $A^3=2A$. Donc :
         \[
          f(\sqrt{2} \, A+A^2)=\sqrt{2} \, A^2 +2A = \sqrt{2}
          (\sqrt{2}\, A+A^2)
         \]
         Donc $\sqrt{2}\, A+A^2$ est bien vecteur propre 
        
        \item On aurait pu
          déterminer $E_{\sqrt{2}}(F)$, $E_{0}(F)$,
          $E_{-\sqrt{2}}(F)$.
          \[
          E_{\sqrt{2}}(F) = \Vect{
            \begin{smatrix}
              0 \\
              \sqrt{2} \\
              1
            \end{smatrix}
          } \quad E_{0}(F) = \Vect{
            \begin{smatrix}
              -2 \\
              0 \\
              1
            \end{smatrix}
          } \quad E_{\sqrt{2}}(F) = \Vect{
            \begin{smatrix}
              0 \\
              -\sqrt{2} \\
              1
            \end{smatrix}
          }
          \]

        \item Insistons une nouvelle fois sur la différence
          coordonnées / matrice représentative.
          \begin{noliste}{$\stimes$}
          \item $M = 0 \cdot I + \sqrt{2} \cdot A + 1 \cdot A^2$ a
            pour coordonnées $(0, \sqrt{2}, 1)$ dans la base $(I, A,
            A^2)$.

          \item la matrice représentative de $M$ dans cette base est :
            \[X = \Mat_{(I, A, A^2)} (M) =
            \begin{smatrix}
              0 \\
              \sqrt{2} \\
              1
            \end{smatrix}
            \]
          \end{noliste}
        \end{noliste}
      \end{remark}~\\[-1.4cm]
\end{proof}
\end{noliste}

\item L'endomorphisme $f$ est-il bijectif ? diagonalisable ?

\begin{proof}~
 \begin{noliste}{$\sbullet$}
  \item $0$ est une valeur propre de $f$, donc $\kr(f) \neq
      \{0_{\M{3}}\}$.\\
      On en conclut que l'application linéaire $f$ n'est pas
      injective.
  \conc{$f$ n'est pas bijectif.}
  \item $f$ est un endomorphisme de $\mathcal{E}$ et on sait que :
  \begin{noliste}{$\stimes$}
  \item $\dim(\mathcal{E})=3$ d'après la question \itbf{5.},
  \item $f$ possède $3$ valeurs propres {\bf distinctes}.
  \end{noliste}
  \conc{$f$ est diagonalisable.}~\\[-1.2cm]
 \end{noliste}
 
 \begin{remark}%~
      Le premier point de cette question amène à plusieurs remarques.
      \begin{noliste}{$\sbullet$}
      \item Tout d'abord, rappelons qu'une application quelconque $f$ 
(pas
        forcément linéaire) est bijective si elle est à la fois
        injective et surjective. Telle que la réponse est rédigée, on
        utilise ici le fait que : 
        \[
        \text{$f$ bijective} \ \Rightarrow \ \text{$f$ injective}
        \]
        et donc, par contraposée : 
        \[
        \text{$f$ non injective} \ \Rightarrow \ \text{$f$ non
          bijective}
        \]
        
      \item Dans le cas où $f$ est une application linéaire, on sait
        de plus que : 
        \[
        \text{$f$ injective} \ \Leftrightarrow \ \kr(f) = \{0\}
        \]
        La combinaison de ces deux points permet de conclure la
        première partie de la question.
    \end{noliste}
 \end{remark}
 
 
 \newpage
        
        
 \begin{remark}
   \begin{noliste}{$\sbullet$}
      \item Dans le cas où $f : {\cal E} \to {\cal E}$ est une
        application linéaire et que {\bf ${\cal E}$ est de dimension
          finie} :
        \[
        \text{$f$ bijective} \ \Leftrightarrow \ \text{$f$ injective}
        \ \Leftrightarrow \ \kr(f) = \{0\} \ \Leftrightarrow \
        \text{$0$ n'est pas valeur propre de $f$}
        \]
        L'hypothèse de dimension finie est primordiale pour la
        première équivalence (les autres sont toujours vraies).
        Autrement dit, il existe des applications linéaires injectives
        et non bijectives si ${\cal E}$ est de dimension infinie (et
        seulement dans ce cas).

      \item Cela signifie que si $f$ est un endomorphisme de ${\cal
          E}$, de dimension finie, qui n'admet pas $0$ comme valeur
        propre, on doit rédiger comme suit :\\[.2cm]
        $0$ n'est pas valeur propre de $f$, donc $\kr(f) = \{0\}$.\\
        On en conclut que l'application linéaire $f$ est injective.\\
        Ainsi, comme {\bf ${\cal E}$ est de dimension finie}, $f$ est
        bijective.\\
        {\it (l'oubli de cette hypothèse risque d'être sanctionné)}
        
      \item Par contre, on peut utiliser, sans citer d'hypothèse,
      l'équivalence :
        \[
        \text{La matrice $A$ est inversible} \ \Leftrightarrow \ \
        \text{$0$ n'est pas valeur propre de $A$}
        \]
        En effet, la notion de matrice sous-entend que l'on se trouve
        en dimension finie.
      \end{noliste}
    \end{remark}~\\[-1.4cm]
\end{proof}

\item Déterminer une base de $\im(f)$ et une base de 
$\kr(f)$.

\begin{proof}~
  \begin{noliste}{$\sbullet$}
   \item \dashuline{Cas de $\kr(f)$} :
   On sait que $\kr(f)=\Vect{-2I+A^2}$ (question \itbf{9.}).\\[.2cm]
   La famille $(-2I+A^2)$ :
   \begin{noliste}{$\stimes$}
    \item engendre $\kr(f)$ d'après la question \itbf{9.},
    \item est libre dans $\mathcal{E}$ car elle est constituée d'une 
    unique matrice non nulle.
   \end{noliste}
   \conc{$(-2I+A^2)$ est une base de $\kr(f)$.}
   
   \item \dashuline{Cas de $\im(f)$} :\\[.1cm]
   On sait que $(I,A,A^2)$ est une base de $\mathcal{E}$. Donc, par 
   caractérisation de l'image d'une application linéaire :
 \[
  \im(f)=\Vect{f(I),f(A),f(A^2)} = \Vect{A,A^2,2A}=\Vect{A,A^2}
 \]
 Donc la famille $(A,A^2)$ :
 \begin{noliste}{$\stimes$}
  \item engendre $\im(f)$,
  \item est libre car elle est constituée de $2$ matrices non 
  proportionnelles.
 \end{noliste}
 \conc{$(A,A^2)$ est une base de $\im(f)$.}
  \end{noliste}
  
  \begin{remark}
   On remarque que :
   \begin{noliste}{$\stimes$}
    \item comme $(-2I+A^2)$ est une base de $\kr(f)$, alors :
    \[
     \dim(\kr(f)) = \Card((-2I+A^2))=1
    \]
    \item comme $(A,A^2)$ est une base de $\im(f)$, alors :
    \[
     \rg(g)=\dim(\im(f)) = \Card((A,A^2))=2
    \]
   \end{noliste}
   On retrouve donc bien le théorème du rang :
   \[
    \dim(\kr(f)) + \rg(f) = 3 = \dim(\mathcal{E})
   \]
  \end{remark}~\\[-1.4cm]
\end{proof}

\item 
\begin{noliste}{a)}
\item Résoudre l'équation $f(M)=I+A^2$, d'inconnue $M\in \mathcal{E}$.

\begin{proof}~\\
 Soit $M\in\mathcal{E}$.\\
      Il existe donc $(a, b, c) \in \R^3$ tel que $M = a \cdot I + b
      \cdot A + c \cdot A^2$. Alors :
      \[
      \begin{array}{cl@{\qquad}>{\it}R{4.2cm}}
        & f(M) = I+A^2 \\[.2cm]
        \Leftrightarrow & f(a \cdot I + b \cdot A + c
        \cdot A^2) = I + A^2 
        \\[.2cm]
        \Leftrightarrow & a \cdot f(I) + b \cdot f(A)+ c \cdot
        f(A^2) = I + A^2
        \\[.2cm]
        \Leftrightarrow & a \cdot A + b \cdot A^2 + 2c \cdot A = I+A^2
        \\[.2cm]
        \Leftrightarrow & I - (a+2c) \cdot A + (1-b) \cdot A^2 = 0
        \\[.2cm]
        \Leftrightarrow & 
        \left\{
          \begin{array}{rcl}
            1 & = & 0\\
            -(a+2c) & = & 0\\
            1-b & = & 0
          \end{array}
        \right.
        & (car la famille \\ $(I, A, A^2)$ est libre)
      \end{array}
      \]
      \conc{Ce système n'admettant pas de solution, l'équation
        $f(M)=I+A^2$ n'admet pas de solution.}

        
      \begin{remark}%~
        On pouvait rédiger différemment.
        \begin{noliste}{$\sbullet$}
        \item Comme $f(M) \in \im(f)$, si l'équation $f(M) = I + A^2$
          est vérifiée alors : $I + A^2 \in \im(f)$.\\
          De plus : $A^2 \in \Vect{A,A^2} = \im(f)$.\\
          On en déduit, par soustraction : $I = (I + A^2) - A^2 \in
          \im(f)$.\\
          Or $(I,A,A^2)$ est une famille libre et $I \neq 0$ donc $I
          \in \Vect{A,A^2}$ est impossible.
        \end{noliste}
      \end{remark}~\\[-1.4cm]
\end{proof}


\newpage


\item Résoudre l'équation $f(N)=A+A^2$, d'inconnue $N\in \mathcal{E}$.

\begin{proof}~\\
 Soit $N \in {\cal E}$.\\
      Il existe donc $(a, b, c) \in \R^3$ tel que $N = a \cdot I + b
      \cdot A + c \cdot A^2$. Alors :
      \[
      \begin{array}{cl@{\qquad}>{\it}R{4.2cm}}
        & f(N) = A + A^2 
        \\[.2cm]
        \Leftrightarrow & f(a \cdot I + b \cdot A + c
        \cdot A^2) = A + A^2
        \\[.2cm]
        \Leftrightarrow & a \cdot f(I) + b \cdot f(A) + c \cdot
        f(A^2)= A + A^2
        \\[.2cm]
        \Leftrightarrow & a \cdot A + b \cdot A^2 + 2c \cdot A = A +
        A^2
        \\[.2cm]
        \Leftrightarrow & (a+2c-1) \cdot A + (b-1) \cdot A^2 = 0
        \\[.2cm]
        \Leftrightarrow & 
        \left\{
          \begin{array}{rcrcrcc}
            a & & & + & 2c & = & 1\\
            & & b & & & = & 1
          \end{array}
        \right.
        & (car $(A, A^2)$ sous-famille de $(I, A, A^2)$ est libre)
        \nl 
        \nl[-.2cm]
        \Leftrightarrow &
        \left\{
          \begin{array}{rcrcl}
            a & & & = & 1-2c\\
            & & b & = & 1
          \end{array}
        \right.
      \end{array}
      \]
      \conc{L'ensemble des solutions de l'équation $f(N)=A+A^2$ est 
	:\\[.2cm]
        $\{(1-2c) \cdot I + A + c \cdot A^2 \ | \
        c\in\R\}$.}~\\[-.7cm]
\end{proof}

\end{noliste}
\end{noliste}


\newpage


\section*{EXERCICE II}

\noindent
On considère l'application $f: [0, +\infty[\to \R$ définie, pour tout
$t$ de $[0, +\infty[$, par :
\[
f(t) =
\left\{
  \begin{array}{cR{1.4cm}}
    t^2-t\ln(t) & si $t\neq 0$ \nl
    0 & si $t = 0$
  \end{array}
\right.
\]
On admet : $0, \ 69 < \ln(2)<0, \ 70$.

\subsection*{PARTIE I :  Étude de la fonction $f$}

\begin{noliste}{1.}
  \setlength{\itemsep}{2mm}
\item Montrer que $f$ est continue sur $[0, +\infty[$.
  
  \begin{proof}~
    \begin{noliste}{$\sbullet$}
    \item La fonction $f$ est continue sur $]0,+\infty[$ car c'est la
      somme $f = f_1 + f_2$ où :
      \begin{noliste}{$\stimes$}
      \item $f_1 : t \mapsto t \ln(t)$ continue sur $]0,+\infty[$
        comme produit de fonctions continues sur $]0,+\infty[$,
      \item $f_2 : t \mapsto t^2$ continue sur $]0,+\infty[$.
      \end{noliste}
      
    \item Par ailleurs : $\dlim{t\to 0} t\ln(t)=0$. De plus, 
      $\dlim{t\to 0} t^2=0$.\\
      Donc $\dlim{t\to 0} f(t)=0=f(0)$.\\
      Donc $f$ est continue en $0$.
    \end{noliste}
    \conc{La fonction $f$ est continue sur $[0,+\infty[$.}~\\[-1cm]
  \end{proof}


\item Justifier que $f$ est de classe $\Cont{2}$ sur $]0,+\infty[$ et
  calculer, pour tout $t$ de $]0,+\infty[$, $f'(t)$ et $f''(t)$.

  \begin{proof}~\\
    La fonction $f$ est de classe $\Cont{2}$ sur $]0,+\infty[$ car
    c'est la somme $f = f_1 + f_2$ où :
    \begin{noliste}{$\stimes$}
    \item $f_1 : t\mapsto t\ln(t)$ de classe $\Cont{2}$ sur
      $]0,+\infty[$ comme produit de fonctions de classe $\Cont{2}$
      sur $]0,+\infty[$,
    \item $f_2 : t \mapsto t^2$ de classe $\Cont{2}$ sur
      $]0,+\infty[$.
    \end{noliste}
    \conc{La fonction $f$ est de classe $\Cont{2}$ sur
      $]0,+\infty[$.}%
    \conc{$\forall t \in \ ]0,+\infty[$, $f'(t)=2t-\left(1\times
        \ln(t) + \bcancel{t}\times \dfrac{1}{\bcancel{t}}\right) =
      2t-\ln(t)-1$}%
    \conc{$\forall t \in \ ]0,+\infty[$, $f''(t) = 2 - \dfrac{1}{t} =
      \dfrac{2t - 1}{t}$}~\\[-1cm]
  \end{proof}

\item Dresser le tableau des variations de $f$. On précisera la limite
  de $f$ en $+\infty$.

\begin{proof}~
  \begin{noliste}{$\sbullet$}
  \item Soit $t\in\ ]0,+\infty[$. Comme $t > 0$, $f''(t) = \dfrac{2t -
      1}{t}$ est du signe de $2t -1$. Or :
    \[
    2t - 1 \geq 0 \ \Leftrightarrow \ 2t \geq 1 \ \Leftrightarrow \ t
    \geq \dfrac{1}{2}
    \]
  
  \item On obtient le tableau de variations suivant :
  
  \begin{center}
      \begin{tikzpicture}[scale=0.8, transform shape]
        \tkzTabInit[lgt=4,espcl=3] %
        { %
        $t$ /1, %
        Signe de $f''(t)$ /1, %
        Variations de $f'$ /2
        } %
        {$0$, $\frac{1}{2}$, $+\infty$} %
        \tkzTabLine{ , - , z , + , } % 
        \tkzTabVar{+/$+\infty$, -/$\ln(2)$ , +/$+\infty$} %
      \end{tikzpicture}
     \end{center}


     \newpage


     \noindent
  Détaillons les éléments de ce tableau.
  \begin{noliste}{$-$}
    \item $f'\left(\frac{1}{2}\right) = \bcancel{2}\times 
    \dfrac{1}{\bcancel{2}} - \ln\left(\frac{1}{2}\right) -1
    =\bcancel{1} +\ln(2)-\bcancel{1} = \ln(2)$
    
    \item $f'(t)=2t-\ln(t)-1 \eq{t}{0} -\ln(t)$. 
    Donc : $\dlim{t\to 0} f'(t)=+\infty$.
    
  \item $f'(t)=2t-\ln(t)-1 = 2t\left(1-\dfrac{\ln(t)}{2t}\right)
    \eq{t}{+\infty} 2t$, car $1-\dfrac{\ln(t)}{2t} \tendd{t}{+\infty}
    1$ par croissances comparées.\\
    Donc : $\dlim{t\to +\infty} f'(t)=+\infty$.
  \end{noliste}
  
\item Or $\ln(2)>0$. Donc : $\forall t \in \ ]0,+\infty[$,
  $f'(t)>0$.\\[.2cm]
  De plus : $f(t)=t^2-t\ln(t) =t^2\left(1-\dfrac{\ln(t)}{t}\right)
  \eq{t}{+\infty} t^2$ par croissances comparées.\\
  Donc : $\dlim{t\to+\infty} f(t)=+\infty$.\\
  On obtient donc le tableau de variations suivant pour $f$ :
  
  \begin{center}
      \begin{tikzpicture}[scale=0.8, transform shape]
        \tkzTabInit[lgt=4,espcl=3] %
        { %
        $t$ /1, %
        Signe de $f'(t)$ /1, %
        Variations de $f$ /2
        } %
        {$0$, $+\infty$} %
        \tkzTabLine{ , + , } % 
        \tkzTabVar{-/$0$, +/$+\infty$} %
      \end{tikzpicture}
     \end{center}~\\[-1.6cm]
 \end{noliste}
\end{proof}

\item On note $C$ la courbe représentative de $f$ dans un repère 
orthonormal $(0, \vec{i} ,\vec{j})$.
\begin{noliste}{a)}
\item Montrer que $C$ admet une tangente en $0$ et préciser celle-ci.

\begin{proof}~
 \begin{noliste}{$\sbullet$}
  \item Soit $t\in \ ]0,+\infty[$. Calculons le taux d'accroissement
  de $f$ en $0$.
  \[
   \dfrac{f(t)-f(0)}{t-0} = \dfrac{t^2-t\ln(t)}{t} = \dfrac{\bcancel{t}
   (t-\ln(t))}{\bcancel{t}} = t-\ln(t)
  \]
  Donc : $\dlim{t\to 0} \dfrac{f(t)-f(0)}{t-0} = +\infty$.%
  \conc{Donc la courbe $C$ admet une demi-tangente verticale en
    $0$.}~\\[-1.4cm]
 \end{noliste}
\end{proof}


\item Montrer que $C$ admet un point d'inflexion et un seul, noté $I$, 
et préciser les coordonnées de $I$.

\begin{proof}~\\
  D'après la question précédente, $f''$ s'annule en changeant de signe
  uniquement en $\dfrac{1}{2}$.\\
  De plus : $f\left(\dfrac{1}{2}\right) = \left(\dfrac{1}{2}\right)^2
  - \dfrac{1}{2}\ln\left(\dfrac{1}{2}\right) =
  \dfrac{1}{4}+\dfrac{1}{2} \ln(2)$.%
  \conc{La fonction $f$ admet un unique point d'inflexion $I$ en
    $\left( \dfrac{1}{2}, \dfrac{1}{4}+\dfrac{1}{2}\ln(2)\right)$.}~\\[-1cm]
\end{proof}


\newpage


\item Tracer l'allure de $C$.

  \begin{proof}~\\
    La courbe $C$ admet pour tangente au point d'abscisse
    $\frac{1}{2}$ la droite d'équation :
    \[
    \begin{array}{rcl}
      y & = & f\left( \frac{1}{2} \right) + f'\left( \frac{1}{2} \right) 
      \left(x - \frac{1}{2} \right)
      \\[.4cm]
      & = & \left( \frac{1}{4} + \frac{1}{2} \ \ln(2) \right) + \ln(2)
      \left(x - \frac{1}{2} \right)  
      \\[.4cm]
      & = & \frac{1}{4} + \ln(2) x
    \end{array}
    \]
    \begin{center}
      %% French babel fout la merde : ne pas oublier shorthandoff
      \shorthandoff{;}
      \begin{tikzpicture}[xscale=1.5, yscale = 1.5, scale = .8, transform
        shape, %
        declare function = { %
          g(\x) = \x^2 - \x * ln(\x); %
          h(\x) = 1/4 + \x * ln(2); %
        },] %
        \pgfplotsset{every tick label/.append style={font=\tiny}}
        \begin{axis}[%
          xmin = -0.1, %
          xmax = 2, %
          % ymin = -0.7, %
          % ymax = 3, %
          ymin = -0.1, %
          ymax = 2, %
          no markers, %
          axis x line = center, %
          axis y line = center, %
          grid = both, %
          extra x ticks = {0.5}, %
          % extra x tick labels = {\color{red} $x_I$}, %
          ] %
          \tikzset{cross/.style={cross out, draw=black, fill=none,
              minimum size=2*(#1-\pgflinewidth), inner sep=0pt, outer
              sep=0pt}, cross/.default={2pt}} %
          \xdef\epsi{0.25}; %
          \addplot[samples=150, domain=0:2, blue, thick] {g(x)}; %
          \addplot[<->, samples=2, domain={1/2-\epsi}:{1/2+\epsi},
          red, very thick]{h(x)}; %
          \addplot[dashed] coordinates {(0.5, -0.02) (0.5, {g(0.5)})}
          node[cross = 1.5, rotate = 45, black] {}; %{$I$}; %
          \addplot[->, red, very thick] coordinates {(0, 0) (0,
            0.25)}; %
          \addplot[dashed] coordinates {(0.5, -0.02) (0.5, {g(0.5)})}
          node[above] {$I$}; %
          % \draw (0.5, {g(0.5)}) node[above] {$I$};
          % \draw[dashed, red, thick] (0.5,) -- (0.5,{g(0.5)}) 
          % node[above] {$I$}; %
        \end{axis}
      \end{tikzpicture}
    \end{center}~\\[-1.2cm]
  \end{proof}
  
\end{noliste}

\item Montrer que l'équation $f(t)=1$, d'inconnue $t\in[0,+\infty[$, 
admet une solution et une seule et que celle-ci est égale à $1$.

\begin{proof}~%
  \begin{noliste}{$\sbullet$}
  \item La fonction $f$ est :
    \begin{noliste}{$\stimes$}
    \item continue sur $[0,+\infty[$,
    \item strictement croissante sur $[0,+\infty[$.
    \end{noliste}
    Ainsi, $f$ réalise une bijection de $[0,+\infty[$ sur
    $f([0,+\infty[)$. De plus :
    \[
    f([0,+\infty[) = \left[f(0), \dlim{t\to+\infty} f(t)\right[
    =[0,+\infty[
    \]
    Or $1\in [0, +\infty[$, donc l'équation $f(t)=1$ admet une unique
    solution $\alpha$ sur $[0,+\infty[$.
    
  \item De plus $f(1)=1^2-1\times \ln(1)=1$. Donc $\alpha=1$.
  \end{noliste}
  \conc{L'équation $f(t)=1$ admet $1$ comme unique solution sur
    $[0,+\infty[$.}
  \begin{remark}
    \begin{noliste}{$\sbullet$}
    \item Il faut tout de suite repérer cette question comme une
      application du théorème de la bijection. Et s'empresser d'y
      répondre !
      
    \item Attention de ne pas seulement vérifier que $1$ est solution
      de l'équation $f(t)=1$.\\
      Il faut bien montrer ici que c'est {\bf la seule} solution sur
      tout l'intervalle $[0,+\infty[$.
    \end{noliste}
  \end{remark}~\\[-1.4cm]
\end{proof}

\end{noliste}


\newpage


\subsection*{PARTIE II : Étude d'une fonction $F$ de deux 
variables réelles}

\noindent
On considère l'application $F: \ ]0,+\infty[^2 \to \R$ de 
classe $\Cont{2}$, définie, pour tout $(x,y)$ de $]0,+\infty[^2$ , par 
: 
\[
F(x,y)=x\ln(y)-y\ln(x)
\]
\begin{noliste}{1.}
  \setlength{\itemsep}{2mm} %
  \setcounter{enumi}{5}
\item Calculer les dérivées partielles premières de $F$ en tout
  $(x,y)$ de $]0,+\infty[^2$.

  \begin{proof}~
    \begin{noliste}{$\sbullet$}
    \item $F$ est de classe $\Cont{2}$ sur $]0,+\infty[^2$, donc elle
      est de classe $\Cont{1}$ sur $]0,+\infty[^2$.
      
    \item Soit $(x,y)\in \ ]0,+\infty[^2$.
      \[
      \dfn{F}{1}(x,y)=\ln(y)-y\times \dfrac{1}{x} = \ln(y) - \dfrac{y}{x}
      \]
      \[
      \dfn{F}{2}(x,y)=x\times \dfrac{1}{y} -\ln(x) = \dfrac{x}{y} - \ln(x)
      \]
    \end{noliste}
    \conc{$\forall (x,y)\in \ ]0,+\infty[^2$, $\dfn{F}{1}(x,y) = \ln(y) - 
      \dfrac{y}{x}$ \ et \ $\dfn{F}{2}(x,y) = \dfrac{x}{y} -
      \ln(x)$}~\\[-1cm] 
  \end{proof}
  

\item
\begin{noliste}{a)}
\item Soit $(x,y)\in \ ]0,+\infty[^2$. Montrer que $(x,y)$ est un point 
critique de $F$ si et seulement si :
\[
x > 1, \quad y=\dfrac{x}{\ln(x)} \quad \text{et} \quad 
f\left(\ln(x)\right)=1
\]

\begin{proof}~\\
 Le couple $(x,y)$ est un point critique de $F$ si et seulement si :
 \[
   \nabla(F)(x,y) = 
   \begin{smatrix}
    0\\
    0
   \end{smatrix}
   \ \Leftrightarrow \
   \left\{
   \begin{array}{rcl}
    \dfn{F}{1}(x,y) &=& 0\\[.2cm]
    \dfn{F}{2}(x,y) &=& 0
   \end{array}
   \right.
   \ \Leftrightarrow \
   \left\{
   \begin{array}{rcl}
    \ln(y) &=& \dfrac{y}{x}\\[.4cm]
    \dfrac{x}{y} &=& \ln(x)
   \end{array}
   \right.
 \]
 \begin{noliste}{$\sbullet$}
 \item On sait que $(x,y)\in \ ]0,+\infty[^2$, c'est-à-dire $x>0$ et
   $y>0$. On en déduit : $\dfrac{x}{y} > 0$.\\
   Ainsi, si $(x,y)$ est un point critique de $F$,
   $\ln(x)=\dfrac{x}{y} >0$, et par stricte croissance de la
   fonction exponentielle, $x>\ee^0=1$.\\
   On a donc déjà $x>1$.
  
  \item On reprend alors la résolution du système.
  \[
    \left\{
    \begin{array}{rcl}
     \ln(y) &=& \dfrac{y}{x}\\[.4cm]
     \dfrac{x}{y} &=& \ln(x)
    \end{array}
    \right.
    \ \Leftrightarrow \
    \left\{
    \begin{array}{rcl}
     x \, \ln(y) &=& y\\[.2cm]
     \dfrac{x}{\ln(x)} &=& y 
    \end{array}
    \right.
    \ \Leftrightarrow \
    \left\{
    \begin{array}{rcl}
     x \, \ln\left(\dfrac{x}{\ln(x)}\right) &=& 
     \dfrac{x}{\ln(x)}\\[.4cm]
     \dfrac{x}{\ln(x)} &=& y
    \end{array}
    \right.
  \]
  La première équation devient alors :
  \[
   \begin{array}{rcl@{\quad}>{\it}R{5cm}}
    x \, \ln\left(\dfrac{x}{\ln(x)}\right) = \dfrac{x}{\ln(x)}
    & \Leftrightarrow &
    x \, \left(\ln(x) - \ln(\ln(x))\right) = \dfrac{x}{\ln(x)}
    \\[.4cm]
    & \Leftrightarrow & 
    \bcancel{x} \, \ln(x) \left(\ln(x) - \ln(\ln(x))\right) = 
    \bcancel{x}
    & (car $x \neq 0$)
    \nl
    \nl[-.2cm]
    & \Leftrightarrow &
    \left(\ln(x)\right)^2 - \ln(x) \, \ln(\ln(x)) = 1
    \\[.2cm]
    & \Leftrightarrow &
    f(\ln(x))=1
   \end{array}
  \]

  
  \newpage
  

  \noindent
  On en déduit alors que, si $(x,y)$ est un point critique de $F$,
  alors :
  \[
   \left\{
   \begin{array}{rcl}
    x & > & 1\\[.2cm]
    f(\ln(x)) &=& 1\\[.2cm]
    y &=& \dfrac{x}{\ln(x)}
   \end{array}
   \right.
  \]
  
  \item Réciproquement, si $(x,y)$ vérifie ces trois conditions, alors 
  $\nabla(F)(x,y)=
  \begin{smatrix}
   0\\
   0
 \end{smatrix}$.\\
 D'où $(x,y)$ est un point critique de $F$.
 \end{noliste}
 \conc{Finalement $(x,y)$ est un point critique de $F$ si et 
 seulement si :
 $\left\{
 \begin{array}{l}
    x > 1\\[.2cm]
    f(\ln(x)) = 1\\[.2cm]
    y = \dfrac{x}{\ln(x)}
   \end{array}
 \right.$}~\\[-1cm]
\end{proof}


\item Établir que $F$ admet un point critique et un seul et qu'il
  s'agit de $(\ee,\ee)$.

  \begin{proof}~%
    \begin{noliste}{$\sbullet$}
    \item Soit $(x,y)\in \ ]0,+\infty[^2$.  D'après la question
      précédente, si $(x,y)$ est un point critique de $F$, alors, en
      particulier $f(\ln(x))=1$.
  
    \item D'après la question \itbf{5.}, l'équation $f(t)=1$ admet une
      unique solution sur $]0,+\infty[$ : $\alpha =1$.\\
      Donc $\ln(x)=1$ et $x = \ee^1 = \ee$.
  
    \item On obtient alors : $y=\dfrac{\ee}{\ln(\ee)}=\ee$. D'où
      $(x,y)=(\ee,\ee)$.
  
    \item Réciproquement :
      \[
      \dfn{F}{1}(\ee,\ee)=\ln(\ee)-\dfrac{\ee}{\ee}=0 \quad 
      \mbox{ et } \quad \dfn{F}{2}(\ee,\ee) = \ln(\ee) - 
      \dfrac{\ee}{\ee} = 0
      \]
      Donc $(\ee,\ee)$ est un point critique de $F$.
    \end{noliste}
    \conc{Finalement, la fonction $F$ admet $(\ee,\ee)$ pour unique point 
      critique.}~\\[-1cm]
  \end{proof}

\end{noliste}

\item La fonction $F$ admet-elle un extremum local en $(\ee,\ee)$?

  \begin{proof}~\\
    Pour savoir si $(\ee,\ee)$ est un extremum local pour $F$, on
    détermine les valeurs propres de la matrice hessienne de $F$ en
    $(\ee,\ee)$.
    \begin{noliste}{$\sbullet$}
    \item La fonction $F$ est de classe $\Cont{2}$ sur
      $]0,+\infty[^2$.\\
      Elle admet donc des dérivées partielles d'ordre $2$ sur cet
      ouvert.
  
    \item Soit $(x,y)\in \ ]0,+\infty[^2$.
      \[
      \nabla^2(F)(x,y) =
      \begin{smatrix}
        \ddfn{F}{1,1}(x,y) & \ddfn{F}{1,2}(x,y)
        \\[.4cm]
        \ddfn{F}{2,1}(x,y) & \ddfn{F}{2,2}(x,y)
      \end{smatrix}
      =
      \begin{smatrix}
        \dfrac{y}{x^2} & \dfrac{1}{y}-\dfrac{1}{x}
        \\[.4cm]
        \dfrac{1}{y}-\dfrac{1}{x} & -\dfrac{x}{y^2}
      \end{smatrix}
      \]
      

      \newpage

      
      \noindent
      Donc on obtient :
      \[
      \nabla^2(F)(\ee,\ee) =
      \begin{smatrix}
        \dfrac{\ee}{\ee^2} & \dfrac{1}{\ee} - \dfrac{1}{\ee}
        \\[.4cm]
        \dfrac{1}{\ee} - \dfrac{1}{\ee} & -\dfrac{\ee}{\ee^2}
      \end{smatrix}
      =
      \begin{smatrix}
        \dfrac{1}{\ee} & 0
        \\[.4cm]
        0 & - \dfrac{1}{\ee}
      \end{smatrix}
      \]
      
    \item La matrice $\nabla^2(F)(\ee,\ee)$ est diagonale, donc ses
      valeurs propres sont ses coefficients diagonaux.\\
      D'où $\spc(\nabla^2(F)(\ee,\ee))=\{\ee^{-1}, -\ee^{-1}\}$.
    \end{noliste}
    \conc{$\nabla^2(F)(\ee,\ee)$ admet deux valeurs propres de signe
      contraire,\\
      donc $(\ee,\ee)$ n'est pas un extremum local pour $F$ (c'est un
      point selle).}
    \begin{remark}
      On rappelle qu'on utilise ici le théorème suivant :\\
      Soit $f$ une fonction de classe $\Cont 2$ sur un {\bf ouvert}
      $U$ et soit $(x_0,y_0)$ un {\bf point critique} de $f$.\\[-.4cm]
      \begin{noliste}{$\sbullet$}
      \item Si les valeurs propres de la matrice hessienne 
        $\nabla^2(f)(x_0,y_0)$ sont strictement 
        positives, alors $f$ admet un minimum local en 
        $(x_0,y_0)$.\\[-.4cm]
        
      \item Si les valeurs propres de la matrice hessienne
        $\nabla^2(f)(x_0,y_0)$ sont strictement négatives, alors $f$
        admet un maximum local en
        $(x_0,y_0)$.\\[-.4cm]
    
      \item Si les valeurs propres de $\nabla^2(f)(x_0,y_0)$ sont non
        nulles et de signes opposés, alors $f$ n'admet pas d'extremum
        local en $(x_0,y_0)$. On parle de {\it point col} ou {\it
          point selle}.\\[-.4cm]
        
      \item Si $0$ est valeur propre de $\nabla^2(f)(x_0,y_0)$, alors 
        on ne peut rien conclure a priori.
      \end{noliste}
    \end{remark}~\\[-1.4cm]
  \end{proof}
  
\end{noliste}

\subsection*{PARTIE III : Étude d'une suite récurrente}

\noindent
On considère la suite $(u_n)_{n\in \N}$ définie par : $u_0 =
\dfrac{1}{2}$ \quad et \quad $\forall n\in \N$, $u_{n+1} = f(u_n)$.

\begin{noliste}{1.}
\setlength{\itemsep}{2mm}
\setcounter{enumi}{8}
\item Montrer : $\forall n\in \N$, $u_n \in \left[\frac{1}{2},
    1\right]$.

  \begin{proof}~\\
    Montrons par récurrence : $\forall n\in\N$, $\PP{n}$, \quad où
    \quad $\PP{n}$ : $u_n$ existe et $u_n\in
    \left[\frac{1}{2},1\right]$.
    \begin{noliste}{\fitem}
    \item {\bf Initialisation} : \\
      $u_0=\dfrac{1}{2} \in \left[\frac{1}{2}, 1\right]$. D'où
      $\PP{0}$.
      
    \item {\bf Hérédité} : soit $n\in\N$.\\
      Supposons $\PP{n}$ et démontrons $\PP{n+1}$ (c'est-à-dire
      $u_{n+1}$ existe et $u_{n+1}\in \left[\frac{1}{2},1\right]$).
 \begin{noliste}{$\sbullet$}
 \item Par hypothèse de récurrence, $u_n$ existe et $u_n\in \left[
 \frac{1}{2},1\right]$.\\
 En particulier $u_n\geq 0$. Donc $f(u_n)$ est bien défini. 
 Ainsi $u_{n+1}=f(u_n)$ existe.
 
 \item On sait que : $\dfrac{1}{2} \leq u_n \leq 1$.\\
 Or, d'après la question \itbf{3.}, la fonction $f$ est croissante
 sur $\R_+$. 
 Donc :
 \[
  f\left(\dfrac{1}{2}\right) \leq f(u_n) \leq f(1) \ 
  \Leftrightarrow \ 
  \dfrac{1}{4} + \dfrac{1}{2}\, \ln(2) \leq u_{n+1} \leq 1
 \]
 On a donc déjà : $u_{n+1} \leq 1$.
 
 \item D'après l'énoncé, on sait que : $\ln(2)>0,69$. Donc :
 \[
  \dfrac{1}{4} + \dfrac{1}{2} \, \ln(2) \ > \ \dfrac{1}{4} + 
  \dfrac{1}{2} \, 0,69 \ > \ \dfrac{1}{4} + 0,34 = 0,25+0,34 = 0,59
 \]
 Donc : $u_{n+1} \geq 0.59 > \dfrac{1}{2}$.
 \end{noliste}
 
 \newpage
 
  Finalement $u_{n+1}\in \left[\frac{1}{2},1\right]$. D'où $\PP{n+1}$.
 \end{noliste}
 \conc{Par principe de récurrence, $(u_n)$ est bien définie et :
 $\forall n\in\N$, $u_n\in\left[\frac{1}{2},1\right]$.}~\\[-1cm]
\end{proof}

\item Montrer que la suite $(u_n)_{n\in \N}$ est croissante.

  \begin{proof}~\\
    Montrons par récurrence que : $\forall n\in\N$, $\PP{n}$, \quad où
    \quad $\PP{n}$ : $u_n\leq u_{n+1}$.
    \begin{noliste}{\fitem}
    \item {\bf Initialisation} : \\
      D'après la question 
      précédente : $u_1\geq \dfrac{1}{2} =u_0$.\\
      D'où $\PP{0}$.
      
    \item {\bf Hérédité} : soit $n\in\N$.\\
      Supposons $\PP{n}$ et démontrons $\PP{n+1}$ (c'est-à-dire
      $u_{n+1} \leq u_{n+2}$).\\
      Par hypothèse de récurrence, $u_n \leq u_{n+1}$.\\
      Or la fonction $f$ est croissante sur $\R_+$. Donc :
      \[
      \begin{array}{ccc}
        f(u_n) & \leq & f(u_{n+1})
        \\[.2cm]
        \shortparallel & & \shortparallel
        \\[.2cm]
        u_{n+1} & & u_{n+2}
      \end{array}
      \]
      D'où $\PP{n+1}$.
    \end{noliste}
    Par principe de récurrence : $\forall n\in\N$, $u_n\leq u_{n+1}$.%
    \conc{La suite $(u_n)$ est croissante.}~\\[-1cm]
  \end{proof}

\item En déduire que la suite $(u_n)_{n\in \N}$ converge et déterminer
  sa limite.\\
  {\it (on pourra étudier les variations de la fonction $t\mapsto
    t-\ln(t)$)}

\begin{proof}~
 \begin{noliste}{$\sbullet$}
  \item La suite $(u_n)$ est :
  \begin{noliste}{$\stimes$}
    \item croissante, d'après la question \itbf{10.},
    \item majorée par $1$, d'après la question \itbf{9.},
  \end{noliste}
  Elle est donc convergent vers un réel $\ell$.
  
\item De plus : $\forall n \in\N$, $\dfrac{1}{2} \leq u_n \leq 1$.\\
  Donc, par passage à la limite dans cette inégalité : $\dfrac{1}{2}
  \leq \ell \leq 1$.
  
\item On a : $\forall n\in\N$, $u_{n+1}=f(u_n)$.\\
  Donc, par passage à la limite et par continuité de $f$ sur
  $\left[\frac{1}{2},1\right]$, on obtient :
  \[
   \begin{array}{rcl@{\qquad}>{\it}R{4cm}}
    \ell = f(\ell) 
    & \Leftrightarrow &
    \ell = \ell^2 - \ell \, \ln(\ell)
    \\[.2cm]
    & \Leftrightarrow &
    \bcancel{\ell} = \bcancel{\ell} (\ell-\ln(\ell))
    & (car $\ell \geq \dfrac{1}{2}$, donc en particulier $\ell \neq 0$)
    \nl
    \nl[-.2cm]
    & \Leftrightarrow &
    1 = \ell - \ln(\ell)
   \end{array}
  \]
  Donc $\ell = f(\ell)$ si et seulement si $g(\ell)=1$ où $g : t
  \mapsto t - \ln(t)$.
  
  \newpage
  
\item Étudions alors la fonction $g$ sur $\left[\frac{1}{2},1\right]$.
  \begin{noliste}{$-$}
  \item La fonction $g$ est dérivable sur $\left[\frac{1}{2},1\right]$
    comme somme de fonctions dérivables sur
    $\left[\frac{1}{2},1\right]$.
    
    \item Soit $t\in\left[\frac{1}{2},1\right]$.
    \[
     g'(t) = 1-\dfrac{1}{t} = \dfrac{t-1}{t}
    \]
    Comme $t > 0$, $g'(t)$ est du signe de $t-1$. Ainsi :
    \[
    g'(t) \geq 0 \ \Leftrightarrow \ t-1 \geq 0 \ \Leftrightarrow \ t
    \geq 1
    \]
    
    \item On obtient alors le tableau de variations suivant :
    
    \begin{center}
      \begin{tikzpicture}[scale=0.8, transform shape]
        \tkzTabInit[lgt=4,espcl=3] %
        { %
        $t$ /1, %
        Signe de $g'(t)$ /1, %
        Variations de $g$ /2
        } %
        {$\frac{1}{2}$, $1$} %
        \tkzTabLine{ , - , } % 
        \tkzTabVar{+/$g\left(\frac{1}{2}\right)$, -/$1$} %
      \end{tikzpicture}
     \end{center}
     
    \item La fonction $g$ est donc :
    \begin{noliste}{$\stimes$}
    \item continue sur $\left[\frac{1}{2},1\right]$ (car dérivable sur
      cet intervalle),
      
    \item strictement décroissante sur $\left[\frac{1}{2},1\right]$.
    \end{noliste}
    Ainsi $g$ réalise une bijection de $\left[\frac{1}{2},1\right]$
    sur $g\left(\left[\frac{1}{2},1\right]\right)$. De plus :
    \[
    g\left(\left[ \mbox{$\frac{1}{2}$}, 1 \right]\right) = \left[g(1),
      g\left( \mbox{$\frac{1}{2}$} \right)\right] = \left[1,
      \mbox{$\frac{1}{2}$} + \ln(2) \right]
    \]
    Or $1\in \left[1, \frac{1}{2}+\ln(2)\right]$, donc l'équation
    $g(t)=1$ admet une unique solution sur
    $\left[\frac{1}{2},1\right]$.\\[.2cm]
    De plus $g(1)=1$. Donc l'unique solution de l'équation $g(t)=1$
    sur $\left[\frac{1}{2},1\right]$ est $1$.
  \end{noliste}
\item Or $\ell\in \left[\frac{1}{2},1\right]$ et $g(\ell)=1$.
  Donc $\ell=1$.
\end{noliste}
\conc{On en déduit que la suite $(u_n)$ converge et 
  $\dlim{n\to+\infty} u_n=1$}~\\[-1cm]
\end{proof}

\item Écrire un programme en \Scilab{} qui calcule et affiche un entier 
naturel $N$ tel que $1-u_N<10^{-4}$.

\begin{proof}~
 \begin{scilab}
   & n = 0 \nl
   & u = 1/2 \nl
   & \tcFor{while} 1 - u >= 10\puis{}(-4) \nl
   & \quad u = u\puis{}2 - u \Sfois{} log(u) \nl
   & \quad n = n + 1 \nl
   & \tcFor{end} \nl
   & disp(n)
 \end{scilab}~\\[-1cm]
\end{proof}

\end{noliste}

\newpage

\section*{EXERCICE III}


\subsection*{PARTIE I : Étude d'une variable aléatoire}

\noindent
On considère l'application $f:\R\rightarrow\R$ définie, 
pour tout $t$ de $\R$, par : 
$f(t)=\dfrac{\ee^{-t}}{(1+\ee^{-t})^2}$.

\begin{noliste}{1.}
\setlength{\itemsep}{2mm}
\item Vérifier que la fonction $f$ est paire.

\begin{proof}~\\
 Soit $t\in\R$. On a bien $-t\in\R$.
 \[
  \begin{array}{rcl}
    f(-t) - f(t) & = & \dfrac{\ee^{-(-t)}}{(1+\ee^{-(-t)})^2}
    - \dfrac{\ee^{-t}}{(1+\ee^{-t})^2}
    \\[.4cm]
    & = & \dfrac{\ee^t}{(1+\ee^t)^2} - \dfrac{\ee^{-t}}{(1+\ee^{-t})^2}
    \\[.6cm]
    & = & \dfrac{\ee^t(1+\ee^{-t})^2-\ee^{-t}(1+\ee^t)^2}
    {(1+\ee^t)^2 (1+\ee^{-t})^2}
    \\[.6cm]
    & = & \dfrac{\ee^t(1+2\ee^{-t}+\ee^{-2t}) 
    - \ee^{-t}(1+2\ee^t+\ee^{2t})}
    {(1+\ee^t)^2 (1+\ee^{-t})^2}
    \\[.6cm]
    & = & \dfrac{\bcancel{\ee^t +2+\ee^{-t}} - \bcancel{(\ee^{-t}
    +2+\ee^t)}}{(1+\ee^t)^2 (1+\ee^{-t})^2} \ = \ 0
  \end{array}
 \]
 Donc $f(-t)=f(t)$.
 \conc{La fonction $f$ est paire.}
 
 \begin{remark}%~\\
  La méthode classique pour démontrer qu'une fonction $f$ est paire
  est de prouver l'égalité :
  \[
   \forall x \in \R, \ f(-x)=f(x)
  \]
  Lorsque ce calcul direct ne semble pas aboutir, on pensera à 
  former $f(-x)-f(x)$.\\[.2cm]
  En règle générale, pour démontrer l'égalité \og $a=b$ \fg{}, on peut :
  \begin{noliste}{$-$}
   \item partir de $a$ et, par une succession d'égalités, arriver à $b$.
   \item partir de $b$ et, par une succession d'égalités, arriver à $a$.
   \item prouver $a=c$, puis $b=c$ par l'une des méthodes précédentes 
   (méthode du \og mi-chemin \fg{}.
   \item calculer $a-b$, pour prouver $a-b=0$.
  \end{noliste}
  Pour choisir entre les deux premières méthodes, on retiendra qu'il 
est plus simple de transformer une expression \og compliquée \fg{} en 
expression \og simple \fg{} que l'inverse.\\
La dernière méthode est souvent efficace, notamment lorsque les autres 
ne semblent pas aboutir.
 \end{remark}~\\[-1.2cm]
\end{proof}



\newpage



\item Montrer que  $f$ est une densité d'une variable aléatoire réelle.

\begin{proof}~
 \begin{noliste}{$\sbullet$}
 \item La fonction $f$ est continue sur $\R$ car c'est le quotient de
   fonctions continues sur $\R$ dont le dénominateur ne s'annule pas.
   Détaillons ce dernier point.\\
   Soit $t\in\R$.
  \[
   \begin{array}{rcl@{\qquad}>{\it}R{5cm}}
     \ee^{-t}>0 & \mbox{donc} & 1+\ee^{-t}>1
     \\[.2cm]
     & \mbox{et} & (1+\ee^{-t})^2 >1^2
     & (car la fonction $x\mapsto x^2$ est croissante sur $[0,+\infty[$)
     \nl
     \nl[-.2cm]
     & \mbox{ainsi} & (1+\ee^{-t})^2>0
   \end{array}
  \]
  \conc{La fonction $f$ est continue sur $\R$.}
  
\item Pour tout $t\in\R$, $\ee^{-t}>0$ donc $(1+\ee^{-t})^2>0$.%
  \conc{$\forall t \in\R$, $f(t)\geq 0$}
  
\item L'intégrale impropre $\dint{-\infty}{+\infty} f(t) \dt$ est
  convergente si les intégrales impropres $\dint{-\infty}{0} f(t) \dt$
  et $\dint{0}{+\infty} f(t) \dt$ le sont. On étudie tout d'abord
  l'intégrale impropre $\dint{0}{+\infty} f(t) \dt$.
  \begin{noliste}{$-$}
  \item La fonction $f$ est continue sur $[0, +\infty[$.
  
  \item Soit $A\in [0, +\infty[$.
  \[
   \dint{0}{A} f(t)\dt = \dint{0}{A} \dfrac{\ee^{-t}}{(1+\ee^{-t})^2} 
   \dt = \Prim{\dfrac{1}{1+\ee^{-t}}}{0}{A} = \dfrac{1}{1+\ee^{-A}} 
   -\dfrac{1}{2} \ \tendd{A}{+\infty} \ 1-\dfrac{1}{2}=\dfrac{1}{2}
  \]
  Ainsi, $\dint{0}{+\infty} f(t)\dt$ converge et vaut $\dfrac{1}{2}$.
  \end{noliste}
  On étudie maintenant l'intégrale impropre $\dint{-\infty}{0} f(t)
  \dt$.
  \begin{noliste}{$-$}
  \item On effectue le changement de variable $\Boxed{u = -t}$.
      \[
      \left|
        \begin{array}{P{7cm}}
          $u = -t$ (donc $t = -u$) \nl
          $\hookrightarrow$ $du = - \dt$ \quad et \quad $dt = - du$ \nl
          \vspace{-.4cm}
          \begin{noliste}{$\sbullet$}
          \item $t = -\infty \ \Rightarrow \ u = +\infty$
          \item $t = 0 \ \Rightarrow \ u = 0$
            \vspace{-.4cm}
          \end{noliste}
        \end{array}
      \right.
      \]
      Ce changement de variable est valide car $\psi:u \mapsto -u$ est
      de classe $\Cont{1}$ sur $[0, +\infty[$.\\
      On a donc :
   \[
   \dint{0}{+\infty} f(t) \dt = \dint{0}{-\infty} f(-u)(-\ du) =
   \dint{-\infty}{0} f(-u) \ du = \dint{-\infty}{0} f(u) \ du
   \]
   La dernière égalité est obtenue car la fonction $f$ est paire
   (d'après la question \itbf{1.}).\\
   On en déduit que l'intégrale impropre $\dint{-\infty}{0} f(u) \ du$
   est convergente et vaut $\dfrac{1}{2}$.
   % Donc $\dint{-\infty}{0}f(t)\dt$ converge et vaut $\dfrac{1}{2}$.
  \end{noliste}
  \conc{Ainsi : $\dint{-\infty}{+\infty} f(t)\dt$ converge et :
    $\dint{-\infty}{+\infty} f(t) \dt = \dint{-\infty}{0} f(t) \dt +
    \dint{0}{+\infty} f(t) \dt = 1$.}
 \end{noliste} 
%  \newpage
%  On a donc montré les trois points suivants :
%  \begin{noliste}{$-$}
%   \item $f$ est continue sur $\R$,
%   \item $\forall t\in \R$, $f(t) \geq 0$,
%   \item l'intégrale $\dint{-\infty}{+\infty} f(t) \dt$ converge et 
%   vaut $1$.
%  \end{noliste}
 \conc{Finalement, la fonction $f$ est une densité d'une 
 variable aléatoire réelle.}%~\\[-1cm]


\newpage


\begin{remark}%~%
  \begin{noliste}{$\sbullet$}
  \item Le programme officiel stipule que \og les changements de
    variables affines pourront être utilisés directement sur des
    intégrales sur un intervalle quelconque \fg{}. Pour autant, ce
    type de changement de variable ne peut se faire qu'après avoir
    démontré la convergence (ce qu'on a fait dans l'exemple). 
  \item Cela signifie aussi que, de manière générale, on ne peut
    effectuer de changement de variable directement sur une intégrale
    impropre : on doit se ramener au préalable sur une intégrale sur
    un segment.
  % \item Attention ! C'est le seul cas où nous ne sommes pas
  %   obligés de revenir sur un intégrale sur un segment.
  \end{noliste}
\end{remark}~\\[-1.3cm]
\end{proof}

\end{noliste}

\noindent
Dans toute la suite de l'exercice, on considère une variable aléatoire 
réelle $X$ à densité, de densité $f$.
\begin{noliste}{1.}
\setlength{\itemsep}{2mm}
\setcounter{enumi}{2}
\item Déterminer la fonction de répartition de $X$.

\begin{proof}~\\
 On note $F_X$ la fonction de répartition de $X$. Soit $x\in\R$.
 \[
  F_X(x) = \Prob(\Ev{X\leq x}) = \dint{-\infty}{x} f(t)\dt
 \]
 La fonction $f$ est continue sur $]-\infty, x]$.\\
 Soit $B \in \ ]-\infty, x]$.
 \[
  \dint{B}{x} f(t)\dt = \dint{B}{x} \dfrac{\ee^{-t}}{(1+\ee^{-t})^2}
  \dt = \Prim{\dfrac{1}{1+\ee^{-t}}}{B}{x} = 
  \dfrac{1}{1+\ee^{-x}} - \dfrac{1}{1+\ee^{-B}} \ \tendd{B}{-\infty} \
  \dfrac{1}{1+\ee^{-x}}
 \]
 Donc $\dint{-\infty}{x}f(t)\dt$ converge et vaut
 $\dfrac{1}{1+\ee^{-x}}$.
 \conc{$\forall x\in\R$, $F_X(x)=\dfrac{1}{1+\ee^{-x}}$.}~\\[-1cm]
\end{proof}

%\newpage

\item
\begin{noliste}{a)}
\item Montrer que l'intégrale $\dint{0}{+\infty}t \, f(t)\dt$ 
converge.

\begin{proof}~
 \begin{noliste}{$\sbullet$}
 \item La fonction $t\mapsto t\, f(t)$ est continue sur $[0,+\infty[$
   comme produit de fonctions continues sur l'intervalle
   $[0,+\infty[$.
  
\item
  \begin{noliste}{$\stimes$}
  \item Tout d'abord : $t\, f(t) = \oo{t}{+\infty}
    \left(\dfrac{1}{t^2}\right)$.\\
    En effet : 
    \[
    \dfrac{t\, f(t)}{\frac{1}{t^2}} = t^2 \, tf(t) 
    = \dfrac{t^3 \, \ee^{-t}}{(1+\ee^{-t})^2} = 
    \dfrac{t^3\, \ee^{-t}}{1+2\ee^{-t}+\ee^{-2t}}
    \ \eq{t}{+\infty} \ \dfrac{t^3 \, \ee^{-t}}{1} = t^3\, \ee^{-t}
    \]
    Or, par croissances comparées, $\dlim{t\to+\infty} t^3\ee^{-t}=0$.
    Donc $\dlim{t\to+\infty} t^2 \, tf(t)=0$.\\[.2cm]
    On en déduit : $t\, f(t) = \oo{t}{+\infty}
    \left(\dfrac{1}{t^2}\right)$.

  \item $\forall t\in[1,+\infty[$, $t\, f(t) \geq 0$ \quad et \quad
    $\dfrac{1}{t^2}\geq 0$.
  \item L'intégrale $\dint{1}{+\infty} \dfrac{1}{t^2}\dt$ est une
    intégrale de Riemann impropre en $+\infty$, d'exposant $2>1$.
    Elle est donc convergente.
  \end{noliste}
  Par critère de négligeabilité des intégrales généralisées de 
  fonctions continues positives, l'intégrale $\dint{1}{+\infty} t\, 
  f(t)\dt$ converge.
  

  
  \newpage
  
  
\item La fonction $t\mapsto t\, f(t)$ est continue sur le {\bf
    segment} $[0,1]$.\\
  Ainsi, l'intégrale $\dint{0}{1} t\, f(t)\dt$ est bien définie.
 \end{noliste}
 \conc{Finalement, l'intégrale $\dint{0}{+\infty} t\, f(t)\dt$
   converge.}
 
 \begin{remark}%~\\%
  L'énoncé demande simplement de déterminer la {\bf nature} de 
  l'intégrale $\dint{0}{+\infty} t\, f(t)\dt$ (sans la calculer). Il 
  faut donc privilégier pour cette question l'utilisation d'un critère 
  de comparaison / équivalence / négligeabilité d'intégrales de 
  fonctions continues positives.\\
  On évitera donc le calcul direct de l'intégrale car :
  \begin{noliste}{\scriptsize 1.}
   \item il est sans doute difficile,
   \item il est peut-être même (souvent) impossible.
  \end{noliste}
 \end{remark}~\\[-1.4cm]
\end{proof}

\item En utilisant l'imparité de la fonction $\R\to \R$, $t\mapsto t
  \, f(t)$, montrer que $X$ admet une espérance et que l'on a :
  $\E(X)=0$.

\begin{proof}~
 \begin{noliste}{$\sbullet$}
 \item La \var $X$ admet une espérance si et seulement si l'intégrale
   impropre $\dint{-\infty}{+\infty} t \ f(t) \dt$ est absolument
   convergente, ce qui équivaut à démontrer la convergence pour un
   calcul de moment du type $\dint{-\infty}{+\infty} t^n \ f(t) \dt$.\\
   L'intégrale impropre $\dint{-\infty}{+\infty} t f(t) \dt$ est
   convergente si les intégrales impropres $\dint{-\infty}{0} t f(t)
   \dt$ et $\dint{0}{+\infty} t f(t) \dt$ le sont. Or, d'après la
   question précédente, l'intégrale $\dint{0}{+\infty} t f(t) \dt$
   converge.
 \item Déterminons la nature de $\dint{-\infty}{0} t f(t) \dt$.\\
   On effectue le changement de variable $\Boxed{u = -t}$.
      \[
      \left|
        \begin{array}{P{6cm}}
          $u = -t$ (donc $t = -u$) \nl
          $\hookrightarrow$ $du = - \dt$ \quad et \quad $dt = - du$ \nl
          \vspace{-.4cm}
          \begin{noliste}{$\sbullet$}
          \item $t = -\infty \ \Rightarrow \ u = +\infty$
          \item $t = 0 \ \Rightarrow \ u = 0$
            \vspace{-.4cm}
          \end{noliste}
        \end{array}
      \right.
      \]
      Ce changement de variable est valide car $\psi:u \mapsto -u$ est
      de classe $\Cont{1}$ sur $[0, +\infty[$.\\
      On a donc :
   \[
   \dint{0}{+\infty} tf(t) \dt = \dint{0}{-\infty} (\bcancel{-}u)
   f(-u) (\bcancel{-}\ du) = \dint{0}{-\infty} u f(u) \ du = -
   \dint{-\infty}{0} u f(u) \ du
   \]
   La deuxième égalité est obtenue car la fonction $f$ est paire
   (d'après la question \itbf{1.}).\\
   On en déduit que l'intégrale impropre $\dint{-\infty}{0} f(u) \
   du$ est convergente.


   \newpage


 \item Finalement, $X$ admet une espérance et :
   \[
   \E(X) = \dint{-\infty}{+\infty} tf(t)\dt
   =\dint{-\infty}{0}tf(t)\dt + \dint{0}{+\infty} tf(t)\dt
   =-\bcancel{\dint{0}{+\infty} tf(t)\dt} + \bcancel{
     \dint{0}{+\infty} tf(t)\dt}=0
   \]
 \end{noliste}
 \conc{La \var $X$ admet une espérance et $\E(X)=0$.}~\\[-1.2cm]
\end{proof}

\end{noliste}
\end{noliste}

\subsection*{PARTIE II. Étude d'une autre variable aléatoire}

\noindent
On considère l'application $\varphi: \R\rightarrow\R$ définie, pour
tout $x$ de $\R$, par : $\varphi(x) = \ln(1+\ee^x)$.
\begin{noliste}{1.}
  \setlength{\itemsep}{2mm}%
  \setcounter{enumi}{4}
\item Montrer que $\varphi$ est une bijection de $\R$ sur un 
  intervalle $I$ à préciser.

\begin{proof}~
 \begin{noliste}{$\sbullet$}
 \item La fonction $\varphi$ est dérivable sur $\R$ car elle est la
   composée $\varphi = h\circ g$ où :
  \begin{noliste}{$\stimes$}
  \item $g$ : $x\mapsto 1+\ee^x$ :
      \begin{noliste}{$-$}
      \item dérivable sur $\R$,
      \item telle que $g(\R)\subset \ ]0,+\infty[$ (pour tout $x\in \
        \R$, $1+\ee^x>0$).
      \end{noliste}

    \item $h:y\mapsto \ln(y)$ dérivable sur 
    $]0,+\infty[$.
  \end{noliste}
  
  
  \item Soit $x\in\R$.
  \[
   \varphi'(x) = \dfrac{\ee^x}{1+\ee^x} >0
  \]
  Donc $\varphi$ est strictement croissante sur $\R$.
%   On obtient le tableau de variations suivant :
  
%   \begin{center}
%       \begin{tikzpicture}[scale=0.8, transform shape]
%         \tkzTabInit[lgt=4,espcl=3] %
%         { %
%         $t$ /1, %
%         Signe de $\varphi'(t)$ /1, %
%         Variations de $\varphi$ /2
%         } %
%         {$-\infty$, $+\infty$} %
%         \tkzTabLine{ , + , } % 
%         \tkzTabVar{-/$0$, +/$+\infty$} %
%       \end{tikzpicture}
%      \end{center}
%   Détaillons les éléments de ce tableau.
%   \begin{noliste}{$-$}
%   \item Tout d'abord : $\dlim{x\to-\infty} (1+\ee^x)=1$. Donc, par
%     continuité de $y\mapsto \ln(y)$ en $1$ :
%    \[
%     \dlim{x\to-\infty}\varphi(x) = \dlim{x\to-\infty} \ln(1+\ee^x)
%     =\ln(1)=0
%    \]
   
%  \item De plus : $\dlim{x\to+\infty} (1+\ee^x)=+\infty$. Donc :
%    $\dlim{x\to+\infty} \varphi(x)= \dlim{x\to+\infty} \ln(1+\ee^x) =
%    +\infty$.
%   \end{noliste}
  
  \item La fonction $\varphi$ est :
  \begin{noliste}{$\stimes$}
    \item continue sur $]-\infty,+\infty[$ (car dérivable sur cet 
    intervalle),
    \item strictement croissante sur $]-\infty,+\infty[$.
  \end{noliste}
  Ainsi, $\varphi$ réalise une bijection de $]-\infty,+\infty[$ sur
  $I=\varphi(]-\infty,+\infty[)$. De plus :
  \[
   \varphi(]-\infty,+\infty[) = \left] \dlim{x\to-\infty} \varphi(x),
   \dlim{x\to+\infty} \varphi(x)\right[ = \ ]0,+\infty[
  \]
 \end{noliste}
 \conc{La fonction $\varphi$ réalise une bijection de $\R$ sur 
 $I= \ ]0,+\infty[$.}~\\[-1cm]
\end{proof}

\item Exprimer, pour tout $y$ de $I$, $\varphi^{-1}(y)$.

\begin{proof}~\\
 Soit $y\in I$. Soit $x\in\R$.
 \[
  \begin{array}{rcl@{\quad}>{\it}R{5.5cm}}
   \varphi^{-1}(y)=x
   & \Leftrightarrow & 
   y = \varphi(x)
   \\[.2cm]
   & \Leftrightarrow &
   y=\ln(1+\ee^x)
   \\[.2cm]
   & \Leftrightarrow & 
   \ee^y = 1+\ee^x
   & (car la fonction $x\mapsto \ee^x$ est bijective sur $\R$)
   \nl
   \nl[-.2cm]
   & \Leftrightarrow &
   \ee^y -1=\ee^x
   \\[.2cm]
   & \Leftrightarrow &
   \ln(\ee^y-1)=x
   & (car la fonction $x\mapsto \ln(x)$ est bijective sur $]0,+\infty[$)
  \end{array}
 \]
 \conc{$\forall y\in I$, $\varphi^{-1}(y)=\ln(\ee^y-1)$}
 
 \begin{remark}%~\\
   On remarque que la composition par $\ln$ est bien autorisée car, si
   $y\in I= \ ]0,+\infty[$, alors $\ee^y\in \ ]1,+\infty[$. Et donc
   $\ee^y-1 \in \ ]0,+\infty[$.
 \end{remark}~\\[-1.3cm]
\end{proof}

\end{noliste}


\newpage


\noindent
On considère la variable aléatoire réelle $Y$ définie par : 
$Y=\varphi(X)$.
\begin{noliste}{1.}
\setlength{\itemsep}{2mm}
\setcounter{enumi}{6}
\item Justifier : $\Prob(\Ev{Y\leq 0})=0$.

\begin{proof}~
 \begin{noliste}{$\sbullet$}
 \item Remarquons tout d'abord : 
   \[
   Y(\Omega) = (\varphi(X))(\Omega) = \varphi\big( X(\Omega)
   \big) \subset \ ]0,+\infty[
   \]
   En effet, $X(\Omega) \subset \R$ et $\varphi(\R)= I = \
   ]0,+\infty[$ (d'après la question \itbf{5.})
 
 \item On obtient alors : $\Ev{Y\leq 0}=\varnothing$. Donc :
 \[
  \Prob(\Ev{Y\leq 0})=\Prob(\varnothing)=0
 \]
 \end{noliste}
 \conc{$\Prob(\Ev{Y\leq 0})=0$}~\\[-1cm]
\end{proof}

\item Déterminer la fonction de répartition de $Y$.

\begin{proof}~
 \begin{noliste}{$\sbullet$}
  \item Tout d'abord : $Y(\Omega) \subset \ ]0,+\infty[$.
  
  \item Soit $x\in\R$. Deux cas se présentent.
  \begin{noliste}{$-$}
  \item \dashuline{Si $x\leq 0$} alors $\Ev{Y\leq x}=\varnothing$ car
    $Y(\Omega) \subset \ ]0,+\infty[$. Donc, on a :
    \[
     F_Y(x)=\Prob(\Ev{Y\leq x})=\Prob(\varnothing)=0
    \]
    
    
    % \newpage
    
    \item \dashuline{Si $x>0$}.
    \[
     \begin{array}{rcl@{\quad}>{\it}R{5.5cm}}
       F_Y(x) & = & \Prob(\Ev{Y\leq x})
       \\[.2cm]
       & = & \Prob(\Ev{\varphi(X) \leq x})
       \\[.2cm]
       & = & \Prob(\Ev{X \leq \varphi^{-1}(x)})
       & (car $\varphi$ est strictement croissante, donc 
       $\varphi^{-1}$ également)
       \nl
       \nl[-.2cm]
       & = & \Prob(\Ev{X\leq \ln(\ee^x-1)})
       \ = \ F_X(\ln(\ee^x-1))
       \\[.2cm]
       & = & \dfrac{1}{1 + \exp(-\ln(\ee^x-1))}
       \\[.4cm]
       & = & \dfrac{1}{1 + \exp\left(\ln\big( (\ee^x-1)^{-1} \big) \right)}
       \\[.4cm]
       & = & \dfrac{1}{1 + \frac{1}{\ee^x-1}}
       \ = \ \dfrac{1}{\frac{(\ee^x - \bcancel{1}) + \bcancel{1}}{\ee^x-1}}
       \ = \ \dfrac{\ee^x-1}{\ee^x}
      \\[.6cm]
      & = & \dfrac{\bcancel{\ee^x} \ (1 - \ee^{-x})}{\bcancel{\ee^x}}
      \ = \ 1 - \ee^{-x} 
     \end{array}
    \]
  \end{noliste}  
  \conc{$F_Y : x \mapsto \left\{
      \begin{array}{cR{1.6cm}}
        1 - \ee^{-x} & \mbox{ si $x>0$}
        \nl
        0 & \mbox{ si $x\leq 0$}
      \end{array}
    \right.$}~\\[-1.4cm]
\end{noliste}
\end{proof}


\item Reconnaître alors la loi de $Y$ et donner, sans calcul, son 
espérance et sa variance.

\begin{proof}~ %
  \conc{On reconnaît une loi exponentielle : $Y\suit \Exp{1}$. On a
    alors $\E(Y)=\dfrac{1}{1}=1$ et
    $\V(Y)=\dfrac{1}{1^2}=1$.}~\\[-.6cm]
\end{proof}

\end{noliste}


\newpage


\subsection*{PARTIE III : Étude d'une convergence en loi}

\noindent
On considère une suite de variables aléatoires réelles $(X_n)_{n\in
  \N^*}$, mutuellement indépendantes, de même densité $f$, où $f$ a
été définie dans la partie I.\\
On pose, pour tout $n$ de $\N^*$ : $T_n=\max(X_1,\ldots,X_n)$ et
$U_n=T_n-\ln(n)$.

\begin{noliste}{1.}
  \setlength{\itemsep}{2mm}
  \setcounter{enumi}{9}
\item
  \begin{noliste}{a)}
  \item Déterminer, pour tout $n$ de $\N^*$, la fonction de
    répartition de $T_n$.

    \begin{proof}~\\
      Soit $n\in\N^*$.
      \begin{noliste}{$\sbullet$}
      \item Soit $x\in\R$. On commence par noter que :
        \[
        \Ev{T_n\leq x} = \Ev{X_1 \leq x} \ \cap \ \cdots \ \cap \Ev{X_n\leq 
          x} = \dcap{i=1}{n} \Ev{X_i \leq x}
        \]
        
      \item On obtient alors :
        \[
        \begin{array}{rcl@{\qquad}>{\it}R{4.5cm}}
          F_{T_n}(x) & = & \Prob(\Ev{T_n \leq x})
          \\[.2cm]
          & = & \Prob\left(\dcap{i=1}{n} \Ev{X_i \leq x}\right)
          \\[.2cm]
          & = & \Prod{i=1}{n} \Prob(\Ev{X_i \leq x})
          & (car les \var $X_1$, $\hdots$, $X_n$ sont indépendantes)
          \nl
          \nl[-.2cm]
          & = & \Prod{i=1}{n} \Prob(\Ev{X_1\leq x})
          & (car les \var $X_1$, $\hdots$, $X_n$ ont même loi)
          \nl
          \nl[-.2cm]
          & = & \big(\Prob(\Ev{X_1 \leq x} \big)^n
        \end{array}
        \]
  
      \item D'après la question \itbf{3.} : $\forall x\in\R$,
        $F_{X_1}(x) = \dfrac{1}{1+\ee^{-x}}$.
      \end{noliste}
      \conc{On en déduit : $\forall x\in\R$, $F_{T_n}(x) =
        \left(\dfrac{1}{1+\ee^{-x}}\right)^n$.}~\\[-1cm]
    \end{proof}
    
  \item En déduire : $\forall n\in \N^*, \ \forall x\in \R, \
    \Prob(\Ev{U_n\leq x}) = \left(1 + \dfrac{\ee^{-x}}{n}
    \right)^{-n}$.
    
    \begin{proof}~\\
      Soit $n\in\N^*$. Soit $x\in\R$.
      \[
      \begin{array}{rcl@{\qquad}>{\it}R{4.5cm}}
        \Prob(\Ev{U_n\leq x}) & = & \Prob(\Ev{T_n - \ln(n)\leq x})
        \\[.2cm]
        & = & \Prob(\Ev{T_n \leq x+\ln(n)})
        \\[.2cm]
        & = & F_{T_n}(x+\ln(n))
        \\[.2cm]
        & = & \left(\dfrac{1}{1 + \exp(-(x+\ln(n)))}\right)^n
        & (d'après la question \itbf{10.a)})
        \nl
        \nl[-.2cm]
        & = & \Big(1 + \exp(-x-\ln(n)) \Big)^{-n}
        % \\[.2cm]
        % & = &
        % \multicolumn{2}{l}{\left(1+\dfrac{\ee^{-x}}{\exp(\ln(n))}\right)^{-n}
        %   \ = \ \left(1+\dfrac{\ee^{-x}}{n}\right)^{-n}}
      \end{array}
      \]
      Or : $\exp(-x-\ln(n)) = \ee^{-x-\ln(n)} = \ee^{-x} \
      \ee^{-\ln(n)} = \ee^{-x} \ \ee^{\ln(n^{-1})} =
      \dfrac{\ee^{-x}}{n}$.

      \conc{$\forall n\in\N^*$, $\forall x\in\R$, $\Prob(\Ev{U_n \leq
          x}) = \left(1+\dfrac{\ee^{-x}}{n}\right)^{-n}$}~\\[-1cm]
    \end{proof}
  \end{noliste}

\item En déduire que la suite de variables aléatoires
  $(U_n)_{n\in\N^*}$ converge en loi vers une variable aléatoire
  réelle à densité dont on précisera la fonction de répartition et une
  densité.

\begin{proof}~\\
  Soit $x\in\R$. On cherche à déterminer, si elle existe,
  $\dlim{n\to+\infty} F_{U_n}(x)$.
 \begin{noliste}{$\sbullet$}
  \item Soit $n\in\N^*$.
  \[
   F_{U_n}(x)=\left(1+\dfrac{\ee^{-x}}{n}\right)^{-n}
   = \exp\left(-n \, \ln\left( 1+\dfrac{\ee^{-x}}{n}\right) 
   \right)
  \]
  
\item De plus : $\ln(1+u) \eq{u}{0} u$.\\[.1cm]
  Or $\dlim{n\to+\infty} \dfrac{\ee^{-x}}{n}=0$, donc
  $\ln\left(1+\dfrac{\ee^{-x}}{n}\right) \eqn \dfrac{\ee^{-x}}{n}$.
  D'où :
  \[
   -n \, \ln\left(1+\dfrac{\ee^{-x}}{n}\right) \eqn 
   -\bcancel{n} \, \dfrac{\ee^{-x}}{\bcancel{n}}=-\ee^{-x}
  \]
  \item Or la fonction $x\mapsto \exp(x)$ est continue sur $\R$, donc, 
  par composition de {\bf limites} :
  \[
   \dlim{n\to+\infty} \exp\left(-n \, \ln\left(1+ 
   \dfrac{\ee^{-x}}{n}\right)\right) = \exp\left(-\ee^{-x}\right)
  \]
  D'où :
  \[
   \dlim{n\to+\infty} F_{U_n}(x) = \exp(-\ee^{-x})
  \]
  \conc{On note $G$ la fonction définie sur $\R$ par $G:x
    \mapsto \exp(-\ee^{-x})$.\\[.1cm]
    On a alors $\dlim{n\to+\infty} F_{U_n}(x)=G(x)$.}
\end{noliste}
Montrons que $G$ est une fonction de répartition.
\begin{noliste}{$\sbullet$}
    \item Tout d'abord : $\dlim{x\to+\infty} -\ee^{-x} = 0$. Donc,
    par continuité de $\exp$ en $0$ : 
    \[
     \dlim{x\to+\infty} G(x)= \dlim{x\to+\infty} \exp(-\ee^{-x})=\ee^0=1
    \]
    
    \item De plus : $\dlim{x\to-\infty} -\ee^{-x}=-\infty$. D'où :
    \[
     \dlim{x\to-\infty} G(x)=\dlim{x\to-\infty} \exp(-\ee^{-x}) =0
    \]
    
  \item La fonction $G$ est continue sur $\R$ car elle est la composée
    $G = h_2\circ h_1$ où :
    \begin{noliste}{$\stimes$}
    \item $h_1$ : $x\mapsto -\ee^{-x}$ est :
      \begin{noliste}{$-$}
      \item continue sur $\R$,
      \item telle que $h_1(\R) \subset \R$.
      \end{noliste}
      
    \item $h_2:y\mapsto \exp(y)$ continue sur $\R$.
    \end{noliste}
    
    \item Elle est dérivable sur $\R$ pour la même raison et, pour 
    tout $x\in\R$ :
    \[
     g(x)=G'(x)=\ee^{-x} \, \exp(-\ee^{-x}) >0
    \]
    Donc la fonction $G$ est croissante sur $\R$.
  \end{noliste}
  \conc{La fonction $G$ est une fonction de répartition.}
  
  
  \newpage


  \noindent
  Montrons que $G$ est la fonction de répartition d'une \var à
  densité.
  \begin{noliste}{$\sbullet$}
  \item On vient de démontrer que $G$ est continue sur $\R$.
  \item La fonction $G$ est aussi de classe $\Cont{1}$ sur $\R$ car
    les fonctions $h_1$ et $h_2$ sont de classe $\Cont{1}$ sur $\R$.
  \end{noliste}
  \conc{La fonction $G$ est la fonction de répartition d'une \var à 
  densité que l'on notera $V$.}

\begin{noliste}{$\sbullet$}  
\item Pour déterminer une densité de $V$, on dérive la fonction $G$
  sur $\R$ ($\R = \ ]-\infty,+\infty[$ est bien un intervalle ouvert).
  On en déduit que $g$ est bien une densité de $V$.
 \end{noliste}
 \conc{La suite $(U_n)$ converge en loi vers la \var $V$ de fonction 
 de répartition $G:x\mapsto \exp(-\ee^{-x})$\\[.1cm] 
 dont une densité est $g:x\mapsto \ee^{-x} \, \exp(-\ee^{-x})$.}
\begin{remark}%~%
  \begin{noliste}{$\sbullet$}
  \item Le programme officiel liste certaines propriétés d'une fonction de
    répartition $F$ :
    \begin{noliste}{\scriptsize 1.}
    \item $F$ est croissante.
    \item $F$ est continue à droite en tout point.
    \item $\dlim{x\to+\infty} F(x)=1$.
    \item $\dlim{x\to-\infty} F(x)=0$.
    \end{noliste}
    Cependant, il n'est pas précisé qu'il s'agit là d'une
    caractérisation d'une fonction de répartition : toute fonction $F
    : \R \to \R$ qui vérifie les propriétés \itbf{1.}, \itbf{2.},
    \itbf{3.} et \itbf{4.} peut être considérée comme la fonction de
    répartition d'une variable aléatoire.

  \item L'utilisation de la caractérisation ci-dessus ne semble
    apparaître que dans ce type de question traitant de la convergence
    en loi. Ce type de question apparaît aussi dans le sujet d'EML
    2017.\\
    Il est donc conseillé de connaître la caractérisation ci-dessus.
%   \item Le programme officiel ne mentionne pas explicitement la
%     caractérisation des fonctions de répartition. Devoir \og montrer
%     qu'une fonction $F$ est une fonction de répartition \fg{} ne
%     semble donc pas pourvoir se réaliser dans ce cadre.
  \end{noliste}
 \end{remark}~\\[-1.3cm]
\end{proof}

\end{noliste} 

%%% VERSION ROXANE %%%
%%% la version au-dessus est une version "corrigé du DS8vA"
% \subsection*{PARTIE I : Étude d'une variable aléatoire}

% \noindent
% On considère l'application $f:\R\rightarrow\R$ définie, 
% pour tout $t$ de $\R$, par : 
% $f(t)=\dfrac{\ee^{-t}}{(1+\ee^{-t})^2}$.

% \begin{noliste}{1.}
% \setlength{\itemsep}{2mm}
% \item Vérifier que la fonction $f$ est paire.

% \begin{proof}~\\
%  Soit $t\in\R$. On a bien $-t\in\R$.
%  \[
%   \begin{array}{rcl}
%     f(-t) - f(t) &=& \dfrac{\ee^{-(-t)}}{(1+\ee^{-(-t)})^2}
%     - \dfrac{\ee^{-t}}{(1+\ee^{-t})^2}
%     \\[.4cm]
%     &=& \dfrac{\ee^t}{(1+\ee^t)^2} - \dfrac{\ee^{-t}}{(1+\ee^{-t})^2}
%     \\[.6cm]
%     &=& \dfrac{\ee^t(1+\ee^{-t})^2-\ee^{-t}(1+\ee^t)^2}
%     {(1+\ee^t)^2 (1+\ee^{-t})^2}
%     \\[.6cm]
%     &=& \dfrac{\ee^t(1+2\ee^{-t}+\ee^{-2t}) 
%     - \ee^{-t}(1+2\ee^t+\ee^{2t})}
%     {(1+\ee^t)^2 (1+\ee^{-t})^2}
%     \\[.6cm]
%     &=& \dfrac{\bcancel{\ee^t +2+\ee^{-t}} - \bcancel{(\ee^{-t}
%     +2+\ee^t)}}{(1+\ee^t)^2 (1+\ee^{-t})^2}
%     \\[.6cm]
%     &=& 0
%   \end{array}
%  \]
%  Donc $f(-t)=f(t)$.
%  \conc{La fonction $f$ est paire.}
 
%  \begin{remark}
%   La méthode classique pour démontrer qu'une fonction $f$ est paire
%   est de prouver l'égalité :
%   \[
%    \forall x \in \R, \ f(-x)=f(x)
%   \]
%   Lorsque ce calcul direct ne semble pas aboutir, on pensera à 
%   calculer $f(-x)-f(x)$.\\[.2cm]
%   En règle générale, pour démontrer l'égalité \og $a=b$ \fg{}, on peut :
%   \begin{noliste}{$-$}
%    \item partir de $a$ et, par une succession d'égalités, arriver à $b$.
%    \item partir de $b$ et, par une succession d'égalités, arriver à $a$.
%    \item prouver $a=c$, puis $b=c$ par l'une des méthodes précédentes 
%    (méthode du \og mi-chemin \fg{}.
%    \item calculer $a-b$, pour prouver $a-b=0$.
%   \end{noliste}
%   Pour choisir entre les deux premières méthodes, on retiendra qu'il 
% est plus simple de transformer une expression \og compliquée \fg{} en 
% expression \og simple \fg{} que l'inverse.\\
% La dernière méthode est souvent efficace, notamment lorsque les autres 
% ne semblent pas aboutir.
%  \end{remark}~\\[-1.4cm]
% \end{proof}



% \newpage



% \item Montrer que  $f$ est une densité d'une variable aléatoire réelle.

% \begin{proof}~
%  \begin{noliste}{$\sbullet$}
%   \item La fonction $f$ est continue sur $\R$ en tant que 
%   quotient de fonctions continues sur $\R$ dont le dénominateur ne 
%   s'annule pas.\\
%   En effet : soit $t\in\R$.
%   \[
%    \begin{array}{rcl@{\quad}>{\it}R{5cm}}
%     \ee^{-t}>0 & \mbox{donc} & 1+\ee^{-t}>1
%     \\[.2cm]
%     & \mbox{d'où} & (1+\ee^{-t})^2 >1^2
%     & (car la fonction $x\mapsto x^2$ est croissante sur $[0,+\infty[$)
%     \nl
%     \nl[-.2cm]
%     & \mbox{ainsi} & (1+\ee^{-t})^2>0
%    \end{array}
%   \]
%   \conc{La fonction $f$ est continue sur $\R$.}
  
  
%   \item Soit $t\in\R$. $\ee^{-t}>0$ et $(1+\ee^{-t})^2>0$. 
%   \conc{Donc $\forall t \in\R$, $f(t)\geq 0$}
  
%   \item 
%   \begin{noliste}{$-$}
%   \item On sait déjà que $f$ est continue sur $\R$.
  
%   \item Soit $A\geq 0$.
%   \[
%    \dint{0}{A} f(t)\dt = \dint{0}{A} \dfrac{\ee^{-t}}{(1+\ee^{-t})^2} 
%    \dt = \Prim{\dfrac{1}{1+\ee^{-t}}}{0}{A} = \dfrac{1}{1+\ee^{-A}} 
%    -\dfrac{1}{2} \ \tendd{A}{+\infty} \ 1-\dfrac{1}{2}=\dfrac{1}{2}
%   \]
%   Donc $\dint{0}{+\infty} f(t)\dt$ converge et vaut $\dfrac{1}{2}$.
  
%   \item Soit $A\geq 0$.\\
%   On effectue le changement de variable $\Boxed{u = -t}$.
%    \[
%    \left|
%      \begin{array}{P{14cm}}
%        $u = -t$ \quad (et donc $t = -u$) \nl 
%        $\hookrightarrow$ $du = -dt$ \quad et \quad $dt = -du$ \nl
%        \vspace{-.4cm}
%        \begin{noliste}{$\sbullet$}
%        \item $t = -A \ \Rightarrow \ u = A$
%        \item $t = 0 \ \Rightarrow \ u = 0$ %
%          \vspace{-.4cm}
%        \end{noliste}
%      \end{array}
%    \right. %
%    \]
%    Ce changement de variable est valide car $\psi:u \mapsto -u$ est 
%    de classe $\Cont{1}$ sur $[0,A]$.\\
%    On a donc :
%    \[
%     \dint{-A}{0}f(t)\dt = \dint{A}{0}f(-u)(-\ du)
%     =\dint{0}{A}f(-u) \ du
%    \]
%    Or, la fonction $f$ est paire (d'après la question \itbf{1.}), donc :
%    \[
%     \dint{-A}{0}f(t)\dt = \dint{0}{A}f(u) \ du \ \tendd{A}{+\infty} \
%     \dint{0}{+\infty} f(u)\ du=\dfrac{1}{2}
%    \]
%    Donc $\dint{-\infty}{0}f(t)\dt$ converge et vaut $\dfrac{1}{2}$.
%   \end{noliste}
%   \conc{D'où $\dint{-\infty}{+\infty} f(t)\dt$ converge et vaut 
%   $1$}
%  \end{noliste}
 
 
%  \newpage
 
 
%  On a donc montré les trois points suivants :
%  \begin{noliste}{$-$}
%   \item $f$ est continue sur $\R$,
%   \item $\forall t\in \R$, $f(t) \geq 0$,
%   \item l'intégrale $\dint{-\infty}{+\infty} f(t) \dt$ converge et 
%   vaut $1$.
%  \end{noliste}
%  \conc{Finalement, la fonction $f$ est une densité d'une 
%  variable aléatoire réelle.}~\\[-1cm]
% \end{proof}

% \end{noliste}

% \noindent
% Dans toute la suite de l'exercice, on considère une variable aléatoire 
% réelle $X$ à densité, de densité $f$.
% \begin{noliste}{1.}
% \setlength{\itemsep}{2mm}
% \setcounter{enumi}{2}
% \item Déterminer la fonction de répartition de $X$.

% \begin{proof}~\\
%  On note $F_X$ la fonction de répartition de $X$. Soit $x\in\R$.
%  \[
%   F_X(x) = \Prob(\Ev{X\leq x}) = \dint{-\infty}{x} f(t)\dt
%  \]
%  Soit $A\leq x$.
%  \[
%   \dint{A}{x} f(t)\dt = \dint{A}{x} \dfrac{\ee^{-t}}{(1+\ee^{-t})^2}
%   \dt = \Prim{\dfrac{1}{1+\ee^{-t}}}{A}{x} = 
%   \dfrac{1}{1+\ee^{-x}} - \dfrac{1}{1+\ee^{-A}} \ \tendd{A}{-\infty} \
%   \dfrac{1}{1+\ee^{-x}}
%  \]
%  Donc $\dint{-\infty}{x}f(t)\dt$ converge et vaut 
%  $\dfrac{1}{1+\ee^{-x}}$.
%  \conc{$\forall x\in\R$, $F_X(x)=\dfrac{1}{1+\ee^{-x}}$.}~\\[-1cm]
% \end{proof}

% %\newpage

% \item
% \begin{noliste}{a)}
% \item Montrer que l'intégrale $\dint{0}{+\infty}t \, f(t)\dt$ 
% converge.

% \begin{proof}~
%  \begin{noliste}{$\sbullet$}
%   \item La fonction $t\mapsto t\, f(t)$ est continue sur $[0,+\infty[$ 
%   en tant que produit de fonctions continues sur l'intervalle
%   $[0,+\infty[$.

%   \item On remarque :
%   \[
%    \dfrac{t\, f(t)}{\frac{1}{t^2}} = t^2 \, tf(t) 
%    = \dfrac{t^3 \, \ee^{-t}}{(1+\ee^{-t})^2} = 
%    \dfrac{t^3\, \ee^{-t}}{1+2\ee^{-t}+\ee^{-2t}}
%    \ \eq{t}{+\infty} \ \dfrac{t^3 \, \ee^{-t}}{1} = t^3\, \ee^{-t}
%   \]
%   Or, par croissances comparées, $\dlim{t\to+\infty} t^3\ee^{-t}=0$.
%   Donc $\dlim{t\to+\infty} t^2 \, tf(t)=0$.\\[.2cm]
%   On en déduit : $t\, f(t) = \oo{t}{+\infty} 
%   \left(\dfrac{1}{t^2}\right)$.
  
%   \item On sait alors que :
%   \begin{noliste}{$\stimes$}
%     \item $t\, f(t) = \oo{t}{+\infty} \left(\dfrac{1}{t^2}\right)$
%     \item $\forall t\in[1,+\infty[$, $t\, f(t) \geq 0$ et 
%     $\dfrac{1}{t^2}\geq 0$.
%     \item l'intégrale $\dint{1}{+\infty} \dfrac{1}{t^2}\dt$ est une 
%     intégrale de Riemann impropre en $+\infty$, d'exposant $2>1$. Donc 
%     elle converge.
%   \end{noliste}
  
  
%   \newpage
  
  
%   Par critère de négligeabilité des intégrales généralisées de 
%   fonctions continues positives, l'intégrale $\dint{1}{+\infty} t\, 
%   f(t)\dt$ converge.
  
%   \item La fonction $t\mapsto t\, f(t)$ est continue sur le {\bf
%   segment} $[0,1]$. Donc l'intégrale $\dint{0}{1} t\, f(t)\dt$ est
%   bien définie.
%  \end{noliste}
%  \conc{Finalement, l'intégrale $\dint{0}{+\infty} t\, f(t)\dt$ 
%  converge.}
 
%  \begin{remark}
%   L'énoncé demande simplement de déterminer la {\bf nature} de 
%   l'intégrale $\dint{0}{+\infty} t\, f(t)\dt$ (sans la calculer). Il 
%   faut donc privilégier pour cette question l'utilisation d'un critère 
%   de comparaison / équivalence / négligeabilité d'intégrales de 
%   fonctions continues positives.\\
%   On évitera donc le calcul direct de l'intégrale car :
%   \begin{noliste}{\scriptsize 1.}
%    \item il est sans doute difficile,
%    \item il est peut-être même (souvent) impossible.
%   \end{noliste}
%  \end{remark}~\\[-1.4cm]
% \end{proof}



% \item En utilisant l'imparité de la fonction $\R\to 
% \R$, $t\mapsto t \, f(t)$, montrer que $X$ admet une espérance et 
% que l'on a : $\E(X)=0$.

% \begin{proof}~
%  \begin{noliste}{$\sbullet$}
%   \item On note $g$ la fonction $g:t\mapsto t\, f(t)$ définie sur 
%   $\R$.\\ 
%   Soit $t\in\R$. Par parité de $f$ (question \itbf{1.}), on obtient :
%   \[
%    g(-t)=(-t)f(-t)=-tf(-t)=-tf(t)=-g(t)
%   \]
%   Donc la fonction $g$ est impaire.
  
%   \item 
%   \begin{noliste}{$-$}
%     \item La fonction $g$ est continue sur $\R$ en tant que produit de 
%     fonctions continues sur $\R$.
    
%     \item Soit $A\geq 0$.\\
%     On effectue le changement de variable 
%     $\Boxed{u = -t}$.~\\[-1cm]
%     \end{noliste}
%     \[
%       \left|
% 	\begin{array}{P{14cm}}
% 	  $u = -t$ \quad (et donc $t = -u$) \nl 
% 	  $\hookrightarrow$ $du = -dt$ \quad et \quad $dt = -du$ \nl
% 	  \vspace{-.4cm}
% 	  \begin{noliste}{$\sbullet$}
% 	  \item $t = -A \ \Rightarrow \ u = A$
% 	  \item $t = 0 \ \Rightarrow \ u = 0$ %
% 	    \vspace{-.4cm}
% 	  \end{noliste}
% 	\end{array}
%       \right. %
%      \]
%      Ce changement de variable est valide car $\psi : u \mapsto -u$ est 
%      de classe $\Cont{1}$ sur $[0,A]$.
%      \begin{noliste}{}
%      On a donc :
%      \[
%       \dint{-A}{0}g(t)\dt = \dint{A}{0}g(-u)(-\ du)
%       =\dint{0}{A}g(-u) \ du
%      \]
%      Or, la fonction $g$ est impaire, donc : 
%      $\dint{-A}{0}g(t)\dt = \dint{0}{A}-g(u) \ du$.\\[.2cm]
%      Or l'intégrale $\dint{0}{+\infty} g(u)\ du$ converge d'après 
%      la question \itbf{4.a)}. D'où :
%      \[
%       \dint{-A}{0}g(t)\dt = \dint{0}{A}-g(u) \ du \ \tendd{A}{+\infty} \
%       -\dint{0}{+\infty} g(u)\ du
%      \]
%      Donc $\dint{-\infty}{0}g(t)\dt$ converge et vaut 
%      $-\dint{0}{+\infty} g(t)\dt$.
%   \end{noliste}
  
  
  
%   \newpage
  
  
  
%   \item Finalement, $X$ admet une espérance et :
%   \[
%    \E(X) = \dint{-\infty}{+\infty} g(t)\dt
%    =\dint{-\infty}{0}g(t)\dt + \dint{0}{+\infty} g(t)\dt
%    =-\bcancel{\dint{0}{+\infty} g(t)\dt} + \bcancel{
%    \dint{0}{+\infty} g(t)\dt}=0
%   \]
%  \end{noliste}
%  \conc{La \var $X$ admet une espérance et $\E(X)=0$.}~\\[-1cm]
% \end{proof}

% \end{noliste}
% \end{noliste}


% \subsection*{PARTIE II. Étude d'une autre variable aléatoire}

% \noindent
% On considère l'application $\varphi: \R\rightarrow\R$ 
% définie, pour tout $x$ de $\R$, par : $\varphi(x)=\ln(1+\ee^x)$.
% \begin{noliste}{1.}
% \setlength{\itemsep}{2mm}
% \setcounter{enumi}{4}
% \item Montrer que $\varphi$ est une bijection de $\R$ sur un 
% intervalle $I$ à préciser.

% \begin{proof}~
%  \begin{noliste}{$\sbullet$}
%   \item La fonction $\varphi$ est dérivable sur 
%   $\R$ car elle est la composée $h\circ g$ des 
%   fonctions :
%   \begin{noliste}{$\stimes$}
%     \item $g$ : $x\mapsto 1+\ee^x$ dérivable sur 
%     $\R$, et telle que 
%     $g(\R)\subset \ 
%     ]0,+\infty[$.\\ 
%     (pour tout $x\in \ \R$, $1+\ee^x>0$)
    
%     \item $h:y\mapsto \ln(y)$ dérivable sur 
%     $]0,+\infty[$.
%   \end{noliste}
  
  
%   \item Soit $x\in\R$.
%   \[
%    \varphi'(x) = \dfrac{\ee^x}{1+\ee^x} >0
%   \]
%   Donc $\varphi$ est strictement croissante sur $\R$.
%   On obtient le tableau de variations suivant :
  
%   \begin{center}
%       \begin{tikzpicture}[scale=0.8, transform shape]
%         \tkzTabInit[lgt=4,espcl=3] %
%         { %
%         $t$ /1, %
%         Signe de $\varphi'(t)$ /1, %
%         Variations de $\varphi$ /2
%         } %
%         {$-\infty$, $+\infty$} %
%         \tkzTabLine{ , + , } % 
%         \tkzTabVar{-/$0$, +/$+\infty$} %
%       \end{tikzpicture}
%      \end{center}

     
%   Détaillons les éléments de ce tableau.
%   \begin{noliste}{$-$}
%    \item On a déjà : $\dlim{x\to-\infty} (1+\ee^x)=1$. Donc, par 
%    continuité de $y\mapsto \ln(y)$ en $1$ :
%    \[
%     \dlim{x\to-\infty}\varphi(x) = \dlim{x\to-\infty} \ln(1+\ee^x)
%     =\ln(1)=0
%    \]
   
%    \item On sait aussi : $\dlim{x\to+\infty} (1+\ee^x)=+\infty$. Donc :
%    $\dlim{x\to+\infty} \varphi(x)= \dlim{x\to+\infty} \ln(1+\ee^x) =
%    +\infty$.
%   \end{noliste}
  
%   \item La fonction $\varphi$ est :
%   \begin{noliste}{$\stimes$}
%     \item continue sur $]-\infty,+\infty[$ (car dérivable sur cet 
%     intervalle),
%     \item strictement croissante sur $]-\infty,+\infty[$.
%   \end{noliste}
%   Ainsi, $\varphi$ réalise une bijection de $]-\infty,+\infty[$ sur 
%   $I=\varphi(]-\infty,+\infty[)$.
%   \[
%    \varphi(]-\infty,+\infty[) = \left] \dlim{x\to-\infty} \varphi(x),
%    \dlim{x\to+\infty} \varphi(x)\right[ = \ ]0,+\infty[
%   \]
%  \end{noliste}
 
%  \conc{La fonction $\varphi$ réalise une bijection de $\R$ sur 
%  $I= \ ]0,+\infty[$.}~\\[-1cm]
% \end{proof}



% \newpage


% \item Exprimer, pour tout $y$ de $I$, $\varphi^{-1}(y)$.

% \begin{proof}~\\
%  Soit $y\in I$. Soit $x\in\R$.
%  \[
%   \begin{array}{rcl@{\quad}>{\it}R{4cm}}
%    \varphi^{-1}(y)=x
%    & \Leftrightarrow & 
%    y = \varphi(x)
%    \\[.2cm]
%    & \Leftrightarrow &
%    y=\ln(1+\ee^x)
%    \\[.2cm]
%    & \Leftrightarrow & 
%    \ee^y = 1+\ee^x
%    & (car la fonction $x\mapsto \ee^x$ est bijective sur $\R$)
%    \nl
%    \nl[-.2cm]
%    & \Leftrightarrow &
%    \ee^y -1=\ee^x
%    \\[.2cm]
%    & \Leftrightarrow &
%    \ln(\ee^y-1)=x
%    & (car la fonction $x\mapsto \ln(x)$ est bijective sur $]0,+\infty[$)
%   \end{array}
%  \]
%  \conc{$\forall y\in I$, $\varphi^{-1}(y)=\ln(\ee^y-1)$}
 
%  \begin{remark}
%   On remarque que la composition par $\ln$ est bien autorisée car, si 
%   $y\in I= \ ]0,+\infty[$,
%   alors $\ee^y\in \ ]1,+\infty[$. Et donc $\ee^y-1 \in 
%   \ ]0,+\infty[$.
%  \end{remark}~\\[-1.4cm]
% \end{proof}

% \end{noliste}



% \noindent
% On considère la variable aléatoire réelle $Y$ définie par : 
% $Y=\varphi(X)$.
% \begin{noliste}{1.}
% \setlength{\itemsep}{2mm}
% \setcounter{enumi}{6}
% \item Justifier : $\Prob(\Ev{Y\leq 0})=0$.

% \begin{proof}~
%  \begin{noliste}{$\sbullet$}
%  \item On sait que :
%  \begin{noliste}{$\stimes$}
%   \item $X(\Omega) \subset \R$
%   \item $\varphi(\R)= I = \ ]0,+\infty[$ (d'après la question 
%   \itbf{5.})
%  \end{noliste}
%  Donc $Y(\Omega)=(\varphi(X))(\Omega) \subset \ ]0,+\infty[$.
 
%  \item On obtient alors que : $\Ev{Y\leq 0}=\varnothing$. Donc :
%  \[
%   \Prob(\Ev{Y\leq 0})=\Prob(\varnothing)=0
%  \]
%  \end{noliste}
%  \conc{$\Prob(\Ev{Y\leq 0})=0$}~\\[-1cm]
% \end{proof}


% \item Déterminer la fonction de répartition de $Y$.

% \begin{proof}~
%  \begin{noliste}{$\sbullet$}
%   \item Tout d'abord : $Y(\Omega) \subset \ ]0,+\infty[$.
  
%   \item Soit $x\in\R$.
%   \begin{noliste}{$-$}
%     \item \dashuline{Si $x\leq 0$} : Alors $\Ev{Y\leq x}=\varnothing$ 
%     car $Y(\Omega) \subset \ ]0,+\infty[$. Donc, on a :
%     \[
%      F_Y(x)=\Prob(\Ev{Y\leq x})=\Prob(\varnothing)=0
%     \]
    
    
%     \newpage
    
%     \item \dashuline{Si $x>0$}.
%     \[
%      \begin{array}{rcl@{\quad}>{\it}R{4cm}}
%       F_Y(x) &=& \Prob(\Ev{Y\leq x})
%       \\[.2cm]
%       &=& \Prob(\Ev{\varphi(X) \leq x})
%       \\[.2cm]
%       &=& \Prob(\Ev{X \leq \varphi^{-1}(x)})
%       & (car $\varphi$ est strictement croissante, donc 
%       $\varphi^{-1}$ également)
%       \nl
%       \nl[-.2cm]
%       &=& \Prob(\Ev{X\leq \ln(\ee^x-1)})
%       \ = \ F_X(\ln(\ee^x-1))
%       \\[.2cm]
%       &=& \dfrac{1}{1+\exp(\ln(\ee^x-1))}
%       \ = \ \dfrac{1}{\bcancel{1}+\ee^x-\bcancel{1}}
%       \ = \ \dfrac{1}{\ee^x}
%       \\[.4cm]
%       &=& \ee^{-x}
%      \end{array}
%     \]
%   \end{noliste}
  
%   \conc{$F_Y : x \mapsto \left\{
%   \begin{array}{ll}
%    \ee^{-x} & \mbox{ si $x>0$}\\
%    0 & \mbox{ si $x\leq 0$}
%   \end{array}
%   \right.$}~\\[-1.4cm]
%  \end{noliste}
% \end{proof}


% \item Reconnaître alors la loi de $Y$ et donner, sans calcul, son 
% espérance et sa variance.

% \begin{proof}~
%  \conc{On reconnaît une loi exponentielle : $Y\suit \Exp{1}$. On a 
%  alors $\E(Y)=\dfrac{1}{1}=1$ et
%  $\V(Y)=\dfrac{1}{1^2}=1$.}~\\[-.6cm]
% \end{proof}

% \end{noliste}

% \subsection*{PARTIE III : Étude d'une convergence en loi}

% \noindent
% On considère une suite de variables aléatoires réelles $(X_n)_{n\in 
% \N^*}$, mutuellement indépendantes, de même densité $f$, où $f$ 
% a été définie dans la partie I.\\
% On pose, pour tout $n$ de $\N^*$ : $T_n=\max(X_1,\ldots,X_n)$ et 
% $U_n=T_n-\ln(n)$.

% \begin{noliste}{1.}
% \setlength{\itemsep}{2mm}
% \setcounter{enumi}{9}
% \item
% \begin{noliste}{a)}
% \item Déterminer, pour tout $n$ de $\N^*$, la fonction de 
% répartition de $T_n$.

% \begin{proof}~\\
%  Soit $n\in\N^*$.
%  \begin{noliste}{$\sbullet$}
%   \item Soit $x\in\R$. On commence par noter que :
%   \[
%    \Ev{T_n\leq x} = \Ev{X_1 \leq x} \ \cap \ \cdots \ \cap \Ev{X_n\leq 
%    x} = \dcap{i=1}{n} \Ev{X_i \leq x}
%   \]
  
%   \item On obtient alors :
%   \[
%    \begin{array}{rcl@{\qquad}>{\it}R{4.5cm}}
%     F_{T_n}(x) &=& \Prob(\Ev{T_n \leq x})
%     \\[.2cm]
%     &=& \Prob\left(\dcap{i=1}{n} \Ev{X_i \leq x}\right)
%     \\[.2cm]
%     &=& \Prod{i=1}{n} \Prob(\Ev{X_i \leq n})
%     & (car les \var $X_1$, $\hdots$, $X_n$ sont indépendantes)
%     \nl
%     \nl[-.2cm]
%     &=& \Prod{i=1}{n} \Prob(\Ev{X_1\leq 1})
%     & (car les \var $X_1$, $\hdots$, $X_n$ ont même loi)
%     \nl
%     \nl[-.2cm]
%     &=& (\Prob(\Ev{X_1\leq x})^n
%    \end{array}
%   \]
  
%   \item D'après la question \itbf{3.} : $\forall x\in\R$, $F_{X_1}(x)
%   = \dfrac{1}{1+\ee^{-x}}$.
%  \end{noliste}
%  \conc{On en déduit : $\forall x\in\R$, $F_{T_n}(x) =
%  \left(\dfrac{1}{1+\ee^{-x}}\right)^n$.}~\\[-1cm]
% \end{proof}



% \newpage


% \item En déduire : $\forall n\in \N^*, \quad \forall x\in 
% \R, \quad \Prob(\Ev{U_n\leq 
% x})=\left(1+\dfrac{\ee^{-x}}{n}\right)^{-n}$.

% \begin{proof}~\\
%  Soit $n\in\N^*$. Soit $x\in\R$.
%  \[
%   \begin{array}{rcl@{\qquad}>{\it}R{4.5cm}}
%    \Prob(\Ev{U_n\leq x}) &=& \Prob(\Ev{T_n - \ln(n)\leq x})
%    \\[.2cm]
%    &=& \Prob(\Ev{T_n \leq x+\ln(n)})
%    \\[.2cm]
%    &=& F_{T_n}(x+\ln(n))
%    \\[.2cm]
%    &=& \left(\dfrac{1}{1+\exp(-(x+\ln(n)))}\right)^n
%    & (d'après la question \itbf{10.a)})
%    \nl
%    \nl[-.2cm]
%    &=& \left(1+\exp(-x-\ln(n))\right)^{-n}
%    \\[.2cm]
%    &=& \left(1+\dfrac{\ee^{-x}}{\exp(\ln(n))}\right)^{-n}
%    \\[.6cm]
%    &=& \left(1+\dfrac{\ee^{-x}}{n}\right)^{-n}
%   \end{array}
%  \]
%  \conc{$\forall n\in\N^*$, $\forall x\in\R$, $\Prob(\Ev{U_n \leq x})
%  =\left(1+\dfrac{\ee^{-x}}{n}\right)^{-n}$}~\\[-1cm]
% \end{proof}
% \end{noliste}



% \item En déduire que la suite de variables aléatoires 
% $(U_n)_{n\in\N^*}$ converge en loi vers une variable aléatoire 
% réelle à densité dont on précisera la fonction de répartition et une 
% densité.

% \begin{proof}~\\
%  Soit $x\in\R$. On cherche à déterminer $\dlim{n\to+\infty}
%  F_{U_n}(x)=\Prob(\Ev{U_n \leq x})$.
%  \begin{noliste}{$\sbullet$}
%   \item Soit $n\in\N^*$.
%   \[
%    F_{U_n}(x)=\left(1+\dfrac{\ee^{-x}}{n}\right)^{-n}
%    = \exp\left(-n \, \ln\left( 1+\dfrac{\ee^{-x}}{n}\right) 
%    \right)
%   \]
  
%   \item De plus, on sait : $\ln(1+u) \eq{u}{0} u$.\\[.1cm]
%   Or $\dlim{n\to+\infty} \dfrac{\ee^{-x}}{n}=0$, donc 
%   $\ln\left(1+\dfrac{\ee^{-x}}{n}\right) \eqn \dfrac{\ee^{-x}}{n}$.
%   D'où :
%   \[
%    -n \, \ln\left(1+\dfrac{\ee^{-x}}{n}\right) \eqn 
%    -\bcancel{n} \, \dfrac{\ee^{-x}}{\bcancel{n}}=-\ee^{-x}
%   \]
%   \item Or la fonction $x\mapsto \exp(x)$ est continue sur $\R$, donc, 
%   par composition de {\bf limites} :
%   \[
%    \dlim{n\to+\infty} \exp\left(-n \, \ln\left(1+ 
%    \dfrac{\ee^{-x}}{n}\right)\right) = \exp\left(-\ee^{-x}\right)
%   \]
%   D'où :
%   \[
%    \dlim{n\to+\infty} F_{U_n}(x) = \exp(-\ee^{-x})
%   \]
%   \conc{On note $G$ la fonction définie sur $\R$ par $G:x
%   \mapsto \exp(-\ee^{-x})$.\\[.1cm]
%   On a alors $\dlim{n\to+\infty} F_{U_n}(x)=G(x)$.}
  
  
%   \newpage
  
  
  
%   \item Montrons que $G$ est une fonction de répartition.
%   \begin{noliste}{$-$}
%     \item Tout d'abord : $\dlim{x\to+\infty} -\ee^{-x} = 0$. Donc,
%     par continuité de $\exp$ en $0$ : 
%     \[
%      \dlim{x\to+\infty} G(x)= \dlim{x\to+\infty} \exp(-\ee^{-x})=\ee^0=1
%     \]
    
%     \item De plus : $\dlim{x\to-\infty} -\ee^{-x}=-\infty$. D'où :
%     \[
%      \dlim{x\to-\infty} G(x)=\dlim{x\to-\infty} \exp(-\ee^{-x}) =0
%     \]
    
%     \item La fonction $G$ est continue sur $\R$ car elle est la 
%     composée $h_2\circ h_1$ des fonctions :
%     \begin{noliste}{$\stimes$}
%       \item $h_1$ : $x\mapsto -\ee^{-x}$ continue sur 
%       $\R$, et telle que $h_1(\R)\subset \R$.
      
%       \item $h_2:y\mapsto \exp(y)$ continue sur $\R$.
%     \end{noliste}
    
%     \item Elle est dérivable sur $\R$ pour la même raison et, pour 
%     tout $x\in\R$ :
%     \[
%      g(x)=G'(x)=\ee^{-x} \, \exp(-\ee^{-x}) >0
%     \]
%     Donc la fonction $G$ est croissante sur $\R$.
%   \end{noliste}
%   \conc{La fonction $G$ est une fonction de répartition.}
  
  
%   \item Montrons que $G$ est la fonction de répartition d'une \var
%   à densité.
%   \begin{noliste}{$-$}
%     \item On sait déjà que $G$ est continue sur $\R$.
%     \item La fonction $G$ est aussi de classe $\Cont{1}$ sur $\R$ car 
%     les fonctions $h_1$ et $h_2$ sont également de classe $\Cont{1}$
%     sur $\R$.
%   \end{noliste}
%   \conc{La fonction $G$ est la fonction de répartition d'une \var à 
%   densité que l'on notera $V$.}
  
%   \item Pour déterminer une densité de $V$, on dérive la fonction $G$
%   sur $\R$ ($\R=]-\infty,+\infty[$ est bien un intervalle ouvert).\\
%   Donc une densité de $V$ est $g$.
%  \end{noliste}
%  \conc{La suite $(U_n)$ converge en loi vers la \var $V$ de fonction 
%  de répartition $G:x\mapsto \exp(-\ee^{-x})$\\[.1cm] 
%  dont une densité est $g:x\mapsto \ee^{-x} \, \exp(-\ee^{-x})$.}
 
%  \begin{remark}
%   Cette question est assez surprenante. En effet, \og montrer qu'une 
%   fonction $F$ est une fonction de répartition \fg{} est une question 
%   qui ne peut pas être résolue en voie ECE.\\
%   Le programme fournit des propriétés d'une fonction de répartition $F$ 
%   :
%   \begin{noliste}{\scriptsize 1.}
%     \item $F$ est croissante.
%     \item $F$ est continue à droite en tout point.
%     \item $\dlim{x\to+\infty} F(x)=1$.
%     \item $\dlim{x\to-\infty} F(x)=0$.
%   \end{noliste}
%   Il ne précise pas qu'il s'agit d'une 
%   caractérisation d'une fonction de répartition.\\
%   Cette question est donc à la limite du programme. Cependant, il faut 
%   apprendre à la résoudre (c'est-à-dire connaître la caractérisation 
%   ci-dessus), puisque ce type de question apparaît aussi dans le 
%   sujet d'EML 2017.
%  \end{remark}~\\[-1.4cm]
% \end{proof}

% \end{noliste} 







\end{document}
