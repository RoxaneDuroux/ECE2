\chapter*{ECRICOME 2016 : le sujet}
  
%
\vspace*{1cm}

\section*{EXERCICE 1}

\subsection*{Partie A}

\noindent Pour tout couple de réels $(x,y)$, on définit la matrice 
$M(x,y)$ par~:
\[ 
M(x,y) = 
\begin{smatrix} 
3x & -2x+2y & 2x - y \\ 
-x-y & 4x-3y & -2x+y \\
-2y & 4x-4y & -x+y 
\end{smatrix}
\]
On appelle $E$ l'ensemble des matrices $M(x,y)$ où $x$ et $y$ décrivent 
$\R$~:
\[ 
E = \{ M(x,y), \ (x,y) \in \R^2 \} 
\]
On note $A = M(1,0)$ et $B=M(0,1)$.
\begin{noliste}{1.}
 \setlength{\itemsep}{4mm}
 \item Montrer que $E$ est un sous-espace vectoriel de 
 $\M{3}$.\\ 
 En déterminer une base et donner sa 
 dimension.
 
 


 \item Montrer que $1$, $2$ et $3$ sont valeurs propres de $A$ et 
 déterminer les espaces propres associés.\\ 
 $A$ est-elle diagonalisable~?
 
 
 
\item Déterminer une matrice inversible $P$ de $\mathcal{M}_3(\R)$
  dont la première ligne est $\begin{smatrix} 1 & -2 & 1
  \end{smatrix}$, et telle que~:
  \[ 
  A = P D_A P^{-1}, \quad \text{où} \quad D_A = 
  \begin{smatrix} 
    1 & 0 & 0 \\ 
    0 & 2 & 0 \\ 
    0 & 0 & 3 
  \end{smatrix} 
  \]
  
  


  %\newpage


 \item Déterminer $P^{-1}$ (faire figurer le détail des calculs sur la 
 copie).
 
 
 
\item En notant $X_1$, $X_2$ et $X_3$ les trois vecteurs colonnes
  formant la matrice $P$, calculer $BX_1$, $BX_2$ et $BX_3$.  En
  déduire l'existence d'une matrice diagonale $D_B$ que l'on
  explicitera telle que~:
  \[ 
  B = PD_BP^{-1} 
  \]
  
  

 
 \item En déduire que pour tout $(x,y) \in \R^2$, il existe une 
 matrice diagonale $D(x,y)$ de $\M{3}$ telle que~:
 \[ 
 M(x,y) = P D(x,y) P^{-1} 
 \]
 
 
 
 
 %\newpage

 
 \item En déduire une condition nécessaire et suffisante sur $(x,y)$ 
 pour que $M(x,y)$ soit inversible.

 
 
\item Montrer que $B^2$ est un élément de $E$. La matrice $A^2$
  est-elle aussi un élément de $E$ ?
 
 
 \end{noliste}

 \newpage
 
 \subsection*{Partie B}

 \noindent 
 On souhaite dans cette partie étudier les suites $(a_n)_{n 
 \in \N}$, $(b_n)_{n \in \N}$ et $(c_n)_{n \in 
 \N}$ définies par les conditions initiales $a_0=1$, $b_0=0$, 
 $c_0=0$ et les relations de récurrence suivantes~:
 \[ 
  \left\{ 
  \begin{array}{rcrrrrr} 
  a_{n+1} & = & 3 a_n & + & 4 b_n & - & c_n \\ 
  b_{n+1} & = & -4 a_n & - & 5 b_n & + & c_n \\ 
  c_{n+1} & = & -6 a_n & - & 8 b_n & + & 2 c_n 
  \end{array} 
  \right. 
 \]
 Pour tout $n \in \N$, on pose $X_n = 
 \begin{smatrix} 
 a_n \\ 
 b_n \\ 
 c_n 
 \end{smatrix}$.
\begin{noliste}{1.}
\setlength{\itemsep}{2mm}
\setcounter{enumi}{8}
\item Que vaut $X_0$ ?




\item Déterminer une matrice $C$ telle que pour tout $n \in \N$, on
  ait~:
  \[ 
  X_{n+1} = C X_n 
  \]
  Déterminer ensuite deux réels $x$ et $y$ tels que $C = M(x,y)$.
  



%\newpage


\item Montrer que, pour tout $n \in \N$, $X_n = C^n X_0$.

  

\item À l'aide des résultats de la partie A, exprimer $a_n$, $b_n$ et 
$c_n$ en fonction de $n$.



\end{noliste}


%\newpage


\section*{EXERCICE 2}

\begin{noliste}{1.}
  \setlength{\itemsep}{2mm}
\item Pour tout $n \in \N$, on définit la fonction $g_n : [0,+\infty[
  \ \rightarrow \R$ par~:
  \[ 
  g_n(x) = \dfrac{( \ln(1+x))^n}{(1+x)^2} 
  \]

\begin{noliste}{a)}
\item Étudier les variations de la fonction $g_0$, définie sur 
$[0,+\infty[$ par~: $g_0(x) = \dfrac{1}{(1+x)^2}$.\\ 
Préciser la limite de $g_0$ en $+\infty$, donner l'équation de la 
tangente en $0$, et donner l'allure de la courbe représentative de 
$g_0$.





%\newpage



\item Pour $n \geq 1$, justifier que $g_n$ est dérivable sur
  $[0,+\infty[$ et montrer que~:
  \[ 
  \forall x \in [0,+\infty[, \ g_n'(x) \geq 0 \ \Leftrightarrow \ n
  \geq 2\ln(1+x)
  \]
  En déduire les variations de la fonction $g_n$ lorsque $n \geq 1$.\\ 
  Calculer soigneusement $\dlim{x \to + \infty} g_n(x)$.
  
  


\item Montrer que, pour $n \geq 1$, $g_n$ admet un maximum sur 
$[0,+\infty[$ qui vaut~:
\[ 
M_n = \left( \dfrac{n}{2 \ee} \right)^n 
\]
et déterminer $\dlim{n \to + \infty} M_n$.




%\newpage


\item Montrer enfin que pour tout $n \geq 1$~:
  \[
  g_n(x)  = \oox{+\infty} \left( 
    \frac{1}{x^{\frac{3}{2}}} \right) 
  \]

  
\end{noliste}


\newpage


\item On pose pour tout $n \in \N$~:
\[ 
I_n = \dint{0}{+\infty} g_n(t) dt 
\]
\begin{noliste}{a)}
\item Montrer que l'intégrale $I_0$ est convergente et la calculer.




%\newpage


\item Montrer que pour tout entier $n \geq 1$, l'intégrale $I_n$ 
est convergente.




\item À l'aide d'une intégration par parties, montrer que~:
\[ 
\forall n \in \N, \ I_{n+1} = (n+1) I_n 
\]



\item En déduire que~:
  \[ 
  \forall n \in \N, \ I_n = n! 
  \]


\end{noliste}

\item Pour tout $n \in \N$, on définit la fonction $f_n$ par~:
\[ 
\forall x \in \R, \ f_n(x) = \left\{ 
\begin{array}{cl} 
0 & \text{ si } x < 0 \\[.2cm] 
\dfrac{1}{n!} \ g_n(x) & \text{ si } 
x \geq 0 \end{array} \right. 
\]
\begin{noliste}{a)}
\item Montrer que pour tout $n \in \N$, $f_n$ est une densité de
  probabilité.

  
\end{noliste}

\noindent \hspace{-0.5cm} On considère à présent, pour tout $n \in 
\N$, $X_n$ une variable aléatoire réelle admettant $f_n$ pour 
densité.

\noindent \hspace{-0.5cm} On notera $F_n$ la fonction de répartition de 
$X_n$.

\begin{noliste}{a)}
\setcounter{enumii}{1}
\item La variable aléatoire $X_n$ admet-elle une espérance~?




\item Que vaut $F_n(x)$ pour $x < 0$ et $n \in \N$~?

  
  
\item Calculer $F_0(x)$ pour $x \geq 0$.
  
  


%\newpage


\item Soit $x \geq 0$ et $k \in \N^*$. Montrer que~:
  \[ 
  F_k(x) - F_{k-1}(x) = - \dfrac{1}{k!} \dfrac{(\ln(1+x))^k}{1+x} 
  \]




\item En déduire une expression de $F_n(x)$ pour $x \geq 0$ et $n 
\in \N^*$ faisant intervenir une somme (on ne cherchera pas à 
calculer cette somme).




\item Pour $x \in \R$ fixé, déterminer la limite de $F_n(x)$  
lorsque $n$ tend vers $+\infty$.




\item La suite de variables aléatoires $(X_n)_{n \in \N}$ 
converge-t-elle en loi~?



\end{noliste}

\item Pour tout $n \in \N$, on note $Y_n = \ln(1+X_n)$.
\begin{noliste}{a)}
\item Justifier que $Y_n$ est bien définie. Quelles sont les valeurs 
prises par $Y_n$~?





\item Justifier que $Y_n$ admet une espérance et la calculer.




\item Justifier que $Y_n$ admet une variance et la calculer.

  


\item On note $H_n$ la fonction de répartition de $Y_n$. Montrer que~:
\[ 
\forall x \in \R, \ H_n(x) = F_n(\ee^x-1) 
\]




\item Montrer que $Y_n$ est une variable aléatoire à densité et donner
  une densité de $Y_n$.




%\newpage


\item Reconnaître la loi de $Y_0$. À l'aide de ce qui précède,
  déterminer le moment d'ordre $k$ de $Y_0$ pour tout $k \in \N^*$.

  
\end{noliste}
\end{noliste}


\newpage


\section*{EXERCICE 3}

\noindent
Dans tout l'exercice, $X$ et $Y$ sont deux variables aléatoires
définies sur le même espace probabilisé et à valeurs dans~$\N$. On dit
que les deux variables $X$ et $Y$ sont \textbf{échangeables} si~:
\[ 
\forall (i, j) \in \N^2, \quad \Prob(\Ev{X=i} \cap \Ev{Y=j}) = 
\Prob(\Ev{X=j} \cap \Ev{Y=i}) 
\]

\subsection*{Résultats préliminaires}

\begin{noliste}{1.}
\setlength{\itemsep}{2mm}
\item On suppose que $X$ et $Y$ sont deux variables indépendantes et de 
même loi.\\
Montrer que $X$ et $Y$ sont échangeables.




\item On suppose que $X$ et $Y$ sont échangeables.\\
  Montrer, à l'aide de la formule des probabilités totales, que~:
  \[ 
  \forall i \in \N, \quad \Prob(\Ev{X=i}) = \Prob(\Ev{Y=i}) 
  \]
  
  
  
\end{noliste}


%\newpage


\subsection*{\'Etude d'un exemple}

\noindent
Soient $n$, $b$ et $c$ trois entiers strictement positifs.\\
Une urne contient initialement $n$ boules noires et $b$ boules
blanches. On effectue l'expérience suivante, en distinguant trois
variantes.

\begin{noliste}{$\sbullet$}
\item On pioche une boule dans l'urne. \\
  On définit $X$ la variable aléatoire qui vaut $1$ si cette boule est
  noire et $2$ si elle est blanche.
  
\item On replace la boule dans l'urne et~:
  \begin{noliste}{$\star$}
  \item Variante 1 : on ajoute dans l'urne $c$ boules de la même 
    couleur que la boule qui vient d'être piochée.
    
  \item Variante 2 : on ajoute dans l'urne $c$ boules de la 
    couleur opposée à celle de la boule qui vient d'être piochée.
    
  \item Variante 3 : on n'ajoute pas de boule supplémentaire 
    dans l'urne.
  \end{noliste}
  
\item On pioche à nouveau une boule dans l'urne.\\
  On définit $Y$ la variable aléatoire qui vaut $1$ si cette seconde 
  boule piochée est noire et $2$ si elle est blanche.
\end{noliste}

\begin{noliste}{1.}
  \setlength{\itemsep}{2mm}
  \setcounter{enumi}{2}
\item
  \begin{noliste}{a)}
  \item Compléter la fonction \Scilab{} suivante, qui simule le tirage 
    d'une boule dans une urne contenant $b$ boules blanches et $n$ boules 
    noires et qui retourne $1$ si la boule tirée est noire, et $2$ si la 
    boule tirée est blanche.
    
    \begin{scilab}
      & \tcFun{function} \tcVar{res} = tirage(\tcVar{b}, \tcVar{n}) \nl %
      & \quad r = rand() \nl %
      & \quad \tcIf{if} ........... \tcIf{then} \nl %
      & \quad \quad \tcVar{res} = 2 \nl %
      & \quad \tcIf{else} \nl %
      & \quad \quad \tcVar{res} = 1 \nl %
      & \quad \tcIf{end} \nl %
      & \tcFun{endfunction}
    \end{scilab}
    
    


   \newpage


  \item Compléter la fonction suivante, qui effectue l'expérience
    étudiée avec une urne contenant initialement $b$ boules blanches,
    $n$ boules noires et qui ajoute éventuellement $c$ boules après le
    premier tirage,
    selon le choix de la variante dont le numéro est \texttt{variante}.\\
    Les paramètres de sortie sont~:
    \begin{noliste}{-}
    \item \texttt{x} : une simulation de la variable aléatoire $X$
    \item \texttt{y} : une simulation de la variable aléatoire $Y$
    \end{noliste}
    
    \begin{scilab}
      & \tcFun{function} [\tcVar{x}, \tcVar{y}] = experience (\tcVar{b}, 
      \tcVar{n}, \tcVar{c}, \tcVar{variante}) \nl %
      & \quad \tcVar{x} = tirage (\tcVar{b}, \tcVar{n}) \nl %
      & \quad \tcIf{if} \tcVar{variante} == 1 \tcIf{then} \nl %
      & \quad \quad \tcIf{if} \tcVar{x} == 1 \tcIf{then} \nl %
      & \quad \quad \quad ........... \nl %
      & \quad \quad \tcIf{else} \nl %
      & \quad \quad \quad ........... \nl %
      & \quad \quad \tcIf{end} \nl %
      & \quad \tcIf{else if} \tcVar{variante} == 2 \tcIf{then} \nl %
      & \quad \quad ........... \nl %
      & \quad \quad ........... \nl %
      & \quad \quad ........... \nl %
      & \quad \quad ........... \nl %
      & \quad \quad ........... \nl %
      & \quad \tcIf{end} \nl %
      & \quad \tcVar{y} = tirage (\tcVar{b}, \tcVar{n}) \nl %
      & \tcFun{endfunction}
    \end{scilab}
    
   
  \item Compléter la fonction suivante, qui simule l'expérience $N$
    fois (avec $N \in \N^*$), et qui estime la loi de $X$, la loi de
    $Y$ et la loi du couple $(X,Y)$.\\
    Les paramètres de sortie sont :
    \begin{noliste}{-}
    \item \texttt{loiX} : un tableau unidimensionnel à deux éléments qui 
      estime [ $\Prob(\Ev{X=1})$, \ $\Prob(\Ev{X=2})$ ]
    \item \texttt{loiY} : un tableau unidimensionnel à deux éléments qui 
      estime [ $\Prob(\Ev{Y=1})$, \ $\Prob(\Ev{Y=2})$ ]
    \item \texttt{loiXY} : un tableau bidimensionnel à deux lignes et
      deux colonnes qui estime :
      \[ 
      \left[ 
        \begin{array}{cc}
          \Prob(\Ev{X=1} \cap \Ev{Y=1}) &  \Prob(\Ev{X=1} \cap \Ev{Y=2})\\[.2cm]
          \Prob(\Ev{X=2} \cap \Ev{Y=2}) &  \Prob(\Ev{X=1} \cap \Ev{Y=2})  
        \end{array} 
      \right] 
      \]
    \end{noliste}
    \begin{scilab}
      & \tcFun{function} [\tcVar{loiX}, \tcVar{loiY}, \tcVar{loiXY}] =
      estimation(\tcVar{b}, \tcVar{n}, \tcVar{c}, \tcVar{variante},
      \tcVar{N}) \nl %
      & \quad \tcVar{loiX} = [0, 0] \nl %
      & \quad \tcVar{loiY} = [0, 0] \nl %
      & \quad \tcVar{loiXY} = [0, 0 ; 0, 0] \nl %
      & \quad \tcFor{for} k = 1 : \tcVar{N} \nl %
      & \quad \quad [x , y] = experience(\tcVar{b}, \tcVar{n}, \tcVar{c},
      \tcVar{variante}) \nl %
      & \quad \quad \tcVar{loiX}(x) = \tcVar{loiX}(x) + 1 \nl %
      & \quad \quad ........... \nl %
      & \quad \quad ........... \nl %
      & \quad \tcIf{end} \nl %
      & \quad \tcVar{loiX} = \tcVar{loiX} / \tcVar{N} \nl %
      & \quad \tcVar{loiY} = \tcVar{loiY} / \tcVar{N} \nl %
      & \quad \tcVar{loiXY} = \tcVar{loiXY} / \tcVar{N} \nl %
      & \tcFun{endfunction}
    \end{scilab}
    
    
    
    \newpage
    

  \item On exécute notre fonction précédente avec $b=1$, $n=2$, $c=1$,
    $N=10000$ et dans chacune des variantes. On obtient~:
    \[
    \begin{console}
      \lInv{[loiX,loiY,loiXY] = estimation(1,2,1,1,10000)} \nl %
      \lDisp{\qquad loiXY =} \nl %
      \lDisp{\qquad \qquad 0.49837 \qquad 0.16785} \nl %
      \lDisp{\qquad \qquad 0.16697 \qquad 0.16681} \nl %
      \lDisp{\qquad loiY =} \nl %
      \lDisp{\qquad \qquad 0.66534 \qquad 0.33466} \nl %
      \lDisp{\qquad loiX =} \nl %
      \lDisp{\qquad \qquad 0.66622 \qquad 0.33378} \nle %
      %%%%%
      \lInv{[loiX,loiY,loiXY] = estimation(1,2,1,2,10000)} \nl %
      \lDisp{\qquad loiXY =} \nl %
      \lDisp{\qquad \qquad 0.33258 \qquad 0.33286} \nl %
      \lDisp{\qquad \qquad 0.25031 \qquad 0.08425} \nl %
      \lDisp{\qquad loiY =} \nl %
      \lDisp{\qquad \qquad 0.58289 \qquad 0.41711} \nl %
      \lDisp{\qquad loiX =} \nl %
      \lDisp{\qquad \qquad 0.66544 \qquad 0.33456} \nle %
      %%%%%
      \lInv{[loiX,loiY,loiXY] = estimation(1,2,1,3,10000)} \nl %
      \lDisp{\qquad loiXY =} \nl %
      \lDisp{\qquad \qquad 0.44466 \qquad 0.22098} \nl %
      \lDisp{\qquad \qquad 0.22312 \qquad 0.11124} \nl %
      \lDisp{\qquad loiY =} \nl %
      \lDisp{\qquad \qquad 0.66778 \qquad 0.33222} \nl %
      \lDisp{\qquad loiX =} \nl %
      \lDisp{\qquad \qquad 0.66564 \qquad 0.33436} \nle %
    \end{console}
    \] 
    En étudiant ces résultats, émettre des conjectures quant à
    l'indépendance et l'échangeabilité de $X$ et $Y$ dans chacune des
    variantes. \\
    On donne les valeurs numériques approchées suivantes~:
    \[ 
    \begin{array}{l}
      0.33 \times 0.33 \simeq 0.11 \\
      0.33 \times 0.41 \simeq 0.14 \\
      0.33 \times 0.58 \simeq 0.19 \\
      0.33 \times 0.66 \simeq 0.22 \\
      0.41 \times 0.66 \simeq 0.27 \\
      0.58 \times 0.66 \simeq 0.38 \\
      0.66 \times 0.66 \simeq 0.44 
    \end{array} 
    \]
    




%\newpage



\end{noliste}
\item On se place dans cette question dans le cadre de la variante $1$.
\begin{noliste}{a)}
\item Donner la loi de $X$.






\item Déterminer la loi du couple $(X,Y)$.




\item Déterminer la loi de $Y$.




\item Montrer que $X$ et $Y$ sont échangeables mais ne sont pas 
indépendantes.



\end{noliste}
\end{noliste}

