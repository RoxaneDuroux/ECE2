\chapter*{EDHEC 2016 : le sujet}
  
%

\section*{Exercice 1}
\noindent
On désigne par $Id$ l'endomorphisme identité de $\R^3$ et par $I$ la
matrice identité de $\M{3}$. \\
On note $\B = (e_{1},e_{2},e_{3})$ la base canonique de $\R^3$ et on
considère l'endomorphisme $f$ de $\R^3$ dont la matrice dans la base
$\B$ est : $A =
\begin{smatrix}
  3 & -1 & 1\\
  2 & 0 & 2\\
  1 & -1 & 3
\end{smatrix}
$. 
\begin{noliste}{1.}
  \setlength{\itemsep}{4mm}
\item Calculer $A^{2}-4A$ puis déterminer un polynôme annulateur de
  $A$ de degré $2$.

  

\item
  \begin{noliste}{a)}
    \setlength{\itemsep}{2mm}
  \item En déduire la seule valeur propre de $A$ (donc aussi de $f$ ).

    		

  \item La matrice $A$ est-elle diagonalisable ? Est-elle inversible ?
  \end{noliste}

    		

\item Déterminer une base $(u_{1},u_{2})$ du sous-espace propre de $f$
  associé à la valeur propre de $f$.

  

\item 
  \begin{noliste}{a)}
    \setlength{\itemsep}{2mm}
  \item On pose $u_{3} = e_{1} + e_{2} + e_{3}$. Montrer que la
    famille $(u_{1},u_{2},u_{3})$ est une base de $\R^{3}$.

    		


    %\newpage


  \item Vérifier que la matrice $T$ de $f$ dans la base
    $(u_{1},u_{2},u_{3})$ est triangulaire et que ses éléments
    diagonaux sont tous égaux à 2.

    

  \item En écrivant $T = 2I + N$, déterminer, pour tout entier naturel
    $n$, la matrice $T^{n}$ comme combinaison linéaire de $I$ et $N$,
    puis de $I$ et $T$.

    
  \end{noliste}


%\newpage


\item
  \begin{noliste}{a)}
    \setlength{\itemsep}{2mm}
  \item Expliquer pourquoi l'on a :
    \[
    \forall n\in\N, \ A^{n} = n2^{n-1} \ A - (n-1) \ 2^{n} \ I
    \]

    
    
  \item Utiliser le polynôme annulateur obtenu à la première question
    pour déterminer $A^{-1}$ en fonction de $I$ et de $A$.

    

  \item Vérifier que la formule trouvée à la question \itbf{5a} reste
    valable pour $n = -1$.

    
  \end{noliste}
\end{noliste}


%\newpage


\section*{Exercice 2}
\noindent
Pour chaque entier naturel $n$, on définit la fonction $f_{n}$ par :
$\forall x\in[n, + \infty[, \ f_{n}(x) = \dint{n}{x} \ee^{\sqrt{t}} \dt$.
\begin{noliste}{1.}
  \setlength{\itemsep}{4mm}
\item Étude de $f_{n}$.
  \begin{noliste}{a)}
    \setlength{\itemsep}{2mm}
  \item Montrer que $f_{n}$ est de classe $\Cont{1}$ sur $[n, +
    \infty[$ puis déterminer $f'_{n}(x)$ pour tout $x$ de $[n,
    +\infty[$.\\
    Donner le sens de variation de $f_{n}$.

    

  \item En minorant $f_{n}(x)$, établir que $\dlim{x \tend +\infty}
    f_{n}(x) = + \infty$.

    

  \item En déduire que pour chaque entier naturel $n$, il existe un
    unique réel, noté $u_n$, élément de $[n, + \infty[$, tel que
    $f_{n}(u_n) = 1$.

    
  \end{noliste}

\item Étude de la suite $(u_n)$.
  \begin{noliste}{a)}
    \setlength{\itemsep}{2mm}
  \item Montrer que $\dlim{n \tend +\infty} u_n = +\infty$.

    

  \item Montrer que : $\forall n \in \N$, $\ee^{-\sqrt{u_n}} \leq
    u_n-n \leq \ee^{-\sqrt{n}}$.

    
  \end{noliste}
  
  
  \newpage
  

\item
  \begin{noliste}{a)}
    \setlength{\itemsep}{2mm}
  \item Utiliser la question \itbf{2.b)} pour compléter les commandes
    \Scilab{} suivantes afin qu'elles permettent d'afficher un entier
    naturel $n$ pour lequel $u_n-n$ est inférieur ou égal à $10^{-4}$.
    %%% ATTENTION COQUILLE ! C'était inscrit v_n dans l'énoncé
    %%% original (défini juste après...)
    \begin{scilab}
      & n = 0 \nl %
      & \tcFor{while} ------------ \nl %
      & \qquad n = ------------ \nl %
      & \tcFor{end} \nl %
      & disp(n)
    \end{scilab}


    %\newpage


    

  \item Le script affiche l'une des trois valeurs $n = 55$, $n = 70$
    et $n = 85$. \\
    Préciser laquelle en prenant $2,3$ comme valeur approchée de
    $\ln(10)$.

        
  \end{noliste}


  %\newpage


\item On pose $v_n = u_n-n$.
  \begin{noliste}{a)}
    \setlength{\itemsep}{2mm}
  \item Montrer que $\dlim{n \tend +\infty} v_n= 0$.

    

  \item Établir que, pour tout réel $x$ supérieur ou égal à $-1$, on a
    : $\sqrt{1 + x} \leq 1 + \dfrac{x}{2}$.

    


    %\newpage


  \item Vérifier ensuite que : $\forall n \in \N^{*}$,
    $\ee^{-\sqrt{u_n}} \geq \ee^{-\sqrt{n}} \exp\left(
      -\dfrac{v_n}{2\sqrt{n}} \right)$.

    


    %\newpage


  \item Déduire de l'encadrement obtenu en \itbf{2.b)} que : $u_n - n
    \eqn{} \ee^{-\sqrt{n}}$.

    
  \end{noliste}
\end{noliste}

\section*{Exercice 3}

\noindent 
\begin{noliste}{$\sbullet$}
\item Dans cet exercice, toutes les variables aléatoires sont
  supposées définies sur un même espace probabilisé $(\Omega, \A,
  \Prob)$. On désigne par $p$ un réel de $]0,1[$.
\item On considère deux variables aléatoires indépendantes $U$ et $V$,
  telles que $U$ suit la loi uniforme sur $[-3,1]$, et $V$ suit la loi
  uniforme sur $[-1,3]$.
\item On considère également une variable aléatoire $Z$, indépendante
  de $U$ et $V$, dont la loi est donnée par :
  \[
  \Prob(\Ev{Z = 1}) = p \quad \text{ et } \quad \Prob(\Ev{Z = -1}) =
  1-p
  \]
\item Enfin,on note $X$ la variable aléatoire, définie par :
  \[
  \forall \omega \in \Omega, \ X(\omega) = %
  \left\{
    \begin{array}{cR{2.4cm}}
      U(\omega) & si $Z(\omega) = 1$ 
      \nl
      \nl[-.2cm]
      V(\omega) & si $Z(\omega) = -1$
    \end{array}
  \right.
  \]
\item On note $F_X$, $F_U$ et $F_V$ les fonctions de répartition
  respectives des variables $X$, $U$ et $V$.
\end{noliste}

\begin{noliste}{1.}
  \setlength{\itemsep}{4mm}
\item Donner les expressions de $F_U(x)$ et $F_V(x)$ selon les valeurs
  de $x$.

  

\item
  \begin{noliste}{a)}
    \setlength{\itemsep}{2mm}
  \item Établir, grâce au système complet d'évènements $\left( \Ev{Z =
        1}, \Ev{Z = -1} \right)$, que :
    \[
    \forall x\in\R, \ F_X(x) = p \ F_U(x) + (1-p) \ F_V(x)
    \]

    
    
  \item Vérifier que $X(\Omega) = [-3,3]$ puis expliciter $F_X(x)$
    dans les cas :
    \[
    x<-3, \quad -3\leq x\leq-1, \quad -1\leq x\leq1, \quad 1 \leq x
    \leq 3 \quad \text{ et } \quad x>3
    \]

    


    %\newpage

    
  \item On admet que X est une variable à densité. Donner une densité
    $f_{X}$ de la variable aléatoire $X$.

    	    

  \item Établir que $X$ admet une espérance $\E(X)$ et une variance
    $\V(X)$, puis les déterminer.

        
  \end{noliste}

  
  \newpage

  
\item On se propose de montrer d'une autre façon que $X$ possède une
  espérance et un moment d'ordre $2$ puis de les déterminer.

  \begin{noliste}{a)}
    \setlength{\itemsep}{2mm}
  \item Vérifier que l'on a :
    \[
    X = U \ \dfrac{1 + Z}{2} + V \ \dfrac{1-Z}{2}
    \]    

    


    %\newpage


  \item Déduire de l'égalité précédente que $X$ possède une espérance
    et retrouver la valeur de $\E(X)$.

    	    


    %\newpage


  \item En déduire également que $X$ possède un moment d'ordre $2$ et
    retrouver la valeur de $\E(X^{2})$.

    

\end{noliste}

\item
  \begin{noliste}{a)}
    \setlength{\itemsep}{2mm}
  \item Soit $T$ une variable aléatoire suivant la loi de Bernoulli de
    paramètre $p$.\\
    Déterminer la loi de $2T-1$.

    

  \item On rappelle que {\tt grand(1,1,\ttq{}unf\ttq{},a,b)} et {\tt
      grand(1,1,\ttq{}bin\ttq{},p)} sont des commandes \Scilab{}
    permettant de simuler respectivement une variable aléatoire à
    densité suivant la loi uniforme sur $[a,b]$ et une variable
    aléatoire suivant la loi de Bernoulli de paramètre $p$.\\
    Écrire des commandes \Scilab{} permettant de simuler $U$, $V$,
    $Z$, puis $X$.

    
  \end{noliste}
\end{noliste}

\section*{Problème}

\subsection*{Partie I : Questions préliminaires.}

\noindent
Dans cette partie, $x$ désigne un réel élément de $[0,1[$.
\begin{noliste}{1.}
  \setlength{\itemsep}{4mm}
\item
  \begin{noliste}{a)}
    \setlength{\itemsep}{2mm}
  \item Pour tout $n$ de $\N^*$ et pour tout $t$ de $[0,x]$,
    simplifier la somme $\Sum{p = 1}{n} t^{p-1}$.

    


    %\newpage


  \item En déduire que : $\Sum{p = 1}{n}\dfrac{x^{p}}{p} = - \ln(1-x)
    - \dint{0}{x} \dfrac{t^n}{1-t} \dt$.

    

  \item Établir par encadrement que l'on a : $\dlim{n \tend +\infty}
    \dint{0}{x} \dfrac{t^n}{1-t} \dt = 0$.
    

  \item En déduire que : $\Sum{k = 1}{+ \infty}\dfrac{x^k}{k} =
    -\ln(1-x)$.

    

  \end{noliste}


  %\newpage


\item Soit $m$ un entier naturel fixé. À l'aide de la formule du
  triangle de Pascal, établir l'égalité :
  \[
  \forall q \geq m, \ \Sum{k = m}{q} \dbinom{k}{m} = \dbinom{q + 1}{m
    + 1}
  \]

  
  
\item Soit $n$ un entier naturel non nul. On considère une suite
  $(X_n)_{n\in\N^{*}}$ de variables aléatoires, mutuellement
  indépendantes, suivant toutes la loi géométrique de paramètre $x$,
  et on pose $S_n = \Sum{k = 1}{n}X_k$.
  \begin{noliste}{a)}
    \setlength{\itemsep}{2mm}
  \item Déterminer $S_n(\Omega)$ puis établir que, pour tout entier
    $k$ supérieur ou égal à $n + 1$, on a :
    \[
    \Prob(\Ev{S_{n + 1} = k}) \ = \ \Sum{j = n}{k-1} \Prob(\Ev{S_n =
      j} \cap \Ev{X_{n + 1} = k-j})
    \]

    

  \item En déduire, par récurrence sur $n$, que la loi de $S_n$ est
    donnée par :
    \[
    \forall k \in \llb n, + \infty \llb, \ \Prob(\Ev{S_n = k}) =
    \dbinom{k-1}{n-1} \ x^n \ (1-x)^{k-n}
    \]

    

  \item En déduire, pour tout $x$ de $]0,1[$ et pour tout entier
    naturel $n$ non nul :
    \[
    \Sum{k = n}{+ \infty} \dbinom{k-1}{n-1} \ (1-x)^{k-n} =
    \dfrac{1}{x^n}
    \]

    
      

    \newpage


  \item On rappelle que la commande {\tt
      grand(1,n,\ttq{}geom\ttq{},p)} permet à \Scilab{} de simuler $n$
    variables aléatoires indépendantes suivant toutes la loi
    géométrique de paramètre $p$.\\
    Compléter les commandes \Scilab{} suivantes pour qu'elles simulent
    la variable aléatoire $S_n$.\\
    \begin{scilab}
      & n = input(\ttq{}entrez une valeur de n supérieure à 1 :\ttq{}) \nl %
      & S = ------------\nl %
      & disp(S)
    \end{scilab}

    
  \end{noliste}
\end{noliste}

\subsection*{Partie 2 : étude d'une variable aléatoire.}

\noindent
Dans cette partie, on désigne par $p$ un réel de $]0,1[$ et on pose $q
= 1-p$.\\
On considère la suite $(u_k)_{k\in\N^{*}}$, définie par :
\[
\forall k\in\N^*, \ u_k = -\dfrac{q^{k}}{k \ \ln(p)}
\]
\begin{noliste}{1.}
  \setlength{\itemsep}{4mm}
\item
  \begin{noliste}{a)}
    \setlength{\itemsep}{2mm}
  \item Vérifier que la suite $(u_k)_{k\in\N^{*}}$ est à termes
    positifs.

    


    %\newpage


  \item Montrer, en utilisant un résultat de la partie $1$, que
    $\Sum{k = 1}{+ \infty} u_k = 1$.

    
  \end{noliste}
  On considère dorénavant une variable aléatoire $X$ dont la loi de
  probabilité est donnée par :
  \[
  \forall k\in\N^{*}, \ \Prob(\Ev{X = k}) = u_k
  \]
  
\item
  \begin{noliste}{a)}
    \setlength{\itemsep}{2mm}
  \item Montrer que $X$ possède une espérance et la déterminer.

    


    %\newpage


  \item Montrer également que $X$ possède une variance et vérifier que
    : $\V(X) = \dfrac{-q \ (q + \ln(p))}{(p \ \ln(p))^{2}}$.

    
  \end{noliste}

\item Soit $k$ un entier naturel non nul. On considère une variable
  aléatoire $Y$ dont la loi, conditionnellement à l'évènement $\Ev{X =
    k}$, est la loi binomiale de paramètres $k$ et $p$.

  \begin{noliste}{a)}
    \setlength{\itemsep}{2mm}
  \item Montrer que $Y(\Omega) = \N$ puis utiliser la formule des
    probabilités totales, ainsi que la question \itbf{1}) de la partie
    $1$, pour montrer que :
    \[
    \Prob(\Ev{Y = 0}) = 1 + \dfrac{\ln(1 + q)}{\ln(p)}
    \]


    %\newpage


    
    
  \item Après avoir montré que, pour tout couple $(k,n)$ de $\N^*
    \times \N^*$, on a : $\dfrac{\binom{k}{n}}{k} =
    \dfrac{\binom{k-1}{n-1}}{n}$, établir que, pour tout entier
    naturel $n$ non nul, on a :
    \[
    \Prob(\Ev{Y = n}) = -\dfrac{p^{n} \ q^{n}}{n \ \ln(p)} \ \Sum{k =
      n}{+ \infty} \dbinom{k-1}{n-1} \ \left(q^{2} \right)^{k-n}
    \]
    En déduire, grâce à la question \itbf{3)} de la première partie,
    l'égalité :
    \[
    \Prob(\Ev{Y = n}) = -\dfrac{q^{n}}{n \ (1 + q)^{n} \ \ln(p)}
    \]

    
    
  \item Vérifier que l'on a $\Sum{k = 0}{+ \infty}\Prob\left(\Ev{Y =
        k}\right) = 1.$

    
    

    %\newpage


  \item Montrer que $Y$ possède une espérance et donner son expression
    en fonction de $\ln(p)$ et $q$.

    

  \item Montrer aussi que $Y$ possède une variance et que l'on a :
    $\V(Y) = -\dfrac{q \ (q + (1 + q) \ \ln(p))}{(\ln(p))^{2}}$.

    
  \end{noliste}
\end{noliste}

