\chapter*{ESSEC-I 2016 : le corrigé}
  
%

\noindent
On s'intéresse dans ce problème à deux mesures du risque utilisées 
par les marchés financiers.\\
Pour cela, on considère des variables aléatoires sur un espace 
probabilisé $(\Omega,\A,\Prob)$, qui modélisent des pertes 
financières subies par des acteurs économiques sur une période donnée.\\
{\bf Toutes les variables aléatoires définies dans ce problème 
sont des variables aléatoires sur cet espace probabilisé.}\\
Soit ${\cal D}$ l'ensemble des variables aléatoires réelles à 
densité $X$ vérifiant :
\begin{noliste}{$\sbullet$}
  \item $X$ admet une espérance notée $\E(X)$.
  \item il existe un intervalle $I_X$ (dont on admet 
  l'unicité) sur lequel la fonction de répartition de $X$, notée 
  $F_X$, réalise une bijection de classe $\Cont{1}$ strictement 
  croissante de $I_X$ sur $]0,1[$.\\
  On note $G_X$ la bijection réciproque, définie de $]0,1[$ sur 
  $I_X$. Les notations $F_X$ et $G_X$ seront utilisées dans tout 
  le sujet.
\end{noliste}
Dans tout le problème $\beta$ est un réel appartenant à $]0,1[$ et 
représentant un niveau de confiance.

\section*{Partie I - Définition et propriétés de la \og Value at 
Risk \fg}

\begin{noliste}{1.}
  \setlength{\itemsep}{4mm}
  \item Soit $X \in {\cal D}$. Montrer qu'il existe un unique réel 
  $v$ tel que $\Prob(\Ev{X \leq v})=\beta$, et que l'on a 
  $v=G_X(\beta)$.
  
  \begin{proof}~
    \begin{noliste}{$\sbullet$}
      \item La \var $X$ appartient à ${\cal D}$.\\
      Donc $F_X$ réalise une bijection de $I_X$ dans $]0,1[$.\\
      Or $\beta \in \ ]0,1[$.
      \conc{Donc il existe un unique $v\in I_X \subset \R$ tel que 
      $\Prob(\Ev{X \leq v})=F_X(v)=\beta$.}
      
      \item D'après l'énoncé, la fonction $G_X$ est la
      bijection réciproque de $F_X$ sur $I_X$.
      \conc{Donc : $v = G_X(F_X(v))=G_X(\beta)$.}~\\[-1.4cm]
    \end{noliste}
  \end{proof}

\end{noliste}

\begin{noliste}{$\sbullet$}
  \item On définit alors $r_\beta(X)$ appelé la \og Value at 
  Risk \fg au niveau de confiance $\beta$ de $X$, par 
  $r_\beta(X)=G_X(\beta)$. C'est une grandeur qui permet d'évaluer 
  le risque pris par l'acteur qui détient l'actif dont les pertes 
  sont modélisées par $X$.
  
  \item {\bf On remarque que $r_\beta(X)$ est égal au 
  capital minimal qu'il faut détenir pour être en mesure de 
  couvrir les pertes de l'actif associé à $X$ avec une probabilité 
  égale à $\beta$.}
\end{noliste}

\begin{noliste}{1.}
  \setlength{\itemsep}{4mm}
  \setcounter{enumi}{1}
  \item On suppose que, dans cette question, $X$ est une variable 
  aléatoire suivant la loi exponentielle de paramètre $\lambda \in 
  \ ]0,+\infty[$.
  \begin{noliste}{a)}
    \setlength{\itemsep}{2mm}
    \item Rappeler la valeur de $F_X(x)$ pour tout réel $x$.
    
    \begin{proof}~
      \conc{$F_X: x \mapsto \left\{
      \begin{array}{cR{1.5cm}}
        1- \ee^{-\lambda \, x} & si $x\geq 0$
        \nl
        0 & si $x<0$
      \end{array}
      \right.$}~\\[-1cm]
    \end{proof}

    
    \item En déduire que $X \in {\cal D}$ et que l'on a 
    $r_\beta(X)=-\dfrac1\lambda \ln(1-\beta)$.
    
    \begin{proof}~
      \begin{noliste}{$\sbullet$}
	\item La \var $X$ est à densité.
	\item Elle admet une espérance ($\E(X) = \dfrac{1}{\lambda}$).
	
	
	\newpage
	
	
	\item La fonction $F_X$ est :
	\begin{noliste}{$\stimes$}
	  \item continue sur $]0,+\infty[$,
	  \item strictement croissante sur $]0,+\infty[$.\\
	  En effet, la fonction $F_X$ est dérivable sur 
	  $]0+\infty[$ et :
	  \[
	    \forall x \in \ ]0,+\infty[, \ 
	    F_X'(x) = -(-\lambda) \ee^{-\lambda \, x} = 
	    \lambda \, \ee^{-\lambda \, x} >0
	  \]
	\end{noliste}
	Ainsi, $F_X$ réalise une bijection de $]0,+\infty[$ sur 
	$F_X(]0,+\infty[)$.\\
	Comme $F_X$ est une fonction de répartition, on a :
	$\dlim{x\to +\infty} F_X(x)=1$.\\
	D'où :
	\[
	  F_X(]0,+\infty[) = \left] \dlim{x\to 0^+} F_X(x), 
	  \dlim{x\to +\infty} F_X(x) \right[ = \ ]0,1[
	\]
	Ainsi, la fonction $F_X$ réalise une bijection de 
	$]0,+\infty[$ sur $]0,1[$.
      \end{noliste}
      \conc{Finalement : $X \in {\cal D}$.}
      
      \begin{noliste}{$\sbullet$}
	\item Déterminons $G_X$.\\
	Soit $x \in \ ]0,+\infty[$. Soit $\beta \in \ ]0,1[$.
	\[
	  \begin{array}{rcccl}
	    G_X(\beta) = x & \Leftrightarrow & 
	    \beta = F_X(x) & \Leftrightarrow &
	    \beta = 1-\ee^{-\lambda \, x}
	    \\[.2cm]
	    & \Leftrightarrow & 1-\beta = \ee^{-\lambda \, x}
	    & \Leftrightarrow & \ln(1-\beta) = - \lambda \, x
	    \\[.2cm]
	    & \Leftrightarrow & -\dfrac{1}{\lambda} \, 
	    \ln(1-\beta) = x
	  \end{array}
	\]
	Ainsi : $\forall \beta \in \ ]0,1[$, $G_X(\beta) = 
	-\dfrac{1}{\lambda} \, \ln(1-\beta)$.
	\conc{$r_\beta(X) = -\dfrac{1}{\lambda} \, 
	\ln(1-\beta)$}~\\[-1.4cm]
      \end{noliste}
    \end{proof}

  \end{noliste}

  \item On suppose dans cette question que $X$ et $Y$ sont deux 
  variables aléatoires indépendantes suivant la loi normale de 
  paramètres $m$ et $\sigma^2$ pour $X$ et de paramètres $\mu$ et 
  $s^2$ pour $Y$.\\
  On note $\Phi$ la fonction de répartition de la loi normale 
  centrée réduite et $\varphi$ sa densité usuelle.
  \begin{noliste}{a)}
    \setlength{\itemsep}{2mm}
    \item 
    \begin{nonoliste}{(i)}
      \item Justifier que $\Phi$ réalise une bijection de $\R$ sur 
      $]0,1[$. On note $\Phi^{-1}$ la bijection réciproque.
      
      \begin{proof}~\\
          La fonction $\varphi : x \mapsto \dfrac{1}{
	  \sqrt{2 \pi}} \, \ee^{-\frac{x^2}{2}}$ est continue
	  sur $\R$.\\
	  Donc $\Phi$ est de classe $\Cont{1}$ sur $\R$ en tant 
	  que primitive de $\varphi$ et :
	  \[
	    \forall x \in \R, \ \Phi'(x) = \varphi(x) >0
	  \]
	  \conc{Donc la fonction $\Phi$ est strictement croissante 
	  sur $\R$.}
	  En résumé, la fonction $\Phi$ est :
	  \begin{noliste}{$\stimes$}
	    \item continue sur $\R$ (car de classe $\Cont{1}$ 
	    sur $\R$),
	    \item strictement croissante sur $\R$.
	  \end{noliste}
	  Ainsi $\Phi$ réalise une bijection de $\R= \ ]-\infty,
	  +\infty[$ sur $\Phi(]-\infty,+\infty[)$.\\
	  Or $\Phi$ est une fonction de répartition, donc :
	  $\dlim{x\to -\infty} \Phi(x)=0$ et $\dlim{x\to + \infty}
	  \Phi(x)=1$.\\
	  On en déduit :
	  \[
	    \Phi(]-\infty, +\infty[) = \left] \dlim{x\to -\infty}
	    \Phi(x), \dlim{x\to +\infty} \Phi(x) \right[ = \
	    ]0,1[
	  \]
        \conc{Ainsi, la fonction $\Phi$ réalise une bijection de $\R$ 
	sur $]0,1[$.}~\\[-1cm]
      \end{proof}
      
      
      
      \newpage
      

      
      \item Pour tout $x \in\R$, exprimer $F_X(x)$ en fonction de 
      $\Phi$, $m$, $\sigma$ et $x$.
      
      \begin{proof}~
        \begin{noliste}{$\sbullet$}
	  \item D'après l'énoncé : $X \suit \Norm{m}{\sigma^2}$.\\
	  On note $Z$ la \var centrée réduite associée à $X$.
	  Alors :
	  \[
	    Z = \dfrac{X-\E(X)}{\sqrt{\V(X)}} = \dfrac{X-m}{
	    \sigma}
	  \]
	  La \var $Z$ est une transformée affine de $X$ qui 
	  suit une loi normale. Donc $Z$ suit également
	  une loi normale.\\
	  De plus : $\E(Z)=0$ et $\V(Z)=1$.
	  \conc{Donc : $Z \suit \Norm{0}{1}$.}
	  
	  \begin{remark}
	    On rappelle le résultat de cours suivant : $\forall
	    (a,b) \in \R^* \times \R$,
	    \[
	      X \suit \Norm{m}{\sigma^2} \ \Leftrightarrow \
	      aX+b \suit \Norm{am+b}{\sigma^2}
	    \]
	    En particulier : $X \suit \Norm{m}{\sigma^2} \
	    \Leftrightarrow \ X^* = \dfrac{X-m}{\sigma} \suit
	    \Norm{0}{1}$.
	  \end{remark}

	  
	  \item Soit $x\in \R$.
	  \[
	  \begin{array}{rcl}
	    F_X(x) &=& \Prob(\Ev{X \leq x}) \ = \ \Prob(\Ev{
	    X-m \leq x-m})
	    \\[.2cm]
	    &=& \Prob\left(\Ev{\dfrac{X-m}{\sigma} \leq 
	    \dfrac{x-m}{\sigma}}\right) \ = \ \Prob\left(\Ev{
	    Z \leq \dfrac{x-m}{\sigma}}\right)
	    \\[.4cm]
	    &=& \Phi \left(\dfrac{x-m}{\sigma}\right)
	  \end{array}
	  \]
        \end{noliste}
        \conc{$\forall x \in \R$, $F_X(x) = \Phi\left( 
        \dfrac{x-m}{\sigma}\right)$}
        
        \begin{remark}
          La démonstration de ce résultat pouvait également 
          s'effectuer par calcul direct.\\
          Soit $x\in \R$.
          \[
            F_X(x) = \Prob(\Ev{X \leq x}) = \dint{-\infty}{x}
            f_X(t) \dt
          \]
          Or $X \suit \Norm{m}{\sigma^2}$. Donc :
          \[
            f_X(x) = \dfrac{1}{\sigma \, \sqrt{2 \pi}} \,
            \ee^{-\frac{(x-m)^2}{2 \sigma^2}} = \dfrac{1}{\sigma}
            \, \varphi\left(\dfrac{x-m}{\sigma}\right)
          \]
          D'où :
          \[
            F_X(x) = \dint{-\infty}{x} \dfrac{1}{\sigma} \,
            \varphi\left(\dfrac{t-m}{\sigma}\right) \dt
          \]
          \end{remark}
          
          
          \newpage
          
          \begin{remarkL}{.95}
          Deux méthodes de résolution sont alors possibles :\\
          {$\sbullet$} \ \dashuline{Méthode 1}. On remarque : 
	    \[
	      \forall t \in \R, \ \Phi\left( \dfrac{t-m}{\sigma}\right)
	      = \dfrac{1}{\sigma} \, \varphi\left( 
	      \dfrac{t-m}{\sigma}\right)
	    \]
	    Soit $A \in \ ]-\infty,x]$.
	    \[
	      \dint{A}{x} \dfrac{1}{\sigma} \, \varphi\left( 
	      \dfrac{t-m}{\sigma} \right) \dt
	      \ = \ \Prim{\Phi\left( \dfrac{t-m}{\sigma}\right)}
	      {A}{x}
	      \ = \ \Phi \left( \dfrac{x-m}{\sigma} \right) - 
	      \Phi \left( \dfrac{A-m}{\sigma} \right)
	    \]
	    Or, comme $\Phi$ est une fonction de répartition :
	    $\dlim{A \to -\infty} \Phi \left( \dfrac{A-m}{\sigma}
	    \right) =0$.\\
	    D'où : $F_X(x) \ = \ \Phi\left( \dfrac{x-m}{\sigma} 
	  \right)$.
    
   {$\sbullet$} \ \dashuline{Méthode 2}. On effectue le changement de 
    variable 
    $\Boxed{u =
    \dfrac{t-m}{\sigma}}$.
      \[
      \left|
        \begin{array}{P{8cm}}
          $u = \dfrac{t-m}{\sigma}$ (et donc $t = \sigma \, u+m$) \nl
          $\hookrightarrow$ $du = \dfrac{1}{\sigma} \dt$ 
          \quad et \quad $dt = \sigma \ du$ 
          \nl
          \vspace{-.2cm}
          \begin{noliste}{$\sbullet$}
          \item $t = -\infty \ \Rightarrow \ u = -\infty$
          \item $t = x \ \Rightarrow \ u = \dfrac{x-m}{\sigma}$ %
            \vspace{-.4cm}
          \end{noliste}
        \end{array}
      \right.
      \]
      Ce changement de variable est valide car $\psi: u \mapsto 
      \sigma \, u +m$ est de classe $\Cont{1}$ sur $]-\infty, x]$.\\
      On obtient alors :
      \[
        F_X(x) \ = \ \dint{-\infty}{x} \varphi\left(\dfrac{t-m}
        {\sigma}\right) \ \dfrac{1}{\sigma} \dt \ = \
        \dint{-\infty}{\frac{x-m}{\sigma}} \varphi(u) \ du
        \ = \ \Phi\left(\dfrac{x-m}{\sigma}\right)
      \]
        \end{remarkL}~\\[-1.4cm]
      \end{proof}

      
      \item En déduire que $X \in {\cal D}$ et que 
      $r_\beta(X)=m+\sigma \Phi^{-1}(\beta)$.
      
      \begin{proof}~
        \begin{noliste}{$\sbullet$}
	  \item La \var $X$ est à densité.
	  \item Elle admet une espérance ($\E(X)=m$).
	  \item On note $f$ la fonction définie par : 
	  \[
	    f : x \mapsto \dfrac{x-m}{\sigma}
	  \]
	  Alors : $F_X=\Phi \circ f$.\\
	  D'après la question \itbf{3.a)(i)}, on sait que 
	  $\Phi$ réalise une bijection de $\R$ sur $]0,1[$.\\
	  Montrons alors que $f$ réalise une bijection de $\R$ sur 
	  $\R$.\\
	  Soit $y\in \R$. Soit $x\in\R$.
	  \[
	    y=f(x) \ \Leftrightarrow \ y = \dfrac{x-m}{\sigma}
	    \ \Leftrightarrow \ \sigma \, y = x-m \ 
	    \Leftrightarrow \ \sigma \, y+m = x
	  \]
	  Ainsi, la fonction $f$ réalise une bijection de $\R$ sur 
	  $\R$, de 
	  réciproque $f^{-1} : x \mapsto \sigma \, x+m$.\\
	  Finalement, la fonction $F_X$ réalise une bijection de 
	  $\R$ dans $]0,1[$.
	  \conc{Ainsi : $X\in {\cal D}$.}
	  
	  
	  
	  \newpage
	  
	  
	  
	  \item Déterminons $G_X$.\\
	  Tout d'abord : $F_X=\Phi \circ f$. Donc :
	  $G_X=F_X^{-1} = f^{-1} \circ \Phi^{-1}$.\\
	  Soit $\beta \in \ ]0,1[$. On obtient alors :
	  \[
	    G_X(\beta) = f^{-1}(\Phi^{-1}(\beta)) = \sigma \,
	    \Phi^{-1}(\beta) +m
	  \]
	  \conc{Ainsi : $r_{\beta}(X) = m + \sigma \, 
	  \Phi^{-1}(\beta)$.}~\\[-1.4cm]
        \end{noliste}
      \end{proof}
    \end{nonoliste}
    
    \item Quelle est la loi de $X+Y$ ?\\
    En déduire $r_\beta(X+Y)$ en fonction de $m$, $\mu$, $\sigma$, 
    $s$ et $\beta$.
    
    \begin{proof}~
      \begin{noliste}{$\sbullet$}
	\item Les \var $X$ et $Y$ :
	\begin{noliste}{$\stimes$}
	  \item sont indépendantes,
	  \item suivent des lois normales ($X \suit \Norm{m}{
	  \sigma^2}$ et $Y \suit \Norm{\mu}{s^2}$).
	\end{noliste}
	Donc par stabilité des lois normales, $X+Y$ suit une loi 
	normale.\\
	De plus, par linéarité de l'espérance :
	\[
	  \E(X+Y) = \E(X) + \E(Y) = m+\mu
	\]
	Par indépendance de $X$ et $Y$ :
	\[
	  \V(X+Y) = \V(X)+\V(Y) = \sigma^2+s^2
	\]
	\conc{D'où : $X+Y \suit \Norm{m+\mu}{\sigma^2+s^2}$}
	
	\item On applique le résultat de la question \itbf{3.a)}
	pour $m=m+\mu$ et $\sigma^2=\sigma^2+s^2$.
	\conc{On obtient : $r_{\beta}(X+Y) = (m+\mu) + 
	\sqrt{\sigma^2+s^2} \ \Phi^{-1}(\beta)$.}~\\[-1.4cm]
      \end{noliste}
    \end{proof}

    
    \item Pour quels $\beta \in \ ]0,1[$ a-t-on $r_{\beta}(X+Y) \leq
    r_\beta(X)+r_\beta(Y)$ ?
    
    \begin{proof}~\\
      Soit $\beta \in \ ]0,1[$.
      \[
        r_{\beta}(X) + r_{\beta}(Y) = (m+\sigma \, \Phi^{-1}(\beta))
        +(\mu + s \, \Phi^{-1}(\beta)) = (m+\mu) + (\sigma + s)
        \Phi^{-1}(\beta)
      \]
      Donc :
      \[
        \begin{array}{rcl}
          r_{\beta}(X+Y) \leq r_{\beta}(X) + r_{\beta}(Y) 
          & \Leftrightarrow & \bcancel{(m+\mu)} + 
          \sqrt{\sigma^2+s^2} \ \Phi^{-1}(\beta) \leq \bcancel{(m+\mu)} 
          +(\sigma + s) \Phi^{-1}(\beta)
          \\[.4cm]
          & \Leftrightarrow & \sqrt{\sigma^2+s^2} \ \Phi^{-1}(\beta)
          \leq (\sigma +s) \Phi^{-1}(\beta)
        \end{array}
      \]
      Trois cas se présentent alors.
      \begin{noliste}{$\sbullet$}
	\item \dashuline{Si $\Phi^{-1}(\beta)> 0$}. Alors :
	\[
	\begin{array}{rcl@{\quad}>{\it}R{4.5cm}}
	  r_{\beta}(X+Y) \leq r_{\beta}(X) + r_{\beta}(Y)
	  & \Leftrightarrow & \sqrt{\sigma^2+s^2} \leq 
	  \sigma + s
	  \\[.2cm]
	  & \Leftrightarrow & \sigma^2+s^2 \leq (\sigma+s)^2
	  & (car $\sigma>0$, $s>0$ et $x\mapsto x^2$ est 
	  strictement croissante sur $]0,+\infty[$)
	  \nl
	  \nl[-.2cm]
	  & \Leftrightarrow & \xcancel{\sigma^2} + \bcancel{s^2} \leq 
	  \xcancel{\sigma^2} +
	  2\, \sigma \, s + \bcancel{s^2}
	  \\[.2cm]
	  & \Leftrightarrow & 0 \leq \sigma \, s
	\end{array}
	\]
	Or $\sigma>0$ et $s>0$, donc la dernière assertion est vraie.\\
	D'où, par équivalence, la première aussi, c'est-à-dire :
	\[
	  r_{\beta}(X+Y) \leq r_{\beta}(X) + r_{\beta}(Y)
	\]
	
	
	\newpage
	
	
	\item \dashuline{Si $\Phi^{-1}(\beta) <0$}. Alors, 
	avec le même raisonnement que précédemment :
	\[
	  r_{\beta}(X+Y) \leq r_{\beta}(X) + r_{\beta}(Y)
	  \ \Leftrightarrow \ \sqrt{\sigma^2+s^2} \geq 
	  \sigma + s
	  \ \Leftrightarrow \ 0 \geq \sigma \, s
	\]
	Cette dernière assertion est fausse. Donc, par équivalence,
	la première aussi.
	
	\item \dashuline{Si $\Phi^{-1}(\beta)= 0$}. Alors 
	l'inégalité : $\sqrt{\sigma^2+s^2} \ \Phi^{-1}(\beta)
	\leq (\sigma +s) \Phi^{-1}(\beta)$, est trivialement vérifiée.\\
	On a donc encore par équivalence :
	\[
	  r_{\beta}(X+Y) \leq r_{\beta}(X) + r_{\beta}(Y)
	\]
      \end{noliste}
      De plus :
	\[
	  \Phi^{-1}(\beta) =0 \ \Leftrightarrow \ \beta=
	  \Phi(0)=\dfrac{1}{2}
	\]
      \conc{Finalement :
      $r_{\beta}(X+Y) \leq r_{\beta}(X) + r_{\beta}(Y)$ \
      $\Leftrightarrow$ \ $\beta \geq \dfrac{1}{2}$.}~\\[-1.2cm]
    \end{proof}
  \end{noliste}
  
  \item Soit $X$ une variable aléatoire appartenant à ${\cal D}$, 
  $c$ un réel et $\lambda$ un réel strictement positif.\\
  On pose $Y=X+c$ et $Z=\lambda X$ et on admet que $Y$ et $Z$ 
  appartiennent à ${\cal D}$.
  \begin{noliste}{a)}
    \setlength{\itemsep}{2mm}
    \item Montrer que $r_\beta(Y)=r_\beta(X)+c$.
    
    \begin{proof}~
      \begin{noliste}{$\sbullet$}
	\item Déterminons le lien entre $F_Y$ et $F_X$.\\
	Soit $x\in \R$.
	\[
	  F_Y(x) \ = \ \Prob(\Ev{Y \leq x})
	  \ = \ \Prob(\Ev{X+c \leq x})
	  \ = \ \Prob(\Ev{X \leq x-c})
	  \ = \ F_X(x-c)
	\]
	\item De plus, par définition de $r_\beta$ :
	\begin{noliste}{$\stimes$}
	  \item $\beta = F_X(r_\beta(X))$
	  \item $\beta = F_Y(r_\beta(Y))=F_X(r_\beta(Y)-c)$, 
	  d'après la relation précédente.
	\end{noliste}
	D'où :
	\[
	  F_X(r_\beta(X)) = F_X(r_\beta(Y)-c)
	\]
	Or, comme $X\in {\cal D}$, la fonction $F_X$ réalise 
	une bijection de $I_X$ sur $]0,1[$. Donc :
	\[
	  r_\beta(X) = r_\beta(Y)-c
	\]
      \end{noliste}
      \conc{$r_\beta(Y) = r_{\beta}(X)+c$}~\\[-1cm]
    \end{proof}

    
    \item Montrer que $r_\beta(Z)=\lambda \, r_\beta(X)$.
    
    \begin{proof}~
      \begin{noliste}{$\sbullet$}
	\item Déterminons le lien entre $F_Z$ et $F_X$.\\
	Soit $x\in \R$.
	\[
	  F_Z(x) \ = \ \Prob(\Ev{Z \leq x})
	  \ = \ \Prob(\Ev{\lambda \,X \leq x})
	  \ = \ \Prob\left(\Ev{X \leq \dfrac{x}{\lambda}}\right)
	  \ = \ F_X\left(\dfrac{x}{\lambda}\right)
	\]
	\item De plus, par définition de $r_\beta$ :
	\begin{noliste}{$\stimes$}
	  \item $\beta = F_X(r_\beta(X))$
	  \item $\beta = F_Z(r_\beta(Z))=F_X\left( 
	  \dfrac{r_\beta(Z)}{\lambda} \right)$, 
	  d'après la relation précédente.
	\end{noliste}
	
	
	\newpage
	
	
	D'où :
	\[
	  F_X(r_\beta(X)) = F_X\left(\dfrac{r_\beta(Z)}{\lambda}
	  \right)
	\]
	Or, comme $X\in {\cal D}$, la fonction $F_X$ réalise 
	une bijection de $I_X$ sur $]0,1[$. Donc :
	\[
	  r_\beta(X) = \dfrac{r_\beta(Z)}{\lambda}
	\]
      \end{noliste}
      \conc{$r_\beta(Z) = \lambda \, r_{\beta}(X)$}~\\[-1cm]
    \end{proof}
  \end{noliste}
  
  \item Soit $X$ et $Y$ deux variables aléatoires appartenant à
  ${\cal D}$ et telles que pour tout $\omega \in \Omega$, 
  $X(\omega) \leq Y(\omega)$.
  \begin{noliste}{a)}
    \setlength{\itemsep}{2mm}
    \item Comparer, pour tout réel $x$, $F_X(x)$ et $F_Y(x)$.
    
    \begin{proof}~\\
      Soit $x\in \R$. Soit $\omega \in \Omega$.\\
      Supposons $\omega \in \Ev{Y \leq x}$, autrement dit :
      $Y(\omega) \leq x$.\\[.1cm]
      Comme $X(\omega) \leq Y(\omega)$, on obtient : $X(\omega) \leq 
      x$, 
      c'est-à-dire $\omega \in \Ev{X \leq x}$.\\
      On en déduit :
      \[
        \Ev{Y \leq x} \subset \Ev{X \leq x}
      \]
      Donc, par croissance de $\Prob$ :
      \[
        \Prob(\Ev{Y \leq x}) \leq \Prob(\Ev{X \leq x})
      \]
      \conc{Ainsi : $\forall x \in \R$, $F_Y(x) \leq F_X(x)$.}
      
      \begin{remark}
        En général, pour comparer des probabilités, on s'efforcera
        de raisonner dans un premier temps sur les événements.
      \end{remark}~\\[-1.4cm]
    \end{proof}

    
    \item En déduire que $r_\beta(X) \leq r_\beta(Y)$. 
    
    \begin{proof}~\\
      Par définition de $r_\beta$ :
      \[
        F_X(r_\beta(X)) \ = \ \beta \ = \
        F_Y(r_\beta(Y))
      \]
      Or, d'après la question précédente : $F_Y(r_\beta(Y)) \ \leq 
      \ F_X(r_\beta(Y))$. Donc :
      \[
        F_X(r_\beta(X)) \ \leq \ F_X(r_\beta(Y))
      \]
      Or, comme $X \in {\cal D}$, la fonction $F_X$ réalise une 
      bijection strictement croissante de $I_X$ sur $]0,1[$.
      \conc{Ainsi : $r_\beta(X) \leq r_\beta(Y)$.}~\\[-1cm]
    \end{proof}
  \end{noliste}
\end{noliste}


\newpage


\section*{Partie II - Estimation de la valeur de $r_\beta(X)$}
\noindent
Dans la pratique la loi de $X$ n'est pas totalement connue et on a 
besoin d'avoir une idée assez précise de la \og Value at Risk \fg{} 
ne connaissant qu'un certain nombre de valeurs de cette variable.\\
On modélise cette situation en supposant, dans cette partie, que 
la loi de $X$ dépend d'un paramètre $\theta$ inconnu appartenant à 
un sous ensemble $\Theta$ de $\R$ ou $\R^2$, que 
$r_\beta(X)=g(\theta)$ où $g$ est une fonction définie sur 
$\Theta$ et que pour tout $\theta \in \Theta$, $X \in {\cal D}$.\\
On utilise aussi les hypothèses et notations suivantes :
\begin{noliste}{$\sbullet$}
  \item $(X_k)_{k\geq 1}$ est une suite de variables 
  aléatoires réelles appartenant à ${\cal D}$, mutuellement 
  indépendantes, de même loi que $X$.
  
  \item pour tout $\omega \in \Omega$ et $n \in \N^*$, on 
  ordonne $X_1(\omega), \ldots, X_n(\omega)$ dans l'ordre
  croissant et on note alors $X_{1,n}(\omega), \ldots, 
  X_{n,n}(\omega)$ les valeurs obtenues.\\
  En particulier, $X_{1,n}(\omega)$ est la plus petite des valeurs   
  $X_1(\omega), \ldots, X_n(\omega)$ et $X_{n,n}(\omega)$ la plus 
  grande.
  
  \item on admet que pour tout $n \in \N^*$ et $k \in \llb 
  1,n \rrb$, les $X_{k,n}$ sont des variables aléatoires.
  
  \item pour tout réel $x$ et tout entier naturel non nul 
  $n$, on définit la variable aléatoire $N_{x,n}$ ainsi :\\
  pour tout $\omega \in \Omega$, $N_{x,n}(\omega)$ est le nombre 
  d'indices $k$ compris entre $1$ et $n$ tels que l'on ait 
  $X_k(\omega) \leq x$.
\end{noliste}

\begin{noliste}{1.}
  \setlength{\itemsep}{4mm}
  \setcounter{enumi}{5}
  \item Montrer que pour tout $x \in \R$ et tout $n \in \N^*$, 
  $N_{x,n}$ suit une loi binomiale dont on précisera les 
  paramètres. En déduire l'espérance et la variance de $N_{x,n}$.
  
  \begin{proof}~
  \begin{noliste}{$\sbullet$}
    \item Soit $\omega \in \Omega$. Pour déterminer le nombre d'indices 
    $k \in \llb 1,n \rrb$ tels que $X_k(\omega) \leq x$, on teste 
    successivement pour chaque $k \in \llb 1, n \rrb$ si $X_k(\omega)$
    est inférieur à $x$.\\
    On considère alors le test à deux issues suivant :
    \begin{noliste}{$\stimes$}
      \item soit $X$ est inférieur à $x$,
      \item soit $X$ est strictement supérieur à $x$.
    \end{noliste}
    Il définit donc une épreuve de Bernoulli.
    
    \item On considère alors l'expérience consistant en la succession 
    de $n$ épreuves de Bernoulli identiques et indépendantes.
    En effet :
    \begin{noliste}{$\stimes$}
      \item il y a autant d'épreuves que de \var $X_1$, $\hdots$, $X_n$,
      \item ces épreuves sont identiques car $X_1$, $\hdots$, $X_n$ 
      ont même loi,
      \item elles sont indépendantes car $X_1$, $\hdots$, $X_n$ sont 
      indépendantes.
    \end{noliste}
    Le succès de celles-ci est d'obtenir $X$ inférieure 
    à $x$, ce qui se produit avec probabilité :
    \[
      \Prob(\Ev{X \leq x}) = F_X(x)
    \]
    \item La \var $N_{x,n}$ est la \var associée au nombre de 
    succès de cette expérience.\\
    Donc $N_{x,n}$ suit la loi binomiale de paramètres $n$ et $F_X(x)$
    \conc{$N_{x,n} \suit \Bin{n}{F_X(x)}$}
  \end{noliste}
  \conc{On en déduit : $\E(N_{x,n}) = n\, F_X(x)$ et 
  $\V(N_{x,n}) = n\, F_X(x) (1-F_X(x))$.}~\\[-1cm]
  \end{proof}
  
  
  
  \newpage
  


  \item Montrer que pour tout $x \in \R$ et $\eps > 0$ :
  \[
    \dlim{n \to +\infty} \Prob\left(\Ev{ \left\vert 
    \dfrac{N_{x,n}}{n} - F_X(x)\right\vert \geq \eps} 
    \right)=0
  \]
  
  \begin{proof}~\\
    Soit $x\in \R$. Soit $\eps >0$.
    \begin{noliste}{$\sbullet$}
    \item La \var $\dfrac{N_{x,n}}{n}$ admet une variance car $N_{x,n}$
    en admet une.\\
    On peut donc appliquer l'inégalité de Bienaymé-Tchebychev. On
    obtient :
    \[
      \Prob\left(\Ev{ \left\vert \dfrac{N_{x,n}}{n} - 
      \E\left( \dfrac{N_{x,n}}{n} \right) \right\vert \geq \eps}
      \right) \leq \dfrac{\V\left(\frac{N_{x,n}}{n}\right)}
      {\eps^2} \qquad (*)
    \]
    
    \item De plus, par linéarité de l'espérance :
    \[
      \E\left(\dfrac{N_{x,n}}{n} \right) \ = \ \dfrac{\E(N_{x,n})}{n} \ 
      = \ \dfrac{\bcancel{n} \, F_X(x)}{\bcancel{n}} \ = \
      F_X(x)
    \]
    
    Par propriété de la variance :
    \[
      \V\left(\dfrac{N_{x,n}}{n} \right) \ = \ \dfrac{\V(N_{x,n})}{n^2} 
      \ = \ \dfrac{\bcancel{n} \, 
      F_X(x)(1-F_X(x))}{n^{\bcancel{2}}} \ = \
      \dfrac{F_X(x)(1-F_X(x))}{n}
    \]
    
    \item L'inégalité $(*)$ devient donc :
    \[
      \Prob\left(\Ev{ \left\vert \dfrac{N_{x,n}}{n} - 
      F_X(x) \right\vert \geq \eps}
      \right) \leq \dfrac{F_X(x)(1-F_X(x))}
      {n \, \eps^2}
    \]
    
    \item Par propriété d'une probabilité :
    \[
      0 \ \leq \ \Prob\left(\Ev{ \left\vert \dfrac{N_{x,n}}{n} - 
      F_X(x) \right\vert \geq \eps}
      \right) \ \leq \ \dfrac{F_X(x)(1-F_X(x))}
      {n \, \eps^2}
    \]
    
    \item Or : $\dlim{n\to +\infty} \dfrac{F_X(x)(1-F_X(x))}{n
    \, \eps^2} = 0$.
    \conc{Donc, par théorème d'encadrement :\\[.1cm]
    $\dlim{n\to +\infty} \Prob\left(\Ev{ \left\vert 
    \dfrac{N_{x,n}}{n} - F_X(x) \right\vert \geq \eps}
      \right) \ = \ 0$.}~\\[-1.2cm]
  \end{noliste}
  \end{proof}

  
  \item Soit $x \in \R$ et $n \in \N^*$.
  \begin{noliste}{a)}
    \setlength{\itemsep}{2mm}
    \item Montrer que pour tout $k \in\llb 1,n \rrb$, il y a 
    égalité entre les événements $\Ev{X_{k,n} \leq x}$ et 
    $\Ev{N_{x,n} \geq k}$.
    
    \begin{proof}~\\
    Soit $k \in \llb 1,n \rrb$.
      \begin{noliste}{$\sbullet$}
	\item Les \var $X_{1,n}$, $\ldots$, $X_{n,n}$ 
	sont ordonnées dans l'ordre croissant, c'est-à-dire :
	\[
	  \forall \omega \in \Omega, \quad
	  X_{1,n}(\omega) \leq X_{2,n}(\omega) \leq \cdots \leq 
	  X_{k,n}(\omega) \leq 
	  X_{k+1,n}(\omega) \leq \cdots \leq X_{n,n}(\omega)
	\]
	
	\item Or, par définition de $N_{x,n}$, l'événement 
	$\Ev{N_{x,n} \geq k}$ est réalisé si et seulement s'il 
	existe au moins $k$ \var parmi $X_1$, $\ldots$, $X_n$ telles 
	que les événements $\Ev{X_i \leq x}$ soient réalisés.\\
	Donc, comme : $\forall \omega \in \Omega, \ X_{1,n}(\omega) 
	\leq \cdots \leq X_{k,n}(\omega) \leq 
	X_{k+1,n}(\omega) \leq \cdots \leq X_{n,n}(\omega)$ :
	\[
	  \Ev{N_{x,n}\geq k} \ = \ \Ev{X_{1,n} \leq x} \cap
	  \cdots \cap \Ev{X_{k,n}\leq x}
	\]
	
	\item Or : pour tout $i\in \llb 1,k\rrb$, 
	$\Ev{X_{k,n} \leq x} \subset \Ev{X_{i,n} \leq x}$ 
	(car : $\forall \omega \in \Omega, \ X_{i,n}(\omega) \leq 
	X_{k,n}(\omega)$).
	\conc{Donc, finalement : $\forall k \in \llb 1,n \rrb$,
	$\Ev{N_{x,n}\geq k} = \Ev{X_{k,n}\leq x}$.}~\\[-1.6cm]
      \end{noliste}
    \end{proof}
    
    
    
    \newpage
    

    
    \item En déduire que pour tout $k \in \llb 1,n \rrb$ :
    \[
      \Prob(\Ev{X_{k,n} \leq x}) = \Sum{r=k}{n} \dbinom{n}{r} 
      \left(F_X(x) \right)^r\left(1-F_X(x)\right)^{n-r}
    \]
    et que $X_{k,n}$ est une variable aléatoire à densité.
    
    \begin{proof}~\\
    Soit $k \in \llb 1,n \rrb$.
    \begin{noliste}{$\sbullet$}
      \item D'après la question précédente et comme $N_{x,n}(\Omega)
      =\llb 0,n \rrb$ :
      \[
        \Prob(\Ev{X_{k,n} \leq x}) \ = \ \Prob(\Ev{N_{x,n}\geq k})
        \ = \ \Prob\left( \dcup{r=k}{n} \Ev{N_{x,n}=r}\right)
      \]
      Or les événements $\Ev{N_{x,n}=k}$, $\ldots$, 
      $\Ev{N_{x,n}=n}$ sont incompatibles. Donc :
      \[
        \Prob\left(\dcup{r=k}{n} \Ev{N_{x,n}=r}\right) 
        \ = \ \Sum{r=k}{n} \Prob(\Ev{N_{x,n}=r})
      \]
      Or, d'après la question \itbf{6.} : $N_{x,n} \suit 
      \Bin{n}{F_X(x)}$. Donc : $\forall r \in \llb k,n \rrb$,
      \[
        \Prob(\Ev{N_{x,n}=r}) = \dbinom{n}{r} (F_X(x))^r
        (1-F_X(x))^{n-r}
      \]
      \conc{D'où :
      $
        \Prob(\Ev{X_{k,n}\leq x}) = \Sum{r=k}{n} 
        \dbinom{n}{r} (F_X(x))^r (1-F_X(x))^{n-r}
      $}
      
      \item Comme $X \in {\cal D}$, la \var $X$ est une \var à 
      densité.\\
      En particulier, sa fonction de répartition $F_X$ est :
      \begin{noliste}{$\stimes$}
	\item continue sur $\R$,
	\item de classe $\Cont{1}$ sur $\R$ sauf éventuellement 
	en un nombre fini de points.
      \end{noliste}
      Or, d'après ce qui précède :
      \[
        F_{X_{k,n}}(x) = \Prob(\Ev{X_{k,n} \leq x})
        = \Sum{r=k}{n} \dbinom{n}{r} (F_X(x))^r 
        (1-F_X(x))^{n-r}
      \]
      Donc la fonction de répartition $F_{X_{k,n}}$ est :
      \begin{noliste}{$\stimes$}
	\item continue sur $\R$ en tant que somme et 
	produit de fonctions continues sur $\R$,
	\item de classe $\Cont{1}$ sur $\R$ sauf éventuellement
	en un nombre fini de points, pour la même raison.
      \end{noliste}
      \conc{La \var $X_{k,n}$ est une \var à densité.}
    \end{noliste}
    
    \begin{remark}
      Il n'est pas nécessaire, dans cette question, de détailler 
      la continuité et le caractère $\Cont{1}$ sur $\R$ de la fonction
      $F_{X_{k,n}}$ : ce qui est testé ici, c'est la connaissance de la 
      définition d'une \var à densité.
    \end{remark}~\\[-1.4cm]
    \end{proof}
  \end{noliste}

  
  
  \newpage
  
  
  
  \item Soit $(U_n)_{n \in \N^*}$ une suite de variables 
  aléatoires et $c$ un réel. On suppose que pour tout $\eps >0$ :
  \[
    \dlim{n \to +\infty} \Prob\left(\Ev{ \left\vert U_n-c 
    \right\vert \geq \eps} \right)=0
  \]
  On considère $(u_n)_{n \geq 1}$ une suite convergente de réels 
  et on pose $\ell = \dlim{n \to +\infty} u_n$.
  \begin{noliste}{a)}
    \setlength{\itemsep}{2mm}
    \item Établir que : $\dlim{n \to +\infty} 
    \Prob(\Ev{U_n \geq t})=\left\{
    \begin{array}{cR{1.5cm}}
      0 & si $t>c$
      \nl
      1 & si $t<c$ 
    \end{array} 
    \right.$.
    
    \begin{proof}~
      \begin{noliste}{$\sbullet$}
	\item Étudions tout d'abord l'hypothèse de l'énoncé.\\
	Soit $\eps >0$. Soit $n\in \N^*$.
	\[
	  \Prob(\Ev{\vert U_n -c \vert \geq \eps}) \ = \
	  1 - \Prob(\Ev{ \vert U_n-c \vert < \eps})
	\]
      
% 	\item Soit $\eps >0$. Soit $n\in \N^*$.
% 	\[
% 	  \begin{array}{rcl}
% 	    \Prob(\Ev{\vert U_n -c \vert \geq \eps})
% 	    &=& 1 - \Prob(\Ev{ \vert U_n-c \vert < \eps})
% 	    \\[.2cm]
% 	    &=& 1 - \Prob(\Ev{-\eps < U_n -c < \eps})
% 	    \\[.2cm]
% 	    &=& 1 - \Prob(\Ev{c-\eps <U_n < c+\eps})
% 	    \\[.2cm]
% 	    &=& \Prob(\Ev{c-\eps \geq U_n} \cup 
% 	    \Ev{U_n \geq c+\eps})
% 	  \end{array}
% 	\]
	L'hypothèse de l'énoncé apporte donc deux informations.
	\begin{noliste}{$\stimes$}
	  \item Tout d'abord : $\dlim{n\to +\infty} 
	  \Prob(\Ev{\vert U_n-c \vert \geq \eps}) = 0$ (évidemment)
	  \item Ensuite : $\dlim{n\to +\infty} \Prob(\Ev{\vert U_n-c 
	  \vert <\eps}) =1$
	\end{noliste}
	
	\item Soit $t>c$. Alors il existe $\eps_0>0$ tel que :
	$t=c+\eps_0$. Donc : 
	\[
	  \Ev{U_n \geq t} \ = \ \Ev{U_n \geq c+\eps_0} \ = \
	  \Ev{U_n - c \geq \eps_0}
	\]
	Montrons : $\Ev{U_n - c \geq \eps_0} \ \subset \ 
	\Ev{\vert U_n -c \vert \geq \eps_0}$.\\
	Soit $\omega \in \Omega$.\\
	Supposons $\omega \in \Ev{U_n - c \geq \eps_0}$, autrement dit
	$U_n(\omega) -c \geq \eps_0$.\\[.1cm]
	Donc, par croissance de la fonction $x \mapsto \vert x \vert$
	sur $[0,+\infty[$ (on a bien $\eps_0>0$) :
	$\vert U_n(\omega) - c \vert \geq \eps_0$, c'est-à-dire 
	$\omega \in \Ev{\vert U_n - c \vert \geq \eps_0}$.\\
	D'où : $\Ev{U_n - c \geq \eps_0} \ \subset \ 
	\Ev{\vert U_n -c \vert \geq \eps_0}$.
	\conc{Ainsi : $\Ev{U_n \geq t} \ \subset \ \Ev{\vert U_n - 
	c\vert \geq \eps_0}$.}
	
	On en déduit, par propriétés d'une probabilité :
	\[
	  0 \ \leq \ \Prob(\Ev{U_n \geq t}) \ \leq \ 
	  \Prob(\Ev{\vert U_n -c \vert \geq \eps_0})
	\]
	Or, d'après l'énoncé :
	\[
	  \dlim{n\to +\infty} \Prob(\Ev{\vert U_n -c \vert \geq 
	  \eps_0}) \ = \ 0
	\]
	\conc{Donc, par théorème d'encadrement : 
	$\forall t>c$, $\dlim{n\to +\infty}
	\Prob(\Ev{U_n \geq t}) =0$.}
	
	\item Soit $t<c$. Alors il existe $\eps_1>0$ tel que :
	$t=c-\eps_1$. Donc :
	\[
	  \Ev{U_n \geq t} = \Ev{U_n \geq c-\eps_1}
	\]
	Montrons : $\Ev{\vert U_n -c \vert < \eps_1} \ \subset \ 
	\Ev{U_n \geq c-\eps_1}$.\\
	Soit $\omega \in \Omega$.\\
	Supposons $\omega \in \Ev{\vert U_n -c\vert < \eps_1}$, 
	autrement dit $\vert U_n(\omega) -c \vert < \eps_1$. Or :
	\[
	   \vert U_n(\omega) -c \vert < \eps_1
	  \ \Leftrightarrow \ -\eps_1 < U_n(\omega) -c < \eps_1
	  \ \Leftrightarrow \ c-\eps_1 < U_n < c+\eps_1
	\]
	En particulier : $U_n(\omega) \geq c-\eps_1$, c'est-à-dire 
	$\omega \in \Ev{U_n \geq c-\eps_1}$.\\
	D'où : $\Ev{\vert U_n - c \vert < \eps_1} \ \subset \
	\Ev{U_n \geq c-\eps_1}$
	\conc{Ainsi : $\Ev{\vert U_n -c \vert < \eps_1} \ 
	\subset \ \Ev{U_n \geq t}$.}
	
	
	
	\newpage
	
	
	
	D'où, par propriétés d'une probabilité :
	\[
	  \Prob(\Ev{\vert U_n -c\vert < \eps_1}) \ \leq \
	  \Prob(\Ev{U_n \geq t}) \ \leq \ 1
	\]
	Or, d'après l'énoncé :
	\[
	  \dlim{n\to +\infty} \Prob(\Ev{\vert U_n - c \vert < \eps_1}) 
	  = 1
	\]
	\conc{Donc, par théorème d'encadrement :
	$\forall t<c$, $\dlim{n\to +\infty}
	\Prob(\Ev{U_n \geq t}) =1$.}
      \end{noliste}
      \conc{Finalement : $\dlim{n\to +\infty} \Prob(\Ev{U_n \geq 
      t}) = \left\{
      \begin{array}{cR{1.5cm}}
        0 & si $t>c$
        \nl
        1 & si $t<c$
      \end{array}
      \right.$.}~\\[-1cm]
    \end{proof}

    
    \item On suppose $\ell >c$ et on pose $\eps = 
    \dfrac{\ell-c}{2}$.\\
    En remarquant que $\ell-\eps=c+\eps$, montrer qu'à 
    partir d'un certain rang, $u_n \geq c+\eps$.\\
    En déduire que $\dlim{n \to +\infty}\Prob(\Ev{U_n \geq 
    u_n})=0$. 
    
    \begin{proof}~
      \begin{noliste}{$\sbullet$}
	\item D'après l'énoncé $\dlim{n\to+\infty}
	u_n = \ell$.\\
	Donc il existe $N\in \N$ tel que, pour tout $n\geq N$ :
	$\vert u_n - \ell \vert \leq \eps$.\\
	Or, pour tout $n \geq N$ :
	\[
	  \vert u_n - \ell \vert \leq \eps \ \Leftrightarrow \
	  -\eps \leq u_n - \ell \leq \eps \ \Leftrightarrow \
	  \ell - \eps \leq u_n \leq \ell + \eps
	\]
	En particulier, pour tout $n \geq N$ :
	\[
	  u_n \geq \ell - \eps = c+\eps
	\]
	\conc{D'où, à partir du rang $N$ : $u_n \geq c +\eps$.}
	
	\item Soit $n\geq N$. Alors : $u_n \geq c +\eps$.\\
	Donc :
	\[
	  \Ev{U_n \geq u_n} \ \subset \ \Ev{U_n \geq c+\eps}
	\]
	D'où, d'après les propriétés d'une probabilité :
	\[
	  0 \ \leq \ \Prob(\Ev{U_n \geq u_n}) \ \leq \ 
	  \Prob(\Ev{U_n \geq c+\eps})
	\]
	Or, d'après la question \itbf{9.a)} : 
	$\dlim{n\to +\infty} \Prob(\Ev{U_n \geq c+\eps}) =0$.
	\conc{Donc, par théorème d'encadrement :
	$\dlim{n\to +\infty} \Prob(\Ev{U_n \geq u_n})=0$.}
      \end{noliste}
      
      \begin{remark}
      \begin{noliste}{$\sbullet$}
        \item On utilise ici la définition de la 
        convergence d'une suite $(u_n)$ vers $\ell$ :
        \[
          \text{($(u_n)$ converge vers $\ell \in \R$)} \ 
          \Leftrightarrow \ (\forall \eps >0, \ \exists N \in 
          \N, \ (n \geq N) \Rightarrow (\vert u_n - \ell \vert
          \leq \eps))
        \]
        
        \item Démontrons : $\ell - \eps = c+\eps$.\\
        D'une part :
        \[
          \ell - \eps \ = \ \ell - \dfrac{\ell - c}{2} \ = \
          \dfrac{2\ell -(\ell -c)}{2} \ = \ 
          \dfrac{\ell + c}{2}
        \]
        D'autre part :
        \[
          c+ \eps \ = \ c + \dfrac{\ell - c}{2} \ = \
          \dfrac{2c +(\ell -c)}{2} \ = \ 
          \dfrac{\ell + c}{2}
        \]
        On a donc bien : $\ell - \eps = c+\eps$.
      \end{noliste}
      \end{remark}~\\[-1.4cm]
    \end{proof}
    
    
    \newpage

    
    \item Montrer de même que si $\ell <c$, $\dlim{n 
    \to +\infty} \Prob(\Ev{U_n \geq u_n})=1$. 
    
    \begin{proof}~\\[.1cm]
    On pose $\eps'=\dfrac{c-\ell}{2}$.
      \begin{noliste}{$\sbullet$}
	\item D'après l'énoncé $\dlim{n\to+\infty}
	u_n = \ell$.\\
	Donc il existe $N'\in \N$ tel que, pour tout $n \geq N'$ :
	\[
	  u_n \leq \ell + \eps'
	\]
	\conc{D'où, puisque $\ell + \eps' = c- \eps'$, à partir du rang 
	$N'$ : $u_n \leq c -\eps'$.}
	
	\item Soit $n\geq N'$. Alors : $u_n \leq c -\eps'$.
	Donc :
	\[
	  \Ev{U_n \geq c-\eps'} \subset \Ev{U_n \geq u_n}
	\]
	D'où, d'après les propriétés d'une probabilité :
	\[
	  \Prob(\Ev{U_n \geq c-\eps'}) \ \leq \ 
	  \Prob(\Ev{U_n \geq u_n}) \ \leq \ 1
	\]
	Or, d'après la question \itbf{9.a)} : 
	$\dlim{n\to +\infty} \Prob(\Ev{U_n \geq c-\eps'}) =1$.
	\conc{Donc, par théorème d'encadrement, si $\ell<c$ :
	$\dlim{n\to +\infty} \Prob(\Ev{U_n \geq u_n})=1$.}
      \end{noliste}
      
      \begin{remark}
        Détaillons l'égalité $\ell + \eps' = c-\eps'$.\\
        D'une part :
        \[
          \ell + \eps' \ = \ \ell + \dfrac{c-\ell}{2} \ = \
          \dfrac{2\ell +(c-\ell)}{2} \ = \ 
          \dfrac{\ell + c}{2}
        \]
        D'autre part :
        \[
          c- \eps' \ = \ c - \dfrac{c-\ell}{2} \ = \
          \dfrac{2c -(c-\ell)}{2} \ = \ 
          \dfrac{\ell + c}{2}
        \]
        On a donc bien : $\ell + \eps' = c-\eps'$.
      \end{remark}~\\[-1.4cm]
    \end{proof}
  \end{noliste}

  \item On définit pour tout $n \in \N^*$ tel que $n \beta \geq 1$, 
  la variable aléatoire $Y_n$ sur $(\Omega,\A,\Prob)$ par 
  $Y_n=X_{\lfloor n\beta \rfloor, n}$ où $\lfloor n\beta \rfloor$ 
  désigne la partie entière de $n\beta$ et on pose 
  $\theta'=r_{\beta}(X)$.\\
  Soit $\eps >0$.
  \begin{noliste}{a)}
    \setlength{\itemsep}{2mm}
    \item Montrer que : $\Prob(\Ev{|Y_n-\theta'| \leq
    \eps})=\Prob(\Ev{Y_n \leq \theta'+\eps}) - \Prob(\Ev{Y_n \leq
    \theta'-\eps})$. 
    
    \begin{proof}~\\
      On a les égalités suivantes :
      \[
        \begin{array}{rcl@{\quad}>{\it}R{6cm}}
          & & \Prob(\Ev{\vert Y_n - \theta' \vert \leq \eps})
          \\[.2cm]
          &=& \Prob(\Ev{-\eps \leq Y_n -\theta' \leq \eps})
          \\[.2cm]
          &=& \Prob(\Ev{\theta' -\eps \leq Y_n \leq \theta' 
          +\eps})
          \\[.2cm]
          &=& \Prob(\Ev{Y_n \leq \theta' +\eps} \setminus 
          \Ev{Y_n < \theta'-\eps})
          \\[.2cm]
          &=& \Prob(\Ev{Y_n \leq \theta' +\eps}) - 
          \Prob(\Ev{Y_n \leq \theta'+\eps} \cap 
          \Ev{Y_n < \theta' -\eps})
          \\[.2cm]
          &=& \Prob(\Ev{Y_n \leq \theta' +\eps}) -
          \Prob(\Ev{Y_n < \theta'-\eps}) 
          & (car $\Ev{Y_n < \theta'-\eps} \subset 
          \Ev{Y_n \leq \theta' +\eps}$)
          \nl
          \nl[-.2cm]
          &=& \Prob(\Ev{Y_n \leq \theta' +\eps}) - 
          \Prob(\Ev{Y_n \leq \theta' -\eps})
          & (car, d'après \itbf{8.b)}, 
          $X_{\lfloor n\, \beta \rfloor, n}$ est une \var 
          à densité)
        \end{array}
      \]
      \conc{$\Prob(\Ev{ \vert Y_n - \theta' \vert \leq \eps})
      = \Prob(\Ev{ Y_n \leq \theta'+\eps}) - 
      \Prob(\Ev{Y_n \leq \theta'-\eps})$}~\\[-1cm]
    \end{proof}
    
    
    
    \newpage

    
    \item En déduire que :
    \[
     \Prob( \Ev{ |Y_n- \theta' | \leq\eps} )=\Prob\left( \Ev{ 
     \dfrac{N_{\theta'+\eps,n}}{n} \geq \dfrac{\lfloor n\beta 
     \rfloor}{n}} \right)-\Prob\left( \Ev{ 
     \dfrac{N_{\theta'-\eps,n}}{n} \geq \dfrac{\lfloor n\beta 
     \rfloor}{n}} \right)
    \]
    
    \begin{proof}~
      \begin{noliste}{$\sbullet$}
	\item Par définition de la \var $Y_n$ :
	\[
	  \begin{array}{rcl@{\quad}>{\it}R{4.5cm}}
	    \Prob(\Ev{Y_n \leq \theta'+\eps}) &=& 
	    \Prob(\Ev{X_{\lfloor n\, \beta \rfloor,n} \leq 
	    \theta'+\eps})
	    \\[.4cm]
	    &=& \Prob(\Ev{N_{\theta'+\eps,n} \geq 
	    \lfloor n\, \beta \rfloor})
	    & (d'après la question \itbf{8.a)})
	    \nl
	    \nl[.2cm]
	    &=& \Prob\left(\Ev{ \dfrac{N_{\theta' +\eps,n}}{n}
	    \geq \dfrac{\lfloor n \, \beta \rfloor}{n}}
	    \right)
	    & (car $n>0$)
	  \end{array}
	\]
	
	\item De même :
	\[
	  \Prob(\Ev{Y_n \leq \theta' -\eps}) \ = \
	  \Prob\left( \Ev{ \dfrac{N_{\theta' -\eps,n}}{n} \geq 
	  \dfrac{\lfloor n\, \beta \rfloor}{n}}\right)
	\]
	
	\item Or, d'après la question précédente :
	\[
	  \Prob(\Ev{ \vert Y_n - \theta' \vert \leq \eps})
	  \ = \ \Prob(\Ev{Y_n \leq \theta' +\eps}) -
	  \Prob(\Ev{Y_n \leq \theta' - \eps})
	\]
	\conc{Donc : $\Prob(\Ev{\vert Y_n - \theta' \vert \leq 
	\eps }) \ = \ \Prob\left( \Ev{ \dfrac{N_{\theta'
	+\eps,n}}{n} \geq \dfrac{\lfloor n\, \beta \rfloor}{n}}
	\right) - \Prob\left( \Ev{\dfrac{N_{\theta'-\eps,n}}{n}
	\geq \dfrac{\lfloor n\, \beta \rfloor}{n}}\right)$.
	}~\\[-1.2cm]
      \end{noliste}
    \end{proof}
    
    \item En déduire : $\dlim{n \to +\infty} 
    \Prob(\Ev{|Y_n - \theta'| \leq \eps})=1$.\\
    Que peut-on en déduire concernant l'estimateur $Y_n$ de 
    $r_\beta(X)$ ?
    
    \begin{proof}~\\
    Soit $n\in \N^*$.
      \begin{noliste}{$\sbullet$}
	\item D'après le résultat obtenu à la question 
	précédente, on souhaite appliquer les résultats des questions 
	\itbf{9.b)} et \itbf{9.c)} à :
	\begin{noliste}{\scriptsize 1)}
	  \item $U_n = \dfrac{N_{\theta'+\eps,n}}{n}$ et $u_n = 
	  \dfrac{\lfloor n \, \beta \rfloor}{n}$
	  \item $U_n = \dfrac{N_{\theta'-\eps,n}}{n}$ et $u_n = 
	  \dfrac{\lfloor n \, \beta \rfloor}{n}$
	\end{noliste}
	Commençons par le premier cas.\\
	On démontre que toutes les hypothèses 
	d'application des questions \itbf{9.b)} ou \itbf{9.c)} sont 
	vérifiées.
	\begin{noliste}{$\stimes$}
	  \item D'après la question \itbf{7.} :
	  \[
	    \dlim{n\to +\infty} \Prob\left( \Ev{ \left\vert
	    \dfrac{N_{\theta' + \eps,n}}{n} - F_X(\theta' +\eps) 
	    \right\vert \geq \eps}\right) = 0
	  \]
	  \conc{En choisissant $c=F_X(\theta'+\eps)$ et $U_n =
	  \dfrac{N_{\theta' +\eps,n}}{n}$, on a bien :\\[.2cm]
	  $\dlim{n\to +\infty} \Prob(\Ev{\vert U_n - c \vert \geq 
	  \eps}) =0$.}
	  
	  
	  
	  \newpage
	  
	  
	  
	  \item Pour tout $n\in \N^*$, on pose : $u_n = 
	  \dfrac{\lfloor n\, \beta \rfloor}{n}$.\\
	  Montrons maintenant que $(u_n)$ converge 
	  et déterminons sa limite.\\
	  Soit $n\in\N^*$. Par définition de la partie entière :
	  \[
	    \lfloor n\, \beta \rfloor \ \leq \ n\, \beta \
	    < \ \lfloor n\, \beta \rfloor +1
	  \]
	  Donc :
	  \[
	    n\, \beta -1 \ < \ \lfloor n\, \beta \rfloor \leq n\, \beta
	  \]
	  Comme $n>0$, on obtient :
	  \[
	    \beta - \dfrac{1}{n} \ < \ \dfrac{\lfloor n\, \beta 
	    \rfloor}{n} \ \leq \ \beta
	  \]
	  Or : $\dlim{n\to +\infty} \left(\beta + \dfrac{1}{n} \right)
	  = \beta$.
	  \conc{Donc, par théorème d'encadrement : $\dlim{n\to 
	  +\infty} u_n=\beta$.\\[.1cm]
	  On choisit alors : $\ell = \beta$.}
	  
	  \item Pour savoir si l'on applique la question \itbf{9.b)}
	  ou la question \itbf{9.c)}, il faut maintenant comparer les 
	  réels $c$ et $\ell$.\\
	  Par définition de $r_\beta(X)$ :
	  \[
	    \ell < c \ \Leftrightarrow \ \beta < F_X(\theta' +\eps)
	    \ \Leftrightarrow \ F_X(r_\beta(X)) < F_X(\theta' +\eps)
	  \]
	  Or, comme $X\in {\cal D}$, la fonction $F_X$ est une 
	  bijection strictement croissante de $I_X$ sur $]0,1[$. Donc :
	  \[
	    \ell < c \ \Leftrightarrow r_\beta(X) < \theta'+\eps
	  \]
	  Or $\theta'=r_\beta(X)$ et $\eps >0$, donc la dernière 
	  assertion est vraie.
	  \conc{Par équivalence, on obtient donc : $\ell < c$.}
	  
	  \item On est donc ici dans le cadre d'application de la 
	question \itbf{9.c)}. Donc :
	\[
	  \dlim{n\to +\infty} \Prob(\Ev{U_n \geq u_n})=1
	\]
	\conc{On en déduit : $\dlim{n\to+\infty} \Prob\left( \Ev{ 
	\dfrac{N_{\theta' + \eps,n}}{n} \geq \dfrac{\lfloor n\, \beta 
	\rfloor}{n}}\right) = 1$.}
	
	\item On démontrerait de même, avec la question \itbf{9.b)}, en 
	choisissant :
	\[
	  U_n = \dfrac{N_{\theta'-\eps,n}}{n}, \quad u_n = 
	  \dfrac{\lfloor n \, \beta \rfloor}{n}, \quad c = F_X(\theta'
	  -\eps) \quad \text{et} \quad \ell = \beta
	\]
	la convergence suivante :
	\[
	  \dlim{n\to +\infty} \Prob(\Ev{U_n \geq u_n})=0
	\]
	\conc{On en déduit : $\dlim{n\to +\infty} \Prob\left( \Ev{ 
	\dfrac{N_{\theta' - \eps,n}}{n} \geq \dfrac{\lfloor n\, \beta 
	\rfloor}{n}}\right) = 0$.}
	\end{noliste}
	\conc{D'après la question précédente, on obtient donc :\\
	$ \dlim{n\to +\infty} \Prob(\Ev{ \vert Y_n - \theta'\vert 
	\leq \eps}) = 1-0=1$.}
	
	
	
	\newpage
	
	
	
	\item On sait :
	\[
	  \Prob(\Ev{\vert Y_n - \theta' \vert >\eps}) = 
	  1-\Prob(\Ev{\vert Y_n - \theta'\vert \leq \eps})
	\]
	Donc, d'après ce qui précède :
	\[
	  \dlim{n\to +\infty} \Prob(\Ev{\vert Y_n - \theta' 
	  \vert >\eps}) = 0
	\]
	D'où $(Y_n)$ converge en probabilité vers $\theta'$.
	\conc{On en déduit que $Y_n$ est un estimateur 
	convergent de $\theta'= r_\beta(X)$.}
      \end{noliste}
      
      \begin{remark}
        Il peut être intéressant de garder en tête le résultat 
        suivant : $\lfloor x \rfloor \eqx{+\infty} x$.\\
        Ce résultat est à savoir redémontrer. Détaillons 
        cette démonstration.\\
        Soit $x\in \ ]0,+\infty[$.\\
        Par définition de la partie entière : $\lfloor x 
        \rfloor \leq x < \lfloor x \rfloor +1$. Donc :
        \[
          x-1< \lfloor x \rfloor \leq x
        \]
        D'où, comme $x >0$ :
        \[
          1-\dfrac{1}{x} < \dfrac{\lfloor x \rfloor}{x} 
          \leq 1
        \]
        Or : $\dlim{x\to +\infty} \left(1-\dfrac{1}{x}\right) 
        =1$.\\
        Ainsi, par théorème d'encadrement : $\dlim{x \to +\infty}
        \dfrac{\lfloor x \rfloor}{x}=1$. C'est-à-dire : 
        $\lfloor x \rfloor \eqx{+\infty} x$.
      \end{remark}~\\[-1.4cm]
    \end{proof}
  \end{noliste}
  
  \item On suppose que l'on a défini un fonction d'en-tête {\tt 
  function R = triCroissant(T)} qui renvoie le tableau des valeurs 
  se trouvant dans {\tt T} rangées dans l'ordre croissant.\\ 
  Par exemple, si {\tt T=[0 -1 0 2 4 2 3]} alors {\tt 
  disp(triCroissant(T))} affiche :
  \[
  \begin{console}
    \lDisp{\qquad ans \ =} \nl %
    \lDisp{\qquad \qquad -1. \quad 0. \quad 0. \quad 2. \quad 2. \quad 
    3. \quad 4.} \nle %
  \end{console}
  \]
  Écrire une fonction \Scilab{} d'en-tête {\tt function r = 
  VaR(X,beta)} qui renvoie la valeur de l'estimation obtenue avec 
  l'estimateur $Y_n$ pour $r_\beta(X)$ si le tableau {\tt X} 
  contient la réalisation de l'échantillon $(X_1, \ldots, X_n)$ et 
  {\tt beta} la valeur de $\beta$. 
  
  \begin{proof}~
    \begin{scilab}
      & \tcFun{function} \tcVar{r} = VaR(\tcVar{X},\tcVar{beta}) \nl %
      & \quad n = length(\tcVar{X}) \nl %
      & \quad Z = triCroissant(\tcVar{X}) \nl %
      & \quad \tcVar{r} = Z(floor(n \Sfois{} \tcVar{beta})) \nl %
      & \tcFun{endfunction}
    \end{scilab}
    
    
    
    \newpage
    
    
    
    Détaillons l'obtention de cette fonction.\\
    D'après la question précédente, on sait que $Y_n
    =X_{\lfloor n\, \beta \rfloor,n}$ est 
    un estimateur convergent de $r_\beta(X)$.\\
    Pour déterminer $Y_n$ à partir de $X_1$, $\ldots$, $X_n$, 
    on souhaite donc :
    \begin{noliste}{\scriptsize 1)}
      \item définir $n$. Il s'agit de la taille de 
      l'échantillon $X_1$, $\ldots$, $X_n$.\\
      Donc on stocke cette valeur dans la variable {\tt n} avec
      la commande suivante :
      \begin{scilabC}{1}
        & \quad n = length(\tcVar{X})
      \end{scilabC}
      
      \item obtenir une réalisation de $X_{1,n}$, $\ldots$, $X_{n,n}$, 
      c'est-à-dire
      ordonner une réalisation de $X_1$, $\ldots$, $X_n$ dans l'ordre 
      croissant.\\
      On stocke ce nouveau vecteur dans la variable {\tt Z}
      avec la commande suivante :
      \begin{scilabC}{2}
        & \quad Z = triCroissant(\tcVar{x})
      \end{scilabC}
      
      \item sélectionner dans ce vecteur la réalisation de $X_{\lfloor 
      n\, \beta \rfloor,n}$, ce qui correspond à la $\eme{\lfloor n \,
      \beta \rfloor}$ coordonnée du vecteur {\tt Z}.\\
      On stocke ce résultat dans la variable de sortie {\tt r},
      ce qui donne :
      \begin{scilabC}{3}
        & \quad \tcVar{r} = Z(floor(n \Sfois{} \tcVar{beta}))
      \end{scilabC}
      où la commande {\tt floor} correspond à la fonction 
      partie entière.
    \end{noliste}~\\[-1cm]
  \end{proof}
\end{noliste}






\section*{Partie III - L'\og Expected Shortfall\fg (ES)}
\noindent
On conserve les notations de la partie I.\\
Pour qu'une mesure de risque soit acceptable, on souhaite qu'elle 
vérifie un certaines propriétés.\\
On dit qu'une fonction $\rho$ définie sur ${\cal D}$ à valeurs 
réelles est une {\bf mesure de risque cohérente} sur ${\cal D}$ 
si elle vérifie les quatre propriétés :
\begin{noliste}{}
  \item $(R_1) \quad \forall X \in {\cal D}, \forall c \in \R, \
  \rho(X+c)=\rho(X)+c$ ;
  
  \item $(R_2) \quad \forall X \in {\cal D}, \forall \lambda \in 
  \R, \ \rho(\lambda X)=\lambda \rho(X)$ ;
  
  \item $(R_3) \quad \forall(X,Y) \in {\cal D}^2$, si pour tout 
  $\omega \in \Omega$, $X(\omega) \leq Y(\omega)$ alors $\rho(X) 
  \leq \rho(Y)$ ;
  
  \item $(R_4) \quad \forall(X,Y) \in {\cal D}^2$, telles que $X+Y 
  \in {\cal D}, \ \rho(X+Y) \leq \rho(X)+\rho(Y)$.
\end{noliste}

\begin{noliste}{1.}
  \setlength{\itemsep}{4mm}
  \setcounter{enumi}{11}
  \item Montrer que l'espérance est une mesure de risque cohérente 
  sur ${\cal D}$.
  
  \begin{proof}~\\
    Montrons que l'espérance vérifie les propriétés $(R_1)$ à $(R_4)$.\\
    Soit $(X,Y) \in {\cal D}^2$. Soit $(c,\lambda) \in \R^2$.
    \begin{noliste}{$\sbullet$}
      \item Par linéarité de l'espérance : 
      \[
        \E(X+c) = \E(X) + \E(c) = \E(X) + c \quad \text{et} \quad
        \E(\lambda \, X)= \lambda \, \E(X)
      \]
      Donc $(R_1)$ et $(R_2)$ sont vérifiées.
      
      \item Par croissance de l'espérance : 
      \[
        \text{si pour tout } \omega \in \Omega, \ X(\omega) \leq 
	Y(\omega), \ \text{alors} \ \E(X) \leq \E(Y)
      \]
      Donc $(R_3)$ est vérifiée.
      
      \item Par linéarité de l'espérance :
      \[
        \E(X+Y) = \E(X) + \E(Y)
      \]
      Donc, en particulier : $\E(X+Y) \leq \E(X) + \E(Y)$.\\
      Ainsi, $(R_4)$ est vérifiée.
    \end{noliste}
    \conc{On en déduit que l'espérance est une mesure de risque sur 
    ${\cal D}$.}~\\[-1cm]
  \end{proof}
  
  
  \newpage

  
  \item La \og Value at Risk \fg{} $r_\beta$ est-elle une mesure de 
  risque cohérente sur ${\cal D}$ pour toute valeur de $\beta \in \
  ]0,1[$ ?\\
  On détaillera si chacune des propriétés de $(R_1)$ à $(R_4)$ est 
  satisfaite ou non.
  
  \begin{proof}~
    \begin{noliste}{$\sbullet$}
      \item D'après la question \itbf{4.a)}, $r_\beta$ vérifie $(R_1)$.
      
      \item D'après la question \itbf{4.b)}, $r_\beta$ vérifie 
      $(R_2)$.
      
      \item D'après la question \itbf{5.}, $r_\beta$ vérifie $(R_3)$.
      
      \item D'après la question \itbf{3.}, si on considère :
      \[
        X \suit \Norm{m}{\sigma^2} \quad \text{et} \quad 
        Y \suit \Norm{\mu}{s^2}
      \]
      Alors $r_{\frac{1}{4}}$ ne vérifie pas $(R_4)$.\\
      {\it (en fait, pour tout $\beta < \dfrac{1}{2}$, $r_\beta$ ne 
      satisfait pas $(R_4)$)}
    \end{noliste}
    \conc{On en déduit que $r_\beta$ n'est pas une mesure de risque
    cohérente pour tout $\beta \in \ ]0,1[$.}
    
    \begin{remark}
      Attention : la question \itbf{3.} démontre :
      \begin{center}
        $\beta <\dfrac{1}{2}$ \ $\Rightarrow$ \ $(R_4)$ n'est pas 
	vérifié.
      \end{center}
      Elle utilise pour cela un contre-exemple.\\
      Cependant, ce résultat n'est pas une équivalence !\\
      En particulier, rien ne dit :
      \begin{center}
        $\beta \geq \dfrac{1}{2}$ \ $\Rightarrow$ \ $(R_4)$ est 
	vérifiée.
      \end{center}
      Pour démontrer un tel résultat, il faudrait considérer deux \var 
      $X$ et $Y$ dans ${\cal D}$ quelconque (et non des \var 
      gaussiennes).
    \end{remark}~\\[-1.4cm]
  \end{proof}

  Soit $X$ une variable aléatoire appartenant à ${\cal D}$, 
  admettant une densité $f_X$. On définit l' \og Expected 
  Shortfall\fg{} de $X$ de niveau de confiance $\beta$ par :
  \[
    ES_\beta(X)=\dfrac1{1-\beta} \int_{r_\beta(X)}^{+\infty} x 
    f_X(x) \dx \quad(1)
  \]
  {\bf Le but de cette partie est de démontrer que, pour tout 
  $\beta \in \ ]0,1[$, $ES_\beta$ est une mesure de risque cohérente 
  sur ${\cal D}$, assez \og proche \fg{} de $r_\beta$.}
  
  \item Soit $X$ une variable aléatoire appartenant à ${\cal D}$.
  \begin{noliste}{a)}
    \setlength{\itemsep}{2mm}
    \item Montrer que $ES_\beta(X)$ est bien définie, et que 
    $ES_\beta(X) \geq r_\beta(X)$.
    
    \begin{proof}~
      \begin{noliste}{$\sbullet$}
	\item Comme $X\in {\cal D}$, en particulier la \var $X$ 
	admet une espérance.\\
	Donc l'intégrale $\dint{-\infty}{+\infty} x \, f_X(x) \dx$
	converge absolument.\\
	Ainsi, l'intégrale $\dint{r_\beta(X)}{+\infty} x\, f_X(x) \dx$
	converge aussi absolument.
	\conc{On en déduit que $ES_\beta(X)$ est bien définie.}
	
	
	\newpage
	
	
	\item Soit $x \geq r_\beta(X)$.\\
	Alors, comme $f_X(x) \geq 0$ (car $f_X$ est une densité), on 
	obtient :
	\[
	  r_\beta(X) \, f_X(x) \ \leq \ x \, f_X(x)
	\]
	Par croissance de l'intégrale (les bornes sont bien 
	ordonnées) et comme les intégrales en présence convergent :
	\[
	  \dint{r_\beta(X)}{+\infty} r_\beta(X) \, f_X(X) \dx
	  \leq \dint{r_\beta(X)}{+\infty} x\, f_X(x) \dx
	\]
	Or :
	\[
	\begin{array}{rcl}
	  \dint{r_\beta(X)}{+\infty} r_\beta(X) \, f_X(x) \dx
	  & = & r_\beta(X) \, \dint{r_\beta(X)}{+\infty} f_X(x) \dx
	  \ = \ r_\beta(X) \, \Prob(\Ev{X \geq r_\beta(X)})
	  \\[.6cm]
	  & = & r_\beta(X) (1-F_X(r_\beta(X)))
	\end{array}
	\]
	De plus, par définition de $r_\beta(X)$ : $F_X(r_\beta(X)) 
	=\beta$. Donc :
	\[
	  \dint{r_\beta(X)}{+\infty} r_\beta(X) \, f_X(x) \dx \ = \
	  r_\beta(X)(1-\beta)
	\]
	D'où :
	\[
	  r_\beta(X) (1-\beta) \ \leq \ \dint{r_\beta(X)}{+\infty}
	  x\, f_X(x) \dx
	\]
	Comme $1-\beta>0$ :
	\[
	  \dfrac{1}{\bcancel{1-\beta}} r_\beta(X) \bcancel{(1-\beta)}
	  \ \leq \ \dfrac{1}{1-\beta} \, \dint{r_\beta(X)}{+\infty}
	  x\, f_X(x) \dx
	\]
	\conc{On en déduit, par définition de $ES_\beta(X)$ :
	$r_\beta(X) \leq ES_\beta(X)$.}~\\[-1.2cm]
      \end{noliste}
    \end{proof}
    
    \item À l'aide du changement de variable $t=F_X(x)$, établir :
    \[
      ES_\beta(X)=\dfrac{1}{1-\beta} \dint{\beta}{1} G_X(t) \dt 
      \quad (2)
    \]
    
    \begin{proof}~\\
      Soit $A \geq r_\beta(X)$.\\
      On effectue le changement de 
      variable $\Boxed{t =F_X(x)}$. 
      \[
      \left|
        \begin{array}{P{8cm}}
          $t = F_X(x)$ (et donc $x = G_X(t)$) \nl
          $\hookrightarrow$ $dt = f_X(x)\dx$ 
          %\quad et \quad $dt
          %= \dfrac{1}{\ee^{-t}}\ du = \dfrac{1}{1-u} \ du$ 
          \nl
          \vspace{-.4cm}
          \begin{noliste}{$\sbullet$}
          \item $x = r_\beta(X) \ \Rightarrow \ t = F_X(r_\beta(X)) =
          \beta$
          \item $x = A \ \Rightarrow \ t = F_X(A)$ %
            \vspace{-.4cm}
          \end{noliste}
        \end{array}
      \right.
      \]
      Ce changement de variable est valide car $\psi : t \mapsto 
      G_X(t)$ est de classe $\Cont{1}$ sur $I_X$ (car $X \in {\cal 
      D}$).\\
      On obtient finalement :
      \[
        \dint{r_\beta(X)}{A} x\, f_X(x) \dx = \dint{\beta}{F_X(A)}
        G_X(t) \dt
      \]
      Or, comme $F_X$ est une fonction de répartition :
      $\dlim{A \to +\infty} F_X(A) = 1$.
      \conc{On en déduit : $ES_\beta(X) = \dfrac{1}{1-\beta}
      \dint{\beta}{1} G_X(t) \dt$.}~\\[-1cm]
    \end{proof}
  \end{noliste}
  
  
  \newpage
  
  
  On pourra utiliser $(1)$ ou $(2)$ au choix dans la suite pour 
  définir $ES_\beta(X)$.
  
  \item
  \begin{noliste}{a)}
    \setlength{\itemsep}{2mm}
    \item Montrer que $ES_\beta$ vérifie la propriété $(R_1)$.
    
    \begin{proof}~\\
      Soit $X\in {\cal D}$. Soit $c\in\R$.
      \begin{noliste}{$\sbullet$}
	\item D'après la question \itbf{4.a)} : $r_\beta(X+c) = 
	r_\beta(X)+c$.
	
	\item De plus, d'après la démonstration de la question 
	\itbf{4.a)} :
	\[
	  \forall x \in \R, \ F_{X+c}(x) = F_X(x-c)
	\]
	On a admis en question \itbf{4.a)} que $X+c \in {\cal D}$. En 
	particulier, $(X+c)$ est une \var à densité. On en déduit :
	\[
	  \forall x \in \R, f_{X+c}(x) = f_X(x-c)
	\]
	
	\item On obtient alors avec $(1)$ :
	\[
	  ES_\beta(X+c) \ = \ \dfrac{1}{1-\beta} 
	  \dint{r_\beta(X+c)}{+\infty} x \, f_{X+c}(x) \dx \ = \
	  \dfrac{1}{1-\beta} \dint{r_\beta(X)+c}{+\infty} x\, 
	  f_X(x-c) \dx
	\]
	On effectue le changement de 
      variable $\Boxed{t =x-c}$.
      \[
      \left|
        \begin{array}{P{8cm}}
          $t = x-c$ (et donc $x = t+c$) \nl
          $\hookrightarrow$ $dt = \dx$ 
%           \quad et \quad $dx = \dt$ 
          \nl
          \vspace{-.4cm}
          \begin{noliste}{$\sbullet$}
          \item $x = r_\beta(X)+c \ \Rightarrow \ t = r_\beta(X)$
          \item $x = +\infty \ \Rightarrow \ t = +\infty$ %
            \vspace{-.4cm}
          \end{noliste}
        \end{array}
      \right.
      \]
      Ce changement de variable est valide car $\psi : t \mapsto 
      t+c$ est de classe $\Cont{1}$ sur $[r_\beta(X), +\infty[$.\\
      On obtient finalement :
      \[
      \begin{array}{rcl@{\quad}>{\it}R{4.5cm}}
        ES_\beta(X+c) & = & \dfrac{1}{1-\beta} 
	\dint{r_\beta(X)}{+\infty}
        (t+c)f_X(t) \dt 
        \\[.6cm]
        & = & \dfrac{1}{1-\beta} \left(
        \dint{r_\beta(X)}{+\infty} t \, f_X(t) \dt + 
        c \, \dint{r_\beta(X)}{+\infty} f_X(t) \dt\right)
        \\[.6cm]
        &=& ES_\beta(X) + \dfrac{c}{1-\beta}(1-F_X(r_\beta(X)))
        \\[.4cm]
        &=& ES_\beta(X) + \dfrac{c}{\bcancel{1-\beta}} 
        \bcancel{(1-\beta)}
        & (par définition de $r_\beta(X)$)
        \nl
        \nl[-.2cm]
        &=& ES_\beta(X)+c
      \end{array}
      \]
      \conc{Ainsi, $ES_\beta$ vérifie $(R_1)$.}~\\[-1.4cm]
      \end{noliste}
    \end{proof}

    
    \item Montrer que $ES_\beta$ vérifie la propriété $(R_2)$.
    
    \begin{proof}~\\
      Soit $X\in {\cal D}$. Soit $\lambda\in\R$.
      \begin{noliste}{$\sbullet$}
	\item D'après la question \itbf{4.b)} : $r_\beta(\lambda \, X) 
	= \lambda \, r_\beta(X)$.
	
	\item De plus, d'après la démonstration de la question 
	\itbf{4.b)} :
	\[
	  \forall x \in \R, \ F_{\lambda \, X}(x) = F_X\left(\dfrac{x}
	  {\lambda}\right)
	\]
	
	
	\newpage
	
	
	On a admis en question \itbf{4.b)} que $\lambda \, X \in {\cal 
	D}$. En 
	particulier, $\lambda \, X$ est une \var à densité. \\
	On en déduit :
	\[
	  \forall x \in \R, f_{\lambda \, X}(x) = \dfrac{1}{\lambda} \, 
	  f_X\left(\dfrac{x}{\lambda}\right)
	\]
	
	\item On obtient alors avec $(1)$ :
	\[
	  ES_\beta(\lambda \, X) \ = \ \dfrac{1}{1-\beta} 
	  \dint{r_\beta(\lambda \,X)}{+\infty} x \, f_{\lambda \,X}(x) 
	  \dx \ = \
	  \dfrac{1}{1-\beta} \dint{\lambda \, r_\beta(X)}{+\infty} 
	  x\, \dfrac{1}{\lambda} f_X\left(\dfrac{x}{\lambda}\right) \dx
	\]
	On effectue le changement de 
      variable $\Boxed{t = \dfrac{x}{\lambda}}$.
      \[
      \left|
        \begin{array}{P{8cm}}
          $t = \dfrac{x}{\lambda}$ (et donc $x = \lambda \, t$) \nl
          $\hookrightarrow$ $dt = \dfrac{1}{\lambda} \dx$ 
          \quad et \quad $dx = \lambda \dt$ 
          \nl
          \vspace{-.4cm}
          \begin{noliste}{$\sbullet$}
          \item $x = \lambda \, r_\beta(X) \ \Rightarrow \ t = 
	  r_\beta(X)$
          \item $x = +\infty \ \Rightarrow \ t = +\infty$ (car $\lambda
          >0$) %
            \vspace{-.4cm}
          \end{noliste}
        \end{array}
      \right.
      \]
      Ce changement de variable est valide car $\psi : t \mapsto 
      \lambda \, t$ est de classe $\Cont{1}$ sur $[r_\beta(X),
      +\infty[$.\\
      On obtient finalement :
      \[
      \begin{array}{rcl@{\quad}>{\it}R{4.5cm}}
        ES_\beta(\lambda \, X) & = & \dfrac{1}{1-\beta} 
	\dint{r_\beta(X)}{+\infty}
        \lambda \, t \, f_X(t) \dt 
        \\[.6cm]
        & = & \lambda \, \dfrac{1}{1-\beta} \,
        \dint{r_\beta(X)}{+\infty} t \, f_X(t) \dt
        \\[.6cm]
        &=& \lambda \, ES_\beta(X)
      \end{array}
      \]
      \conc{Ainsi, $ES_\beta$ vérifie $(R_2)$.}~\\[-1.4cm]
      \end{noliste}
    \end{proof}
  \end{noliste}
  
  \item On suppose dans cette question que $X$ suit la loi 
  exponentielle de paramètre $\lambda>0$.
  \begin{noliste}{a)}
    \setlength{\itemsep}{2mm}
    \item Montrer que $ES_\beta(X)=r_\beta(X)+\dfrac{1}{\lambda}$.
    
    \begin{proof}~
      \begin{noliste}{$\sbullet$}
	\item D'après la question \itbf{2.b)} : $r_\beta(X) = -
	\dfrac{1}{\lambda} \, \ln(1-\beta)$.\\
	Donc, en particulier : $r_\beta(X) \geq 0$. D'où :
	\[
	  ES_\beta(X) \ = \ \dfrac{1}{1-\beta} \, \dint{r_\beta(X)}
	  {+\infty} x \, f_X(x) \dx \ = \ \dfrac{1}{1-\beta} \,
	  \dint{r_\beta(X)}{+\infty} \lambda \, x \, \ee^{-\lambda \, 
	  x} \dx
	\]
	
	\item Soit $A\geq r_\beta(X)$. On effectue l'intégration par 
	parties (IPP) suivante.
      \[
	\renewcommand{\arraystretch}{2.2}
	\begin{array}{|rcl@{\qquad}rcl}
	  u(x) & = & x & u'(x) & = & 1 \\
	  v'(x) & = & \lambda \, \ee^{-\lambda \, x} & v(x) & = & -
	  \ee^{-\lambda \, x}
	\end{array}
      \]
      
      
      \newpage
      
      
      Cette IPP est valide car les fonctions $u$ et $v$ sont de classe 
      $\Cont{1}$ sur $[r_\beta(X),A]$. On obtient :
      \[
        \begin{array}{rcl}
          \dint{r_\beta(X)}{A} \lambda \, x \, \ee^{-\lambda \, x} \dx
          &=& \Prim{-x \, \ee^{-\lambda \, x}}{r_\beta(X)}{A}
          + \dint{r_\beta(X)}{A} \ee^{-\lambda \, x} \dx
          \\[.6cm]
          &=& -A \ee^{-\lambda \, A} + r_\beta(X) \, \ee^{-\lambda \,
          r_\beta(X)} + \Prim{-\dfrac{1}{\lambda} \, \ee^{-\lambda \,
          x}}{r_\beta(X)}{A}
          \\[.6cm]
          &=& -A \ee^{-\lambda \, A} + r_\beta(X) \, \ee^{-\lambda \,
          r_\beta(X)} - \dfrac{1}{\lambda} \, \ee^{-\lambda \, A} 
          + \dfrac{1}{\lambda} \ee^{-\lambda \, r_\beta(X)}
          \\[.6cm]
          &=& \ee^{-\lambda \, r_\beta(X)}\left(r_\beta(X) +
          \dfrac{1}{\lambda}\right) -A \ee^{-\lambda \, A} - 
          \dfrac{1}{\lambda} \, \ee^{-\lambda \, A}
        \end{array}
      \]
      
      \item Or : $\dlim{A\to +\infty} \ee^{-\lambda \, A} =0$.\\[.2cm]
      De plus, par croissances comparées : $\dlim{A\to +\infty}
      A \ee^{-\lambda \, A} =0$.\\
      Donc :
      \[
        \dint{r_\beta(X)}{+\infty} \lambda \, x \, \ee^{-\lambda \, x}
        \dx \ = \ \ee^{-\lambda \, r_\beta(X)} \left( r_\beta(X) +
        \dfrac{1}{\lambda}\right)
      \]
      
      \item Or, d'après la question \itbf{2.b)} : $r_\beta(X) = -
	\dfrac{1}{\lambda} \, \ln(1-\beta)$. Donc :
	\[
	    \ee^{-\lambda \, r_\beta(X)} \ = \ 
	    \ee^{-\bcancel{\lambda} \times \left(-\frac{1}
	    {\bcancel{\lambda}} \, \ln(1-\beta)\right)}
	    \ = \ \ee^{\ln(1-\beta)} \ = \ 1-\beta
	\]
	D'où : 
	\[
	  \dint{r_\beta(X)}{+\infty} \lambda \, x \, \ee^{-\lambda \, x}
        \dx \ = \ (1-\beta) \left( r_\beta(X) +
        \dfrac{1}{\lambda}\right)
	\]
	
	\item On en déduit :
	\[
	  ES_\beta(X) \ = \ \dfrac{1}{\bcancel{1-\beta}} \times 
	  \bcancel{(1-\beta)} \left( r_\beta(X) +
        \dfrac{1}{\lambda}\right) \ = \ r_\beta(X) +
        \dfrac{1}{\lambda}
	\]
      \end{noliste}
      \conc{$ES_\beta(X) = r_\beta(X) + \dfrac{1}{\lambda}$}~\\[-1cm]
    \end{proof}

    
    \item En déduire que $ES_\beta(X) \eq{\beta}{1} r_\beta(X)$.
    
    \begin{proof}~\\
      D'après la question précédente :
      \[
        \dfrac{ES_\beta(X)}{r_\beta(X)} \ = \ \dfrac{r_\beta(X) + 
        \frac{1}{\lambda}}{r_\beta(X)} \ = \ 1 + 
        \dfrac{1}{\lambda \, r_\beta(X)}
      \]
      Or, d'après la question \itbf{2.b)} : $r_\beta(X) = -\dfrac{1}
      {\lambda} \, \ln(1-\beta)$. Donc :
      \[
        \dfrac{ES_\beta(X)}{r_\beta(X)} \ = \ 1 + \dfrac{1}{
        \bcancel{\lambda} \left(-\frac{1}{\bcancel{\lambda}} \, 
        \ln(1-\beta)\right)} \ = \ 1 - \dfrac{1}{\ln(1-\beta)}
      \]
      De plus : $\dlim{\beta \to 1} \ln(1-\beta) = -\infty$. D'où :
      \[
        \dlim{\beta \to 1} \dfrac{ES_\beta(X)}{r_\beta(X)} = 
        1-0 =1
      \]
      \conc{On en déduit : $ES_\beta(X) \eq{\beta}{1} 
      r_\beta(X)$}~\\[-1cm]
    \end{proof}
  \end{noliste}
  
  
  
  \newpage
  
  
  
  \item On suppose dans cette question que $X$ suit la loi normale 
  centrée réduite.
  \begin{noliste}{a)}
    \setlength{\itemsep}{2mm}
    \item Montrer $ES_\beta(X) = 
    \dfrac{\varphi(r_\beta(X))}{1-\Phi(r_\beta(X))}$.
    
    \begin{proof}~\\
      Par définition de $\varphi$ :
      \[
        ES_\beta(X) = \dfrac{1}{1-\beta} \, \dint{r_\beta(X)}
        {+\infty} x \, \dfrac{1}{\sqrt{2 \pi}} \, \ee^{-\frac{x^2}{2}}
        \dx \ = \ \dfrac{1}{1-\beta} \, \dfrac{1}{\sqrt{2 \pi}} \,
        \dint{r_\beta(X)}{+\infty} x \, \ee^{-\frac{x^2}{2}} \dx
      \]
      Soit $A \in [r_\beta(X), +\infty[$.
      \[
        \dint{r_\beta(X)}{A} x\, \ee^{-\frac{x^2}{2}} \dx \ = \
          \Prim{-\ee^{-\frac{x^2}{2}}}{r_\beta(X)}{A}
          \ = \ -\ee^{-\frac{A^2}{2}} + \ee^{-\frac{(r_\beta(X))^2}{2}}
      \]
      Or : $\dlim{A \to +\infty} \ee^{-\frac{A^2}{2}} =
      0$.\\
      Donc l'intégrale $\dint{r_\beta(X)}{+\infty} x \, 
      \ee^{-\frac{x^2}{2}} \dx$ converge et : 
      $\dint{r_\beta(X)}{+\infty} x \, \ee^{-\frac{x^2}{2}} \dx = 
      \ee^{-\frac{(r_\beta(X))^2}{2}}$.\\
      De plus, par définition de $r_\beta(X)$ : $\beta = 
      \Phi(r_\beta(X))$. Donc :
      \[
        ES_\beta(X) \ = \ \dfrac{1}{1-\Phi(r_\beta(X))} \, 
        \dfrac{1}{\sqrt{2 \pi}} \, \ee^{-\frac{(r_\beta(X))^2}{2}}
        \ = \ \dfrac{\varphi(r_\beta(X))}{1-\Phi(r_\beta(X))}
      \]
      \conc{$ES_\beta(X) = \dfrac{\varphi(r_\beta(X))}{1- 
      \Phi(r_\beta(X))}$}~\\[-1cm]
    \end{proof}
    
    \item Pour tout $x >0$, établir l'égalité : $1-\Phi(x)=
    \dfrac{\varphi(x)}{x}-\dint{x}{+\infty} \dfrac{\varphi(t)}{t^2} 
    \dt$.
    
    \begin{proof}~\\
      Soit $x>0$.
      \begin{noliste}{$\sbullet$}
	\item Par définition de $\Phi$ :
	\[
	  1-\Phi(x) \ = \ 1- \dint{-\infty}{x} \varphi(t) \dt
	  \ = \ \dint{x}{+\infty} \varphi(t) \dt \ = \
	  \dfrac{1}{\sqrt{2 \pi}} \dint{x}{+\infty} 
	  \ee^{-\frac{t^2}{2}} \dt \ = \ \dfrac{1}{\sqrt{2 \pi}}
	  \dint{x}{+\infty} \dfrac{1}{t} \times t \, 
	  \ee^{-\frac{t^2}{2}} \dt
	\]
	
	  \item Soit $A \geq x$. On effectue l'intégration par 
	parties (IPP) suivante.
      \[
	\renewcommand{\arraystretch}{2.2}
	\begin{array}{|rcl@{\qquad}rcl}
	  u(t) & = & \dfrac{1}{t} & u'(t) & = & -\dfrac{1}{t^2} \\
	  v'(t) & = & t \, \ee^{-\frac{t^2}{2}} & v(t) & = & -
	  \ee^{-\frac{t^2}{2}}
	\end{array}
      \]
      Cette IPP est valide car les fonctions $u$ et $v$ sont de classe 
      $\Cont{1}$ sur $[x,A]$. On obtient :
      \[
        \begin{array}{rcl}
          \dint{x}{A} \dfrac{1}{t} \, t \, \ee^{-\frac{t^2}{2}} \dt
          &=& \Prim{-\dfrac{1}{t} \, \ee^{-\frac{t^2}{2}}}{x}{A} - 
          \dint{x}{A} \dfrac{\ee^{-\frac{t^2}{2}}}{t^2} \dt
          \\[.6cm]
          &=& - \dfrac{\ee^{-\frac{-A^2}{2}}}{A} + \dfrac{\ee^{-
          \frac{x^2}{2}}}{x} - \dint{x}{A} \dfrac{\ee^{-
          \frac{t^2}{2}}}{t^2} \dt
        \end{array}
      \]
      
      
      
      \newpage
      
      
      
      \item Or : $\dlim{A \to +\infty} \dfrac{\ee^{-\frac{A^2}{2}}}
      {A} =0$.\\[.2cm]
      De plus :
      \[
        \dint{x}{A} \dfrac{\ee^{-
          \frac{t^2}{2}}}{t^2} \dt \ = \
        - \dfrac{\ee^{-\frac{-A^2}{2}}}{A} + \dfrac{\ee^{-
          \frac{x^2}{2}}}{x}
        - \dint{x}{A} \dfrac{1}{t} \, t \, \ee^{-\frac{t^2}{2}} \dt
        \ = \ - \dfrac{\ee^{-\frac{-A^2}{2}}}{A} + \dfrac{\ee^{-
          \frac{x^2}{2}}}{x} - \sqrt{2 \pi} \,
          \dint{x}{A} \varphi(t) \dt
      \]
      Donc l'intégrale $\dint{x}{+\infty} \dfrac{\ee^{-
          \frac{t^2}{2}}}{t^2} \dt$ converge, car l'intégrale 
      $\dint{x}{+\infty} \varphi(t) \dt$ converge.
      
      \item On en déduit :
      \[
        \begin{array}{rcl}
          1- \Phi(x) &=& \dfrac{1}{\sqrt{2 \pi}} \left(
          \dfrac{\ee^{-\frac{x^2}{2}}}{x} - \dint{x}{+\infty}
          \dfrac{\ee^{-\frac{t^2}{2}}}{t^2} \dt \right)
          \ = \ \dfrac{\frac{1}{\sqrt{2 \pi}} \, 
	  \ee^{-\frac{x^2}{2}}}{x} - \dint{x}{+\infty}
          \dfrac{\frac{1}{\sqrt{2 \pi}} \, \ee^{-\frac{t^2}{2}}}{t^2} 
	  \dt
	  \\[.6cm]
	  &=& \dfrac{\varphi(x)}{x} - \dint{x}{+\infty} 
	  \dfrac{\varphi(t)}{t^2} \dt
        \end{array}
      \]
      \end{noliste}
      \conc{$\forall x >0$, $1-\Phi(x) \ = \ \dfrac{\varphi(x)}{x} -
      \dint{x}{+\infty} \dfrac{\varphi(t)}{t^2} \dt$}~\\[-1cm]      
    \end{proof}

    \item Montrer que, pour tout $x>0$ : $0 \ \leq \ \dint{x}{+\infty}  
    \dfrac{\varphi(t)}{t^2} \dt \ \leq \ \dfrac{1}{x^2}\, 
    (1-\Phi(x))$.\\
    En déduire que : $1-\Phi(x) \eq{x}{+\infty} 
    \dfrac{\varphi(x)}{x}$.
    
    \begin{proof}~\\
      Soit $x>0$.
      \begin{noliste}{$\sbullet$}
      \item Soit $t \geq x$. Alors, par décroissance de $t\mapsto 
      \dfrac{1}{t^2}$ sur $]0,+\infty[$ : $ \dfrac{1}{t^2} \leq 
      \dfrac{1}{x^2}$.\\
      De plus : $\dfrac{1}{t^2} \geq 0$. Donc : 
      \[
        0 \ \leq \ \dfrac{1}{t^2} \ \leq \ \dfrac{1}{x^2}
      \]
      Or, comme $\varphi$ est une densité, en particulier : 
      $\varphi(t) \geq 0$. D'où :
      \[
        0 \ \leq \ \dfrac{\varphi(t)}{t^2} \ \leq \ 
        \dfrac{\varphi(t)}{x^2}
      \]
      
    \item Par croissance de l'intégrale (les bornes sont bien
      ordonnées) et comme les intégrales en présence sont convergentes
      :
      \[
        0 \ \leq \ \dint{x}{+\infty} \dfrac{\varphi(t)}{t^2} \dt \
        \leq \ \dint{x}{+\infty} \dfrac{\varphi(t)}{x^2} \dt
      \]
      
      \item Or :
      \[
        \dint{x}{+\infty} \dfrac{\varphi(t)}{x^2} \dt \ = \
        \dfrac{1}{x^2} \, \dint{x}{+\infty} \varphi(t) \dt \ = \
        \dfrac{1}{x^2} \, \Prob(\Ev{X \geq x}) \ = \
        \dfrac{1}{x^2} (1-\Phi(x))
      \]
      \end{noliste}
      \conc{On en déduit : $\forall x >0$, $0 \ \leq \ 
      \dint{x}{+\infty} \dfrac{\varphi(t)}{t^2} \dt \ \leq \
      \dfrac{1}{x^2} (1-\Phi(x))$.}
      
      
      \newpage
      
      
      \begin{remark}
        Le schéma de résolution de cette question est plutôt 
	classique.\\
        Afin de comparer deux intégrales $\dint{a}{b} f(t) \dt$ et 
        $\dint{a}{b} g(t) \dt$ :
        \begin{noliste}{\scriptsize 1)}
	  \item on cherche d'abord à comparer leurs 
	  intégrandes, c'est-à-dire montrer :
	  \[
	    \forall t \in [a,b], \ f(t) \leq g(t) \quad \text{ou} \quad 
	    \forall t \in [a,b], \ f(t) \geq g(t)
	  \]
	  \item on utilise ensuite la croissance de l'intégrale (si 
	  les bornes $a$ et $b$ sont bien ordonnées, c'est-à-dire 
	  $a \leq b$) pour conclure :
	  \[
	    \dint{a}{b} f(t) \dt \leq \dint{a}{b} g(t) \dt \quad 
	    \text{ou} \quad \dint{a}{b} f(t) \dt \geq 
	    \dint{a}{b} g(t) \dt
	  \]
        \end{noliste}
      \end{remark}
      
      Soit $x>0$.
      \begin{noliste}{$\sbullet$}
	\item D'après la question précédente :
	\[
	  -\dfrac{1}{x^2}(1-\Phi(x)) \ \leq \ - \dint{x}{+\infty}
	  \dfrac{\varphi(t)}{t^2} \dt \ \leq \ 0
	\]
	Donc, d'après la question \itbf{17.b)}, en ajoutant $\dfrac{
	\varphi(x)}{x}$ à chaque membre de cet encadrement :
	\[
	  \dfrac{\varphi(x)}{x} - \dfrac{1}{x^2}(1-\Phi(x)) \ \leq \
	  1-\Phi(x) \ \leq \ \dfrac{\varphi(x)}{x}
	\]
	
	\item On en déduit :
	\[
	  1-\Phi(x) \ \leq \ \dfrac{\varphi(x)}{x} \ \leq \
	  1-\Phi(x) + \dfrac{1}{x^2}(1-\Phi(x)) = (1-\Phi(x))
	  \left(1+\dfrac{1}{x^2}\right)
	\]
	D'où, comme $1-\Phi(x) >0$ :
	\[
	  1 \ \leq \ \dfrac{\dfrac{\varphi(x)}{x}}{1-\Phi(x)} \ \leq \
	  1+ \dfrac{1}{x^2}
	\]
	\item Or : $\dlim{x\to +\infty} \left(1+ \dfrac{1}{x^2}\right) 
	=1$.\\
	Donc, par théorème d'encadrement : $\dlim{x\to + \infty} 
	\dfrac{\frac{\varphi(x)}{x}}{1-\Phi(x)}=1$.
	\conc{Ainsi : $1- \Phi(x) \eqx{+\infty} 
	\dfrac{\varphi(x)}{x}$.}~\\[-1.2cm]
      \end{noliste}
    \end{proof}

    
    \item En conclure que l'on a aussi dans ce cas : $ES_\beta(X) 
    \eq{\beta}{1} r_\beta(X)$.
    
    \begin{proof}~
      \begin{noliste}{$\sbullet$}
	\item D'après la question \itbf{3.a)(iii)} : $r_\beta(X) = 
	\Phi^{-1}(\beta)$.\\
	Or, d'après la question \itbf{3.a)(i)}, la fonction $\Phi$
	réalise une bijection strictement croissante de 
	$]-\infty, +\infty[$ sur $]0,1[$.\\
	On en déduit : $\dlim{\beta \to 1} \Phi^{-1}(\beta)=+\infty$.
	D'où : $\dlim{\beta \to 1} r_\beta(X)=+\infty$.
	
	
	\newpage
	
	
	\item De plus, d'après la question précédente : $1-\Phi(x) 
	\eqx{+\infty} \dfrac{\varphi(x)}{x}$.\\
	D'où : $1-\Phi(r_\beta(X)) \eq{\beta}{1} \dfrac{\varphi(r_\beta
	(X))}{r_\beta(X)}$.\\[.1cm]
	Ainsi, d'après la question \itbf{17.a)} :
	\[
        ES_\beta(X) = \dfrac{\varphi(r_\beta(X))}{1-\Phi(r_\beta(X))}
        \eq{\beta}{1}
        \dfrac{\varphi(r_\beta(X))}{\frac{\varphi(r_\beta(X))}{r_\beta(X)}}
        = \bcancel{\varphi(r_{\beta}(X))} \
        \dfrac{r_{\beta}(X)}{\bcancel{\varphi(r_{\beta}(X))}} =
        r_\beta(X)
	\]
      \end{noliste}
      \conc{$ES_\beta(X) \eq{\beta}{1} r_\beta(X)$}~\\[-1cm]
    \end{proof}
  \end{noliste}
  Dans les questions qui suivent, $X$ est une variable aléatoire 
  appartenant à ${\cal D}$.
  \begin{noliste}{$\sbullet$}
    \item On note $h$ la fonction définie par : $\forall x 
    \in \R$, $h(x)=\max(x,0)$.
    
    \item On admet que si $U$ et $V$ sont deux variables 
    aléatoires telles que, $0 \leq U \leq V$ et $\E(V)$ existe 
    alors $\E(U)$ existe et $0 \leq \E(U) \leq \E(V)$.
    
    \item On note pour tout événement $A$, 
    $\unq_A$ la variable aléatoire indicatrice de 
    l'événement $A$. Rappelons qu'il s'agit de la variable 
    aléatoire prenant la valeur $1$ si $A$ est réalisé, et la 
    valeur $0$ sinon.
  \end{noliste}
  
  \item
  \begin{noliste}{a)}
    \setlength{\itemsep}{2mm}
    \item Montrer que $h(X-r_\beta(X))$ admet une espérance, et 
    que l'on a :
    \[
      \E\left(h(X-r_\beta(X))\right)=\dint{r_\beta(X)}{+\infty} t 
      f_X(t) \dt - (1-\beta)r_\beta(X)
    \]
    où $f_X$ désigne une densité de $X$.
    
    \begin{proof}~
      \begin{noliste}{$\sbullet$}
	\item Soit $t\in \R$.\\
	Par définition de la fonction $h$ : 
	\[
	  h(t-r_\beta(X)) \ = \ \max(t- r_\beta(X),0) \ = \
	  \left\{
	  \begin{array}{cR{3cm}}
	    t - r_\beta(X) & si $t-r_\beta(X) \geq 0$
	    \nl
	    0 & sinon
	  \end{array}
	  \right.
	\]
	\conc{D'où : $\forall t \in \R$,
	$
	  h(t-r_\beta(X)) \ = \ \left\{
	  \begin{array}{cR{2.1cm}}
	    t-r_\beta(X) & si $t\geq r_\beta(X)$
	    \nl
	    0 & sinon
	  \end{array}
	  \right.
	$.}
	
    \item D'après le théorème de transfert, la \var $h(X- r_\beta(X))$
      admet une espérance si et seulement si l'intégrale
      $\dint{-\infty}{+\infty} h(t- r_\beta(X)) f_X(t) \dt$ converge
      absolument. Or :
	\begin{noliste}{$\stimes$}
	  \item comme $f_X$ est une densité : $\forall t\in \R$, 
	  $f_X(t) \geq 0$ ;
	  \item d'après ce qui précède : $\forall t \in \R$, 
	  $h(t- r_\beta(X)) \geq 0$.
	\end{noliste}
	Donc cela revient à montrer que l'intégrale $\dint{-\infty}
	{+\infty} h(t-r_\beta(X)) f_X(t) \dt$ converge.
	
	\item Toujours d'après ce qui précède, la fonction $t \mapsto 
	h(t-r_\beta(X))$ est nulle en dehors de $[r_\beta(X), 
	+\infty[$. Donc :
	\[
	  \dint{-\infty}{+\infty} h(t-r_\beta(X)) f_X(t) \dt \ = \
	  \dint{r_\beta(X)}{+\infty} h(t-r_\beta(X)) f_X(t) \dt
	\]
	
	
	
	\newpage
	
	
	\item Soit $A \geq r_\beta(X)$.\\
	Par définition de la fonction $t\mapsto h(t- r_\beta(X))$ 
	sur $[r_\beta(X), +\infty[$ :
	\[
	  \begin{array}{rcl}
	    \dint{r_\beta(X)}{A} h(t-r_\beta(X)) f_X(t) \dt &=& 
	    \dint{r_\beta(X)}{A} (t-r_\beta(X)) f_X(t) \dt
	    \\[.6cm]
	    &=& \dint{r_\beta(X)}{A} t \, f_X(t) \dt - 
	    r_\beta(X) \dint{r_\beta(X)}{A} f_X(t) \dt
% 	    \\[.6cm]
% 	    &=& \dint{r_\beta(X)}{A} t\, f_X(t) \dt - r_\beta(X)
% 	    \, \Prob(\Ev{r_\beta(X) \leq X \leq A})
% 	    \\[.6cm]
% 	    &=& \dint{r_\beta(X)}{A} t\, f_X(t) \dt - r_\beta(X)
% 	    (F_X(A) - F_X(r_\beta(X))
% 	    \\[.6cm]
% 	    &=& \dint{r_\beta(X)}{A} t\, f_X(t) \dt - r_\beta(X)
% 	    (F_X(A)-\beta)
	  \end{array}
	\]
	
	\item Or, $X\in {\cal D}$, donc la \var $X$ admet une 
	espérance.\\[.1cm]
	Ainsi l'intégrale $\dint{-\infty}{+\infty} t \, f_X(t) \dt$
	converge absolument.\\[.1cm]
	Donc l'intégrale $\dint{r_\beta(X)}{+\infty} t\, f_X(t) \dt$
	converge absolument.
	
	\item De plus, par définition de $r_\beta(X)$ :
	\[
	  \dint{r_\beta(X)}{A} f_X(t) \dt \ = \ F_X(A) - F_X(r_\beta(X))
	  \ = \ F_X(A) - \beta \ \tendd{A}{+\infty} \ 1-\beta
	\]
	
	\item On en déduit que l'intégrale $\dint{r_\beta(X)}{+\infty}
	h(t-r_\beta(X)) f_X(t) \dt$ converge.
	\conc{Donc la \var $h(X-r_\beta(X))$ admet une espérance.}
	
	De plus :
	\[
	  \E(h(X-r_\beta(X))) \ = \
	  \dint{r_\beta(X)}{+\infty} h(t- r_\beta(X)) f_X(t) \dt 
	  \ = \ \dint{r_\beta(X)}{+\infty} t \, f_X(t) \dt - 
	  r_\beta(X) (1-\beta)
	\]
	\conc{$\E(h(X-r_\beta(X)) \ = \ \dint{r_\beta(X)}{+\infty}
	t\, f_X(t) \dt - (1-\beta) r_\beta(X)$}~\\[-1.4cm]
      \end{noliste}
    \end{proof}

    
    \item En déduire : 
    \[
      ES_\beta(X)=r_\beta(X)+\dfrac{1}{1-\beta} 
      \E\left(h(X-r_\beta(X))\right)
    \]
    
    \begin{proof}~\\
      D'après l'équation $(1)$ :
      \[
        ES_\beta(X) \ = \ \dfrac{1}{1-\beta} \dint{r_\beta(X)}{+\infty}
        t \, f_X(t) \dt
      \]
      Or, d'après la question précédente :
      \[
        \dint{r_\beta(X)}{+\infty} t \, f_X(t) \dt \ = \ 
        \E(h(X- r_\beta(X))) + (1-\beta) r_\beta(X)
      \]
      Donc : 
      \[
        ES_\beta(X) \ = \ \dfrac{1}{1-\beta} \Big(
        \E(h(X- r_\beta(X))) + (1-\beta) r_\beta(X)) \Big) \ = \
        \dfrac{1}{1-\beta} \E(h(X- r_\beta(X))) + r_\beta(X)
      \]
      \conc{$ES_\beta(X) \ = \ r_\beta(X) + \dfrac{1}{1-\beta} 
      \E(h(X- r_\beta(X)))$}~\\[-1cm]
    \end{proof}
  \end{noliste}
  
  
  
  \newpage
  
  
  
  \item En utilisant la méthode de Monte-Carlo, dont on supposera 
  la validité, et la fonction {\tt VaR} définie dans la question 
  \itbf{11.}, écrire une fonction \Scilab{} qui calcule une valeur 
  approchée de $ES_\beta(X)$ à partir de la réalisation d'un 
  échantillon de taille $n$ de la loi de $X$ dont les valeurs se 
  trouvent dans le tableau \Scilab{} {\tt X} et de la valeur de 
  $\beta$ se trouvant dans la variable \Scilab{} {\tt beta}.
  
  \begin{proof}~\\%
    On souhaite obtenir une valeur approchée de $ES_{\beta}(X)$. Pour
    ce faire, on doit tout d'abord obtenir une valeur approchée de
    l'espérance de la \var $V = h(X- r_\beta(X))$.
    \begin{noliste}{$\sbullet$}
    \item L'idée naturelle pour obtenir une approximation de cette
      espérance est :
      \begin{noliste}{$\stimes$}
      \item de simuler un grand nombre de fois ($N = 10000$ par
        exemple) la \var $V$.\\
        Formellement, on souhaite obtenir un $N$-uplet $(v_1,
        \ldots, v_N)$ qui correspond à l'observation d'un
        $N$-échantillon $(V_1, \ldots, V_N)$ de la \var $V$.
      \item de réaliser la moyenne des résultats de cette
        observation.
      \end{noliste}
      Cette idée est justifiée par la loi faible des grands nombres
      (LfGN) qui affirme :
      \[
      \mbox{moyenne de l'observation} = \dfrac{1}{N} \ \Sum{i =
        1}{N} v_i \ \simeq \ \E(V)
      \]

    \item Il reste à obtenir le $N$-uplet $(v_1, \ldots, v_N)$. Pour
      ce faire, on dispose d'un $N$-uplet $(x_1, \ldots, x_N)$ qui
      correspond à l'observation d'un $N$-échantillon $(X_1, \ldots,
      X_N)$ de la \var $X$.\\
      Cette observation permet d'obtenir $(v_1, \hdots, v_N)$ en 
      écrivant :
      \[
      v_i \ = \ h\big(x_i - r_{\beta}(X)\big)
      \]
      Il est à noter que, comme le suggère l'énoncé, la valeur de
      $r_{\beta}(X)$ sera obtenue par l'appel à la fonction {\tt VaR}.

    \item La valeur approchée de $ES_{\beta}(X)$ est alors obtenue par
      le calcul : $r_{\beta}(X) + \dfrac{1}{1-\beta} \ \dfrac{1}{N} \
      \Sum{i = 1}{N} v_i$.

    \item Ce qui aboutit à la fonction
      suivante :\\[-.2cm]
        \begin{scilab}
          & \tcFun{function} \tcVar{ES} =
          \underline{ExpectedShortfall}(\tcVar{X}, \tcVar{beta}) \nl %
          & \qquad N = length(\tcVar{X}) \nl %
          & \qquad V = zeros(1, N) \nl %
          & \qquad r = VaR(\tcVar{X}, \tcVar{beta}) \nl %
          & \qquad \tcFor{for} i=1:N \nl %
          & \qquad \qquad V(i) = max(\tcVar{X}(i)-r, 0) \nl %
          & \qquad \tcFor{end} \nl %
          & \qquad \tcVar{ES} = r + 1/(1-beta) \Sfois{} 1/N \Sfois{}
          sum(V) \nl %
          & \tcFun{endfunction}
        \end{scilab}~\\%[-.2cm]
        Détaillons les différents éléments de ce code :
        \begin{noliste}{$\stimes$}
        \item le paramètre \tcVar{X} correspond à l'observation $(x_1,
          \ldots, x_N)$.
        \item en ligne \ligne{2}, on récupère la taille de
          l'échantillon fourni en paramètre.
        \item en ligne \ligne{3}, on crée une matrice ligne de taille
          $N$ permettant à terme de stocker l'observation $(v_1,
          \ldots, v_N)$.
        \item en lignes \ligne{5} à \ligne{7}, on effectue une boucle
          permettant d'obtenir successivement chaque $v_i$ à l'aide de
          la valeur $x_i$ fournit en paramètre.
        \item en ligne \ligne{8}, on obtient la valeur approchée de
          $ES_{\beta}(X)$ en effectuant le calcul annoncé.
        \end{noliste}
      \end{noliste}
      
      
      \newpage
      
      
      \begin{remark}%~%
        \begin{noliste}{$\sbullet$}
        \item Afin de permettre une bonne compréhension des mécanismes
          en jeu, on a détaillé la réponse à cette
          question. Cependant, écrire correctement la fonction
          \Scilab{} démontre la bonne compréhension et permet
          certainement d'obtenir tous les points alloués.
        \item On s'est servi en ligne \ligne{6} de la fonction {\tt
            max} prédéfinie en \Scilab{}.\\
          Il était aussi possible d'utiliser une structure
          conditionnelle :%\\[-.2cm]
          \begin{scilabC}{5}
            & \qquad \qquad \tcIf{if} \tcVar{X}(i)-r > 0
            \tcIf{then} \nl %
            & \qquad \qquad \qquad V(i) = \tcVar{X}(i)-r \nl %
            & \qquad \qquad \tcIf{else} \nl %
            & \qquad \qquad \qquad V(i) = 0 \nl %
            & \qquad \qquad \tcIf{end}
          \end{scilabC}
        % On procédera de même dans les autres questions \Scilab{}.
      \item Enfin, il est possible de calculer la somme $\Sum{i =
          1}{N} v_i$ sans créer une matrice de taille $N$. Pour ce
        faire, on crée une variable {\tt S} qu'on initialise à $0$ et
        qu'on met à jour dans la boucle. On obtient le programme
        suivant :
        \begin{scilab}
          & \tcFun{function} \tcVar{ES} =
          \underline{ExpectedShortfall}(\tcVar{X}, \tcVar{beta}) \nl %
          & \qquad N = length(\tcVar{X}) \nl %
          & \qquad S = 0 \nl %
          & \qquad r = VaR(\tcVar{X}, \tcVar{beta}) \nl %
          & \qquad \tcFor{for} i=1:N \nl %
          & \qquad \qquad S = S + max(\tcVar{X}(i)-r, 0) \nl %
          & \qquad \tcFor{end} \nl %
          & \qquad \tcVar{ES} = r + 1/(1-beta) \Sfois{} 1/N \Sfois{}
          S \nl %
          & \tcFun{endfunction}
        \end{scilab}
      \end{noliste}
    \end{remark}~\\[-1.4cm]
  \end{proof}
  
  \item Soit $Z$ une variable aléatoire telle que : 
  $\E(Z)=1$ et $0\leq Z \leq \dfrac1{1-\beta}$.
  \begin{noliste}{a)}
    \setlength{\itemsep}{2mm}
    \item Justifier l'égalité entre variables aléatoires : 
    $h(X-r_\beta(X))=(X-r_\beta(X)) \times \unq_{\Ev{X > 
    r_\beta(X)}}$.
    
    \begin{proof}~\\
      Soit $\omega \in \Omega$.\\
      Deux cas se présentent.
      \begin{noliste}{$\sbullet$}
	\item \dashuline{Si $X(\omega) > r_\beta(X)$}, alors :
	\begin{noliste}{$\stimes$}
	  \item par définition de la \var indicatrice $\unq_{\Ev{ X >
	  r_\beta(X)}}$ : $\unq_{\Ev{ X> r_\beta(X)}}(\omega) = 1$.\\
	  Donc :
	  \[
	    (X(\omega) - r_\beta(X)) \times \unq_{\Ev{ X > r_\beta(X)}}
	    (\omega) \ = \ X(\omega) - r_\beta(X)
	  \]
	  
	  \item par définition de $t \mapsto h(t- r_\beta(X))$ : 
	  $h(X(\omega) - r_\beta(X)) = X(\omega) - r_\beta(X)$.
	\end{noliste}
	D'où :
	\[
	  h(X(\omega) - r_\beta(X)) \ = \ (X(\omega) - r_\beta(X))
	  \times \unq_{\Ev{ X> r_\beta(X)}}(\omega)
	\]
	
	\item \dashuline{Si $X(\omega) \leq r_\beta(X)$}, alors :
	\begin{noliste}{$\stimes$}
	  \item par définition de la \var indicatrice $\unq_{\Ev{ X >
	  r_\beta(X)}}$ : $\unq_{\Ev{ X> r_\beta(X)}}(\omega) = 0$.\\
	  Donc :
	  \[
	    (X(\omega) - r_\beta(X)) \times \unq_{\Ev{ X > r_\beta(X)}}
	    (\omega) \ = \ 0
	  \]
	  
	  \item par définition de $t \mapsto h(t- r_\beta(X))$ : 
	  $h(X(\omega) - r_\beta(X)) = 0$.
	\end{noliste}
	
	
	\newpage
	
	
	D'où :
	\[
	  h(X(\omega) - r_\beta(X)) \ = \ (X(\omega) - r_\beta(X))
	  \times \unq_{\Ev{ X> r_\beta(X)}}(\omega)
	\]
      \end{noliste}
      Finalement : $\forall \omega \in \Omega$, $h(X(\omega) - 
      r_\beta(X)) = (X(\omega)- r_\beta(X)) \times \unq_{\Ev{X > 
      r_\beta(X)}}(\omega)$.
      \conc{$h(X- r_\beta(X)) = (X- r_\beta(X)) \times \unq_{ \Ev{
      X > r_\beta(X)}}$}~\\[-1cm]
    \end{proof}

    
    \item Montrer que $\E(XZ)$ existe et établir l'égalité :
    \[
    ES_\beta(X) -\E(XZ) = \dfrac{1}{1-\beta} \E\left[(X-r_\beta(X))
      (\unq_{\Ev{X > r_\beta(X)}}-(1-\beta)Z)\right]
    \]
    
    \begin{proof}~
      \begin{noliste}{$\sbullet$}
        % \item \begin{noliste}{$\stimes$}
      \item D'après l'énoncé : $0 \leq Z \leq \dfrac{1}{1-\beta}$.\\
        Donc, comme $\vert X \vert$ est à valeurs positives : $ 0 \
        \leq \ \vert X \vert \, Z \ \leq \ \dfrac{1}{1-\beta} \,
	\vert X \vert$.\\
	D'où, comme $Z$ est aussi à valeurs positives :
	\[
	  0 \ \leq \ \vert X Z \vert \ \leq \dfrac{1}{1-\beta} \, \vert 
	  X \vert \quad (\star)
	\]

      \item Démontrons alors que $|X|$ admet une espérance.\\
        D'après le théorème de transfert, $|X|$ admet une espérance si
        et seulement si l'intégrale impropre $\dint{-\infty}{+\infty}
        |t| \ f_X(t) \dt$ est absolument convergente.\\[.2cm]
        Or, comme $X \in {\cal D}$, alors $X$ admet une
        espérance. Cela signifie que l'intégrale impropre
        $\dint{-\infty}{+\infty} t \ f_X(t) \dt$ est absolument
        convergente. Enfin, comme $f$ est à valeurs positives :
        \[
        \forall t \in \R, \ |t \ f_X(t)| \ = \ |t| \ |f_X(t)| \ = \
        |t| \ f_X(t)
        \]
        On en conclut que $|X|$ admet une espérance si et seulement si
        $X$ admet une espérance.%
        \conc{Et comme $X$ admet une espérance, il en est de même de $|X|$.}

%       \item De plus, comme $X$ est une \var à densité, elle admet une
%         espérance si et seulement si l'intégrale
%         $\dint{-\infty}{+\infty} t\, f_X(t) \dt$ converge absolument,
%         c'est-à-dire $\dint{-\infty}{+\infty} \vert t \vert \, f_X(t)
%         \dt$ converge (car, comme $f_X$ est une densité : $\forall t
%         \in \R$, $f_X(t) \geq 0$).\\

% 	Finalement, la \var $X$ admet une espérance ssi $\vert X
%         \vert$ admet une espérance.
	
      \item Or, il est précisé dans l'énoncé la propriété suivante :
	\[
	  \left\{
	  \begin{array}{l}
	    0 \leq U \leq V \\[.1cm]
	    \text{$\E(V)$ existe}
	  \end{array}
	  \right. 
	  \ \Rightarrow \ 
	  \left\{
	  \begin{array}{l}
	    \text{$\E(U)$ existe}\\[.1cm]
	    0 \leq \E(U) \leq \E(V)
	  \end{array}
	  \right. \qquad (\star \star)
	\]
        La propriété $(\star)$ étant vérifiée, et comme $|X|$ admet
        une espérance, on peut appliquer la propriété rappelée
        ci-dessus à $U= \vert XZ \vert$ et $V=\dfrac{1}{1- \beta}
        \vert X \vert$.\\
        On en déduit que la \var $\vert XZ \vert$ admet une
        espérance.%
        \conc{Ainsi, $XZ$ admet une espérance.}
        % \end{noliste}
	
      \item Il s'agit maintenant d'établir l'égalité énoncée. \\
        Tout d'abord, d'après la question \itbf{20.a)} :
	\[
	  \begin{array}{rcl@{\quad}>{\it}R{4cm}}
	    (X- r_\beta(X)) \Big(\unq_{\Ev{X > r_\beta(X)}} - (1-\beta)
	    Z \Big) & = & (X-r_\beta(X)) \, \unq_{\Ev{X > r_\beta(X)}}
	    - (X- r_\beta(X))(1-\beta) Z
	    \\[.4cm]
	    & = & h(X- r_\beta(X)) - (1-\beta) XZ + (1-\beta)r_\beta(X)
	    Z
	  \end{array}
	\]
	
	
	\newpage
	
	
      \item De plus les \var $h(X- r_\beta(X))$, $XZ$ et $Z$ admettent
	une espérance.\\
	Ainsi, la \var $(X- r_\beta(X)) \Big(\unq_{\Ev{X >
            r_\beta(X)}} - (1-\beta)Z \Big)$ admet une espérance en
        tant que combinaison linéaire de \var qui en admettent une.\\
        Enfin, par linéarité de l'espérance :
	\[
	  \begin{array}{rcl@{\quad}>{\it}R{5cm}}
	    & & \E\Big((X- r_\beta(X)) (\unq_{\Ev{X > r_\beta(X)}} - 
	    (1-\beta) Z )\Big) 
	    \\[.4cm]
	    &=& \E(h(X-r_\beta(X))) - 
	    (1-\beta) \E(XZ) + (1-\beta) r_\beta(X) \E(Z)
	    \\[.4cm]
	    &=& (1-\beta) (ES_\beta(X)- \bcancel{r_\beta(X)})
	    -(1-\beta) \E(XZ)
	    +\bcancel{(1-\beta) r_\beta(X)}
	    & (d'après \itbf{18.b)} et car $\E(Z)=1$ d'après l'énoncé)
	    \nl
	    \nl[-.2cm]
	    &=& (1-\beta)\Big(ES_\beta(X) - \E(XZ)\Big)
	  \end{array}
	\]
	\conc{$ES_\beta(X) - \E(XZ) \ = \ \dfrac{1}{1-\beta} \, 
	\E\Big( (X- r_\beta(X)) (\unq_{\Ev{X > r_\beta(X)}} - 
	    (1-\beta) Z )\Big)$}
      \end{noliste}
      
      \begin{remark}
        \begin{noliste}{$\sbullet$}
	  \item Pour les \var à densité ou discrètes, la propriété :
	  \[
	    \text{$X$ admet une espérance \ $\Leftrightarrow$ \ $\vert 
	    X \vert$ admet une espérance}
	  \]
	  est immédiate par définition de l'espérance (comme 
	  démontré plus haut).
	  
	  \item Elle est cependant bien moins triviale pour des \var 
	  quelconques puisque l'on n'a pas accès à une expression 
	  explicite de l'espérance.\\
	  On peut cependant démontrer l'implication :
	  \[
	    \text{$\vert X \vert$ admet une espérance \ $\Rightarrow$ \ 
	    $X$ admet une espérance}
	  \]
        \item Détaillons cette démonstration.\\
          On suppose que $\vert X \vert$ admet une espérance.
	  \begin{noliste}{$\stimes$}
	    \item On commence par définir les \var suivantes :
	    \[
	      X^+ \ = \ \dfrac{\vert X \vert +X}{2} \quad \text{et} 
	      \quad X^- \ = \ \dfrac{\vert X \vert -X}{2}
	    \]
	    La \var $X^+=\max(X,0)$ est la partie positive de $X$.\\
	    La \var $X^-=-\min(X,0)$ est la partie négative de $X$.
	    
	    \item Par définition de $X^+$ et $X^-$, on a :
	    \[
	      0 \leq X^+ \leq \vert X \vert \quad \text{et} \quad 
	      0 \leq X^- \leq \vert X \vert
	    \]
	    Or $\vert X \vert$ admet une espérance.\\
	    Donc, d'après la propriété $(\star \star)$ de l'énoncé, les 
	    \var 
	    $X^+$ et 
	    $X^-$ admettent des espérances.
	    
	    \item Enfin, on remarque :
	    \[
	      X = X^+ - X^-
	    \]
	    Donc $X$ admet une espérance en tant que différence de \var 
	    en admettant une.
	  \end{noliste}
	\end{noliste}
      \end{remark}~\\[-1.4cm]
    \end{proof}
    
    
    \newpage
    
    
    \item En déduire que $ES_\beta(X)-\E(XZ) \geq 0$.\\
    Comment choisir $Z$ pour que $ES_\beta(X)=\E(XZ)$ ?
    
    \begin{proof}~\\
      D'après la question précédente :
      \[
        ES_\beta(X) - \E(XZ) \ = \ \dfrac{1}{1-\beta} \, 
	\E\Big( (X- r_\beta(X)) (\unq_{\Ev{X > r_\beta(X)}} - 
	(1-\beta) Z )\Big)
      \]
      Montrons donc :
      \[
        (X- r_\beta(X)) (\unq_{\Ev{X > r_\beta(X)}} - 
	(1-\beta) Z ) \geq 0
      \]
      Soit $\omega \in \Omega$. D'après l'énoncé : $0 \leq Z(\omega) 
      \leq \dfrac{1}{1-\beta}$. Or :
      \[
        0 \leq Z(\omega) \leq \dfrac{1}{1-\beta} \ \Leftrightarrow \
        0 \leq (1-\beta)Z(\omega) \leq 1 \ \Leftrightarrow \ 
        -1 \leq -(1-\beta)Z(\omega) \leq 0
      \]
      Deux cas se présentent alors.
      \begin{noliste}{$\sbullet$}
	\item \dashuline{Si $X(\omega) > r_\beta(X)$}, alors :
	\begin{noliste}{$\stimes$}
	  \item $X(\omega) - r_\beta(X) > 0$ ;
	  \item $\unq_{\Ev{X >r_\beta(X)}}(\omega) =1$. Donc :
	  $0 \ \leq \ \unq_{\Ev{X > r_\beta(X)}}(\omega) - (1-\beta)Z 
	  (\omega) \ \leq \ 1$.
	\end{noliste}
	D'où : $(X(\omega) - r_\beta(X))(\unq_{\Ev{ X > r_\beta(X)}}
	(\omega) - (1-\beta) Z(\omega)) \ \geq \ 0$.
	
	\item \dashuline{Si $X(\omega) \leq r_\beta(X)$}, alors :
	\begin{noliste}{$\stimes$}
	  \item $X(\omega) - r_\beta(X) \leq 0$ ;
	  \item $\unq_{\Ev{X >r_\beta(X)}}(\omega) =0$. Donc :
	  $-1 \ \leq \ \unq_{\Ev{X > r_\beta(X)}}(\omega) - (1-\beta)Z 
	  (\omega) \ \leq \ 0$.
	\end{noliste}
	D'où : $(X(\omega) - r_\beta(X))(\unq_{\Ev{ X > r_\beta(X)}}
	(\omega) - (1-\beta) Z(\omega)) \ \geq \ 0$.
      \end{noliste}
      Finalement : $\forall \omega \in \Omega, \ 
      (X(\omega) - r_\beta(X))(\unq_{\Ev{ X > r_\beta(X)}}
      (\omega) - (1-\beta) Z(\omega)) \ \geq \ 0$.
      \conc{$(X - r_\beta(X))(\unq_{\Ev{ X > r_\beta(X)}}
      - (1-\beta) Z) \ \geq \ 0$}
      
      Par positivité de l'espérance :
      \[
        \E\Big( (X - r_\beta(X))(\unq_{\Ev{ X > r_\beta(X)}}
	- (1-\beta) Z) \Big) \ \geq \ 0
      \]
      De plus : $\beta \in \ ]0,1[$. Donc $\dfrac{1}{1-\beta} >0$.
      \conc{Ainsi : $ES_\beta(X) - \E(XZ) \geq 0$.}
      
      Tout d'abord :
      \[
        \begin{array}{rcl@{\quad}>{\it}R{6cm}}
          & & ES_\beta(X) - \E(XZ) = 0 
          \\[.2cm]
          & \Leftrightarrow & 
          \dfrac{1}{1-\beta} \, \E\Big( (X - r_\beta(X))(\unq_{\Ev{ X > 
	  r_\beta(X)}} - (1-\beta) Z) \Big)= 0
	  \\[.4cm]
	  & \Leftrightarrow & \E\Big( (X - r_\beta(X))(\unq_{\Ev{ X > 
	  r_\beta(X)}} - (1-\beta) Z) \Big) =0
	  & (car $\dfrac{1}{1-\beta} \neq 0$)
	  \nl
	  \nl[-.2cm]
	  & \Leftrightarrow & (X - r_\beta(X))(\unq_{\Ev{ X > 
	  r_\beta(X)}} - (1-\beta) Z) = 0
	  & (par positivité de l'espérance, car $(X - 
	  r_\beta(X))(\unq_{\Ev{ X > 
	  r_\beta(X)}} - (1-\beta) Z)$ est une \var positive)
	  \nl
	  \nl[-.2cm]
	  & \Leftrightarrow & \unq_{\Ev{X > r_\beta(X)}} - 
	  (1-\beta)Z = 0
	  & (car $X$ est une \var à densité, donc n'est pas la \var 
	  constante égale à $r_\beta(X)$)
	  \nl
	  \nl[-.2cm]
	  & \Leftrightarrow & Z = \dfrac{1}{1-\beta} 
	  \unq_{\Ev{X > r_\beta(X)}}
        \end{array}
      \]
      Vérifions que $Z_0= \dfrac{1}{1-\beta}$ vérifie les deux 
      propriétés énoncées en début de question.
      \begin{noliste}{$\sbullet$}
	\item Par définition d'une \var indicatrice : $0 \ \leq \
	\unq_{\Ev{X > r_\beta(X)}} \ \leq \ 1$.\\
	Donc $0 \ \leq \ \dfrac{1}{1-\beta} \unq_{\Ev{ X> r_\beta(X)}}
	\ \leq \ \dfrac{1}{1-\beta}$.
	\conc{D'où : $0 \ \leq \ Z_0 \ \leq \ \dfrac{1}{1-\beta}$.}
	
	\item Montrons que $\E(Z_0)=1$.\\
	Pour cela, on note : $Y= \unq_{\Ev{X > r_\beta(X)}}$.
	\begin{noliste}{$\stimes$}
	  \item Par définition d'une \var indicatrice : $Y(\Omega)
	  =\{0,1\}$.
	  
	  \item Donc la \var $Y$ suit une loi de Bernoulli de paramètre 
	  :
	  \[
	   p= \Prob(\Ev{Y=1}) \ = \ \Prob\Big(\Ev{\unq_{\Ev{ X > 
	    r_\beta(X)}}=1}\Big)
	    \ = \ \Prob(\Ev{X> r_\beta(X)}) \ = \ 1-F_X(r_\beta(X))
	    \ = \ 1-\beta
	  \]
	\end{noliste}
	\conc{Ainsi : $Y \suit \Bern{1-\beta}$}
	On en déduit :
	\[
	  \E(Z_0) \ = \ \E\left(\dfrac{1}{1-\beta} \, Y\right)
	  \ = \ \dfrac{1}{1-\beta} \, \E(Y) \ = \
	  \dfrac{1}{\bcancel{1-\beta}} \bcancel{(1-\beta)} \ = \
	  1
	\]
	\conc{Ainsi : $\E(Z_0)=1$}
      \end{noliste}
      \conc{Finalement, en choisissant $Z_0= \dfrac{1}{1-\beta} 
      \unq_{\Ev{X > r_\beta(X)}}$, on obtient :
      $ES_\beta(X) \ = \ \E(XZ_0)$.}~\\[-1cm]
    \end{proof}
  \end{noliste}
  
  
  
  \newpage
  
  

  \item On note ${\cal K}$ l'ensemble des variables aléatoires $Z$ 
  sur $(\Omega,\A,\Prob)$ telles que $\E(Z)=1$ et \\
  $0\leq Z \leq \dfrac{1}{1-\beta}$.\\[.1cm]
  Justifier l'égalité : $ES_\beta(X)= \dmax{Z \in {\cal K}} 
  \E(XZ).$
  
  \begin{proof}~
    \begin{noliste}{$\sbullet$}
      \item D'après la question \itbf{20.c)} :
      \[
        \forall \, T \in {\cal K}, \ ES_\beta(X) \geq \E(XT)
      \]
      Cette inégalité est vérifiée pour tout $T \in {\cal K}$.\\[.1cm]
      Donc : $ES_\beta(X) \geq \dmax{Z \in {\cal K}}(\E(XZ))$.
      
      \item De plus : $ES_\beta(X) = \E(X Z_0)$ et $Z_0 \in {\cal K}$.
      Donc : $\dmax{Z \in {\cal K}}(\E(XZ)) = \E(XZ_0)$.
      \conc{D'où : $ES_\beta(X) \ = \ \dmax{Z \in {\cal K}} 
      (\E(XZ))$}~\\[-1.6cm]
    \end{noliste}
  \end{proof}
  
  
  \item Démontrer que, pour tout $\beta \in \ ]0,1[$, la fonction 
  $ES_\beta$ est une mesure de risque cohérente sur ${\cal D}$.
  
  \begin{proof}~\\
    Soit $\beta \in \ ]0,1[$.
    \begin{noliste}{$\sbullet$}
      \item On sait déjà, d'après les questions \itbf{15.a)} et 
      \itbf{15.b)} que $ES_\beta$ vérifie $(R_1)$ et $(R_2)$.
      
      \item Soit $(X,Y) \in {\cal D}^2$ tel que $X \leq Y$.\\
      Soit $T \in {\cal K}$. Alors, comme $T\geq 0$ : $XT \leq YT$.\\
      Donc, par croissance de l'espérance :
      \[
        \E(XT) \ \leq \ \E(YT) \ \leq \ \dmax{Z\in {\cal K}}
        (\E(YZ))
      \]
      Cette inégalité est réalisée pour tout $T\in {\cal K}$.\\
      D'où : $\dmax{Z\in {\cal K}}(\E(XZ)) \leq \dmax{Z\in {\cal K}}
      (\E(YZ))$.\\
      D'après la question précédente, on en déduit :
      \[
        ES_\beta(X) \leq ES_\beta(Y)
      \]
      \conc{Ainsi, $ES_\beta$ vérifie $(R_3)$.}
      
      \item Soit $(X,Y) \in {\cal D}^2$ tel que $X +Y\in {\cal D}$.\\
      Soit $T \in {\cal D}$. Alors, par linéarité de l'espérance :
      \[
        \E((X+Y)T) \ = \ \E(XT) + \E(YT) \ \leq \ \dmax{Z\in {\cal K}}
        (\E(XZ)) + \dmax{Z\in {\cal K}} (\E(YZ))
      \]
      Cette inégalité est réalisée pour tout $T\in {\cal K}$.
      D'où : 
      \[
	\dmax{Z\in {\cal K}}(\E((X+Y)Z)) \ 
	\leq \ \dmax{Z\in {\cal K}}(\E(XZ)) +
        \dmax{Z\in {\cal K}}(\E(YZ))
      \]
      D'après la question précédente, on en déduit :
      \[
        ES_\beta(X+Y) \ \leq \ ES_\beta(X) + ES_\beta(Y)
      \]
      \conc{Ainsi, $ES_\beta$ vérifie $(R_4)$.}
    \end{noliste}
    \conc{On en déduit que $ES_\beta$ est une mesure de risque 
    cohérente sur ${\cal D}$.}~\\[-1cm]
  \end{proof}
\end{noliste}

