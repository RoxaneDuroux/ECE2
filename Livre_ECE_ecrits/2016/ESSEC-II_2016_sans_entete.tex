\chapter*{ESSEC-II 2016 : le sujet}
  
%

\noindent
Le but du problème est d'étudier le renouvellement d'un des composants 
d'un système complexe (une machine, un réseau de distribution d'énergie 
etc...) formé d'un assemblage de différentes pièces susceptibles de 
tomber en panne. On s'intéresse donc à une de ces pièces susceptibles de 
se casser ou de tomber en panne et on se place dans la situation idéale 
où dès que la pièce est défectueuse, elle est immédiatement remplacée. 
Dans une première partie, on étudie quelques propriétés fondamentales 
des variables aléatoires discrètes. Puis, dans une deuxième partie, on 
étudie la probabilité de devoir changer la pièce un certain jour donné. 
Enfin, dans une troisième partie on cherche à estimer le temps de 
fonctionnement du système avec un certain nombre de pièces de rechange à 
disposition.\\
Dans tout le problème, on considère un espace probabilisé $(\Omega, 
\A, \Prob)$. Pour toute variable aléatoire réelle $X$ définie sur 
$(\Omega, \A, \Prob)$, on note, sous réserve d'existence, $\E(X)$ 
l'espérance de $X$ et $\V(X)$ sa variance.\\
La deuxième partie peut être traitée en admettant si besoin les 
résultats de la première partie.

\subsection*{Première partie}

\noindent
Dans cette première partie, on étudie les propriétés asymptotiques
d'une variable aléatoire $X$ à valeurs dans $\N^*$.
\begin{noliste}{1.}
  \setlength{\itemsep}{2mm}
\item 
  \begin{noliste}{a)}
  \item Montrer que pour tout entier naturel $j$ non nul : 
    \[
    \Prob(\Ev{X=j}) = \Prob(\Ev{X > j-1}) - \Prob(\Ev{X>j})
    \]
    
    


    %\newpage

    
  \item Soit $p$ un entier naturel non nul. Montrer que :
    \[
    \Sum{j=1}{p} j \, \Prob(\Ev{X=j})= \Sum{j=0}{p-1} 
    \Prob(\Ev{X>j})-p \, \Prob(\Ev{X>p})
    \]
    
    
\end{noliste}


%\newpage


\item 
  \begin{noliste}{a)}
  \item On suppose que $X$ admet une espérance $\E(X)= \mu$.
    \begin{nonoliste}{i.}
    \item Justifier la convergence de la série de terme général $k \,
      \Prob(\Ev{X=k})$.
      
      

    \item Montrer que : 
      \[
      \dlim{p \to +\infty} \Sum{k=p+1}{+\infty} k \, \Prob(\Ev{X=k})=0
      \]
      
      


      %\newpage


    \item En déduire que :
      \[
      \dlim{p \to +\infty} p \ \Prob(\Ev{X>p}) = 0
      \]

      

    \item Montrer que la série de terme général $\Prob(\Ev{X>j})$
      converge.

      


    %\newpage


  \item Montrer que : $\mu= \Sum{j=0}{+\infty} \Prob(\Ev{X >j})$.

      
    \end{nonoliste}
    
  \item On suppose que $\Sum{j=0}{+\infty} \Prob(\Ev{X >j})$ converge.
    \begin{nonoliste}{i.}
    \item Déterminer le sens de variation de la suite $(v_p)_{p \geq
        1}$ définie par :
      \[
      v_p= \Sum{j=0}{p-1} \Prob(\Ev{X >j})
      \]
      
      

    \item Comparer $\Sum{j=1}{p} j \, \Prob(\Ev{X =j})$ et
      $\Sum{j=0}{+\infty} \Prob(\Ev{X >j}).$
    \end{nonoliste}

      

    \begin{nonoliste}{i.}
      \setcounter{enumiii}{2}
    \item En déduire que $X$ admet une espérance.

      

    \end{nonoliste}
   
  \item Conclure des questions précédentes que $X$ admet une espérance
    si et seulement si la série de terme général $\Prob(\Ev{X>j})$
    converge.

    
%     
  \end{noliste}

\item On suppose dans cette question qu'il existe un réel $\alpha$
  strictement positif tel que pour tout entier naturel $j$ on ait :
  \[
  \Prob(\Ev{X >j})= \dfrac{1}{(j+1)^{\alpha}} \qquad (\ast)
  \]
  \begin{noliste}{a)}
  \item Légitimer que $(\ast)$ définit bien une loi de probabilité d'une 
    variable aléatoire à valeurs dans $\N^*$.
    
    


    % %\newpage


  \item Montrer que $X$ admet une espérance si et seulement si $\alpha$ 
    est strictement supérieur à 1.

    

  \item Montrer que pour tout entier naturel $j$ non nul :
    \[
    \Prob(\Ev{X=j}) = \dfrac{1}{j^{\alpha}} \left( 1 -
      \dfrac{1}{(1+\frac{1}{j})^{\alpha}}\right)
    \]
    
    

  \item
    \begin{nonoliste}{i.}
    \item Étudier les variations de $f : x \mapsto 1-(1+x)^{-\alpha}-\alpha 
      x$ sur $[0,1]$. 

      
      
    \item Montrer que pour tout entier naturel $j$ non nul :
      \[
      \Prob(\Ev{X=j}) \leq \frac{\alpha}{j^{1+\alpha}}
      \]

      

    \end{nonoliste}


    %\newpage

    
  \item Montrer, en utilisant le résultat de \itbf{3.c)}, que :
    \[
    \dlim{j \to +\infty} j^{\alpha+1} \, \Prob(\Ev{X=j}) = \alpha
    \]

    

    
    %\newpage


  \item Montrer que $X$ admet une variance si et seulement si $\alpha 
    >2$. 

    

  \end{noliste}
\end{noliste}

\subsection*{Deuxième partie : Étude de la probabilité de 
panne un jour donné.}
\noindent
Dans cette deuxième partie, on suppose donnée une suite de variables
aléatoires $(X_i)_{i \geq 1}$ mutuellement indépendantes et de même
loi à valeurs dans $\N^*$.\\
Pour tout entier $i$ non nul, $X_i$ représente la durée de vie en
jours du $\eme{i}$ composant en fonctionnement.\\
Soit $k$ un entier naturel non nul. On note $T_k= X_1+...+X_k$. $T_k$
représente donc le jour où le $\eme{k}$ composant tombe en panne. On
fixe un entier naturel $n$ non nul représentant un jour donné et on
considère l'événement $A_n$ : \og le composant en place le jour $n$
tombe en panne \fg{} c'est-à-dire $A_n$ : \og il existe $k$ entier
naturel non nul tel que $T_k=n$ \fg{}, et on se propose d'étudier
$\Prob(A_n)$ .

\begin{noliste}{1.}
  \setlength{\itemsep}{2mm} \setcounter{enumi}{3}
\item Pour tout entier naturel non nul $j$, on note
  $p_j=\Prob(\Ev{X_1=j})$ et $u_j=\Prob(A_j)$. On suppose que pour
  tout entier naturel non nul $j$, on a $p_j \neq 0$. On pose de plus
  par convention $u_0=1$.
  \begin{noliste}{a)}
  \item Montrer que : $u_1 = p_1$.%\\[-.8cm]

    
    

    %\newpage


  \item
    \begin{nonoliste}{i.}
    \item Montrer que : $A_2 \ = \ \Ev{X_1=2} \ \cup \
      \left(\Ev{X_1=1} \cap \Ev{X_2=1}\right)$.
      \end{nonoliste}

      

      \begin{nonoliste}{i.}
        \setcounter{enumiii}{1}
    \item En déduire $u_2$ en fonction de $p_1$ et $p_2$.

      
    \end{nonoliste}


    %\newpage


  \item Pour tout entier naturel $i$, on pose $\tilde{X}_i= X_{i+1}$.
    \begin{nonoliste}{i.}
    \item Montrer que les variables $\tilde{X}_i$ sont mutuellement
      indépendantes, indépendantes de $X_1$ et de même loi que $X_1$.

      

    \item Soit $k$ un entier naturel non nul strictement inférieur à
      $n$. Montrer que :
      \[
      A_n \cap \Ev{X_1=k}=\Ev{X_1=k} \ \cap \ \dcup{j \geq 1}{}
      \Ev{\tilde{X}_1+\tilde{X}_2+ \ldots +\tilde{X}_j=n-k}
      \]

      

    \item En déduire que pour tout entier naturel $k$ non nul
      strictement inférieur à $n$ :
      \[
      \Prob_{\Ev{X_1=k}}(A_n) = \Prob(A_{n-k})
      \]

      

    \end{nonoliste}

  \item Montrer que :
    \[
    u_n = u_{n-1} \ p_1 + \ldots + u_0 \ p_n
    \]
    
    

  \item En \Scilab{}, soit $P=[p_1,p_2,...,p_n]$ le vecteur ligne tel
    que $P(j)=p_j$ pour $j$ dans $\llb 1,n \rrb$.\\
    Écrire un programme en \Scilab{} qui calcule $u_n$ à partir de
    $P$.

    
\end{noliste}


%\newpage

  
\item Soit $\lambda$ un réel appartenant à $]0,1[$.\\[.2cm]
  \textbf{Dans cette question}, on suppose que $X_1$ suit la loi
  géométrique de paramètre $\lambda$.\\
  Pour tout entier naturel $j$ non nul, on a donc $\Prob(\Ev{X_1=j})=
  \lambda (1-\lambda)^{j-1}$.
  \begin{noliste}{a)}
  \item Calculer $\Prob(\Ev{X_1>k})$ pour tout entier naturel $k$ non nul.

    

  \item Calculer $\Prob_{\Ev{X_1 >k}}(\Ev{X_1=k+1})$.

    

  \item Montrer que pour tout entier naturel $n$ non nul : $\Prob(A_n)
    = \lambda$.

    
  \end{noliste}

\item On suppose dans cette question que $p_1$ vérifie $0<p_1<1$ et
  que $p_2 = 1 - p_1$.\\
  Pour simplifier, on posera $p = p_1 = 1 - p_2$.
  \begin{noliste}{a)}
  \item Que vaut $p_i$ pour $i$ supérieur ou égal à 3 ?

    


    %\newpage


  \item Soit la matrice $M =
    \begin{smatrix}
      p & 1-p\\
      1 & 0 
    \end{smatrix}$.\\
    Montrer que pour tout entier naturel $n$ supérieur ou égal à 2 : $
    \begin{smatrix}
      u_n\\ 
      u_{n-1} 
    \end{smatrix}
    = M \ 
    \begin{smatrix}
      u_{n-1}\\
      u_{n-2} 
    \end{smatrix}
    $.

    
    
  \item 
    \begin{nonoliste}{i.}
    \item Diagonaliser la matrice $M$.
      
      
      
    \item Montrer que :
      \[
      M^{n-1}=\dfrac{1}{2-p}
      \begin{smatrix} 
        1 & 1-p\\
        1 & 1-p
      \end{smatrix}
      + \dfrac{(p-1)^{n-1}}{2-p}
      \begin{smatrix} 
        1-p & p-1\\
        -1 & 1 
      \end{smatrix}
      \]

      

    \end{nonoliste}
    
  \item 
    \begin{nonoliste}{i.}
    \item Exprimer $u_n$ en fonction de $p$ et de $n$. 

      

    \item Déterminer $\dlim{n \to +\infty} u_n$. 

      

    \end{nonoliste}
  \end{noliste}
\end{noliste}

\subsection*{Troisième partie : Étude de la durée de fonctionnement.}

\noindent 
Comme dans la partie précédente, on suppose donnée une suite de
variables aléatoires $(X_{i})_{i \geq 1}$ indépendantes et de même
loi, telle que pour tout entier $i$ non nul, $X_{i}$ représente la
durée de vie en jours du $i$-ème composant en fonctionnement.\\
Soit $k$ un entier naturel non nul. On étudie dans cette partie la
durée de fonctionnement prévisible du système si on a $k$ composants à
disposition (y compris celui installé au départ). \\
On notera toujours $T_{k} = X_{1} +... + X_{k}$.\\
On suppose dans cette partie qu'il existe un réel $\alpha >1$ tel que
pour tout entier naturel $j$ on ait :
\[
\Prob\left(\Ev{X_{1} >j}\right) = \frac{1}{(j + 1)^{\alpha}}
\]
En particulier, dans toute cette partie, $X_{1}$ admet une espérance,
on l'on notera $\mu = \E(X_{1})$.
\begin{noliste}{1.}
  \setlength{\itemsep}{4mm} %
  \setcounter{enumi}{6}
\item Que vaut $\E(T_{k})$ ? 
  
  


  %\newpage


\item On suppose, \textbf{dans cette question}, que $\alpha$ est
  strictement supérieur à 2. La variable aléatoire $X_{1}$ admet donc
  une variance $\sigma^{2}$.
  \begin{noliste}{a)}
    \setlength{\itemsep}{2mm}
  \item Calculer $\V(T_{k})$.

    

  \item Montrer que pour tout réel $\eps$ strictement positif, 
    \[
    \Prob\left(\Ev{ |T_{k} - k\mu| \geq k \eps}\right) \leq
    \dfrac{\sigma^{2}}{k \eps^{2}}
    \]
    
    

    
    %\newpage


  \item Déduire que, pour tout réel strictement positif $\eps$, on a :
    \[
    \dlim{k \to + \infty} \Prob\left(\Ev{\frac{T_{k}}{k} \in \ ] \mu -
        \eps, \mu + \eps[} \ \right) = 1
    \]

    
  \end{noliste}

\item On suppose maintenant uniquement que $\alpha >1$ et donc que
  $X_{1}$ n'a pas nécessairement de variance d'où l'impossibilité
  d'appliquer la méthode précédente. On va mettre en \oe{}uvre ce
  qu'on appelle une méthode de troncation.\\
  On fixe un entier naturel $m$ strictement positif. Pour tout entier
  naturel non nul $i$, on définit deux variables aléatoires
  $Y_{i}^{(m)}$ et $Z_{i}^{(m)}$ de la façon suivante
  \[
  Y_{i}^{(m)} = \left\{
    \begin{array}{cl}
      X_{i} & \text{ si } X_{i} \leq m, \\
      0 & \text{ sinon}.
    \end{array}
  \right. %
  \qquad %
  Z_{i}^{(m)} = %
  \left\{
    \begin{array}{cl}
      X_{i} & \text{ si } X_{i} > m, \\
      0 & \text{ sinon}.
    \end{array}
  \right.
  \]
\end{noliste}
\begin{liste}{a)}
  \setlength{\itemsep}{2mm}
\item Montrer que $X_{i} = Y_{i}^{(m)} + Z_{i}^{(m)}$.
  
  

\item ~\\[-1.15cm]
\end{liste}
\begin{liste}{\ i.}
\item En utilisant la question \itbf{3.d)ii.}, montrer que :
  \[
  \E\big(Z_{1}^{(m)} \big) \ \leq \ \Sum{i = m + 1}{+ \infty}
  \dfrac{\alpha}{i^{\alpha}}
  \]
  
  


  %\newpage


\item Montrer que :
  \[
  \E\big(Z_{1}^{(m)} \big) \ \leq \ \dint{m}{+ \infty}
  \dfrac{\alpha}{x^{\alpha}} \dx
  \]

  
  
\item Calculer :
  \[
  \dint{m}{+ \infty} \dfrac{\alpha}{x^{\alpha}} \dx
  \]

  

\item En déduire que :
  \[
  \dlim{m \to + \infty} \E\big(Z_{1}^{(m)} \big) = 0
  \]

  
  
\item Montrer que
  \[
  \dlim{m \to + \infty} \E\big(Y_{1}^{(m)} \big) = \mu
  \]

  
\end{liste}


%\newpage


\begin{liste}{a)}
  \setcounter{enumi}{2}
\item ~\\[-1.15cm]
\end{liste}
\begin{liste}{\ i.}
\item Montrer que
  \[
  \big( Y_{1}^{(m)} \big)^{2} \leq m X_{1}
  \]
  
  
  
\item En déduire que
  \[
  \V\big(Y_{1}^{(m)} \big) \leq m \mu
  \]
  
  
\end{liste}

\begin{liste}{a)}
  \setcounter{enumi}{3}
\item Soit $\eps$ un réel strictement positif. Montrer qu'il existe
  un entier naturel $m_{0}$ non nul tel que pour tout entier naturel
  $m$ supérieur ou égal à $m_{0}$,
  \[
  \dfrac{\alpha}{\alpha-1} m^{1-\alpha} \leq \eps
  \]
  
  
  
  
  %\newpage
  
  
  \noindent
  \textbf{Jusqu'à la fin du problème, $m$ désignera un entier
    supérieur ou égal à $m_{0}$.}
  
\item On note, pour tout entier naturel $k$ non nul
  \[
  U_{k}^{(m)} = \Sum{i = 1}{k} Y_{i}^{(m)} \quad \text{ et } \quad
  V_{k}^{(m)} = \Sum{i = 1}{k} Z_{i}^{(m)}
  \]
  Vérifier que :
  \[
  T_{k} = U_{k}^{(m)} + V_{k}^{(m)}
  \]
  
  
  
  % \begin{liste}{a)}
  %   \setcounter{enumi}{2}
  % \item ~\\[-1.15cm]
  % \end{liste}
  % \begin{liste}{\ i.}
  
\item ~\\[-1.15cm]
\end{liste}
\begin{liste}{\ i.}
\item Montrer que :
  \[
  \E\big(V_{k}^{(m)} \big) \ \leq \ k \ \dfrac{\alpha}{\alpha-1}
  m^{1-\alpha}
  \]
  
  
  
\item En déduire que :
  \[
  \Prob\left(\Ev{V_{k}^{(m)} \geq k \eps} \right) \ \leq \ 
  \dfrac{\alpha}{\alpha-1} \dfrac{m^{1-\alpha}}{\eps}
  \]

  
\end{liste}

\begin{liste}{a)}
  \setcounter{enumi}{6}
\item
  \begin{nonoliste}{i.}
  \item Montrer que :
    \[
    \E\big(U_{k}^{(m)} \big) \ \geq \ k \mu - k
    \dfrac{\alpha}{\alpha-1} m^{1-\alpha}
    \]

    

  \item En déduire que :
    \[
    \left| \ \E(U_{k}^{(m)}) -k \mu \ \right| \ \leq \  k \eps
    \]

    


    %\newpage


  \item Montrer que :
    \[
    \Prob\left(\Ev{ |U_{k}^{(m)} - k \mu| \ \geq \ 2k \eps}\right) \
    \leq \ \Prob\left(\Ev{ |U_{k}^{(m)} - \E(U_{k}^{(m)})| \ \geq \ k
        \eps}\right)
    \]

    

  \item Montrer que :
    \[
    \V\big(U_{k}^{(m)} \big) \ \leq \ km \mu
    \]

    

  \item En déduire que :
    \[
    \Prob\left(\Ev{ |U_{k}^{(m)} - k \mu| \ \geq \  2k \eps}\right) \ \leq
    \ \dfrac{m \mu}{k \eps^{2}}
    \]

    
  \end{nonoliste}

\item
  \begin{nonoliste}{i.}
  \item Montrer que pour tout couple d'événements $A$ et $B$ dans
    $\mathcal{A}$, on a :
    \[
    \Prob(A \cap B) \ \geq \  \Prob\left(A\right) + \Prob\left(B \right)-1
    \]

    


    %\newpage


  \item En appliquant l'inégalité précédente aux événements :
    \[
    A = \Ev{V_{k}^{(m)} < k \eps} \quad \text{ et } \quad B =
    \Ev{U_{k}^{(m)} \in \ ] k(\mu - 2 \eps), k(\mu + 2 \eps)[ }
    \]
    montrer que :
    \[
    \Prob\left(\Ev{T_{k} \in \ ]k(\mu - 3 \eps), k(\mu + 3 \eps)[ \ }
      \ \right) \ \geq \ \Prob\left(\Ev{ V_{k}^{(m)} < k \eps}\right)
    + \Prob\left(\Ev{U_{k}^{(m)} \in \ ]k(\mu - 2 \eps), k(\mu + 2
        \eps)[} \ \right) - 1
    \]

    


    %\newpage


  \item Déduire des questions précédentes que pour tout réel $\eps$
    strictement positif, et pour tout entier $m$ supérieur ou égal à
    $m_{0}$, on a pour tout entier naturel $k$ non nul :
    \[
    \Prob\left(\Ev{\ T_{k} \in \ ]k(\mu - 3 \eps), k(\mu + 3 \eps)[ \
      } \ \right) \ \geq \ 1 - \dfrac{\alpha}{\alpha-1}
    \dfrac{m^{1-\alpha}}{\eps} - \dfrac{m \mu}{k \eps^{2}}
    \]

    

  \item Pour $k$ assez grand, appliquer l'inégalité précédente à un
    entier $m_{k} \in [\sqrt{k}, 2\sqrt{k}]$ et conclure que :
    \[
    \dlim{k \to + \infty} \Prob\left(\Ev{\frac{T_{k}}{k} \in \ ] \mu -
        3 \eps, \mu + 3 \eps[} \ \right) = 1
    \]

    
  \end{nonoliste}
\end{liste}
%\end{noliste}




