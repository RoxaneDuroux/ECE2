\documentclass[11pt]{article}%
\usepackage{geometry}%
\geometry{a4paper,
  lmargin=2cm,rmargin=2cm,tmargin=2.5cm,bmargin=2.5cm}

% \usepackage{array}
\usepackage{paralist}

\usepackage[svgnames, usenames, dvipsnames]{xcolor}
\xdefinecolor{RecColor}{named}{Aqua}
\xdefinecolor{IncColor}{named}{Aqua}
\xdefinecolor{ImpColor}{named}{PaleGreen}

% \usepackage{frcursive}

\usepackage{adjustbox}

%%%%%%%%%%%
\newcommand{\cRB}[1]{{\color{Red} \pmb{#1}}} %
\newcommand{\cR}[1]{{\color{Red} {#1}}} %
\newcommand{\cBB}[1]{{\color{Blue} \pmb{#1}}}
\newcommand{\cB}[1]{{\color{Blue} {#1}}}
\newcommand{\cGB}[1]{{\color{LimeGreen} \pmb{#1}}}
\newcommand{\cG}[1]{{\color{LimeGreen} {#1}}}

%%%%%%%%%%

\usepackage{diagbox} %
\usepackage{colortbl} %
\usepackage{multirow} %
\usepackage{pgf} %
\usepackage{environ} %
\usepackage{fancybox} %
\usepackage{textcomp} %
\usepackage{marvosym} %

%%%%%%%%%% pour qu'une cellcolor ne recouvre pas le trait du tableau
\usepackage{hhline}%

\usepackage{pgfplots}
\pgfplotsset{compat=1.10}
\usepgfplotslibrary{patchplots}
\usepgfplotslibrary{fillbetween}
\usepackage{tikz,tkz-tab}
\usepackage{ifthen}
\usepackage{calc}
\usetikzlibrary{calc,decorations.pathreplacing,arrows,positioning} 
\usetikzlibrary{fit,shapes,backgrounds}
\usepackage[nomessages]{fp}% http://ctan.org/pkg/fp

\usetikzlibrary{matrix,arrows,decorations.pathmorphing,
  decorations.pathreplacing} 

\newcommand{\myunit}{1 cm}
\tikzset{
    node style sp/.style={draw,circle,minimum size=\myunit},
    node style ge/.style={circle,minimum size=\myunit},
    arrow style mul/.style={draw,sloped,midway,fill=white},
    arrow style plus/.style={midway,sloped,fill=white},
}

%%%%%%%%%%%%%%
%%%%% écrire des inférieur égal ou supérieur égal avec typographie
%%%%% francaise
%%%%%%%%%%%%%

\renewcommand{\geq}{\geqslant}
\renewcommand{\leq}{\leqslant}
\renewcommand{\emptyset}{\varnothing}

\newcommand{\Leq}{\leqslant}
\newcommand{\Geq}{\geqslant}

%%%%%%%%%%%%%%
%%%%% Macro Celia
%%%%%%%%%%%%%

\newcommand{\ff}[2]{\left[#1, #2\right]} %
\newcommand{\fo}[2]{\left[#1, #2\right[} %
\newcommand{\of}[2]{\left]#1, #2\right]} %
\newcommand{\soo}[2]{\left]#1, #2\right[} %
\newcommand{\abs}[1]{\left|#1\right|} %
\newcommand{\Ent}[1]{\left\lfloor #1 \right\rfloor} %


%%%%%%%%%%%%%%
%%%%% tikz : comment dessiner un "oeil"
%%%%%%%%%%%%%

\newcommand{\eye}[4]% size, x, y, rotation
{ \draw[rotate around={#4:(#2,#3)}] (#2,#3) -- ++(-.5*55:#1) (#2,#3)
  -- ++(.5*55:#1); \draw (#2,#3) ++(#4+55:.75*#1) arc
  (#4+55:#4-55:.75*#1);
  % IRIS
  \draw[fill=gray] (#2,#3) ++(#4+55/3:.75*#1) arc
  (#4+180-55:#4+180+55:.28*#1);
  % PUPIL, a filled arc
  \draw[fill=black] (#2,#3) ++(#4+55/3:.75*#1) arc
  (#4+55/3:#4-55/3:.75*#1);%
}


%%%%%%%%%%
%% discontinuité fonction
\newcommand\pointg[2]{%
  \draw[color = red, very thick] (#1+0.15, #2-.04)--(#1, #2-.04)--(#1,
  #2+.04)--(#1+0.15, #2+.04);%
}%

\newcommand\pointd[2]{%
  \draw[color = red, very thick] (#1-0.15, #2+.04)--(#1, #2+.04)--(#1,
  #2-.04)--(#1-0.15, #2-.04);%
}%

%%%%%%%%%%
%%% 1 : position abscisse, 2 : position ordonnée, 3 : taille, 4 : couleur
%%%%%%%%%%
% \newcommand\pointG[4]{%
%   \draw[color = #4, very thick] (#1+#3, #2-(#3/3.75))--(#1,
%   #2-(#3/3.75))--(#1, #2+(#3/3.75))--(#1+#3, #2+(#3/3.75)) %
% }%

\newcommand\pointG[4]{%
  \draw[color = #4, very thick] ({#1+#3/3.75}, {#2-#3})--(#1,
  {#2-#3})--(#1, {#2+#3})--({#1+#3/3.75}, {#2+#3}) %
}%

\newcommand\pointD[4]{%
  \draw[color = #4, very thick] ({#1-#3/3.75}, {#2+#3})--(#1,
  {#2+#3})--(#1, {#2-#3})--({#1-#3/3.75}, {#2-#3}) %
}%

\newcommand\spointG[4]{%
  \draw[color = #4, very thick] ({#1+#3/1.75}, {#2-#3})--(#1,
  {#2-#3})--(#1, {#2+#3})--({#1+#3/1.75}, {#2+#3}) %
}%

\newcommand\spointD[4]{%
  \draw[color = #4, very thick] ({#1-#3/2}, {#2+#3})--(#1,
  {#2+#3})--(#1, {#2-#3})--({#1-#3/2}, {#2-#3}) %
}%

%%%%%%%%%%

\newcommand{\Pb}{\mathtt{P}}

%%%%%%%%%%%%%%%
%%% Pour citer un précédent item
%%%%%%%%%%%%%%%
\newcommand{\itbf}[1]{{\small \bf \textit{#1}}}


%%%%%%%%%%%%%%%
%%% Quelques couleurs
%%%%%%%%%%%%%%%

\xdefinecolor{cancelcolor}{named}{Red}
\xdefinecolor{intI}{named}{ProcessBlue}
\xdefinecolor{intJ}{named}{ForestGreen}

%%%%%%%%%%%%%%%
%%%%%%%%%%%%%%%
% barrer du texte
\usetikzlibrary{shapes.misc}

\makeatletter
% \definecolor{cancelcolor}{rgb}{0.127,0.372,0.987}
\newcommand{\tikz@bcancel}[1]{%
  \begin{tikzpicture}[baseline=(textbox.base), inner sep=0pt]
    \node[strike out, draw] (textbox) {#1}[thick, color=cancelcolor];
    \useasboundingbox (textbox);
  \end{tikzpicture}%
}
\newcommand{\bcancel}[1]{%
  \relax\ifmmode
    \mathchoice{\tikz@bcancel{$\displaystyle#1$}}
               {\tikz@bcancel{$\textstyle#1$}}
               {\tikz@bcancel{$\scriptstyle#1$}}
               {\tikz@bcancel{$\scriptscriptstyle#1$}}
  \else
    \tikz@bcancel{\strut#1}%
  \fi
}
\newcommand{\tikz@xcancel}[1]{%
  \begin{tikzpicture}[baseline=(textbox.base),inner sep=0pt]
  \node[cross out,draw] (textbox) {#1}[thick, color=cancelcolor];
  \useasboundingbox (textbox);
  \end{tikzpicture}%
}
\newcommand{\xcancel}[1]{%
  \relax\ifmmode
    \mathchoice{\tikz@xcancel{$\displaystyle#1$}}
               {\tikz@xcancel{$\textstyle#1$}}
               {\tikz@xcancel{$\scriptstyle#1$}}
               {\tikz@xcancel{$\scriptscriptstyle#1$}}
  \else
    \tikz@xcancel{\strut#1}%
  \fi
}
\makeatother

\newcommand{\xcancelRA}{\xcancel{\rule[-.15cm]{0cm}{.5cm} \Rightarrow
    \rule[-.15cm]{0cm}{.5cm}}}

%%%%%%%%%%%%%%%%%%%%%%%%%%%%%%%%%%%
%%%%%%%%%%%%%%%%%%%%%%%%%%%%%%%%%%%

\newcommand{\vide}{\multicolumn{1}{c}{}}

%%%%%%%%%%%%%%%%%%%%%%%%%%%%%%%%%%%
%%%%%%%%%%%%%%%%%%%%%%%%%%%%%%%%%%%


\usepackage{multicol}
% \usepackage[latin1]{inputenc}
% \usepackage[T1]{fontenc}
\usepackage[utf8]{inputenc}
\usepackage[T1]{fontenc}
\usepackage[normalem]{ulem}
\usepackage[french]{babel}

\usepackage{url}    
\usepackage{hyperref}
\hypersetup{
  backref=true,
  pagebackref=true,
  hyperindex=true,
  colorlinks=true,
  breaklinks=true,
  urlcolor=blue,
  linkcolor=black,
  %%%%%%%%
  % ATTENTION : red changé en black pour le Livre !
  %%%%%%%%
  bookmarks=true,
  bookmarksopen=true
}

%%%%%%%%%%%%%%%%%%%%%%%%%%%%%%%%%%%%%%%%%%%
%% Pour faire des traits diagonaux dans les tableaux
%% Nécessite slashbox.sty
%\usepackage{slashbox}

\usepackage{tipa}
\usepackage{verbatim,listings}
\usepackage{graphicx}
\usepackage{fancyhdr}
\usepackage{mathrsfs}
\usepackage{pifont}
\usepackage{tablists}
\usepackage{dsfont,amsfonts,amssymb,amsmath,amsthm,stmaryrd,upgreek,manfnt}
\usepackage{enumerate}

%\newcolumntype{M}[1]{p{#1}}
\newcolumntype{C}[1]{>{\centering}m{#1}}
\newcolumntype{R}[1]{>{\raggedright}m{#1}}
\newcolumntype{L}[1]{>{\raggedleft}m{#1}}
\newcolumntype{P}[1]{>{\raggedright}p{#1}}
\newcolumntype{B}[1]{>{\raggedright}b{#1}}
\newcolumntype{Q}[1]{>{\raggedright}t{#1}}

\newcommand{\alias}[2]{
\providecommand{#1}{}
\renewcommand{#1}{#2}
}
\alias{\R}{\mathbb{R}}
\alias{\N}{\mathbb{N}}
\alias{\Z}{\mathbb{Z}}
\alias{\Q}{\mathbb{Q}}
\alias{\C}{\mathbb{C}}
\alias{\K}{\mathbb{K}}

%%%%%%%%%%%%
%% rendre +infty et -infty plus petits
%%%%%%%%%%%%
\newcommand{\sinfty}{{\scriptstyle \infty}}

%%%%%%%%%%%%%%%%%%%%%%%%%%%%%%%
%%%%% macros TP Scilab %%%%%%%%
\newcommand{\Scilab}{\textbf{Scilab}} %
\newcommand{\Scinotes}{\textbf{SciNotes}} %
\newcommand{\faire}{\noindent $\blacktriangleright$ } %
\newcommand{\fitem}{\scalebox{.8}{$\blacktriangleright$}} %
\newcommand{\entree}{{\small\texttt{ENTRÉE}}} %
\newcommand{\tab}{{\small\texttt{TAB}}} %
\newcommand{\mt}[1]{\mathtt{#1}} %
% guillemets droits

\newcommand{\ttq}{\textquotesingle} %

\newcommand{\reponse}[1]{\longboxed{
    \begin{array}{C{0.9\textwidth}}
      \nl[#1]
    \end{array}
  }} %

\newcommand{\reponseR}[1]{\longboxed{
    \begin{array}{R{0.9\textwidth}}
      #1
    \end{array}
  }} %

\newcommand{\reponseC}[1]{\longboxed{
    \begin{array}{C{0.9\textwidth}}
      #1
    \end{array}
  }} %

\colorlet{pyfunction}{Blue}
\colorlet{pyCle}{Magenta}
\colorlet{pycomment}{LimeGreen}
\colorlet{pydoc}{Cyan}
% \colorlet{SansCo}{white}
% \colorlet{AvecCo}{black}

\newcommand{\visible}[1]{{\color{ASCo}\colorlet{pydoc}{pyDo}\colorlet{pycomment}{pyCo}\colorlet{pyfunction}{pyF}\colorlet{pyCle}{pyC}\colorlet{function}{sciFun}\colorlet{var}{sciVar}\colorlet{if}{sciIf}\colorlet{comment}{sciComment}#1}} %

%%%% à changer ????
\newcommand{\invisible}[1]{{\color{ASCo}\colorlet{pydoc}{pyDo}\colorlet{pycomment}{pyCo}\colorlet{pyfunction}{pyF}\colorlet{pyCle}{pyC}\colorlet{function}{sciFun}\colorlet{var}{sciVar}\colorlet{if}{sciIf}\colorlet{comment}{sciComment}#1}} %

\newcommand{\invisibleCol}[2]{{\color{#1}#2}} %

\NewEnviron{solution} %
{ %
  \Boxed{
    \begin{array}{>{\color{ASCo}} R{0.9\textwidth}}
      \colorlet{pycomment}{pyCo}
      \colorlet{pydoc}{pyDo}
      \colorlet{pyfunction}{pyF}
      \colorlet{pyCle}{pyC}
      \colorlet{function}{sciFun}
      \colorlet{var}{sciVar}
      \colorlet{if}{sciIf}
      \colorlet{comment}{sciComment}
      \BODY
    \end{array}
  } %
} %

\NewEnviron{solutionC} %
{ %
  \Boxed{
    \begin{array}{>{\color{ASCo}} C{0.9\textwidth}}
      \colorlet{pycomment}{pyCo}
      \colorlet{pydoc}{pyDo}
      \colorlet{pyfunction}{pyF}
      \colorlet{pyCle}{pyC}
      \colorlet{function}{sciFun}
      \colorlet{var}{sciVar}
      \colorlet{if}{sciIf}
      \colorlet{comment}{sciComment}
      \BODY
    \end{array}
  } %
} %

\newcommand{\invite}{--\!\!>} %

%%%%% nouvel environnement tabular pour retour console %%%%
\colorlet{ConsoleColor}{Black!12}
\colorlet{function}{Red}
\colorlet{var}{Maroon}
\colorlet{if}{Magenta}
\colorlet{comment}{LimeGreen}

\newcommand{\tcVar}[1]{\textcolor{var}{\bf \small #1}} %
\newcommand{\tcFun}[1]{\textcolor{function}{#1}} %
\newcommand{\tcIf}[1]{\textcolor{if}{#1}} %
\newcommand{\tcFor}[1]{\textcolor{if}{#1}} %

\newcommand{\moins}{\!\!\!\!\!\!- }
\newcommand{\espn}{\!\!\!\!\!\!}

\usepackage{booktabs,varwidth} \newsavebox\TBox
\newenvironment{console}
{\begin{lrbox}{\TBox}\varwidth{\linewidth}
    \tabular{>{\tt\small}R{0.84\textwidth}}
    \nl[-.4cm]} {\endtabular\endvarwidth\end{lrbox}%
  \fboxsep=1pt\colorbox{ConsoleColor}{\usebox\TBox}}

\newcommand{\lInv}[1]{%
  $\invite$ #1} %

\newcommand{\lAns}[1]{%
  \qquad ans \ = \nl %
  \qquad \qquad #1} %

\newcommand{\lVar}[2]{%
  \qquad #1 \ = \nl %
  \qquad \qquad #2} %

\newcommand{\lDisp}[1]{%
  #1 %
} %

\newcommand{\ligne}[1]{\underline{\small \tt #1}} %

\newcommand{\ligneAns}[2]{%
  $\invite$ #1 \nl %
  \qquad ans \ = \nl %
  \qquad \qquad #2} %

\newcommand{\ligneVar}[3]{%
  $\invite$ #1 \nl %
  \qquad #2 \ = \nl %
  \qquad \qquad #3} %

\newcommand{\ligneErr}[3]{%
  $\invite$ #1 \nl %
  \quad !-{-}error #2 \nl %
  #3} %
%%%%%%%%%%%%%%%%%%%%%% 

\newcommand{\bs}[1]{\boldsymbol{#1}} %
\newcommand{\nll}{\nl[.4cm]} %
\newcommand{\nle}{\nl[.2cm]} %
%% opérateur puissance copiant l'affichage Scilab
%\newcommand{\puis}{\!\!\!~^{\scriptscriptstyle\pmb{\wedge}}}
\newcommand{\puis}{\mbox{$\hspace{-.1cm}~^{\scriptscriptstyle\pmb{\wedge}}
    \hspace{0.05cm}$}} %
\newcommand{\pointpuis}{.\mbox{$\hspace{-.15cm}~^{\scriptscriptstyle\pmb{\wedge}}$}} %
\newcommand{\Sfois}{\mbox{$\mt{\star}$}} %

%%%%% nouvel environnement tabular pour les encadrés Scilab %%%%
\newenvironment{encadre}
{\begin{lrbox}{\TBox}\varwidth{\linewidth}
    \tabular{>{\tt\small}C{0.1\textwidth}>{\small}R{0.7\textwidth}}}
  {\endtabular\endvarwidth\end{lrbox}%
  \fboxsep=1pt\longboxed{\usebox\TBox}}

\newenvironment{encadreL}
{\begin{lrbox}{\TBox}\varwidth{\linewidth}
    \tabular{>{\tt\small}C{0.25\textwidth}>{\small}R{0.6\textwidth}}}
  {\endtabular\endvarwidth\end{lrbox}%
  \fboxsep=1pt\longboxed{\usebox\TBox}}

\newenvironment{encadreF}
{\begin{lrbox}{\TBox}\varwidth{\linewidth}
    \tabular{>{\tt\small}C{0.2\textwidth}>{\small}R{0.70\textwidth}}}
  {\endtabular\endvarwidth\end{lrbox}%
  \fboxsep=1pt\longboxed{\usebox\TBox}}

\newenvironment{encadreLL}[2]
{\begin{lrbox}{\TBox}\varwidth{\linewidth}
    \tabular{>{\tt\small}C{#1\textwidth}>{\small}R{#2\textwidth}}}
  {\endtabular\endvarwidth\end{lrbox}%
  \fboxsep=1pt\longboxed{\usebox\TBox}}

%%%%% nouvel environnement tabular pour les script et fonctions %%%%
\newcommand{\commentaireDL}[1]{\multicolumn{1}{l}{\it
    \textcolor{comment}{$\slash\slash$ #1}}}

\newcommand{\commentaire}[1]{{\textcolor{comment}{$\slash\slash$ #1}}}

\newcounter{cptcol}

\newcommand{\nocount}{\multicolumn{1}{c}{}}

\newcommand{\sciNo}[1]{{\small \underbar #1}}

\NewEnviron{scilab}{ %
  \setcounter{cptcol}{0}
  \begin{center}
    \longboxed{
      \begin{tabular}{>{\stepcounter{cptcol}{\tiny \underbar
              \thecptcol}}c>{\tt}l}
        \BODY
      \end{tabular}
    }
  \end{center}
}

\NewEnviron{scilabNC}{ %
  \begin{center}
    \longboxed{
      \begin{tabular}{>{\tt}l} %
          \BODY
      \end{tabular}
    }
  \end{center}
}

\NewEnviron{scilabC}[1]{ %
  \setcounter{cptcol}{#1}
  \begin{center}
    \longboxed{
      \begin{tabular}{>{\stepcounter{cptcol}{\tiny \underbar
              \thecptcol}}c>{\tt}l}
        \BODY
      \end{tabular}
    }
  \end{center}
}

\newcommand{\scisol}[1]{ %
  \setcounter{cptcol}{0}
  \longboxed{
    \begin{tabular}{>{\stepcounter{cptcol}{\tiny \underbar
            \thecptcol}}c>{\tt}l}
      #1
    \end{tabular}
  }
}

\newcommand{\scisolNC}[1]{ %
  \longboxed{
    \begin{tabular}{>{\tt}l}
      #1
    \end{tabular}
  }
}

\newcommand{\scisolC}[2]{ %
  \setcounter{cptcol}{#1}
  \longboxed{
    \begin{tabular}{>{\stepcounter{cptcol}{\tiny \underbar
            \thecptcol}}c>{\tt}l}
      #2
    \end{tabular}
  }
}

\NewEnviron{syntaxe}{ %
  % \fcolorbox{black}{Yellow!20}{\setlength{\fboxsep}{3mm}
  \shadowbox{
    \setlength{\fboxsep}{3mm}
    \begin{tabular}{>{\tt}l}
      \BODY
    \end{tabular}
  }
}

%%%%% fin macros TP Scilab %%%%%%%%
%%%%%%%%%%%%%%%%%%%%%%%%%%%%%%%%%%%

%%%%%%%%%%%%%%%%%%%%%%%%%%%%%%%%%%%
%%%%% TP Python - listings %%%%%%%%
%%%%%%%%%%%%%%%%%%%%%%%%%%%%%%%%%%%
\newcommand{\Python}{\textbf{Python}} %

\lstset{% general command to set parameter(s)
basicstyle=\ttfamily\small, % print whole listing small
keywordstyle=\color{blue}\bfseries\underbar,
%% underlined bold black keywords
frame=lines,
xleftmargin=10mm,
numbers=left,
numberstyle=\tiny\underbar,
numbersep=10pt,
%identifierstyle=, % nothing happens
commentstyle=\color{green}, % white comments
%%stringstyle=\ttfamily, % typewriter type for strings
showstringspaces=false}

\newcommand{\pysolCpt}[2]{
  \setcounter{cptcol}{#1}
  \longboxed{
    \begin{tabular}{>{\stepcounter{cptcol}{\tiny \underbar
            \thecptcol}}c>{\tt}l}
        #2
      \end{tabular}
    }
} %

\newcommand{\pysol}[1]{
  \setcounter{cptcol}{0}
  \longboxed{
    \begin{tabular}{>{\stepcounter{cptcol}{\tiny \underbar
            \thecptcol}}c>{\tt}l}
        #1
      \end{tabular}
    }
} %

% \usepackage[labelsep=endash]{caption}

% avec un caption
\NewEnviron{pythonCap}[1]{ %
  \renewcommand{\tablename}{Programme}
  \setcounter{cptcol}{0}
  \begin{center}
    \longboxed{
      \begin{tabular}{>{\stepcounter{cptcol}{\tiny \underbar
              \thecptcol}}c>{\tt}l}
        \BODY
      \end{tabular}
    }
    \captionof{table}{#1}
  \end{center}
}

\NewEnviron{python}{ %
  \setcounter{cptcol}{0}
  \begin{center}
    \longboxed{
      \begin{tabular}{>{\stepcounter{cptcol}{\tiny \underbar
              \thecptcol}}c>{\tt}l}
        \BODY
      \end{tabular}
    }
  \end{center}
}

\newcommand{\pyVar}[1]{\textcolor{var}{\bf \small #1}} %
\newcommand{\pyFun}[1]{\textcolor{pyfunction}{#1}} %
\newcommand{\pyCle}[1]{\textcolor{pyCle}{#1}} %
\newcommand{\pyImp}[1]{{\bf #1}} %

%%%%% commentaire python %%%%
\newcommand{\pyComDL}[1]{\multicolumn{1}{l}{\textcolor{pycomment}{\#
      #1}}}

\newcommand{\pyCom}[1]{{\textcolor{pycomment}{\# #1}}}
\newcommand{\pyDoc}[1]{{\textcolor{pydoc}{#1}}}

\newcommand{\pyNo}[1]{{\small \underbar #1}}

%%%%%%%%%%%%%%%%%%%%%%%%%%%%%%%%%%%
%%%%%% Système linéaire paramétré : écrire les opérations au-dessus
%%%%%% d'un symbole équivalent
%%%%%%%%%%%%%%%%%%%%%%%%%%%%%%%%%%%

\usepackage{systeme}

\NewEnviron{arrayEq}{ %
  \stackrel{\scalebox{.6}{$
      \begin{array}{l} 
        \BODY \\[.1cm]
      \end{array}$}
  }{\Longleftrightarrow}
}

\NewEnviron{arrayEg}{ %
  \stackrel{\scalebox{.6}{$
      \begin{array}{l} 
        \BODY \\[.1cm]
      \end{array}$}
  }{=}
}

\NewEnviron{operationEq}{ %
  \scalebox{.6}{$
    \begin{array}{l} 
      \scalebox{1.6}{$\mbox{Opérations :}$} \\[.2cm]
      \BODY \\[.1cm]
    \end{array}$}
}

% \NewEnviron{arraySys}[1]{ %
%   \sysdelim\{.\systeme[#1]{ %
%     \BODY %
%   } %
% }

%%%%%

%%%%%%%%%%
%%%%%%%%%% ESSAI
\newlength\fboxseph
\newlength\fboxsepva
\newlength\fboxsepvb

\setlength\fboxsepva{0.2cm}
\setlength\fboxsepvb{0.2cm}
\setlength\fboxseph{0.2cm}

\makeatletter

\def\longboxed#1{\leavevmode\setbox\@tempboxa\hbox{\color@begingroup%
\kern\fboxseph{\m@th$\displaystyle #1 $}\kern\fboxseph%
\color@endgroup }\my@frameb@x\relax}

\def\my@frameb@x#1{%
  \@tempdima\fboxrule \advance\@tempdima \fboxsepva \advance\@tempdima
  \dp\@tempboxa\hbox {%
    \lower \@tempdima \hbox {%
      \vbox {\hrule\@height\fboxrule \hbox{\vrule\@width\fboxrule #1
          \vbox{%
            \vskip\fboxsepva \box\@tempboxa \vskip\fboxsepvb}#1
          \vrule\@width\fboxrule }%
        \hrule \@height \fboxrule }}}}

\newcommand{\boxedhv}[3]{\setlength\fboxseph{#1cm}
  \setlength\fboxsepva{#2cm}\setlength\fboxsepvb{#2cm}\longboxed{#3}}

\newcommand{\boxedhvv}[4]{\setlength\fboxseph{#1cm}
  \setlength\fboxsepva{#2cm}\setlength\fboxsepvb{#3cm}\longboxed{#4}}

\newcommand{\Boxed}[1]{{\setlength\fboxseph{0.2cm}
  \setlength\fboxsepva{0.2cm}\setlength\fboxsepvb{0.2cm}\longboxed{#1}}}

\newcommand{\mBoxed}[1]{{\setlength\fboxseph{0.2cm}
  \setlength\fboxsepva{0.2cm}\setlength\fboxsepvb{0.2cm}\longboxed{\mbox{#1}}}}

\newcommand{\mboxed}[1]{{\setlength\fboxseph{0.2cm}
  \setlength\fboxsepva{0.2cm}\setlength\fboxsepvb{0.2cm}\boxed{\mbox{#1}}}}

\newsavebox{\fmbox}
\newenvironment{fmpage}[1]
     {\begin{lrbox}{\fmbox}\begin{minipage}{#1}}
     {\end{minipage}\end{lrbox}\fbox{\usebox{\fmbox}}}

%%%%%%%%%%
%%%%%%%%%%

\DeclareMathOperator{\ch}{ch}
\DeclareMathOperator{\sh}{sh}

%%%%%%%%%%
%%%%%%%%%%

\newcommand{\norme}[1]{\Vert #1 \Vert}

%\newcommand*\widefbox[1]{\fbox{\hspace{2em}#1\hspace{2em}}}

\newcommand{\nl}{\tabularnewline}

\newcommand{\hand}{\noindent\ding{43}\ }
\newcommand{\ie}{\textit{i.e. }}
\newcommand{\cf}{\textit{cf }}

\newcommand{\Card}{\operatorname{Card}}

\newcommand{\aire}{\mathcal{A}}

\newcommand{\LL}[1]{\mathscr{L}(#1)} %
\newcommand{\B}{\mathscr{B}} %
\newcommand{\Bc}[1]{B_{#1}} %
\newcommand{\M}[1]{\mathscr{M}_{#1}(\mathbb{R})}

\DeclareMathOperator{\im}{Im}
\DeclareMathOperator{\kr}{Ker}
\DeclareMathOperator{\rg}{rg}
\DeclareMathOperator{\spc}{Sp}
\DeclareMathOperator{\sgn}{sgn}
\DeclareMathOperator{\supp}{Supp}

\newcommand{\Mat}{{\rm{Mat}}}
\newcommand{\Vect}[1]{{\rm{Vect}}\left(#1\right)}

\newenvironment{smatrix}{%
  \begin{adjustbox}{width=.9\width}
    $
    \begin{pmatrix}
    }{%      
    \end{pmatrix}
    $
  \end{adjustbox}
}

\newenvironment{sarray}[1]{%
  \begin{adjustbox}{width=.9\width}
    $
    \begin{array}{#1}
    }{%      
    \end{array}
    $
  \end{adjustbox}
}

\newcommand{\vd}[2]{
  \scalebox{.8}{
    $\left(\!
      \begin{array}{c}
        #1 \\
        #2
      \end{array}
    \!\right)$
    }}

\newcommand{\vt}[3]{
  \scalebox{.8}{
    $\left(\!
      \begin{array}{c}
        #1 \\
        #2 \\
        #3 
      \end{array}
    \!\right)$
    }}

\newcommand{\vq}[4]{
  \scalebox{.8}{
    $\left(\!
      \begin{array}{c}
        #1 \\
        #2 \\
        #3 \\
        #4 
      \end{array}
    \!\right)$
    }}

\newcommand{\vc}[5]{
  \scalebox{.8}{
    $\left(\!
      \begin{array}{c}
        #1 \\
        #2 \\
        #3 \\
        #4 \\
        #5 
      \end{array}
    \!\right)$
    }}

\newcommand{\ee}{\text{e}}

\newcommand{\dd}{\text{d}}

%%% Ensemble de définition
\newcommand{\Df}{\mathscr{D}}
\newcommand{\Cf}{\mathscr{C}}
\newcommand{\Ef}{\mathscr{C}}

\newcommand{\rond}[1]{\,\overset{\scriptscriptstyle \circ}{\!#1}}

\newcommand{\df}[2]{\dfrac{\partial #1}{\partial #2}} %
\newcommand{\dfn}[2]{\partial_{#2}(#1)} %
\newcommand{\ddfn}[2]{\partial^2_{#2}(#1)} %
\newcommand{\ddf}[2]{\dfrac{\partial^2 #1}{\partial #2^2}} %
\newcommand{\ddfr}[3]{\dfrac{\partial^2 #1}{\partial #2 \partial
    #3}} %


\newcommand{\dlim}[1]{{\displaystyle \lim_{#1} \ }}
\newcommand{\dlimPlus}[2]{
  \dlim{
    \scalebox{.6}{
      $
      \begin{array}{l}
        #1 \rightarrow #2\\
        #1 > #2
      \end{array}
      $}}}
\newcommand{\dlimMoins}[2]{
  \dlim{
    \scalebox{.6}{
      $
      \begin{array}{l}
        #1 \rightarrow #2\\
        #1 < #2
      \end{array}
      $}}}

%%%%%%%%%%%%%%
%% petit o, développement limité
%%%%%%%%%%%%%%

\newcommand{\oo}[2]{{\underset {{\overset {#1\rightarrow #2}{}}}{o}}} %
\newcommand{\oox}[1]{{\underset {{\overset {x\rightarrow #1}{}}}{o}}} %
\newcommand{\oon}{{\underset {{\overset {n\rightarrow +\infty}{}}}{o}}} %
\newcommand{\po}[1]{{\underset {{\overset {#1}{}}}{o}}} %
\newcommand{\neqx}[1]{{\ \underset {{\overset {x \to #1}{}}}{\not\sim}\ }} %
\newcommand{\eqx}[1]{{\ \underset {{\overset {x \to #1}{}}}{\sim}\ }} %
\newcommand{\eqn}{{\ \underset {{\overset {n \to +\infty}{}}}{\sim}\ }} %
\newcommand{\eq}[2]{{\ \underset {{\overset {#1 \to #2}{}}}{\sim}\ }} %
\newcommand{\DL}[1]{{\rm{DL}}_1 (#1)} %
\newcommand{\DLL}[1]{{\rm{DL}}_2 (#1)} %

\newcommand{\negl}{<<}

\newcommand{\neglP}[1]{\begin{array}{c}
    \vspace{-.2cm}\\
    << \\
    \vspace{-.7cm}\\
    {\scriptstyle #1}
  \end{array}}

%%%%%%%%%%%%%%
%% borne sup, inf, max, min
%%%%%%%%%%%%%%
\newcommand{\dsup}[1]{\displaystyle \sup_{#1} \ }
\newcommand{\dinf}[1]{\displaystyle \inf_{#1} \ }
\newcommand{\dmax}[1]{\max\limits_{#1} \ }
\newcommand{\dmin}[1]{\min\limits_{#1} \ }

\newcommand{\dcup}[2]{{\textstyle\bigcup\limits_{#1}^{#2}}\hspace{.1cm}}
%\displaystyle \bigcup_{#1}^{#2}}
\newcommand{\dcap}[2]{{\textstyle\bigcap\limits_{#1}^{#2}}\hspace{.1cm}}
% \displaystyle \bigcap_{#1}^{#2}
%%%%%%%%%%%%%%
%% opérateurs logiques
%%%%%%%%%%%%%%
\newcommand{\NON}[1]{\mathop{\small \tt{NON}} (#1)}
\newcommand{\ET}{\mathrel{\mathop{\small \mathtt{ET}}}}
\newcommand{\OU}{\mathrel{\mathop{\small \tt{OU}}}}
\newcommand{\XOR}{\mathrel{\mathop{\small \tt{XOR}}}}

\newcommand{\id}{{\rm{id}}}

\newcommand{\sbullet}{\scriptstyle \bullet}
\newcommand{\stimes}{\scriptstyle \times}

%%%%%%%%%%%%%%%%%%
%% Probabilités
%%%%%%%%%%%%%%%%%%
\newcommand{\Prob}{\mathbb{P}}
\newcommand{\Ev}[1]{\left[ {#1} \right]}
\newcommand{\Evmb}[1]{[ {#1} ]}
\newcommand{\E}{\mathbb{E}}
\newcommand{\V}{\mathbb{V}}
\newcommand{\Cov}{{\rm{Cov}}}
\newcommand{\U}[2]{\mathcal{U}(\llb #1, #2\rrb)}
\newcommand{\Uc}[2]{\mathcal{U}([#1, #2])}
\newcommand{\Ucof}[2]{\mathcal{U}(]#1, #2])}
\newcommand{\Ucoo}[2]{\mathcal{U}(]#1, #2[)}
\newcommand{\Ucfo}[2]{\mathcal{U}([#1, #2[)}
\newcommand{\Bern}[1]{\mathcal{B}\left(#1\right)}
\newcommand{\Bin}[2]{\mathcal{B}\left(#1, #2\right)}
\newcommand{\G}[1]{\mathcal{G}\left(#1\right)}
\newcommand{\Pois}[1]{\mathcal{P}\left(#1\right)}
\newcommand{\HG}[3]{\mathcal{H}\left(#1, #2, #3\right)}
\newcommand{\Exp}[1]{\mathcal{E}\left(#1\right)}
\newcommand{\Norm}[2]{\mathcal{N}\left(#1, #2\right)}

\DeclareMathOperator{\cov}{Cov}

\newcommand{\var}{v.a.r. }
\newcommand{\suit}{\hookrightarrow}

\newcommand{\flecheR}[1]{\rotatebox{90}{\scalebox{#1}{\color{red}
      $\curvearrowleft$}}}


\newcommand{\partie}[1]{\mathcal{P}(#1)}
\newcommand{\Cont}[1]{\mathcal{C}^{#1}}
\newcommand{\Contm}[1]{\mathcal{C}^{#1}_m}

\newcommand{\llb}{\llbracket}
\newcommand{\rrb}{\rrbracket}

%\newcommand{\im}[1]{{\rm{Im}}(#1)}
\newcommand{\imrec}[1]{#1^{- \mathds{1}}}

\newcommand{\unq}{\mathds{1}}

\newcommand{\Hyp}{\mathtt{H}}

\newcommand{\eme}[1]{#1^{\scriptsize \mbox{ème}}}
\newcommand{\er}[1]{#1^{\scriptsize \mbox{er}}}
\newcommand{\ere}[1]{#1^{\scriptsize \mbox{ère}}}
\newcommand{\nd}[1]{#1^{\scriptsize \mbox{nd}}}
\newcommand{\nde}[1]{#1^{\scriptsize \mbox{nde}}}

\newcommand{\truc}{\mathop{\top}}
\newcommand{\fois}{\mathop{\ast}}

\newcommand{\f}[1]{\overrightarrow{#1}}

\newcommand{\checked}{\textcolor{green}{\checkmark}}

\def\restriction#1#2{\mathchoice
              {\setbox1\hbox{${\displaystyle #1}_{\scriptstyle #2}$}
              \restrictionaux{#1}{#2}}
              {\setbox1\hbox{${\textstyle #1}_{\scriptstyle #2}$}
              \restrictionaux{#1}{#2}}
              {\setbox1\hbox{${\scriptstyle #1}_{\scriptscriptstyle #2}$}
              \restrictionaux{#1}{#2}}
              {\setbox1\hbox{${\scriptscriptstyle #1}_{\scriptscriptstyle #2}$}
              \restrictionaux{#1}{#2}}}
\def\restrictionaux#1#2{{#1\,\smash{\vrule height .8\ht1 depth .85\dp1}}_{\,#2}}

\makeatletter
\newcommand*{\da@rightarrow}{\mathchar"0\hexnumber@\symAMSa 4B }
\newcommand*{\da@leftarrow}{\mathchar"0\hexnumber@\symAMSa 4C }
\newcommand*{\xdashrightarrow}[2][]{%
  \mathrel{%
    \mathpalette{\da@xarrow{#1}{#2}{}\da@rightarrow{\,}{}}{}%
  }%
}
\newcommand{\xdashleftarrow}[2][]{%
  \mathrel{%
    \mathpalette{\da@xarrow{#1}{#2}\da@leftarrow{}{}{\,}}{}%
  }%
}
\newcommand*{\da@xarrow}[7]{%
  % #1: below
  % #2: above
  % #3: arrow left
  % #4: arrow right
  % #5: space left 
  % #6: space right
  % #7: math style 
  \sbox0{$\ifx#7\scriptstyle\scriptscriptstyle\else\scriptstyle\fi#5#1#6\m@th$}%
  \sbox2{$\ifx#7\scriptstyle\scriptscriptstyle\else\scriptstyle\fi#5#2#6\m@th$}%
  \sbox4{$#7\dabar@\m@th$}%
  \dimen@=\wd0 %
  \ifdim\wd2 >\dimen@
    \dimen@=\wd2 %   
  \fi
  \count@=2 %
  \def\da@bars{\dabar@\dabar@}%
  \@whiledim\count@\wd4<\dimen@\do{%
    \advance\count@\@ne
    \expandafter\def\expandafter\da@bars\expandafter{%
      \da@bars
      \dabar@ 
    }%
  }%  
  \mathrel{#3}%
  \mathrel{%   
    \mathop{\da@bars}\limits
    \ifx\\#1\\%
    \else
      _{\copy0}%
    \fi
    \ifx\\#2\\%
    \else
      ^{\copy2}%
    \fi
  }%   
  \mathrel{#4}%
}
\makeatother



\newcount\depth

\newcount\depth
\newcount\totaldepth

\makeatletter
\newcommand{\labelsymbol}{%
      \ifnum\depth=0
        %
      \else
        \rlap{\,$\bullet$}%
      \fi
}

\newcommand*\bernoulliTree[1]{%
    \depth=#1\relax            
    \totaldepth=#1\relax
    \draw node(root)[bernoulli/root] {\labelsymbol}[grow=right] \draw@bernoulli@tree;
    \draw \label@bernoulli@tree{root};                                   
}                                                                        

\def\draw@bernoulli@tree{%
    \ifnum\depth>0 
      child[parent anchor=east] foreach \type/\label in {left child/$E$,right child/$S$} {%
          node[bernoulli/\type] {\label\strut\labelsymbol} \draw@bernoulli@tree
      }
      coordinate[bernoulli/increment] (dummy)
   \fi%
}

\def\label@bernoulli@tree#1{%
    \ifnum\depth>0
      ($(#1)!0.5!(#1-1)$) node[fill=white,bernoulli/decrement] {\tiny$p$}
      \label@bernoulli@tree{#1-1}
      ($(#1)!0.5!(#1-2)$) node[fill=white] {\tiny$q$}
      \label@bernoulli@tree{#1-2}
      coordinate[bernoulli/increment] (dummy)
   \fi%
}

\makeatother

\tikzset{bernoulli/.cd,
         root/.style={},
         decrement/.code=\global\advance\depth by-1\relax,
         increment/.code=\global\advance\depth by 1\relax,
         left child/.style={bernoulli/decrement},
         right child/.style={}}


\newcommand{\eps}{\varepsilon}

% \newcommand{\tendi}[1]{\xrightarrow[\footnotesize #1 \rightarrow
%   +\infty]{}}%

\newcommand{\tend}{\rightarrow}%
\newcommand{\tendn}{\underset{n\to +\infty}{\longrightarrow}} %
\newcommand{\ntendn}{\underset{n\to
    +\infty}{\not\hspace{-.15cm}\longrightarrow}} %
% \newcommand{\tendn}{\xrightarrow[\footnotesize n \rightarrow
%   +\infty]{}}%
\newcommand{\Tendx}[1]{\xrightarrow[\footnotesize x \rightarrow
  #1]{}}%
\newcommand{\tendx}[1]{\underset{x\to #1}{\longrightarrow}}%
\newcommand{\ntendx}[1]{\underset{x\to #1}{\not\!\!\longrightarrow}}%
\newcommand{\tendd}[2]{\underset{#1\to #2}{\longrightarrow}}%
% \newcommand{\tendd}[2]{\xrightarrow[\footnotesize #1 \rightarrow
%   #2]{}}%
\newcommand{\tendash}[1]{\xdashrightarrow[\footnotesize #1 \rightarrow
  +\infty]{}}%
\newcommand{\tendashx}[1]{\xdashrightarrow[\footnotesize x \rightarrow
  #1]{}}%
\newcommand{\tendb}[1]{\underset{#1 \to +\infty}{\longrightarrow}}%
\newcommand{\tendL}{\overset{\mathscr L}{\underset{n \to
      +\infty}{\longrightarrow}}}%
\newcommand{\tendP}{\overset{\Prob}{\underset{n \to
      +\infty}{\longrightarrow}}}%
\newcommand{\tenddL}[1]{\overset{\mathscr L}{\underset{#1 \to
      +\infty}{\longrightarrow}}}%

\NewEnviron{attention}{ %
  ~\\[-.2cm]\noindent
  \begin{minipage}{\linewidth}
  \setlength{\fboxsep}{3mm}%
  \ \ \dbend \ \ %
  \fbox{\parbox[t]{.88\linewidth}{\BODY}} %
  \end{minipage}\\
}

\NewEnviron{sattention}[1]{ %
  ~\\[-.2cm]\noindent
  \begin{minipage}{#1\linewidth}
  \setlength{\fboxsep}{3mm}%
  \ \ \dbend \ \ %
  \fbox{\parbox[t]{.88\linewidth}{\BODY}} %
  \end{minipage}\\
}

%%%%% OBSOLETE %%%%%%

% \newcommand{\attention}[1]{
%   \noindent
%   \begin{tabular}{@{}l|p{11.5cm}|}
%     \cline{2-2}
%     \vspace{-.2cm} 
%     & \nl
%     \dbend & #1 \nl
%     \cline{2-2}
%   \end{tabular}
% }

% \newcommand{\attentionv}[2]{
%   \noindent
%   \begin{tabular}{@{}l|p{11.5cm}|}
%     \cline{2-2}
%     \vspace{-.2cm} 
%     & \nl
%     \dbend & #2 \nl[#1 cm]
%     \cline{2-2}
%   \end{tabular}
% }

\newcommand{\explainvb}[2]{
  \noindent
  \begin{tabular}{@{}l|p{11.5cm}|}
    \cline{2-2}
    \vspace{-.2cm} 
    & \nl
    \hand & #2 \nl [#1 cm]
    \cline{2-2}
  \end{tabular}
}


% \noindent
% \begin{tabular}{@{}l|lp{11cm}|}
%   \cline{3-3} 
%   \multicolumn{1}{@{}l@{\dbend}}{} & & #1 \nl
%   \multicolumn{1}{l}{} & & \nl [-.8cm]
%   & & #2 \nl
%   \cline{2-3}
% \end{tabular}

% \newcommand{\attention}[1]{
%   \noindent
%   \begin{tabular}{@{}@{}cp{11cm}}
%     \dbend & #1 \nl
%   \end{tabular}
% }

\newcommand{\PP}[1]{\mathcal{P}(#1)}
\newcommand{\HH}[1]{\mathcal{H}(#1)}
\newcommand{\FF}[1]{\mathcal{F}(#1)}

\newcommand{\DSum}[2]{\displaystyle\sum\limits_{#1}^{#2}\hspace{.1cm}}
\newcommand{\Sum}[2]{{\textstyle\sum\limits_{#1}^{#2}}\hspace{.1cm}}
\newcommand{\Serie}{\textstyle\sum\hspace{.1cm}}
\newcommand{\Prod}[2]{\textstyle\prod\limits_{#1}^{#2}}

\newcommand{\Prim}[3]{\left[\ {#1} \ \right]_{\scriptscriptstyle
   \hspace{-.15cm} ~_{#2}\, }^ {\scriptscriptstyle \hspace{-.15cm} ~^{#3}\, }}

% \newcommand{\Prim}[3]{\left[\ {#1} \ \right]_{\scriptscriptstyle
%     \!\!~_{#2}}^ {\scriptscriptstyle \!\!~^{#3}}}

\newcommand{\dint}[2]{\displaystyle \int_{#1}^{#2}\ }
\newcommand{\Int}[2]{{\rm{Int}}_{\scriptscriptstyle #1, #2}}
\newcommand{\dt}{\ dt}
\newcommand{\dx}{\ dx}

\newcommand{\llpar}[1]{\left(\!\!\!
    \begin{array}{c}
      \rule{0pt}{#1}
    \end{array}
  \!\!\!\right.}

\newcommand{\rrpar}[1]{\left.\!\!\!
    \begin{array}{c}
      \rule{0pt}{#1}
    \end{array}
  \!\!\!\right)}

\newcommand{\llacc}[1]{\left\{\!\!\!
    \begin{array}{c}
      \rule{0pt}{#1}
    \end{array}
  \!\!\!\right.}

\newcommand{\rracc}[1]{\left.\!\!\!
    \begin{array}{c}
      \rule{0pt}{#1}
    \end{array}
  \!\!\!\right\}}

\newcommand{\ttacc}[1]{\mbox{\rotatebox{-90}{\hspace{-.7cm}$\llacc{#1}$}}}
\newcommand{\bbacc}[1]{\mbox{\rotatebox{90}{\hspace{-.5cm}$\llacc{#1}$}}}

\newcommand{\comp}[1]{\overline{#1}}%

\newcommand{\dcomp}[2]{\stackrel{\mbox{\ \ \----}{\scriptscriptstyle
      #2}}{#1}}%

% \newcommand{\Comp}[2]{\stackrel{\mbox{\ \
%       \-------}{\scriptscriptstyle #2}}{#1}}

% \newcommand{\dcomp}[2]{\stackrel{\mbox{\ \
%       \-------}{\scriptscriptstyle #2}}{#1}}

\newcommand{\A}{\mathscr{A}}

\newcommand{\conc}[1]{
  \begin{center}
    \fbox{
      \begin{tabular}{c}
        #1
      \end{tabular}
    }
  \end{center}
}

\newcommand{\concC}[1]{
  \begin{center}
    \fbox{
    \begin{tabular}{C{10cm}}
      \quad #1 \quad
    \end{tabular}
    }
  \end{center}
}

\newcommand{\concL}[2]{
  \begin{center}
    \fbox{
    \begin{tabular}{C{#2cm}}
      \quad #1 \quad
    \end{tabular}
    }
  \end{center}
}


% \newcommand{\lims}[2]{\prod\limits_{#1}^{#2}}

\newtheorem{theorem}{Théorème}[]
\newtheorem{lemma}{Lemme}[]
\newtheorem{proposition}{Proposition}[]
\newtheorem{corollary}{Corollaire}[]

% \newenvironment{proof}[1][Démonstration]{\begin{trivlist}
% \item[\hskip \labelsep {\bfseries #1}]}{\end{trivlist}}
\newenvironment{definition}[1][Définition]{\begin{trivlist}
\item[\hskip \labelsep {\bfseries #1}]}{\end{trivlist}}
\newenvironment{example}[1][Exemple]{\begin{trivlist}
\item[\hskip \labelsep {\bfseries #1}]}{\end{trivlist}}
\newenvironment{examples}[1][Exemples]{\begin{trivlist}
\item[\hskip \labelsep {\bfseries #1}]}{\end{trivlist}}
\newenvironment{notation}[1][Notation]{\begin{trivlist}
\item[\hskip \labelsep {\bfseries #1}]}{\end{trivlist}}
\newenvironment{propriete}[1][Propriété]{\begin{trivlist}
\item[\hskip \labelsep {\bfseries #1}]}{\end{trivlist}}
\newenvironment{proprietes}[1][Propriétés]{\begin{trivlist}
\item[\hskip \labelsep {\bfseries #1}]}{\end{trivlist}}
% \newenvironment{remark}[1][Remarque]{\begin{trivlist}
% \item[\hskip \labelsep {\bfseries #1}]}{\end{trivlist}}
\newenvironment{application}[1][Application]{\begin{trivlist}
\item[\hskip \labelsep {\bfseries #1}]}{\end{trivlist}}

% Environnement pour les réponses des DS
\newenvironment{answer}{\par\emph{Réponse :}\par{}}
{\vspace{-.6cm}\hspace{\stretch{1}}\rule{1ex}{1ex}\vspace{.3cm}}

\newenvironment{answerTD}{\vspace{.2cm}\par\emph{Réponse :}\par{}}
{\hspace{\stretch{1}}\rule{1ex}{1ex}\vspace{.3cm}}

\newenvironment{answerCours}{\noindent\emph{Réponse :}}
{\rule{1ex}{1ex}}%\vspace{.3cm}}


% footnote in footer
\newcommand{\fancyfootnotetext}[2]{%
  \fancypagestyle{dingens}{%
    \fancyfoot[LO,RE]{\parbox{0.95\textwidth}{\footnotemark[#1]\footnotesize
        #2}}%
  }%
  \thispagestyle{dingens}%
}

%%% définit le style (arabic : 1,2,3...) et place des parenthèses
%%% autour de la numérotation
\renewcommand*{\thefootnote}{(\arabic{footnote})}
% http://www.tuteurs.ens.fr/logiciels/latex/footnote.html

%%%%%%%% tikz axis
% \pgfplotsset{every axis/.append style={
%                     axis x line=middle,    % put the x axis in the middle
%                     axis y line=middle,    % put the y axis in the middle
%                     axis line style={<->,color=blue}, % arrows on the axis
%                     xlabel={$x$},          % default put x on x-axis
%                     ylabel={$y$},          % default put y on y-axis
%             }}

%%%% s'utilise comme suit

% \begin{axis}[
%   xmin=-8,xmax=4,
%   ymin=-8,ymax=4,
%   grid=both,
%   ]
%   \addplot [domain=-3:3,samples=50]({x^3-3*x},{3*x^2-9}); 
% \end{axis}

%%%%%%%%



%%%%%%%%%%%% Pour avoir des numéros de section qui correspondent à
%%%%%%%%%%%% ceux du tableau
\renewcommand{\thesection}{\Roman{section}.\hspace{-.3cm}}
\renewcommand{\thesubsection}{\Roman{section}.\arabic{subsection}.\hspace{-.2cm}}
\renewcommand{\thesubsubsection}{\Roman{section}.\arabic{subsection}.\alph{subsubsection})\hspace{-.2cm}}
%%%%%%%%%%%% 

%%% Changer le nom des figures : Fig. au lieu de Figure
\usepackage[font=small,labelfont=bf,labelsep=space]{caption}
\captionsetup{%
  figurename=Fig.,
  tablename=Tab.
}
% \renewcommand{\thesection}{\Roman{section}.\hspace{-.2cm}}
% \renewcommand{\thesubsection}{\Roman{section}
%   .\hspace{.2cm}\arabic{subsection}\ .\hspace{-.3cm}}
% \renewcommand{\thesubsubsection}{\alph{subsection})}

\newenvironment{tabliste}[1]
{\begin{tabenum}[\bfseries\small\itshape #1]}{\end{tabenum}} 

%%%% ESSAI contre le too deeply nested %%%%
%%%% ATTENTION au package enumitem qui se comporte mal avec les
%%%% noliste, à redéfinir !
% \usepackage{enumitem}

% \setlistdepth{9}

% \newlist{myEnumerate}{enumerate}{9}
% \setlist[myEnumerate,1]{label=(\arabic*)}
% \setlist[myEnumerate,2]{label=(\Roman*)}
% \setlist[myEnumerate,3]{label=(\Alph*)}
% \setlist[myEnumerate,4]{label=(\roman*)}
% \setlist[myEnumerate,5]{label=(\alph*)}
% \setlist[myEnumerate,6]{label=(\arabic*)}
% \setlist[myEnumerate,7]{label=(\Roman*)}
% \setlist[myEnumerate,8]{label=(\Alph*)}
% \setlist[myEnumerate,9]{label=(\roman*)}

%%%%%

\newenvironment{noliste}[1] %
{\begin{enumerate}[\bfseries\small\itshape #1]} %
  {\end{enumerate}}

\newenvironment{nonoliste}[1] %
{\begin{enumerate}[\hspace{-12pt}\bfseries\small\itshape #1]} %
  {\end{enumerate}}

\newenvironment{arrayliste}[1]{ 
  % List with minimal white space to fit in small areas, e.g. table
  % cell
  \begin{minipage}[t]{\linewidth} %
    \begin{enumerate}[\bfseries\small\itshape #1] %
      {\leftmargin=0.5em \rightmargin=0em
        \topsep=0em \parskip=0em \parsep=0em
        \listparindent=0em \partopsep=0em \itemsep=0pt
        \itemindent=0em \labelwidth=\leftmargin\labelsep+0.25em}
    }{
    \end{enumerate}\end{minipage}
}

\newenvironment{nolistes}[2]
{\begin{enumerate}[\bfseries\small\itshape
    #1]\setlength{\itemsep}{#2 mm}}{\end{enumerate}}

\newenvironment{liste}[1]
{\begin{enumerate}[\hspace{12pt}\bfseries\small\itshape
    #1]}{\end{enumerate}}   


%%%%%%%% Pour les programmes de colle %%%%%%%

\newcommand{\cours}{{\small \tt (COURS)}} %
\newcommand{\poly}{{\small \tt (POLY)}} %
\newcommand{\exo}{{\small \tt (EXO)}} %
\newcommand{\culture}{{\small \tt (CULTURE)}} %
\newcommand{\methodo}{{\small \tt (MÉTHODO)}} %
\newcommand{\methodob}{\Boxed{\mbox{\tt MÉTHODO}}} %

%%%%%%%% Pour les TD %%%%%%%
\newtheoremstyle{exostyle} {\topsep} % espace avant
{.6cm} % espace apres
{} % Police utilisee par le style de thm
{} % Indentation (vide = aucune, \parindent = indentation paragraphe)
{\bfseries} % Police du titre de thm
{} % Signe de ponctuation apres le titre du thm
{ } % Espace apres le titre du thm (\newline = linebreak)
{\thmname{#1}\thmnumber{ #2}\thmnote{.
    \normalfont{\textit{#3}}}} % composants du titre du thm : \thmname
                               % = nom du thm, \thmnumber = numéro du
                               % thm, \thmnote = sous-titre du thm
 
\theoremstyle{exostyle}
\newtheorem{exercice}{Exercice}
\newtheorem*{exoCours}{Exercice}

%%%%%%%% Pour des théorèmes Sans Espaces APRÈS %%%%%%%
\newtheoremstyle{exostyleSE} {\topsep} % espace avant
{} % espace apres
{} % Police utilisee par le style de thm
{} % Indentation (vide = aucune, \parindent = indentation paragraphe)
{\bfseries} % Police du titre de thm
{} % Signe de ponctuation apres le titre du thm
{ } % Espace apres le titre du thm (\newline = linebreak)
{\thmname{#1}\thmnumber{ #2}\thmnote{.
    \normalfont{\textit{#3}}}} % composants du titre du thm : \thmname
                               % = nom du thm, \thmnumber = numéro du
                               % thm, \thmnote = sous-titre du thm
 
\theoremstyle{exostyleSE}
\newtheorem{exerciceSE}{Exercice}
\newtheorem*{exoCoursSE}{Exercice}

% \newcommand{\lims}[2]{\prod\limits_{#1}^{#2}}

\newtheorem{theoremSE}{Théorème}[]
\newtheorem{lemmaSE}{Lemme}[]
\newtheorem{propositionSE}{Proposition}[]
\newtheorem{corollarySE}{Corollaire}[]

% \newenvironment{proofSE}[1][Démonstration]{\begin{trivlist}
% \item[\hskip \labelsep {\bfseries #1}]}{\end{trivlist}}
\newenvironment{definitionSE}[1][Définition]{\begin{trivlist}
  \item[\hskip \labelsep {\bfseries #1}]}{\end{trivlist}}
\newenvironment{exampleSE}[1][Exemple]{\begin{trivlist} 
  \item[\hskip \labelsep {\bfseries #1}]}{\end{trivlist}}
\newenvironment{examplesSE}[1][Exemples]{\begin{trivlist}
\item[\hskip \labelsep {\bfseries #1}]}{\end{trivlist}}
\newenvironment{notationSE}[1][Notation]{\begin{trivlist}
\item[\hskip \labelsep {\bfseries #1}]}{\end{trivlist}}
\newenvironment{proprieteSE}[1][Propriété]{\begin{trivlist}
\item[\hskip \labelsep {\bfseries #1}]}{\end{trivlist}}
\newenvironment{proprietesSE}[1][Propriétés]{\begin{trivlist}
\item[\hskip \labelsep {\bfseries #1}]}{\end{trivlist}}
\newenvironment{remarkSE}[1][Remarque]{\begin{trivlist}
\item[\hskip \labelsep {\bfseries #1}]}{\end{trivlist}}
\newenvironment{applicationSE}[1][Application]{\begin{trivlist}
\item[\hskip \labelsep {\bfseries #1}]}{\end{trivlist}}

%%%%%%%%%%% Obtenir les étoiles sans charger le package MnSymbol
%%%%%%%%%%%
\DeclareFontFamily{U} {MnSymbolC}{}
\DeclareFontShape{U}{MnSymbolC}{m}{n}{
  <-6> MnSymbolC5
  <6-7> MnSymbolC6
  <7-8> MnSymbolC7
  <8-9> MnSymbolC8
  <9-10> MnSymbolC9
  <10-12> MnSymbolC10
  <12-> MnSymbolC12}{}
\DeclareFontShape{U}{MnSymbolC}{b}{n}{
  <-6> MnSymbolC-Bold5
  <6-7> MnSymbolC-Bold6
  <7-8> MnSymbolC-Bold7
  <8-9> MnSymbolC-Bold8
  <9-10> MnSymbolC-Bold9
  <10-12> MnSymbolC-Bold10
  <12-> MnSymbolC-Bold12}{}

\DeclareSymbolFont{MnSyC} {U} {MnSymbolC}{m}{n}

\DeclareMathSymbol{\filledlargestar}{\mathrel}{MnSyC}{205}
\DeclareMathSymbol{\largestar}{\mathrel}{MnSyC}{131}

\newcommand{\facile}{\rm{(}$\scriptstyle\largestar$\rm{)}} %
\newcommand{\moyen}{\rm{(}$\scriptstyle\filledlargestar$\rm{)}} %
\newcommand{\dur}{\rm{(}$\scriptstyle\filledlargestar\filledlargestar$\rm{)}} %
\newcommand{\costaud}{\rm{(}$\scriptstyle\filledlargestar\filledlargestar\filledlargestar$\rm{)}}

%%%%%%%%%%%%%%%%%%%%%%%%%

%%%%%%%%%%%%%%%%%%%%%%%%%
%%%%%%%% Fin de la partie TD

%%%%%%%%%%%%%%%%
%%%%%%%%%%%%%%%%
\makeatletter %
\newenvironment{myitemize}{%
  \setlength{\topsep}{0pt} %
  \setlength{\partopsep}{0pt} %
  \renewcommand*{\@listi}{\leftmargin\leftmargini \parsep\z@
    \topsep\z@ \itemsep\z@} \let\@listI\@listi %
  \itemize %
}{\enditemize} %
\makeatother
%%%%%%%%%%%%%%%%
%%%%%%%%%%%%%%%%

%% Commentaires dans la correction du livre

\newcommand{\Com}[1]{
% Define box and box title style
\tikzstyle{mybox} = [draw=black!50,
very thick,
    rectangle, rounded corners, inner sep=10pt, inner ysep=8pt]
\tikzstyle{fancytitle} =[rounded corners, fill=black!80, text=white]
\tikzstyle{fancylogo} =[ text=white]
\begin{center}

\begin{tikzpicture}
\node [mybox] (box){%

    \begin{minipage}{0.90\linewidth}
\vspace{6pt}  #1
    \end{minipage}
};
\node[fancytitle, right=10pt] at (box.north west) 
{\bfseries\normalsize{Commentaire}};

\end{tikzpicture}%

\end{center}
%
}

\NewEnviron{remark}{%
  % Define box and box title style
  \tikzstyle{mybox} = [draw=black!50, very thick, rectangle, rounded
  corners, inner sep=10pt, inner ysep=8pt] %
  \tikzstyle{fancytitle} = [rounded corners , fill=black!80,
  text=white] %
  \tikzstyle{fancylogo} =[ text=white]
  \begin{center}
    \begin{tikzpicture}
      \node [mybox] (box){%
        \begin{minipage}{0.90\linewidth}
          \vspace{6pt} \BODY
        \end{minipage}
      }; %
      \node[fancytitle, right=10pt] at (box.north west) %
      {\bfseries\normalsize{Commentaire}}; %
    \end{tikzpicture}%
  \end{center}
}

\NewEnviron{remarkST}{%
  % Define box and box title style
  \tikzstyle{mybox} = [draw=black!50, very thick, rectangle, rounded
  corners, inner sep=10pt, inner ysep=8pt] %
  \tikzstyle{fancytitle} = [rounded corners , fill=black!80,
  text=white] %
  \tikzstyle{fancylogo} =[ text=white]
  \begin{center}
    \begin{tikzpicture}
      \node [mybox] (box){%
        \begin{minipage}{0.90\linewidth}
          \vspace{6pt} \BODY
        \end{minipage}
      }; %
      % \node[fancytitle, right=10pt] at (box.north west) %
%       {\bfseries\normalsize{Commentaire}}; %
    \end{tikzpicture}%
  \end{center}
}

\NewEnviron{remarkL}[1]{%
  % Define box and box title style
  \tikzstyle{mybox} = [draw=black!50, very thick, rectangle, rounded
  corners, inner sep=10pt, inner ysep=8pt] %
  \tikzstyle{fancytitle} =[rounded corners, fill=black!80,
  text=white] %
  \tikzstyle{fancylogo} =[ text=white]
  \begin{center}
    \begin{tikzpicture}
      \node [mybox] (box){%
        \begin{minipage}{#1\linewidth}
          \vspace{6pt} \BODY
        \end{minipage}
      }; %
      \node[fancytitle, right=10pt] at (box.north west) %
      {\bfseries\normalsize{Commentaire}}; %
    \end{tikzpicture}%
  \end{center}
}

\NewEnviron{remarkSTL}[1]{%
  % Define box and box title style
  \tikzstyle{mybox} = [draw=black!50, very thick, rectangle, rounded
  corners, inner sep=10pt, inner ysep=8pt] %
  \tikzstyle{fancytitle} =[rounded corners, fill=black!80,
  text=white] %
  \tikzstyle{fancylogo} =[ text=white]
  \begin{center}
    \begin{tikzpicture}
      \node [mybox] (box){%
        \begin{minipage}{#1\linewidth}
          \vspace{6pt} \BODY
        \end{minipage}
      }; %
%       \node[fancytitle, right=10pt] at (box.north west) %
%       {\bfseries\normalsize{Commentaire}}; %
    \end{tikzpicture}%
  \end{center}
}

\NewEnviron{titre} %
{ %
  ~\\[-1.8cm]
  \begin{center}
    \bf \LARGE \BODY
  \end{center}
  ~\\[-.6cm]
  \hrule %
  \vspace*{.2cm}
} %

\NewEnviron{titreL}[2] %
{ %
  ~\\[-#1cm]
  \begin{center}
    \bf \LARGE \BODY
  \end{center}
  ~\\[-#2cm]
  \hrule %
  \vspace*{.2cm}
} %



%%%%%%%%%%% Redefinition \chapter



\usepackage[explicit]{titlesec}
\usepackage{color}
\titleformat{\chapter}
{\gdef\chapterlabel{}
\selectfont\huge\bf}
%\normalfont\sffamily\Huge\bfseries\scshape}
{\gdef\chapterlabel{\thechapter)\ }}{0pt}
{\begin{tikzpicture}[remember picture,overlay]
\node[yshift=-3cm] at (current page.north west)
{\begin{tikzpicture}[remember picture, overlay]
\draw (.1\paperwidth,0) -- (.9\paperwidth,0);
\draw (.1\paperwidth,2) -- (.9\paperwidth,2);
%(\paperwidth,3cm);
\node[anchor=center,xshift=.5\paperwidth,yshift=1cm, rectangle,
rounded corners=20pt,inner sep=11pt]
{\color{black}\chapterlabel#1};
\end{tikzpicture}
};
\end{tikzpicture}
}
\titlespacing*{\chapter}{0pt}{50pt}{-75pt}




%%%%%%%%%%%%%% Affichage chapter dans Table des matieres

\makeatletter
\renewcommand*\l@chapter[2]{%
  \ifnum \c@tocdepth >\m@ne
    \addpenalty{-\@highpenalty}%
    \vskip 1.0em \@plus\p@
    \setlength\@tempdima{1.5em}%
    \begingroup
      \parindent \z@ \rightskip \@pnumwidth
      \parfillskip -\@pnumwidth
      \leavevmode %\bfseries
      \advance\leftskip\@tempdima
      \hskip -\leftskip
      #1\nobreak\ 
       \leaders\hbox{$\m@th
        \mkern \@dotsep mu\hbox{.}\mkern \@dotsep
        mu$}\hfil\nobreak\hb@xt@\@pnumwidth{\hss #2}\par
      \penalty\@highpenalty
    \endgroup
  \fi}
\makeatother




%%%%%%%%%%%%%%%%% Redefinition part



\usepackage{array}
\usepackage{paralist}

\usepackage[svgnames, usenames, dvipsnames]{xcolor}
\xdefinecolor{RecColor}{named}{Aqua}
\xdefinecolor{IncColor}{named}{Aqua}
\xdefinecolor{ImpColor}{named}{PaleGreen}

% \usepackage{frcursive}

\usepackage{adjustbox}

%%%%%%%%%%%
\newcommand{\cRB}[1]{{\color{Red} \pmb{#1}}} %
\newcommand{\cR}[1]{{\color{Red} {#1}}} %
\newcommand{\cBB}[1]{{\color{Blue} \pmb{#1}}}
\newcommand{\cB}[1]{{\color{Blue} {#1}}}
\newcommand{\cGB}[1]{{\color{LimeGreen} \pmb{#1}}}
\newcommand{\cG}[1]{{\color{LimeGreen} {#1}}}

%%%%%%%%%%

\usepackage{diagbox} %
\usepackage{colortbl} %
\usepackage{multirow} %
\usepackage{pgf} %
\usepackage{environ} %
\usepackage{fancybox} %
\usepackage{textcomp} %
\usepackage{marvosym} %

%%%%%%%%%% pour pouvoir faire des dashedline dans les tableaux
\usepackage{arydshln}

%%%%%%%%%% pour qu'une cellcolor ne recouvre pas le trait du tableau
\usepackage{hhline}%

\usepackage{pgfplots}
\pgfplotsset{compat=1.10}
\usepgfplotslibrary{patchplots}
\usepgfplotslibrary{fillbetween}
\usepackage{tikz,tkz-tab}
\usepackage{ifthen}
\usepackage{calc}
\usetikzlibrary{calc,decorations.pathreplacing,arrows,positioning} 
\usetikzlibrary{fit,shapes,backgrounds}
% \usepackage[nomessages]{fp}% http://ctan.org/pkg/fp

\usetikzlibrary{matrix,arrows,decorations.pathmorphing,
  decorations.pathreplacing} 

\newcommand{\myunit}{1 cm}
\tikzset{
    node style sp/.style={draw,circle,minimum size=\myunit},
    node style ge/.style={circle,minimum size=\myunit},
    arrow style mul/.style={draw,sloped,midway,fill=white},
    arrow style plus/.style={midway,sloped,fill=white},
}

%%%%%%%%%%%%%%
%%%%% écrire des inférieur égal ou supérieur égal avec typographie
%%%%% francaise
%%%%%%%%%%%%%

\renewcommand{\geq}{\geqslant}
\renewcommand{\leq}{\leqslant}
\renewcommand{\emptyset}{\varnothing}

\newcommand{\Leq}{\leqslant}
\newcommand{\Geq}{\geqslant}

%%%%%%%%%%%%%%
%%%%% Macro Celia
%%%%%%%%%%%%%

\newcommand{\ff}[2]{\left[#1, #2\right]} %
\newcommand{\fo}[2]{\left[#1, #2\right[} %
\newcommand{\of}[2]{\left]#1, #2\right]} %
\newcommand{\soo}[2]{\left]#1, #2\right[} %
\newcommand{\abs}[1]{\left|#1\right|} %
\newcommand{\Ent}[1]{\left\lfloor #1 \right\rfloor} %


%%%%%%%%%%%%%%
%%%%% tikz : comment dessiner un "oeil"
%%%%%%%%%%%%%

\newcommand{\eye}[4]% size, x, y, rotation
{ \draw[rotate around={#4:(#2,#3)}] (#2,#3) -- ++(-.5*55:#1) (#2,#3)
  -- ++(.5*55:#1); \draw (#2,#3) ++(#4+55:.75*#1) arc
  (#4+55:#4-55:.75*#1);
  % IRIS
  \draw[fill=gray] (#2,#3) ++(#4+55/3:.75*#1) arc
  (#4+180-55:#4+180+55:.28*#1);
  % PUPIL, a filled arc
  \draw[fill=black] (#2,#3) ++(#4+55/3:.75*#1) arc
  (#4+55/3:#4-55/3:.75*#1);%
}


%%%%%%%%%%
%% discontinuité fonction
\newcommand\pointg[2]{%
  \draw[color = red, very thick] (#1+0.15, #2-.04)--(#1, #2-.04)--(#1,
  #2+.04)--(#1+0.15, #2+.04);%
}%

\newcommand\pointd[2]{%
  \draw[color = red, very thick] (#1-0.15, #2+.04)--(#1, #2+.04)--(#1,
  #2-.04)--(#1-0.15, #2-.04);%
}%

%%%%%%%%%%
%%% 1 : position abscisse, 2 : position ordonnée, 3 : taille, 4 : couleur
%%%%%%%%%%
% \newcommand\pointG[4]{%
%   \draw[color = #4, very thick] (#1+#3, #2-(#3/3.75))--(#1,
%   #2-(#3/3.75))--(#1, #2+(#3/3.75))--(#1+#3, #2+(#3/3.75)) %
% }%

\newcommand\pointG[4]{%
  \draw[color = #4, very thick] ({#1+#3/3.75}, {#2-#3})--(#1,
  {#2-#3})--(#1, {#2+#3})--({#1+#3/3.75}, {#2+#3}) %
}%

\newcommand\pointD[4]{%
  \draw[color = #4, very thick] ({#1-#3/3.75}, {#2+#3})--(#1,
  {#2+#3})--(#1, {#2-#3})--({#1-#3/3.75}, {#2-#3}) %
}%

\newcommand\spointG[4]{%
  \draw[color = #4, very thick] ({#1+#3/1.75}, {#2-#3})--(#1,
  {#2-#3})--(#1, {#2+#3})--({#1+#3/1.75}, {#2+#3}) %
}%

\newcommand\spointD[4]{%
  \draw[color = #4, very thick] ({#1-#3/2}, {#2+#3})--(#1,
  {#2+#3})--(#1, {#2-#3})--({#1-#3/2}, {#2-#3}) %
}%

%%%%%%%%%%

\newcommand{\Pb}{\mathtt{P}}

%%%%%%%%%%%%%%%
%%% Pour citer un précédent item
%%%%%%%%%%%%%%%
\newcommand{\itbf}[1]{{\small \bf \textit{#1}}}


%%%%%%%%%%%%%%%
%%% Quelques couleurs
%%%%%%%%%%%%%%%

\xdefinecolor{cancelcolor}{named}{Red}
\xdefinecolor{intI}{named}{ProcessBlue}
\xdefinecolor{intJ}{named}{ForestGreen}

%%%%%%%%%%%%%%%
%%%%%%%%%%%%%%%
% barrer du texte
\usetikzlibrary{shapes.misc}

\makeatletter
% \definecolor{cancelcolor}{rgb}{0.127,0.372,0.987}
\newcommand{\tikz@bcancel}[1]{%
  \begin{tikzpicture}[baseline=(textbox.base),inner sep=0pt]
  \node[strike out,draw] (textbox) {#1}[thick, color=cancelcolor];
  \useasboundingbox (textbox);
  \end{tikzpicture}%
}
\newcommand{\bcancel}[1]{%
  \relax\ifmmode
    \mathchoice{\tikz@bcancel{$\displaystyle#1$}}
               {\tikz@bcancel{$\textstyle#1$}}
               {\tikz@bcancel{$\scriptstyle#1$}}
               {\tikz@bcancel{$\scriptscriptstyle#1$}}
  \else
    \tikz@bcancel{\strut#1}%
  \fi
}
\newcommand{\tikz@xcancel}[1]{%
  \begin{tikzpicture}[baseline=(textbox.base),inner sep=0pt]
  \node[cross out,draw] (textbox) {#1}[thick, color=cancelcolor];
  \useasboundingbox (textbox);
  \end{tikzpicture}%
}
\newcommand{\xcancel}[1]{%
  \relax\ifmmode
    \mathchoice{\tikz@xcancel{$\displaystyle#1$}}
               {\tikz@xcancel{$\textstyle#1$}}
               {\tikz@xcancel{$\scriptstyle#1$}}
               {\tikz@xcancel{$\scriptscriptstyle#1$}}
  \else
    \tikz@xcancel{\strut#1}%
  \fi
}
\makeatother

\newcommand{\xcancelRA}{\xcancel{\rule[-.15cm]{0cm}{.5cm} \Rightarrow
    \rule[-.15cm]{0cm}{.5cm}}}

%%%%%%%%%%%%%%%%%%%%%%%%%%%%%%%%%%%
%%%%%%%%%%%%%%%%%%%%%%%%%%%%%%%%%%%

\newcommand{\vide}{\multicolumn{1}{c}{}}

%%%%%%%%%%%%%%%%%%%%%%%%%%%%%%%%%%%
%%%%%%%%%%%%%%%%%%%%%%%%%%%%%%%%%%%


\usepackage{multicol}
% \usepackage[latin1]{inputenc}
% \usepackage[T1]{fontenc}
\usepackage[utf8]{inputenc}
\usepackage[T1]{fontenc}
\usepackage[normalem]{ulem}
\usepackage[french]{babel}

\usepackage{url}    
\usepackage{hyperref}
\hypersetup{
  backref=true,
  pagebackref=true,
  hyperindex=true,
  colorlinks=true,
  breaklinks=true,
  urlcolor=blue,
  linkcolor=red,
  bookmarks=true,
  bookmarksopen=true
}

%%%%%%%%%%%%%%%%%%%%%%%%%%%%%%%%%%%%%%%%%%%
%% Pour faire des traits diagonaux dans les tableaux
%% Nécessite slashbox.sty
%\usepackage{slashbox}

\usepackage{tipa}
\usepackage{verbatim,listings}
\usepackage{graphicx}
\usepackage{fancyhdr}
\usepackage{mathrsfs}
\usepackage{pifont}
\usepackage{tablists}
\usepackage{dsfont,amsfonts,amssymb,amsmath,amsthm,stmaryrd,upgreek,manfnt}
\usepackage{enumerate}

%\newcolumntype{M}[1]{p{#1}}
\newcolumntype{C}[1]{>{\centering}m{#1}}
\newcolumntype{R}[1]{>{\raggedright}m{#1}}
\newcolumntype{L}[1]{>{\raggedleft}m{#1}}
\newcolumntype{P}[1]{>{\raggedright}p{#1}}
\newcolumntype{B}[1]{>{\raggedright}b{#1}}
\newcolumntype{Q}[1]{>{\raggedright}t{#1}}

\newcommand{\alias}[2]{
\providecommand{#1}{}
\renewcommand{#1}{#2}
}
\alias{\R}{\mathbb{R}}
\alias{\N}{\mathbb{N}}
\alias{\Z}{\mathbb{Z}}
\alias{\Q}{\mathbb{Q}}
\alias{\C}{\mathbb{C}}
\alias{\K}{\mathbb{K}}

%%%%%%%%%%%%
%% rendre +infty et -infty plus petits
%%%%%%%%%%%%
\newcommand{\sinfty}{{\scriptstyle \infty}}

%%%%%%%%%%%%%%%%%%%%%%%%%%%%%%%
%%%%% macros TP Scilab %%%%%%%%
\newcommand{\Scilab}{\textbf{Scilab}} %
\newcommand{\Scinotes}{\textbf{SciNotes}} %
\newcommand{\faire}{\noindent $\blacktriangleright$ } %
\newcommand{\fitem}{\scalebox{.8}{$\blacktriangleright$}} %
\newcommand{\entree}{{\small\texttt{ENTRÉE}}} %
\newcommand{\tab}{{\small\texttt{TAB}}} %
\newcommand{\mt}[1]{\mathtt{#1}} %
% guillemets droits

\newcommand{\ttq}{\textquotesingle} %

\newcommand{\reponse}[1]{\longboxed{
    \begin{array}{C{0.9\textwidth}}
      \nl[#1]
    \end{array}
  }} %

\newcommand{\reponseL}[2]{\longboxed{
    \begin{array}{C{#2\textwidth}}
      \nl[#1]
    \end{array}
  }} %

\newcommand{\reponseR}[1]{\longboxed{
    \begin{array}{R{0.9\textwidth}}
      #1
    \end{array}
  }} %

\newcommand{\reponseRL}[2]{\longboxed{
    \begin{array}{R{#2\textwidth}}
      #1
    \end{array}
  }} %

\newcommand{\reponseC}[1]{\longboxed{
    \begin{array}{C{0.9\textwidth}}
      #1
    \end{array}
  }} %

\colorlet{pyfunction}{Blue}
\colorlet{pyCle}{Magenta}
\colorlet{pycomment}{LimeGreen}
\colorlet{pydoc}{Cyan}
% \colorlet{SansCo}{white}
% \colorlet{AvecCo}{black}

\newcommand{\visible}[1]{{\color{ASCo}\colorlet{pydoc}{pyDo}\colorlet{pycomment}{pyCo}\colorlet{pyfunction}{pyF}\colorlet{pyCle}{pyC}\colorlet{function}{sciFun}\colorlet{var}{sciVar}\colorlet{if}{sciIf}\colorlet{comment}{sciComment}#1}} %

%%%% à changer ????
\newcommand{\invisible}[1]{{\color{ASCo}\colorlet{pydoc}{pyDo}\colorlet{pycomment}{pyCo}\colorlet{pyfunction}{pyF}\colorlet{pyCle}{pyC}\colorlet{function}{sciFun}\colorlet{var}{sciVar}\colorlet{if}{sciIf}\colorlet{comment}{sciComment}#1}} %

\newcommand{\invisibleCol}[2]{{\color{#1}#2}} %

\NewEnviron{solution} %
{ %
  \Boxed{
    \begin{array}{>{\color{ASCo}} R{0.9\textwidth}}
      \colorlet{pycomment}{pyCo}
      \colorlet{pydoc}{pyDo}
      \colorlet{pyfunction}{pyF}
      \colorlet{pyCle}{pyC}
      \colorlet{function}{sciFun}
      \colorlet{var}{sciVar}
      \colorlet{if}{sciIf}
      \colorlet{comment}{sciComment}
      \BODY
    \end{array}
  } %
} %

\NewEnviron{solutionC} %
{ %
  \Boxed{
    \begin{array}{>{\color{ASCo}} C{0.9\textwidth}}
      \colorlet{pycomment}{pyCo}
      \colorlet{pydoc}{pyDo}
      \colorlet{pyfunction}{pyF}
      \colorlet{pyCle}{pyC}
      \colorlet{function}{sciFun}
      \colorlet{var}{sciVar}
      \colorlet{if}{sciIf}
      \colorlet{comment}{sciComment}
      \BODY
    \end{array}
  } %
} %

\newcommand{\invite}{--\!\!>} %

%%%%% nouvel environnement tabular pour retour console %%%%
\colorlet{ConsoleColor}{Blue!15}
\colorlet{function}{Red}
\colorlet{var}{Maroon}
\colorlet{if}{Magenta}
\colorlet{comment}{LimeGreen}

\newcommand{\tcVar}[1]{\textcolor{var}{\bf \small #1}} %
\newcommand{\tcFun}[1]{\textcolor{function}{#1}} %
\newcommand{\tcIf}[1]{\textcolor{if}{#1}} %
\newcommand{\tcFor}[1]{\textcolor{if}{#1}} %

\newcommand{\moins}{\!\!\!\!\!\!- }
\newcommand{\espn}{\!\!\!\!\!\!}

\usepackage{booktabs,varwidth} \newsavebox\TBox
\newenvironment{console}
{\begin{lrbox}{\TBox}\varwidth{\linewidth}
    \tabular{>{\tt\small}R{0.84\textwidth}}
    \nl[-.4cm]} {\endtabular\endvarwidth\end{lrbox}%
  \fboxsep=1pt\colorbox{ConsoleColor}{\usebox\TBox}}

\newcommand{\lInv}[1]{%
  $\invite$ #1} %

\newcommand{\lAns}[1]{%
  \qquad ans \ = \nl %
  \qquad \qquad #1} %

\newcommand{\lVar}[2]{%
  \qquad #1 \ = \nl %
  \qquad \qquad #2} %

\newcommand{\lDisp}[1]{%
  #1 %
} %

\newcommand{\ligne}[1]{\underline{\small \tt #1}} %

\newcommand{\ligneAns}[2]{%
  $\invite$ #1 \nl %
  \qquad ans \ = \nl %
  \qquad \qquad #2} %

\newcommand{\ligneVar}[3]{%
  $\invite$ #1 \nl %
  \qquad #2 \ = \nl %
  \qquad \qquad #3} %

\newcommand{\ligneErr}[3]{%
  $\invite$ #1 \nl %
  \quad !-{-}error #2 \nl %
  #3} %
%%%%%%%%%%%%%%%%%%%%%% 

\newcommand{\bs}[1]{\boldsymbol{#1}} %
\newcommand{\nll}{\nl[.4cm]} %
\newcommand{\nle}{\nl[.2cm]} %
%% opérateur puissance copiant l'affichage Scilab
%\newcommand{\puis}{\!\!\!~^{\scriptscriptstyle\pmb{\wedge}}}
\newcommand{\puis}{\mbox{$\hspace{-.1cm}~^{\scriptscriptstyle\pmb{\wedge}}
    \hspace{0.05cm}$}} %
\newcommand{\pointpuis}{.\mbox{$\hspace{-.15cm}~^{\scriptscriptstyle\pmb{\wedge}}$}} %
\newcommand{\Sfois}{\mbox{$\mt{\star}$}} %

%%%%% nouvel environnement tabular pour les encadrés Scilab %%%%
\newenvironment{encadre}
{\begin{lrbox}{\TBox}\varwidth{\linewidth}
    \tabular{>{\tt\small}C{0.1\textwidth}>{\small}R{0.7\textwidth}}}
  {\endtabular\endvarwidth\end{lrbox}%
  \fboxsep=1pt\longboxed{\usebox\TBox}}

\newenvironment{encadreL}
{\begin{lrbox}{\TBox}\varwidth{\linewidth}
    \tabular{>{\tt\small}C{0.25\textwidth}>{\small}R{0.6\textwidth}}}
  {\endtabular\endvarwidth\end{lrbox}%
  \fboxsep=1pt\longboxed{\usebox\TBox}}

\newenvironment{encadreF}
{\begin{lrbox}{\TBox}\varwidth{\linewidth}
    \tabular{>{\tt\small}C{0.2\textwidth}>{\small}R{0.70\textwidth}}}
  {\endtabular\endvarwidth\end{lrbox}%
  \fboxsep=1pt\longboxed{\usebox\TBox}}

\newenvironment{encadreLL}[2]
{\begin{lrbox}{\TBox}\varwidth{\linewidth}
    \tabular{>{\tt\small}C{#1\textwidth}>{\small}R{#2\textwidth}}}
  {\endtabular\endvarwidth\end{lrbox}%
  \fboxsep=1pt\longboxed{\usebox\TBox}}

%%%%% nouvel environnement tabular pour les script et fonctions %%%%
\newcommand{\commentaireDL}[1]{\multicolumn{1}{l}{\it
    \textcolor{comment}{$\slash\slash$ #1}}}

\newcommand{\commentaire}[1]{{\textcolor{comment}{$\slash\slash$ #1}}}

\newcounter{cptcol}

\newcommand{\nocount}{\multicolumn{1}{c}{}}

\newcommand{\sciNo}[1]{{\small \underbar #1}}

\NewEnviron{scilab}{ %
  \setcounter{cptcol}{0}
  \begin{center}
    \longboxed{
      \begin{tabular}{>{\stepcounter{cptcol}{\tiny \underbar
              \thecptcol}}c>{\tt}l}
        \BODY
      \end{tabular}
    }
  \end{center}
}

\NewEnviron{scilabNC}{ %
  \begin{center}
    \longboxed{
      \begin{tabular}{>{\tt}l} %
          \BODY
      \end{tabular}
    }
  \end{center}
}

\NewEnviron{scilabC}[1]{ %
  \setcounter{cptcol}{#1}
  \begin{center}
    \longboxed{
      \begin{tabular}{>{\stepcounter{cptcol}{\tiny \underbar
              \thecptcol}}c>{\tt}l}
        \BODY
      \end{tabular}
    }
  \end{center}
}

\newcommand{\scisol}[1]{ %
  \setcounter{cptcol}{0}
  \longboxed{
    \begin{tabular}{>{\stepcounter{cptcol}{\tiny \underbar
            \thecptcol}}c>{\tt}l}
      #1
    \end{tabular}
  }
}

\newcommand{\scisolNC}[1]{ %
  \longboxed{
    \begin{tabular}{>{\tt}l}
      #1
    \end{tabular}
  }
}

\newcommand{\scisolC}[2]{ %
  \setcounter{cptcol}{#1}
  \longboxed{
    \begin{tabular}{>{\stepcounter{cptcol}{\tiny \underbar
            \thecptcol}}c>{\tt}l}
      #2
    \end{tabular}
  }
}

\NewEnviron{syntaxe}{ %
  % \fcolorbox{black}{Yellow!20}{\setlength{\fboxsep}{3mm}
  \shadowbox{
    \setlength{\fboxsep}{3mm}
    \begin{tabular}{>{\tt}l}
      \BODY
    \end{tabular}
  }
}

%%%%% fin macros TP Scilab %%%%%%%%
%%%%%%%%%%%%%%%%%%%%%%%%%%%%%%%%%%%

%%%%%%%%%%%%%%%%%%%%%%%%%%%%%%%%%%%
%%%%% TP Python - listings %%%%%%%%
%%%%%%%%%%%%%%%%%%%%%%%%%%%%%%%%%%%
\newcommand{\Python}{\textbf{Python}} %

\lstset{% general command to set parameter(s)
basicstyle=\ttfamily\small, % print whole listing small
keywordstyle=\color{blue}\bfseries\underbar,
%% underlined bold black keywords
frame=lines,
xleftmargin=10mm,
numbers=left,
numberstyle=\tiny\underbar,
numbersep=10pt,
%identifierstyle=, % nothing happens
commentstyle=\color{green}, % white comments
%%stringstyle=\ttfamily, % typewriter type for strings
showstringspaces=false}

\newcommand{\pysolCpt}[2]{
  \setcounter{cptcol}{#1}
  \longboxed{
    \begin{tabular}{>{\stepcounter{cptcol}{\tiny \underbar
            \thecptcol}}c>{\tt}l}
        #2
      \end{tabular}
    }
} %

\newcommand{\pysol}[1]{
  \setcounter{cptcol}{0}
  \longboxed{
    \begin{tabular}{>{\stepcounter{cptcol}{\tiny \underbar
            \thecptcol}}c>{\tt}l}
        #1
      \end{tabular}
    }
} %

% \usepackage[labelsep=endash]{caption}

% avec un caption
\NewEnviron{pythonCap}[1]{ %
  \renewcommand{\tablename}{Programme}
  \setcounter{cptcol}{0}
  \begin{center}
    \longboxed{
      \begin{tabular}{>{\stepcounter{cptcol}{\tiny \underbar
              \thecptcol}}c>{\tt}l}
        \BODY
      \end{tabular}
    }
    \captionof{table}{#1}
  \end{center}
}

\NewEnviron{python}{ %
  \setcounter{cptcol}{0}
  \begin{center}
    \longboxed{
      \begin{tabular}{>{\stepcounter{cptcol}{\tiny \underbar
              \thecptcol}}c>{\tt}l}
        \BODY
      \end{tabular}
    }
  \end{center}
}

\newcommand{\pyVar}[1]{\textcolor{var}{\bf \small #1}} %
\newcommand{\pyFun}[1]{\textcolor{pyfunction}{#1}} %
\newcommand{\pyCle}[1]{\textcolor{pyCle}{#1}} %
\newcommand{\pyImp}[1]{{\bf #1}} %

%%%%% commentaire python %%%%
\newcommand{\pyComDL}[1]{\multicolumn{1}{l}{\textcolor{pycomment}{\#
      #1}}}

\newcommand{\pyCom}[1]{{\textcolor{pycomment}{\# #1}}}
\newcommand{\pyDoc}[1]{{\textcolor{pydoc}{#1}}}

\newcommand{\pyNo}[1]{{\small \underbar #1}}

%%%%%%%%%%%%%%%%%%%%%%%%%%%%%%%%%%%
%%%%%% Système linéaire paramétré : écrire les opérations au-dessus
%%%%%% d'un symbole équivalent
%%%%%%%%%%%%%%%%%%%%%%%%%%%%%%%%%%%

\usepackage{systeme}

\NewEnviron{arrayEq}{ %
  \stackrel{\scalebox{.6}{$
      \begin{array}{l} 
        \BODY \\[.1cm]
      \end{array}$}
  }{\Longleftrightarrow}
}

\NewEnviron{arrayEg}{ %
  \stackrel{\scalebox{.6}{$
      \begin{array}{l} 
        \BODY \\[.1cm]
      \end{array}$}
  }{=}
}

\NewEnviron{operationEq}{ %
  \scalebox{.6}{$
    \begin{array}{l} 
      \scalebox{1.6}{$\mbox{Opérations :}$} \\[.2cm]
      \BODY \\[.1cm]
    \end{array}$}
}

% \NewEnviron{arraySys}[1]{ %
%   \sysdelim\{.\systeme[#1]{ %
%     \BODY %
%   } %
% }

%%%%%

%%%%%%%%%%
%%%%%%%%%% ESSAI
\newlength\fboxseph
\newlength\fboxsepva
\newlength\fboxsepvb

\setlength\fboxsepva{0.2cm}
\setlength\fboxsepvb{0.2cm}
\setlength\fboxseph{0.2cm}

\makeatletter

\def\longboxed#1{\leavevmode\setbox\@tempboxa\hbox{\color@begingroup%
\kern\fboxseph{\m@th$\displaystyle #1 $}\kern\fboxseph%
\color@endgroup }\my@frameb@x\relax}

\def\my@frameb@x#1{%
  \@tempdima\fboxrule \advance\@tempdima \fboxsepva \advance\@tempdima
  \dp\@tempboxa\hbox {%
    \lower \@tempdima \hbox {%
      \vbox {\hrule\@height\fboxrule \hbox{\vrule\@width\fboxrule #1
          \vbox{%
            \vskip\fboxsepva \box\@tempboxa \vskip\fboxsepvb}#1
          \vrule\@width\fboxrule }%
        \hrule \@height \fboxrule }}}}

\newcommand{\boxedhv}[3]{\setlength\fboxseph{#1cm}
  \setlength\fboxsepva{#2cm}\setlength\fboxsepvb{#2cm}\longboxed{#3}}

\newcommand{\boxedhvv}[4]{\setlength\fboxseph{#1cm}
  \setlength\fboxsepva{#2cm}\setlength\fboxsepvb{#3cm}\longboxed{#4}}

\newcommand{\Boxed}[1]{{\setlength\fboxseph{0.2cm}
  \setlength\fboxsepva{0.2cm}\setlength\fboxsepvb{0.2cm}\longboxed{#1}}}

\newcommand{\mBoxed}[1]{{\setlength\fboxseph{0.2cm}
  \setlength\fboxsepva{0.2cm}\setlength\fboxsepvb{0.2cm}\longboxed{\mbox{#1}}}}

\newcommand{\mboxed}[1]{{\setlength\fboxseph{0.2cm}
  \setlength\fboxsepva{0.2cm}\setlength\fboxsepvb{0.2cm}\boxed{\mbox{#1}}}}

\newsavebox{\fmbox}
\newenvironment{fmpage}[1]
     {\begin{lrbox}{\fmbox}\begin{minipage}{#1}}
     {\end{minipage}\end{lrbox}\fbox{\usebox{\fmbox}}}

%%%%%%%%%%
%%%%%%%%%%

\DeclareMathOperator{\ch}{ch}
\DeclareMathOperator{\sh}{sh}

%%%%%%%%%%
%%%%%%%%%%

\newcommand{\norme}[1]{\Vert #1 \Vert}

%\newcommand*\widefbox[1]{\fbox{\hspace{2em}#1\hspace{2em}}}

\newcommand{\nl}{\tabularnewline}

\newcommand{\hand}{\noindent\ding{43}\ }
\newcommand{\ie}{\textit{i.e. }}
\newcommand{\cf}{\textit{cf }}

\newcommand{\Card}{\operatorname{Card}}

\newcommand{\aire}{\mathcal{A}}

\newcommand{\LL}[1]{\mathscr{L}(#1)} %
\newcommand{\B}{\mathscr{B}} %
\newcommand{\Bc}[1]{B_{#1}} %
\newcommand{\M}[1]{\mathscr{M}_{#1}(\mathbb{R})}

\DeclareMathOperator{\im}{Im}
\DeclareMathOperator{\kr}{Ker}
\DeclareMathOperator{\rg}{rg}
\DeclareMathOperator{\tr}{tr}
\DeclareMathOperator{\spc}{Sp}
\DeclareMathOperator{\sgn}{sgn}
\DeclareMathOperator{\supp}{Supp}

\newcommand{\Mat}{{\rm{Mat}}}
\newcommand{\Vect}[1]{{\rm{Vect}}\left(#1\right)}

\newenvironment{smatrix}{%
  \begin{adjustbox}{width=.9\width}
    $
    \begin{pmatrix}
    }{%      
    \end{pmatrix}
    $
  \end{adjustbox}
}

\newenvironment{sarray}[1]{%
  \begin{adjustbox}{width=.9\width}
    $
    \begin{array}{#1}
    }{%      
    \end{array}
    $
  \end{adjustbox}
}

\newcommand{\vd}[2]{
  \scalebox{.8}{
    $\left(\!
      \begin{array}{c}
        #1 \\
        #2
      \end{array}
    \!\right)$
    }}

\newcommand{\vt}[3]{
  \scalebox{.8}{
    $\left(\!
      \begin{array}{c}
        #1 \\
        #2 \\
        #3 
      \end{array}
    \!\right)$
    }}

\newcommand{\vq}[4]{
  \scalebox{.8}{
    $\left(\!
      \begin{array}{c}
        #1 \\
        #2 \\
        #3 \\
        #4 
      \end{array}
    \!\right)$
    }}

\newcommand{\vc}[5]{
  \scalebox{.8}{
    $\left(\!
      \begin{array}{c}
        #1 \\
        #2 \\
        #3 \\
        #4 \\
        #5 
      \end{array}
    \!\right)$
    }}

\newcommand{\ee}{\text{e}}

\newcommand{\dd}{\text{d}}

%%% Ensemble de définition
\newcommand{\Df}{\mathscr{D}}
\newcommand{\Cf}{\mathscr{C}}
\newcommand{\Ef}{\mathscr{C}}

\newcommand{\rond}[1]{\,\overset{\scriptscriptstyle \circ}{\!#1}}

\newcommand{\df}[2]{\dfrac{\partial #1}{\partial #2}} %
\newcommand{\dfn}[2]{\partial_{#2}(#1)} %
\newcommand{\ddfn}[2]{\partial^2_{#2}(#1)} %
\newcommand{\ddf}[2]{\dfrac{\partial^2 #1}{\partial #2^2}} %
\newcommand{\ddfr}[3]{\dfrac{\partial^2 #1}{\partial #2 \partial
    #3}} %


\newcommand{\dlim}[1]{{\displaystyle \lim_{#1} \ }}
\newcommand{\dlimPlus}[2]{
  \dlim{
    \scalebox{.6}{
      $
      \begin{array}{l}
        #1 \rightarrow #2\\
        #1 > #2
      \end{array}
      $}}}
\newcommand{\dlimMoins}[2]{
  \dlim{
    \scalebox{.6}{
      $
      \begin{array}{l}
        #1 \rightarrow #2\\
        #1 < #2
      \end{array}
      $}}}

%%%%%%%%%%%%%%
%% petit o, développement limité
%%%%%%%%%%%%%%

\newcommand{\oo}[2]{{\underset {{\overset {#1\rightarrow #2}{}}}{o}}} %
\newcommand{\oox}[1]{{\underset {{\overset {x\rightarrow #1}{}}}{o}}} %
\newcommand{\oon}{{\underset {{\overset {n\rightarrow +\infty}{}}}{o}}} %
\newcommand{\po}[1]{{\underset {{\overset {#1}{}}}{o}}} %
\newcommand{\neqx}[1]{{\ \underset {{\overset {x \to #1}{}}}{\not\sim}\ }} %
\newcommand{\eqx}[1]{{\ \underset {{\overset {x \to #1}{}}}{\sim}\ }} %
\newcommand{\eqn}{{\ \underset {{\overset {n \to +\infty}{}}}{\sim}\ }} %
\newcommand{\eq}[2]{{\ \underset {{\overset {#1 \to #2}{}}}{\sim}\ }} %
\newcommand{\DL}[1]{{\rm{DL}}_1 (#1)} %
\newcommand{\DLL}[1]{{\rm{DL}}_2 (#1)} %

\newcommand{\negl}{<<}

\newcommand{\neglP}[1]{\begin{array}{c}
    \vspace{-.2cm}\\
    << \\
    \vspace{-.7cm}\\
    {\scriptstyle #1}
  \end{array}}

%%%%%%%%%%%%%%
%% borne sup, inf, max, min
%%%%%%%%%%%%%%
\newcommand{\dsup}[1]{\displaystyle \sup_{#1} \ }
\newcommand{\dinf}[1]{\displaystyle \inf_{#1} \ }
\newcommand{\dmax}[1]{\max\limits_{#1} \ }
\newcommand{\dmin}[1]{\min\limits_{#1} \ }

\newcommand{\dcup}[2]{{\textstyle\bigcup\limits_{#1}^{#2}}\hspace{.1cm}}
%\displaystyle \bigcup_{#1}^{#2}}
\newcommand{\dcap}[2]{{\textstyle\bigcap\limits_{#1}^{#2}}\hspace{.1cm}}
% \displaystyle \bigcap_{#1}^{#2}
%%%%%%%%%%%%%%
%% opérateurs logiques
%%%%%%%%%%%%%%
\newcommand{\NON}[1]{\mathop{\small \tt{NON}} (#1)}
\newcommand{\ET}{\mathrel{\mathop{\small \mathtt{ET}}}}
\newcommand{\OU}{\mathrel{\mathop{\small \tt{OU}}}}
\newcommand{\XOR}{\mathrel{\mathop{\small \tt{XOR}}}}

\newcommand{\id}{{\rm{id}}}

\newcommand{\sbullet}{\scriptstyle \bullet}
\newcommand{\stimes}{\scriptstyle \times}

%%%%%%%%%%%%%%%%%%
%% Probabilités
%%%%%%%%%%%%%%%%%%
\newcommand{\Prob}{\mathbb{P}}
\newcommand{\Ev}[1]{\left[ {#1} \right]}
\newcommand{\E}{\mathbb{E}}
\newcommand{\V}{\mathbb{V}}
\newcommand{\Cov}{{\rm{Cov}}}
\newcommand{\U}[2]{\mathcal{U}(\llb #1, #2\rrb)}
\newcommand{\Uc}[2]{\mathcal{U}([#1, #2])}
\newcommand{\Ucof}[2]{\mathcal{U}(]#1, #2])}
\newcommand{\Ucoo}[2]{\mathcal{U}(]#1, #2[)}
\newcommand{\Ucfo}[2]{\mathcal{U}([#1, #2[)}
\newcommand{\Bern}[1]{\mathcal{B}\left(#1\right)}
\newcommand{\Bin}[2]{\mathcal{B}\left(#1, #2\right)}
\newcommand{\G}[1]{\mathcal{G}\left(#1\right)}
\newcommand{\Pois}[1]{\mathcal{P}\left(#1\right)}
\newcommand{\HG}[3]{\mathcal{H}\left(#1, #2, #3\right)}
\newcommand{\Exp}[1]{\mathcal{E}\left(#1\right)}
\newcommand{\Norm}[2]{\mathcal{N}\left(#1, #2\right)}

\DeclareMathOperator{\cov}{Cov}

\newcommand{\var}{v.a.r. }
\newcommand{\suit}{\hookrightarrow}

\newcommand{\flecheR}[1]{\rotatebox{90}{\scalebox{#1}{\color{red}
      $\curvearrowleft$}}}


\newcommand{\partie}[1]{\mathcal{P}(#1)}
\newcommand{\Cont}[1]{\mathcal{C}^{#1}}
\newcommand{\Contm}[1]{\mathcal{C}^{#1}_m}

\newcommand{\llb}{\llbracket}
\newcommand{\rrb}{\rrbracket}

%\newcommand{\im}[1]{{\rm{Im}}(#1)}
\newcommand{\imrec}[1]{#1^{- \mathds{1}}}

\newcommand{\unq}{\mathds{1}}

\newcommand{\Hyp}{\mathtt{H}}

\newcommand{\eme}[1]{#1^{\scriptsize \mbox{ème}}}
\newcommand{\er}[1]{#1^{\scriptsize \mbox{er}}}
\newcommand{\ere}[1]{#1^{\scriptsize \mbox{ère}}}
\newcommand{\nd}[1]{#1^{\scriptsize \mbox{nd}}}
\newcommand{\nde}[1]{#1^{\scriptsize \mbox{nde}}}

\newcommand{\truc}{\mathop{\top}}
\newcommand{\fois}{\mathop{\ast}}

\newcommand{\f}[1]{\overrightarrow{#1}}

\newcommand{\checked}{\textcolor{green}{\checkmark}}

\def\restriction#1#2{\mathchoice
              {\setbox1\hbox{${\displaystyle #1}_{\scriptstyle #2}$}
              \restrictionaux{#1}{#2}}
              {\setbox1\hbox{${\textstyle #1}_{\scriptstyle #2}$}
              \restrictionaux{#1}{#2}}
              {\setbox1\hbox{${\scriptstyle #1}_{\scriptscriptstyle #2}$}
              \restrictionaux{#1}{#2}}
              {\setbox1\hbox{${\scriptscriptstyle #1}_{\scriptscriptstyle #2}$}
              \restrictionaux{#1}{#2}}}
\def\restrictionaux#1#2{{#1\,\smash{\vrule height .8\ht1 depth .85\dp1}}_{\,#2}}

\makeatletter
\newcommand*{\da@rightarrow}{\mathchar"0\hexnumber@\symAMSa 4B }
\newcommand*{\da@leftarrow}{\mathchar"0\hexnumber@\symAMSa 4C }
\newcommand*{\xdashrightarrow}[2][]{%
  \mathrel{%
    \mathpalette{\da@xarrow{#1}{#2}{}\da@rightarrow{\,}{}}{}%
  }%
}
\newcommand{\xdashleftarrow}[2][]{%
  \mathrel{%
    \mathpalette{\da@xarrow{#1}{#2}\da@leftarrow{}{}{\,}}{}%
  }%
}
\newcommand*{\da@xarrow}[7]{%
  % #1: below
  % #2: above
  % #3: arrow left
  % #4: arrow right
  % #5: space left 
  % #6: space right
  % #7: math style 
  \sbox0{$\ifx#7\scriptstyle\scriptscriptstyle\else\scriptstyle\fi#5#1#6\m@th$}%
  \sbox2{$\ifx#7\scriptstyle\scriptscriptstyle\else\scriptstyle\fi#5#2#6\m@th$}%
  \sbox4{$#7\dabar@\m@th$}%
  \dimen@=\wd0 %
  \ifdim\wd2 >\dimen@
    \dimen@=\wd2 %   
  \fi
  \count@=2 %
  \def\da@bars{\dabar@\dabar@}%
  \@whiledim\count@\wd4<\dimen@\do{%
    \advance\count@\@ne
    \expandafter\def\expandafter\da@bars\expandafter{%
      \da@bars
      \dabar@ 
    }%
  }%  
  \mathrel{#3}%
  \mathrel{%   
    \mathop{\da@bars}\limits
    \ifx\\#1\\%
    \else
      _{\copy0}%
    \fi
    \ifx\\#2\\%
    \else
      ^{\copy2}%
    \fi
  }%   
  \mathrel{#4}%
}
\makeatother



\newcount\depth

\newcount\depth
\newcount\totaldepth

\makeatletter
\newcommand{\labelsymbol}{%
      \ifnum\depth=0
        %
      \else
        \rlap{\,$\bullet$}%
      \fi
}

\newcommand*\bernoulliTree[1]{%
    \depth=#1\relax            
    \totaldepth=#1\relax
    \draw node(root)[bernoulli/root] {\labelsymbol}[grow=right] \draw@bernoulli@tree;
    \draw \label@bernoulli@tree{root};                                   
}                                                                        

\def\draw@bernoulli@tree{%
    \ifnum\depth>0 
      child[parent anchor=east] foreach \type/\label in {left child/$E$,right child/$S$} {%
          node[bernoulli/\type] {\label\strut\labelsymbol} \draw@bernoulli@tree
      }
      coordinate[bernoulli/increment] (dummy)
   \fi%
}

\def\label@bernoulli@tree#1{%
    \ifnum\depth>0
      ($(#1)!0.5!(#1-1)$) node[fill=white,bernoulli/decrement] {\tiny$p$}
      \label@bernoulli@tree{#1-1}
      ($(#1)!0.5!(#1-2)$) node[fill=white] {\tiny$q$}
      \label@bernoulli@tree{#1-2}
      coordinate[bernoulli/increment] (dummy)
   \fi%
}

\makeatother

\tikzset{bernoulli/.cd,
         root/.style={},
         decrement/.code=\global\advance\depth by-1\relax,
         increment/.code=\global\advance\depth by 1\relax,
         left child/.style={bernoulli/decrement},
         right child/.style={}}


\newcommand{\eps}{\varepsilon}

% \newcommand{\tendi}[1]{\xrightarrow[\footnotesize #1 \rightarrow
%   +\infty]{}}%

\newcommand{\tend}{\rightarrow}%
\newcommand{\tendn}{\underset{n\to +\infty}{\longrightarrow}} %
\newcommand{\ntendn}{\underset{n\to
    +\infty}{\not\hspace{-.15cm}\longrightarrow}} %
% \newcommand{\tendn}{\xrightarrow[\footnotesize n \rightarrow
%   +\infty]{}}%
\newcommand{\Tendx}[1]{\xrightarrow[\footnotesize x \rightarrow
  #1]{}}%
\newcommand{\tendx}[1]{\underset{x\to #1}{\longrightarrow}}%
\newcommand{\ntendx}[1]{\underset{x\to #1}{\not\!\!\longrightarrow}}%
\newcommand{\tendd}[2]{\underset{#1\to #2}{\longrightarrow}}%
% \newcommand{\tendd}[2]{\xrightarrow[\footnotesize #1 \rightarrow
%   #2]{}}%
\newcommand{\tendash}[1]{\xdashrightarrow[\footnotesize #1 \rightarrow
  +\infty]{}}%
\newcommand{\tendashx}[1]{\xdashrightarrow[\footnotesize x \rightarrow
  #1]{}}%
\newcommand{\tendb}[1]{\underset{#1 \to +\infty}{\longrightarrow}}%
\newcommand{\tendL}{\overset{\mathscr L}{\underset{n \to
      +\infty}{\longrightarrow}}}%
\newcommand{\tendP}{\overset{\Prob}{\underset{n \to
      +\infty}{\longrightarrow}}}%
\newcommand{\tenddL}[1]{\overset{\mathscr L}{\underset{#1 \to
      +\infty}{\longrightarrow}}}%

\NewEnviron{attention}{ %
  ~\\[-.2cm]\noindent
  \begin{minipage}{\linewidth}
  \setlength{\fboxsep}{3mm}%
  \ \ \dbend \ \ %
  \fbox{\parbox[t]{.88\linewidth}{\BODY}} %
  \end{minipage}\\
}

\NewEnviron{sattention}[1]{ %
  ~\\[-.2cm]\noindent
  \begin{minipage}{#1\linewidth}
  \setlength{\fboxsep}{3mm}%
  \ \ \dbend \ \ %
  \fbox{\parbox[t]{.88\linewidth}{\BODY}} %
  \end{minipage}\\
}

%%%%% OBSOLETE %%%%%%

% \newcommand{\attention}[1]{
%   \noindent
%   \begin{tabular}{@{}l|p{11.5cm}|}
%     \cline{2-2}
%     \vspace{-.2cm} 
%     & \nl
%     \dbend & #1 \nl
%     \cline{2-2}
%   \end{tabular}
% }

% \newcommand{\attentionv}[2]{
%   \noindent
%   \begin{tabular}{@{}l|p{11.5cm}|}
%     \cline{2-2}
%     \vspace{-.2cm} 
%     & \nl
%     \dbend & #2 \nl[#1 cm]
%     \cline{2-2}
%   \end{tabular}
% }

\newcommand{\explainvb}[2]{
  \noindent
  \begin{tabular}{@{}l|p{11.5cm}|}
    \cline{2-2}
    \vspace{-.2cm} 
    & \nl
    \hand & #2 \nl [#1 cm]
    \cline{2-2}
  \end{tabular}
}


% \noindent
% \begin{tabular}{@{}l|lp{11cm}|}
%   \cline{3-3} 
%   \multicolumn{1}{@{}l@{\dbend}}{} & & #1 \nl
%   \multicolumn{1}{l}{} & & \nl [-.8cm]
%   & & #2 \nl
%   \cline{2-3}
% \end{tabular}

% \newcommand{\attention}[1]{
%   \noindent
%   \begin{tabular}{@{}@{}cp{11cm}}
%     \dbend & #1 \nl
%   \end{tabular}
% }

\newcommand{\PP}[1]{\mathcal{P}(#1)}
\newcommand{\FF}[1]{\mathcal{F}(#1)}

\newcommand{\DSum}[2]{\displaystyle\sum\limits_{#1}^{#2}\hspace{.1cm}}
\newcommand{\Sum}[2]{{\textstyle\sum\limits_{#1}^{#2}}\hspace{.1cm}}
\newcommand{\Serie}{\textstyle\sum\ }
\newcommand{\Prod}[2]{\textstyle\prod\limits_{#1}^{#2}}

\newcommand{\Prim}[3]{\left[\ {#1} \ \right]_{\scriptscriptstyle
   \hspace{-.15cm} ~_{#2}\, }^ {\scriptscriptstyle \hspace{-.15cm} ~^{#3}\, }}

% \newcommand{\Prim}[3]{\left[\ {#1} \ \right]_{\scriptscriptstyle
%     \!\!~_{#2}}^ {\scriptscriptstyle \!\!~^{#3}}}

\newcommand{\dint}[2]{\displaystyle \int_{#1}^{#2}\ }
\newcommand{\Int}[2]{{\rm{Int}}_{\scriptscriptstyle #1, #2}}
\newcommand{\dt}{\ dt}
\newcommand{\dx}{\ dx}

\newcommand{\llpar}[1]{\left(\!\!\!
    \begin{array}{c}
      \rule{0pt}{#1}
    \end{array}
  \!\!\!\right.}

\newcommand{\rrpar}[1]{\left.\!\!\!
    \begin{array}{c}
      \rule{0pt}{#1}
    \end{array}
  \!\!\!\right)}

\newcommand{\llacc}[1]{\left\{\!\!\!
    \begin{array}{c}
      \rule{0pt}{#1}
    \end{array}
  \!\!\!\right.}

\newcommand{\rracc}[1]{\left.\!\!\!
    \begin{array}{c}
      \rule{0pt}{#1}
    \end{array}
  \!\!\!\right\}}

\newcommand{\ttacc}[1]{\mbox{\rotatebox{-90}{\hspace{-.7cm}$\llacc{#1}$}}}
\newcommand{\bbacc}[1]{\mbox{\rotatebox{90}{\hspace{-.5cm}$\llacc{#1}$}}}

\newcommand{\comp}[1]{\overline{#1}}%

\newcommand{\dcomp}[2]{\stackrel{\mbox{\ \ \----}{\scriptscriptstyle
      #2}}{#1}}%

% \newcommand{\Comp}[2]{\stackrel{\mbox{\ \
%       \-------}{\scriptscriptstyle #2}}{#1}}

% \newcommand{\dcomp}[2]{\stackrel{\mbox{\ \
%       \-------}{\scriptscriptstyle #2}}{#1}}

\newcommand{\A}{\mathscr{A}}

\newcommand{\conc}[1]{
  \begin{center}
    \fbox{
      \begin{tabular}{c}
        #1
      \end{tabular}
    }
  \end{center}
}

\newcommand{\concC}[1]{
  \begin{center}
    \fbox{
    \begin{tabular}{C{10cm}}
      \quad #1 \quad
    \end{tabular}
    }
  \end{center}
}

\newcommand{\concL}[2]{
  \begin{center}
    \fbox{
    \begin{tabular}{C{#2cm}}
      \quad #1 \quad
    \end{tabular}
    }
  \end{center}
}

% \newcommand{\lims}[2]{\prod\limits_{#1}^{#2}}

\newtheorem{theorem}{Théorème}[]
\newtheorem{lemma}{Lemme}[]
\newtheorem{proposition}{Proposition}[]
\newtheorem{corollary}{Corollaire}[]

% \newenvironment{proof}[1][Démonstration]{\begin{trivlist}
% \item[\hskip \labelsep {\bfseries #1}]}{\end{trivlist}}
\newenvironment{definition}[1][Définition]{\begin{trivlist}
\item[\hskip \labelsep {\bfseries #1}]}{\end{trivlist}}
\newenvironment{example}[1][Exemple]{\begin{trivlist}
\item[\hskip \labelsep {\bfseries #1}]}{\end{trivlist}}
\newenvironment{examples}[1][Exemples]{\begin{trivlist}
\item[\hskip \labelsep {\bfseries #1}]}{\end{trivlist}}
\newenvironment{notation}[1][Notation]{\begin{trivlist}
\item[\hskip \labelsep {\bfseries #1}]}{\end{trivlist}}
\newenvironment{propriete}[1][Propriété]{\begin{trivlist}
\item[\hskip \labelsep {\bfseries #1}]}{\end{trivlist}}
\newenvironment{proprietes}[1][Propriétés]{\begin{trivlist}
\item[\hskip \labelsep {\bfseries #1}]}{\end{trivlist}}
\newenvironment{remarkSC}[1][Remarque]{\begin{trivlist}
\item[\hskip \labelsep {\bfseries #1}]}{\end{trivlist}}
\newenvironment{application}[1][Application]{\begin{trivlist}
\item[\hskip \labelsep {\bfseries #1}]}{\end{trivlist}}

% Environnement pour les réponses des DS
\newenvironment{answer}{\par\emph{Réponse :}\par{}}
{\vspace{-.6cm}\hspace{\stretch{1}}\rule{1ex}{1ex}\vspace{.3cm}}

\newenvironment{answerTD}{\vspace{.2cm}\par\emph{Réponse :}\par{}}
{\hspace{\stretch{1}}\rule{1ex}{1ex}\vspace{.3cm}}

\newenvironment{answerCours}{\noindent\emph{Réponse :}}
{\rule{1ex}{1ex}}%\vspace{.3cm}}


% footnote in footer
\newcommand{\fancyfootnotetext}[2]{%
  \fancypagestyle{dingens}{%
    \fancyfoot[LO,RE]{\parbox{0.95\textwidth}{\footnotemark[#1]\footnotesize
        #2}}%
  }%
  \thispagestyle{dingens}%
}

%%% définit le style (arabic : 1,2,3...) et place des parenthèses
%%% autour de la numérotation
\renewcommand*{\thefootnote}{(\arabic{footnote})}
% http://www.tuteurs.ens.fr/logiciels/latex/footnote.html

%%%%%%%% tikz axis
% \pgfplotsset{every axis/.append style={
%                     axis x line=middle,    % put the x axis in the middle
%                     axis y line=middle,    % put the y axis in the middle
%                     axis line style={<->,color=blue}, % arrows on the axis
%                     xlabel={$x$},          % default put x on x-axis
%                     ylabel={$y$},          % default put y on y-axis
%             }}

%%%% s'utilise comme suit

% \begin{axis}[
%   xmin=-8,xmax=4,
%   ymin=-8,ymax=4,
%   grid=both,
%   ]
%   \addplot [domain=-3:3,samples=50]({x^3-3*x},{3*x^2-9}); 
% \end{axis}

%%%%%%%%



%%%%%%%%%%%% Pour avoir des numéros de section qui correspondent à
%%%%%%%%%%%% ceux du tableau
\renewcommand{\thesection}{\Roman{section}.\hspace{-.3cm}}
\renewcommand{\thesubsection}{\Roman{section}.\arabic{subsection}.\hspace{-.2cm}}
\renewcommand{\thesubsubsection}{\Roman{section}.\arabic{subsection}.\alph{subsubsection})\hspace{-.2cm}}
%%%%%%%%%%%% 

%%% Changer le nom des figures : Fig. au lieu de Figure
\usepackage[font=small,labelfont=bf,labelsep=space]{caption}
\captionsetup{%
  figurename=Fig.,
  tablename=Tab.
}
% \renewcommand{\thesection}{\Roman{section}.\hspace{-.2cm}}
% \renewcommand{\thesubsection}{\Roman{section}
%   .\hspace{.2cm}\arabic{subsection}\ .\hspace{-.3cm}}
% \renewcommand{\thesubsubsection}{\alph{subsection})}

\newenvironment{tabliste}[1]
{\begin{tabenum}[\bfseries\small\itshape #1]}{\end{tabenum}} 

\newenvironment{noliste}[1] %
{\begin{enumerate}[\bfseries\small\itshape #1]} %
  {\end{enumerate}}

\newenvironment{nonoliste}[1] %
{\begin{enumerate}[\hspace{-12pt}\bfseries\small\itshape #1]} %
  {\end{enumerate}}

\newenvironment{arrayliste}[1]{ 
  % List with minimal white space to fit in small areas, e.g. table
  % cell
  \begin{minipage}[t]{\linewidth} %
    \begin{enumerate}[\bfseries\small\itshape #1] %
      {\leftmargin=0.5em \rightmargin=0em
        \topsep=0em \parskip=0em \parsep=0em
        \listparindent=0em \partopsep=0em \itemsep=0pt
        \itemindent=0em \labelwidth=\leftmargin\labelsep+0.25em}
    }{
    \end{enumerate}\end{minipage}
}

\newenvironment{nolistes}[2]
{\begin{enumerate}[\bfseries\small\itshape
    #1]\setlength{\itemsep}{#2 mm}}{\end{enumerate}}

\newenvironment{liste}[1]
{\begin{enumerate}[\hspace{12pt}\bfseries\small\itshape #1]}{\end{enumerate}} 



%%%%%%%% Pour les programmes de colle %%%%%%%

\newcommand{\cours}{{\small \tt (COURS)}} %
\newcommand{\poly}{{\small \tt (POLY)}} %
\newcommand{\exo}{{\small \tt (EXO)}} %
\newcommand{\culture}{{\small \tt (CULTURE)}} %
\newcommand{\methodo}{{\small \tt (MÉTHODO)}} %
\newcommand{\methodob}{\Boxed{\mbox{\tt MÉTHODO}}} %

%%%%%%%% Pour les TD %%%%%%%
\newtheoremstyle{exostyle} {\topsep} % espace avant
{.6cm} % espace apres
{} % Police utilisee par le style de thm
{} % Indentation (vide = aucune, \parindent = indentation paragraphe)
{\bfseries} % Police du titre de thm
{} % Signe de ponctuation apres le titre du thm
{ } % Espace apres le titre du thm (\newline = linebreak)
{\thmname{#1}\thmnumber{ #2}\thmnote{.
    \normalfont{\textit{#3}}}} % composants du titre du thm : \thmname
                               % = nom du thm, \thmnumber = numéro du
                               % thm, \thmnote = sous-titre du thm
 
\theoremstyle{exostyle}
\newtheorem{exercice}{Exercice}
\newtheorem{exerciceAP}{Exercice avec préparation}
\newtheorem{exerciceSP}{Exercice sans préparation}
\newtheorem*{exoCours}{Exercice}

%%%%%%%% Pour des théorèmes Sans Espaces APRÈS %%%%%%%
\newtheoremstyle{exostyleSE} {\topsep} % espace avant
{} % espace apres
{} % Police utilisee par le style de thm
{} % Indentation (vide = aucune, \parindent = indentation paragraphe)
{\bfseries} % Police du titre de thm
{} % Signe de ponctuation apres le titre du thm
{ } % Espace apres le titre du thm (\newline = linebreak)
{\thmname{#1}\thmnumber{ #2}\thmnote{.
    \normalfont{\textit{#3}}}} % composants du titre du thm : \thmname
                               % = nom du thm, \thmnumber = numéro du
                               % thm, \thmnote = sous-titre du thm
 
\theoremstyle{exostyleSE}
\newtheorem{exerciceSE}{Exercice}
\newtheorem*{exoCoursSE}{Exercice}

% \newcommand{\lims}[2]{\prod\limits_{#1}^{#2}}

\newtheorem{theoremSE}{Théorème}[]
\newtheorem{lemmaSE}{Lemme}[]
\newtheorem{propositionSE}{Proposition}[]
\newtheorem{corollarySE}{Corollaire}[]

% \newenvironment{proofSE}[1][Démonstration]{\begin{trivlist}
% \item[\hskip \labelsep {\bfseries #1}]}{\end{trivlist}}
\newenvironment{definitionSE}[1][Définition]{\begin{trivlist}
  \item[\hskip \labelsep {\bfseries #1}]}{\end{trivlist}}
\newenvironment{exampleSE}[1][Exemple]{\begin{trivlist} 
  \item[\hskip \labelsep {\bfseries #1}]}{\end{trivlist}}
\newenvironment{examplesSE}[1][Exemples]{\begin{trivlist}
\item[\hskip \labelsep {\bfseries #1}]}{\end{trivlist}}
\newenvironment{notationSE}[1][Notation]{\begin{trivlist}
\item[\hskip \labelsep {\bfseries #1}]}{\end{trivlist}}
\newenvironment{proprieteSE}[1][Propriété]{\begin{trivlist}
\item[\hskip \labelsep {\bfseries #1}]}{\end{trivlist}}
\newenvironment{proprietesSE}[1][Propriétés]{\begin{trivlist}
\item[\hskip \labelsep {\bfseries #1}]}{\end{trivlist}}
\newenvironment{remarkSE}[1][Remarque]{\begin{trivlist}
\item[\hskip \labelsep {\bfseries #1}]}{\end{trivlist}}
\newenvironment{applicationSE}[1][Application]{\begin{trivlist}
\item[\hskip \labelsep {\bfseries #1}]}{\end{trivlist}}

%%%%%%%%%%% Obtenir les étoiles sans charger le package MnSymbol
%%%%%%%%%%%
\DeclareFontFamily{U} {MnSymbolC}{}
\DeclareFontShape{U}{MnSymbolC}{m}{n}{
  <-6> MnSymbolC5
  <6-7> MnSymbolC6
  <7-8> MnSymbolC7
  <8-9> MnSymbolC8
  <9-10> MnSymbolC9
  <10-12> MnSymbolC10
  <12-> MnSymbolC12}{}
\DeclareFontShape{U}{MnSymbolC}{b}{n}{
  <-6> MnSymbolC-Bold5
  <6-7> MnSymbolC-Bold6
  <7-8> MnSymbolC-Bold7
  <8-9> MnSymbolC-Bold8
  <9-10> MnSymbolC-Bold9
  <10-12> MnSymbolC-Bold10
  <12-> MnSymbolC-Bold12}{}

\DeclareSymbolFont{MnSyC} {U} {MnSymbolC}{m}{n}

\DeclareMathSymbol{\filledlargestar}{\mathrel}{MnSyC}{205}
\DeclareMathSymbol{\largestar}{\mathrel}{MnSyC}{131}

\newcommand{\facile}{\rm{(}$\scriptstyle\largestar$\rm{)}} %
\newcommand{\moyen}{\rm{(}$\scriptstyle\filledlargestar$\rm{)}} %
\newcommand{\dur}{\rm{(}$\scriptstyle\filledlargestar\filledlargestar$\rm{)}} %
\newcommand{\costaud}{\rm{(}$\scriptstyle\filledlargestar\filledlargestar\filledlargestar$\rm{)}}

%%%%%%%%%%%%%%%%%%%%%%%%%

%%%%%%%%%%%%%%%%%%%%%%%%%
%%%%%%%% Fin de la partie TD

%%%%%%%%%%%%%%%%
%%%%%%%%%%%%%%%%
\makeatletter %
\newenvironment{myitemize}{%
  \setlength{\topsep}{0pt} %
  \setlength{\partopsep}{0pt} %
  \renewcommand*{\@listi}{\leftmargin\leftmargini \parsep\z@
    \topsep\z@ \itemsep\z@} \let\@listI\@listi %
  \itemize %
}{\enditemize} %
\makeatother
%%%%%%%%%%%%%%%%
%%%%%%%%%%%%%%%%

%% Commentaires dans la correction du livre

\newcommand{\Com}[1]{
% Define box and box title style
\tikzstyle{mybox} = [draw=black!50,
very thick,
    rectangle, rounded corners, inner sep=10pt, inner ysep=8pt]
\tikzstyle{fancytitle} =[rounded corners, fill=black!80, text=white]
\tikzstyle{fancylogo} =[ text=white]
\begin{center}

\begin{tikzpicture}
\node [mybox] (box){%

    \begin{minipage}{0.90\linewidth}
\vspace{6pt}  #1
    \end{minipage}
};
\node[fancytitle, right=10pt] at (box.north west) 
{\bfseries\normalsize{Commentaire}};

\end{tikzpicture}%

\end{center}
%
}

% \NewEnviron{remark}{%
%   % Define box and box title style
%   \tikzstyle{mybox} = [draw=black!50, very thick, rectangle, rounded
%   corners, inner sep=10pt, inner ysep=8pt] %
%   \tikzstyle{fancytitle} =[rounded corners, fill=black!80,
%   text=white] %
%   \tikzstyle{fancylogo} =[ text=white]
%   \begin{center}
%     \begin{tikzpicture}
%       \node [mybox] (box){%
%         \begin{minipage}{0.90\linewidth}
%           \vspace{6pt} \BODY
%         \end{minipage}
%       }; %
%       \node[fancytitle, right=10pt] at (box.north west) %
%       {\bfseries\normalsize{Commentaire}}; %
%     \end{tikzpicture}%
%   \end{center}
% }
% \makeatother


%%%
% Remarques comme dans le livre
%%%

\NewEnviron{remark}{%
  % Define box and box title style
  \tikzstyle{mybox} = [draw=black!50, very thick, rectangle, rounded
  corners, inner sep=10pt, inner ysep=8pt] %
  \tikzstyle{fancytitle} = [rounded corners , fill=black!80,
  text=white] %
  \tikzstyle{fancylogo} =[ text=white]
  \begin{center}
    \begin{tikzpicture}
      \node [mybox] (box){%
        \begin{minipage}{0.90\linewidth}
          \vspace{6pt} \BODY
        \end{minipage}
      }; %
      \node[fancytitle, right=10pt] at (box.north west) %
      {\bfseries\normalsize{Commentaire}}; %
    \end{tikzpicture}%
  \end{center}
}

\NewEnviron{remarkST}{%
  % Define box and box title style
  \tikzstyle{mybox} = [draw=black!50, very thick, rectangle, rounded
  corners, inner sep=10pt, inner ysep=8pt] %
  \tikzstyle{fancytitle} = [rounded corners , fill=black!80,
  text=white] %
  \tikzstyle{fancylogo} =[ text=white]
  \begin{center}
    \begin{tikzpicture}
      \node [mybox] (box){%
        \begin{minipage}{0.90\linewidth}
          \vspace{6pt} \BODY
        \end{minipage}
      }; %
      % \node[fancytitle, right=10pt] at (box.north west) %
%       {\bfseries\normalsize{Commentaire}}; %
    \end{tikzpicture}%
  \end{center}
}

\NewEnviron{remarkL}[1]{%
  % Define box and box title style
  \tikzstyle{mybox} = [draw=black!50, very thick, rectangle, rounded
  corners, inner sep=10pt, inner ysep=8pt] %
  \tikzstyle{fancytitle} =[rounded corners, fill=black!80,
  text=white] %
  \tikzstyle{fancylogo} =[ text=white]
  \begin{center}
    \begin{tikzpicture}
      \node [mybox] (box){%
        \begin{minipage}{#1\linewidth}
          \vspace{6pt} \BODY
        \end{minipage}
      }; %
      \node[fancytitle, right=10pt] at (box.north west) %
      {\bfseries\normalsize{Commentaire}}; %
    \end{tikzpicture}%
  \end{center}
}

\NewEnviron{remarkSTL}[1]{%
  % Define box and box title style
  \tikzstyle{mybox} = [draw=black!50, very thick, rectangle, rounded
  corners, inner sep=10pt, inner ysep=8pt] %
  \tikzstyle{fancytitle} =[rounded corners, fill=black!80,
  text=white] %
  \tikzstyle{fancylogo} =[ text=white]
  \begin{center}
    \begin{tikzpicture}
      \node [mybox] (box){%
        \begin{minipage}{#1\linewidth}
          \vspace{6pt} \BODY
        \end{minipage}
      }; %
%       \node[fancytitle, right=10pt] at (box.north west) %
%       {\bfseries\normalsize{Commentaire}}; %
    \end{tikzpicture}%
  \end{center}
}

\NewEnviron{titre} %
{ %
  ~\\[-1.8cm]
  \begin{center}
    \bf \LARGE \BODY
  \end{center}
  ~\\[-.6cm]
  \hrule %
  \vspace*{.2cm}
} %

\NewEnviron{titreL}[2] %
{ %
  ~\\[-#1cm]
  \begin{center}
    \bf \LARGE \BODY
  \end{center}
  ~\\[-#2cm]
  \hrule %
  \vspace*{.2cm}
} %




% \renewcommand{\thesection}{\Roman{section}.\hspace{-.3cm}}
% \renewcommand{\thesubsection}{\Alph{subsection}.\hspace{-.2cm}}

\pagestyle{fancy} %
\lhead{ECE2 \hfill Mathématiques \\} %
\chead{\hrule} %
\rhead{} %
\lfoot{} %
\cfoot{} %
\rfoot{\thepage} %

\renewcommand{\headrulewidth}{0pt}% : Trace un trait de séparation
                                    % de largeur 0,4 point. Mettre 0pt
                                    % pour supprimer le trait.

\renewcommand{\footrulewidth}{0.4pt}% : Trace un trait de séparation
                                    % de largeur 0,4 point. Mettre 0pt
                                    % pour supprimer le trait.

\setlength{\headheight}{14pt}

\title{\bf \vspace{-1.6cm} EML 2019} %
\author{} %
\date{} %
\begin{document}

\maketitle %
\vspace{-1.2cm}\hrule %
\thispagestyle{fancy}

\vspace*{-.2cm}

%%DEBUT

\section*{EXERCICE 1}

\noindent %
Dans ce problème, toutes les variables aléatoires sont supposées
définies sur un même espace probabilisé noté $(\Omega, \A, \Prob)$. %

\subsection*{PARTIE A : Des résultats préliminaires} %

\noindent %
Soient $U$ et $V$ deux variables aléatoires à densité indépendantes,
de densités respectives $f_U$ et $f_V$ et de fonctions de répartition
respectives $F_U$ et $F_V$.\\
On suppose que les fonctions $f_U$ et $f_V$ sont nulles sur $]-\infty,
0[$ et continues sur $[0,+\infty[$.
\begin{noliste}{1.}
  \setlength{\itemsep}{4mm}
\item
  \begin{noliste}{a)}
    \setlength{\itemsep}{2mm}
  \item Justifier : $\forall t \in [0,+\infty[$, $0 \ \leq \ F_U(t) \,
    f_V(t) \ \leq \ f_V(t)$.
    \begin{proof}~\\
      Soit $t \in [0,+\infty[$.\\
      Comme la fonction $F_U$ est une fonction de répartition :
      \[
        0 \ \leq \ F_U(t) \ \leq \ 1
      \]
      De plus, la fonction $f_V$ est une densité de probabilité, donc
      : $f_V(t) \geq 0$. Ainsi :
      \[
        0 \times f_V(t) \ \leq \ F_U(t) \times f_V(t)
        \ \leq \ 1 \times f_V(t)
      \]
      \conc{On obtient : $\forall t \in [0,+\infty[$, $0 \ \leq \ F_U(t)
        \, f_V(t) \ \leq \ f_V(t)$.}~\\[-1cm]
    \end{proof}
    
  \item En déduire que l'intégrale $\dint{0}{+\infty} F_U(t) \, f_V(t)
    \dt$ converge.
    \begin{proof}~
      \begin{noliste}{$\sbullet$}
      \item La fonction $t \mapsto F_U(t) \, f_V(t)$ est continue sur
        $[0,+ \infty[$ en tant que produit de fonctions continues sur
        cet intervalle. En effet :
        \begin{noliste}{$\stimes$}
        \item la fonction $F_U$ est continue sur $\R$ (et donc sur
          $[0,+\infty[$) en tant que fonction de répartition d'une \var
          à densité.
          
        \item la fonction $f_V$ est continue sur $[0,+\infty[$ d'après
          l'énoncé.
        \end{noliste}
        
      \item De plus :
        \begin{noliste}{$\stimes$}
        \item $\forall t \in [0,+\infty[$, $0 \ \leq \ F_U(t) \,
          f_V(t) \ \leq \ f_V(t)$
          
        \item l'intégrale $\dint{0}{+\infty} f_V(t) \dt$ est
          convergente. En effet, comme $f_V$ est une densité de
          probabilité, l'intégrale $\dint{-\infty}{+\infty} f_V(t)
          \dt$ est convergente (et vaut $1$). Ainsi, en
          particulier, l'intégrale $\dint{0}{+\infty} f_V(t) \dt$ converge.
        \end{noliste}
        \conc{Par critère de comparaison d'intégrales généralisées de
          fonctions continues positives,\\
          $\dint{0}{+\infty} F_U(t) \, f_V(t) \dt$ est convergente.}~\\[-1.4cm]
      \end{noliste}
    \end{proof}
  \end{noliste}


  \newpage


  \noindent %
  On admet le résultat suivant :
  \[
    \dint{0}{+\infty} F_U(t) \, f_V(t) \dt \ = \ \Prob(\Ev{U \leq V})
  \]
  
\item En déduire : $\Prob(\Ev{U > V}) \ = \ \dint{0}{+\infty} \big(1 -
  F_U(t)\big) \, f_V(t) \dt$.
  \begin{proof}~
    \begin{noliste}{$\sbullet$}
    \item Tout d'abord, pour tout $t \in [0,+\infty[$ :
      \[
        \big(1- F_U(t)\big) \, f_V(t) \ = \ f_V(t) - F_U(t) \, f_V(t)
      \]
      Or les intégrales $\dint{0}{+\infty} f_V(t) \dt$ et
      $\dint{0}{+\infty} F_U(t) \, f_V(t) \dt$ sont convergentes
      d'après la question précédente.
      On en déduit que l'intégrale $\dint{0}{+\infty} \big(1-
      F_U(t)\big) \, f_V(t) \dt$ est convergente.
      
    \item De plus :
      \[
        \begin{array}{rcl@{\quad}>{\it}R{4cm}}
          \dint{0}{+\infty} \big(1- F_U(t)\big) \, f_V(t) \dt
          & = & \dint{0}{+\infty} f_V(t) - F_U(t) \, f_V(t) \dt
          \\[.6cm]
          & = & \dint{0}{+\infty} f_V(t) \dt - \dint{0}{+\infty}
                F_U(t) \, f_V(t) \dt
          \\[.6cm]
          & = & \dint{0}{+\infty} f_V(t) \dt - \Prob(\Ev{U \leq V})
          & (d'après le résultat admis de l'énoncé)
          \nl
          \nl[-.2cm]
          & = & \dint{-\infty}{+\infty} f_V(t) \dt - \Prob(\Ev{U \leq
                V})
          & (car $f_V$ est nulle en dehors de $[0,+\infty[$)
          \nl
          \nl[-.2cm]
          & = & 1 - \Prob(\Ev{U \leq V})
          & (car $f_V$ est une densité de probabilité)
          \nl
          \nl[-.2cm]
          & = & \Prob(\Ev{ U > V}) 
        \end{array}
      \]
    \end{noliste}
    \conc{$\Prob(\Ev{U > V}) \ = \ \dint{0}{+\infty} \big(1 -
      F_U(t)\big) \, f_V(t) \dt$}~\\[-1cm]
  \end{proof}
  
\item {\bf Exemple} : Soient $\lambda, \mu \in \R^2$. On suppose dans
  cette question que $U$ suit la loi exponentielle de paramètre
  $\lambda$ et que $V$ suit la loi exponentielle de paramètre $\mu$.
  \begin{noliste}{a)}
    \setlength{\itemsep}{2mm}
  \item Rappeler, pour tout $t$ de $\R_+$, une expression de $F_U(t)$
    et de $f_V(t)$.
    \begin{proof}~
      \conc{Comme $U \suit \Exp{\lambda}$ et $V \suit \Exp{\mu}$, pour
        tout $t \in [0,+\infty[$ :\\[.2cm]
        $F_U(t) = 1 - \ee^{-\lambda \, t}$ \quad et \quad $f_V(t) =
        \mu \, \ee^{-\mu \, t}$.}~\\[-1cm]
    \end{proof}


  \newpage

    
  \item En déduire : $\Prob(\Ev{U > V}) \ = \ \dfrac{\mu}{\lambda +
      \mu}$.
    \begin{proof}~
      \begin{noliste}{$\sbullet$}
      \item Tout d'abord, on précise que les \var $U$ et $V$ sont des
        \var à densité. On est donc bien dans le cadre d'application de la question
        \itbf{2.}
        
      \item On obtient alors :
        \[
          \begin{array}{rcl@{\quad}>{\it}R{4cm}}
            \Prob(\Ev{U > V})
            & = & \dint{0}{+\infty} \big(1 - F_U(t)\big) \, f_V(t) \dt
            \\[.6cm]
            & = & \dint{0}{+\infty} \left(\bcancel{1}-
                  \left(\bcancel{1} - \ee^{-\lambda \,
                  t}\right)\right) \, \mu \, \ee^{-\mu \, t} \dt
            & (d'après la question précédente)
            \nl
            \nl[-.2cm]
            & = & \mu \ \dint{0}{+\infty} \ee^{-\lambda \, t} \
                  \ee^{-\mu \, t} \dt
            \\[.6cm]
            & = & \mu \ \dint{0}{+\infty} \ee^{-(\lambda + \mu) t} \dt
          \end{array}
        \]
        
      \item De plus, soit $A \in [0,+\infty[$ :
        \[
          \dint{0}{A} \ee^{-(\lambda + \mu) t} \dt \ = \ \Prim{-
            \dfrac{1}{\lambda + \mu} \ \ee^{-(\lambda + \mu) t}}{0}{A}
          \ = \ - \dfrac{1}{\lambda + \mu} \left(\ee^{-(\lambda + \mu)
              A} - 1\right) \ = \ -\dfrac{1}{\lambda + \mu} \
          \ee^{-(\lambda + \mu) A} + \dfrac{1}{\lambda + \mu}
        \]
        Or, comme $\lambda >0$ et $\mu >0$ : $\dlim{A \to +\infty}
        \ee^{-(\lambda + \mu) A} = 0$. Ainsi :
        \[
          \dint{0}{+\infty} \ee^{-(\lambda + \mu) t} \dt =
          \dfrac{1}{\lambda + \mu}
        \]
        \conc{On en déduit : $\Prob(\Ev{U > V}) \ = \ \mu \
          \dint{0}{+\infty} \ee^{-(\lambda + \mu) t} \dt \ = \
          \dfrac{\mu}{\lambda + \mu}$.}~\\[-1.4cm]
      \end{noliste}
    \end{proof}
  \end{noliste}
\end{noliste}


\subsection*{PARTIE B : Une application} %

\noindent %
Soit $\lambda \in \R_+^*$. On considère une suite $(T_n)_{n \in \N}$
de variables aléatoires indépendantes, suivant toutes la loi
exponentielle de paramètre $\lambda$.\\
On définit ensuite la variable aléatoire $N$ égale au plus petit
entier $k$ de $\N^*$ tel que $T_k \leq T_0$ si un tel entier existe et
égale à $0$ sinon.
\begin{noliste}{1.}
  \setlength{\itemsep}{4mm}
  \setcounter{enumi}{3}
\item Soit $n \in \N^*$. On définit la variable aléatoire $M_n$ par :
  $M_n  = \min(T_1, \ldots, T_n)$.
  \begin{noliste}{a)}
    \setlength{\itemsep}{2mm}
  \item Calculer, pour tout $t$ de $\R_+$, $\Prob(\Ev{M_n > t})$.
    \begin{proof}~\\
      Soit $t \in [0,+\infty[$.
      \begin{noliste}{$\sbullet$}
      \item Tout d'abord :
        \[
          \begin{array}{rcl}
            \Ev{M_n >t}
            & = & \Ev{\min(T_1, \ldots, T_n) > t}
            \\[.2cm]
            & = & \dcap{i=1}{n} \Ev{T_i > t}
          \end{array}
        \]


        \newpage
        
        
      \item On en déduit :
        \[
          \begin{array}{rcl@{\quad}>{\it}R{5cm}}
            \Prob(\Ev{M_n > t})
            & = & \Prob\left( \dcap{i=1}{n} \Ev{T_i >t}\right)
            \\[.4cm]
            & = & \Prod{i=1}{n} \Prob(\Ev{T_i > t})
            & (car les \var $T_1$, $\ldots$, $T_n$ sont indépendantes)
            \nl
            \nl[-.2cm]
            & = & \Prod{i=1}{n} \left(1 - F_{T_i}(t)\right)
            \\[.4cm]
            & = & \Prod{i=1}{n} \left(\bcancel{1}- \left(\bcancel{1} -
                  \ee^{-\lambda \, t}\right) \right)
            & (car, pour tout $i \in \llb 1,n \rrb$, $T_i \suit
              \Exp{\lambda}$ et $t \geq 0$)
            \nl
            \nl[-.2cm]
            & = & \Prod{i=1}{n} \ee^{-\lambda \, t}
            \\[.6cm]
            & = & \left(\ee^{-\lambda \, t}\right)^n
          \end{array}
        \]
        \conc{$\forall t \in [0,+\infty[$, $\Prob(\Ev{M_n > t}) =
          \ee^{-n \lambda \, t}$}
      \end{noliste}
      \begin{remark}
        \begin{noliste}{$\sbullet$}
        \item On fait ici l'étude de la loi d'un minimum de
          \var. Cette étude est classique et commence toujours par
          l'énoncé de l'égalité entre événements, pour tout $t \in \R$ :
          \[
            \Ev{\min(T_1, \ldots, T_n) > t} \ = \ \dcap{i=1}{n}
            \Ev{T_i > t}
          \]
          
        \item On rappelle que dans le cas de l'étude d'un maximum de \var, on
          commence par une égalité similaire, pour tout $t \in \R$ :
          \[
            \Ev{\max(T_1, \ldots, T_n) \leq t} \ = \ \dcap{i=1}{n}
            \Ev{T_i \leq t}
          \]
        \end{noliste}
      \end{remark}~\\[-1.4cm]
    \end{proof}
    
  \item En déduire la fonction de répartition de $M_n$ sur $\R$.\\
    Reconnaître la loi de $M_n$ et préciser son (ses) paramètre(s).
    \begin{proof}~
      \begin{noliste}{$\sbullet$}
      \item Tout d'abord, comme pour tout $i \in \llb 1,n \rrb$, $T_i
        \suit \Exp{\lambda}$, on a : $T_i(\Omega) = [0,+\infty[$.
        \conc{On en déduit : $M_n(\Omega) \subset [0,+\infty[$.}
        
      \item Soit $t \in \R$. Deux cas se présentent :
        \begin{noliste}{$\stimes$}
        \item \dashuline{si $t \in \ ]-\infty, 0[$}, alors : $\Ev{M_n
            \leq t} = \emptyset$ car $M_n(\Omega) \subset
          [0,+\infty[$. Donc :
          \[
            F_{M_n}(t) \ = \ \Prob(\Ev{M_n \leq t}) \ = \
            \Prob(\emptyset) \ = \ 0
          \]
          
        \item \dashuline{si $t \in [0,+\infty[$}, alors :
          \[
            \begin{array}{rcl@{\quad}>{\it}R{4cm}}
              F_{M_n}(t)
              & = & \Prob(\Ev{M_n \leq t})
                    \ = \ 1 - \Prob(\Ev{M_n > t})
              \\[.2cm]
              & = & 1 - \ee^{-n\lambda \, t}
              & (d'après la question précédente)
            \end{array}
          \]
        \end{noliste}
        \conc{Finalement : $F_{M_n} : t \mapsto \left\{
            \begin{array}{cR{2.5cm}}
              0 & si $t \in \ ]-\infty, 0[$
              \nl
              \nl[-.2cm]
              1 - \ee^{-n \lambda \, t} & si $t \in [0,+\infty[$    
            \end{array}
          \right.$}


        \newpage
        
        
      \item On reconnaît la fonction de répartition d'une \var de loi
        $\Exp{n \lambda}$.\\
        Or la fonction de répartition caractérise la loi.
        \conc{On en déduit : $M_n \suit \Exp{n \lambda}$.}~\\[-1.4cm]
      \end{noliste}
    \end{proof}
  \end{noliste}
  
\item
  \begin{noliste}{a)}
    \setlength{\itemsep}{2mm}
  \item Montrer : $\Prob(\Ev{N = 1}) \ = \ \Prob(\Ev{T_1 \leq T_0}) \
    = \ \dfrac{1}{2}$.
    \begin{proof}~
      \begin{noliste}{$\sbullet$}
      \item Tout d'abord, notons que l'événement $\Ev{N = 1}$ est
        réalisé si et seulement si le plus petit entier $k$ de $\N^*$
        tel que $T_k \leq T_0$ est $1$, autrement dit, si et seulement
        si l'événement $\Ev{T_1 \leq T_0}$ est réalisé.
        \conc{On en déduit : $\Ev{N=1} = \Ev{T_1 \leq T_0}$. D'où :\\[.2cm]
          $\Prob(\Ev{N=1}) \ = \ \Prob(\Ev{T_1 \leq T_0})$.}
        
      \item De plus :
        \[
          \begin{array}{rcl@{\quad}>{\it}R{6.5cm}}
            \Prob(\Ev{T_1 \leq T_0})
            & = & 1 - \Prob(\Ev{T_1 > T_0})
            \\[.2cm]
            & = & 1 - \dfrac{\lambda}{\lambda + \lambda}
            & (d'après \itbf{3.b)} car $T_1$ et $T_0$ sont
              indépendantes et suivent la loi $\Exp{\lambda}$)
            \nl
            \nl[-.2cm]  
            & = & 1 - \dfrac{\bcancel{\lambda}}{2 \,
                  \bcancel{\lambda}}
            \\[.4cm]
            & = & 1 - \dfrac{1}{2}
          \end{array}
        \]
        \conc{D'où : $\Prob(\Ev{T_1 \leq T_0}) = \dfrac{1}{2}$.}
      \end{noliste}
      \begin{remark}
        On utilise ici le résultat d'une question précédente. On
        n'omettra donc pas de préciser que toutes les hypothèses
        nécessaires à son application sont vérifiées.
      \end{remark}~\\[-1.4cm]
    \end{proof}
    
  \item Justifier : $\forall n \in \N^*$, $\Ev{N > n} \cup \Ev{N = 0}
    \ = \ \Ev{M_n >  T_0}$.\\
    En déduire, pour tout $n$ de $\N^* $, une expression de
    $\Prob(\Ev{N > n} \cup \Ev{N = 0})$ en fonction de $n$.
  \end{noliste}
    \begin{proof}~
      \begin{noliste}{$\sbullet$}
      \item Soit $n \in \N^*$.\\ %
        Tout d'abord :
        \[
          \Ev{M_n > T_0} \ = \ \dcap{i=1}{n} \Ev{T_i > T_0}
        \]
        Démontrons donc : $\dcap{i=1}{n} \Ev{T_i > T_0} = \Ev{N > n}
        \cup \Ev{N = 0}$.\\
        Soit $\omega \in \Omega$.
        \begin{liste}{}
        \item[$(\subset)$] Supposons $\omega \in \dcap{i=1}{n} \Ev{T_i > T_0}$,
          \ie, pour tout $i \in \llb 1,n \rrb$, $T_i(\omega) >
          T_0(\omega)$.\\ %
          Deux cas se présentent alors :
          \begin{noliste}{-}
          \item soit il existe un entier $k > n$ tel que $T_k(\omega)
            \leq T_0(\omega)$.\\
            Alors : $N(\omega) > n$, c'est-à-dire
            $\omega \in \Ev{N > n}$.
            
          \item soit, pour tout $k \in \N^*$, $T_k(\omega) >
            T_0(\omega)$. \\
            Alors : $N(\omega) = 0$, c'est-à-dire
            $\omega \in \Ev{N = 0}$.
          \end{noliste}
          Finalement : $\omega \in \Ev{N > n}$ $\OU$ $\omega \in \Ev{N =
            0}$. Donc : $\omega \in \Ev{N > n} \cup \Ev{N = 0}$.
          \conc{$\dcap{i=1}{n} \Ev{T_i > T_0} \ \subset \ \Ev{N > n}
            \cup \Ev{N = 0}$}
          

          \newpage
          
          
        \item[$(\supset)$] Supposons $\omega \in \Ev{N > n} \cup \Ev{N
            = 0}$, \ie $N(\omega) > n$ ou $N(\omega) = 0$.\\
          Alors, dans les deux cas, par définition de la \var $N$,
          pour tout $i \in \llb 1,n \rrb$ : $T_i(\omega) >
          T_0(\omega)$. Autrement dit : $\omega \in \dcap{i=1}{n}
          \Ev{T_i > T_0}$.
          \conc{$\dcap{i=1}{n} \Ev{T_i > T_o} \ \supset \ \Ev{N > n}
            \cup \Ev{N = 0}$} 
        \end{liste}
        \conc{Finalement : $\Ev{M_n > T_0} \ = \ \dcap{i=1}{n} \Ev{T_i > T_0}
          \ = \ \Ev{N > n} \cup \Ev{N = 0}$.}
        
      \item Les \var $M_n$ et $T_0$ sont :
        \begin{noliste}{$\stimes$}
        \item des \var à densité,
          
        \item indépendantes par lemme des coalitions.
        \end{noliste}
        On peut donc appliquer le résultat de la question \itbf{2.} :
        \[
          \begin{array}{rcl@{\quad}>{\it}R{5cm}}
            \Prob(\Ev{M_n > T_0})
            & = & \dint{0}{+\infty} \big(1- F_{M_n}(t)\big) \, f_{T_0}(t)
                  \dt
            \\[.6cm]
            & = & \dint{0}{+\infty} \left(\bcancel{1} -
                  \left(\bcancel{1} - \ee^{-n \lambda \, t}\right)\right) \,
                  \lambda \, \ee^{-\lambda \, t} \dt
            & (d'après la question \itbf{4.b)})
            \nl
            \nl[-.2cm]
            & = & \lambda \, \dint{0}{+\infty} \ee^{-n \lambda \, t}
                  \, \ee^{-\lambda \, t} \dt
            \\[.6cm]
            & = & \lambda \, \dint{0}{+\infty} \ee^{-(n+1)\lambda \,
                  t} \dt
          \end{array}
        \]
        
      \item De plus, soit $A \in [0,+\infty[$ :
        \[
          \begin{array}{rcl}
            \dint{0}{A} \ee^{-(n+1) \lambda \, t} \dt
            & = & \Prim{- \dfrac{1}{(n+1) \lambda} \ \ee^{-(n+1)
                  \lambda  \, t}}{0}{A}
            \\[.6cm]
            & = & -\dfrac{1}{(n+1) \lambda} \left(\ee^{-(n+1) \lambda
                  \, A}  - 1\right)
            \\[.6cm]
            & = & -\dfrac{1}{(n+1) \lambda} \ \ee^{-(n+1) \lambda \, A}
                  + \dfrac{1}{(n+1) \lambda}
          \end{array}
        \]
        Or, comme $\lambda >0$ et $n>0$ : $\dlim{A \to + \infty}
        \ee^{-(n+1) \lambda \, A} = 0$. Ainsi :
        \[
          \Prob(\Ev{M_n > T_0}) \ = \ \bcancel{\lambda} \
          \dfrac{1}{(n+1) \bcancel{\lambda}} \ = \ \dfrac{1}{n+1}
        \]
        \conc{Finalement, pour tout $n \in \N^*$ :$ \Prob(\Ev{N > n}
          \cup \Ev{N = 0}) \ = \ \Prob(\Ev{M_n > T_0}) \ = \
          \dfrac{1}{n+1}$.}
      \end{noliste}
      \begin{remark}
        L'énoncé original proposait de démontrer l'égalité : $\Ev{N >
          n} = \Ev{M_n > T_0}$. Cependant : $\Ev{M_n > T_0}
        \not\subset \Ev{N > n}$. En effet, considérons
        l'événement $\dcap{i=1}{+\infty} \Ev{T_i > T_0}$.
        \begin{noliste}{$\stimes$}
        \item Tout d'abord :
          \[
            \dcap{i=1}{+\infty} \Ev{T_i > T_0} \ \subset \
            \dcap{i=1}{n} \Ev{T_i > T_0} \ = \ \Ev{M_n > T_0}
          \]
          
        \item Soit $\omega \in \Omega$.\\
          On sait que $\omega \in \dcap{i=1}{+\infty} \Ev{T_i
            > T_0}$ si et seulement si, pour tout $i \in \N^*$, $T_i(\omega) >
          T_0(\omega)$.\\
          Par définition de $N$, cela équivaut à
          $N(\omega) = 0$, \ie $\omega \in \Ev{N = 0}$. Ainsi :
          \[
            \dcap{i=1}{+\infty} \Ev{T_i > T_0} \ = \ \Ev{N = 0}
          \]
          Or : $\Ev{N = 0} \not\subset \Ev{N > n}$. Donc :
          $\dcap{i=1}{+\infty} \Ev{T_i > T_0} \not\subset \Ev{N > n}$.
          
        \item Si $\Ev{M_n > T_0} \subset \Ev{N > n}$, on obtiendrait
          par transitivité :
          \[
            \dcap{i=1}{+\infty} \Ev{T_i > T_0} \ \subset \ \Ev{M_n >
              T_0} \ \subset \ \Ev{N >n}
          \]
          Comme ce n'est pas le cas, on a bien : $\Ev{M_n > T_0}
          \not\subset \Ev{N > n}$.
        \end{noliste}
      \end{remark}~\\[-1.4cm]
    \end{proof}

  \begin{noliste}{a)}
    \setcounter{enumii}{2}  
  \item Montrer alors : $\forall n \in \N \, \setminus \, \{0,1\}$,
    $\Prob(\Ev{ N = n}) = \dfrac{1}{n(n+1)}$.
    \begin{proof}~\\
      Soit $n \in \N \, \setminus \, \{0,1\}$.
      \begin{noliste}{$\sbullet$}
      \item Notons tout d'abord l'égalité suivante :
        \[
          \begin{array}{rcl@{\quad}>{\it}R{4cm}}
            \Ev{N > n-1}
            & = & \Ev{N \geq n}
            & (car $N$ est à valeurs entières)
            \nl
            \nl[-.2cm]
            & = & \Ev{N  = n} \cup \Ev{ N > n}
          \end{array}
        \]
        On en déduit :
        \[
          \begin{array}{rcl@{\quad}>{\it}R{4cm}}
            \Ev{N > n-1} \cup \Ev{N = 0}
            & = & \big(\Ev{N = n} \cup \Ev{N > n}\big) \cup \Ev{N = 0}
            \\[.4cm]
            & = & \Ev{N = n} \cup \big(\Ev{N > n} \cup \Ev{N = 0}\big)
            & (par associativité de $\cup$)
          \end{array}
        \]
        
      \item De plus, les événements $\Ev{N = n}$ et $\Ev{N > n} \cup
        \Ev{N = 0}$ sont incompatibles. Donc :
        \[
          \Prob(\Ev{N > n-1} \cup \Ev{N = 0}) \ = \ \Prob(\Ev{N = n})
          + \Prob(\Ev{N > n} \cup \Ev{N = 0})
        \]


        \newpage
        
        
      \item On en déduit :
        \[
          \begin{array}{rcl@{\quad}>{\it}R{4.5cm}}
            \Prob(\Ev{N = n})
            & = & \Prob(\Ev{N > n-1} \cup \Ev{N = 0}) - \Prob(\Ev{N >
                  n} \cup \Ev{N = 0})
            \\[.2cm]
            & = & \dfrac{1}{(n-\bcancel{1})+\bcancel{1}} -
                  \dfrac{1}{n+1}
            & (d'après \itbf{5.b)} car $n\in \N^*$ et $n-1 \in \N^*$)
            \nl
            \nl[-.2cm]
            & = & \dfrac{1}{n} - \dfrac{1}{n+1}
            \\[.6cm]
            & = & \dfrac{(\bcancel{n} +1) - \bcancel{n}}{n(n+1)}
          \end{array}
        \]
      \end{noliste}
      \conc{Finalement : $\forall n \in \N \, \setminus \, \{0,1\}$,
        $\Prob(\Ev{N=n}) = \dfrac{1}{n(n+1)}$.}~\\[-1cm]
    \end{proof}
    
  \item En déduire la valeur de $\Prob(\Ev{N = 0})$.
    \begin{proof}~
      \begin{noliste}{$\sbullet$}
      \item La famille $(\Ev{N = n})_{n \in \N}$ forme un système
        complet d'événements. Ainsi :
        \[
          \begin{array}{rcl@{\quad}>{\it}R{5cm}}
            \Prob(\Ev{N = 0})
            & = & 1 - \Sum{n=1}{+\infty} \Prob(\Ev{N = n})
            \\[.4cm]
            & = & 1 - \left(\Prob(\Ev{N = 1}) + \Sum{n=2}{+\infty}
                  \Prob(\Ev{N = n})\right)
            \\[.4cm]
            & = & 1 - \dfrac{1}{2} - \Sum{n=2}{+\infty}
                  \dfrac{1}{n(n+1)}
            & (d'après \itbf{5.a)} et \itbf{5.c)})
            \nl
            \nl[-.2cm]
            & = & \dfrac{1}{2} - \Sum{n=2}{+\infty} \dfrac{1}{n(n+1)}
          \end{array}
        \]
        
      \item Notons d'abord que, pour tout $n \in \N^*$ :
        \[
          \dfrac{1}{n} - \dfrac{1}{n+1} \ = \ \dfrac{(\bcancel{n}+1) -
            \bcancel{n}}{n(n+1)} \ = \ \dfrac{1}{n(n+1)}
        \]
        Ainsi, soit $N \geq 2$ :
        \[
          \begin{array}{rcl@{\quad}>{\it}R{5cm}}
            \Sum{n=2}{N} \dfrac{1}{n(n+1)}
            & = & \Sum{n=2}{N} \left(\dfrac{1}{n} -
                  \dfrac{1}{n+1}\right)
            \\[.6cm]
            & = & \dfrac{1}{2} - \dfrac{1}{N+1}
            & (par télescopage)
            \nl
            \nl[-.2cm]
            & \tendd{N}{+\infty} & \dfrac{1}{2}
          \end{array}
        \]
        On en déduit que la série $\Sum{n \geq 2}{} \dfrac{1}{n(n+1)}$
        est convergente et que sa somme vaut $\dfrac{1}{2}$.
      \end{noliste}
      \conc{Ainsi : $\Prob(\Ev{N = 0}) \ = \ \dfrac{1}{2} -
        \dfrac{1}{2} \ = \ 0$.}


      \newpage

      
      \begin{remark}
        Prendre l'initiative de la décomposition :
        \[
          \dfrac{1}{n(n+1)} \ = \ \dfrac{1}{n} - \dfrac{1}{n+1}
        \]
        pour faire apparaître la somme télescopique, peut paraître
        ardu. On s'efforcera donc de garder en mémoire cet exemple
        classique de somme télescopique.
      \end{remark}~\\[-1.4cm]
    \end{proof}
  \end{noliste}
  
\item La variable aléatoire $N$ admet-elle une espérance ?
  \begin{proof}~
    \begin{noliste}{$\sbullet$}
    \item La \var $N$ admet une espérance si et seulement si la série
      $\Sum{n \geq 0}{} n \, \Prob(\Ev{N = n})$ est absolument
      convergente. Cette série étant à termes positifs, cela revient à
      démontrer qu'elle est convergente.
      
    \item Soit $N \geq 2$.
      \[
        \begin{array}{rcl@{\quad}>{\it}R{5cm}}
          \Sum{n=0}{N} n \, \Prob(\Ev{N = n})
          & = & \bcancel{0 \times \Prob(\Ev{N = 0})} + 1 \times
                \Prob(\Ev{N = 1}) + \Sum{n=2}{N} n \, \Prob(\Ev{N =n})
          \\[.4cm]
          & = & \dfrac{1}{2} + \Sum{n=2}{N} \bcancel{n} \
                \dfrac{1}{\bcancel{n}(n+1)}
          & (d'après \itbf{5.a)} et \itbf{5.c)})
          \nl
          \nl[-.2cm]
          & = & \dfrac{1}{2} + \Sum{n=3}{N+1} \dfrac{1}{n}
          & (par décalage d'indice)      
        \end{array}
      \]
      
    \item Or la série $\Sum{n \geq 3}{} \dfrac{1}{n}$ est une série de
      Riemann d'exposant $1$ ($1 \ngtr 2$). Elle est donc
      divergente.\\ %
      Ainsi, la série $\Sum{n \geq 0}{} n \, \Prob(\Ev{N = n})$ est
      divergente.
      \conc{On en déduit que la \var $N$ n'admet pas d'espérance.}~\\[-1.4cm]
    \end{noliste}
  \end{proof}
\end{noliste}


\newpage %


\section*{Exercice 2}

\noindent
On rappelle que deux matrices $A$ et $B$ de $\M{3}$ sont dites
semblables lorsqu'il existe $P$ de $\M{3}$ inversible telle que : 
\[
B \ = \ P^{-1} \, A \, P
\]~\\[-1.4cm]
\begin{remarkL}{.72}
  On peut noter que cette égalité est équivalente à l'égalité : $A \ =
  \ P \, B \, P^{-1}$.
\end{remarkL}~\\[-.6cm]
L'objectif de cet exercice est d'étudier des exemples de matrices
inversibles qui sont semblables à leur inverse. Les trois parties de
cet exercice sont indépendantes entre elles.

\subsection*{PARTIE A : Premier exemple}

\noindent
On considère la matrice $A$ de $\M{3}$ définie par : $A \ = \ 
\begin{smatrix}
  1 & -1 & 1 \\
  0 & \frac{1}{2} & 0 \\
  0 & 0 & 2
\end{smatrix}
$.

\begin{noliste}{1.}
  \setlength{\itemsep}{4mm} %
\item Déterminer les valeurs propres de $A$.\\
  Justifier que $A$ est inversible et diagonalisable.

  \begin{proof}~%
    \begin{noliste}{$\sbullet$}
    \item La matrice $A$ est triangulaire supérieure. Ses valeurs
      propres sont donc ses coefficients diagonaux. %
      \conc{$\spc(A) = \{\frac{1}{2}, 1, 2\}$}

    \item Le réel $0$ n'est pas valeur propre de $A$. %
      \conc{On en déduit que $A$ est inversible.}

    \item On a :
      \begin{noliste}{$\stimes$}
      \item $A \in \M{3}$,
      \item $A$ possède trois valeurs propres distinctes.
      \end{noliste}
      \conc{On en déduit que $A$ est diagonalisable.}~\\[-1.65cm]
    \end{noliste}
  \end{proof}

\item Déterminer une matrice $D$ de $\M{3}$ diagonale où les
  coefficients diagonaux sont rangés dans l'ordre croissant, et une
  matrice $P$ de $\M{3}$ inversible telles que : $A \ = \ P \, D \,
  P^{-1}$.\\
  Expliciter la matrice $D^{-1}$.
  
  \begin{proof}~%
    \begin{noliste}{$\sbullet$}    
    \item Déterminons $E_{\frac{1}{2}}(A)$ le sous-espace propre de
      $A$ associé à la valeur propre $\frac{1}{2}$.\\
      % \item Déterminons une base de $E_{\frac{1}{2}}(A)$ le sous-espace
      %   propre de $A$ associé à la valeur propre $\frac{1}{2}$.\\
      Soit $X =
      \begin{smatrix}
        x \\ 
        y \\
        z
      \end{smatrix} 
      \in \M{3,1}$.
      \[
      \begin{array}{rcl}
        X\in E_{\frac{1}{2}}(A)
        & \Longleftrightarrow & (A - \frac{1}{2} \, I_3) \, X = 0
        \\[.2cm]
        & \Longleftrightarrow & 
        \begin{smatrix}
          \frac{1}{2} & -1 & 1 \\
          0 & 0 & 0 \\
          0 & 0 & \frac{3}{2}
        \end{smatrix}
        \begin{smatrix}
          x \\
          y \\
          z
        \end{smatrix}
        =
        \begin{smatrix}
          0 \\
          0 \\
          0
        \end{smatrix}
        \\[.8cm]
        & \Longleftrightarrow & 
        \left\{
	  \begin{array}{rcrcrcl}
            \frac{1}{2} \, x & - & y & + & z & = & 0 \\
            & & & & 0 & = & 0 \\
            & & & & \frac{3}{2} \, z & = & 0
	  \end{array}
        \right.
        \\[.8cm]
        & 
        \begin{arrayEq}
          L_1 \leftarrow L_1 - \frac{2}{3} \, L_3
        \end{arrayEq}
        & 
        \left\{
	  \begin{array}{rcrcrcl}
            \frac{1}{2} \, x & - & y & & & = & 0 \\
            & & & & \frac{3}{2} \, z & = & 0
	  \end{array}
        \right.
        \\[.8cm]
        &
        \Longleftrightarrow
        &
        \left\{
	  \begin{array}{rcrcl}
            x & & & = & 2 \, y \\
            & & z & = & 0 
	  \end{array}
        \right.
      \end{array}
      \]


      \newpage


      \noindent
      Finalement on obtient l'expression de $E_{\frac{1}{2}}(A)$
      suivante :
      \[
      \begin{array}{rcl}
        E_{\frac{1}{2}}(A) & = & 
        \{ 
        X \in \M{3,1} \ | \ AX = \frac{1}{2} \, X \}
        \ = \ 
        \{
        \begin{smatrix}
          x \\ 
          y \\
          z
        \end{smatrix}
        \ | \
        x = 2 y \ \text{ et } \ z = 0
        \}
        \\[.6cm]
        & = & 
        \{
        \begin{smatrix}
          2 y \\ 
          y \\ 
          0
        \end{smatrix}
        \ | \
        y \in \R 
        \}
        \ = \ 
        \{
        y \cdot
        \begin{smatrix}
          2 \\ 
          1 \\ 
          0
        \end{smatrix}
        \ | \ y \in \R 
        \}
        % \\[.6cm]
        \ = \ 
        \Vect{
	  \begin{smatrix}
            2 \\
            1 \\ 
            0
	  \end{smatrix}
        }
      \end{array}
      \]
      \conc{$E_{\frac{1}{2}}(A) = \Vect{
          \begin{smatrix}
            2 \\
            1 \\
            0
          \end{smatrix}
        }$} %
      La famille ${\cal F} = \left(
        \begin{smatrix} 
          2 \\
          1 \\ 
          0
        \end{smatrix}
      \right)$ :
      \begin{noliste}{$\stimes$}
      \item engendre $E_{\frac{1}{2}}(A)$,
      \item est une famille libre de $\M{3,1}$ car elle est uniquement
        constituée d'un vecteur non nul.
      \end{noliste}      
      \conc{Ainsi, ${\cal F}$ est une base de $E_{\frac{1}{2}}(A)$.}

    \item Déterminons $E_1(A)$ le sous-espace propre de $A$ associé à
      la valeur propre $1$.\\
%     \item Déterminons une base de $E_1(A)$ le sous-espace propre de
%       $A$ associé à la valeur propre $1$.\\
      Soit $X =
      \begin{smatrix}
        x \\ 
        y \\
        z
      \end{smatrix}
      \in \M{3,1}$.
      \[
      \begin{array}{rcl}
        X \in E_{1}(A)
        & \Longleftrightarrow & (A - I_3) \, X = 0
        \\[.2cm]
        & \Longleftrightarrow & 
        \begin{smatrix}
          0 & -1 & 1 \\
          0 & -\frac{1}{2} & 0 \\[.1cm]
          0 & 0 & 1
        \end{smatrix}
        \begin{smatrix}
          x \\
          y \\
          z
        \end{smatrix}
        =
        \begin{smatrix}
          0 \\
          0 \\
          0
        \end{smatrix}
        \\[.8cm]
        & \Longleftrightarrow & 
        \left\{
	  \begin{array}{crcrcl}
            - & y & + & z & = & 0 \\
            - & \frac{1}{2} \, y & & & = & 0 \\
            & & & z & = & 0
	  \end{array}
        \right.
        \\[.8cm]
        &
        \Longleftrightarrow
        &
        \left\{
	  \begin{array}{crcrcl}
            & y & & & = & 0 \\
            & & & z & = & 0
	  \end{array}
        \right.
      \end{array}
      \]
      Finalement on obtient l'expression de $E_1(A)$ suivante :
      \[
      \begin{array}{rcl}
        E_{1}(A) & = & \{ X \in \M{3,1} \ | \ AX = X\}
        \ = \ 
        \{
        \begin{smatrix}
          x \\ 
          y \\
          z
        \end{smatrix}
        \ | \
        y = 0 \ \text{ et } \ z = 0
        \}
        \\[.6cm]
        & = & 
        \{
        \begin{smatrix}
          x \\ 
          0 \\ 
          0
        \end{smatrix}
        \ | \
        x \in \R 
        \}
        \ = \ 
        \{
        x \cdot
        \begin{smatrix}
          1 \\ 
          0 \\ 
          0
        \end{smatrix}
        \ | \ x \in \R 
        \}
        % \\[.6cm]
        \ = \
        \Vect{
	  \begin{smatrix}
            1 \\
            0 \\ 
            0
	  \end{smatrix}
        }
      \end{array}
      \]
%       On sait donc que la famille $\left( 
%         \begin{smatrix} 
%           1 \\
%           0 \\ 
%           0
%         \end{smatrix}
%       \right)$ :
%       \begin{noliste}{$\stimes$}
%       \item engendre $E_{1}(A)$,
%       \item est une famille libre de $\M{3,1}$ car elle est uniquement
%         constituée d'un vecteur non nul.
%       \end{noliste}      
      \conc{$E_{1}(A) = \Vect{
          \begin{smatrix}
            1 \\
            0 \\
            0
          \end{smatrix}
        }$} %
      \conc{Par un raisonnement analogue au précédent, la famille
        $\left(
          \begin{smatrix}
            1 \\
            0 \\
            0
          \end{smatrix} 
        \right)$ est une base de $E_{1}(A)$.}


      \newpage


    \item Déterminons $E_{2}(A)$ le sous-espace propre de $A$ associé
      à la valeur propre $2$.\\
%     \item Déterminons une base de $E_{2}(A)$ le sous-espace propre de
%       $A$ associé à la valeur propre $2$.\\
      Soit $X =
      \begin{smatrix}
        x \\ 
        y \\
        z
      \end{smatrix} 
      \in \M{3,1}$.
      \[
      \begin{array}{rcl}
        X \in E_{2}(A)
        & \Longleftrightarrow & (A - 2 \, I_3) \, X = 0
        \\[.2cm]
        & \Longleftrightarrow & 
        \begin{smatrix}
          -1 & -1 & 1 \\
          0 & -\frac{3}{2} & 0 \\
          0 & 0 & 0
        \end{smatrix}
        \begin{smatrix}
          x \\
          y \\
          z
        \end{smatrix}
        =
        \begin{smatrix}
          0 \\
          0 \\
          0
        \end{smatrix}
        \\[.8cm]
        & \Longleftrightarrow & 
        \left\{
	  \begin{array}{rcrcrcl}
            - \, x & - & y & + & z & = & 0 \\
            & - & \frac{3}{2} \, y & & & = & 0 \\
            & & & & 0 & = & 0
	  \end{array}
        \right.
        \\[.8cm]
        & \Longleftrightarrow & 
        \left\{
	  \begin{array}{rcrcl}
            x & + & y & = & z \\[.1cm]
            & - & \frac{3}{2} \, y & = & 0 
	  \end{array}
        \right.
        \\[.8cm]
        &
        \begin{arrayEq}
          L_1 \leftarrow L_1 + \frac{2}{3} \, L_2
        \end{arrayEq}
        &
        \left\{
	  \begin{array}{rcrcl}
            x & & & = & z \\
            & & y & = & 0 
	  \end{array}
        \right.
      \end{array}
      \]
      Finalement on obtient l'expression de $E_{2}(A)$ suivante :
      \[
      \begin{array}{rcl}
        E_{2}(A) & = & 
        \{ 
        X \in \M{3,1} \ | \ AX = 2 \, X \}
        \ = \ 
        \{
        \begin{smatrix}
          x \\ 
          y \\
          z
        \end{smatrix}
        \ | \
        x = z \ \text{ et } \ y = 0
        \}
        \\[.6cm]
        & = & 
        \{
        \begin{smatrix}
          z \\ 
          0 \\ 
          z
        \end{smatrix}
        \ | \
        z \in \R 
        \}
        \ = \ 
        \{
        z \cdot
        \begin{smatrix}
          1 \\ 
          0 \\ 
          1
        \end{smatrix}
        \ | \ z \in \R 
        \}
        % \\[.6cm]
        \ = \
        \Vect{
	  \begin{smatrix}
            1 \\
            0 \\ 
            1
	  \end{smatrix}
        }
      \end{array}
      \]
      % On sait donc que la famille $\left(
      %   \begin{smatrix}
      %     1 \\
      %     0 \\ 
      %     1
      %   \end{smatrix}
      % \right)$ :
      % \begin{noliste}{$\stimes$}
      % \item engendre $E_{2}(A)$,
      % \item est une famille libre de $\M{3,1}$ car elle est uniquement
      %   constituée d'un vecteur non nul.
      % \end{noliste}      
      \conc{$E_{2}(A) = \Vect{
          \begin{smatrix}
            1 \\
            0 \\
            1
          \end{smatrix}
        }$} %
      \conc{Par un raisonnement analogue au précédent, la famille
        $\left(
          \begin{smatrix}
            1 \\
            0 \\
            1
          \end{smatrix}
        \right)$ est une base de $E_{2}(A)$.}
      
    \item D'après la question \itbf{1.}, la matrice $A$ est
      diagonalisable.\\
      Il existe donc une matrice $P$ inversible et une matrice $D$
      diagonale telles que $A = P \, D \, P^{-1}$.\\
      Plus précisément :
      \begin{noliste}{$\stimes$}
      \item la matrice $P$ est obtenue par concaténation de bases des
        sous-espaces propres de $A$,
      \item la matrice $D$ est la matrice diagonale dont les
        coefficients diagonaux sont les valeurs propres de $A$ (dans
        le même ordre d'apparition que les vecteurs propres).
      \end{noliste}
      Comme $\left(
        \begin{smatrix}
          2 \\
          1 \\
          0
        \end{smatrix}
      \right)$, %
      $\left(
        \begin{smatrix}
          1 \\
          0 \\
          0
        \end{smatrix}
      \right)$ et %
      $\left(
        \begin{smatrix}
          1 \\
          0 \\
          1
        \end{smatrix}
      \right)$ sont des bases respectives de $E_{\frac{1}{2}}(A)$,
      $E_1(A)$ et $E_2(A)$ :
      \[
      P=
      \begin{smatrix}
        2 & 1 & 1 \\
        1 & 0 & 0 \\
        0 & 0 & 1
      \end{smatrix}
      \quad \text{ et } \quad %
      D =
      \begin{smatrix}
        \frac{1}{2} & 0 & 0\\
        0 & 1 & 0\\
        0 & 0 & 2
      \end{smatrix}
      \]
      \conc{On obtient bien : $A = P \, D \, P^{-1}$.}%~\\[-1cm]
      

      \newpage


    \item La matrice $D$ est inversible car elle est diagonale et à
      coefficients diagonaux tous non nuls.
    \end{noliste}
    \conc{Enfin : $D^{-1} =
      \begin{smatrix}
        2 & 0 & 0 \\
        0 & 1 & 0 \\
        0 & 0 & \frac{1}{2}
      \end{smatrix}
      $.}~\\[-1.1cm]
    \begin{remarkL}{.96}%~%
      \begin{noliste}{$\sbullet$}
      \item Il faut s'habituer à déterminer les ensembles $E_\lambda(A)$
        par lecture de la matrice $A - \lambda \, I$.

      \item Illustrons la méthode avec la matrice de l'exercice et
        $\lambda = \frac{1}{2}$.\\
        On cherche les vecteurs $X =
        \begin{smatrix}
          x \\
          y \\
          z
        \end{smatrix}
        $ de $E_{\frac{1}{2}}(A) $ c'est-à-dire les vecteurs tels que
        : \\
        $(A - \frac{1}{2} \, I_3) \, X = 0_{\M{3,1}}$. Or :\\[-.1cm]
        \[
        \begin{array}{rcl}
          \begin{smatrix}
            \frac{1}{2} & -1 & 1 \\
            0 & 0 & 0 \\
            0 & 0 & \frac{3}{2}
          \end{smatrix}
          \begin{smatrix}
            x \\
            y \\
            z 
          \end{smatrix}
          & = & x \cdot C_1 + y \cdot C_2 + z \cdot C_3
          \\[-.2cm]
          & = & 
          x \cdot
          \begin{smatrix}
            \frac{1}{2} \\
            0 \\
            0
          \end{smatrix}
          + y \cdot
          \begin{smatrix}
            -1 \\
            0 \\
            0 
          \end{smatrix}
          + z \cdot
          \begin{smatrix}
            1 \\
            0 \\
            \frac{3}{2}
          \end{smatrix}
        \end{array}
        \]~\\[-.6cm]
        Pour obtenir le vecteur $
        \begin{smatrix}
          0 \\
          0 \\
          0
        \end{smatrix}
        $ à l'aide de cette combinaison linéaire, on doit forcément
        choisir $z = 0$. En effet, si $z$ est non nul, les
        coefficients en $\eme{2}$ et $\eme{3}$ position de la
        combinaison linéaire sont non nuls. On prend alors $z = 0$.\\
        La combinaison linéaire restante est nulle si $\frac{1}{2} \,
        x - y = 0$, c'est-à-dire si $x = 2y$.\\
        En prenant par exemple $y = 1$, on obtient : $
        E_{\frac{1}{2}}(A) \ \supset \ \Vect{
          \begin{smatrix}
            2 \\
            1 \\
            0 
          \end{smatrix}
        } $.\\
        Et l'égalité est vérifiée car ces deux espaces vectoriels sont
        de même dimension.

      \item Explicitons maintenant la construction de la matrice
        $P$.\\
        Notons $\B = (e_1, e_2, e_3)$ la base canonique de $\R^3$ et
        introduisons l'endomorphisme $f$ dont la représentation dans
        la base canonique est $\B$ (cet endomorphisme est unique par
        bijectivité de l'application $\Mat_{\B}(\cdot)$). Notons aussi
        : $v_1 = (2, 1, 0)$, \ $v_2 = (1, 0, 0)$, et $v_3 = (1, 0,
        1)$.\\[.1cm]
        La famille $\B' = (v_1, v_2, v_3)$ :
        \begin{noliste}{$\stimes$}
        \item est libre car est la concaténation de familles libres de
          vecteurs propres associés à des valeurs propres distinctes
          de l'endomorphisme $f$. 

        \item de cardinal $\Card( \B' ) = 3 = \dim\big( \R^3 \big)$.
        \end{noliste}
        On en déduit que $\B'$ est une base de vecteurs propres de
        $f$.\\
        Notons alors $D = \Mat_{\B'}(f)$. Par la formule du changement
        de base on a :
        \[
        \begin{array}{ccccccc}
          \Mat_{\B}(f) & = & P_{\B, \B'} & \times & \Mat_{\B'}(f) &
          \times & P_{\B', \B}
          \\[.2cm]
          \shortparallel & & \shortparallel & & \shortparallel & &
          \shortparallel  
          \\[.2cm]
          A & = & P & \times & D & \times & P^{-1}
        \end{array}
        \]
        On obtient bien la formule annoncée.

      \item Rappelons que si $\Delta$ est une matrice diagonale à
        coefficients diagonaux $a$, $b$, $c$ tous non nuls
        \[
        \text{alors} \quad 
        \Delta^{-1}
        \ = \ 
        \begin{smatrix}
          \frac{1}{a} & 0 & 0 \\
          0 & \frac{1}{b} & 0 \\
          0 & 0 & \frac{1}{c}
        \end{smatrix}
        \quad \text{ puisque } \quad
        \begin{smatrix}
          a & 0 & 0 \\
          0 & b & 0 \\
          0 & 0 & c
        \end{smatrix}
        \begin{smatrix}
          \frac{1}{a} & 0 & 0 \\
          0 & \frac{1}{b} & 0 \\
          0 & 0 & \frac{1}{c}
        \end{smatrix}
        \ = \
        \begin{smatrix}
          1 & 0 & 0 \\
          0 & 1 & 0 \\
          0 & 0 & 1
        \end{smatrix}
        \]
      \end{noliste}
    \end{remarkL}~\\[-1.6cm]
  \end{proof}
  

  \newpage


\item On note $Q \ = \
  \begin{smatrix}
    0 & 0 & 1 \\
    0 & 1 & 0 \\
    1 & 0 & 0
  \end{smatrix}
  $. Calculer $Q^2$ et $Q \, D \, Q$.

  \begin{proof}~%
    \begin{noliste}{$\sbullet$}
    \item Tout d'abord :
      \[
      Q^2 \ = \
      \begin{smatrix}
        0 & 0 & 1 \\
        0 & 1 & 0 \\
        1 & 0 & 0
      \end{smatrix}
      \begin{smatrix}
        0 & 0 & 1 \\
        0 & 1 & 0 \\
        1 & 0 & 0
      \end{smatrix}
      \ = \ 
      \begin{smatrix}
        1 & 0 & 0 \\
        0 & 1 & 0 \\
        0 & 0 & 1
      \end{smatrix}
      \]
      \conc{Ainsi : $Q^2 = Q \times Q = I_3$. On en déduit que la
        matrice $Q$ est inversible d'inverse $Q^{-1} = Q$.}

    \item Ensuite :
      \[
      \begin{array}{rcl}
        Q \, D \, Q & = & 
        \begin{smatrix}
          0 & 0 & 1 \\
          0 & 1 & 0 \\
          1 & 0 & 0
        \end{smatrix}
        \begin{smatrix}
          \frac{1}{2} & 0 & 0 \\
          0 & 1 & 0 \\
          0 & 0 & 2
        \end{smatrix}
        \begin{smatrix}
          0 & 0 & 1 \\
          0 & 1 & 0 \\
          1 & 0 & 0
        \end{smatrix}
        % \\[.8cm]
        \ = \ 
        \begin{smatrix}
          0 & 0 & 2 \\
          0 & 1 & 0 \\
          \frac{1}{2} & 0 & 0
        \end{smatrix}
        \begin{smatrix}
          0 & 0 & 1 \\
          0 & 1 & 0 \\
          1 & 0 & 0
        \end{smatrix}
        % \\[.8cm]
        \ = \
        \begin{smatrix}
          2 & 0 & 0 \\
          0 & 1 & 0 \\
          0 & 0 & \frac{1}{2}
        \end{smatrix}
      \end{array}      
      \]
      \conc{$Q \, D \, Q = D^{-1}$}~\\[-1cm]
    \begin{remarkL}{.98}%
      \begin{noliste}{$\sbullet$}
      \item Les coefficients de la matrice $Q$ sont tous dans $\{ 0, 1
        \}$. De plus, la matrice $Q$ admet un seul $1$ par ligne et
        par colonne. Une telle matrice est appelée {\bf matrice de
          permutation}.

      \item Ces matrices permettent de formaliser les opérations
        élémentaires permettant d'échanger des lignes (resp. des
        colonnes) lors de l'algorithme du pivot de Gauss. Plus
        précisément, multiplier une matrice $A$ à gauche (resp. à
        droite) par une matrice de permutation permet d'échanger des
        lignes (des colonnes). Dans notre exemple, la matrice $Q$
        permet d'échanger les éléments en $\ere{1}$ et $\eme{3}$
        position. Ainsi :
        \[
        \begin{array}{rC{1cm}l}
          \begin{smatrix}
            0 & 0 & 1 \\
            0 & 1 & 0 \\
            1 & 0 & 0 
          \end{smatrix}
          \begin{smatrix}
            1 & -1 & 1 \\
            0 & \frac{1}{2} & 0 \\
            0 & 0 & 2
          \end{smatrix}
          \ = \ 
          \begin{smatrix}
            0 & 0 & 2 \\
            0 & \frac{1}{2} & 0 \\
            1 & -1 & 1 
          \end{smatrix}      
          & et &
          \begin{smatrix}
            1 & -1 & 1 \\
            0 & \frac{1}{2} & 0 \\
            0 & 0 & 2
          \end{smatrix}
          \begin{smatrix}
            0 & 0 & 1 \\
            0 & 1 & 0 \\
            1 & 0 & 0 
          \end{smatrix}
          \ = \ 
          \begin{smatrix}
            1 & -1 & 1 \\
            0 & \frac{1}{2} & 0 \\
            2 & 0 & 0
          \end{smatrix}      
          \\[.8cm]
          \text{\it (on échange les lignes $1$ et $3$ de $A$)}
          & & 
          \text{\it (on échange les colonnes $1$ et $3$ de $A$)}
        \end{array}
        \]
        On peut retenir l'idée développée dans le paragraphe par la
        forme :
        \[
        L \ A \ C
        \]
        qui signifie qu'avec une multiplication à gauche, on
        effectue une opération sur les (L)ignes, tandis qu'avec une
        multiplication à droite, on effectue une multiplication sur
        les (C)olonnes.

      \item L'écriture $Q \, D \, Q^{-1} = D^{-1}$ réfère à un
        changement de base. Pour bien comprendre ce point,
        introduisons la famille $\B'' = (w_1, w_2, w_3)$ définie par :
        $w_1 = u_3$, $w_2 = u_2$, et $w_3 = u_1$.\\[.1cm]
        La famille $(w_1, w_2, w_3)$ est :
        \begin{noliste}{$\stimes$}
        \item génératrice de $\R^3$ puisque : $\Vect{ w_1, w_2, w_3} =
          \Vect{ u_3, u_2, u_1} = \Vect{ u_1, u_2, u_3} = \R^3$.
        \item de cardinal $\Card\big( (w_1, w_2, w_3) \big) = 3 =
          \dim\big( \R^3 \big)$.
        \end{noliste}
        Ainsi, la famille $\B''$ est une base de $\R^3$.\\
        Remarquons alors : $Q^{-1} = P_{\B', \B''}$ (matrice de
        passage de la base $\B'$ à la base $\B''$). Ainsi :
        \[
        \begin{array}{rccccccl@{\qquad}>{\it}R{3.5cm}}
          D^{-1} & = & Q & \times & D & \times & Q^{-1}
          \\%[.2cm]
          & = & P_{\B'', \B'} & \times & \Mat_{\B'}(f) & \times &
          P_{\B', \B''} & = \ \Mat_{\B'}(f)
          & (par formule de changement de base)
          % \\[.2cm]
          % & = & \multicolumn{5}{l}{\Mat_{\B'}(f)}
        \end{array}
        \]
        Ainsi, la matrice $D^{-1}$ apparaît comme matrice
        représentative de $f$ dans la base $\B'$.
      \end{noliste}
    \end{remarkL}~\\[-1.7cm]
    \end{noliste}
  \end{proof}


\newpage


\item En déduire que les matrices $A$ et $A^{-1}$ sont semblables.

  \begin{proof}~%
    \begin{noliste}{$\sbullet$}
    \item D'après la question \itbf{2.} : $A = P \, D \, P^{-1}$. On
      en déduit :
      \begin{noliste}{$\stimes$}
      \item $P^{-1} \, A = D \, P^{-1}$ \ puis \ $P^{-1} \, A \, P =
        D$.

%       \item $      
%         \begin{array}[t]{rcl@{\qquad}>{\it}R{5cm}}
%           A^{-1} & = & \Big( P \, \big( D \, P^{-1} \big) \Big)^{-1}
%           \\[.2cm]
%           & = & \big( D \, P^{-1} \big)^{-1} \, P^{-1}
%           \\[.2cm]
%           & = & \big( P^{-1} \big)^{-1} \, D^{-1} \, P^{-1}
%           \\[.2cm]
%           & = & P \, D^{-1} \, P^{-1}
%         \end{array}
%         $\\[.2cm]

      \item $A^{-1} \ = \ \Big( P \, \big( D \, P^{-1} \big)
        \Big)^{-1}
        \ = \ \big( D \, P^{-1} \big)^{-1} \, P^{-1}
        \ = \ \big( P^{-1} \big)^{-1} \, D^{-1} \, P^{-1}
        \ = \ P \, D^{-1} \, P^{-1}
        $.
        % \\[.1cm]
        % Et ainsi : $D^{-1} = P^{-1} \, A^{-1} \, P$.
      \end{noliste}

    \item En combinant ce résultat avec ce qui précède :
      \[
      \begin{array}{C{2cm}rcl@{\qquad}>{\it}R{6.5cm}}
        & A^{-1} & = & P \, D^{-1} \, P^{-1}
        \\%[.2cm]
        & & = & 
        P \, \big( Q \, D \, Q^{-1} \big) \, P^{-1}
        & (d'après la question précédente, \\ en remarquant $Q = Q^{-1}$)
        \nl
        %\nl[-.2cm]
        & & = & 
        P \, Q \, \big( P^{-1} \, A \, P \big) \, Q^{-1} \, P^{-1}
        \\[.2cm]
        & & = & 
        \big( P \, Q \, P^{-1} \big) \, A \, \big( P \, Q^{-1} \,
        P^{-1} \big)
        \\[.2cm]
        & & = & 
        \big( P \, Q \, P^{-1} \big) \, A \, \big( P \, Q \, P^{-1}
        \big)^{-1} 
      \end{array}
      \]
      La dernière ligne est obtenue en remarquant :
      \[
      \big( P \, Q \, P^{-1} \big)^{-1} \ = \ \big( Q \, P^{-1}
      \big)^{-1} \, P^{-1} \ = \ \big( P^{-1} \big)^{-1} \, Q^{-1} \,
      P^{-1} \ = \ P \, Q^{-1} \, P^{-1}
      \]
    \end{noliste}
    \concL{En posant $H = P \, Q \, P^{-1}$, on obtient $A^{-1} = H \,
      A \, H^{-1}$ ou encore $A = H^{-1} \, A^{-1} \, H$.\\
      Les matrices $A$ et $A^{-1}$ sont donc semblables.
    }{14.4}~\\[-1cm]
    \begin{remarkL}{.98}%
      \begin{noliste}{$\sbullet$}
%       \item En début d'exercice, il est précisé que les matrices $A
%         \in \M{3}$ et $B \in \M{3}$ sont semblables s'il existe une
%         matrice $P \in \M{3}$ inversible telle que : $B \ = \ P^{-1}
%         \, A \, P$.\\
%         Notons que cette égalité est équivalente à l'égalité : $A \ =
%         \ P \, B \, P^{-1}$.

      \item D'après le cours :
        \[
        \begin{array}{L{5.2cm}cR{7.2cm}}
          Deux matrices $A \in \M{3}$ et $B \in \M{3}$ sont
          semblables & \Leftrightarrow & Les matrices $A$ et $B$
          représentent le même endomorphisme dans des bases
          différentes
        \end{array}
        \]
        Cela revient à démontrer que toute matrice inversible peut
        être considérée comme une matrice de changement de base.
%         \\[.2cm]
%         Or, on a noté dans les précédentes remarques :
%         \[
%         A = \Mat_{\B}(f) \ \qquad \ D = \Mat_{\B'}(f) \ \qquad \
%         D^{-1} = \Mat_{\B''}(f)
%         \]
%         Ces matrices représentent toutes le même endomorphisme $f$
%         dans des bases différentes. Elles sont donc semblables deux à
%         deux.

      \item On peut illustrer ce point à l'aide de la question. On
        démontre : $A^{-1} = H \, A \, H^{-1}$, ce qui signifie que
        $A$ et $A^{-1}$ sont semblables. On peut aussi faire
        apparaître $H$ et $H^{-1}$ comme des matrices de changement de
        base ce qui permettra d'interpréter cette égalité comme une
        formule de changement de base. Pour ce faire, introduisons la
        famille $\B''' = (z_1, z_2, z_3)$ définie par : $z_1 = e_1$,
        $z_2 = -e_1 + e_3$, et $z_3 = e_1 + e_2$.\\[.1cm]
        Cette famille est une base de $\R^3$ (car génératrice de
        $\R^3$ et de cardinal $3$).\\
        Remarquons alors : $H^{-1} = P_{\B, \B'''}$ (matrice de
        passage de la base $\B$ à la base $\B'''$). Ainsi :
        \[
        \begin{array}{rccccccl@{\quad}>{\it}R{3.5cm}}
          A^{-1} & = & H & \times & A & \times & H^{-1}
          \\%[.2cm]
          & = & P_{\B''', \B} & \times & \Mat_{\B}(f) & \times &
          P_{\B, \B'''} & = \ \Mat_{\B'''}(f)
          & (par formule de changement de base)
          \nl[-.2cm]
          % & = & \multicolumn{5}{l}{\Mat_{\B}(f)}
        \end{array}
        \]
        Ainsi, les matrices $A$ et $A^{-1}$ représentent le même
        endomorphisme $f$ dans des bases différentes.

      \item Notons que si $A$ est une représentation matricielle d'un
        endomorphisme $f$ alors $A^{-1}$ représente naturellement
        l'endomorphisme $f^{-1}$. En effet :
        \[
        \text{si \ $A = \Mat_{\B}(f)$ \quad alors \quad $A^{-1} =
          \big(\Mat_{\B}(f) \big)^{-1} = \Mat_{\B}\big( f^{-1} \big)$
        }
        \]
        On a ainsi trouvé une base $\B'''$ dans laquelle $A^{-1}$ est
        une représentation de $f$ et une base $\B$ dans laquelle
        $A^{-1}$ est une représentation de $f^{-1}$. On ne peut
        conclure pour autant : $f \xcancel{\rule[0cm]{0cm}{.3cm} =
          \rule[-.1cm]{0cm}{.3cm}} f^{-1}$.\\
        {\it (cela sera vrai si les bases $\B$ et $\B'''$ étaient égales)}
      \end{noliste}
    \end{remarkL}~\\[-1.5cm]
  \end{proof}
\end{noliste}


\newpage


\subsection*{PARTIE B : Deuxième exemple}

\noindent
On considère $f$ l'endomorphisme de $\R^3$ défini par : 
\[
\forall (x, y, z) \in \R^3, \quad f(x, y, z) \ = \ (x, -z, y + 2z)
\]
On note $M$ la matrice de $f$ dans la base canonique de $\R^3$.\\
On considère également les vecteurs $u_1$ et $u_2$ de $\R^3$ définis
par : $u_1 = (1, 0, 0)$ et $u_2 = (0, 1, -1)$.

\begin{noliste}{1.}
  \setlength{\itemsep}{4mm} %
  \setcounter{enumi}{4}
\item Expliciter la matrice $M$ et montrer que $M$ est inversible.

  \begin{proof}~\\%
    Notons $\B = (e_1, e_2, e_3)$ la base canonique de $\R^3$.
    % Autrement dit :
    % \[
    % e_1 = (1, 0, 0), \qquad e_2 = (0, 1, 0) \qquad \text{ et } \qquad
    % e_3 = (0, 0, 1)
    % \]
    \begin{noliste}{$\sbullet$}
    \item $f(e_1) = f(1, 0, 0) = (1, 0, 0) = e_1 = 1 \cdot e_1 + 0
      \cdot e_2 + 0 \cdot e_3$.\\[.1cm]
      Ainsi : $\Mat_{\B}\big( f(e_1) \big) =
      \begin{smatrix}
        1 \\
        0 \\
        0
      \end{smatrix}
      $.
      
    \item $f(e_2) = f(0, 1, 0) = (0, 0, 1) = e_3 = 0 \cdot e_1 + 0
      \cdot e_2 + 1 \cdot e_3$.\\[.1cm]
      Ainsi : $\Mat_{\B}\big( f(e_2) \big) =
      \begin{smatrix}
        0 \\
        0 \\
        1
      \end{smatrix}
      $.
      
    \item $f(e_3) = f(0, 0, 1) = (0, -1, 2) = 0 \cdot e_1 - 1 \cdot
      e_2 + 2 \cdot e_3$.\\[.1cm]
      Ainsi : $\Mat_{\B}\big( f(e_3) \big) =
      \begin{smatrix}
        0 \\
        -1 \\
        2
      \end{smatrix}
      $.
    \end{noliste}
    \conc{Finalement : $M = \Mat_{\B}(f)=
      \begin{smatrix}
        1 & 0 & 0 \\
        0 & 0 & -1 \\
        0 & 1 & 2
      \end{smatrix}
      $.}%~\\[-1cm]
    \begin{noliste}{$\sbullet$}
    \item D'autre part :
      \[
      \rg(M) \ = \ \rg \left( 
        \begin{smatrix}
          1 & 0 & 0 \\
          0 & 0 & -1 \\
          0 & 1 & 2
        \end{smatrix}
      \right)
      \
      \begin{arrayEg}
        L_2 \leftrightarrow L_3
      \end{arrayEg}
      \ 
      \rg \left( 
        \begin{smatrix}
          1 & 0 & 0 \\
          0 & 1 & 2 \\
          0 & 0 & -1
        \end{smatrix}
      \right)
      \]
      La réduite obtenue est triangulaire supérieure et à coefficients
      diagonaux non nuls.\\
      Elle est donc inversible et il en est de même de la matrice
      initiale $M$.\\[-.2cm] %
      \conc{La matrice $M$ est inversible.}~\\[-1.5cm]
    \end{noliste}
  \end{proof}

\item
  \begin{noliste}{a.}
    \setlength{\itemsep}{2mm} %
  \item Vérifier que $1$ est valeur propre de $f$ et $(u_1, u_2)$ est
    une base du sous-espace propre associée.

    \begin{proof}~%
      \begin{noliste}{$\sbullet$}
      \item Tout d'abord :
        \[
        \begin{array}{rcl@{\qquad}>{\it}R{5cm}}
        \rg(M - I_3) %
        & = & \rg\left(
          \begin{smatrix}
            0 & 0 & 0 \\
            0 & -1 & -1 \\
            0 & 1 & 1
          \end{smatrix}
        \right) %
        \\[.8cm]
        &
        \begin{arrayEg}
          L_3 \leftarrow L_3 + L_2
        \end{arrayEg}
        &
        \rg\left(
          \begin{smatrix}
            0 & 0 & 0 \\
            0 & -1 & -1 \\
            0 & 0 & 0 
          \end{smatrix}
        \right) %
        \\[.8cm]
        & 
        \begin{arrayEg}
          C_3 \leftarrow C_3 - C_2
        \end{arrayEg}
        &
        \rg\left(
          \begin{smatrix}
            0 & 0 & 0 \\
            0 & -1 & 0 \\
            0 & 0 & 0 
          \end{smatrix}
        \right) %
        % \\[.8cm]        
        \ = \ 1
      \end{array}
      \]
      \conc{On en déduit : $\rg\big( f - \id_{\R^3} \big) \ = \
        \rg\big( M - I_3 \big) \ = \ 1$.}


      \newpage


      \item Or, par théorème du rang :
        \[
        \begin{array}{C{2cm}ccccc}
          & \dim\big( \R^3 \big) & = & \dim\Big( \kr\big( f - \id_{\R^3}
          \big) \Big) & + & \rg\big( f - \id_{\R^3} \big)
          \\[.2cm]
          donc & \dim\big( \R^3 \big) & = & \dim\big( E_1(f) \big) & +
          & \rg\big( M - I_3 \big) 
          \\[.2cm]
          & \shortparallel & & & & \shortparallel
          \\[.2cm]
          & 3 & & & & 1
        \end{array}
        \]
        On en déduit : $\dim\big( E_{1}(f) \big) \ = \ 3 - 1 \ = \
        2$.%
        \conc{$\dim\big( E_{1}(f) \big) \ = \ 2$}

      \item D'autre part :
        \begin{noliste}{$\stimes$}
        \item $f(u_1) = f(1, 0, 0) = (1, 0, 0) = u_1 = 1 \cdot u_1$.
        \item $f(u_2) = f(0, 1, -1) = (0, 1, 1-2) = u_2 = 1 \cdot u_2$.
        \end{noliste}
        \conc{Ainsi $u_1$ et $u_2$ sont deux vecteurs propres de $f$
          associés à la valeur propre $1$.}

      \item Ainsi, ${\cal F} = (u_1, u_2)$ est une famille d'éléments
        de $E_1(f)$ qui est :
        \begin{noliste}{$\stimes$}
        \item libre car constituée de deux vecteurs non colinéaires.
        \item de cardinal $\Card\big( {\cal F} \big) = 2 = \dim\big(
          E_1(f) \big)$. %\\[-.8cm]
        \end{noliste}
        \concL{On en déduit que ${\cal F}$ est une base de $E_1(f)$,
          sous-espace propre de $f$ associé à la valeur propre
          $1$.}{13.4}
      \end{noliste}
      \begin{remarkL}{.98}%~
        \begin{noliste}{$\sbullet$}
        \item La formulation de l'énoncé, à savoir \og {\bf Vérifier}
          que $1$ est valeur propre \fg{} oriente vers une méthode
          directe consistant généralement en un calcul simple. Il ne
          s'agit donc pas de mettre en place une démonstration
          théorique. Ici, il serait malvenu de chercher toutes les
          valeurs propres en effectuant le calcul de $\rg(M - \lambda
          \, I_3)$.
        \item L'énoncé demande de démontrer que la famille $(u_1, u_2)$
          est une base de $E_1(f)$.\\
          En démontrant que $u_1$ et $u_2$ sont deux vecteurs propres de
          $f$ associés à la valeur propre $1$, on démontre :
          \[
          E_1(f) \ \supset \ \Vect{u_1, u_2}
          \]
          On conclut à l'égalité de ces deux espaces vectoriels à l'aide
          d'un argument de dimension fournit par le théorème du rang.
          
        \item On peut aussi rédiger tout autrement. On ne démontre pas
          que $1$ est valeur propre mais on détermine directement, par
          la méthode usuelle $E_{1}(f)$. Détaillons cette méthode.\\[.1cm]
          Soit $u = (x, y, z) \in \R^3$. Notons $U = \Mat_{\B}(u) =
          \begin{smatrix}
            x \\
            y \\
            z
          \end{smatrix}
          $.
          \[
          \begin{array}{rcl@{\quad}>{\it}R{5cm}}
            u \in E_{1}(f)
            & \Longleftrightarrow & 
            U \in E_{1}(M)
            \\[.2cm]
            & \Longleftrightarrow & 
            (M - I_3) \, U = 0_{\M{3,1}}
            \\[.2cm]
            & \Longleftrightarrow & 
            \begin{smatrix}
              0 & 0 & 0 \\
              0 & - 1 & -1 \\
              0 & 1 & 1
            \end{smatrix}
            \begin{smatrix}
              x \\
              y \\
              z
            \end{smatrix}
            =
            \begin{smatrix}
              0 \\
              0 \\
              0
            \end{smatrix}
            \\[.8cm]
            & \Longleftrightarrow & 
            \left\{
              \begin{array}{crcrcr}
                & & & 0 & = & 0 
                \\[.2cm]
                - & y & - & z & = & 0 
                \\[.2cm]
                & y & + & z & = & 0 
              \end{array}
            \right.
            % \\[1cm]
            % & 
            % \begin{arrayEq}
            %   L_3 \leftarrow L_3 + L_2
            % \end{arrayEq}
            % & 
            \
            \begin{arrayEq}
              L_3 \leftarrow L_3 + L_2
            \end{arrayEq}
            \ 
            \left\{
              \begin{array}{crcrcr}
                & & & 0 & = & 0 
                \\[.2cm]
                - & y & - & z & = & 0 
                \\[.2cm]
                & & & 0 & = & 0 
              \end{array}
            \right.
          \end{array}
          \]
        \end{noliste}
      \end{remarkL}%~\\[-5.4cm]
      \begin{remarkL}{.98}
        On obtient alors :
        \[
        \begin{array}{rcl@{\quad}>{\it}R{5cm}}
          E_{1}(f) & = & 
          \{ \ 
          (x, y, z) \in \R^3
          \ | \ \ y = -z \ \}
          \\[.2cm]
          & = & 
          \{ \ 
          (x, -z, z) \in \R^3
          \ | \ \ (x, z) \in \R^2 
          \ \}
          \\[.2cm]
          & = & 
          \{ \ 
          x \cdot (1, 0, 0) + y \cdot (0, -1, 1)
          \ | \ \ (x, z) \in \R^2 
          \ \}
          \\[.2cm]
          & = &
          \Vect{\ (1, 0, 0), \ (0, -1, 1) \ }
          \\[.2cm]
          & = &
          \Vect{\ (1, 0, 0), \ (0, 1, -1) \ } \ = \ \Vect{u_1, u_2}
        \end{array}
        \]
        Comme : $E_{1}(f) \neq \{ 0_{\M{3,1}} \}$, le réel $1$ est
        bien valeur propre de $f$ et $E_{1}(f)$ est le sous-espace
        propre associé à cette valeur propre.
      \end{remarkL}~\\[-1.4cm]
    \end{proof}
    
  \item Déterminer un vecteur $u_3$ de $\R^3$ tel que : $f(u_3) - u_3 =
    u_2$.
    
    \begin{proof}~\\%
      Soit $u_3 = (x, y, z) \in \R^3$. Notons $U_3 = \Mat_{\B}(u_3) =
      \begin{smatrix}
        x \\
        y \\
        z
      \end{smatrix}
      $ \ et \ $U_2 = \Mat_{\B}(u_2) =
      \begin{smatrix}
        0 \\
        1 \\
        -1
      \end{smatrix}
      $.
      \[
      \begin{array}{rcl@{\quad}>{\it}R{5cm}}
        f(u_3) - u_3 = u_2
        & \Longleftrightarrow & 
        \big( f - \id_{\R^3} \big) (u_3) = u_2
        \\[.2cm]
        & \Longleftrightarrow & 
        (M - I_3) \, U_3 = U_2
        \\[.2cm]
        & \Longleftrightarrow & 
        \begin{smatrix}
          0 & 0 & 0 \\
          0 & - 1 & -1 \\
          0 & 1 & 1
        \end{smatrix}
        \begin{smatrix}
          x \\
          y \\
          z
        \end{smatrix}
        =
        \begin{smatrix}
          0 \\
          1 \\
          -1
        \end{smatrix}
        \\[.8cm]
        & \Longleftrightarrow & 
        \left\{
          \begin{array}{crcrcr}
            & & & 0 & = & 0 
            \\[.2cm]
            - & y & - & z & = & 1
            \\[.2cm]
            & y & + & z & = & -1
          \end{array}
        \right.
        \\[1cm]
        &
        \begin{arrayEq}
          L_3 \leftarrow L_3 + L_2
        \end{arrayEq}
        & 
        \left\{
          \begin{array}{crcrcr}
            & & & 0 & = & 0 
            \\[.2cm]
            - & y & - & z & = & 1
            \\[.2cm]
            & & & 0 & = & 0 
          \end{array}
        \right.
      \end{array}
      \]
      \conc{En prenant $x = 0$ et $y = 0$ (par exemple), on obtient
        $u_3 = (0, 0, -1)$.}
      \begin{remarkL}{.98}%
        On démontre ici que tous les vecteurs $(x, y, z)$ vérifiant
        $y+z = -1$ sont solutions de l'équation énoncée. Autrement
        dit, tous les vecteurs s'écrivant sous la forme $(x, -1 - z,
        z)$, où $x$ et $y$ sont des réels quelconques, sont
        solutions. On a choisi $x = 0$ et $z= -1$ (ou $y= 0$) par
        simplicité. Mais on aurait très bien pu faire d'autres choix
        comme : 
        \[
        (1, 0, -1), \ (1, -2, 1), \ (1, -3, 2), \ (0, -2, 1), \ (0,
        -1, 0) \ldots
        \]
      \end{remarkL}~\\[-1.5cm]
    \end{proof}
    

    \newpage


  \item Montrer que la famille $\B_1 = (u_1, u_2, u_3)$ est une base
    de $\R^3$.

    \begin{proof}~
      \begin{noliste}{$\sbullet$}
      \item Montrons que la famille $(u_1, u_2, u_3)$ est libre.\\
	Soit $(\lambda_1, \lambda_2, \lambda_3) \in \R^3$. Supposons :
        $\lambda_1 \cdot u_1 +\lambda_2 \cdot u_2 + \lambda_3 \cdot
        u_3 \ = \ 0_{\R^3}$.\\[.2cm]
	Les équivalences suivantes sont vérifiées.
	\[
	\begin{array}{cl@{\qquad}>{\it}R{3cm}}
          & \lambda_1 \cdot u_1 + \lambda_2 \cdot u_2 + \lambda_3 \cdot 
          u_3 = 0_{\R^3} 
          \\[.2cm]
          \Longleftrightarrow &  
          (\lambda_1, \ \lambda_2, \ -\lambda_2 - \lambda_3) \ = \ (0,
          0, 0)  
          \\[.2cm]
          \Longleftrightarrow &
          \left\{
            \begin{array}{rcrcrcl}
              \lambda_1 & & & & & = & 0 \\
              & & \lambda_2 & & & = & 0 \\
              & - & \lambda_2 & - & \lambda_3 & = & 0 
            \end{array}
          \right.
          \\[.8cm]            
          \begin{arrayEq}
            L_3 \leftarrow L_3 + L_2
          \end{arrayEq}
          &  
          \left\{
            \begin{array}{rcrcrcl}
              \lambda_1 & & & & & = & 0 \\
              & & \lambda_2 & & & = & 0 \\
              & & & - & \lambda_3 & = & 0 
            \end{array}
          \right.
          \\[.8cm]
          \Longleftrightarrow &  
          \left\{
            \lambda_1 = \lambda_2 = \lambda_3 = 0
          \right.
	\end{array}
	\]
	\conc{Ainsi, $\B_1 = (u_1, u_2, u_3)$ est une famille libre
          de $\R^3$.}
		
      \item On a alors :
	\begin{noliste}{$\stimes$}
        \item la famille $(u_1, u_2, u_3)$ est une famille libre,
        \item $\Card\big( (u_1, u_2, u_3) \big) = 3 = \dim(\R^3)$.
	\end{noliste}
      \end{noliste}
      \conc{Ainsi, $\B_1 = (u_1, u_2, u_3)$ est une base de
        $\R^3$.}
      \begin{remark}%~
        \begin{noliste}{$\sbullet$}
        \item Le terme {\bf cardinal} est réservé aux ensembles
          finis. La famille $(u_1, u_2, u_3)$ est un ensemble qui
          contient $3$ vecteurs. Elle est donc finie, de cardinal $3$
          (ce qu'on note $\Card((u_1, u_2, u_3)) = 3$).

        \item $\Vect{u_1, u_2, u_3}$ est l'espace vectoriel constitué
          de toutes les combinaisons linéaires des vecteurs $(u_1,
          u_2, u_3)$. C'est un ensemble {\bf infini} de vecteurs, on
          ne peut parler de son cardinal. Par contre, si l'on dispose
          d'une base $(u_1, u_2, u_3)$ d'un espace vectoriel, tout
          vecteur se décompose de manière unique sur cette base. Ceci
          permet de donner une représentation finie de cet ensemble
          infini.

        \item Les notations : $\bcancel{\Card\big( \Vect{u_1, u_2,
              u_3} \big)}$ et $\bcancel{\dim((u_1, u_2, u_3))}$ n'ont
          aucun sens !
        \end{noliste}
      \end{remark}~\\[-1.4cm]
    \end{proof}

  \end{noliste}
\end{noliste}
On admet que $\B_2 = (u_1, -u_2, u_3)$ est également une base de $\R^3$.
\begin{remarkL}{.78}%
  Ce résultat se démontre sans pein. En effet, la famille $\B_2 =
  (u_1, -u_2, u_3)$ est :
  \begin{noliste}{$\stimes$}
  \item génératrice de $\R^3$ puisque : $\Vect{ u_1, -u_2, u_3 } =
    \Vect{ u_1, u_2, u_3 } = \R^3$.
  \item de cardinal $\Card\big( (u_1, -u_2, u_3) \big) = 3 = \dim\big(
    \R^3 \big)$.
  \end{noliste}
  Ainsi, la famille $\B_2$ est bien une base de $\R^3$.  
\end{remarkL}


\newpage


\begin{noliste}{1.}
  \setlength{\itemsep}{4mm} %
  \setcounter{enumi}{6}
\item
  \begin{noliste}{a.}
    \setlength{\itemsep}{2mm} %
  \item Écrire la matrice $M_1$ de $f$ dans la base $\B_1$ et la
    matrice $M_2$ de $f$ dans la base $\B_2$.
    
    \begin{proof}~\\%
      Déterminons tout d'abord $M_1 = \Mat_{\B_1}(f)$.
      \begin{noliste}{$\sbullet$}
      \item $f(u_1) = u_1 = 1 \cdot u_1 + 0 \cdot u_2 + 0 \cdot u_3$.\\[.1cm]
        Ainsi : $\Mat_{\B}\big( f(u_1) \big) =
        \begin{smatrix}
          1 \\
          0 \\
          0
        \end{smatrix}
        $.
        
      \item $f(u_2) = u_2 = 0 \cdot u_1 + 1 \cdot u_2 + 0 \cdot u_3$.\\[.1cm]
        Ainsi : $\Mat_{\B}\big( f(u_2) \big) =
        \begin{smatrix}
          0 \\
          1 \\
          0
        \end{smatrix}
        $.
        
      \item $f(u_3) = u_2 + u_3 = 0 \cdot u_1 + 1 \cdot u_2 + 1 \cdot
        u_3$.\\[.1cm]
        Ainsi : $\Mat_{\B}\big( f(u_3) \big) =
        \begin{smatrix}
          0 \\
          1 \\
          1
        \end{smatrix}
        $.
      \end{noliste}
      \conc{Finalement : $M_1 = \Mat_{\B_1}(f) =
        \begin{smatrix}
          1 & 0 & 0 \\
          0 & 1 & 1 \\
          0 & 0 & 1
        \end{smatrix}
        $.}%~\\[-1cm]
      \begin{remarkL}{.98}%
        \begin{noliste}{$\sbullet$}
        \item Cette écriture permet de démontrer que $1$ est l'unique
          valeur propre de l'endomorphisme $f$. En effet, comme $M_1$
          est une matrice triangulaire supérieure, ses valeurs propres
          sont ses coefficients diagonaux. Ainsi :
          \[
          \spc(f) \ = \ \spc(M_1) \ = \ \{ 1 \}
          \]
          Ainsi, l'endomorphisme $f$ possède une unique valeur
          propre. On en déduit, en procédant par l'absurde, que $f$
          n'est pas diagonalisable. En effet, si on suppose $f$
          diagonalisable, $M_1 = \Mat_{\B_1}(f)$ l'est aussi. Il
          existe donc une matrice inversible $P \in \M{3}$ et une
          matrice diagonale $D \in \M{3}$ dont les coefficients
          diagonaux sont les valeurs propres de $M_1$ (ainsi $D =
          I_3$) telles que $M_1 = P \, D \, P^{-1} = P \, I_3 \,
          P^{-1} = I_3$. Ce qui est absurde.

        \item L'endomorphisme $f$ n'étant pas diagonalisable, on se
          rabat sur une propriété plus faible : existe-t-il une base
          dans laquelle la représentation matricielle de $f$ est
          triangulaire supérieure ? Cette propriété est beaucoup plus
          simple à obtenir notamment si l'on accepte d'utiliser des
          matrices dont les coefficients sont complexes (hors de
          portée en ECE).\\
          On parle alors de {\bf trigonaliser} (on dit aussi {\bf
            triangulariser}) l'endomorphisme $f$.

        \item Si un endomorphisme $f : E \to E$ est triangularisable,
          comment le triangularise-t-on ? \\
          Notons $\lambda_1$, \ldots, $\lambda_m$ les valeurs propres
          de $f$. On cherche alors une base de chaque sous-espace
          propre $E_{\lambda_i}$ et on considère la famille obtenue en
          concaténant toutes ces bases.\\
          Cette famille {\tt N'EST PAS} une base de $E$. Si tel était
          le cas, on aurait formé une base de vecteurs prores et donc
          $E$ serait diagonalisable. \\
          Par contre, cette famille est libre. On peut alors la
          compléter en une base de $E$.\\
          Sans entrer dans les détails, on peut faire en sorte (en
          choisissant correctement les vecteurs qu'on ajoute) que la
          matrice représentative de $f$ dans la base $\B'$ soit
          triangulaire supérieure.

        \item C'est la méthode développée dans cet énoncé. Ici, $f$
          n'a qu'une valeur propre. Le sous-espace propre $E_1(f)$ a
          pour base la famille $\big( u_1, u_2 \big)$. On complète
          alors cette famille en ajoutant $u_3$. La matrice
          représentant $f$ dans la base $\B_1$ ainsi formée est
          bien triangulaire supérieure.

%       \item On peut remarquer : 
%         \begin{noliste}{$\stimes$}
%         \item comme $f(e_2') = e_1'$ alors $f\big( f(e_2') \big) =
%           f(e_1') = 0_{\R^3}$.\\[.1cm]
%           Autrement dit : $e_2' \in \kr( f^2 )$.
%         \item comme $f(e_3') = e_2'$ alors $f^2\big( f(e_3') \big) =
%           f^2\big( e_2' \big) = f\big( f(e_2') \big) = f\big( e_1'
%           \big) = 0_{\R^3}$.\\[.1cm]
%           Autrement dit : $e_3' \in \kr( f^3 )$.          
%         \end{noliste}        
      \end{noliste}
    \end{remarkL}%~\\[-1.5cm]


    \newpage

    
    \noindent
      Déterminons maintenant $M_2 = \Mat_{\B_2}(f)$.
      \begin{noliste}{$\sbullet$}
      \item $f(u_1) = u_1 = 1 \cdot u_1 + 0 \cdot (-u_2) + 0 \cdot
        u_3$.\\[.1cm] 
        Ainsi : $\Mat_{\B}\big( f(u_1) \big) =
        \begin{smatrix}
          1 \\
          0 \\
          0
        \end{smatrix}
        $.
        
      \item $f(-u_2) = -u_2 = 0 \cdot u_1 + 1 \cdot (-u_2) + 0 \cdot
        u_3$.\\[.1cm]
        Ainsi : $\Mat_{\B}\big( f(-u_2) \big) =
        \begin{smatrix}
          0 \\
          1 \\
          0
        \end{smatrix}
        $.
        
      \item $f(u_3) = u_2 + u_3 = 0 \cdot u_1 - 1 \cdot (-u_2) + 1
        \cdot u_3$.\\[.1cm]
        Ainsi : $\Mat_{\B}\big( f(u_3) \big) =
        \begin{smatrix}
          0 \\
          -1 \\
          1
        \end{smatrix}
        $.
      \end{noliste}
      \conc{Finalement : $M_2 = \Mat_{\B_2}(f) =
        \begin{smatrix}
          1 & 0 & 0 \\
          0 & 1 & -1 \\
          0 & 0 & 1
        \end{smatrix}
        $.}~\\[-1.2cm]
    \end{proof}
    
  \item Justifier que les matrices $M_1$ et $M_2$ sont semblables, et
    calculer $M_1 M_2$.

    \begin{proof}~%
      \begin{noliste}{$\sbullet$}
      \item Les matrices $M_1$ et $M_2$ sont semblables car elles
        représentent le même endomorphisme $f$ dans des bases
        différentes.

      \item Par ailleurs : 
        \[
        M_1 \, M_2 \ = \ 
        \begin{smatrix}
          1 & 0 & 0 \\
          0 & 1 & 1 \\
          0 & 0 & 1
        \end{smatrix}
        \begin{smatrix}
          1 & 0 & 0 \\
          0 & 1 & -1 \\
          0 & 0 & 1
        \end{smatrix}
        \ = \ 
        \begin{smatrix}
          1 & 0 & 0 \\
          0 & 1 & 0 \\
          0 & 0 & 1
        \end{smatrix}
        \]
        \concL{Ainsi : $M_1 \, M_2 = I_3$. On en déduit que les
          matrices $M_1$ et $M_2$ sont inversibles et inverses l'une
          de l'autre.}{13.4}~\\[-1.4cm]
      \end{noliste}
    \end{proof}
  \end{noliste}

\item En déduire que les matrices $M$ et $M^{-1}$ sont semblables.

  \begin{proof}~\\%
    Notons $P = P_{\B, \B_1}$ et $Q = P_{\B, \B_2}$.
    \begin{noliste}{$\sbullet$}
    \item D'après la formule de changement de base :
      \[
      \begin{array}{ccccccc@{\quad}>{\it}R{3.5cm}}
        \Mat_{\B}(f) & = & P_{\B, \B_1} & \times & \Mat_{\B_1}(f) &
        \times & P_{\B_1, \B} 
        \\[.2cm]
        \shortparallel & & \shortparallel & & \shortparallel & &
        \shortparallel 
        \\[.2cm]
        M & = & P & \times & M_1 & \times & P^{-1}
      \end{array}
      \]
      \conc{On en déduit : $M^{-1} = P \, M_1^{-1} \, P^{-1}$.}

    \item Ainsi : 
      \[
      \begin{array}{rcl@{\qquad}>{\it}R{5cm}}
        M^{-1} & = & P \, M_1^{-1} \, P^{-1}
        \\[.2cm]
        & = & P \, M_2 \, P^{-1}
        & (d'après la question \itbf{7.b)})
        \nl
        \nl[-.2cm]
        & = & P \, \big( Q \, M \, Q^{-1} \big) \, P^{-1}
        & (d'après la question \itbf{7.a)})
        \nl
        \nl[-.2cm]
        & = & \big(P \, Q \big) \, M \, \big( P \, Q \big)^{-1}
      \end{array}
      \]
    \end{noliste}
    \conc{Les matrices $M$ et $M^{-1}$ sont semblables.}~\\[-1cm]
  \end{proof}
\end{noliste}


\newpage


\subsection*{PARTIE C : Troisième exemple}

\noindent 
On considère la matrice $T$ de $\M{3}$ définie par : $T =
\begin{smatrix}
  1 & -1 & 1 \\
  0 & 1 & -1 \\
  0 & 0 & 1
\end{smatrix}
$.\\
On note $I_3$ la matrice identité de $\M{3}$ et on pose : $N = T -
I_3$.

\begin{noliste}{1.}
  \setlength{\itemsep}{4mm} %
  \setcounter{enumi}{8}
\item Justifier que la matrice $T$ est inversible. Est-elle
  diagonalisable ?

  \begin{proof}~%
    \begin{noliste}{$\sbullet$}
    \item La matrice $T$ est triangulaire supérieure. Ses valeurs
      propres sont donc ses coefficients diagonaux. %
      \conc{$\spc(T) = \{ 1 \}$}

    \item Le réel $0$ n'est pas valeur propre de $T$. %
      \conc{On en déduit que $T$ est inversible.}

    \item Démontrons que $T$ n'est pas diagonalisable. On procède par
      l'absurde.\\
      Supposons que $T$ est diagonalisable.\\
      Il existe donc une matrice inversible $P \in \M{3}$ et une
      matrice diagonale $D \in \M{3}$ dont les coefficients diagonaux
      sont les valeurs propres de $T$ telles que $T = PDP^{-1}$.\\
      Or $1$ est la seule valeur propre de $T$. Ainsi $D = I_3$ et :
      \[
      T \ = \ P \, I_3 \, P^{-1} \ = \ P \, I_3 \, P^{-1} \ = \ I_3
      \]
      Absurde !%      
      \conc{La matrice $T$ n'est pas diagonalisable.}~\\[-1.6cm]
    \end{noliste}
  \end{proof}

\item 
  \begin{noliste}{a.}
    \setlength{\itemsep}{2mm} %
  \item Calculer $N^3$ puis $(I_3 + N) \, (I_3 - N + N^2)$.

    \begin{proof}~%
      \begin{noliste}{$\sbullet$}
      \item Tout d'abord :
        \[
        \begin{array}{rcl}
          N^3 & = & (T - I_3)^3 \ = \
          \begin{smatrix}
            0 & -1 & 1 \\
            0 & 0 & -1 \\
            0 & 0 & 0
          \end{smatrix}
          \begin{smatrix}
            0 & -1 & 1 \\
            0 & 0 & -1 \\
            0 & 0 & 0
          \end{smatrix}
          \begin{smatrix}
            0 & -1 & 1 \\
            0 & 0 & -1 \\
            0 & 0 & 0
          \end{smatrix}
          \\[.8cm]
          & = & 
          \begin{smatrix}
            0 & 0 & 1 \\
            0 & 0 & 0 \\
            0 & 0 & 0
          \end{smatrix}
          \begin{smatrix}
            0 & -1 & 1 \\
            0 & 0 & -1 \\
            0 & 0 & 0
          \end{smatrix}
          \ = \ 
          \begin{smatrix}
            0 & 0 & 0 \\
            0 & 0 & 0 \\
            0 & 0 & 0
          \end{smatrix}          
        \end{array}
        \]
        \conc{Ainsi : $N^3 = 0_{\M{3}}$.}

      \item D'autre part :
        \[
        \begin{array}{rcl@{\qquad}>{\it}R{5cm}}
          (I_3 + N) \, (I_3 - N + N^2) & = & (I_3 - \bcancel{N} +
          \xcancel{N^2}) + (\bcancel{N} - \xcancel{N^2} + N^3) 
          \\[.2cm]
          & = & I_3 - N^3 \ = \ I_3
          & (car $N^3 = 0_{\M{3}}$)
        \end{array}
        \]
        \conc{$(I_3 + N) \, (I_3 - N + N^2) = I_3$}~\\[-1.6cm]
      \end{noliste}
    \end{proof}

  \item En déduire une expression de $T^{-1}$ en fonction de $I_3$,
    $N$ et $N^2$.

    \begin{proof}~\\%
      Par définition : $T = I_3 + N$. Ainsi, d'après ce qui précède :
      \[
      T \, (I_3 - N + N^2) = I_3
      \]
      \conc{On en déduit que $T$ est inversible d'inverse $T^{-1} =
        I_3 - N + N^2$.}~\\[-1.2cm]
    \end{proof}
  \end{noliste}


  \newpage


\item On note $g$ l'endomorphisme de $\R^3$ dont la matrice dans la
  base canonique est $N$.
  \begin{noliste}{a.}
    \setlength{\itemsep}{2mm} %
  \item Justifier qu'il existe un vecteur $u$ de $\R^3$ tel que : $g
    \circ g (u) \neq 0_{\R^3}$ \ et \ $g \circ g \circ g (u) =
    0_{\R^3}$.

    \begin{proof}~%
      \begin{noliste}{$\sbullet$}
      \item Tout d'abord, d'après la question \itbf{10.a)} : $N^2 =
        \begin{smatrix}
          0 & 0 & 1 \\
          0 & 0 & -1 \\
          0 & 0 & 0
        \end{smatrix}
        \ \neq \ 0_{\M{3}}$.

      \item Or : 
        \[
        \begin{array}{rcl@{\qquad}>{\it}R{5cm}}
          N^2 \neq 0_{\M{3}} & \Leftrightarrow & g^2 \neq
          0_{\LL{\R^3}}
          & (par isomorphisme \\ de représentation)
          \nl
          \nl[-.2cm]
          & \Leftrightarrow & \exists u \in \R^3, \ g^2(u) \neq
          0_{\R^3} 
          \nl
          \nl[-.2cm]
          & \Leftrightarrow & \exists u \in \R^3, \ g \circ g(u) \neq
          0_{\R^3} 
        \end{array}
        \]
        \conc{Ainsi, il existe $u \in \R^3$ tel que : $g \circ g(u) =
          0_{\R^3}$.}

      \item Enfin, comme $N^3 = 0_{\M{3}}$, on a : $g^3 =
        0_{\LL{\R^3}}$ (toujours par isomorphisme de représentation).%
        \conc{On en déduit : $g \circ g \circ g(u) =
          0_{\R^3}$.}~\\[-1.4cm] 
      \end{noliste}
    \end{proof}

  \item Montrer que la famille $\B_3 = (g \circ g(u), g(u), u)$ est
    une base de $\R^3$.

    \begin{proof}~%
      \begin{noliste}{$\sbullet$}
      \item Montrons que la famille $(g \circ g(u), g(u), u)$ est libre.\\
	Soit $(\lambda_1, \lambda_2, \lambda_3) \in \R^3$.\\
        Supposons : $\lambda_1 \cdot g \circ g(u) +\lambda_2 \cdot
        g(u) + \lambda_3 \cdot u \ = \ 0_{\R^3}$ $(*)$.
        \begin{noliste}{$\stimes$}
        \item Par linéarité de $g^2$, on obtient, en appliquant $g^2$ de
          part et d'autre de l'égalité :
          \[
          \begin{array}{ccc}
            \lambda_1 \cdot \bcancel{g^4(u)} + \lambda_2 \cdot
            \bcancel{g^3(u)} + \lambda_3 \cdot g^{2}(u) & = & g^2\big(
            0_{\R^3} \big) 
            \\[.2cm]
            \shortparallel & & \shortparallel
            \\[.2cm]
            \lambda_3 \cdot g^2(u) & & 0_{\R^3}
          \end{array}
          \]
          En effet, comme $g^3 = 0_{\LL{\R^3}}$ alors $g^4 = g \circ
          0_{\LL{\R^3}} = 0_{\LL{\R^3}}$.\\[.1cm]
          Et en particulier : $g^3(u) = 0_{\R^3}$ et $g^4(u) =
          0_{\R^3}$. %
          \conc{Comme $\lambda_1 \cdot g^2(u) = 0_{\R^3}$ \ et \
            $g^2(u) \neq 0_{\R^3}$, alors : $\lambda_3 = 0$.}

        \item L'égalité $(*)$ se réécrit alors : $\lambda_1 \cdot
          g^2(u) + \lambda_2 \cdot g(u) = 0_{\R^3}$.\\
          En appliquant $g$ de part et d'autre, on obtient alors :
          \[
          \lambda_1 \cdot \bcancel{g^3(u)} + \lambda_2 \cdot g^2(u) =
          0_{\R^3}
          \]
          \conc{On en déduit, comme ci-dessus : $\lambda_2 = 0$.}

        \item L'égalité $(*)$ se réécrit alors : $\lambda_1 \cdot
          g^{2}(u) = 0_{\R^3}$.%
          \conc{On en conclut une nouvelle fois : $\lambda_3 = 0$.}
        \end{noliste}
        \conc{Ainsi, la famille $(g^{2}(u), g(u), u)$ est bien libre.}


        \newpage


      \item La famille ${\cal F} = (g^{2}(u), g(u), u)$ est :
        \begin{noliste}{$\stimes$}
        \item libre.
        \item de cardinal : $\Card({\cal F}) = 3 = \dim(\R^3)$.
        \end{noliste}
        \conc{Ainsi, la famille ${\cal F}$ est une base de
          $\R^3$.}~\\[-1.6cm]
      \end{noliste}
    \end{proof}

  \item Écrire la matrice de $g$ dans la base $\B_3$.

    \begin{proof}~%
      \begin{noliste}{$\sbullet$}
      \item $g\big( g^2(u) \big) = g^3(u) = 0_{\R^3} = 0 \cdot g^2(u)
        + 0 \cdot g(u) + 0 \cdot u$.\\[.1cm] 
        Ainsi : $\Mat_{\B_3}\Big( g\big( g^2(u) \big) \Big) =
        \begin{smatrix}
          0 \\
          0 \\
          0
        \end{smatrix}
        $.

      \item $g\big( g(u) \big) = g^2(u) = 1 \cdot g^2(u) + 0 \cdot
        g(u) + 0 \cdot u$.\\[.1cm] 
        Ainsi : $\Mat_{\B_3}\Big( g\big( g(u) \big) \Big) =
        \begin{smatrix}
          1 \\
          0 \\
          0
        \end{smatrix}
        $.

      \item $g\big( u \big) = g(u) = 0 \cdot g^2(u) + 1 \cdot g(u) + 0
        \cdot u$.\\[.1cm] 
        Ainsi : $\Mat_{\B_3}\big( g(u) \big) =
        \begin{smatrix}
          0 \\
          1 \\
          0
        \end{smatrix}
        $.
      \end{noliste}
      \conc{Finalement : $\Mat_{\B_3}(g) =
        \begin{smatrix}
          0 & 1 & 0 \\
          0 & 0 & 1 \\
          0 & 0 & 0
        \end{smatrix}
        $.}~\\[-1.2cm]
    \end{proof}

  \item Calculer $N^2 - N$ et en déduire que les matrices $N$ et $N^2
    - N$ sont semblables.

    \begin{proof}~%
      \begin{noliste}{$\sbullet$}
      \item Tout d'abord :
        \[
        N^2 - N \ = \
        \begin{smatrix}
          0 & 0 & 1 \\
          0 & 0 & -1 \\
          0 & 0 & 0
        \end{smatrix}
        -
        \begin{smatrix}
          0 & -1 & 1 \\
          0 & 0 & -1 \\
          0 & 0 & 0
        \end{smatrix}
        \ = \
        \begin{smatrix}
          0 & 1 & 0 \\
          0 & 0 & 1 \\
          0 & 0 & 0
        \end{smatrix}        
        \]
        \conc{Ainsi : $N^2 - N = \Mat_{\B_3}(g)$.}

      \item Or, d'après la formule de changement de base :
        \[
        \begin{array}{R{2cm}ccccccc@{\quad}>{\it}R{5.5cm}}
          & \Mat_{\B}(g) & = & P_{\B, \B_3} & \times & \Mat_{\B_3}(g) &
          \times & P_{\B_3, \B} 
          \\[.2cm]
          & \shortparallel & & \shortparallel & & \shortparallel & &
          \shortparallel 
          \\[.2cm]
          & N & = & R & \times & (N^2 - N) & \times & R^{-1}
          & (en notant $R = P_{\B, \B_3}$)
        \end{array}
        \]
        \conc{Ainsi, les matrices $N$ et $N^2 - N$ sont
          semblables.}~\\[-1.5cm]
      \end{noliste}
    \end{proof}
  \end{noliste}

\item Montrer que les matrices $T$ et $T^{-1}$ sont semblables.

  \begin{proof}~\\%
    D'après la question précédente :
    \[
    \begin{array}{rcl@{\qquad}>{\it}R{5cm}}
      R \, T^{-1} \, R^{-1} & = & R \, \big( I_3 - N + N^2 \big) \,
      R^{-1}
      & (d'après la question \itbf{10.a)})
      \nl
      \nl[-.2cm]
      & = & R \, I_3 \, R^{-1} + R \, \big(N^2 - N \big) \, R^{-1}
      % \\[.2cm]
      \ = \ I_3 + N \ = \ T
    \end{array}
    \]
    \conc{Ainsi, $T = R \, T^{-1} \, R^{-1}$ et les matrices $T$ et
      $T^{-1}$ sont semblables.}~\\[-1.2cm]
  \end{proof}
\end{noliste}


\newpage


\section*{EXERCICE 3} %

\noindent %
On considère la fonction $f$ définie sur $]0,+\infty[$ par :
\[
  \forall t \in \ ]0,+\infty[, \ f(t) = t + \dfrac{1}{t}
\]

\subsection*{PARTIE A : \'Etude d'une fonction d'une variable} %

\begin{noliste}{1.}
  \setlength{\itemsep}{4mm}
\item \'Etudier les variations de la fonction $f$ sur $]0,+\infty[$.\\
  Dresser le tableau de variations de $f$ en précisant les limites en
  $0$ et $+\infty$.
  \begin{proof}~
    \begin{noliste}{$\sbullet$}
    \item La fonction $f$ est dérivable sur $]0,+\infty[$ en tant que
      somme $f_1+f_2$ avec :
      \begin{noliste}{$\stimes$}
      \item $f_1 : t \mapsto t$ dérivable sur $]0,+\infty[$ en tant
        que fonction polynomiale,
        
      \item $f_2 : t \mapsto \dfrac{1}{t}$ dérivable sur $]0,+\infty$
        en tant qu'inverse d'une fonction polynomiale qui ne s'annule
        pas sur cet intervalle.
      \end{noliste}
      
    \item Soit $t \in \ ]0,+\infty[$.
      \[
        f'(t) \ = \ 1 - \dfrac{1}{t^2}
      \]
      De plus :
      \[
        \begin{array}{rcl@{\quad}>{\it}R{5cm}}
          f'(t) \geq 0
          & \Leftrightarrow & 1-\dfrac{1}{t^2} \geq 0
          \\[.6cm]
          & \Leftrightarrow & 1 \geq \dfrac{1}{t^2}
          \\[.6cm]
          & \Leftrightarrow & 1 \leq t^2
          & (par stricte décroissance de la fonction $t \mapsto
            \dfrac{1}{t}$ sur $]0,+\infty[$)
          \nl
          \nl[-.2cm]
          & \Leftrightarrow & 1 \leq t
          & (par stricte croissance de la fonction $t \mapsto
            \sqrt{t}$ sur $[0,+\infty[$)                               
        \end{array}
      \]
      
    \item On obtient alors le tableau de variations suivant :\\[-.2cm]
      \begin{center}
        \begin{tikzpicture}[scale=0.8, transform shape]
          \tkzTabInit[lgt=4,espcl=3] 
          {$t$ /1, Signe de $f'(t)$ /1, Variations de $f$
          /2} 
          {$0$, $1$, $+\infty$}%
          \tkzTabLine{ , - ,z, + , } 
          \tkzTabVar{+/$+\infty$, -/$2$, +/$+\infty$}
        \end{tikzpicture}
      \end{center}
      
    \item Détaillons les éléments de ce tableau :
      \begin{noliste}{$\stimes$}
      \item $f(1) \ = \ 1+ \dfrac{1}{1} \ = \ 2$.
        
      \item comme $\dlim{t\to 0^+} \dfrac{1}{t} = +\infty$, alors :
        $\dlim{t\to 0^+} f(t) = +\infty$.
        
      \item comme $\dlim{t\to +\infty} \dfrac{1}{t} = 0$, alors :
        $\dlim{t\to +\infty} f(t) = +\infty$.
      \end{noliste}
    \end{noliste}
  \end{proof}


  \newpage
  
  
\item Montrer que $f$ réalise une bijection de $[1,+\infty[$ vers
  $[2,+\infty[$.
  \begin{proof}~\\
    D'après la question précédent, la fonction $f$ est :
    \begin{noliste}{$\stimes$}
    \item continue (car dérivable) sur $[1,+\infty[$,
      
    \item strictement croissante sur $[1,+\infty[$.
    \end{noliste}
    Ainsi, la fonction $f$ réalise une bijection de $[1,+\infty[$ sur
    $f\big([1, +\infty[\big)$ avec :
    \[
      f\big([1, +\infty[\big) \ = \ \left[ f(1), \dlim{t\to +\infty}
        f(t) \right[ \ = \ [2,+\infty[
    \]
    \conc{Finalement, la fonction $f$ réalise une bijection de
      $[1,+\infty[$ sur $[2,+\infty[$.}~\\[-1cm]
  \end{proof}
\end{noliste}
On note $g : [2,+\infty[ \ \to [1,+\infty[$ la bijection réciproque de
la restriction de $f$ à $[1,+\infty[$.
\begin{noliste}{1.}
  \setlength{\itemsep}{4mm}
  \setcounter{enumi}{2}
\item
  \begin{noliste}{a)}
    \setlength{\itemsep}{2mm}
  \item Dresser le tableau de variations de $g$.
    \begin{proof}~\\
      D'après le théorème de la bijection, la fonction $g$ est
      continue sur $[2,+\infty[$ et strictement monotone sur
      $[2,+\infty[$, de même sens de variation que $f$ sur
      $[1,+\infty[$.\\
      On obtient alors le tableau de variations suivant :\\[-.2cm]
      \begin{center}
        \begin{tikzpicture}[scale=0.8, transform shape]
          \tkzTabInit[lgt=4,espcl=3] 
          {$t$ /1, Variations de $g$ /2} 
          {$2$, $+\infty$}% 
          \tkzTabVar{-/$1$, +/$+\infty$}
        \end{tikzpicture}
      \end{center}
    \end{proof}
    
  \item Justifier que la fonction $g$ est dérivable sur $]2,+\infty[$.
    \begin{proof}~\\
      La fonction $f$ :
      \begin{noliste}{$\stimes$}
      \item réalise une bijection de $]1,+\infty[$ sur $]2,+\infty[$,
        
      \item est dérivable sur $]1,+\infty[$ (d'après \itbf{1.}),
        
      \item est telle que : $\forall t \in \ ]1,+\infty[$, $f(t) \neq
        0$. En effet : $\forall t \in \ ]1,+\infty[$, $f(t) \geq 2 > 0$.
      \end{noliste}
      \conc{On en déduit que $g$ est dérivable sur $]2,+\infty[$.}
      \begin{remark}
        \begin{noliste}{$\sbullet$}
        \item On connaît même l'expression de $g'$ : $g' =
          \dfrac{1}{f' \circ g}$.
          
        \item On peut retrouver cette formule via l'égalité $f \circ g
          = \id$.\\
          En effet, en dérivant formellement cette égalité, on obtient
          :
          \[
            (f' \circ g) \times g' \ = \ 1
          \]
        \end{noliste}
      \end{remark}~\\[-1.4cm]
    \end{proof}


    \newpage
    
    
  \item Soit $y \in [2,+\infty[$.
    En se ramenant à une équation du second degré, résoudre l'équation
    $f(t) = y$ d'inconnue $t \in \ ]0,+\infty[$. En déduire une
    expression de $g(y)$ en fonction de $y$.
    \begin{proof}~\\
      Soit $y \in [2,+\infty[$. Soit $t \in \ ]0,+\infty[$.
      \begin{noliste}{$\sbullet$}
      \item Tout d'abord : 
        \[
          \begin{array}{rcl@{\quad}>{\it}R{5cm}}
            f(t) = y
            & \Leftrightarrow & t + \dfrac{1}{t} = y
            \\[.4cm]
            & \Leftrightarrow & t^2 + 1 = y \, t
            & (car $t \neq 0$)
            \nl
            \nl[-.2cm]
            & \Leftrightarrow & t^2 - y \, t +1 = 0
          \end{array}
        \]
        
      \item On note $P\in \R[X]$ le polynôme défini par : $P(X) = X^2 - y \, X +1$.
        \begin{noliste}{-}
        \item Calculons le discriminant du polynôme $P$ :
          \[
            \Delta \ = \ (-y)^2 - 4 \times 1 \times 1 \ = \ y^2 - 4
          \]
          Comme $y\geq 2$, alors, par croissance de la foncion $x
          \mapsto x^2$ sur $[0,+\infty[$ : $y^2 \geq 4$. Ainsi : $\Delta \geq 0$.
          
        \item Distinguons les cas suivant le nombre de racines de
          $P$. Deux cas se présentent alors :
        \end{noliste}
        \begin{liste}{$\stimes$}
        \item \dashuline{si $\Delta >0$} (\ie si $y>2$), alors $P$ admet exactement
          deux racines notées $r_1$ et $r_2$ :
          \[
            r_1 = \dfrac{-(-y) - \sqrt{\Delta}}{2 \times 1} =
            \dfrac{y- \sqrt{y^2 - 4}}{2} \quad \text{et} \quad r_2 =
            \dfrac{-(-y) + \sqrt{\Delta}}{2 \times 1} = \dfrac{y+
              \sqrt{y^2 -4}}{2}
          \]
          \conc{Si $y>2$, l'équation $f(t)=y$ admet exactement deux
            solutions :\\[.2cm] %
            $r_1 = \dfrac{y- \sqrt{y^2-4}}{2}$ \quad et \quad $r_2 = \dfrac{y+
              \sqrt{y^2-4}}{2}$.}
        \item \dashuline{si $\Delta = 0$} (\ie si $y=2$), alors $P$ admet exactement
          une racine notée $r_0$ :
          \[
            r_0 \ = \ \dfrac{-(-y)}{2 \times 1} \ = \ \dfrac{y}{2} \ =
            \ \dfrac{2}{2} \ = \ 1
          \]
          \conc{Si $y=2$, l'équation $f(t)=y$ admet une unique
            solution : $r_0 = 1$.}
        \end{liste}
      \begin{remark}
        Remarquons que, pour $y=2$, $r_1$ et $r_2$ prennent la valeur $1$. Les
        expressions de $r_0$, $r_1$ et $r_2$ coïncident donc en $y=2$.
      \end{remark}


      \newpage
      
      
      \item Comme $g$ est la bijection réciproque de la restriction de
        $f$ à $[1,+\infty[$, pour tout $y \in [2,+\infty[$ et tout $t
        \in [1,+\infty[$ :
        \[
          g(y) = t \quad \Leftrightarrow \quad f\big(g(y)\big) = f(t) \quad
          \Leftrightarrow \quad y = f(t)
        \]
        On en déduit que, pour tout $y \in [2,+\infty[$, $g(y)$ est
        la solution de l'équation $f(t)=y$ sur l'intervalle
        $[1,+\infty[$.


        % \newpage


        \noindent %
        Soit $y \in [2,+\infty[$. Deux cas se présentent donc :
        \begin{noliste}{$\stimes$}
        \item \dashuline{si $y>2$}, alors on cherche à déterminer
          quelle solution de $r_1$ ou de $r_2$ appartient à
          $[1,+\infty[$.
        \end{noliste}
        \begin{liste}{-}
        \item Tout d'abord :
          \[
            \begin{array}{rcl@{\quad}>{\it}R{6.5cm}}
              r_1 \geq 1
              & \Leftrightarrow & \dfrac{y- \sqrt{y^2-4}}{2} \geq 1
              \\[.4cm]
              & \Leftrightarrow & y-\sqrt{y^2-4} \geq 2
              \\[.4cm]
              & \Leftrightarrow & y-2 \geq \sqrt{y^2-4}
              \\[.4cm]
              & \Leftrightarrow & (y-2)^2 \geq y^2-4
              & (par stricte croissance de la fonction $x \mapsto x^2$
                sur $[0,+\infty[$, et $y \geq 2$)
              \nl
              \nl[-.2cm]
              & \Leftrightarrow & \bcancel{y^2}-4y+4 \geq
                                  \bcancel{y^2}-4
              \\[.4cm]
              & \Leftrightarrow & 8 \geq 4y \ \Leftrightarrow \ 2 \geq y
            \end{array}
          \]
          Or cette dernière inégalité est fausse (on a supposé $y>2$), donc, par
          équivalence : $r_1 < 1$.
        \item Ensuite : $\sqrt{y^2-4} \geq 0$. Donc, comme $y \geq 2$ : $y
          + \sqrt{y^2-4} \geq 2$.\\[.1cm] %
          Ainsi : $r_2 = \dfrac{y+ \sqrt{y^2-4}}{2} \geq 1$.
          \conc{On en déduit, pour tout $y \in \ ]2,+\infty[$, $g(y) =
            \dfrac{y + \sqrt{y^2-4}}{2}$.}
        \end{liste}

        \begin{noliste}{$\stimes$}
        \item \dashuline{si $y =2$}, alors $g(y) = r_0 = 1$.
          \conc{On en déduit : $g(2)=1$.}
        \end{noliste}.
        \conc{Comme les expressions de $r_0$ et de $r_2$ coïncident en
          $2$, on obtient finalement :\\[.2cm] %
          $g : y \mapsto \dfrac{y+ \sqrt{y^2-4}}{2}$.}~\\[-1.4cm]
      \end{noliste}
    \end{proof}
  \end{noliste}
\end{noliste}


\newpage


\subsection*{PARTIE B : Étude d'une fonction de deux variables} %

\noindent %
On considère la fonction $h$ de classe $\Cont{2}$ sur l'ouvert $U = \
]0,+\infty[ \, \times \, ]0,+\infty[$ définie par :
\[
\forall (x,y) \in U, \ h(x,y) = \left( \dfrac{1}{x} +
  \dfrac{1}{y}\right) (1+x) (1+y)
\]
\begin{noliste}{1.}
  \setlength{\itemsep}{4mm} %
  \setcounter{enumi}{3}
\item Calculer les dérivées partielles d'ordre $1$ de $h$ en tout
  $(x,y)$ de $U$.

  \begin{proof}~%
    \begin{noliste}{$\sbullet$}
    \item La fonction $h$ est de classe $\Cont{1}$ (car de classe
      $\Cont{2}$) sur $]0, +\infty[ \ \times \ ]0, +\infty[$. \\
      Elle admet donc des dérivées partielles premières sur cet
      ensemble.

    \item Soit $(x, y) \in \ ]0, +\infty[ \ \times \ ]0,
      +\infty[$. Tout d'abord : %
      \[
      \begin{array}{rcl@{\qquad}>{\it}R{5cm}}
        \dfn{h}{1}(x, y) & = & (1+y) \ \left( - \dfrac{1}{x^2} \ (1+x)
          + \left( \dfrac{1}{x} + \dfrac{1}{y}\right) \right)
        \\[.6cm]
        & = & (1+y) \ \left( - \dfrac{1}{x^2} - \bcancel{\dfrac{1}{x}}
          + \left( \bcancel{\dfrac{1}{x}} + \dfrac{1}{y} \right) \right) 
        \ = \ (1+y) \ \left( \dfrac{1}{y} - \dfrac{1}{x^2} \right)
      \end{array}      
      \]
      Par ailleurs :
      \[
      \dfn{h}{2}(x, y) \ = \ (1+x) \ \left( - \dfrac{1}{y^2} \ (1+y) +
        \left( \dfrac{1}{x} + \dfrac{1}{y}\right) \right) \ = \ (1+x)
      \ \left( \dfrac{1}{x} - \dfrac{1}{y^2} \right)
      \]
      \conc{$\dfn{h}{1}(x, y) \ = \ (1+y) \ \left( \dfrac{1}{y} -
          \dfrac{1}{x^2} \right)$ \quad et \quad $\dfn{h}{2}(x, y) \ =
        \ (1+x) \ \left( \dfrac{1}{x} - \dfrac{1}{y^2} \right)$}
    \end{noliste}
    \begin{remarkL}{.98}%
      \begin{noliste}{$\sbullet$}
      \item Il faut noter que l'expression $(1+y)$ est une constante
        par rapport à $x$. Ainsi, lorsque l'on dérive $h$ par rapport
        à la première variable, on met en facteur cette constante
        $(1+y)$ et on dérive le produit $\left( \dfrac{1}{x} +
          \dfrac{1}{y}\right) (1+x)$.

      \item Dans l'expression de $h$, les variables $x$ et $y$ jouent
        des rôles symétriques.\\
        Plus précisément : $\forall (x, y) \in U, \ h(x, y) = h(y,
        x)$. On peut alors en déduire :
        \[
        \forall (x, y) \in U, \ \dfn{h}{2}(x, y) \ = \ \dfn{h}{1}(y, x)
        \]
        {\it (ce qu'on a retrouvé par le calcul)}\\[.1cm]
        On peut démontrer ce résultat en remarquant :
        \[
        \dfn{h}{2}(x, y) \ = \ \dlim{h \tend 0} \dfrac{f(x, y + h) -
          f(x, y)}{h} \ = \ \dlim{h \tend 0} \dfrac{f(y + h, x) - f(y,
          x)}{h} \ = \ \dfn{h}{1}(y, x)
        \]

      \item Ce cas est relativement fréquent lors de l'étude de
        fonction de deux variables et le résultat qui en découle est
        plutôt naturel. Même si ce résultat ne figure pas
        explicitement dans le programme, il est conseillé de le
        connaître. On peut alors concevoir ce résultat comme un
        procédé de vérification permettant de vérifier que le deuxième
        calcul effectué est juste.
      \end{noliste}
    \end{remarkL}~\\[-1.4cm]
  \end{proof}
  

  \newpage


\item Soit $(x,y) \in U$. Montrer :
  \[
  \text{$(x,y)$ est un point critique de $h$} \ \Leftrightarrow \
  \left\{
    \begin{array}{l}
      y \ = \ x^2 \\[.1cm]
      x \ = \ y^2
    \end{array}
  \right.
  \]
  
  \begin{proof}~%
    \begin{noliste}{$\sbullet$}
    \item Rappelons tout d'abord :
      \[
      \text{$(x, y)$ est un point critique de $h$} \ \Leftrightarrow \
      \nabla(h)(x, y) = 0_{\M{2,1}} \ \Leftrightarrow \ \left\{
        \begin{array}{l}
          \dfn{h}{1}(x, y) \ = \ 0
          \\[.2cm]
          \dfn{h}{2}(x, y) \ = \ 0
        \end{array}
      \right.
      \]
      % Ainsi :

    \item
      $
      \begin{array}[t]{R{.8cm}C{3.4cm}cl@{\qquad}>{\it}R{6.8cm}}
        Ainsi : & $(x, y)$ est un point critique de $h$ 
        & \Leftrightarrow & 
        \multicolumn{2}{l}{
        \left\{
          \begin{array}{rcl} 
            (1+y) \ \left( \dfrac{1}{y} - \dfrac{1}{x^2} \right) & = &
            0 
            \\[.5cm]
            (1+x) \ \left( \dfrac{1}{x} - \dfrac{1}{y^2} \right) & = &
            0 
          \end{array}
        \right.
        }
        \\[1.2cm]
        & & \Leftrightarrow & 
        \left\{
          \begin{array}{rcl} 
            \left( \dfrac{1}{y} - \dfrac{1}{x^2} \right) & = & 0 
            \\[.5cm]
            \left( \dfrac{1}{x} - \dfrac{1}{y^2} \right) & = & 0
          \end{array}
        \right.
        & (en divisant la $\ere{1}$ équation par $(1+y) \neq 0$ et la
        $\eme{2}$ par $(1+x) \neq 0$)
        \nl
        \nl[-.2cm]
        & & \Leftrightarrow & 
        \left\{
          \begin{array}{rcl} 
            \dfrac{1}{y} & = & \dfrac{1}{x^2}
            \\[.5cm]
            \dfrac{1}{x} & = & \dfrac{1}{y^2}
          \end{array}
        \right.
        \\[1.2cm]
        & & \Leftrightarrow & 
        \left\{
          \begin{array}{rcl}
            x^2 & = & y
            \\[.1cm]
            y^2 & = & x
          \end{array}
        \right.
        & (en multipliant la $\ere{1}$ équation par $x^2 \, y \neq 0$ et la
        $\eme{2}$ par $x \, y^2 \neq 0$)
      \end{array}
      $ %\]
      \conc{On a bien : $\text{$(x,y)$ est un point critique de $h$} \
        \Leftrightarrow \ \left\{
          \begin{array}{l}
            y \ = \ x^2 \\[.1cm]
            x \ = \ y^2
          \end{array}
        \right.$.}
    \end{noliste}
    \begin{remarkL}{.98}%~%
      \begin{noliste}{$\sbullet$}
      \item On a démontré ci-dessus, à l'aide d'un calcul :
        $\dfrac{1}{y} \ = \ \dfrac{1}{x^2} \ \ \Leftrightarrow \ \ y \
        = \ x^2$.\\
        On peut aussi démontrer cette équivalence en argumentant par
        le caractère bijectif de la fonction $x \mapsto \frac{1}{x}$
        sur $]0, +\infty[$.
        
      \item La difficulté de la recherche de points critiques réside
        dans le fait qu'il n'existe pas de méthode générale pour
        résoudre l'équation $\nabla(h)(x,y) = 0_{\M{2,1}}$.\\
        On est donc confronté à une question bien plus complexe qu'une
        résolution de système d'équations linéaires (que l'on résout
        aisément à l'aide de la méthode du pivot de Gauss).

      \item Lors de la recherche de points critiques, on doit faire
        appel à des méthodes ad hoc. Ici, on fait apparaître une
        équation du type :
        \[
        y = \psi(x)
        \]
        En injectant cette égalité dans la seconde équation, on
        obtient une nouvelle équation qui ne dépend plus que d'une
        variable et qu'il est donc plus simple de résoudre.\\
        C'est la stratégie qu'on adopte ci-dessous.
      \end{noliste}
    \end{remarkL}~\\[-1.4cm]
  \end{proof}


  \newpage


\item En déduire que $h$ admet un unique point critique sur $U$ dont
  on précisera les coordonnées $(a,b)$.

  \begin{proof}~\\%
    D'après ce qui précède :
    \[
    \begin{array}{C{2cm}C{3.4cm}cl@{\qquad}>{\it}R{4.8cm}}
      & $(x, y) \in U$ est un point critique de $h$ 
      & \Leftrightarrow & 
      \left\{
        \begin{array}{rcl}
          x^2 & = & y
          \\[.1cm]
          y^2 & = & x
        \end{array}
      \right.
      \\[.8cm]
      & & \Leftrightarrow & 
      \left\{
        \begin{array}{rcl}
          x^2 & = & y
          \\[.1cm]
          x^4 & = & x
        \end{array}
      \right.
      \\[.8cm]
      & & \Leftrightarrow & 
      \left\{
        \begin{array}{rcl}
          x^2 & = & y
          \\[.1cm]
          x^3 & = & 1
        \end{array}
      \right.
      & (en divisant la $\eme{2}$ \\ équation par $x \neq 0$)
      \nl
      \nl[-.2cm]
      & & \Leftrightarrow & 
      \left\{
        \begin{array}{rcl}
          x^2 & = & y
          \\[.1cm]
          x^3 & = & 1^3
        \end{array}
      \right.
      \\[.8cm]
      & & \Leftrightarrow & 
      \left\{
        \begin{array}{rcl}
          x^2 & = & y
          \\[.1cm]
          x & = & 1
        \end{array}
      \right.
      & (par bijectivité de la fonction $x \mapsto x^3$)
      \nl
      \nl[-.2cm]
      & & \Leftrightarrow & 
      \left\{
        \begin{array}{rcl}
          1 & = & y
          \\[.1cm]
          x & = & 1
        \end{array}
      \right.
      & (en remplaçant $x$ par $1$ \\ dans la $\ere{1}$ équation)
    \end{array}
    \]
    \conc{Ainsi, $h$ admet pour unique point critique $(1, 1)$.}~\\[-1.2cm]
  \end{proof}

\item
  \begin{noliste}{a)}
    \setlength{\itemsep}{2mm}
  \item Vérifier : $\forall (x,y) \in U$, $h(x, y) = 2 + f(x) + f(y) +
    f\left( \dfrac{x}{y} \right)$.

    \begin{proof}~\\%
      Soit $(x, y) \in U$.
      \[
      \begin{array}{R{2cm}rcl@{\qquad}>{\it}R{3cm}}
        & h(x, y) & = & \left( \dfrac{1}{x} + \dfrac{1}{y} \right) \,
        (1+x) \, (1+y)
        \\[.6cm]
        & & = & \left( \left( \dfrac{1}{x} + \dfrac{1}{y} \right) \,
          + 1 + \dfrac{x}{y} \right) \, (1+y)
        \\[.6cm]
        & & = & \left( \left( \dfrac{1}{x} + \dfrac{1}{y} \right) \,
          + 1 + \dfrac{x}{y} \right) + \left( \left( \dfrac{y}{x} +
            1 \right) \, + y + x \right)
        \\[.6cm]
        & & = & 2 + \left( x + \dfrac{1}{x} \right) + \left( y +
          \dfrac{1}{y} \right) + \left( \dfrac{x}{y} + \dfrac{y}{x}
        \right) 
        & (en réordonnant)
        \nl
        \nl[-.2cm]
        & & = &  
        2 + f(x) + f(y) + f\left( \dfrac{x}{y} \right)
      \end{array}
      \]
      \conc{$\forall (x,y) \in U$, $h(x, y) = 2 + f(x) + f(y) +
        f\left( \dfrac{x}{y} \right)$}~\\[-1cm]
    \end{proof}
    
    
    \newpage


  \item En déduire que $h$ admet en $(a,b)$ un minimum global sur $U$.

    \begin{proof}~%
      \begin{noliste}{$\sbullet$}
      \item Tout d'abord :
        \[
        h(1, 1) \ = \ \left( \dfrac{1}{1} + \dfrac{1}{1} \right) \,
        (1+1) \, (1+1) \ = \ 2 \times 2 \times 2 \ = \ 8
        \]
        \conc{$h(1, 1) = 8$}

      \item Rappelons, d'après la question \itbf{1.}, que la fonction $f$ :
        \begin{noliste}{$\stimes$}
        \item est décroissante sur $]0, 1[$.
        \item croissante sur $[1, +\infty[$.          
        \end{noliste}
        On en déduit que $f$ admet pour minimum $f(1) = 2$.%
        \conc{Autrement dit : $\forall t \in \ ]0, +\infty[$, $f(t)
          \geq 2$.}

      \item Soit $(x, y) \in U$. D'après la question précédente :
        \[
        h(x, y) \ = \ 2 + f(x) + f(y) + f\left( \dfrac{x}{y} \right)
        \ \geq \ 2 + 2 + 2 + 2 \ = \ 8 \ = \ h(1, 1)
        \]
        Cette minoration est obtenue en appliquant le résultat
        précédent successivement en $t = x > 0$, $t = y > 0$ et $t =
        \frac{x}{y} > 0$. %
        \conc{On en conclut : $\forall (x, y) \in U$, $h(1, 1) \ \leq
          \ h(x, y)$.\\
        Ainsi, $h$ admet un minimum global sur $U$ au point $(1,
        1)$.}~\\[-1.4cm] 
      \end{noliste}
    \end{proof}
  \end{noliste}
\end{noliste}


\subsection*{PARTIE C : \'Etude d'une suite} %

\noindent %
On introduit la suite $(u_n)_{n \in \N^*}$ définie par :
\[
  u_1 = 1 \quad \text{et} \quad \forall n \in \N^*, \ u_{n+1} = u_n +
  \dfrac{1}{n^2 \, u_n} = \dfrac{1}{n} \, f(n \, u_n)
\]
\begin{noliste}{1.}
  \setlength{\itemsep}{4mm}
  \setcounter{enumi}{7}
\item Montrer que, pour tout $n$ de $\N^*$, $u_n$ existe et $u_n \geq
  1$.
  \begin{proof}~\\ %
    Démontrons par récurrence : $\forall n \in \N^*$, $\PP{n}$ \quad
    où \quad $\PP{n} : \left\{
      \begin{array}{l}
        \text{$u_n$ existe}\\[.2cm]
        u_n \geq 1
      \end{array}
    \right. $.
    \begin{noliste}{\fitem}
    \item {\bf Initialisation} :\\ %
      D'après l'énoncé : $u_1 = 1$. Donc : $u_1 \geq 1$.\\ %
      D'où $\PP{1}$.
      
    \item {\bf Hérédité} : soit $n \in \N^*$.\\
      Supposons $\PP{n}$ et démontrons $\PP{n+1}$ (\ie $\left\{
        \begin{array}{l}
          \text{$u_{n+1}$ existe}\\[.2cm]
          u_{n+1} \geq 1
        \end{array}
      \right.$)\\[.1cm] %
      Par hypothèse de récurrence, le réel $u_n$ existe et $u_n \geq
      1$.
      \begin{noliste}{$\sbullet$}
      \item Comme $u_n \geq 1 >0$ et $n>0$, alors : $n \, u_n >0$.\\ %
        Or $f$ est définie sur $]0,+\infty[$, donc le réel $f(n \,
        u_n)$ existe.\\ %
        On en déduit que $u_{n+1}$ existe.


        \newpage
        
        
      \item Ensuite :
        \[
          \begin{array}{R{1cm}c@{\quad}>{\it}R{6cm}}
            & u_n \ \geq \ 1
            \\[.2cm]
            donc & n \, u_n \ \geq \ n
            & (car $n >0$)
            \nl
            \nl[-.2cm]
            d'où & f(n \, u_n) \ \geq \ f(n)
            & (par croissance de $f$ sur $[1,+\infty[$,\\
              et $n \, u_n \geq n \geq 1$)
            \nl
            \nl[-.2cm]
            ainsi & \dfrac{1}{n} \ f(n \, u_n) \ \geq \ \dfrac{1}{n} \
                    f(n)
            & (car $\dfrac{1}{n} >0$)                    
          \end{array}
        \]
        On en déduit : $u_{n+1} \geq \dfrac{1}{n} \ f(n)$. Or :
        \[
          \dfrac{1}{n} \ f(n) \ = \ \dfrac{1}{n}\left(n +
            \dfrac{1}{n}\right) \ = \ 1 + \dfrac{1}{n^2} \ \geq \ 1
        \]
        Ainsi, par transitivité : $u_{n+1} \ \geq \ \dfrac{1}{n} \
        f(n) \ \geq \ 1$.\\
        D'où $\PP{n+1}$.
      \end{noliste}
    \end{noliste}
    \conc{Par principe de récurrence, on en déduit que, pour tout $n
      \in \N^*$, $u_n$ existe et $u_n \geq 1$.}
    \begin{remarkL}{.8}
      Cette question est un classique des suites
      récurrentes. Elle se traite presque toujours par récurrence.
    \end{remarkL}~\\[-1.4cm]
  \end{proof}


  % \newpage
  
  
\item Recopier et compléter les lignes \ligne{3} et \ligne{4} de la
  fonction \Scilab{} suivante afin que, prenant en argument un entier
  $n$ de $\N^*$, elle renvoie la valeur de $u_n$.
  \begin{scilab}
    & \tcFun{function} \tcVar{u}=\underline{suite}(\tcVar{n}) \nl %
    & \quad \tcVar{u} = 1 \nl %
    & \quad \tcFor{for} k = .................. \nl %
    & \quad \quad \tcVar{u} = .................. \nl %
    & \quad \tcFor{end} \nl %
    & \tcFun{endfunction}
  \end{scilab}
  \begin{proof}~
    \begin{scilab}
      & \tcFun{function} \tcVar{u}=\underline{suite}(\tcVar{n}) \nl %
      & \quad \tcVar{u} = 1 \nl %
      & \quad \tcFor{for} k = 1:(\tcVar{n}-1) \nl %
      & \quad \quad \tcVar{u} = (1/\tcVar{n}) \Sfois{}
      (\tcVar{n}\Sfois{}\tcVar{u} + 1/(\tcVar{n}\Sfois{}\tcVar{u})) \nl %
      & \quad \tcFor{end} \nl %
      & \tcFun{endfunction}
    \end{scilab}
    Détaillons l'obtention de ce programme.\\ %
    La variable {\tt u} est crée pour contenir successivement les
    valeurs $u_1$, $\ldots$, $u_n$.
    \begin{noliste}{$\sbullet$}
    \item On initialise donc cette variable à $u_1=1$ avec la ligne
      \ligne{2}.
      \begin{scilabC}{1}
        & \quad \tcVar{u} = 1 
      \end{scilabC}


      \newpage
      
      
    \item On met ensuite à jour {\tt u} à l'aide d'une
      structure itérative (boucle \tcFor{for}) avec les lignes
      \ligne{3} à \ligne{5}.
      \begin{scilabC}{2}
         & \quad \tcFor{for} k = 1:(\tcVar{n}-1) \nl %
         & \quad \quad \tcVar{u} = (1/\tcVar{n}) \Sfois{}
         (\tcVar{n}\Sfois{}\tcVar{u} + 1/(\tcVar{n}\Sfois{}\tcVar{u})) \nl %
         & \quad \tcFor{end}
      \end{scilabC}
    \end{noliste}
    \begin{remark}
      \begin{noliste}{$\sbullet$}
      \item On décrit ici de manière précise les instructions afin
        d'aider le lecteur un peu moins habile en
        \Scilab{}. Cependant, l'écriture du script démontre la
        compréhension de toutes les commandes en question et permet
        sans doute d'obtenir la totalité des points alloués à cette question.
        
      \item On pouvait également coder la fonction $f$ dans un script
        à part. On aurait alors obtenu les deux programmes suivants :
        \begin{scilab}
          & \tcFun{function} \tcVar{y}=\underline{f}(\tcVar{t}) \nl %
          & \quad y = t + 1/t \nl %
          & \tcFun{endfunction}
        \end{scilab}
        \begin{scilab}
           & \tcFun{function} \tcVar{u}=\underline{suite}(\tcVar{n}) \nl %
           & \quad \tcVar{u} = 1 \nl %
           & \quad \tcFor{for} k = 1:(\tcVar{n}-1) \nl %
           & \quad \quad \tcVar{u} = (1/\tcVar{n}) \Sfois{}
           f(\tcVar{n}\Sfois{}\tcVar{u}) \nl %
           & \quad \tcFor{end} \nl %
           & \tcFun{endfunction}
        \end{scilab}
      \end{noliste}
    \end{remark}~\\[-1.4cm]
  \end{proof}


  % \newpage
  
  
\item On pose, pour tout $n$ de $\N^*$, $v_n = u_{n+1} - u_n$.
  \begin{noliste}{a)}
    \setlength{\itemsep}{2mm}
  \item Montrer : $\forall n \in \N^*$, $0 \leq v_n \leq
    \dfrac{1}{n^2}$.
    \begin{proof}~\\
      Soit $n \in \N^*$.
      \begin{noliste}{$\sbullet$}
      \item Tout d'abord :
        \[
          v_n \ = \ u_{n+1} - u_n \ = \ \bcancel{u_n} + \dfrac{1}{n^2 \,
            u_n} - \bcancel{u_n} \ = \ \dfrac{1}{n^2 \, u_n}
        \]
        
      \item D'après la question précédente : $u_n \geq 1 >0$. Donc :
        $v_n = \dfrac{1}{n^2 \, u_n} \geq 0$.
      
      \item Toujours d'après la question précédente :
        \[
          \begin{array}{R{1cm}c@{\quad}>{\it}R{6cm}}
            & u_n \ \geq \ 1
            \\[.2cm]
            donc & \dfrac{1}{u_n} \ \leq \ 1
            & (par décroissance de la fonction inverse sur
              $]0,+\infty[$)
            \nl
            \nl[-.2cm]
            d'où & \dfrac{1}{n^2 \, u_n} \ \leq \ \dfrac{1}{n^2}
            & (car $\dfrac{1}{n^2} >0$)
            \nl
            \nl[-.2cm]
            ainsi & v_n \ \leq \ \dfrac{1}{n^2}
          \end{array}
        \]
      \end{noliste}
      \conc{Finalement : $\forall n \in \N^*$, $0 \leq v_n \leq
        \dfrac{1}{n^2}$.}~\\[-1cm]
    \end{proof}


    \newpage
    
    
  \item En déduire la nature de la série $\Sum{n \geq 1}{} v_n$.
    \begin{proof}~\\
      D'après la question précédente :
      \begin{noliste}{$\stimes$}
      \item $\forall n \in \N^*$, $0 \leq v_n \leq \dfrac{1}{n^2}$.
        
      \item la série $\Sum{n \geq 1}{} \dfrac{1}{n^2}$ est une série
        de Riemann d'exposant $2$ ($2>1$). Elle est donc convergente.
      \end{noliste}
      \conc{Par critère de comparaison des séries à termes positifs,
        la série $\Sum{n \geq 1}{} v_n$ est convergente.}
      \begin{remark}
        La seule difficulté de cette démonstration réside dans la
        rédaction du critère des séries à termes positifs (les
        arguments à utiliser ont tous été démontrés dans les questions
        précédentes). C'est donc une question d'application directe du
        cours qu'il convient de savoir traiter.
      \end{remark}~\\[-1.4cm]
    \end{proof}
    
  \item Calculer, pour tout $n$ supérieur ou égal à $2$,
    $\Sum{k=1}{n-1} v_k$.\\
    En déduire que la suite $(u_n)_{n\in \N^*}$ converge vers un réel
    $\ell$, que l'on ne cherchera pas à déterminer.
    \begin{proof}~
      \begin{noliste}{$\sbullet$}
      \item Soit $n \geq 2$. Par sommation télescopique :
        \[
          \Sum{k=1}{n-1} v_k \ = \ \Sum{k=1}{n-1} (u_{k+1} - u_k) \ =
          \ u_{(n-1)+1} - u_1 \ = \ u_n - 1
        \]
        \conc{$\forall n \geq 2$, $\Sum{k=1}{n-1} v_k = u_n-1$}
        
      \item Soit $n \geq 2$. D'après ce qui précède :
        \[
          u_n \ = \ 1 + \Sum{k=1}{n-1} v_k
        \]
        Or, d'après la question précédente, la série $\Sum{n \geq 1}{}
        v_n$ est convergente.\\ %
        On déduit de l'écriture précédente de $u_n$ que la suite
        $(u_n)$ est convergente, de limite :
        \[
          \dlim{n \to + \infty} u_n \ = \ 1+ \Sum{k=1}{+\infty} v_k
        \]
        \conc{Ainsi, la suite $(u_n)$ converge vers un réel noté
          $\ell$.}~\\[-1.4cm]
      \end{noliste}
    \end{proof}
  \end{noliste}


  \newpage
  
  
\item
  \begin{noliste}{a)}
    \setlength{\itemsep}{2mm}
  \item Montrer que, pour tout entier $k$ supérieur ou égal à $2$, on
    a : $\dfrac{1}{k^2} \ \leq \ \dint{k-1}{k} \dfrac{1}{t^2} \dt$.
    \begin{proof}~\\
      Soit $k \geq 2$.
      \begin{noliste}{$\sbullet$}
      \item Soit $t \in [k-1, k]$.
        \[
          \begin{array}{C{1cm}ccccc@{\quad}>{\it}R{5cm}}
            Comme & k-1 & \leq & t & \leq & k
            \\[.2cm]
            alors & \dfrac{1}{(k-1)^2} & \geq & \dfrac{1}{t^2} & \geq
            & \dfrac{1}{k^2}
            & (par décroissance de la fonction $t \mapsto
              \dfrac{1}{t^2}$ sur $]0,+\infty[$)
          \end{array}
        \]
        
      \item La fonction $t \mapsto \dfrac{1}{t^2}$ est continue sur le
        segment $[k-1,k]$.\\
        Par croissance de l'intégrale, les bornes étant dans l'ordre
        croissant ($k-1 \leq k$) :
        \[
          \begin{array}{ccccc}
            \dint{k-1}{k} \dfrac{1}{(k-1)^2} \dt & \geq
            & \dint{k-1}{k} \dfrac{1}{t^2} \dt & \geq
            & \dint{k-1}{k} \dfrac{1}{k^2} \dt
            \\[.6cm]
            \shortparallel & & & & \shortparallel
            \\[.2cm]
            \big(k-(k-1)\big) \dfrac{1}{(k-1)^2} & & &
            & \big(k-(k-1)\big) \dfrac{1}{k^2}
          \end{array}
        \]
        \conc{En particulier, pour tout $k \geq 2$ : $\dfrac{1}{k^2}
          \leq \dint{k-1}{k} \dfrac{1}{t^2} \dt$.}~\\[-1.4cm]
      \end{noliste}
    \end{proof}
    
  \item Pour tous entiers $n$ et $p$ tels que $2 \leq p < n$, calculer
    $\Sum{k=p}{n-1} v_k$ et en déduire :
    \[
      0 \ \leq \ u_n - u_p \ \leq \ \dint{p-1}{n-1} \dfrac{1}{t^2} \dt
    \]
    \begin{proof}~\\
      Soit $(n,p) \in \N^2$ tel que : $2 \leq p < n$.
      \begin{noliste}{$\sbullet$}
      \item Tout d'abord, par sommation télescopique :
        \[
          \Sum{k=p}{n-1} v_k \ = \ \Sum{k=p}{n-1} (u_{k+1} - u_k) \ =
          \ u_{(n-1)+1} - u_p \ = \ u_n - u_p
        \]
        \conc{$\Sum{k=p}{n-1} v_k \ = \ u_n - u_p$}
        
      \item Soit $k \geq 2$.
        \begin{noliste}{-}
        \item D'après la question \itbf{10.a)} : $0 \ \leq \ v_k \
          \leq \ \dfrac{1}{k^2}$.
          
        \item D'après la question précédente, on obtient par
          transitivité :
          \[
            0 \ \leq \ v_k \ \leq \ \dfrac{1}{k^2} \ \leq \
            \dint{k-1}{k} \dfrac{1}{t^2} \dt
          \]
        \end{noliste}
        \conc{$\forall k \geq 2$, $0 \ \leq \ v_k \ \leq \
          \dint{k-1}{k} \dfrac{1}{t^2} \dt$}


        \newpage
        
        
      \item On obtient, par sommation :
        \[
          \begin{array}{ccccc@{\quad}>{\it}R{5cm}}
            0 & \leq & \Sum{k=p}{n-1} v_k & \leq & \Sum{k=p}{n-1}
            \dint{k-1}{k} \dfrac{1}{t^2} \dt
            \\[.4cm]
            & & & & \shortparallel
            \\[.2cm]
            & & & & \dint{p-1}{n-1} \dfrac{1}{t^2} \dt
            & (par relation de Chasles)
          \end{array}
        \]
        \conc{Ainsi, d'après ce qui précède, pour tout $(n,p) \in
          \N^2$ tel que $2 \leq p < n$ :\\[.2cm] $0 \ \leq \ u_n - u_p \ \leq
          \ \dint{p-1}{n-1} \dfrac{1}{t^2} \dt$.}
      \end{noliste}~\\[-.8cm]
      \begin{remark}
        Les questions \itbf{11.a)} et \itbf{11.b)} sont en fait une
        comparaison série-intégrale dont on rappelle le résultat ci-dessous.
        \begin{noliste}{$\sbullet$}
        \item On considère une fonction $f: [0,+\infty[ \ \to \R$
          continue sur $[0,+\infty[$.
        \item On suppose de plus que $f$ est décroissante sur
          $[0,+\infty[$. Alors :
          \[
            \forall k \in \N^*, \quad f(k) \ \leq \ \dint{k-1}{k} f(t)
            \dt \ \leq \ f(k-1)
          \]
          On en déduit par sommation :
          \[
            \forall n \in \N, \quad \Sum{k=1}{n} f(k) \ \leq \
            \dint{0}{n} f(t) \dt \ \leq \ \Sum{k=1}{n} f(k-1)
          \]
        \end{noliste}
      \end{remark}~\\[-1.4cm]
    \end{proof}
    
  \item En déduire, pour tout entier $n$ supérieur ou égal à $3$ :
    $u_2 \leq u_n \leq 1 + u_2$.\\
    Montrer alors que $\ell$ appartient à l'intervalle $[2,3]$.
    \begin{proof}~
      \begin{noliste}{$\sbullet$}
      \item Soit $n \geq 3$. On applique le résultat de la question
        précédente avec $p=2$ (on a bien : $2 \leq p < n$) :
        \[
          0 \ \leq \ u_n - u_2 \ \leq \ \dint{1}{n-1} \dfrac{1}{t^2}
          \dt
        \]
        Or :
        \[
          \dint{1}{n-1} \dfrac{1}{t^2} \dt \ = \
          \Prim{-\dfrac{1}{t}}{1}{n-1} \ = \ -\left(\dfrac{1}{n-1} -
            1\right) \ = \ 1 - \dfrac{1}{n-1} \ \leq \ 1
        \]
        
      \item On en déduit, par transitivité :
        \[
          0 \ \leq \ u_n - u_2 \ \leq \ \dint{1}{n-1} \dfrac{1}{t^2}
          \dt \ \leq \ 1
        \]
        \conc{Ainsi, pour tout $n \geq 3$ : $u_2 \ \leq \ u_n \ \leq \
          1 + u_2$.}
        
      \item Par définition de la suite $(u_n)$ :
        \[
          u_2 \ = \ u_1 + \dfrac{1}{1^2 \, u_1} \ = \ 1 + \dfrac{1}{1
            \times 1} \ = \ 2
        \]
        On en déduit, avec l'encadrement précédent, pour tout $n \geq
        3$ :
        \[
          2 \ \leq \ u_n \ \leq \ 3
        \]
        \conc{En passant à la limite quand $n$ tend vers $+\infty$
          dans l'encadrement précédent,\\ on obtient : $\ell \in [2,3]$.}~\\[-1.4cm]
      \end{noliste}
    \end{proof}


    \newpage
    
    
  \item Montrer, pour tout entier $p$ supérieur ou égal à $2$ :
    \[
      0 \ \leq \ \ell - u_p \ \leq \ \dfrac{1}{p-1}
    \]
    \begin{proof}~\\
      Soit $p \geq 2$.
      \begin{noliste}{$\sbullet$}
      \item Soit $n >p$. D'après la question \itbf{11.b)} :
        \[
          0 \ \leq \ u_n - u_p \ \leq \ \dint{p-1}{n-1} \dfrac{1}{t^2} \dt
        \]
        
      \item Or :
        \[
          \dint{p-1}{n-1} \dfrac{1}{t^2} \dt \ = \
          \Prim{-\dfrac{1}{t}}{p-1}{n-1} \ = \ -\left( \dfrac{1}{n-1}
            - \dfrac{1}{p-1}\right) \ = \ \dfrac{1}{p-1} - \dfrac{1}{n-1}
        \]
        Ainsi :
        \[
          0 \ \leq \ u_n - u_p \ \leq \ \dfrac{1}{p-1} -
          \dfrac{1}{n-1}
        \]
        \conc{En passant à la limite quand $n$ tend vers $+\infty$
          dans l'encadrement précédent,\\ %
          on obtient : $0 \ \leq \ \ell - u_p \ \leq \ \dfrac{1}{p-1}$.}~\\[-1.4cm]
      \end{noliste}
    \end{proof}
    
  \item En déduire une fonction \Scilab{} qui renvoie une valeur
    approchée de $\ell$ à $10^{-4}$ près.
    \begin{proof}~
      \begin{noliste}{$\sbullet$}
      \item On cherche ici à trouver un entier $N$ tel que $u_N$ est
        une valeur approchée de $\ell$ à $10^{-4}$ près. Autrement
        dit, on souhaite exhiber $N \in \N$ tel que :
        \[
          |\ell - u_N| \ \leq \ 10^{-4}
        \]
        
      \item Or, d'après la question précédente : $ \forall p \geq 2,
        \quad 0 \ \leq \ \ell - u_p \ \leq \ \dfrac{1}{p-1} $.
        
      \item Il suffit alors de trouver $N \in \N$ tel que :
        $\dfrac{1}{N-1} \leq 10^{-4}$.\\[.1cm]
        Si c'est le cas, on obtient alors par transitivité :
        \[
          0 \ \leq \ \ell - u_N \ \leq \ 10^{-4}
        \]
        
      \item On propose alors le programme suivant :
        \begin{scilab}
          & \tcFun{function} \tcVar{l} =
          \underline{valeur\_approchee}() \nl %
          & \quad n = 2 \nl %
          & \quad \tcFor{while} 1 / (n-1) > 10\puis{}(-4) \nl %
          & \quad \quad n = n + 1 \nl %
          & \quad \tcFor{end} \nl %
          & \quad \tcVar{l} = suite(n) \nl %
          & \tcFun{endfunction}
        \end{scilab}


        \newpage


        \noindent
        Détaillons les éléments de ce script.
        \begin{noliste}{-}
        \item {\bf Début du script}\\
          La variable {\tt n} est initialisée à $2$. En effet, on
          souhaite pouvoir effectuer le calcul : $\dfrac{1}{{\tt n}
            -1}$.
          \begin{scilabC}{1}
            & \quad n = 2
          \end{scilabC}
          
        \item {\bf Structure itérative}\\
          Les lignes \ligne{3} à \ligne{5} consistent à déterminer le
          plus petit entier $n$ tel que $\dfrac{1}{n-1} \leq
          10^{-4}$. On doit donc comparer les valeurs successives de la
          suite $\left(\dfrac{1}{n-1}\right)_{n \geq 2}$ au réel
          $10^{-4}$ jusqu'à ce que $\dfrac{1}{n-1} \leq 10^{-4}$.
          Autrement dit, on doit comparer ces valeurs successives à $10^{-4}$
          tant que $\dfrac{1}{n-1} > 10^{-4}$. Pour cela on met en
          place une structure itérative (boucle \tcFor{while}) :
          \begin{scilabC}{2}
            & \quad \tcFor{while} 1 / (n-1) > 10\puis{}(-4)
          \end{scilabC}
          On met alors à jour en conséquence la variable {\tt n} :
          on ajoute $1$ pour signaler qu'on va comparer le terme
          suivant de la suite $\left(\dfrac{1}{n-1}\right)_{n\geq
            2}$ à $10^{-4}$.
          \begin{scilabC}{3}
            & \quad \quad n = n + 1
          \end{scilabC}
          
        \item {\bf Fin du script}\\
          À la fin de cette boucle, on est assuré que :
          $\dfrac{1}{n-1} \leq 10^{-4}$ (on itère tant que ce n'est
          pas le cas).\\
          Il reste alors à calculer la valeur approchée de $\ell$ : on
          l'obtient par le calcul de $u_n$ où $n$ est la valeur
          obtenue à l'issue de cette boucle.
          \begin{scilabC}{5}
            & \quad \tcVar{l} = suite(n)
          \end{scilabC}
        \end{noliste}
      \end{noliste}
      \begin{remark}
        \begin{noliste}{$\sbullet$}
        \item Lorsqu'on écrit une boucle {\tt while}, il est
          préférable de s'assurer en amont de sa terminaison. C'est
          bien le cas ici. En effet, la suite
          $\left(\dfrac{1}{n-1}\right)_{n \geq 2}$ est convergente de
          limite $0$. Ce qui signifie :
          \[
            \forall \eps > 0, \ \exists n_0 \in \N, \ \forall n \geq
            n_0, \ \left\vert \dfrac{1}{n-1} - 0 \right\vert < \eps
          \]
          Ainsi, quelle que soit la précision $\eps >0$ choisie au
          départ (ici $10^{-4}$), on est toujours en mesure de trouver
          un rang $n_0$ à partir duquel on aura : $\dfrac{1}{n-1} < 10^{-4}$.
          
        \item On pouvait déterminer, sans utiliser de boucle, un
          entier $N$ tel que $u_N$ est une valeur approchée
          à $10^{-4}$ près de $\ell$. Pour ce faire, on remarque :
          \[
            \dfrac{1}{n-1} \leq 10^{-4} \ \Leftrightarrow \ n-1 \geq
            10^4 \ \Leftrightarrow \ n \geq 10^4 +1
          \]
          L'entier $N = \lceil 10^4 +1 \rceil$ convient.
        \end{noliste}
      \end{remark}~\\[-1.4cm]
    \end{proof}
  \end{noliste}
\end{noliste}

\end{document}