\documentclass[11pt]{article}%
\usepackage{geometry}%
\geometry{a4paper,
  lmargin=2cm,rmargin=2cm,tmargin=2.5cm,bmargin=2.5cm}

% \usepackage{array}
\usepackage{paralist}

\usepackage[svgnames, usenames, dvipsnames]{xcolor}
\xdefinecolor{RecColor}{named}{Aqua}
\xdefinecolor{IncColor}{named}{Aqua}
\xdefinecolor{ImpColor}{named}{PaleGreen}

% \usepackage{frcursive}

\usepackage{adjustbox}

%%%%%%%%%%%
\newcommand{\cRB}[1]{{\color{Red} \pmb{#1}}} %
\newcommand{\cR}[1]{{\color{Red} {#1}}} %
\newcommand{\cBB}[1]{{\color{Blue} \pmb{#1}}}
\newcommand{\cB}[1]{{\color{Blue} {#1}}}
\newcommand{\cGB}[1]{{\color{LimeGreen} \pmb{#1}}}
\newcommand{\cG}[1]{{\color{LimeGreen} {#1}}}

%%%%%%%%%%

\usepackage{diagbox} %
\usepackage{colortbl} %
\usepackage{multirow} %
\usepackage{pgf} %
\usepackage{environ} %
\usepackage{fancybox} %
\usepackage{textcomp} %
\usepackage{marvosym} %

%%%%%%%%%% pour qu'une cellcolor ne recouvre pas le trait du tableau
\usepackage{hhline}%

\usepackage{pgfplots}
\pgfplotsset{compat=1.10}
\usepgfplotslibrary{patchplots}
\usepgfplotslibrary{fillbetween}
\usepackage{tikz,tkz-tab}
\usepackage{ifthen}
\usepackage{calc}
\usetikzlibrary{calc,decorations.pathreplacing,arrows,positioning} 
\usetikzlibrary{fit,shapes,backgrounds}
\usepackage[nomessages]{fp}% http://ctan.org/pkg/fp

\usetikzlibrary{matrix,arrows,decorations.pathmorphing,
  decorations.pathreplacing} 

\newcommand{\myunit}{1 cm}
\tikzset{
    node style sp/.style={draw,circle,minimum size=\myunit},
    node style ge/.style={circle,minimum size=\myunit},
    arrow style mul/.style={draw,sloped,midway,fill=white},
    arrow style plus/.style={midway,sloped,fill=white},
}

%%%%%%%%%%%%%%
%%%%% écrire des inférieur égal ou supérieur égal avec typographie
%%%%% francaise
%%%%%%%%%%%%%

\renewcommand{\geq}{\geqslant}
\renewcommand{\leq}{\leqslant}
\renewcommand{\emptyset}{\varnothing}

\newcommand{\Leq}{\leqslant}
\newcommand{\Geq}{\geqslant}

%%%%%%%%%%%%%%
%%%%% Macro Celia
%%%%%%%%%%%%%

\newcommand{\ff}[2]{\left[#1, #2\right]} %
\newcommand{\fo}[2]{\left[#1, #2\right[} %
\newcommand{\of}[2]{\left]#1, #2\right]} %
\newcommand{\soo}[2]{\left]#1, #2\right[} %
\newcommand{\abs}[1]{\left|#1\right|} %
\newcommand{\Ent}[1]{\left\lfloor #1 \right\rfloor} %


%%%%%%%%%%%%%%
%%%%% tikz : comment dessiner un "oeil"
%%%%%%%%%%%%%

\newcommand{\eye}[4]% size, x, y, rotation
{ \draw[rotate around={#4:(#2,#3)}] (#2,#3) -- ++(-.5*55:#1) (#2,#3)
  -- ++(.5*55:#1); \draw (#2,#3) ++(#4+55:.75*#1) arc
  (#4+55:#4-55:.75*#1);
  % IRIS
  \draw[fill=gray] (#2,#3) ++(#4+55/3:.75*#1) arc
  (#4+180-55:#4+180+55:.28*#1);
  % PUPIL, a filled arc
  \draw[fill=black] (#2,#3) ++(#4+55/3:.75*#1) arc
  (#4+55/3:#4-55/3:.75*#1);%
}


%%%%%%%%%%
%% discontinuité fonction
\newcommand\pointg[2]{%
  \draw[color = red, very thick] (#1+0.15, #2-.04)--(#1, #2-.04)--(#1,
  #2+.04)--(#1+0.15, #2+.04);%
}%

\newcommand\pointd[2]{%
  \draw[color = red, very thick] (#1-0.15, #2+.04)--(#1, #2+.04)--(#1,
  #2-.04)--(#1-0.15, #2-.04);%
}%

%%%%%%%%%%
%%% 1 : position abscisse, 2 : position ordonnée, 3 : taille, 4 : couleur
%%%%%%%%%%
% \newcommand\pointG[4]{%
%   \draw[color = #4, very thick] (#1+#3, #2-(#3/3.75))--(#1,
%   #2-(#3/3.75))--(#1, #2+(#3/3.75))--(#1+#3, #2+(#3/3.75)) %
% }%

\newcommand\pointG[4]{%
  \draw[color = #4, very thick] ({#1+#3/3.75}, {#2-#3})--(#1,
  {#2-#3})--(#1, {#2+#3})--({#1+#3/3.75}, {#2+#3}) %
}%

\newcommand\pointD[4]{%
  \draw[color = #4, very thick] ({#1-#3/3.75}, {#2+#3})--(#1,
  {#2+#3})--(#1, {#2-#3})--({#1-#3/3.75}, {#2-#3}) %
}%

\newcommand\spointG[4]{%
  \draw[color = #4, very thick] ({#1+#3/1.75}, {#2-#3})--(#1,
  {#2-#3})--(#1, {#2+#3})--({#1+#3/1.75}, {#2+#3}) %
}%

\newcommand\spointD[4]{%
  \draw[color = #4, very thick] ({#1-#3/2}, {#2+#3})--(#1,
  {#2+#3})--(#1, {#2-#3})--({#1-#3/2}, {#2-#3}) %
}%

%%%%%%%%%%

\newcommand{\Pb}{\mathtt{P}}

%%%%%%%%%%%%%%%
%%% Pour citer un précédent item
%%%%%%%%%%%%%%%
\newcommand{\itbf}[1]{{\small \bf \textit{#1}}}


%%%%%%%%%%%%%%%
%%% Quelques couleurs
%%%%%%%%%%%%%%%

\xdefinecolor{cancelcolor}{named}{Red}
\xdefinecolor{intI}{named}{ProcessBlue}
\xdefinecolor{intJ}{named}{ForestGreen}

%%%%%%%%%%%%%%%
%%%%%%%%%%%%%%%
% barrer du texte
\usetikzlibrary{shapes.misc}

\makeatletter
% \definecolor{cancelcolor}{rgb}{0.127,0.372,0.987}
\newcommand{\tikz@bcancel}[1]{%
  \begin{tikzpicture}[baseline=(textbox.base), inner sep=0pt]
    \node[strike out, draw] (textbox) {#1}[thick, color=cancelcolor];
    \useasboundingbox (textbox);
  \end{tikzpicture}%
}
\newcommand{\bcancel}[1]{%
  \relax\ifmmode
    \mathchoice{\tikz@bcancel{$\displaystyle#1$}}
               {\tikz@bcancel{$\textstyle#1$}}
               {\tikz@bcancel{$\scriptstyle#1$}}
               {\tikz@bcancel{$\scriptscriptstyle#1$}}
  \else
    \tikz@bcancel{\strut#1}%
  \fi
}
\newcommand{\tikz@xcancel}[1]{%
  \begin{tikzpicture}[baseline=(textbox.base),inner sep=0pt]
  \node[cross out,draw] (textbox) {#1}[thick, color=cancelcolor];
  \useasboundingbox (textbox);
  \end{tikzpicture}%
}
\newcommand{\xcancel}[1]{%
  \relax\ifmmode
    \mathchoice{\tikz@xcancel{$\displaystyle#1$}}
               {\tikz@xcancel{$\textstyle#1$}}
               {\tikz@xcancel{$\scriptstyle#1$}}
               {\tikz@xcancel{$\scriptscriptstyle#1$}}
  \else
    \tikz@xcancel{\strut#1}%
  \fi
}
\makeatother

\newcommand{\xcancelRA}{\xcancel{\rule[-.15cm]{0cm}{.5cm} \Rightarrow
    \rule[-.15cm]{0cm}{.5cm}}}

%%%%%%%%%%%%%%%%%%%%%%%%%%%%%%%%%%%
%%%%%%%%%%%%%%%%%%%%%%%%%%%%%%%%%%%

\newcommand{\vide}{\multicolumn{1}{c}{}}

%%%%%%%%%%%%%%%%%%%%%%%%%%%%%%%%%%%
%%%%%%%%%%%%%%%%%%%%%%%%%%%%%%%%%%%


\usepackage{multicol}
% \usepackage[latin1]{inputenc}
% \usepackage[T1]{fontenc}
\usepackage[utf8]{inputenc}
\usepackage[T1]{fontenc}
\usepackage[normalem]{ulem}
\usepackage[french]{babel}

\usepackage{url}    
\usepackage{hyperref}
\hypersetup{
  backref=true,
  pagebackref=true,
  hyperindex=true,
  colorlinks=true,
  breaklinks=true,
  urlcolor=blue,
  linkcolor=black,
  %%%%%%%%
  % ATTENTION : red changé en black pour le Livre !
  %%%%%%%%
  bookmarks=true,
  bookmarksopen=true
}

%%%%%%%%%%%%%%%%%%%%%%%%%%%%%%%%%%%%%%%%%%%
%% Pour faire des traits diagonaux dans les tableaux
%% Nécessite slashbox.sty
%\usepackage{slashbox}

\usepackage{tipa}
\usepackage{verbatim,listings}
\usepackage{graphicx}
\usepackage{fancyhdr}
\usepackage{mathrsfs}
\usepackage{pifont}
\usepackage{tablists}
\usepackage{dsfont,amsfonts,amssymb,amsmath,amsthm,stmaryrd,upgreek,manfnt}
\usepackage{enumerate}

%\newcolumntype{M}[1]{p{#1}}
\newcolumntype{C}[1]{>{\centering}m{#1}}
\newcolumntype{R}[1]{>{\raggedright}m{#1}}
\newcolumntype{L}[1]{>{\raggedleft}m{#1}}
\newcolumntype{P}[1]{>{\raggedright}p{#1}}
\newcolumntype{B}[1]{>{\raggedright}b{#1}}
\newcolumntype{Q}[1]{>{\raggedright}t{#1}}

\newcommand{\alias}[2]{
\providecommand{#1}{}
\renewcommand{#1}{#2}
}
\alias{\R}{\mathbb{R}}
\alias{\N}{\mathbb{N}}
\alias{\Z}{\mathbb{Z}}
\alias{\Q}{\mathbb{Q}}
\alias{\C}{\mathbb{C}}
\alias{\K}{\mathbb{K}}

%%%%%%%%%%%%
%% rendre +infty et -infty plus petits
%%%%%%%%%%%%
\newcommand{\sinfty}{{\scriptstyle \infty}}

%%%%%%%%%%%%%%%%%%%%%%%%%%%%%%%
%%%%% macros TP Scilab %%%%%%%%
\newcommand{\Scilab}{\textbf{Scilab}} %
\newcommand{\Scinotes}{\textbf{SciNotes}} %
\newcommand{\faire}{\noindent $\blacktriangleright$ } %
\newcommand{\fitem}{\scalebox{.8}{$\blacktriangleright$}} %
\newcommand{\entree}{{\small\texttt{ENTRÉE}}} %
\newcommand{\tab}{{\small\texttt{TAB}}} %
\newcommand{\mt}[1]{\mathtt{#1}} %
% guillemets droits

\newcommand{\ttq}{\textquotesingle} %

\newcommand{\reponse}[1]{\longboxed{
    \begin{array}{C{0.9\textwidth}}
      \nl[#1]
    \end{array}
  }} %

\newcommand{\reponseR}[1]{\longboxed{
    \begin{array}{R{0.9\textwidth}}
      #1
    \end{array}
  }} %

\newcommand{\reponseC}[1]{\longboxed{
    \begin{array}{C{0.9\textwidth}}
      #1
    \end{array}
  }} %

\colorlet{pyfunction}{Blue}
\colorlet{pyCle}{Magenta}
\colorlet{pycomment}{LimeGreen}
\colorlet{pydoc}{Cyan}
% \colorlet{SansCo}{white}
% \colorlet{AvecCo}{black}

\newcommand{\visible}[1]{{\color{ASCo}\colorlet{pydoc}{pyDo}\colorlet{pycomment}{pyCo}\colorlet{pyfunction}{pyF}\colorlet{pyCle}{pyC}\colorlet{function}{sciFun}\colorlet{var}{sciVar}\colorlet{if}{sciIf}\colorlet{comment}{sciComment}#1}} %

%%%% à changer ????
\newcommand{\invisible}[1]{{\color{ASCo}\colorlet{pydoc}{pyDo}\colorlet{pycomment}{pyCo}\colorlet{pyfunction}{pyF}\colorlet{pyCle}{pyC}\colorlet{function}{sciFun}\colorlet{var}{sciVar}\colorlet{if}{sciIf}\colorlet{comment}{sciComment}#1}} %

\newcommand{\invisibleCol}[2]{{\color{#1}#2}} %

\NewEnviron{solution} %
{ %
  \Boxed{
    \begin{array}{>{\color{ASCo}} R{0.9\textwidth}}
      \colorlet{pycomment}{pyCo}
      \colorlet{pydoc}{pyDo}
      \colorlet{pyfunction}{pyF}
      \colorlet{pyCle}{pyC}
      \colorlet{function}{sciFun}
      \colorlet{var}{sciVar}
      \colorlet{if}{sciIf}
      \colorlet{comment}{sciComment}
      \BODY
    \end{array}
  } %
} %

\NewEnviron{solutionC} %
{ %
  \Boxed{
    \begin{array}{>{\color{ASCo}} C{0.9\textwidth}}
      \colorlet{pycomment}{pyCo}
      \colorlet{pydoc}{pyDo}
      \colorlet{pyfunction}{pyF}
      \colorlet{pyCle}{pyC}
      \colorlet{function}{sciFun}
      \colorlet{var}{sciVar}
      \colorlet{if}{sciIf}
      \colorlet{comment}{sciComment}
      \BODY
    \end{array}
  } %
} %

\newcommand{\invite}{--\!\!>} %

%%%%% nouvel environnement tabular pour retour console %%%%
\colorlet{ConsoleColor}{Black!12}
\colorlet{function}{Red}
\colorlet{var}{Maroon}
\colorlet{if}{Magenta}
\colorlet{comment}{LimeGreen}

\newcommand{\tcVar}[1]{\textcolor{var}{\bf \small #1}} %
\newcommand{\tcFun}[1]{\textcolor{function}{#1}} %
\newcommand{\tcIf}[1]{\textcolor{if}{#1}} %
\newcommand{\tcFor}[1]{\textcolor{if}{#1}} %

\newcommand{\moins}{\!\!\!\!\!\!- }
\newcommand{\espn}{\!\!\!\!\!\!}

\usepackage{booktabs,varwidth} \newsavebox\TBox
\newenvironment{console}
{\begin{lrbox}{\TBox}\varwidth{\linewidth}
    \tabular{>{\tt\small}R{0.84\textwidth}}
    \nl[-.4cm]} {\endtabular\endvarwidth\end{lrbox}%
  \fboxsep=1pt\colorbox{ConsoleColor}{\usebox\TBox}}

\newcommand{\lInv}[1]{%
  $\invite$ #1} %

\newcommand{\lAns}[1]{%
  \qquad ans \ = \nl %
  \qquad \qquad #1} %

\newcommand{\lVar}[2]{%
  \qquad #1 \ = \nl %
  \qquad \qquad #2} %

\newcommand{\lDisp}[1]{%
  #1 %
} %

\newcommand{\ligne}[1]{\underline{\small \tt #1}} %

\newcommand{\ligneAns}[2]{%
  $\invite$ #1 \nl %
  \qquad ans \ = \nl %
  \qquad \qquad #2} %

\newcommand{\ligneVar}[3]{%
  $\invite$ #1 \nl %
  \qquad #2 \ = \nl %
  \qquad \qquad #3} %

\newcommand{\ligneErr}[3]{%
  $\invite$ #1 \nl %
  \quad !-{-}error #2 \nl %
  #3} %
%%%%%%%%%%%%%%%%%%%%%% 

\newcommand{\bs}[1]{\boldsymbol{#1}} %
\newcommand{\nll}{\nl[.4cm]} %
\newcommand{\nle}{\nl[.2cm]} %
%% opérateur puissance copiant l'affichage Scilab
%\newcommand{\puis}{\!\!\!~^{\scriptscriptstyle\pmb{\wedge}}}
\newcommand{\puis}{\mbox{$\hspace{-.1cm}~^{\scriptscriptstyle\pmb{\wedge}}
    \hspace{0.05cm}$}} %
\newcommand{\pointpuis}{.\mbox{$\hspace{-.15cm}~^{\scriptscriptstyle\pmb{\wedge}}$}} %
\newcommand{\Sfois}{\mbox{$\mt{\star}$}} %

%%%%% nouvel environnement tabular pour les encadrés Scilab %%%%
\newenvironment{encadre}
{\begin{lrbox}{\TBox}\varwidth{\linewidth}
    \tabular{>{\tt\small}C{0.1\textwidth}>{\small}R{0.7\textwidth}}}
  {\endtabular\endvarwidth\end{lrbox}%
  \fboxsep=1pt\longboxed{\usebox\TBox}}

\newenvironment{encadreL}
{\begin{lrbox}{\TBox}\varwidth{\linewidth}
    \tabular{>{\tt\small}C{0.25\textwidth}>{\small}R{0.6\textwidth}}}
  {\endtabular\endvarwidth\end{lrbox}%
  \fboxsep=1pt\longboxed{\usebox\TBox}}

\newenvironment{encadreF}
{\begin{lrbox}{\TBox}\varwidth{\linewidth}
    \tabular{>{\tt\small}C{0.2\textwidth}>{\small}R{0.70\textwidth}}}
  {\endtabular\endvarwidth\end{lrbox}%
  \fboxsep=1pt\longboxed{\usebox\TBox}}

\newenvironment{encadreLL}[2]
{\begin{lrbox}{\TBox}\varwidth{\linewidth}
    \tabular{>{\tt\small}C{#1\textwidth}>{\small}R{#2\textwidth}}}
  {\endtabular\endvarwidth\end{lrbox}%
  \fboxsep=1pt\longboxed{\usebox\TBox}}

%%%%% nouvel environnement tabular pour les script et fonctions %%%%
\newcommand{\commentaireDL}[1]{\multicolumn{1}{l}{\it
    \textcolor{comment}{$\slash\slash$ #1}}}

\newcommand{\commentaire}[1]{{\textcolor{comment}{$\slash\slash$ #1}}}

\newcounter{cptcol}

\newcommand{\nocount}{\multicolumn{1}{c}{}}

\newcommand{\sciNo}[1]{{\small \underbar #1}}

\NewEnviron{scilab}{ %
  \setcounter{cptcol}{0}
  \begin{center}
    \longboxed{
      \begin{tabular}{>{\stepcounter{cptcol}{\tiny \underbar
              \thecptcol}}c>{\tt}l}
        \BODY
      \end{tabular}
    }
  \end{center}
}

\NewEnviron{scilabNC}{ %
  \begin{center}
    \longboxed{
      \begin{tabular}{>{\tt}l} %
          \BODY
      \end{tabular}
    }
  \end{center}
}

\NewEnviron{scilabC}[1]{ %
  \setcounter{cptcol}{#1}
  \begin{center}
    \longboxed{
      \begin{tabular}{>{\stepcounter{cptcol}{\tiny \underbar
              \thecptcol}}c>{\tt}l}
        \BODY
      \end{tabular}
    }
  \end{center}
}

\newcommand{\scisol}[1]{ %
  \setcounter{cptcol}{0}
  \longboxed{
    \begin{tabular}{>{\stepcounter{cptcol}{\tiny \underbar
            \thecptcol}}c>{\tt}l}
      #1
    \end{tabular}
  }
}

\newcommand{\scisolNC}[1]{ %
  \longboxed{
    \begin{tabular}{>{\tt}l}
      #1
    \end{tabular}
  }
}

\newcommand{\scisolC}[2]{ %
  \setcounter{cptcol}{#1}
  \longboxed{
    \begin{tabular}{>{\stepcounter{cptcol}{\tiny \underbar
            \thecptcol}}c>{\tt}l}
      #2
    \end{tabular}
  }
}

\NewEnviron{syntaxe}{ %
  % \fcolorbox{black}{Yellow!20}{\setlength{\fboxsep}{3mm}
  \shadowbox{
    \setlength{\fboxsep}{3mm}
    \begin{tabular}{>{\tt}l}
      \BODY
    \end{tabular}
  }
}

%%%%% fin macros TP Scilab %%%%%%%%
%%%%%%%%%%%%%%%%%%%%%%%%%%%%%%%%%%%

%%%%%%%%%%%%%%%%%%%%%%%%%%%%%%%%%%%
%%%%% TP Python - listings %%%%%%%%
%%%%%%%%%%%%%%%%%%%%%%%%%%%%%%%%%%%
\newcommand{\Python}{\textbf{Python}} %

\lstset{% general command to set parameter(s)
basicstyle=\ttfamily\small, % print whole listing small
keywordstyle=\color{blue}\bfseries\underbar,
%% underlined bold black keywords
frame=lines,
xleftmargin=10mm,
numbers=left,
numberstyle=\tiny\underbar,
numbersep=10pt,
%identifierstyle=, % nothing happens
commentstyle=\color{green}, % white comments
%%stringstyle=\ttfamily, % typewriter type for strings
showstringspaces=false}

\newcommand{\pysolCpt}[2]{
  \setcounter{cptcol}{#1}
  \longboxed{
    \begin{tabular}{>{\stepcounter{cptcol}{\tiny \underbar
            \thecptcol}}c>{\tt}l}
        #2
      \end{tabular}
    }
} %

\newcommand{\pysol}[1]{
  \setcounter{cptcol}{0}
  \longboxed{
    \begin{tabular}{>{\stepcounter{cptcol}{\tiny \underbar
            \thecptcol}}c>{\tt}l}
        #1
      \end{tabular}
    }
} %

% \usepackage[labelsep=endash]{caption}

% avec un caption
\NewEnviron{pythonCap}[1]{ %
  \renewcommand{\tablename}{Programme}
  \setcounter{cptcol}{0}
  \begin{center}
    \longboxed{
      \begin{tabular}{>{\stepcounter{cptcol}{\tiny \underbar
              \thecptcol}}c>{\tt}l}
        \BODY
      \end{tabular}
    }
    \captionof{table}{#1}
  \end{center}
}

\NewEnviron{python}{ %
  \setcounter{cptcol}{0}
  \begin{center}
    \longboxed{
      \begin{tabular}{>{\stepcounter{cptcol}{\tiny \underbar
              \thecptcol}}c>{\tt}l}
        \BODY
      \end{tabular}
    }
  \end{center}
}

\newcommand{\pyVar}[1]{\textcolor{var}{\bf \small #1}} %
\newcommand{\pyFun}[1]{\textcolor{pyfunction}{#1}} %
\newcommand{\pyCle}[1]{\textcolor{pyCle}{#1}} %
\newcommand{\pyImp}[1]{{\bf #1}} %

%%%%% commentaire python %%%%
\newcommand{\pyComDL}[1]{\multicolumn{1}{l}{\textcolor{pycomment}{\#
      #1}}}

\newcommand{\pyCom}[1]{{\textcolor{pycomment}{\# #1}}}
\newcommand{\pyDoc}[1]{{\textcolor{pydoc}{#1}}}

\newcommand{\pyNo}[1]{{\small \underbar #1}}

%%%%%%%%%%%%%%%%%%%%%%%%%%%%%%%%%%%
%%%%%% Système linéaire paramétré : écrire les opérations au-dessus
%%%%%% d'un symbole équivalent
%%%%%%%%%%%%%%%%%%%%%%%%%%%%%%%%%%%

\usepackage{systeme}

\NewEnviron{arrayEq}{ %
  \stackrel{\scalebox{.6}{$
      \begin{array}{l} 
        \BODY \\[.1cm]
      \end{array}$}
  }{\Longleftrightarrow}
}

\NewEnviron{arrayEg}{ %
  \stackrel{\scalebox{.6}{$
      \begin{array}{l} 
        \BODY \\[.1cm]
      \end{array}$}
  }{=}
}

\NewEnviron{operationEq}{ %
  \scalebox{.6}{$
    \begin{array}{l} 
      \scalebox{1.6}{$\mbox{Opérations :}$} \\[.2cm]
      \BODY \\[.1cm]
    \end{array}$}
}

% \NewEnviron{arraySys}[1]{ %
%   \sysdelim\{.\systeme[#1]{ %
%     \BODY %
%   } %
% }

%%%%%

%%%%%%%%%%
%%%%%%%%%% ESSAI
\newlength\fboxseph
\newlength\fboxsepva
\newlength\fboxsepvb

\setlength\fboxsepva{0.2cm}
\setlength\fboxsepvb{0.2cm}
\setlength\fboxseph{0.2cm}

\makeatletter

\def\longboxed#1{\leavevmode\setbox\@tempboxa\hbox{\color@begingroup%
\kern\fboxseph{\m@th$\displaystyle #1 $}\kern\fboxseph%
\color@endgroup }\my@frameb@x\relax}

\def\my@frameb@x#1{%
  \@tempdima\fboxrule \advance\@tempdima \fboxsepva \advance\@tempdima
  \dp\@tempboxa\hbox {%
    \lower \@tempdima \hbox {%
      \vbox {\hrule\@height\fboxrule \hbox{\vrule\@width\fboxrule #1
          \vbox{%
            \vskip\fboxsepva \box\@tempboxa \vskip\fboxsepvb}#1
          \vrule\@width\fboxrule }%
        \hrule \@height \fboxrule }}}}

\newcommand{\boxedhv}[3]{\setlength\fboxseph{#1cm}
  \setlength\fboxsepva{#2cm}\setlength\fboxsepvb{#2cm}\longboxed{#3}}

\newcommand{\boxedhvv}[4]{\setlength\fboxseph{#1cm}
  \setlength\fboxsepva{#2cm}\setlength\fboxsepvb{#3cm}\longboxed{#4}}

\newcommand{\Boxed}[1]{{\setlength\fboxseph{0.2cm}
  \setlength\fboxsepva{0.2cm}\setlength\fboxsepvb{0.2cm}\longboxed{#1}}}

\newcommand{\mBoxed}[1]{{\setlength\fboxseph{0.2cm}
  \setlength\fboxsepva{0.2cm}\setlength\fboxsepvb{0.2cm}\longboxed{\mbox{#1}}}}

\newcommand{\mboxed}[1]{{\setlength\fboxseph{0.2cm}
  \setlength\fboxsepva{0.2cm}\setlength\fboxsepvb{0.2cm}\boxed{\mbox{#1}}}}

\newsavebox{\fmbox}
\newenvironment{fmpage}[1]
     {\begin{lrbox}{\fmbox}\begin{minipage}{#1}}
     {\end{minipage}\end{lrbox}\fbox{\usebox{\fmbox}}}

%%%%%%%%%%
%%%%%%%%%%

\DeclareMathOperator{\ch}{ch}
\DeclareMathOperator{\sh}{sh}

%%%%%%%%%%
%%%%%%%%%%

\newcommand{\norme}[1]{\Vert #1 \Vert}

%\newcommand*\widefbox[1]{\fbox{\hspace{2em}#1\hspace{2em}}}

\newcommand{\nl}{\tabularnewline}

\newcommand{\hand}{\noindent\ding{43}\ }
\newcommand{\ie}{\textit{i.e. }}
\newcommand{\cf}{\textit{cf }}

\newcommand{\Card}{\operatorname{Card}}

\newcommand{\aire}{\mathcal{A}}

\newcommand{\LL}[1]{\mathscr{L}(#1)} %
\newcommand{\B}{\mathscr{B}} %
\newcommand{\Bc}[1]{B_{#1}} %
\newcommand{\M}[1]{\mathscr{M}_{#1}(\mathbb{R})}

\DeclareMathOperator{\im}{Im}
\DeclareMathOperator{\kr}{Ker}
\DeclareMathOperator{\rg}{rg}
\DeclareMathOperator{\spc}{Sp}
\DeclareMathOperator{\sgn}{sgn}
\DeclareMathOperator{\supp}{Supp}

\newcommand{\Mat}{{\rm{Mat}}}
\newcommand{\Vect}[1]{{\rm{Vect}}\left(#1\right)}

\newenvironment{smatrix}{%
  \begin{adjustbox}{width=.9\width}
    $
    \begin{pmatrix}
    }{%      
    \end{pmatrix}
    $
  \end{adjustbox}
}

\newenvironment{sarray}[1]{%
  \begin{adjustbox}{width=.9\width}
    $
    \begin{array}{#1}
    }{%      
    \end{array}
    $
  \end{adjustbox}
}

\newcommand{\vd}[2]{
  \scalebox{.8}{
    $\left(\!
      \begin{array}{c}
        #1 \\
        #2
      \end{array}
    \!\right)$
    }}

\newcommand{\vt}[3]{
  \scalebox{.8}{
    $\left(\!
      \begin{array}{c}
        #1 \\
        #2 \\
        #3 
      \end{array}
    \!\right)$
    }}

\newcommand{\vq}[4]{
  \scalebox{.8}{
    $\left(\!
      \begin{array}{c}
        #1 \\
        #2 \\
        #3 \\
        #4 
      \end{array}
    \!\right)$
    }}

\newcommand{\vc}[5]{
  \scalebox{.8}{
    $\left(\!
      \begin{array}{c}
        #1 \\
        #2 \\
        #3 \\
        #4 \\
        #5 
      \end{array}
    \!\right)$
    }}

\newcommand{\ee}{\text{e}}

\newcommand{\dd}{\text{d}}

%%% Ensemble de définition
\newcommand{\Df}{\mathscr{D}}
\newcommand{\Cf}{\mathscr{C}}
\newcommand{\Ef}{\mathscr{C}}

\newcommand{\rond}[1]{\,\overset{\scriptscriptstyle \circ}{\!#1}}

\newcommand{\df}[2]{\dfrac{\partial #1}{\partial #2}} %
\newcommand{\dfn}[2]{\partial_{#2}(#1)} %
\newcommand{\ddfn}[2]{\partial^2_{#2}(#1)} %
\newcommand{\ddf}[2]{\dfrac{\partial^2 #1}{\partial #2^2}} %
\newcommand{\ddfr}[3]{\dfrac{\partial^2 #1}{\partial #2 \partial
    #3}} %


\newcommand{\dlim}[1]{{\displaystyle \lim_{#1} \ }}
\newcommand{\dlimPlus}[2]{
  \dlim{
    \scalebox{.6}{
      $
      \begin{array}{l}
        #1 \rightarrow #2\\
        #1 > #2
      \end{array}
      $}}}
\newcommand{\dlimMoins}[2]{
  \dlim{
    \scalebox{.6}{
      $
      \begin{array}{l}
        #1 \rightarrow #2\\
        #1 < #2
      \end{array}
      $}}}

%%%%%%%%%%%%%%
%% petit o, développement limité
%%%%%%%%%%%%%%

\newcommand{\oo}[2]{{\underset {{\overset {#1\rightarrow #2}{}}}{o}}} %
\newcommand{\oox}[1]{{\underset {{\overset {x\rightarrow #1}{}}}{o}}} %
\newcommand{\oon}{{\underset {{\overset {n\rightarrow +\infty}{}}}{o}}} %
\newcommand{\po}[1]{{\underset {{\overset {#1}{}}}{o}}} %
\newcommand{\neqx}[1]{{\ \underset {{\overset {x \to #1}{}}}{\not\sim}\ }} %
\newcommand{\eqx}[1]{{\ \underset {{\overset {x \to #1}{}}}{\sim}\ }} %
\newcommand{\eqn}{{\ \underset {{\overset {n \to +\infty}{}}}{\sim}\ }} %
\newcommand{\eq}[2]{{\ \underset {{\overset {#1 \to #2}{}}}{\sim}\ }} %
\newcommand{\DL}[1]{{\rm{DL}}_1 (#1)} %
\newcommand{\DLL}[1]{{\rm{DL}}_2 (#1)} %

\newcommand{\negl}{<<}

\newcommand{\neglP}[1]{\begin{array}{c}
    \vspace{-.2cm}\\
    << \\
    \vspace{-.7cm}\\
    {\scriptstyle #1}
  \end{array}}

%%%%%%%%%%%%%%
%% borne sup, inf, max, min
%%%%%%%%%%%%%%
\newcommand{\dsup}[1]{\displaystyle \sup_{#1} \ }
\newcommand{\dinf}[1]{\displaystyle \inf_{#1} \ }
\newcommand{\dmax}[1]{\max\limits_{#1} \ }
\newcommand{\dmin}[1]{\min\limits_{#1} \ }

\newcommand{\dcup}[2]{{\textstyle\bigcup\limits_{#1}^{#2}}\hspace{.1cm}}
%\displaystyle \bigcup_{#1}^{#2}}
\newcommand{\dcap}[2]{{\textstyle\bigcap\limits_{#1}^{#2}}\hspace{.1cm}}
% \displaystyle \bigcap_{#1}^{#2}
%%%%%%%%%%%%%%
%% opérateurs logiques
%%%%%%%%%%%%%%
\newcommand{\NON}[1]{\mathop{\small \tt{NON}} (#1)}
\newcommand{\ET}{\mathrel{\mathop{\small \mathtt{ET}}}}
\newcommand{\OU}{\mathrel{\mathop{\small \tt{OU}}}}
\newcommand{\XOR}{\mathrel{\mathop{\small \tt{XOR}}}}

\newcommand{\id}{{\rm{id}}}

\newcommand{\sbullet}{\scriptstyle \bullet}
\newcommand{\stimes}{\scriptstyle \times}

%%%%%%%%%%%%%%%%%%
%% Probabilités
%%%%%%%%%%%%%%%%%%
\newcommand{\Prob}{\mathbb{P}}
\newcommand{\Ev}[1]{\left[ {#1} \right]}
\newcommand{\Evmb}[1]{[ {#1} ]}
\newcommand{\E}{\mathbb{E}}
\newcommand{\V}{\mathbb{V}}
\newcommand{\Cov}{{\rm{Cov}}}
\newcommand{\U}[2]{\mathcal{U}(\llb #1, #2\rrb)}
\newcommand{\Uc}[2]{\mathcal{U}([#1, #2])}
\newcommand{\Ucof}[2]{\mathcal{U}(]#1, #2])}
\newcommand{\Ucoo}[2]{\mathcal{U}(]#1, #2[)}
\newcommand{\Ucfo}[2]{\mathcal{U}([#1, #2[)}
\newcommand{\Bern}[1]{\mathcal{B}\left(#1\right)}
\newcommand{\Bin}[2]{\mathcal{B}\left(#1, #2\right)}
\newcommand{\G}[1]{\mathcal{G}\left(#1\right)}
\newcommand{\Pois}[1]{\mathcal{P}\left(#1\right)}
\newcommand{\HG}[3]{\mathcal{H}\left(#1, #2, #3\right)}
\newcommand{\Exp}[1]{\mathcal{E}\left(#1\right)}
\newcommand{\Norm}[2]{\mathcal{N}\left(#1, #2\right)}

\DeclareMathOperator{\cov}{Cov}

\newcommand{\var}{v.a.r. }
\newcommand{\suit}{\hookrightarrow}

\newcommand{\flecheR}[1]{\rotatebox{90}{\scalebox{#1}{\color{red}
      $\curvearrowleft$}}}


\newcommand{\partie}[1]{\mathcal{P}(#1)}
\newcommand{\Cont}[1]{\mathcal{C}^{#1}}
\newcommand{\Contm}[1]{\mathcal{C}^{#1}_m}

\newcommand{\llb}{\llbracket}
\newcommand{\rrb}{\rrbracket}

%\newcommand{\im}[1]{{\rm{Im}}(#1)}
\newcommand{\imrec}[1]{#1^{- \mathds{1}}}

\newcommand{\unq}{\mathds{1}}

\newcommand{\Hyp}{\mathtt{H}}

\newcommand{\eme}[1]{#1^{\scriptsize \mbox{ème}}}
\newcommand{\er}[1]{#1^{\scriptsize \mbox{er}}}
\newcommand{\ere}[1]{#1^{\scriptsize \mbox{ère}}}
\newcommand{\nd}[1]{#1^{\scriptsize \mbox{nd}}}
\newcommand{\nde}[1]{#1^{\scriptsize \mbox{nde}}}

\newcommand{\truc}{\mathop{\top}}
\newcommand{\fois}{\mathop{\ast}}

\newcommand{\f}[1]{\overrightarrow{#1}}

\newcommand{\checked}{\textcolor{green}{\checkmark}}

\def\restriction#1#2{\mathchoice
              {\setbox1\hbox{${\displaystyle #1}_{\scriptstyle #2}$}
              \restrictionaux{#1}{#2}}
              {\setbox1\hbox{${\textstyle #1}_{\scriptstyle #2}$}
              \restrictionaux{#1}{#2}}
              {\setbox1\hbox{${\scriptstyle #1}_{\scriptscriptstyle #2}$}
              \restrictionaux{#1}{#2}}
              {\setbox1\hbox{${\scriptscriptstyle #1}_{\scriptscriptstyle #2}$}
              \restrictionaux{#1}{#2}}}
\def\restrictionaux#1#2{{#1\,\smash{\vrule height .8\ht1 depth .85\dp1}}_{\,#2}}

\makeatletter
\newcommand*{\da@rightarrow}{\mathchar"0\hexnumber@\symAMSa 4B }
\newcommand*{\da@leftarrow}{\mathchar"0\hexnumber@\symAMSa 4C }
\newcommand*{\xdashrightarrow}[2][]{%
  \mathrel{%
    \mathpalette{\da@xarrow{#1}{#2}{}\da@rightarrow{\,}{}}{}%
  }%
}
\newcommand{\xdashleftarrow}[2][]{%
  \mathrel{%
    \mathpalette{\da@xarrow{#1}{#2}\da@leftarrow{}{}{\,}}{}%
  }%
}
\newcommand*{\da@xarrow}[7]{%
  % #1: below
  % #2: above
  % #3: arrow left
  % #4: arrow right
  % #5: space left 
  % #6: space right
  % #7: math style 
  \sbox0{$\ifx#7\scriptstyle\scriptscriptstyle\else\scriptstyle\fi#5#1#6\m@th$}%
  \sbox2{$\ifx#7\scriptstyle\scriptscriptstyle\else\scriptstyle\fi#5#2#6\m@th$}%
  \sbox4{$#7\dabar@\m@th$}%
  \dimen@=\wd0 %
  \ifdim\wd2 >\dimen@
    \dimen@=\wd2 %   
  \fi
  \count@=2 %
  \def\da@bars{\dabar@\dabar@}%
  \@whiledim\count@\wd4<\dimen@\do{%
    \advance\count@\@ne
    \expandafter\def\expandafter\da@bars\expandafter{%
      \da@bars
      \dabar@ 
    }%
  }%  
  \mathrel{#3}%
  \mathrel{%   
    \mathop{\da@bars}\limits
    \ifx\\#1\\%
    \else
      _{\copy0}%
    \fi
    \ifx\\#2\\%
    \else
      ^{\copy2}%
    \fi
  }%   
  \mathrel{#4}%
}
\makeatother



\newcount\depth

\newcount\depth
\newcount\totaldepth

\makeatletter
\newcommand{\labelsymbol}{%
      \ifnum\depth=0
        %
      \else
        \rlap{\,$\bullet$}%
      \fi
}

\newcommand*\bernoulliTree[1]{%
    \depth=#1\relax            
    \totaldepth=#1\relax
    \draw node(root)[bernoulli/root] {\labelsymbol}[grow=right] \draw@bernoulli@tree;
    \draw \label@bernoulli@tree{root};                                   
}                                                                        

\def\draw@bernoulli@tree{%
    \ifnum\depth>0 
      child[parent anchor=east] foreach \type/\label in {left child/$E$,right child/$S$} {%
          node[bernoulli/\type] {\label\strut\labelsymbol} \draw@bernoulli@tree
      }
      coordinate[bernoulli/increment] (dummy)
   \fi%
}

\def\label@bernoulli@tree#1{%
    \ifnum\depth>0
      ($(#1)!0.5!(#1-1)$) node[fill=white,bernoulli/decrement] {\tiny$p$}
      \label@bernoulli@tree{#1-1}
      ($(#1)!0.5!(#1-2)$) node[fill=white] {\tiny$q$}
      \label@bernoulli@tree{#1-2}
      coordinate[bernoulli/increment] (dummy)
   \fi%
}

\makeatother

\tikzset{bernoulli/.cd,
         root/.style={},
         decrement/.code=\global\advance\depth by-1\relax,
         increment/.code=\global\advance\depth by 1\relax,
         left child/.style={bernoulli/decrement},
         right child/.style={}}


\newcommand{\eps}{\varepsilon}

% \newcommand{\tendi}[1]{\xrightarrow[\footnotesize #1 \rightarrow
%   +\infty]{}}%

\newcommand{\tend}{\rightarrow}%
\newcommand{\tendn}{\underset{n\to +\infty}{\longrightarrow}} %
\newcommand{\ntendn}{\underset{n\to
    +\infty}{\not\hspace{-.15cm}\longrightarrow}} %
% \newcommand{\tendn}{\xrightarrow[\footnotesize n \rightarrow
%   +\infty]{}}%
\newcommand{\Tendx}[1]{\xrightarrow[\footnotesize x \rightarrow
  #1]{}}%
\newcommand{\tendx}[1]{\underset{x\to #1}{\longrightarrow}}%
\newcommand{\ntendx}[1]{\underset{x\to #1}{\not\!\!\longrightarrow}}%
\newcommand{\tendd}[2]{\underset{#1\to #2}{\longrightarrow}}%
% \newcommand{\tendd}[2]{\xrightarrow[\footnotesize #1 \rightarrow
%   #2]{}}%
\newcommand{\tendash}[1]{\xdashrightarrow[\footnotesize #1 \rightarrow
  +\infty]{}}%
\newcommand{\tendashx}[1]{\xdashrightarrow[\footnotesize x \rightarrow
  #1]{}}%
\newcommand{\tendb}[1]{\underset{#1 \to +\infty}{\longrightarrow}}%
\newcommand{\tendL}{\overset{\mathscr L}{\underset{n \to
      +\infty}{\longrightarrow}}}%
\newcommand{\tendP}{\overset{\Prob}{\underset{n \to
      +\infty}{\longrightarrow}}}%
\newcommand{\tenddL}[1]{\overset{\mathscr L}{\underset{#1 \to
      +\infty}{\longrightarrow}}}%

\NewEnviron{attention}{ %
  ~\\[-.2cm]\noindent
  \begin{minipage}{\linewidth}
  \setlength{\fboxsep}{3mm}%
  \ \ \dbend \ \ %
  \fbox{\parbox[t]{.88\linewidth}{\BODY}} %
  \end{minipage}\\
}

\NewEnviron{sattention}[1]{ %
  ~\\[-.2cm]\noindent
  \begin{minipage}{#1\linewidth}
  \setlength{\fboxsep}{3mm}%
  \ \ \dbend \ \ %
  \fbox{\parbox[t]{.88\linewidth}{\BODY}} %
  \end{minipage}\\
}

%%%%% OBSOLETE %%%%%%

% \newcommand{\attention}[1]{
%   \noindent
%   \begin{tabular}{@{}l|p{11.5cm}|}
%     \cline{2-2}
%     \vspace{-.2cm} 
%     & \nl
%     \dbend & #1 \nl
%     \cline{2-2}
%   \end{tabular}
% }

% \newcommand{\attentionv}[2]{
%   \noindent
%   \begin{tabular}{@{}l|p{11.5cm}|}
%     \cline{2-2}
%     \vspace{-.2cm} 
%     & \nl
%     \dbend & #2 \nl[#1 cm]
%     \cline{2-2}
%   \end{tabular}
% }

\newcommand{\explainvb}[2]{
  \noindent
  \begin{tabular}{@{}l|p{11.5cm}|}
    \cline{2-2}
    \vspace{-.2cm} 
    & \nl
    \hand & #2 \nl [#1 cm]
    \cline{2-2}
  \end{tabular}
}


% \noindent
% \begin{tabular}{@{}l|lp{11cm}|}
%   \cline{3-3} 
%   \multicolumn{1}{@{}l@{\dbend}}{} & & #1 \nl
%   \multicolumn{1}{l}{} & & \nl [-.8cm]
%   & & #2 \nl
%   \cline{2-3}
% \end{tabular}

% \newcommand{\attention}[1]{
%   \noindent
%   \begin{tabular}{@{}@{}cp{11cm}}
%     \dbend & #1 \nl
%   \end{tabular}
% }

\newcommand{\PP}[1]{\mathcal{P}(#1)}
\newcommand{\HH}[1]{\mathcal{H}(#1)}
\newcommand{\FF}[1]{\mathcal{F}(#1)}

\newcommand{\DSum}[2]{\displaystyle\sum\limits_{#1}^{#2}\hspace{.1cm}}
\newcommand{\Sum}[2]{{\textstyle\sum\limits_{#1}^{#2}}\hspace{.1cm}}
\newcommand{\Serie}{\textstyle\sum\hspace{.1cm}}
\newcommand{\Prod}[2]{\textstyle\prod\limits_{#1}^{#2}}

\newcommand{\Prim}[3]{\left[\ {#1} \ \right]_{\scriptscriptstyle
   \hspace{-.15cm} ~_{#2}\, }^ {\scriptscriptstyle \hspace{-.15cm} ~^{#3}\, }}

% \newcommand{\Prim}[3]{\left[\ {#1} \ \right]_{\scriptscriptstyle
%     \!\!~_{#2}}^ {\scriptscriptstyle \!\!~^{#3}}}

\newcommand{\dint}[2]{\displaystyle \int_{#1}^{#2}\ }
\newcommand{\Int}[2]{{\rm{Int}}_{\scriptscriptstyle #1, #2}}
\newcommand{\dt}{\ dt}
\newcommand{\dx}{\ dx}

\newcommand{\llpar}[1]{\left(\!\!\!
    \begin{array}{c}
      \rule{0pt}{#1}
    \end{array}
  \!\!\!\right.}

\newcommand{\rrpar}[1]{\left.\!\!\!
    \begin{array}{c}
      \rule{0pt}{#1}
    \end{array}
  \!\!\!\right)}

\newcommand{\llacc}[1]{\left\{\!\!\!
    \begin{array}{c}
      \rule{0pt}{#1}
    \end{array}
  \!\!\!\right.}

\newcommand{\rracc}[1]{\left.\!\!\!
    \begin{array}{c}
      \rule{0pt}{#1}
    \end{array}
  \!\!\!\right\}}

\newcommand{\ttacc}[1]{\mbox{\rotatebox{-90}{\hspace{-.7cm}$\llacc{#1}$}}}
\newcommand{\bbacc}[1]{\mbox{\rotatebox{90}{\hspace{-.5cm}$\llacc{#1}$}}}

\newcommand{\comp}[1]{\overline{#1}}%

\newcommand{\dcomp}[2]{\stackrel{\mbox{\ \ \----}{\scriptscriptstyle
      #2}}{#1}}%

% \newcommand{\Comp}[2]{\stackrel{\mbox{\ \
%       \-------}{\scriptscriptstyle #2}}{#1}}

% \newcommand{\dcomp}[2]{\stackrel{\mbox{\ \
%       \-------}{\scriptscriptstyle #2}}{#1}}

\newcommand{\A}{\mathscr{A}}

\newcommand{\conc}[1]{
  \begin{center}
    \fbox{
      \begin{tabular}{c}
        #1
      \end{tabular}
    }
  \end{center}
}

\newcommand{\concC}[1]{
  \begin{center}
    \fbox{
    \begin{tabular}{C{10cm}}
      \quad #1 \quad
    \end{tabular}
    }
  \end{center}
}

\newcommand{\concL}[2]{
  \begin{center}
    \fbox{
    \begin{tabular}{C{#2cm}}
      \quad #1 \quad
    \end{tabular}
    }
  \end{center}
}


% \newcommand{\lims}[2]{\prod\limits_{#1}^{#2}}

\newtheorem{theorem}{Théorème}[]
\newtheorem{lemma}{Lemme}[]
\newtheorem{proposition}{Proposition}[]
\newtheorem{corollary}{Corollaire}[]

% \newenvironment{proof}[1][Démonstration]{\begin{trivlist}
% \item[\hskip \labelsep {\bfseries #1}]}{\end{trivlist}}
\newenvironment{definition}[1][Définition]{\begin{trivlist}
\item[\hskip \labelsep {\bfseries #1}]}{\end{trivlist}}
\newenvironment{example}[1][Exemple]{\begin{trivlist}
\item[\hskip \labelsep {\bfseries #1}]}{\end{trivlist}}
\newenvironment{examples}[1][Exemples]{\begin{trivlist}
\item[\hskip \labelsep {\bfseries #1}]}{\end{trivlist}}
\newenvironment{notation}[1][Notation]{\begin{trivlist}
\item[\hskip \labelsep {\bfseries #1}]}{\end{trivlist}}
\newenvironment{propriete}[1][Propriété]{\begin{trivlist}
\item[\hskip \labelsep {\bfseries #1}]}{\end{trivlist}}
\newenvironment{proprietes}[1][Propriétés]{\begin{trivlist}
\item[\hskip \labelsep {\bfseries #1}]}{\end{trivlist}}
% \newenvironment{remark}[1][Remarque]{\begin{trivlist}
% \item[\hskip \labelsep {\bfseries #1}]}{\end{trivlist}}
\newenvironment{application}[1][Application]{\begin{trivlist}
\item[\hskip \labelsep {\bfseries #1}]}{\end{trivlist}}

% Environnement pour les réponses des DS
\newenvironment{answer}{\par\emph{Réponse :}\par{}}
{\vspace{-.6cm}\hspace{\stretch{1}}\rule{1ex}{1ex}\vspace{.3cm}}

\newenvironment{answerTD}{\vspace{.2cm}\par\emph{Réponse :}\par{}}
{\hspace{\stretch{1}}\rule{1ex}{1ex}\vspace{.3cm}}

\newenvironment{answerCours}{\noindent\emph{Réponse :}}
{\rule{1ex}{1ex}}%\vspace{.3cm}}


% footnote in footer
\newcommand{\fancyfootnotetext}[2]{%
  \fancypagestyle{dingens}{%
    \fancyfoot[LO,RE]{\parbox{0.95\textwidth}{\footnotemark[#1]\footnotesize
        #2}}%
  }%
  \thispagestyle{dingens}%
}

%%% définit le style (arabic : 1,2,3...) et place des parenthèses
%%% autour de la numérotation
\renewcommand*{\thefootnote}{(\arabic{footnote})}
% http://www.tuteurs.ens.fr/logiciels/latex/footnote.html

%%%%%%%% tikz axis
% \pgfplotsset{every axis/.append style={
%                     axis x line=middle,    % put the x axis in the middle
%                     axis y line=middle,    % put the y axis in the middle
%                     axis line style={<->,color=blue}, % arrows on the axis
%                     xlabel={$x$},          % default put x on x-axis
%                     ylabel={$y$},          % default put y on y-axis
%             }}

%%%% s'utilise comme suit

% \begin{axis}[
%   xmin=-8,xmax=4,
%   ymin=-8,ymax=4,
%   grid=both,
%   ]
%   \addplot [domain=-3:3,samples=50]({x^3-3*x},{3*x^2-9}); 
% \end{axis}

%%%%%%%%



%%%%%%%%%%%% Pour avoir des numéros de section qui correspondent à
%%%%%%%%%%%% ceux du tableau
\renewcommand{\thesection}{\Roman{section}.\hspace{-.3cm}}
\renewcommand{\thesubsection}{\Roman{section}.\arabic{subsection}.\hspace{-.2cm}}
\renewcommand{\thesubsubsection}{\Roman{section}.\arabic{subsection}.\alph{subsubsection})\hspace{-.2cm}}
%%%%%%%%%%%% 

%%% Changer le nom des figures : Fig. au lieu de Figure
\usepackage[font=small,labelfont=bf,labelsep=space]{caption}
\captionsetup{%
  figurename=Fig.,
  tablename=Tab.
}
% \renewcommand{\thesection}{\Roman{section}.\hspace{-.2cm}}
% \renewcommand{\thesubsection}{\Roman{section}
%   .\hspace{.2cm}\arabic{subsection}\ .\hspace{-.3cm}}
% \renewcommand{\thesubsubsection}{\alph{subsection})}

\newenvironment{tabliste}[1]
{\begin{tabenum}[\bfseries\small\itshape #1]}{\end{tabenum}} 

%%%% ESSAI contre le too deeply nested %%%%
%%%% ATTENTION au package enumitem qui se comporte mal avec les
%%%% noliste, à redéfinir !
% \usepackage{enumitem}

% \setlistdepth{9}

% \newlist{myEnumerate}{enumerate}{9}
% \setlist[myEnumerate,1]{label=(\arabic*)}
% \setlist[myEnumerate,2]{label=(\Roman*)}
% \setlist[myEnumerate,3]{label=(\Alph*)}
% \setlist[myEnumerate,4]{label=(\roman*)}
% \setlist[myEnumerate,5]{label=(\alph*)}
% \setlist[myEnumerate,6]{label=(\arabic*)}
% \setlist[myEnumerate,7]{label=(\Roman*)}
% \setlist[myEnumerate,8]{label=(\Alph*)}
% \setlist[myEnumerate,9]{label=(\roman*)}

%%%%%

\newenvironment{noliste}[1] %
{\begin{enumerate}[\bfseries\small\itshape #1]} %
  {\end{enumerate}}

\newenvironment{nonoliste}[1] %
{\begin{enumerate}[\hspace{-12pt}\bfseries\small\itshape #1]} %
  {\end{enumerate}}

\newenvironment{arrayliste}[1]{ 
  % List with minimal white space to fit in small areas, e.g. table
  % cell
  \begin{minipage}[t]{\linewidth} %
    \begin{enumerate}[\bfseries\small\itshape #1] %
      {\leftmargin=0.5em \rightmargin=0em
        \topsep=0em \parskip=0em \parsep=0em
        \listparindent=0em \partopsep=0em \itemsep=0pt
        \itemindent=0em \labelwidth=\leftmargin\labelsep+0.25em}
    }{
    \end{enumerate}\end{minipage}
}

\newenvironment{nolistes}[2]
{\begin{enumerate}[\bfseries\small\itshape
    #1]\setlength{\itemsep}{#2 mm}}{\end{enumerate}}

\newenvironment{liste}[1]
{\begin{enumerate}[\hspace{12pt}\bfseries\small\itshape
    #1]}{\end{enumerate}}   


%%%%%%%% Pour les programmes de colle %%%%%%%

\newcommand{\cours}{{\small \tt (COURS)}} %
\newcommand{\poly}{{\small \tt (POLY)}} %
\newcommand{\exo}{{\small \tt (EXO)}} %
\newcommand{\culture}{{\small \tt (CULTURE)}} %
\newcommand{\methodo}{{\small \tt (MÉTHODO)}} %
\newcommand{\methodob}{\Boxed{\mbox{\tt MÉTHODO}}} %

%%%%%%%% Pour les TD %%%%%%%
\newtheoremstyle{exostyle} {\topsep} % espace avant
{.6cm} % espace apres
{} % Police utilisee par le style de thm
{} % Indentation (vide = aucune, \parindent = indentation paragraphe)
{\bfseries} % Police du titre de thm
{} % Signe de ponctuation apres le titre du thm
{ } % Espace apres le titre du thm (\newline = linebreak)
{\thmname{#1}\thmnumber{ #2}\thmnote{.
    \normalfont{\textit{#3}}}} % composants du titre du thm : \thmname
                               % = nom du thm, \thmnumber = numéro du
                               % thm, \thmnote = sous-titre du thm
 
\theoremstyle{exostyle}
\newtheorem{exercice}{Exercice}
\newtheorem*{exoCours}{Exercice}

%%%%%%%% Pour des théorèmes Sans Espaces APRÈS %%%%%%%
\newtheoremstyle{exostyleSE} {\topsep} % espace avant
{} % espace apres
{} % Police utilisee par le style de thm
{} % Indentation (vide = aucune, \parindent = indentation paragraphe)
{\bfseries} % Police du titre de thm
{} % Signe de ponctuation apres le titre du thm
{ } % Espace apres le titre du thm (\newline = linebreak)
{\thmname{#1}\thmnumber{ #2}\thmnote{.
    \normalfont{\textit{#3}}}} % composants du titre du thm : \thmname
                               % = nom du thm, \thmnumber = numéro du
                               % thm, \thmnote = sous-titre du thm
 
\theoremstyle{exostyleSE}
\newtheorem{exerciceSE}{Exercice}
\newtheorem*{exoCoursSE}{Exercice}

% \newcommand{\lims}[2]{\prod\limits_{#1}^{#2}}

\newtheorem{theoremSE}{Théorème}[]
\newtheorem{lemmaSE}{Lemme}[]
\newtheorem{propositionSE}{Proposition}[]
\newtheorem{corollarySE}{Corollaire}[]

% \newenvironment{proofSE}[1][Démonstration]{\begin{trivlist}
% \item[\hskip \labelsep {\bfseries #1}]}{\end{trivlist}}
\newenvironment{definitionSE}[1][Définition]{\begin{trivlist}
  \item[\hskip \labelsep {\bfseries #1}]}{\end{trivlist}}
\newenvironment{exampleSE}[1][Exemple]{\begin{trivlist} 
  \item[\hskip \labelsep {\bfseries #1}]}{\end{trivlist}}
\newenvironment{examplesSE}[1][Exemples]{\begin{trivlist}
\item[\hskip \labelsep {\bfseries #1}]}{\end{trivlist}}
\newenvironment{notationSE}[1][Notation]{\begin{trivlist}
\item[\hskip \labelsep {\bfseries #1}]}{\end{trivlist}}
\newenvironment{proprieteSE}[1][Propriété]{\begin{trivlist}
\item[\hskip \labelsep {\bfseries #1}]}{\end{trivlist}}
\newenvironment{proprietesSE}[1][Propriétés]{\begin{trivlist}
\item[\hskip \labelsep {\bfseries #1}]}{\end{trivlist}}
\newenvironment{remarkSE}[1][Remarque]{\begin{trivlist}
\item[\hskip \labelsep {\bfseries #1}]}{\end{trivlist}}
\newenvironment{applicationSE}[1][Application]{\begin{trivlist}
\item[\hskip \labelsep {\bfseries #1}]}{\end{trivlist}}

%%%%%%%%%%% Obtenir les étoiles sans charger le package MnSymbol
%%%%%%%%%%%
\DeclareFontFamily{U} {MnSymbolC}{}
\DeclareFontShape{U}{MnSymbolC}{m}{n}{
  <-6> MnSymbolC5
  <6-7> MnSymbolC6
  <7-8> MnSymbolC7
  <8-9> MnSymbolC8
  <9-10> MnSymbolC9
  <10-12> MnSymbolC10
  <12-> MnSymbolC12}{}
\DeclareFontShape{U}{MnSymbolC}{b}{n}{
  <-6> MnSymbolC-Bold5
  <6-7> MnSymbolC-Bold6
  <7-8> MnSymbolC-Bold7
  <8-9> MnSymbolC-Bold8
  <9-10> MnSymbolC-Bold9
  <10-12> MnSymbolC-Bold10
  <12-> MnSymbolC-Bold12}{}

\DeclareSymbolFont{MnSyC} {U} {MnSymbolC}{m}{n}

\DeclareMathSymbol{\filledlargestar}{\mathrel}{MnSyC}{205}
\DeclareMathSymbol{\largestar}{\mathrel}{MnSyC}{131}

\newcommand{\facile}{\rm{(}$\scriptstyle\largestar$\rm{)}} %
\newcommand{\moyen}{\rm{(}$\scriptstyle\filledlargestar$\rm{)}} %
\newcommand{\dur}{\rm{(}$\scriptstyle\filledlargestar\filledlargestar$\rm{)}} %
\newcommand{\costaud}{\rm{(}$\scriptstyle\filledlargestar\filledlargestar\filledlargestar$\rm{)}}

%%%%%%%%%%%%%%%%%%%%%%%%%

%%%%%%%%%%%%%%%%%%%%%%%%%
%%%%%%%% Fin de la partie TD

%%%%%%%%%%%%%%%%
%%%%%%%%%%%%%%%%
\makeatletter %
\newenvironment{myitemize}{%
  \setlength{\topsep}{0pt} %
  \setlength{\partopsep}{0pt} %
  \renewcommand*{\@listi}{\leftmargin\leftmargini \parsep\z@
    \topsep\z@ \itemsep\z@} \let\@listI\@listi %
  \itemize %
}{\enditemize} %
\makeatother
%%%%%%%%%%%%%%%%
%%%%%%%%%%%%%%%%

%% Commentaires dans la correction du livre

\newcommand{\Com}[1]{
% Define box and box title style
\tikzstyle{mybox} = [draw=black!50,
very thick,
    rectangle, rounded corners, inner sep=10pt, inner ysep=8pt]
\tikzstyle{fancytitle} =[rounded corners, fill=black!80, text=white]
\tikzstyle{fancylogo} =[ text=white]
\begin{center}

\begin{tikzpicture}
\node [mybox] (box){%

    \begin{minipage}{0.90\linewidth}
\vspace{6pt}  #1
    \end{minipage}
};
\node[fancytitle, right=10pt] at (box.north west) 
{\bfseries\normalsize{Commentaire}};

\end{tikzpicture}%

\end{center}
%
}

\NewEnviron{remark}{%
  % Define box and box title style
  \tikzstyle{mybox} = [draw=black!50, very thick, rectangle, rounded
  corners, inner sep=10pt, inner ysep=8pt] %
  \tikzstyle{fancytitle} = [rounded corners , fill=black!80,
  text=white] %
  \tikzstyle{fancylogo} =[ text=white]
  \begin{center}
    \begin{tikzpicture}
      \node [mybox] (box){%
        \begin{minipage}{0.90\linewidth}
          \vspace{6pt} \BODY
        \end{minipage}
      }; %
      \node[fancytitle, right=10pt] at (box.north west) %
      {\bfseries\normalsize{Commentaire}}; %
    \end{tikzpicture}%
  \end{center}
}

\NewEnviron{remarkST}{%
  % Define box and box title style
  \tikzstyle{mybox} = [draw=black!50, very thick, rectangle, rounded
  corners, inner sep=10pt, inner ysep=8pt] %
  \tikzstyle{fancytitle} = [rounded corners , fill=black!80,
  text=white] %
  \tikzstyle{fancylogo} =[ text=white]
  \begin{center}
    \begin{tikzpicture}
      \node [mybox] (box){%
        \begin{minipage}{0.90\linewidth}
          \vspace{6pt} \BODY
        \end{minipage}
      }; %
      % \node[fancytitle, right=10pt] at (box.north west) %
%       {\bfseries\normalsize{Commentaire}}; %
    \end{tikzpicture}%
  \end{center}
}

\NewEnviron{remarkL}[1]{%
  % Define box and box title style
  \tikzstyle{mybox} = [draw=black!50, very thick, rectangle, rounded
  corners, inner sep=10pt, inner ysep=8pt] %
  \tikzstyle{fancytitle} =[rounded corners, fill=black!80,
  text=white] %
  \tikzstyle{fancylogo} =[ text=white]
  \begin{center}
    \begin{tikzpicture}
      \node [mybox] (box){%
        \begin{minipage}{#1\linewidth}
          \vspace{6pt} \BODY
        \end{minipage}
      }; %
      \node[fancytitle, right=10pt] at (box.north west) %
      {\bfseries\normalsize{Commentaire}}; %
    \end{tikzpicture}%
  \end{center}
}

\NewEnviron{remarkSTL}[1]{%
  % Define box and box title style
  \tikzstyle{mybox} = [draw=black!50, very thick, rectangle, rounded
  corners, inner sep=10pt, inner ysep=8pt] %
  \tikzstyle{fancytitle} =[rounded corners, fill=black!80,
  text=white] %
  \tikzstyle{fancylogo} =[ text=white]
  \begin{center}
    \begin{tikzpicture}
      \node [mybox] (box){%
        \begin{minipage}{#1\linewidth}
          \vspace{6pt} \BODY
        \end{minipage}
      }; %
%       \node[fancytitle, right=10pt] at (box.north west) %
%       {\bfseries\normalsize{Commentaire}}; %
    \end{tikzpicture}%
  \end{center}
}

\NewEnviron{titre} %
{ %
  ~\\[-1.8cm]
  \begin{center}
    \bf \LARGE \BODY
  \end{center}
  ~\\[-.6cm]
  \hrule %
  \vspace*{.2cm}
} %

\NewEnviron{titreL}[2] %
{ %
  ~\\[-#1cm]
  \begin{center}
    \bf \LARGE \BODY
  \end{center}
  ~\\[-#2cm]
  \hrule %
  \vspace*{.2cm}
} %



%%%%%%%%%%% Redefinition \chapter



\usepackage[explicit]{titlesec}
\usepackage{color}
\titleformat{\chapter}
{\gdef\chapterlabel{}
\selectfont\huge\bf}
%\normalfont\sffamily\Huge\bfseries\scshape}
{\gdef\chapterlabel{\thechapter)\ }}{0pt}
{\begin{tikzpicture}[remember picture,overlay]
\node[yshift=-3cm] at (current page.north west)
{\begin{tikzpicture}[remember picture, overlay]
\draw (.1\paperwidth,0) -- (.9\paperwidth,0);
\draw (.1\paperwidth,2) -- (.9\paperwidth,2);
%(\paperwidth,3cm);
\node[anchor=center,xshift=.5\paperwidth,yshift=1cm, rectangle,
rounded corners=20pt,inner sep=11pt]
{\color{black}\chapterlabel#1};
\end{tikzpicture}
};
\end{tikzpicture}
}
\titlespacing*{\chapter}{0pt}{50pt}{-75pt}




%%%%%%%%%%%%%% Affichage chapter dans Table des matieres

\makeatletter
\renewcommand*\l@chapter[2]{%
  \ifnum \c@tocdepth >\m@ne
    \addpenalty{-\@highpenalty}%
    \vskip 1.0em \@plus\p@
    \setlength\@tempdima{1.5em}%
    \begingroup
      \parindent \z@ \rightskip \@pnumwidth
      \parfillskip -\@pnumwidth
      \leavevmode %\bfseries
      \advance\leftskip\@tempdima
      \hskip -\leftskip
      #1\nobreak\ 
       \leaders\hbox{$\m@th
        \mkern \@dotsep mu\hbox{.}\mkern \@dotsep
        mu$}\hfil\nobreak\hb@xt@\@pnumwidth{\hss #2}\par
      \penalty\@highpenalty
    \endgroup
  \fi}
\makeatother




%%%%%%%%%%%%%%%%% Redefinition part



\usepackage{array}
\usepackage{paralist}

\usepackage[svgnames, usenames, dvipsnames]{xcolor}
\xdefinecolor{RecColor}{named}{Aqua}
\xdefinecolor{IncColor}{named}{Aqua}
\xdefinecolor{ImpColor}{named}{PaleGreen}

% \usepackage{frcursive}

\usepackage{adjustbox}

%%%%%%%%%%%
\newcommand{\cRB}[1]{{\color{Red} \pmb{#1}}} %
\newcommand{\cR}[1]{{\color{Red} {#1}}} %
\newcommand{\cBB}[1]{{\color{Blue} \pmb{#1}}}
\newcommand{\cB}[1]{{\color{Blue} {#1}}}
\newcommand{\cGB}[1]{{\color{LimeGreen} \pmb{#1}}}
\newcommand{\cG}[1]{{\color{LimeGreen} {#1}}}

%%%%%%%%%%

\usepackage{diagbox} %
\usepackage{colortbl} %
\usepackage{multirow} %
\usepackage{pgf} %
\usepackage{environ} %
\usepackage{fancybox} %
\usepackage{textcomp} %
\usepackage{marvosym} %

%%%%%%%%%% pour pouvoir faire des dashedline dans les tableaux
\usepackage{arydshln}

%%%%%%%%%% pour qu'une cellcolor ne recouvre pas le trait du tableau
\usepackage{hhline}%

\usepackage{pgfplots}
\pgfplotsset{compat=1.10}
\usepgfplotslibrary{patchplots}
\usepgfplotslibrary{fillbetween}
\usepackage{tikz,tkz-tab}
\usepackage{ifthen}
\usepackage{calc}
\usetikzlibrary{calc,decorations.pathreplacing,arrows,positioning} 
\usetikzlibrary{fit,shapes,backgrounds}
% \usepackage[nomessages]{fp}% http://ctan.org/pkg/fp

\usetikzlibrary{matrix,arrows,decorations.pathmorphing,
  decorations.pathreplacing} 

\newcommand{\myunit}{1 cm}
\tikzset{
    node style sp/.style={draw,circle,minimum size=\myunit},
    node style ge/.style={circle,minimum size=\myunit},
    arrow style mul/.style={draw,sloped,midway,fill=white},
    arrow style plus/.style={midway,sloped,fill=white},
}

%%%%%%%%%%%%%%
%%%%% écrire des inférieur égal ou supérieur égal avec typographie
%%%%% francaise
%%%%%%%%%%%%%

\renewcommand{\geq}{\geqslant}
\renewcommand{\leq}{\leqslant}
\renewcommand{\emptyset}{\varnothing}

\newcommand{\Leq}{\leqslant}
\newcommand{\Geq}{\geqslant}

%%%%%%%%%%%%%%
%%%%% Macro Celia
%%%%%%%%%%%%%

\newcommand{\ff}[2]{\left[#1, #2\right]} %
\newcommand{\fo}[2]{\left[#1, #2\right[} %
\newcommand{\of}[2]{\left]#1, #2\right]} %
\newcommand{\soo}[2]{\left]#1, #2\right[} %
\newcommand{\abs}[1]{\left|#1\right|} %
\newcommand{\Ent}[1]{\left\lfloor #1 \right\rfloor} %


%%%%%%%%%%%%%%
%%%%% tikz : comment dessiner un "oeil"
%%%%%%%%%%%%%

\newcommand{\eye}[4]% size, x, y, rotation
{ \draw[rotate around={#4:(#2,#3)}] (#2,#3) -- ++(-.5*55:#1) (#2,#3)
  -- ++(.5*55:#1); \draw (#2,#3) ++(#4+55:.75*#1) arc
  (#4+55:#4-55:.75*#1);
  % IRIS
  \draw[fill=gray] (#2,#3) ++(#4+55/3:.75*#1) arc
  (#4+180-55:#4+180+55:.28*#1);
  % PUPIL, a filled arc
  \draw[fill=black] (#2,#3) ++(#4+55/3:.75*#1) arc
  (#4+55/3:#4-55/3:.75*#1);%
}


%%%%%%%%%%
%% discontinuité fonction
\newcommand\pointg[2]{%
  \draw[color = red, very thick] (#1+0.15, #2-.04)--(#1, #2-.04)--(#1,
  #2+.04)--(#1+0.15, #2+.04);%
}%

\newcommand\pointd[2]{%
  \draw[color = red, very thick] (#1-0.15, #2+.04)--(#1, #2+.04)--(#1,
  #2-.04)--(#1-0.15, #2-.04);%
}%

%%%%%%%%%%
%%% 1 : position abscisse, 2 : position ordonnée, 3 : taille, 4 : couleur
%%%%%%%%%%
% \newcommand\pointG[4]{%
%   \draw[color = #4, very thick] (#1+#3, #2-(#3/3.75))--(#1,
%   #2-(#3/3.75))--(#1, #2+(#3/3.75))--(#1+#3, #2+(#3/3.75)) %
% }%

\newcommand\pointG[4]{%
  \draw[color = #4, very thick] ({#1+#3/3.75}, {#2-#3})--(#1,
  {#2-#3})--(#1, {#2+#3})--({#1+#3/3.75}, {#2+#3}) %
}%

\newcommand\pointD[4]{%
  \draw[color = #4, very thick] ({#1-#3/3.75}, {#2+#3})--(#1,
  {#2+#3})--(#1, {#2-#3})--({#1-#3/3.75}, {#2-#3}) %
}%

\newcommand\spointG[4]{%
  \draw[color = #4, very thick] ({#1+#3/1.75}, {#2-#3})--(#1,
  {#2-#3})--(#1, {#2+#3})--({#1+#3/1.75}, {#2+#3}) %
}%

\newcommand\spointD[4]{%
  \draw[color = #4, very thick] ({#1-#3/2}, {#2+#3})--(#1,
  {#2+#3})--(#1, {#2-#3})--({#1-#3/2}, {#2-#3}) %
}%

%%%%%%%%%%

\newcommand{\Pb}{\mathtt{P}}

%%%%%%%%%%%%%%%
%%% Pour citer un précédent item
%%%%%%%%%%%%%%%
\newcommand{\itbf}[1]{{\small \bf \textit{#1}}}


%%%%%%%%%%%%%%%
%%% Quelques couleurs
%%%%%%%%%%%%%%%

\xdefinecolor{cancelcolor}{named}{Red}
\xdefinecolor{intI}{named}{ProcessBlue}
\xdefinecolor{intJ}{named}{ForestGreen}

%%%%%%%%%%%%%%%
%%%%%%%%%%%%%%%
% barrer du texte
\usetikzlibrary{shapes.misc}

\makeatletter
% \definecolor{cancelcolor}{rgb}{0.127,0.372,0.987}
\newcommand{\tikz@bcancel}[1]{%
  \begin{tikzpicture}[baseline=(textbox.base),inner sep=0pt]
  \node[strike out,draw] (textbox) {#1}[thick, color=cancelcolor];
  \useasboundingbox (textbox);
  \end{tikzpicture}%
}
\newcommand{\bcancel}[1]{%
  \relax\ifmmode
    \mathchoice{\tikz@bcancel{$\displaystyle#1$}}
               {\tikz@bcancel{$\textstyle#1$}}
               {\tikz@bcancel{$\scriptstyle#1$}}
               {\tikz@bcancel{$\scriptscriptstyle#1$}}
  \else
    \tikz@bcancel{\strut#1}%
  \fi
}
\newcommand{\tikz@xcancel}[1]{%
  \begin{tikzpicture}[baseline=(textbox.base),inner sep=0pt]
  \node[cross out,draw] (textbox) {#1}[thick, color=cancelcolor];
  \useasboundingbox (textbox);
  \end{tikzpicture}%
}
\newcommand{\xcancel}[1]{%
  \relax\ifmmode
    \mathchoice{\tikz@xcancel{$\displaystyle#1$}}
               {\tikz@xcancel{$\textstyle#1$}}
               {\tikz@xcancel{$\scriptstyle#1$}}
               {\tikz@xcancel{$\scriptscriptstyle#1$}}
  \else
    \tikz@xcancel{\strut#1}%
  \fi
}
\makeatother

\newcommand{\xcancelRA}{\xcancel{\rule[-.15cm]{0cm}{.5cm} \Rightarrow
    \rule[-.15cm]{0cm}{.5cm}}}

%%%%%%%%%%%%%%%%%%%%%%%%%%%%%%%%%%%
%%%%%%%%%%%%%%%%%%%%%%%%%%%%%%%%%%%

\newcommand{\vide}{\multicolumn{1}{c}{}}

%%%%%%%%%%%%%%%%%%%%%%%%%%%%%%%%%%%
%%%%%%%%%%%%%%%%%%%%%%%%%%%%%%%%%%%


\usepackage{multicol}
% \usepackage[latin1]{inputenc}
% \usepackage[T1]{fontenc}
\usepackage[utf8]{inputenc}
\usepackage[T1]{fontenc}
\usepackage[normalem]{ulem}
\usepackage[french]{babel}

\usepackage{url}    
\usepackage{hyperref}
\hypersetup{
  backref=true,
  pagebackref=true,
  hyperindex=true,
  colorlinks=true,
  breaklinks=true,
  urlcolor=blue,
  linkcolor=red,
  bookmarks=true,
  bookmarksopen=true
}

%%%%%%%%%%%%%%%%%%%%%%%%%%%%%%%%%%%%%%%%%%%
%% Pour faire des traits diagonaux dans les tableaux
%% Nécessite slashbox.sty
%\usepackage{slashbox}

\usepackage{tipa}
\usepackage{verbatim,listings}
\usepackage{graphicx}
\usepackage{fancyhdr}
\usepackage{mathrsfs}
\usepackage{pifont}
\usepackage{tablists}
\usepackage{dsfont,amsfonts,amssymb,amsmath,amsthm,stmaryrd,upgreek,manfnt}
\usepackage{enumerate}

%\newcolumntype{M}[1]{p{#1}}
\newcolumntype{C}[1]{>{\centering}m{#1}}
\newcolumntype{R}[1]{>{\raggedright}m{#1}}
\newcolumntype{L}[1]{>{\raggedleft}m{#1}}
\newcolumntype{P}[1]{>{\raggedright}p{#1}}
\newcolumntype{B}[1]{>{\raggedright}b{#1}}
\newcolumntype{Q}[1]{>{\raggedright}t{#1}}

\newcommand{\alias}[2]{
\providecommand{#1}{}
\renewcommand{#1}{#2}
}
\alias{\R}{\mathbb{R}}
\alias{\N}{\mathbb{N}}
\alias{\Z}{\mathbb{Z}}
\alias{\Q}{\mathbb{Q}}
\alias{\C}{\mathbb{C}}
\alias{\K}{\mathbb{K}}

%%%%%%%%%%%%
%% rendre +infty et -infty plus petits
%%%%%%%%%%%%
\newcommand{\sinfty}{{\scriptstyle \infty}}

%%%%%%%%%%%%%%%%%%%%%%%%%%%%%%%
%%%%% macros TP Scilab %%%%%%%%
\newcommand{\Scilab}{\textbf{Scilab}} %
\newcommand{\Scinotes}{\textbf{SciNotes}} %
\newcommand{\faire}{\noindent $\blacktriangleright$ } %
\newcommand{\fitem}{\scalebox{.8}{$\blacktriangleright$}} %
\newcommand{\entree}{{\small\texttt{ENTRÉE}}} %
\newcommand{\tab}{{\small\texttt{TAB}}} %
\newcommand{\mt}[1]{\mathtt{#1}} %
% guillemets droits

\newcommand{\ttq}{\textquotesingle} %

\newcommand{\reponse}[1]{\longboxed{
    \begin{array}{C{0.9\textwidth}}
      \nl[#1]
    \end{array}
  }} %

\newcommand{\reponseL}[2]{\longboxed{
    \begin{array}{C{#2\textwidth}}
      \nl[#1]
    \end{array}
  }} %

\newcommand{\reponseR}[1]{\longboxed{
    \begin{array}{R{0.9\textwidth}}
      #1
    \end{array}
  }} %

\newcommand{\reponseRL}[2]{\longboxed{
    \begin{array}{R{#2\textwidth}}
      #1
    \end{array}
  }} %

\newcommand{\reponseC}[1]{\longboxed{
    \begin{array}{C{0.9\textwidth}}
      #1
    \end{array}
  }} %

\colorlet{pyfunction}{Blue}
\colorlet{pyCle}{Magenta}
\colorlet{pycomment}{LimeGreen}
\colorlet{pydoc}{Cyan}
% \colorlet{SansCo}{white}
% \colorlet{AvecCo}{black}

\newcommand{\visible}[1]{{\color{ASCo}\colorlet{pydoc}{pyDo}\colorlet{pycomment}{pyCo}\colorlet{pyfunction}{pyF}\colorlet{pyCle}{pyC}\colorlet{function}{sciFun}\colorlet{var}{sciVar}\colorlet{if}{sciIf}\colorlet{comment}{sciComment}#1}} %

%%%% à changer ????
\newcommand{\invisible}[1]{{\color{ASCo}\colorlet{pydoc}{pyDo}\colorlet{pycomment}{pyCo}\colorlet{pyfunction}{pyF}\colorlet{pyCle}{pyC}\colorlet{function}{sciFun}\colorlet{var}{sciVar}\colorlet{if}{sciIf}\colorlet{comment}{sciComment}#1}} %

\newcommand{\invisibleCol}[2]{{\color{#1}#2}} %

\NewEnviron{solution} %
{ %
  \Boxed{
    \begin{array}{>{\color{ASCo}} R{0.9\textwidth}}
      \colorlet{pycomment}{pyCo}
      \colorlet{pydoc}{pyDo}
      \colorlet{pyfunction}{pyF}
      \colorlet{pyCle}{pyC}
      \colorlet{function}{sciFun}
      \colorlet{var}{sciVar}
      \colorlet{if}{sciIf}
      \colorlet{comment}{sciComment}
      \BODY
    \end{array}
  } %
} %

\NewEnviron{solutionC} %
{ %
  \Boxed{
    \begin{array}{>{\color{ASCo}} C{0.9\textwidth}}
      \colorlet{pycomment}{pyCo}
      \colorlet{pydoc}{pyDo}
      \colorlet{pyfunction}{pyF}
      \colorlet{pyCle}{pyC}
      \colorlet{function}{sciFun}
      \colorlet{var}{sciVar}
      \colorlet{if}{sciIf}
      \colorlet{comment}{sciComment}
      \BODY
    \end{array}
  } %
} %

\newcommand{\invite}{--\!\!>} %

%%%%% nouvel environnement tabular pour retour console %%%%
\colorlet{ConsoleColor}{Blue!15}
\colorlet{function}{Red}
\colorlet{var}{Maroon}
\colorlet{if}{Magenta}
\colorlet{comment}{LimeGreen}

\newcommand{\tcVar}[1]{\textcolor{var}{\bf \small #1}} %
\newcommand{\tcFun}[1]{\textcolor{function}{#1}} %
\newcommand{\tcIf}[1]{\textcolor{if}{#1}} %
\newcommand{\tcFor}[1]{\textcolor{if}{#1}} %

\newcommand{\moins}{\!\!\!\!\!\!- }
\newcommand{\espn}{\!\!\!\!\!\!}

\usepackage{booktabs,varwidth} \newsavebox\TBox
\newenvironment{console}
{\begin{lrbox}{\TBox}\varwidth{\linewidth}
    \tabular{>{\tt\small}R{0.84\textwidth}}
    \nl[-.4cm]} {\endtabular\endvarwidth\end{lrbox}%
  \fboxsep=1pt\colorbox{ConsoleColor}{\usebox\TBox}}

\newcommand{\lInv}[1]{%
  $\invite$ #1} %

\newcommand{\lAns}[1]{%
  \qquad ans \ = \nl %
  \qquad \qquad #1} %

\newcommand{\lVar}[2]{%
  \qquad #1 \ = \nl %
  \qquad \qquad #2} %

\newcommand{\lDisp}[1]{%
  #1 %
} %

\newcommand{\ligne}[1]{\underline{\small \tt #1}} %

\newcommand{\ligneAns}[2]{%
  $\invite$ #1 \nl %
  \qquad ans \ = \nl %
  \qquad \qquad #2} %

\newcommand{\ligneVar}[3]{%
  $\invite$ #1 \nl %
  \qquad #2 \ = \nl %
  \qquad \qquad #3} %

\newcommand{\ligneErr}[3]{%
  $\invite$ #1 \nl %
  \quad !-{-}error #2 \nl %
  #3} %
%%%%%%%%%%%%%%%%%%%%%% 

\newcommand{\bs}[1]{\boldsymbol{#1}} %
\newcommand{\nll}{\nl[.4cm]} %
\newcommand{\nle}{\nl[.2cm]} %
%% opérateur puissance copiant l'affichage Scilab
%\newcommand{\puis}{\!\!\!~^{\scriptscriptstyle\pmb{\wedge}}}
\newcommand{\puis}{\mbox{$\hspace{-.1cm}~^{\scriptscriptstyle\pmb{\wedge}}
    \hspace{0.05cm}$}} %
\newcommand{\pointpuis}{.\mbox{$\hspace{-.15cm}~^{\scriptscriptstyle\pmb{\wedge}}$}} %
\newcommand{\Sfois}{\mbox{$\mt{\star}$}} %

%%%%% nouvel environnement tabular pour les encadrés Scilab %%%%
\newenvironment{encadre}
{\begin{lrbox}{\TBox}\varwidth{\linewidth}
    \tabular{>{\tt\small}C{0.1\textwidth}>{\small}R{0.7\textwidth}}}
  {\endtabular\endvarwidth\end{lrbox}%
  \fboxsep=1pt\longboxed{\usebox\TBox}}

\newenvironment{encadreL}
{\begin{lrbox}{\TBox}\varwidth{\linewidth}
    \tabular{>{\tt\small}C{0.25\textwidth}>{\small}R{0.6\textwidth}}}
  {\endtabular\endvarwidth\end{lrbox}%
  \fboxsep=1pt\longboxed{\usebox\TBox}}

\newenvironment{encadreF}
{\begin{lrbox}{\TBox}\varwidth{\linewidth}
    \tabular{>{\tt\small}C{0.2\textwidth}>{\small}R{0.70\textwidth}}}
  {\endtabular\endvarwidth\end{lrbox}%
  \fboxsep=1pt\longboxed{\usebox\TBox}}

\newenvironment{encadreLL}[2]
{\begin{lrbox}{\TBox}\varwidth{\linewidth}
    \tabular{>{\tt\small}C{#1\textwidth}>{\small}R{#2\textwidth}}}
  {\endtabular\endvarwidth\end{lrbox}%
  \fboxsep=1pt\longboxed{\usebox\TBox}}

%%%%% nouvel environnement tabular pour les script et fonctions %%%%
\newcommand{\commentaireDL}[1]{\multicolumn{1}{l}{\it
    \textcolor{comment}{$\slash\slash$ #1}}}

\newcommand{\commentaire}[1]{{\textcolor{comment}{$\slash\slash$ #1}}}

\newcounter{cptcol}

\newcommand{\nocount}{\multicolumn{1}{c}{}}

\newcommand{\sciNo}[1]{{\small \underbar #1}}

\NewEnviron{scilab}{ %
  \setcounter{cptcol}{0}
  \begin{center}
    \longboxed{
      \begin{tabular}{>{\stepcounter{cptcol}{\tiny \underbar
              \thecptcol}}c>{\tt}l}
        \BODY
      \end{tabular}
    }
  \end{center}
}

\NewEnviron{scilabNC}{ %
  \begin{center}
    \longboxed{
      \begin{tabular}{>{\tt}l} %
          \BODY
      \end{tabular}
    }
  \end{center}
}

\NewEnviron{scilabC}[1]{ %
  \setcounter{cptcol}{#1}
  \begin{center}
    \longboxed{
      \begin{tabular}{>{\stepcounter{cptcol}{\tiny \underbar
              \thecptcol}}c>{\tt}l}
        \BODY
      \end{tabular}
    }
  \end{center}
}

\newcommand{\scisol}[1]{ %
  \setcounter{cptcol}{0}
  \longboxed{
    \begin{tabular}{>{\stepcounter{cptcol}{\tiny \underbar
            \thecptcol}}c>{\tt}l}
      #1
    \end{tabular}
  }
}

\newcommand{\scisolNC}[1]{ %
  \longboxed{
    \begin{tabular}{>{\tt}l}
      #1
    \end{tabular}
  }
}

\newcommand{\scisolC}[2]{ %
  \setcounter{cptcol}{#1}
  \longboxed{
    \begin{tabular}{>{\stepcounter{cptcol}{\tiny \underbar
            \thecptcol}}c>{\tt}l}
      #2
    \end{tabular}
  }
}

\NewEnviron{syntaxe}{ %
  % \fcolorbox{black}{Yellow!20}{\setlength{\fboxsep}{3mm}
  \shadowbox{
    \setlength{\fboxsep}{3mm}
    \begin{tabular}{>{\tt}l}
      \BODY
    \end{tabular}
  }
}

%%%%% fin macros TP Scilab %%%%%%%%
%%%%%%%%%%%%%%%%%%%%%%%%%%%%%%%%%%%

%%%%%%%%%%%%%%%%%%%%%%%%%%%%%%%%%%%
%%%%% TP Python - listings %%%%%%%%
%%%%%%%%%%%%%%%%%%%%%%%%%%%%%%%%%%%
\newcommand{\Python}{\textbf{Python}} %

\lstset{% general command to set parameter(s)
basicstyle=\ttfamily\small, % print whole listing small
keywordstyle=\color{blue}\bfseries\underbar,
%% underlined bold black keywords
frame=lines,
xleftmargin=10mm,
numbers=left,
numberstyle=\tiny\underbar,
numbersep=10pt,
%identifierstyle=, % nothing happens
commentstyle=\color{green}, % white comments
%%stringstyle=\ttfamily, % typewriter type for strings
showstringspaces=false}

\newcommand{\pysolCpt}[2]{
  \setcounter{cptcol}{#1}
  \longboxed{
    \begin{tabular}{>{\stepcounter{cptcol}{\tiny \underbar
            \thecptcol}}c>{\tt}l}
        #2
      \end{tabular}
    }
} %

\newcommand{\pysol}[1]{
  \setcounter{cptcol}{0}
  \longboxed{
    \begin{tabular}{>{\stepcounter{cptcol}{\tiny \underbar
            \thecptcol}}c>{\tt}l}
        #1
      \end{tabular}
    }
} %

% \usepackage[labelsep=endash]{caption}

% avec un caption
\NewEnviron{pythonCap}[1]{ %
  \renewcommand{\tablename}{Programme}
  \setcounter{cptcol}{0}
  \begin{center}
    \longboxed{
      \begin{tabular}{>{\stepcounter{cptcol}{\tiny \underbar
              \thecptcol}}c>{\tt}l}
        \BODY
      \end{tabular}
    }
    \captionof{table}{#1}
  \end{center}
}

\NewEnviron{python}{ %
  \setcounter{cptcol}{0}
  \begin{center}
    \longboxed{
      \begin{tabular}{>{\stepcounter{cptcol}{\tiny \underbar
              \thecptcol}}c>{\tt}l}
        \BODY
      \end{tabular}
    }
  \end{center}
}

\newcommand{\pyVar}[1]{\textcolor{var}{\bf \small #1}} %
\newcommand{\pyFun}[1]{\textcolor{pyfunction}{#1}} %
\newcommand{\pyCle}[1]{\textcolor{pyCle}{#1}} %
\newcommand{\pyImp}[1]{{\bf #1}} %

%%%%% commentaire python %%%%
\newcommand{\pyComDL}[1]{\multicolumn{1}{l}{\textcolor{pycomment}{\#
      #1}}}

\newcommand{\pyCom}[1]{{\textcolor{pycomment}{\# #1}}}
\newcommand{\pyDoc}[1]{{\textcolor{pydoc}{#1}}}

\newcommand{\pyNo}[1]{{\small \underbar #1}}

%%%%%%%%%%%%%%%%%%%%%%%%%%%%%%%%%%%
%%%%%% Système linéaire paramétré : écrire les opérations au-dessus
%%%%%% d'un symbole équivalent
%%%%%%%%%%%%%%%%%%%%%%%%%%%%%%%%%%%

\usepackage{systeme}

\NewEnviron{arrayEq}{ %
  \stackrel{\scalebox{.6}{$
      \begin{array}{l} 
        \BODY \\[.1cm]
      \end{array}$}
  }{\Longleftrightarrow}
}

\NewEnviron{arrayEg}{ %
  \stackrel{\scalebox{.6}{$
      \begin{array}{l} 
        \BODY \\[.1cm]
      \end{array}$}
  }{=}
}

\NewEnviron{operationEq}{ %
  \scalebox{.6}{$
    \begin{array}{l} 
      \scalebox{1.6}{$\mbox{Opérations :}$} \\[.2cm]
      \BODY \\[.1cm]
    \end{array}$}
}

% \NewEnviron{arraySys}[1]{ %
%   \sysdelim\{.\systeme[#1]{ %
%     \BODY %
%   } %
% }

%%%%%

%%%%%%%%%%
%%%%%%%%%% ESSAI
\newlength\fboxseph
\newlength\fboxsepva
\newlength\fboxsepvb

\setlength\fboxsepva{0.2cm}
\setlength\fboxsepvb{0.2cm}
\setlength\fboxseph{0.2cm}

\makeatletter

\def\longboxed#1{\leavevmode\setbox\@tempboxa\hbox{\color@begingroup%
\kern\fboxseph{\m@th$\displaystyle #1 $}\kern\fboxseph%
\color@endgroup }\my@frameb@x\relax}

\def\my@frameb@x#1{%
  \@tempdima\fboxrule \advance\@tempdima \fboxsepva \advance\@tempdima
  \dp\@tempboxa\hbox {%
    \lower \@tempdima \hbox {%
      \vbox {\hrule\@height\fboxrule \hbox{\vrule\@width\fboxrule #1
          \vbox{%
            \vskip\fboxsepva \box\@tempboxa \vskip\fboxsepvb}#1
          \vrule\@width\fboxrule }%
        \hrule \@height \fboxrule }}}}

\newcommand{\boxedhv}[3]{\setlength\fboxseph{#1cm}
  \setlength\fboxsepva{#2cm}\setlength\fboxsepvb{#2cm}\longboxed{#3}}

\newcommand{\boxedhvv}[4]{\setlength\fboxseph{#1cm}
  \setlength\fboxsepva{#2cm}\setlength\fboxsepvb{#3cm}\longboxed{#4}}

\newcommand{\Boxed}[1]{{\setlength\fboxseph{0.2cm}
  \setlength\fboxsepva{0.2cm}\setlength\fboxsepvb{0.2cm}\longboxed{#1}}}

\newcommand{\mBoxed}[1]{{\setlength\fboxseph{0.2cm}
  \setlength\fboxsepva{0.2cm}\setlength\fboxsepvb{0.2cm}\longboxed{\mbox{#1}}}}

\newcommand{\mboxed}[1]{{\setlength\fboxseph{0.2cm}
  \setlength\fboxsepva{0.2cm}\setlength\fboxsepvb{0.2cm}\boxed{\mbox{#1}}}}

\newsavebox{\fmbox}
\newenvironment{fmpage}[1]
     {\begin{lrbox}{\fmbox}\begin{minipage}{#1}}
     {\end{minipage}\end{lrbox}\fbox{\usebox{\fmbox}}}

%%%%%%%%%%
%%%%%%%%%%

\DeclareMathOperator{\ch}{ch}
\DeclareMathOperator{\sh}{sh}

%%%%%%%%%%
%%%%%%%%%%

\newcommand{\norme}[1]{\Vert #1 \Vert}

%\newcommand*\widefbox[1]{\fbox{\hspace{2em}#1\hspace{2em}}}

\newcommand{\nl}{\tabularnewline}

\newcommand{\hand}{\noindent\ding{43}\ }
\newcommand{\ie}{\textit{i.e. }}
\newcommand{\cf}{\textit{cf }}

\newcommand{\Card}{\operatorname{Card}}

\newcommand{\aire}{\mathcal{A}}

\newcommand{\LL}[1]{\mathscr{L}(#1)} %
\newcommand{\B}{\mathscr{B}} %
\newcommand{\Bc}[1]{B_{#1}} %
\newcommand{\M}[1]{\mathscr{M}_{#1}(\mathbb{R})}

\DeclareMathOperator{\im}{Im}
\DeclareMathOperator{\kr}{Ker}
\DeclareMathOperator{\rg}{rg}
\DeclareMathOperator{\tr}{tr}
\DeclareMathOperator{\spc}{Sp}
\DeclareMathOperator{\sgn}{sgn}
\DeclareMathOperator{\supp}{Supp}

\newcommand{\Mat}{{\rm{Mat}}}
\newcommand{\Vect}[1]{{\rm{Vect}}\left(#1\right)}

\newenvironment{smatrix}{%
  \begin{adjustbox}{width=.9\width}
    $
    \begin{pmatrix}
    }{%      
    \end{pmatrix}
    $
  \end{adjustbox}
}

\newenvironment{sarray}[1]{%
  \begin{adjustbox}{width=.9\width}
    $
    \begin{array}{#1}
    }{%      
    \end{array}
    $
  \end{adjustbox}
}

\newcommand{\vd}[2]{
  \scalebox{.8}{
    $\left(\!
      \begin{array}{c}
        #1 \\
        #2
      \end{array}
    \!\right)$
    }}

\newcommand{\vt}[3]{
  \scalebox{.8}{
    $\left(\!
      \begin{array}{c}
        #1 \\
        #2 \\
        #3 
      \end{array}
    \!\right)$
    }}

\newcommand{\vq}[4]{
  \scalebox{.8}{
    $\left(\!
      \begin{array}{c}
        #1 \\
        #2 \\
        #3 \\
        #4 
      \end{array}
    \!\right)$
    }}

\newcommand{\vc}[5]{
  \scalebox{.8}{
    $\left(\!
      \begin{array}{c}
        #1 \\
        #2 \\
        #3 \\
        #4 \\
        #5 
      \end{array}
    \!\right)$
    }}

\newcommand{\ee}{\text{e}}

\newcommand{\dd}{\text{d}}

%%% Ensemble de définition
\newcommand{\Df}{\mathscr{D}}
\newcommand{\Cf}{\mathscr{C}}
\newcommand{\Ef}{\mathscr{C}}

\newcommand{\rond}[1]{\,\overset{\scriptscriptstyle \circ}{\!#1}}

\newcommand{\df}[2]{\dfrac{\partial #1}{\partial #2}} %
\newcommand{\dfn}[2]{\partial_{#2}(#1)} %
\newcommand{\ddfn}[2]{\partial^2_{#2}(#1)} %
\newcommand{\ddf}[2]{\dfrac{\partial^2 #1}{\partial #2^2}} %
\newcommand{\ddfr}[3]{\dfrac{\partial^2 #1}{\partial #2 \partial
    #3}} %


\newcommand{\dlim}[1]{{\displaystyle \lim_{#1} \ }}
\newcommand{\dlimPlus}[2]{
  \dlim{
    \scalebox{.6}{
      $
      \begin{array}{l}
        #1 \rightarrow #2\\
        #1 > #2
      \end{array}
      $}}}
\newcommand{\dlimMoins}[2]{
  \dlim{
    \scalebox{.6}{
      $
      \begin{array}{l}
        #1 \rightarrow #2\\
        #1 < #2
      \end{array}
      $}}}

%%%%%%%%%%%%%%
%% petit o, développement limité
%%%%%%%%%%%%%%

\newcommand{\oo}[2]{{\underset {{\overset {#1\rightarrow #2}{}}}{o}}} %
\newcommand{\oox}[1]{{\underset {{\overset {x\rightarrow #1}{}}}{o}}} %
\newcommand{\oon}{{\underset {{\overset {n\rightarrow +\infty}{}}}{o}}} %
\newcommand{\po}[1]{{\underset {{\overset {#1}{}}}{o}}} %
\newcommand{\neqx}[1]{{\ \underset {{\overset {x \to #1}{}}}{\not\sim}\ }} %
\newcommand{\eqx}[1]{{\ \underset {{\overset {x \to #1}{}}}{\sim}\ }} %
\newcommand{\eqn}{{\ \underset {{\overset {n \to +\infty}{}}}{\sim}\ }} %
\newcommand{\eq}[2]{{\ \underset {{\overset {#1 \to #2}{}}}{\sim}\ }} %
\newcommand{\DL}[1]{{\rm{DL}}_1 (#1)} %
\newcommand{\DLL}[1]{{\rm{DL}}_2 (#1)} %

\newcommand{\negl}{<<}

\newcommand{\neglP}[1]{\begin{array}{c}
    \vspace{-.2cm}\\
    << \\
    \vspace{-.7cm}\\
    {\scriptstyle #1}
  \end{array}}

%%%%%%%%%%%%%%
%% borne sup, inf, max, min
%%%%%%%%%%%%%%
\newcommand{\dsup}[1]{\displaystyle \sup_{#1} \ }
\newcommand{\dinf}[1]{\displaystyle \inf_{#1} \ }
\newcommand{\dmax}[1]{\max\limits_{#1} \ }
\newcommand{\dmin}[1]{\min\limits_{#1} \ }

\newcommand{\dcup}[2]{{\textstyle\bigcup\limits_{#1}^{#2}}\hspace{.1cm}}
%\displaystyle \bigcup_{#1}^{#2}}
\newcommand{\dcap}[2]{{\textstyle\bigcap\limits_{#1}^{#2}}\hspace{.1cm}}
% \displaystyle \bigcap_{#1}^{#2}
%%%%%%%%%%%%%%
%% opérateurs logiques
%%%%%%%%%%%%%%
\newcommand{\NON}[1]{\mathop{\small \tt{NON}} (#1)}
\newcommand{\ET}{\mathrel{\mathop{\small \mathtt{ET}}}}
\newcommand{\OU}{\mathrel{\mathop{\small \tt{OU}}}}
\newcommand{\XOR}{\mathrel{\mathop{\small \tt{XOR}}}}

\newcommand{\id}{{\rm{id}}}

\newcommand{\sbullet}{\scriptstyle \bullet}
\newcommand{\stimes}{\scriptstyle \times}

%%%%%%%%%%%%%%%%%%
%% Probabilités
%%%%%%%%%%%%%%%%%%
\newcommand{\Prob}{\mathbb{P}}
\newcommand{\Ev}[1]{\left[ {#1} \right]}
\newcommand{\E}{\mathbb{E}}
\newcommand{\V}{\mathbb{V}}
\newcommand{\Cov}{{\rm{Cov}}}
\newcommand{\U}[2]{\mathcal{U}(\llb #1, #2\rrb)}
\newcommand{\Uc}[2]{\mathcal{U}([#1, #2])}
\newcommand{\Ucof}[2]{\mathcal{U}(]#1, #2])}
\newcommand{\Ucoo}[2]{\mathcal{U}(]#1, #2[)}
\newcommand{\Ucfo}[2]{\mathcal{U}([#1, #2[)}
\newcommand{\Bern}[1]{\mathcal{B}\left(#1\right)}
\newcommand{\Bin}[2]{\mathcal{B}\left(#1, #2\right)}
\newcommand{\G}[1]{\mathcal{G}\left(#1\right)}
\newcommand{\Pois}[1]{\mathcal{P}\left(#1\right)}
\newcommand{\HG}[3]{\mathcal{H}\left(#1, #2, #3\right)}
\newcommand{\Exp}[1]{\mathcal{E}\left(#1\right)}
\newcommand{\Norm}[2]{\mathcal{N}\left(#1, #2\right)}

\DeclareMathOperator{\cov}{Cov}

\newcommand{\var}{v.a.r. }
\newcommand{\suit}{\hookrightarrow}

\newcommand{\flecheR}[1]{\rotatebox{90}{\scalebox{#1}{\color{red}
      $\curvearrowleft$}}}


\newcommand{\partie}[1]{\mathcal{P}(#1)}
\newcommand{\Cont}[1]{\mathcal{C}^{#1}}
\newcommand{\Contm}[1]{\mathcal{C}^{#1}_m}

\newcommand{\llb}{\llbracket}
\newcommand{\rrb}{\rrbracket}

%\newcommand{\im}[1]{{\rm{Im}}(#1)}
\newcommand{\imrec}[1]{#1^{- \mathds{1}}}

\newcommand{\unq}{\mathds{1}}

\newcommand{\Hyp}{\mathtt{H}}

\newcommand{\eme}[1]{#1^{\scriptsize \mbox{ème}}}
\newcommand{\er}[1]{#1^{\scriptsize \mbox{er}}}
\newcommand{\ere}[1]{#1^{\scriptsize \mbox{ère}}}
\newcommand{\nd}[1]{#1^{\scriptsize \mbox{nd}}}
\newcommand{\nde}[1]{#1^{\scriptsize \mbox{nde}}}

\newcommand{\truc}{\mathop{\top}}
\newcommand{\fois}{\mathop{\ast}}

\newcommand{\f}[1]{\overrightarrow{#1}}

\newcommand{\checked}{\textcolor{green}{\checkmark}}

\def\restriction#1#2{\mathchoice
              {\setbox1\hbox{${\displaystyle #1}_{\scriptstyle #2}$}
              \restrictionaux{#1}{#2}}
              {\setbox1\hbox{${\textstyle #1}_{\scriptstyle #2}$}
              \restrictionaux{#1}{#2}}
              {\setbox1\hbox{${\scriptstyle #1}_{\scriptscriptstyle #2}$}
              \restrictionaux{#1}{#2}}
              {\setbox1\hbox{${\scriptscriptstyle #1}_{\scriptscriptstyle #2}$}
              \restrictionaux{#1}{#2}}}
\def\restrictionaux#1#2{{#1\,\smash{\vrule height .8\ht1 depth .85\dp1}}_{\,#2}}

\makeatletter
\newcommand*{\da@rightarrow}{\mathchar"0\hexnumber@\symAMSa 4B }
\newcommand*{\da@leftarrow}{\mathchar"0\hexnumber@\symAMSa 4C }
\newcommand*{\xdashrightarrow}[2][]{%
  \mathrel{%
    \mathpalette{\da@xarrow{#1}{#2}{}\da@rightarrow{\,}{}}{}%
  }%
}
\newcommand{\xdashleftarrow}[2][]{%
  \mathrel{%
    \mathpalette{\da@xarrow{#1}{#2}\da@leftarrow{}{}{\,}}{}%
  }%
}
\newcommand*{\da@xarrow}[7]{%
  % #1: below
  % #2: above
  % #3: arrow left
  % #4: arrow right
  % #5: space left 
  % #6: space right
  % #7: math style 
  \sbox0{$\ifx#7\scriptstyle\scriptscriptstyle\else\scriptstyle\fi#5#1#6\m@th$}%
  \sbox2{$\ifx#7\scriptstyle\scriptscriptstyle\else\scriptstyle\fi#5#2#6\m@th$}%
  \sbox4{$#7\dabar@\m@th$}%
  \dimen@=\wd0 %
  \ifdim\wd2 >\dimen@
    \dimen@=\wd2 %   
  \fi
  \count@=2 %
  \def\da@bars{\dabar@\dabar@}%
  \@whiledim\count@\wd4<\dimen@\do{%
    \advance\count@\@ne
    \expandafter\def\expandafter\da@bars\expandafter{%
      \da@bars
      \dabar@ 
    }%
  }%  
  \mathrel{#3}%
  \mathrel{%   
    \mathop{\da@bars}\limits
    \ifx\\#1\\%
    \else
      _{\copy0}%
    \fi
    \ifx\\#2\\%
    \else
      ^{\copy2}%
    \fi
  }%   
  \mathrel{#4}%
}
\makeatother



\newcount\depth

\newcount\depth
\newcount\totaldepth

\makeatletter
\newcommand{\labelsymbol}{%
      \ifnum\depth=0
        %
      \else
        \rlap{\,$\bullet$}%
      \fi
}

\newcommand*\bernoulliTree[1]{%
    \depth=#1\relax            
    \totaldepth=#1\relax
    \draw node(root)[bernoulli/root] {\labelsymbol}[grow=right] \draw@bernoulli@tree;
    \draw \label@bernoulli@tree{root};                                   
}                                                                        

\def\draw@bernoulli@tree{%
    \ifnum\depth>0 
      child[parent anchor=east] foreach \type/\label in {left child/$E$,right child/$S$} {%
          node[bernoulli/\type] {\label\strut\labelsymbol} \draw@bernoulli@tree
      }
      coordinate[bernoulli/increment] (dummy)
   \fi%
}

\def\label@bernoulli@tree#1{%
    \ifnum\depth>0
      ($(#1)!0.5!(#1-1)$) node[fill=white,bernoulli/decrement] {\tiny$p$}
      \label@bernoulli@tree{#1-1}
      ($(#1)!0.5!(#1-2)$) node[fill=white] {\tiny$q$}
      \label@bernoulli@tree{#1-2}
      coordinate[bernoulli/increment] (dummy)
   \fi%
}

\makeatother

\tikzset{bernoulli/.cd,
         root/.style={},
         decrement/.code=\global\advance\depth by-1\relax,
         increment/.code=\global\advance\depth by 1\relax,
         left child/.style={bernoulli/decrement},
         right child/.style={}}


\newcommand{\eps}{\varepsilon}

% \newcommand{\tendi}[1]{\xrightarrow[\footnotesize #1 \rightarrow
%   +\infty]{}}%

\newcommand{\tend}{\rightarrow}%
\newcommand{\tendn}{\underset{n\to +\infty}{\longrightarrow}} %
\newcommand{\ntendn}{\underset{n\to
    +\infty}{\not\hspace{-.15cm}\longrightarrow}} %
% \newcommand{\tendn}{\xrightarrow[\footnotesize n \rightarrow
%   +\infty]{}}%
\newcommand{\Tendx}[1]{\xrightarrow[\footnotesize x \rightarrow
  #1]{}}%
\newcommand{\tendx}[1]{\underset{x\to #1}{\longrightarrow}}%
\newcommand{\ntendx}[1]{\underset{x\to #1}{\not\!\!\longrightarrow}}%
\newcommand{\tendd}[2]{\underset{#1\to #2}{\longrightarrow}}%
% \newcommand{\tendd}[2]{\xrightarrow[\footnotesize #1 \rightarrow
%   #2]{}}%
\newcommand{\tendash}[1]{\xdashrightarrow[\footnotesize #1 \rightarrow
  +\infty]{}}%
\newcommand{\tendashx}[1]{\xdashrightarrow[\footnotesize x \rightarrow
  #1]{}}%
\newcommand{\tendb}[1]{\underset{#1 \to +\infty}{\longrightarrow}}%
\newcommand{\tendL}{\overset{\mathscr L}{\underset{n \to
      +\infty}{\longrightarrow}}}%
\newcommand{\tendP}{\overset{\Prob}{\underset{n \to
      +\infty}{\longrightarrow}}}%
\newcommand{\tenddL}[1]{\overset{\mathscr L}{\underset{#1 \to
      +\infty}{\longrightarrow}}}%

\NewEnviron{attention}{ %
  ~\\[-.2cm]\noindent
  \begin{minipage}{\linewidth}
  \setlength{\fboxsep}{3mm}%
  \ \ \dbend \ \ %
  \fbox{\parbox[t]{.88\linewidth}{\BODY}} %
  \end{minipage}\\
}

\NewEnviron{sattention}[1]{ %
  ~\\[-.2cm]\noindent
  \begin{minipage}{#1\linewidth}
  \setlength{\fboxsep}{3mm}%
  \ \ \dbend \ \ %
  \fbox{\parbox[t]{.88\linewidth}{\BODY}} %
  \end{minipage}\\
}

%%%%% OBSOLETE %%%%%%

% \newcommand{\attention}[1]{
%   \noindent
%   \begin{tabular}{@{}l|p{11.5cm}|}
%     \cline{2-2}
%     \vspace{-.2cm} 
%     & \nl
%     \dbend & #1 \nl
%     \cline{2-2}
%   \end{tabular}
% }

% \newcommand{\attentionv}[2]{
%   \noindent
%   \begin{tabular}{@{}l|p{11.5cm}|}
%     \cline{2-2}
%     \vspace{-.2cm} 
%     & \nl
%     \dbend & #2 \nl[#1 cm]
%     \cline{2-2}
%   \end{tabular}
% }

\newcommand{\explainvb}[2]{
  \noindent
  \begin{tabular}{@{}l|p{11.5cm}|}
    \cline{2-2}
    \vspace{-.2cm} 
    & \nl
    \hand & #2 \nl [#1 cm]
    \cline{2-2}
  \end{tabular}
}


% \noindent
% \begin{tabular}{@{}l|lp{11cm}|}
%   \cline{3-3} 
%   \multicolumn{1}{@{}l@{\dbend}}{} & & #1 \nl
%   \multicolumn{1}{l}{} & & \nl [-.8cm]
%   & & #2 \nl
%   \cline{2-3}
% \end{tabular}

% \newcommand{\attention}[1]{
%   \noindent
%   \begin{tabular}{@{}@{}cp{11cm}}
%     \dbend & #1 \nl
%   \end{tabular}
% }

\newcommand{\PP}[1]{\mathcal{P}(#1)}
\newcommand{\FF}[1]{\mathcal{F}(#1)}

\newcommand{\DSum}[2]{\displaystyle\sum\limits_{#1}^{#2}\hspace{.1cm}}
\newcommand{\Sum}[2]{{\textstyle\sum\limits_{#1}^{#2}}\hspace{.1cm}}
\newcommand{\Serie}{\textstyle\sum\ }
\newcommand{\Prod}[2]{\textstyle\prod\limits_{#1}^{#2}}

\newcommand{\Prim}[3]{\left[\ {#1} \ \right]_{\scriptscriptstyle
   \hspace{-.15cm} ~_{#2}\, }^ {\scriptscriptstyle \hspace{-.15cm} ~^{#3}\, }}

% \newcommand{\Prim}[3]{\left[\ {#1} \ \right]_{\scriptscriptstyle
%     \!\!~_{#2}}^ {\scriptscriptstyle \!\!~^{#3}}}

\newcommand{\dint}[2]{\displaystyle \int_{#1}^{#2}\ }
\newcommand{\Int}[2]{{\rm{Int}}_{\scriptscriptstyle #1, #2}}
\newcommand{\dt}{\ dt}
\newcommand{\dx}{\ dx}

\newcommand{\llpar}[1]{\left(\!\!\!
    \begin{array}{c}
      \rule{0pt}{#1}
    \end{array}
  \!\!\!\right.}

\newcommand{\rrpar}[1]{\left.\!\!\!
    \begin{array}{c}
      \rule{0pt}{#1}
    \end{array}
  \!\!\!\right)}

\newcommand{\llacc}[1]{\left\{\!\!\!
    \begin{array}{c}
      \rule{0pt}{#1}
    \end{array}
  \!\!\!\right.}

\newcommand{\rracc}[1]{\left.\!\!\!
    \begin{array}{c}
      \rule{0pt}{#1}
    \end{array}
  \!\!\!\right\}}

\newcommand{\ttacc}[1]{\mbox{\rotatebox{-90}{\hspace{-.7cm}$\llacc{#1}$}}}
\newcommand{\bbacc}[1]{\mbox{\rotatebox{90}{\hspace{-.5cm}$\llacc{#1}$}}}

\newcommand{\comp}[1]{\overline{#1}}%

\newcommand{\dcomp}[2]{\stackrel{\mbox{\ \ \----}{\scriptscriptstyle
      #2}}{#1}}%

% \newcommand{\Comp}[2]{\stackrel{\mbox{\ \
%       \-------}{\scriptscriptstyle #2}}{#1}}

% \newcommand{\dcomp}[2]{\stackrel{\mbox{\ \
%       \-------}{\scriptscriptstyle #2}}{#1}}

\newcommand{\A}{\mathscr{A}}

\newcommand{\conc}[1]{
  \begin{center}
    \fbox{
      \begin{tabular}{c}
        #1
      \end{tabular}
    }
  \end{center}
}

\newcommand{\concC}[1]{
  \begin{center}
    \fbox{
    \begin{tabular}{C{10cm}}
      \quad #1 \quad
    \end{tabular}
    }
  \end{center}
}

\newcommand{\concL}[2]{
  \begin{center}
    \fbox{
    \begin{tabular}{C{#2cm}}
      \quad #1 \quad
    \end{tabular}
    }
  \end{center}
}

% \newcommand{\lims}[2]{\prod\limits_{#1}^{#2}}

\newtheorem{theorem}{Théorème}[]
\newtheorem{lemma}{Lemme}[]
\newtheorem{proposition}{Proposition}[]
\newtheorem{corollary}{Corollaire}[]

% \newenvironment{proof}[1][Démonstration]{\begin{trivlist}
% \item[\hskip \labelsep {\bfseries #1}]}{\end{trivlist}}
\newenvironment{definition}[1][Définition]{\begin{trivlist}
\item[\hskip \labelsep {\bfseries #1}]}{\end{trivlist}}
\newenvironment{example}[1][Exemple]{\begin{trivlist}
\item[\hskip \labelsep {\bfseries #1}]}{\end{trivlist}}
\newenvironment{examples}[1][Exemples]{\begin{trivlist}
\item[\hskip \labelsep {\bfseries #1}]}{\end{trivlist}}
\newenvironment{notation}[1][Notation]{\begin{trivlist}
\item[\hskip \labelsep {\bfseries #1}]}{\end{trivlist}}
\newenvironment{propriete}[1][Propriété]{\begin{trivlist}
\item[\hskip \labelsep {\bfseries #1}]}{\end{trivlist}}
\newenvironment{proprietes}[1][Propriétés]{\begin{trivlist}
\item[\hskip \labelsep {\bfseries #1}]}{\end{trivlist}}
\newenvironment{remarkSC}[1][Remarque]{\begin{trivlist}
\item[\hskip \labelsep {\bfseries #1}]}{\end{trivlist}}
\newenvironment{application}[1][Application]{\begin{trivlist}
\item[\hskip \labelsep {\bfseries #1}]}{\end{trivlist}}

% Environnement pour les réponses des DS
\newenvironment{answer}{\par\emph{Réponse :}\par{}}
{\vspace{-.6cm}\hspace{\stretch{1}}\rule{1ex}{1ex}\vspace{.3cm}}

\newenvironment{answerTD}{\vspace{.2cm}\par\emph{Réponse :}\par{}}
{\hspace{\stretch{1}}\rule{1ex}{1ex}\vspace{.3cm}}

\newenvironment{answerCours}{\noindent\emph{Réponse :}}
{\rule{1ex}{1ex}}%\vspace{.3cm}}


% footnote in footer
\newcommand{\fancyfootnotetext}[2]{%
  \fancypagestyle{dingens}{%
    \fancyfoot[LO,RE]{\parbox{0.95\textwidth}{\footnotemark[#1]\footnotesize
        #2}}%
  }%
  \thispagestyle{dingens}%
}

%%% définit le style (arabic : 1,2,3...) et place des parenthèses
%%% autour de la numérotation
\renewcommand*{\thefootnote}{(\arabic{footnote})}
% http://www.tuteurs.ens.fr/logiciels/latex/footnote.html

%%%%%%%% tikz axis
% \pgfplotsset{every axis/.append style={
%                     axis x line=middle,    % put the x axis in the middle
%                     axis y line=middle,    % put the y axis in the middle
%                     axis line style={<->,color=blue}, % arrows on the axis
%                     xlabel={$x$},          % default put x on x-axis
%                     ylabel={$y$},          % default put y on y-axis
%             }}

%%%% s'utilise comme suit

% \begin{axis}[
%   xmin=-8,xmax=4,
%   ymin=-8,ymax=4,
%   grid=both,
%   ]
%   \addplot [domain=-3:3,samples=50]({x^3-3*x},{3*x^2-9}); 
% \end{axis}

%%%%%%%%



%%%%%%%%%%%% Pour avoir des numéros de section qui correspondent à
%%%%%%%%%%%% ceux du tableau
\renewcommand{\thesection}{\Roman{section}.\hspace{-.3cm}}
\renewcommand{\thesubsection}{\Roman{section}.\arabic{subsection}.\hspace{-.2cm}}
\renewcommand{\thesubsubsection}{\Roman{section}.\arabic{subsection}.\alph{subsubsection})\hspace{-.2cm}}
%%%%%%%%%%%% 

%%% Changer le nom des figures : Fig. au lieu de Figure
\usepackage[font=small,labelfont=bf,labelsep=space]{caption}
\captionsetup{%
  figurename=Fig.,
  tablename=Tab.
}
% \renewcommand{\thesection}{\Roman{section}.\hspace{-.2cm}}
% \renewcommand{\thesubsection}{\Roman{section}
%   .\hspace{.2cm}\arabic{subsection}\ .\hspace{-.3cm}}
% \renewcommand{\thesubsubsection}{\alph{subsection})}

\newenvironment{tabliste}[1]
{\begin{tabenum}[\bfseries\small\itshape #1]}{\end{tabenum}} 

\newenvironment{noliste}[1] %
{\begin{enumerate}[\bfseries\small\itshape #1]} %
  {\end{enumerate}}

\newenvironment{nonoliste}[1] %
{\begin{enumerate}[\hspace{-12pt}\bfseries\small\itshape #1]} %
  {\end{enumerate}}

\newenvironment{arrayliste}[1]{ 
  % List with minimal white space to fit in small areas, e.g. table
  % cell
  \begin{minipage}[t]{\linewidth} %
    \begin{enumerate}[\bfseries\small\itshape #1] %
      {\leftmargin=0.5em \rightmargin=0em
        \topsep=0em \parskip=0em \parsep=0em
        \listparindent=0em \partopsep=0em \itemsep=0pt
        \itemindent=0em \labelwidth=\leftmargin\labelsep+0.25em}
    }{
    \end{enumerate}\end{minipage}
}

\newenvironment{nolistes}[2]
{\begin{enumerate}[\bfseries\small\itshape
    #1]\setlength{\itemsep}{#2 mm}}{\end{enumerate}}

\newenvironment{liste}[1]
{\begin{enumerate}[\hspace{12pt}\bfseries\small\itshape #1]}{\end{enumerate}} 



%%%%%%%% Pour les programmes de colle %%%%%%%

\newcommand{\cours}{{\small \tt (COURS)}} %
\newcommand{\poly}{{\small \tt (POLY)}} %
\newcommand{\exo}{{\small \tt (EXO)}} %
\newcommand{\culture}{{\small \tt (CULTURE)}} %
\newcommand{\methodo}{{\small \tt (MÉTHODO)}} %
\newcommand{\methodob}{\Boxed{\mbox{\tt MÉTHODO}}} %

%%%%%%%% Pour les TD %%%%%%%
\newtheoremstyle{exostyle} {\topsep} % espace avant
{.6cm} % espace apres
{} % Police utilisee par le style de thm
{} % Indentation (vide = aucune, \parindent = indentation paragraphe)
{\bfseries} % Police du titre de thm
{} % Signe de ponctuation apres le titre du thm
{ } % Espace apres le titre du thm (\newline = linebreak)
{\thmname{#1}\thmnumber{ #2}\thmnote{.
    \normalfont{\textit{#3}}}} % composants du titre du thm : \thmname
                               % = nom du thm, \thmnumber = numéro du
                               % thm, \thmnote = sous-titre du thm
 
\theoremstyle{exostyle}
\newtheorem{exercice}{Exercice}
\newtheorem{exerciceAP}{Exercice avec préparation}
\newtheorem{exerciceSP}{Exercice sans préparation}
\newtheorem*{exoCours}{Exercice}

%%%%%%%% Pour des théorèmes Sans Espaces APRÈS %%%%%%%
\newtheoremstyle{exostyleSE} {\topsep} % espace avant
{} % espace apres
{} % Police utilisee par le style de thm
{} % Indentation (vide = aucune, \parindent = indentation paragraphe)
{\bfseries} % Police du titre de thm
{} % Signe de ponctuation apres le titre du thm
{ } % Espace apres le titre du thm (\newline = linebreak)
{\thmname{#1}\thmnumber{ #2}\thmnote{.
    \normalfont{\textit{#3}}}} % composants du titre du thm : \thmname
                               % = nom du thm, \thmnumber = numéro du
                               % thm, \thmnote = sous-titre du thm
 
\theoremstyle{exostyleSE}
\newtheorem{exerciceSE}{Exercice}
\newtheorem*{exoCoursSE}{Exercice}

% \newcommand{\lims}[2]{\prod\limits_{#1}^{#2}}

\newtheorem{theoremSE}{Théorème}[]
\newtheorem{lemmaSE}{Lemme}[]
\newtheorem{propositionSE}{Proposition}[]
\newtheorem{corollarySE}{Corollaire}[]

% \newenvironment{proofSE}[1][Démonstration]{\begin{trivlist}
% \item[\hskip \labelsep {\bfseries #1}]}{\end{trivlist}}
\newenvironment{definitionSE}[1][Définition]{\begin{trivlist}
  \item[\hskip \labelsep {\bfseries #1}]}{\end{trivlist}}
\newenvironment{exampleSE}[1][Exemple]{\begin{trivlist} 
  \item[\hskip \labelsep {\bfseries #1}]}{\end{trivlist}}
\newenvironment{examplesSE}[1][Exemples]{\begin{trivlist}
\item[\hskip \labelsep {\bfseries #1}]}{\end{trivlist}}
\newenvironment{notationSE}[1][Notation]{\begin{trivlist}
\item[\hskip \labelsep {\bfseries #1}]}{\end{trivlist}}
\newenvironment{proprieteSE}[1][Propriété]{\begin{trivlist}
\item[\hskip \labelsep {\bfseries #1}]}{\end{trivlist}}
\newenvironment{proprietesSE}[1][Propriétés]{\begin{trivlist}
\item[\hskip \labelsep {\bfseries #1}]}{\end{trivlist}}
\newenvironment{remarkSE}[1][Remarque]{\begin{trivlist}
\item[\hskip \labelsep {\bfseries #1}]}{\end{trivlist}}
\newenvironment{applicationSE}[1][Application]{\begin{trivlist}
\item[\hskip \labelsep {\bfseries #1}]}{\end{trivlist}}

%%%%%%%%%%% Obtenir les étoiles sans charger le package MnSymbol
%%%%%%%%%%%
\DeclareFontFamily{U} {MnSymbolC}{}
\DeclareFontShape{U}{MnSymbolC}{m}{n}{
  <-6> MnSymbolC5
  <6-7> MnSymbolC6
  <7-8> MnSymbolC7
  <8-9> MnSymbolC8
  <9-10> MnSymbolC9
  <10-12> MnSymbolC10
  <12-> MnSymbolC12}{}
\DeclareFontShape{U}{MnSymbolC}{b}{n}{
  <-6> MnSymbolC-Bold5
  <6-7> MnSymbolC-Bold6
  <7-8> MnSymbolC-Bold7
  <8-9> MnSymbolC-Bold8
  <9-10> MnSymbolC-Bold9
  <10-12> MnSymbolC-Bold10
  <12-> MnSymbolC-Bold12}{}

\DeclareSymbolFont{MnSyC} {U} {MnSymbolC}{m}{n}

\DeclareMathSymbol{\filledlargestar}{\mathrel}{MnSyC}{205}
\DeclareMathSymbol{\largestar}{\mathrel}{MnSyC}{131}

\newcommand{\facile}{\rm{(}$\scriptstyle\largestar$\rm{)}} %
\newcommand{\moyen}{\rm{(}$\scriptstyle\filledlargestar$\rm{)}} %
\newcommand{\dur}{\rm{(}$\scriptstyle\filledlargestar\filledlargestar$\rm{)}} %
\newcommand{\costaud}{\rm{(}$\scriptstyle\filledlargestar\filledlargestar\filledlargestar$\rm{)}}

%%%%%%%%%%%%%%%%%%%%%%%%%

%%%%%%%%%%%%%%%%%%%%%%%%%
%%%%%%%% Fin de la partie TD

%%%%%%%%%%%%%%%%
%%%%%%%%%%%%%%%%
\makeatletter %
\newenvironment{myitemize}{%
  \setlength{\topsep}{0pt} %
  \setlength{\partopsep}{0pt} %
  \renewcommand*{\@listi}{\leftmargin\leftmargini \parsep\z@
    \topsep\z@ \itemsep\z@} \let\@listI\@listi %
  \itemize %
}{\enditemize} %
\makeatother
%%%%%%%%%%%%%%%%
%%%%%%%%%%%%%%%%

%% Commentaires dans la correction du livre

\newcommand{\Com}[1]{
% Define box and box title style
\tikzstyle{mybox} = [draw=black!50,
very thick,
    rectangle, rounded corners, inner sep=10pt, inner ysep=8pt]
\tikzstyle{fancytitle} =[rounded corners, fill=black!80, text=white]
\tikzstyle{fancylogo} =[ text=white]
\begin{center}

\begin{tikzpicture}
\node [mybox] (box){%

    \begin{minipage}{0.90\linewidth}
\vspace{6pt}  #1
    \end{minipage}
};
\node[fancytitle, right=10pt] at (box.north west) 
{\bfseries\normalsize{Commentaire}};

\end{tikzpicture}%

\end{center}
%
}

% \NewEnviron{remark}{%
%   % Define box and box title style
%   \tikzstyle{mybox} = [draw=black!50, very thick, rectangle, rounded
%   corners, inner sep=10pt, inner ysep=8pt] %
%   \tikzstyle{fancytitle} =[rounded corners, fill=black!80,
%   text=white] %
%   \tikzstyle{fancylogo} =[ text=white]
%   \begin{center}
%     \begin{tikzpicture}
%       \node [mybox] (box){%
%         \begin{minipage}{0.90\linewidth}
%           \vspace{6pt} \BODY
%         \end{minipage}
%       }; %
%       \node[fancytitle, right=10pt] at (box.north west) %
%       {\bfseries\normalsize{Commentaire}}; %
%     \end{tikzpicture}%
%   \end{center}
% }
% \makeatother


%%%
% Remarques comme dans le livre
%%%

\NewEnviron{remark}{%
  % Define box and box title style
  \tikzstyle{mybox} = [draw=black!50, very thick, rectangle, rounded
  corners, inner sep=10pt, inner ysep=8pt] %
  \tikzstyle{fancytitle} = [rounded corners , fill=black!80,
  text=white] %
  \tikzstyle{fancylogo} =[ text=white]
  \begin{center}
    \begin{tikzpicture}
      \node [mybox] (box){%
        \begin{minipage}{0.90\linewidth}
          \vspace{6pt} \BODY
        \end{minipage}
      }; %
      \node[fancytitle, right=10pt] at (box.north west) %
      {\bfseries\normalsize{Commentaire}}; %
    \end{tikzpicture}%
  \end{center}
}

\NewEnviron{remarkST}{%
  % Define box and box title style
  \tikzstyle{mybox} = [draw=black!50, very thick, rectangle, rounded
  corners, inner sep=10pt, inner ysep=8pt] %
  \tikzstyle{fancytitle} = [rounded corners , fill=black!80,
  text=white] %
  \tikzstyle{fancylogo} =[ text=white]
  \begin{center}
    \begin{tikzpicture}
      \node [mybox] (box){%
        \begin{minipage}{0.90\linewidth}
          \vspace{6pt} \BODY
        \end{minipage}
      }; %
      % \node[fancytitle, right=10pt] at (box.north west) %
%       {\bfseries\normalsize{Commentaire}}; %
    \end{tikzpicture}%
  \end{center}
}

\NewEnviron{remarkL}[1]{%
  % Define box and box title style
  \tikzstyle{mybox} = [draw=black!50, very thick, rectangle, rounded
  corners, inner sep=10pt, inner ysep=8pt] %
  \tikzstyle{fancytitle} =[rounded corners, fill=black!80,
  text=white] %
  \tikzstyle{fancylogo} =[ text=white]
  \begin{center}
    \begin{tikzpicture}
      \node [mybox] (box){%
        \begin{minipage}{#1\linewidth}
          \vspace{6pt} \BODY
        \end{minipage}
      }; %
      \node[fancytitle, right=10pt] at (box.north west) %
      {\bfseries\normalsize{Commentaire}}; %
    \end{tikzpicture}%
  \end{center}
}

\NewEnviron{remarkSTL}[1]{%
  % Define box and box title style
  \tikzstyle{mybox} = [draw=black!50, very thick, rectangle, rounded
  corners, inner sep=10pt, inner ysep=8pt] %
  \tikzstyle{fancytitle} =[rounded corners, fill=black!80,
  text=white] %
  \tikzstyle{fancylogo} =[ text=white]
  \begin{center}
    \begin{tikzpicture}
      \node [mybox] (box){%
        \begin{minipage}{#1\linewidth}
          \vspace{6pt} \BODY
        \end{minipage}
      }; %
%       \node[fancytitle, right=10pt] at (box.north west) %
%       {\bfseries\normalsize{Commentaire}}; %
    \end{tikzpicture}%
  \end{center}
}

\NewEnviron{titre} %
{ %
  ~\\[-1.8cm]
  \begin{center}
    \bf \LARGE \BODY
  \end{center}
  ~\\[-.6cm]
  \hrule %
  \vspace*{.2cm}
} %

\NewEnviron{titreL}[2] %
{ %
  ~\\[-#1cm]
  \begin{center}
    \bf \LARGE \BODY
  \end{center}
  ~\\[-#2cm]
  \hrule %
  \vspace*{.2cm}
} %




% \renewcommand{\thesection}{\Roman{section}.\hspace{-.3cm}}
% \renewcommand{\thesubsection}{\Alph{subsection}.\hspace{-.2cm}}

\pagestyle{fancy} %
\lhead{ECE2 \hfill Mathématiques \\} %
\chead{\hrule} %
\rhead{} %
\lfoot{} %
\cfoot{} %
\rfoot{\thepage} %

\renewcommand{\headrulewidth}{0pt}% : Trace un trait de séparation
                                    % de largeur 0,4 point. Mettre 0pt
                                    % pour supprimer le trait.

\renewcommand{\footrulewidth}{0.4pt}% : Trace un trait de séparation
                                    % de largeur 0,4 point. Mettre 0pt
                                    % pour supprimer le trait.

\setlength{\headheight}{14pt}

\title{\bf \vspace{-1.6cm} ESSEC II 2019} %
\author{} %
\date{} %
\begin{document}

\maketitle %
\vspace{-1.2cm}\hrule %
\thispagestyle{fancy}

\vspace*{.4cm}

%%DEBUT

\noindent %
Un modèle probabiliste d'une expérience aléatoire représente dans un
certain sens le désordre qui intervient dans l'expérience et il est
donc naturel que des outils soient introduits qui permettent de
mesurer l'intensité de ce désordre. C'est le cas de la notion
d'entropie qui fait l'objet du présent problème. On considèrera
différentes situations et notamment la façon dont on mesure
l'information que deux variables aléatoires s'apportent
mutuellement.\\
Dans la première partie on étudie le cas plus simple techniquement de
variables dont la loi admet une densité. Les deuxièmes et troisièmes
parties sont consacrées au cas discret. Dans la deuxième partie, on
introduit les différentes notions d'entropie pour le cas de variables
discrètes et dans la troisième partie, on examine comment on peut
mesurer l'information apportée mutuellement par deux variables
aléatoires.\\
Toutes les variables aléatoires intervenant dans le problème sont
définies sur un espace probabilisé $(\Omega, \A, \Prob)$.\\
Pour toute variable aléatoire $Y$, on notera $\E(Y)$ son espérance
lorsqu'elle existe.


\subsection*{Première partie : Entropie différentielle d'une variable
  à densité}

\begin{noliste}{1.}
  \setlength{\itemsep}{4mm}
\item La fonction logarithme de base $2$, notée $\log_2$, est définie
  sur $\R_+^*$ par : $\log_2(x) = \dfrac{\ln(x)}{\ln(2)}$.
  \begin{noliste}{a)}
    \setlength{\itemsep}{2mm}
  \item Montrer que pour tout $(x,y)$ élément de $\R_+^* \times
    \R_+^*$, on a : $\log_2(x \, y) = \log_2(x) + \log_2(y)$.
    
    \begin{proof}~\\
      Soit $(x,y) \in \R_+^* \times \R_+^*$.
      \[
        \begin{array}{rcl@{\qquad}>{\it}R{3cm}}
          \log_2 (x \, y)
          & = & \dfrac{\ln(x \, y)}{\ln(2)} 
          \\[.6cm]
          & = & \dfrac{\ln(x) + \ln(y)}{\ln(2)}
          & (par propriété de la fonction $\ln$)
            \nl
            \nl[-.2cm]
          & = & \dfrac{\ln(x)}{\ln(2)} + \dfrac{\ln(y)}{\ln(2)}
          \\[.6cm]
          & = & \log_2(x) + \log_2(y)
        \end{array}
      \]
      \conc{$\forall (x,y) \in \R_+^* \times \R_+^*$, $\log_2(x \, y)
        = \log_2(x) + \log_2(y)$}~\\[-1cm]
    \end{proof}
    
  \item Vérifier que pour tout réel $\alpha$ :
    $\log_2\left(2^\alpha\right) = \alpha$.

    \begin{proof}~\\
      Soit $\alpha \in \R$.
      \[
        \log_2\left(2^\alpha \right) \ = \ \dfrac{\ln\left( 2^\alpha
          \right)}{ \ln(2)} \ = \ \dfrac{\alpha \, \bcancel{\ln(2)}}{
          \bcancel{\ln(2)}} \ = \ \alpha
      \]
      \conc{$\forall \alpha \in \R$, $\log_2\left( 2^\alpha \right) =
        \alpha$}~\\[-1cm]
    \end{proof}

    
    \newpage

    
  \item Montrer que la fonction $\log_2$ est concave sur $\R_+^*$.

    \begin{proof}~\\
      La fonction $\log_2$ est de classe $\Cont{2}$ sur
      $\R_+^*$ car la fonction $\ln$ l'est.\\
      Soit $x \in \R_+^*$.
      \[
        (\log_2)'(x) \ = \ \dfrac{1}{x \, \ln(2)}
      \]
      \[
        (\log_2)''(x) \ = \ -\dfrac{1}{x^2 \, \ln(2)} \ < \ 0
      \]
      \conc{On en déduit que la fonction $\log_2$ est concave sur
        $\R_+^*$.}
      \begin{remarkL}{.95}
        On pouvait répondre à cette question grâce à d'autres
        caractérisations de la concavité.
        \begin{noliste}{\scriptsize 1.}
        \item La fonction $\log_2$ est de classe $\Cont{1}$ sur
          $\R_+^*$ car la fonction $\ln$ l'est.\\
          Soit $x \in \R_+^*$.
          \[
            (\log_2)'(x) \ = \ \dfrac{1}{x \, \ln(2)}
          \]
          La fonction $(\log_2)'$ est donc décroissante sur $\R_+^*$.\\
          On en déduit que la fonction $\log_2$ est concave sur
          $\R_+^*$.
          
        \item
          \begin{noliste}{$\sbullet$}
          \item Tout d'abord :
            \[
              \begin{array}{c}
                \text{La fonction $\log_2$ est concave sur $\R_+^*$}
                \\[.2cm]
                \rotatebox{90}{$\Leftrightarrow$}
                \\[.2cm]
                \forall \lambda \in [0,1], \ \forall (x,y)
                \in \R_+^* \times \R_+^*, \ \log_2\big(\lambda \, x + (1-
                \lambda) y\big) \geq \lambda \, \log_2(x) + (1- \lambda)
                \log_2(y)
              \end{array}
            \]
            
          \item Soit $\lambda \in [0,1]$. Soit $(x,y) \in \R_+^* \times
            \R_+^*$.
            \[
              \begin{array}{cl@{\qquad}>{\it}R{3cm}}
                & \log_2\big(\lambda \, x + (1-
                  \lambda) y\big) \ \geq \ \lambda \, \log_2(x) + (1- \lambda)
                  \log_2(y)
                \\[.4cm]
                \Leftrightarrow & \dfrac{\ln\big(\lambda \, x + (1-
                                  \lambda) y\big)}{\ln(2)} \ \geq \ \lambda \,
                                  \dfrac{\ln(x)}{ \ln(2)} + (1- \lambda)
                                  \dfrac{\ln(y)}{\ln(2)}
                \\[.6cm]
                \Leftrightarrow & \ln\big(\lambda \, x + (1-
                \lambda) y\big) \ \geq \ \lambda \, \ln(x) + (1- \lambda)
                \ln(y)
                & (car $\ln(2) >0$)                  
              \end{array}
            \]
            Cette dernière inégalité est vraie car la fonction $\ln$ est
            concave sur $\R_+^*$. Par équivalence, la première est vraie.
          \end{noliste}
          On en déduit que la fonction $\log_2$ est concave sur $\R_+^*$.
        \end{noliste}
      \end{remarkL}~\\[-1.4cm]
    \end{proof}
  \end{noliste}


  \newpage
  
  
\item Soit $X$ une variable aléatoire réelle à densité, et soit $f$
  une densité de $X$. On appelle {\bf support} de $f$ l'ensemble $I=
  \{x \in \R \ | \ f(x) >0\}$, et on suppose que $I$ est un intervalle
  de $\R$ d'extrémités $a$ et $b$ ($a<b$, $a$ et $b$ finis ou
  infinis). L'{\bf entropie différentielle} de $X$ est, sous réserve
  d'existence, le réel :
  \[
    h(X) \ = \ - \dint{a}{b} f(x) \, \log_2 \big(f(x)\big) \dx
  \]
  Montrer : $h(X) = - \E\Big(\log_2\big(f(X)\big)\Big)$.

  \begin{proof}~
    \begin{noliste}{$\sbullet$}
    \item Par théorème de transfert, la \var $\log_2\big(f(X)\big)$
      admet une espérance si et seulement si l'intégrale
      $\dint{-\infty}{+\infty} \log_2\big(f(x)\big) \, f(x) \dx$ est
      absolument convergente.
      
    \item De plus, par définition du support, la fonction $f$ est
      nulle en dehors de l'intervalle $I$ d'extrémités $a$ et $b$, donc :
      \[
        \dint{-\infty}{+\infty} f(x) \, \log_2\big(f(x)\big) \dx \ = \
        \dint{a}{b} f(x) \, \log_2\big(f(x)\big) \dx
      \]
    \end{noliste}
    \conc{Ainsi, sous réserve de convergence absolue :\\
      $h(X) \ = \ - \dint{a}{b} f(x) \, \log_2\big(f(x)\big) \dx \ = \
      - \E\Big(\log_2 \big(f(x) \big) \Big)$.}

    \begin{remark}
      L'énoncé précise que $I$ est un intervalle de $\R$ {\bf
        d'extrémités} $a$ et $b$. C'est donc l'un des intervalles
      suivants : $]a,b[$, $]a,b]$, $[a,b[$ ou $[a,b]$.
    \end{remark}~\\[-1.4cm]
  \end{proof}
  
\item Soit $X$ une variable aléatoire de densité $f$ de support $I$,
  intervalle de $\R$ d'extrémités $a$ et $b$. On suppose que $X$ admet
  une entropie différentielle.
  \begin{noliste}{a)}
    \setlength{\itemsep}{2mm}
  \item Soit $c$ un réel, et soit $Y$ la variable aléatoire définie
    par $Y = c+X$.
    \begin{nonoliste}{(i)}
      \setlength{\itemsep}{2mm}
    \item Déterminer une densité de $Y$.
    \end{nonoliste}
  \end{noliste}
      
      \begin{proof}~
        \begin{noliste}{$\sbullet$}
        \item Déterminons la fonction de répartition de la \var $Y$.\\
          Soit $x \in \R$.
          \[
            F_Y(x) \ = \ \Prob(\Ev{Y \leq x}) \ = \ \Prob(\Ev{c+X \leq
              x}) \ = \ \Prob(\Ev{X \leq x-c}) \ = \ F_X(x-c)
          \]
          \conc{Finalement : $F_Y : x \mapsto F_X(x-c)$.}
          
        \item La fonction $F_Y$ est continue sur $\R$ car elle est la composée
          $F_X \circ g_1$ de :
          \begin{noliste}{$\stimes$}
          \item $g_1 : x \mapsto x-c$ qui :~\\[-.6cm]
          \end{noliste}
          \begin{liste}{-}
          \item est continue sur $\R$,
            
          \item vérifie : $g_1(\R) \subset \R$.
          \end{liste}
          \begin{noliste}{$\stimes$}
          \item $F_X$ qui est continue sur $\R$, car $X$ est une \var
            à densité.
          \end{noliste}

        \item La fonction $F_Y$ est de classe $\Cont{1}$ sur $\R$ sauf
          (éventuellement) en un nombre fini de points avec des
          arguments similaires à ceux de la continuité sur $\R$.
          \conc{On en déduit que $Y$ est une \var à densité.}

          
          \newpage
          
          
          \begin{remark}
            \begin{noliste}{$\sbullet$}
            \item Si $F_X$ est de classe $\Cont{1}$ sur $\R$, alors la
              fonction $F_Y$ est également de classe $\Cont{1}$ sur
              $\R$ en tant composée $F_X \circ g_1$ (même démonstration
              que la continuité de $F_Y$ sur $\R$).
              
            \item Si $F_X$ n'est pas de classe $\Cont{1}$ sur $\R$,
              on peut déterminer plus précisément les points
              où la fonction $F_Y$ n'est pas de classe $\Cont{1}$.
              \begin{noliste}{$\stimes$}
              \item Comme $X$ est une \var à densité, sa fonction de
                répartition $F_X$ est de classe $\Cont{1}$ sur $\R$ sauf
                en un nombre fini de points. Notons
                $\{x_1, \ldots, x_n\}$ l'ensemble des
                points où la fonction $F_X$ n'est pas de classe $\Cont{1}$.
              
              \item Comme $F_Y : x \mapsto F_X(x-c)$, on en déduit que
                la fonction $F_Y$ est de classe $\Cont{1}$ sur $\R$ sauf
                en $x_1+c$, $\ldots$, $x_n +c$.
              \end{noliste}
            \end{noliste}
          \end{remark}
          
        \item Pour déterminer une densité $f_Y$ de $Y$, on dérive la
          fonction $F_Y$ en les points où elle est
          de classe $\Cont{1}$. Soit $x \in \R$.
          \begin{noliste}{$\stimes$}
          \item \dashuline{Si $F_X$ est de classe $\Cont{1}$ en $x$}.
            \[
              f_Y(x) \ = \ F_Y'(x) \ = \ (F_X \circ g_1)'(x) \ = \ (F_X'
              \circ g_1)(x) \times g_1'(x)\ = \ F_X'\big(g_1(x) \big) \times 1
              \ = \  F_X'(x-c) \ = \ f(x-c)
            \]
            
          \item \dashuline{Si $F_X$ n'est pas de classe $\Cont{1}$ en
              $x$}, on choisit : $f_Y(x) = f(x-c)$.
          \end{noliste}
          \conc{Finalement : $f_Y : x \mapsto f(x-c)$.}
        \end{noliste}
        \begin{remark}
          Afin de permettre une bonne compréhension du calcul, on a
          rédigé ici en détails la dérivation de la
          composée. Cependant, fournir le résultat correct permet
          sans aucun doute d'obtenir la totalité des points alloués à
          ce calcul.
        \end{remark}~\\[-1.4cm]
      \end{proof}

    \begin{noliste}{(i)}
      \setcounter{enumii}{1}
    \item Justifier l'existence de l'entropie différentielle $h(Y)$,
      et la déterminer en fonction de $h(X)$.
    \end{noliste}

    \begin{proof}~
      \begin{noliste}{$\sbullet$}
      \item D'après la question \itbf{2.}, la \var $Y$ admet une
        entropie différentielle si et seulement si la \var
        $\log_2\big(f_Y(Y) \big)$ admet une espérance.
        
      \item On remarque de plus :
        \[
          \begin{array}{rcl@{\qquad}>{\it}R{3.5cm}}
            \log_2\big( f_Y(Y) \big)
            & = & \log_2\big( f(Y-c) \big)
            & (d'après la question précédente)
              \nl
              \nl[-.2cm]
            & = & \log_2\Big( f\big( (X+ \bcancel{c}) - \bcancel{c}
                  \big) \Big)
            & (par définition de\\ la \var $Y$)
              \nl
              \nl[-.2cm]
            & = & \log_2 \big( f(X) \big)
          \end{array}
        \]
        
      \item Or, d'après l'énoncé, la \var $X$ admet une entropie
        différentielle. Donc la \var $\log_2\big( f(X) \big)$ admet
        une espérance.\\
        On en déduit que la \var $\log_2\big( f_Y(Y) \big)$ admet une
        espérance.
        \conc{Ainsi, la \var $Y$ admet une entropie différentielle.}
        
      \item De plus, toujours d'après la question \itbf{2.} :
        \[
          h(Y) \ = \ -\E \Big( \log_2 \big( f_Y(Y) \big) \Big) \ = \ -
          \E\Big( \log_2 \big( f(X) \big) \Big) \ = \ h(X)
        \]
        \conc{$h(Y) \ = \ h(X)$}~\\[-1.4cm]
      \end{noliste}
    \end{proof}

    
    \newpage

    
  \begin{noliste}{a)}
    \setcounter{enumii}{1}
  \item Soit $\alpha$ un réel strictement positif, et soit $Z$ la
    variable aléatoire définie par $Z = \alpha \, X$.
    \begin{nonoliste}{(i)}
      \setlength{\itemsep}{2mm}
    \item Déterminer une densité de $Z$.
    \end{nonoliste}
  \end{noliste}

  \begin{proof}~
    \begin{noliste}{$\sbullet$}
    \item Déterminons la fonction de répartition de la \var $Z$.\\
      Soit $x \in \R$.
      \[
        \begin{array}{rcl@{\qquad}>{\it}R{3.5cm}}
          F_Z(x)
          & = & \Prob(\Ev{Z \leq x}) \ = \ \Prob(\Ev{\alpha \, X \leq
                x})
          \\[.2cm]
          & = & \Prob\left( \Ev{X \leq \dfrac{x}{\alpha}} \right)
          & (car $\alpha >0$)
            \nl
            \nl[-.2cm]
          & = & F_X\left( \dfrac{x}{\alpha} \right)
        \end{array}
      \]
      \conc{Finalement : $F_Z : x \mapsto F_X\left( \dfrac{x}{\alpha} \right)$.}
      
    \item La fonction $F_Z$ est continue sur $\R$ car elle est la
      composée $F_X \circ g_2$ de :
      \begin{noliste}{$\stimes$}
      \item $g_2 : x \mapsto \dfrac{x}{\alpha}$ qui :
        \begin{noliste}{-}
        \item est continue sur $\R$,
          
        \item vérifie : $g_2(\R) \subset \R$.
        \end{noliste}
        
      \item $F_X$ qui est continue sur $\R$, car $X$ est une \var à densité.
      \end{noliste}
      
    \item La fonction $F_Z$ est de classe $\Cont{1}$ sur $\R$ sauf
      (éventuellement) en un nombre fini de points avec des arguments
      similaires à ceux de la continuité sur $\R$.
      \begin{remark}
        \begin{noliste}{$\sbullet$}
        \item Si $F_X$ est de classe $\Cont{1}$ sur $\R$, alors la
          fonction $F_Z$ est également de classe $\Cont{1}$ sur $\R$
          en tant que composée $F_X \circ g_2$ (même démonstration que
          la continuité de $F_Z$ sur $\R$).
          
        \item Si $F_X$ n'est pas de classe $\Cont{1}$ sur $\R$,
          \begin{noliste}{$\stimes$}
          \item on note comme précédemment $\{x_1, \ldots, x_n\}$ l'ensemble
            des points où la fonction $F_X$ n'est pas de classe $\Cont{1}$,
            
          \item comme $F_Z : x \mapsto F_X \left( \dfrac{x}{\alpha}
            \right)$, on en déduit que la fonction $F_Z$ est de classe
            $\Cont{1}$ sur $\R$ sauf en $\alpha \, x_1$, $\ldots$,
            $\alpha \, x_n$.
          \end{noliste}
        \end{noliste}
      \end{remark}

    \item Pour déterminer une densité $f_Z$ de $Z$, on dérive la
      fonction $F_Z$ en les points où elle est de classe
      $\Cont{1}$. Soit $x \in \R$.
      \begin{noliste}{$\stimes$}
      \item \dashuline{Si $F_X$ est de classe $\Cont{1}$ en $x$}.
        \[
          f_Z(x) \ = \ F_Z'(x) \ = \ (F_X \circ g_2)'(x) \ = \ (F_X'
          \circ g_2)(x) \times g_2'(x) \ = \ F_X'\big(g_2(x)\big)
          \times \dfrac{1}{\alpha} \ = \ \dfrac{1}{\alpha} \
          F_X'\left( \dfrac{x}{\alpha} \right) \ = \ \dfrac{1}{\alpha}
          \ f\left( \dfrac{x}{\alpha} \right)
        \]
        
      \item \dashuline{Si $F_X$ n'est pas de classe $\Cont{1}$ en
          $x$}, on choisit : $f_Z(x) = \dfrac{1}{\alpha} \ f\left(
          \dfrac{x}{\alpha} \right)$.
      \end{noliste}
      \conc{Finalement : $f_Z : x \mapsto \dfrac{1}{\alpha} \ f\left(
          \dfrac{x}{\alpha} \right)$.}~\\[-1.4cm]
    \end{noliste}
  \end{proof}


  \newpage
  
      
  \begin{noliste}{(i)}
    \setcounter{enumii}{1}
    \item Justifier l'existence de l'entropie différentielle $h(Z)$,
      et la déterminer en fonction de $h(X)$.
  \end{noliste}

  \begin{proof}~
    \begin{noliste}{$\sbullet$}
    \item D'après la question \itbf{2.}, la \var $Y$ admet une
      entropie différentielle si et seulement si la \var
      $\log_2\big(f_Z(Z) \big)$ admet une espérance.
      
    \item On remarque de plus :
      \[
        \begin{array}{rcl@{\qquad}>{\it}R{3.5cm}}
          \log_2\big( f_Z(Z) \big)
          & = & \log_2\left( \dfrac{1}{\alpha} \ f\left(
                \dfrac{X}{\alpha} \right) \right)
          & (d'après la question précédente)
            \nl
            \nl[-.2cm]
          & = & \log_2\left( \dfrac{1}{\alpha} \ f\left( \dfrac{
                \bcancel{\alpha} \, X}{\bcancel{\alpha}} \right) \right)
          & (par définition de\\ la \var $Y$)
          \nl
          \nl[-.2cm]
          & = & \log_2 \left( \dfrac{1}{\alpha} \ f(X) \right)
          \\[.6cm]
          & = & \log_2\left( \dfrac{1}{\alpha} \right) + \log_2 \big(
                f(X) \big)
          & (d'après la question \itbf{1.a)})
        \end{array}
      \]
      Enfin :
      \[
        \log_2\left( \dfrac{1}{\alpha} \right) \ = \ \dfrac{\ln\left(
            \frac{1}{\alpha} \right)}{\ln(2)} \ = \ \dfrac{-
          \ln(\alpha)}{\ln(2)} \ = \ -\log_2(\alpha)
      \]
      \conc{Ainsi : $\log_2 \big( f_Z(Z) \big) \ = \ \log_2\big( f(X) \big) -
        \log_2( \alpha)$.}
      
    \item Or, d'après l'énoncé, la \var $X$ admet une entropie
      différentielle. Donc la \var $\log_2\big( f(X) \big)$ admet
      une espérance.\\
      On en déduit que la \var $\log_2\big( f_Z(Z) \big)$ admet une
      espérance en tant que transformée affine d'une \var qui en admet une.
      \conc{Ainsi, la \var $Z$ admet une entropie différentielle.}
      
    \item De plus, toujours d'après la question \itbf{2.} :
      \[
        \begin{array}{rcl@{\qquad}>{\it}R{3.5cm}}
          h(Z)
          & = & -\E \Big( \log_2 \big( f_Z(Z) \big) \Big)
          \\[.4cm]
          & = & - \E\Big( \log_2 \big( f(X) \big) - \log_2(\alpha)
                \Big)
          \\[.4cm]
          & = & - \Big( \E\left( \log_2 \big( f(X) \big) \right) -
                \log_2(\alpha) \Big)
          & (par linéarité de l'espérance)
            \nl
            \nl[-.2cm]
          & = & h(X) + \log_2(\alpha)
        \end{array}
      \]
      \conc{$h(Z) \ = \ h(X) + \log_2(\alpha)$}~\\[-1.4cm]
    \end{noliste}
  \end{proof}

  
\item On détermine dans cette question l'entropie différentielle de
  quelques variables aléatoires suivant des lois classiques.
  \begin{noliste}{a)}
    \setlength{\itemsep}{2mm}
  \item Soit $a>0$. On considère $X$ une variable aléatoire de loi
    uniforme sur $[0,a]$.
    \begin{nonoliste}{(i)}
      \setlength{\itemsep}{2mm}
    \item Donner une densité de $X$.
    \end{nonoliste}

    \begin{proof}~
      \conc{Une densité $f$ de $X$ est : $f : x \mapsto \left\{
          \begin{array}{cR{2.5cm}}
            0 & si $x \in \ ]-\infty, 0[$
            \nl
            \nl[-.2cm]
            \dfrac{1}{a} & si $x \in [0,a]$
            \nl
            \nl[-.2cm]
            0 & si $x \in \ ]a, +\infty[$
          \end{array}
          \right.$}~\\[-1cm]
    \end{proof}

    
    \newpage

    
    \begin{nonoliste}{(i)}
      \setcounter{enumiii}{1}
    \item Justifier l'existence de l'entropie différentielle $h(X)$,
      et la déterminer.
    \end{nonoliste}

    \begin{proof}~
      \begin{noliste}{$\sbullet$}
      \item On commence par remarquer que, par définition de $f$, son
        support $I$ est l'intervalle $[0,a]$.\\
        Alors, d'après la définition de l'entropie, $X$ admet une
        entropie différentielle si et seulement si l'intégrale
        $\dint{0}{a} f(x) \, \log_2\big( f(x) \big) \dx$ est convergente.
        
      \item Or la fonction $x \mapsto f(x) \, \log_2 \big( f(x) \big)$
        est continue par morceaux sur le segment $[0,a]$. On en déduit
        que l'intégrale $\dint{0}{a} f(x) \, \log_2 \big(f(x) \big)
        \dx$ converge.
        \conc{Ainsi, la \var $X$ admet une entropie différentielle.}
        
      \item De plus :
        \[
          \begin{array}{rcl@{\qquad}>{\it}R{3.5cm}}
            h(X)
            & = & -\dint{0}{a} f(x) \log_2 \big( f(x) \big) \dx
            \\[.4cm]
            & = & -\dint{0}{a} \dfrac{1}{a} \ \log_2 \left( \dfrac{1}{a}
                  \right) \dx
            & (par définition de $f$ sur $[0,a]$)
              \nl
              \nl[-.2cm]
            & = & \dfrac{1}{a} \, \log_2(a) \ \dint{0}{a} \dx
            \\[.6cm]
            & = & \dfrac{1}{a} \, \log_2(a) \ \Prim{x}{0}{a}
            \\[.6cm]
            & = & \dfrac{1}{a} \, \log_2(a) \, (a-0)
            \\[.6cm]
            & = & \log_2(a)
          \end{array}
        \]
        \conc{$h(X) = \log_2(a)$}
      \end{noliste}
      % \begin{remark}
      %   On aurait aussi pu rédiger la preuve de l'existence de $h(X)$
      %   en exploitant complètement la question \itbf{2.} de la manière
      %   suivante.
      %   \begin{noliste}{$\sbullet$}
      %   \item D'après la question \itbf{2.}, la \var $X$ admet une
      %     entropie différentielle si et seulement la \var $\log_2
      %     \big( f(X) \big)$ admet une espérance.
          
      %   \item Par théorème de transfert, la \var $\log_2\big(f(X)\big)$
      %     admet une espérance si et seulement si l'intégrale
      %     $\dint{-\infty}{+\infty} \log_2\big(f(x)\big) \, f(x) \dx$ est
      %     absolument convergente.
          
      %   \item De plus, la fonction $f$ est
      %     nulle en dehors de l'intervalle $[0,a]$, donc :
      %     \[
      %       \dint{-\infty}{+\infty} f(x) \, \log_2\big(f(x)\big) \dx \ = \
      %       \dint{0}{a} f(x) \, \log_2\big(f(x)\big) \dx
      %     \] 
      %   \end{noliste}
      % \end{remark}
    \end{proof}

    \begin{nonoliste}{(i)}
      \setcounter{enumiii}{2}
    \item Déterminer une condition nécessaire et suffisante sur $a$
      pour que $h(X)>0$.
    \end{nonoliste}

    \begin{proof}~\\
      D'après la question précédente :
      \[
        \begin{array}{rcl@{\qquad}>{\it}R{5cm}}
          h(X) > 0
          & \Leftrightarrow & \log_2(a) > 0
          \\[.2cm]
          & \Leftrightarrow & \dfrac{\ln(a)}{\ln(2)} >0
          \\[.6cm]
          & \Leftrightarrow & \ln(a) >0
          & (car $\ln(2) >0$)
            \nl
            \nl[-.2cm]
          & \Leftrightarrow & a > 1
          & (par stricte croissance de la fonction $\exp$ sur $\R$)
        \end{array}
      \]
      \conc{$h(X) >0 \ \Leftrightarrow \ a >1$}~\\[-1cm]
    \end{proof}


    \newpage
    
    
  \item On considère $Y$ une variable aléatoire de loi normale centrée
    réduite. Montrer que $Y$ admet une entropie différentielle et :
    $h(Y) = \dfrac{1}{2} \ \log_2(2 \pi \ee)$.

    \begin{proof}~
      \begin{noliste}{$\sbullet$}
      \item Une densité de $Y$ est : $f_Y : x \mapsto
        \dfrac{1}{\sqrt{2 \pi}} \ \ee^{-\frac{x^2}{2}}$.\\
        On en déduit que le support de $f_Y$ est l'intervalle $I = \
        ]-\infty, + \infty[$.\\
        Alors, la \var $Y$ admet une entropie différentielle si et
        seulement si l'intégrale $\dint{-\infty}{+\infty} f_Y(x) \,
        \log_2 \big( f_Y(x) \big) \dx$ est convergente.
        
      \item Soit $x \in \R$.
        \[
          \begin{array}{rcl@{\qquad}>{\it}R{3cm}}
            \log_2 \big( f_Y(x) \big)
            & = & \log_2\left( \dfrac{1}{\sqrt{2 \pi}} \, \ee^{-
                  \frac{x^2}{2}} \right)
            \\[.6cm]
            & = & \log_2\left(\dfrac{1}{\sqrt{2 \pi}}\right) +
                  \log_2\left( \ee^{- \frac{x^2}{2}} \right)
            & (d'après la question \itbf{1.a)})
              \nl
              \nl[-.2cm]
            & = & \dfrac{\ln\left( \frac{1}{\sqrt{ 2 \pi}}
                  \right)}{\ln(2)} + \dfrac{\ln \left( \ee^{-
                  \frac{x^2}{2}} \right)}{\ln(2)}
            \\[.6cm]
            & = & \dfrac{-\frac{1}{2} \, \ln(2 \pi)}{\ln(2)} +
                  \dfrac{- \frac{x^2}{2}}{\ln(2)}
            \\[.6cm]
            & = & - \dfrac{1}{2} \ \dfrac{\ln(2 \pi)}{\ln(2)} -
                  \dfrac{1}{\ln(2)} \ \dfrac{x^2}{2} 
            \\[.6cm]
            & = & - \dfrac{1}{2} \ \log_2(2 \pi)- \dfrac{1}{2 \ln(2)} \
                 x^2
          \end{array}
        \]
        Ainsi :
        \[
          \begin{array}{rcl@{\qquad}>{\it}R{3.5cm}}
            f_Y(x) \, \log_2 \big( f_Y(x) \big)
            & = & f_Y(x) \left( - \dfrac{1}{2} \ \log_2(2 \pi)-
                  \dfrac{1}{2 \ln(2)} \ x^2 \right)
            \\[.6cm]
            & = & - \dfrac{1}{2} \ \log_2(2 \pi) \ f_Y(x) -
                  \dfrac{1}{2 \ln(2)} \ x^2 \, f_Y(x)
          \end{array}
        \]
        
      \item Or,
        \begin{noliste}{$\stimes$}
        \item l'intégrale $\dint{-\infty}{+\infty} f_Y(x) \dx$ est
          convergente car $f_Y$ est une densité. De plus :
          \[
            \dint{-\infty}{+\infty} f_Y(x) \dx \ = \  1
          \]
          
        \item l'intégrale $\dint{-\infty}{+\infty} x^2 \, f_Y(x) \dx$
          est convergente car la \var $Y$ admet un moment d'ordre
          $2$. De plus :
          \[
            \dint{-\infty}{+\infty} x^2 \, f_Y(x) \dx \ = \ \E(Y^2)
          \]
          D'après la formule de Koenig-Huygens : $\V(Y) \ = \ \E(Y^2)
          - \big( \E(Y)\big)^2$. Ainsi :
          \[
            \dint{-\infty}{+\infty} x^2 \, f_Y(x) \dx \ = \ \E(Y^2) \
            = \ \V(X) + \big(\E(Y)\big)^2 \ = \ 1 - 0 \ = \ 1
          \]


          \newpage

          
        \item On en déduit que l'intégrale $\dint{-\infty}{+\infty}
          f_Y(x) \, \log_2\big( f_Y(x) \big) \dx$ est convergente en tant que
          combinaison linéaire d'intégrales convergentes.
          \conc{Ainsi, la \var $Y$ admet une entropie différentielle.}
          
        \item De plus :
          \[
            \begin{array}{cl@{\qquad}>{\it}R{5cm}}
              & h(Y)
              \\[.2cm]
              = & -\dint{-\infty}{+\infty} f_Y(x) \, \log_2 \big(
                    f_Y(x) \big) \dx
              \\[.4cm]
              = & -\dint{-\infty}{+\infty} - \dfrac{1}{2} \ \log_2(2
                    \pi) \, f_Y(x) - \dfrac{1}{2 \ln(2)} x^2 \, f_Y(x)
                    \dx
              & (d'après les calculs précédents)
                \nl
                \nl[-.2cm]
              = & \dfrac{1}{2} \ \log_2 (2 \pi) \
                    \dint{-\infty}{+\infty} f_Y(x) \dx + \dfrac{1}{2
                    \ln(2)} \ \dint{-\infty}{+\infty} x^2 \, f_Y(x)
                    \dx
              & (par linéarité de l'intégrale, car les intégrales en
                présence sont convergentes)
                \nl
                \nl[-.2cm]
              = & \dfrac{1}{2} \ \log_2 (2 \pi) \times 1 + \dfrac{1}{2
                  \ln(2)} \times 1
              \\[.6cm]
              = & \dfrac{1}{2} \left( \log_2(2 \pi) +
                  \dfrac{1}{\ln(2)} \right)
              \\[.6cm]
              = & \dfrac{1}{2} \left( \log_2(2 \pi) +
                  \dfrac{\ln(\ee)}{ \ln(2)} \right)
              \\[.6cm]
              = & \dfrac{1}{2} \big( \log_2(2 \pi) + \log_2(\ee) \big)
            \end{array}
          \]
        \end{noliste}
        \conc{D'après la question \itbf{1.a)}, on obtient : $h(Y) \ =
          \ \dfrac{1}{2} \ \log_2(2 \pi \ee)$.}
      \end{noliste}
      \begin{remark}
        On a en particulier démontré dans nos calculs l'égalité
        suivante :
        \[
          \dfrac{1}{\ln(\ee)} \ = \ \log_2(\ee)
        \]
      \end{remark}~\\[-1.4cm]
    \end{proof}
    
  \item On considère $Z$ une variable aléatoire de loi exponentielle
    de paramètre $\lambda$ ($\lambda >0$). Justifier l'existence de
    l'entropie différentielle $h(Z)$ et la déterminer.

    \begin{proof}~
      \begin{noliste}{$\sbullet$}
      \item Une densité de $Z$ est :
        \[
          f_Z : x \mapsto \left\{
            \begin{array}{cR{2.5cm}}
              0 & si $x \in \ ]-\infty, 0[$
                  \nl
                  \nl[-.2cm]
              \lambda \, \ee^{-\lambda \, x} & si $x \in [0,+\infty[$
            \end{array}
          \right.
        \]
        On en déduit que le support de $f_Z$ est l'intervalle $I = \
        [0, + \infty[$.\\
        Alors, la \var $Z$ admet une entropie différentielle si et
        seulement si l'intégrale $\dint{0}{+\infty} f_Z(x) \,
        \log_2 \big( f_Z(x) \big) \dx$ est convergente.
        
      \item Soit $x \in [0,+\infty[$.
        \[
          \begin{array}{rcl@{\qquad}>{\it}R{3cm}}
            \log_2 \big( f_Z(x) \big)
            & = & \log_2\left( \lambda \, \ee^{-
                  \lambda \, x} \right)
            \\[.6cm]
            & = & \log_2\left(\lambda\right) +
                  \log_2\left( \ee^{- \lambda \, x} \right)
            & (d'après la question \itbf{1.a)})
              \nl
              \nl[-.2cm]
            & = & \log_2(\lambda) + \dfrac{\ln \left( \ee^{-
                  \lambda \, x} \right)}{\ln(2)}
            \\[.6cm]
            & = & \log_2(\lambda) +
                  \dfrac{- \lambda \, x}{\ln(2)}
            \\[.6cm]
            & = & \log_2(\lambda)- \dfrac{\lambda}{\ln(2)} \ x
          \end{array}
        \]
        Ainsi :
        \[
          \begin{array}{rcl}
            f_Z(x) \, \log_2 \big( f_Z(x) \big)
            & = & f_Y(x) \left( \log_2(\lambda)-
                  \dfrac{\lambda}{\ln(2)} \ x \right) 
            \\[.6cm]
            & = & \log_2(\lambda) \ f_Z(x) -
                  \dfrac{\lambda}{\ln(2)} \ x \, f_Z(x)
          \end{array}
        \]

      \item Or,
        \begin{noliste}{$\stimes$}
        \item l'intégrale $\dint{0}{+\infty} f_Z(x) \dx$ est
          convergente car $f_Z$ est une densité. De plus, comme $f_Z$
          est nulle en dehors de $[0,+\infty[$ :
          \[
            \dint{0}{+\infty} f_Z(x) \dx \ = \
            \dint{-\infty}{+\infty} f_Z(x) \dx \ = \  1
          \]
          
        \item l'intégrale $\dint{0}{+\infty} x \, f_Z(x) \dx$
          est convergente car la \var $Z$ admet une espérance. De
          plus, toujours comme $f_Z$ est nulle en dehors de $[0,+\infty[$ :
          \[
            \dint{0}{+\infty} x \, f_Z(x) \dx \ = \ 
            \dint{-\infty}{+\infty} x \, f_Z(x) \dx \ = \ \E(Z) \ = \
            \dfrac{1}{\lambda}
          \]
        \end{noliste}
          
        \item On en déduit que l'intégrale $\dint{0}{+\infty}
          f_Z(x) \, \log_2\big( f_Z(x) \big) \dx$ est convergente en tant que
          combinaison linéaire d'intégrales convergentes.
          \conc{Ainsi, la \var $Z$ admet une entropie différentielle.}

          
          \newpage
          
          
        \item De plus :
          \[
            \begin{array}{cl@{\qquad}>{\it}R{5cm}}
              & h(Z)
              \\[.2cm]
              = & -\dint{0}{+\infty} f_Z(x) \, \log_2\big( f_Z(x)
                    \big) \dx
              \\[.4cm]
              = & - \dint{0}{+\infty} \log_2(\lambda) \ f_Z(x) -
                  \dfrac{\lambda}{\ln(2)} \ x \, f_Z(x) \dx
              & (d'après les calculs précédents)
                \nl
                \nl[-.2cm]
              = & - \log_2(\lambda) \ \dint{0}{+\infty} f_Z(x) \dx +
                    \dfrac{\lambda}{\ln(2)} \ \dint{0}{+\infty} x \,
                    f_Z(x) \dx
              & (par linéarité de l'intégrale, car les intégrales en
                présence sont convergentes)
                \nl
                \nl[-.2cm]
              = & - \log_2(\lambda) \times 1 + \dfrac{\bcancel{\lambda}}{\ln(2)}
                  \times \dfrac{1}{\bcancel{\lambda}}
              \\[.6cm]
              = & - \log_2(\lambda) + \log_2(\ee)
            \end{array}
          \]
          \conc{D'après la question \itbf{1.a)}, on obtient : $h(Z) \
            = \ \log_2 \left( \dfrac{\ee}{\lambda} \right)$.}~\\[-1.4cm]
      \end{noliste}
    \end{proof}
    
  \item Soit $f$ la fonction définie sur $\R$ par $f(x) = \dfrac{1}{2}
    \ \lambda \, \ee^{-\lambda \, |x|}$ ($\lambda >0$).
    \begin{nonoliste}{(i)}
    \item Montrer que $f$ est une densité de probabilité sur $\R$.
    \end{nonoliste}
  \end{noliste}
  
  \begin{proof}~
    \begin{noliste}{$\sbullet$}
    \item La fonction $x \mapsto \ee^{-\lambda | x |}$ est continue
      sur $\R$ car elle est la composée $\exp \circ g_3$ de :
      \begin{noliste}{$\stimes$}
      \item $g_3 : x \mapsto -\lambda | x|$ qui :
        \begin{noliste}{-}
        \item est continue sur $\R$ en tant que produit de fonctions
          continues sur $\R$,
          
        \item vérifie : $g_3(\R) \subset \R$ (en fait on a même :
          $g_3(\R) \subset \ ]-\infty, 0]$).
        \end{noliste}
        
      \item $\exp$ qui est continue sur $\R$.
      \end{noliste}
      \conc{On en déduit que la fonction $f$ est continue sur $\R$.}

      \begin{remarkL}{.95}
        Afin de permettre une bonne compréhension de ce point, on a
        rédigé ici en détails la continuiré de la
        composée. Cependant, à ce stade du sujet, une phrase plus
        succinte du type \og la
        fonction $f$ est continue sur $\R$ en tant que composée et
        produit de fonctions continues sur $\R$ \fg{} permettait sans
        doute d'obtenir la totalité des points alloués à cette
        question.
      \end{remarkL}
      
    \item Soit $x \in \R$.\\
      Comme $\lambda >0$ et $\ee^{-\lambda | x |} >0$, on obtient :
      $f(x) >0$.
      \conc{$\forall x \in \R$, $f(x) \geq 0$}


      \newpage
      
      
    \item Montrons que l'intégrale $\dint{-\infty}{+\infty} f(x) \dx$
      converge et vaut $1$.
      \begin{noliste}{$\stimes$}
      \item Soit $A \in [0, +\infty[$.
        \[
          \begin{array}{rcl}
            \dint{0}{A} f(x) \dx
            & = & \dint{0}{A} \dfrac{1}{2} \ \lambda \, \ee^{- \lambda
                  | x |} \dx
            \\[.6cm]
            & = & \dfrac{1}{2} \ \dint{0}{A} \lambda \, \ee^{- \lambda
                  \, x} \dx
            \\[.6cm]
            & = & \dfrac{1}{2} \ \Prim{- \ee^{-\lambda x}}{0}{A}
            \\[.6cm]
            & = & \dfrac{1}{2} \left(-\ee^{-\lambda A} + 1\right)
          \end{array}
        \]
        Or, comme $\lambda >0$ : $\dlim{A \to +\infty} \ee^{- \lambda
          A} = 0$.
        \conc{On en déduit que l'intégrale $\dint{0}{+\infty} f(x)
          \dx$ est convergente et vaut $\dfrac{1}{2}$.}
        
      \item Soit $A \in \ ]-\infty, 0]$.
        \[
          \begin{array}{rcl}
            \dint{A}{0} f(x) \dx
            & = & \dint{A}{0} \dfrac{1}{2} \ \lambda \, \ee^{- \lambda
                  | x |} \dx
            \\[.6cm]
            & = & \dfrac{1}{2} \ \dint{A}{0} \lambda \, \ee^{ \lambda
                  \, x} \dx
            \\[.6cm]
            & = & \dfrac{1}{2} \ \Prim{\ee^{\lambda x}}{0}{A}
            \\[.6cm]
            & = & \dfrac{1}{2} \left(1 - \ee^{\lambda A}\right)
          \end{array}
        \]
        Or, comme $\lambda >0$ : $\dlim{A \to -\infty} \ee^{\lambda
          A} = 0$.
        \conc{On en déduit que l'intégrale $\dint{-\infty}{0} f(x)
          \dx$ est convergente et vaut $\dfrac{1}{2}$.}
        \conc{Finalement, l'intégrale $\dint{-\infty}{+\infty} f(x)
          \dx$ est convergente.}
        
      \item De plus :
        \[
          \dint{-\infty}{+\infty} f(x) \dx \ = \ \dint{-\infty}{0}
          f(x) \dx + \dint{0}{+\infty} f(x) \dx \ = \ \dfrac{1}{2} +
          \dfrac{1}{2} \ = \ 1
        \]
        \conc{$\dint{-\infty}{+\infty} f(x) \dx \ = \ 1$}
      \end{noliste}
      \conc{On en déduit que la fonction $f$ est une densité de probabilité.}
    \end{noliste}
    \begin{remark}
      L'intégrale $\dint{-\infty}{+\infty} f(x) \dx$ est impropre en
      ses deux bornes. On rappelle que pour démontrer que de telles
      intégrales sont convergentes, on démontre :
      \begin{noliste}{\scriptsize 1.}
      \item que l'intégrale $\dint{c}{+\infty} f(x) \dx$ est convergente,
        
      \item que l'intégrale $\dint{-\infty}{c} f(x) \dx$ est convergente.
      \end{noliste}
      où $c$ est un réel fixé. Fréquemment, on choisit $c=0$, comme
      c'est le cas pour cette question.
    \end{remark}~\\[-1.4cm]
  \end{proof}
  
  \begin{noliste}{(i)}
    \setcounter{enumii}{1}
  \item Soit $W$ une variable aléatoire de densité $f$. Justifier
    l'existence de l'entropie différentielle $h(W)$ et la déterminer.
  \end{noliste}
  
  \begin{proof}~
    \begin{noliste}{$\sbullet$}
    \item Par définition de $f$, son support est $]-\infty,
      +\infty[$.\\
      Ainsi, la \var $W$ admet une entropie différentielle si et
      seulement si l'intégrale $\dint{-\infty}{+\infty} f(x) \, \log_2
      \big( f(x) \big) \dx$ est convergente.
      
    \item Soit $x \in \R$.
      \[
        \begin{array}{rcl@{\qquad}>{\it}R{3cm}}
          \log_2 \big( f(x) \big)
          & = & \log_2\left( \dfrac{1}{2} \, \lambda \, \ee^{-
                \lambda \, \vert x \vert} \right)
          \\[.6cm]
          & = & -\log_2(2) + \log_2\left(\lambda\right) +
                \log_2\left( \ee^{- \lambda \, \vert x \vert} \right)
          & (d'après la question \itbf{1.a)})
            \nl
            \nl[-.2cm]
          & = & -1 + \log_2(\lambda) + \dfrac{\ln \left( \ee^{-
                \lambda \, \vert x \vert} \right)}{\ln(2)}
          & (car $\log_2(2^1) = 1$ d'après \itbf{1.b)})
            \nl
            \nl[-.2cm]
          & = & -1+ \log_2(\lambda) +
                \dfrac{- \lambda \, \vert x \vert}{\ln(2)}
          \\[.6cm]
          & = & -1+ \log_2(\lambda)- \dfrac{\lambda}{\ln(2)} \ \vert x \vert
        \end{array}
      \]
      Ainsi :
      \[
        \begin{array}{rcl}
          f(x) \, \log_2 \big( f(x) \big)
          & = & f(x) \left( -1+ \log_2(\lambda)-
                \dfrac{\lambda}{\ln(2)} \ \vert x \vert\right)
          \\[.6cm]
          & = & \big(-1 + \log_2(\lambda)\big) \, f(x) -
                \dfrac{\lambda}{\ln(2)} \ \vert x \vert \, f(x)
        \end{array}
      \]
      
    \item Or l'intégrale $\dint{-\infty}{+\infty} f(x) \dx$ est
      convergente car $f$ est une densité. De plus :
      \[
        \dint{-\infty}{+\infty} f(x) \dx \ = \ 1
      \]
        
    \item Démontrons maintenant que l'intégrale
      $\dint{-\infty}{+\infty} \vert x \vert \, f(x) \dx$ est convergente.
      \begin{noliste}{$\stimes$}
      \item La fonction $x \mapsto \vert x \vert \, f(x)$ est continue
        sur $\R$ en tant que produit de fonctions continues sur $\R$.


        \newpage

        
      \item Soit $A \in [0,+\infty[$.
        \[
          \begin{array}{rcl@{\qquad}>{\it}R{3cm}}
            \dint{0}{A} \vert x \vert \, f(x) \dx
            & = & \dfrac{1}{2} \ \dint{0}{A} x \ \lambda \, \ee^{-
                  \lambda x} \dx
            \\[.6cm]
            & = & \dfrac{1}{2} \ \dint{0}{A} x \, f_Z(x) \dx
            & (car $Z \suit \Exp{\lambda}$)
          \end{array}
        \]
        Or la \var $Z$ admet une espérance. Ainsi, avec le même raisonnement
        qu'en question \itbf{4.c)}, on en déduit que l'intégrale
        $\dint{0}{+\infty} x \, f_Z(x) \dx$ est convergente et :
        \[
          \dint{0}{+\infty} x \, f_Z(x) \dx \ = \ \E(Z) \ = \
          \dfrac{1}{\lambda}
        \]
        \conc{Finalement, l'intégrale $\dint{0}{+\infty} \vert x \vert
          f(x) \dx$ est convergente et :\\
          $\dint{0}{+\infty} \vert x \vert \, f(x) \dx \ = \
          \dfrac{1}{2} \times \dfrac{1}{\lambda}$.}
        
      \item Soit $A \in \ ]-\infty, 0]$.
        \[
          \begin{array}{rcl}
            \dint{A}{0} \vert x \vert \, f(x) \dx
            & = & \dfrac{1}{2} \ \dint{A}{0} (-x) \, \lambda \ee^{-
                  \lambda \, (-x)} \dx
            \\[.6cm]
            & = & - \dfrac{1}{2} \ \dint{A}{0} x \, \lambda
                  \ee^{\lambda x} \dx
          \end{array}
        \]
        Pour calculer l'intégrale $\dint{A}{0} x \, \lambda
        \ee^{\lambda x} \dx$, on procède par intégration par parties (IPP).
        \[
          \renewcommand{\arraystretch}{2.2}
          \begin{array}{|rcl@{\qquad}rcl}
            u(x) & = & x & u'(x) & = & 1 \\
            v'(x) & = & \lambda \ee^{\lambda x} & v(x) & = & \ee^{\lambda x}
          \end{array}
        \]
        Cette IPP est valide car les fonctions $u$ et $v$ sont 
        de classe $\Cont{1}$
        sur $[A, 0]$. On obtient :
        \[
          \begin{array}{rcl}
            \dint{A}{0} x \, \lambda \ee^{\lambda x} \dx
            & = & \Prim{x \ \lambda \ee^{\lambda x}}{A}{0} -
                  \dint{A}{0} \ee^{\lambda x} \dx
            \\[.6cm]
            & = & \left(0 -\lambda \, A \, \ee^{\lambda \, A}\right) -
                  \Prim{ \dfrac{1}{\lambda} \, \ee^{\lambda x}}{A}{0}
            \\[.6cm]
            & = & - \lambda \, A \, \ee^{\lambda \, A}
                  -\dfrac{1}{\lambda} \, (1-\ee^{\lambda \, A})
            \\[.6cm]
            & = & -\dfrac{1}{\lambda} + \dfrac{1}{\lambda} \
                  \ee^{\lambda A} - \lambda \, A \, \ee^{\lambda \, A}
          \end{array}
        \]


        \newpage

        
        \noindent
        Or, comme $\lambda >0$ :
        \begin{noliste}{-}
        \item $\dlim{A \to - \infty} \ee^{\lambda \, A} \ = \ 0$,
          
        \item $\dlim{A \to - \infty} A \, \ee^{\lambda \, A} \ = \ 0$,
          par croissances comparées.
        \end{noliste}
        On en déduit que l'intégrale $\dint{-\infty}{0} x \, \lambda
        \ee^{\lambda x} \dx$ est convergente et vaut $-
        \dfrac{1}{\lambda}$.
        \conc{Ainsi, l'intégrale $\dint{-\infty}{0} \vert x \vert \,
          f(x) \dx$ est convergente et :\\
          $\dint{-\infty}{0} \vert x \vert \, f(x) \dx \ = \
          -\dfrac{1}{2} \times \left(- \dfrac{1}{\lambda}\right) \ = \
          \dfrac{1}{2} \ \dfrac{1}{\lambda}$.}
      \end{noliste}
      \conc{On en déduit que l'intégrale $\dint{-\infty}{+\infty}
        \vert x \vert \, f(x) \dx$ est convergente.}
      De plus :
      \[
        \dint{-\infty}{+\infty} \vert x \vert \, f(x) \dx \ = \
        \dint{-\infty}{0} \vert x \vert \, f(x) \dx +
        \dint{0}{+\infty} \vert x \vert \, f(x) \dx
        \ = \ \dfrac{1}{2} \ \dfrac{1}{\lambda} + \dfrac{1}{2} \
        \dfrac{1}{\lambda} \ = \ \dfrac{1}{\lambda}
      \]
      \conc{$\dint{-\infty}{+\infty} \vert x \vert \, f(x) \dx \ = \
        \dfrac{1}{\lambda}$}
      
    \item On en déduit que l'intégrale $\dint{-\infty}{+\infty} f(x)
      \, \log_2 \big( f(x) \big) \dx$ est convergente en tant que
      combinaison linéaire d'intégrales convergentes.
      \conc{Ainsi, la \var $W$ admet une entropie différentielle.}
      
    \item De plus :
      \[
        \begin{array}{cl@{\qquad}>{\it}R{5cm}}
          & h(W)
          \\[.2cm]
          = & - \dint{-\infty}{+\infty} f(x) \, \log_2\big( f(x) \big)
              \dx
          \\[.4cm]
          = & - \dint{-\infty}{+\infty} \big(-1+ \log_2(\lambda) \big)
              \, f(x) - \dfrac{\lambda}{\ln(2)} \, f(x) \dx
          & (d'après les calculs précédents)
            \nl
            \nl[-.2cm]
          = & \big(1- \log_2(\lambda) \big) \ \dint{-\infty}{+\infty}
              f(x) \dx - \dfrac{\lambda}{\ln(2)} \
              \dint{-\infty}{+\infty} \vert x \vert \, f(x) \dx
          & (par linéarité de l'intégrale, car les intégrales en
            présence sont convergentes)
            \nl
            \nl[-.2cm]
          = & \big(1- \log_2(\lambda)\big) \times 1 -
              \dfrac{\bcancel{\lambda}}{ \ln(2)} \times
              \dfrac{1}{\bcancel{\lambda}}
          \\[.6cm]
          = & 1 - \log_2(\lambda) + \log_2(\ee)
        \end{array}
      \]
      \conc{D'après la question \itbf{1.a)}, on obtient : $h(W) = 1 +
        \log_2 \left( \dfrac{\ee}{\lambda} \right)$.}~\\[-1.4cm]
    \end{noliste}
  \end{proof}
  \begin{remark}
    On dit que la \var $W$ suit la {\it loi de Laplace de paramètre
      $(0,1)$}. L'étude de cette loi est classique aux concours. On
    pourra par exemple regarder l'épreuve ESSEC I 2017 pour une
    étude plus détaillée de cette loi.
  \end{remark}
  
  
\item On dit qu'un couple $(X,Y)$ de variables aléatoires est un
  couple gaussien centré si, pour tout $(\alpha, \beta) \in \R^2$,
  $\alpha \, X + \beta \, Y$ est une variable de loi normale centrée,
  c'est-à-dire qu'il existe $\gamma \in \R$ et une variable aléatoire
  $Z$ de loi normale centrée réduite tels que $\alpha \, X + \beta \,
  Y$ a même loi que $\gamma \, Z$. On considère un tel couple $(X,Y)$
  et on noté $\sigma^2$ la variance de $X$. On suppose : $\sigma^2
  >0$.
  \begin{noliste}{a)}
    \setlength{\itemsep}{2mm}
  \item Montrer que $X$ suit une loi normale centrée.

    \begin{proof}~\\
      Comme $(X,Y)$ est un couple gaussien centré, alors {\bf pour tout}
      $(\alpha, \beta) \in \R^2$, $\alpha \, X + \beta \, Y$ suit une
      loi normale centrée. {\bf En particulier}, la \var $1 \cdot X + 0 \cdot
      Y=X$ suit une loi normale centrée.\\
      De plus, d'après l'énoncé : $\V(X) = \sigma^2$.
      \conc{Finalement : $X \suit \Norm{0}{\sigma^2}$.}~\\[-1cm]
    \end{proof}
    
  \item Calculer $h(X)$.

    \begin{proof}~
      \begin{noliste}{$\sbullet$}
      \item D'après la question précédente : $X \suit \Norm{0}{\sigma^2}$.
        \conc{On en déduit que $X$ suit la même loi que la \var
          $\sigma \, Z$, où : $Z \suit \Norm{0}{1}$.}

        \begin{remark}
          \begin{noliste}{$\sbullet$}
          \item On rappelle qu'on a le résultat suivant :
            \[
              X \suit \Norm{m}{\sigma^2} \quad \Leftrightarrow \quad a
              \, X + b \suit \Norm{am +b}{a^2 \, \sigma^2}
            \]
            
          \item En particulier :
            \[
              Z \suit \Norm{0}{1} \quad \Leftrightarrow \quad \sigma
              \, Z \suit \Norm{0}{\sigma^2}
            \]
          \end{noliste}
        \end{remark}
        
      \item De plus, d'après la question \itbf{4.b)}, la \var $Z$
        admet une entropie différentielle et :
        \[
          h(Z) \ = \ \dfrac{1}{2} \, \log_2(2 \pi \ee)
        \]
        
      \item Enfin, d'après la question \itbf{3.b)(ii)} :
        \begin{noliste}{$\stimes$}
        \item la \var $X$ admet une entropie différentielle,
          
        \item on obtient de plus :
          \[
            h(X) \ = \ h(Z) + \log_2(\sigma) \ = \ \dfrac{1}{2} \,
            \log(2 \pi \ee) + \dfrac{1}{2} \, \log_2(\sigma^2) \ = \
            \dfrac{1}{2} \, \log_2(2 \pi \, \ee \, \sigma^2)
          \]
        \end{noliste}
        \conc{$h(X) \ = \ \dfrac{1}{2} \, \log_2(2 \pi \, \ee \,
          \sigma^2)$}~\\[-1.4cm]
      \end{noliste}
    \end{proof}
    
  \item On suppose désormais que $X$ et $Y$ suivent la même loi
    normale centrée de variance $\sigma^2$ et on admet que les
    propriétés de l'espérance des variables discrètes se généralisent
    aux variables aléatoires quelconques.
    \begin{nonoliste}{(i)}
      \setlength{\itemsep}{2mm}
    \item Montrer que $\E(XY)$ existe.

      \begin{proof}~\\
        Les \var $X$ et $Y$ suivent une loi normale.\\
        Elles admettent donc chacune un moment d'ordre $2$.
        \conc{On en déduit que la \var $XY$ admet
          une espérance.}


        \newpage
        

        \begin{remark}
          \begin{noliste}{$\sbullet$}
          \item On rappelle que, dans le cas général, la \var produit
            $XY$ admet une espérance si les \var $X$ et $Y$ admettent
            {\bf un moment d'ordre $2$}.
            
          \item On peut se demander d'où provient cette hypothèse liée
            aux moments d'ordre $2$.\\
            Elle est issue d'un théorème de domination. Détaillons ce
            point.\\
            Remarquons tout d'abord : $(X-Y)^2 \geq 0$.\\
            On en déduit : $X^2 -2 \, XY + Y^2 \geq 0$.\\
            Et, en réordonnant : $XY \leq \dfrac{1}{2} \ X^2 +
            \dfrac{1}{2} \ Y^2$. Ou encore :
            \[
              0 \ \leq \ \vert XY \vert \ \leq \ \dfrac{1}{2} \ X^2 +
              \dfrac{1}{2} \ Y^2
            \]
            Comme $X$ et $Y$ admettent un moment d'ordre $2$, la \var
            $\frac{1}{2} \ X^2 + \frac{1}{2} \ Y^2$ admet une
            espérance comme combinaison linéaire de \var qui en
            admettent une.\\
            Ainsi, par théorème de domination (présenté seulement dans
            le programme ECS), la \var $|XY|$ admet une espérance. Il
            en est de même de la \var $XY$.
          \end{noliste}
        \end{remark}~\\[-1.4cm]
      \end{proof}
      
    \item Montrer de plus, pour tout réel $\lambda$ : $\lambda^2 \,
      \E\left( Y^2 \right) + 2 \lambda \, \E(XY) + \E\left( X^2
      \right) \ \geq \ 0$.
      
      \begin{proof}~\\
        Soit $\lambda \in \R$.
        \[
          \begin{array}{rcl@{\qquad}>{\it}R{3cm}}
            \lambda^2 \, \E(Y^2) + 2 \lambda \, \E(XY) + \E(X^2)
            & = & \E\left( \lambda^2 \, Y^2 + 2 \lambda \, XY + X^2
                  \right)
            & (par linéarité de l'espérance)
              \nl
              \nl[-.2cm]
            & = & \E\Big( (\lambda \, Y + X)^2 \Big)
          \end{array}
        \]
        Or : $(\lambda \, Y + X)^2 \geq 0$.\\
        Ainsi, par croissance de l'espérance : $\E \Big( (\lambda Y +
        X)^2 \Big) \geq 0$.
        \conc{On en déduit : $\lambda^2 \, \E(Y^2) + 2 \lambda \,
          \E(XY) + \E(X^2) \geq 0$.}
        \begin{remark}
          On prêtera attention à la différence de nature des deux
          inégalités citées plus haut : 
          \begin{noliste}{$\stimes$}
          \item l'inégalité $(\lambda \, Y + X)^2 = 0$ est une inégalité
            entre {\bf variables aléatoires}. Ainsi, \og $0$ \fg{} est ici
            la variable aléatoire constante égale à $0$.
            
          \item l'inégalité $\E\Big( (\lambda \, Y +X)^2 \Big) \geq 0$
            est une inégalité entre {\bf réels}. Ainsi, \og $0$ \fg{}
            ici tout simplement le réel $0$.
          \end{noliste}
        \end{remark}~\\[-1.4cm]
      \end{proof}
      
    \item En déduire : $\big( \E(XY) \big)^2 \ \leq \ \E\left( X^2
      \right) \, \E\left( Y^2 \right)$.

      \begin{proof}~
        \begin{noliste}{$\sbullet$}
        \item La fonction $g$ définie par : 
          \[
            \forall \lambda \in \R, \quad P(\lambda) = \lambda^2 \,
            \E(Y^2) + 2 \lambda \, \E(XY) + \E(X^2)
          \]
          est une fonction polynomiale de degré $2$ en $\lambda$.

          
          \newpage

          
        \item De plus, d'après la question précédente : $\forall
          \lambda \in \R$, $g(\lambda) \geq 0$.\\
          Cette fonction polynomiale étant de signe constant, on en
          déduit que le  discriminant du polynôme associé est négatif.
          Or :
          \[
             \Delta \ = \ \big(2 \, \E(XY)\big)^2 - 4 \, \E(Y^2) \,
             \E(X^2) \ = \ 4 \, \Big(\big(\E(XY)\big)^2 - \E(Y^2) \, \E(X^2)\Big)
          \]
          On obtient donc :
          \[
            \Delta \leq 0 \quad \Leftrightarrow \quad
            \big(\E(XY)\big)^2 \ \leq \  \E(Y^2) \, \E(X^2)
          \]
        \end{noliste}
        \conc{Ainsi : $\big(\E(XY)\big)^2 \ \leq \ \E(Y^2) \, \E(X^2)$.}
        \begin{remarkL}{.8}
          On reconnaît ici exactement le principe de démonstration de l'inégalité
          de Cauchy-Schwarz :
          \[
            \big(\cov(X,Y)\big)^2 \ \leq \ \V(X) \, \V(Y)
          \]
        \end{remarkL}~\\[-1.4cm]
      \end{proof}
      
    \item On pose $\rho = \dfrac{\E(XY)}{\sigma^2}$. Montrer : $\rho
      \in [-1,1]$.
    \end{nonoliste}

    \begin{proof}~
      \begin{noliste}{$\sbullet$}
      \item D'après la question précédente :
        \[
          \begin{array}{rcl@{\qquad}>{\it}R{4cm}}
            \big(\E(XY)\big)^2 \ \leq \ \E(Y^2) \, \E(X^2)
            & \Leftrightarrow & \sqrt{\big(\E(XY)\big)^2} \ \leq \
                                \sqrt{\E(Y^2) \, \E(X^2)}
            & (par stricte croissance de $\sqrt{\cdot}$ sur
              $[0,+\infty[$)
              \nl
              \nl[-.2cm]
            & \Leftrightarrow & \left\vert \E(XY) \right\vert \ \leq \
                                \sqrt{\E(Y^2)} \, \sqrt{\E(X^2)}
          \end{array}
        \]
        
      \item De plus, par fomule de Koenig-Huygens :
        \[
          \begin{array}{ccccc@{\qquad}>{\it}R{3cm}}
            \V(X) & = & \E(X^2) & - & \big(\E(X)\big)^2
            \\[.2cm]
            \shortparallel & & & & \shortparallel
            \\
            \sigma^2 & & & & 0
            & (car $X$ est une \var centrée)
          \end{array}
        \]
        Ainsi : $\E(X^2) = \sigma^2$. De même : $\E(Y^2) = \sigma^2$.
        
      \item On en déduit :
        \[
          \begin{array}{rcl@{\qquad}>{\it}R{3cm}}
            \left\vert \E(XY) \right\vert \ \leq \
            \sqrt{\E(Y^2)} \, \sqrt{\E(X^2)}
            & \Leftrightarrow & \left\vert \E(XY) \right\vert \ \leq \
                                \sqrt{\sigma^2} \, \sqrt{\sigma^2}
            \\[.2cm]
            & \Leftrightarrow & \left\vert \E(XY) \right\vert \ \leq \
                                \sigma^2
            \\[.2cm]
            & \Leftrightarrow & \dfrac{\left\vert \E(XY)
                                \right\vert}{\sigma^2} \ \leq \ 1
            \\[.6cm]
            & \Leftrightarrow & \left\vert \dfrac{\E(XY)}{\sigma^2}
                                \right\vert \ \leq \ 1
            & (car $\sigma^2 \geq 0$)
              \nl
              \nl[-.2cm]
            & \Leftrightarrow & | \rho | \ \leq \ 1
          \end{array}
        \]
      \end{noliste}
      \conc{Ainsi : $-1 \ \leq \ \rho \ \leq \ 1$.}


      \newpage

      
        \begin{remark}
          \begin{noliste}{$\sbullet$}
          \item On reconnaît ici le {\it coefficient de corrélation linéaire} défini
            par :
            \[
              \rho \ = \ \dfrac{\cov(X,Y)}{\sqrt{\V(X)} \, \sqrt{\V(Y)}}
            \]
            dans le cas où $X$ et $Y$ sont deux \var :
            \begin{noliste}{$\stimes$}
            \item centrées (dans ce cas, par formule de Koenig-Huygens
              : $\Cov(X,Y) = \E(XY)$),
              
            \item de variance $\sigma^2$.
            \end{noliste}
            
          \item Le programme d'ECE ne définit ce coefficient que pour des
          \var discrètes. On peut cependant généraliser cette
          définition au cas de \var quelconques. Cette question et la
          suivante montrent que l'on conserve les grandes propriétés
          du coefficient de corrélation linéaire défini en cours.
          \end{noliste}
        \end{remark}~\\[-1.4cm]
      \end{proof}

    \begin{nonoliste}{(i)}
      \setcounter{enumiii}{4}
    \item Que vaut $\rho$ si $X$ et $Y$ sont indépendantes ?
    \end{nonoliste}

    \begin{proof}~\\
      Si les \var $X$ et $Y$ sont indépendantes, alors :
      \[
        \begin{array}{rcl@{\qquad}>{\it}R{3cm}}
          \E(XY)
          & = & \E(X) \, \E(Y)
          \\[.2cm]
          & = & 0
          & (car $X$ et $Y$ sont des \var centrées)
        \end{array}
      \]
      \conc{Ainsi, si les \var $X$ et $Y$ sont indépendantes, alors : $\rho \
        = \ \dfrac{\E(XY)}{\sigma^2} \ = \ 0$.}
      \begin{remark}
        Remarquons bien qu'on démontre ici {\bf l'implication} :
        \[
          \text{$X$ et $Y$ indépendantes} \quad \Rightarrow \quad \rho =0
        \]
        Ce n'est en rien une équivalence.
      \end{remark}~\\[-1.4cm]
    \end{proof}
    
  \item On suppose $|\rho| <1$. On appelle {\bf entropie jointe} du
    couple $(X,Y)$ le réel :
    \[
      h(X,Y) \ = \ \log_2\left(2 \pi \, \ee \, \sigma^2 \, \sqrt{1-
          \rho^2} \right)
    \]
    \begin{nonoliste}{(i)}
      \setlength{\itemsep}{2mm}
    \item À quelle condition $h(X,Y)$ est-elle nulle ?

      \begin{proof}~\\
        On a les équivalences suivantes :
        \[
          \begin{array}{rcl@{\qquad}>{\it}R{4cm}}
            h(X,Y) = 0
            & \Leftrightarrow & \log_2 \left( 2 \pi \, \ee \, \sigma^2
                                \, \sqrt{1- \rho^2}\right) \ = \ 0
            \\[.4cm]
            & \Leftrightarrow & \dfrac{\ln\left( 2 \pi \, \ee \,
                                \sigma^2 \, \sqrt{1- \rho^2}\right)}{
                                \ln(2)} \ = \ 0
            \\[.6cm]
            & \Leftrightarrow & \ln\left( 2 \pi \, \ee \,
                                \sigma^2 \, \sqrt{1- \rho^2}\right) \
                                = \ 0
            \\[.6cm]
            & \Leftrightarrow & 2 \pi \, \ee \, \sigma^2 \, \sqrt{1-
                                \rho^2} \ = \ 1
            & (par bijectivité de la fonction $\exp$ sur $\R$)
          \end{array}
        \]


        \newpage


        \noindent
        De plus :
        \[
          \begin{array}{rcl@{\qquad}>{\it}R{5cm}}
            2 \pi \, \ee \, \sigma^2 \, \sqrt{1-
            \rho^2} \ = \ 1
            & \Leftrightarrow & \sqrt{1- \rho^2} \ = \ \dfrac{1}{2 \pi
                                \ee \sigma^2}
            \\[.6cm]
            & \Leftrightarrow & 1- \rho^2 \ = \ \left( \dfrac{1}{2 \pi
                                \ee \sigma^2} \right)^2
            & (par bijectivité de la fonction $x \mapsto x^2$ sur
              $[0,+\infty[$)
              \nl
              \nl[-.2cm]
            & \Leftrightarrow & \rho^2 \ = \ 1 - \dfrac{1}{\big( 2 \pi
                                \ee \sigma^2\big)^2}
          \end{array}
        \]
        \conc{Finalement : $h(X,Y) = 0 \ \Leftrightarrow \ \rho \in
          \left\{- \sqrt{1 - \dfrac{1}{\big( 2 \pi \ee
                \sigma^2\big)^2}} \ , \ \sqrt{1 - \dfrac{1}{\big( 2 \pi
                  \ee \sigma^2\big)^2}} \right\} $.}
        \begin{remark}
          Cette question est un peu vague :
          \begin{noliste}{$\stimes$}
          \item demande-t-on une condition
            nécessaire ? suffisante ? nécessaire et suffisante ?
            
          \item demande-t-on une condition sur $\rho$ ou sur un autre
            paramètre ?
          \end{noliste}
          Sans plus de précisions, on considèrera que les termes \og
          nécessaire et suffisante \fg{} sont sous-entendus. On choisit
          par ailleurs dans ce corrigé de considérer qu'on demande
          bien une condition sur le paramètre $\rho$.
        \end{remark}~\\[-1.4cm]
      \end{proof}
      
    \item L'{\bf information mutuelle} de $X$ et $Y$ est définie par :
      \[
        I(X,Y) \ = \ h(X) + h(Y) - h(X,Y)
      \]
      Calculer $I(X,Y)$.

      \begin{proof}~\\
        On calcule :
        \[
          \begin{array}{rcl@{\qquad}>{\it}R{3cm}}
            I(X,Y)
            & = & h(X) + h(Y) - h(X,Y)
            \\[.2cm]
            & = & \dfrac{1}{2} \, \log_2(2 \pi \, \ee \, \sigma^2) +
                  \dfrac{1}{2} \, \log_2(2 \pi \, \ee \, \sigma^2) -
                  \log_2\left( 2 \pi \, \ee \, \sigma^2 \, \sqrt{1-
                  \rho^2} \right)
            & (d'après la question \itbf{5.b)})
              \nl
              \nl[-.2cm]
            & = & \log_2(2 \pi \, \ee \, \sigma^2) - \log_2\left(2 \pi
                  \, \ee \, \sigma^2 \, \sqrt{1- \rho^2}\right)
            \\[.4cm]
            & = & -\log_2\left( \dfrac{ \bcancel{2\pi \, \ee \, \sigma^2} \,
                  \sqrt{1- \rho^2}}{\bcancel{2\pi \, \ee \,
                  \sigma^2}} \right)
            & (d'après la question \itbf{1.a)})
              \nl
              \nl[-.2cm]
            & = & -\log_2\left(\sqrt{1- \rho^2}\right)
          \end{array}
        \]
        \conc{Finalement : $I(X,Y) \ = \ -\dfrac{1}{2} \, \log_2(1-
          \rho^2)$}~\\[-1cm]
      \end{proof}


      \newpage
      
      
    \item Montrer : $I(X,Y) \geq 0$.

      \begin{proof}~\\
        On a la succession d'équivalences suivante :
        \[
          \begin{array}{rcl@{\qquad}>{\it}R{5cm}}
            I(X,Y) \geq 0
            & \Leftrightarrow & -\dfrac{1}{2} \, \log_2(1- \rho^2)
                                \geq 0
            & (d'après la question précédente)
              \nl
              \nl[-.2cm]
            & \Leftrightarrow & \log_2(1- \rho^2) \leq 0
            \\[.4cm]
            & \Leftrightarrow & \dfrac{\ln(1- \rho^2)}{\ln(2)} \leq 0
            \\[.6cm]
            & \Leftrightarrow & \ln(1- \rho^2) \leq 0
            & (car $\ln(2) >0$)
              \nl
              \nl[-.2cm]
            & \Leftrightarrow & 1- \rho^2 \leq 1
            & (par stricte croissance de la fonction $\exp$ sur $\R$)
              \nl
              \nl[-.2cm]
            & \Leftrightarrow & 0 \leq \rho^2
          \end{array}
        \]
        La dernière inégalité est vraie. Ainsi, par équivalence, la
        première inégalité aussi.
        \conc{Finalement : $I(X,Y) \geq 0$.}~\\[-1cm]
      \end{proof}
      
    \item Quelle est la limite de $I(X,Y)$ quand $\rho$ tend vers $1$ ?
    \end{nonoliste}

    \begin{proof}~
      \begin{noliste}{$\sbullet$}
      \item D'après la question \itbf{5.d)(ii)} :
        \[
          I(X,Y) \ = \ -\dfrac{1}{2} \, \log_2(1- \rho^2) \ = \
          -\dfrac{1}{2} \ \dfrac{\ln(1- \rho^2)}{\ln(2)}
        \]
        
      \item Tout d'abord, comme $\dlim{\rho \to 1} 1- \rho^2 = 0$,
        alors : $\dlim{\rho \to 1} \ln(1-\rho^2) \ = \ - \infty$.
      \end{noliste}
      \conc{Comme de plus $-\dfrac{1}{2 \, \ln(2)} <0$, on obtient :
        $\dlim{\rho \to 1} I(X,Y) = + \infty$.}~\\[-1cm]
    \end{proof}
  \end{noliste}
\end{noliste}


\newpage


\subsection*{Deuxième partie : Généralités sur l'entropie des
  variables discrètes}

\noindent
Soit $A$ un ensemble fini non vide. On dit que $X$ est une variable
aléatoire dont la loi est à support $A$, si $X$ est à valeurs dans $A$
et si pour tout $x \in A$ : $\Prob(\Ev{X = x}) >0$.
\begin{noliste}{1.}
  \setlength{\itemsep}{4mm}
  \setcounter{enumi}{5}
\item Soit $X$ une variable aléatoire de loi à support $\{0,1,2,
  \ldots, n\}$ où $n$ est un entier naturel. On appelle {\bf entropie}
  de $X$ le réel :
  \[
    H(X) \ = \ - \Sum{k=0}{n} \Prob(\Ev{X = k}) \, \log_2\big(
    \Prob(\Ev{X = k})\big)
  \]
  \begin{noliste}{a)}
    \setlength{\itemsep}{2mm}
  \item On définit la fonction $g : \{0, \ldots,n\} \to \R$ en posant
    $g(k) = \log_2 \big( \Prob(\Ev{X = k}) \big)$ pour $k$ élément de
    $\{0, 1, \ldots, n\}$. Montrer : $H(X) = - \E \big(g(X)\big)$.
    
  \item Montrer : $H(X) \geq 0$.
    
  \item Soit $p$ un réel tel que $0 < p < 1$.\\
    On suppose dans cette
    question que $X$ suit la loi de Bernoulli $\Bern{p}$.
    \begin{nonoliste}{(i)}
      \setlength{\itemsep}{2mm}
    \item Calculer $H(X)$ en fonction de $p$. On note $\psi$ la
      fonction qui, à $p$, associe $H(X)$.
      
    \item Montrer que $\psi$ est concave sur $]0,1[$.
      
    \item Déterminer la valeur $p_0$ où $\psi$ est maximale.
    \end{nonoliste}
    
  \item On suppose dans cette question que la loi de $X$ est à support
    $\{0,1,2,3\}$ avec les probabilités :
    \[
      \Prob(\Ev{X = 0}) = \dfrac{1}{2} \quad ; \quad \Prob(\Ev{X = 1})
      = \dfrac{1}{4} \quad ; \quad \Prob(\Ev{X = 2}) = \Prob(\Ev{X =
        3}) = \dfrac{1}{8}
    \]
    Calculer $H(X)$.
  \end{noliste}
  
\item On souhaite écrire une fonction en \Scilab{} pour calculer
  l'entropie d'une variable aléatoire $X$ dont le support de la loi
  est de la forme $A = \{0,1, \ldots, n\}$ où $n$ est un entier
  naturel. On suppose que le vecteur {\tt P} de \Scilab{} est tel que
  pour tout $k$ de $A$, ${\tt P(k + 1)} = \Prob(\Ev{X =
    k})$. Compléter la fonction ci-dessous d'argument {\tt P} qui
  renvoie l'entropie de $X$, c'est-à-dire $-\Sum{k=0}{n} \Prob(\Ev{X =
    k}) \, \log_2\big( \Prob(\Ev{X = k}) \big)$.
  \begin{scilab}
    & \tcFun{function} \tcVar{h} = \underline{Entropie}(\tcVar{P})
    \nl %
    & ... \nl %
    & \tcFun{endfunction}
  \end{scilab}
  Si nécessaire, on pourra utiliser l'instruction {\tt length(P)} qui
  donne le nombre d'éléments de {\tt P}.\\
  On souhaite maintenant démontrer quelques inégalités concernant
  l'entropie.
  
\item On commence par une inégalité générale, appelée {\bf Inégalité
    de Jensen}.
  \begin{noliste}{a)}
    \setlength{\itemsep}{2mm}
  \item Soit $N \geq 2$. Soit $X$ une variable aléatoire de loi à
    support $\{x_1, x_2, \ldots, x_N\}$ où les $x_i$ sont des éléments
    distincts de $\R_+$. On pose $\Prob(\Ev{X = x_i}) = p_i$.\\
    Montrer que, pour tout $1 \leq i \leq N$, on a : $p_i <
    1$.\\[.4cm]
    On désire démontrer par récurrence la propriété suivante :
    \[
      \begin{array}{rc@{\qquad}>{\it}R{12cm}}
        \PP{N} & : & {\bf Pour toute fonction $\varphi$ convexe sur
                     $\R_+$, si $X$ est une variable aléatoire de loi
                     à support $A \subset \R_+$ avec $\Card(A) = N$,
                     on a : $\E\big(\varphi(X)\big) \geq \varphi\big(
                     \E(X) \big)$}
      \end{array}
    \]
    
  \item Montrer que $\PP{2}$ est vraie.


    \newpage


  \item Soit $N \geq 3$. On suppose que $\PP{N-1}$ est vérifiée. Soit
    $X$ une variable aléatoire de loi à support $A= \{x_1, x_2,
    \ldots, x_N\}$ où les $x_i$ sont des éléments distincts de
    $\R_+$. On pose : $\Prob(\Ev{X = x_i}) = p_i$.\\
    Pour $i$ tel que $1 \leq i \leq N-1$, on pose : $p_i' =
    \dfrac{p_i}{1 - p_N}$.
    \begin{nonoliste}{(i)}
      \setlength{\itemsep}{2mm}
    \item Montrer : $\Sum{i=1}{N-1} p_i' = 1$ et $ \forall i \in \llb 1,
      N-1 \rrb$, $0< p_i' <1$.
      
    \item Soit $Y$ une variable aléatoire de loi à support $\{x_1,
      \ldots, x_{N-1}\}$ telle que $\Prob(\Ev{Y = x_i}) = p_i'$ pour
      $1 \leq i \leq N-1$. Montrer : $\Sum{i=1}{N-1} p_i' \,
      \varphi(x_i) \ \geq \ \varphi\left( \Sum{i=1}{N-1} p_i' \, x_i
      \right)$.
      
    \item Montrer : $\E\big( \varphi(X) \big) \ \geq \ \varphi\big( \E(X)
      \big)$.
    \end{nonoliste}
    
  \item Montrer que, si $\varphi$ est {\it concave} sur $\R_+$, on a :
    $\E\big( \varphi(X)\big) \ \leq \ \varphi\big( \E(X) \big)$.
  \end{noliste}
  
\item Soit $X$ une variable aléatoire de loi à support $\{0,1, \ldots,
  n\}$. On pose, pour $k$ tel que $0 \leq k \leq n$, $p_k =
  \Prob(\Ev{X = k})$.
  \begin{noliste}{a)}
    \setlength{\itemsep}{2mm}
  \item Montrer : $\Sum{k=0}{n} p_k \, \log_2\left( \dfrac{1}{(n+1) \,
        p_k} \right) \ \leq \ \log_2\left( \Sum{k=0}{n}
      \dfrac{p_k}{(n+1) \, p_k}\right) \ = \ 0$.
    
  \item Montrer : $\Sum{k=0}{n} p_k \, \log_2 \big( (n+1) \, p_k \big)
    \ = \ \log_2(n+1) - H(X)$.
    
  \item Montrer : $H(X) \leq \log_2(n+1)$.
    
  \item On suppose que $X$ suit la loi uniforme sur $\{0,1, \ldots,
    N\}$. Calculer $H(X)$.
  \end{noliste}
  
\item Soient $X$ et $Y$ deux variables aléatoires {\it de même loi} à
  support $\{0,1, \ldots, n\}$. On suppose en outre $X$ et $Y$
  indépendantes.
  \begin{noliste}{a)}
    \setlength{\itemsep}{2mm}
  \item Montrer : $\Prob(\Ev{X = Y}) \ = \ \Sum{k=0}{n} \big(
    \Prob(\Ev{X = k})\big)^2$.
    
  \item On pose $v(k) = \Prob(\Ev{X = k})$ pour tout $k$ élément de
    $\{0,1, \ldots, n\}$. Montrer :
    \[
      2^{\E\left( \log_2\big(v(X)\big)\right)} \ \leq \ \E\left(
      2^{\log_2 \big(v(X)\big)}\right) \ = \ \E\big(v(X)\big)
    \]
      
  \item En déduire : $2^{-H(X)} \ \leq \ \Prob(\Ev{X = Y})$.
    
  \item Donner un exemple de loi où l'inégalité précédente est une
    égalité.
  \end{noliste}
\end{noliste}


\subsection*{Troisième partie : Entropie jointe et information
  mutuelle de deux variables discrètes}

\noindent
Soient $X$ et $Y$ deux variables aléatoires de lois à support $\{0,1,
\ldots, n\}$. On appelle {\bf entropie jointe} de $X$ et $Y$ le réel :
\[
  H(X,Y) \ = \ - \Sum{k=0}{n} \, \Sum{j=0}{n} \Prob(\Ev{X = k} \cap
  \Ev{Y = j}) \, \log_2 \big( \Prob(\Ev{X = k} \cap \Ev{Y = j}) \big)
\]
avec la convention : $0 \times \log_2(0) = 0$.
\begin{noliste}{1.}
  \setlength{\itemsep}{4mm}
  \setcounter{enumi}{10}
\item
  \begin{noliste}{a)}
    \setlength{\itemsep}{2mm}
  \item On définit la fonction $g : \{0,1, \ldots , n \}^2 \to \R \up
    \{-\infty\}$ en posant pour $(k,j) \in \{0,1, \ldots, n\}^2$ :
    \[
      g(k,j) \ = \ \log_2 \big( \Prob(\Ev{X = k} \cap \Ev{Y = j})\big)
    \]
    Montrer : $H(X,Y) = -\E\big( g(X,Y) \big)$.
    
  \item Montrer : $H(X,Y) = H(Y,X)$.
    
  \item Pour tout $k$ tel que $0 \leq k \leq n$, on pose :
    \[
      H(Y \, | \, X = k) \ = \ -\Sum{j=0}{n} \Prob_{\Ev{X = k}} (\Ev{Y
        = j}) \, \log_2 \left( \Prob_{\Ev{X = k}}(\Ev{Y = j}) \right)
    \]
    On appelle {\bf entropie conditionnelle} de $Y$ sachant $X$ le
    réel :
    \[
      H(Y \, | \, X) \ = \ \Sum{k=0}{n} \Prob(\Ev{X = k}) \, H(Y \, |
      \, X = k)
    \]
    Montrer : $H(X,Y) \ = \ H(X) + H(Y \, | \, X)$.
    
  \item Montrer que pour tout couple de variables aléatoires $X$ et
    $Y$ de lois à support $\{0,1, \ldots ,n\}$, on a :
    \[
      H(X) - H(X \, | \, Y) \ = \ H(Y) - H(Y \, | \, X)
    \]
  \end{noliste}
  
\item On considère dans cette question deux variables aléatoires de
  lois à support $\{0,1,2,3\}$. On suppose que la loi conjointe de
  $(X,Y)$ est donnée par le tableau suivant :
  \[
    % \renewcommand{\arraystretch}{1.5}
    \begin{tabular}{|>{\centering\small}c||
      *{4}{>{\centering\arraybackslash\small$}m{7mm}<{$}|
      }}
    % \hhline{~|*{3}{-}}
      \hline
      \diagbox[width=1.2cm, height=1.2cm]
      {\scalebox{1}{$j$}}
      {\scalebox{1}{$k$}}
      & \cellcolor{gray!20} 0 & \cellcolor{gray!20} 1  
      & \cellcolor{gray!20} 2 & \cellcolor{gray!20} 3 \\
      \hline
      \hline
      \rule[18pt]{0pt}{0pt}
      \cellcolor{gray!20} 0
      \rule[-15pt]{0pt}{0pt} 
      & \dfrac{1}{8} & \dfrac{1}{16} & \dfrac{1}{32} 
      & \dfrac{1}{32} \\
      \hline
      \rule[18pt]{0pt}{0pt}
      \cellcolor{gray!20} 1
      \rule[-15pt]{0pt}{0pt} 
      & \dfrac{1}{16} & \dfrac{1}{8} & \dfrac{1}{32}
      & \dfrac{1}{32} \\
      \hline
      \rule[18pt]{0pt}{0pt}
      \cellcolor{gray!20} 2
      \rule[-15pt]{0pt}{0pt} 
      & \dfrac{1}{16} & \dfrac{1}{16} & \dfrac{1}{16} 
      & \dfrac{1}{16} \\
      \hline
      \rule[18pt]{0pt}{0pt}
      \cellcolor{gray!20} 3
      \rule[-15pt]{0pt}{0pt} 
      & \dfrac{1}{4} & 0 & 0 & 0 \\
      \hline
    \end{tabular}
  \]
  (on lit dans la $\eme{k}$ colonne et la $\eme{j}$ ligne la valeur de
  $\Prob(\Ev{X = k} \cap \Ev{Y = j})$)
  \begin{noliste}{a)}
    \setlength{\itemsep}{2mm}
  \item Déterminer la loi de $X$ et montrer : $H(X) = \dfrac{7}{4}$.
    
  \item Déterminer la loi de $Y$ et calculer $H(Y)$.
    
  \item Montrer : $H(X \, | \, Y) = \dfrac{11}{8}$.
    
  \item Que vaut $H(Y \, | \, X)$ ?
    
  \item Calculer $H(X,Y)$.
  \end{noliste}
  
\item Soient $X$ et $Y$ deux variables aléatoires de lois à support
  $\{0,1, \ldots, n\}$. On appelle {\bf information mutuelle} de $X$
  et de $Y$ le réel :
  \[
    I(X,Y) \ = \ \Sum{k=0}{n} \, \Sum{j=0}{n} \Prob(\Ev{X = k} \cap
    \Ev{Y = j}) \, \log_2 \left( \dfrac{\Prob(\Ev{X = k} \cap \Ev{Y =
          j})}{\Prob(\Ev{X = k}) \, \Prob(\Ev{Y = j})} \right)
  \]
  \begin{noliste}{a)}
    \setlength{\itemsep}{2mm}
  \item Montrer : $I(X,Y) = I(Y,X)$.
    
  \item Montrer : $I(X,Y) = H(X) - H(X \, | \, Y)$.
    
  \item Montrer : $I(X,X) = H(X)$.
    
  \item Que vaut $I(X,Y)$ si $X$ et $Y$ sont indépendantes ?
  \end{noliste}


  \newpage
  
  
\item Soient $X$ et $Y$ deux variables aléatoires de lois à support
  $\{0,1, \ldots, n\}$. On fixe $0 \leq k \leq n$. Pour $0 \leq j \leq
  n$, on pose : $p_j = \dfrac{\Prob(\Ev{X = k} \cap \Ev{Y = j})}{
    \Prob(\Ev{X = k})}$. On suppose que $p_j >0$ pour tout $0 \leq j
  \leq n$ et on pose : $x_j = \dfrac{\Prob(\Ev{X = k}) \, \Prob(\Ev{Y
      = j})}{\Prob(\Ev{X = k} \cap \Ev{Y = j})}$.
  \begin{noliste}{a)}
    \setlength{\itemsep}{2mm}
  \item Montrer : $\Sum{j=0}{n} p_j = 1$.
    
  \item Soit $Z_k$ une variable aléatoire de loi à support $\{x_0,
    \ldots, x_n \}$ dont la loi est donnée par $\Prob(\Ev{Z_k = x_j})
    =~p_j$ pour $0 \leq j \leq n$. Montrer :
    \[
      \E\big( \log_2(Z_k) \big) \ \leq \ 0
    \]
    
  \item En déduire : $I(X,Y) \geq 0$.
  \end{noliste}
\end{noliste}
\end{document}