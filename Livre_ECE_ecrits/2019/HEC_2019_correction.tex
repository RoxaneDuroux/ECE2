\documentclass[11pt]{article}%
\usepackage{geometry}%
\geometry{a4paper,
  lmargin=2cm,rmargin=2cm,tmargin=2.5cm,bmargin=2.5cm}

% \usepackage{array}
\usepackage{paralist}

\usepackage[svgnames, usenames, dvipsnames]{xcolor}
\xdefinecolor{RecColor}{named}{Aqua}
\xdefinecolor{IncColor}{named}{Aqua}
\xdefinecolor{ImpColor}{named}{PaleGreen}

% \usepackage{frcursive}

\usepackage{adjustbox}

%%%%%%%%%%%
\newcommand{\cRB}[1]{{\color{Red} \pmb{#1}}} %
\newcommand{\cR}[1]{{\color{Red} {#1}}} %
\newcommand{\cBB}[1]{{\color{Blue} \pmb{#1}}}
\newcommand{\cB}[1]{{\color{Blue} {#1}}}
\newcommand{\cGB}[1]{{\color{LimeGreen} \pmb{#1}}}
\newcommand{\cG}[1]{{\color{LimeGreen} {#1}}}

%%%%%%%%%%

\usepackage{diagbox} %
\usepackage{colortbl} %
\usepackage{multirow} %
\usepackage{pgf} %
\usepackage{environ} %
\usepackage{fancybox} %
\usepackage{textcomp} %
\usepackage{marvosym} %

%%%%%%%%%% pour qu'une cellcolor ne recouvre pas le trait du tableau
\usepackage{hhline}%

\usepackage{pgfplots}
\pgfplotsset{compat=1.10}
\usepgfplotslibrary{patchplots}
\usepgfplotslibrary{fillbetween}
\usepackage{tikz,tkz-tab}
\usepackage{ifthen}
\usepackage{calc}
\usetikzlibrary{calc,decorations.pathreplacing,arrows,positioning} 
\usetikzlibrary{fit,shapes,backgrounds}
\usepackage[nomessages]{fp}% http://ctan.org/pkg/fp

\usetikzlibrary{matrix,arrows,decorations.pathmorphing,
  decorations.pathreplacing} 

\newcommand{\myunit}{1 cm}
\tikzset{
    node style sp/.style={draw,circle,minimum size=\myunit},
    node style ge/.style={circle,minimum size=\myunit},
    arrow style mul/.style={draw,sloped,midway,fill=white},
    arrow style plus/.style={midway,sloped,fill=white},
}

%%%%%%%%%%%%%%
%%%%% écrire des inférieur égal ou supérieur égal avec typographie
%%%%% francaise
%%%%%%%%%%%%%

\renewcommand{\geq}{\geqslant}
\renewcommand{\leq}{\leqslant}
\renewcommand{\emptyset}{\varnothing}

\newcommand{\Leq}{\leqslant}
\newcommand{\Geq}{\geqslant}

%%%%%%%%%%%%%%
%%%%% Macro Celia
%%%%%%%%%%%%%

\newcommand{\ff}[2]{\left[#1, #2\right]} %
\newcommand{\fo}[2]{\left[#1, #2\right[} %
\newcommand{\of}[2]{\left]#1, #2\right]} %
\newcommand{\soo}[2]{\left]#1, #2\right[} %
\newcommand{\abs}[1]{\left|#1\right|} %
\newcommand{\Ent}[1]{\left\lfloor #1 \right\rfloor} %


%%%%%%%%%%%%%%
%%%%% tikz : comment dessiner un "oeil"
%%%%%%%%%%%%%

\newcommand{\eye}[4]% size, x, y, rotation
{ \draw[rotate around={#4:(#2,#3)}] (#2,#3) -- ++(-.5*55:#1) (#2,#3)
  -- ++(.5*55:#1); \draw (#2,#3) ++(#4+55:.75*#1) arc
  (#4+55:#4-55:.75*#1);
  % IRIS
  \draw[fill=gray] (#2,#3) ++(#4+55/3:.75*#1) arc
  (#4+180-55:#4+180+55:.28*#1);
  % PUPIL, a filled arc
  \draw[fill=black] (#2,#3) ++(#4+55/3:.75*#1) arc
  (#4+55/3:#4-55/3:.75*#1);%
}


%%%%%%%%%%
%% discontinuité fonction
\newcommand\pointg[2]{%
  \draw[color = red, very thick] (#1+0.15, #2-.04)--(#1, #2-.04)--(#1,
  #2+.04)--(#1+0.15, #2+.04);%
}%

\newcommand\pointd[2]{%
  \draw[color = red, very thick] (#1-0.15, #2+.04)--(#1, #2+.04)--(#1,
  #2-.04)--(#1-0.15, #2-.04);%
}%

%%%%%%%%%%
%%% 1 : position abscisse, 2 : position ordonnée, 3 : taille, 4 : couleur
%%%%%%%%%%
% \newcommand\pointG[4]{%
%   \draw[color = #4, very thick] (#1+#3, #2-(#3/3.75))--(#1,
%   #2-(#3/3.75))--(#1, #2+(#3/3.75))--(#1+#3, #2+(#3/3.75)) %
% }%

\newcommand\pointG[4]{%
  \draw[color = #4, very thick] ({#1+#3/3.75}, {#2-#3})--(#1,
  {#2-#3})--(#1, {#2+#3})--({#1+#3/3.75}, {#2+#3}) %
}%

\newcommand\pointD[4]{%
  \draw[color = #4, very thick] ({#1-#3/3.75}, {#2+#3})--(#1,
  {#2+#3})--(#1, {#2-#3})--({#1-#3/3.75}, {#2-#3}) %
}%

\newcommand\spointG[4]{%
  \draw[color = #4, very thick] ({#1+#3/1.75}, {#2-#3})--(#1,
  {#2-#3})--(#1, {#2+#3})--({#1+#3/1.75}, {#2+#3}) %
}%

\newcommand\spointD[4]{%
  \draw[color = #4, very thick] ({#1-#3/2}, {#2+#3})--(#1,
  {#2+#3})--(#1, {#2-#3})--({#1-#3/2}, {#2-#3}) %
}%

%%%%%%%%%%

\newcommand{\Pb}{\mathtt{P}}

%%%%%%%%%%%%%%%
%%% Pour citer un précédent item
%%%%%%%%%%%%%%%
\newcommand{\itbf}[1]{{\small \bf \textit{#1}}}


%%%%%%%%%%%%%%%
%%% Quelques couleurs
%%%%%%%%%%%%%%%

\xdefinecolor{cancelcolor}{named}{Red}
\xdefinecolor{intI}{named}{ProcessBlue}
\xdefinecolor{intJ}{named}{ForestGreen}

%%%%%%%%%%%%%%%
%%%%%%%%%%%%%%%
% barrer du texte
\usetikzlibrary{shapes.misc}

\makeatletter
% \definecolor{cancelcolor}{rgb}{0.127,0.372,0.987}
\newcommand{\tikz@bcancel}[1]{%
  \begin{tikzpicture}[baseline=(textbox.base), inner sep=0pt]
    \node[strike out, draw] (textbox) {#1}[thick, color=cancelcolor];
    \useasboundingbox (textbox);
  \end{tikzpicture}%
}
\newcommand{\bcancel}[1]{%
  \relax\ifmmode
    \mathchoice{\tikz@bcancel{$\displaystyle#1$}}
               {\tikz@bcancel{$\textstyle#1$}}
               {\tikz@bcancel{$\scriptstyle#1$}}
               {\tikz@bcancel{$\scriptscriptstyle#1$}}
  \else
    \tikz@bcancel{\strut#1}%
  \fi
}
\newcommand{\tikz@xcancel}[1]{%
  \begin{tikzpicture}[baseline=(textbox.base),inner sep=0pt]
  \node[cross out,draw] (textbox) {#1}[thick, color=cancelcolor];
  \useasboundingbox (textbox);
  \end{tikzpicture}%
}
\newcommand{\xcancel}[1]{%
  \relax\ifmmode
    \mathchoice{\tikz@xcancel{$\displaystyle#1$}}
               {\tikz@xcancel{$\textstyle#1$}}
               {\tikz@xcancel{$\scriptstyle#1$}}
               {\tikz@xcancel{$\scriptscriptstyle#1$}}
  \else
    \tikz@xcancel{\strut#1}%
  \fi
}
\makeatother

\newcommand{\xcancelRA}{\xcancel{\rule[-.15cm]{0cm}{.5cm} \Rightarrow
    \rule[-.15cm]{0cm}{.5cm}}}

%%%%%%%%%%%%%%%%%%%%%%%%%%%%%%%%%%%
%%%%%%%%%%%%%%%%%%%%%%%%%%%%%%%%%%%

\newcommand{\vide}{\multicolumn{1}{c}{}}

%%%%%%%%%%%%%%%%%%%%%%%%%%%%%%%%%%%
%%%%%%%%%%%%%%%%%%%%%%%%%%%%%%%%%%%


\usepackage{multicol}
% \usepackage[latin1]{inputenc}
% \usepackage[T1]{fontenc}
\usepackage[utf8]{inputenc}
\usepackage[T1]{fontenc}
\usepackage[normalem]{ulem}
\usepackage[french]{babel}

\usepackage{url}    
\usepackage{hyperref}
\hypersetup{
  backref=true,
  pagebackref=true,
  hyperindex=true,
  colorlinks=true,
  breaklinks=true,
  urlcolor=blue,
  linkcolor=black,
  %%%%%%%%
  % ATTENTION : red changé en black pour le Livre !
  %%%%%%%%
  bookmarks=true,
  bookmarksopen=true
}

%%%%%%%%%%%%%%%%%%%%%%%%%%%%%%%%%%%%%%%%%%%
%% Pour faire des traits diagonaux dans les tableaux
%% Nécessite slashbox.sty
%\usepackage{slashbox}

\usepackage{tipa}
\usepackage{verbatim,listings}
\usepackage{graphicx}
\usepackage{fancyhdr}
\usepackage{mathrsfs}
\usepackage{pifont}
\usepackage{tablists}
\usepackage{dsfont,amsfonts,amssymb,amsmath,amsthm,stmaryrd,upgreek,manfnt}
\usepackage{enumerate}

%\newcolumntype{M}[1]{p{#1}}
\newcolumntype{C}[1]{>{\centering}m{#1}}
\newcolumntype{R}[1]{>{\raggedright}m{#1}}
\newcolumntype{L}[1]{>{\raggedleft}m{#1}}
\newcolumntype{P}[1]{>{\raggedright}p{#1}}
\newcolumntype{B}[1]{>{\raggedright}b{#1}}
\newcolumntype{Q}[1]{>{\raggedright}t{#1}}

\newcommand{\alias}[2]{
\providecommand{#1}{}
\renewcommand{#1}{#2}
}
\alias{\R}{\mathbb{R}}
\alias{\N}{\mathbb{N}}
\alias{\Z}{\mathbb{Z}}
\alias{\Q}{\mathbb{Q}}
\alias{\C}{\mathbb{C}}
\alias{\K}{\mathbb{K}}

%%%%%%%%%%%%
%% rendre +infty et -infty plus petits
%%%%%%%%%%%%
\newcommand{\sinfty}{{\scriptstyle \infty}}

%%%%%%%%%%%%%%%%%%%%%%%%%%%%%%%
%%%%% macros TP Scilab %%%%%%%%
\newcommand{\Scilab}{\textbf{Scilab}} %
\newcommand{\Scinotes}{\textbf{SciNotes}} %
\newcommand{\faire}{\noindent $\blacktriangleright$ } %
\newcommand{\fitem}{\scalebox{.8}{$\blacktriangleright$}} %
\newcommand{\entree}{{\small\texttt{ENTRÉE}}} %
\newcommand{\tab}{{\small\texttt{TAB}}} %
\newcommand{\mt}[1]{\mathtt{#1}} %
% guillemets droits

\newcommand{\ttq}{\textquotesingle} %

\newcommand{\reponse}[1]{\longboxed{
    \begin{array}{C{0.9\textwidth}}
      \nl[#1]
    \end{array}
  }} %

\newcommand{\reponseR}[1]{\longboxed{
    \begin{array}{R{0.9\textwidth}}
      #1
    \end{array}
  }} %

\newcommand{\reponseC}[1]{\longboxed{
    \begin{array}{C{0.9\textwidth}}
      #1
    \end{array}
  }} %

\colorlet{pyfunction}{Blue}
\colorlet{pyCle}{Magenta}
\colorlet{pycomment}{LimeGreen}
\colorlet{pydoc}{Cyan}
% \colorlet{SansCo}{white}
% \colorlet{AvecCo}{black}

\newcommand{\visible}[1]{{\color{ASCo}\colorlet{pydoc}{pyDo}\colorlet{pycomment}{pyCo}\colorlet{pyfunction}{pyF}\colorlet{pyCle}{pyC}\colorlet{function}{sciFun}\colorlet{var}{sciVar}\colorlet{if}{sciIf}\colorlet{comment}{sciComment}#1}} %

%%%% à changer ????
\newcommand{\invisible}[1]{{\color{ASCo}\colorlet{pydoc}{pyDo}\colorlet{pycomment}{pyCo}\colorlet{pyfunction}{pyF}\colorlet{pyCle}{pyC}\colorlet{function}{sciFun}\colorlet{var}{sciVar}\colorlet{if}{sciIf}\colorlet{comment}{sciComment}#1}} %

\newcommand{\invisibleCol}[2]{{\color{#1}#2}} %

\NewEnviron{solution} %
{ %
  \Boxed{
    \begin{array}{>{\color{ASCo}} R{0.9\textwidth}}
      \colorlet{pycomment}{pyCo}
      \colorlet{pydoc}{pyDo}
      \colorlet{pyfunction}{pyF}
      \colorlet{pyCle}{pyC}
      \colorlet{function}{sciFun}
      \colorlet{var}{sciVar}
      \colorlet{if}{sciIf}
      \colorlet{comment}{sciComment}
      \BODY
    \end{array}
  } %
} %

\NewEnviron{solutionC} %
{ %
  \Boxed{
    \begin{array}{>{\color{ASCo}} C{0.9\textwidth}}
      \colorlet{pycomment}{pyCo}
      \colorlet{pydoc}{pyDo}
      \colorlet{pyfunction}{pyF}
      \colorlet{pyCle}{pyC}
      \colorlet{function}{sciFun}
      \colorlet{var}{sciVar}
      \colorlet{if}{sciIf}
      \colorlet{comment}{sciComment}
      \BODY
    \end{array}
  } %
} %

\newcommand{\invite}{--\!\!>} %

%%%%% nouvel environnement tabular pour retour console %%%%
\colorlet{ConsoleColor}{Black!12}
\colorlet{function}{Red}
\colorlet{var}{Maroon}
\colorlet{if}{Magenta}
\colorlet{comment}{LimeGreen}

\newcommand{\tcVar}[1]{\textcolor{var}{\bf \small #1}} %
\newcommand{\tcFun}[1]{\textcolor{function}{#1}} %
\newcommand{\tcIf}[1]{\textcolor{if}{#1}} %
\newcommand{\tcFor}[1]{\textcolor{if}{#1}} %

\newcommand{\moins}{\!\!\!\!\!\!- }
\newcommand{\espn}{\!\!\!\!\!\!}

\usepackage{booktabs,varwidth} \newsavebox\TBox
\newenvironment{console}
{\begin{lrbox}{\TBox}\varwidth{\linewidth}
    \tabular{>{\tt\small}R{0.84\textwidth}}
    \nl[-.4cm]} {\endtabular\endvarwidth\end{lrbox}%
  \fboxsep=1pt\colorbox{ConsoleColor}{\usebox\TBox}}

\newcommand{\lInv}[1]{%
  $\invite$ #1} %

\newcommand{\lAns}[1]{%
  \qquad ans \ = \nl %
  \qquad \qquad #1} %

\newcommand{\lVar}[2]{%
  \qquad #1 \ = \nl %
  \qquad \qquad #2} %

\newcommand{\lDisp}[1]{%
  #1 %
} %

\newcommand{\ligne}[1]{\underline{\small \tt #1}} %

\newcommand{\ligneAns}[2]{%
  $\invite$ #1 \nl %
  \qquad ans \ = \nl %
  \qquad \qquad #2} %

\newcommand{\ligneVar}[3]{%
  $\invite$ #1 \nl %
  \qquad #2 \ = \nl %
  \qquad \qquad #3} %

\newcommand{\ligneErr}[3]{%
  $\invite$ #1 \nl %
  \quad !-{-}error #2 \nl %
  #3} %
%%%%%%%%%%%%%%%%%%%%%% 

\newcommand{\bs}[1]{\boldsymbol{#1}} %
\newcommand{\nll}{\nl[.4cm]} %
\newcommand{\nle}{\nl[.2cm]} %
%% opérateur puissance copiant l'affichage Scilab
%\newcommand{\puis}{\!\!\!~^{\scriptscriptstyle\pmb{\wedge}}}
\newcommand{\puis}{\mbox{$\hspace{-.1cm}~^{\scriptscriptstyle\pmb{\wedge}}
    \hspace{0.05cm}$}} %
\newcommand{\pointpuis}{.\mbox{$\hspace{-.15cm}~^{\scriptscriptstyle\pmb{\wedge}}$}} %
\newcommand{\Sfois}{\mbox{$\mt{\star}$}} %

%%%%% nouvel environnement tabular pour les encadrés Scilab %%%%
\newenvironment{encadre}
{\begin{lrbox}{\TBox}\varwidth{\linewidth}
    \tabular{>{\tt\small}C{0.1\textwidth}>{\small}R{0.7\textwidth}}}
  {\endtabular\endvarwidth\end{lrbox}%
  \fboxsep=1pt\longboxed{\usebox\TBox}}

\newenvironment{encadreL}
{\begin{lrbox}{\TBox}\varwidth{\linewidth}
    \tabular{>{\tt\small}C{0.25\textwidth}>{\small}R{0.6\textwidth}}}
  {\endtabular\endvarwidth\end{lrbox}%
  \fboxsep=1pt\longboxed{\usebox\TBox}}

\newenvironment{encadreF}
{\begin{lrbox}{\TBox}\varwidth{\linewidth}
    \tabular{>{\tt\small}C{0.2\textwidth}>{\small}R{0.70\textwidth}}}
  {\endtabular\endvarwidth\end{lrbox}%
  \fboxsep=1pt\longboxed{\usebox\TBox}}

\newenvironment{encadreLL}[2]
{\begin{lrbox}{\TBox}\varwidth{\linewidth}
    \tabular{>{\tt\small}C{#1\textwidth}>{\small}R{#2\textwidth}}}
  {\endtabular\endvarwidth\end{lrbox}%
  \fboxsep=1pt\longboxed{\usebox\TBox}}

%%%%% nouvel environnement tabular pour les script et fonctions %%%%
\newcommand{\commentaireDL}[1]{\multicolumn{1}{l}{\it
    \textcolor{comment}{$\slash\slash$ #1}}}

\newcommand{\commentaire}[1]{{\textcolor{comment}{$\slash\slash$ #1}}}

\newcounter{cptcol}

\newcommand{\nocount}{\multicolumn{1}{c}{}}

\newcommand{\sciNo}[1]{{\small \underbar #1}}

\NewEnviron{scilab}{ %
  \setcounter{cptcol}{0}
  \begin{center}
    \longboxed{
      \begin{tabular}{>{\stepcounter{cptcol}{\tiny \underbar
              \thecptcol}}c>{\tt}l}
        \BODY
      \end{tabular}
    }
  \end{center}
}

\NewEnviron{scilabNC}{ %
  \begin{center}
    \longboxed{
      \begin{tabular}{>{\tt}l} %
          \BODY
      \end{tabular}
    }
  \end{center}
}

\NewEnviron{scilabC}[1]{ %
  \setcounter{cptcol}{#1}
  \begin{center}
    \longboxed{
      \begin{tabular}{>{\stepcounter{cptcol}{\tiny \underbar
              \thecptcol}}c>{\tt}l}
        \BODY
      \end{tabular}
    }
  \end{center}
}

\newcommand{\scisol}[1]{ %
  \setcounter{cptcol}{0}
  \longboxed{
    \begin{tabular}{>{\stepcounter{cptcol}{\tiny \underbar
            \thecptcol}}c>{\tt}l}
      #1
    \end{tabular}
  }
}

\newcommand{\scisolNC}[1]{ %
  \longboxed{
    \begin{tabular}{>{\tt}l}
      #1
    \end{tabular}
  }
}

\newcommand{\scisolC}[2]{ %
  \setcounter{cptcol}{#1}
  \longboxed{
    \begin{tabular}{>{\stepcounter{cptcol}{\tiny \underbar
            \thecptcol}}c>{\tt}l}
      #2
    \end{tabular}
  }
}

\NewEnviron{syntaxe}{ %
  % \fcolorbox{black}{Yellow!20}{\setlength{\fboxsep}{3mm}
  \shadowbox{
    \setlength{\fboxsep}{3mm}
    \begin{tabular}{>{\tt}l}
      \BODY
    \end{tabular}
  }
}

%%%%% fin macros TP Scilab %%%%%%%%
%%%%%%%%%%%%%%%%%%%%%%%%%%%%%%%%%%%

%%%%%%%%%%%%%%%%%%%%%%%%%%%%%%%%%%%
%%%%% TP Python - listings %%%%%%%%
%%%%%%%%%%%%%%%%%%%%%%%%%%%%%%%%%%%
\newcommand{\Python}{\textbf{Python}} %

\lstset{% general command to set parameter(s)
basicstyle=\ttfamily\small, % print whole listing small
keywordstyle=\color{blue}\bfseries\underbar,
%% underlined bold black keywords
frame=lines,
xleftmargin=10mm,
numbers=left,
numberstyle=\tiny\underbar,
numbersep=10pt,
%identifierstyle=, % nothing happens
commentstyle=\color{green}, % white comments
%%stringstyle=\ttfamily, % typewriter type for strings
showstringspaces=false}

\newcommand{\pysolCpt}[2]{
  \setcounter{cptcol}{#1}
  \longboxed{
    \begin{tabular}{>{\stepcounter{cptcol}{\tiny \underbar
            \thecptcol}}c>{\tt}l}
        #2
      \end{tabular}
    }
} %

\newcommand{\pysol}[1]{
  \setcounter{cptcol}{0}
  \longboxed{
    \begin{tabular}{>{\stepcounter{cptcol}{\tiny \underbar
            \thecptcol}}c>{\tt}l}
        #1
      \end{tabular}
    }
} %

% \usepackage[labelsep=endash]{caption}

% avec un caption
\NewEnviron{pythonCap}[1]{ %
  \renewcommand{\tablename}{Programme}
  \setcounter{cptcol}{0}
  \begin{center}
    \longboxed{
      \begin{tabular}{>{\stepcounter{cptcol}{\tiny \underbar
              \thecptcol}}c>{\tt}l}
        \BODY
      \end{tabular}
    }
    \captionof{table}{#1}
  \end{center}
}

\NewEnviron{python}{ %
  \setcounter{cptcol}{0}
  \begin{center}
    \longboxed{
      \begin{tabular}{>{\stepcounter{cptcol}{\tiny \underbar
              \thecptcol}}c>{\tt}l}
        \BODY
      \end{tabular}
    }
  \end{center}
}

\newcommand{\pyVar}[1]{\textcolor{var}{\bf \small #1}} %
\newcommand{\pyFun}[1]{\textcolor{pyfunction}{#1}} %
\newcommand{\pyCle}[1]{\textcolor{pyCle}{#1}} %
\newcommand{\pyImp}[1]{{\bf #1}} %

%%%%% commentaire python %%%%
\newcommand{\pyComDL}[1]{\multicolumn{1}{l}{\textcolor{pycomment}{\#
      #1}}}

\newcommand{\pyCom}[1]{{\textcolor{pycomment}{\# #1}}}
\newcommand{\pyDoc}[1]{{\textcolor{pydoc}{#1}}}

\newcommand{\pyNo}[1]{{\small \underbar #1}}

%%%%%%%%%%%%%%%%%%%%%%%%%%%%%%%%%%%
%%%%%% Système linéaire paramétré : écrire les opérations au-dessus
%%%%%% d'un symbole équivalent
%%%%%%%%%%%%%%%%%%%%%%%%%%%%%%%%%%%

\usepackage{systeme}

\NewEnviron{arrayEq}{ %
  \stackrel{\scalebox{.6}{$
      \begin{array}{l} 
        \BODY \\[.1cm]
      \end{array}$}
  }{\Longleftrightarrow}
}

\NewEnviron{arrayEg}{ %
  \stackrel{\scalebox{.6}{$
      \begin{array}{l} 
        \BODY \\[.1cm]
      \end{array}$}
  }{=}
}

\NewEnviron{operationEq}{ %
  \scalebox{.6}{$
    \begin{array}{l} 
      \scalebox{1.6}{$\mbox{Opérations :}$} \\[.2cm]
      \BODY \\[.1cm]
    \end{array}$}
}

% \NewEnviron{arraySys}[1]{ %
%   \sysdelim\{.\systeme[#1]{ %
%     \BODY %
%   } %
% }

%%%%%

%%%%%%%%%%
%%%%%%%%%% ESSAI
\newlength\fboxseph
\newlength\fboxsepva
\newlength\fboxsepvb

\setlength\fboxsepva{0.2cm}
\setlength\fboxsepvb{0.2cm}
\setlength\fboxseph{0.2cm}

\makeatletter

\def\longboxed#1{\leavevmode\setbox\@tempboxa\hbox{\color@begingroup%
\kern\fboxseph{\m@th$\displaystyle #1 $}\kern\fboxseph%
\color@endgroup }\my@frameb@x\relax}

\def\my@frameb@x#1{%
  \@tempdima\fboxrule \advance\@tempdima \fboxsepva \advance\@tempdima
  \dp\@tempboxa\hbox {%
    \lower \@tempdima \hbox {%
      \vbox {\hrule\@height\fboxrule \hbox{\vrule\@width\fboxrule #1
          \vbox{%
            \vskip\fboxsepva \box\@tempboxa \vskip\fboxsepvb}#1
          \vrule\@width\fboxrule }%
        \hrule \@height \fboxrule }}}}

\newcommand{\boxedhv}[3]{\setlength\fboxseph{#1cm}
  \setlength\fboxsepva{#2cm}\setlength\fboxsepvb{#2cm}\longboxed{#3}}

\newcommand{\boxedhvv}[4]{\setlength\fboxseph{#1cm}
  \setlength\fboxsepva{#2cm}\setlength\fboxsepvb{#3cm}\longboxed{#4}}

\newcommand{\Boxed}[1]{{\setlength\fboxseph{0.2cm}
  \setlength\fboxsepva{0.2cm}\setlength\fboxsepvb{0.2cm}\longboxed{#1}}}

\newcommand{\mBoxed}[1]{{\setlength\fboxseph{0.2cm}
  \setlength\fboxsepva{0.2cm}\setlength\fboxsepvb{0.2cm}\longboxed{\mbox{#1}}}}

\newcommand{\mboxed}[1]{{\setlength\fboxseph{0.2cm}
  \setlength\fboxsepva{0.2cm}\setlength\fboxsepvb{0.2cm}\boxed{\mbox{#1}}}}

\newsavebox{\fmbox}
\newenvironment{fmpage}[1]
     {\begin{lrbox}{\fmbox}\begin{minipage}{#1}}
     {\end{minipage}\end{lrbox}\fbox{\usebox{\fmbox}}}

%%%%%%%%%%
%%%%%%%%%%

\DeclareMathOperator{\ch}{ch}
\DeclareMathOperator{\sh}{sh}

%%%%%%%%%%
%%%%%%%%%%

\newcommand{\norme}[1]{\Vert #1 \Vert}

%\newcommand*\widefbox[1]{\fbox{\hspace{2em}#1\hspace{2em}}}

\newcommand{\nl}{\tabularnewline}

\newcommand{\hand}{\noindent\ding{43}\ }
\newcommand{\ie}{\textit{i.e. }}
\newcommand{\cf}{\textit{cf }}

\newcommand{\Card}{\operatorname{Card}}

\newcommand{\aire}{\mathcal{A}}

\newcommand{\LL}[1]{\mathscr{L}(#1)} %
\newcommand{\B}{\mathscr{B}} %
\newcommand{\Bc}[1]{B_{#1}} %
\newcommand{\M}[1]{\mathscr{M}_{#1}(\mathbb{R})}

\DeclareMathOperator{\im}{Im}
\DeclareMathOperator{\kr}{Ker}
\DeclareMathOperator{\rg}{rg}
\DeclareMathOperator{\spc}{Sp}
\DeclareMathOperator{\sgn}{sgn}
\DeclareMathOperator{\supp}{Supp}

\newcommand{\Mat}{{\rm{Mat}}}
\newcommand{\Vect}[1]{{\rm{Vect}}\left(#1\right)}

\newenvironment{smatrix}{%
  \begin{adjustbox}{width=.9\width}
    $
    \begin{pmatrix}
    }{%      
    \end{pmatrix}
    $
  \end{adjustbox}
}

\newenvironment{sarray}[1]{%
  \begin{adjustbox}{width=.9\width}
    $
    \begin{array}{#1}
    }{%      
    \end{array}
    $
  \end{adjustbox}
}

\newcommand{\vd}[2]{
  \scalebox{.8}{
    $\left(\!
      \begin{array}{c}
        #1 \\
        #2
      \end{array}
    \!\right)$
    }}

\newcommand{\vt}[3]{
  \scalebox{.8}{
    $\left(\!
      \begin{array}{c}
        #1 \\
        #2 \\
        #3 
      \end{array}
    \!\right)$
    }}

\newcommand{\vq}[4]{
  \scalebox{.8}{
    $\left(\!
      \begin{array}{c}
        #1 \\
        #2 \\
        #3 \\
        #4 
      \end{array}
    \!\right)$
    }}

\newcommand{\vc}[5]{
  \scalebox{.8}{
    $\left(\!
      \begin{array}{c}
        #1 \\
        #2 \\
        #3 \\
        #4 \\
        #5 
      \end{array}
    \!\right)$
    }}

\newcommand{\ee}{\text{e}}

\newcommand{\dd}{\text{d}}

%%% Ensemble de définition
\newcommand{\Df}{\mathscr{D}}
\newcommand{\Cf}{\mathscr{C}}
\newcommand{\Ef}{\mathscr{C}}

\newcommand{\rond}[1]{\,\overset{\scriptscriptstyle \circ}{\!#1}}

\newcommand{\df}[2]{\dfrac{\partial #1}{\partial #2}} %
\newcommand{\dfn}[2]{\partial_{#2}(#1)} %
\newcommand{\ddfn}[2]{\partial^2_{#2}(#1)} %
\newcommand{\ddf}[2]{\dfrac{\partial^2 #1}{\partial #2^2}} %
\newcommand{\ddfr}[3]{\dfrac{\partial^2 #1}{\partial #2 \partial
    #3}} %


\newcommand{\dlim}[1]{{\displaystyle \lim_{#1} \ }}
\newcommand{\dlimPlus}[2]{
  \dlim{
    \scalebox{.6}{
      $
      \begin{array}{l}
        #1 \rightarrow #2\\
        #1 > #2
      \end{array}
      $}}}
\newcommand{\dlimMoins}[2]{
  \dlim{
    \scalebox{.6}{
      $
      \begin{array}{l}
        #1 \rightarrow #2\\
        #1 < #2
      \end{array}
      $}}}

%%%%%%%%%%%%%%
%% petit o, développement limité
%%%%%%%%%%%%%%

\newcommand{\oo}[2]{{\underset {{\overset {#1\rightarrow #2}{}}}{o}}} %
\newcommand{\oox}[1]{{\underset {{\overset {x\rightarrow #1}{}}}{o}}} %
\newcommand{\oon}{{\underset {{\overset {n\rightarrow +\infty}{}}}{o}}} %
\newcommand{\po}[1]{{\underset {{\overset {#1}{}}}{o}}} %
\newcommand{\neqx}[1]{{\ \underset {{\overset {x \to #1}{}}}{\not\sim}\ }} %
\newcommand{\eqx}[1]{{\ \underset {{\overset {x \to #1}{}}}{\sim}\ }} %
\newcommand{\eqn}{{\ \underset {{\overset {n \to +\infty}{}}}{\sim}\ }} %
\newcommand{\eq}[2]{{\ \underset {{\overset {#1 \to #2}{}}}{\sim}\ }} %
\newcommand{\DL}[1]{{\rm{DL}}_1 (#1)} %
\newcommand{\DLL}[1]{{\rm{DL}}_2 (#1)} %

\newcommand{\negl}{<<}

\newcommand{\neglP}[1]{\begin{array}{c}
    \vspace{-.2cm}\\
    << \\
    \vspace{-.7cm}\\
    {\scriptstyle #1}
  \end{array}}

%%%%%%%%%%%%%%
%% borne sup, inf, max, min
%%%%%%%%%%%%%%
\newcommand{\dsup}[1]{\displaystyle \sup_{#1} \ }
\newcommand{\dinf}[1]{\displaystyle \inf_{#1} \ }
\newcommand{\dmax}[1]{\max\limits_{#1} \ }
\newcommand{\dmin}[1]{\min\limits_{#1} \ }

\newcommand{\dcup}[2]{{\textstyle\bigcup\limits_{#1}^{#2}}\hspace{.1cm}}
%\displaystyle \bigcup_{#1}^{#2}}
\newcommand{\dcap}[2]{{\textstyle\bigcap\limits_{#1}^{#2}}\hspace{.1cm}}
% \displaystyle \bigcap_{#1}^{#2}
%%%%%%%%%%%%%%
%% opérateurs logiques
%%%%%%%%%%%%%%
\newcommand{\NON}[1]{\mathop{\small \tt{NON}} (#1)}
\newcommand{\ET}{\mathrel{\mathop{\small \mathtt{ET}}}}
\newcommand{\OU}{\mathrel{\mathop{\small \tt{OU}}}}
\newcommand{\XOR}{\mathrel{\mathop{\small \tt{XOR}}}}

\newcommand{\id}{{\rm{id}}}

\newcommand{\sbullet}{\scriptstyle \bullet}
\newcommand{\stimes}{\scriptstyle \times}

%%%%%%%%%%%%%%%%%%
%% Probabilités
%%%%%%%%%%%%%%%%%%
\newcommand{\Prob}{\mathbb{P}}
\newcommand{\Ev}[1]{\left[ {#1} \right]}
\newcommand{\Evmb}[1]{[ {#1} ]}
\newcommand{\E}{\mathbb{E}}
\newcommand{\V}{\mathbb{V}}
\newcommand{\Cov}{{\rm{Cov}}}
\newcommand{\U}[2]{\mathcal{U}(\llb #1, #2\rrb)}
\newcommand{\Uc}[2]{\mathcal{U}([#1, #2])}
\newcommand{\Ucof}[2]{\mathcal{U}(]#1, #2])}
\newcommand{\Ucoo}[2]{\mathcal{U}(]#1, #2[)}
\newcommand{\Ucfo}[2]{\mathcal{U}([#1, #2[)}
\newcommand{\Bern}[1]{\mathcal{B}\left(#1\right)}
\newcommand{\Bin}[2]{\mathcal{B}\left(#1, #2\right)}
\newcommand{\G}[1]{\mathcal{G}\left(#1\right)}
\newcommand{\Pois}[1]{\mathcal{P}\left(#1\right)}
\newcommand{\HG}[3]{\mathcal{H}\left(#1, #2, #3\right)}
\newcommand{\Exp}[1]{\mathcal{E}\left(#1\right)}
\newcommand{\Norm}[2]{\mathcal{N}\left(#1, #2\right)}

\DeclareMathOperator{\cov}{Cov}

\newcommand{\var}{v.a.r. }
\newcommand{\suit}{\hookrightarrow}

\newcommand{\flecheR}[1]{\rotatebox{90}{\scalebox{#1}{\color{red}
      $\curvearrowleft$}}}


\newcommand{\partie}[1]{\mathcal{P}(#1)}
\newcommand{\Cont}[1]{\mathcal{C}^{#1}}
\newcommand{\Contm}[1]{\mathcal{C}^{#1}_m}

\newcommand{\llb}{\llbracket}
\newcommand{\rrb}{\rrbracket}

%\newcommand{\im}[1]{{\rm{Im}}(#1)}
\newcommand{\imrec}[1]{#1^{- \mathds{1}}}

\newcommand{\unq}{\mathds{1}}

\newcommand{\Hyp}{\mathtt{H}}

\newcommand{\eme}[1]{#1^{\scriptsize \mbox{ème}}}
\newcommand{\er}[1]{#1^{\scriptsize \mbox{er}}}
\newcommand{\ere}[1]{#1^{\scriptsize \mbox{ère}}}
\newcommand{\nd}[1]{#1^{\scriptsize \mbox{nd}}}
\newcommand{\nde}[1]{#1^{\scriptsize \mbox{nde}}}

\newcommand{\truc}{\mathop{\top}}
\newcommand{\fois}{\mathop{\ast}}

\newcommand{\f}[1]{\overrightarrow{#1}}

\newcommand{\checked}{\textcolor{green}{\checkmark}}

\def\restriction#1#2{\mathchoice
              {\setbox1\hbox{${\displaystyle #1}_{\scriptstyle #2}$}
              \restrictionaux{#1}{#2}}
              {\setbox1\hbox{${\textstyle #1}_{\scriptstyle #2}$}
              \restrictionaux{#1}{#2}}
              {\setbox1\hbox{${\scriptstyle #1}_{\scriptscriptstyle #2}$}
              \restrictionaux{#1}{#2}}
              {\setbox1\hbox{${\scriptscriptstyle #1}_{\scriptscriptstyle #2}$}
              \restrictionaux{#1}{#2}}}
\def\restrictionaux#1#2{{#1\,\smash{\vrule height .8\ht1 depth .85\dp1}}_{\,#2}}

\makeatletter
\newcommand*{\da@rightarrow}{\mathchar"0\hexnumber@\symAMSa 4B }
\newcommand*{\da@leftarrow}{\mathchar"0\hexnumber@\symAMSa 4C }
\newcommand*{\xdashrightarrow}[2][]{%
  \mathrel{%
    \mathpalette{\da@xarrow{#1}{#2}{}\da@rightarrow{\,}{}}{}%
  }%
}
\newcommand{\xdashleftarrow}[2][]{%
  \mathrel{%
    \mathpalette{\da@xarrow{#1}{#2}\da@leftarrow{}{}{\,}}{}%
  }%
}
\newcommand*{\da@xarrow}[7]{%
  % #1: below
  % #2: above
  % #3: arrow left
  % #4: arrow right
  % #5: space left 
  % #6: space right
  % #7: math style 
  \sbox0{$\ifx#7\scriptstyle\scriptscriptstyle\else\scriptstyle\fi#5#1#6\m@th$}%
  \sbox2{$\ifx#7\scriptstyle\scriptscriptstyle\else\scriptstyle\fi#5#2#6\m@th$}%
  \sbox4{$#7\dabar@\m@th$}%
  \dimen@=\wd0 %
  \ifdim\wd2 >\dimen@
    \dimen@=\wd2 %   
  \fi
  \count@=2 %
  \def\da@bars{\dabar@\dabar@}%
  \@whiledim\count@\wd4<\dimen@\do{%
    \advance\count@\@ne
    \expandafter\def\expandafter\da@bars\expandafter{%
      \da@bars
      \dabar@ 
    }%
  }%  
  \mathrel{#3}%
  \mathrel{%   
    \mathop{\da@bars}\limits
    \ifx\\#1\\%
    \else
      _{\copy0}%
    \fi
    \ifx\\#2\\%
    \else
      ^{\copy2}%
    \fi
  }%   
  \mathrel{#4}%
}
\makeatother



\newcount\depth

\newcount\depth
\newcount\totaldepth

\makeatletter
\newcommand{\labelsymbol}{%
      \ifnum\depth=0
        %
      \else
        \rlap{\,$\bullet$}%
      \fi
}

\newcommand*\bernoulliTree[1]{%
    \depth=#1\relax            
    \totaldepth=#1\relax
    \draw node(root)[bernoulli/root] {\labelsymbol}[grow=right] \draw@bernoulli@tree;
    \draw \label@bernoulli@tree{root};                                   
}                                                                        

\def\draw@bernoulli@tree{%
    \ifnum\depth>0 
      child[parent anchor=east] foreach \type/\label in {left child/$E$,right child/$S$} {%
          node[bernoulli/\type] {\label\strut\labelsymbol} \draw@bernoulli@tree
      }
      coordinate[bernoulli/increment] (dummy)
   \fi%
}

\def\label@bernoulli@tree#1{%
    \ifnum\depth>0
      ($(#1)!0.5!(#1-1)$) node[fill=white,bernoulli/decrement] {\tiny$p$}
      \label@bernoulli@tree{#1-1}
      ($(#1)!0.5!(#1-2)$) node[fill=white] {\tiny$q$}
      \label@bernoulli@tree{#1-2}
      coordinate[bernoulli/increment] (dummy)
   \fi%
}

\makeatother

\tikzset{bernoulli/.cd,
         root/.style={},
         decrement/.code=\global\advance\depth by-1\relax,
         increment/.code=\global\advance\depth by 1\relax,
         left child/.style={bernoulli/decrement},
         right child/.style={}}


\newcommand{\eps}{\varepsilon}

% \newcommand{\tendi}[1]{\xrightarrow[\footnotesize #1 \rightarrow
%   +\infty]{}}%

\newcommand{\tend}{\rightarrow}%
\newcommand{\tendn}{\underset{n\to +\infty}{\longrightarrow}} %
\newcommand{\ntendn}{\underset{n\to
    +\infty}{\not\hspace{-.15cm}\longrightarrow}} %
% \newcommand{\tendn}{\xrightarrow[\footnotesize n \rightarrow
%   +\infty]{}}%
\newcommand{\Tendx}[1]{\xrightarrow[\footnotesize x \rightarrow
  #1]{}}%
\newcommand{\tendx}[1]{\underset{x\to #1}{\longrightarrow}}%
\newcommand{\ntendx}[1]{\underset{x\to #1}{\not\!\!\longrightarrow}}%
\newcommand{\tendd}[2]{\underset{#1\to #2}{\longrightarrow}}%
% \newcommand{\tendd}[2]{\xrightarrow[\footnotesize #1 \rightarrow
%   #2]{}}%
\newcommand{\tendash}[1]{\xdashrightarrow[\footnotesize #1 \rightarrow
  +\infty]{}}%
\newcommand{\tendashx}[1]{\xdashrightarrow[\footnotesize x \rightarrow
  #1]{}}%
\newcommand{\tendb}[1]{\underset{#1 \to +\infty}{\longrightarrow}}%
\newcommand{\tendL}{\overset{\mathscr L}{\underset{n \to
      +\infty}{\longrightarrow}}}%
\newcommand{\tendP}{\overset{\Prob}{\underset{n \to
      +\infty}{\longrightarrow}}}%
\newcommand{\tenddL}[1]{\overset{\mathscr L}{\underset{#1 \to
      +\infty}{\longrightarrow}}}%

\NewEnviron{attention}{ %
  ~\\[-.2cm]\noindent
  \begin{minipage}{\linewidth}
  \setlength{\fboxsep}{3mm}%
  \ \ \dbend \ \ %
  \fbox{\parbox[t]{.88\linewidth}{\BODY}} %
  \end{minipage}\\
}

\NewEnviron{sattention}[1]{ %
  ~\\[-.2cm]\noindent
  \begin{minipage}{#1\linewidth}
  \setlength{\fboxsep}{3mm}%
  \ \ \dbend \ \ %
  \fbox{\parbox[t]{.88\linewidth}{\BODY}} %
  \end{minipage}\\
}

%%%%% OBSOLETE %%%%%%

% \newcommand{\attention}[1]{
%   \noindent
%   \begin{tabular}{@{}l|p{11.5cm}|}
%     \cline{2-2}
%     \vspace{-.2cm} 
%     & \nl
%     \dbend & #1 \nl
%     \cline{2-2}
%   \end{tabular}
% }

% \newcommand{\attentionv}[2]{
%   \noindent
%   \begin{tabular}{@{}l|p{11.5cm}|}
%     \cline{2-2}
%     \vspace{-.2cm} 
%     & \nl
%     \dbend & #2 \nl[#1 cm]
%     \cline{2-2}
%   \end{tabular}
% }

\newcommand{\explainvb}[2]{
  \noindent
  \begin{tabular}{@{}l|p{11.5cm}|}
    \cline{2-2}
    \vspace{-.2cm} 
    & \nl
    \hand & #2 \nl [#1 cm]
    \cline{2-2}
  \end{tabular}
}


% \noindent
% \begin{tabular}{@{}l|lp{11cm}|}
%   \cline{3-3} 
%   \multicolumn{1}{@{}l@{\dbend}}{} & & #1 \nl
%   \multicolumn{1}{l}{} & & \nl [-.8cm]
%   & & #2 \nl
%   \cline{2-3}
% \end{tabular}

% \newcommand{\attention}[1]{
%   \noindent
%   \begin{tabular}{@{}@{}cp{11cm}}
%     \dbend & #1 \nl
%   \end{tabular}
% }

\newcommand{\PP}[1]{\mathcal{P}(#1)}
\newcommand{\HH}[1]{\mathcal{H}(#1)}
\newcommand{\FF}[1]{\mathcal{F}(#1)}

\newcommand{\DSum}[2]{\displaystyle\sum\limits_{#1}^{#2}\hspace{.1cm}}
\newcommand{\Sum}[2]{{\textstyle\sum\limits_{#1}^{#2}}\hspace{.1cm}}
\newcommand{\Serie}{\textstyle\sum\hspace{.1cm}}
\newcommand{\Prod}[2]{\textstyle\prod\limits_{#1}^{#2}}

\newcommand{\Prim}[3]{\left[\ {#1} \ \right]_{\scriptscriptstyle
   \hspace{-.15cm} ~_{#2}\, }^ {\scriptscriptstyle \hspace{-.15cm} ~^{#3}\, }}

% \newcommand{\Prim}[3]{\left[\ {#1} \ \right]_{\scriptscriptstyle
%     \!\!~_{#2}}^ {\scriptscriptstyle \!\!~^{#3}}}

\newcommand{\dint}[2]{\displaystyle \int_{#1}^{#2}\ }
\newcommand{\Int}[2]{{\rm{Int}}_{\scriptscriptstyle #1, #2}}
\newcommand{\dt}{\ dt}
\newcommand{\dx}{\ dx}

\newcommand{\llpar}[1]{\left(\!\!\!
    \begin{array}{c}
      \rule{0pt}{#1}
    \end{array}
  \!\!\!\right.}

\newcommand{\rrpar}[1]{\left.\!\!\!
    \begin{array}{c}
      \rule{0pt}{#1}
    \end{array}
  \!\!\!\right)}

\newcommand{\llacc}[1]{\left\{\!\!\!
    \begin{array}{c}
      \rule{0pt}{#1}
    \end{array}
  \!\!\!\right.}

\newcommand{\rracc}[1]{\left.\!\!\!
    \begin{array}{c}
      \rule{0pt}{#1}
    \end{array}
  \!\!\!\right\}}

\newcommand{\ttacc}[1]{\mbox{\rotatebox{-90}{\hspace{-.7cm}$\llacc{#1}$}}}
\newcommand{\bbacc}[1]{\mbox{\rotatebox{90}{\hspace{-.5cm}$\llacc{#1}$}}}

\newcommand{\comp}[1]{\overline{#1}}%

\newcommand{\dcomp}[2]{\stackrel{\mbox{\ \ \----}{\scriptscriptstyle
      #2}}{#1}}%

% \newcommand{\Comp}[2]{\stackrel{\mbox{\ \
%       \-------}{\scriptscriptstyle #2}}{#1}}

% \newcommand{\dcomp}[2]{\stackrel{\mbox{\ \
%       \-------}{\scriptscriptstyle #2}}{#1}}

\newcommand{\A}{\mathscr{A}}

\newcommand{\conc}[1]{
  \begin{center}
    \fbox{
      \begin{tabular}{c}
        #1
      \end{tabular}
    }
  \end{center}
}

\newcommand{\concC}[1]{
  \begin{center}
    \fbox{
    \begin{tabular}{C{10cm}}
      \quad #1 \quad
    \end{tabular}
    }
  \end{center}
}

\newcommand{\concL}[2]{
  \begin{center}
    \fbox{
    \begin{tabular}{C{#2cm}}
      \quad #1 \quad
    \end{tabular}
    }
  \end{center}
}


% \newcommand{\lims}[2]{\prod\limits_{#1}^{#2}}

\newtheorem{theorem}{Théorème}[]
\newtheorem{lemma}{Lemme}[]
\newtheorem{proposition}{Proposition}[]
\newtheorem{corollary}{Corollaire}[]

% \newenvironment{proof}[1][Démonstration]{\begin{trivlist}
% \item[\hskip \labelsep {\bfseries #1}]}{\end{trivlist}}
\newenvironment{definition}[1][Définition]{\begin{trivlist}
\item[\hskip \labelsep {\bfseries #1}]}{\end{trivlist}}
\newenvironment{example}[1][Exemple]{\begin{trivlist}
\item[\hskip \labelsep {\bfseries #1}]}{\end{trivlist}}
\newenvironment{examples}[1][Exemples]{\begin{trivlist}
\item[\hskip \labelsep {\bfseries #1}]}{\end{trivlist}}
\newenvironment{notation}[1][Notation]{\begin{trivlist}
\item[\hskip \labelsep {\bfseries #1}]}{\end{trivlist}}
\newenvironment{propriete}[1][Propriété]{\begin{trivlist}
\item[\hskip \labelsep {\bfseries #1}]}{\end{trivlist}}
\newenvironment{proprietes}[1][Propriétés]{\begin{trivlist}
\item[\hskip \labelsep {\bfseries #1}]}{\end{trivlist}}
% \newenvironment{remark}[1][Remarque]{\begin{trivlist}
% \item[\hskip \labelsep {\bfseries #1}]}{\end{trivlist}}
\newenvironment{application}[1][Application]{\begin{trivlist}
\item[\hskip \labelsep {\bfseries #1}]}{\end{trivlist}}

% Environnement pour les réponses des DS
\newenvironment{answer}{\par\emph{Réponse :}\par{}}
{\vspace{-.6cm}\hspace{\stretch{1}}\rule{1ex}{1ex}\vspace{.3cm}}

\newenvironment{answerTD}{\vspace{.2cm}\par\emph{Réponse :}\par{}}
{\hspace{\stretch{1}}\rule{1ex}{1ex}\vspace{.3cm}}

\newenvironment{answerCours}{\noindent\emph{Réponse :}}
{\rule{1ex}{1ex}}%\vspace{.3cm}}


% footnote in footer
\newcommand{\fancyfootnotetext}[2]{%
  \fancypagestyle{dingens}{%
    \fancyfoot[LO,RE]{\parbox{0.95\textwidth}{\footnotemark[#1]\footnotesize
        #2}}%
  }%
  \thispagestyle{dingens}%
}

%%% définit le style (arabic : 1,2,3...) et place des parenthèses
%%% autour de la numérotation
\renewcommand*{\thefootnote}{(\arabic{footnote})}
% http://www.tuteurs.ens.fr/logiciels/latex/footnote.html

%%%%%%%% tikz axis
% \pgfplotsset{every axis/.append style={
%                     axis x line=middle,    % put the x axis in the middle
%                     axis y line=middle,    % put the y axis in the middle
%                     axis line style={<->,color=blue}, % arrows on the axis
%                     xlabel={$x$},          % default put x on x-axis
%                     ylabel={$y$},          % default put y on y-axis
%             }}

%%%% s'utilise comme suit

% \begin{axis}[
%   xmin=-8,xmax=4,
%   ymin=-8,ymax=4,
%   grid=both,
%   ]
%   \addplot [domain=-3:3,samples=50]({x^3-3*x},{3*x^2-9}); 
% \end{axis}

%%%%%%%%



%%%%%%%%%%%% Pour avoir des numéros de section qui correspondent à
%%%%%%%%%%%% ceux du tableau
\renewcommand{\thesection}{\Roman{section}.\hspace{-.3cm}}
\renewcommand{\thesubsection}{\Roman{section}.\arabic{subsection}.\hspace{-.2cm}}
\renewcommand{\thesubsubsection}{\Roman{section}.\arabic{subsection}.\alph{subsubsection})\hspace{-.2cm}}
%%%%%%%%%%%% 

%%% Changer le nom des figures : Fig. au lieu de Figure
\usepackage[font=small,labelfont=bf,labelsep=space]{caption}
\captionsetup{%
  figurename=Fig.,
  tablename=Tab.
}
% \renewcommand{\thesection}{\Roman{section}.\hspace{-.2cm}}
% \renewcommand{\thesubsection}{\Roman{section}
%   .\hspace{.2cm}\arabic{subsection}\ .\hspace{-.3cm}}
% \renewcommand{\thesubsubsection}{\alph{subsection})}

\newenvironment{tabliste}[1]
{\begin{tabenum}[\bfseries\small\itshape #1]}{\end{tabenum}} 

%%%% ESSAI contre le too deeply nested %%%%
%%%% ATTENTION au package enumitem qui se comporte mal avec les
%%%% noliste, à redéfinir !
% \usepackage{enumitem}

% \setlistdepth{9}

% \newlist{myEnumerate}{enumerate}{9}
% \setlist[myEnumerate,1]{label=(\arabic*)}
% \setlist[myEnumerate,2]{label=(\Roman*)}
% \setlist[myEnumerate,3]{label=(\Alph*)}
% \setlist[myEnumerate,4]{label=(\roman*)}
% \setlist[myEnumerate,5]{label=(\alph*)}
% \setlist[myEnumerate,6]{label=(\arabic*)}
% \setlist[myEnumerate,7]{label=(\Roman*)}
% \setlist[myEnumerate,8]{label=(\Alph*)}
% \setlist[myEnumerate,9]{label=(\roman*)}

%%%%%

\newenvironment{noliste}[1] %
{\begin{enumerate}[\bfseries\small\itshape #1]} %
  {\end{enumerate}}

\newenvironment{nonoliste}[1] %
{\begin{enumerate}[\hspace{-12pt}\bfseries\small\itshape #1]} %
  {\end{enumerate}}

\newenvironment{arrayliste}[1]{ 
  % List with minimal white space to fit in small areas, e.g. table
  % cell
  \begin{minipage}[t]{\linewidth} %
    \begin{enumerate}[\bfseries\small\itshape #1] %
      {\leftmargin=0.5em \rightmargin=0em
        \topsep=0em \parskip=0em \parsep=0em
        \listparindent=0em \partopsep=0em \itemsep=0pt
        \itemindent=0em \labelwidth=\leftmargin\labelsep+0.25em}
    }{
    \end{enumerate}\end{minipage}
}

\newenvironment{nolistes}[2]
{\begin{enumerate}[\bfseries\small\itshape
    #1]\setlength{\itemsep}{#2 mm}}{\end{enumerate}}

\newenvironment{liste}[1]
{\begin{enumerate}[\hspace{12pt}\bfseries\small\itshape
    #1]}{\end{enumerate}}   


%%%%%%%% Pour les programmes de colle %%%%%%%

\newcommand{\cours}{{\small \tt (COURS)}} %
\newcommand{\poly}{{\small \tt (POLY)}} %
\newcommand{\exo}{{\small \tt (EXO)}} %
\newcommand{\culture}{{\small \tt (CULTURE)}} %
\newcommand{\methodo}{{\small \tt (MÉTHODO)}} %
\newcommand{\methodob}{\Boxed{\mbox{\tt MÉTHODO}}} %

%%%%%%%% Pour les TD %%%%%%%
\newtheoremstyle{exostyle} {\topsep} % espace avant
{.6cm} % espace apres
{} % Police utilisee par le style de thm
{} % Indentation (vide = aucune, \parindent = indentation paragraphe)
{\bfseries} % Police du titre de thm
{} % Signe de ponctuation apres le titre du thm
{ } % Espace apres le titre du thm (\newline = linebreak)
{\thmname{#1}\thmnumber{ #2}\thmnote{.
    \normalfont{\textit{#3}}}} % composants du titre du thm : \thmname
                               % = nom du thm, \thmnumber = numéro du
                               % thm, \thmnote = sous-titre du thm
 
\theoremstyle{exostyle}
\newtheorem{exercice}{Exercice}
\newtheorem*{exoCours}{Exercice}

%%%%%%%% Pour des théorèmes Sans Espaces APRÈS %%%%%%%
\newtheoremstyle{exostyleSE} {\topsep} % espace avant
{} % espace apres
{} % Police utilisee par le style de thm
{} % Indentation (vide = aucune, \parindent = indentation paragraphe)
{\bfseries} % Police du titre de thm
{} % Signe de ponctuation apres le titre du thm
{ } % Espace apres le titre du thm (\newline = linebreak)
{\thmname{#1}\thmnumber{ #2}\thmnote{.
    \normalfont{\textit{#3}}}} % composants du titre du thm : \thmname
                               % = nom du thm, \thmnumber = numéro du
                               % thm, \thmnote = sous-titre du thm
 
\theoremstyle{exostyleSE}
\newtheorem{exerciceSE}{Exercice}
\newtheorem*{exoCoursSE}{Exercice}

% \newcommand{\lims}[2]{\prod\limits_{#1}^{#2}}

\newtheorem{theoremSE}{Théorème}[]
\newtheorem{lemmaSE}{Lemme}[]
\newtheorem{propositionSE}{Proposition}[]
\newtheorem{corollarySE}{Corollaire}[]

% \newenvironment{proofSE}[1][Démonstration]{\begin{trivlist}
% \item[\hskip \labelsep {\bfseries #1}]}{\end{trivlist}}
\newenvironment{definitionSE}[1][Définition]{\begin{trivlist}
  \item[\hskip \labelsep {\bfseries #1}]}{\end{trivlist}}
\newenvironment{exampleSE}[1][Exemple]{\begin{trivlist} 
  \item[\hskip \labelsep {\bfseries #1}]}{\end{trivlist}}
\newenvironment{examplesSE}[1][Exemples]{\begin{trivlist}
\item[\hskip \labelsep {\bfseries #1}]}{\end{trivlist}}
\newenvironment{notationSE}[1][Notation]{\begin{trivlist}
\item[\hskip \labelsep {\bfseries #1}]}{\end{trivlist}}
\newenvironment{proprieteSE}[1][Propriété]{\begin{trivlist}
\item[\hskip \labelsep {\bfseries #1}]}{\end{trivlist}}
\newenvironment{proprietesSE}[1][Propriétés]{\begin{trivlist}
\item[\hskip \labelsep {\bfseries #1}]}{\end{trivlist}}
\newenvironment{remarkSE}[1][Remarque]{\begin{trivlist}
\item[\hskip \labelsep {\bfseries #1}]}{\end{trivlist}}
\newenvironment{applicationSE}[1][Application]{\begin{trivlist}
\item[\hskip \labelsep {\bfseries #1}]}{\end{trivlist}}

%%%%%%%%%%% Obtenir les étoiles sans charger le package MnSymbol
%%%%%%%%%%%
\DeclareFontFamily{U} {MnSymbolC}{}
\DeclareFontShape{U}{MnSymbolC}{m}{n}{
  <-6> MnSymbolC5
  <6-7> MnSymbolC6
  <7-8> MnSymbolC7
  <8-9> MnSymbolC8
  <9-10> MnSymbolC9
  <10-12> MnSymbolC10
  <12-> MnSymbolC12}{}
\DeclareFontShape{U}{MnSymbolC}{b}{n}{
  <-6> MnSymbolC-Bold5
  <6-7> MnSymbolC-Bold6
  <7-8> MnSymbolC-Bold7
  <8-9> MnSymbolC-Bold8
  <9-10> MnSymbolC-Bold9
  <10-12> MnSymbolC-Bold10
  <12-> MnSymbolC-Bold12}{}

\DeclareSymbolFont{MnSyC} {U} {MnSymbolC}{m}{n}

\DeclareMathSymbol{\filledlargestar}{\mathrel}{MnSyC}{205}
\DeclareMathSymbol{\largestar}{\mathrel}{MnSyC}{131}

\newcommand{\facile}{\rm{(}$\scriptstyle\largestar$\rm{)}} %
\newcommand{\moyen}{\rm{(}$\scriptstyle\filledlargestar$\rm{)}} %
\newcommand{\dur}{\rm{(}$\scriptstyle\filledlargestar\filledlargestar$\rm{)}} %
\newcommand{\costaud}{\rm{(}$\scriptstyle\filledlargestar\filledlargestar\filledlargestar$\rm{)}}

%%%%%%%%%%%%%%%%%%%%%%%%%

%%%%%%%%%%%%%%%%%%%%%%%%%
%%%%%%%% Fin de la partie TD

%%%%%%%%%%%%%%%%
%%%%%%%%%%%%%%%%
\makeatletter %
\newenvironment{myitemize}{%
  \setlength{\topsep}{0pt} %
  \setlength{\partopsep}{0pt} %
  \renewcommand*{\@listi}{\leftmargin\leftmargini \parsep\z@
    \topsep\z@ \itemsep\z@} \let\@listI\@listi %
  \itemize %
}{\enditemize} %
\makeatother
%%%%%%%%%%%%%%%%
%%%%%%%%%%%%%%%%

%% Commentaires dans la correction du livre

\newcommand{\Com}[1]{
% Define box and box title style
\tikzstyle{mybox} = [draw=black!50,
very thick,
    rectangle, rounded corners, inner sep=10pt, inner ysep=8pt]
\tikzstyle{fancytitle} =[rounded corners, fill=black!80, text=white]
\tikzstyle{fancylogo} =[ text=white]
\begin{center}

\begin{tikzpicture}
\node [mybox] (box){%

    \begin{minipage}{0.90\linewidth}
\vspace{6pt}  #1
    \end{minipage}
};
\node[fancytitle, right=10pt] at (box.north west) 
{\bfseries\normalsize{Commentaire}};

\end{tikzpicture}%

\end{center}
%
}

\NewEnviron{remark}{%
  % Define box and box title style
  \tikzstyle{mybox} = [draw=black!50, very thick, rectangle, rounded
  corners, inner sep=10pt, inner ysep=8pt] %
  \tikzstyle{fancytitle} = [rounded corners , fill=black!80,
  text=white] %
  \tikzstyle{fancylogo} =[ text=white]
  \begin{center}
    \begin{tikzpicture}
      \node [mybox] (box){%
        \begin{minipage}{0.90\linewidth}
          \vspace{6pt} \BODY
        \end{minipage}
      }; %
      \node[fancytitle, right=10pt] at (box.north west) %
      {\bfseries\normalsize{Commentaire}}; %
    \end{tikzpicture}%
  \end{center}
}

\NewEnviron{remarkST}{%
  % Define box and box title style
  \tikzstyle{mybox} = [draw=black!50, very thick, rectangle, rounded
  corners, inner sep=10pt, inner ysep=8pt] %
  \tikzstyle{fancytitle} = [rounded corners , fill=black!80,
  text=white] %
  \tikzstyle{fancylogo} =[ text=white]
  \begin{center}
    \begin{tikzpicture}
      \node [mybox] (box){%
        \begin{minipage}{0.90\linewidth}
          \vspace{6pt} \BODY
        \end{minipage}
      }; %
      % \node[fancytitle, right=10pt] at (box.north west) %
%       {\bfseries\normalsize{Commentaire}}; %
    \end{tikzpicture}%
  \end{center}
}

\NewEnviron{remarkL}[1]{%
  % Define box and box title style
  \tikzstyle{mybox} = [draw=black!50, very thick, rectangle, rounded
  corners, inner sep=10pt, inner ysep=8pt] %
  \tikzstyle{fancytitle} =[rounded corners, fill=black!80,
  text=white] %
  \tikzstyle{fancylogo} =[ text=white]
  \begin{center}
    \begin{tikzpicture}
      \node [mybox] (box){%
        \begin{minipage}{#1\linewidth}
          \vspace{6pt} \BODY
        \end{minipage}
      }; %
      \node[fancytitle, right=10pt] at (box.north west) %
      {\bfseries\normalsize{Commentaire}}; %
    \end{tikzpicture}%
  \end{center}
}

\NewEnviron{remarkSTL}[1]{%
  % Define box and box title style
  \tikzstyle{mybox} = [draw=black!50, very thick, rectangle, rounded
  corners, inner sep=10pt, inner ysep=8pt] %
  \tikzstyle{fancytitle} =[rounded corners, fill=black!80,
  text=white] %
  \tikzstyle{fancylogo} =[ text=white]
  \begin{center}
    \begin{tikzpicture}
      \node [mybox] (box){%
        \begin{minipage}{#1\linewidth}
          \vspace{6pt} \BODY
        \end{minipage}
      }; %
%       \node[fancytitle, right=10pt] at (box.north west) %
%       {\bfseries\normalsize{Commentaire}}; %
    \end{tikzpicture}%
  \end{center}
}

\NewEnviron{titre} %
{ %
  ~\\[-1.8cm]
  \begin{center}
    \bf \LARGE \BODY
  \end{center}
  ~\\[-.6cm]
  \hrule %
  \vspace*{.2cm}
} %

\NewEnviron{titreL}[2] %
{ %
  ~\\[-#1cm]
  \begin{center}
    \bf \LARGE \BODY
  \end{center}
  ~\\[-#2cm]
  \hrule %
  \vspace*{.2cm}
} %



%%%%%%%%%%% Redefinition \chapter



\usepackage[explicit]{titlesec}
\usepackage{color}
\titleformat{\chapter}
{\gdef\chapterlabel{}
\selectfont\huge\bf}
%\normalfont\sffamily\Huge\bfseries\scshape}
{\gdef\chapterlabel{\thechapter)\ }}{0pt}
{\begin{tikzpicture}[remember picture,overlay]
\node[yshift=-3cm] at (current page.north west)
{\begin{tikzpicture}[remember picture, overlay]
\draw (.1\paperwidth,0) -- (.9\paperwidth,0);
\draw (.1\paperwidth,2) -- (.9\paperwidth,2);
%(\paperwidth,3cm);
\node[anchor=center,xshift=.5\paperwidth,yshift=1cm, rectangle,
rounded corners=20pt,inner sep=11pt]
{\color{black}\chapterlabel#1};
\end{tikzpicture}
};
\end{tikzpicture}
}
\titlespacing*{\chapter}{0pt}{50pt}{-75pt}




%%%%%%%%%%%%%% Affichage chapter dans Table des matieres

\makeatletter
\renewcommand*\l@chapter[2]{%
  \ifnum \c@tocdepth >\m@ne
    \addpenalty{-\@highpenalty}%
    \vskip 1.0em \@plus\p@
    \setlength\@tempdima{1.5em}%
    \begingroup
      \parindent \z@ \rightskip \@pnumwidth
      \parfillskip -\@pnumwidth
      \leavevmode %\bfseries
      \advance\leftskip\@tempdima
      \hskip -\leftskip
      #1\nobreak\ 
       \leaders\hbox{$\m@th
        \mkern \@dotsep mu\hbox{.}\mkern \@dotsep
        mu$}\hfil\nobreak\hb@xt@\@pnumwidth{\hss #2}\par
      \penalty\@highpenalty
    \endgroup
  \fi}
\makeatother




%%%%%%%%%%%%%%%%% Redefinition part



\usepackage{array}
\usepackage{paralist}

\usepackage[svgnames, usenames, dvipsnames]{xcolor}
\xdefinecolor{RecColor}{named}{Aqua}
\xdefinecolor{IncColor}{named}{Aqua}
\xdefinecolor{ImpColor}{named}{PaleGreen}

% \usepackage{frcursive}

\usepackage{adjustbox}

%%%%%%%%%%%
\newcommand{\cRB}[1]{{\color{Red} \pmb{#1}}} %
\newcommand{\cR}[1]{{\color{Red} {#1}}} %
\newcommand{\cBB}[1]{{\color{Blue} \pmb{#1}}}
\newcommand{\cB}[1]{{\color{Blue} {#1}}}
\newcommand{\cGB}[1]{{\color{LimeGreen} \pmb{#1}}}
\newcommand{\cG}[1]{{\color{LimeGreen} {#1}}}

%%%%%%%%%%

\usepackage{diagbox} %
\usepackage{colortbl} %
\usepackage{multirow} %
\usepackage{pgf} %
\usepackage{environ} %
\usepackage{fancybox} %
\usepackage{textcomp} %
\usepackage{marvosym} %

%%%%%%%%%% pour pouvoir faire des dashedline dans les tableaux
\usepackage{arydshln}

%%%%%%%%%% pour qu'une cellcolor ne recouvre pas le trait du tableau
\usepackage{hhline}%

\usepackage{pgfplots}
\pgfplotsset{compat=1.10}
\usepgfplotslibrary{patchplots}
\usepgfplotslibrary{fillbetween}
\usepackage{tikz,tkz-tab}
\usepackage{ifthen}
\usepackage{calc}
\usetikzlibrary{calc,decorations.pathreplacing,arrows,positioning} 
\usetikzlibrary{fit,shapes,backgrounds}
% \usepackage[nomessages]{fp}% http://ctan.org/pkg/fp

\usetikzlibrary{matrix,arrows,decorations.pathmorphing,
  decorations.pathreplacing} 

\newcommand{\myunit}{1 cm}
\tikzset{
    node style sp/.style={draw,circle,minimum size=\myunit},
    node style ge/.style={circle,minimum size=\myunit},
    arrow style mul/.style={draw,sloped,midway,fill=white},
    arrow style plus/.style={midway,sloped,fill=white},
}

%%%%%%%%%%%%%%
%%%%% écrire des inférieur égal ou supérieur égal avec typographie
%%%%% francaise
%%%%%%%%%%%%%

\renewcommand{\geq}{\geqslant}
\renewcommand{\leq}{\leqslant}
\renewcommand{\emptyset}{\varnothing}

\newcommand{\Leq}{\leqslant}
\newcommand{\Geq}{\geqslant}

%%%%%%%%%%%%%%
%%%%% Macro Celia
%%%%%%%%%%%%%

\newcommand{\ff}[2]{\left[#1, #2\right]} %
\newcommand{\fo}[2]{\left[#1, #2\right[} %
\newcommand{\of}[2]{\left]#1, #2\right]} %
\newcommand{\soo}[2]{\left]#1, #2\right[} %
\newcommand{\abs}[1]{\left|#1\right|} %
\newcommand{\Ent}[1]{\left\lfloor #1 \right\rfloor} %


%%%%%%%%%%%%%%
%%%%% tikz : comment dessiner un "oeil"
%%%%%%%%%%%%%

\newcommand{\eye}[4]% size, x, y, rotation
{ \draw[rotate around={#4:(#2,#3)}] (#2,#3) -- ++(-.5*55:#1) (#2,#3)
  -- ++(.5*55:#1); \draw (#2,#3) ++(#4+55:.75*#1) arc
  (#4+55:#4-55:.75*#1);
  % IRIS
  \draw[fill=gray] (#2,#3) ++(#4+55/3:.75*#1) arc
  (#4+180-55:#4+180+55:.28*#1);
  % PUPIL, a filled arc
  \draw[fill=black] (#2,#3) ++(#4+55/3:.75*#1) arc
  (#4+55/3:#4-55/3:.75*#1);%
}


%%%%%%%%%%
%% discontinuité fonction
\newcommand\pointg[2]{%
  \draw[color = red, very thick] (#1+0.15, #2-.04)--(#1, #2-.04)--(#1,
  #2+.04)--(#1+0.15, #2+.04);%
}%

\newcommand\pointd[2]{%
  \draw[color = red, very thick] (#1-0.15, #2+.04)--(#1, #2+.04)--(#1,
  #2-.04)--(#1-0.15, #2-.04);%
}%

%%%%%%%%%%
%%% 1 : position abscisse, 2 : position ordonnée, 3 : taille, 4 : couleur
%%%%%%%%%%
% \newcommand\pointG[4]{%
%   \draw[color = #4, very thick] (#1+#3, #2-(#3/3.75))--(#1,
%   #2-(#3/3.75))--(#1, #2+(#3/3.75))--(#1+#3, #2+(#3/3.75)) %
% }%

\newcommand\pointG[4]{%
  \draw[color = #4, very thick] ({#1+#3/3.75}, {#2-#3})--(#1,
  {#2-#3})--(#1, {#2+#3})--({#1+#3/3.75}, {#2+#3}) %
}%

\newcommand\pointD[4]{%
  \draw[color = #4, very thick] ({#1-#3/3.75}, {#2+#3})--(#1,
  {#2+#3})--(#1, {#2-#3})--({#1-#3/3.75}, {#2-#3}) %
}%

\newcommand\spointG[4]{%
  \draw[color = #4, very thick] ({#1+#3/1.75}, {#2-#3})--(#1,
  {#2-#3})--(#1, {#2+#3})--({#1+#3/1.75}, {#2+#3}) %
}%

\newcommand\spointD[4]{%
  \draw[color = #4, very thick] ({#1-#3/2}, {#2+#3})--(#1,
  {#2+#3})--(#1, {#2-#3})--({#1-#3/2}, {#2-#3}) %
}%

%%%%%%%%%%

\newcommand{\Pb}{\mathtt{P}}

%%%%%%%%%%%%%%%
%%% Pour citer un précédent item
%%%%%%%%%%%%%%%
\newcommand{\itbf}[1]{{\small \bf \textit{#1}}}


%%%%%%%%%%%%%%%
%%% Quelques couleurs
%%%%%%%%%%%%%%%

\xdefinecolor{cancelcolor}{named}{Red}
\xdefinecolor{intI}{named}{ProcessBlue}
\xdefinecolor{intJ}{named}{ForestGreen}

%%%%%%%%%%%%%%%
%%%%%%%%%%%%%%%
% barrer du texte
\usetikzlibrary{shapes.misc}

\makeatletter
% \definecolor{cancelcolor}{rgb}{0.127,0.372,0.987}
\newcommand{\tikz@bcancel}[1]{%
  \begin{tikzpicture}[baseline=(textbox.base),inner sep=0pt]
  \node[strike out,draw] (textbox) {#1}[thick, color=cancelcolor];
  \useasboundingbox (textbox);
  \end{tikzpicture}%
}
\newcommand{\bcancel}[1]{%
  \relax\ifmmode
    \mathchoice{\tikz@bcancel{$\displaystyle#1$}}
               {\tikz@bcancel{$\textstyle#1$}}
               {\tikz@bcancel{$\scriptstyle#1$}}
               {\tikz@bcancel{$\scriptscriptstyle#1$}}
  \else
    \tikz@bcancel{\strut#1}%
  \fi
}
\newcommand{\tikz@xcancel}[1]{%
  \begin{tikzpicture}[baseline=(textbox.base),inner sep=0pt]
  \node[cross out,draw] (textbox) {#1}[thick, color=cancelcolor];
  \useasboundingbox (textbox);
  \end{tikzpicture}%
}
\newcommand{\xcancel}[1]{%
  \relax\ifmmode
    \mathchoice{\tikz@xcancel{$\displaystyle#1$}}
               {\tikz@xcancel{$\textstyle#1$}}
               {\tikz@xcancel{$\scriptstyle#1$}}
               {\tikz@xcancel{$\scriptscriptstyle#1$}}
  \else
    \tikz@xcancel{\strut#1}%
  \fi
}
\makeatother

\newcommand{\xcancelRA}{\xcancel{\rule[-.15cm]{0cm}{.5cm} \Rightarrow
    \rule[-.15cm]{0cm}{.5cm}}}

%%%%%%%%%%%%%%%%%%%%%%%%%%%%%%%%%%%
%%%%%%%%%%%%%%%%%%%%%%%%%%%%%%%%%%%

\newcommand{\vide}{\multicolumn{1}{c}{}}

%%%%%%%%%%%%%%%%%%%%%%%%%%%%%%%%%%%
%%%%%%%%%%%%%%%%%%%%%%%%%%%%%%%%%%%


\usepackage{multicol}
% \usepackage[latin1]{inputenc}
% \usepackage[T1]{fontenc}
\usepackage[utf8]{inputenc}
\usepackage[T1]{fontenc}
\usepackage[normalem]{ulem}
\usepackage[french]{babel}

\usepackage{url}    
\usepackage{hyperref}
\hypersetup{
  backref=true,
  pagebackref=true,
  hyperindex=true,
  colorlinks=true,
  breaklinks=true,
  urlcolor=blue,
  linkcolor=red,
  bookmarks=true,
  bookmarksopen=true
}

%%%%%%%%%%%%%%%%%%%%%%%%%%%%%%%%%%%%%%%%%%%
%% Pour faire des traits diagonaux dans les tableaux
%% Nécessite slashbox.sty
%\usepackage{slashbox}

\usepackage{tipa}
\usepackage{verbatim,listings}
\usepackage{graphicx}
\usepackage{fancyhdr}
\usepackage{mathrsfs}
\usepackage{pifont}
\usepackage{tablists}
\usepackage{dsfont,amsfonts,amssymb,amsmath,amsthm,stmaryrd,upgreek,manfnt}
\usepackage{enumerate}

%\newcolumntype{M}[1]{p{#1}}
\newcolumntype{C}[1]{>{\centering}m{#1}}
\newcolumntype{R}[1]{>{\raggedright}m{#1}}
\newcolumntype{L}[1]{>{\raggedleft}m{#1}}
\newcolumntype{P}[1]{>{\raggedright}p{#1}}
\newcolumntype{B}[1]{>{\raggedright}b{#1}}
\newcolumntype{Q}[1]{>{\raggedright}t{#1}}

\newcommand{\alias}[2]{
\providecommand{#1}{}
\renewcommand{#1}{#2}
}
\alias{\R}{\mathbb{R}}
\alias{\N}{\mathbb{N}}
\alias{\Z}{\mathbb{Z}}
\alias{\Q}{\mathbb{Q}}
\alias{\C}{\mathbb{C}}
\alias{\K}{\mathbb{K}}

%%%%%%%%%%%%
%% rendre +infty et -infty plus petits
%%%%%%%%%%%%
\newcommand{\sinfty}{{\scriptstyle \infty}}

%%%%%%%%%%%%%%%%%%%%%%%%%%%%%%%
%%%%% macros TP Scilab %%%%%%%%
\newcommand{\Scilab}{\textbf{Scilab}} %
\newcommand{\Scinotes}{\textbf{SciNotes}} %
\newcommand{\faire}{\noindent $\blacktriangleright$ } %
\newcommand{\fitem}{\scalebox{.8}{$\blacktriangleright$}} %
\newcommand{\entree}{{\small\texttt{ENTRÉE}}} %
\newcommand{\tab}{{\small\texttt{TAB}}} %
\newcommand{\mt}[1]{\mathtt{#1}} %
% guillemets droits

\newcommand{\ttq}{\textquotesingle} %

\newcommand{\reponse}[1]{\longboxed{
    \begin{array}{C{0.9\textwidth}}
      \nl[#1]
    \end{array}
  }} %

\newcommand{\reponseL}[2]{\longboxed{
    \begin{array}{C{#2\textwidth}}
      \nl[#1]
    \end{array}
  }} %

\newcommand{\reponseR}[1]{\longboxed{
    \begin{array}{R{0.9\textwidth}}
      #1
    \end{array}
  }} %

\newcommand{\reponseRL}[2]{\longboxed{
    \begin{array}{R{#2\textwidth}}
      #1
    \end{array}
  }} %

\newcommand{\reponseC}[1]{\longboxed{
    \begin{array}{C{0.9\textwidth}}
      #1
    \end{array}
  }} %

\colorlet{pyfunction}{Blue}
\colorlet{pyCle}{Magenta}
\colorlet{pycomment}{LimeGreen}
\colorlet{pydoc}{Cyan}
% \colorlet{SansCo}{white}
% \colorlet{AvecCo}{black}

\newcommand{\visible}[1]{{\color{ASCo}\colorlet{pydoc}{pyDo}\colorlet{pycomment}{pyCo}\colorlet{pyfunction}{pyF}\colorlet{pyCle}{pyC}\colorlet{function}{sciFun}\colorlet{var}{sciVar}\colorlet{if}{sciIf}\colorlet{comment}{sciComment}#1}} %

%%%% à changer ????
\newcommand{\invisible}[1]{{\color{ASCo}\colorlet{pydoc}{pyDo}\colorlet{pycomment}{pyCo}\colorlet{pyfunction}{pyF}\colorlet{pyCle}{pyC}\colorlet{function}{sciFun}\colorlet{var}{sciVar}\colorlet{if}{sciIf}\colorlet{comment}{sciComment}#1}} %

\newcommand{\invisibleCol}[2]{{\color{#1}#2}} %

\NewEnviron{solution} %
{ %
  \Boxed{
    \begin{array}{>{\color{ASCo}} R{0.9\textwidth}}
      \colorlet{pycomment}{pyCo}
      \colorlet{pydoc}{pyDo}
      \colorlet{pyfunction}{pyF}
      \colorlet{pyCle}{pyC}
      \colorlet{function}{sciFun}
      \colorlet{var}{sciVar}
      \colorlet{if}{sciIf}
      \colorlet{comment}{sciComment}
      \BODY
    \end{array}
  } %
} %

\NewEnviron{solutionC} %
{ %
  \Boxed{
    \begin{array}{>{\color{ASCo}} C{0.9\textwidth}}
      \colorlet{pycomment}{pyCo}
      \colorlet{pydoc}{pyDo}
      \colorlet{pyfunction}{pyF}
      \colorlet{pyCle}{pyC}
      \colorlet{function}{sciFun}
      \colorlet{var}{sciVar}
      \colorlet{if}{sciIf}
      \colorlet{comment}{sciComment}
      \BODY
    \end{array}
  } %
} %

\newcommand{\invite}{--\!\!>} %

%%%%% nouvel environnement tabular pour retour console %%%%
\colorlet{ConsoleColor}{Blue!15}
\colorlet{function}{Red}
\colorlet{var}{Maroon}
\colorlet{if}{Magenta}
\colorlet{comment}{LimeGreen}

\newcommand{\tcVar}[1]{\textcolor{var}{\bf \small #1}} %
\newcommand{\tcFun}[1]{\textcolor{function}{#1}} %
\newcommand{\tcIf}[1]{\textcolor{if}{#1}} %
\newcommand{\tcFor}[1]{\textcolor{if}{#1}} %

\newcommand{\moins}{\!\!\!\!\!\!- }
\newcommand{\espn}{\!\!\!\!\!\!}

\usepackage{booktabs,varwidth} \newsavebox\TBox
\newenvironment{console}
{\begin{lrbox}{\TBox}\varwidth{\linewidth}
    \tabular{>{\tt\small}R{0.84\textwidth}}
    \nl[-.4cm]} {\endtabular\endvarwidth\end{lrbox}%
  \fboxsep=1pt\colorbox{ConsoleColor}{\usebox\TBox}}

\newcommand{\lInv}[1]{%
  $\invite$ #1} %

\newcommand{\lAns}[1]{%
  \qquad ans \ = \nl %
  \qquad \qquad #1} %

\newcommand{\lVar}[2]{%
  \qquad #1 \ = \nl %
  \qquad \qquad #2} %

\newcommand{\lDisp}[1]{%
  #1 %
} %

\newcommand{\ligne}[1]{\underline{\small \tt #1}} %

\newcommand{\ligneAns}[2]{%
  $\invite$ #1 \nl %
  \qquad ans \ = \nl %
  \qquad \qquad #2} %

\newcommand{\ligneVar}[3]{%
  $\invite$ #1 \nl %
  \qquad #2 \ = \nl %
  \qquad \qquad #3} %

\newcommand{\ligneErr}[3]{%
  $\invite$ #1 \nl %
  \quad !-{-}error #2 \nl %
  #3} %
%%%%%%%%%%%%%%%%%%%%%% 

\newcommand{\bs}[1]{\boldsymbol{#1}} %
\newcommand{\nll}{\nl[.4cm]} %
\newcommand{\nle}{\nl[.2cm]} %
%% opérateur puissance copiant l'affichage Scilab
%\newcommand{\puis}{\!\!\!~^{\scriptscriptstyle\pmb{\wedge}}}
\newcommand{\puis}{\mbox{$\hspace{-.1cm}~^{\scriptscriptstyle\pmb{\wedge}}
    \hspace{0.05cm}$}} %
\newcommand{\pointpuis}{.\mbox{$\hspace{-.15cm}~^{\scriptscriptstyle\pmb{\wedge}}$}} %
\newcommand{\Sfois}{\mbox{$\mt{\star}$}} %

%%%%% nouvel environnement tabular pour les encadrés Scilab %%%%
\newenvironment{encadre}
{\begin{lrbox}{\TBox}\varwidth{\linewidth}
    \tabular{>{\tt\small}C{0.1\textwidth}>{\small}R{0.7\textwidth}}}
  {\endtabular\endvarwidth\end{lrbox}%
  \fboxsep=1pt\longboxed{\usebox\TBox}}

\newenvironment{encadreL}
{\begin{lrbox}{\TBox}\varwidth{\linewidth}
    \tabular{>{\tt\small}C{0.25\textwidth}>{\small}R{0.6\textwidth}}}
  {\endtabular\endvarwidth\end{lrbox}%
  \fboxsep=1pt\longboxed{\usebox\TBox}}

\newenvironment{encadreF}
{\begin{lrbox}{\TBox}\varwidth{\linewidth}
    \tabular{>{\tt\small}C{0.2\textwidth}>{\small}R{0.70\textwidth}}}
  {\endtabular\endvarwidth\end{lrbox}%
  \fboxsep=1pt\longboxed{\usebox\TBox}}

\newenvironment{encadreLL}[2]
{\begin{lrbox}{\TBox}\varwidth{\linewidth}
    \tabular{>{\tt\small}C{#1\textwidth}>{\small}R{#2\textwidth}}}
  {\endtabular\endvarwidth\end{lrbox}%
  \fboxsep=1pt\longboxed{\usebox\TBox}}

%%%%% nouvel environnement tabular pour les script et fonctions %%%%
\newcommand{\commentaireDL}[1]{\multicolumn{1}{l}{\it
    \textcolor{comment}{$\slash\slash$ #1}}}

\newcommand{\commentaire}[1]{{\textcolor{comment}{$\slash\slash$ #1}}}

\newcounter{cptcol}

\newcommand{\nocount}{\multicolumn{1}{c}{}}

\newcommand{\sciNo}[1]{{\small \underbar #1}}

\NewEnviron{scilab}{ %
  \setcounter{cptcol}{0}
  \begin{center}
    \longboxed{
      \begin{tabular}{>{\stepcounter{cptcol}{\tiny \underbar
              \thecptcol}}c>{\tt}l}
        \BODY
      \end{tabular}
    }
  \end{center}
}

\NewEnviron{scilabNC}{ %
  \begin{center}
    \longboxed{
      \begin{tabular}{>{\tt}l} %
          \BODY
      \end{tabular}
    }
  \end{center}
}

\NewEnviron{scilabC}[1]{ %
  \setcounter{cptcol}{#1}
  \begin{center}
    \longboxed{
      \begin{tabular}{>{\stepcounter{cptcol}{\tiny \underbar
              \thecptcol}}c>{\tt}l}
        \BODY
      \end{tabular}
    }
  \end{center}
}

\newcommand{\scisol}[1]{ %
  \setcounter{cptcol}{0}
  \longboxed{
    \begin{tabular}{>{\stepcounter{cptcol}{\tiny \underbar
            \thecptcol}}c>{\tt}l}
      #1
    \end{tabular}
  }
}

\newcommand{\scisolNC}[1]{ %
  \longboxed{
    \begin{tabular}{>{\tt}l}
      #1
    \end{tabular}
  }
}

\newcommand{\scisolC}[2]{ %
  \setcounter{cptcol}{#1}
  \longboxed{
    \begin{tabular}{>{\stepcounter{cptcol}{\tiny \underbar
            \thecptcol}}c>{\tt}l}
      #2
    \end{tabular}
  }
}

\NewEnviron{syntaxe}{ %
  % \fcolorbox{black}{Yellow!20}{\setlength{\fboxsep}{3mm}
  \shadowbox{
    \setlength{\fboxsep}{3mm}
    \begin{tabular}{>{\tt}l}
      \BODY
    \end{tabular}
  }
}

%%%%% fin macros TP Scilab %%%%%%%%
%%%%%%%%%%%%%%%%%%%%%%%%%%%%%%%%%%%

%%%%%%%%%%%%%%%%%%%%%%%%%%%%%%%%%%%
%%%%% TP Python - listings %%%%%%%%
%%%%%%%%%%%%%%%%%%%%%%%%%%%%%%%%%%%
\newcommand{\Python}{\textbf{Python}} %

\lstset{% general command to set parameter(s)
basicstyle=\ttfamily\small, % print whole listing small
keywordstyle=\color{blue}\bfseries\underbar,
%% underlined bold black keywords
frame=lines,
xleftmargin=10mm,
numbers=left,
numberstyle=\tiny\underbar,
numbersep=10pt,
%identifierstyle=, % nothing happens
commentstyle=\color{green}, % white comments
%%stringstyle=\ttfamily, % typewriter type for strings
showstringspaces=false}

\newcommand{\pysolCpt}[2]{
  \setcounter{cptcol}{#1}
  \longboxed{
    \begin{tabular}{>{\stepcounter{cptcol}{\tiny \underbar
            \thecptcol}}c>{\tt}l}
        #2
      \end{tabular}
    }
} %

\newcommand{\pysol}[1]{
  \setcounter{cptcol}{0}
  \longboxed{
    \begin{tabular}{>{\stepcounter{cptcol}{\tiny \underbar
            \thecptcol}}c>{\tt}l}
        #1
      \end{tabular}
    }
} %

% \usepackage[labelsep=endash]{caption}

% avec un caption
\NewEnviron{pythonCap}[1]{ %
  \renewcommand{\tablename}{Programme}
  \setcounter{cptcol}{0}
  \begin{center}
    \longboxed{
      \begin{tabular}{>{\stepcounter{cptcol}{\tiny \underbar
              \thecptcol}}c>{\tt}l}
        \BODY
      \end{tabular}
    }
    \captionof{table}{#1}
  \end{center}
}

\NewEnviron{python}{ %
  \setcounter{cptcol}{0}
  \begin{center}
    \longboxed{
      \begin{tabular}{>{\stepcounter{cptcol}{\tiny \underbar
              \thecptcol}}c>{\tt}l}
        \BODY
      \end{tabular}
    }
  \end{center}
}

\newcommand{\pyVar}[1]{\textcolor{var}{\bf \small #1}} %
\newcommand{\pyFun}[1]{\textcolor{pyfunction}{#1}} %
\newcommand{\pyCle}[1]{\textcolor{pyCle}{#1}} %
\newcommand{\pyImp}[1]{{\bf #1}} %

%%%%% commentaire python %%%%
\newcommand{\pyComDL}[1]{\multicolumn{1}{l}{\textcolor{pycomment}{\#
      #1}}}

\newcommand{\pyCom}[1]{{\textcolor{pycomment}{\# #1}}}
\newcommand{\pyDoc}[1]{{\textcolor{pydoc}{#1}}}

\newcommand{\pyNo}[1]{{\small \underbar #1}}

%%%%%%%%%%%%%%%%%%%%%%%%%%%%%%%%%%%
%%%%%% Système linéaire paramétré : écrire les opérations au-dessus
%%%%%% d'un symbole équivalent
%%%%%%%%%%%%%%%%%%%%%%%%%%%%%%%%%%%

\usepackage{systeme}

\NewEnviron{arrayEq}{ %
  \stackrel{\scalebox{.6}{$
      \begin{array}{l} 
        \BODY \\[.1cm]
      \end{array}$}
  }{\Longleftrightarrow}
}

\NewEnviron{arrayEg}{ %
  \stackrel{\scalebox{.6}{$
      \begin{array}{l} 
        \BODY \\[.1cm]
      \end{array}$}
  }{=}
}

\NewEnviron{operationEq}{ %
  \scalebox{.6}{$
    \begin{array}{l} 
      \scalebox{1.6}{$\mbox{Opérations :}$} \\[.2cm]
      \BODY \\[.1cm]
    \end{array}$}
}

% \NewEnviron{arraySys}[1]{ %
%   \sysdelim\{.\systeme[#1]{ %
%     \BODY %
%   } %
% }

%%%%%

%%%%%%%%%%
%%%%%%%%%% ESSAI
\newlength\fboxseph
\newlength\fboxsepva
\newlength\fboxsepvb

\setlength\fboxsepva{0.2cm}
\setlength\fboxsepvb{0.2cm}
\setlength\fboxseph{0.2cm}

\makeatletter

\def\longboxed#1{\leavevmode\setbox\@tempboxa\hbox{\color@begingroup%
\kern\fboxseph{\m@th$\displaystyle #1 $}\kern\fboxseph%
\color@endgroup }\my@frameb@x\relax}

\def\my@frameb@x#1{%
  \@tempdima\fboxrule \advance\@tempdima \fboxsepva \advance\@tempdima
  \dp\@tempboxa\hbox {%
    \lower \@tempdima \hbox {%
      \vbox {\hrule\@height\fboxrule \hbox{\vrule\@width\fboxrule #1
          \vbox{%
            \vskip\fboxsepva \box\@tempboxa \vskip\fboxsepvb}#1
          \vrule\@width\fboxrule }%
        \hrule \@height \fboxrule }}}}

\newcommand{\boxedhv}[3]{\setlength\fboxseph{#1cm}
  \setlength\fboxsepva{#2cm}\setlength\fboxsepvb{#2cm}\longboxed{#3}}

\newcommand{\boxedhvv}[4]{\setlength\fboxseph{#1cm}
  \setlength\fboxsepva{#2cm}\setlength\fboxsepvb{#3cm}\longboxed{#4}}

\newcommand{\Boxed}[1]{{\setlength\fboxseph{0.2cm}
  \setlength\fboxsepva{0.2cm}\setlength\fboxsepvb{0.2cm}\longboxed{#1}}}

\newcommand{\mBoxed}[1]{{\setlength\fboxseph{0.2cm}
  \setlength\fboxsepva{0.2cm}\setlength\fboxsepvb{0.2cm}\longboxed{\mbox{#1}}}}

\newcommand{\mboxed}[1]{{\setlength\fboxseph{0.2cm}
  \setlength\fboxsepva{0.2cm}\setlength\fboxsepvb{0.2cm}\boxed{\mbox{#1}}}}

\newsavebox{\fmbox}
\newenvironment{fmpage}[1]
     {\begin{lrbox}{\fmbox}\begin{minipage}{#1}}
     {\end{minipage}\end{lrbox}\fbox{\usebox{\fmbox}}}

%%%%%%%%%%
%%%%%%%%%%

\DeclareMathOperator{\ch}{ch}
\DeclareMathOperator{\sh}{sh}

%%%%%%%%%%
%%%%%%%%%%

\newcommand{\norme}[1]{\Vert #1 \Vert}

%\newcommand*\widefbox[1]{\fbox{\hspace{2em}#1\hspace{2em}}}

\newcommand{\nl}{\tabularnewline}

\newcommand{\hand}{\noindent\ding{43}\ }
\newcommand{\ie}{\textit{i.e. }}
\newcommand{\cf}{\textit{cf }}

\newcommand{\Card}{\operatorname{Card}}

\newcommand{\aire}{\mathcal{A}}

\newcommand{\LL}[1]{\mathscr{L}(#1)} %
\newcommand{\B}{\mathscr{B}} %
\newcommand{\Bc}[1]{B_{#1}} %
\newcommand{\M}[1]{\mathscr{M}_{#1}(\mathbb{R})}

\DeclareMathOperator{\im}{Im}
\DeclareMathOperator{\kr}{Ker}
\DeclareMathOperator{\rg}{rg}
\DeclareMathOperator{\tr}{tr}
\DeclareMathOperator{\spc}{Sp}
\DeclareMathOperator{\sgn}{sgn}
\DeclareMathOperator{\supp}{Supp}

\newcommand{\Mat}{{\rm{Mat}}}
\newcommand{\Vect}[1]{{\rm{Vect}}\left(#1\right)}

\newenvironment{smatrix}{%
  \begin{adjustbox}{width=.9\width}
    $
    \begin{pmatrix}
    }{%      
    \end{pmatrix}
    $
  \end{adjustbox}
}

\newenvironment{sarray}[1]{%
  \begin{adjustbox}{width=.9\width}
    $
    \begin{array}{#1}
    }{%      
    \end{array}
    $
  \end{adjustbox}
}

\newcommand{\vd}[2]{
  \scalebox{.8}{
    $\left(\!
      \begin{array}{c}
        #1 \\
        #2
      \end{array}
    \!\right)$
    }}

\newcommand{\vt}[3]{
  \scalebox{.8}{
    $\left(\!
      \begin{array}{c}
        #1 \\
        #2 \\
        #3 
      \end{array}
    \!\right)$
    }}

\newcommand{\vq}[4]{
  \scalebox{.8}{
    $\left(\!
      \begin{array}{c}
        #1 \\
        #2 \\
        #3 \\
        #4 
      \end{array}
    \!\right)$
    }}

\newcommand{\vc}[5]{
  \scalebox{.8}{
    $\left(\!
      \begin{array}{c}
        #1 \\
        #2 \\
        #3 \\
        #4 \\
        #5 
      \end{array}
    \!\right)$
    }}

\newcommand{\ee}{\text{e}}

\newcommand{\dd}{\text{d}}

%%% Ensemble de définition
\newcommand{\Df}{\mathscr{D}}
\newcommand{\Cf}{\mathscr{C}}
\newcommand{\Ef}{\mathscr{C}}

\newcommand{\rond}[1]{\,\overset{\scriptscriptstyle \circ}{\!#1}}

\newcommand{\df}[2]{\dfrac{\partial #1}{\partial #2}} %
\newcommand{\dfn}[2]{\partial_{#2}(#1)} %
\newcommand{\ddfn}[2]{\partial^2_{#2}(#1)} %
\newcommand{\ddf}[2]{\dfrac{\partial^2 #1}{\partial #2^2}} %
\newcommand{\ddfr}[3]{\dfrac{\partial^2 #1}{\partial #2 \partial
    #3}} %


\newcommand{\dlim}[1]{{\displaystyle \lim_{#1} \ }}
\newcommand{\dlimPlus}[2]{
  \dlim{
    \scalebox{.6}{
      $
      \begin{array}{l}
        #1 \rightarrow #2\\
        #1 > #2
      \end{array}
      $}}}
\newcommand{\dlimMoins}[2]{
  \dlim{
    \scalebox{.6}{
      $
      \begin{array}{l}
        #1 \rightarrow #2\\
        #1 < #2
      \end{array}
      $}}}

%%%%%%%%%%%%%%
%% petit o, développement limité
%%%%%%%%%%%%%%

\newcommand{\oo}[2]{{\underset {{\overset {#1\rightarrow #2}{}}}{o}}} %
\newcommand{\oox}[1]{{\underset {{\overset {x\rightarrow #1}{}}}{o}}} %
\newcommand{\oon}{{\underset {{\overset {n\rightarrow +\infty}{}}}{o}}} %
\newcommand{\po}[1]{{\underset {{\overset {#1}{}}}{o}}} %
\newcommand{\neqx}[1]{{\ \underset {{\overset {x \to #1}{}}}{\not\sim}\ }} %
\newcommand{\eqx}[1]{{\ \underset {{\overset {x \to #1}{}}}{\sim}\ }} %
\newcommand{\eqn}{{\ \underset {{\overset {n \to +\infty}{}}}{\sim}\ }} %
\newcommand{\eq}[2]{{\ \underset {{\overset {#1 \to #2}{}}}{\sim}\ }} %
\newcommand{\DL}[1]{{\rm{DL}}_1 (#1)} %
\newcommand{\DLL}[1]{{\rm{DL}}_2 (#1)} %

\newcommand{\negl}{<<}

\newcommand{\neglP}[1]{\begin{array}{c}
    \vspace{-.2cm}\\
    << \\
    \vspace{-.7cm}\\
    {\scriptstyle #1}
  \end{array}}

%%%%%%%%%%%%%%
%% borne sup, inf, max, min
%%%%%%%%%%%%%%
\newcommand{\dsup}[1]{\displaystyle \sup_{#1} \ }
\newcommand{\dinf}[1]{\displaystyle \inf_{#1} \ }
\newcommand{\dmax}[1]{\max\limits_{#1} \ }
\newcommand{\dmin}[1]{\min\limits_{#1} \ }

\newcommand{\dcup}[2]{{\textstyle\bigcup\limits_{#1}^{#2}}\hspace{.1cm}}
%\displaystyle \bigcup_{#1}^{#2}}
\newcommand{\dcap}[2]{{\textstyle\bigcap\limits_{#1}^{#2}}\hspace{.1cm}}
% \displaystyle \bigcap_{#1}^{#2}
%%%%%%%%%%%%%%
%% opérateurs logiques
%%%%%%%%%%%%%%
\newcommand{\NON}[1]{\mathop{\small \tt{NON}} (#1)}
\newcommand{\ET}{\mathrel{\mathop{\small \mathtt{ET}}}}
\newcommand{\OU}{\mathrel{\mathop{\small \tt{OU}}}}
\newcommand{\XOR}{\mathrel{\mathop{\small \tt{XOR}}}}

\newcommand{\id}{{\rm{id}}}

\newcommand{\sbullet}{\scriptstyle \bullet}
\newcommand{\stimes}{\scriptstyle \times}

%%%%%%%%%%%%%%%%%%
%% Probabilités
%%%%%%%%%%%%%%%%%%
\newcommand{\Prob}{\mathbb{P}}
\newcommand{\Ev}[1]{\left[ {#1} \right]}
\newcommand{\E}{\mathbb{E}}
\newcommand{\V}{\mathbb{V}}
\newcommand{\Cov}{{\rm{Cov}}}
\newcommand{\U}[2]{\mathcal{U}(\llb #1, #2\rrb)}
\newcommand{\Uc}[2]{\mathcal{U}([#1, #2])}
\newcommand{\Ucof}[2]{\mathcal{U}(]#1, #2])}
\newcommand{\Ucoo}[2]{\mathcal{U}(]#1, #2[)}
\newcommand{\Ucfo}[2]{\mathcal{U}([#1, #2[)}
\newcommand{\Bern}[1]{\mathcal{B}\left(#1\right)}
\newcommand{\Bin}[2]{\mathcal{B}\left(#1, #2\right)}
\newcommand{\G}[1]{\mathcal{G}\left(#1\right)}
\newcommand{\Pois}[1]{\mathcal{P}\left(#1\right)}
\newcommand{\HG}[3]{\mathcal{H}\left(#1, #2, #3\right)}
\newcommand{\Exp}[1]{\mathcal{E}\left(#1\right)}
\newcommand{\Norm}[2]{\mathcal{N}\left(#1, #2\right)}

\DeclareMathOperator{\cov}{Cov}

\newcommand{\var}{v.a.r. }
\newcommand{\suit}{\hookrightarrow}

\newcommand{\flecheR}[1]{\rotatebox{90}{\scalebox{#1}{\color{red}
      $\curvearrowleft$}}}


\newcommand{\partie}[1]{\mathcal{P}(#1)}
\newcommand{\Cont}[1]{\mathcal{C}^{#1}}
\newcommand{\Contm}[1]{\mathcal{C}^{#1}_m}

\newcommand{\llb}{\llbracket}
\newcommand{\rrb}{\rrbracket}

%\newcommand{\im}[1]{{\rm{Im}}(#1)}
\newcommand{\imrec}[1]{#1^{- \mathds{1}}}

\newcommand{\unq}{\mathds{1}}

\newcommand{\Hyp}{\mathtt{H}}

\newcommand{\eme}[1]{#1^{\scriptsize \mbox{ème}}}
\newcommand{\er}[1]{#1^{\scriptsize \mbox{er}}}
\newcommand{\ere}[1]{#1^{\scriptsize \mbox{ère}}}
\newcommand{\nd}[1]{#1^{\scriptsize \mbox{nd}}}
\newcommand{\nde}[1]{#1^{\scriptsize \mbox{nde}}}

\newcommand{\truc}{\mathop{\top}}
\newcommand{\fois}{\mathop{\ast}}

\newcommand{\f}[1]{\overrightarrow{#1}}

\newcommand{\checked}{\textcolor{green}{\checkmark}}

\def\restriction#1#2{\mathchoice
              {\setbox1\hbox{${\displaystyle #1}_{\scriptstyle #2}$}
              \restrictionaux{#1}{#2}}
              {\setbox1\hbox{${\textstyle #1}_{\scriptstyle #2}$}
              \restrictionaux{#1}{#2}}
              {\setbox1\hbox{${\scriptstyle #1}_{\scriptscriptstyle #2}$}
              \restrictionaux{#1}{#2}}
              {\setbox1\hbox{${\scriptscriptstyle #1}_{\scriptscriptstyle #2}$}
              \restrictionaux{#1}{#2}}}
\def\restrictionaux#1#2{{#1\,\smash{\vrule height .8\ht1 depth .85\dp1}}_{\,#2}}

\makeatletter
\newcommand*{\da@rightarrow}{\mathchar"0\hexnumber@\symAMSa 4B }
\newcommand*{\da@leftarrow}{\mathchar"0\hexnumber@\symAMSa 4C }
\newcommand*{\xdashrightarrow}[2][]{%
  \mathrel{%
    \mathpalette{\da@xarrow{#1}{#2}{}\da@rightarrow{\,}{}}{}%
  }%
}
\newcommand{\xdashleftarrow}[2][]{%
  \mathrel{%
    \mathpalette{\da@xarrow{#1}{#2}\da@leftarrow{}{}{\,}}{}%
  }%
}
\newcommand*{\da@xarrow}[7]{%
  % #1: below
  % #2: above
  % #3: arrow left
  % #4: arrow right
  % #5: space left 
  % #6: space right
  % #7: math style 
  \sbox0{$\ifx#7\scriptstyle\scriptscriptstyle\else\scriptstyle\fi#5#1#6\m@th$}%
  \sbox2{$\ifx#7\scriptstyle\scriptscriptstyle\else\scriptstyle\fi#5#2#6\m@th$}%
  \sbox4{$#7\dabar@\m@th$}%
  \dimen@=\wd0 %
  \ifdim\wd2 >\dimen@
    \dimen@=\wd2 %   
  \fi
  \count@=2 %
  \def\da@bars{\dabar@\dabar@}%
  \@whiledim\count@\wd4<\dimen@\do{%
    \advance\count@\@ne
    \expandafter\def\expandafter\da@bars\expandafter{%
      \da@bars
      \dabar@ 
    }%
  }%  
  \mathrel{#3}%
  \mathrel{%   
    \mathop{\da@bars}\limits
    \ifx\\#1\\%
    \else
      _{\copy0}%
    \fi
    \ifx\\#2\\%
    \else
      ^{\copy2}%
    \fi
  }%   
  \mathrel{#4}%
}
\makeatother



\newcount\depth

\newcount\depth
\newcount\totaldepth

\makeatletter
\newcommand{\labelsymbol}{%
      \ifnum\depth=0
        %
      \else
        \rlap{\,$\bullet$}%
      \fi
}

\newcommand*\bernoulliTree[1]{%
    \depth=#1\relax            
    \totaldepth=#1\relax
    \draw node(root)[bernoulli/root] {\labelsymbol}[grow=right] \draw@bernoulli@tree;
    \draw \label@bernoulli@tree{root};                                   
}                                                                        

\def\draw@bernoulli@tree{%
    \ifnum\depth>0 
      child[parent anchor=east] foreach \type/\label in {left child/$E$,right child/$S$} {%
          node[bernoulli/\type] {\label\strut\labelsymbol} \draw@bernoulli@tree
      }
      coordinate[bernoulli/increment] (dummy)
   \fi%
}

\def\label@bernoulli@tree#1{%
    \ifnum\depth>0
      ($(#1)!0.5!(#1-1)$) node[fill=white,bernoulli/decrement] {\tiny$p$}
      \label@bernoulli@tree{#1-1}
      ($(#1)!0.5!(#1-2)$) node[fill=white] {\tiny$q$}
      \label@bernoulli@tree{#1-2}
      coordinate[bernoulli/increment] (dummy)
   \fi%
}

\makeatother

\tikzset{bernoulli/.cd,
         root/.style={},
         decrement/.code=\global\advance\depth by-1\relax,
         increment/.code=\global\advance\depth by 1\relax,
         left child/.style={bernoulli/decrement},
         right child/.style={}}


\newcommand{\eps}{\varepsilon}

% \newcommand{\tendi}[1]{\xrightarrow[\footnotesize #1 \rightarrow
%   +\infty]{}}%

\newcommand{\tend}{\rightarrow}%
\newcommand{\tendn}{\underset{n\to +\infty}{\longrightarrow}} %
\newcommand{\ntendn}{\underset{n\to
    +\infty}{\not\hspace{-.15cm}\longrightarrow}} %
% \newcommand{\tendn}{\xrightarrow[\footnotesize n \rightarrow
%   +\infty]{}}%
\newcommand{\Tendx}[1]{\xrightarrow[\footnotesize x \rightarrow
  #1]{}}%
\newcommand{\tendx}[1]{\underset{x\to #1}{\longrightarrow}}%
\newcommand{\ntendx}[1]{\underset{x\to #1}{\not\!\!\longrightarrow}}%
\newcommand{\tendd}[2]{\underset{#1\to #2}{\longrightarrow}}%
% \newcommand{\tendd}[2]{\xrightarrow[\footnotesize #1 \rightarrow
%   #2]{}}%
\newcommand{\tendash}[1]{\xdashrightarrow[\footnotesize #1 \rightarrow
  +\infty]{}}%
\newcommand{\tendashx}[1]{\xdashrightarrow[\footnotesize x \rightarrow
  #1]{}}%
\newcommand{\tendb}[1]{\underset{#1 \to +\infty}{\longrightarrow}}%
\newcommand{\tendL}{\overset{\mathscr L}{\underset{n \to
      +\infty}{\longrightarrow}}}%
\newcommand{\tendP}{\overset{\Prob}{\underset{n \to
      +\infty}{\longrightarrow}}}%
\newcommand{\tenddL}[1]{\overset{\mathscr L}{\underset{#1 \to
      +\infty}{\longrightarrow}}}%

\NewEnviron{attention}{ %
  ~\\[-.2cm]\noindent
  \begin{minipage}{\linewidth}
  \setlength{\fboxsep}{3mm}%
  \ \ \dbend \ \ %
  \fbox{\parbox[t]{.88\linewidth}{\BODY}} %
  \end{minipage}\\
}

\NewEnviron{sattention}[1]{ %
  ~\\[-.2cm]\noindent
  \begin{minipage}{#1\linewidth}
  \setlength{\fboxsep}{3mm}%
  \ \ \dbend \ \ %
  \fbox{\parbox[t]{.88\linewidth}{\BODY}} %
  \end{minipage}\\
}

%%%%% OBSOLETE %%%%%%

% \newcommand{\attention}[1]{
%   \noindent
%   \begin{tabular}{@{}l|p{11.5cm}|}
%     \cline{2-2}
%     \vspace{-.2cm} 
%     & \nl
%     \dbend & #1 \nl
%     \cline{2-2}
%   \end{tabular}
% }

% \newcommand{\attentionv}[2]{
%   \noindent
%   \begin{tabular}{@{}l|p{11.5cm}|}
%     \cline{2-2}
%     \vspace{-.2cm} 
%     & \nl
%     \dbend & #2 \nl[#1 cm]
%     \cline{2-2}
%   \end{tabular}
% }

\newcommand{\explainvb}[2]{
  \noindent
  \begin{tabular}{@{}l|p{11.5cm}|}
    \cline{2-2}
    \vspace{-.2cm} 
    & \nl
    \hand & #2 \nl [#1 cm]
    \cline{2-2}
  \end{tabular}
}


% \noindent
% \begin{tabular}{@{}l|lp{11cm}|}
%   \cline{3-3} 
%   \multicolumn{1}{@{}l@{\dbend}}{} & & #1 \nl
%   \multicolumn{1}{l}{} & & \nl [-.8cm]
%   & & #2 \nl
%   \cline{2-3}
% \end{tabular}

% \newcommand{\attention}[1]{
%   \noindent
%   \begin{tabular}{@{}@{}cp{11cm}}
%     \dbend & #1 \nl
%   \end{tabular}
% }

\newcommand{\PP}[1]{\mathcal{P}(#1)}
\newcommand{\FF}[1]{\mathcal{F}(#1)}

\newcommand{\DSum}[2]{\displaystyle\sum\limits_{#1}^{#2}\hspace{.1cm}}
\newcommand{\Sum}[2]{{\textstyle\sum\limits_{#1}^{#2}}\hspace{.1cm}}
\newcommand{\Serie}{\textstyle\sum\ }
\newcommand{\Prod}[2]{\textstyle\prod\limits_{#1}^{#2}}

\newcommand{\Prim}[3]{\left[\ {#1} \ \right]_{\scriptscriptstyle
   \hspace{-.15cm} ~_{#2}\, }^ {\scriptscriptstyle \hspace{-.15cm} ~^{#3}\, }}

% \newcommand{\Prim}[3]{\left[\ {#1} \ \right]_{\scriptscriptstyle
%     \!\!~_{#2}}^ {\scriptscriptstyle \!\!~^{#3}}}

\newcommand{\dint}[2]{\displaystyle \int_{#1}^{#2}\ }
\newcommand{\Int}[2]{{\rm{Int}}_{\scriptscriptstyle #1, #2}}
\newcommand{\dt}{\ dt}
\newcommand{\dx}{\ dx}

\newcommand{\llpar}[1]{\left(\!\!\!
    \begin{array}{c}
      \rule{0pt}{#1}
    \end{array}
  \!\!\!\right.}

\newcommand{\rrpar}[1]{\left.\!\!\!
    \begin{array}{c}
      \rule{0pt}{#1}
    \end{array}
  \!\!\!\right)}

\newcommand{\llacc}[1]{\left\{\!\!\!
    \begin{array}{c}
      \rule{0pt}{#1}
    \end{array}
  \!\!\!\right.}

\newcommand{\rracc}[1]{\left.\!\!\!
    \begin{array}{c}
      \rule{0pt}{#1}
    \end{array}
  \!\!\!\right\}}

\newcommand{\ttacc}[1]{\mbox{\rotatebox{-90}{\hspace{-.7cm}$\llacc{#1}$}}}
\newcommand{\bbacc}[1]{\mbox{\rotatebox{90}{\hspace{-.5cm}$\llacc{#1}$}}}

\newcommand{\comp}[1]{\overline{#1}}%

\newcommand{\dcomp}[2]{\stackrel{\mbox{\ \ \----}{\scriptscriptstyle
      #2}}{#1}}%

% \newcommand{\Comp}[2]{\stackrel{\mbox{\ \
%       \-------}{\scriptscriptstyle #2}}{#1}}

% \newcommand{\dcomp}[2]{\stackrel{\mbox{\ \
%       \-------}{\scriptscriptstyle #2}}{#1}}

\newcommand{\A}{\mathscr{A}}

\newcommand{\conc}[1]{
  \begin{center}
    \fbox{
      \begin{tabular}{c}
        #1
      \end{tabular}
    }
  \end{center}
}

\newcommand{\concC}[1]{
  \begin{center}
    \fbox{
    \begin{tabular}{C{10cm}}
      \quad #1 \quad
    \end{tabular}
    }
  \end{center}
}

\newcommand{\concL}[2]{
  \begin{center}
    \fbox{
    \begin{tabular}{C{#2cm}}
      \quad #1 \quad
    \end{tabular}
    }
  \end{center}
}

% \newcommand{\lims}[2]{\prod\limits_{#1}^{#2}}

\newtheorem{theorem}{Théorème}[]
\newtheorem{lemma}{Lemme}[]
\newtheorem{proposition}{Proposition}[]
\newtheorem{corollary}{Corollaire}[]

% \newenvironment{proof}[1][Démonstration]{\begin{trivlist}
% \item[\hskip \labelsep {\bfseries #1}]}{\end{trivlist}}
\newenvironment{definition}[1][Définition]{\begin{trivlist}
\item[\hskip \labelsep {\bfseries #1}]}{\end{trivlist}}
\newenvironment{example}[1][Exemple]{\begin{trivlist}
\item[\hskip \labelsep {\bfseries #1}]}{\end{trivlist}}
\newenvironment{examples}[1][Exemples]{\begin{trivlist}
\item[\hskip \labelsep {\bfseries #1}]}{\end{trivlist}}
\newenvironment{notation}[1][Notation]{\begin{trivlist}
\item[\hskip \labelsep {\bfseries #1}]}{\end{trivlist}}
\newenvironment{propriete}[1][Propriété]{\begin{trivlist}
\item[\hskip \labelsep {\bfseries #1}]}{\end{trivlist}}
\newenvironment{proprietes}[1][Propriétés]{\begin{trivlist}
\item[\hskip \labelsep {\bfseries #1}]}{\end{trivlist}}
\newenvironment{remarkSC}[1][Remarque]{\begin{trivlist}
\item[\hskip \labelsep {\bfseries #1}]}{\end{trivlist}}
\newenvironment{application}[1][Application]{\begin{trivlist}
\item[\hskip \labelsep {\bfseries #1}]}{\end{trivlist}}

% Environnement pour les réponses des DS
\newenvironment{answer}{\par\emph{Réponse :}\par{}}
{\vspace{-.6cm}\hspace{\stretch{1}}\rule{1ex}{1ex}\vspace{.3cm}}

\newenvironment{answerTD}{\vspace{.2cm}\par\emph{Réponse :}\par{}}
{\hspace{\stretch{1}}\rule{1ex}{1ex}\vspace{.3cm}}

\newenvironment{answerCours}{\noindent\emph{Réponse :}}
{\rule{1ex}{1ex}}%\vspace{.3cm}}


% footnote in footer
\newcommand{\fancyfootnotetext}[2]{%
  \fancypagestyle{dingens}{%
    \fancyfoot[LO,RE]{\parbox{0.95\textwidth}{\footnotemark[#1]\footnotesize
        #2}}%
  }%
  \thispagestyle{dingens}%
}

%%% définit le style (arabic : 1,2,3...) et place des parenthèses
%%% autour de la numérotation
\renewcommand*{\thefootnote}{(\arabic{footnote})}
% http://www.tuteurs.ens.fr/logiciels/latex/footnote.html

%%%%%%%% tikz axis
% \pgfplotsset{every axis/.append style={
%                     axis x line=middle,    % put the x axis in the middle
%                     axis y line=middle,    % put the y axis in the middle
%                     axis line style={<->,color=blue}, % arrows on the axis
%                     xlabel={$x$},          % default put x on x-axis
%                     ylabel={$y$},          % default put y on y-axis
%             }}

%%%% s'utilise comme suit

% \begin{axis}[
%   xmin=-8,xmax=4,
%   ymin=-8,ymax=4,
%   grid=both,
%   ]
%   \addplot [domain=-3:3,samples=50]({x^3-3*x},{3*x^2-9}); 
% \end{axis}

%%%%%%%%



%%%%%%%%%%%% Pour avoir des numéros de section qui correspondent à
%%%%%%%%%%%% ceux du tableau
\renewcommand{\thesection}{\Roman{section}.\hspace{-.3cm}}
\renewcommand{\thesubsection}{\Roman{section}.\arabic{subsection}.\hspace{-.2cm}}
\renewcommand{\thesubsubsection}{\Roman{section}.\arabic{subsection}.\alph{subsubsection})\hspace{-.2cm}}
%%%%%%%%%%%% 

%%% Changer le nom des figures : Fig. au lieu de Figure
\usepackage[font=small,labelfont=bf,labelsep=space]{caption}
\captionsetup{%
  figurename=Fig.,
  tablename=Tab.
}
% \renewcommand{\thesection}{\Roman{section}.\hspace{-.2cm}}
% \renewcommand{\thesubsection}{\Roman{section}
%   .\hspace{.2cm}\arabic{subsection}\ .\hspace{-.3cm}}
% \renewcommand{\thesubsubsection}{\alph{subsection})}

\newenvironment{tabliste}[1]
{\begin{tabenum}[\bfseries\small\itshape #1]}{\end{tabenum}} 

\newenvironment{noliste}[1] %
{\begin{enumerate}[\bfseries\small\itshape #1]} %
  {\end{enumerate}}

\newenvironment{nonoliste}[1] %
{\begin{enumerate}[\hspace{-12pt}\bfseries\small\itshape #1]} %
  {\end{enumerate}}

\newenvironment{arrayliste}[1]{ 
  % List with minimal white space to fit in small areas, e.g. table
  % cell
  \begin{minipage}[t]{\linewidth} %
    \begin{enumerate}[\bfseries\small\itshape #1] %
      {\leftmargin=0.5em \rightmargin=0em
        \topsep=0em \parskip=0em \parsep=0em
        \listparindent=0em \partopsep=0em \itemsep=0pt
        \itemindent=0em \labelwidth=\leftmargin\labelsep+0.25em}
    }{
    \end{enumerate}\end{minipage}
}

\newenvironment{nolistes}[2]
{\begin{enumerate}[\bfseries\small\itshape
    #1]\setlength{\itemsep}{#2 mm}}{\end{enumerate}}

\newenvironment{liste}[1]
{\begin{enumerate}[\hspace{12pt}\bfseries\small\itshape #1]}{\end{enumerate}} 



%%%%%%%% Pour les programmes de colle %%%%%%%

\newcommand{\cours}{{\small \tt (COURS)}} %
\newcommand{\poly}{{\small \tt (POLY)}} %
\newcommand{\exo}{{\small \tt (EXO)}} %
\newcommand{\culture}{{\small \tt (CULTURE)}} %
\newcommand{\methodo}{{\small \tt (MÉTHODO)}} %
\newcommand{\methodob}{\Boxed{\mbox{\tt MÉTHODO}}} %

%%%%%%%% Pour les TD %%%%%%%
\newtheoremstyle{exostyle} {\topsep} % espace avant
{.6cm} % espace apres
{} % Police utilisee par le style de thm
{} % Indentation (vide = aucune, \parindent = indentation paragraphe)
{\bfseries} % Police du titre de thm
{} % Signe de ponctuation apres le titre du thm
{ } % Espace apres le titre du thm (\newline = linebreak)
{\thmname{#1}\thmnumber{ #2}\thmnote{.
    \normalfont{\textit{#3}}}} % composants du titre du thm : \thmname
                               % = nom du thm, \thmnumber = numéro du
                               % thm, \thmnote = sous-titre du thm
 
\theoremstyle{exostyle}
\newtheorem{exercice}{Exercice}
\newtheorem{exerciceAP}{Exercice avec préparation}
\newtheorem{exerciceSP}{Exercice sans préparation}
\newtheorem*{exoCours}{Exercice}

%%%%%%%% Pour des théorèmes Sans Espaces APRÈS %%%%%%%
\newtheoremstyle{exostyleSE} {\topsep} % espace avant
{} % espace apres
{} % Police utilisee par le style de thm
{} % Indentation (vide = aucune, \parindent = indentation paragraphe)
{\bfseries} % Police du titre de thm
{} % Signe de ponctuation apres le titre du thm
{ } % Espace apres le titre du thm (\newline = linebreak)
{\thmname{#1}\thmnumber{ #2}\thmnote{.
    \normalfont{\textit{#3}}}} % composants du titre du thm : \thmname
                               % = nom du thm, \thmnumber = numéro du
                               % thm, \thmnote = sous-titre du thm
 
\theoremstyle{exostyleSE}
\newtheorem{exerciceSE}{Exercice}
\newtheorem*{exoCoursSE}{Exercice}

% \newcommand{\lims}[2]{\prod\limits_{#1}^{#2}}

\newtheorem{theoremSE}{Théorème}[]
\newtheorem{lemmaSE}{Lemme}[]
\newtheorem{propositionSE}{Proposition}[]
\newtheorem{corollarySE}{Corollaire}[]

% \newenvironment{proofSE}[1][Démonstration]{\begin{trivlist}
% \item[\hskip \labelsep {\bfseries #1}]}{\end{trivlist}}
\newenvironment{definitionSE}[1][Définition]{\begin{trivlist}
  \item[\hskip \labelsep {\bfseries #1}]}{\end{trivlist}}
\newenvironment{exampleSE}[1][Exemple]{\begin{trivlist} 
  \item[\hskip \labelsep {\bfseries #1}]}{\end{trivlist}}
\newenvironment{examplesSE}[1][Exemples]{\begin{trivlist}
\item[\hskip \labelsep {\bfseries #1}]}{\end{trivlist}}
\newenvironment{notationSE}[1][Notation]{\begin{trivlist}
\item[\hskip \labelsep {\bfseries #1}]}{\end{trivlist}}
\newenvironment{proprieteSE}[1][Propriété]{\begin{trivlist}
\item[\hskip \labelsep {\bfseries #1}]}{\end{trivlist}}
\newenvironment{proprietesSE}[1][Propriétés]{\begin{trivlist}
\item[\hskip \labelsep {\bfseries #1}]}{\end{trivlist}}
\newenvironment{remarkSE}[1][Remarque]{\begin{trivlist}
\item[\hskip \labelsep {\bfseries #1}]}{\end{trivlist}}
\newenvironment{applicationSE}[1][Application]{\begin{trivlist}
\item[\hskip \labelsep {\bfseries #1}]}{\end{trivlist}}

%%%%%%%%%%% Obtenir les étoiles sans charger le package MnSymbol
%%%%%%%%%%%
\DeclareFontFamily{U} {MnSymbolC}{}
\DeclareFontShape{U}{MnSymbolC}{m}{n}{
  <-6> MnSymbolC5
  <6-7> MnSymbolC6
  <7-8> MnSymbolC7
  <8-9> MnSymbolC8
  <9-10> MnSymbolC9
  <10-12> MnSymbolC10
  <12-> MnSymbolC12}{}
\DeclareFontShape{U}{MnSymbolC}{b}{n}{
  <-6> MnSymbolC-Bold5
  <6-7> MnSymbolC-Bold6
  <7-8> MnSymbolC-Bold7
  <8-9> MnSymbolC-Bold8
  <9-10> MnSymbolC-Bold9
  <10-12> MnSymbolC-Bold10
  <12-> MnSymbolC-Bold12}{}

\DeclareSymbolFont{MnSyC} {U} {MnSymbolC}{m}{n}

\DeclareMathSymbol{\filledlargestar}{\mathrel}{MnSyC}{205}
\DeclareMathSymbol{\largestar}{\mathrel}{MnSyC}{131}

\newcommand{\facile}{\rm{(}$\scriptstyle\largestar$\rm{)}} %
\newcommand{\moyen}{\rm{(}$\scriptstyle\filledlargestar$\rm{)}} %
\newcommand{\dur}{\rm{(}$\scriptstyle\filledlargestar\filledlargestar$\rm{)}} %
\newcommand{\costaud}{\rm{(}$\scriptstyle\filledlargestar\filledlargestar\filledlargestar$\rm{)}}

%%%%%%%%%%%%%%%%%%%%%%%%%

%%%%%%%%%%%%%%%%%%%%%%%%%
%%%%%%%% Fin de la partie TD

%%%%%%%%%%%%%%%%
%%%%%%%%%%%%%%%%
\makeatletter %
\newenvironment{myitemize}{%
  \setlength{\topsep}{0pt} %
  \setlength{\partopsep}{0pt} %
  \renewcommand*{\@listi}{\leftmargin\leftmargini \parsep\z@
    \topsep\z@ \itemsep\z@} \let\@listI\@listi %
  \itemize %
}{\enditemize} %
\makeatother
%%%%%%%%%%%%%%%%
%%%%%%%%%%%%%%%%

%% Commentaires dans la correction du livre

\newcommand{\Com}[1]{
% Define box and box title style
\tikzstyle{mybox} = [draw=black!50,
very thick,
    rectangle, rounded corners, inner sep=10pt, inner ysep=8pt]
\tikzstyle{fancytitle} =[rounded corners, fill=black!80, text=white]
\tikzstyle{fancylogo} =[ text=white]
\begin{center}

\begin{tikzpicture}
\node [mybox] (box){%

    \begin{minipage}{0.90\linewidth}
\vspace{6pt}  #1
    \end{minipage}
};
\node[fancytitle, right=10pt] at (box.north west) 
{\bfseries\normalsize{Commentaire}};

\end{tikzpicture}%

\end{center}
%
}

% \NewEnviron{remark}{%
%   % Define box and box title style
%   \tikzstyle{mybox} = [draw=black!50, very thick, rectangle, rounded
%   corners, inner sep=10pt, inner ysep=8pt] %
%   \tikzstyle{fancytitle} =[rounded corners, fill=black!80,
%   text=white] %
%   \tikzstyle{fancylogo} =[ text=white]
%   \begin{center}
%     \begin{tikzpicture}
%       \node [mybox] (box){%
%         \begin{minipage}{0.90\linewidth}
%           \vspace{6pt} \BODY
%         \end{minipage}
%       }; %
%       \node[fancytitle, right=10pt] at (box.north west) %
%       {\bfseries\normalsize{Commentaire}}; %
%     \end{tikzpicture}%
%   \end{center}
% }
% \makeatother


%%%
% Remarques comme dans le livre
%%%

\NewEnviron{remark}{%
  % Define box and box title style
  \tikzstyle{mybox} = [draw=black!50, very thick, rectangle, rounded
  corners, inner sep=10pt, inner ysep=8pt] %
  \tikzstyle{fancytitle} = [rounded corners , fill=black!80,
  text=white] %
  \tikzstyle{fancylogo} =[ text=white]
  \begin{center}
    \begin{tikzpicture}
      \node [mybox] (box){%
        \begin{minipage}{0.90\linewidth}
          \vspace{6pt} \BODY
        \end{minipage}
      }; %
      \node[fancytitle, right=10pt] at (box.north west) %
      {\bfseries\normalsize{Commentaire}}; %
    \end{tikzpicture}%
  \end{center}
}

\NewEnviron{remarkST}{%
  % Define box and box title style
  \tikzstyle{mybox} = [draw=black!50, very thick, rectangle, rounded
  corners, inner sep=10pt, inner ysep=8pt] %
  \tikzstyle{fancytitle} = [rounded corners , fill=black!80,
  text=white] %
  \tikzstyle{fancylogo} =[ text=white]
  \begin{center}
    \begin{tikzpicture}
      \node [mybox] (box){%
        \begin{minipage}{0.90\linewidth}
          \vspace{6pt} \BODY
        \end{minipage}
      }; %
      % \node[fancytitle, right=10pt] at (box.north west) %
%       {\bfseries\normalsize{Commentaire}}; %
    \end{tikzpicture}%
  \end{center}
}

\NewEnviron{remarkL}[1]{%
  % Define box and box title style
  \tikzstyle{mybox} = [draw=black!50, very thick, rectangle, rounded
  corners, inner sep=10pt, inner ysep=8pt] %
  \tikzstyle{fancytitle} =[rounded corners, fill=black!80,
  text=white] %
  \tikzstyle{fancylogo} =[ text=white]
  \begin{center}
    \begin{tikzpicture}
      \node [mybox] (box){%
        \begin{minipage}{#1\linewidth}
          \vspace{6pt} \BODY
        \end{minipage}
      }; %
      \node[fancytitle, right=10pt] at (box.north west) %
      {\bfseries\normalsize{Commentaire}}; %
    \end{tikzpicture}%
  \end{center}
}

\NewEnviron{remarkSTL}[1]{%
  % Define box and box title style
  \tikzstyle{mybox} = [draw=black!50, very thick, rectangle, rounded
  corners, inner sep=10pt, inner ysep=8pt] %
  \tikzstyle{fancytitle} =[rounded corners, fill=black!80,
  text=white] %
  \tikzstyle{fancylogo} =[ text=white]
  \begin{center}
    \begin{tikzpicture}
      \node [mybox] (box){%
        \begin{minipage}{#1\linewidth}
          \vspace{6pt} \BODY
        \end{minipage}
      }; %
%       \node[fancytitle, right=10pt] at (box.north west) %
%       {\bfseries\normalsize{Commentaire}}; %
    \end{tikzpicture}%
  \end{center}
}

\NewEnviron{titre} %
{ %
  ~\\[-1.8cm]
  \begin{center}
    \bf \LARGE \BODY
  \end{center}
  ~\\[-.6cm]
  \hrule %
  \vspace*{.2cm}
} %

\NewEnviron{titreL}[2] %
{ %
  ~\\[-#1cm]
  \begin{center}
    \bf \LARGE \BODY
  \end{center}
  ~\\[-#2cm]
  \hrule %
  \vspace*{.2cm}
} %




% \renewcommand{\thesection}{\Roman{section}.\hspace{-.3cm}}
% \renewcommand{\thesubsection}{\Alph{subsection}.\hspace{-.2cm}}
\pagestyle{fancy} %
\lhead{ECE2 \hfill Mathématiques \\} %
\chead{\hrule} %
\rhead{} %
\lfoot{} %
\cfoot{} %
\rfoot{\thepage} %

\renewcommand{\headrulewidth}{0pt}% : Trace un trait de séparation
                                    % de largeur 0,4 point. Mettre 0pt
                                    % pour supprimer le trait.

\renewcommand{\footrulewidth}{0.4pt}% : Trace un trait de séparation
                                    % de largeur 0,4 point. Mettre 0pt
                                    % pour supprimer le trait.

\setlength{\headheight}{14pt}

\title{\bf \vspace{-1.6cm} HEC 2019} %
\author{} %
\date{} %
\begin{document}

\maketitle %
\vspace{-1.2cm}\hrule %
\thispagestyle{fancy}

\vspace*{-.2cm}

%%DEBUT

\section*{Exercice}

\begin{noliste}{1.}
  \setlength{\itemsep}{4mm} %
\item Dans cette question, on considère les matrices $C =
  \begin{smatrix}
    0 \\
    1 \\
    2
  \end{smatrix}
  \in \M{3, 1}$, $L =
  \begin{smatrix}
    1 & 2 & -1
  \end{smatrix}
  \in \M{3, 1}$ et le produit matriciel $M = CL$.

  \begin{noliste}{a)}
    \setlength{\itemsep}{2mm} %
  \item
    \begin{nonoliste}{(i)}
      \setlength{\itemsep}{2mm} %
    \item Calculer $M$ et $M^2$.

      \begin{proof}~%
        \begin{noliste}{$\sbullet$}
        \item Tout d'abord : $M \ = \
          \begin{smatrix}
            0 \\
            1 \\
            2
          \end{smatrix}
          \begin{smatrix}
            1 & 2 & -1
          \end{smatrix}
          \ = \ 
          \begin{smatrix}
            0 & 0 & 0 \\
            1 & 2 & -1 \\
            2 & 4 & -2 
          \end{smatrix}
          $. %
          \conc{$M =
            \begin{smatrix}
              0 & 0 & 0 \\
              1 & 2 & -1 \\
              2 & 4 & -2
            \end{smatrix}$}
          \begin{remarkL}{.98}%
            Si on note $C_1$, $C_2$ et $C_3$ les colonnes de $M$, on
            remarque :
            \[
            C_1 = C, \quad C_2 = 2 \, C \quad \text{ et } \quad C_3 =
            - \, C
            \]
            La matrice $M$ est donc obtenue par concaténation de
            copies, à coefficients multiplicatifs près, de la colonne
            $C$. L'objectif de l'énoncé est l'étude des propriétés de
            telles matrices.
          \end{remarkL}

        \item $
          \begin{array}[t]{R{1.4cm}rcl@{\qquad}>{\it}R{5cm}}
            Ensuite : & M^2 & = & M \times M \ = \ CL \times CL 
            \\[.2cm]
            & & = & C \times (LC) \times L
            & (par associativité)
            \nl
            % \nl[-.2cm]
            & & = & (LC) \cdot C \times L
            & (car comme $L \in \M{1, 3}$ et $C \in \M{3, 1}$, $LC$
            est un réel)
            \nl
            %\nl[-.2cm]
            & & = & 0 \cdot M \ = \ 0_{\M{3}}
          \end{array}
          $\\
          En effet : $LC \ = \
          \begin{smatrix}
            1 & 2 & -1
          \end{smatrix}
          \begin{smatrix}
            0 \\
            1 \\
            2
          \end{smatrix}
          \ = \ 1 \times 0 + 2 \times 1 - 1 \times 2 = 2 - 2 = 0$. %
          \conc{Ainsi : $M^{2} = 0_{\M{3}}$.}
        \end{noliste}
        \begin{remarkL}{.98}%
          Évidemment, on peut aussi effectuer le calcul de $M^2$
          directement :
          \[
          M^2 \ = \             
          \begin{smatrix}
            0 & 0 & 0 \\
            1 & 2 & -1 \\
            2 & 4 & -2
          \end{smatrix}
          \begin{smatrix}
            0 & 0 & 0 \\
            1 & 2 & -1 \\
            2 & 4 & -2
          \end{smatrix}
          \ = \ 
          \begin{smatrix}
            0 & 0 & 0 \\
            0 & 0 & 0 \\
            0 & 0 & 0
          \end{smatrix}
          \]
          Un tel calcul permet assurément d'obtenir tous les points de
          la question mais n'est pas dans l'esprit de la construction
          très particulière de la matrice $M$. Il s'agit ici de faire
          apparaître sur un exemple simple de petite taille
          (manipulation d'une matrice ligne et d'une matrice colonne à
          $3$ éléments), des propriétés qu'on généralisera à des
          matrices de tailles quelconques.
        \end{remarkL}~\\[-1.4cm]
      \end{proof}


      \newpage


    \item Déterminer le rang de $M$.

      \begin{proof}~%
        \[
        \begin{array}{rcl}
          \rg\left( 
            \begin{smatrix} 
              0 & 0 & 0 \\
              1 & 2 & -1 \\
              2 & 4 & -2 
            \end{smatrix}
          \right)
          & = & 
          \rg\left( 
            \begin{smatrix} 
              0 \\ 
              1 \\
              2 
            \end{smatrix}
            ,
            \begin{smatrix} 
              0 \\ 
              2 \\
              4 
            \end{smatrix}
            ,
            \begin{smatrix} 
              0 \\ 
              -1 \\
              -2 
            \end{smatrix}
          \right)
          \\[1cm]
          & = &
          \rg\left( 
            \begin{smatrix} 
              0 \\ 
              1 \\
              2 
            \end{smatrix}
            ,
            2 \, 
            \begin{smatrix} 
              0 \\ 
              1 \\
              2 
            \end{smatrix}
            ,
            - \,
            \begin{smatrix} 
              0 \\ 
              1 \\
              2 
            \end{smatrix}
          \right)
          \ = \ 
          \rg\left( 
            \begin{smatrix} 
              0 \\ 
              1 \\
              2 
            \end{smatrix}
          \right) \ = \ 1
        \end{array}
        \]
        \conc{$\rg(M) = 1$}~\\[-1.2cm]
      \end{proof}

    \item La matrice $M$ est-elle diagonalisable ?

      \begin{proof}~%
        \begin{noliste}{$\sbullet$}
        \item D'après la question \itbf{1.a)(i)}, le polynôme $Q(X) =
          X^2$ est un polynôme annulateur de la matrice $M$. Ainsi :
          \[
          \spc(M) \subset \{\text{racines de $Q$}\} = \{0\}
          \]
          \conc{Le réel $0$ est la seule valeur propre possible de
            $M$.}

        \item D'après la question précédente : $\rg(M) = 1 \neq 3 =
          \dim\big( \M{3,1} \big)$. %
          \conc{Ainsi, la matrice $M$ n'est pas inversible et $0$ est
            la seule valeur propre de $M$.}

        \item Notons $\B$ la base canonique de $\R^3$. Considérons
          l'endomorphisme $f \in \LL{\R^3}$ dont la représentation
          dans la base $\B$ est $M$. Par le théorème du rang :
          \[
          \begin{array}{ccccc}
            \dim\big( \R^3 \big) & = & \dim\big( \kr(f) \big) & + &
            \dim\big( \im(f) \big)
            \\[.2cm]
            \shortparallel & & \shortparallel & & \shortparallel
            \\[.2cm]
            3 & & \dim\big(  E_0(f) \big) & & \rg(f)
            \\[.2cm]
            \shortparallel & & \shortparallel & & \shortparallel
            \\[.2cm]
            \dim\big( \M{3,1} \big) & = & \dim\big( E_0(M) \big) & + & 
            \rg(M)
          \end{array}
          \]
          \conc{Ainsi : $\dim\big( E_0(M) \big) = \dim\big( \M{3,1}
            \big) - \rg(M) = 3 - 1 = 2$.} %
        \end{noliste}
        \conc{Comme $\dim\big( E_0(M) \big) = 2 \neq 3 = \dim\big(
          \M{3,1} \big)$, la matrice $M$ n'est pas diagonalisable.}
        \begin{remarkL}{.98}
          \begin{noliste}{$\sbullet$}
          \item On a démontré que la matrice $M$ possédait une unique
            valeur propre. Dans ce cas, il est classique de procéder
            par l'absurde pour démontrer que $M$ n'est pas
            diagonalisable.

          \item Supposons que $M$ est diagonalisable.\\
            Il existe donc une matrice inversible $P \in \M{3}$ et une
            matrice diagonale $D \in \M{3}$ dont les coefficients
            diagonaux sont les valeurs propres de $M$ telles que $% =
            PDP^{-1}$.\\
            Or $0$ est la seule valeur propre de $M$. Ainsi $D =
            0_{\M{3}}$ et :
            \[
            M = PDP^{-1} = P \, 0_{\M{3}} \, P^{-1} = 0_{\M{3}}
            \]
            Absurde !
          \end{noliste}
        \end{remarkL}~\\[-1.4cm]
      \end{proof}
    \end{nonoliste}


    \newpage


    \begin{remarkL}{.99}
      \begin{noliste}{$\sbullet$}
      \item Il était aussi possible de déterminer le sous-espace
        propre de $M$ associé à la valeur propre $0$.\\
        Soit $X =
        \begin{smatrix}
          x \\ 
          y \\
          z
        \end{smatrix} 
        \in \M{3,1}$.\\[-.1cm]
        \[
        \begin{array}{rcl}
          X\in E_{0}(M)
          & \Longleftrightarrow & M \, X = 0_{\M{3}}
          \\[.2cm]
          & \Longleftrightarrow & 
          \begin{smatrix}
            0 & 0 & 0 \\
            1 & 2 & -1 \\
            2 & 4 & -2
          \end{smatrix}
          \begin{smatrix}
            x \\
            y \\
            z
          \end{smatrix}
          =
          \begin{smatrix}
            0 \\
            0 \\
            0
          \end{smatrix}
          \\[.6cm]
          & \Longleftrightarrow & 
          \left\{
            \begin{array}{rcrcrcl}
              & & & & 0 & = & 0 \\
              x & + & 2 \, y & - & z & = & 0 \\
              2 \, x & + & 4 \, y & - & 2 \, z & = & 0
            \end{array}
          \right.
          \\[.6cm]
          & 
          \begin{arrayEq}
            L_3 \leftarrow L_3 - 2 \, L_2
          \end{arrayEq}
          & 
          \left\{
            \begin{array}{rcrcrcl}
              & & & & 0 & = & 0 \\
              x & + & 2 \, y & - & z & = & 0 \\
              & & & & 0 & = & 0 
            \end{array}
          \right.
          \\[.6cm]
          &
          \Longleftrightarrow
          &
          \left\{
            \begin{array}{rcrcr}
              x & = & -2 \, y & + & z \\
            \end{array}
          \right.
        \end{array}
        \]        
        Finalement on obtient l'expression de $E_0(M)$ suivante :\\[-.2cm]
        \[
        \begin{array}{rcl}
          E_{0}(M) & = & 
          \{ 
          X \in \M{3,1} \ | \ MX = 0_{\M{3}} \}
          \ = \ 
          \{
          \begin{smatrix}
            x \\ 
            y \\
            z
          \end{smatrix}
          \ | \
          x = - 2 y + z
          \}
          \\[.6cm]
          & = & 
          \{
          \begin{smatrix}
            - 2 y + z \\ 
            y \\ 
            z
          \end{smatrix}
          \ | \
          (y, z) \in \R^2
          \}
          \ = \ 
          \{
          y \cdot
          \begin{smatrix}
            -2 \\ 
            1 \\ 
            0
          \end{smatrix}
          +
          z \cdot
          \begin{smatrix}
            1 \\ 
            0 \\ 
            1
          \end{smatrix}
          \ | \ (y, z) \in \R^2
          \}
          \\[.6cm]
          & = & 
          \Vect{
            \begin{smatrix}
              -2 \\
              1 \\ 
              0
            \end{smatrix}
            ,
            \begin{smatrix}
              1 \\
              0 \\ 
              1
            \end{smatrix}
          }
        \end{array}
        \]

      \item Il faut s'habituer à déterminer les ensembles
        $E_{\lambda}(M)$ par lecture de la matrice $M - \lambda \, I_3$.\\
        Ici, on a $\lambda = 0$. On cherche donc les vecteurs $X =
        \begin{smatrix}
          x \\
          y \\
          z
        \end{smatrix}
        $ de $E_{0}(M) $ c'est-à-dire les vecteurs tels que : $M \, X
        = 0_{\M{3,1}}$. Or :\\[-.1cm]
        \[
        \begin{array}{rcl}
          \begin{smatrix}
            0 & 0 & 0 \\
            1 & 2 & -1 \\
            2 & 4 & -2
          \end{smatrix}
          \begin{smatrix}
            x \\
            y \\
            z 
          \end{smatrix}
          & = & x \cdot C_1 + y \cdot C_2 + z \cdot C_3
          \\[-.2cm]
          & = & 
          x \cdot
          \begin{smatrix}
            0 \\
            1 \\
            2
          \end{smatrix}
          + y \cdot
          \begin{smatrix}
            0 \\
            2 \\
            4 
          \end{smatrix}
          + z \cdot
          \begin{smatrix}
            0 \\
            -1 \\
            -2
          \end{smatrix}
        \end{array}
        \]~\\[-.6cm]
        Pour obtenir le vecteur $
        \begin{smatrix}
          0 \\
          0 \\
          0
        \end{smatrix}
        $ à l'aide de cette combinaison linéaire, plusieurs choix sont
        possibles. Plus précisément :
        \begin{noliste}{$\stimes$}
        \item si l'on choisit $y = 0$, il suffit de prendre $x = z$
          pour obtenir le vecteur nul.\\
          En prenant (par exemple) $z = 1$, on obtient : $x = 1$.
        \item si l'on choisit $z = 0$, il suffit de prendre $x = -2y$
          pour obtenir le vecteur nul.\\
          En prenant (par exemple) $y = 1$, on obtient : $y = -2$.
        \end{noliste}
        On obtient ainsi : $E_0(M) \supset \Vect{
            \begin{smatrix}
              -2 \\
              1 \\ 
              0
            \end{smatrix}
            ,
            \begin{smatrix}
              1 \\
              0 \\ 
              1
            \end{smatrix}
          }$.\\
          Et l'égalité est vérifiée car ces deux espaces vectoriels
          sont de même dimension.
        \end{noliste}      
      \end{remarkL}

  \item Soit $P =
    \begin{smatrix}
      0 & 1 & 0 \\
      1 & 0 & 0 \\
      0 & -2 & 1
    \end{smatrix}
    $. Justifier que la matrice $P$ est inversible et calculer le
    produit $P
    \begin{smatrix}
      0 \\
      1 \\
      2
    \end{smatrix}
    $.

    \begin{proof}~%
      \begin{noliste}{$\sbullet$}
      \item Tout d'abord :
        \[
        \rg(P) \ = \ \rg\left(
          \begin{smatrix}
            0 & 1 & 0 \\
            1 & 0 & 0 \\
            0 & -2 & 1
          \end{smatrix}
        \right)
        \begin{arrayEg}
          L_1 \leftrightarrow L_2
        \end{arrayEg}
        \rg\left(
          \begin{smatrix}
            1 & 0 & 0 \\
            0 & 1 & 0 \\
            0 & -2 & 1
          \end{smatrix}
        \right)
        \begin{arrayEg}
          L_3 \leftarrow L_3 + 2 \, L_2
        \end{arrayEg}
        \rg\left(
          \begin{smatrix}
            1 & 0 & 0 \\
            0 & 1 & 0 \\
            0 & 0 & 1
          \end{smatrix}
        \right) \ = \ 3
        \]
        \conc{On en conclut que la matrice $P$ est inversible.}

      \item Ensuite : 
        \[
        P
        \begin{smatrix}
          0 \\
          1 \\
          2
        \end{smatrix}
        \ = \ 
        \begin{smatrix}
          0 & 1 & 0 \\
          1 & 0 & 0 \\
          0 & -2 & 1
        \end{smatrix}
        \begin{smatrix}
          0 \\
          1 \\
          2
        \end{smatrix}
        \ = \ 
        \begin{smatrix}
          1 \\
          0 \\
          0
        \end{smatrix}
        \]        
      \end{noliste}
      \conc{$P
        \begin{smatrix}
          0 \\
          1 \\
          2
        \end{smatrix}
        \ = \ 
        \begin{smatrix}
          1 \\
          0 \\
          0
        \end{smatrix}$}~\\[-1.2cm]
    \end{proof}

  \item Trouver une matrice inversible $Q$ dont la transposée ${}^tQ$
    vérifie : ${}^tQ
    \begin{smatrix}
      1 \\
      2 \\
      -1
    \end{smatrix}
    =
    \begin{smatrix}
      1 \\
      0 \\
      0
    \end{smatrix}
    $.

    \begin{proof}~%
      \begin{noliste}{$\sbullet$}
      \item Notons ${}^tQ =
        \begin{smatrix}
          x & u & a \\
          y & v & b \\
          z & w & c
        \end{smatrix}
        $ où $(x, y, z, u, v, w, a, b, c) \in \R^9$. Remarquons tout
        d'abord :
        \[
        {}^tQ
        \begin{smatrix}
          1 \\
          2 \\
          -1
        \end{smatrix}
        \ = \ 
        \begin{smatrix}
          x & u & a \\
          y & v & b \\
          z & w & c
        \end{smatrix}
        \begin{smatrix}
          1 \\
          2 \\
          -1
        \end{smatrix}
        \ = \ 1 \cdot
        \begin{smatrix}
          x \\
          y \\
          z
        \end{smatrix}
        + 2 \cdot 
        \begin{smatrix}
          u \\
          v \\
          w
        \end{smatrix}
        - 1 \cdot 
        \begin{smatrix}
          a \\
          b \\
          c
        \end{smatrix}
        \]

      \item Pour obtenir le vecteur $
        \begin{smatrix}
          1 \\
          0 \\
          0
        \end{smatrix}
        $ à l'aide de cette combinaison linéaire, plusieurs choix sont
        possibles. Plus précisément :
        \begin{noliste}{$\stimes$}
        \item pour obtenir : $x + 2 \, u - a = 1$, on peut prendre $x
          = 1$ et $u = a = 0$.

        \item pour obtenir : $y + 2 \, v - b = 0$, on peut prendre $y
          = 2$ et $v = -1$ et $b = 0$.

        \item pour obtenir : $z + 2 \, w - c = 0$, on peut prendre $z
          = -1$ et $w = 1$ et $c = 1$.      
        \end{noliste}
        On construit ainsi la matrice ${}^tQ =
        \begin{smatrix}
          1 & 0 & 0 \\
          2 & -1 & 0 \\
          -1 & 1 & 1 
        \end{smatrix}$ et donc $Q =         
        \begin{smatrix}
          1 & 2 & -1 \\
          0 & -1 & 1 \\
          0 & 0 & 1
        \end{smatrix}$.\\
        Cette matrice est triangulaire supérieure et à coefficients
        diagonaux tous non nuls.\\
        Elle est donc inversible.
      \end{noliste}
      \conc{La matrice $Q = 
        \begin{smatrix}
          1 & 2 & -1 \\
          0 & -1 & 1 \\
          0 & 0 & 1
        \end{smatrix}$ est inversible et vérifie ${}^tQ
        \begin{smatrix}
          1 \\
          2 \\
          -1
        \end{smatrix}
        =
        \begin{smatrix}
          1 \\
          0 \\
          0
        \end{smatrix}
        $.}


      \newpage


      \begin{remarkL}{.98}%
        \begin{noliste}{$\sbullet$}
        \item Remarquons tout d'abord, par propriété de l'application
          transposée :
          \[
          {}^tQ 
          \begin{smatrix}
            1 \\
            2 \\
            -1
          \end{smatrix}
          =
          \begin{smatrix}
            1 \\
            0 \\
            0
          \end{smatrix}          
          \quad \Leftrightarrow \quad
          \begin{smatrix}
            1 & 2 & -1
          \end{smatrix}
          \, Q =
          \begin{smatrix}
            1 & 0 & 0 
          \end{smatrix}
          \]
          L'introduction de la transposée a donc pour but ici de faire
          apparaître un calcul sur des lignes plutôt que sur des
          colonnes. Si on travaille directement sur l'égalité de
          droite, on obtient, avec les notations précédentes :
          \[
          \begin{smatrix}
            1 & 2 & -1
          \end{smatrix}
          \, Q \ = \ 1 \cdot 
          \begin{smatrix}
            x & y & z
          \end{smatrix}
          + 2 \cdot 
          \begin{smatrix}
            u & v & w
          \end{smatrix}
          - 1 \cdot 
          \begin{smatrix}
            a & b & c
          \end{smatrix}
          \]
          On obtient évidemment les mêmes équations que précdemment.

        \item On retiendra qu'en multipliant $Q \in \M{3}$ à droite
          (resp. gauche) par une matrice colonne (resp. ligne), on
          obtient une combinaison linéaire des colonnes (resp. lignes)
          de la matrice $Q$. On peut retenir l'idée développée dans le
          paragraphe par la forme :
          \[
          L \ A \ C
          \]
          qui signifie qu'avec une multiplication à gauche, on
          effectue une opération sur les (L)ignes, tandis qu'avec une
          multiplication à droite, on effectue une multiplication sur
          les (C)olonnes.

        \item D'autres choix étaients possibles pour la matrice
          $Q$. Par exemple, on pouvait choisir :
          \[
          Q =
          \begin{smatrix}
            1 & -1 & 1 \\
            0 & 1 & 0 \\
            0 & 1 & 1
          \end{smatrix}
          \quad \text{ ou encore } \quad 
          Q =
          \begin{smatrix}
            1 & -1 & 0 \\
            0 & 1 & 1 \\
            0 & 1 & 2
          \end{smatrix}
          \]
          Il est à noter qu'il ne suffit pas de résoudre les
          contraintes issues des équations pour exhiber une matrice
          $Q$ satisfaisante. Il est précisé dans l'énoncé que $Q$ est
          une matrice inversible. Cela explique la direction prise par
          la résolution proposée initialement : les choix effectués
          permettent de construire une matrice qui est visiblement
          inversible (triangulaire supérieure et à coefficients
          diagonaux tous non nuls).
        \end{noliste}
      \end{remarkL}~\\[-1.4cm]
    \end{proof}

  \item Pour une telle matrice $Q$, calculer le produit $P \, M \, Q$.

    \begin{proof}~%
      \[
      \begin{array}{rcl@{\qquad}>{\it}R{5cm}}
        P \, M \, Q & = & P \, (CL) \, Q
        \\[.2cm]
        & = & (PC) \, (LQ)
        \\[.2cm]
        & = &
        \begin{smatrix}
          1 \\
          0 \\
          0
        \end{smatrix}
        \, 
        \begin{smatrix}
          1 & 0 & 0
        \end{smatrix}
        & (d'après les calculs effectués en \itbf{1.b)} et en
        \itbf{1.c)})
        \nl
        \nl[-.2cm]
        & = &
        \begin{smatrix}
          1 & 0 & 0 \\
          0 & 0 & 0 \\
          0 & 0 & 0
        \end{smatrix}
        \ = \ E_{1,1}
      \end{array}      
      \]
      \conc{Ainsi : $P \, M \, Q = E_{1,1}$.}~\\[-1.2cm]
    \end{proof}
  \end{noliste}


  \newpage


\item La fonction \Scilab{} suivante permet de multiplier la $\eme{i}$
  ligne $L_i$ d'une matrice $A$ par une réel sans modifier ses autres
  lignes, c'est-à-dire de lui appliquer l'opération élémentaire $L_i
  \leftarrow a \, L_i$ (où $a \neq 0$).
  \begin{scilab}
    & \tcFun{function} \tcVar{B} = \underline{multilig}(\tcVar{a},
    \tcVar{i}, \tcVar{A}) \nl %
    & \qquad [n, p] = size(\tcVar{A}) \nl %
    & \qquad \tcVar{B} = \tcVar{A} \nl %
    & \qquad \tcFor{for} j = 1:p \nl %
    & \qquad \qquad \tcVar{B}(\tcVar{i}, j) = \tcVar{a} \Sfois{}
    \tcVar{B}(\tcVar{i}, j) \nl %
    & \qquad \tcFor{end} \nl %
    & \tcFun{endfunction}
  \end{scilab}
  \begin{remarkL}{.98}%
    \begin{noliste}{$\sbullet$}
    \item Le code de ce programme est plutôt simple à comprendre : 
      \begin{noliste}{$\stimes$}
      \item on crée une copie de la matrice {\tt A} que l'on stocke
        dans la variable {\tt B},
      \item on met à jour la $\eme{\text{\tt i}}$ ligne de {\tt B} en
        modifiant un par un les éléments de cette ligne à l'aide de la
        boucle {\tt for}.        
      \end{noliste}

    \item Pour être plus proche de l'opération élémentaire $L_i
      \leftarrow a \, L_i$, on pouvait opter pour une présentation ne
      nécissitant pas l'utilisation de la boucle {\tt for} :\\[-.2cm]
      \begin{scilab}
        & \tcFun{function} \tcVar{B} = \underline{multilig}(\tcVar{a},
        \tcVar{i}, \tcVar{A}) \nl %
        & \qquad [n, p] = size(\tcVar{A}) \nl %
        & \qquad \tcVar{B} = \tcVar{A} \nl %
        & \qquad \tcVar{B}(\tcVar{i}, :) = \tcVar{a} .\Sfois{}
        \tcVar{B}(i, :) \nl %
        & \qquad \tcFor{end} \nl %
        & \tcFun{endfunction}
      \end{scilab}
      L'appel {\tt B(i, :)} permet d'accéder à la $\eme{\text{\tt i}}$
      ligne de {\tt B}. On peut modifier cette ligne en lui assignant
      une matrice ligne de même taille, ce qu'on fait ici.
    \end{noliste}
  \end{remarkL}

  \begin{noliste}{a)}
    \setlength{\itemsep}{2mm} %
  \item Donner le code \Scilab{} de deux fonctions {\tt adlig}
    (d'arguments {\tt b}, {\tt i}, {\tt j}, {\tt A}) et {\tt echlig}
    (d'arguments {\tt i}, {\tt j}, {\tt A}) permettant d'effectuer
    respectivement les autres opérations sur les lignes d'une matrice
    :
    \[
    Li \leftarrow L_i + b \, L_j \ (i \neq j) \quad \text{ et } \quad
    L_i \leftrightarrow L_j \ (i \neq j)
    \]

    \begin{proof}~%
      \begin{noliste}{$\sbullet$}
      \item On s'inspire de la fonction donnée pour créer {\tt adlig} :
        \begin{scilab}
          & \tcFun{function} \tcVar{B} =
          \underline{adlig}(\tcVar{b}, \tcVar{i}, \tcVar{j}, \tcVar{A}) \nl %
          & \qquad [n, p] = size(\tcVar{A}) \nl %
          & \qquad \tcVar{B} = \tcVar{A} \nl %
          & \qquad \tcFor{for} k = 1:p \nl %
          & \qquad \qquad \tcVar{B}(\tcVar{i}, k) =
          \tcVar{B}(\tcVar{i}, k) + \tcVar{b} \Sfois{}
          \tcVar{B}(\tcVar{j}, k) \nl %
          & \qquad \tcFor{end} \nl %
          & \tcFun{endfunction}
        \end{scilab}
        \begin{remark}%
          On note que la variable {\tt j} est ici une variable
          d'entrée du programme (on l'utilise pour désigner la ligne
          ajoutée dans l'opération élémentaire $Li \leftarrow L_i + b
          \, L_j \ (i \neq j)$). Cela oblige à renommer la variable
          d'itération du programme {\tt multilig}.
        \end{remark}


        \newpage


      \item La fonction {\tt echlig} est créée suivant le même
        principe :
        \begin{scilab}
          & \tcFun{function} \tcVar{B} =
          \underline{echlig}(\tcVar{i}, \tcVar{j}, \tcVar{A}) \nl %
          & \qquad [n, p] = size(\tcVar{A}) \nl %
          & \qquad \tcVar{B} = \tcVar{A} \nl %
          & \qquad \tcVar{aux} = 0 \nl %
          & \qquad \tcFor{for} k = 1:p \nl %
          & \qquad \qquad \tcVar{aux} = \tcVar{B}(\tcVar{i}, k) \nl %
          & \qquad \qquad \tcVar{B}(\tcVar{i}, k) =
          \tcVar{B}(\tcVar{j}, k) \nl % 
          & \qquad \qquad \tcVar{B}(\tcVar{j}, k) = aux \nl % 
          & \qquad \tcFor{end} \nl %
          & \tcFun{endfunction}
        \end{scilab}
      \end{noliste}
      \begin{remarkL}{.98}
        \begin{noliste}{$\sbullet$}
        \item On a introduit ici une variable auxiliaire appelée {\tt
            aux}. Le but de cette variable est de ne pas perdre
          d'information lors de l'échange des valeurs des deux
          coefficients de la même colonne. Plus précisément :
          \begin{noliste}{$\stimes$}
          \item l'instruction de la ligne \ligne{6} permet de stocker
            la valeur de {\tt B(i, k)}.
          \item en ligne \ligne{7}, on écrase la valeur du coefficient
            $B(i, k)$ en lui affectant la valeur $B(j, k)$.
          \item enfin, en ligne \ligne{8}, on affecte à {\tt B(i, k)}
            la valeur de {\tt aux}, c'est-à-dire la valeur {\bf
              initiale} (et pas la nouvelle valeur) du coefficient
            $B(i, k)$.
          \end{noliste}

        \item On pouvait aussi tirer parti du fait que l'on travaille
          sur une copie {\tt B} de la matrice {\tt A} d'entrée (jamais
          modifiée) pour ne pas introduire de variable auxiliaire {\tt
          aux}.\\[-.2cm]
          \begin{scilab}
            & \tcFun{function} \tcVar{B} =
            \underline{echlig}(\tcVar{i}, \tcVar{j}, \tcVar{A}) \nl %
            & \qquad [n, p] = size(\tcVar{A}) \nl %
            & \qquad \tcVar{B} = \tcVar{A} \nl %
            & \qquad \tcFor{for} k = 1:p \nl %
            & \qquad \qquad \tcVar{B}(\tcVar{i}, k) =
            \tcVar{A}(\tcVar{j}, k) \nl %
            & \qquad \qquad \tcVar{B}(\tcVar{j}, k) =
            \tcVar{A}(\tcVar{i}, k) \nl %
            & \qquad \tcFor{end} \nl %
            & \tcFun{endfunction}
          \end{scilab}
        \end{noliste}
      \end{remarkL}~\\[-1.5cm]
    \end{proof}

  \item Expliquer pourquoi la fonction {\tt multligmat} suivante
    retourne le même résultat {\tt B} que la fonction {\tt multlig}.
    \begin{scilab}
      & \tcFun{function} \tcVar{B} =
      \underline{multiligmat}(\tcVar{a}, \tcVar{i}, \tcVar{A}) \nl %
      & \qquad [n, p] = size(\tcVar{A}) \nl %
      & \qquad D = eye(n, n) \nl %
      & \qquad D(\tcVar{i}, \tcVar{i}) = a \nl %
      & \qquad \tcVar{B} = D \Sfois{} \tcVar{A} \nl %
      & \tcFun{endfunction}
    \end{scilab}    


    \newpage


    \begin{proof}~\\%
      Nommons $A$, $B$, $i$, $a$, $n$ et $p$ les éléments codés par
      les variables du programme correspondantes.
      \begin{noliste}{$\sbullet$}
      \item L'instruction en ligne \ligne{3} : {\tt D = eye(n, n)}
        permet de créer la matrice identité $I_{n}$.\\
        {\it (le nom {\tt eye} provient d'un jeu sur les sonorités :
          on crée à l'aide de cette instruction la matrice identité
          qui en anglais se dit \og identity matrix \fg{} qu'il faut
          lire {\tt eye}-dentity matrix)}

      \item L'instruction en ligne \ligne{4} : {\tt D(i, i) = a}
        permet de remplacer le coefficient $D_{i,i}$ par la valeur
        $a$. On crée ainsi une matrice $D$ diagonale carrée d'ordre
        $n$ dont :
        \begin{noliste}{$\stimes$}
        \item le $\eme{i}$ coefficient diagonal est la valeur $a$,
        \item les autres coefficients diagonaux ont tous la même
          valeur $1$.
        \end{noliste}

      \item L'instruction en ligne \ligne{5} : {\tt B = D \Sfois{} A}
        permet de stocker, dans la variable {\tt B}, le résultat de la
        multiplication $D \times A$. Détaillons ce calcul.\\[.2cm]
        Soit $(i', j) \in \llb 1, n \rrb \times \llb 1, p \rrb$. Par
        la formule de multiplication matricielle, on a :
        \[
        \begin{array}{rcl@{\qquad}>{\it}R{5cm}}
          B_{i', j} & = & \Sum{k = 1}{n} D_{i', k} \times A_{k, j}
          %\\[.2cm]
          \ = \ D_{i', i'} \times A_{i', j}
          & (car $D_{i', k} = 0$ si $k \neq i'$)
        \end{array}
        \]
        Deux cas se présentent alors :
        \begin{noliste}{$\stimes$}
        \item \dashuline{si $i' = i$} alors $D_{i', i'} = D_{i, i} =
          \text{a}$ et ainsi : $B_{i, j} = a \times A_{i, j}$.%
          \conc{La $\eme{i}$ ligne de $B$ est obtenue en multipliant
            la $\eme{i}$ ligne de $A$ par $a$.}
        \item \dashuline{si $i' \neq i$} alors $D_{i', i'} = 1$ et
          ainsi : $B_{i', j} = 1 \times A_{i', j} = A_{i', j}$.%
          \conc{Les autres lignes de $B$ sont des copies des lignes
            correspondantes de la matrice $A$.}
        \end{noliste}
        \concL{On en conclut que la fonction {\tt multiligmat} permet
          de calculer la matrice obtenue en appliquant à $A$
          l'opération élémentaire $L_i \leftarrow a \, L_i$. Cela
          correspond bien au calcul effectué par la fonction {\tt
            multilig}.}{12.4}
      \end{noliste}
      \begin{remark}%
        \begin{noliste}{$\sbullet$}
        \item La matrice $D \in \M{n}$ décrite dans cette question est
          une matrice {\bf de dilatation}. L'opération élémentaire
          $L_i \leftarrow a \, L_i$ (resp. $C_i \leftarrow a \, C_i$)
          se traduit par la multiplication matricielle à gauche
          (resp. à droite) de la matrice initiale $A$ par la matrice
          $D$.

        \item Illustrons de point par un exemple simple. Considérons
          $D =
          \begin{smatrix}
            1 & 0 & 0 \\
            0 & 1 & 0 \\
            0 & 0 & 3
          \end{smatrix}
          $. Alors :
          \[
          \begin{array}{C{1cm}rcl}
            & D \, M & = & 
            \begin{smatrix}
              1 & 0 & 0 \\
              0 & 1 & 0 \\
              0 & 0 & 5
            \end{smatrix}
            \begin{smatrix}
              0 & 0 & 0 \\
              1 & 2 & -1 \\
              2 & 4 & -2
            \end{smatrix}
            \ = \ 
            \begin{smatrix}
              0 & 0 & 0 \\
              1 & 2 & -1 \\
              10 & 20 & -10
            \end{smatrix}
            \\[.6cm]
            & & & \text{\it (on multiplie la $\eme{3}$ ligne par $5$)}
            \\[.2cm]
            et & M \, D & = & 
            \begin{smatrix}
              0 & 0 & 0 \\
              1 & 2 & -1 \\
              2 & 4 & -2
            \end{smatrix}
            \begin{smatrix}
              1 & 0 & 0 \\
              0 & 1 & 0 \\
              0 & 0 & 5
            \end{smatrix}
            \ = \ 
            \begin{smatrix}
              0 & 0 & 0 \\
              1 & 2 & -5 \\
              2 & 4 & -10
            \end{smatrix}
            \\[.6cm]
            & & & \text{\it (on multiplie la $\eme{3}$ colonne par $5$)}
          \end{array}
          \]          
        \end{noliste}
      \end{remark}~\\[-1.4cm]
    \end{proof}
  \end{noliste}


  \newpage


\item Dans cette question, on note $n$ un entier supérieur ou égal à
  $2$ et $M$ une matrice de $\M{n}$ de rang $1$. Pour tout couple $(i,
  j) \in \llb 1, n \rrb^2$, on note $E_{i, j}$ la matrice de $\M{n}$
  dont tous les coefficients sont nuls sauf celui situé à
  l'intersection de sa $\eme{i}$ ligne et de sa $\eme{j}$ colonne, et
  qui vaut $1$.
  \begin{noliste}{a)}
    \setlength{\itemsep}{2mm} %
  \item
    \begin{nonoliste}{(i)}
      \setlength{\itemsep}{2mm} %
    \item Justifier l'existence d'une matrice colonne non nulle $C =
      \begin{smatrix}
        c_1 \\
        \vdots \\
        c_n
      \end{smatrix}
      \in \M{n, 1}$ et d'une matrice ligne non nulle $
      L_1 = 
      \begin{smatrix}
        \ell_1 & \ldots & \ell_n
      \end{smatrix}
      \in \M{1, n}
      $ telles que $M = C \, L$.

      \begin{proof}~%
        \begin{noliste}{$\sbullet$}
        \item Remarquons tout d'abord :
          \[
          \begin{array}{rcl@{\qquad}>{\it}R{4cm}}
            C \, L & = &
            \begin{smatrix}
              c_{1} \, \ell_1 & c_{1} \, \ell_2 & \ldots & c_{1} \,
              \ell_{n-1} & c_{1} \ell_n
              \\
              c_{2} \, \ell_1 & c_{2} \, \ell_2 & \ldots & c_{2} \,
              \ell_{n-1} & c_{2} \ell_n
              \\
              \vdots & \vdots & & \vdots & \vdots
              \\
              c_{n-1} \, \ell_1 & c_{n-1} \, \ell_2 & \ldots & c_{n-1}
              \, \ell_{n-1} & c_{n-1} \, \ell_n
              \\
              c_{n} \, \ell_1 & c_{n} \, \ell_2 & \ldots & c_{n} \,
              \ell_{n-1} & c_{n} \ell_n
            \end{smatrix}
            \\[1.2cm]
            & = & 
            \left( 
              \ell_1 \, 
              \begin{smatrix}
                c_{1} \\
                c_2 \\
                \vdots \\
                c_{n-1} \\
                c_{n}
              \end{smatrix}
              \quad
              \ell_2 \, 
              \begin{smatrix}
                c_{1} \\
                c_2 \\
                \vdots \\
                c_{n-1} \\
                c_{n}
              \end{smatrix}
              \quad \ldots \quad
              \ell_{n-1} \, 
              \begin{smatrix}
                c_{1} \\
                c_2 \\
                \vdots \\
                c_{n-1} \\
                c_{n}
              \end{smatrix}
              \quad
              \ell_{n} \, 
              \begin{smatrix}
                c_{1} \\
                c_2 \\
                \vdots \\
                c_{n-1} \\
                c_{n}
              \end{smatrix}
            \right)  
            \ = \ 
            \begin{smatrix}
              \ell_1 \, C & \ldots & \ell_n \, C
            \end{smatrix}
          \end{array}      
          \]
          Il s'agit donc de démontrer que toute matrice de rang $1$
          apparaît comme concaténation de colonnes colinéaires à une
          matrice colonne $C$ non nulle.\\
          Ce résultat se montre en deux étapes.

        \item Tout d'abord, comme la matrice $M$ est de rang $1$ elle
          est forcément non nulle (si c'était le cas, elle serait de
          rang $0$). Ainsi, $M$ admet (au moins) une colonne non
          nulle. On note $C$ la première colonne non nulle de $M$.

        \item Démontrons maintenant que toutes les colonnes de $M$
          sont colinéaires à $C$.\\
          On procède par l'absurde.\\[.1cm]
          Supposons que le matrice $M$ possède une colonne non
          colinéaire à $C$.\\
          Notons $k \in \llb 1, n \rrb$ l'indice de cette
          colonne. Alors :
          \[
          \begin{array}{rcl@{\qquad}>{\it}R{4cm}}
            \rg( M ) & = & \rg\big( C_1(M), \ldots, C_n(M) \big)
            \\[.2cm]
            & \geq & \rg\big( C_{k}(M), C \big)
            & (car $C$ et $C_k(M)$ sont des colonnes de $M$)
          \end{array}
          \]
          La famille $\big( C_{k}(M), C \big)$ est libre car
          constituée de deux vecteurs non colinéaires.\\
          On en déduit :
          \[
          \rg( M ) \geq \rg\big( C_{k}(M), C \big) = 2
          \]
          Absurde ! %
          \conc{Ainsi, toute colonne de $M$ est colinéaire à $C$} %

        \item Il existe donc un $n$-uplet $(\ell_1, \ldots, \ell_n)
          \in \R^n$ tel que :
          \[
          M \ = \ 
          \begin{smatrix}
            \ell_1 \, C & \ldots & \ell_n \, C
          \end{smatrix}
          \]
          Notons que ce $n$-uplet est forcément différent du $n$-uplet
          $(0, \ldots, 0)$ (si c'était le cas, la matrice $M$ serait
          nulle).
        \end{noliste}~\\[-1cm]
        \concL{Ainsi, si $M \in \M{n}$ est de rang $1$, il existe un
          matrice colonne $C \in \M{n, 1}$ et une matrice ligne non
          nulle $L \in \M{1, n}$ telles que : $M = C \,
          L$.}{14.4}~\\[-1cm]
        % \begin{remarkL}{.98}
        %   Les énoncés de type {\tt HEC} / {\tt ESSEC} se distinguent
        %   des énoncés {\tt EML} / {\tt EDHEC} par un découpage plus
        %   faible des questions qui oblige à prendre plus
        %   d'initiatives. Ici, la formulation de la question \og En
        %   déduire que \ldots \fg{} doit aider à comprendre qu'il
        %   s'agit de se servir du résultat précédent. En question
        %   précédente, on exhibe $(n-1)$ vecteurs de $\im(f)$. Il
        %   s'agit alors de tester si la famille constituée de ces
        %   vecteurs est une base de $\im(f)$.
        % \end{remarkL}~\\[-1.4cm]        
      \end{proof}


      \newpage


    \item Calculer la matrice $M \, C$ et en déduire une valeur propre
      de $M$.

      \begin{proof}~%
        \[
        \begin{array}{rcl@{\qquad}>{\it}R{5cm}}
          M \times C & = & (C \, L) \times C
          \\[.4cm]
          & = & C \times (L \, C)
          & (par associativité)
          \nl
          \nl[-.2cm]
          & = & (LC) \cdot C 
          & (car comme $L \in \M{1, 3}$ et $C \in \M{3, 1}$, $LC$ est
          un réel)        
          \nl
          & = & \left( \Sum{i=1}{n} \ell_i \, c_i \right) \cdot C
        \end{array}
        \]
        \concL{Comme $C \neq 0_{\M{n,1}}$, le vecteur $C$ est un
          vecteur propre de $M$ associé à la valeur propre
          $\Sum{i=1}{n} \ell_i \, c_i$.}{12.4}~\\[-1cm]
      \end{proof}

    \item Montrer que si le réel $\Sum{i=1}{n} c_i \, \ell_i$ est
      différent de $0$, alors la matrice $M$ est diagonalisable.

      \begin{proof}~%
        \begin{noliste}{$\sbullet$}
        \item Par définition :
          \[
          \rg(M) = 1 \neq n = \dim\big( \M{n, 1} \big)
          \]
          En effet, il est précisé dans l'énoncé : $n \geq 2$. %
          \conc{On en déduit que $M$ n'est pas inversible. Ainsi, le
            réel $0$ est valeur propre de $M$.}

        \item Notons $\B$ la base canonique de $\R^n$. Considérons
          l'endomorphisme $f \in \LL{\R^n}$ dont la représentation
          dans la base $\B$ est $M$. Par le théorème du rang :
          \[
          \begin{array}{ccccc}
            \dim\big( \R^n \big) & = & \dim\big( \kr(f) \big) & + &
            \dim\big( \im(f) \big)
            \\[.2cm]
            \shortparallel & & \shortparallel & & \shortparallel
            \\[.2cm]
            n & & \dim\big(  E_0(f) \big) & & \rg(f)
            \\[.2cm]
            \shortparallel & & \shortparallel & & \shortparallel
            \\[.2cm]
            \dim\big( \M{n,1} \big) & = & \dim\big( E_0(M) \big) & + & 
            \rg(M)
          \end{array}
          \]
          \conc{Ainsi : $\dim\big( E_0(M) \big) = \dim\big( \M{n,1}
            \big) - \rg(M) = n - 1$.} %

        \item Notons $\alpha = \Sum{i=1}{n} c_i \, l_i$. On suppose
          $\alpha$ non nul.\\
          D'après la question précédente, $\alpha$ est valeur propre
          de $M$. On en déduit : $\dim\big( E_{\alpha}(M) \big) \geq
          1$. Ainsi :
          \[
          \dim\big( E_{0}(M) \big) + \dim\big( E_{\alpha}(M) \big)
          \geq (n-1) + 1 \geq n
          \]
          Et comme on a forcément : $\dim\big( E_{0}(M) \big) +
          \dim\big( E_{\alpha}(M) \big) \leq n = \dim\big( \M{n,1}
          \big)$, on en conclut :
          \[
          \dim\big( E_{0}(M) \big) + \dim\big( E_{\alpha}(M) \big) = n
          \]
        \end{noliste}
        \conc{Ainsi, si $\alpha \neq 0$, la matrice $M$ possède deux
          valeurs propres distinctes $0$ et $\alpha$.\\
          Comme : $\dim\big( E_{0}(M) \big) + \dim\big( E_{\alpha}(M)
          \big) = \dim\big( \M{n,1} \big)$, la matrice $M$ est
          diagonalisable.}~\\[-1cm]
      \end{proof}
    \end{nonoliste}


    \newpage


  \item
    \begin{nonoliste}{(i)}
      \setlength{\itemsep}{2mm} %
    \item À l'aide de l'égalité $M = C \, L$, établir l'existence de
      deux matrices inversibles $P$ et $Q$ telles que $P \, M \, Q =
      E_{1, 1}$.

      \begin{proof}~%
        \begin{noliste}{$\sbullet$}
        \item D'après un calcul analogue à celui fait en
          \itbf{3.a)(i)} :$
          \begin{smatrix}
            1 \\
            0 \\
            \vdots \\
            0
          \end{smatrix}
          \begin{smatrix}
            1 & 0 & \ldots & 0 
          \end{smatrix}
          \ = \ E_{1, 1}
          $.

        \item Pour résoudre la question, il suffit donc de trouver
          deux matrices inversibles $P$ et $Q$ de $\M{n}$ telles que :
          \[
          P \, C \ = \           
          \begin{smatrix}
            1 \\
            0 \\
            \vdots \\
            0
          \end{smatrix}
          \qquad \text{ et } \qquad
          L \, Q \ = \           
          \begin{smatrix}
            1 & 0 & \ldots & 0 
          \end{smatrix}
          \]
          En effet, si c'est le cas, on a :
          \[
          \begin{array}{R{2cm}rcl@{\qquad}>{\it}R{5cm}}
            & P \, M \, Q & = & P \, (CL) \, Q
            \\[.2cm]
            & & = & (PC) \, (LQ)
            & (par associativité)
            \nl
            \nl[-.2cm]
            & & = & 
            \begin{smatrix}
              1 \\
              0 \\
              \vdots \\
              0
            \end{smatrix}
            \begin{smatrix}
              1 & 0 & \ldots & 0 
            \end{smatrix}
            \ = \ E_{1, 1}
          \end{array}
          \]

        \item Il reste à démontrer l'existence des matrices $P$ et
          $Q$. Remarquons dans un premier temps :
          \[
          \begin{array}{ccccR{8cm}}
            PC =         
            \begin{smatrix}
              1 \\
              0 \\
              \vdots \\
              0
            \end{smatrix}
            & \Leftrightarrow & 
            P^{-1}           
            \begin{smatrix}
              1 \\
              0 \\
              \vdots \\
              0
            \end{smatrix}
            = C
            & \Leftrightarrow &
            La première colonne de $P^{-1}$ est le vecteur $C$
          \end{array}
          \]
          Il s'agit donc de construire une matrice {\bf inversible}
          dont la première colonne est $C$. \\
          La première colonne étant fixée, il reste à choisir les
          suivantes.

        \item Construisons une telle matrice. Deux cas se présentent :
        \end{noliste}
          \begin{liste}{$\stimes$}
          \item \dashuline{si $c_1 \neq 0$}, on pose :
            \[
            P^{-1} \ = \
            \begin{smatrix}
              c_1 & 0 & \ldots & \ldots & 0 \\
              c_2 & 1 & \ddots & & \vdots \\
              \vdots & 0 & \ddots & \ddots & \vdots \\
              \vdots & \vdots & \ddots & \ddots & 0 \\
              c_2 & 0 & \ldots & 0 & 1
            \end{smatrix}
            \]
            Cette matrice est bien inversible car elle est
            trinagulaire inférieure et de coefficients diagonaux tous
            non nuls.
            
          \item \dashuline{si $c_1 = 0$}, on note $i_0$ l'indice du
            premier coefficient non nul du vecteur $C$. On pose alors
            :
            \[
            P^{-1} \ = \
            \begin{smatrix}
              0 & 0 & \ldots & 0 & 1 & 0 & \ldots & 0 \\
              \vdots & 1 & \ddots & \vdots & 0 & \vdots & & \vdots \\
              \vdots & 0 & \ddots & 0 & \vdots & \vdots & & \vdots \\
              0 & \vdots & \ddots & 1 & \vdots & \vdots & & \vdots \\ 
              c_{i_0} & \vdots & & \ddots & 0 & 0 & & \vdots \\
              \vdots & \vdots & & & \ddots & 1 & \ddots & \vdots \\
              \vdots & \vdots & & & & \ddots & \ddots & 0 \\
              c_n & 0 & \ldots & \ldots & \ldots & \ldots & 0 & 1 
            \end{smatrix}
            \]
            Démontrons que cette matrice est inversible :
            \[
            \rg\left( P^{-1} \right)
            \ 
            \begin{arrayEg}
              C_1 \leftrightarrow C_{i_0}
            \end{arrayEg}
            \ 
            \rg\left( 
              \begin{smatrix}
                1 & 0 & \ldots & \ldots & \ldots & \ldots & & 0 \\
                0 & \ddots & \ddots & & & & & \vdots \\
                \vdots & \ddots & 1 & \ddots & & & & \vdots \\
                \vdots & & 0 & c_{i_0} & \ddots & & & \vdots \\
                \vdots & & \vdots & \vdots & 1 & \ddots & & \vdots \\
                \vdots & & \vdots & \vdots & 0 & \ddots & \ddots & \vdots \\
                \vdots & & \vdots & \vdots & \vdots & \ddots & \ddots & 0 \\
                0 & \ldots & 0 & c_n & 0 & \ldots & 0 & 1 
              \end{smatrix}
            \right)
            \]
            La réduite obtenue est triangulaire inférieure et à
            coefficients diagonaux tous non nuls.\\
            Elle est donc inversible et il en est de même de la
            matrice initiale $P^{-1}$.
          \end{liste}
          \begin{noliste}{$\sbullet$}
          \item Il reste alors à construire $Q$. Remarquons : 
          \[
          \begin{array}{ccccccR{3.8cm}}
            LQ =         
            \begin{smatrix}
              1 & 0 & \ldots & 0 
            \end{smatrix}
            & \Leftrightarrow & 
            {}^tQ \ {}^t L = 
            \begin{smatrix}
              1 \\
              0 \\
              \vdots \\
              0
            \end{smatrix}
            & \Leftrightarrow &
            \big({}^tQ \big)^{-1}
            \begin{smatrix}
              1 \\
              0 \\
              \vdots \\
              0
            \end{smatrix}
            = {}^tL
            & \Leftrightarrow &
            La première colonne de $\big({}^tQ \big)^{-1} = {}^t\big(Q
            \big)^{-1}$ est le vecteur ${}^tL$ 
          \end{array}
          \]       
          On construit alors ${}^t\big(Q \big)^{-1}$ par la méthode
          ayant permis la construction de $P^{-1}$. 
        \end{noliste}
        \conc{On a bien démontré l'existence de deux matrices
          inversibles $P$ et $Q$ telles que $P \, M \, Q =
          E_{1,1}$.}~\\[-1cm] 
      \end{proof}
      \begin{remarkL}{1.05}%
        Pour construire $P^{-1}$, on peut aussi faire appel au
        théorème dit de la base incomplète qui stipule :
        \[
        \Boxed{
          \begin{array}{C{12cm}}
            Toute famille libre d'un espace vectoriel $E$ de dimension
            finie $n$ peut être complétée en une base de $E$ 
          \end{array}
        }
        \]
        Ici, comme le vecteur $C$ est non nul, la famille $(C)$ est
        une famille libre de $\M{n, 1}$. \\
        On peut donc compléter cette famille en une base de $\M{n,
          1}$. Autrement dit, il existe des vecteurs colonnes $C_2$,
        \ldots $C_n$ tels que la famille $\big( C, C_2, \ldots, C_n
        \big)$ est une base de $\M{n,1}$.\\
        On peut alors construire la matrice $P^{-1}$ par concaténation
        de ces vecteurs :
        \[
        P^{-1} \ = \
        \begin{smatrix}
          C & C_2 & \ldots & C_n
        \end{smatrix}
        \]
        Cette matrice est bien inversible. En effet :
        \[
        \rg\left(
          \begin{smatrix}
            C & C_2 & \ldots & C_n
          \end{smatrix}
        \right) \ = \ \rg \big( C, C_2, \ldots, C_n \big) \ = \ n
        \quad \text{\it (car $\big( C, C_2, \ldots, C_n \big)$ est
          une famille libre)}
        \]
      \end{remarkL}


      \newpage


    \item En déduire que pour tout couple $(i, j) \in \llb 1,
      n\rrb^2$, il existe deux matrices inversibles $P_i$ et $Q_j$
      telles que $P_i \, M \, Q_j = E_{i, j}$.

      \begin{proof}~\\%
        Soit $(i, j) \in \llb 1, n\rrb^2$. On raisonne comme
        précédemment.
        \begin{noliste}{$\sbullet$}
        \item Remarquons tout d'abord que $E_{i,j}$ peut s'obtenir
          comme produit :
        \end{noliste}
        \begin{liste}{$\stimes$}
        \item de la matrice colonne contenant uniquement des $0$
          sauf en ligne $i$ où il contient un $1$.
        \item de la matrice ligne contenant uniquement des $0$ sauf
          en colonne $j$ où il contient un $1$.
        \end{liste}
        \begin{noliste}{$\sbullet$}
        \item Pour résoudre la question, il suffit donc de trouver
          deux matrices inversibles $P$ et $Q$ de $\M{n}$ telles que :
          \[
          P_i \, C \ = \
          \begin{smatrix}
            0 \\
            \vdots \\
            0 \\
            1 \\
            0 \\
            \vdots \\
            0
          \end{smatrix}
          \qquad \text{ et } \qquad L \, Q_j \ = \
          \begin{smatrix}
            0 & \ldots & 0 & 1 & 0 & \ldots & 0
          \end{smatrix}
          \]
        
        \item Pour un raisonnement analogue à celui de la question
          précédente, il suffit alors de trouver :
        \end{noliste}
        \begin{liste}{$\stimes$}
        \item une matrice $P_i^{-1}$ inversible dont la $\eme{i}$
          colonne est $C$.

        \item une matrice ${}^t \big( Q_j \big)^{-1}$ dont la $\eme{j}$
          colonne est ${}^tL$.         
        \end{liste}
        On obtient ces deux matrices par une construction similaire à
        la précédente. %
        \concL{Ainsi, pour tout couple $(i, j) \in \llb 1, n\rrb^2$, il
          existe deux matrices inversibles $P_i$ et $Q_j$ telles que
          $P_i \, M \, Q_j = E_{i, j}$.}{15.4}~\\[-1.2cm]
      \end{proof}
    \end{nonoliste}
    
  \end{noliste}
\end{noliste}


\newpage


\section*{Problème}

\noindent %
{\it Dans ce problème, on définit et on étudie les fonctions
  génératrices des cumulants de variables aléatoires discrètes ou à
  densité.\\
  Les cumulants d'ordre $3$ et $4$ permettent de définir des
  paramètres d'asymétrie et d'aplatissement qui viennent compléter la
  description usuelle d'une loi de probabilité par son espérance
  (paramètre de position) et sa variance (paramètre de dispersion) ;
  ces cumulants sont notamment utilisés pour l'évaluation des risques
  financiers.}\\[.2cm]
{\bf Dans tout le problème} :
\begin{noliste}{$\sbullet$}
\item on note $(\Omega, \A, \Prob)$ un espace probabilisé et toutes
  les variables aléatoires introduites dans l'énoncé sont des
  variables aléatoires réelles définies sur $(\Omega, \A)$ ;
  
\item sous réserve d'existence, l'espérance et la variance d'une
  variable aléatoire $X$ sont respectivement notées $\E(X)$ et $\V(X)$
  ;
  
\item pour tout variable aléatoire $X$ et pour tout réel $t$ pour
  lesquels la variable aléatoire $\ee^{t \, X}$ admet une espérance,
  on pose :
  \[
    M_X(t) = \E\left(\ee^{t \, X}\right) \quad \text{et} \quad K_X(t)
    = \ln\big(M_X(t)\big) ;
  \]
  (les fonctions $M_X$ et $K_X$ sont respectivement appelées la {\it
    fonction génératrice des moments} et la {\it fonction génératrice
    des cumulants} de $X$)
  
\item lorsque, pour un entier $p \in \N^*$, la fonction $K_X$ est de
  classe $\Cont{p}$ sur un intervalle ouvert contenant l'origine, on
  appelle {\it cumulant d'ordre $p$ de $X$}, noté $Q_p(X)$, la valeur
  de la dérivée $\eme{p}$ de $K_X$ en $0$ :
  \[
    Q_p(X) \ = \ K_X^{(p)}(0).
  \]
\end{noliste}

\begin{remark}
  \begin{noliste}{$\sbullet$}
  \item La fonction génératrice des moments d'une \var $X$ est un
    objet classique en probabilités. Comme son nom l'indique, cette
    fonction permet de retrouver les moments de $X$ (sous réserve
    d'existence). Plus précisément, si la \var $X$ admet un moment
    d'ordre $n$, alors :
    \[
      \E(X^n) \ = \ M_X^{(n)}(0)
    \]
    
  \item On ne confondra pas cette fonction avec une autre fonction
    classique en probabilités : la fonction génératrice (des
    probabilités). Cette dernière n'est définie que pour des \var $X$
    à valeurs entières et positives par la formule :
    \[
      \forall s \in [0,1], \quad G_X(s) \ = \ \E(s^X) \ = \
      \Sum{k=0}{+\infty} s^k \, \Prob(\Ev{X = k})
    \]
    Cette fonction caractérise quant à elle, non pas les moments de
    $X$, mais sa loi.\\
    Plus précisément :
    \[
      \forall n \in \N, \quad \Prob(\Ev{X = n}) \ = \
      \dfrac{G_X^{(n)}(0)}{n!}
    \]
    
  \item L'énoncé propose de démontrer quelques propriétés de la
    fonction génératrice des moments pour des \var particulières. Par
    exemple, dans le cas d'une \var $X$ à valeurs dans $\llb -n, n \rrb$ :
    \begin{noliste}{$\stimes$}
    \item $\forall p \in \N^*$, $M_X^{(p)}(0) = \E(X^p)$,
      
    \item $\forall t \in \R$, $G_X(\ee^t) = \ee^{nt} \, M_X(t)$.\\
      {\it (lien entre la fonction génératrice et la fonction génératrice
      des moments)}
    \end{noliste}
  \end{noliste}
\end{remark}


\newpage


\subsection*{Partie I. Fonction génératrice des moments de variables
  aléatoires discrètes}

\noindent
{\it Dans toute cette partie} :
\begin{noliste}{$\sbullet$}
\item on note $n$ un entier supérieur ou égal à $2$ ;
  
\item toutes les variables aléatoires considérées sont discrètes à
  valeurs entières ;
  
\item on note $S$ une variable aléatoire à valeurs dans $\{-1,1\}$
  dont la loi est donnée par :
  \[
    \Prob(\Ev{S = -1}) \ = \ \Prob(\Ev{S = +1}) \ = \ \dfrac{1}{2}.
  \]
\end{noliste}


% \newpage


\begin{noliste}{1.}
  \setlength{\itemsep}{4mm}
\item Soit $X$ une variable aléatoire à valeurs dans $\llb -n, n\rrb$.
  \begin{noliste}{a)}
    \setlength{\itemsep}{2mm}
  \item Pour tout $t\in \R$, écrire $M_X(t)$ sous la forme d'une somme
    et en déduire que la fonction $M_X$ est de classe $\Cont{\infty}$
    sur $\R$.
    \begin{proof}~
      \begin{noliste}{$\sbullet$}
      \item Soit $t \in \R$.\\
        La \var $X$ est une \var finie. Ainsi, la \var $\ee^{t \,
          X}$ est également une \var finie. Elle admet donc des
        moments à tout ordre, en particulier une espérance.
        \conc{On en déduit que, pour tout $t \in \R$, le réel
          $M_X(t)$ est bien défini.}
        \begin{remark}
          On rappelle que, d'après l'énoncé : $X(\Omega) \subset \llb
          -n, n \rrb = \{k \ | \ k \in \llb -n, n \rrb\}$.\\
          On en déduit, pour tout $t \in \R$ :
          \[
            \left(\ee^{t \, X}\right)(\Omega) \ = \ \ee^{t \,
              X(\Omega)} \ = \ \{\ee^{t \, k} \ | \ k \in \llb -n, n
            \rrb \}
          \]
          On retrouve bien que la \var $\ee^{t \, X}$ est finie.
        \end{remark}
        
      \item Soit $t\in \R$. Comme $X$ est une \var discrète, par
        théorème de transfert :
        \[
          M_X(t) \ = \ \E\left(\ee^{t \, X}\right) \ = \ \Sum{k=-n}{n}
          \ee^{t \, k} \, \Prob(\Ev{X = k})
        \]
        \conc{$\forall t \in \R$, $M_X(t) = \Sum{k=-n}{n} \ee^{t \, k}
          \, \Prob(\Ev{X = k})$}
        
      \item La fonction $M_X$ est de classe $\Cont{\infty}$ sur $\R$
        en tant que somme, pour tout $k \in \llb -n,n \rrb$ des
        fonctions $t \mapsto \ee^{t \, k} \, \Prob(\Ev{X=k})$ de
        classe $\Cont{\infty}$ sur $\R$.
        \conc{La fonction $M_X$ est de classe $\Cont{\infty}$ sur
          $\R$.}
      \end{noliste}
      \begin{remark}
        Soit $k \in \llb -n, n \rrb$.\\
        Il faut noter que l'expression $\Prob(\Ev{X = k})$ est une
        constante par rapport à $t$. Ainsi, pour étudier la régularité de
        la fonction $t \mapsto \ee^{t \, k} \, \Prob(\Ev{X = k})$
        on peut se contenter d'étudier la régularité de la fonction $t \mapsto
        \ee^{t \, k}$ (qui, elle, est trivialement de classe
        $\Cont{\infty}$ sur $\R$).
      \end{remark}~\\[-1.4cm]
    \end{proof}


    \newpage
    
    
  \item Justifier pour tout $p \in \N^*$, l'égalité : $M_X^{(p)}(0) =
    \E(X^p)$.
    \begin{proof}~
      \begin{noliste}{$\sbullet$}
      \item D'après la question précédente, la fonction $M_X$ est de
        classe $\Cont{\infty}$, donc, pour tout $p \in \N^*$, la
        fonction $M_X^{(p)}$ existe.
        
      \item Commençons par déterminer, pour tout $p \in \N^*$,
        l'expression de $M_X^{(p)}$.\\
        Démontrons par récurrence : $\forall p \in \N^*$, $\PP{p}$\\ 
        où \quad $\PP{p}$ : $\forall t \in \R$, $M_X^{(p)}(t) =
        \Sum{k=-n}{n} k^p \, \ee^{k \, t} \, \Prob(\Ev{X = k})$.
        \begin{noliste}{\fitem}
        \item {\bf Initialisation} :\\
          D'après la question précédente :
          \[
            M_X : t \mapsto \Sum{k=-n}{n} \ee^{k \, t} \, \Prob(\Ev{X
              = k})
          \]
          Ainsi, pour tout $t \in \R$.
          \[
            M_X^{(1)}(t) \ = \ M_X'(t) \ = \ \Sum{k=-n}{n} k \, \ee^{k
              \, t} \, \Prob(\Ev{X = k})
          \]
          D'où $\PP{1}$.
          
        \item {\bf Hérédité} : soit $p \in \N^* $.\\
          Supposons $\PP{p}$ et démontrons $\PP{p+1}$ (\ie $\forall t
          \in \R$, $M_X^{(p+1)}(t) = \Sum{k=-n}{n} k^{p+1} \, \ee^{k
            \, t} \, \Prob(\Ev{X = k})$).\\
          Par hypothèse de récurrence :
          \[
            M_X^{(p)} : t \mapsto \Sum{k=-n}{n} k^p \, \ee^{k \, t} \,
            \Prob(\Ev{X = k})
          \]
          Ainsi, pour tout $t \in \R$ :
          \[
            M_X^{(p+1)}(t) \ = \ \left(M_X^{(p)}\right)'(t) \ = \
            \Sum{k=-n}{n} k^p \, \left(k \, \ee^{k \, t}\right) \,
            \Prob(\Ev{X = k}) \ = \ \Sum{k=-n}{n} k^{p+1} \, \ee^{k \,
              t} \, \Prob(\Ev{X = k})
          \]
          D'où $\PP{p+1}$.
        \end{noliste}
        Par principe de récurrence : $\forall t \in \R$, $M_X^{(p)}(t)
        = \Sum{k=-n}{n} k^p \, \ee^{k \, t} \, \Prob(\Ev{X = k})$.
        
      \item Soit $p \in \N^*$. On en déduit :
        \[
          M_X^{(p)}(0) \ = \ \Sum{k=-n}{n} k^p \, \ee^{k \times 0} \,
          \Prob(\Ev{X = k}) \ = \ \Sum{k=-n}{n} k^p \, \Prob(\Ev{X =
            k}) \ = \ \E(X^p)
        \]
        où la dernière égalité est obtenue par théorème de transfert.
        \conc{Finalement, pour tout $p \in \N^*$ : $M_X^{(p)}(0) =
          \E(X^p)$.}
      \end{noliste}


      \newpage

      
      \begin{remark}
        L'obtention de l'expression de $M_X^{(p)}$ s'effectue au
        brouillon :
        \begin{noliste}{\scriptsize 1)}
        \item on commence par déterminer les premières dérivées
          successives de la fonction $M_X$ :
          \[
            \begin{array}{lccl}
              \forall t \in \R, &  M_X'(t) & = & \Sum{k=-n}{n} k \,
              \ee^{k \, t} \, \Prob(\Ev{X = k})
              \\[.4cm]
              \forall t \in \R, &  M_X''(t) & = & \Sum{k=-n}{n} k \left(
              k \, \ee^{k \, t}\right) \, \Prob(\Ev{X = k}) \ = \
              \Sum{k=-n}{n} k^2 \ee^{k \, t} \, \Prob(\Ev{X = k})
              \\[.4cm]
              \forall t \in \R, &  M_X^{(3)}(t) & = & \Sum{k=-n}{n} k^2
              \left(k \, \ee^{k \, t}\right) \, \Prob(\Ev{X = k}) \ = \
              \Sum{k=-n}{n} k^3 \ee^{k \, t} \, \Prob(\Ev{X = k})
            \end{array}
          \]
          
        \item on en déduit une formule générale :
          \[
            \forall t \in \R, \quad
            M_X^{(p)}(t) \ = \ \Sum{k=-n}{n} k^p \, \ee^{k \, t} \,
            \Prob(\Ev{X = k})
          \]
        \end{noliste}
      \end{remark}~\\[-1.4cm]
    \end{proof}
    
  \item Soit $Y$ une variable aléatoire à valeurs dans $\llb -n,n
    \rrb$ dont la fonction génératrice des moments $M_Y$ est la même
    que celle de $X$.\\
    On note $G_X$ et $G_Y$ les deux polynômes définis par :
    \[
      \forall x \in \R, \ \left\{
        \begin{array}{l}
          G_X(x) \ = \ \Sum{k=0}{2n} \Prob(\Ev{X = k-n}) \, x^k
          \\[.4cm]
          G_Y(x) \ = \ \Sum{k=0}{2n} \Prob(\Ev{Y = k-n}) \, x^k
        \end{array}
      \right.
    \]
    \begin{nonoliste}{(i)}
      \setlength{\itemsep}{2mm}
    \item Vérifier pour tout $t\in \R$, l'égalité : $G_X(\ee^t) =
      \ee^{nt} \, M_X(t)$.
      \begin{proof}~\\
        Soit $t \in \R$.
        \[
          \begin{array}{rcl@{\quad}>{\it}R{4cm}}
            G_X\left(\ee^t\right)
            & = & \Sum{k=0}{2n} \Prob(\Ev{X = k-n}) \,
                  \left(\ee^t\right)^k
            \\[.6cm]
            & = & \Sum{k=0}{2n} \Prob(\Ev{X = k-n}) \, \ee^{k \, t}
            \\[.6cm]
            & = & \Sum{k=0-n}{2n-n} \Prob(\Ev{X = k}) \, \ee^{(k+n) \, t}
            & (par décalage d'indice)
            \nl
            \nl[-.2cm]
            & = & \Sum{k=-n}{n} \left(\Prob(\Ev{X = k}) \, \ee^{k \, t} \,
                  \ee^{n \, t}\right)
            \\[.6cm]
            & = & \ee^{n \, t} \ \Sum{k=-n}{n} \Prob(\Ev{X = k}) \,
                  \ee^{k \, t}
            \\[.6cm]
            & = & \ee^{n \, t} \, M_X(t)
          \end{array}
        \]
        \conc{$\forall t \in \R$, $G_X\left(\ee^t\right) = \ee^{nt} \,
          M_X(t)$}~\\[-1cm]
      \end{proof}


      \newpage
      
      
    \item Justifier la relation : $\forall t \in \R$, $G_X(\ee^t) =
      G_Y(\ee^t)$.
      \begin{proof}~\\
        Soit $t \in \R$.
        \[
          \begin{array}{rcl@{\quad}>{\it}R{5cm}}
            G_X\left(\ee^t\right)
            & = & \ee^{n \, t} \, M_X(t)
            & (d'après la question précédente)
            \nl
            \nl[-.2cm]
            & = & \ee^{n \, t} \, M_Y(t)
            & (car, d'après l'énoncé : $M_X = M_Y$)
            \nl
            \nl[-.2cm]
            & = & G_Y\left(\ee^t\right)
            & (avec le même raisonnement qu'en question précédente)
          \end{array}
        \]
        \conc{$\forall t \in \R$, $G_X\left(\ee^t\right) = G_Y\left(
              \ee^t\right)$}~\\[-1cm]
      \end{proof}
      
    \item En déduire que la variable aléatoire $Y$ suit la même loi
      que $X$.
      \begin{proof}~
        \begin{noliste}{$\sbullet$}
        \item Tout d'abord : $X(\Omega) \subset \llb -n,n \rrb$ et
          $Y(\Omega) \subset \llb -n,n \rrb$.
          
        \item Ensuite, d'après la question précédente :
          \[
            \forall t \in \R, \ G_X\left(\ee^t\right) =
            G_Y\left(\ee^t\right)
          \]
          Ainsi, pour tout $x \in \ ]0,+\infty[$ :
          \[
            \begin{array}{ccc}
              G_X\left(\ee^{\ln(x)}\right) & =
              & G_Y\left(\ee^{\ln(x)}\right)
              \\[.2cm]
              \shortparallel & & \shortparallel
              \\[.2cm]
              G_X(x) & & G_Y(x)
            \end{array}
          \]
          
        \item Soit $x \in \ ]0,+\infty[$, on en déduit :
          $(G_X-G_Y)(x) = 0$.\\
          Or :
          \[
            \begin{array}{rcl}
              (G_X-G_Y)(x)
              & = &  G_X(x) - G_Y(x)
              \\[.2cm]
              & = & \Sum{k=0}{2n} \Prob(\Ev{X = k-n}) \,
                    x^k - \Sum{k=0}{2n} \Prob(\Ev{Y = k-n}) \, x^k
              \\[.6cm]
              & = & \Sum{k=0}{2n} \left(\Prob(\Ev{X = k-n}) \, x^k -
                    \Prob(\Ev{Y = k-n}) \, x^k \right)
              \\[.6cm]
              & = & \Sum{k=0}{2n} \big( \Prob(\Ev{X = k-n}) -
                    \Prob(\Ev{Y = k-n})\big) \, x^k
            \end{array}
          \]
          Ainsi, le polynôme $G_X-G_Y$ admet une infinité de
          racines. C'est donc le polynôme nul, \ie :
          \[
            \forall k \in \llb 0,2n \rrb, \ \Prob(\Ev{X = k-n}) -
            \Prob(\Ev{Y = k-n}) =0
          \]
          D'où :
          \[
            \forall k \in \llb -n,n \rrb, \ \Prob(\Ev{X = k}) -
            \Prob(\Ev{Y = k}) =0
          \]
          Ainsi :
          \[
            \forall k \in \llb -n,n \rrb, \ \Prob(\Ev{X = k}) =
            \Prob(\Ev{Y = k})
          \]
          \conc{Finalement, les \var $X$ et $Y$ suivent la même
            loi.}~\\[-1.4cm]
        \end{noliste}
      \end{proof}
    \end{nonoliste}
  \end{noliste}


  \newpage
  
  
\item Dans cette question, on note $X_2$ une variable aléatoire qui
  suit la loi binomiale $\Bin{2}{\dfrac{1}{2}}$.\\
  On suppose que les variables aléatoires $X_2$ et $S$ sont
  indépendantes et on pose $Y_2 = S \, X_2$.
  \begin{noliste}{a)}
    \setlength{\itemsep}{2mm}
  \item
    \begin{nonoliste}{(i)}
      \setlength{\itemsep}{2mm}
    \item Préciser l'ensemble des valeurs possibles de la variable
      aléatoire $Y_2$.
      \begin{proof}~
        \begin{noliste}{$\sbullet$}
        \item Comme $X_2 \suit \Bin{2}{\dfrac{1}{2}}$, on a : $X_2(\Omega) =
          \llb 0,2 \rrb = \{0,1,2\}$.
          
        \item De plus : $S(\Omega) = \{-1,1\}$.
        \end{noliste}
        \conc{On en déduit : $Y_2(\Omega) = (S \, X_2)(\Omega) \subset
          \{-2,-1,0,1,2\} = \llb -2, 2\rrb$.}~\\[-1cm]
      \end{proof}
      
    \item Calculer les probabilités $\Prob(\Ev{Y_2 = y})$ attachées
      aux diverses valeurs possibles $y$ de $Y_2$.
      \begin{proof}~
        \begin{noliste}{$\sbullet$}
        \item Tout d'abord, on a l'égalité entre événements :
          \[
            \Ev{Y_2 = -2} \ = \ \Ev{S \, X_2 = -2} \ = \ \Ev{S = -1}
            \cap \Ev{X_2 = 2}
          \]
          On en déduit :
          \[
            \begin{array}{rcl@{\quad}>{\it}R{4cm}}
              \Prob(\Ev{Y_2 = -2})
              & = & \Prob(\Ev{S = -1} \cap \Ev{X_2 = 2})
              \\[.2cm]
              & = & \Prob(\Ev{S = -1}) \ \Prob(\Ev{X_2 = 2})
              & (car $S$ et $X_2$ sont indépendantes)
              \nl
              \nl[-.2cm]
              & = & \dfrac{1}{2} \times \dbinom{2}{2}
                    \left(\dfrac{1}{2}\right)^2 \
                    \left(\dfrac{1}{2}\right)^{2-2}
              \\[.6cm]
              & = & \dfrac{1}{2} \times 1 \times \dfrac{1}{4}
            \end{array}
          \]
          \conc{$\Prob(\Ev{Y_2 = -2}) = \dfrac{1}{8}$}
          
        \item Ensuite :
          \[
            \Ev{Y_2 = -1} \ = \ \Ev{S = -1} \cap \Ev{X_2 = 1}
          \]
          On obtient donc :
          \[
            \begin{array}{rcl@{\quad}>{\it}R{4cm}}
              \Prob(\Ev{Y_2 = -1})
              & = & \Prob(\Ev{S = -1} \cap \Ev{X_2 = 1})
              \\[.2cm]
              & = & \Prob(\Ev{S = -1}) \ \Prob(\Ev{X_2 = 1})
              & (car $S$ et $X_2$ sont indépendantes)
              \nl
              \nl[-.2cm]
              & = & \dfrac{1}{2} \times \dbinom{2}{1}
                    \left(\dfrac{1}{2}\right)^1 \
                    \left(\dfrac{1}{2}\right)^{2-1}
              \\[.6cm]
              & = & \dfrac{1}{2} \times 2 \times \dfrac{1}{4}
            \end{array}
          \]
          \conc{$\Prob(\Ev{Y_2 = -1}) = \dfrac{1}{4}$}


          \newpage

          
        \item De même :
          \[
            \Ev{Y_2 = 1} = \Ev{S = 1} \cap \Ev{X_2 = 1} \quad
            \text{et} \quad \Ev{Y_2 = 2} = \Ev{S = 1} \cap \Ev{X_2 =
              2}
          \]
          \conc{Ainsi : $\Prob(\Ev{Y_2 = 1}) = \dfrac{1}{4}$ \quad et
            \quad $\Prob(\Ev{Y_2 = 2}) = \dfrac{1}{8}$.}
          
        \item Enfin : $\Ev{Y_2 = 0} = \Ev{X_2 = 0}$. Ainsi :
          \[
            \Prob(\Ev{Y_2  = 0}) \ = \ \Prob(\Ev{X_2 = 0}) \ = \
            \dbinom{2}{0} \left(\dfrac{1}{2}\right)^0 \,
            \left(\dfrac{1}{2}\right)^2 \ = \ \dfrac{1}{4}
          \]
          \conc{$\Prob(\Ev{Y_2 = 0}) = \dfrac{1}{4}$}
        \end{noliste}
        \begin{remark}
          On pouvait aussi remarquer que la famille $(\Ev{Y_2 = k})_{k
            \in \llb -2, 2\rrb}$ est un système complet
          d'événements. D'où :
          \[
            \begin{array}{rcl}
              \Prob(\Ev{Y_2 = 0})
              & = & 1-\Big(\Prob(\Ev{Y_2 = -2}) +
                    \Prob(\Ev{Y_2 = -1}) + \Prob(\Ev{Y_2 = 1}) +
                    \Prob(\Ev{Y_2  = 2})\Big)
              \\[.4cm]
              & = & 1 - \left(\dfrac{1}{8} + \dfrac{1}{4} +
                    \dfrac{1}{4} + \dfrac{1}{8}\right) \ = \ \dfrac{1}{4}
            \end{array}
          \]
        \end{remark}~\\[-1.4cm]
      \end{proof}
    \end{nonoliste}
    
  \item Vérifier que la variable aléatoire $X_2 - (S+1)$ suit la même
    loi que $Y_2$.
    \begin{proof}~
      \begin{noliste}{$\sbullet$}
      \item Tout d'abord, comme $S(\Omega) = \{-1,1\}$, on a :
        $(S+1)(\Omega) = \{0,2\}$.\\
        De plus $X_2(\Omega) = \{0,1,2\}$. D'où, en notant $T_2 = X_2
        - (S+1)$ : $T_2(\Omega) \subset \{-2,-1,0,1,2\}$.
        \conc{$T_2(\Omega) \subset \llb -2,2 \rrb$ (on rappelle :
          $Y_2(\Omega) \subset \llb -2, 2 \rrb$ d'après \itbf{2.a)(i)})}
        
      \item Soit $k \in \llb -2,2 \rrb$.\\
        La famille $(\Ev{S = -1}, \Ev{S = 1})$ forme un système
        complet d'événements.\\
        Ainsi, par formule des probabilités totales :
        \[
          \begin{array}{cl@{\quad}>{\it}R{4cm}}
            & \Prob(\Ev{T_2 = k})
            \\[.4cm]
            = & \Prob(\Ev{T_2 = k} \cap \Ev{S = -1}) + \Prob(\Ev{T_2
                  = k} \cap \Ev{S = 1})
            \\[.4cm]
            = & \Prob(\Ev{X_2 - (S+1) = k} \cap \Ev{S = -1}) +
                  \Prob(\Ev{X_2 - (S+1) = k} \cap \Ev{S = 1})
            \\[.4cm]
            = & \Prob(\Ev{X_2 - (-1+1) = k} \cap \Ev{S = -1}) +
                  \Prob(\Ev{X_2 - (1+1) = k} \cap \Ev{S = 1})
            \\[.4cm]
            = & \Prob(\Ev{X_2 = k} \cap \Ev{S = -1}) +
                  \Prob(\Ev{X_2 = k+2} \cap \Ev{S = 1})
            \\[.4cm]
            = & \Prob(\Ev{X_2 = k}) \ \Prob(\Ev{S = -1}) +
                  \Prob(\Ev{X_2 = k+2}) \ \Prob(\Ev{S = 1})
            & (car $X_2$ et $S$ sont indépendantes)
            \nl
            \nl[-.2cm]
            = & \dfrac{1}{2} \ \Prob(\Ev{X_2 = k}) + \dfrac{1}{2} \
                \Prob(\Ev{X_2 = k+2})
          \end{array}
        \]


        \newpage

        
      \item On obtient ainsi :
        \[
          \begin{array}{rcl}
            \Prob(\Ev{T_2 = -2}) & = & \dfrac{1}{2} \ \Prob(\Ev{X_2 = -2})
            + \dfrac{1}{2} \ \Prob(\Ev{X_2 = 0}) \ = \ \bcancel{\dfrac{1}{2} \
            \Prob(\emptyset)} + \dfrac{1}{2} \times \dfrac{1}{4} \ = \
            \dfrac{1}{8} \ = \ \Prob(\Ev{Y_2 = -2})
            \\[.6cm]
            \Prob(\Ev{T_2 = -1}) & = & \dfrac{1}{2} \ \Prob(\Ev{X_2 = -1})
            + \dfrac{1}{2} \ \Prob(\Ev{X_2 = 1}) \ = \ \bcancel{\dfrac{1}{2} \
            \Prob(\emptyset)} + \dfrac{1}{2} \times \dfrac{1}{2} \ = \
            \dfrac{1}{4} \ = \ \Prob(\Ev{Y_2 = -1})
            \\[.6cm]
            \Prob(\Ev{T_2 = 0}) & = & \dfrac{1}{2} \ \Prob(\Ev{X_2 = 0})
            + \dfrac{1}{2} \ \Prob(\Ev{X_2 = 2}) \ = \ \dfrac{1}{2} \times
            \dfrac{1}{4} + \dfrac{1}{2} \times \dfrac{1}{4} \ = \
            \dfrac{1}{4} \ = \ \Prob(\Ev{Y_2 = 0})
            \\[.6cm]                          
            \Prob(\Ev{T_2 = 1}) & = & \dfrac{1}{2} \ \Prob(\Ev{X_2 = 1})
            + \dfrac{1}{2} \ \Prob(\Ev{X_2 = 3}) \ = \ \dfrac{1}{2} \times
            \dfrac{1}{2} + \bcancel{\dfrac{1}{2} \ \Prob(\emptyset)} \ = \
            \dfrac{1}{4} \ = \ \Prob(\Ev{Y_2 = 1})
            \\[.6cm]
            \Prob(\Ev{T_2 = 2}) & = & \dfrac{1}{2} \ \Prob(\Ev{X_2 = 2})
            + \dfrac{1}{2} \ \Prob(\Ev{X_2 = 4}) \ = \ \dfrac{1}{2} \times
            \dfrac{1}{4} + \bcancel{\dfrac{1}{2} \ \Prob(\emptyset)} \ = \
            \dfrac{1}{8} \ = \ \Prob(\Ev{Y_2 = 2})
          \end{array}
        \]
        \conc{On en déduit que $T_2 = X_2 - (S+1)$ et $Y_2$ suivent la
          même loi.}
      \end{noliste}
      \begin{remark}
        \begin{noliste}{$\sbullet$}
        \item Lors d'une rédaction classique de détermination de la loi
          d'une somme, après obtention de la relation :
          \[
            \Prob(\Ev{T_2 = k}) \ = \ \dfrac{1}{2} \ \Prob(\Ev{X_2 = k})
            + \dfrac{1}{2} \ \Prob(\Ev{X_2 = k+2})
          \]
          On cherche à déterminer si les événements $\Ev{X_2 = k}$ et/ou
          $\Ev{X_2 = k+2}$ sont l'événement impossible.
          
        \item Ici, cela varie suivant les valeurs de $k$. C'est pour
          cela qu'on effectue le calcul direct des probabilités
          $\Prob(\Ev{T_2 = k})$ après l'obtention de la formule précédente.
        \end{noliste}
      \end{remark}~\\[-1.4cm]
    \end{proof}
  \end{noliste}
  
\item Le script \Scilab{} suivant permet d'effectuer des simulations
  de la variable aléatoire $Y_2$ définie dans la question précédente.
  \begin{scilab}
    & n = 10 \nl %
    & X = grand(n,2,\ttq{}bin\ttq{},2,0.5) \nl %
    & B = grand(n,2,\ttq{}bin\ttq{},1,0.5) \nl %
    & S = 2 \Sfois{} B - ones(n,2) \nl %
    & Z1 = [S(1:n,1) .\Sfois{} X(1:n,1) , X(1:n,1) - S(1:n,1) -
    ones(n,1)] \nl %
    & Z2 = [S(1:n,1) .\Sfois{} X(1:n,1) , X(1:n,2) - S(1:n,2) -
    ones(n,1)]
  \end{scilab}
  \begin{noliste}{a)}
    \setlength{\itemsep}{2mm}
  \item Que contiennent les variables {\tt X} et {\tt S} après
    l'exécution des quatre premières instructions ?
    \begin{proof}~
      \begin{noliste}{$\sbullet$}
      \item
        % La variable {\tt X} est définie par la commande :
        % \begin{scilabC}{1}
        %   & X = grand(n,2,\ttq{}bin\ttq{},2,0.5)
        % \end{scilabC}
        L'instruction {\tt X = grand(n,2,\ttq{}bin\ttq{},2,0.5)}
        permet de stocker dans la variable {\tt X} une matrice
        à {\tt n} lignes
        et $2$ colonnes où chaque colonne contient une observation
        d'un {\tt n}-échantillon de loi $\Bin{2}{\dfrac{1}{2}}$,
        c'est-à-dire un {\tt n}-échantillon de la \var $X$. Plus précisément,
        la première colonne de {\tt X} contient une observation $(c_1,
        \ldots, c_n)$ d'un {\tt n}-échantillon $(C_1, \ldots, C_n)$ de
        $X$, et la deuxième colonne de {\tt X} contient une
        observation $(c_1', \ldots, c_n')$ d'un {\tt n}-échantillon
        $(c_1', \ldots, C_n')$ de {\tt X}.\\
        {\it (les \var $C_i$ et $C_i'$ sont indépendantes et ont même
          loi que la \var $X$)}
        

        \newpage
        
        
      \item
        % La variable {\tt B} est définie par la commande :
        % \begin{scilabC}{2}
        %   & B = grand(n,2,\ttq{}bin\ttq{},1,0.5)
        % \end{scilabC}
        L'instruction {\tt B = grand(n,2,\ttq{}bin\ttq{},1,0.5)}
        permet de stocker dans la variable {\tt B} une matrice à {\tt n}
        lignes et $2$ colonnes où la première colonne contient une
        observation $(b_1, \ldots, b_n)$ d'un {\tt n}-échantillon
        $(B_1, \ldots, B_n)$ de loi $\Bern{\dfrac{1}{2}}$, et la
        deuxième colonne contient une observation $(b_1', \ldots,
        b_n')$ d'un {\tt n}-échantillon $(B_1', \ldots, B_n')$ de loi
        $\Bern{\dfrac{1}{2}}$.\\
        {\it (les \var $B_i$ et $B_i'$ sont indépendantes et ont même
          loi $\Bern{\dfrac{1}{2}}$)}
        
      \item On commence par rappeler que l'instruction {\tt ones(n,2)}
        permet d'obtenir une matrice à {\tt n} lignes et $2$ colonnes
        dont tous les coefficients sont égaux à $1$.\\
        On en déduit que l'instruction {\tt S = 2 \Sfois{} B -
          ones(n,2)} permet de stocker dans la variable {\tt S} une
        matrice à {\tt n} lignes et $2$ colonnes.\\
        La première colonne contient l'observation
        $(s_1, \ldots, s_n) = (2 \, b_1 -1, \ldots, 2 \, b_n -1)$ du
        {\tt n}-échantillon $(2 \, B_1 - 1, \ldots, 2 \, B_n -1)$.\\
        La deuxième colonne contient l'observation $(s_1', \ldots,
        s_n') = (2 \, b_1' -1, \ldots, 2 \, b_n' -1)$ du {\tt
          n}-échantillon $(2 \, B_1' -1, \ldots, 2 \, B_n' -1)$.
        
      \item Les \var $2 \, B_i - 1$ et $2 \, B_i' -1$ sont
        indépendantes et de même loi. Déterminons cette loi : on note
        $B$ une \var de loi $\Bern{\dfrac{1}{2}}$ et on cherche la loi
        de $V = 2 \, B -1$.
        \begin{noliste}{$\stimes$}
        \item Tout d'abord, comme $B \suit \Bern{\dfrac{1}{2}}$, on a :
          $B(\Omega) = \{0,1\}$.\\
          On en déduit : $V(\Omega) = \{2 \times 0 -1, 2 \times 1 -
          1\} = \{-1,1\}$.
          
        \item De plus :
          \[
            \Ev{V = 1} \ = \ \Ev{2 \, B -1 = 1} \ = \ \Ev{2 \, B =
              2} \ = \ \Ev{B = 1}
          \]
          D'où : $\Prob(\Ev{V = 1}) = \Prob(\Ev{B = 1}) =
          \dfrac{1}{2}$
            
        \item Enfin, comme la famille $(\Ev{V = -1}, \Ev{V = 1})$
          est un système complet d'événements :
          \[
            \Prob(\Ev{V = -1}) \ = \ 1 - \Prob(\Ev{V = 1}) \ = \ 1 -
            \dfrac{1}{2} \ = \ \dfrac{1}{2}
          \]
        \end{noliste}
        On en déduit que la \var $V = 2 \, B -1$ suit la même loi que
        la \var $S$.
        
      \item La variable {\tt S} contient donc une
        matrice à {\tt n} lignes et $2$ colonnes, où
        la première colonne contient une observation
        $(s_1, \ldots, s_n)$ d'un
        {\tt n}-échantillon $(S_1, \ldots, S_n)$ de la \var $S$, et la
        deuxième colonne contient une observation $(s_1', \ldots,
        s_n')$ du {\tt n}-échantillon $(S_1', \ldots, S_n')$ de la
        \var $S$.\\
        {\it (les \var $S_i$ et $S_i'$ sont indépendantes et ont même
          loi que la \var $S$)}
      \end{noliste}
      \begin{remark}
        \begin{noliste}{$\sbullet$}
        \item Comme souvent dans les sujets HEC, l'une des difficultés
          provient de la prise d'initiatives nécessaires pour répondre à
          une question.
          
        \item Dans cette question par exemple, déterminer la loi de la
          \var $V = 2 \, B-1$ n'est pas difficile. C'est prendre
          l'initiative de déterminer cette loi qui constitue la difficulté.
        \end{noliste}
      \end{remark}~\\[-1.4cm]
    \end{proof}


    \newpage
    
    
  \item Expliquer pourquoi, après l'exécution des six instructions,
    chacun des coefficients des matrices {\tt Z1} et {\tt Z2} contient
    une simulation de la variable aléatoire $Y_2$.
    \begin{proof}~
      \begin{noliste}{$\sbullet$}
      \item L'instruction {\tt  Z1 = [S(1:n,1) .\Sfois{} X(1:n,1) ,
          X(1:n,1) - S(1:n,1) - ones(n,1)]} permet de stocker dans la
        variable {\tt Z1} une matrice à {\tt n} lignes et $2$
        colonnes.\\
        Le contenu de la première colonne est donné par la commande
        {\tt S(1:n,1) .\Sfois{} X(1:n,1)} et celui de la deuxième
        colonne par {\tt  X(1:n,1) - S(1:n,1) - ones(n,1)}.

      \item L'instruction {\tt S(1:n,1) .\Sfois{} X(1:n,1)}
        permet d'obtenir une matrice colonne à {\tt n} lignes
        contenant l'observation $(d_1, \ldots, d_n) =
        (s_1 \times c_1, \ldots , s_n \times c_n)$
        du {\tt n}-échantillon ($S_1 \, C_1, \ldots, S_n \ C_n)$.\\
        Or, les $S_i$ suivent la même loi que $S$ et les $C_i$ suivent
        la même loi que $X_2$. Ainsi, les $S_i \, C_i$ suivent la même
        loi que $Y_2$.\\
        La première colonne de {\tt Z1} contient donc une observation $(d_1,
        \ldots, d_n)$ d'un {\tt n}-échantillon $(D_1, \ldots, D_n)$ de
        $Y_2$.\\
        {\it (les \var $D_i$ sont indépendantes et ont même
          loi que la \var $Y_2$)}


        % \newpage


        
      \item L'instruction {\tt X(1:n,1) - S(1:n,1) - ones(n,1)}
        permet d'obtenir une matrice colonne à {\tt n} lignes
        contenant l'observation $(d_1', \ldots, d_n') =
        (c_1 - s_1 - 1, \ldots , c_n - s_n -1)$
        du {\tt n}-échantillon ($C_1 - S_1-1, \ldots, C_n- S_n -1)$.\\
        Or, les $S_i$ suivent la même loi que $S$ et les $C_i$ suivent
        la même loi que $X_2$. Ainsi, les $C_i-S_i-1$ suivent la même
        loi que $X_2 - S-1$. De plus, d'après la question \itbf{2.b)},
        la \var $X_2-S-1$ suit la même loi que $Y_2$.\\ 
        La deuxième colonne de {\tt Z1} contient donc une observation $(d_1',
        \ldots, d_n')$ d'un {\tt n}-échantillon $(D_1', \ldots, D_n')$ de
        $Y_2$.\\
        {\it (les \var $D_i'$ sont indépendantes et ont même
          loi que la \var $Y_2$)}
        
      \item De même, l'instruction {\tt  Z2 = [S(1:n,1) .\Sfois{} X(1:n,1) ,
          X(1:n,2) - S(1:n,2) - ones(n,1)]} permet de stocker dans la
        variable {\tt Z2} une matrice à {\tt n} lignes et $2$
        colonnes.\\
        Le contenu de la première colonne est donné par la commande
        {\tt S(1:n,1) .\Sfois{} X(1:n,1)} et celui de la deuxième
        colonne par {\tt  X(1:n,2) - S(1:n,2) - ones(n,1)}.
        
      \item La première colonne de {\tt Z2} est identique à celle de
        {\tt Z1}, donc la première colonne de {\tt Z2} contient
        l'observation $(d_1, \ldots, d_n)$ du {\tt n}-échantillon
        $(D_1, \ldots, D_n)$ de $Y_2$.
        
      \item L'instruction {\tt X(1:n,2) - S(1:n,2) - ones(n,1)}
        permet d'obtenir une matrice colonne à {\tt n} lignes
        contenant l'observation $(d_1'', \ldots, d_n'') =
        (c_1' - s_1' - 1, \ldots , c_n' - s_n' -1)$
        du {\tt n}-échantillon ($C_1' - S_1'-1, \ldots, C_n'- S_n' -1)$.\\
        Or, les $S_i'$ suivent la même loi que $S$ et les $C_i'$ suivent
        la même loi que $X_2$. Ainsi, les $C_i'-S_i'-1$ suivent la même
        loi que $X_2 - S-1$, donc que $Y_2$.\\ 
        La deuxième colonne de {\tt Z2} contient donc une observation $(d_1'',
        \ldots, d_n'')$ d'un {\tt n}-échantillon $(D_1'', \ldots, D_n'')$ de
        $Y_2$.\\
        {\it (les \var $D_i''$ sont indépendantes et ont même
          loi que la \var $Y_2$)}
      \end{noliste}
      \conc{Finalement, chacun des coefficients des matrices {\tt Z1}
        et {\tt Z2} contient une simulation de la \var $Y_2$.}
      \begin{remarkL}{.95}
        On aurait pu utiliser davantage les commandes Scilab{}. En
        effet, l'appel classique permettant d'extraire la première
        colonne de la matrice {\tt S} est plutôt {\tt S(:,1)}
        (que {\tt S(1:n,1)}).
      \end{remarkL}~\\[-1.4cm]
    \end{proof}


    \newpage
    
    
  \item On modifie la première ligne du script précédent en affectant
    à {\tt n} une valeur beaucoup plus grande que $10$ (par exemple,
    $100000$) et en lui adjoignant les deux instructions \ligne{7} et
    \ligne{8} suivantes :
    \begin{scilabC}{6}
      & p1 = length(find(Z1(1:n,1) == Z1(1:n,2))) / n \nl %
      & p2 = length(find(Z2(1:n,1) == Z2(1:n,2))) / n
    \end{scilabC}
    Quelles valeurs numériques approchées la loi faible des grands
    nombres permet-elle de fournir pour {\tt p1} et {\tt p2} après
    l'exécution des huit lignes du nouveau script ?


    % \newpage


    \noindent
    Dans le langage \Scilab{}, la fonction {\tt length} fournit la \og
    longueur \fg{} d'un vecteur ou d'une matrice et la fonction {\tt
      find} calcule les positions des coefficients d'une matrice pour
    lesquels une propriété est vraie, comme l'illustre le script
    suivant :
    \[
      \begin{console}
        \lInv{A = [1 ; 2 ; 0 ; 4]} \nl %
        \lInv{B = [2 ; 2 ; 4 ; 3]} \nl %
        \lInv{length(A)} \nl %
        \lDisp{\qquad ans = 4.} \nl %
        \lInv{length([A , B])} \nl %
        \lDisp{\qquad ans = 8.} \nl %
        \lInv{find(A < B)} \nl %
        \lDisp{\qquad ans = 1. 3. // \textit{car $1<2$ et $0<4$, alors
            que $2 \geq 2$ et $4 \geq 3$}}
      \end{console}
    \]


    % \newpage

    
    \begin{proof}~
      \begin{noliste}{$\sbullet$}
      \item Commençons par commenter l'instruction :
        \begin{scilabC}{6}
          & p1 = length(find(Z1(1:n,1) == Z1(1:n,2))) / n
        \end{scilabC}
        \begin{noliste}{$\stimes$}
        \item L'instruction {\tt find(Z1(1:n,1) == Z1(1:n,2))} permet
          d'obtenir une matrice ligne contenant les positions des
          coefficients des matrices {\tt Z1(1:n,1)} et {\tt Z1(1:n,2)}
          égaux. Autrement dit, on obtient une matrice ligne contenant
          les indices $i$ tels que $d_i = d_i'$.\\
          {\it (on rappelle que $(d_1, \ldots, d_n)$ est une
            observation d'un {\tt n}-échantillon $(D_1, \ldots, D_n)$
            de $S \, X_2$ et $(d_1', \ldots, d_n')$ est une
            observation d'un {\tt n}-échantillon $(D_1', \ldots, D_n')$
            de $X_2 - S-1$)}
          
        \item L'instruction {\tt length(find(Z1(1:n,1) == Z1(1:n,2)))}
          permet d'obtenir la longueur de la matrice
          précédente. Ainsi, on obtient le nombre de fois où $d_i =
          d_i'$, pour $i \in \llb 1,{\tt n} \rrb$.
          
        \item Enfin, on divise ce nombre par la taille {\tt n} de
          l'observation.\\
          Or, par loi faible des grands nombres (LfGN) :
          \[
            \dfrac{\text{nombre de fois où $d_i = d_i'$}}{\text{taille
                de l'observation}} \ \simeq \ \Prob(\Ev{S \, X_2 = X_2
              - S-1})
          \]
        \end{noliste}
        \conc{La variable {\tt p1} contient une valeur approchée de
          $\Prob(\Ev{S \, X_2 = X_2 -S-1})$.}
        
      \item Commentons ensuite l'instruction :
        \begin{scilabC}{7}
          & p2 = length(find(Z2(1:n,1) == Z2(1:n,2))) / n
        \end{scilabC}
        \begin{noliste}{$\stimes$}
        \item L'instruction {\tt find(Z2(1:n,1) == Z2(1:n,2))} permet
          d'obtenir une matrice ligne contenant les positions des
          coefficients des matrices {\tt Z1(2:n,1)} et {\tt Z1(2:n,2)}
          égaux. Autrement dit, on obtient une matrice ligne contenant
          les indices $i$ tels que $d_i = d_i''$.\\
          {\it (on rappelle que $(d_1, \ldots, d_n)$ est une
            observation d'un {\tt n}-échantillon $(D_1, \ldots, D_n)$
            de $S \, X_2$ et $(d_1'', \ldots, d_n'')$ est une
            observation d'un {\tt n}-échantillon $(D_1'', \ldots, D_n'')$
            de $X_2' - S-1$, où la \var $X_2'$ suit la même loi que $X_2$)}
          
        \item L'instruction {\tt length(find(Z2(1:n,1) == Z2(1:n,2)))}
          permet d'obtenir la longueur de la matrice
          précédente. Ainsi, on obtient le nombre de fois où $d_i =
          d_i''$, pour $i \in \llb 1,{\tt n} \rrb$.

          
          \newpage
          
          
        \item Enfin, on divise ce nombre par la taille {\tt n} de
          l'observation.\\
          Or, par loi faible des grands nombres (LfGN) :
          \[
            \dfrac{\text{nombre de fois où $d_i = d_i''$}}{\text{taille
                de l'observation}} \ \simeq \ \Prob(\Ev{S \, X_2 = X_2'
              - S-1})
          \]
        \end{noliste}
        \conc{La variable {\tt p2} contient une valeur approchée de
          $\Prob(\Ev{S \, X_2 = X_2' -S-1})$\\
          où la \var $X_2'$ suit la même loi que $X_2$.}
      \end{noliste}
      \begin{remarkL}{.98}
        \begin{noliste}{$\sbullet$}
        \item Le programme proposé par l'énoncé n'est ici rien d'autre
          qu'une illustration de l'idée naturelle pour obtenir une
          approximation de $\Prob(\Ev{S \, X_2 = X_2 - S-1})$ :
          \begin{noliste}{$\stimes$}
          \item simuler un grand nombre de fois ({\tt n} $= 100000$)
            les \var $S \, X_2$ et $X_2 - S-1$.\\
            Formellement, on souhaite obtenir une observation $(d_1,
            \ldots, d_n)$ d'un {\tt n}-échantillon
            $(D_1, \ldots, D_n)$ de la \var $S \, X_2$, et une
            observation $(d_1', \ldots, d_n')$ d'un {\tt
              n}-échantillon $(D_1', \ldots, D_n')$ de la \var $X_2 -S-1$. 
            
          \item de compter le nombre de fois où $d_i = d_i'$, pour $i
            \in \llb 1, {\tt n} \rrb$.
          \end{noliste}
          
        \item L'objectif de cette question \Scilab{} est de revenir sur
          un point important en probabilités :
          \[
            \mbox{$X$ et $Y$ ont même loi \quad
            \bcancel{\rule[-.2cm]{0cm}{.2cm} \!\!\! $\Rightarrow$ \!\!\!
            \rule[-.2cm]{0cm}{.2cm}} \quad $X=Y$}
          \]
          (mais bien sûr, si $X=Y$, alors $X$ et $Y$ ont même loi)
          
        \item On peut avoir la confirmation du point précédent en
          exécutant le programme \Scilab{}.\\
          On obtient :
          \[
            \begin{console}
              \lDisp{\qquad p1 =} \nl %
              \lDisp{\qquad \qquad 0.1261} \nl %
              \lDisp{\qquad p2 =} \nl %
              \lDisp{\qquad \qquad 0.2179}
            \end{console}
          \]
          On constate qu'on a ${\tt p1} \neq 1$ et ${\tt p2} \neq
          1$. Ainsi, ici :
          \begin{noliste}{$\stimes$}
          \item les \var $S \, X_2$ et $X_2 -S-1$ ont même loi,
            
          \item mais $\Prob(\Ev{S \, X_2 = X_2 -S-1}) \neq 1$. Cela
            démontre en particulier : $S \, X_2 \neq X_2 - S-1$.
          \end{noliste}
        \end{noliste}
      \end{remarkL}~\\[-1.4cm]
    \end{proof}
  \end{noliste}
  
\item Dans cette question, on note $X_n$ une variable aléatoire qui
  suit la loi binomiale $\Bin{n}{\dfrac{1}{2}}$.\\
  On suppose que les variables aléatoires $X_n$ et $S$ sont
  indépendantes et on pose $Y_n = S \, X_n$.
  \begin{noliste}{a)}
    \setlength{\itemsep}{2mm}
  \item Justifier que la fonction $M_{X_n}$ est définie sur $\R$ et
    calculer $M_{X_n}(t)$ pour tout $t\in \R$.
    \begin{proof}~
      \begin{noliste}{$\sbullet$}
      \item  Soit $t \in \R$.\\
        La \var $X_n$ est une \var finie. Ainsi, la \var $\ee^{t \,
          X_n}$ est également une \var finie. Elle admet donc des
        moments à tout ordre, en particulier une espérance.
        \conc{On en déduit que la fonction $M_{X_n}$ défini sur $\R$.}


        \newpage
        
        
      \item Soit $t \in \R$.
        \[
          \begin{array}{rcl@{\quad}>{\it}R{5cm}}
            M_{X_n}(t)
            & = & \E\left(\ee^{t \, X_n}\right)
            \\[.2cm]
            & = & \Sum{k=0}{n} \ee^{k \, t} \ \Prob(\Ev{X_n = k})
            & (par théorème de transfert ($X_n(\Omega) = \llb 0,n \rrb$))
            \nl
            \nl[-.2cm]
            & = & \Sum{k=0}{n} \ee^{k \, t} \dbinom{n}{k}
                  \left(\dfrac{1}{2}\right)^k
                  \left(\dfrac{1}{2}\right)^{n-k}
            & (car $X_n \suit \Bin{n}{\dfrac{1}{2}}$)
            \nl
            \nl[-.2cm]
            & = & \Sum{k=0}{n} \dbinom{n}{k} \ee^{k \, t}
                  \left(\dfrac{1}{2}\right)^n
            \\[.6cm]
            & = & \dfrac{1}{2^n} \ \Sum{k=0}{n} \dbinom{n}{k} \left(
                  \ee^t \right)^k \ 1^{n-k}
            \\[.6cm]
            & = & \dfrac{1}{2^n} \left(\ee^t +1\right)^n
            & (d'après la formule du binôme de Newton)
          \end{array}
        \]
        \conc{$\forall t \in \R$, $M_{X_n}(t) = \dfrac{1}{2^n}
          \left(1+ \ee^t\right)^n$}~\\[-1.4cm]
      \end{noliste}
    \end{proof}
    
  \item Montrer que la fonction $M_{Y_n}$ est donnée par : $\forall t
    \in \R$, $M_{Y_n}(t) = \dfrac{1}{2^{n+1}} \ \big((1+\ee^t)^n + (1+
    \ee^{-t})^n\big)$.
    \begin{proof}~
      \begin{noliste}{$\sbullet$}
      \item Soit $t \in \R$.\\
        La \var $Y_n$ est une \var finie (car les \var $X_n$ et
        $S$ le sont). Ainsi, la \var $\ee^{t \,
          Y_n}$ est également une \var finie. Elle admet donc des
        moments à tout ordre, en particulier une espérance.
        \conc{On en déduit que la fonction $M_{Y_n}$ défini sur $\R$.}
        
      \item Déterminons la loi de $Y_n$.
        \begin{noliste}{$\stimes$}
        \item Tout d'abord, comme $X_n(\Omega) = \llb 0,n \rrb$ et
          $S(\Omega) = \{-1,1\}$, on a : $Y_n(\Omega) \subset \llb -n,
          n \rrb$.
          
        \item Soit $k \in \llb -n, n \rrb$.\\
          La famille $(\Ev{S=-1}, \Ev{S=1})$ forme un système complet
          d'événements.\\
          Ainsi, par formule des probabilités totales :
          \[
            \begin{array}{rcl@{\quad}>{\it}R{4cm}}
              \Prob(\Ev{Y_n = k})
              & = & \Prob(\Ev{S = -1} \cap \Ev{Y_n = k}) + \Prob(\Ev{S
                    = 1} \cap \Ev{Y_n = k})
              \\[.4cm]
              & = & \Prob(\Ev{S = -1} \cap \Ev{S \, X_n = k}) +
                    \Prob(\Ev{S = 1} \cap \Ev{S \, X_n = k})
              \\[.4cm]
              & = & \Prob(\Ev{S = -1} \cap \Ev{-X_n = k}) +
                    \Prob(\Ev{S = 1} \cap \Ev{X_n = k})
              \\[.4cm]
              & = & \Prob(\Ev{S = -1}) \ \Prob(\Ev{X_n = -k}) +
                    \Prob(\Ev{S = 1}) \ \Prob(\Ev{X_n = k})
              & (car les \var $S$ et $X_n$ sont indépendantes)
              \nl
              \nl[-.2cm]
              & = & \dfrac{1}{2} \ \Prob(\Ev{X_n = -k}) + \dfrac{1}{2}
                    \ \Prob(\Ev{X_n = k})
            \end{array}
          \]
          Trois cas se présentent alors :
        \end{noliste}
        \begin{liste}{-}
        \item \dashuline{si $k \in \llb -n, 0 \llb$}, alors $\Ev{X_n = k} =
          \emptyset$. Donc :
          \[
            \Prob(\Ev{Y_n = k}) \ = \ \dfrac{1}{2} \ \Prob(\Ev{X_n =
              -k}) + \bcancel{\dfrac{1}{2} \ \Prob(\emptyset)} \ = \
            \dfrac{1}{2} \ \Prob(\Ev{X_n = -k})
          \]


          \newpage
          
          
        \item \dashuline{si $k=0$}, alors :
          \[
            \begin{array}{rcl}
              \Prob(\Ev{Y_n = 0})
              & = & \dfrac{1}{2} \ \Prob(\Ev{X_n = 0}) + \dfrac{1}{2}
                    \ \Prob(\Ev{X_n = 0})
              \\[.6cm]
              & = & \Prob(\Ev{X_n = 0})
              \\[.4cm]
              & = & \dbinom{n}{0} \left(\dfrac{1}{2}\right)^0
                    \left(\dfrac{1}{2}\right)^{n-0}
              \\[.6cm]
              & = & \dfrac{1}{2^n}
            \end{array}
          \]
          
        \item \dashuline{si $k \in \rrb 0, n \rrb$}, alors $\Ev{X_n = -k} =
          \emptyset$. Donc :
          \[
            \Prob(\Ev{Y_n = k}) \ = \ \bcancel{\dfrac{1}{2} \ \Prob(\emptyset)}
            + \dfrac{1}{2} \ \Prob(\Ev{X_n = k}) \ = \
            \dfrac{1}{2} \ \Prob(\Ev{X_n = k})
          \]
        \end{liste}
        
      \item Soit $t \in \R$.
        \[
          \begin{array}{rcl@{\quad}>{\it}R{4cm}}
            M_{Y_n}(t)
            & = & \E\left(\ee^{t \, Y_n}\right)
            \\[.2cm]
            & = & \Sum{k=-n}{n} \ee^{k \, t} \ \Prob(\Ev{Y_n = k})
            \\[.6cm]
            & = & \Sum{k=-n}{-1} \left(\ee^{k \, t} \ \Prob(\Ev{Y_n =
                  k})\right) + \ee^{0 \, t} \ \Prob(\Ev{Y_n = 0}) +
                  \Sum{k=1}{n} \left(\ee^{k \, t} \ \Prob(\Ev{Y_n =
                  k})\right)
            \\[.6cm]
            & = & \dfrac{1}{2} \ \Sum{k=-n}{-1} \left(\ee^{k \, t} \
                  \Prob(\Ev{X_n = -k})\right) + \dfrac{1}{2^n} +
                  \dfrac{1}{2} \ \Sum{k=1}{n} \left(\ee^{k \, t} \
                  \Prob(\Ev{X_n = k})\right)
          \end{array}
        \]
        \begin{noliste}{$\stimes$}
        \item Tout d'abord :
          \[
            \begin{array}{rcl@{\quad}>{\it}R{4cm}}
              \Sum{k=-n}{-1} \ee^{k \, t} \ \Prob(\Ev{X_n = -k})
              & = & \Sum{j=1}{n} \ee^{-j \, t} \ \Prob(\Ev{X_n = j})
              & (par changement d'indice $j=-k$)
              \nl
              \nl[-.2cm]
              & = &\Sum{j=1}{n} \ee^{-j \, t} \ \dbinom{n}{j}
                    \left(\dfrac{1}{2}\right)^j
                    \left(\dfrac{1}{2}\right)^{n-j}
              \\[.6cm]
              & = & \dfrac{1}{2^n} \ \Sum{j=1}{n} \dbinom{n}{j} \
                    \left(\ee^{-t}\right)^j \ 1^{n-j}
              \\[.6cm]
              & = & \dfrac{1}{2^n} \ \left(\Sum{j=0}{n} \left(\dbinom{n}{j} \
                    \left(\ee^{-t}\right)^j \ 1^{n-j}\right) -1\right)
              \\[.6cm]
              & = & \dfrac{1}{2^n} \left((\ee^{-t}+1)^n -1\right)
              & (d'après la formule du binôme de Newton)
            \end{array}
          \]

        \item De même :
          \[
            \begin{array}{rcl@{\quad}>{\it}R{4cm}}
              \Sum{k=1}{n} \ee^{k \, t} \ \Prob(\Ev{X_n = k})
              & = &\Sum{k=1}{n} \ee^{k \, t} \ \dbinom{n}{k}
                    \left(\dfrac{1}{2}\right)^k
                    \left(\dfrac{1}{2}\right)^{n-k}
              \\[.6cm]
              & = & \dfrac{1}{2^n} \ \Sum{k=1}{n} \dbinom{n}{k} \
                    \left(\ee^{t}\right)^k \ 1^{n-k}
              \\[.6cm]
              & = & \dfrac{1}{2^n} \ \left(\Sum{k=0}{n} \left(\dbinom{n}{k} \
                    \left(\ee^{t}\right)^k \ 1^{n-k}\right) -1\right)
              \\[.6cm]
              & = & \dfrac{1}{2^n} \left((\ee^{t}+1)^n -1\right)
              & (d'après la formule du binôme de Newton)
            \end{array}
          \]
        \end{noliste}


        \newpage
        
        
      \item On obtient alors :
        \[
          \begin{array}{rcl@{\quad}>{\it}R{4cm}}
            M_{Y_n}(t)
            & = & \dfrac{1}{2} \times \dfrac{1}{2^n}
                  \left((1+\ee^{-t})^n -1\right) + \dfrac{1}{2^n} +
                  \dfrac{1}{2} \times \dfrac{1}{2^n} \left((1+
                  \ee^t)^n -1\right)
            \\[.6cm]
            & = & \dfrac{1}{2^{n+1}} (1+ \ee^{-t})^n -
                  \dfrac{1}{2^{n+1}} + \dfrac{1}{2^n} +
                  \dfrac{1}{2^{n+1}} (1+ \ee^t)^n - \dfrac{1}{2^{n+1}}
            \\[.6cm]
            & = & \dfrac{1}{2^{n+1}} \left((1+ \ee^{-t})^n + (1+
                  \ee^t)^n \right) - \dfrac{\bcancel{2}}{2^{n +
                  \bcancel{1}}} + \dfrac{1}{2^n}
            \\[.6cm]
            & = & \dfrac{1}{2^{n+1}} \left((1+ \ee^{-t})^n + (1+
                  \ee^t)^n \right) - \bcancel{\dfrac{1}{2^n}} +
                  \bcancel{\dfrac{1}{2^n}}
          \end{array}
        \]
        \conc{$\forall t \in \R$, $M_{Y_n}(t) = \dfrac{1}{2^{n+1}} \
          \left((1+ \ee^t)^n + (1+ \ee^{-t})^n\right)$}~\\[-1.4cm]
      \end{noliste}
    \end{proof}
    
  \item En utilisant l'égalité $(1+\ee^{-t})^n = \ee^{-nt} \,
    (1+\ee^t)^n$, montrer que $Y_n$ suit la même loi que la différence
    $X_n - H_n$, où $H_n$ est une variable aléatoire indépendante de
    $X_n$ dont on précisera la loi.
    \begin{proof}~
      \begin{noliste}{$\sbullet$}
      \item  Soit $t \in \R$. Démontrons l'égalité : $(1+ \ee^{-t})^n
        =  \ee^{-nt} (1+ \ee^t)^n$.
        \[
          \begin{array}{rcl@{\quad}>{\it}R{4cm}}
            \ee^{-n \, t} \ (1+ \ee^t)^n
            & = & \left(\ee^{-t}\right)^n \ (1+\ee^t)^n
            \\[.2cm]
            & = & \big(\ee^{-t}(1+\ee^t)\big)^n
            \\[.2cm]
            & = & (\ee^{-t} + 1)^n
          \end{array}
        \]
        \conc{$\forall t \in \R$, $(1+ \ee^{-t})^n = \ee^{-nt} (1+\ee^t)^n$}
        
      \item Il s'agit de démontrer que les \var $Y_n$ et $X_n - H_n$
        ont même loi. On peut déjà remarquer : $Y(\Omega) \subset \llb
        -n,n \rrb$. Or, en question \itbf{1.c)}, on a démontré que, si
        deux \var $X$ et $Y$ vérifient :
        \begin{noliste}{$\stimes$}
        \item $X(\Omega) \subset \llb -n,n \rrb$,
          
        \item $Y(\Omega) \subset \llb -n,n \rrb$,
          
        \item $M_X \ = \ M_Y$,
        \end{noliste}
        alors $X$ et $Y$ ont même loi.\\
        Pour répondre à cette question, il suffit donc de trouver une
        \var $H_n$ indépendante de $X_n$ telle que :
        % Comme $Y_n(\Omega) \subset \llb -n,n \rrb$, on souhaite
        % appliquer la question \itbf{1.c)}, \ie on souhaite trouver
        % une \var $H_n$ indépendante de $X_n$ telle que :
        \[
          (X_n - H_n)(\Omega) \subset \llb -n,n \rrb \quad \text{et}
          \quad M_{Y_n} = M_{X_n - H_n}
        \]
        Pour déterminer une telle \var $H_n$, on commencera par
        supposer qu'elle existe pour en déduire des propriétés sur sa loi,
        puis on choisira une \var vérifiant ces propriétés, et enfin
        on établira que l'on est bien dans le cadre d'application de
        la question \itbf{1.c)}.
        
      \item Supposons qu'il existe une telle \var $H_n$. Alors, soit
        $t \in \R$ :
        \begin{noliste}{$\stimes$}
        \item d'une part :
          \[
            \begin{array}{rcl@{\quad}>{\it}R{7cm}}
              M_{X_n - H_n}(t)
              & = & \E\left( \ee^{t \, (X_n - H_n)}\right)
              \\[.4cm]
              & = & \E\left(\ee^{t \, X_n} \ \ee^{-t \, H_n}\right)
              \\[.4cm]
              & = & \E\left(\ee^{t \, X_n}\right) \ \E\left(\ee^{-t \,
                    H_n}\right)
              & (car, par lemme des coalitions, les \var $\ee^{t \,
                X_n}$ et $\ee^{-t \, H_n}$ sont indépendantes)
              \nl
              \nl[-.2cm]
              & = & M_{X_n}(t) \ \E\left(\ee^{-t \, H_n}\right)
              \\[.4cm]
              & = & \dfrac{1}{2^n} \ (1+\ee^t)^n \ \E\left(\ee^{-t \,
                    H_n}\right)
              & (d'après la question \itbf{4.a)})
            \end{array}
          \]


          \newpage
          
          
        \item d'autre part :
          \[
            \begin{array}{rcl@{\quad}>{\it}R{4cm}}
              M_{Y_n}(t)
              & = & \dfrac{1}{2^{n+1}} \ \big((1+ \ee^t)^n + (1+
                    \ee^{-t})^n\big)
              & (d'après la question précédente)
              \nl
              \nl[-.2cm]
              & = & \dfrac{1}{2^{n+1}} \big((1+\ee^t)^n + \ee^{-nt}
                    (1+ \ee^t)^n\big)
              & (d'après l'indication de l'énoncé)
              \nl
              \nl[-.2cm]
              & = & \dfrac{1}{2^{n+1}} \ (1+ \ee^t)^n \left(1 +
                    \ee^{-nt}\right)
              \\[.6cm]
              & = & \dfrac{1}{2^n} \ (1+ \ee^t)^n \times \dfrac{1}{2}
                    \ \left(1+ \ee^{-nt}\right)
            \end{array}
          \]
        \end{noliste}
        Ainsi, si $H_n$ vérifie les propriétés souhaitées :
        \[
          \begin{array}{rcl}
            M_{X_n - H_n}(t) = M_{Y_n}(t)
            & \Leftrightarrow & \bcancel{\dfrac{1}{2^n} \ (1+\ee^t)^n}
                                \  \E\left(\ee^{-t \, H_n}\right) =
                                \bcancel{\dfrac{1}{2^n} \ (1+ \ee^t)^n}
                                \times \dfrac{1}{2} \ \left(1+
                                \ee^{-nt}\right)
            \\[.6cm]
            & \Leftrightarrow & \E\left(\ee^{-t \, H_n}\right) =
                                \dfrac{1}{2} \left(1+ \ee^{-nt}\right)
            \\[.6cm]
            & \Leftrightarrow & \E\left(\ee^{-t \, H_n}\right) =
                                \ee^{-0 \, t} \ \dfrac{1}{2} +
                                \ee^{-nt} \ \dfrac{1}{2}
          \end{array}
        \]


        % \newpage
        
        
      \item On semble faire apparaître ici le théorème de
        transfert. On choisit donc une \var $H_n$ qui vérifie cette égalité.
        On choisit alors une \var $H_n$, indépendante de $X_n$
        telle que :
        \begin{noliste}{$\stimes$}
        \item $H_n(\Omega) = \{0,n\}$,
          
        \item $\Prob(\Ev{H_n = 0}) = \Prob(\Ev{H_n = n}) = \dfrac{1}{2}$.
        \end{noliste}
        
      \item Vérifions qu'avec cette \var $H_n$, nous sommes dans
        le cadre d'application de la question \itbf{1.c)}.
        \begin{noliste}{$\stimes$}
        \item Tout d'abord, comme $X_n(\Omega) = \llb 0,n \rrb$ et
          $H_n(\Omega) = \{0,n \}$, alors : $(X_n - H_n)(\Omega)
          \subset \llb -n,n \rrb$.
          
        \item Ensuite, la \var $\ee^{-t \, H_n}$ admet une espérance
          en tant que \var finie (car $H_n$ est une \var finie).
          Ainsi, par théorème de transfert, soit $t \in \R$ :
          \[
            \E\left( \ee^{-t \, H_n}\right)
            \ = \ \ee^{-0 \, t} \ \Prob(\Ev{H_n = 0}) + \ee^{-nt} \
            \Prob(\Ev{H_n = n})
            \ = \ \dfrac{1}{2} + \ee^{-nt} \ \dfrac{1}{2}
            \ = \ \dfrac{1}{2} \left(1+ \ee^{-nt}\right)
          \]
          Ainsi :
          \[
            \begin{array}{rcl@{\quad}>{\it}R{5cm}}
              M_{Y_n}(t)
              & = & \dfrac{1}{2^n} \ (1+ \ee^t)^n \times \dfrac{1}{2} \
                    (1+ \ee^{-t})^n
              \\[.6cm]
              & = & M_{X_n}(t) \times \E(\ee^{-t \, H_n})
              \\[.4cm]
              & = & \E(\ee^{t \, X_n}) \ \E(\ee^{-t \, H_n})
              \\[.4cm]
              & = & \E(\ee^{t \, X_n} \ \ee^{-t \, H_n})
              & (car les \var $\ee^{t \, X_n}$ et $\ee^{-t \, H_n}$
                sont indépendantes)
              \nl
              \nl[-.2cm]
              & = & M_{X_n-H_n}(t)
            \end{array}
          \]
          D'où : $M_{Y_n} = M_{X_n - H_n}$
        \end{noliste}
        \conc{D'après \itbf{1.c)}, on en déduit que les \var $Y_n$ et
          $X_n - H_n$ suivent la même loi.}
      \end{noliste}


      % \newpage
      

      \begin{remark}
       Les termes \og en utilisant ... \fg{} de cette
       question font hésiter quant à la nécessité de démontrer
       l'égalité énoncée. Dans le doute, on le démontre.
      \end{remark}~\\[-1.4cm]
    \end{proof}
  \end{noliste}
\end{noliste}


\newpage


\subsection*{Partie II. Propriétés générales des fonctions
  génératrices des cumulants et quelques exemples}

\begin{noliste}{1.}
  \setlength{\itemsep}{4mm}
  \setcounter{enumi}{4}
\item Soit $X$ une variable aléatoire et ${\cal D}_X$ le domaine de
  définition de la fonction $K_X$.
  \begin{noliste}{a)}
    \setlength{\itemsep}{2mm}
  \item Donner la valeur de $K_X(0)$.
    \begin{proof}~
      \begin{noliste}{$\sbullet$}
      \item Tout d'abord, la \var $\ee^{0 \, X} = 1$ (\var constante
        égale à $1$) est une \var finie. Elle admet donc une
        espérance. Donc $M_X(0)$ est bien défini. De plus :
        \[
          M_X(0) \ = \ \E\left(\ee^{0 \, X}\right) \ = \ \E(1) \ = \ 1
        \]
        
      \item Ainsi, $K_X(0)$ est bien défini et :
        \[
          K_X(0) \ = \ \ln\big(M_X(0)\big) \ = \ \ln(1) \ = \ 0
        \]
      \end{noliste}
      \conc{$K_X(0) = 0$}~\\[-1cm]
    \end{proof}
    
  \item Soit $(a,b) \in \R^2$ et $Y= a \, X +b$. Justifier pour tout
    réel $t$ pour lequel $a \, t$ appartient à ${\cal D}_X$, l'égalité
    :
    \[
      K_Y(t) \ = \ b \, t + K_X(a \, t)
    \]
    \begin{proof}~\\
      Soit $t \in \R$ tel que $a \, t \in {\cal D}_X$.
      \begin{noliste}{$\sbullet$}
      \item Comme $a \, t \in {\cal D}_X$, alors $K_X(a \, t)$ est
        bien défini, donc $M_X(a \, t)$ également et : $M_X(a \, t) >0$.
        
      \item De plus :
        \[
          M_X(a \, t)
          \ = \ \E\left(\ee^{a \, t}\right) \ = \ \E\left(\ee^{t \,
              (a \, X)}\right)
          \ = \ \E\left(\ee^{t \, (a \, X +b) -b \, t}\right)
          \ = \ \E\left(\ee^{t \, Y} \, \ee^{-b \, t}\right)
          \ = \ \ee^{-b \, t} \ \E\left(\ee^{t \, Y}\right)
        \]
        Or, par définition de $M_Y(t)$ (si cette quantité existe) :
        $M_Y(t) = \E\left(\ee^{t \, T}\right)$.\\
        D'après le calcul précédent, on en déduit que le réel $M_Y(t)$
        est bien défini et :
        \[
          M_Y(t) \ = \ \ee^{b \, t} \ M_X(a \, t)
        \]
        
      \item Comme : $M_X(a \, t) >0$ et $\ee^{b \, t} >0$, alors :
        $M_Y(t) = \ee^{b \, t} \ M_X(a \, t) >0$.\\
        On en déduit que $K_Y(t)$ est bien défini. Et enfin :
        \[
          K_Y(t) \ = \ \ln\big(M_Y(t)\big)
          \ = \ \ln\left(\ee^{b \, t} \ M_X(a \, t)\right)
          \ = \ b \, t + \ln\big(M_X(a \, t)\big)
          \ = \ b \, t + K_X(a \, t)
        \]
      \end{noliste}
      \conc{Pour tout $t \in \R$ tel que $a \, t \in {\cal D}_X$ :
        $K_Y(t) = b \, t + K_X(a \, t)$.}


      % \newpage


      
      \begin{remark}
        On notera que la difficulté de cette question ne réside pas
        dans la démonstration de la relation entre $K_Y(t)$ et $K_X(a
        \, t)$ mais bien dans la démonstration de l'existence de tous
        les objets manipulés.
      \end{remark}~\\[-1.4cm]
    \end{proof}


    \newpage
    
    
  \item On suppose ici que les variables aléatoires $X$ et $-X$
    suivent la même loi.\\
    Que peut-on dire dans ce cas des cumulants d'ordre impair de la
    variables aléatoire $X$ ?
    \begin{proof}~
      \begin{noliste}{$\sbullet$}
      \item Soit $t \in {\cal D}_X$.\\
        Les \var $X$ et $-X$ ont même loi. On en déduit que les
        \var $\ee^{t \, X}$ et $\ee^{-t \, X}$ ont même loi.\\
        Or, comme $t \in {\cal D}_X$, $K_X(t)$ existe, donc $\ee^{t \,
          X}$ admet une espérance ($M_X(t)$) et : $M_X(t)>0$.
        On en déduit que la \var $\ee^{-t \, X}$ admet une espérance
        ($M_{-X}(t) = M_X(-t)$) et : $M_{X}(-t) >0$. D'où :
        $-t \in {\cal D}_X$.
        \conc{On en déduit que, pour tout $t \in {\cal D}_X$, on a :
          $-t \in {\cal D}_X$.} 
        
      \item Soit $t \in {\cal D}_X$, alors $-t \in {\cal D}_X$.
        \begin{noliste}{-}
        \item Ainsi, d'après la question précédente (appliquée à $a=-1$ et
          $b=0$) :
          \[
            K_{-X}(t) \ = \ 0\times t + K_X(-t) \ = \ K_X(-t)
          \]
          
        \item De plus, comme $X$ et $-X$ ont même loi (on rappelle que
          $M_X(t)$ et $M_{-X}(t)$ sont bien définis car $t \in {\cal
            D}_X$ et $-t \in {\cal D}_X$) :
          \[
            M_{-X}(t) \ = \ \E\left(\ee^{t \, (-X)}\right) \ = \
            \E\left(\ee^{t \, X}\right) \ = \ M_X(t)
          \]
          Ainsi : $K_{-X}(t) \ = \ \ln\big(M_{-X}(t)\big) \ = \
          \ln\big(M_X(t)\big) \ = \ K_X(t)$.
        \end{noliste}
        On en déduit : $K_X(t) = K_{-X}(t) = K_X(-t)$.
        \conc{Ainsi, si les \var $X$ et $-X$ suivent la même loi :
          $\forall t \in {\cal D}_X$, $K_X(t) = K_X(-t)$.}
        
      \item Supposons maintenant que la fonction $K_X$ est de classe
        $\Cont{\infty}$ sur ${\cal D}_X$. Soit $p \in \N$ :
        \begin{noliste}{$\stimes$}
        \item tout d'abord : $Q^{(2p+1)}(X) = K_X^{(2p+1)}(0)$,
          
        \item ensuite : $\forall t \in {\cal D}_X$,
          $K_X(t) = K_X(-t)$.\\
          Alors, par récurrence immédiate : $\forall n \in \N$,
          $\forall t \in {\cal D}_X$,
          $K_X^{(n)}(t) = (-1)^n \, K_X^{(n)}(-t)$.\\
          En particulier, pour tout $t \in {\cal D}_X$ :
          \[
            K_X^{(2p+1)}(t) \ = \ (-1)^{2p+1} \, K_X^{(2p+1)}(-t) \ =
            \ -K_X^{(2p+1)}(-t)
          \]
          Comme $0 \in {\cal D}_X$ (d'après \itbf{5.a)}), on en déduit :
          \[
            K_X^{(2p+1)}(0) = -K^{(2p+1)}(0) \ \Leftrightarrow \ 2 \,
            K_X^{(2p+1)}(0) = 0 \ \Leftrightarrow \ K_X^{(2p+1)}(0) =
            0
          \]
        \end{noliste}
        \conc{Ainsi, si $X$ et $-X$ suivent la même loi, sous réserve
          d'existence : $\forall p \in \N$, $Q^{(2p+1)}(X) = 0$.}
      \end{noliste}
  %   \end{proof}
  % \end{noliste}


      \newpage

      
      \begin{remarkL}{.98}
        \begin{noliste}{$\sbullet$}
        \item Remarquons que si $K_X$ n'est pas dérivable sur ${\cal
            D}_X$ alors on ne peut rien dire des cumulants de $X$,
          puisque ces derniers n'existent pas.
          
        \item Dans le programme ECE, on trouve la propriété :
            \[
              \Boxed{
                \begin{array}{C{5.6cm}cC{5cm}}
                  Les \var $X$ et $Y$ ont même loi & \Leftrightarrow & Les
                  \var $X$ et $Y$ ont même fonction de répartition
                \end{array}
              }
            \]

        \item On peut alors démontrer que si $X$ et $Y$ sont des \var
          discrètes (resp. à densité), on a :
          \[
          \Boxed{
            \begin{array}{rcl}
              \left.
                \begin{array}{R{6cm}}
                  \begin{noliste}{$\sbullet$}
                  \item Les \var $X$ et $Y$ ont même loi
                     \item Les \var $X$ et $Y$ admettent un moment d'ordre
                    $n \in \N^*$\\[-.4cm]
                  \end{noliste}             
                \end{array}
              \right\}
              & \Rightarrow & \E\big( X^n \big) = \E\big( Y^n \big)
            \end{array}
          }
          \]
          Cette dernière propriété est aussi vérifiée pour les \var
          quelconques mais n'est pas explicitement écrite dans ce cas
          précis. Toutefois, on peut considérer qu'on y a accès
          puisque le programme précise : \og on admettra que les
          propriétés opératoires usuelles de l'espérance et de la
          variance se généralisent aux variables aléatoires
          quelconques \fg{}.
          
        \item Enfin, on utilise dans cette question la propriété :
          \[
            \text{$X$ et $Y$ ont même loi} \quad \Rightarrow \quad
            \text{$f(X)$ et $f(Y)$ ont même loi}
          \]
          où $f$ est une fonction (ici $x \mapsto \ee^{t \, x}$).\\
          Cette propriété est facile à démontrer dans le cas de \var
          discrètes ou à densité. On l'admettra pour le cas de \var
          quelconques.
        \end{noliste}
      \end{remarkL}~\\[-1.4cm]
    \end{proof}
  \end{noliste}


  % \newpage
  
  
\item Soit $X$ et $Y$ deux variables aléatoires indépendantes et
  ${\cal D}_X$ et ${\cal D}_Y$ les domaines de définition respectifs
  des fonctions $K_X$ et $K_Y$.
  \begin{noliste}{a)}
    \setlength{\itemsep}{2mm}
  \item Monter que pour tout réel $t$ appartenant à la fois à ${\cal
      D}_X$ et ${\cal D}_Y$, on a : $K_{X+Y}(t) = K_X(t) + K_Y(t)$.
    \begin{proof}~\\
      Soit $t \in {\cal D}_X \cap {\cal D}_Y$.
      \begin{noliste}{$\sbullet$}
      \item Comme $t \in {\cal D}_X \cap {\cal D}_Y$, les quantités
        $K_X(t)$ et $K_Y(t)$ sont bien définies.\\
        En particulier, $M_X(t)$ et $M_Y(t)$ sont bien définies et :
        $M_X(t) >0$ et $M_Y(t) >0$.\\
        Ainsi, les \var $\ee^{t \, X}$ et $\ee^{t \, Y}$ :
        \begin{noliste}{$\stimes$}
        \item admettent une espérance,
          
        \item sont indépendantes par lemme des coalitions, car $X$ et
          $Y$ sont indépendantes.
        \end{noliste}
        On en déduit que la \var $\ee^{t \, X} \times \ee^{t \, Y} =
        \ee^{t \, (X+Y)}$ admet une espérance. %
        \conc{Ainsi, la quantité $M_{X+Y}(t)$ est bien définie.}
        \begin{remarkL}{.98}
          \begin{noliste}{$\sbullet$}
          \item On utilise ici le fait que si deux \var $U$ et $V$
            sont indépendantes et admettent une espérance alors, la
            \var produit $UV$ admet une espérance donnée par :
            $\E\big( UV \big) = \E(U) \, \E(V)$.\\
            L'hypothèse d'indépendance est ici cruciale pour démontrer
            l'existence de l'espérance du produit et pour obtenir sa
            valeur.
          \end{noliste}
        \end{remarkL}


        \newpage


        \begin{remarkL}{.99}
          \begin{noliste}{$\sbullet$}
          \item Dans le cas général, la \var produit $UV$ admet une
            espérance si les \var $U$ et $V$ admettent {\bf un moment
              d'ordre $2$}.
            
          \item On peut se demander d'où provient cette hypothèse liée
            aux moments d'ordre $2$.\\
            Elle est issue d'un théorème de domination. Détaillons ce
            point.\\
            Remarquons tout d'abord : $(U - V)^2 \geq 0$. \\[.2cm]
            On en déduit : $U^2 - 2 \, UV + V^2 \geq 0$.\\[.2cm]
            Et, en réordonnant : $UV \leq \dfrac{1}{2} \, U^2 +
            \dfrac{1}{2} \, V^2$. Ou encore :
            \[
            0 \ \leq \ |UV| \ \leq \ \dfrac{1}{2} \, U^2 +
            \dfrac{1}{2} \, V^2
            \]
            Comme $U$ et $V$ admettent un moment d'ordre $2$, la \var
            $\frac{1}{2} \, U^2 + \frac{1}{2} \, V^2$ admet une
            espérance comme combinaison linéaire de \var qui en
            admettent une.\\
            Ainsi, par théorème de domination (présenté seulement dans
            le programme ECS), la \var $|UV|$ admet une espérance. Il
            en est de même de la \var $UV$.
            
%           \item Si $X$ et $Y$ sont indépendantes et admettent une
%             espérance, on a l'égalité :
%             \[
%             \E(X \, Y) \ = \ \E(X) \, \E(Y)
%             \]
%             L'hypothèse d'existence du moment d'ordre $2$ n'est donc
%             pas nécessaire dans ce cas.
          \end{noliste}
        \end{remarkL}
        
      \item De plus :
        \[
          \begin{array}{rcl@{\quad}>{\it}R{6cm}}
            M_{X+Y}(t)
            & = & \E\left(\ee^{t \, (X+Y)}\right)
            \\[.2cm]
            & = & \E\left(\ee^{t \, X} \ \ee^{t \, Y}\right)
            \\[.2cm]
            & = & \E\left(\ee^{t \, X}\right) \ \E\left(\ee^{t \,
                  Y}\right)
            & (car les \var $\ee^{t \, X}$ et
              $\ee^{t \, Y}$ sont indépendantes)
            \nl
            \nl[-.2cm]
            & = & M_X(t) \ M_Y(t)
          \end{array}
        \]

        
      \item Enfin, comme $M_X(t) >0$ et $M_Y(t) >0$, alors :
        $M_{X+Y}(t) >0$. Donc $K_{X+Y}(t)$ est bien définie. On
        obtient :
        \[
          K_{X+Y}(t) \ = \ \ln\big(M_{X+Y}(t)\big) \ = \
          \ln\big(M_X(t) \, M_Y(t)\big) \ = \ \ln\big(M_X(t)\big) +
          \ln\big(M_Y(t)\big) \ = \ K_X(t) + K_Y(t)
        \]
        \conc{$\forall t \in {\cal D}_X \cap {\cal D}_Y$, $K_{X+Y}(t)
          = K_X(t) + K_Y(t)$}~\\[-1.4cm]
      \end{noliste}
    \end{proof}


    % \newpage
    
    
  \item En déduire une relation entre les cumulants des variables
    aléatoires $X$, $Y$ et $X+Y$.
    \begin{proof}~\\
      Soit $p \in \N^*$.\\
      Si $K_X$ est de classe $\Cont{p}$ sur ${\cal D}_X$ et $K_Y$ est
      de classe $\Cont{p}$ sur ${\cal D}_Y$, alors, d'après la
      question précédente, $K_{X+Y}$ est de
      classe $\Cont{p}$ sur ${\cal D}_X \cap {\cal D}_Y$ et :
      \[
        \forall t \in {\cal D}_X \cap {\cal D}_Y, \ \ K_{X+Y}^{(p)}(t)
        \ = \ K_X^{(p)}(t) + K_X^{(p)}(t)
      \]
      Or $0 \in {\cal D}_X \cap {\cal D}_Y$. Donc :
      \[
        Q_p(X+Y) \ = \ K_{X+Y}^{(p)}(0) \ = \ K_X^{(p)}(0) +
        K_Y^{(p)}(0) \ = \ Q_p(X) + Q_p(Y)
      \]
      \conc{Finalement, pour tout $p \in \N^*$, si $Q_p(X)$ et
        $Q_p(Y)$ sont bien définis, alors :\\
        $Q_p(X+Y) \ = \ Q_p(X) + Q_p(Y)$.}~\\[-1cm]
    \end{proof}
  \end{noliste}


  \newpage
  
  
\item Soit $U$ une variable aléatoire suivant la loi uniforme sur
  l'intervalle $[0,1]$.
  \begin{noliste}{a)}
    \setlength{\itemsep}{2mm}
  \item Montrer que la fonction $M_U$ est définie sur $\R$ et donnée
    par : $\forall t \in \R$, $M_U(t) = \left\{
      \begin{array}{cR{1.5cm}}
        \dfrac{\ee^t - 1}{t} & si $t \neq 0$
        \nl
        \nl[-.2cm]
        1 & si $t=0$
      \end{array}
    \right.$.
    \begin{proof}~
      \begin{noliste}{$\sbullet$}
      \item Soit $t \in \R$.\\
        Le réel $M_U(t)$ existe si et seulement si la \var $\ee^{t \,
          U}$ admet une espérance.\\
        Par théorème de transfert, la \var $\ee^{t \, U}$ admet une
        espérance si et seulement si l'intégrale
        $\dint{-\infty}{+\infty} \ee^{t \, x} f_U(x) \dx$ est
        absolument convergente.\\
        Les fonctions $x \mapsto \ee^{t \,x}$ et $f_U$ étant à valeurs
        positives sur $\R$ ($f_U$ est une densité de probabilité),
        cela revient à démonter que cette intégrale est convergente.

        
      \item De plus, la fonction $f_u$ est nulle en dehors de $[0,1]$,
        donc :
        \[
          \dint{-\infty}{+\infty} \ee^{t \, x} f_U(x) \dx \ = \
          \dint{0}{1} \ee^{t \, x} f_U(x) \dx
        \]
        
      \item La fonction $x \mapsto \ee^{t \, x} f_U(x)$ est continue
        par morceaux sur le segment $[0,1]$. On en déduit que
        l'intégrale $\dint{0}{1} \ee^{t \, x} f_U(x) \dx$ est bien
        définie.\\
        Ainsi, pour tout $t \in \R$, la \var $\ee^{t \, U}$ admet une
        espérance.%
        \conc{La fonction $M_U$ est donc définie sur $\R$.}


        % \newpage
        
        
      \item Soit $t \in \R$. Deux cas se présentent :
        \begin{noliste}{$\stimes$}
        \item \dashuline{si $t=0$}, alors d'après la question \itbf{5.a)}
          : $M_U(0) = 1$.
          
        \item \dashuline{si $t \neq 0$}, alors :
          \[
            \begin{array}{rcl@{\quad}>{\it}R{4cm}}
              M_U(t)
              & = & \E\left(\ee^{t \, U}\right)
              \\[.2cm]
              & = & \dint{0}{1} \ee^{t \, x} f_U(x) \dx
              & (après les points précédents)
              \nl
              \nl[-.2cm]
              & = & \dint{0}{1} \ee^{t \, x} \dx
              \\[.4cm]
              & = & \Prim{\dfrac{1}{t} \ \ee^{t \, x}}{0}{1}
              & (car $t \neq 0$)
              \nl
              \nl[-.2cm]
              & = & \dfrac{1}{t}\left(\ee^t - 1\right)
            \end{array}
          \]
        \end{noliste}
        \conc{Finalement : $M_U : t \mapsto \left\{
            \begin{array}{cR{2cm}}
              \dfrac{\ee^t - 1}{t} & si $t \neq 0$
              \nl
              \nl[-.2cm]
              1 & si $t=0$
            \end{array}
          \right.$.}~\\[-1.4cm]
      \end{noliste}
    \end{proof}


    \newpage
    
    
  \item Calculer la dérivée de la fonction $M_U$ en tout point $t \neq
    0$.
    \begin{proof}~
      \begin{noliste}{$\sbullet$}
      \item La fonction $M_U$ est dérivable sur $]-\infty, 0[$ et sur
        $]0,+\infty[$ en tant que quotient $\dfrac{f_1}{f_2}$ avec :
        \begin{noliste}{$\stimes$}
        \item $f_1 : t \mapsto \ee^t - 1$ dérivable sur $]-\infty, 0[$
          et sur $]0,+\infty[$,
          
        \item $f_2 : t \mapsto t$ dérivable sur $]-\infty,0[$ et sur
          $]0,+\infty[$ et qui ne s'annule pas sur cet intervalle.
        \end{noliste}
        
      \item Soit $t \in \ ]-\infty,0[ \ \cup \ ]0,+\infty[$.
        \[
          M_U'(t) \ = \ \dfrac{\ee^t \times t - (\ee^t -1) \times
            1}{t^2}
        \]
        \conc{$\forall t \in \ ]-\infty, 0[ \ \cup \ ]0,+\infty[$,
          $M_U(t) = \dfrac{t \, \ee^t - \ee^t +1}{t^2}$}~\\[-1.4cm]
      \end{noliste}
    \end{proof}
    
  \item Trouver la limite du quotient $\dfrac{M_U(t) -1}{t}$ lorsque
    $t$ tend vers $0$.
    \begin{proof}~
      \begin{noliste}{$\sbullet$}
      \item Soit $t \in \ \R^*$.
        \[
          \dfrac{M_U(t) - 1}{t} \ = \ \dfrac{\frac{\ee^t -1}{t} -1}{t}
          \ = \ \dfrac{\frac{\ee^t - 1 -t}{t}}{t} \ = \ \dfrac{\ee^t
            -1 -t}{t^2}
        \]
        
      \item Déterminons, si elle existe, $\dlim{t\to 0} \dfrac{\ee^t
          -1-t}{t^2}$.
        \begin{noliste}{$\stimes$}
        \item Tout d'abord : $\ee^t = 1+t + \dfrac{t^2}{2} + \oo{t}{0}
            (t^2)$. Ainsi : $\ee^t - 1 -t = \dfrac{t^2}{2} +
          \oo{t}{0}(t^2)$. D'où :
          \[
            \ee^t - 1 - t \eq{t}{0} \dfrac{t^2}{2}
          \]
          
        \item On en déduit :
          \[
          \dfrac{\ee^t -1-t}{t^2} \eq{t}{0} \dfrac{\frac{t^2}{2}}{t^2}
          = \dfrac{1}{\bcancel{t^2}} \ \dfrac{\bcancel{t^2}}{2} =
          \dfrac{1}{2}
          \]
        \end{noliste}
        \conc{On en déduit : $\dlim{t\to 0} \dfrac{M_U(t) - 1}{t} =
            \dfrac{1}{2}$.}~\\[-1.4cm]
      \end{noliste}
    \end{proof}
    
  \item Montrer que la fonction $M_U$ est de classe $\Cont{1}$ sur
    $\R$.
    \begin{proof}~
      \begin{noliste}{$\sbullet$}
      \item D'après la question précédente : $\dlim{t\to 0}
        \dfrac{M_U(t) - M_U(0)}{t-0} \ = \ \dfrac{1}{2}$.
        \conc{On en déduit que la fonction $M_U$ est dérivable en $0$
          et $M_U'(0) = \dfrac{1}{2}$.}
        
      \item De plus, d'après la question \itbf{7.b)}, la fonction
        $M_U$ est dérivable sur $]-\infty,0[$ et sur $]0,+\infty[$.%
        \conc{On en déduit que la fonction $M_U$ est dérivable sur
          $\R$.}


        \newpage
        
        
      \item La fonction $M_U$ est de classe $\Cont{1}$ sur
        $]-\infty,0[$ et sur $]0,+\infty[$ avec des arguments
        similaires à ceux de la dérivabilité sur ces intervalles
        (question \itbf{7.b)}).
        
      \item On cherche enfin à montrer que la fonction $M_U$ est de classe
        $\Cont{1}$ en $0$, c'est-à-dire que la fonction $M_U'$ est
        continue en $0$. On rappelle :
        \[
          M_U' : t \mapsto \left\{
            \begin{array}{cR{2cm}}
              \dfrac{t \, \ee^t - \ee^t +1}{t^2} & si $t \neq 0$
              \nl
              \nl[-.2cm]
              \dfrac{1}{2} & si $t=0$
            \end{array}
          \right.
        \]
        On cherche donc à montrer : $\dlim{t \to 0} \dfrac{t \, \ee^t
          - \ee^t +1}{t^2} = \dfrac{1}{2}$.
        \begin{noliste}{$\stimes$}
        \item On sait déjà : $\ee^t = 1+t + \dfrac{t^2}{2} + \oo{t}{0}
            (t^2)$. Ainsi :
          \[
            t \, \ee^t \ = \ t \left(1+t+ \dfrac{t^2}{2} + \oo{t}{0}
            (t^2)\right) \ = \ t+ t^2 + \dfrac{t^3}{2} +
            \oo{t}{0}(t^3)
          \]
          On en déduit :
          \[
            \begin{array}{rcl@{\quad}>{\it}R{4cm}}
              t \, \ee^t - \ee^t +1
              & = & \bcancel{t}+t^2 + \dfrac{t^3}{2} + \oo{t}{0}(t^3) -
                    \left(\bcancel{1}+\bcancel{t} + \dfrac{t^2}{2} +
                    \oo{t}{0}(t^2)\right) +\bcancel{1}
              \\[.6cm]
              & = & \dfrac{t^2}{2} + \oo{t}{0}(t^2)
              & (car $t^3 = \oo{t}{0}(t^2)$)
            \end{array}
          \]
          Ainsi : $t \, \ee^t - \ee^t +1 \eq{t}{0} \dfrac{t^2}{2}$.

        \item On obtient :
          \[
            M_U'(t) \ = \ \dfrac{t \, \ee^t - \ee^t +1}{t^2} \eq{t}{0}
            \dfrac{\frac{\bcancel{t^2}}{2}}{\bcancel{t^2}} \ = \
            \dfrac{1}{2}
          \]
          On en déduit : $\dlim{t\to 0} M_U'(t) \ = \ \dfrac{1}{2} \ =
          \ M_U'(0)$.\\
          La fonction $M_U'$ est donc continue en $0$.
        \end{noliste}
        \conc{Finalement, la fonction $M_U$ est de classe $\Cont{1}$
          sur $\R$.}
      \end{noliste}
      \begin{remark}
        \begin{noliste}{$\sbullet$}
        \item On pouvait également déterminer le DL à l'ordre $2$ \og
          $t \, \ee^t - \ee^t +1 = \dfrac{t^2}{2} + \oo{t}{0}(t^2)$
          \fg{} en appliquant la formule de Taylor-Young à la fonction
          $t \mapsto t \, \ee^t - \ee^t +1$ (qui est bien de classe
          $\Cont{2}$ sur $\R$).
          
        \item L'utilisation de la formule de Taylor-Young serait un
          peu plus cohérente au regard du programme d'ECE. On ne
          privilégie cependant  pas cette méthode ici
          puisqu'elle est bien plus chronophage que celle présentée.
        \end{noliste}
      \end{remark}~\\[-1.4cm]
    \end{proof}
  \end{noliste}


  \newpage
  
  
\item Soit $\alpha$ et $\beta$ deux réels tels que $\alpha < \beta$.\\
  Dans cette question, on note $X$ une variable aléatoire qui suit la
  loi uniforme sur l'intervalle $[\alpha, \beta]$.
  \begin{noliste}{a)}
    \setlength{\itemsep}{2mm}
  \item Exprimer $K_X$ en fonction de $M_U$, où la variable aléatoire
    $U$ a été définie dans la question \itbf{7.}
    \begin{proof}~
      \begin{noliste}{$\sbullet$}
      \item Montrons que la fonction $K_U$ est bien définie, \ie :
        $\forall t \in \R$, $M_U(t) >0$.\\
        Soit $t \in \R$. Trois cas se présentent :
        \begin{noliste}{$\stimes$}
        \item \dashuline{si $t \in \ ]-\infty, 0[$} :
          \[
            \begin{array}{rcl@{\quad}>{\it}R{5cm}}
              M_u(t) >0
              & \Leftrightarrow & \dfrac{\ee^t - 1}{t} >0
              \\[.6cm]
              & \Leftrightarrow & \ee^t - 1 < 0
              & (car $t<0$)
              \nl
              \nl[-.2cm]
              & \Leftrightarrow & \ee^t < 1
              \\[.2cm]
              & \Leftrightarrow & t < 0
              & (par stricte croissance de la fonction $\ln$ sur
                $]0,+\infty[$)
            \end{array}
          \]
          La dernière inégalité est vraie, donc, par équivalence, la
          première aussi.
          
        \item \dashuline{si $t=0$}, alors : $M_U(0) = 1 >0$.
          
        \item \dashuline{si $t \in \ ]0,+\infty[$}, alors, avec un raisonnement
          similaire au cas $t \in \ ]-\infty, 0[$, on obtient
          également :
          $M_U(t) >0$.
        \end{noliste}
        \conc{La fonction $K_U$ est bien définie sur $\R$.}
        \begin{remarkL}{.95}
          \begin{noliste}{$\sbullet$}
          \item L'esprit du sujet est ici de faire l'étude des
            fonctions $M_X$ et $K_X$ dans le cas particulier de
            certaines lois usuelles. C'est pourquoi on exploite
            l'expression explicite de $M_U$ pour démontrer l'existence
            de $K_U$.
            
          \item On pourrait en fait démontrer dans un cadre très
            général que si $M_X(t)$ est bien définie, alors, comme la
            \var $\ee^{t \, X}$ est à valeurs {\bf strictement}
            positives : $M_X(t) >0$.\\
            On démontrera cette implication dans la partie III qui,
            elle traite du cas général.
          \end{noliste}
        \end{remarkL}
        
      \item De plus, rappelons :
        \[
          U \suit \Uc{0}{1} \quad \Leftrightarrow \quad Y=(\beta -
          \alpha) \, U + \alpha \suit \Uc{\alpha}{\beta}
        \]
        Ainsi, en notant $Y = (\beta - \alpha) \, U + \alpha$, on
        obtient que les \var $Y$ et $X$ ont même loi.
        
      \item Soit $t \in \R$, alors $(\beta-\alpha) \, t \in \R$. On en déduit,
        d'après la question \itbf{5.b)}, que $K_Y(t)$ existe et :
        \[
          K_Y(t) \ = \ \alpha \, t + K_U\big((\beta-\alpha) \, t) \ = \
          \alpha \, t + \ln\Big(M_U\big((\beta-\alpha) \, t\big)\Big)
        \]
        {\it (on rappelle que, comme la fonction $K_Y$ est bien
          définie sur $\R$, la fonction $M_Y$ aussi et : $\forall t
          \in \R$, $M_Y(t) >0$)}
        
      \item Enfin, comme les \var $X$ et $Y$ ont même loi, alors les
        \var $\ee^{t \, X}$ et $\ee^{t \, Y}$ ont même loi.\\
        Or, comme la fonction $M_Y$ est bien définie sur
        $\R$, la \var $\ee^{t \, Y}$ admet une espérance strictement
        positive. Ainsi, la \var $\ee^{t \, X}$ admet la même
        espérance strictement positive, \ie la fonction $M_X$ est
        définie sur $\R$ et : $\forall t \in \R$, $M_X(t) = M_Y(t)
        >0$.
        \conc{On en déduit que la fonction $K_X$ est définie sur $\R$
          et :\\
          $\forall t \in \R$, \quad $K_X(t) \ = \ \alpha \, t +
          \ln\Big(M_U\big((\beta-\alpha) \, t\big)\Big)$.}~\\[-1.4cm]
      \end{noliste}
    \end{proof}


    \newpage
    
    
  \item Justifier que la fonction $K_X$ est de classe $\Cont{1}$ sur
    $\R$ et établir l'égalité : $Q_1(X) = \E(X)$.
  \end{noliste}
    \begin{proof}~
      \begin{noliste}{$\sbullet$}
      \item Tout d'abord, la fonction $\widetilde{M_U} :t \mapsto M_u\big((\beta -
        \alpha) \, t\big)$ est de classe $\Cont{1}$ sur $\R$ en tant
        que composée de fonctions de classe $\Cont{1}$ sur $\R$ (on rappelle que
            $M_U$ est de classe $\Cont{1}$ sur $\R$ d'après \itbf{7.d)}).\\
        De plus, comme $M_U(]-\infty, +\infty[)
        \subset \ ]0,+\infty[$ (démontré dans la question précédente),
        pour tout $t \in \R$ :
        :
        \[
          \widetilde{M_U}(t) = M_U\big((\beta-\alpha) \, t\big) >0
        \]
        % \begin{noliste}{$\stimes$}
        % \item $f_1 : t \mapsto (\beta - \alpha) \, t$ qui :
        %   \begin{noliste}{-}
        %   \item est de classe $\Cont{1}$ sur $\R$,
            
        %   \item vérifie : $f(\R) \subset \ \R$
        %   \end{noliste}
          
        % \item $\M_U$ qui est de classe $\Cont{1}$ sur $\R$, d'après \itbf{7.d)}.
        % \end{noliste}
        
      \item Ensuite, la fonction $t \mapsto \ln \Big( M_u \big((\beta
        - \alpha) \, t\big)\Big)$ est de classe
        $\Cont{1}$ sur $\R$ car la composée $\ln \circ \tilde{M_U}$ de :
        \begin{noliste}{$\stimes$}
        \item $\tilde{M_U} : t \mapsto M_U\big((\beta- \alpha) \, t\big)$ qui :
          \begin{noliste}{-}
          \item est de classe $\Cont{1}$ sur $\R$,
            
          \item vérifie : $\tilde{M_U}(]-\infty, +\infty[) \subset \
            ]0,+\infty[$ 
          \end{noliste}
          
        \item $\ln$ qui est de classe $\Cont{1}$ sur $]0,+\infty[$.
        \end{noliste}
        \conc{On en déduit que la fonction $K_X$ est de classe
          $\Cont{1}$ sur $\R$\\ en tant que somme de fonctions de classe
          $\Cont{1}$ sur $\R$.}
        
      \item Par définition de $Q_1(X)$, on a : $Q_1(X) = K_X'(0)$.\\
        Or, d'après la question précédente, pour tout $t \in \R$ :
        \[
          K_X'(t) \ = \ \alpha + \dfrac{(\beta-\alpha) \,
            M_U'\big((\beta- \alpha) \, t\big)}{M_U\big((\beta-\alpha)
            \, t\big)}
        \]
        On en déduit :
        \[
          \begin{array}{rcl@{\quad}>{\it}R{5cm}}
            Q_1(X)
            & = & K_X'(0) \ = \ \alpha + \dfrac{(\beta-\alpha) \, M_U'(0)}{M_U(0)}
            \\[.6cm]
            & = & \alpha + \dfrac{(\beta-\alpha) \ \frac{1}{2}}{1}
            & (d'après \itbf{7.a)} et \itbf{7.d)})
            \nl
            \nl[-.2cm]
            & = & \alpha + \dfrac{\beta-\alpha}{2}
            \\[.6cm]
            & = & \dfrac{\alpha + \beta}{2}
          \end{array}
        \]
        \conc{Finalement, comme $X \suit \Uc{\alpha}{\beta}$ : $\E(X)
          = \dfrac{\alpha + \beta}{2} = Q_1(X)$.}~\\[-1.4cm]
      \end{noliste}
    \end{proof}
  
\item Soit un réel $\lambda >0$ et soit $T$ une variable aléatoire qui
  suit la loi de Poisson de paramètre $\lambda$.
  \begin{noliste}{a)}
    \setlength{\itemsep}{2mm}
  \item Déterminer les fonctions $M_T$ et $K_T$.
    \begin{proof}~
      \begin{noliste}{$\sbullet$}
      \item Soit $t \in \R$. Tout d'abord, la quantité $M_T(t)$ est
        bien définie si la \var $\ee^{t \, T}$ admet une espérance.\\
        Or, par théorème de transfert, la \var $\ee^{t \, T}$ admet
        une espérance si et seulement si la série $\Sum{n \geq 0}{}
        \ee^{t \, n} \ \Prob(\Ev{T = n})$ est absolument
        convergente.\\
        Cette série étant à termes positifs, cela revient à démontrer
        qu'elle est convergente.


        \newpage

        
      \item Soit $N \in \N$.
        \[
          \begin{array}{rcl@{\quad}>{\it}R{5cm}}
            \Sum{k=0}{N} \ee^{k \, t} \ \Prob(\Ev{T = k})
            & = & \Sum{k=0}{N} \ee^{k \, t} \ \dfrac{\lambda^k}{k!} \
                  \ee^{-\lambda}
            & (car $T \suit \Pois{\lambda}$)
            \nl
            \nl[-.2cm]
            & = & \ee^{-\lambda} \ \Sum{k=0}{N}
                  \dfrac{\left(\ee^t\right)^k \, \lambda^k}{k!}
            \\[.6cm]
            & = & \ee^{-\lambda} \ \Sum{k=0}{N} \dfrac{\left(\lambda
                  \, \ee^t\right)^k}{k!}
          \end{array}
        \]
        On reconnaît la somme partielle d'ordre $N$ de la série
        exponentielle $\Sum{n \geq 0}{} \dfrac{x^n}{n!}$ en $x =
        \lambda \, \ee^t$. Cette série est convergente et :
        \[
          \Sum{k=0}{N} \dfrac{\left(\lambda \, \ee^t\right)^k}{k!} \
          \tendd{N}{+\infty} \  \Sum{k=0}{+\infty} \dfrac{\left(\lambda
          \, \ee^t\right)^k}{k!} = \exp\left(\lambda \, \ee^t\right)
        \]
        On en déduit que la série $\Sum{n \geq 0}{} \ee^{t \, n} \
        \Prob(\Ev{T = n})$ converge.
        Ainsi, pour tout $t \in \R$, la \var $\ee^{t \, T}$ admet une espérance.
        \conc{On en déduit que la fonction $M_T$ est définie sur
          $\R$.}
        
      \item De plus, soit $t \in \R$ :
        \[
          M_T(t) \ = \ \E\left(\ee^{t \, T}\right) \ = \
          \ee^{-\lambda} \ \exp\left(\lambda \, \ee^t\right) \ = \
          \exp\left(- \lambda + \lambda \, \ee^t\right) \ = \
          \exp\left(\lambda \, (\ee^t -1)\right)
        \]
        \conc{$M_T : t \mapsto \exp\left(\lambda \, (\ee^t
            -1)\right)$}
        
      \item On a bien, pour tout $t \in \R$ : $M_T(t) =
        \exp\left(\lambda \, (\ee^t-1)\right) >0$.
        \conc{La fonction $K_T$ est donc définie sur $\R$.}
        
      \item Soit $t \in \R$.
        \[
          K_T(t) \ = \ \ln\big(M_T(t)\big) \ = \ \ln\left(
            \exp\left(\lambda \, (\ee^t -1)\right)\right) \ = \
          \lambda \, (\ee^t -1)
        \]
        \conc{$K_T : t \mapsto \lambda \, (\ee^t -1)$}~\\[-1.4cm]
      \end{noliste}
    \end{proof}
    
  \item En déduire les cumulants de $T$.
    \begin{proof}~
      \begin{noliste}{$\sbullet$}
      \item Tout d'abord, la fonction $K_T$ est de classe
        $\Cont{\infty}$ sur $\R$ en tant que transformée affine de la
        fonction $\exp$ de classe $\Cont{\infty}$  sur $\R$.
        \conc{Ainsi, la \var $T$ admet des cumulants à tout ordre.}
        
      \item Ensuite, pour tout $t \in \R$ : $K_T'(t) = \lambda \,
        \ee^t$.\\
        On en déduit, par récurrence immédiate, que pour tout $n \in
        \N^*$ : $\forall t \in \R$, $K_T^{(n)}(t) = \lambda \, \ee^t$.
        \conc{Ainsi, pour tout $n \in \N^*$ : $Q_n(X) = K_T^{(n)}(0) =
          \lambda$.}~\\[-1.4cm]
      \end{noliste}
    \end{proof}
  \end{noliste}


  \newpage
  


\item Soit $Z$ une variable aléatoire qui suit la loi normale centrée
  réduite.
  \begin{noliste}{a)}
    \setlength{\itemsep}{2mm}
  \item Justifier pour tout $t \in \R$, la convergence de l'intégrale
    $\dint{-\infty}{+\infty} \exp\left(t \, x - \dfrac{x^2}{2}\right)
    \dx$.
    
    \begin{proof}~\\%
      Soit $t \in \R$.
      \begin{noliste}{$\sbullet$}
      \item L'intégrale impropre $\dint{-\infty}{+\infty} \exp\left(t
          \, x - \dfrac{x^2}{2}\right) \dx$ est convergente si et
        seulement si :
        \begin{noliste}{$-$}
        \item l'intégrale impropre $\dint{-\infty}{0} \exp\left(t \, x
            - \dfrac{x^2}{2}\right) \dx$ est convergente.
        \item et l'intégrale impropre $\dint{0}{+\infty} \exp\left(t
            \, x - \dfrac{x^2}{2}\right) \dx$ est convergente.
        \end{noliste}
        
      \item Démontrons tout d'abord la convergence de l'intégrale
        impropre $\dint{0}{+\infty} \exp\left(t \, x -
          \dfrac{x^2}{2}\right) \dx$.\\
        La fonction $h_t : x \mapsto \exp\left(t \, x -
          \dfrac{x^2}{2}\right)$ est continue sur $[0, +\infty[$.\\
        Ainsi, l'intégrale $\dint{0}{+\infty} h_t(x) \dx$ est impropre
        seulement en $+\infty$.
      \end{noliste}
      \begin{liste}{$-$}
      \item Tout d'abord, la fonction $h_t$ est continue sur le {\bf
          segment} $[0,1]$.\\
        On en déduit que l'intégrale $\dint{0}{1} h_t(x) \dx$ est bien
        définie.
        
      \item Par ailleurs :
        \begin{noliste}{$\stimes$}
        \item $\forall x \in [1, +\infty[$, \ $\exp\left(t \, x -
            \dfrac{x^2}{2}\right) \geq 0$ \quad et \quad
          $\dfrac{1}{x^2} \geq 0$.
          
        \item $\exp\left(t \, x - \dfrac{x^2}{2}\right) =
          \oox{+\infty} \left( \dfrac{1}{x^2} \right)$          
          
        \item $\dint{1}{+\infty} \dfrac{1}{x^2} \dx$ est une intégrale
          de Riemann, impropre en $+\infty$, d'exposant $2$ ($>
          1)$.\\[.2cm]
          Elle est donc convergente.
        \end{noliste}
        Par critère de négligeabilité d'intégrales généralisées de
        fonctions continues positives, l'intégrale $\dint{1}{+\infty}
        h_t(x) \dx$ est convergente.
      \end{liste}
      \conc{Ainsi, pour tout $t \in \R$, l'intégrale impropre
        $\dint{0}{+\infty} h_t(x) \dx$ est convergente.}
      \begin{noliste}{$\sbullet$}
      \item Il reste à démontrer la convergence de l'intégrale
        impropre $\dint{0}{+\infty} \exp\left(t \, x -
          \dfrac{x^2}{2}\right) \dx$.\\
        La fonction $h_t$ est continue sur $]-\infty, 0]$ donc cette
        intégrale est impropre seulement en $-\infty$.\\
        En effectuant le changement de variable affine $\Boxed{u =
          -x}$, on obtient :
        \[
        \begin{array}{rcl@{\qquad}>{\it}R{5cm}}
          \dint{-\infty}{0} \exp\left( t \, x - \dfrac{x^2}{2} \right)
          \dx & = & \dint{0}{+\infty} \exp\left( -t \, u -
            \dfrac{(-u)^2}{2} \right) (- \ du) 
          \\[1cm]
          & = & - \dint{0}{+\infty} \exp\left( -t \, u - \dfrac{u^2}{2}
          \right) \ du \ = \ - \dint{0}{+\infty} f_{-t}(u) \ du
        \end{array}
        \]
        
        
        \newpage
        
          
        \noindent
        En appliquant le résultat précédent en $-t \in \R$, on conclut
        que l'intégrale impropre $\dint{0}{+\infty} f_{-t}(u) \ du$
        est convergente. %
        \conc{Ainsi, pour tout $t \in \R$, l'intégrale impropre
          $\dint{-\infty}{0} h_t(x) \dx$ est convergente.}~\\[-.6cm]
      \end{noliste}
      \conc{Pour tout $t \in \R$, l'intégrale impropre
        $\dint{-\infty}{+\infty} h_t(x) \dx$ est bien convergente.}
      \begin{remarkL}{.99}
        \begin{noliste}{$\sbullet$}
        \item Le programme officiel précise que \og les changements de
          variables {\bf non affines} ne seront pratiqués qu'avec des
          intégrales sur un segment \fg{}. Il est donc autorisé, sous
          réserve de convergence, d'effectuer un changement de
          variable affine sur une intégrale généralisée (ce qui est
          fait dans cette question).
          
        \item Ici, on pose le changement de variable affine $\Boxed{u
            = -x}$. Il faut s'habituer à effectuer ce changement de
          variable classique à la volée. Rappelons comment le
          présenter formellement.
          \[
          \left|
            \begin{array}{P{6cm}}
              $u = -x$ (et donc $x = -u$) \nl
              $\hookrightarrow$ $du = -\dx$ \quad et \quad $\dx = -
              \ du$ \nl 
              \vspace{-.2cm}
              \begin{noliste}{$\sbullet$}
              \item $x = -\infty \ \Rightarrow \ u = +\infty$\\[.2cm]
              \item $x = 0 \ \Rightarrow \ u = 0$ %
                \vspace{-.8cm}
              \end{noliste}
            \end{array}
          \right.
          \]
          Ce changement de variable est valide car $\psi : u \mapsto -u$
          est de classe $\Cont{1}$ sur $[0, +\infty[$.
          
        \item La fonction intégrande $h_t : x \mapsto \ee^{t \, x -
            \frac{x^2}{2}}$ :
          \begin{noliste}{$\stimes$}
          \item n'admet pas d'équivalent plus simple en $+\infty$,
          \item tend très rapidement vers $0$ en $+\infty$.
          \end{noliste}         
          Du fait de ces deux points, on opte dans la démonstration
          pour un critère de négligeabilité. De manière informelle, la
          présence du terme $\ee^{-\frac{x^2}{2}}$ produit une
          convergence extrêmement rapide de $h_t$ vers $0$ en
          $+\infty$. L'intégrande $h_t$ apparaît donc suffisamment
          petite en $+\infty$ pour que l'intégrale sur $[1, +\infty[$
          associée soit convergente. Formellement, cette idée est
          concrétisée en comparant $h_t$ à l'intégrande $x \mapsto
          \frac{1}{x^2}$, positive et dont l'intégrale sur $[1,
          +\infty[$ associée est convergente.

        \item La démonstration de la propriété $\exp\left(t \, x -
            \dfrac{x^2}{2}\right) = \oox{+\infty} \left(
            \dfrac{1}{x^2} \right)$ n'est pas forcément un attendu de
          la question. Prendre l'initiative de la comparaion à la
          fonction $x \mapsto \frac{1}{x^2}$ démontre la bonne
          compréhension des mécanismes en jeu. Il faut évidemment
          savoir comment faire cette démonstration dont on donne
          ci-dessous les détails :
          \[
          \dfrac{\ee^{t \, x - \frac{x^2}{2}}}{\frac{1}{x^2}} \ = \
          x^2 \ \ee^{t \, x - \frac{x^2}{2}} \ = \
          \dfrac{x^2}{\ee^{\frac{x^2}{4}}} \ \ee^{t \, x -
            \frac{x^2}{4}} \tendx{+\infty} 0
          \]
          En effet : 
          \begin{noliste}{$\stimes$}
          \item en posant $u = \frac{x^2}{4}$, on obtient : $\dlim{x
              \tend +\infty} \dfrac{x^2}{\ee^{\frac{x^2}{4}}} =
            \dlim{u \tend +\infty} 4 \, \dfrac{u}{\ee^{u}} = 0$ {\it
              (par croissances comparées)}.

          \item $\dlim{x \tend +\infty} \ee^{t \, x - \frac{x^2}{2}} =
            0$.            
          \end{noliste}
        \end{noliste}
      \end{remarkL}~\\[-1.6cm]
    \end{proof}
    

    \newpage


  \item Montrer que la fonction $M_Z$ est définie sur $\R$ et donnée
    par : $\forall t \in \R$, $M_Z(t) = \exp\left( \dfrac{t^2}{2}
    \right)$.

    \begin{proof}~\\%
      Soit $t \in \R$.
      \begin{noliste}{$\sbullet$}
      \item Le réel $M_Z(t)$ existe si et seulement si la \var $\ee^{t
          \, Z}$ admet une espérance.\\[.1cm]
        Or, par théorème de transfert, la \var $\ee^{t \, Z}$ admet
        une espérance si et seulement si l'intégrale
        $\dint{-\infty}{+\infty} \ee^{t \, x} \ f_Z(x) \dx$ est
        absolument convergente.\\
        Les fonctions $x \mapsto \ee^{t \,x}$ \ et \ $f_Z$ étant à
        valeurs positives sur $\R$ ($f_Z$ est une densité de
        probabilité), cela revient à démonter que cette intégrale est
        convergente.%
        \concL{D'après la question précédente, cette intégrale est
          convergente pour tout $t \in \R$.\\
          Ainsi, $M_Z$ est définie sur $\R$.}{14}
        
      \item Remarquons que pour tout $x \in \ ]-\infty, +\infty[$ :
        \[
        \ee^{t \, x} \ f_Z(x) \ = \ \dfrac{1}{\sqrt{2 \, \pi}} \
        \ee^{t \, x} \ \ee^{-\frac{1}{2} \, x^2} \ = \
        \dfrac{1}{\sqrt{2 \, \pi}} \ \ee^{- \frac{1}{2} \, ( x^2 - 2
          \, tx)} \ = \ \dfrac{1}{\sqrt{2 \, \pi}} \ \ee^{-
          \frac{1}{2} \, ( x - t)^2 + \frac{1}{2} \, t^2} \ = \
        \dfrac{1}{\sqrt{2 \, \pi}} \ \ee^{\frac{1}{2} \, t^2} \ \ee^{-
          \frac{1}{2} \, ( x - t)^2}
        \]

      \item On effectue alors le changement de variable affine
        $\Boxed{u = x - t}$.
        \[
        \left|
          \begin{array}{P{6cm}}
            $u = x-t$ (et donc $x = u - t$) \nl
            $\hookrightarrow$ $du = \dx$ \quad et \quad $\dx = \ du$
            \nl  
            \vspace{-.2cm}
            \begin{noliste}{$\sbullet$}
            \item $x = -\infty \ \Rightarrow \ u = -\infty$\\[.2cm]
            \item $x = +\infty \ \Rightarrow \ u = +\infty$ %
              \vspace{-.8cm}
            \end{noliste}
          \end{array}
        \right.
        \]
        On obtient alors : 
        \[
        \begin{array}{rcl@{\qquad}>{\it}R{5cm}}
          \dint{-\infty}{+\infty} \ee^{t \, x} \ f_Z(x) \dx & = & 
          \ee^{\frac{1}{2} \, t^2} \ \dint{-\infty}{+\infty}
          \dfrac{1}{\sqrt{2 \, \pi}} \ \ee^{-\frac{1}{2} \, ( x -
            t)^2} \dx  
          \\[.6cm]
          & = & 
          \ee^{\frac{1}{2} \, t^2} \ \dint{-\infty}{+\infty}
          \dfrac{1}{\sqrt{2 \, \pi}} \ \ee^{-\frac{1}{2} \, u^2} \ du 
          \\[.6cm]            
          & = & \ee^{\frac{1}{2} \, t^2} \ \dint{-\infty}{+\infty}
          f_Z(u) \ du
          % \\[.4cm]
          \ = \ \ee^{\frac{1}{2} \, t^2}
          & (car $f_Z$ est une \\ densité de probabilité)
        \end{array}
        \]        
      \end{noliste}
      \conc{$\forall t \in \R$, $M_Z(t) = \exp\left( \dfrac{t^2}{2}
        \right)$}~\\[-1.4cm]
%       \begin{remarkL}{.98}%
%         \begin{noliste}{$\sbullet$}
%         \item Cette question est difficile car demande une prise
%           d'initiative importante. On peut penser, dans un premier
%           temps, à utiliser une intégration par parties pour calculer
%           l'intégrale considérée. Cependant, cette méthode a peu de
%           chances d'aboutir (si c'était le cas, on pourrait calculer
%           $\dint{-\infty}{+\infty} f_Z(x) \dx$ par intrégrations par
%           parties).

%         \item L'idée est donc de se ramener, par un changement de
%           variable, à un calcul d'intégrale de la densité $f_Z$.
%         \end{noliste}
%       \end{remarkL}
    \end{proof}
    
  \item En déduire la valeur de tous les cumulants d'une variable
    aléatoire qui suit une loi normale d'espérance $\mu \in \R$ et
    d'écart-type $\sigma \in \R_+^*$.

    \begin{proof}~%
      \begin{noliste}{$\sbullet$}
      \item Tout d'abord, remarquons que pour tout $t \in \R$ :
        $M_Z(t) = \ee^{\frac{1}{2} \, t^2} > 0$.\\
        Ainsi, $K_2$ est définie sur $\R$ et pour tout $t \in \R$ :
        \[
        K_Z(t) \ = \ \ln\left( \ee^{\frac{1}{2} \, t^2} \right) \ = \
        \frac{1}{2} \, t^2
        \]
        La fonction $K_Z$ est de classe $\Cont{\infty}$ sur $\R$ en
        tant que fonction polynomiale.%
        \conc{Ainsi, la \var $Z$ admet des cumulants à tout ordre.}


        \newpage


      \item On a alors, pour tout $t \in \R$ :
        \begin{noliste}{$\stimes$}
        \item $K'_Z(t) = t$ et ainsi : $Q_1(Z) = K'_Z(0) = 0$.
        \item $K''_Z(t) = 1$ et ainsi : $Q_2(Z) = K''_Z(0) = 1$.
        \item pour tout $p \in \llb 2, +\infty \llb$, $K^{(p)}_Z(t) =
          0$ et ainsi : $Q_p(Z) = K^{(p)}_Z(0) = 0$.
        \end{noliste}
        \conc{En conclusion : $Q_2(Z) = 1$ et pour tout $p \in \N^*
          \setminus \{2\}$, $Q_p(Z) = 0$.}~\\[-1.4cm]
      \end{noliste}
    \end{proof}
  \end{noliste}
  
\item Soit $(T_n)_{n \in \N^*}$ une suite de variables aléatoires
  telles que, pour tout $n \in \N^*$, la variable aléatoire $T_n$ suit
  la loi de Poisson de paramètre $n$. Pour tout $n \in \N^*$, on pose
  : $W_n = \dfrac{T_n - n}{\sqrt{n}}$.
  \begin{noliste}{a)}
    \setlength{\itemsep}{2mm}
  \item Justifier la convergence en loi de la suite de variables
    aléatoires $(W_n)_{n \in \N^*}$ vers une variable aléatoire $W$.
    \begin{proof}~
      \begin{noliste}{$\sbullet$}
      \item Pour tout $i \in \N^*$, notons $X_i$ une \var de loi de Poisson
        $\Pois{1}$. On suppose de plus que les \var $X_i$ sont indépendantes.\\
        Alors, par stabilité des lois de Poisson, la \var $S_n =
        \Sum{i=1}{n} X_i$ suit la loi de Poisson $\Pois{n}$.\\
        On en déduit que $S_n$ et $T_n$ suivent la même loi. Ainsi,
        les \var $\dfrac{S_n -n}{\sqrt{n}}$ et $\dfrac{T_n
          -n}{\sqrt{n}}=W_n$ ont même loi.
        
      \item La suite $(X_i)_{i \in \N^*}$ est constituée de \var :
        \begin{noliste}{$\stimes$}
        \item indépendantes,
          
        \item de même loi,
          
        \item admettant la même espérance $1$,
          
        \item admettant la même variance $1 \neq 0$.
        \end{noliste}
        Ainsi, comme $S_n \suit \Pois{n}$, la \var centrée réduite
        associée à $S_n$ est :
        \[
          S_n^* \ = \ \dfrac{S_n - \E(S_n)}{\sqrt{\V(S_n)}} \ = \
          \dfrac{S_n - n}{\sqrt{n}}
        \]
        
      \item Alors, d'après le théorème central limite :
        \[
          S_n^* \tendL W \quad \text{où} \quad W \suit \Norm{0}{1}
        \]
        On a donc : $\forall t \in \R$, $\dlim{n\to +\infty}
        F_{S_n^*}(t) = \Phi(t)$.\\
        {\it (la fonction $\Phi$ est la fonction de répartition de
          $W$)}
        
      \item Or les \var $S_n^* = \dfrac{S_n-n}{\sqrt{n}}$ et $W_n =
        \dfrac{T_n - n}{\sqrt{n}}$ ont
        même loi, donc même fonction de répartition. On en déduit :
        $\forall x \in \R$, $\dlim{n\to +\infty} F_{W_n}(t) =
        \Phi(t)$.
        \conc{On en conclut que la suite de \var $(W_n)_{n\in \N^*}$
          converge en loi vers $W$ où $W \suit \Norm{0}{1}$.}
      \end{noliste}


      \newpage


      \begin{remarkL}{1}
        \begin{noliste}{$\sbullet$}
        \item La forme de la \var $W_n = \dfrac{T_n -n}{\sqrt{n}}$ dans le
          contexte de convergence de \var doit faire penser {\bf
          systématiquement} au théorème central limite (TCL).
          
        \item De manière générale, si les $X_i$ ont pour espérance $m$
          et variance $\sigma^2$, alors :
          \[
            S_n^* \ = \ \dfrac{S_n - n \, m}{\sigma \, \sqrt{n}}
          \]
          On voit bien apparaître la division par $\sqrt{n}$ qui est
          très caractéristique de l'utilisation du TCL.
          
        \item Rappelons aussi que si on pose : $V_n = \dfrac{1}{n} \
          \Sum{i=1}{n} X_i$, alors :
          \[
            V_n^* \ = \ \sqrt{n} \ \dfrac{V_n - m}{\sigma}
          \]
          {\it (on notera : $V_n^* = S_n^*$)}\\
          On voit alors apparaître le produit par $\sqrt{n}$ qui est aussi
          caractéristique du TCL.
          
        \item Pour pouvoir justifier de l'application de ce théorème
          dans cette question, on introduit la \var $S_n$ qui suit la
          même loi que $T_n$ et satisfait bien aux hypothèses du
          TCL.\\
          On rappelle, comme remarqué en question \itbf{3.c)}, que deux
          \var qui suivent la même loi ne sont pas forcément égales
          (on n'a d'ailleurs pas du tout besoin dans cette question de
          l'égalité $S_n = T_n$).
          Cependant on peut raisonnablement penser qu'un candidat
          précisant \og $T_n = \Sum{i=1}{n} X_i$ \fg{} (et non : $S_n
          = \Sum{i=1}{n} X_i$) ne serait pas sanctionné.
        \end{noliste}
      \end{remarkL}~\\[-1.4cm]
    \end{proof}
    
  \item Déterminer la fonction $K_{W_n}$.
    \begin{proof}~
      \begin{noliste}{$\sbullet$}
      \item Comme $T_n \suit \Pois{n}$, d'après la question
        \itbf{9.a)} :
        \[
          K_{T_n} : t \mapsto n(\ee^t -1)
        \]
        
      \item De plus : $W_n = \dfrac{T_n -n}{\sqrt{n}} \ = \
        \dfrac{1}{\sqrt{n}} \ T_n - \dfrac{n}{\sqrt{n}} \ = \
        \dfrac{1}{\sqrt{n}} \ T_n - \sqrt{n}$.\\
        Ainsi, soit $t \in \R$, alors $\dfrac{1}{\sqrt{n}} \ t \in \R$
        et, d'après la question \itbf{5.b)} :
        \[
          K_{W_n}(t) \ = \ -\sqrt{n} \, t + K_{T_n}\left(
            \dfrac{1}{\sqrt{n}} \ t\right) \ = \ -\sqrt{n} \, t + n
          \left(\ee^{\frac{1}{\sqrt{n}} \ t} -1\right)
        \]
      \end{noliste}
      \conc{Finalement : $K_{W_n} : t \mapsto -\sqrt{n} \, t + n
        \left(\ee^{\frac{t}{\sqrt{n}}} -1\right)$.}~\\[-1cm]
    \end{proof}
    

    \newpage


  \item Montrer que pour tout $t \in \R$, on a : $\dlim{n\to +\infty}
    K_{W_n}(t) = K_W(t)$.
    \begin{proof}~\\
      Soit $t \in \R$.
      \begin{noliste}{$\sbullet$}
      \item Tout d'abord : $\ee^x = 1+x+ \dfrac{x^2}{2} +
        \oox{0}(x^2)$.\\
        Or : $\dlim{n \to +\infty} \dfrac{t}{\sqrt{n}} = 0$. On peut
        donc appliquer le développement limité précédent en
        choisissant $x = \dfrac{t}{\sqrt{n}}$. On obtient :
        \[
          \ee^{\frac{t}{\sqrt{n}}} = 1 + \dfrac{t}{\sqrt{n}} +
          \dfrac{\left(\frac{t}{\sqrt{n}}\right)^2}{2} +
          \oon \left(\left(\dfrac{t}{\sqrt{n}}
          \right)^2\right) \ = \ 1+ \dfrac{t}{\sqrt{n}} +
          \dfrac{t^2}{2n} + \oon\left( \dfrac{1}{n}\right)
        \]
        
      \item On en déduit, d'après la question précédente :
        \[
          \begin{array}{rcl}
            K_{W_n}(t)
            & = & -\sqrt{n} \, t + n \left(\bcancel{1} +
                  \dfrac{t}{\sqrt{n}} + \dfrac{t^2}{2n} + \oon\left(
                  \dfrac{1}{n}\right) - \bcancel{1} \right)
            \\[.6cm]
            & = & -\sqrt{n} \, t + n \ \dfrac{t}{\sqrt{n}} +
                  \dfrac{t^2}{2} + \oon (1)
            \\[.6cm]
            & = & -\bcancel{\sqrt{n} \, t} + \bcancel{\sqrt{n} \, t} +
                  \dfrac{t^2}{2} + \oon(1)
            \\[.6cm]
            & = & \dfrac{t^2}{2} + \oon(1)
          \end{array}
        \]
        On en déduit : $\dlim{n \to +\infty} K_{W_n}(t) =
        \dfrac{t^2}{2}$.
        
      \item Or, comme $W \suit \Norm{0}{1}$, d'après la question
        \itbf{10.c)} : $K_W(t) = \dfrac{t^2}{2}$.
      \end{noliste}
      \conc{Finalement, pour tout $t\in \R$ : $\dlim{n \to +\infty}
        K_{W_n}(t) = \dfrac{t^2}{2} = K_W(t)$.}
      \begin{remark}
        \begin{noliste}{$\sbullet$}
        \item Soit $f$ une fonction et $(a,x_0) \in \R^2$.\\
          On rappelle la propriété utilisée dans le deuxième point
          :
          \[
            \dlim{x \to x_0} f(x) = a \quad \Leftrightarrow \quad f(x)
            = a + \oox{x_0}(1)
          \]
          
        \item On peut remarquer que cette question \itbf{11.} nous
          fait démontrer, {\bf dans le cas particulier d'une loi de
            Poisson}, l'implication suivante :
          \[
            W_n \tendL W \quad \Rightarrow \quad \forall t \in \R, \
            K_{W_n}(t) \tendn K_W(t)
          \]
          On peut se poser la question de la généralisation de cette
          propriété. En effet, il existe un lien entre $M_X$ et $F_X$,
          mais cela serait hors de portée du programme ECE.
        \end{noliste}
      \end{remark}~\\[-1.4cm]
    \end{proof}
  \end{noliste}
\end{noliste}


\newpage


\subsection*{Partie III. Cumulant d'ordre 4}

\noindent
Dans cette partie, on considère une variable aléatoire $X$ telle que
$M_X$ est de classe $\Cont{4}$ sur un intervalle ouvert $I$ contenant
l'origine.\\
{\it On admet} alors que $X$ possède des moments jusqu'à l'ordre $4$
qui coïncident avec les dérivées successives de la fonction $M_X$ en
$0$. Autrement dit, pour tout $k \in \llb 1,4 \rrb$, on a :
$M_X^{(k)}(0) = \E(X^k)$.\\[.1cm]
De plus, on pose : $\mu_4(X) = \E\left( \big(X - \E(X)\big)^4
\right)$.
\begin{remarkL}{.98}%
  Dans la définition de $\mu_4$, on sous-entend que la \var $\big(X -
  \E(X)\big)^4$ admet une espérance.\\
  Dans le cours, on démontre les propriétés suivantes :
  \begin{noliste}{$\stimes$}
  \item si $X$ admet une espérance, il en est de même de $X - \E(X)$
    (c'est la \var centrée associée à $X$).\\
    Dans ce cas, on a : $\mu_1(X) = \E\big( X - \E(X) \big) = 0$.

  \item si $X$ admet un moment d'ordre $2$, il en est de même de $X -
    \E(X)$.\\
    Dans ce cas, on a : $\mu_2(X) = \E\Big( \big(X - \E(X) \big)^2
    \Big) = \V(X)$.\\
    {\it (la variance est le moment centré d'ordre $2$)}
  \end{noliste}
  De manière générale si $X$ admet un moment d'ordre $n \in \N$, il en
  est de même de $X - \E(X)$.\\
  En effet :
  \[
  \begin{array}{rcl}
    \big( X - \E(X) \big)^n & = & \Sum{k=0}{n} \dbinom{n}{k} \ \big( -
    \E(X) \big)^{n-k} \ X^k 
  \end{array}
  \]
  Ainsi, la \var $\big( X - \E(X) \big)^n$ admet une espérance en tant
  que combinaison linéaire des \var $X^0$, $X_1$, \ldots, $X^n$, qui
  admettent toutes une espérance. 
\end{remarkL}
\begin{noliste}{1.}
  \setlength{\itemsep}{4mm} %
  \setcounter{enumi}{11}
\item Justifier les égalités : $Q_1(X) = \E(X)$ et $Q_2(X) = \V(X)$.
  
  \begin{proof}~%
    \begin{noliste}{$\sbullet$}
    \item La fonction $K_X$ est de classe $\Cont{4}$ sur $I$ car elle
      est la composée $K_X = \ln \circ M_X$ où :
      \begin{noliste}{$\stimes$}
      \item la fonction $M_X$ est :
        \begin{noliste}{$-$}
        \item est de classe $\Cont{4}$ sur $I$,
        \item telle que $M_X(I) \subset \ ]0, +\infty[$.\\[.1cm]
          En effet, comme : $\forall t \in I$, $\ee^{tX} > 0$, on a :
          $\forall t \in I$, $M_X(t) = \E(\ee^{tX}) > 0$.          
        \end{noliste}

      \item la fonction $\ln$ est de classe $\Cont{4}$ sur $]0,
        +\infty[$.
      \end{noliste}

    \item Déterminons les dérivées successives de $K_X$. Soit $t \in
      I$.
      \begin{noliste}{$\stimes$}
      \item Tout d'abord : $K'_X(t) = \dfrac{1}{M_X(t)} \times
        M'_X(t)$. 
        \[
        \begin{array}{rcl@{\qquad}>{\it}R{5cm}}
          Q_1(X) & = & K'_X(0)
          & (par définition)
          \nl
          \nl[-.2cm]          
          % \\[.2cm]
          & = & \dfrac{1}{M_X(0)} \times M'_X(0)
          & (d'après le résultat \\ précédent en $t = 0 \in I$)
          \nl
          \nl[-.2cm]          
          & = & \dfrac{1}{M_X(0)} \times \E(X)
          & (d'après le résultat \\ admis dans l'énoncé)
          \nl
          \nl[-.2cm]
          & = & \dfrac{1}{\E\big( \ee^{0} \big)} \ \E(X) 
          % \\[.6cm] 
          \ = \ \E(X)
        \end{array}
        \]
        \conc{$Q_1(X) = \E(X)$}


        \newpage


      \item Ensuite : $K''_X(t) = \left( -\dfrac{1}{\big( M_X(t)
            \big)^2} \times M'_X(t) \right) \ M'_X(t) +
        \dfrac{1}{M_X(t)} \times M''_X(t)$.
        % \\[.4cm]
        \[
        \begin{array}{rcl@{\qquad}>{\it}R{5cm}}
          Q_1(X) & = & K''_X(0)
          & (par définition)
          \nl
          \nl[-.2cm]          
           & = & \left( -\dfrac{1}{\big( M_X(0) \big)^2} \times
            M'_X(0) \right) \ M'_X(0) + \dfrac{1}{M_X(0)} \times
          M''_X(0)
          & (d'après le résultat \\ précédent en $t = 0 \in I$)
          \nl
          \nl[-.2cm]          
          & = & -\dfrac{1}{(1)^2} \times \E(X) \times \E(X) +
          \dfrac{1}{1} \times \E\big( X^2 \big)
          & (d'après le résultat \\ admis dans l'énoncé)
          \nl
          \nl[-.2cm]
          & = & \E\big( X^2 \big) - \big( \E(X) \big)^2
          % \\[.2cm]
          \ = \ \V(X) 
          & (d'après la formule \\ de K\oe{}nig-Huygens)
        \end{array}
        \]~\\[-1cm]
        \conc{$Q_2(X) = \V(X)$}
      \end{noliste}
    \end{noliste}~\\[-1.1cm]
    \begin{remarkL}{.98}
      \begin{noliste}{$\sbullet$}
      \item On se sert ici du résultat stipulant que toute \var $X$
        qui admet une espérance vérifie :
        \[
        \Boxed{X > 0 \quad \Rightarrow \quad \E(X) > 0} \ (*)
        \]
        Ce résultat n'est pas officiellement au programme des classes
        préparatoires commerciales. \\
        Seul le résultat plus faible suivant (nommé parfois positivité
        de l'espérance) figure :
        \[
        \Boxed{X \geq 0 \quad \Rightarrow \quad \E(X) \geq 0} \ (**)
        \]
        On notera au passage que l'on obtient le même résultat en
        supposant l'hypothèse plus faible : $\Prob\big( \Ev{X \geq 0}
        \big) = 1$.

      \item Pour démontrer le résultat $(*)$, on peut procéder par
        l'absurde.\\[.1cm]
        Supposons $X > 0$ et $\NON{\E(X) > 0}$ (autrement dit $\E(X)
        \leq 0)$.
        \begin{noliste}{$\stimes$}
        \item Comme $X > 0 \geq 0$, alors, d'après $(**)$ : $\E(X)
          \geq 0$.\\[.1cm]
          Comme $\E(X) \leq 0$, on en déduit $\E(X) = 0$.

        \item D'après l'inégalité de Markov : $\forall a > 0$,
          \ $\Prob\big( \Ev{X > a} \big) \leq \dfrac{\E(X)}{a} = 0$.\\[.1cm]
          Une probabilité étant toujours positive, on démontre ainsi :
          $\forall a > 0$, \ $\Prob\big( \Ev{X > a} \big) = 0$.

        \item Démontrons, à l'aide de cette dernière égalité :
          $\Prob\big( \Ev{X > 0} \big) = 0$.\\[.1cm]
          Pour ce faire, on remarque tout d'abord : $
          \begin{array}{c@{\quad}>{\it}R{4cm}}
            \Ev{X > 0} = \dcup{n = 1}{+\infty} \Ev{X > \frac{1}{n}}
            & (démonstration par double inclusion)
          \end{array}
          $\\
          On en déduit :
          \[
          \begin{array}{rcl@{\qquad}>{\it}R{6.5cm}}
            \Prob\big( \Ev{X > 0} \big) & = & \Prob\big( \dcup{n =
              1}{+\infty} \Ev{X > \frac{1}{n}} \big)
            \\[.4cm]
            & = & \dlim{N \tend +\infty} \Prob\big( \dcup{n =
              1}{N} \Ev{X > \frac{1}{n}} \big)
            & (d'après la propriété de \\ la limite monotone)
            \nl
            \nl[-.2cm]
            & = & \dlim{N \tend +\infty} \Prob\big( \Ev{X >
              \frac{1}{N}} \big)  
            & (car $\big( \Ev{X > \frac{1}{N}} \big)_{N \in \N^*}$ est une
            \\ suite croissante d'événements)
            \nl
            \nl[-.2cm]
            & = & \dlim{N \tend +\infty} 0 \ = \ 0
            & (car $\Prob\big( \Ev{X > \frac{1}{N}} \big) = 0$)
          \end{array}
          \]
          Absurde !
        \end{noliste}
      \end{noliste}
    \end{remarkL}~\\[-1.6cm]
  \end{proof}


  \newpage


\item Soit $X_1$ et $X_2$ deux variables aléatoires indépendantes et
  de même loi que $X$. On pose : $S = X_1 - X_2$.
  \begin{noliste}{a)}
    \setlength{\itemsep}{2mm}
  \item Montrer que la variable aléatoire $S$ possède un moment
    d'ordre $4$ et établir l'égalité :
    \[
    \E\big( S^4 \big) \ = \ 2 \, \mu_4(X) + 6 \, \big(\V(X)\big)^2
    \]

    \begin{proof}~%
      \begin{noliste}{$\sbullet$}
      \item D'après l'énoncé, les \var $X_1$ et $X_2$ ont même loi que
        $X$, \var qui admet une espérance.\\
        Ainsi $\E(X_1) = \E(X) = \E(X_2)$ et :
        \[
        \begin{array}{rcl@{\qquad}>{\it}R{5cm}}
          S & = & X_1 - X_2 
          \\[.2cm]
          & = & X_1 - X_2 - \E(X_1) + \E(X_2)
          \\[.2cm]
          & = & \big( X_1 - \E(X_1) \big) - \big(X_2 - \E(X_2) \big)
        \end{array}
        \]

      \item On en déduit, à l'aide de la formule du binôme de Newton :
        \[
        \begin{array}{rcl@{\qquad}>{\it}R{5cm}}
          S^4 & = & \Big( \big( X_1 - \E(X_1) \big) - \big(X_2 -
          \E(X_2) \big) \Big)
          \\[.4cm]
          & = & \Sum{k=0}{4} \dbinom{4}{k} \ \big( X_1 - \E(X_1)
          \big)^{4-k} \ \Big( - \big( X_2 - \E(X_2) \big)^{k} \Big)
          \\[.6cm]
          & = & \Sum{k=0}{4} \dbinom{4}{k} \ (-1)^k \ \big( X_1 - \E(X_1)
          \big)^{4-k} \ \big( X_2 - \E(X_2) \big)^{k}
        \end{array}        
        \]
        Remarquons alors que :
        \begin{noliste}{$\stimes$}
        \item les \var $X_1 - \E(X_1)$ et $X_2 - \E(X_2)$ admettent
          des moments à tout ordre $r \in \llb 0, 4 \rrb$ car suivent
          la même loi que $X - \E(X)$ qui admet un moment d'ordre $4$.

        \item d'après le lemme des coalitions, pour tout entier $k \in
          \llb 0, 4 \rrb$, les \var $\big( X_1 - \E(X_1) \big)^{4-k}$
          et $\big( X_2 - \E(X_2) \big)^{k}$ sont indépendantes car
          $X_1$ et $X_2$ le sont.
        \end{noliste}        
        On en déduit que pour tout $k \in \llb 0, 4 \rrb$, la \var
        $\big( X_1 - \E(X_1) \big)^{4-k} \ \big( X_2 - \E(X_2)
        \big)^{k}$ admet une espérance. Et par propriété de
        l'espérance :
        \[
        \E\Big( \big( X_1 - \E(X_1) \big)^{4-k} \ \big( X_2 - \E(X_2)
        \big)^{k} \Big) \ = \ \E\Big( \big( X_1 - \E(X_1) \big)^{4-k}
        \Big) \ \E\Big( \big( X_2 - \E(X_2) \big)^{k} \Big)
        \]
        \concL{La \var $S^4$ admet une espérance comme combinaison
          linéaire de \var qui admettent une espérance.}{14}

      \item De plus, par linéarité de l'espérance :
        \[
        \begin{array}{rcrl@{\qquad}>{\it}R{5cm}}
          \E \big( S^4 \big) & = & \multicolumn{3}{l}{
            \Sum{k=0}{4} \dbinom{4}{k} \ (-1)^k
          \ \E\Big( \big( X_1 - \E(X_1) \big)^{4-k} \Big) \ 
          \E\Big( \big( X_2 - \E(X_2) \big)^{k} \Big)
          }
          \\[.6cm]
          & = & 1 & \E\Big( \big( X_1 - \E(X_1) \big)^{4} \Big)
          \\[.4cm]
          & & -\, 4 & \E\Big( \big( X_1 - \E(X_1) \big)^{3} \Big) \ 
          \bcancel{\E\Big( \big( X_2 - \E(X_2) \big)^{1} \Big)}
          & (car $X_2-\E(X_2)$ est la \\ \var centrée associée à $X_2$)
          \nl
          \nl[-.2cm]
          & & +\, 6 & \E\Big( \big( X_1 - \E(X_1) \big)^{2} \Big) \ 
          \E\Big( \big( X_2 - \E(X_2) \big)^{2} \Big)
          \\[.4cm]
          & & -\, 4 & \bcancel{\E\Big( \big( X_1 - \E(X_1) \big)^{1}
            \Big)} \ \E\Big( \big( X_2 - \E(X_2) \big)^{3} \Big)
          & (car $X_1-\E(X_1)$ est la \\ \var centrée associée à $X_1$)
          \nl
          \nl[-.2cm]
          & & +\, 1 & \E\Big( \big( X_2 - \E(X_2) \big)^{4} \Big)
          \\[.4cm]
          & = & \multicolumn{3}{l}{
            \mu_4(X_1) \ + \ 6 \, \V(X_1) \ \V(X_2) \ + \ \mu_4(X_2)
            }
        \end{array}
        \]


        \newpage


        \noindent 
        Enfin, comme $X_1$ et $X_2$ ont même loi que $X$ :
        \[
        \V(X_1) \ = \ \V(X) \ = \ \V(X_2) \quad \text{ et } \quad
        \mu_4(X_1) \ = \ \mu_4(X) \ = \ \mu_4(X_2)
        \]
        \conc{Et ainsi : $\E \big( S^4 \big) \ = \ 2 \, \mu_4(X_1) + 6
          \ \big( \V(X) \big)^2$.}~\\[-1.5cm]
      \end{noliste}
%       \begin{remarkL}{.98}%~
%         \begin{noliste}{$\sbullet$}
%         \item Dans le programme ECE, on trouve la propriété :
%           \[
%           \Boxed{
%             \begin{array}{C{5.6cm}cC{5cm}}
%               Les \var $X$ et $Y$ ont même loi & \Leftrightarrow & Les
%               \var $X$ et $Y$ ont même fonction de répartition
%             \end{array}
%           }
%           \]

%         \item On peut alors démontrer que si $X$ et $Y$ sont des \var
%           discrètes (resp. à densité), on a :
%           \[
%           \Boxed{
%             \begin{array}{rcl}
%               \left.
%                 \begin{array}{R{6cm}}
%                   \begin{noliste}{$\sbullet$}
%                   \item Les \var $X$ et $Y$ ont même loi
%                   \item Les \var $X$ et $Y$ admettent un moment d'ordre
%                     $n \in \N^*$\\[-.4cm]
%                   \end{noliste}             
%                 \end{array}
%               \right\}
%               & \Rightarrow & \E\big( X^n \big) = \E\big( Y^n \big)
%             \end{array}
%           }
%           \]
%           Cette dernière propriété est aussi vérifiée pour les \var
%           quelconques mais n'est pas explicitement écrite dans ce cas
%           précis. Toutefois, on peut considérer qu'on y a accès
%           puisque le programme précise : \og on admettra que les
%           propriétés opératoires usuelles de l'espérance et de la
%           variance se généralisent aux variables aléatoires
%           quelconques \fg{}.
%         \end{noliste}        
%       \end{remarkL}~\\[-1.5cm]
    \end{proof}
    
  \item Montrer que les fonctions $M_S$ et $K_S$ sont de classe
    $\Cont{4}$ sur $I$ et que pour tout $t \in I$, on a :
    \[
    M_S^{(4)}(t) \ = \ K_S^{(4)}(t) \, M_S(t) + 3 \, K_S^{(3)}(t) \,
    M_S'(t) + 3 \, K_S''(t) \, M_S''(t) + K_S'(t) \, M_S^{(3)}(t)
    \]

    \begin{proof}~\\%
      Soit $t \in I$ tel que $-t \in I$. 
      \begin{noliste}{$\sbullet$}
      \item Comme $-t \in I$, la quantité $M_X(-t)$ est bien
        définie. On a alors :
        \[
        \begin{array}{R{2cm}ccccccc@{\qquad}>{\it}R{3cm}}
          & M_X(-t) & = & \E\big( \ee^{-t \, X} \big) & = & \E\big(
          \ee^{t \, (-X)} \big) & = & M_{-X}(t)
          \\[.2cm]
          & \shortparallel & & & & & & \shortparallel
          \\ %[.2cm]
          & M_{X_2}(-t) & & & & & & M_{-X_2}(t)
          & (car $X$ et $X_2$ \\ ont même loi)
        \end{array}
        \]
        \conc{Ainsi, pour tout $t \in I$ tel que $-t \in I$, on a :
          $M_{-X_2}(t) = M_{X_2}(-t)$.} %{13.4}

      \item D'autre part :
        \[
        \ee^{t \, S} \ = \ \ee^{t \, (X_1 - X_2)} \ = \ \ee^{t \, X_1}
        \times \ee^{-t \, X_2}
        \]
        Les \var $\ee^{t \, X_1}$ et $\ee^{-t \, X_2}$ :
        \begin{noliste}{$\stimes$}
        \item admettent toutes les deux une espérance car $M_{X_1}$
          est définie en $t$ et $M_{X_2}$ est définie en $-t$ (car on
          a supposé $-t \in I$).

        \item sont indépendantes d'après le lemme des coalitions
          puisque $X_1$ et $X_2$ le sont.          
        \end{noliste}
        On en déduit que la \var $\ee^{t \, X_1} \times \ee^{-t \,
          X_2}$ admet une espérance, donnée par : 
        \[
        \begin{array}{R{2cm}ccccc@{\qquad}>{\it}R{4cm}}
          & \E\big( \ee^{t \, S} \big) & = & \E\big( \ee^{t \, X_1} \times
          \ee^{-t \, X_2} \big) & = & \E\big( \ee^{t \, X_1} \big)
          \times \E\big( \ee^{-t \, X_2} \big)
          \\[.2cm]
          & \shortparallel & & & & \shortparallel
          \\ %[.2cm]
          & M_S(t) & & & & M_{X_1}(t) \times M_{X_2}(-t)
          & (d'après le \\ point précédent)
          \nl
          %\nl[-.2cm]
          & & & & & \shortparallel
          \\ %[.2cm]
          & & & & & M_{X}(t) \times M_{X}(-t)
          & (car les \var $X_2$ et $X_1$ ont même loi que $X$)
        \end{array}
        \]
        \conc{Ainsi, pour tout intervalle $J \subset I$ symétrique par
          rapport à l'origine, on a : \\[.1cm]
          $\forall t \in J$, $M_S(t) \ = \ M_{X}(t) \times
          M_{X}(-t)$.}
      \end{noliste}
      \concL{En particulier, la fonction $M_S$ est de classe
        $\Cont{4}$ sur tout intervalle $J \subset I$ symétrique par
        rapport à l'origine, comme produit de fonctions de classe
        $\Cont{4}$ sur cet intervalle. On en déduit, comme en question
        \itbf{12}, que la fonction $K_S$ est elle aussi de classe
        $\Cont{4}$ sur $J$.}{15.4}%
      Dans la suite, on note $J \subset I$ un intervalle symétrique
      par rapport à l'origine. Soit $t \in J$.
      \begin{noliste}{$\sbullet$}
      \item Tout d'abord : $K'_S(t) \ = \ \dfrac{1}{M_S(t)} \times
        M'_S(t)$. %
        \conc{On en déduit : $\forall t \in J$, $M'_S(t) \ = \ K'_S(t)
          \times M_S(t)$.}


        \newpage


      \item En remarquant : $\big( M_S \big)^{(4)}(t) = \big( M'_S
        \big)^{(3)}(t)$, on obtient :
        \[
        \begin{array}{cl@{\qquad}>{\it}R{5cm}}
          & \big( M_S \big)^{(4)}(t) 
          \\[.4cm]
%           \ = \ \big( M'_S \big)^{(3)}(t)
%           \\[.4cm]
          = & \big(K'_S \times M_S \big)^{(3)}(t)
          \\[.4cm]
          = & \Big( \big(K'_S \times M_S \big)' \Big)^{(2)}(t)
          \\[.4cm]
          = & \Big( K''_S \times M_S + K'_S \times M'_S \Big)^{(2)}(t)
          \\[.4cm]
          = & \Big( \big( K''_S \times M_S + K'_S \times M'_S \big)'
          \Big)^{(1)}(t)
          \\[.4cm]
          = & \Big( \big( K'''_S \times M_S + K''_S \times M'_S
          \big) + \big( K''_S \times M'_S + K'_S \times M''_S \big)
          \Big)^{(1)}(t)
          \\[.4cm]
          = & \Big( K'''_S \times M_S + 2 \, K''_S \times M'_S +
          K'_S \times M''_S \Big)^{(1)}(t)
          \\[.4cm]
          = & \Big( \big( K''''_S \times M_S + K'''_S \times M'_S \big) + 2
          \, \big( K'''_S \times M'_S +  K''_S \times M''_S \big) + 
          \big( K''_S \times M''_S + K'_S \times M'''_S \big) \Big) (t)
          \\[.4cm]
          = & \Big( K''''_S \times M_S + 3 \, K'''_S \times M'_S + 3
          \, K''_S \times M''_S +  K'_S \times M'''_S \Big) (t)
          \\[.4cm]
          = & K^{(4)}_S(t) \, M_S(t) + 3 \, K^{(3)}_S(t) \, M'_S(t) + 3
          \, K''_S(t) \, M''_S(t) +  K'_S(t) \, M^{(3)}_S(t) 
        \end{array}
        \]
        \conc{$\forall t \in J$, $M_S^{(4)}(t) = K_S^{(4)}(t) \,
          M_S(t) + 3 \, K_S^{(3)}(t) \, M_S'(t) + 3 \, K_S''(t) \,
          M_S''(t) + K_S'(t) \, M_S^{(3)}(t)$}~\\[-1cm]
      \end{noliste}
      \begin{remarkL}{.98}%
        \begin{noliste}{$\sbullet$}
        \item Lors de l'étude de $M_S(t)$ (pour $t \in I$) la quantité
          $M_{X_2}(-t)$ apparaît naturellement. Or, cette quantité
          existe seulement si $-t \in I$. C'est pourquoi on a décidé
          dans cette question de restreindre à la démonstration à un
          intervalle $J \subset I$ symétrique par rapport à
          l'origine.
          % On suit d'ailleurs l'idée présentée en question
          % \itbf{5.b)} à savoir que pour tout $t \in I$ tel que $t
          % \in I$ : $K_{-X}(t) = K_X(-t)$.

        \item Dans la correction, on a déterminé la dérivée quatrième
          de $M_S$ en dérivant successivement trois fois la fonction
          $M_S'$. On aurait pu utiliser directement la formule de
          Leibniz qui stipule que si $f, g : I \to \R$ sont deux
          fonctions $n$ fois dérivables sur un intervalle $I$, on a :
          \[
          \Boxed{ %
            \big( f \times g \big)^{(n)} \ = \ \Sum{k=0}{n}
            \dbinom{n}{k} \ f^{(n-k)} \times g^{(k)} %
          }
          \]
          Cette formule n'apparaît pas explicitement dans le programme
          ECE (mais est bien présente dans le programme ECS de
          première année). Il est toutefois très classique de la
          présenter lors de la première année ECE. On pouvait
          l'utiliser directement ici et écrire :
          \[
          \begin{array}{rcl@{\qquad}>{\it}R{5cm}}
            \big( M_S \big)^{(4)}(t)             
            & = & \big( M'_S \big)^{(3)}(t)
            % \\[.4cm]
            \ = \ \Sum{k=0}{3} \dbinom{3}{k} \ {K'_S}^{(3-k)} \times
            \big( M_S \big)^{(k)}
            \\[.3cm]
            & = & \Sum{k=0}{3} \dbinom{3}{k} \ K_S^{(4-k)} \times
            \big( M_S \big)^{(k)}
            \\[.5cm]
            & = & K^{(4)}_S(t) \, M_S(t) + 3 \, K^{(3)}_S(t) \, M'_S(t) + 3
            \, K''_S(t) \, M''_S(t) +  K'_S(t) \, M^{(3)}_S(t) 
          \end{array}
          \]
          La présence des coefficients $1$, $3$, $3$, $1$ dans la
          formule à démontrer doit mettre sur la piste de
          l'utilisation d'une formule utilisant les coefficients
          binomiaux.
        \end{noliste}
      \end{remarkL}~\\[-1.6cm]
    \end{proof}


    \newpage
    

  \item En déduire l'égalité : $\E(S^4) \ = \ Q_4(S) + 3 \, \big(
    \V(S) \big)^2$.

    \begin{proof}~%
      \begin{noliste}{$\sbullet$}
      \item Notons tout d'abord que la fonction $M_S$ est de classe
        $\Cont{4}$ sur l'intervalle $J$ (qu'on peut choisir ouvert)
        qui contient l'origine. On admet donc, comme dans l'énoncé :
        \[
        \forall k \in \llb 1, 4 \rrb, \ M_S^{(k)}(0) = \E\big( S^k
        \big)
        \]

      \item La \var $S$ vérifiant les mêmes hypothèses que la \var $X$
        de début de la {\bf Partie III}, on en déduit, que le résultat
        de la question \itbf{12} est vérifié pour la \var $S$. Plus
        précisément :
        \[
        Q_1(S) = \E(S) \quad \text{ et } \quad Q_2(S) = \V(S)
        \]
        Remarquons au passage : 
        \[
        \begin{array}{rcl@{\qquad}>{\it}R{6cm}}
          \E(S) & = & \E(X_1 - X_2) 
          \\[.2cm]
          & = & \E(X_1) - \E(X_2) 
          & (par linéarité de l'espérance)
          \nl
          \nl[-.2cm]
          & = & \E(X) - \E(X) \ = \ 0
          & (car $X_1$ et $X_2$ ont même loi que $X$)
        \end{array}
        \]
        Et ainsi :
        \[
        \begin{array}{rcl@{\qquad}>{\it}R{6.6cm}}
          \V(S) & = & \E\big( S^2 \big) - \bcancel{\big( \E(S) \big)^2}
          & (d'après la formule de K\oe{}nig-Huygens)
          \nl
          \nl[-.2cm]
          & = & \E\big( S^2 \big)
        \end{array}
        \]

      \item On applique alors l'égalité précédente à $t = 0 \in J$ :
        \[
        \begin{array}{cccccccc@{\qquad}>{\it}R{5cm}}
          & \multicolumn{7}{l}{
            M_S^{(4)}(0) 
          }
          \\[.4cm]
          = & K^{(4)}_S(0) \times M_S(0) & + & 3 \, K^{(3)}_S(0) \times
          M'_S(0) & + & 3 \, K''_S(0) \times M''_S(0) & + & K'_S(0)
          \times M^{(3)}_S(0)  
          \\[.3cm]
          = & Q_4(S) \times M_S(0) & + & 3 \, Q_3(S) \times M'_S(0) & +
          & 3 \, Q_2(S) \times M''_S(0) & + & Q_1(S) 
          \times M^{(3)}_S(0)  
          \\[.3cm]
          = & Q_4(S) \times \E\big( \ee^{0} \big) & + & 3 \, Q_3(S)
          \times \E(S) & + & 3 \, Q_2(S) \times \E\big( S^2
          \big) & + & Q_1(S) \times \E\big( S^3 \big)
          \\[.3cm]
          = & Q_4(S) & + & 3 \, Q_3(S) \times \bcancel{\E(S)} & + & 3
          \, \V(S) \times \E\big( S^2 \big) & + & \bcancel{\E(S)}
          \times \E\big( S^3 \big)  
          \\[.3cm]
          = & 
          Q_4(S) & + & 3 \, \V(S) \times \V(S)
        \end{array}
        \]
      \end{noliste}
      \conc{On a bien : $\E(S^4) \ = \ M_S^{(4)}(0) \ = \ Q_4(S) + 3
        \, \big( \V(S) \big)^2$.}~\\[-1.2cm]
    \end{proof}
  \end{noliste}
  
\item Justifier que le cumulant d'ordre $4$ de $X$ est donné par la
  relation : $Q_4(X) \ = \ \mu_4(X) - 3 \, \big(\V(X)\big)^2$.
  
  \begin{proof}~%
    \begin{noliste}{$\sbullet$}
    \item Tout d'abord : 
      \[
      \begin{array}{rcl@{\qquad}>{\it}R{7cm}}
        Q_4(S) & = & \E(S^4) - 3 \, \big( \V(S) \big)^2
        & (d'après la question \itbf{13.c)})
        \nl
        \nl[-.2cm]
        & = & \Big( 2 \, \mu_4(X) + 6 \, \big(\V(X)\big)^2 \Big) - 3
        \, \big( \V(S) \big)^2 
        & (d'après la question \itbf{13.a)})
        \nl
        \nl[-.2cm]
        & = & \Big( 2 \, \mu_4(X) + 6 \, \big(\V(X)\big)^2 \Big) - 3
        \, \big( \V( X_1 - X_2) \big)^2 
        \\[.2cm]
        & = & \Big( 2 \, \mu_4(X) + 6 \, \big(\V(X)\big)^2 \Big) - 3
        \, \big( \V(X_1) + \V(- X_2) \big)^2 
        & (car $X_1$ et $-X_2$ sont indépendantes d'après le
        lemme des coalitions)
        \nl
        \nl[-.3cm]
        & = & 
        \multicolumn{2}{l}{
          \Big( 2 \, \mu_4(X) + 6 \, \big(\V(X)\big)^2 \Big) - 3
          \, \big( \V(X_1) + (-1)^2 \, \V(X_2) \big)^2 
        }
        \\[.4cm]
        & = & \Big( 2 \, \mu_4(X) + 6 \, \big(\V(X)\big)^2 \Big) - 3
        \, \big( \V(X) + \V(X) \big)^2 
        & (car $X_1$ et $X_2$ ont même loi que $X$)
        \nl
        \nl[-.3cm]
        & = & 
        \multicolumn{2}{l}{
          2 \, \mu_4(X) + 6 \, \big( \V(X) \big)^2 - 12 \, \big(
          \V(X) \big)^2 \ = \ 2 \, \mu_4(X) - 6 \, \big( \V(X) \big)^2 
        }
      \end{array}
      \]


      \newpage


    \item Il reste alors à exprimer $Q_4(S)$ en fonction de
      $Q_4(X)$.\\[.1cm]
      Rappelons tout d'abord qu'on a démontré en question \itbf{13.b)}
      que pour tout $t \in J$ : 
      \[
      M_S(t) \ = \ M_X(t) \ M_{X}(-t)
      \]
      On en conclut :
      \[
      \begin{array}{rcl@{\qquad}>{\it}R{5cm}}
        K_S(t) & = & \ln\big( M_S(t) \big) 
        \\[.3cm]
        & = & \ln\big( M_X(t) \times M_{X}(-t) \big)
        \\[.3cm]
        & = & \ln\big( M_X(t) \big) + \ln\big( M_X(-t) \big)
        \\[.3cm]
        & = & K_X(t) + K_X(-t)
      \end{array}
      \]
      Ainsi, par dérivations successives des deux membres de cette
      égalité :
      \[
      \begin{array}{R{1.4cm}rcl@{\qquad}>{\it}R{5cm}}
        & K'_S(t) & = & K'_X(t) - K'_X(-t)
        \\[.4cm]
        donc & K''_S(t) & = & K''_X(t) - \big(- K''_X(-t) \big)
        \\[.2cm]
        & & = & K'_X(t) + K'_X(-t)
        \\[.4cm]
        et & K^{(3)}_S(t) & = & K^{(3)}_X(t) - K^{(3)}_X(-t)
        \\[.4cm]
        enfin & K^{(4)}_S(t) & = & K^{(4)}_X(t) + K^{(4)}_X(-t)
      \end{array}
      \]
      En particulier, pour $t = 0 \in J$ :
      \[
      \begin{array}{ccccc@{\qquad}>{\it}R{5cm}}
        K^{(4)}_S(0) & = & K^{(4)}_X(0) + K^{(4)}_X(0) & = & 2 \,
        K^{(4)}_X(0) 
        \\[.2cm]
        \shortparallel & & & & \shortparallel 
        \\[.2cm]
        Q_4(S) & & & & 2 \, Q_4(X)
      \end{array}
      \]

    \item On en conclut, d'après ce qui précède :
      \[
      2 \, Q_4(X) \ = \ 2 \, \mu_4(X) - 6 \, \big( \V(X) \big)^2 
      \]
    \end{noliste}
    \conc{Finalement : $Q_4(X) \ = \ \mu_4(X) - 3 \, \big( \V(X)
      \big)^2$.}~\\[-1.2cm]  
  \end{proof}
\end{noliste}


\end{document}