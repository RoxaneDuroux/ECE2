\documentclass[11pt]{article}%
\usepackage{geometry}%
\geometry{a4paper,
 lmargin = 2cm,rmargin = 2cm,tmargin = 2.5cm,bmargin = 2.5cm}

\usepackage{array}
\usepackage{paralist}

\usepackage[svgnames, usenames, dvipsnames]{xcolor}
\xdefinecolor{RecColor}{named}{Aqua}
\xdefinecolor{IncColor}{named}{Aqua}
\xdefinecolor{ImpColor}{named}{PaleGreen}

% \usepackage{frcursive}

\usepackage{adjustbox}

%%%%%%%%%%%
\newcommand{\cRB}[1]{{\color{Red} \pmb{#1}}} %
\newcommand{\cR}[1]{{\color{Red} {#1}}} %
\newcommand{\cBB}[1]{{\color{Blue} \pmb{#1}}}
\newcommand{\cB}[1]{{\color{Blue} {#1}}}
\newcommand{\cGB}[1]{{\color{LimeGreen} \pmb{#1}}}
\newcommand{\cG}[1]{{\color{LimeGreen} {#1}}}

%%%%%%%%%%

\usepackage{diagbox} %
\usepackage{colortbl} %
\usepackage{multirow} %
\usepackage{pgf} %
\usepackage{environ} %
\usepackage{fancybox} %
\usepackage{textcomp} %
\usepackage{marvosym} %

%%%%%%%%%% pour pouvoir faire des dashedline dans les tableaux
\usepackage{arydshln}

%%%%%%%%%% pour qu'une cellcolor ne recouvre pas le trait du tableau
\usepackage{hhline}%

\usepackage{pgfplots}
\pgfplotsset{compat=1.10}
\usepgfplotslibrary{patchplots}
\usepgfplotslibrary{fillbetween}
\usepackage{tikz,tkz-tab}
\usepackage{ifthen}
\usepackage{calc}
\usetikzlibrary{calc,decorations.pathreplacing,arrows,positioning} 
\usetikzlibrary{fit,shapes,backgrounds}
% \usepackage[nomessages]{fp}% http://ctan.org/pkg/fp

\usetikzlibrary{matrix,arrows,decorations.pathmorphing,
  decorations.pathreplacing} 

\newcommand{\myunit}{1 cm}
\tikzset{
    node style sp/.style={draw,circle,minimum size=\myunit},
    node style ge/.style={circle,minimum size=\myunit},
    arrow style mul/.style={draw,sloped,midway,fill=white},
    arrow style plus/.style={midway,sloped,fill=white},
}

%%%%%%%%%%%%%%
%%%%% écrire des inférieur égal ou supérieur égal avec typographie
%%%%% francaise
%%%%%%%%%%%%%

\renewcommand{\geq}{\geqslant}
\renewcommand{\leq}{\leqslant}
\renewcommand{\emptyset}{\varnothing}

\newcommand{\Leq}{\leqslant}
\newcommand{\Geq}{\geqslant}

%%%%%%%%%%%%%%
%%%%% Macro Celia
%%%%%%%%%%%%%

\newcommand{\ff}[2]{\left[#1, #2\right]} %
\newcommand{\fo}[2]{\left[#1, #2\right[} %
\newcommand{\of}[2]{\left]#1, #2\right]} %
\newcommand{\soo}[2]{\left]#1, #2\right[} %
\newcommand{\abs}[1]{\left|#1\right|} %
\newcommand{\Ent}[1]{\left\lfloor #1 \right\rfloor} %


%%%%%%%%%%%%%%
%%%%% tikz : comment dessiner un "oeil"
%%%%%%%%%%%%%

\newcommand{\eye}[4]% size, x, y, rotation
{ \draw[rotate around={#4:(#2,#3)}] (#2,#3) -- ++(-.5*55:#1) (#2,#3)
  -- ++(.5*55:#1); \draw (#2,#3) ++(#4+55:.75*#1) arc
  (#4+55:#4-55:.75*#1);
  % IRIS
  \draw[fill=gray] (#2,#3) ++(#4+55/3:.75*#1) arc
  (#4+180-55:#4+180+55:.28*#1);
  % PUPIL, a filled arc
  \draw[fill=black] (#2,#3) ++(#4+55/3:.75*#1) arc
  (#4+55/3:#4-55/3:.75*#1);%
}


%%%%%%%%%%
%% discontinuité fonction
\newcommand\pointg[2]{%
  \draw[color = red, very thick] (#1+0.15, #2-.04)--(#1, #2-.04)--(#1,
  #2+.04)--(#1+0.15, #2+.04);%
}%

\newcommand\pointd[2]{%
  \draw[color = red, very thick] (#1-0.15, #2+.04)--(#1, #2+.04)--(#1,
  #2-.04)--(#1-0.15, #2-.04);%
}%

%%%%%%%%%%
%%% 1 : position abscisse, 2 : position ordonnée, 3 : taille, 4 : couleur
%%%%%%%%%%
% \newcommand\pointG[4]{%
%   \draw[color = #4, very thick] (#1+#3, #2-(#3/3.75))--(#1,
%   #2-(#3/3.75))--(#1, #2+(#3/3.75))--(#1+#3, #2+(#3/3.75)) %
% }%

\newcommand\pointG[4]{%
  \draw[color = #4, very thick] ({#1+#3/3.75}, {#2-#3})--(#1,
  {#2-#3})--(#1, {#2+#3})--({#1+#3/3.75}, {#2+#3}) %
}%

\newcommand\pointD[4]{%
  \draw[color = #4, very thick] ({#1-#3/3.75}, {#2+#3})--(#1,
  {#2+#3})--(#1, {#2-#3})--({#1-#3/3.75}, {#2-#3}) %
}%

\newcommand\spointG[4]{%
  \draw[color = #4, very thick] ({#1+#3/1.75}, {#2-#3})--(#1,
  {#2-#3})--(#1, {#2+#3})--({#1+#3/1.75}, {#2+#3}) %
}%

\newcommand\spointD[4]{%
  \draw[color = #4, very thick] ({#1-#3/2}, {#2+#3})--(#1,
  {#2+#3})--(#1, {#2-#3})--({#1-#3/2}, {#2-#3}) %
}%

%%%%%%%%%%

\newcommand{\Pb}{\mathtt{P}}

%%%%%%%%%%%%%%%
%%% Pour citer un précédent item
%%%%%%%%%%%%%%%
\newcommand{\itbf}[1]{{\small \bf \textit{#1}}}


%%%%%%%%%%%%%%%
%%% Quelques couleurs
%%%%%%%%%%%%%%%

\xdefinecolor{cancelcolor}{named}{Red}
\xdefinecolor{intI}{named}{ProcessBlue}
\xdefinecolor{intJ}{named}{ForestGreen}

%%%%%%%%%%%%%%%
%%%%%%%%%%%%%%%
% barrer du texte
\usetikzlibrary{shapes.misc}

\makeatletter
% \definecolor{cancelcolor}{rgb}{0.127,0.372,0.987}
\newcommand{\tikz@bcancel}[1]{%
  \begin{tikzpicture}[baseline=(textbox.base),inner sep=0pt]
  \node[strike out,draw] (textbox) {#1}[thick, color=cancelcolor];
  \useasboundingbox (textbox);
  \end{tikzpicture}%
}
\newcommand{\bcancel}[1]{%
  \relax\ifmmode
    \mathchoice{\tikz@bcancel{$\displaystyle#1$}}
               {\tikz@bcancel{$\textstyle#1$}}
               {\tikz@bcancel{$\scriptstyle#1$}}
               {\tikz@bcancel{$\scriptscriptstyle#1$}}
  \else
    \tikz@bcancel{\strut#1}%
  \fi
}
\newcommand{\tikz@xcancel}[1]{%
  \begin{tikzpicture}[baseline=(textbox.base),inner sep=0pt]
  \node[cross out,draw] (textbox) {#1}[thick, color=cancelcolor];
  \useasboundingbox (textbox);
  \end{tikzpicture}%
}
\newcommand{\xcancel}[1]{%
  \relax\ifmmode
    \mathchoice{\tikz@xcancel{$\displaystyle#1$}}
               {\tikz@xcancel{$\textstyle#1$}}
               {\tikz@xcancel{$\scriptstyle#1$}}
               {\tikz@xcancel{$\scriptscriptstyle#1$}}
  \else
    \tikz@xcancel{\strut#1}%
  \fi
}
\makeatother

\newcommand{\xcancelRA}{\xcancel{\rule[-.15cm]{0cm}{.5cm} \Rightarrow
    \rule[-.15cm]{0cm}{.5cm}}}

%%%%%%%%%%%%%%%%%%%%%%%%%%%%%%%%%%%
%%%%%%%%%%%%%%%%%%%%%%%%%%%%%%%%%%%

\newcommand{\vide}{\multicolumn{1}{c}{}}

%%%%%%%%%%%%%%%%%%%%%%%%%%%%%%%%%%%
%%%%%%%%%%%%%%%%%%%%%%%%%%%%%%%%%%%


\usepackage{multicol}
% \usepackage[latin1]{inputenc}
% \usepackage[T1]{fontenc}
\usepackage[utf8]{inputenc}
\usepackage[T1]{fontenc}
\usepackage[normalem]{ulem}
\usepackage[french]{babel}

\usepackage{url}    
\usepackage{hyperref}
\hypersetup{
  backref=true,
  pagebackref=true,
  hyperindex=true,
  colorlinks=true,
  breaklinks=true,
  urlcolor=blue,
  linkcolor=red,
  bookmarks=true,
  bookmarksopen=true
}

%%%%%%%%%%%%%%%%%%%%%%%%%%%%%%%%%%%%%%%%%%%
%% Pour faire des traits diagonaux dans les tableaux
%% Nécessite slashbox.sty
%\usepackage{slashbox}

\usepackage{tipa}
\usepackage{verbatim,listings}
\usepackage{graphicx}
\usepackage{fancyhdr}
\usepackage{mathrsfs}
\usepackage{pifont}
\usepackage{tablists}
\usepackage{dsfont,amsfonts,amssymb,amsmath,amsthm,stmaryrd,upgreek,manfnt}
\usepackage{enumerate}

%\newcolumntype{M}[1]{p{#1}}
\newcolumntype{C}[1]{>{\centering}m{#1}}
\newcolumntype{R}[1]{>{\raggedright}m{#1}}
\newcolumntype{L}[1]{>{\raggedleft}m{#1}}
\newcolumntype{P}[1]{>{\raggedright}p{#1}}
\newcolumntype{B}[1]{>{\raggedright}b{#1}}
\newcolumntype{Q}[1]{>{\raggedright}t{#1}}

\newcommand{\alias}[2]{
\providecommand{#1}{}
\renewcommand{#1}{#2}
}
\alias{\R}{\mathbb{R}}
\alias{\N}{\mathbb{N}}
\alias{\Z}{\mathbb{Z}}
\alias{\Q}{\mathbb{Q}}
\alias{\C}{\mathbb{C}}
\alias{\K}{\mathbb{K}}

%%%%%%%%%%%%
%% rendre +infty et -infty plus petits
%%%%%%%%%%%%
\newcommand{\sinfty}{{\scriptstyle \infty}}

%%%%%%%%%%%%%%%%%%%%%%%%%%%%%%%
%%%%% macros TP Scilab %%%%%%%%
\newcommand{\Scilab}{\textbf{Scilab}} %
\newcommand{\Scinotes}{\textbf{SciNotes}} %
\newcommand{\faire}{\noindent $\blacktriangleright$ } %
\newcommand{\fitem}{\scalebox{.8}{$\blacktriangleright$}} %
\newcommand{\entree}{{\small\texttt{ENTRÉE}}} %
\newcommand{\tab}{{\small\texttt{TAB}}} %
\newcommand{\mt}[1]{\mathtt{#1}} %
% guillemets droits

\newcommand{\ttq}{\textquotesingle} %

\newcommand{\reponse}[1]{\longboxed{
    \begin{array}{C{0.9\textwidth}}
      \nl[#1]
    \end{array}
  }} %

\newcommand{\reponseL}[2]{\longboxed{
    \begin{array}{C{#2\textwidth}}
      \nl[#1]
    \end{array}
  }} %

\newcommand{\reponseR}[1]{\longboxed{
    \begin{array}{R{0.9\textwidth}}
      #1
    \end{array}
  }} %

\newcommand{\reponseRL}[2]{\longboxed{
    \begin{array}{R{#2\textwidth}}
      #1
    \end{array}
  }} %

\newcommand{\reponseC}[1]{\longboxed{
    \begin{array}{C{0.9\textwidth}}
      #1
    \end{array}
  }} %

\colorlet{pyfunction}{Blue}
\colorlet{pyCle}{Magenta}
\colorlet{pycomment}{LimeGreen}
\colorlet{pydoc}{Cyan}
% \colorlet{SansCo}{white}
% \colorlet{AvecCo}{black}

\newcommand{\visible}[1]{{\color{ASCo}\colorlet{pydoc}{pyDo}\colorlet{pycomment}{pyCo}\colorlet{pyfunction}{pyF}\colorlet{pyCle}{pyC}\colorlet{function}{sciFun}\colorlet{var}{sciVar}\colorlet{if}{sciIf}\colorlet{comment}{sciComment}#1}} %

%%%% à changer ????
\newcommand{\invisible}[1]{{\color{ASCo}\colorlet{pydoc}{pyDo}\colorlet{pycomment}{pyCo}\colorlet{pyfunction}{pyF}\colorlet{pyCle}{pyC}\colorlet{function}{sciFun}\colorlet{var}{sciVar}\colorlet{if}{sciIf}\colorlet{comment}{sciComment}#1}} %

\newcommand{\invisibleCol}[2]{{\color{#1}#2}} %

\NewEnviron{solution} %
{ %
  \Boxed{
    \begin{array}{>{\color{ASCo}} R{0.9\textwidth}}
      \colorlet{pycomment}{pyCo}
      \colorlet{pydoc}{pyDo}
      \colorlet{pyfunction}{pyF}
      \colorlet{pyCle}{pyC}
      \colorlet{function}{sciFun}
      \colorlet{var}{sciVar}
      \colorlet{if}{sciIf}
      \colorlet{comment}{sciComment}
      \BODY
    \end{array}
  } %
} %

\NewEnviron{solutionC} %
{ %
  \Boxed{
    \begin{array}{>{\color{ASCo}} C{0.9\textwidth}}
      \colorlet{pycomment}{pyCo}
      \colorlet{pydoc}{pyDo}
      \colorlet{pyfunction}{pyF}
      \colorlet{pyCle}{pyC}
      \colorlet{function}{sciFun}
      \colorlet{var}{sciVar}
      \colorlet{if}{sciIf}
      \colorlet{comment}{sciComment}
      \BODY
    \end{array}
  } %
} %

\newcommand{\invite}{--\!\!>} %

%%%%% nouvel environnement tabular pour retour console %%%%
\colorlet{ConsoleColor}{Blue!15}
\colorlet{function}{Red}
\colorlet{var}{Maroon}
\colorlet{if}{Magenta}
\colorlet{comment}{LimeGreen}

\newcommand{\tcVar}[1]{\textcolor{var}{\bf \small #1}} %
\newcommand{\tcFun}[1]{\textcolor{function}{#1}} %
\newcommand{\tcIf}[1]{\textcolor{if}{#1}} %
\newcommand{\tcFor}[1]{\textcolor{if}{#1}} %

\newcommand{\moins}{\!\!\!\!\!\!- }
\newcommand{\espn}{\!\!\!\!\!\!}

\usepackage{booktabs,varwidth} \newsavebox\TBox
\newenvironment{console}
{\begin{lrbox}{\TBox}\varwidth{\linewidth}
    \tabular{>{\tt\small}R{0.84\textwidth}}
    \nl[-.4cm]} {\endtabular\endvarwidth\end{lrbox}%
  \fboxsep=1pt\colorbox{ConsoleColor}{\usebox\TBox}}

\newcommand{\lInv}[1]{%
  $\invite$ #1} %

\newcommand{\lAns}[1]{%
  \qquad ans \ = \nl %
  \qquad \qquad #1} %

\newcommand{\lVar}[2]{%
  \qquad #1 \ = \nl %
  \qquad \qquad #2} %

\newcommand{\lDisp}[1]{%
  #1 %
} %

\newcommand{\ligne}[1]{\underline{\small \tt #1}} %

\newcommand{\ligneAns}[2]{%
  $\invite$ #1 \nl %
  \qquad ans \ = \nl %
  \qquad \qquad #2} %

\newcommand{\ligneVar}[3]{%
  $\invite$ #1 \nl %
  \qquad #2 \ = \nl %
  \qquad \qquad #3} %

\newcommand{\ligneErr}[3]{%
  $\invite$ #1 \nl %
  \quad !-{-}error #2 \nl %
  #3} %
%%%%%%%%%%%%%%%%%%%%%% 

\newcommand{\bs}[1]{\boldsymbol{#1}} %
\newcommand{\nll}{\nl[.4cm]} %
\newcommand{\nle}{\nl[.2cm]} %
%% opérateur puissance copiant l'affichage Scilab
%\newcommand{\puis}{\!\!\!~^{\scriptscriptstyle\pmb{\wedge}}}
\newcommand{\puis}{\mbox{$\hspace{-.1cm}~^{\scriptscriptstyle\pmb{\wedge}}
    \hspace{0.05cm}$}} %
\newcommand{\pointpuis}{.\mbox{$\hspace{-.15cm}~^{\scriptscriptstyle\pmb{\wedge}}$}} %
\newcommand{\Sfois}{\mbox{$\mt{\star}$}} %

%%%%% nouvel environnement tabular pour les encadrés Scilab %%%%
\newenvironment{encadre}
{\begin{lrbox}{\TBox}\varwidth{\linewidth}
    \tabular{>{\tt\small}C{0.1\textwidth}>{\small}R{0.7\textwidth}}}
  {\endtabular\endvarwidth\end{lrbox}%
  \fboxsep=1pt\longboxed{\usebox\TBox}}

\newenvironment{encadreL}
{\begin{lrbox}{\TBox}\varwidth{\linewidth}
    \tabular{>{\tt\small}C{0.25\textwidth}>{\small}R{0.6\textwidth}}}
  {\endtabular\endvarwidth\end{lrbox}%
  \fboxsep=1pt\longboxed{\usebox\TBox}}

\newenvironment{encadreF}
{\begin{lrbox}{\TBox}\varwidth{\linewidth}
    \tabular{>{\tt\small}C{0.2\textwidth}>{\small}R{0.70\textwidth}}}
  {\endtabular\endvarwidth\end{lrbox}%
  \fboxsep=1pt\longboxed{\usebox\TBox}}

\newenvironment{encadreLL}[2]
{\begin{lrbox}{\TBox}\varwidth{\linewidth}
    \tabular{>{\tt\small}C{#1\textwidth}>{\small}R{#2\textwidth}}}
  {\endtabular\endvarwidth\end{lrbox}%
  \fboxsep=1pt\longboxed{\usebox\TBox}}

%%%%% nouvel environnement tabular pour les script et fonctions %%%%
\newcommand{\commentaireDL}[1]{\multicolumn{1}{l}{\it
    \textcolor{comment}{$\slash\slash$ #1}}}

\newcommand{\commentaire}[1]{{\textcolor{comment}{$\slash\slash$ #1}}}

\newcounter{cptcol}

\newcommand{\nocount}{\multicolumn{1}{c}{}}

\newcommand{\sciNo}[1]{{\small \underbar #1}}

\NewEnviron{scilab}{ %
  \setcounter{cptcol}{0}
  \begin{center}
    \longboxed{
      \begin{tabular}{>{\stepcounter{cptcol}{\tiny \underbar
              \thecptcol}}c>{\tt}l}
        \BODY
      \end{tabular}
    }
  \end{center}
}

\NewEnviron{scilabNC}{ %
  \begin{center}
    \longboxed{
      \begin{tabular}{>{\tt}l} %
          \BODY
      \end{tabular}
    }
  \end{center}
}

\NewEnviron{scilabC}[1]{ %
  \setcounter{cptcol}{#1}
  \begin{center}
    \longboxed{
      \begin{tabular}{>{\stepcounter{cptcol}{\tiny \underbar
              \thecptcol}}c>{\tt}l}
        \BODY
      \end{tabular}
    }
  \end{center}
}

\newcommand{\scisol}[1]{ %
  \setcounter{cptcol}{0}
  \longboxed{
    \begin{tabular}{>{\stepcounter{cptcol}{\tiny \underbar
            \thecptcol}}c>{\tt}l}
      #1
    \end{tabular}
  }
}

\newcommand{\scisolNC}[1]{ %
  \longboxed{
    \begin{tabular}{>{\tt}l}
      #1
    \end{tabular}
  }
}

\newcommand{\scisolC}[2]{ %
  \setcounter{cptcol}{#1}
  \longboxed{
    \begin{tabular}{>{\stepcounter{cptcol}{\tiny \underbar
            \thecptcol}}c>{\tt}l}
      #2
    \end{tabular}
  }
}

\NewEnviron{syntaxe}{ %
  % \fcolorbox{black}{Yellow!20}{\setlength{\fboxsep}{3mm}
  \shadowbox{
    \setlength{\fboxsep}{3mm}
    \begin{tabular}{>{\tt}l}
      \BODY
    \end{tabular}
  }
}

%%%%% fin macros TP Scilab %%%%%%%%
%%%%%%%%%%%%%%%%%%%%%%%%%%%%%%%%%%%

%%%%%%%%%%%%%%%%%%%%%%%%%%%%%%%%%%%
%%%%% TP Python - listings %%%%%%%%
%%%%%%%%%%%%%%%%%%%%%%%%%%%%%%%%%%%
\newcommand{\Python}{\textbf{Python}} %

\lstset{% general command to set parameter(s)
basicstyle=\ttfamily\small, % print whole listing small
keywordstyle=\color{blue}\bfseries\underbar,
%% underlined bold black keywords
frame=lines,
xleftmargin=10mm,
numbers=left,
numberstyle=\tiny\underbar,
numbersep=10pt,
%identifierstyle=, % nothing happens
commentstyle=\color{green}, % white comments
%%stringstyle=\ttfamily, % typewriter type for strings
showstringspaces=false}

\newcommand{\pysolCpt}[2]{
  \setcounter{cptcol}{#1}
  \longboxed{
    \begin{tabular}{>{\stepcounter{cptcol}{\tiny \underbar
            \thecptcol}}c>{\tt}l}
        #2
      \end{tabular}
    }
} %

\newcommand{\pysol}[1]{
  \setcounter{cptcol}{0}
  \longboxed{
    \begin{tabular}{>{\stepcounter{cptcol}{\tiny \underbar
            \thecptcol}}c>{\tt}l}
        #1
      \end{tabular}
    }
} %

% \usepackage[labelsep=endash]{caption}

% avec un caption
\NewEnviron{pythonCap}[1]{ %
  \renewcommand{\tablename}{Programme}
  \setcounter{cptcol}{0}
  \begin{center}
    \longboxed{
      \begin{tabular}{>{\stepcounter{cptcol}{\tiny \underbar
              \thecptcol}}c>{\tt}l}
        \BODY
      \end{tabular}
    }
    \captionof{table}{#1}
  \end{center}
}

\NewEnviron{python}{ %
  \setcounter{cptcol}{0}
  \begin{center}
    \longboxed{
      \begin{tabular}{>{\stepcounter{cptcol}{\tiny \underbar
              \thecptcol}}c>{\tt}l}
        \BODY
      \end{tabular}
    }
  \end{center}
}

\newcommand{\pyVar}[1]{\textcolor{var}{\bf \small #1}} %
\newcommand{\pyFun}[1]{\textcolor{pyfunction}{#1}} %
\newcommand{\pyCle}[1]{\textcolor{pyCle}{#1}} %
\newcommand{\pyImp}[1]{{\bf #1}} %

%%%%% commentaire python %%%%
\newcommand{\pyComDL}[1]{\multicolumn{1}{l}{\textcolor{pycomment}{\#
      #1}}}

\newcommand{\pyCom}[1]{{\textcolor{pycomment}{\# #1}}}
\newcommand{\pyDoc}[1]{{\textcolor{pydoc}{#1}}}

\newcommand{\pyNo}[1]{{\small \underbar #1}}

%%%%%%%%%%%%%%%%%%%%%%%%%%%%%%%%%%%
%%%%%% Système linéaire paramétré : écrire les opérations au-dessus
%%%%%% d'un symbole équivalent
%%%%%%%%%%%%%%%%%%%%%%%%%%%%%%%%%%%

\usepackage{systeme}

\NewEnviron{arrayEq}{ %
  \stackrel{\scalebox{.6}{$
      \begin{array}{l} 
        \BODY \\[.1cm]
      \end{array}$}
  }{\Longleftrightarrow}
}

\NewEnviron{arrayEg}{ %
  \stackrel{\scalebox{.6}{$
      \begin{array}{l} 
        \BODY \\[.1cm]
      \end{array}$}
  }{=}
}

\NewEnviron{operationEq}{ %
  \scalebox{.6}{$
    \begin{array}{l} 
      \scalebox{1.6}{$\mbox{Opérations :}$} \\[.2cm]
      \BODY \\[.1cm]
    \end{array}$}
}

% \NewEnviron{arraySys}[1]{ %
%   \sysdelim\{.\systeme[#1]{ %
%     \BODY %
%   } %
% }

%%%%%

%%%%%%%%%%
%%%%%%%%%% ESSAI
\newlength\fboxseph
\newlength\fboxsepva
\newlength\fboxsepvb

\setlength\fboxsepva{0.2cm}
\setlength\fboxsepvb{0.2cm}
\setlength\fboxseph{0.2cm}

\makeatletter

\def\longboxed#1{\leavevmode\setbox\@tempboxa\hbox{\color@begingroup%
\kern\fboxseph{\m@th$\displaystyle #1 $}\kern\fboxseph%
\color@endgroup }\my@frameb@x\relax}

\def\my@frameb@x#1{%
  \@tempdima\fboxrule \advance\@tempdima \fboxsepva \advance\@tempdima
  \dp\@tempboxa\hbox {%
    \lower \@tempdima \hbox {%
      \vbox {\hrule\@height\fboxrule \hbox{\vrule\@width\fboxrule #1
          \vbox{%
            \vskip\fboxsepva \box\@tempboxa \vskip\fboxsepvb}#1
          \vrule\@width\fboxrule }%
        \hrule \@height \fboxrule }}}}

\newcommand{\boxedhv}[3]{\setlength\fboxseph{#1cm}
  \setlength\fboxsepva{#2cm}\setlength\fboxsepvb{#2cm}\longboxed{#3}}

\newcommand{\boxedhvv}[4]{\setlength\fboxseph{#1cm}
  \setlength\fboxsepva{#2cm}\setlength\fboxsepvb{#3cm}\longboxed{#4}}

\newcommand{\Boxed}[1]{{\setlength\fboxseph{0.2cm}
  \setlength\fboxsepva{0.2cm}\setlength\fboxsepvb{0.2cm}\longboxed{#1}}}

\newcommand{\mBoxed}[1]{{\setlength\fboxseph{0.2cm}
  \setlength\fboxsepva{0.2cm}\setlength\fboxsepvb{0.2cm}\longboxed{\mbox{#1}}}}

\newcommand{\mboxed}[1]{{\setlength\fboxseph{0.2cm}
  \setlength\fboxsepva{0.2cm}\setlength\fboxsepvb{0.2cm}\boxed{\mbox{#1}}}}

\newsavebox{\fmbox}
\newenvironment{fmpage}[1]
     {\begin{lrbox}{\fmbox}\begin{minipage}{#1}}
     {\end{minipage}\end{lrbox}\fbox{\usebox{\fmbox}}}

%%%%%%%%%%
%%%%%%%%%%

\DeclareMathOperator{\ch}{ch}
\DeclareMathOperator{\sh}{sh}

%%%%%%%%%%
%%%%%%%%%%

\newcommand{\norme}[1]{\Vert #1 \Vert}

%\newcommand*\widefbox[1]{\fbox{\hspace{2em}#1\hspace{2em}}}

\newcommand{\nl}{\tabularnewline}

\newcommand{\hand}{\noindent\ding{43}\ }
\newcommand{\ie}{\textit{i.e. }}
\newcommand{\cf}{\textit{cf }}

\newcommand{\Card}{\operatorname{Card}}

\newcommand{\aire}{\mathcal{A}}

\newcommand{\LL}[1]{\mathscr{L}(#1)} %
\newcommand{\B}{\mathscr{B}} %
\newcommand{\Bc}[1]{B_{#1}} %
\newcommand{\M}[1]{\mathscr{M}_{#1}(\mathbb{R})}

\DeclareMathOperator{\im}{Im}
\DeclareMathOperator{\kr}{Ker}
\DeclareMathOperator{\rg}{rg}
\DeclareMathOperator{\tr}{tr}
\DeclareMathOperator{\spc}{Sp}
\DeclareMathOperator{\sgn}{sgn}
\DeclareMathOperator{\supp}{Supp}

\newcommand{\Mat}{{\rm{Mat}}}
\newcommand{\Vect}[1]{{\rm{Vect}}\left(#1\right)}

\newenvironment{smatrix}{%
  \begin{adjustbox}{width=.9\width}
    $
    \begin{pmatrix}
    }{%      
    \end{pmatrix}
    $
  \end{adjustbox}
}

\newenvironment{sarray}[1]{%
  \begin{adjustbox}{width=.9\width}
    $
    \begin{array}{#1}
    }{%      
    \end{array}
    $
  \end{adjustbox}
}

\newcommand{\vd}[2]{
  \scalebox{.8}{
    $\left(\!
      \begin{array}{c}
        #1 \\
        #2
      \end{array}
    \!\right)$
    }}

\newcommand{\vt}[3]{
  \scalebox{.8}{
    $\left(\!
      \begin{array}{c}
        #1 \\
        #2 \\
        #3 
      \end{array}
    \!\right)$
    }}

\newcommand{\vq}[4]{
  \scalebox{.8}{
    $\left(\!
      \begin{array}{c}
        #1 \\
        #2 \\
        #3 \\
        #4 
      \end{array}
    \!\right)$
    }}

\newcommand{\vc}[5]{
  \scalebox{.8}{
    $\left(\!
      \begin{array}{c}
        #1 \\
        #2 \\
        #3 \\
        #4 \\
        #5 
      \end{array}
    \!\right)$
    }}

\newcommand{\ee}{\text{e}}

\newcommand{\dd}{\text{d}}

%%% Ensemble de définition
\newcommand{\Df}{\mathscr{D}}
\newcommand{\Cf}{\mathscr{C}}
\newcommand{\Ef}{\mathscr{C}}

\newcommand{\rond}[1]{\,\overset{\scriptscriptstyle \circ}{\!#1}}

\newcommand{\df}[2]{\dfrac{\partial #1}{\partial #2}} %
\newcommand{\dfn}[2]{\partial_{#2}(#1)} %
\newcommand{\ddfn}[2]{\partial^2_{#2}(#1)} %
\newcommand{\ddf}[2]{\dfrac{\partial^2 #1}{\partial #2^2}} %
\newcommand{\ddfr}[3]{\dfrac{\partial^2 #1}{\partial #2 \partial
    #3}} %


\newcommand{\dlim}[1]{{\displaystyle \lim_{#1} \ }}
\newcommand{\dlimPlus}[2]{
  \dlim{
    \scalebox{.6}{
      $
      \begin{array}{l}
        #1 \rightarrow #2\\
        #1 > #2
      \end{array}
      $}}}
\newcommand{\dlimMoins}[2]{
  \dlim{
    \scalebox{.6}{
      $
      \begin{array}{l}
        #1 \rightarrow #2\\
        #1 < #2
      \end{array}
      $}}}

%%%%%%%%%%%%%%
%% petit o, développement limité
%%%%%%%%%%%%%%

\newcommand{\oo}[2]{{\underset {{\overset {#1\rightarrow #2}{}}}{o}}} %
\newcommand{\oox}[1]{{\underset {{\overset {x\rightarrow #1}{}}}{o}}} %
\newcommand{\oon}{{\underset {{\overset {n\rightarrow +\infty}{}}}{o}}} %
\newcommand{\po}[1]{{\underset {{\overset {#1}{}}}{o}}} %
\newcommand{\neqx}[1]{{\ \underset {{\overset {x \to #1}{}}}{\not\sim}\ }} %
\newcommand{\eqx}[1]{{\ \underset {{\overset {x \to #1}{}}}{\sim}\ }} %
\newcommand{\eqn}{{\ \underset {{\overset {n \to +\infty}{}}}{\sim}\ }} %
\newcommand{\eq}[2]{{\ \underset {{\overset {#1 \to #2}{}}}{\sim}\ }} %
\newcommand{\DL}[1]{{\rm{DL}}_1 (#1)} %
\newcommand{\DLL}[1]{{\rm{DL}}_2 (#1)} %

\newcommand{\negl}{<<}

\newcommand{\neglP}[1]{\begin{array}{c}
    \vspace{-.2cm}\\
    << \\
    \vspace{-.7cm}\\
    {\scriptstyle #1}
  \end{array}}

%%%%%%%%%%%%%%
%% borne sup, inf, max, min
%%%%%%%%%%%%%%
\newcommand{\dsup}[1]{\displaystyle \sup_{#1} \ }
\newcommand{\dinf}[1]{\displaystyle \inf_{#1} \ }
\newcommand{\dmax}[1]{\max\limits_{#1} \ }
\newcommand{\dmin}[1]{\min\limits_{#1} \ }

\newcommand{\dcup}[2]{{\textstyle\bigcup\limits_{#1}^{#2}}\hspace{.1cm}}
%\displaystyle \bigcup_{#1}^{#2}}
\newcommand{\dcap}[2]{{\textstyle\bigcap\limits_{#1}^{#2}}\hspace{.1cm}}
% \displaystyle \bigcap_{#1}^{#2}
%%%%%%%%%%%%%%
%% opérateurs logiques
%%%%%%%%%%%%%%
\newcommand{\NON}[1]{\mathop{\small \tt{NON}} (#1)}
\newcommand{\ET}{\mathrel{\mathop{\small \mathtt{ET}}}}
\newcommand{\OU}{\mathrel{\mathop{\small \tt{OU}}}}
\newcommand{\XOR}{\mathrel{\mathop{\small \tt{XOR}}}}

\newcommand{\id}{{\rm{id}}}

\newcommand{\sbullet}{\scriptstyle \bullet}
\newcommand{\stimes}{\scriptstyle \times}

%%%%%%%%%%%%%%%%%%
%% Probabilités
%%%%%%%%%%%%%%%%%%
\newcommand{\Prob}{\mathbb{P}}
\newcommand{\Ev}[1]{\left[ {#1} \right]}
\newcommand{\E}{\mathbb{E}}
\newcommand{\V}{\mathbb{V}}
\newcommand{\Cov}{{\rm{Cov}}}
\newcommand{\U}[2]{\mathcal{U}(\llb #1, #2\rrb)}
\newcommand{\Uc}[2]{\mathcal{U}([#1, #2])}
\newcommand{\Ucof}[2]{\mathcal{U}(]#1, #2])}
\newcommand{\Ucoo}[2]{\mathcal{U}(]#1, #2[)}
\newcommand{\Ucfo}[2]{\mathcal{U}([#1, #2[)}
\newcommand{\Bern}[1]{\mathcal{B}\left(#1\right)}
\newcommand{\Bin}[2]{\mathcal{B}\left(#1, #2\right)}
\newcommand{\G}[1]{\mathcal{G}\left(#1\right)}
\newcommand{\Pois}[1]{\mathcal{P}\left(#1\right)}
\newcommand{\HG}[3]{\mathcal{H}\left(#1, #2, #3\right)}
\newcommand{\Exp}[1]{\mathcal{E}\left(#1\right)}
\newcommand{\Norm}[2]{\mathcal{N}\left(#1, #2\right)}

\DeclareMathOperator{\cov}{Cov}

\newcommand{\var}{v.a.r. }
\newcommand{\suit}{\hookrightarrow}

\newcommand{\flecheR}[1]{\rotatebox{90}{\scalebox{#1}{\color{red}
      $\curvearrowleft$}}}


\newcommand{\partie}[1]{\mathcal{P}(#1)}
\newcommand{\Cont}[1]{\mathcal{C}^{#1}}
\newcommand{\Contm}[1]{\mathcal{C}^{#1}_m}

\newcommand{\llb}{\llbracket}
\newcommand{\rrb}{\rrbracket}

%\newcommand{\im}[1]{{\rm{Im}}(#1)}
\newcommand{\imrec}[1]{#1^{- \mathds{1}}}

\newcommand{\unq}{\mathds{1}}

\newcommand{\Hyp}{\mathtt{H}}

\newcommand{\eme}[1]{#1^{\scriptsize \mbox{ème}}}
\newcommand{\er}[1]{#1^{\scriptsize \mbox{er}}}
\newcommand{\ere}[1]{#1^{\scriptsize \mbox{ère}}}
\newcommand{\nd}[1]{#1^{\scriptsize \mbox{nd}}}
\newcommand{\nde}[1]{#1^{\scriptsize \mbox{nde}}}

\newcommand{\truc}{\mathop{\top}}
\newcommand{\fois}{\mathop{\ast}}

\newcommand{\f}[1]{\overrightarrow{#1}}

\newcommand{\checked}{\textcolor{green}{\checkmark}}

\def\restriction#1#2{\mathchoice
              {\setbox1\hbox{${\displaystyle #1}_{\scriptstyle #2}$}
              \restrictionaux{#1}{#2}}
              {\setbox1\hbox{${\textstyle #1}_{\scriptstyle #2}$}
              \restrictionaux{#1}{#2}}
              {\setbox1\hbox{${\scriptstyle #1}_{\scriptscriptstyle #2}$}
              \restrictionaux{#1}{#2}}
              {\setbox1\hbox{${\scriptscriptstyle #1}_{\scriptscriptstyle #2}$}
              \restrictionaux{#1}{#2}}}
\def\restrictionaux#1#2{{#1\,\smash{\vrule height .8\ht1 depth .85\dp1}}_{\,#2}}

\makeatletter
\newcommand*{\da@rightarrow}{\mathchar"0\hexnumber@\symAMSa 4B }
\newcommand*{\da@leftarrow}{\mathchar"0\hexnumber@\symAMSa 4C }
\newcommand*{\xdashrightarrow}[2][]{%
  \mathrel{%
    \mathpalette{\da@xarrow{#1}{#2}{}\da@rightarrow{\,}{}}{}%
  }%
}
\newcommand{\xdashleftarrow}[2][]{%
  \mathrel{%
    \mathpalette{\da@xarrow{#1}{#2}\da@leftarrow{}{}{\,}}{}%
  }%
}
\newcommand*{\da@xarrow}[7]{%
  % #1: below
  % #2: above
  % #3: arrow left
  % #4: arrow right
  % #5: space left 
  % #6: space right
  % #7: math style 
  \sbox0{$\ifx#7\scriptstyle\scriptscriptstyle\else\scriptstyle\fi#5#1#6\m@th$}%
  \sbox2{$\ifx#7\scriptstyle\scriptscriptstyle\else\scriptstyle\fi#5#2#6\m@th$}%
  \sbox4{$#7\dabar@\m@th$}%
  \dimen@=\wd0 %
  \ifdim\wd2 >\dimen@
    \dimen@=\wd2 %   
  \fi
  \count@=2 %
  \def\da@bars{\dabar@\dabar@}%
  \@whiledim\count@\wd4<\dimen@\do{%
    \advance\count@\@ne
    \expandafter\def\expandafter\da@bars\expandafter{%
      \da@bars
      \dabar@ 
    }%
  }%  
  \mathrel{#3}%
  \mathrel{%   
    \mathop{\da@bars}\limits
    \ifx\\#1\\%
    \else
      _{\copy0}%
    \fi
    \ifx\\#2\\%
    \else
      ^{\copy2}%
    \fi
  }%   
  \mathrel{#4}%
}
\makeatother



\newcount\depth

\newcount\depth
\newcount\totaldepth

\makeatletter
\newcommand{\labelsymbol}{%
      \ifnum\depth=0
        %
      \else
        \rlap{\,$\bullet$}%
      \fi
}

\newcommand*\bernoulliTree[1]{%
    \depth=#1\relax            
    \totaldepth=#1\relax
    \draw node(root)[bernoulli/root] {\labelsymbol}[grow=right] \draw@bernoulli@tree;
    \draw \label@bernoulli@tree{root};                                   
}                                                                        

\def\draw@bernoulli@tree{%
    \ifnum\depth>0 
      child[parent anchor=east] foreach \type/\label in {left child/$E$,right child/$S$} {%
          node[bernoulli/\type] {\label\strut\labelsymbol} \draw@bernoulli@tree
      }
      coordinate[bernoulli/increment] (dummy)
   \fi%
}

\def\label@bernoulli@tree#1{%
    \ifnum\depth>0
      ($(#1)!0.5!(#1-1)$) node[fill=white,bernoulli/decrement] {\tiny$p$}
      \label@bernoulli@tree{#1-1}
      ($(#1)!0.5!(#1-2)$) node[fill=white] {\tiny$q$}
      \label@bernoulli@tree{#1-2}
      coordinate[bernoulli/increment] (dummy)
   \fi%
}

\makeatother

\tikzset{bernoulli/.cd,
         root/.style={},
         decrement/.code=\global\advance\depth by-1\relax,
         increment/.code=\global\advance\depth by 1\relax,
         left child/.style={bernoulli/decrement},
         right child/.style={}}


\newcommand{\eps}{\varepsilon}

% \newcommand{\tendi}[1]{\xrightarrow[\footnotesize #1 \rightarrow
%   +\infty]{}}%

\newcommand{\tend}{\rightarrow}%
\newcommand{\tendn}{\underset{n\to +\infty}{\longrightarrow}} %
\newcommand{\ntendn}{\underset{n\to
    +\infty}{\not\hspace{-.15cm}\longrightarrow}} %
% \newcommand{\tendn}{\xrightarrow[\footnotesize n \rightarrow
%   +\infty]{}}%
\newcommand{\Tendx}[1]{\xrightarrow[\footnotesize x \rightarrow
  #1]{}}%
\newcommand{\tendx}[1]{\underset{x\to #1}{\longrightarrow}}%
\newcommand{\ntendx}[1]{\underset{x\to #1}{\not\!\!\longrightarrow}}%
\newcommand{\tendd}[2]{\underset{#1\to #2}{\longrightarrow}}%
% \newcommand{\tendd}[2]{\xrightarrow[\footnotesize #1 \rightarrow
%   #2]{}}%
\newcommand{\tendash}[1]{\xdashrightarrow[\footnotesize #1 \rightarrow
  +\infty]{}}%
\newcommand{\tendashx}[1]{\xdashrightarrow[\footnotesize x \rightarrow
  #1]{}}%
\newcommand{\tendb}[1]{\underset{#1 \to +\infty}{\longrightarrow}}%
\newcommand{\tendL}{\overset{\mathscr L}{\underset{n \to
      +\infty}{\longrightarrow}}}%
\newcommand{\tendP}{\overset{\Prob}{\underset{n \to
      +\infty}{\longrightarrow}}}%
\newcommand{\tenddL}[1]{\overset{\mathscr L}{\underset{#1 \to
      +\infty}{\longrightarrow}}}%

\NewEnviron{attention}{ %
  ~\\[-.2cm]\noindent
  \begin{minipage}{\linewidth}
  \setlength{\fboxsep}{3mm}%
  \ \ \dbend \ \ %
  \fbox{\parbox[t]{.88\linewidth}{\BODY}} %
  \end{minipage}\\
}

\NewEnviron{sattention}[1]{ %
  ~\\[-.2cm]\noindent
  \begin{minipage}{#1\linewidth}
  \setlength{\fboxsep}{3mm}%
  \ \ \dbend \ \ %
  \fbox{\parbox[t]{.88\linewidth}{\BODY}} %
  \end{minipage}\\
}

%%%%% OBSOLETE %%%%%%

% \newcommand{\attention}[1]{
%   \noindent
%   \begin{tabular}{@{}l|p{11.5cm}|}
%     \cline{2-2}
%     \vspace{-.2cm} 
%     & \nl
%     \dbend & #1 \nl
%     \cline{2-2}
%   \end{tabular}
% }

% \newcommand{\attentionv}[2]{
%   \noindent
%   \begin{tabular}{@{}l|p{11.5cm}|}
%     \cline{2-2}
%     \vspace{-.2cm} 
%     & \nl
%     \dbend & #2 \nl[#1 cm]
%     \cline{2-2}
%   \end{tabular}
% }

\newcommand{\explainvb}[2]{
  \noindent
  \begin{tabular}{@{}l|p{11.5cm}|}
    \cline{2-2}
    \vspace{-.2cm} 
    & \nl
    \hand & #2 \nl [#1 cm]
    \cline{2-2}
  \end{tabular}
}


% \noindent
% \begin{tabular}{@{}l|lp{11cm}|}
%   \cline{3-3} 
%   \multicolumn{1}{@{}l@{\dbend}}{} & & #1 \nl
%   \multicolumn{1}{l}{} & & \nl [-.8cm]
%   & & #2 \nl
%   \cline{2-3}
% \end{tabular}

% \newcommand{\attention}[1]{
%   \noindent
%   \begin{tabular}{@{}@{}cp{11cm}}
%     \dbend & #1 \nl
%   \end{tabular}
% }

\newcommand{\PP}[1]{\mathcal{P}(#1)}
\newcommand{\FF}[1]{\mathcal{F}(#1)}

\newcommand{\DSum}[2]{\displaystyle\sum\limits_{#1}^{#2}\hspace{.1cm}}
\newcommand{\Sum}[2]{{\textstyle\sum\limits_{#1}^{#2}}\hspace{.1cm}}
\newcommand{\Serie}{\textstyle\sum\ }
\newcommand{\Prod}[2]{\textstyle\prod\limits_{#1}^{#2}}

\newcommand{\Prim}[3]{\left[\ {#1} \ \right]_{\scriptscriptstyle
   \hspace{-.15cm} ~_{#2}\, }^ {\scriptscriptstyle \hspace{-.15cm} ~^{#3}\, }}

% \newcommand{\Prim}[3]{\left[\ {#1} \ \right]_{\scriptscriptstyle
%     \!\!~_{#2}}^ {\scriptscriptstyle \!\!~^{#3}}}

\newcommand{\dint}[2]{\displaystyle \int_{#1}^{#2}\ }
\newcommand{\Int}[2]{{\rm{Int}}_{\scriptscriptstyle #1, #2}}
\newcommand{\dt}{\ dt}
\newcommand{\dx}{\ dx}

\newcommand{\llpar}[1]{\left(\!\!\!
    \begin{array}{c}
      \rule{0pt}{#1}
    \end{array}
  \!\!\!\right.}

\newcommand{\rrpar}[1]{\left.\!\!\!
    \begin{array}{c}
      \rule{0pt}{#1}
    \end{array}
  \!\!\!\right)}

\newcommand{\llacc}[1]{\left\{\!\!\!
    \begin{array}{c}
      \rule{0pt}{#1}
    \end{array}
  \!\!\!\right.}

\newcommand{\rracc}[1]{\left.\!\!\!
    \begin{array}{c}
      \rule{0pt}{#1}
    \end{array}
  \!\!\!\right\}}

\newcommand{\ttacc}[1]{\mbox{\rotatebox{-90}{\hspace{-.7cm}$\llacc{#1}$}}}
\newcommand{\bbacc}[1]{\mbox{\rotatebox{90}{\hspace{-.5cm}$\llacc{#1}$}}}

\newcommand{\comp}[1]{\overline{#1}}%

\newcommand{\dcomp}[2]{\stackrel{\mbox{\ \ \----}{\scriptscriptstyle
      #2}}{#1}}%

% \newcommand{\Comp}[2]{\stackrel{\mbox{\ \
%       \-------}{\scriptscriptstyle #2}}{#1}}

% \newcommand{\dcomp}[2]{\stackrel{\mbox{\ \
%       \-------}{\scriptscriptstyle #2}}{#1}}

\newcommand{\A}{\mathscr{A}}

\newcommand{\conc}[1]{
  \begin{center}
    \fbox{
      \begin{tabular}{c}
        #1
      \end{tabular}
    }
  \end{center}
}

\newcommand{\concC}[1]{
  \begin{center}
    \fbox{
    \begin{tabular}{C{10cm}}
      \quad #1 \quad
    \end{tabular}
    }
  \end{center}
}

\newcommand{\concL}[2]{
  \begin{center}
    \fbox{
    \begin{tabular}{C{#2cm}}
      \quad #1 \quad
    \end{tabular}
    }
  \end{center}
}

% \newcommand{\lims}[2]{\prod\limits_{#1}^{#2}}

\newtheorem{theorem}{Théorème}[]
\newtheorem{lemma}{Lemme}[]
\newtheorem{proposition}{Proposition}[]
\newtheorem{corollary}{Corollaire}[]

% \newenvironment{proof}[1][Démonstration]{\begin{trivlist}
% \item[\hskip \labelsep {\bfseries #1}]}{\end{trivlist}}
\newenvironment{definition}[1][Définition]{\begin{trivlist}
\item[\hskip \labelsep {\bfseries #1}]}{\end{trivlist}}
\newenvironment{example}[1][Exemple]{\begin{trivlist}
\item[\hskip \labelsep {\bfseries #1}]}{\end{trivlist}}
\newenvironment{examples}[1][Exemples]{\begin{trivlist}
\item[\hskip \labelsep {\bfseries #1}]}{\end{trivlist}}
\newenvironment{notation}[1][Notation]{\begin{trivlist}
\item[\hskip \labelsep {\bfseries #1}]}{\end{trivlist}}
\newenvironment{propriete}[1][Propriété]{\begin{trivlist}
\item[\hskip \labelsep {\bfseries #1}]}{\end{trivlist}}
\newenvironment{proprietes}[1][Propriétés]{\begin{trivlist}
\item[\hskip \labelsep {\bfseries #1}]}{\end{trivlist}}
\newenvironment{remarkSC}[1][Remarque]{\begin{trivlist}
\item[\hskip \labelsep {\bfseries #1}]}{\end{trivlist}}
\newenvironment{application}[1][Application]{\begin{trivlist}
\item[\hskip \labelsep {\bfseries #1}]}{\end{trivlist}}

% Environnement pour les réponses des DS
\newenvironment{answer}{\par\emph{Réponse :}\par{}}
{\vspace{-.6cm}\hspace{\stretch{1}}\rule{1ex}{1ex}\vspace{.3cm}}

\newenvironment{answerTD}{\vspace{.2cm}\par\emph{Réponse :}\par{}}
{\hspace{\stretch{1}}\rule{1ex}{1ex}\vspace{.3cm}}

\newenvironment{answerCours}{\noindent\emph{Réponse :}}
{\rule{1ex}{1ex}}%\vspace{.3cm}}


% footnote in footer
\newcommand{\fancyfootnotetext}[2]{%
  \fancypagestyle{dingens}{%
    \fancyfoot[LO,RE]{\parbox{0.95\textwidth}{\footnotemark[#1]\footnotesize
        #2}}%
  }%
  \thispagestyle{dingens}%
}

%%% définit le style (arabic : 1,2,3...) et place des parenthèses
%%% autour de la numérotation
\renewcommand*{\thefootnote}{(\arabic{footnote})}
% http://www.tuteurs.ens.fr/logiciels/latex/footnote.html

%%%%%%%% tikz axis
% \pgfplotsset{every axis/.append style={
%                     axis x line=middle,    % put the x axis in the middle
%                     axis y line=middle,    % put the y axis in the middle
%                     axis line style={<->,color=blue}, % arrows on the axis
%                     xlabel={$x$},          % default put x on x-axis
%                     ylabel={$y$},          % default put y on y-axis
%             }}

%%%% s'utilise comme suit

% \begin{axis}[
%   xmin=-8,xmax=4,
%   ymin=-8,ymax=4,
%   grid=both,
%   ]
%   \addplot [domain=-3:3,samples=50]({x^3-3*x},{3*x^2-9}); 
% \end{axis}

%%%%%%%%



%%%%%%%%%%%% Pour avoir des numéros de section qui correspondent à
%%%%%%%%%%%% ceux du tableau
\renewcommand{\thesection}{\Roman{section}.\hspace{-.3cm}}
\renewcommand{\thesubsection}{\Roman{section}.\arabic{subsection}.\hspace{-.2cm}}
\renewcommand{\thesubsubsection}{\Roman{section}.\arabic{subsection}.\alph{subsubsection})\hspace{-.2cm}}
%%%%%%%%%%%% 

%%% Changer le nom des figures : Fig. au lieu de Figure
\usepackage[font=small,labelfont=bf,labelsep=space]{caption}
\captionsetup{%
  figurename=Fig.,
  tablename=Tab.
}
% \renewcommand{\thesection}{\Roman{section}.\hspace{-.2cm}}
% \renewcommand{\thesubsection}{\Roman{section}
%   .\hspace{.2cm}\arabic{subsection}\ .\hspace{-.3cm}}
% \renewcommand{\thesubsubsection}{\alph{subsection})}

\newenvironment{tabliste}[1]
{\begin{tabenum}[\bfseries\small\itshape #1]}{\end{tabenum}} 

\newenvironment{noliste}[1] %
{\begin{enumerate}[\bfseries\small\itshape #1]} %
  {\end{enumerate}}

\newenvironment{nonoliste}[1] %
{\begin{enumerate}[\hspace{-12pt}\bfseries\small\itshape #1]} %
  {\end{enumerate}}

\newenvironment{arrayliste}[1]{ 
  % List with minimal white space to fit in small areas, e.g. table
  % cell
  \begin{minipage}[t]{\linewidth} %
    \begin{enumerate}[\bfseries\small\itshape #1] %
      {\leftmargin=0.5em \rightmargin=0em
        \topsep=0em \parskip=0em \parsep=0em
        \listparindent=0em \partopsep=0em \itemsep=0pt
        \itemindent=0em \labelwidth=\leftmargin\labelsep+0.25em}
    }{
    \end{enumerate}\end{minipage}
}

\newenvironment{nolistes}[2]
{\begin{enumerate}[\bfseries\small\itshape
    #1]\setlength{\itemsep}{#2 mm}}{\end{enumerate}}

\newenvironment{liste}[1]
{\begin{enumerate}[\hspace{12pt}\bfseries\small\itshape #1]}{\end{enumerate}} 



%%%%%%%% Pour les programmes de colle %%%%%%%

\newcommand{\cours}{{\small \tt (COURS)}} %
\newcommand{\poly}{{\small \tt (POLY)}} %
\newcommand{\exo}{{\small \tt (EXO)}} %
\newcommand{\culture}{{\small \tt (CULTURE)}} %
\newcommand{\methodo}{{\small \tt (MÉTHODO)}} %
\newcommand{\methodob}{\Boxed{\mbox{\tt MÉTHODO}}} %

%%%%%%%% Pour les TD %%%%%%%
\newtheoremstyle{exostyle} {\topsep} % espace avant
{.6cm} % espace apres
{} % Police utilisee par le style de thm
{} % Indentation (vide = aucune, \parindent = indentation paragraphe)
{\bfseries} % Police du titre de thm
{} % Signe de ponctuation apres le titre du thm
{ } % Espace apres le titre du thm (\newline = linebreak)
{\thmname{#1}\thmnumber{ #2}\thmnote{.
    \normalfont{\textit{#3}}}} % composants du titre du thm : \thmname
                               % = nom du thm, \thmnumber = numéro du
                               % thm, \thmnote = sous-titre du thm
 
\theoremstyle{exostyle}
\newtheorem{exercice}{Exercice}
\newtheorem{exerciceAP}{Exercice avec préparation}
\newtheorem{exerciceSP}{Exercice sans préparation}
\newtheorem*{exoCours}{Exercice}

%%%%%%%% Pour des théorèmes Sans Espaces APRÈS %%%%%%%
\newtheoremstyle{exostyleSE} {\topsep} % espace avant
{} % espace apres
{} % Police utilisee par le style de thm
{} % Indentation (vide = aucune, \parindent = indentation paragraphe)
{\bfseries} % Police du titre de thm
{} % Signe de ponctuation apres le titre du thm
{ } % Espace apres le titre du thm (\newline = linebreak)
{\thmname{#1}\thmnumber{ #2}\thmnote{.
    \normalfont{\textit{#3}}}} % composants du titre du thm : \thmname
                               % = nom du thm, \thmnumber = numéro du
                               % thm, \thmnote = sous-titre du thm
 
\theoremstyle{exostyleSE}
\newtheorem{exerciceSE}{Exercice}
\newtheorem*{exoCoursSE}{Exercice}

% \newcommand{\lims}[2]{\prod\limits_{#1}^{#2}}

\newtheorem{theoremSE}{Théorème}[]
\newtheorem{lemmaSE}{Lemme}[]
\newtheorem{propositionSE}{Proposition}[]
\newtheorem{corollarySE}{Corollaire}[]

% \newenvironment{proofSE}[1][Démonstration]{\begin{trivlist}
% \item[\hskip \labelsep {\bfseries #1}]}{\end{trivlist}}
\newenvironment{definitionSE}[1][Définition]{\begin{trivlist}
  \item[\hskip \labelsep {\bfseries #1}]}{\end{trivlist}}
\newenvironment{exampleSE}[1][Exemple]{\begin{trivlist} 
  \item[\hskip \labelsep {\bfseries #1}]}{\end{trivlist}}
\newenvironment{examplesSE}[1][Exemples]{\begin{trivlist}
\item[\hskip \labelsep {\bfseries #1}]}{\end{trivlist}}
\newenvironment{notationSE}[1][Notation]{\begin{trivlist}
\item[\hskip \labelsep {\bfseries #1}]}{\end{trivlist}}
\newenvironment{proprieteSE}[1][Propriété]{\begin{trivlist}
\item[\hskip \labelsep {\bfseries #1}]}{\end{trivlist}}
\newenvironment{proprietesSE}[1][Propriétés]{\begin{trivlist}
\item[\hskip \labelsep {\bfseries #1}]}{\end{trivlist}}
\newenvironment{remarkSE}[1][Remarque]{\begin{trivlist}
\item[\hskip \labelsep {\bfseries #1}]}{\end{trivlist}}
\newenvironment{applicationSE}[1][Application]{\begin{trivlist}
\item[\hskip \labelsep {\bfseries #1}]}{\end{trivlist}}

%%%%%%%%%%% Obtenir les étoiles sans charger le package MnSymbol
%%%%%%%%%%%
\DeclareFontFamily{U} {MnSymbolC}{}
\DeclareFontShape{U}{MnSymbolC}{m}{n}{
  <-6> MnSymbolC5
  <6-7> MnSymbolC6
  <7-8> MnSymbolC7
  <8-9> MnSymbolC8
  <9-10> MnSymbolC9
  <10-12> MnSymbolC10
  <12-> MnSymbolC12}{}
\DeclareFontShape{U}{MnSymbolC}{b}{n}{
  <-6> MnSymbolC-Bold5
  <6-7> MnSymbolC-Bold6
  <7-8> MnSymbolC-Bold7
  <8-9> MnSymbolC-Bold8
  <9-10> MnSymbolC-Bold9
  <10-12> MnSymbolC-Bold10
  <12-> MnSymbolC-Bold12}{}

\DeclareSymbolFont{MnSyC} {U} {MnSymbolC}{m}{n}

\DeclareMathSymbol{\filledlargestar}{\mathrel}{MnSyC}{205}
\DeclareMathSymbol{\largestar}{\mathrel}{MnSyC}{131}

\newcommand{\facile}{\rm{(}$\scriptstyle\largestar$\rm{)}} %
\newcommand{\moyen}{\rm{(}$\scriptstyle\filledlargestar$\rm{)}} %
\newcommand{\dur}{\rm{(}$\scriptstyle\filledlargestar\filledlargestar$\rm{)}} %
\newcommand{\costaud}{\rm{(}$\scriptstyle\filledlargestar\filledlargestar\filledlargestar$\rm{)}}

%%%%%%%%%%%%%%%%%%%%%%%%%

%%%%%%%%%%%%%%%%%%%%%%%%%
%%%%%%%% Fin de la partie TD

%%%%%%%%%%%%%%%%
%%%%%%%%%%%%%%%%
\makeatletter %
\newenvironment{myitemize}{%
  \setlength{\topsep}{0pt} %
  \setlength{\partopsep}{0pt} %
  \renewcommand*{\@listi}{\leftmargin\leftmargini \parsep\z@
    \topsep\z@ \itemsep\z@} \let\@listI\@listi %
  \itemize %
}{\enditemize} %
\makeatother
%%%%%%%%%%%%%%%%
%%%%%%%%%%%%%%%%

%% Commentaires dans la correction du livre

\newcommand{\Com}[1]{
% Define box and box title style
\tikzstyle{mybox} = [draw=black!50,
very thick,
    rectangle, rounded corners, inner sep=10pt, inner ysep=8pt]
\tikzstyle{fancytitle} =[rounded corners, fill=black!80, text=white]
\tikzstyle{fancylogo} =[ text=white]
\begin{center}

\begin{tikzpicture}
\node [mybox] (box){%

    \begin{minipage}{0.90\linewidth}
\vspace{6pt}  #1
    \end{minipage}
};
\node[fancytitle, right=10pt] at (box.north west) 
{\bfseries\normalsize{Commentaire}};

\end{tikzpicture}%

\end{center}
%
}

% \NewEnviron{remark}{%
%   % Define box and box title style
%   \tikzstyle{mybox} = [draw=black!50, very thick, rectangle, rounded
%   corners, inner sep=10pt, inner ysep=8pt] %
%   \tikzstyle{fancytitle} =[rounded corners, fill=black!80,
%   text=white] %
%   \tikzstyle{fancylogo} =[ text=white]
%   \begin{center}
%     \begin{tikzpicture}
%       \node [mybox] (box){%
%         \begin{minipage}{0.90\linewidth}
%           \vspace{6pt} \BODY
%         \end{minipage}
%       }; %
%       \node[fancytitle, right=10pt] at (box.north west) %
%       {\bfseries\normalsize{Commentaire}}; %
%     \end{tikzpicture}%
%   \end{center}
% }
% \makeatother


%%%
% Remarques comme dans le livre
%%%

\NewEnviron{remark}{%
  % Define box and box title style
  \tikzstyle{mybox} = [draw=black!50, very thick, rectangle, rounded
  corners, inner sep=10pt, inner ysep=8pt] %
  \tikzstyle{fancytitle} = [rounded corners , fill=black!80,
  text=white] %
  \tikzstyle{fancylogo} =[ text=white]
  \begin{center}
    \begin{tikzpicture}
      \node [mybox] (box){%
        \begin{minipage}{0.90\linewidth}
          \vspace{6pt} \BODY
        \end{minipage}
      }; %
      \node[fancytitle, right=10pt] at (box.north west) %
      {\bfseries\normalsize{Commentaire}}; %
    \end{tikzpicture}%
  \end{center}
}

\NewEnviron{remarkST}{%
  % Define box and box title style
  \tikzstyle{mybox} = [draw=black!50, very thick, rectangle, rounded
  corners, inner sep=10pt, inner ysep=8pt] %
  \tikzstyle{fancytitle} = [rounded corners , fill=black!80,
  text=white] %
  \tikzstyle{fancylogo} =[ text=white]
  \begin{center}
    \begin{tikzpicture}
      \node [mybox] (box){%
        \begin{minipage}{0.90\linewidth}
          \vspace{6pt} \BODY
        \end{minipage}
      }; %
      % \node[fancytitle, right=10pt] at (box.north west) %
%       {\bfseries\normalsize{Commentaire}}; %
    \end{tikzpicture}%
  \end{center}
}

\NewEnviron{remarkL}[1]{%
  % Define box and box title style
  \tikzstyle{mybox} = [draw=black!50, very thick, rectangle, rounded
  corners, inner sep=10pt, inner ysep=8pt] %
  \tikzstyle{fancytitle} =[rounded corners, fill=black!80,
  text=white] %
  \tikzstyle{fancylogo} =[ text=white]
  \begin{center}
    \begin{tikzpicture}
      \node [mybox] (box){%
        \begin{minipage}{#1\linewidth}
          \vspace{6pt} \BODY
        \end{minipage}
      }; %
      \node[fancytitle, right=10pt] at (box.north west) %
      {\bfseries\normalsize{Commentaire}}; %
    \end{tikzpicture}%
  \end{center}
}

\NewEnviron{remarkSTL}[1]{%
  % Define box and box title style
  \tikzstyle{mybox} = [draw=black!50, very thick, rectangle, rounded
  corners, inner sep=10pt, inner ysep=8pt] %
  \tikzstyle{fancytitle} =[rounded corners, fill=black!80,
  text=white] %
  \tikzstyle{fancylogo} =[ text=white]
  \begin{center}
    \begin{tikzpicture}
      \node [mybox] (box){%
        \begin{minipage}{#1\linewidth}
          \vspace{6pt} \BODY
        \end{minipage}
      }; %
%       \node[fancytitle, right=10pt] at (box.north west) %
%       {\bfseries\normalsize{Commentaire}}; %
    \end{tikzpicture}%
  \end{center}
}

\NewEnviron{titre} %
{ %
  ~\\[-1.8cm]
  \begin{center}
    \bf \LARGE \BODY
  \end{center}
  ~\\[-.6cm]
  \hrule %
  \vspace*{.2cm}
} %

\NewEnviron{titreL}[2] %
{ %
  ~\\[-#1cm]
  \begin{center}
    \bf \LARGE \BODY
  \end{center}
  ~\\[-#2cm]
  \hrule %
  \vspace*{.2cm}
} %




\pagestyle{fancy} %
\lhead{ECE2 \hfill septembre 2017 \\
 Mathématiques\\[.2cm]} %
\chead{\hrule} %
\rhead{} %
\lfoot{} %
\cfoot{} %
\rfoot{\thepage} %

\renewcommand{\headrulewidth}{0pt}% : Trace un trait de séparation
 % de largeur 0,4 point. Mettre 0pt
 % pour supprimer le trait.

\renewcommand{\footrulewidth}{0.4pt}% : Trace un trait de séparation
 % de largeur 0,4 point. Mettre 0pt
 % pour supprimer le trait.

\setlength{\headheight}{14pt}

\title{\bf \vspace{-1cm} ORAUX HEC 2008} %
\author{} %
\date{} %

\begin{document}

\maketitle %
\vspace{-1.2cm}\hrule %
\thispagestyle{fancy}

\vspace*{.4cm}

% DEBUT DU DOC À MODIFIER : tout virer jusqu'au début de l'exo

\section{Annales 2008}

 %\setcounter{exercice}{0}

 \begin{exercice}{\it (Exercice avec préparation)}~
 \begin{noliste}{1.}
 \setlength{\itemsep}{4mm}
 \item Toutes les variables admettent alors une variance et on a : \\
 $\V( \Sum{k = 1}{n} X_{k} ) = \Sum{k = 1}{n} \V(X_{k}) + 2 \Sum{1 \leq
i < j \leq n} \Cov (X_{i}, X_{j})$. \\
\\
 Soit $(X_{n})_{n \geq 1}$ une suite de variables aléatoires
indépendantes de même loi, à valeurs dans $\{-1 ; 1\}$, définies sur
une même espace probabilisé $(\Omega, \mathcal{A}, P)$. On pose, pour
tout $n \in \N^*$, $p = \Prob\left(\Ev{\Ev{[X_{n} = 1]}}\right)$, et on
suppose que $p \in \ ] 0 ; 1[$. \\
 \item Pour tout $n \in \N^*$, on pose $Y_{n} = \prod\limits_{i = 1}{n}
X_{i}$.
 \begin{noliste}{a)}
 \setlength{\itemsep}{2mm} 
 \item $Y_{2} (\Omega) = \{ -1 ; 1\}$ et $\Prob\left(\Ev{\Ev{Y_{2} =
1}}\right) = \Prob\left(\Ev{\Ev{ X_{1} = Y_{1}}}\right) = p^{2} +
(1-p)^{2}$, et $\Prob\left(\Ev{\Ev{Y_{2} = -1}}\right) =
\Prob\left(\Ev{ X_{1} \neq Y_{1}}\right) = 2 p (1-p)$. \\
\\
 $Y_{3} (\Omega) = \{ -1 ; 1\}$, $\Prob\left(\Ev{\Ev{Y_{3} = 1}}\right)
= \Prob\left(\Ev{\Ev{Y_{2} = X_{3}}}\right) = p (p^{2} + (1-p)^{2} ) +
2 p (1-p)^{2}$ et $\Prob\left(\Ev{\Ev{Y_{3} = -1}}\right) =
\Prob\left(\Ev{ Y_{2} \neq X_{3}}\right) = p^{2} (1-p) + (1-p)^{3} + 2
p^{2} (1-p)$. \\
 On peut s'amuser à simplifier ces résultats mais cela n'a pas grand
intérêt. \\
 \item De même $Y_{n} ( \Omega) = \{-1 ; 1\}$ pour tout $n$ et on a :
\\
 $\Prob\left(\Ev{Y_{n + 1} = 1}\right) = P \left(\Ev{Y_{n} X_{n + 1} =
1}\right) = \Prob\left(\Ev{Y_{n} = X_{n + 1}}\right) = p_{n} \times p +
(1-p_{n}) (1-p) = p_{n} (2p -1) + 1 - p$. \\
\\
 C'est une suite arithmético-géométrique, on résout l'équation $k = k
(2p-1) + 1-p \Leftrightarrow 2 k (1-p) = 1-p \Leftrightarrow k =
\frac{1}{2}$. \\
\\
 Puis on considère $u_{n} = p_{n} - \frac{1}{2}$, on montre qu'elle est
géométrique de raison $2p-1$ et on obtient $u_{n} = (p-1)^{n-1} u_{1}$,
donc $p_{n} = (2p-1)^{n-1} \left( p_{1} + \frac{1}{2} \right) +
\frac{1}{2} = (2p-1)^{n-1} \left( p + \frac{1}{2} \right) +
\frac{1}{2}$. \\
 \item Il faut pour cela $\Prob\left(\Ev{Y_{n} = i, Y_{n + 1} =
j}\right) = \Prob\left(\Ev{\Ev{Y_{n} = i}}\right) \Prob\left(\Ev{ Y_{n
+ 1} = j}\right)$ pour $i$ et $j$ dans $\{ 1 ; 2\}$. \\
 Or $\Prob\left(\Ev{Y_{n} = 1, Y_{n + 1} = 1}\right) = \Prob\left(\Ev{
Y_{n} = 1, X_{n + 1} = 1 }\right) = p_{n} p$ donc il faut que
$\Prob\left(\Ev{ Y_{n + 1} = 1}\right) = p_{n + 1} = p$ ou que $p_{n} =
0$. \\
\\
 De même $\Prob\left(\Ev{Y_{n} = -1, Y_{n + 1} = -1}\right) =
\Prob\left(\Ev{Y_{n} = -1, X_{n + 1} = 1}\right) = (1-p_{n}) p$ donc il
faut $\Prob\left(\Ev{Y_{n + 1} = -1}\right) = 1 - p_{n + 1} = p$ ou
$p_{n} = 1$.
 $\Prob\left(\Ev{Y_{n} = 1, Y_{n + 1} = -1}\right) =
\Prob\left(\Ev{Y_{n} = 1, X_{n + 1} = -1}\right) = p_{n} (1-p)$ donc il
faut $p_{n} = 0$ ou $1 - p_{n + 1} = 1-p$; cela ne donne rien de plus.
De même pour la dernière. \\
\\
 Comme on ne peut avoir $p_{n} = 0 = 1$, il y a deux possibilités :
soit $p_{n} = 0$ et $p_{n + 1} = 1-p$, soit $p_{n} = 1$ et $p_{n + 1} =
p$. \\
\\
 $p_{n} = 0$ donne $(2p-1)^{n-1} = - \frac{1}{2 \left( p + \frac{1}{2}
\right)} $ qui impose $2p - 1 < 0$ et $n-1$ impair, et $(n-1) \ln
(1-2p) = - \ln 2 - \ln \left( p + \frac{1}{2} \right)$ donc $n = 1 -
\frac{ \ln 2 + \ln \left( p + \frac{1}{2} \right)}{ \ln (1-2p)}$, $n$
impair et $p < \frac{1}{2}$. \\
\\
 On a alors $p_{n + 1} = p_{n} ( 2p-1) + 1 - p = 0 (2p-1) + 1-p = 1-p$
et la deuxième condition est bien vérifiée. \\
\\
 $p_{n} = 1$ donne $(2p-1)^{n-1} = \frac{1}{2 \left( p + \frac{1}{2}
\right)} $ qui impose $2p - 1 > 0$ ou $n-1$ pair; \\
 si $2p-1 <0$ on a $(n-1) \ln (1-2p) = \ln 2 + \ln \left( p +
\frac{1}{2} \right)$ donc $n = 1 + \frac{ \ln 2 + \ln \left( p +
\frac{1}{2} \right)}{ \ln (1-2p)}$, $n$ pair et $p < \frac{1}{2}$. \\
 si $2p-1 >0$ on a $(n-1) \ln (2p-1) = \ln 2 + \ln \left( p +
\frac{1}{2} \right)$ donc $n = 1 + \frac{ \ln 2 + \ln \left( p +
\frac{1}{2} \right)}{ \ln (2p-1)}$, $n$ pair ou impair et $p >
\frac{1}{2}$. \\
\\
 On a alors $p_{n + 1} = p_{n} ( 2p-1) + 1 - p = (2p-1) + 1-p = p$ et
la deuxième condition est bien vérifiée. \\
 \end{noliste}
 \item On détermine les valeurs de $S_{n}$ en considérant que si $k$
variables $X_{i}$ valent 1, alors $n-k$ valent $-1$ et $S_{n} = k -
(n-k) = 2k-n$. \\
 Comme la fonction de $k$ obtenue est bijective, la probabilité que
$S_{n} = 2k - n$ est celle que $k$ variables $X_{i}$ valent 1, et en
posant $X_{i}' = 1$ si $X_{i} = 1$ et 0 si $X_{i} = -1$,
$\Prob\left(\Ev{S_{n} = 2k - n}\right) = \Prob\left(\Ev{ \Sum{i = 1}{n}
X_{i}' = k}\right)$ qui est une loi binomiale. \\
 D'où $\Prob\left(\Ev{\Ev{S_{n} = 2k-n}}\right) = \binom{n}{k} p^{k}
(1-p)^{n-k}$, $\E(S_{n}) = 2 \E( \Sum{k = 1}{n} X_{i}' ) -n = 2 n p - n
= n (2p - 1)$ puis $\V(S_{n}) = 4 \V(\Sum{k = 1}{n} X_{i}') = 4 n
p(1-p)$. \\
\\
 Deuxième solution : on exprime tout de suite les $X_{i}'$ sous la
forme $X_{i}' = \frac{X_{i} + 1}{2}$ donc $X_{i} = 2 X_{i}' - 1$ et on
a $S_{n} = 2 S_{n}' -n$, avec $S_{n}'$ qui suit la loi binomiale de
paramètres $n$ et $p$. On retrouve le même résultat. \\
 \item La deuxième modélisation ci-dessus permet de le faire. \\
 var S, x, k, n : integer; p : real; \\
 begin ; \\
 readln (p); readln (n); S : = 0; \\
randomize ;\\
 for k : = 1 to n do \\
 begin \\
 x : = random (1); x : = 2x - 1 ; S : = S + x; \\
 end ; end. \\
\\
 Si on utilise la première modélisation, on remplace (x : = 2x-1 ;) par
(if x : = 0 then x : = -1;) et on obtient le même résultat.
 \\
 \end{noliste}
 \noindent \textbf{\underline{Exercice sans préparation}} \\
\\
 La question suivante permet de conjecturer que la limite est 1, et
$u_{n}$ est trivialement supérieure à 1, on va donc chercher à majorer
$u_{n} -1 = \frac{u_{n-1}}{n + 1}$ par une suite qui tend vers 0; cela
conduit à essayer de majorer $u_{n}$ par une constante. \\
 Le calcul des premiers termes ($u_{0} < 1$, $u_{1} = 1 + u_{0} < 2$,
$u_{2} = 1 + \frac{1 + u_{0}}{2} < 2$, etc...) permet de conjecturer
$u_{n} \leq 2$, qu'on prouve par récurrence : \\
 $u_{0} \leq 2$, $u_{1} \leq 2$ et $u_{2} \leq 2$ viennent d'être
prouvés. \\
 Si il existe $n \geq 2$ tel que $u_{n} \leq 2$, alors $u_{n + 1} = 1 +
\frac{u_{n}}{n + 1} \leq 1 + \frac{u_{n}}{3} \leq 1 + \frac{2}{3} \leq
2$ et le résultat est prouvé pour tout $n$. \\
\\
 Attention l'hérédité doit être faite à partir de $n = 1$ au plus bas;
en effet si on suppose $u_{0} \leq 2$ seulement on a $u_{1} \leq 1 + 2
\leq 3$ qui n'est pas suffisant. Mais comme l'énoncé donnait $u_{0}
\leq 1$ on a résolu le problème en initialisant aux valeurs $0$ et $1$.
\\
\\
 Enfin on obtient pour tout $n \geq 1$, $1 \leq u_{n} \leq 1 +
\frac{u_{n-1} }{n} \leq 1 + \frac{2}{n}$ et par théorème de
comparaison, $\lim u_{n} = 1$.
 \\
\\
 Pour trouver la valeur de $a$ on regarde $ (u_{n} -1) \times n$ qui
doit converger vers $a$. \\
 On a $n (u_{n} - 1) = \frac{n u_{n-1} }{n + 1} \xrightarrow[ n
\rightarrow + \infty]{} 1$ donc $a = 1$. \\
 Ensuite doit avoir $n[n(u_{n} -1) -1 ] \rightarrow b$ donc on étudie :
\\
 $n[ n (u_{n} - 1) - 1] = n \left( \frac{n u_{n-1} }{n + 1} -1 \right)
= n \frac{ n u_{n-1} - n -1}{n + 1} \sim n u_{n-1} - n - 1 = n (
u_{n-1} - 1) -1 = (n-1 + 1) ( u_{n-1} -1 ) - 1 = (n-1) (u_{n-1} - 1) +
u_{n-1} - 2 \xrightarrow[ n \rightarrow + \infty]{} 1 + 1 - 2 = 0$ donc
$b = 0$. \\
\\
 On obtient $n[ n (u_{n} - 1) - 1] = o(1)$ donc $n (u_{n} -1) -1 = o
\left( \frac{1}{n} \right)$, $n (u_{n} -1) = 1 + o \left( \frac{1}{n}
\right)$, $u_{n} - 1 = \frac{1}{n} + o \left( \frac{1}{n^{2}} \right)$,
et enfin $u_{n} = 1 + \frac{1}{n} + o \left( \frac{1}{n^{2}} \right)$.

 \end{exercice}

 \newpage

 \begin{exercice}{\it (Exercice avec préparation)}~
 \begin{noliste}{1.}
 \setlength{\itemsep}{4mm}
 \item Une variable suit la loi de Bernouilli si elle a deux issues
possibles : le succès pour lequel elle vaut 1, et l'échec pour lequel
elle vaut 0. \\
 On a alors $\Prob\left(\Ev{\Ev{X = 1}}\right) = p = \E(X)$ et $\V(X) =
p (1-p)$. \\
 On réalise une succession de $n$ épreuves de Bernouilli indépendantes
et identiques de paramètre $p$, et on compte le nombre de succès : on
obtient alors une loi binomiale de paramètres $n$ et $p$. \\
 La variable $X$ associée vérifie alors $X(\Omega) = \llb 0 ; b \rrb$.
\\
 Pour calculer $\Prob\left(\Ev{\Ev{X = k}}\right)$, on compte le nombre
de possibilités amenant à ce résultat et la probabilité de chacune. \\
 Il faut obtenir $k$ succès et $n-k$ échecs : on place les $k$ succès
parmi les $n$ épreuves pour obtenir toutes les possibilités : il y en a
donc $\binom{n}{k}$. \\
 Dans chacun de ces cas, on obtient de manière indépendante $k$ succès
et $n$ échecs avec une probabilité $p^{k} q^{n-k}$. \\
 On obtient alors $\Prob\left(\Ev{\Ev{X = k}}\right) = \binom{n}{k}
p^{k} q^{n-k}$. \\
 L'espérance s'obtient en écrivant les $n$ $X_{i}$ variables de
Bernouilli, avec la linéarité de l'espérance : $\E(X) = n p$. \\
 De même grâce à l'indépendance on calcule facilement la variance de la
somme : $\V(X) = n p (1-p)$. \\
 \item $X_{i}$ suit une loi de Bernouilli, et $\Prob\left(\Ev{\Ev{X_{i}
= 1}}\right) = \frac{ \binom{1}{1} \times \binom{2n-1}{n-1} }{
\binom{2n}{n} } = \frac{ (2n-1)! \times (n!)^{2} }{(2n)! (n-1)! (n!)} =
\frac{n}{ 2n } = \frac{1}{2 }$. \\
 \item $X_ i X_{j}$ est la variable de Bernouilli qui vaut 1 si la
boule $i$ et la boule $j$ sont dans la poignée. \\
 D'où $\E( X_{i} X_{j}) = \Prob\left(\Ev{ X_{i} X_{j} = 1}\right) =
\frac{ \binom{2}{2} \times \binom{2n-2}{n-2} }{ \binom{2n}{n} } =
\frac{ (2n-2)! \times (n!)^{2} }{(2n)! (n-2)! (n!)} = \frac{n(n-1)}{ 2n
(2n-1) } = \frac{n-1}{4n-2 }$. \\
\\
 On en déduit que $\Cov (X_{i}, X_{j}) = \frac{n-1}{4n-2} - \frac{1}{4}
= \frac{ 2 n - 2 - ( 2n - 1) }{8n - 4} = -\frac{1}{4 (2n -1)}$. \\
 \item On note $S$ la variable aléatoire réelle prenant pour valeur la
somme des numéros portés par les boules figurant dans la poignée. 
 \begin{noliste}{a)}
 \setlength{\itemsep}{2mm}
 \item $S = \Sum{i = 1}{n} i X_{i}$. \\
 \item Par linéarité de l'espérance, $\E(S) = \Sum{i = 1}{n}
\frac{i}{2} = \frac{1}{2} \times \frac{n (n + 1)}{2} = \frac{n (n +
1)}{4}$. \\
 Par indépendance des $X_{i}$ on a $\V( S) = \Sum{i = 1}{n} i^{2}
\V(X_{i}) = \frac{1}{4} \times \frac{n (n + 1) (2n-1)}{6} = \frac{ n (n
+ 1) (2n-1)}{24}$. \\
 \end{noliste}
 \item On note $Y = \Sum{i = 1}{n} X_{i}$ qui compte le nombre de
boules non numérotées $0$ dans la poignée, et qui suit une loi
binomiale de paramètres $n$ et $\frac{1}{2}$. \\
 On a alors $Z = n - Y$ donc $Z(\Omega) = \llb 0 ; n \rrb$ et
$\Prob\left(\Ev{\Ev{Z = k}}\right) = \Prob\left(\Ev{\Ev{ Y =
n-k}}\right) = \binom{n}{k} \frac{1}{2^{n}}$. \\
 On voit que $Z$ suit une loi binomiale, et on a $\E(Z) = \frac{n}{2}$.
\\
 \end{noliste}
 \noindent \textbf{\underline{Exercice sans préparation}} \\
\\
 Soit $f$ la fonction définie par : 
 
\[
 \forall (x,y) \in \R^{2},\ f(x,y) = x^{3} + y^{3} - 3 xy + 1.
 
\]
 \begin{noliste}{1.}
 \setlength{\itemsep}{4mm}
 \item Aucune difficulté ici : \\
 $f_{x}'(x,y) = 3 x^{2} - 3 y$, $f_{y}'(x,y) = 3 y^{2} - 3 x$,
$f_{x,x}''(x,y) = 6 x$, $f_{x,y}''(x,y) = -3$ et $f_{y,y}''(x,y) = 6
y^{2}$. \\
 \item $f_{x}'(x,y) = f_{y}'(x,y) = 0$ donne $x^{2} = y$ (donc $y$
positif) et $y^{2} = x$ (donc $x$ positif), puis $y^{2} = x^{4} = x$
donc $x = 0$ ou $x = 1$, puis $x = 0$ donne $y = 0$ et $x = 1$ donne $y
= 1$. \\
 Les deux points critiques sont $(0,0)$ et $(1,1)$. \\
 \item En $(0,0)$ on a $r = t = 0$ et $s = -3$ donc $rt - s^{2} = -9 <
0$, c'est un point selle. \\
 En $(1,1)$ on a $r = t = 6$ et $s = -3$ donc $rt-s^{2} = 36-9 = 27>0$,
avec $r>0$ donc c'est un minimum local.
 \end{noliste}
 \end{exercice}


 \newpage


 \begin{exercice}{\it (Exercice avec préparation)}~
 \begin{noliste}{1.}
 \setlength{\itemsep}{4mm}
 \item Un estimateur d'un paramètre $\theta$ de la loi $P_{X}$ d'une
variable aléatoire $X$ dont on dispose d'un échantillon $(X_{n})$ est
une suite de variables aléatoires $(T_{n})$ où pour tout $n$, $T_{n}$
est une fonction des variables $X_{1},\ \dots\, X_{n})$. \\
 On définit alors son risque quadratique comme l'espérance des écarts à
$\theta$ mis au carré : \\
$R = E ( [X - \theta]^{2})$. \\
\\
 \item On pose, pour tout $n \in \N^*$, $M_{n} = \dfrac{1}{n} \Sum{k =
1}{n} Z_{k}$. \\
 On a $Z_{k} (\Omega) = \llb 1 ; N \rrb$ et pour tout $i \in \llb 1 ; N
\rrb$, $\Prob\left(\Ev{\Ev{Z_{k} = i}}\right) = \frac{1}{N}$. \\
 D'où $\E(Z_{k}) = \frac{1}{N} \Sum{i = 1}{N} i = \frac{N + 1}{2}$. \\
 On a donc par linéarité de l'espérance, $\E(M_{n}) = \frac{N + 1}{2}$
et en posant $T_{n} = 2 M_{n} -1$, on obtient un estimateur $(T_{n})$
sans biais de $N$, car $\E(T_{n}) = 2 \E(M_{n}) - 1 = N + 1 - 1 = N$.
\\
 De plus on a $R(T_{n}) = b^{2} + \V(T_{n}) = \V(T_{n}) = 4 \V(M_{n}) =
\frac{4}{n^{2}} \times n \V(Z_{1}) = \frac{4}{n} \V(Z_{k})
\xrightarrow[n \rightarrow + \infty]{} 0$. \\
 \item On note $S_{n} = \max ( Z_{1}, Z_{2},\ \dots\, Z_{n})$.
 \begin{noliste}{a)}
 \setlength{\itemsep}{2mm} 
 \item Pour tout $x \in \R$, on a $F_{S_{n}} (x) = \Prob\left(\Ev{\Ev{
S_{n} \leq x}}\right) = P\left(\Ev{ \max (Z_{1},\ \dots\, Z_{n} ) \leq
x }\right)\left(\Ev{Z_{1},\ \dots\, Z_{n} }\right) \leq x ) = P
\left(\Ev{ \bigcap\limits_{k = 1}{n} \Ev{ Z_{k} \leq x}}\right) =
\prod\limits_{k = 1}{n} F_{Z_{k}} (x) = \left( F_{Z_{1}} (x)
\right)^{n}$. \\
 Or pour tout $k \in \llb 1 ; N \rrb$ on a $F_{Z_{1}} (k) =
\Prob\left(\Ev{\Ev{ Z_{1} \leq k}}\right) = \frac{k}{N}$ donc pour tout
$k \in \llb 1 ; N \rrb$, $F_{S_{n}} (k) = \left( \frac{k}{N}
\right)^{n}$. \\
 \item $ \Sum{k = 1}{N} \Prob\left(\Ev{\Ev{ Y \geq k}}\right) = \Sum{k
= 1}{N} \Sum{i = k}{N} \Prob\left(\Ev{\Ev{ Y = i}}\right) = \Sum{ i
\geq k, 1 \leq k \leq N, 1 \leq i \leq N} \Prob\left(\Ev{\Ev{ Y =
i}}\right) = \Sum{ k \leq i, 1 \leq k \leq N, 1 \leq i \leq N}
\Prob\left(\Ev{\Ev{ Y = i}}\right) = \Sum{i = 1}{N} \Sum{k = 1}{i}
\Prob\left(\Ev{\Ev{ Y = i}}\right) = \Sum{i = 1}{N} i
\Prob\left(\Ev{\Ev{ Y = i}}\right) = \E(Y) $. \\
 \item On a donc $\E(S_{n} ) = \Sum{k = 1}{N} \Prob\left(\Ev{\Ev{ S_{n}
\geq k}}\right) = \Sum{k = 1}{N} [ 1 - \Prob\left(\Ev{\Ev{ S_{n} <
k}}\right) ] = N - \Sum{k = 1}{N} \Prob\left(\Ev{\Ev{ S_{n} \leq k
-1}}\right) = N - \Sum{k = 1}{N} \left( \frac{k-1}{N} \right)^{n}$. \\
\\
 $\E(S_{n}) \geq N - \frac{N}{n + 1}$ ? ?? ? J'obtiens la majoration
$\E(S_{n}) \geq N - N \left( 1 - \frac{1}{N} \right)^{n}$ qui donne le
résultat pour la question suivante, mais je ne vois absolument pas
comment obtenir la majoration demandée. \\
 \item $S_{n}$ est n estimateur de $N$ par définition, et on a $N \geq
\E(S_{n}) \geq N - \frac{N}{n + 1} \xrightarrow[n \rightarrow +
\infty]{} N$ donc par théorème d'encadrement, $\E(S_{n}) \rightarrow
N$. \\
 \end{noliste}
 \end{noliste}
 \noindent \textbf{\underline{Exercice sans préparation}} \\
\\
 Soit $A$ la matrice de $\mathcal{M}_{3} (\R)$ telle que :
 
\[
 A = \begin{smatrix}
0 & 1 & -1 \\
-1 & 2 & -1 \\
1 & -1 & 2 \\
\end{smatrix}. 
 
\]
 
 \begin{noliste}{1.}
 \setlength{\itemsep}{4mm}
 \item \begin{noliste}{a)}
 \setlength{\itemsep}{2mm}
 \item On calcule $A^{2}$ et on obtient $A^{2} = 3 A - 2I$. \\
 \item On en déduit que $A \left[ \ \frac{1}{2} ( 3I - A) \right] = I$
donc $A$ est inversible et $A^{-1} = \frac{1}{2} ( 3I - A)$. \\
 \end{noliste}
 \item Le polynôme $P\left(\Ev{x}\right) = x^{2} - 3 x + 2$ est
annulateur de $A$ et a pour racines $1$ et $2$ donc $\spc A \subset \{
1 ; 2\}$.
 \item On a $A - I = \begin{smatrix}
-1 & 1 & -1 \\
-1 & 1 & -1 \\
1 & -1 & 1 \\
\end{smatrix}
$ est de rang 1 (toutes les colonnes sont colinéaires) donc $1$ est
valeur propre et $\dim E_{1} (A) = \dim \ker (A-I) = 2$ par théorème du
rang. \\
 On a $A - 2I = \begin{smatrix}
-2 & 1 & -1 \\
-1 & 0 & -1 \\
1 & -1 & 0 \\
\end{smatrix}
$ n'est pas inversible car $C_{1} + C_{2} - C_{3} = 0$ donc $2$ est
valeur propre et $A$ est de rang 2 car les deux premières colonnes ne
sont pas colinéaires, donc $\dim E_{2} (A) = \dim \ker (A-2I) = 1$ par
théorème du rang. \\
\\
 Enfin la somme des dimensions des sous-espaces propres vaut 3 donc $A$
est diagonalisable.
 \end{noliste}
 \end{exercice}

 \newpage


 \begin{exercice}{\it (Exercice avec préparation)}~\\
 Dans cet exercice, on note $C^{0}$ l'espace vectoriel des fonctions
continues de $\R$ dans $\R$.
 \begin{noliste}{1.}
 \setlength{\itemsep}{4mm}
 \item Soit $u$ un endomorphisme de $E$, un réel $\lambda \in \R$ est
valeur propre de $u$ s'il existe $x \in E$ non nul tel que $u(x) =
\lambda x$. Tout vecteur non nul vérifiant $u(x) = \lambda x$ est
appelé vecteur propre de $u$ associé à la valeur propre $\lambda$. \\
\\
 Soit $\Phi$ l'application définie sur $C^{0}$ qui, à toute fonction
$f$ de $C^{0}$, associe la fonction $g = \Phi(f)$ définie par :
 
\[
 \forall x \in \R,\ \ g(x) = \dint{0}{x} f(t)\ dt.
 
\]
 \item $f$ est continue donc admet des primitives, donc en notant $F$
une primitive de $f$ on a $\Phi (f) (x) = F(x) - F(0)$ est dérivable,
de dérivée $f$. \\
 \item Pour tout $f$, $\Phi(f)$ est dérivable sur $\R$ donc continue
sur $\R$, et $\Phi (f) \in C^{0}$. \\
 La linéarité est évidente par linéarité de l'intégrale. \\
 \item La fonction valeur absolue est une fonction continue sur $\R$ et
non dérivable sur $\R$, car elle n'est pas dérivable en 0. \\
 $\Phi$ n'est donc pas surjective puisque la fonction valeur absolue,
qui est dans $C^{0}$, ne peut être atteinte par $\Phi$. \\
 Pour l'injectivité, on résout $\Phi (f) = 0$. \\
 Supposons $\Phi (f) = 0$, alors $f = \Phi'(f) = 0$ donc $\ker \Phi =
\{ 0 \}$ et $\Phi$ est injective. \\
\\
 Soit $\lambda$ un réel quelconque. On dit que $\lambda$ est une valeur
propre de $\Phi$ s'il existe une fonction $f$ non nulle de $C^{0}$,
telle que $\Phi(f) = \lambda f$. Une telle fonction est appelée
fonction propre associée à la valeur propre $\lambda$. \\
 \item Recherche des valeurs propres non nulles de $\Phi$. \\
 On suppose, dans cette question, que $\Phi$ admet une valeur propre
$\lambda$ non nulle. \\
\\
 Soit $f$ une fonction propre associée à $\lambda$. 
 \begin{noliste}{a)}
 \setlength{\itemsep}{2mm}
 \item $\Phi (f) = \lambda f$ est dérivable sur $\R$ donc $f =
\frac{1}{\lambda} \Phi (f)$ est dérivable sur $\R$. \\
 \item $h'(x) = e^{ - \frac{x}{\lambda} } \left( f'(x) -
\frac{1}{\lambda} f(x) \right)$. \\
 Or $\Phi(f) '(x) = f(x)$ donc $(\lambda f)'(x) = f(x)$ et enfin $f'(x)
= \frac{1}{\lambda} f(x)$, donc $h'(x) = 0$, et $h(x)$ est égale à une
constante $K$, et enfin $f(x) = K e^{ \frac{x}{\lambda} }$. \\
 Or on a $\Phi (f) (0) = \lambda f(0) = \dint{0}{0} f(t)\ dt = 0$ donc
$f(0) = 0$, $K = 0$ et enfin $f(x) = 0$ pour tout $x \in \R$. \\
 \item La seule valeur propre possible est donc 0. \\
 Or on a vu que $\ker \Phi = \{ 0 \}$ donc $0$ n'est pas valeur propre,
et $\Phi$ n'admet donc aucune valeur propre. \\
 \end{noliste}
 \item Pour toute fonction $f$ de $C^{0}$, on pose : $F_{0} = \Phi(f)$
et $\forall n \in \N^*,\ F_{n} = \Phi ( F_{n-1} )$. \begin{noliste}{a)}
 \setlength{\itemsep}{2mm}
 \item Par récurrence évidente on obtient que $F_{n}$ est de classe
$C^{n}$, puis on écrit $F_{n} ( 0) = \Phi (F_{n-1} ) (0) = \dint{0}{0}
F_{n-1} (t)\ dt = 0$. Seule $F_{0} (0) = f(0)$ peut être différent de
0. \\
 Enfin pour tout $k \in \llb 0 ; n \rrb$, $F_{n}{(k)} = F_{n-k}$ donc
$F_{n}{(k)} (0) = 0$ pour $0 \leq k \leq n-1$ et $F_{n}{(n)} (0) =
f(0)$. \\
 \item $F_{n}$ est de classe $C^{n}$, on utilise la formule de Taylor
avec reste intégral à l'ordre $n-1$, qui donne immédiatement le
résultat puisque les $n-1$ premières dérivées en 0 de $F_{n}$ sont
nulles. \\
 \end{noliste}
 \end{noliste}
 \noindent \textbf{\underline{Exercice sans préparation}} \\
\\
 $X$ et $Y$ sont deux variables aléatoires réelles indépendantes
définies sur le même espace probabilisé $(\Omega, \mathcal{A}, P)$ et
ayant la même loi de densité $\varphi$, définie par :
 
\[
 \forall x \in \R,\ \ \ \varphi(x) = k e^{ - | x |}.
 
\]
 \begin{noliste}{1.}
 \setlength{\itemsep}{4mm}
 \item $\phi$ est continue et positive sur $\R$, il faut que
$\dint{-\infty}{+ \infty} \phi(t)\ dt = 1$. \\
 Or $\phi$ est paire donc il suffit de prouver que $\dint{0}{+ \infty}
\phi (t)\ dt = \frac{1}{2}$. \\
 On a $\dint{0}{+ \infty} \phi (t)\ dt = k \dint{0}{+ \infty} e^{-t}\
dt = k$ (loi exponentielle de paramètre $\lambda = 1$), donc il faut
que $k = \frac{1}{2}$. \\
 \item Pour tout $x \leq 0$, on a $F (x) = \dint{-\infty}{x} \phi(t)\
dt = \dint{-\infty}{0} f(t)\ dt + \dint{0}{x} \phi(t)\ dt = \frac{1}{2}
+ \frac{1}{2} \dint{0}{x} e^{t}\ dt = \frac{1}{2} + \frac{1}{2} ( e^{x}
- 1) = \frac{1}{2} e^{x}$. \\
 Pour tout $x > 0$, $F(x) = \dint{-\infty}{x} \phi(t)\ dt =
\dint{-\infty}{0} f(t)\ dt + \dint{0}{x} \phi(t)\ dt = \frac{1}{2} +
\frac{1}{2} \dint{0}{x} e^{-t}\ dt = \frac{1}{2} + \frac{1}{2} (
-e^{-x} + 1) = 1 - \frac{1}{2} e^{-x}$. \\
 \item Par parité de $f$ on obtient que le moment d'ordre 2 existe si
et seulement si $\dint{0}{+ \infty} \phi (t)\ dt$ existe, ce qui est le
cas (on reconnaît le moment d'ordre deux de la loi exponentielle de
paramètre $\lambda = 1$). \\
 On en déduit que $e(X)$ et $\V(X)$ existent. \\
 De plus la fonction $t \rightarrow t \phi(t)\ dt$ est impaire donc
$\E(X) = \dint{-\infty}{+ \infty} t \phi (t)\ dt = 0$. \\
 Enfin la fonction $t \rightarrow t^{2} \phi(t)\ dt$ est paire donc
$\E(X^{2}) = 2 \dint{0}{+ \infty} t^{2} \phi (t)\ dt = \dint{0}{+
\infty} t^{2} e^{-t}\ dt = \E(Y^{2}) = \V(Y) + [ (\E(Y)]^{2}$, où $Y
\suit \mathcal{E} (1)$, donc : \\
 $\V(X) = \E(X^{2}) - [ \ \E(X)]^{2} = \V(Y) + [ \ \E(Y)]^{2} =
\frac{1}{1^{2}} + \frac{1}{1^{2}} = 2$. 
 \end{noliste}
 \end{exercice}

 \newpage

 \begin{exercice}{\it (Exercice avec préparation)}~\\
 Pour tout nombre réel $a$, on note $A(a)$ la matrice
 
\[
 A(a) = \begin{smatrix}
2 & 1 & a \\
1 & 1 + a & 1 \\
a & 1 & 2 \\
\end{smatrix}
 
\]
 \begin{noliste}{1.}
 \setlength{\itemsep}{4mm}
 \item \begin{noliste}{a)}
 \setlength{\itemsep}{2mm}
 \item Une matrice est diagonalisable si et seulement si elle est
semblable à une matrice diagonale. \\
 \item Soit $M$ une matrice diagonalisable, et $D$ diagonale et $P$
inversible telles que $M = P D P^{-1}$. \\
 On a alors ${}{t} M = {}{t}(P D P^{-1} ) = {}{t} (P^{-1}) {}{t} D
{}{t} P = ({}{t} P)^{-1} D {}{t} P$ donc ${}{t}M$ est diagonalisable
car elle est semblable à la matrice diagonale $D$. \\
 \end{noliste} 
 \item 
 \begin{noliste}{a)}
 \setlength{\itemsep}{2mm}
 \item $A(a)$ est symétrique donc diagonalisable. \\
 \item $A(a) - a I = \begin{smatrix}
2 & 1 & a \\
1 & 1 + a & 1 \\
a & 1 & 2 \\
\end{smatrix}
$ n'est pas inversible car $C_{1} + C_{3} - 2C_{2} = 0$ donc $a$ est
valeur propre de $A(a)$. \\
 D'autre part la matrice obtenue est de rang 1 si et seulement si les
trois colonnes sont colinéaires ce qui donne (la 2e ligne vaut 1 pour
les trois) que les trois colonnes sont égales et donc $a = 2- a = 1$.
\\
 D'où pour $a = 1$ la matrice est de rang 1 et le sous-espace propre
est de dimension 2 par théorème du rang, égale à $\begin{smatrix}
1 & 1 & 1 \\
1 & 1 & 1 \\
1 & 1 & 1 \\
\end{smatrix}
$ donc vérifie que $f(e_{2} - e_{1}) = f(e_{3} - e_{1}) = 0$ et $(e_{2}
- e_{1}, e_{3} - e_{1})$ est libre car échelonnée donc c'est une base
de $E_{a} (A(a) )$. \\
 Pour $a \neq 1$ la matrice est de rang 2 donc $E_{a} (A(a) )$ est de
dimension 1, et la relation $f(e_{1}) + f(e_{3}) - 2 f(e_{2}) = 0$
donne $f(e_{1} - 2 e_{2} + e_{3}) = 0$ donc $(e_{1} - 2e_{2} + e_{3})$
famille libre (car d'un vecteur non nul) de $E_{a} (A(a) )$ donc c'est
une base de $E_{a} (A(a) )$.
 \\
 \item $A(a) \begin{smatrix}
1 \\
1 \\
1 \\
\end{smatrix}
 = (3 + a) \begin{smatrix}
1 \\
1 \\
1 \\
\end{smatrix}
$ et $A(a) \begin{smatrix}
1 \\
0 \\
-1 \\
\end{smatrix}
 = (2-a) \begin{smatrix}
1 \\
0 \\
-1 \\
\end{smatrix}
$. \\
 \item Dans tous les cas, la famille $[ ( 1, -2, 1), (1, 1, 1), (1,,0,
-1)]$ est une base de vecteurs propres de $A$ donc on peut diagonaliser
dans cette base : \\
 On pose $P = \begin{smatrix}
a & 0 & 0 \\
0 & 3 + a & 0 \\
0 & 0 & 2-a \\
\end{smatrix}
$ et $D = \begin{smatrix}
2 & 1 & a \\
1 & 1 + a & 1 \\
a & 1 & 2 \\
\end{smatrix}
$ et on a $A(a) = P D P^{-1}$. \\
 \end{noliste}
 \item Soit $(x_{n})_{n \in \N}$, $(y_{n})_{n \in \N}$, $(z_{n})_{n \in
\N}$ trois suites réelles vérifiant, pour tout $n$ entier naturel,
 
\[
 \left\{ 
\begin{array}{l}
 x_{n + 1} = 2 x_{n} + y_{n} \\
y_{n + 1} = x_{n} + y_{n} + z_{n} \\
z_{n + 1} = y_{n} + 2 z_{n} \\
\end{array}
\right. 
 
\]
 \begin{noliste}{a)}
 \setlength{\itemsep}{2mm}
 \item $X_{n + 1} = A(0) X_{n}$. \\
 \item Les valeurs propres de $A(0)$ sont 0, 2 et 3; or on aura $X_{n}
= A(0)^{n} X_{0} = P D^{n} P^{-1} X_{0}$ avec $D^{n}$ comportant les
valeurs 0, $2^{n}$ et $3^{n}$ sur la diagonale; il faut donc que
celles-ci ne rentrent pas en compte. \\
 Chaque suite s'écrit comme combinaison linéaire de $0^{n}$, $2^{n}$ et
$3^{n}$, il faut donc qu'elles ne soient combinaisons linéaires que de
$0^{n}$, donc que $x_{n} = y_{n} = z_{n} = 0$ pour $n \geq 1$,
c'est-à-dire que $X_{1} = A (0) X_{0} = 0$ donc $X_{0} \in \ker A(0)$,
donc $X_{0} \in \Vect[ (1, -2, 1)]$. \\
 La condition cherchée est donc $x_{0} + z_{0} - 2 y_{0} = 0$. \\
 \end{noliste}
 \item
 \begin{noliste}{a)}
 \setlength{\itemsep}{2mm} 
 \item $B = P B' P^{-1}$ et $C^{2} = B$, donc $B' = P^{-1} B P = P^{-1}
C^{2} P = ( P^{-1} C P)^{2}$ donc avec $C' = P^{-1} C P$ on a bien
$C'^{2} = B'$. \\
 \item $ BC = C^{2} C = C^{3} = C C^{2} = C B$. \\
 \item $\begin{smatrix}
3 & 0 & 0 \\
0 & 6 & 0 \\
0 & 0 & -1 \\
\end{smatrix}
\begin{smatrix}
a & b & c \\
d & e & f \\
g & h & i \\
\end{smatrix}
 = \begin{smatrix}
a & b & c \\
d & e & f \\
g & h & i \\
\end{smatrix}
\begin{smatrix}
3 & 0 & 0 \\
0 & 6 & 0 \\
0 & 0 & -1 \\
\end{smatrix}
\Leftrightarrow \begin{smatrix}
3a & 3b & 3c \\
6d & 6e & 6f \\
- g & -h & -i \\
\end{smatrix}
 = \begin{smatrix}
3a & 6b & -c \\
3d & 6e & -f \\
3g & 6h & - i \\
\end{smatrix}
\Leftrightarrow b = c = d = f = g = h = 0 \Leftrightarrow
\begin{smatrix}
a & b & c \\
d & e & f \\
g & h & i \\
\end{smatrix}
 = \begin{smatrix}
a & 0 & 0 \\
0 & e & 0 \\
0 & 0 & i \\
\end{smatrix}
$ est diagonale. \\
 \item D'après les questions précédentes cela revient à cherchez une
matrice $N$ vérifiant $N^{2} = \begin{smatrix}
3 & 0 & 0 \\
0 & 6 & 0 \\
0 & 0 & -1 \\
\end{smatrix}
 = D$ qui est semblable à $A(3)$, et $N$ commute alors avec $D$, donc
d'aptrèsd la question précédente il faut la chercher diagonale. \\
 En posant $N = \operatorname{diag} ( a, b, c)$ on obtient $N^{2} =
\operatorname{diag}(a^{2}, b^{2}, c^{2}) = \operatorname{diag}( 3, 6,
-1)$ ce qui est impossible car un carré est toujours positif. Il n'y a
donc pas de solution à l'équation matricielle $M^{2} = A(3)$. \\
 \end{noliste} 
 \end{noliste}
 \noindent \textbf{\underline{Exercice sans préparation}} \\
 \begin{noliste}{1.}
 \setlength{\itemsep}{4mm}
 \item A priori on en demande pas de vérifier que c'est une fonction de
répartition de variable à densité. \\
 Il faut prouver que $F$ est croissante (évident avec sa dérivée), de
limites 0 en $-\infty$ (évident) et 1 et $ + \infty$ (évident encore)
et continue à droite (évident puisque $F$ est continue sur $\R$). Cela
ne coûte pas grand-chose de préciser que $F$ est de classe $c^{1}$ sur
$\R$ et donc continue sur $\R$ donc que c'est une fonction de
répartition de variable à densité. \\
 \item Pour cette question classique on obtient $G(x) = F(x)^{2} =
\frac{1}{ (1 + e^{-x} )^{2}}$ et $G_{n} (x) = F(x)^{n} = \frac{1}{(1 +
e^{-x})^{n}}$. 
 \end{noliste}
 \end{exercice}

 \newpage

 \begin{exercice}{\it (Exercice avec préparation)}~
 \begin{noliste}{1.}
 \setlength{\itemsep}{4mm}
 \item $f(b) = \Sum{k = 0}{n} \frac{f^{(k)} (a)}{k!} (b-a)^{k} +
\dint{a}{b} \frac{(b-t)^{n}}{n!} f^{n + 1} (t)\ dt$. \\
\\
 Soit $f$ la fonction définie sur $\R$ par $f(x) = e^{-x^{2}}$, et $F$
la primitive de $f$ qui vérifie $F(0) = 0$. \\
 \item $F$ est dérivable sur $\R$, et $F'(x) = f(x) = e^{ - x^{2}} > 0$
donc $F$ est strictement croissante sur $\R$, de plus par parité de
$f$, $F(x) = \dint{0}{x} f(t)\ dt$ est impaire, passe par 0 en 0 et sa
limite en $ + \infty$ vaut $\dint{0}{+ \infty} e^{-x^{2}}\ dt =
\frac{\sqrt{\pi}}{\sqrt{2} }$ (avec la loi normale). \\
 \item \begin{noliste}{a)}
 \setlength{\itemsep}{2mm}
 \item L'intégrale n'est pas généralisée et la fonction intégrée est
continue, l'intégrale existe bien. \\
 On définit alors la fonction $G$ par :
 
\[
 G(x) = \dint{0}{1} e^{-(xt)^{2}}\ dt. 
 
\]
 \item Le changement de variable $u = xt$ donne $G(x) = \dint{0}{x}
\frac{1}{x} e^{ -u^{2}}\ du = \frac{1}{x} \dint{0}{x} e^{-u^{2}}\ du =
\frac{F(x)}{x}$. \\
 On en déduit que $G$ est dérivable sur $\R^*$ et que $G'(x) = \frac{x
F'(x) - F(x) }{x^{2}} = \frac{x e^{-x^{2}} - F(x)}{x^{2}}$. \\
 Le signe de $G'$ est celui de $x e^{-x^{2}} - F(x) = x e^{-x^{2}} -
\dint{0}{x} e^{-t^{2}}\ dt$, qui est impaire comme somme de deux
fonctions impaires. Etudions son signe sur $\R_+^*$ : \\
 Pour tout $t \in [0 ; x]$, $ 0 \leq t \leq x$ donc $0 \leq t^{2} \leq
x^{2}$ et $- x^{2} \leq -t^{2} \leq 0$, on compose par exp qui est
croissante pour obtenir $e^{-x^{2}} \leq e^{-t^{2}}$ et enfin
$\dint{0}{x} e^{-x^{2}}\ dt = x e^{-x^{2}} \leq \dint{0}{x} e^{-t^{2}}\
dt = F(x)$, donc $G'(x) \leq 0$. \\
 On en déduit par imparité que $G'(x) \geq 0$ sur $\R_-*$, puis que $G$
est croissante sur $\R_-^*$ et décroissante sur $\R_+^*$. \\
 \item $\dlim{x \rightarrow 0, x \neq 0} G(x) = \dlim{x \rightarrow 0,
x \neq 0} \frac{F(x)}{x} = \dlim{x \rightarrow 0, x \neq 0} \frac{ F(x)
- F(0)}{x-0}$, or $F$ est dérivable en 0 donc $\dlim{x \rightarrow 0, x
\neq 0} G(x)$ existe et vaut $\dlim{x \rightarrow 0, x \neq 0} G(x) =
F'(0) = f(0) = 1$. \\
 D'autre part $G(0) = \dint{0}{1} e^{0}\ dt = 1$ donc $G$ est continue
en 0. \\
 D'autre part $\dlim{x \rightarrow + \infty} F(x) = \sqrt{ \frac{ \pi
}{2} }$ est finie donc par quotient de limites, $\dlim{x \rightarrow +
\infty} G(x) = \dlim{x \rightarrow + \infty} \frac{F(x)}{x} = 0$. \\
 \item On étudie $\dlim{x \rightarrow 0, x \neq 0} G'(x)$ pour conclure
avec le théorème de prolongement de la dérivée. \\
 On a $G'(x) = \frac{ \dint{0}{x} ( e^{-x^{2}} - e^{-t^{2}} )\ dt
}{x^{2}}$, on étudie donc la fonction $f(x) - f(t)$. \\
 La formule de Taylor avec reste intégral à l'ordre 2 donne $f(t) -f(x)
= f'(x) (t-x) + f''(x) \frac{(t-x)^{2}}{2} + \dint{x}{t}
\frac{(u-x)^{2}}{2} f''(u)\ du$ donc $f(x) - f(t) = f'(x) (x-t) -
f''(x) \frac{(x-t)^{2}}{2} + g(x,t)$, avec $| g(x) | \leq \dint{t}{x}
\frac{x^{2}}{2} \times M\ du \leq \frac{x^{3} M}{2}$, où $M$ est un
majorant de $| f''(u) |$ sur $[0 ; x]$ ou $[x ; 0]$. \\
\\
 On obtient alors $\dint{0}{x} ( e^{-x^{2}} - e^{-t^{2}} )\ dt = f'(x)
\dint{0}{x} (x-t)\ dt - \frac{f''(x)}{2} \dint{0}{x} ( x-t)^{2}\ dt +
\dint{0}{x} g(x,t)\ dt = f'(x) \frac{x^{2}}{2} - \frac{f''(x)}{2}
\frac{x^{3}}{3} + h(x)$, avec $| h(x) | \leq | \dint{0}{x} M
\frac{x^{3}}{2} | = M \frac{ | x^{4} |}{2}$. \\
 Enfin on obtient $G'(x) = \frac{f'(x)}{2} + o (1)$, donc $ \dlim{x
\rightarrow 0, x \neq 0} G'(x) = \frac{f'(0)}{2} = 0$. \\
\\
 La fonction $G'$ admet donc une limite finie en 0 à gauche et à droite
et celles-ci sont égales; le théorème de prolongement de la dérivée
permet de conclure, et donne $G'(0) = 0$. \\
\\\
\
 Autre possibilité : écrire puis sommer les développements limités de
$x f(x)$ et $F(x)$ obtenus à l'aide de la formule de Taylor.
 \end{noliste}
 \item
 \begin{noliste}{a)}
 \setlength{\itemsep}{2mm} 
 \item Pour tout $x \neq 0$, on a $x G'(x) + G(x) = \frac{x f(x) -
F(x)}{x} + \frac{F(x)}{x} = \frac{x f(x)}{x} = f(x)$. \\
 De plus comme les fonction $x \rightarrow x G'(x) + G(x)$ et $x
\rightarrow f(x)$ sont continues, on obtient par continuité en 0 la
relation en $x = 0$. \\
 \item On veut prouver que $G$ est l'unique fonction $g$ dérivable sur
$\R$ telle que : 
 
\[
 \forall x \in \R,\ x g'(x) + g(x) = f(x) \ \ \ (E). 
 
\]
 Soit $G_{1}$ une fonction réelle dérivable sur $\R$ et vérifiant
l'équation $(E)$. On pose $H = G - G_{1}$. Déterminer $H(x)$ pour $x >
0$ puis pour $x < 0$. conclure en utilisant la continuité de $H$ en 0.
\\
\\
 On a $H(x) = G (x) - G_{1} (x)$ vérifie $x H'(x) + H(x) = 0$, donc en
posant $A(x) = x H(x)$ on a $ A'(x) = x H'(x) + H(x) = 0$, donc $A(x)$
est constante égale à $K$ sur $\R$. \\
\\
 On en déduit que $H(x) = \frac{K}{x}$ sur $\R_+^*$ et sur $\R_-^*$, et
la continuité de $H$ en 0 impose que $\frac{K}{x}$ admette une limite
finie en 0. Or si $K \neq 0$, on obtient une limite infinie : ceci
impose que $K = 0$, puis $H(x) = 0$ sur $\R^*$ puis sur $\R$ par
continuité de $H$. \\
\\
 Enfin on obtient $G_{1} (x) = G(x)$, et $G$ est bien l'unique solution
de l'équation différentielle $(E)$. \\
 \end{noliste} 
 \end{noliste}
 \noindent \textbf{\underline{Exercice sans préparation}} \\
\\
 Les variables aléatoires considérées dans cet exercice sont définies
sur un espace probabilisé $(\Omega, \mathcal{A}, P)$. Soit $a$ un réel
strictement positif et $X$ une variable aléatoire de loi uniforme sur
$[0 ; 2a]$. \begin{noliste}{1.}
 \setlength{\itemsep}{4mm}
 \item Soit $n \in \N^*$. On considère $n$ variables aléatoires
indépendantes $X_{1},\ \dots\, X_{n}$ qui ont toutes la même loi que
$X$. On pose : 
 
\[
 M_{n} = \max ( X_{1},\ \dots\, X_{n}). 
 
\]
 De manière classique $F_{M_{n}} (x) = F_{X}(x)^{n} = \left\{
\begin{array}{cl}
 0 \text{ si } x \leq 0 \\
\frac{x^{n}}{(2a)^{n}} \text{ si } 0 \leq x \leq 2a \\
1 \text{ si } x \geq 1
\end{array}
\right.$. \\
 On en déduit que $f_{M_{n}} (x) = 0$ si $x \notin [ 0 ; 2a]$, et
$f_{M_{n}} (x) = n \frac{x^{n-1}}{(2a)^{n}}$ sinon. \\
\\
 Ensuite on a $\E(M_{n}) = \frac{n}{(2a)^{n}} \dint{0}{2a} x^{n}\ dx =
\frac{ n (2a)^{n + 1} }{ (n + 1) (2a)^{n}} = \frac{2a n}{n + 1}$. \\
 De même $\E(M_{n}{2}) = \frac{ n (2a)^{n + 2} }{(n + 2) (2a)^{n}} =
\frac{4 a^{2} n}{n + 2}$, et enfin : \\
 $\V(M_{n}) = \frac{4 a^{2} n}{n + 2} - \frac{4a^{2} n^{2}}{(n +
1)^{2}}$ donc $\V(M_{n}) = 4 a^{2} n \left( \frac{1}{n + 2} -
\frac{n}{(n + 1)^{2}} \right)$. \\
 \item On a $\E(U_{n}) = \frac{n + 1}{2n} \E(M_{n}) = a = \E(X)$ donc
$U_{n}$ est un estimateur sans biais de $a = \E(X)$. \\
 Pour comparer les estimateurs, on compare leurs risques quadratiques :
\\
 $R (V_{n}) = V (V_{n}) = \frac{1}{n^{2}} \times n \V(X) =
\frac{\V(X)}{n} = \frac{ 4 a^{2}}{12 n} \rightarrow 0$ et proportionnel
à $\frac{1}{n}$. \\
 D'autre part $R(U_{n}) = V ( U_{n}) = \frac{(n + 1)^{2}}{4 n^{2}}
\V(M_{n}) = \frac{a^{2}}{n} \left( \frac{(n + 1)^{2}}{n + 2} - n
\right) = a^{2} \left( \frac{(n + 1)^{2} - n (n + 2)}{n (n + 2)}
\right) = a^{2} \left( \frac{1}{n (n + 2)} \right) \rightarrow 0$ mais
proportionnel à $\frac{1}{n^{2}}$, donc il tend plus vite vers 0. \\
 $U_{n}$ est donc un meilleur estimateur que $V_{n}$ de $a = \E(X)$.
 \end{noliste}
 \end{exercice}

 \newpage

 

\end{document}