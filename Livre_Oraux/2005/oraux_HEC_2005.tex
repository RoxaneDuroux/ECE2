\documentclass[11pt]{article}%
\usepackage{geometry}%
\geometry{a4paper,
  lmargin=2cm,rmargin=2cm,tmargin=2.5cm,bmargin=2.5cm}

\input{../../../macros.tex}

% \renewcommand{\thesection}{\Roman{section}.\hspace{-.3cm}}
% \renewcommand{\thesubsection}{\Alph{subsection}.\hspace{-.2cm}}

\pagestyle{fancy} %
\lhead{ECE2 \hfill Mathématiques \\} %
\chead{\hrule} %
\rhead{} %
\lfoot{} %
\cfoot{} %
\rfoot{\thepage} %

\renewcommand{\headrulewidth}{0pt}% : Trace un trait de séparation
                                    % de largeur 0,4 point. Mettre 0pt
                                    % pour supprimer le trait.

\renewcommand{\footrulewidth}{0.4pt}% : Trace un trait de séparation
                                    % de largeur 0,4 point. Mettre 0pt
                                    % pour supprimer le trait.

\setlength{\headheight}{14pt}

\title{\bf \vspace{-1.6cm} HEC 2005} %
\author{} %
\date{} %
\begin{document}
\maketitle %
\vspace{-1.2cm}\hrule %
\thispagestyle{fancy}

\vspace*{.2cm}

%%DEBUT

\begin{center} 
  Petits exercices à utiliser en cas de besoin. \\ La forme de l'oral
  a changé depuis, ils ne sont pas tellement caractéristiques dans
  leur forme \\ mais intéressants sur le fond 
\end{center}

\section*{\textbf{Questions avec préparation.}}

\begin{noliste}{1.}
\item À tout triplet de nombres réels $\left( a,b,c\right) ,$ on
  associe la matrice
  \[
  M\left( a,b,c\right) =\left( 
    \begin{array}{rrr}
      1 & 0 & 0 \\ 
      a & 1 & 0 \\ 
      b & c & 1%
    \end{array}%
  \right) 
  \]

  \begin{noliste}{a)}
  \item Une telle matrice $M\left( a,b,c\right) $ est-elle diagonalisable ?

  \item Calculer $\left( M\left( a,b,c\right) -I\right) ^{n}$ pour tout $n$
    entier naturel non nul

  \item Déterminer $M^{n}$ en fonction de $I\;,\,M$ et $M^{2}$ pour $n\in 
    \N$ puis pour $n\in \mathbb{Z}$
  \end{noliste}

\item 
  \begin{noliste}{a)}
  \item La matrice $A=\left( 
      \begin{array}{rrr}
        13 & -9 & 45 \\ 
        -3 & 3 & -11 \\ 
        -3 & 2 & -10%
      \end{array}%
    \right) $,est -elle diagonalisable ?

  \item Déterminer $P$ inversible telle que $P^{-1}AP=\left( 
      \begin{array}{rrr}
        a & 0 & 0 \\ 
        0 & b & c \\ 
        0 & 0 & b%
      \end{array}%
    \right) =T$
  \end{noliste}

\item Soit $t$ un nombre réel et $A\left( t\right) $ la matrice : 
  \[
  A\left( t\right) =\left( 
    \begin{array}{ccc}
      1-t & -t & 0 \\ 
      -t & 1-t & 0 \\ 
      -t & t & 1-2t%
    \end{array}%
  \right) 
  \]

  On note $\mathcal{M}$ l'ensemble de ces matrices quand $t$ décrit
  $\R$.

  \begin{noliste}{a)}
  \item Montrer que $\mathcal{M}$ est stable par produit matricielle.

  \item Déterminer les valeurs de $t$ pour lesquelles $A\left( t\right) $
    est inversible .

    Montrer que $A\left( t\right) ^{-1}$ appartient encore à $\mathcal{M}$

  \item résoudre l'équation $X^{2}=A\left( \frac{-3}{2}\right) $
    d'inconnue $X$ appartenant à $\mathcal{M}$

  \item Soit $C=A\left( -1\right) .$ déterminer $C^{n}$ pour tout entier
    naturel $n$que)
  \end{noliste}

\item Étudier la fonction

  \[
  f:x\rightarrow \dint{x}{2x}\frac{1}{\sqrt{t^{4}+t^{2}+1}}dt 
  \]

  Ensemble de définition, continuité, dérivée, graphe.

\item Pour tout $n\in \N^{\ast }$ on pose : 
  \[
  I_{n}=\dint{0}{1}\frac{1}{\left( 1+x\right) \left( 1+\frac{x}{2}\right)
    \cdots \left( 1+\frac{x}{n}\right) }dx 
  \]

  Étudier la suite $\left( I_{n}\right) _{n\in \N^{\ast }}$, montrer
  que
  \[
  I_{n}\leq \dint{0}{1}\frac{dx}{1+x\ln \left( n\right) }dx 
  \]

  et déterminer la limite de la suite $\left( I_{n}\right) _{n\in
    \mathbb{N}}$.

\item Soit $a$ un réel strictement positif. On se propose de
  déterminer les fonctions $f$ trois fois dérivables sur un intervalle
  $\left[ 0,2a\right] $ à valeurs réelles et telles que
  \[
  \forall x\in \left[ 0,2a\right] ,\frac{f\left( x\right) }{2}=f\left( \frac{x%
    }{2}\right) +f\left( a-\frac{x}{2}\right) 
  \]

  Montrer qu'il existe $c\in \left[ 0,a\right] $ tel que $f^{\prime \prime
  }\left( c\right) =\max_{t\in \left[ 0,2a\right] }\left( f^{\prime \prime
    }\left( t\right) \right) $ et prouver que $f^{\prime \prime }\left( c\right)
  =f^{\prime \prime }\left( \frac{c}{2}\right) $

  Déterminer alors les solutions $f$.

\item On se donne $n$ variables aléatoires mutuellement indépendantes
  $\left( U_{i}\right) _{1\leq i\leq n}$ de même loi de Bernoulli de
  paramètre $p\in \left] 0,1.\right[ $

  \begin{noliste}{a)}
  \item Déterminer l'espérance et la variance de la variable aléatoire
    $Y = \Sum{i=1}{n}U_{i}$.

  \item On suppose que $n\geq 4.$ calculer , pour chaque entier $k\in
    \mathbb{N}$ la probabilité de l'événement $\left[ Y=k\right] $
    conditionné par l'événement $\left[ U_{2}=0\right] \cap \left[
      U_{4}=1 \right]$.

  \item Calculer , pour chaque entier $k\in \N,$ la probabilité de
    l'événement $\left[ Y=k\right] $ conditionné à l'évé%
    nement $\left[ Y>0\right] $
  \end{noliste}

\item Soit $X$ une variable aléatoire définie sur un espace
  probabilité $\left( \Omega ,A,\Prob\right) $ et à valeurs dans
  $\N$. Pour tout $\omega $ de $\Omega $, on pose $Y\left( \omega
  \right) =\frac{X\left( \omega \right) }{2}$ si $X\left( \omega
  \right) $ est pair et $Y\left( \omega \right) =\frac{1-X\left(
      \omega \right) }{2}$ sinon.

  \begin{noliste}{a)}
  \item Déterminer $\left[ Y=0\right] $ et, pour chaque $k\in \mathbb{Z},$ 
    $\left[ Y=k\right] .$

  \item Soit $p\in \left] 0,1\right[ .$ On suppose que la loi de $X$
    est donnée par :
    \[
    \forall k\in \N,\ \Prob\left( X=k\right) =p\left( 1-p\right)
    ^{k} 
    \]%
    Déterminer alors la loi de $Y$ ainsi que son espérance
    mathématique.
  \end{noliste}

\item \textbf{Dénombrement. }

  Une urne contient des boules numérotées de $1$ à $n$. On effectue
  des tirages avec remise tant que les numéros obtenus forment une
  suite strictement décroissante.

  \begin{noliste}{a)}
  \item Déterminer la loi de la variable aléatoire $X$ représentant le
    nombre de tirages effectués.

  \item Déterminer son espérance mathématique et la limite de cette
    espérance quand $n\rightarrow +\infty$.

    On pourra utiliser sans démonstration la formule suivante :

    si $X$ admet une espérance alors $\E\left( X\right) =
    \Sum{k=0}{+\infty } \Prob\left( X>k\right)$.
  \end{noliste}
\end{noliste}

\section*{\textbf{Question sans préparation}}

\begin{noliste}{1.}
\item Extrema de 
  \[
  f\left( x,y\right) =x^{2}+y^{2}-\left( x-y\right) ^{2} 
  \]

\item Montrer que la famille $\left( p_{i,j}\right) _{\left(
      i,j\right) \in \mathbb{Z}^{2}}$ définie par
  \begin{eqnarray*}
    p_{1,1} & = & p_{-1,1}=\frac{1}{32} \\
    p_{-1,-1} & = & p_{1,-1}=p_{1,0}=p_{0,1}=\frac{3}{32} \\
    p_{-1},_{0} & = & p_{0,-1}=\frac{5}{32} \\
    p_{0,0} & = & \frac{8}{32}
  \end{eqnarray*}%
  et $p_{i,j}=0$ sinon, peut être considérée comme la loi d'un couple
  $\left( X,Y\right)$.

  Étudier l'indépendance de $X$ et de $Y;$ de $X^{2}$ et de $Y^{2}. =
  \dots $

\item On considère la fonction définie par 
  \[
  f\left( x\right) =\left( x-1\right) ^{1/\left( x-2\right) } 
  \]

  Quel est l'ensemble de définition de $f$? Quelle valeur attribuer à
  $f\left( 2\right) $ pour prolonger $f$ en une fonction continue en
  $x=2?$ Tracer l'allure du graphe de $f$ au voisinage de $x=2$.

\item On considère trois variables aléatoires mutuellement
  indépendantes $U$, $V$, et $W$ suivant des lois de Poisson de
  paramêtre respectifs $\alpha ,$ $\beta $ et $\gamma$.

  On pose $X=U+V$ et $Y=V+W$. Calculer $\Cov(X, Y)$.

\item Soit $X$ une variable aléatoire suivant une loi uniforme sur
  $\left[ 0,1\right]$.\\
  Déterminer les fonctions de répartitions et les espérances
  mathématiques des variables aléatoires
  \[
  Y=\inf \left( X,1-X\right) ,\quad Z=\sup \left( X,1-X\right) \text{
    et }R = \frac{Y}{Z}\text{ }
  \]

\item Une matrice $N\in \mathcal{M}_{n}\left( \R\right) $ est dite
  nilpotente s'il existe ne puissance $p\in \N^{\ast }$ telle que
  $N^{p}=0$.

  À quelle condition une matrice nilpotente est-elle diagonalisable ?

  % Si elle est diagonalisable alors $N=PDP^{-1}$ avec $N^{p}=PD^{p}P^{-1}=0$
  % pour un $p$ précédent.

  % Alors $D^{n}=0$ et $D$ étant diagonale, tous ses termes diagonaux sont
  % nuls.

  % Donc $N=0$.
\end{noliste}

%%FIN

\end{document}
