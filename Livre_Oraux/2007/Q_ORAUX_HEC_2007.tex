\documentclass[11pt]{article}%
\usepackage{geometry}%
\geometry{a4paper,
 lmargin = 2cm,rmargin = 2cm,tmargin = 2.5cm,bmargin = 2.5cm}

\usepackage{array}
\usepackage{paralist}

\usepackage[svgnames, usenames, dvipsnames]{xcolor}
\xdefinecolor{RecColor}{named}{Aqua}
\xdefinecolor{IncColor}{named}{Aqua}
\xdefinecolor{ImpColor}{named}{PaleGreen}

% \usepackage{frcursive}

\usepackage{adjustbox}

%%%%%%%%%%%
\newcommand{\cRB}[1]{{\color{Red} \pmb{#1}}} %
\newcommand{\cR}[1]{{\color{Red} {#1}}} %
\newcommand{\cBB}[1]{{\color{Blue} \pmb{#1}}}
\newcommand{\cB}[1]{{\color{Blue} {#1}}}
\newcommand{\cGB}[1]{{\color{LimeGreen} \pmb{#1}}}
\newcommand{\cG}[1]{{\color{LimeGreen} {#1}}}

%%%%%%%%%%

\usepackage{diagbox} %
\usepackage{colortbl} %
\usepackage{multirow} %
\usepackage{pgf} %
\usepackage{environ} %
\usepackage{fancybox} %
\usepackage{textcomp} %
\usepackage{marvosym} %

%%%%%%%%%% pour pouvoir faire des dashedline dans les tableaux
\usepackage{arydshln}

%%%%%%%%%% pour qu'une cellcolor ne recouvre pas le trait du tableau
\usepackage{hhline}%

\usepackage{pgfplots}
\pgfplotsset{compat=1.10}
\usepgfplotslibrary{patchplots}
\usepgfplotslibrary{fillbetween}
\usepackage{tikz,tkz-tab}
\usepackage{ifthen}
\usepackage{calc}
\usetikzlibrary{calc,decorations.pathreplacing,arrows,positioning} 
\usetikzlibrary{fit,shapes,backgrounds}
% \usepackage[nomessages]{fp}% http://ctan.org/pkg/fp

\usetikzlibrary{matrix,arrows,decorations.pathmorphing,
  decorations.pathreplacing} 

\newcommand{\myunit}{1 cm}
\tikzset{
    node style sp/.style={draw,circle,minimum size=\myunit},
    node style ge/.style={circle,minimum size=\myunit},
    arrow style mul/.style={draw,sloped,midway,fill=white},
    arrow style plus/.style={midway,sloped,fill=white},
}

%%%%%%%%%%%%%%
%%%%% écrire des inférieur égal ou supérieur égal avec typographie
%%%%% francaise
%%%%%%%%%%%%%

\renewcommand{\geq}{\geqslant}
\renewcommand{\leq}{\leqslant}
\renewcommand{\emptyset}{\varnothing}

\newcommand{\Leq}{\leqslant}
\newcommand{\Geq}{\geqslant}

%%%%%%%%%%%%%%
%%%%% Macro Celia
%%%%%%%%%%%%%

\newcommand{\ff}[2]{\left[#1, #2\right]} %
\newcommand{\fo}[2]{\left[#1, #2\right[} %
\newcommand{\of}[2]{\left]#1, #2\right]} %
\newcommand{\soo}[2]{\left]#1, #2\right[} %
\newcommand{\abs}[1]{\left|#1\right|} %
\newcommand{\Ent}[1]{\left\lfloor #1 \right\rfloor} %


%%%%%%%%%%%%%%
%%%%% tikz : comment dessiner un "oeil"
%%%%%%%%%%%%%

\newcommand{\eye}[4]% size, x, y, rotation
{ \draw[rotate around={#4:(#2,#3)}] (#2,#3) -- ++(-.5*55:#1) (#2,#3)
  -- ++(.5*55:#1); \draw (#2,#3) ++(#4+55:.75*#1) arc
  (#4+55:#4-55:.75*#1);
  % IRIS
  \draw[fill=gray] (#2,#3) ++(#4+55/3:.75*#1) arc
  (#4+180-55:#4+180+55:.28*#1);
  % PUPIL, a filled arc
  \draw[fill=black] (#2,#3) ++(#4+55/3:.75*#1) arc
  (#4+55/3:#4-55/3:.75*#1);%
}


%%%%%%%%%%
%% discontinuité fonction
\newcommand\pointg[2]{%
  \draw[color = red, very thick] (#1+0.15, #2-.04)--(#1, #2-.04)--(#1,
  #2+.04)--(#1+0.15, #2+.04);%
}%

\newcommand\pointd[2]{%
  \draw[color = red, very thick] (#1-0.15, #2+.04)--(#1, #2+.04)--(#1,
  #2-.04)--(#1-0.15, #2-.04);%
}%

%%%%%%%%%%
%%% 1 : position abscisse, 2 : position ordonnée, 3 : taille, 4 : couleur
%%%%%%%%%%
% \newcommand\pointG[4]{%
%   \draw[color = #4, very thick] (#1+#3, #2-(#3/3.75))--(#1,
%   #2-(#3/3.75))--(#1, #2+(#3/3.75))--(#1+#3, #2+(#3/3.75)) %
% }%

\newcommand\pointG[4]{%
  \draw[color = #4, very thick] ({#1+#3/3.75}, {#2-#3})--(#1,
  {#2-#3})--(#1, {#2+#3})--({#1+#3/3.75}, {#2+#3}) %
}%

\newcommand\pointD[4]{%
  \draw[color = #4, very thick] ({#1-#3/3.75}, {#2+#3})--(#1,
  {#2+#3})--(#1, {#2-#3})--({#1-#3/3.75}, {#2-#3}) %
}%

\newcommand\spointG[4]{%
  \draw[color = #4, very thick] ({#1+#3/1.75}, {#2-#3})--(#1,
  {#2-#3})--(#1, {#2+#3})--({#1+#3/1.75}, {#2+#3}) %
}%

\newcommand\spointD[4]{%
  \draw[color = #4, very thick] ({#1-#3/2}, {#2+#3})--(#1,
  {#2+#3})--(#1, {#2-#3})--({#1-#3/2}, {#2-#3}) %
}%

%%%%%%%%%%

\newcommand{\Pb}{\mathtt{P}}

%%%%%%%%%%%%%%%
%%% Pour citer un précédent item
%%%%%%%%%%%%%%%
\newcommand{\itbf}[1]{{\small \bf \textit{#1}}}


%%%%%%%%%%%%%%%
%%% Quelques couleurs
%%%%%%%%%%%%%%%

\xdefinecolor{cancelcolor}{named}{Red}
\xdefinecolor{intI}{named}{ProcessBlue}
\xdefinecolor{intJ}{named}{ForestGreen}

%%%%%%%%%%%%%%%
%%%%%%%%%%%%%%%
% barrer du texte
\usetikzlibrary{shapes.misc}

\makeatletter
% \definecolor{cancelcolor}{rgb}{0.127,0.372,0.987}
\newcommand{\tikz@bcancel}[1]{%
  \begin{tikzpicture}[baseline=(textbox.base),inner sep=0pt]
  \node[strike out,draw] (textbox) {#1}[thick, color=cancelcolor];
  \useasboundingbox (textbox);
  \end{tikzpicture}%
}
\newcommand{\bcancel}[1]{%
  \relax\ifmmode
    \mathchoice{\tikz@bcancel{$\displaystyle#1$}}
               {\tikz@bcancel{$\textstyle#1$}}
               {\tikz@bcancel{$\scriptstyle#1$}}
               {\tikz@bcancel{$\scriptscriptstyle#1$}}
  \else
    \tikz@bcancel{\strut#1}%
  \fi
}
\newcommand{\tikz@xcancel}[1]{%
  \begin{tikzpicture}[baseline=(textbox.base),inner sep=0pt]
  \node[cross out,draw] (textbox) {#1}[thick, color=cancelcolor];
  \useasboundingbox (textbox);
  \end{tikzpicture}%
}
\newcommand{\xcancel}[1]{%
  \relax\ifmmode
    \mathchoice{\tikz@xcancel{$\displaystyle#1$}}
               {\tikz@xcancel{$\textstyle#1$}}
               {\tikz@xcancel{$\scriptstyle#1$}}
               {\tikz@xcancel{$\scriptscriptstyle#1$}}
  \else
    \tikz@xcancel{\strut#1}%
  \fi
}
\makeatother

\newcommand{\xcancelRA}{\xcancel{\rule[-.15cm]{0cm}{.5cm} \Rightarrow
    \rule[-.15cm]{0cm}{.5cm}}}

%%%%%%%%%%%%%%%%%%%%%%%%%%%%%%%%%%%
%%%%%%%%%%%%%%%%%%%%%%%%%%%%%%%%%%%

\newcommand{\vide}{\multicolumn{1}{c}{}}

%%%%%%%%%%%%%%%%%%%%%%%%%%%%%%%%%%%
%%%%%%%%%%%%%%%%%%%%%%%%%%%%%%%%%%%


\usepackage{multicol}
% \usepackage[latin1]{inputenc}
% \usepackage[T1]{fontenc}
\usepackage[utf8]{inputenc}
\usepackage[T1]{fontenc}
\usepackage[normalem]{ulem}
\usepackage[french]{babel}

\usepackage{url}    
\usepackage{hyperref}
\hypersetup{
  backref=true,
  pagebackref=true,
  hyperindex=true,
  colorlinks=true,
  breaklinks=true,
  urlcolor=blue,
  linkcolor=red,
  bookmarks=true,
  bookmarksopen=true
}

%%%%%%%%%%%%%%%%%%%%%%%%%%%%%%%%%%%%%%%%%%%
%% Pour faire des traits diagonaux dans les tableaux
%% Nécessite slashbox.sty
%\usepackage{slashbox}

\usepackage{tipa}
\usepackage{verbatim,listings}
\usepackage{graphicx}
\usepackage{fancyhdr}
\usepackage{mathrsfs}
\usepackage{pifont}
\usepackage{tablists}
\usepackage{dsfont,amsfonts,amssymb,amsmath,amsthm,stmaryrd,upgreek,manfnt}
\usepackage{enumerate}

%\newcolumntype{M}[1]{p{#1}}
\newcolumntype{C}[1]{>{\centering}m{#1}}
\newcolumntype{R}[1]{>{\raggedright}m{#1}}
\newcolumntype{L}[1]{>{\raggedleft}m{#1}}
\newcolumntype{P}[1]{>{\raggedright}p{#1}}
\newcolumntype{B}[1]{>{\raggedright}b{#1}}
\newcolumntype{Q}[1]{>{\raggedright}t{#1}}

\newcommand{\alias}[2]{
\providecommand{#1}{}
\renewcommand{#1}{#2}
}
\alias{\R}{\mathbb{R}}
\alias{\N}{\mathbb{N}}
\alias{\Z}{\mathbb{Z}}
\alias{\Q}{\mathbb{Q}}
\alias{\C}{\mathbb{C}}
\alias{\K}{\mathbb{K}}

%%%%%%%%%%%%
%% rendre +infty et -infty plus petits
%%%%%%%%%%%%
\newcommand{\sinfty}{{\scriptstyle \infty}}

%%%%%%%%%%%%%%%%%%%%%%%%%%%%%%%
%%%%% macros TP Scilab %%%%%%%%
\newcommand{\Scilab}{\textbf{Scilab}} %
\newcommand{\Scinotes}{\textbf{SciNotes}} %
\newcommand{\faire}{\noindent $\blacktriangleright$ } %
\newcommand{\fitem}{\scalebox{.8}{$\blacktriangleright$}} %
\newcommand{\entree}{{\small\texttt{ENTRÉE}}} %
\newcommand{\tab}{{\small\texttt{TAB}}} %
\newcommand{\mt}[1]{\mathtt{#1}} %
% guillemets droits

\newcommand{\ttq}{\textquotesingle} %

\newcommand{\reponse}[1]{\longboxed{
    \begin{array}{C{0.9\textwidth}}
      \nl[#1]
    \end{array}
  }} %

\newcommand{\reponseL}[2]{\longboxed{
    \begin{array}{C{#2\textwidth}}
      \nl[#1]
    \end{array}
  }} %

\newcommand{\reponseR}[1]{\longboxed{
    \begin{array}{R{0.9\textwidth}}
      #1
    \end{array}
  }} %

\newcommand{\reponseRL}[2]{\longboxed{
    \begin{array}{R{#2\textwidth}}
      #1
    \end{array}
  }} %

\newcommand{\reponseC}[1]{\longboxed{
    \begin{array}{C{0.9\textwidth}}
      #1
    \end{array}
  }} %

\colorlet{pyfunction}{Blue}
\colorlet{pyCle}{Magenta}
\colorlet{pycomment}{LimeGreen}
\colorlet{pydoc}{Cyan}
% \colorlet{SansCo}{white}
% \colorlet{AvecCo}{black}

\newcommand{\visible}[1]{{\color{ASCo}\colorlet{pydoc}{pyDo}\colorlet{pycomment}{pyCo}\colorlet{pyfunction}{pyF}\colorlet{pyCle}{pyC}\colorlet{function}{sciFun}\colorlet{var}{sciVar}\colorlet{if}{sciIf}\colorlet{comment}{sciComment}#1}} %

%%%% à changer ????
\newcommand{\invisible}[1]{{\color{ASCo}\colorlet{pydoc}{pyDo}\colorlet{pycomment}{pyCo}\colorlet{pyfunction}{pyF}\colorlet{pyCle}{pyC}\colorlet{function}{sciFun}\colorlet{var}{sciVar}\colorlet{if}{sciIf}\colorlet{comment}{sciComment}#1}} %

\newcommand{\invisibleCol}[2]{{\color{#1}#2}} %

\NewEnviron{solution} %
{ %
  \Boxed{
    \begin{array}{>{\color{ASCo}} R{0.9\textwidth}}
      \colorlet{pycomment}{pyCo}
      \colorlet{pydoc}{pyDo}
      \colorlet{pyfunction}{pyF}
      \colorlet{pyCle}{pyC}
      \colorlet{function}{sciFun}
      \colorlet{var}{sciVar}
      \colorlet{if}{sciIf}
      \colorlet{comment}{sciComment}
      \BODY
    \end{array}
  } %
} %

\NewEnviron{solutionC} %
{ %
  \Boxed{
    \begin{array}{>{\color{ASCo}} C{0.9\textwidth}}
      \colorlet{pycomment}{pyCo}
      \colorlet{pydoc}{pyDo}
      \colorlet{pyfunction}{pyF}
      \colorlet{pyCle}{pyC}
      \colorlet{function}{sciFun}
      \colorlet{var}{sciVar}
      \colorlet{if}{sciIf}
      \colorlet{comment}{sciComment}
      \BODY
    \end{array}
  } %
} %

\newcommand{\invite}{--\!\!>} %

%%%%% nouvel environnement tabular pour retour console %%%%
\colorlet{ConsoleColor}{Blue!15}
\colorlet{function}{Red}
\colorlet{var}{Maroon}
\colorlet{if}{Magenta}
\colorlet{comment}{LimeGreen}

\newcommand{\tcVar}[1]{\textcolor{var}{\bf \small #1}} %
\newcommand{\tcFun}[1]{\textcolor{function}{#1}} %
\newcommand{\tcIf}[1]{\textcolor{if}{#1}} %
\newcommand{\tcFor}[1]{\textcolor{if}{#1}} %

\newcommand{\moins}{\!\!\!\!\!\!- }
\newcommand{\espn}{\!\!\!\!\!\!}

\usepackage{booktabs,varwidth} \newsavebox\TBox
\newenvironment{console}
{\begin{lrbox}{\TBox}\varwidth{\linewidth}
    \tabular{>{\tt\small}R{0.84\textwidth}}
    \nl[-.4cm]} {\endtabular\endvarwidth\end{lrbox}%
  \fboxsep=1pt\colorbox{ConsoleColor}{\usebox\TBox}}

\newcommand{\lInv}[1]{%
  $\invite$ #1} %

\newcommand{\lAns}[1]{%
  \qquad ans \ = \nl %
  \qquad \qquad #1} %

\newcommand{\lVar}[2]{%
  \qquad #1 \ = \nl %
  \qquad \qquad #2} %

\newcommand{\lDisp}[1]{%
  #1 %
} %

\newcommand{\ligne}[1]{\underline{\small \tt #1}} %

\newcommand{\ligneAns}[2]{%
  $\invite$ #1 \nl %
  \qquad ans \ = \nl %
  \qquad \qquad #2} %

\newcommand{\ligneVar}[3]{%
  $\invite$ #1 \nl %
  \qquad #2 \ = \nl %
  \qquad \qquad #3} %

\newcommand{\ligneErr}[3]{%
  $\invite$ #1 \nl %
  \quad !-{-}error #2 \nl %
  #3} %
%%%%%%%%%%%%%%%%%%%%%% 

\newcommand{\bs}[1]{\boldsymbol{#1}} %
\newcommand{\nll}{\nl[.4cm]} %
\newcommand{\nle}{\nl[.2cm]} %
%% opérateur puissance copiant l'affichage Scilab
%\newcommand{\puis}{\!\!\!~^{\scriptscriptstyle\pmb{\wedge}}}
\newcommand{\puis}{\mbox{$\hspace{-.1cm}~^{\scriptscriptstyle\pmb{\wedge}}
    \hspace{0.05cm}$}} %
\newcommand{\pointpuis}{.\mbox{$\hspace{-.15cm}~^{\scriptscriptstyle\pmb{\wedge}}$}} %
\newcommand{\Sfois}{\mbox{$\mt{\star}$}} %

%%%%% nouvel environnement tabular pour les encadrés Scilab %%%%
\newenvironment{encadre}
{\begin{lrbox}{\TBox}\varwidth{\linewidth}
    \tabular{>{\tt\small}C{0.1\textwidth}>{\small}R{0.7\textwidth}}}
  {\endtabular\endvarwidth\end{lrbox}%
  \fboxsep=1pt\longboxed{\usebox\TBox}}

\newenvironment{encadreL}
{\begin{lrbox}{\TBox}\varwidth{\linewidth}
    \tabular{>{\tt\small}C{0.25\textwidth}>{\small}R{0.6\textwidth}}}
  {\endtabular\endvarwidth\end{lrbox}%
  \fboxsep=1pt\longboxed{\usebox\TBox}}

\newenvironment{encadreF}
{\begin{lrbox}{\TBox}\varwidth{\linewidth}
    \tabular{>{\tt\small}C{0.2\textwidth}>{\small}R{0.70\textwidth}}}
  {\endtabular\endvarwidth\end{lrbox}%
  \fboxsep=1pt\longboxed{\usebox\TBox}}

\newenvironment{encadreLL}[2]
{\begin{lrbox}{\TBox}\varwidth{\linewidth}
    \tabular{>{\tt\small}C{#1\textwidth}>{\small}R{#2\textwidth}}}
  {\endtabular\endvarwidth\end{lrbox}%
  \fboxsep=1pt\longboxed{\usebox\TBox}}

%%%%% nouvel environnement tabular pour les script et fonctions %%%%
\newcommand{\commentaireDL}[1]{\multicolumn{1}{l}{\it
    \textcolor{comment}{$\slash\slash$ #1}}}

\newcommand{\commentaire}[1]{{\textcolor{comment}{$\slash\slash$ #1}}}

\newcounter{cptcol}

\newcommand{\nocount}{\multicolumn{1}{c}{}}

\newcommand{\sciNo}[1]{{\small \underbar #1}}

\NewEnviron{scilab}{ %
  \setcounter{cptcol}{0}
  \begin{center}
    \longboxed{
      \begin{tabular}{>{\stepcounter{cptcol}{\tiny \underbar
              \thecptcol}}c>{\tt}l}
        \BODY
      \end{tabular}
    }
  \end{center}
}

\NewEnviron{scilabNC}{ %
  \begin{center}
    \longboxed{
      \begin{tabular}{>{\tt}l} %
          \BODY
      \end{tabular}
    }
  \end{center}
}

\NewEnviron{scilabC}[1]{ %
  \setcounter{cptcol}{#1}
  \begin{center}
    \longboxed{
      \begin{tabular}{>{\stepcounter{cptcol}{\tiny \underbar
              \thecptcol}}c>{\tt}l}
        \BODY
      \end{tabular}
    }
  \end{center}
}

\newcommand{\scisol}[1]{ %
  \setcounter{cptcol}{0}
  \longboxed{
    \begin{tabular}{>{\stepcounter{cptcol}{\tiny \underbar
            \thecptcol}}c>{\tt}l}
      #1
    \end{tabular}
  }
}

\newcommand{\scisolNC}[1]{ %
  \longboxed{
    \begin{tabular}{>{\tt}l}
      #1
    \end{tabular}
  }
}

\newcommand{\scisolC}[2]{ %
  \setcounter{cptcol}{#1}
  \longboxed{
    \begin{tabular}{>{\stepcounter{cptcol}{\tiny \underbar
            \thecptcol}}c>{\tt}l}
      #2
    \end{tabular}
  }
}

\NewEnviron{syntaxe}{ %
  % \fcolorbox{black}{Yellow!20}{\setlength{\fboxsep}{3mm}
  \shadowbox{
    \setlength{\fboxsep}{3mm}
    \begin{tabular}{>{\tt}l}
      \BODY
    \end{tabular}
  }
}

%%%%% fin macros TP Scilab %%%%%%%%
%%%%%%%%%%%%%%%%%%%%%%%%%%%%%%%%%%%

%%%%%%%%%%%%%%%%%%%%%%%%%%%%%%%%%%%
%%%%% TP Python - listings %%%%%%%%
%%%%%%%%%%%%%%%%%%%%%%%%%%%%%%%%%%%
\newcommand{\Python}{\textbf{Python}} %

\lstset{% general command to set parameter(s)
basicstyle=\ttfamily\small, % print whole listing small
keywordstyle=\color{blue}\bfseries\underbar,
%% underlined bold black keywords
frame=lines,
xleftmargin=10mm,
numbers=left,
numberstyle=\tiny\underbar,
numbersep=10pt,
%identifierstyle=, % nothing happens
commentstyle=\color{green}, % white comments
%%stringstyle=\ttfamily, % typewriter type for strings
showstringspaces=false}

\newcommand{\pysolCpt}[2]{
  \setcounter{cptcol}{#1}
  \longboxed{
    \begin{tabular}{>{\stepcounter{cptcol}{\tiny \underbar
            \thecptcol}}c>{\tt}l}
        #2
      \end{tabular}
    }
} %

\newcommand{\pysol}[1]{
  \setcounter{cptcol}{0}
  \longboxed{
    \begin{tabular}{>{\stepcounter{cptcol}{\tiny \underbar
            \thecptcol}}c>{\tt}l}
        #1
      \end{tabular}
    }
} %

% \usepackage[labelsep=endash]{caption}

% avec un caption
\NewEnviron{pythonCap}[1]{ %
  \renewcommand{\tablename}{Programme}
  \setcounter{cptcol}{0}
  \begin{center}
    \longboxed{
      \begin{tabular}{>{\stepcounter{cptcol}{\tiny \underbar
              \thecptcol}}c>{\tt}l}
        \BODY
      \end{tabular}
    }
    \captionof{table}{#1}
  \end{center}
}

\NewEnviron{python}{ %
  \setcounter{cptcol}{0}
  \begin{center}
    \longboxed{
      \begin{tabular}{>{\stepcounter{cptcol}{\tiny \underbar
              \thecptcol}}c>{\tt}l}
        \BODY
      \end{tabular}
    }
  \end{center}
}

\newcommand{\pyVar}[1]{\textcolor{var}{\bf \small #1}} %
\newcommand{\pyFun}[1]{\textcolor{pyfunction}{#1}} %
\newcommand{\pyCle}[1]{\textcolor{pyCle}{#1}} %
\newcommand{\pyImp}[1]{{\bf #1}} %

%%%%% commentaire python %%%%
\newcommand{\pyComDL}[1]{\multicolumn{1}{l}{\textcolor{pycomment}{\#
      #1}}}

\newcommand{\pyCom}[1]{{\textcolor{pycomment}{\# #1}}}
\newcommand{\pyDoc}[1]{{\textcolor{pydoc}{#1}}}

\newcommand{\pyNo}[1]{{\small \underbar #1}}

%%%%%%%%%%%%%%%%%%%%%%%%%%%%%%%%%%%
%%%%%% Système linéaire paramétré : écrire les opérations au-dessus
%%%%%% d'un symbole équivalent
%%%%%%%%%%%%%%%%%%%%%%%%%%%%%%%%%%%

\usepackage{systeme}

\NewEnviron{arrayEq}{ %
  \stackrel{\scalebox{.6}{$
      \begin{array}{l} 
        \BODY \\[.1cm]
      \end{array}$}
  }{\Longleftrightarrow}
}

\NewEnviron{arrayEg}{ %
  \stackrel{\scalebox{.6}{$
      \begin{array}{l} 
        \BODY \\[.1cm]
      \end{array}$}
  }{=}
}

\NewEnviron{operationEq}{ %
  \scalebox{.6}{$
    \begin{array}{l} 
      \scalebox{1.6}{$\mbox{Opérations :}$} \\[.2cm]
      \BODY \\[.1cm]
    \end{array}$}
}

% \NewEnviron{arraySys}[1]{ %
%   \sysdelim\{.\systeme[#1]{ %
%     \BODY %
%   } %
% }

%%%%%

%%%%%%%%%%
%%%%%%%%%% ESSAI
\newlength\fboxseph
\newlength\fboxsepva
\newlength\fboxsepvb

\setlength\fboxsepva{0.2cm}
\setlength\fboxsepvb{0.2cm}
\setlength\fboxseph{0.2cm}

\makeatletter

\def\longboxed#1{\leavevmode\setbox\@tempboxa\hbox{\color@begingroup%
\kern\fboxseph{\m@th$\displaystyle #1 $}\kern\fboxseph%
\color@endgroup }\my@frameb@x\relax}

\def\my@frameb@x#1{%
  \@tempdima\fboxrule \advance\@tempdima \fboxsepva \advance\@tempdima
  \dp\@tempboxa\hbox {%
    \lower \@tempdima \hbox {%
      \vbox {\hrule\@height\fboxrule \hbox{\vrule\@width\fboxrule #1
          \vbox{%
            \vskip\fboxsepva \box\@tempboxa \vskip\fboxsepvb}#1
          \vrule\@width\fboxrule }%
        \hrule \@height \fboxrule }}}}

\newcommand{\boxedhv}[3]{\setlength\fboxseph{#1cm}
  \setlength\fboxsepva{#2cm}\setlength\fboxsepvb{#2cm}\longboxed{#3}}

\newcommand{\boxedhvv}[4]{\setlength\fboxseph{#1cm}
  \setlength\fboxsepva{#2cm}\setlength\fboxsepvb{#3cm}\longboxed{#4}}

\newcommand{\Boxed}[1]{{\setlength\fboxseph{0.2cm}
  \setlength\fboxsepva{0.2cm}\setlength\fboxsepvb{0.2cm}\longboxed{#1}}}

\newcommand{\mBoxed}[1]{{\setlength\fboxseph{0.2cm}
  \setlength\fboxsepva{0.2cm}\setlength\fboxsepvb{0.2cm}\longboxed{\mbox{#1}}}}

\newcommand{\mboxed}[1]{{\setlength\fboxseph{0.2cm}
  \setlength\fboxsepva{0.2cm}\setlength\fboxsepvb{0.2cm}\boxed{\mbox{#1}}}}

\newsavebox{\fmbox}
\newenvironment{fmpage}[1]
     {\begin{lrbox}{\fmbox}\begin{minipage}{#1}}
     {\end{minipage}\end{lrbox}\fbox{\usebox{\fmbox}}}

%%%%%%%%%%
%%%%%%%%%%

\DeclareMathOperator{\ch}{ch}
\DeclareMathOperator{\sh}{sh}

%%%%%%%%%%
%%%%%%%%%%

\newcommand{\norme}[1]{\Vert #1 \Vert}

%\newcommand*\widefbox[1]{\fbox{\hspace{2em}#1\hspace{2em}}}

\newcommand{\nl}{\tabularnewline}

\newcommand{\hand}{\noindent\ding{43}\ }
\newcommand{\ie}{\textit{i.e. }}
\newcommand{\cf}{\textit{cf }}

\newcommand{\Card}{\operatorname{Card}}

\newcommand{\aire}{\mathcal{A}}

\newcommand{\LL}[1]{\mathscr{L}(#1)} %
\newcommand{\B}{\mathscr{B}} %
\newcommand{\Bc}[1]{B_{#1}} %
\newcommand{\M}[1]{\mathscr{M}_{#1}(\mathbb{R})}

\DeclareMathOperator{\im}{Im}
\DeclareMathOperator{\kr}{Ker}
\DeclareMathOperator{\rg}{rg}
\DeclareMathOperator{\tr}{tr}
\DeclareMathOperator{\spc}{Sp}
\DeclareMathOperator{\sgn}{sgn}
\DeclareMathOperator{\supp}{Supp}

\newcommand{\Mat}{{\rm{Mat}}}
\newcommand{\Vect}[1]{{\rm{Vect}}\left(#1\right)}

\newenvironment{smatrix}{%
  \begin{adjustbox}{width=.9\width}
    $
    \begin{pmatrix}
    }{%      
    \end{pmatrix}
    $
  \end{adjustbox}
}

\newenvironment{sarray}[1]{%
  \begin{adjustbox}{width=.9\width}
    $
    \begin{array}{#1}
    }{%      
    \end{array}
    $
  \end{adjustbox}
}

\newcommand{\vd}[2]{
  \scalebox{.8}{
    $\left(\!
      \begin{array}{c}
        #1 \\
        #2
      \end{array}
    \!\right)$
    }}

\newcommand{\vt}[3]{
  \scalebox{.8}{
    $\left(\!
      \begin{array}{c}
        #1 \\
        #2 \\
        #3 
      \end{array}
    \!\right)$
    }}

\newcommand{\vq}[4]{
  \scalebox{.8}{
    $\left(\!
      \begin{array}{c}
        #1 \\
        #2 \\
        #3 \\
        #4 
      \end{array}
    \!\right)$
    }}

\newcommand{\vc}[5]{
  \scalebox{.8}{
    $\left(\!
      \begin{array}{c}
        #1 \\
        #2 \\
        #3 \\
        #4 \\
        #5 
      \end{array}
    \!\right)$
    }}

\newcommand{\ee}{\text{e}}

\newcommand{\dd}{\text{d}}

%%% Ensemble de définition
\newcommand{\Df}{\mathscr{D}}
\newcommand{\Cf}{\mathscr{C}}
\newcommand{\Ef}{\mathscr{C}}

\newcommand{\rond}[1]{\,\overset{\scriptscriptstyle \circ}{\!#1}}

\newcommand{\df}[2]{\dfrac{\partial #1}{\partial #2}} %
\newcommand{\dfn}[2]{\partial_{#2}(#1)} %
\newcommand{\ddfn}[2]{\partial^2_{#2}(#1)} %
\newcommand{\ddf}[2]{\dfrac{\partial^2 #1}{\partial #2^2}} %
\newcommand{\ddfr}[3]{\dfrac{\partial^2 #1}{\partial #2 \partial
    #3}} %


\newcommand{\dlim}[1]{{\displaystyle \lim_{#1} \ }}
\newcommand{\dlimPlus}[2]{
  \dlim{
    \scalebox{.6}{
      $
      \begin{array}{l}
        #1 \rightarrow #2\\
        #1 > #2
      \end{array}
      $}}}
\newcommand{\dlimMoins}[2]{
  \dlim{
    \scalebox{.6}{
      $
      \begin{array}{l}
        #1 \rightarrow #2\\
        #1 < #2
      \end{array}
      $}}}

%%%%%%%%%%%%%%
%% petit o, développement limité
%%%%%%%%%%%%%%

\newcommand{\oo}[2]{{\underset {{\overset {#1\rightarrow #2}{}}}{o}}} %
\newcommand{\oox}[1]{{\underset {{\overset {x\rightarrow #1}{}}}{o}}} %
\newcommand{\oon}{{\underset {{\overset {n\rightarrow +\infty}{}}}{o}}} %
\newcommand{\po}[1]{{\underset {{\overset {#1}{}}}{o}}} %
\newcommand{\neqx}[1]{{\ \underset {{\overset {x \to #1}{}}}{\not\sim}\ }} %
\newcommand{\eqx}[1]{{\ \underset {{\overset {x \to #1}{}}}{\sim}\ }} %
\newcommand{\eqn}{{\ \underset {{\overset {n \to +\infty}{}}}{\sim}\ }} %
\newcommand{\eq}[2]{{\ \underset {{\overset {#1 \to #2}{}}}{\sim}\ }} %
\newcommand{\DL}[1]{{\rm{DL}}_1 (#1)} %
\newcommand{\DLL}[1]{{\rm{DL}}_2 (#1)} %

\newcommand{\negl}{<<}

\newcommand{\neglP}[1]{\begin{array}{c}
    \vspace{-.2cm}\\
    << \\
    \vspace{-.7cm}\\
    {\scriptstyle #1}
  \end{array}}

%%%%%%%%%%%%%%
%% borne sup, inf, max, min
%%%%%%%%%%%%%%
\newcommand{\dsup}[1]{\displaystyle \sup_{#1} \ }
\newcommand{\dinf}[1]{\displaystyle \inf_{#1} \ }
\newcommand{\dmax}[1]{\max\limits_{#1} \ }
\newcommand{\dmin}[1]{\min\limits_{#1} \ }

\newcommand{\dcup}[2]{{\textstyle\bigcup\limits_{#1}^{#2}}\hspace{.1cm}}
%\displaystyle \bigcup_{#1}^{#2}}
\newcommand{\dcap}[2]{{\textstyle\bigcap\limits_{#1}^{#2}}\hspace{.1cm}}
% \displaystyle \bigcap_{#1}^{#2}
%%%%%%%%%%%%%%
%% opérateurs logiques
%%%%%%%%%%%%%%
\newcommand{\NON}[1]{\mathop{\small \tt{NON}} (#1)}
\newcommand{\ET}{\mathrel{\mathop{\small \mathtt{ET}}}}
\newcommand{\OU}{\mathrel{\mathop{\small \tt{OU}}}}
\newcommand{\XOR}{\mathrel{\mathop{\small \tt{XOR}}}}

\newcommand{\id}{{\rm{id}}}

\newcommand{\sbullet}{\scriptstyle \bullet}
\newcommand{\stimes}{\scriptstyle \times}

%%%%%%%%%%%%%%%%%%
%% Probabilités
%%%%%%%%%%%%%%%%%%
\newcommand{\Prob}{\mathbb{P}}
\newcommand{\Ev}[1]{\left[ {#1} \right]}
\newcommand{\E}{\mathbb{E}}
\newcommand{\V}{\mathbb{V}}
\newcommand{\Cov}{{\rm{Cov}}}
\newcommand{\U}[2]{\mathcal{U}(\llb #1, #2\rrb)}
\newcommand{\Uc}[2]{\mathcal{U}([#1, #2])}
\newcommand{\Ucof}[2]{\mathcal{U}(]#1, #2])}
\newcommand{\Ucoo}[2]{\mathcal{U}(]#1, #2[)}
\newcommand{\Ucfo}[2]{\mathcal{U}([#1, #2[)}
\newcommand{\Bern}[1]{\mathcal{B}\left(#1\right)}
\newcommand{\Bin}[2]{\mathcal{B}\left(#1, #2\right)}
\newcommand{\G}[1]{\mathcal{G}\left(#1\right)}
\newcommand{\Pois}[1]{\mathcal{P}\left(#1\right)}
\newcommand{\HG}[3]{\mathcal{H}\left(#1, #2, #3\right)}
\newcommand{\Exp}[1]{\mathcal{E}\left(#1\right)}
\newcommand{\Norm}[2]{\mathcal{N}\left(#1, #2\right)}

\DeclareMathOperator{\cov}{Cov}

\newcommand{\var}{v.a.r. }
\newcommand{\suit}{\hookrightarrow}

\newcommand{\flecheR}[1]{\rotatebox{90}{\scalebox{#1}{\color{red}
      $\curvearrowleft$}}}


\newcommand{\partie}[1]{\mathcal{P}(#1)}
\newcommand{\Cont}[1]{\mathcal{C}^{#1}}
\newcommand{\Contm}[1]{\mathcal{C}^{#1}_m}

\newcommand{\llb}{\llbracket}
\newcommand{\rrb}{\rrbracket}

%\newcommand{\im}[1]{{\rm{Im}}(#1)}
\newcommand{\imrec}[1]{#1^{- \mathds{1}}}

\newcommand{\unq}{\mathds{1}}

\newcommand{\Hyp}{\mathtt{H}}

\newcommand{\eme}[1]{#1^{\scriptsize \mbox{ème}}}
\newcommand{\er}[1]{#1^{\scriptsize \mbox{er}}}
\newcommand{\ere}[1]{#1^{\scriptsize \mbox{ère}}}
\newcommand{\nd}[1]{#1^{\scriptsize \mbox{nd}}}
\newcommand{\nde}[1]{#1^{\scriptsize \mbox{nde}}}

\newcommand{\truc}{\mathop{\top}}
\newcommand{\fois}{\mathop{\ast}}

\newcommand{\f}[1]{\overrightarrow{#1}}

\newcommand{\checked}{\textcolor{green}{\checkmark}}

\def\restriction#1#2{\mathchoice
              {\setbox1\hbox{${\displaystyle #1}_{\scriptstyle #2}$}
              \restrictionaux{#1}{#2}}
              {\setbox1\hbox{${\textstyle #1}_{\scriptstyle #2}$}
              \restrictionaux{#1}{#2}}
              {\setbox1\hbox{${\scriptstyle #1}_{\scriptscriptstyle #2}$}
              \restrictionaux{#1}{#2}}
              {\setbox1\hbox{${\scriptscriptstyle #1}_{\scriptscriptstyle #2}$}
              \restrictionaux{#1}{#2}}}
\def\restrictionaux#1#2{{#1\,\smash{\vrule height .8\ht1 depth .85\dp1}}_{\,#2}}

\makeatletter
\newcommand*{\da@rightarrow}{\mathchar"0\hexnumber@\symAMSa 4B }
\newcommand*{\da@leftarrow}{\mathchar"0\hexnumber@\symAMSa 4C }
\newcommand*{\xdashrightarrow}[2][]{%
  \mathrel{%
    \mathpalette{\da@xarrow{#1}{#2}{}\da@rightarrow{\,}{}}{}%
  }%
}
\newcommand{\xdashleftarrow}[2][]{%
  \mathrel{%
    \mathpalette{\da@xarrow{#1}{#2}\da@leftarrow{}{}{\,}}{}%
  }%
}
\newcommand*{\da@xarrow}[7]{%
  % #1: below
  % #2: above
  % #3: arrow left
  % #4: arrow right
  % #5: space left 
  % #6: space right
  % #7: math style 
  \sbox0{$\ifx#7\scriptstyle\scriptscriptstyle\else\scriptstyle\fi#5#1#6\m@th$}%
  \sbox2{$\ifx#7\scriptstyle\scriptscriptstyle\else\scriptstyle\fi#5#2#6\m@th$}%
  \sbox4{$#7\dabar@\m@th$}%
  \dimen@=\wd0 %
  \ifdim\wd2 >\dimen@
    \dimen@=\wd2 %   
  \fi
  \count@=2 %
  \def\da@bars{\dabar@\dabar@}%
  \@whiledim\count@\wd4<\dimen@\do{%
    \advance\count@\@ne
    \expandafter\def\expandafter\da@bars\expandafter{%
      \da@bars
      \dabar@ 
    }%
  }%  
  \mathrel{#3}%
  \mathrel{%   
    \mathop{\da@bars}\limits
    \ifx\\#1\\%
    \else
      _{\copy0}%
    \fi
    \ifx\\#2\\%
    \else
      ^{\copy2}%
    \fi
  }%   
  \mathrel{#4}%
}
\makeatother



\newcount\depth

\newcount\depth
\newcount\totaldepth

\makeatletter
\newcommand{\labelsymbol}{%
      \ifnum\depth=0
        %
      \else
        \rlap{\,$\bullet$}%
      \fi
}

\newcommand*\bernoulliTree[1]{%
    \depth=#1\relax            
    \totaldepth=#1\relax
    \draw node(root)[bernoulli/root] {\labelsymbol}[grow=right] \draw@bernoulli@tree;
    \draw \label@bernoulli@tree{root};                                   
}                                                                        

\def\draw@bernoulli@tree{%
    \ifnum\depth>0 
      child[parent anchor=east] foreach \type/\label in {left child/$E$,right child/$S$} {%
          node[bernoulli/\type] {\label\strut\labelsymbol} \draw@bernoulli@tree
      }
      coordinate[bernoulli/increment] (dummy)
   \fi%
}

\def\label@bernoulli@tree#1{%
    \ifnum\depth>0
      ($(#1)!0.5!(#1-1)$) node[fill=white,bernoulli/decrement] {\tiny$p$}
      \label@bernoulli@tree{#1-1}
      ($(#1)!0.5!(#1-2)$) node[fill=white] {\tiny$q$}
      \label@bernoulli@tree{#1-2}
      coordinate[bernoulli/increment] (dummy)
   \fi%
}

\makeatother

\tikzset{bernoulli/.cd,
         root/.style={},
         decrement/.code=\global\advance\depth by-1\relax,
         increment/.code=\global\advance\depth by 1\relax,
         left child/.style={bernoulli/decrement},
         right child/.style={}}


\newcommand{\eps}{\varepsilon}

% \newcommand{\tendi}[1]{\xrightarrow[\footnotesize #1 \rightarrow
%   +\infty]{}}%

\newcommand{\tend}{\rightarrow}%
\newcommand{\tendn}{\underset{n\to +\infty}{\longrightarrow}} %
\newcommand{\ntendn}{\underset{n\to
    +\infty}{\not\hspace{-.15cm}\longrightarrow}} %
% \newcommand{\tendn}{\xrightarrow[\footnotesize n \rightarrow
%   +\infty]{}}%
\newcommand{\Tendx}[1]{\xrightarrow[\footnotesize x \rightarrow
  #1]{}}%
\newcommand{\tendx}[1]{\underset{x\to #1}{\longrightarrow}}%
\newcommand{\ntendx}[1]{\underset{x\to #1}{\not\!\!\longrightarrow}}%
\newcommand{\tendd}[2]{\underset{#1\to #2}{\longrightarrow}}%
% \newcommand{\tendd}[2]{\xrightarrow[\footnotesize #1 \rightarrow
%   #2]{}}%
\newcommand{\tendash}[1]{\xdashrightarrow[\footnotesize #1 \rightarrow
  +\infty]{}}%
\newcommand{\tendashx}[1]{\xdashrightarrow[\footnotesize x \rightarrow
  #1]{}}%
\newcommand{\tendb}[1]{\underset{#1 \to +\infty}{\longrightarrow}}%
\newcommand{\tendL}{\overset{\mathscr L}{\underset{n \to
      +\infty}{\longrightarrow}}}%
\newcommand{\tendP}{\overset{\Prob}{\underset{n \to
      +\infty}{\longrightarrow}}}%
\newcommand{\tenddL}[1]{\overset{\mathscr L}{\underset{#1 \to
      +\infty}{\longrightarrow}}}%

\NewEnviron{attention}{ %
  ~\\[-.2cm]\noindent
  \begin{minipage}{\linewidth}
  \setlength{\fboxsep}{3mm}%
  \ \ \dbend \ \ %
  \fbox{\parbox[t]{.88\linewidth}{\BODY}} %
  \end{minipage}\\
}

\NewEnviron{sattention}[1]{ %
  ~\\[-.2cm]\noindent
  \begin{minipage}{#1\linewidth}
  \setlength{\fboxsep}{3mm}%
  \ \ \dbend \ \ %
  \fbox{\parbox[t]{.88\linewidth}{\BODY}} %
  \end{minipage}\\
}

%%%%% OBSOLETE %%%%%%

% \newcommand{\attention}[1]{
%   \noindent
%   \begin{tabular}{@{}l|p{11.5cm}|}
%     \cline{2-2}
%     \vspace{-.2cm} 
%     & \nl
%     \dbend & #1 \nl
%     \cline{2-2}
%   \end{tabular}
% }

% \newcommand{\attentionv}[2]{
%   \noindent
%   \begin{tabular}{@{}l|p{11.5cm}|}
%     \cline{2-2}
%     \vspace{-.2cm} 
%     & \nl
%     \dbend & #2 \nl[#1 cm]
%     \cline{2-2}
%   \end{tabular}
% }

\newcommand{\explainvb}[2]{
  \noindent
  \begin{tabular}{@{}l|p{11.5cm}|}
    \cline{2-2}
    \vspace{-.2cm} 
    & \nl
    \hand & #2 \nl [#1 cm]
    \cline{2-2}
  \end{tabular}
}


% \noindent
% \begin{tabular}{@{}l|lp{11cm}|}
%   \cline{3-3} 
%   \multicolumn{1}{@{}l@{\dbend}}{} & & #1 \nl
%   \multicolumn{1}{l}{} & & \nl [-.8cm]
%   & & #2 \nl
%   \cline{2-3}
% \end{tabular}

% \newcommand{\attention}[1]{
%   \noindent
%   \begin{tabular}{@{}@{}cp{11cm}}
%     \dbend & #1 \nl
%   \end{tabular}
% }

\newcommand{\PP}[1]{\mathcal{P}(#1)}
\newcommand{\FF}[1]{\mathcal{F}(#1)}

\newcommand{\DSum}[2]{\displaystyle\sum\limits_{#1}^{#2}\hspace{.1cm}}
\newcommand{\Sum}[2]{{\textstyle\sum\limits_{#1}^{#2}}\hspace{.1cm}}
\newcommand{\Serie}{\textstyle\sum\ }
\newcommand{\Prod}[2]{\textstyle\prod\limits_{#1}^{#2}}

\newcommand{\Prim}[3]{\left[\ {#1} \ \right]_{\scriptscriptstyle
   \hspace{-.15cm} ~_{#2}\, }^ {\scriptscriptstyle \hspace{-.15cm} ~^{#3}\, }}

% \newcommand{\Prim}[3]{\left[\ {#1} \ \right]_{\scriptscriptstyle
%     \!\!~_{#2}}^ {\scriptscriptstyle \!\!~^{#3}}}

\newcommand{\dint}[2]{\displaystyle \int_{#1}^{#2}\ }
\newcommand{\Int}[2]{{\rm{Int}}_{\scriptscriptstyle #1, #2}}
\newcommand{\dt}{\ dt}
\newcommand{\dx}{\ dx}

\newcommand{\llpar}[1]{\left(\!\!\!
    \begin{array}{c}
      \rule{0pt}{#1}
    \end{array}
  \!\!\!\right.}

\newcommand{\rrpar}[1]{\left.\!\!\!
    \begin{array}{c}
      \rule{0pt}{#1}
    \end{array}
  \!\!\!\right)}

\newcommand{\llacc}[1]{\left\{\!\!\!
    \begin{array}{c}
      \rule{0pt}{#1}
    \end{array}
  \!\!\!\right.}

\newcommand{\rracc}[1]{\left.\!\!\!
    \begin{array}{c}
      \rule{0pt}{#1}
    \end{array}
  \!\!\!\right\}}

\newcommand{\ttacc}[1]{\mbox{\rotatebox{-90}{\hspace{-.7cm}$\llacc{#1}$}}}
\newcommand{\bbacc}[1]{\mbox{\rotatebox{90}{\hspace{-.5cm}$\llacc{#1}$}}}

\newcommand{\comp}[1]{\overline{#1}}%

\newcommand{\dcomp}[2]{\stackrel{\mbox{\ \ \----}{\scriptscriptstyle
      #2}}{#1}}%

% \newcommand{\Comp}[2]{\stackrel{\mbox{\ \
%       \-------}{\scriptscriptstyle #2}}{#1}}

% \newcommand{\dcomp}[2]{\stackrel{\mbox{\ \
%       \-------}{\scriptscriptstyle #2}}{#1}}

\newcommand{\A}{\mathscr{A}}

\newcommand{\conc}[1]{
  \begin{center}
    \fbox{
      \begin{tabular}{c}
        #1
      \end{tabular}
    }
  \end{center}
}

\newcommand{\concC}[1]{
  \begin{center}
    \fbox{
    \begin{tabular}{C{10cm}}
      \quad #1 \quad
    \end{tabular}
    }
  \end{center}
}

\newcommand{\concL}[2]{
  \begin{center}
    \fbox{
    \begin{tabular}{C{#2cm}}
      \quad #1 \quad
    \end{tabular}
    }
  \end{center}
}

% \newcommand{\lims}[2]{\prod\limits_{#1}^{#2}}

\newtheorem{theorem}{Théorème}[]
\newtheorem{lemma}{Lemme}[]
\newtheorem{proposition}{Proposition}[]
\newtheorem{corollary}{Corollaire}[]

% \newenvironment{proof}[1][Démonstration]{\begin{trivlist}
% \item[\hskip \labelsep {\bfseries #1}]}{\end{trivlist}}
\newenvironment{definition}[1][Définition]{\begin{trivlist}
\item[\hskip \labelsep {\bfseries #1}]}{\end{trivlist}}
\newenvironment{example}[1][Exemple]{\begin{trivlist}
\item[\hskip \labelsep {\bfseries #1}]}{\end{trivlist}}
\newenvironment{examples}[1][Exemples]{\begin{trivlist}
\item[\hskip \labelsep {\bfseries #1}]}{\end{trivlist}}
\newenvironment{notation}[1][Notation]{\begin{trivlist}
\item[\hskip \labelsep {\bfseries #1}]}{\end{trivlist}}
\newenvironment{propriete}[1][Propriété]{\begin{trivlist}
\item[\hskip \labelsep {\bfseries #1}]}{\end{trivlist}}
\newenvironment{proprietes}[1][Propriétés]{\begin{trivlist}
\item[\hskip \labelsep {\bfseries #1}]}{\end{trivlist}}
\newenvironment{remarkSC}[1][Remarque]{\begin{trivlist}
\item[\hskip \labelsep {\bfseries #1}]}{\end{trivlist}}
\newenvironment{application}[1][Application]{\begin{trivlist}
\item[\hskip \labelsep {\bfseries #1}]}{\end{trivlist}}

% Environnement pour les réponses des DS
\newenvironment{answer}{\par\emph{Réponse :}\par{}}
{\vspace{-.6cm}\hspace{\stretch{1}}\rule{1ex}{1ex}\vspace{.3cm}}

\newenvironment{answerTD}{\vspace{.2cm}\par\emph{Réponse :}\par{}}
{\hspace{\stretch{1}}\rule{1ex}{1ex}\vspace{.3cm}}

\newenvironment{answerCours}{\noindent\emph{Réponse :}}
{\rule{1ex}{1ex}}%\vspace{.3cm}}


% footnote in footer
\newcommand{\fancyfootnotetext}[2]{%
  \fancypagestyle{dingens}{%
    \fancyfoot[LO,RE]{\parbox{0.95\textwidth}{\footnotemark[#1]\footnotesize
        #2}}%
  }%
  \thispagestyle{dingens}%
}

%%% définit le style (arabic : 1,2,3...) et place des parenthèses
%%% autour de la numérotation
\renewcommand*{\thefootnote}{(\arabic{footnote})}
% http://www.tuteurs.ens.fr/logiciels/latex/footnote.html

%%%%%%%% tikz axis
% \pgfplotsset{every axis/.append style={
%                     axis x line=middle,    % put the x axis in the middle
%                     axis y line=middle,    % put the y axis in the middle
%                     axis line style={<->,color=blue}, % arrows on the axis
%                     xlabel={$x$},          % default put x on x-axis
%                     ylabel={$y$},          % default put y on y-axis
%             }}

%%%% s'utilise comme suit

% \begin{axis}[
%   xmin=-8,xmax=4,
%   ymin=-8,ymax=4,
%   grid=both,
%   ]
%   \addplot [domain=-3:3,samples=50]({x^3-3*x},{3*x^2-9}); 
% \end{axis}

%%%%%%%%



%%%%%%%%%%%% Pour avoir des numéros de section qui correspondent à
%%%%%%%%%%%% ceux du tableau
\renewcommand{\thesection}{\Roman{section}.\hspace{-.3cm}}
\renewcommand{\thesubsection}{\Roman{section}.\arabic{subsection}.\hspace{-.2cm}}
\renewcommand{\thesubsubsection}{\Roman{section}.\arabic{subsection}.\alph{subsubsection})\hspace{-.2cm}}
%%%%%%%%%%%% 

%%% Changer le nom des figures : Fig. au lieu de Figure
\usepackage[font=small,labelfont=bf,labelsep=space]{caption}
\captionsetup{%
  figurename=Fig.,
  tablename=Tab.
}
% \renewcommand{\thesection}{\Roman{section}.\hspace{-.2cm}}
% \renewcommand{\thesubsection}{\Roman{section}
%   .\hspace{.2cm}\arabic{subsection}\ .\hspace{-.3cm}}
% \renewcommand{\thesubsubsection}{\alph{subsection})}

\newenvironment{tabliste}[1]
{\begin{tabenum}[\bfseries\small\itshape #1]}{\end{tabenum}} 

\newenvironment{noliste}[1] %
{\begin{enumerate}[\bfseries\small\itshape #1]} %
  {\end{enumerate}}

\newenvironment{nonoliste}[1] %
{\begin{enumerate}[\hspace{-12pt}\bfseries\small\itshape #1]} %
  {\end{enumerate}}

\newenvironment{arrayliste}[1]{ 
  % List with minimal white space to fit in small areas, e.g. table
  % cell
  \begin{minipage}[t]{\linewidth} %
    \begin{enumerate}[\bfseries\small\itshape #1] %
      {\leftmargin=0.5em \rightmargin=0em
        \topsep=0em \parskip=0em \parsep=0em
        \listparindent=0em \partopsep=0em \itemsep=0pt
        \itemindent=0em \labelwidth=\leftmargin\labelsep+0.25em}
    }{
    \end{enumerate}\end{minipage}
}

\newenvironment{nolistes}[2]
{\begin{enumerate}[\bfseries\small\itshape
    #1]\setlength{\itemsep}{#2 mm}}{\end{enumerate}}

\newenvironment{liste}[1]
{\begin{enumerate}[\hspace{12pt}\bfseries\small\itshape #1]}{\end{enumerate}} 



%%%%%%%% Pour les programmes de colle %%%%%%%

\newcommand{\cours}{{\small \tt (COURS)}} %
\newcommand{\poly}{{\small \tt (POLY)}} %
\newcommand{\exo}{{\small \tt (EXO)}} %
\newcommand{\culture}{{\small \tt (CULTURE)}} %
\newcommand{\methodo}{{\small \tt (MÉTHODO)}} %
\newcommand{\methodob}{\Boxed{\mbox{\tt MÉTHODO}}} %

%%%%%%%% Pour les TD %%%%%%%
\newtheoremstyle{exostyle} {\topsep} % espace avant
{.6cm} % espace apres
{} % Police utilisee par le style de thm
{} % Indentation (vide = aucune, \parindent = indentation paragraphe)
{\bfseries} % Police du titre de thm
{} % Signe de ponctuation apres le titre du thm
{ } % Espace apres le titre du thm (\newline = linebreak)
{\thmname{#1}\thmnumber{ #2}\thmnote{.
    \normalfont{\textit{#3}}}} % composants du titre du thm : \thmname
                               % = nom du thm, \thmnumber = numéro du
                               % thm, \thmnote = sous-titre du thm
 
\theoremstyle{exostyle}
\newtheorem{exercice}{Exercice}
\newtheorem{exerciceAP}{Exercice avec préparation}
\newtheorem{exerciceSP}{Exercice sans préparation}
\newtheorem*{exoCours}{Exercice}

%%%%%%%% Pour des théorèmes Sans Espaces APRÈS %%%%%%%
\newtheoremstyle{exostyleSE} {\topsep} % espace avant
{} % espace apres
{} % Police utilisee par le style de thm
{} % Indentation (vide = aucune, \parindent = indentation paragraphe)
{\bfseries} % Police du titre de thm
{} % Signe de ponctuation apres le titre du thm
{ } % Espace apres le titre du thm (\newline = linebreak)
{\thmname{#1}\thmnumber{ #2}\thmnote{.
    \normalfont{\textit{#3}}}} % composants du titre du thm : \thmname
                               % = nom du thm, \thmnumber = numéro du
                               % thm, \thmnote = sous-titre du thm
 
\theoremstyle{exostyleSE}
\newtheorem{exerciceSE}{Exercice}
\newtheorem*{exoCoursSE}{Exercice}

% \newcommand{\lims}[2]{\prod\limits_{#1}^{#2}}

\newtheorem{theoremSE}{Théorème}[]
\newtheorem{lemmaSE}{Lemme}[]
\newtheorem{propositionSE}{Proposition}[]
\newtheorem{corollarySE}{Corollaire}[]

% \newenvironment{proofSE}[1][Démonstration]{\begin{trivlist}
% \item[\hskip \labelsep {\bfseries #1}]}{\end{trivlist}}
\newenvironment{definitionSE}[1][Définition]{\begin{trivlist}
  \item[\hskip \labelsep {\bfseries #1}]}{\end{trivlist}}
\newenvironment{exampleSE}[1][Exemple]{\begin{trivlist} 
  \item[\hskip \labelsep {\bfseries #1}]}{\end{trivlist}}
\newenvironment{examplesSE}[1][Exemples]{\begin{trivlist}
\item[\hskip \labelsep {\bfseries #1}]}{\end{trivlist}}
\newenvironment{notationSE}[1][Notation]{\begin{trivlist}
\item[\hskip \labelsep {\bfseries #1}]}{\end{trivlist}}
\newenvironment{proprieteSE}[1][Propriété]{\begin{trivlist}
\item[\hskip \labelsep {\bfseries #1}]}{\end{trivlist}}
\newenvironment{proprietesSE}[1][Propriétés]{\begin{trivlist}
\item[\hskip \labelsep {\bfseries #1}]}{\end{trivlist}}
\newenvironment{remarkSE}[1][Remarque]{\begin{trivlist}
\item[\hskip \labelsep {\bfseries #1}]}{\end{trivlist}}
\newenvironment{applicationSE}[1][Application]{\begin{trivlist}
\item[\hskip \labelsep {\bfseries #1}]}{\end{trivlist}}

%%%%%%%%%%% Obtenir les étoiles sans charger le package MnSymbol
%%%%%%%%%%%
\DeclareFontFamily{U} {MnSymbolC}{}
\DeclareFontShape{U}{MnSymbolC}{m}{n}{
  <-6> MnSymbolC5
  <6-7> MnSymbolC6
  <7-8> MnSymbolC7
  <8-9> MnSymbolC8
  <9-10> MnSymbolC9
  <10-12> MnSymbolC10
  <12-> MnSymbolC12}{}
\DeclareFontShape{U}{MnSymbolC}{b}{n}{
  <-6> MnSymbolC-Bold5
  <6-7> MnSymbolC-Bold6
  <7-8> MnSymbolC-Bold7
  <8-9> MnSymbolC-Bold8
  <9-10> MnSymbolC-Bold9
  <10-12> MnSymbolC-Bold10
  <12-> MnSymbolC-Bold12}{}

\DeclareSymbolFont{MnSyC} {U} {MnSymbolC}{m}{n}

\DeclareMathSymbol{\filledlargestar}{\mathrel}{MnSyC}{205}
\DeclareMathSymbol{\largestar}{\mathrel}{MnSyC}{131}

\newcommand{\facile}{\rm{(}$\scriptstyle\largestar$\rm{)}} %
\newcommand{\moyen}{\rm{(}$\scriptstyle\filledlargestar$\rm{)}} %
\newcommand{\dur}{\rm{(}$\scriptstyle\filledlargestar\filledlargestar$\rm{)}} %
\newcommand{\costaud}{\rm{(}$\scriptstyle\filledlargestar\filledlargestar\filledlargestar$\rm{)}}

%%%%%%%%%%%%%%%%%%%%%%%%%

%%%%%%%%%%%%%%%%%%%%%%%%%
%%%%%%%% Fin de la partie TD

%%%%%%%%%%%%%%%%
%%%%%%%%%%%%%%%%
\makeatletter %
\newenvironment{myitemize}{%
  \setlength{\topsep}{0pt} %
  \setlength{\partopsep}{0pt} %
  \renewcommand*{\@listi}{\leftmargin\leftmargini \parsep\z@
    \topsep\z@ \itemsep\z@} \let\@listI\@listi %
  \itemize %
}{\enditemize} %
\makeatother
%%%%%%%%%%%%%%%%
%%%%%%%%%%%%%%%%

%% Commentaires dans la correction du livre

\newcommand{\Com}[1]{
% Define box and box title style
\tikzstyle{mybox} = [draw=black!50,
very thick,
    rectangle, rounded corners, inner sep=10pt, inner ysep=8pt]
\tikzstyle{fancytitle} =[rounded corners, fill=black!80, text=white]
\tikzstyle{fancylogo} =[ text=white]
\begin{center}

\begin{tikzpicture}
\node [mybox] (box){%

    \begin{minipage}{0.90\linewidth}
\vspace{6pt}  #1
    \end{minipage}
};
\node[fancytitle, right=10pt] at (box.north west) 
{\bfseries\normalsize{Commentaire}};

\end{tikzpicture}%

\end{center}
%
}

% \NewEnviron{remark}{%
%   % Define box and box title style
%   \tikzstyle{mybox} = [draw=black!50, very thick, rectangle, rounded
%   corners, inner sep=10pt, inner ysep=8pt] %
%   \tikzstyle{fancytitle} =[rounded corners, fill=black!80,
%   text=white] %
%   \tikzstyle{fancylogo} =[ text=white]
%   \begin{center}
%     \begin{tikzpicture}
%       \node [mybox] (box){%
%         \begin{minipage}{0.90\linewidth}
%           \vspace{6pt} \BODY
%         \end{minipage}
%       }; %
%       \node[fancytitle, right=10pt] at (box.north west) %
%       {\bfseries\normalsize{Commentaire}}; %
%     \end{tikzpicture}%
%   \end{center}
% }
% \makeatother


%%%
% Remarques comme dans le livre
%%%

\NewEnviron{remark}{%
  % Define box and box title style
  \tikzstyle{mybox} = [draw=black!50, very thick, rectangle, rounded
  corners, inner sep=10pt, inner ysep=8pt] %
  \tikzstyle{fancytitle} = [rounded corners , fill=black!80,
  text=white] %
  \tikzstyle{fancylogo} =[ text=white]
  \begin{center}
    \begin{tikzpicture}
      \node [mybox] (box){%
        \begin{minipage}{0.90\linewidth}
          \vspace{6pt} \BODY
        \end{minipage}
      }; %
      \node[fancytitle, right=10pt] at (box.north west) %
      {\bfseries\normalsize{Commentaire}}; %
    \end{tikzpicture}%
  \end{center}
}

\NewEnviron{remarkST}{%
  % Define box and box title style
  \tikzstyle{mybox} = [draw=black!50, very thick, rectangle, rounded
  corners, inner sep=10pt, inner ysep=8pt] %
  \tikzstyle{fancytitle} = [rounded corners , fill=black!80,
  text=white] %
  \tikzstyle{fancylogo} =[ text=white]
  \begin{center}
    \begin{tikzpicture}
      \node [mybox] (box){%
        \begin{minipage}{0.90\linewidth}
          \vspace{6pt} \BODY
        \end{minipage}
      }; %
      % \node[fancytitle, right=10pt] at (box.north west) %
%       {\bfseries\normalsize{Commentaire}}; %
    \end{tikzpicture}%
  \end{center}
}

\NewEnviron{remarkL}[1]{%
  % Define box and box title style
  \tikzstyle{mybox} = [draw=black!50, very thick, rectangle, rounded
  corners, inner sep=10pt, inner ysep=8pt] %
  \tikzstyle{fancytitle} =[rounded corners, fill=black!80,
  text=white] %
  \tikzstyle{fancylogo} =[ text=white]
  \begin{center}
    \begin{tikzpicture}
      \node [mybox] (box){%
        \begin{minipage}{#1\linewidth}
          \vspace{6pt} \BODY
        \end{minipage}
      }; %
      \node[fancytitle, right=10pt] at (box.north west) %
      {\bfseries\normalsize{Commentaire}}; %
    \end{tikzpicture}%
  \end{center}
}

\NewEnviron{remarkSTL}[1]{%
  % Define box and box title style
  \tikzstyle{mybox} = [draw=black!50, very thick, rectangle, rounded
  corners, inner sep=10pt, inner ysep=8pt] %
  \tikzstyle{fancytitle} =[rounded corners, fill=black!80,
  text=white] %
  \tikzstyle{fancylogo} =[ text=white]
  \begin{center}
    \begin{tikzpicture}
      \node [mybox] (box){%
        \begin{minipage}{#1\linewidth}
          \vspace{6pt} \BODY
        \end{minipage}
      }; %
%       \node[fancytitle, right=10pt] at (box.north west) %
%       {\bfseries\normalsize{Commentaire}}; %
    \end{tikzpicture}%
  \end{center}
}

\NewEnviron{titre} %
{ %
  ~\\[-1.8cm]
  \begin{center}
    \bf \LARGE \BODY
  \end{center}
  ~\\[-.6cm]
  \hrule %
  \vspace*{.2cm}
} %

\NewEnviron{titreL}[2] %
{ %
  ~\\[-#1cm]
  \begin{center}
    \bf \LARGE \BODY
  \end{center}
  ~\\[-#2cm]
  \hrule %
  \vspace*{.2cm}
} %




\pagestyle{fancy} %
\lhead{ECE2 \hfill septembre 2017 \\
 Mathématiques\\[.2cm]} %
\chead{\hrule} %
\rhead{} %
\lfoot{} %
\cfoot{} %
\rfoot{\thepage} %

\renewcommand{\headrulewidth}{0pt}% : Trace un trait de séparation
 % de largeur 0,4 point. Mettre 0pt
 % pour supprimer le trait.

\renewcommand{\footrulewidth}{0.4pt}% : Trace un trait de séparation
 % de largeur 0,4 point. Mettre 0pt
 % pour supprimer le trait.

\setlength{\headheight}{14pt}

\title{\bf \vspace{-1cm} ORAUX HEC 2007} %
\author{} %
\date{} %

\begin{document}

\maketitle %
\vspace{-1.2cm}\hrule %
\thispagestyle{fancy}

\vspace*{.4cm}

% DEBUT DU DOC À MODIFIER : tout virer jusqu'au début de l'exo

\section{Annales 2007}

 %\setcounter{exercice}{0}

 \begin{exercice} \indent \\
\\
 Question de cours : un estimateur d'un paramètre $\theta$ de la loi
$P_{X}$ d'une variable aléatoire $X$ dont on dispose d'un échantillon
$(X_{n})$ est une suite de variables aléatoires $(T_{n})$ où pour tout
$n$, $T_{n}$ est une fonction des variables $X_{1},\ \dots\, X_{n})$.
\\
 On définit alors son biais comme $\E(X - \theta)$, et son risque
quadratique comme l'espérance des écarts à $\theta$ mis au carré : \\
$R = E ( [X - \theta]^{2})$. \\
\\
 On considère $n$ ($n>2$) variables aléatoires réelles indépendantes
 $X_{1}$,..., $X_{n}$ de même loi de densité 
\[
 f_{0}\left( x\right) = \left\{  
\begin{array}{cc}
 \dfrac{3x^{2}}{\theta ^{3}} & \text{si }x\in \left[ 0,\theta \right]
\\
 0 & \text{sinon}
\end{array}
\right. 
\]
 où $\theta $ est un paramètre strictement positif. On pose 
\[
 S = X_{1} + \cdots + X_{n}\text{\quad et\quad }T = \max \left(
X_{1},\cdots,X_{n}\right)
\]

 \begin{noliste}{1.}
 \setlength{\itemsep}{4mm}
 \item $\E(X) = \frac{1}{\theta^{3}} \dint{0}{x} 3 x^{3}\ dx =
\frac{3}{4} \theta$ donc $\E(S) = \frac{3}{4} n \theta$. \\
 $\E(X^{2}) = \frac{1}{\theta^{3}} \dint{0}{x} x^{4}\ dx = \frac{3}{5}
\theta^{2}$. \\
 Enfin $\V(X) = \theta^{2} \left( \frac{3}{5} - \frac{9}{16} \right) =
\theta^{2} \frac{48 - 45}{80} = \frac{3}{80} \theta^{2}$ donc $\V(S) =
n \V(X) = \frac{3}{80} n \theta^{2}$. \\

 \item $\Prob\left(\Ev{\Ev{ T \leq t}}\right) = P \Ev{ X \leq t}{n} =
0$ si $x <0$, $1$ si $x > \theta$ et $\left( \frac{x}{\theta}
\right)^{3n}$ si $0 \leq x \leq \theta$. \\
\\
 On en déduit une densité $f_{T}(x) = 0$ si $x \notin [0 ; \theta]$ et
$f_{T}(x) = \frac{3n x^{3n-1} }{ \theta^{3n} }$. \\
\\
 Ensuite $\E(T) = \dint{0}{\theta} t f_{T}(t)\ dt = \frac{3n \theta
}{(3n + 1) }$. \\
\\
 De même $\E(T^{2}) = \frac{3 n \theta^{2} }{ (3n + 2) }$, et enfin :
\\
\\
 $\V(T) = \frac{3n \theta^{2} }{ (3n + 2) } - \frac{9 n^{2} \theta^{2}
}{ (3n + 1)^{2} }$. \\

 \item On suppose maintenant que $\theta $ est un paramètre inconnu
qu'on
 se propose d'estimer.

 \begin{noliste}{a)}
 \setlength{\itemsep}{2mm}
 \item Les variables $S' = \frac{4}{3n} S$ et $T' = \frac{3n + 1}{3n}
T$ sont des estimateurs sans biais de $\theta$. \\
 On a $\V(S') = \frac{16}{9 n^{2}} \V(S) = \frac{\theta^{2}}{15 n}$ et
$\V(T') = \frac{(3n + 1)^{2}}{9 n^{2}} \V(T) = \frac{ (3n + 1)^{2}
\theta^{2} }{ 3n (3n + 2) } - \theta^{2} = \theta^{2} \frac{(3n +
1)^{2} - 3n (3n + 2) }{3n (3n + 2)}$. \\
 $\V(T') = \frac{ \theta^{2}}{3n (3n + 2)}$. \\

 \item Les deux risques quadratiques tendent vers 0 mais le premier est
équivalent à $\frac{\theta^{2}}{15 n}$ et le deuxième à
$\frac{\theta^{2}}{ 9 n^{2}}$, donc $\V(T_{n}')$ converge plus
rapidement vers 0, et $T'$ est donc le meilleur des deux estimateurs.
 \end{noliste}
 \end{noliste}

 

 \noindent \textbf{\underline{Exercice sans préparation}} \\
\\
 Un agriculteur souhaite améliorer le rendement de son exploitation en
 utilisant de l'engrais. Une étude a montré que le rendement, en
 tonnes par hectare, pour la variété de blé cultivée est donn é par 
\[
 f\left( B,N\right) = 120B-8B^{2} + 4BN-2N^{2} 
\]
 $B$ désigne la, quantité de semences de blé utilisée, $N$ la
 quantité d'engrais utilisée.

 \begin{noliste}{1.}
 \setlength{\itemsep}{4mm}
 \item $f$ est de classe $C^{2}$, et on a : \\
 $f_{B}'(B,B) = 120 - 16 B + 4N$, $f_{N}'(B,N) = 4 B - 4 N$,
$f_{B,B}''(B,N) = -16$, $f_{B,N}''(B,N) = 4$ et $f_{N,N}'' (B,N) = -4$.
\\
\\
 Les points vérifient $120 - 16 B + 4 N = 0$ et $4B - 4 N = 0$, donc $B
= N$ et $120-12B = 0$, et enfin $B = N = 10$. \\
 Au point $(10,10)$ on a $r = -16$, $s = 4$ et $t = -4$ donc $rt -
s^{2} = 64 - 16 = 48 > 0$ et $r = -16 < 0$, c'est donc un maximum
local. \\
 De plus on peut écrire $f(B,N) = 120 B - 6 B^{2} -2 ( B^{2} - 2 B N +
N^{2}) = 120 B - 6 B^{2} - 2 ( B- N)^{2} = -6 [ (B-10)^{2} -100]- 2
(B-N)^{2} = -6 (B-10)^{2} - 2 (B-N)^{2} + 600 = -6 (B-10)^{2} - 2
(B-N)^{2} + f (10,10)$ donc $(10,10)$ est un maximum global de $f$. \\

 \item On traduit la contrainte : on a $B = 23 - 2N$ donc le rendement
est donné par $g(N) = f( 23 - 2N, N) = 120 (23 - 2N) - 8 (23 - 2 N)^{2}
+ 4 (23 - 2N) N - 2 N^{2} = N^{2} ( -32 -8 - 2) + N ( - 240 + 736 + 92)
+ (120 - 184) \times 23 = -42 N^{2} + 588 N - 64 \times 23$. \\
 On dérive : $g'(N) = - 84 N + 588$, qui s'annule pour $N =
\frac{588}{84} = \frac{294}{42} = \frac{147}{21} = \frac{21}{3} = 7$ et
$B = 23 - 14 = 9$ donc le rendement optimum vaut $f( 9, 7) = 1080 - 648
+ 252 - 98 = 432 + 154 = 586$.
 \end{noliste}
 \end{exercice}

 \newpage

 \begin{exercice} \indent \\
\\
 Soit $E$ un $\R$-espace vectoriel de dimension finie. On note pour
 tout endomorphisme $u$ de $E$ et pour tout $\in \N^*,$ $u^{0} =
\mathrm{Id}_{E}$ et 
 
\[
 u^{r} = \underset{r\text{ termes}}{\underbrace{u\circ \cdots \circ u}}
\]

 \noindent On commence par considérer un endomorphisme non nul de $E$
tel que pour
 tout $x\in E$ il existe $r\left( x\right) \in \N^*$ tel que $
u^{r\left( x\right) } = 0$

 \begin{noliste}{1.}
 \setlength{\itemsep}{4mm}
 \item Une matrice carrée d'ordre $n$ admettant $n$ valeurs propres
distinctes est diagonalisable. Une matrice telle que la somme des
dimensions de ses sous-espaces propres vaut $n$ est aussi
diagonalisable. On peut encore citer la base de vecteurs propres. \\

 \item On se donne une base $(e_{1},\ \dots\, e_{n})$ de $E$, et on
pose $r_{0} = \max \{ r( e_{i})\ |\ 1 \leq i \leq n \}$. \\
 Alors pour tout $i$, $u^{r_{0}} (e_{i}) = 0$ et pour tout $x \in E$
s'écrivant $\Sum{i = 1}{n} x_{i} e_{i}$, on a : \\
 $u^{r_{0}} (x) = \Sum{i = 1}{n} x_{i} u^{r_{0}} (e_{i}) = \Sum{i =
1}{n} 0 = 0$. \\
 Soit maintenant $A = \{ k \in \N\ | u^{k} = 0 \}$. \\
 $A$ est un sous-ensemble de $\R$ non vide donc il admet un minimum. On
note $r$ ce minimum, on a alors $r-1 \notin A$ donc $u^{r-1} \neq 0$ et
$u^{r} = 0$. \\
 Enfin comme $u \neq 0$ on sait que $r \geq 2$. \\

 \item Le polynôme $P\left(\Ev{x}\right) = x^{r}$ est annulateur de $u$
et admet pour unique racine 0 donc on a $\spc (u) \subset \{ 0 \}$. \\
 De plus 0 est valeur propre car $u$ n'est pas inversible. \\
 En effet, sinon en composant $r-1$ fois par $u^{-1}$ l'égalité $u^{r}
= 0$ on aurait $u = 0$, ce qui est absurde. \\
 Enfin comme $u$ admet pour unique valeur propre 0, on montre par
l'absurde que si il était diagonalisable, il admettrait pour matrice la
matrice nulle et serait nul. $u$ n'est donc pas diagonalisable. \\

 \item On suppose que $v(x) = 0$, on a alors $\Sum{k = 0}{r-1}
\frac{u^{k} (x)}{k!} = 0$. \\
 On compose par $u^{r-1}$, cela donne $u^{0} (x) = 0$ donc $x = 0$. \\
 L'application $v$ est donc injective, et comme c'est un endomorphisme,
c'est bien un isomorphisme de $E$. \\
 L'idée pour l'inverse est de remarquer l'analogie avec la série
exponentielle, et comme $e^{x} \times e^{-x} = 1$, on va poser $w =
\Sum{k = 0}{r-1} \frac{(-u)^{k} }{k!}$, et calculer $w \circ v$. \\
 On a $w \circ v = \Sum{k = 0}{r-1} \Sum{i = 0}{r-1} (-1)^{i}
\frac{u^{k + i}}{k! i!} \\
\\ = \Sum{k = 0}{r-1} \Sum{i = 0}{r-1} \frac{1}{(i + k)!} \binom{i +
k}{i} (-1)^{i} u^{k + i}$. \\
\\
 On réordonne la somme en posant $n = i + k$, on a $i \leq n$ car $k
\geq 0$, cela donne : \\
 $w \circ v = \Sum{n = 0}{r-1} \Sum{i = 0}{n} \frac{1}{n!} \binom{n}{i}
(-1)^{i} u^{n} $ car $u^{n} = 0$ pour $n > r-1$. \\
 $w \circ v = \Sum{n = 0}{r-1} \frac{u^{n}}{n!} \Sum{i = 0}{n}
\binom{n}{i} (-1)^{i} \times 1^{n-i}$. \\
 $w \circ v = \frac{u^{0}}{0!} + \Sum{n = 1}{r-1} \frac{u^{n}}{n!} (1 -
1)^{n} = \id$. \\
\\
 On obtient alors $v^{-1} = w = \Sum{k = 0}{r-1} \frac{(-u)^{k} }{k!}$.
\\

 \item Si $u(x) = 0$, alors $v(x) = x + \Sum{k = 1}{n} u^{k} (x) = x$
donc $( v -\id) (x) = 0$. \\
 On obtient $\ker u \subset \ker (v - \id)$. \\

 \item 0 est valeur propre de $u$ donc $\ker u \neq \{0 \}$, donc $\ker
(v - \id) \neq 0$, et 1 est valeur propre de $v$. \\
 Plus généralement, on se donne $\lambda \neq 1$, et on étudie : \\
 $v(x) = \lambda x \Leftrightarrow \lambda x = \Sum{k = 0}{r-1}
\frac{u^{k} (x)}{k!}$. \\
 On compose par $u^{r-1}$ et on obtient : \\
 $\lambda u^{r-1} (x) = u^{r-1} (x)$, donc $( \lambda -1) u^{r-1} (x) =
0$. \\
 On a $\lambda \neq 1$ donc $u^{r-1} (x) = 0$. \\
 On a alors $(\lambda -1) x = \Sum{k = 1}{r-2} \frac{u^{k} (x)}{k!}$,
et on compose par $u^{r-2}$, cela donne : \\
 $(\lambda -1) u^{r-2} (x) = 0$, donc $u^{r-2} (x) = 0$ et ainsi de
suite.... \\
 On obtient finalement $u(x) = 0$, puis $\lambda x = x$, $(\lambda -1)
x = 0$ et enfin $x = 0$ car $\lambda -1 \neq 0$. \\
 Il n'y a donc pas de solutions autres que 0, et la seule valeur propre
de $v$ est 1. \\
 \end{noliste}

 

 \noindent \textbf{\underline{Exercice sans préparation}} \\
\\
 Soient $n$ et $N$ des entiers non nuls. \\
 Une urne contient $n$ jetons numérotés de 1 à $n$. On effectue $ N $
tirages avec remise dans cette urne.

 \begin{noliste}{1.}
 \setlength{\itemsep}{4mm}
 \item $F_{i} \suit \mathcal{B} \left(N, \frac{1}{n} \right)$.

 On pose : 
 
\[
 F = \Sum{i = 1}{n}F_{i}
\]
 $F$ est une variable certaine égale à $N$, car elle compte le nombre
total de jetons tirés. \\
 On en déduit que $\E(F) = N$ et $\V(F) = 0$. \\

 Les variables $F_{i}$ ne sont pas deux à deux indépendantes. \\
 Par exemple pour $i \neq j$, $\Prob\left(\Ev{ F_{i} = N, F_{j} =
N}\right) = 0 \neq \Prob\left(\Ev{\Ev{ F_{i} = N}}\right)
\Prob\left(\Ev{\Ev{ F_{j} = N}}\right)$ (on ne peut pas tirer $2 N$
jetons). \\

 \item $X_{i}$ est une variable de Bernouilli de paramètre
$\Prob\left(\Ev{\Ev{ X_{i} = 1}}\right) = 1 - \Prob\left(\Ev{\Ev{ X_{i}
= 0}}\right) = 1 - P \Ev{ F_{i} = 0} = 1 - \left( 1 - \frac{1}{n}
\right)^{n}$. \\
 D'où $\E(X_{i}) = p = 1 - \left( 1 - \frac{1}{n} \right)^{n}$ et
$\V(X_{i}) = p (1-p) = \left[1 - \left( 1 - \frac{1}{n} \right)^{n}
\right] \times \left( 1 - \frac{1}{n} \right)^{n}$. \\
\\
 $P_{\Ev{X_{i} = 0}} \Ev{X_{j} = 0} = P_{F_{i} = 0} \Ev{F_{j} = 0} =
\left( 1 - \frac{1}{N-1} \right)^{n}$ car on tire parmi $N-1$ boules
(on sait que la boule $i$ n'est jamais tirée, on peut faire les calculs
en considérant qu'elle n'est pas là. \\
 on obtient $P_{\Ev{X_{i} = 0}} \Ev{X_{j} = 0} \neq \Prob\left(\Ev{\Ev{
X_{j} = 0}}\right)$ donc $X_{i}$ et $X_{j}$ ne sont pas indépendantes.
 \end{noliste}
 \end{exercice}

 \newpage

 \begin{exercice} \indent \\

 \begin{noliste}{1.}
 \setlength{\itemsep}{4mm}
 \item Une densité de probabilité est positive, continue sauf en un
nombre fini de points (c'est la cas ici à condition que $c \geq 0$ et
vérifie $\dint{-\infty}{+ \infty} f(t)\ dt = 1$. \\
 Ici on calcule $\dint{10}{x} \frac{x}{t^{2}}\ dt = \left[ \
\frac{-c}{t} \right]_{10}{x} = \frac{-c}{x} + \frac{c}{10}
\xrightarrow[x \rightarrow + \infty]{} \frac{c}{10}$ donc il faut $c =
10$, et $f$ est bien positive. \\

 \item Cela signifie que $F(m) = 1 - F(m)$, donc $F(m) = \frac{1}{2}$.
\\
 Or $F(x) = 0$ pour $x < 10$ (pas de solution dans cet intervalle) et
$F(x) = 1 - \frac{10}{x}$ sinon. \\
 D'où $m$ vérifie $\frac{10}{m} = \frac{1}{2}$, et enfin $m = 20$. \\

 \item On définit la loi binomiale de paramètres $5$ et
$\Prob\left(\Ev{\Ev{ X \geq 15}}\right) = \frac{10}{15} = \frac{2}{3}$.
\\
 On cherche $\Prob\left(\Ev{\Ev{ X \geq 3}}\right) = \binom{5}{3}
\left( \frac{2}{3} \right)^{3} \left( \frac{1}{3} \right)^{2} +
\binom{5}{4} \left( \frac{2}{3} \right)^{4} \left( \frac{1}{3} \right)
+ \binom{5}{5} \left( \frac{2}{3} \right)^{5} \left( \frac{1}{3}
\right)^{0} = 10 \frac{8}{3^{5}} + 5 \frac{16}{3^{5}} +
\frac{32}{3^{5}} = \frac{80 + 80 + 32}{3^{5}} = \frac{192}{3^{5}} =
\frac{64}{3^{4}} = \frac{64}{81}$. \\
 \\
 Deux machines $A$ et $B$ sont équipées de composants du type préc
édent. Plus précisément.

 \begin{noliste}{$\sbullet$}
 \item $A$ contient deux composants et cesse de fonctionner dès que
l'un
 de ces composants est défectueux,

 \item $B$ contient également deux composants mais un seul de ces
 composants suffit à la faire fonctionner
 \end{noliste}

 On note $T_{A}$, $T_{B}$ les durées de fonctionnement de ces machines.

 \item $T_{A} = \min (X_{1}, X_{2})$ donc $F_{T_{A}} (x) = 1 - \left( 1
- F_{X}(x) \right)^{2} = 1 - \frac{100}{x^{2}}$ si $x \geq 10$ et $0$
sinon, et $f_{T_{A}} (x) = \frac{200}{x^{3}}$ si $x \geq 10$ et $0$
sinon. \\
\\
 $T_{B} = \max (X_{1}, X_{2})$ donc $F_{T_{B}} (x) = ( F(x) )^{n} =
\left( 1 - \frac{10}{x} \right)^{2}$ si $x \geq 10$ et $0$ sinon, et $
f_{T_{B}} (x) = \frac{ 20}{x^{2}} \left( 1 - \frac{10}{x} \right)$ si
$x \geq 10$ et $0$ sinon. \\

 \item Pour $T_{A}$ comme $T_{B}$, le seul problème est en $ + \infty$.
\\
 Or $x t_{T_{A}} (x) = \frac{200}{x^{2}}$ est une fonction intégrable
en $ + \infty$ (Riemann) et $x f_{T_{B}} (x) \sim \frac{20}{x}$ en $ +
\infty$, qui n'est pas intégrable en $ + \infty$ (Riemann à nouveau)
donc par théorème de comparaison des intégrales de fonctions positives,
l'intégrale de $t f(t)$ diverge. \\
 D'où $T_{A}$ admet une espérance, mais pas $T_{B}$. \\
 Un calcul d'intégrale facile donne $\E(T_{A}) = 20$.
 \end{noliste}

 

 \noindent \textbf{\underline{Exercice sans préparation}} \\
\\
 \begin{noliste}{1.}
 \setlength{\itemsep}{4mm}
 \item Les vecteurs $(1,0)$, $(0,1)$ et $(1,1)$ remplissent cette
condition. \\
\\
 1er cas : il y a deux valeurs propres distinctes, alors les
sous-espaces propres sont de dimension 1. Alors si $e_{1}$ est vecteur
propre associé à $\lambda_{1}$, $E_{\lambda_{1}} = \Vect{ e_{1}}$ et
les deux autres ne sont pas vecteurs propres associés à $\lambda_{1}$.
Si l'un des deux autres est vecteur propre associé à $\lambda_{2}$, le
troisième ne sera alors pas vecteur propre. \\
 2e cas : une seule valeur propre, et un sous-espace propre associé de
dimension 2 ($u = \lambda \id$, alors les trois vecteurs (et tout
vecteur non nul de $\R^{2}$) sont vecteurs propres associé à la valeur
propre $\lambda$. \\

 \item Soit $f$ un endomorphisme de $\R^{n}$ et $\mathcal{F}$ une
 famille de $n + 1$ vecteurs propres de $f$ s'il en existe.

 \begin{noliste}{a)}
 \setlength{\itemsep}{2mm}
 \item Non, car $\Card \mathcal{F} > \dim \R^{n}$. \\

 \item Supposons qu'il existe plusieurs valeurs propres distinctes ($k$
valeurs propres), alors il existe une valeur propre $\lambda$ telle que
$\dim E_{\lambda} \leq n-1$. \\
 Soit $a$ un vecteur propre de $\mathcal{F}$ associé à $\lambda$, alors
la famille restante est une base de vecteurs propres. \\
 Il y a donc une sous-famille de cette famille qui est une base de
$E_{\lambda}$, et comme $a \in E_{\lambda}$, $a$ s'écrit $a = \sum
a_{i} e_{i}$ avec les $a_{i}$ non tous nuls car $a \neq 0$. \\
 On obtient que la famille $(a, e_{1},\ \dots\, e_{p})$ est liée, avec
$p \leq n-1$ donc il y a moins de $n$ vecteurs. On la complète avec
d'autres vecteurs de $\mathcal{F}$ et on obtient une sous-famille de
$\mathcal{F}$ constituée de $n$ vecteurs et liée, ce qui contredit
l'hypothèse de départ. \\
 On en déduit que $f$ admet une unique valeur propre, et comme elle
admet des bases de vecteurs propres, elle est diagonalisable et $f =
\lambda \id$.
 \end{noliste}
 \end{noliste}

 \end{exercice}

 \newpage

 \begin{exercice} \indent \\
\\
 \textbf{Question de cours.} Soit $(A_{i})_{i \in I}$ un système
complet d'évènements, c'est-à-dire une famille d'évènements deux à deux
incompatibles, dont la réunion est $\Omega$ l'univers de l'expérience
et dont aucun n'est négligeable (ils ont des probabilités non nulles).
\\
 Alors pour tout évènement $B \in \Omega$, $\Prob\left(\Ev{B}\right) =
\Sum{i \in I} \Prob( B \cap A_{i}) = \Sum{i \in I}
\Prob\left(\Ev{A_{i}}\right) P_{A_{i}} (B)$.
 \\
\\
 On lance deux pièces truquées : La pièce 1 donne pile avec une
 probabilité $p_{1}$ et la pièce 2 donne pile avec une probabilité,
$p_{2}$. \\
 On effectue les lancers de la façon suivante : on choisit une pièce
 uniformément au hasard et on lance la pièce choisie. Si on obtient
 pile, on relance la même pièce et ainsi de suite jusqu'a ce que l'on
 obtienne face; à ce moment on change de pièce - plus géné ralement,
dès que l'on obtient face, on change de pièce. On suppose
 que $p_{1}$ et $p_{2}$ sont dans $\left] 0,1\right[ $

%%%LALALA 

 \begin{noliste}{1.}
 \setlength{\itemsep}{4mm}
 \item On pose $A_{n}$, de probabilité $a_{n}$, l'évènement : "lancer
la pièce 1 au $n$-ième lancer" et $B_{n}$, de probabilité $b_{n}$,
l'évènement : "lancer la pièce 2 au $n$-ième lancer". \\
 Comme $(A_{n}, B_{n})$ est un système complet d'évènements on a : \\
 $a_{n + 1} = \Prob\left(\Ev{A_{n}}\right) P_{A_{n}} (A_{n + 1}) +
\Prob\left(\Ev{B_{n}}\right) P_{B_{n}} (A_{n + 1}) = a_{n} p_{1} +
(1-a_{n}) (1-p_{2}) = a_{n} (p_{1} + p_{2} -1) + 1 - p_{2}$. \\
\\
 On reconnaît une suite arithmético-géométrique, on résout $k = k
(p_{1} + p_{2} -1) + 1 - p_{2} \Leftrightarrow k ( 2 - p_{1} - p_{2}) =
1 - p_{2} \Leftrightarrow k = \frac{1-p_{2}}{2 - p_{1}-p_{2}}$. \\
 La suite $u_{n} = a_{n} - k$ est alors géométrique de raison $(p_{1} +
p_{2} -1)$, donc pour tout $n$ on a :\\
 $a_{n} = u_{n} + \frac{1-p_{2}}{2 - p_{1}-p_{2}} = (p_{1} + p_{2}
-1)^{n-1} u_{1} + \frac{1-p_{2}}{2 - p_{1}-p_{2}} = (p_{1} + p_{2}
-1)^{n-1} \left(a_{1} -\frac{1-p_{2}}{2 - p_{1}-p_{2}} \right) +
\frac{1-p_{2}}{2 - p_{1}-p_{2}} = (p_{1} + p_{2} -1)^{n-1}
\left(\frac{1}{2}-\frac{1-p_{2}}{2 - p_{1}-p_{2}} \right) +
\frac{1-p_{2}}{2 - p_{1}-p_{2}}$. \\

 \item Avec le même système complet on a en posant $P_{n}$ : "on
obtient pile au $n$-ième tirage" : \\
 $r_{n} = \Prob\left(\Ev{A_{n} }\right) P_{A_{n} } (P_{n}) +
\Prob\left(\Ev{B_{n}}\right) P_{B_{n} } (P_{n}) = a_{n} p_{1} + (1 -
a_{n} ) p_{2} = a_{n} ( p_{1} - p_{2}) + p_{2} = (p_{1} - p_{2}) \left[
(p_{1} + p_{2} -1)^{n-1} \left(\frac{1}{2}-\frac{1-p_{2}}{2 -
p_{1}-p_{2}} \right) + \frac{1-p_{2}}{2 - p_{1}-p_{2}} \right] +
p_{2}$. \\

 \item On a $0 < p_{1} < 1$ et $0 < p_{2} < 1$ donc $0 < p_{1} + p_{2}
< 2$ et enfin $-1 < 1 - p_{1} - p_{2} < 1$. \\
 D'où $\lim\ (p_{1} + p_{2} -1)^{n-1} = 0$, et on obtient $L =
\frac{(p_{1} - p_{2}) (1- p_{2})}{2 - p_{1} - p_{2}} + p_{2}$. \\

%%%LALALA 

 \item On calcule $L = \frac{ \frac{1}{6} \times \frac{5}{6} }{ {3}{2}
} + \frac{1}{6} = \frac{5}{36} \frac{2}{3} + \frac{1}{6} = \frac{5}{54}
+ \frac{9}{54} = \frac{14}{54} = \frac{7}{27}$. \\
 Ensuite on simplifie $r_{n} = \frac{1}{6} \left[ \ \left( -
\frac{1}{2} \right)^{n-1} \frac{-1}{18} + \frac{5}{9} \right] +
\frac{1}{6}$. \\
 Petit problème avec l'énoncé : on ne peut pas calculer un rang à
partir duquel... avec un programme! On peut par contre calculer la
première valeur telle que..., mais sans assurance que les suivantes
vérifieront toutes la propriété (penser au fait que la suite n'est pas
monotone). Par contre si on veut un rang à partir duquel... il faut
réaliser une étude mathématique pour majorer ( ou calculer) $| r_{n} -
L | = \frac{1}{54} \left( \frac{1}{2} \right)^{n-1}$; dans ce cas
l'intervention de l'informatique est bien inutile : un petit passage au
logarithme permet de déterminer $n$ sans problème. \\
 On obtient le programme suivant : \\
 var n : integer; r : real ; \\
 n : = 1; r : = 1/54 ;\\
 repeat \\
 n : = n + 1; r : = r/2 ; \\
 until r < exp (-6 * ln ( 10) ) ; \\
 writeln (n) ; \\
 (programme qui sert à calculer une valeur qu'une calculatrice et un
calcul de logarithme donnerait sans problème...) \\
 \end{noliste}

 

 \noindent \textbf{\underline{Exercice sans préparation}} \\
\\
 Soit $u$ et $v$ deux endomorphismes de $\R^{2}$ dont les matrices
 respectives dans la base canonique $\left( e_{1},e_{2}\right) $ de
$\mathbb{R }{2}$ sont notées $A$ et $B$. On suppose que $AB =
\begin{smatrix}
 0 & 0 \\
 0 & 0
\end{smatrix}
$ et $BA = \begin{smatrix}
 0 & 1 \\
 0 & 0
\end{smatrix}
$

%%%LALALA 
 \begin{noliste}{1.}
 \setlength{\itemsep}{4mm}
 \item Si $v$ était bijectif, $B$ serait inversible et on aurait $A B
B^{-1} = 0 B^{-1}$ donc $A = 0$, puis $B A = B 0 = 0$ ce qui est
absurde. \\
 Donc $v$ n'est pas bijectif. \\
 On en déduit que $\im(v)$ est de dimension inférieure ou égale à 1. \\
 D'autre part on a $\im(v) \circ u \subset \im(v)$ et $\im(v) \circ u =
\Vect{ e_{1}}$ est de dimension 1. \\
 On en déduit que $\dim (\im(v)) \geq 1$, puis $\dim (\im(v)) = 1$ et avec
l'égalité des dimensions, $\im(v) = \im(v) \circ u = \Vect{ e_{1}}$. \\

 \item De la même manière $u$ n'est ni bijectif ni nul donc $\dim \ker
u = 1$. \\
 Or $\ker (v \circ u) \subset \ker u$ est de dimension 1, donc $\ker u
= \ker (v \circ u)$ et $e_{1} \in \ker (v \circ u)$, donc $(e_{1})$
(famille libre car constituée d'un vecteur non nulle, et de cardinal
égal à la dimension) est une base de $\ker (v \circ u)$, et enfin $\ker
u = \Vect{ e_{1}} = \im(v)$. \\

%%%LALALA 

 \item On a $u(e_{1}) = 0$ donc $A = \begin{smatrix}
0 & a \\
0 & b \\
\end{smatrix}
$ et $\im(v) = \Vect{ e_{1}}$ donc $B =
\begin{smatrix}
c & d \\
0 & 0 \\
\end{smatrix}
$. \\
 En calculant $AB$ et $BA$ on obtient bien $AB = 0$ et $BA =
\begin{smatrix}
0 & ac + bd \\
0 & 0 \\
\end{smatrix}
$ donc il faut rajouter la condition $ac + bd = 1$ pour satisfaire les
hypothèses.
 \end{noliste}
 \end{exercice}


 \newpage

%%%LALALA 

\begin{exercice} \indent \\
\\
 Soit $\alpha >0,$ $x_{0}>0$ et $f$ la fonction de $\R$ dans $\mathbb{
R}$ définie par : 
\[
 f\left( x\right) = \frac{\alpha }{x_{0}}\left( \frac{x_{0}}{x}\right)
 ^{\alpha + 1}\text{ si }x\geq x_{0}\text{\quad et\quad \ }f\left(
x\right) = 0 \text{ sinon}
\]
 \begin{noliste}{1.}
 \setlength{\itemsep}{4mm}
 \item 
 \begin{noliste}{a)}
 \setlength{\itemsep}{2mm}
 \item Une variable aléatoire réelle de fonction de répartition $F$ est
dite à densité s'il existe une fonction $f$ définie sur $\R$ positive,
continue sauf éventuellement en un nombre fini de points, et vérifiant
$\dint{-\infty}{+ \infty} f(t)\ dt = 1$, telle que pour tout $x \in
\R$, $F(x) = \dint{-\infty}{+ \infty} f(t)\ dt$. \\
\\
 Ici on a $\dint{-\infty}{+ \infty} f(t)\ dt = \dlim{x \rightarrow +
\infty} \dint{x_{0}}{x} \frac{\alpha}{x_{0}} \left( \frac{x_{0}}{t}
\right)^{\alpha + 1}\ dt = \dlim{x \rightarrow + \infty} x_{0}{\alpha}
\left[ - \frac{1}{t^{\alpha}} \right]_{x_{0}}{x} = \dlim{x \rightarrow
+ \infty} \left( - \frac{x_{0}{\alpha} }{x^{\alpha} } +
\frac{x_{0}{\alpha}}{x_{0}{\alpha}} \right) = 1$. \\
\\
 Ensuite on a $F(x) = 0$ si $x < x_{0}$, et $F(x) = 1 - \left(
\frac{x_{0}}{x} \right)^{\alpha}$ si $x \geq x_{0}$. \\

%%%LALALA 

 \item Seule l'intégrabilité en $ + \infty$ pose problème. \\
 De plus on a $t f(t) = \alpha x_{0}{\alpha} \frac{1}{x^{\alpha}}$ qui
est intégrable en $ + \infty$ si et seulement si $\alpha > 1$ (Riemann)
donc $X$ admet une espérance si et seulement si $\alpha > 1$. \\
 De même $X$ admet un moment d'ordre 2 donc une variance si et
seulement si $\alpha > 2$. \\
 Enfin si $\alpha > 1$, un calcul d'intégrale facile donne $\E(X) =
\frac{\alpha}{\alpha - 1} x_{0}$. \\

 \item $M_{X}(x) = \frac{ \E(X) - \dint{-\infty}{x} t f(t)\ dt }{ 1 -
F_{X}(x) } = \frac{\E(X) - \dint{x_{0}}{x} t f(t)\ dt}{ \left(
\frac{x_{0}}{x} \right)^{\alpha}} = \frac{ \alpha x_{0}{\alpha} }{
(\alpha - 1) x^{\alpha -1} } \times \frac{x^{\alpha}}{x_{0}{\alpha}} =
\frac{ \alpha}{\alpha -1} x$. \\
 \end{noliste}

%%%LALALA 

\item On se propose d'établir une réciproque de la propriété
  précédente. Soit $x_{0}>0$ et $Y$ une variable aléatoire à valeurs
  dans $\left[ x_{0}, + \infty \right[ $ de densité $h$ continue, à
  valeurs strictement positives, admettant une espérance et telle
  qu'il existe un réel $k>1$ vérifiant : 
  \[
  \forall x>x_{0}, \ M_{Y}\left( x \right) = \dfrac{\dint{x}{+\infty}
    th\left( t\right) dt}{\dint{x}{+ \infty} h\left( t\right) dt} = kx
  \]
  

%%%LALALA 

 \begin{noliste}{a)}
 \setlength{\itemsep}{2mm}
 \item On pose, pour tout $x>x_{0}$ 
\[
 G\left( x\right) = \dint{x}{+ \infty }h\left( t\right) dt 
\]
On a $G'(x) = - h(x)$, puis en dérivant l'égalité $\dint{x}{+ \infty
}t h\left( t\right) dt = k x \dint{x}{+ \infty }h\left( t\right) dt$
on a $- x h(x) = - k x h(x) + k G(x)$, donc $ x G'(x) - k x G'(x) = k
G(x)$ en enfin $G(x) = \frac{1-k}{k} x G'(x)$. \\


%%%LALALA 

 \item $A'(x) = x^{ \frac{k}{k-1} } G'(x) + \frac{k}{k-1} x^{
\frac{k}{k-1} - 1} G(x) = G'(x) x^{ \frac{k}{k-1} } \left( 1 +
\frac{k}{k-1} \frac{1}{x} \frac{1-k}{k} x \right) = G'(x) x^{
\frac{k}{k-1} } ( 1 - 1) = 0$. \\
\\
 On en déduit que pour tout $x \in \R$, $A(x) = K$ une constante
réelle, puis que $G(x) = K x^{-\frac{k}{k-1}}$. \\
 Or en la valeur $x = x_{0}$, on a $G(x_{0}) = 1$ donc $K = x_{0}{
\frac{k}{k-1} } $, et enfin : \\
 $F_{Y}(x) = 1 - G(x) = 1 - \left( \frac{x_{0}}{ x}
\right)^{\frac{k}{k-1}}$, et on obtient la loi de Pareto de paramètres
$x_{0}$ et $\frac{k}{k-1}$. \\
\\
 Enfin avec la valeur en $x_{0}$ de $M_{Y}$ on peut retrouver
$\frac{\E(Y)}{1} = k x_{0}$, donc $\E(Y) = k x_{0}$ (on a bien $\frac{
\frac{k}{k-1 } }{ \frac{k}{k-1} - 1} = k$, et on retrouve les résultats
précédents.
 \end{noliste}
 \end{noliste}


%%%LALALA 
 

 \noindent \textbf{\underline{Exercice sans préparation}} \\
\\
 Soient $X$ et $Y$ deux variables aléatoires binomiales de paramètres 
 $\left( n,1/2\right) $ indépendantes. Calculer $\Prob\left(\Ev{
 X = Y}\right).$ \\
\\
 On paramètre par le système complet d'évènements $\Ev{X = k}_{0 \leq k
\leq n}$ : \\
 $\Prob\left(\Ev{\Ev{X = Y}}\right) = \Sum{k = 0}{n} \Prob\left(\Ev{ X
= k, Y = k}\right) = \Sum{k = 0}{n} \binom{n}{k} ^{2} \frac{1}{2^{n}}
\frac{1}{2^{n}} = \frac{1}{4^{n}} \Sum{k = 0}{n} \binom{n}{k} ^{2}$. \\
\\
 On ne sait pas calculer cette somme, la méthode échoue. L'idée est en
fait d'utiliser le fait que $\Prob\left(\Ev{\Ev{ X = k}}\right) =
\Prob\left(\Ev{\Ev{ X = n-k}}\right)$. \\
 On a alors $\Prob\left(\Ev{\Ev{ X = Y}}\right) = \Sum{k = 0}{n}
\Prob\left(\Ev{\Ev{ X = k}}\right) \Prob\left(\Ev{\Ev{ Y = k}}\right) =
\Sum{k = 0}{n} \Prob\left(\Ev{\Ev{ X = k }}\right) \Prob\left(\Ev{\Ev{
Y = n-k}}\right) = \Prob\left(\Ev{\Ev{ X + Y = n}}\right)$. \\
 On sait que $X + Y \suit \mathcal{B} \left( 2n, \frac{1}{2} \right)$,
on en déduit que $\Prob\left(\Ev{\Ev{ X = Y}}\right) = \binom{2n}{n}
\frac{1}{2^{2n}} = \binom{2n}{n} \frac{1}{4^{n}}$. \\
 \end{exercice}

 \newpage


%%%LALALA 

 \begin{exercice} \indent \\
 \begin{noliste}{1.}
 \setlength{\itemsep}{4mm}
 \item Une famille $(f_{i})_{i \in I}$ de vecteurs de $E$ est dite
génératrice de $E$ si $\Vect ( (f_{i})_{i \in I} ) = E$, c'est-à-dire
que tout vecteur de $E$ peut s'écrire comme combinaison linéaire de la
famille. \\
 Elle est dite libre si toute combinaison linéaire est unique,
c'est-à-dire : $\Sum{i \in I} a_{i} f_{i} = 0 \Rightarrow \forall i \in
I,\ a_{i} = 0$. \\

 \hspace{-1cm}Dans l'espace vectoriel $C^{O}\left( \R\right) $ des
 fonctions continues de $\R$ dans $\R$, on considère les
 trois fonctions 
\[
 f_{1} :x\rightarrow 1\quad f_{2} :x\rightarrow x\quad \text{et}
 \quad f_{0} :x\rightarrow \left\{ 
\begin{array}{cc}
 x\ln \left| x\right| & \text{si }x\neq 0 \\
 0 & \text{si }x = 0
\end{array}
\right. 
\]


%%%LALALA 

 \hspace{-1cm}Soit $E$ le sous-espace vectoriel de $C^{0}\left( \R
\right) $ engendré par $\left( f_{1},f_{2},f_{3}\right) $

 \item Elle est génératrice de $E$ par définition, montrons qu'elle est
libre. \\
 Soient $a, b,c$ tels que $a f_{1} + b f_{2} + c f_{3} = 0$, donc $a +
b x + c x \ln | x | = 0$ pour tout $x \neq 0$, et $a + 0 + 0 = 0$ pour
$x = 0$. \\
 On vient d'obtenir $a = 0$, il reste $b x + c x \ln | x | = 0$ pour
tout $x \neq 0$. \\
 On a alors $x ( b + c \ln | x |) = 0$, donc $b + c x \ln | x | = 0$
pour tout $x \neq 0$. \\
 En $x = 1$ on obtient $b + c \ln 1 = b = 0$, donc $b = 0$. \\
 On en déduit que $c \ln | x | = 0$, pour $x \neq 0$, on prend $x = 2$
en on obtient $c \ln 2 = 0$, donc $c = 0$. \\

 \item A toute fonction $f$ de $E$ on associe la fonction $\Phi \left(
 f\right) $ définie par $\Phi \left( f\right) = \left( xf\right)
^{\prime
 },$ dérivée de la fonction $x\rightarrow x~f\left( x\right) $ \\
\\
 On a $\Phi (f_{1}) (x) = 1$ donc $\Phi (f_{1}) = f_{1} \in E$, $\Phi
(f_{2}) (x) = 2x$ donc $\Phi (f_{2}) = 2 f_{2} \in E$, et $\Phi (f_{3})
(x) = 2 x \ln | x | + x^{2} \frac{1}{x} = 2 x \ln | x | + x$ pour $x
\neq 0$ (qui est pour l'instant un abus de notation car on n'a pas
prouvé que $x \rightarrow x f_{3} (x)$ est dérivable sur $\R$. \\
 On vérifie alors que cette dérivée est prolongeable par continuité, on
en déduit par prolongement de la dérivée que $x \rightarrow x f_{3}
(x)$ est dérivable sur $\R$, de dérivée $2 f_{3}(x) + f_{2} (x)$. \\
 D'où $\Phi (f_{3})$ est bien définie et $\Phi (f_{3}) = 2 f_{3} +
f_{2} \in E$. \\
 Toute combinaison linéaire de $f_{1}$, $f_{2}$ et $f_{3}$ admet donc
une image par $\Phi$ et cette image est combinaison linéaire (car
$\Phi$ est trivialement linéaire) de $\Phi (f_{1}), \Phi (f_{2}), \Phi
(f_{3})$ qui sont tous dans $E$, donc elle est dans $E$. \\
 On obtient bien que $\Phi$ est un endomorphisme de $E$, et on a : $M =
\begin{smatrix}
1 & 0 & 0 \\
0 & 2 & 1 \\
0 & 0 & 2 \\
\end{smatrix}
$. \\

 \begin{noliste}{a)}
 \setlength{\itemsep}{2mm}
 \item $M$ est inversible car triangulaire sans 0 sur la diagonale. \\
 Pour son inverse, on peut réaliser : \\
 - un pivot très rapide. \\
 - conjecturer que $M^{-1} = \begin{smatrix}
1 & 0 & 0 \\
0 & \frac{1}{2} & a \\
0 & 0 & \frac{1}{2} \\
\end{smatrix}
$ et montrer que $a = -\frac{1}{4}$ en calculant $M M^{-1}$. \\
 - calculer $M^{2}$ et essayer d'en tirer un polynôme annulateur (cela
échoue). \\


%%%LALALA 

 \item L'équation $\Phi (f) = g$ est équivalente à $\Phi^{-1} (g) = f$,
donc $Mat (f) = M^{-1} Mat (g) = \begin{smatrix}
1 & 0 & 0 \\
0 & \frac{1}{2} & -\frac{1}{4} \\
0 & 0 & \frac{1}{2} \\
\end{smatrix}
\begin{smatrix}
a \\
b \\
1 \\
\end{smatrix}
 = \begin{smatrix}
a \\
\frac{1}{2} b - \frac{1}{4} \\
\frac{1}{2} \\
\end{smatrix}
$. \\
 La solution de l'équation est donc $f = a f_{1} + \left( \frac{1}{2} b
- \frac{1}{4} \right) f_{2} + \frac{1}{2} f_{3}$. \\
\\
 En particulier on voit que $\Phi \left( \frac{1}{2} f_{3} -
\frac{1}{4} f_{2} \right) = f_{3}$, donc une primitive de $f_{3}$ est :
\\
 $h(x) = \frac{1}{2} x \ln | x | - \frac{1}{4} x^{2}$. \\
 \end{noliste}
 \end{noliste}

 

 \noindent \textbf{\underline{Exercice sans préparation}} \\
\\
 Soient $X$ et $Y$ deux variables aléatoires binomiales de paramètres 
 $\left( n,1/2\right) $ indépendantes. Calculer $\Prob\left(\Ev{
 X = Y}\right).$ \\
\\
 On paramètre par le système complet d'évènements $\Ev{X = k}_{0 \leq k
\leq n}$ : \\
 $\Prob\left(\Ev{\Ev{X = Y}}\right) = \Sum{k = 0}{n} \Prob\left(\Ev{ X
= k, Y = k}\right) = \Sum{k = 0}{n} \binom{n}{k} ^{2} \frac{1}{2^{n}}
\frac{1}{2^{n}} = \frac{1}{4^{n}} \Sum{k = 0}{n} \binom{n}{k} ^{2}$. \\
\\
 On ne sait pas calculer cette somme, la méthode échoue. L'idée est en
fait d'utiliser le fait que $\Prob\left(\Ev{\Ev{ X = k}}\right) =
\Prob\left(\Ev{\Ev{ X = n-k}}\right)$. \\
 On a alors $\Prob\left(\Ev{\Ev{ X = Y}}\right) = \Sum{k = 0}{n}
\Prob\left(\Ev{\Ev{ X = k}}\right) \Prob\left(\Ev{\Ev{ Y = k}}\right) =
\Sum{k = 0}{n} \Prob\left(\Ev{\Ev{ X = k }}\right) \Prob\left(\Ev{\Ev{
Y = n-k}}\right) = \Prob\left(\Ev{\Ev{ X + Y = n}}\right)$. \\
 On sait que $X + Y \suit \mathcal{B} \left( 2n, \frac{1}{2} \right)$,
on en déduit que $\Prob\left(\Ev{\Ev{ X = Y}}\right) = \binom{2n}{n}
\frac{1}{2^{2n}} = \binom{2n}{n} \frac{1}{4^{n}}$. \\
 \end{exercice}


 \newpage

 \begin{exercice} \indent \\
\\
 \textbf{Question de cours :} Une variable aléatoire suit la loi
uniforme sur $[ a ;b]$ si c'est une variable à densité dont une densité
est une fonction constante sur $[ a ; b]$ et nulle ailleurs. \\
 De plus pour que ce soit une densité de probabilité, un calcul
d'intégrale trivial donne cette constante égale à $\frac{1}{b-a}$. \\
\\
 On obtient alors que la fonction $f(x) = \left\{
\begin{array}{cl}
 \frac{1}{b-a} \text{ si } x \in [a ;b ] \\
0 \text{ sinon}
\end{array}
\right.$ est une densité de $X$ et la fonction de répartition de $X$
est $F(x) = \left\{
\begin{array}{cl}
 0 \text{ si } x \leq a \\
\frac{x-a}{b-a} \text{ si } x \in [a ;b ] \\
1 \text{ si } x \geq b
\end{array}
\right.$.
 \\
 Soit $X$ une variable aléatoire suivant une loi uniforme sur $\left[
0,1 \right] $. On pose $Y = \min \left( X,1-X\right) $ et $Z = \left(
X,1-X\right) $.

 \begin{noliste}{1.}
 \setlength{\itemsep}{4mm}
 \item On a $Y (\Omega) = \left[ 0 ; \frac{1}{2} \right]$ car si $x
\geq \frac{1}{2}$, alors $1- x \leq \frac{1}{2}$ donc le minimum est
toujours plus petit que $\frac{1}{2}$. \\
 On a donc $F_{Y}(x) = 0$ si $x < 0$ et $1$ si $x > \frac{1}{2}$, et
pour tout $ x \in \left[ 0 ; \frac{1}{2} \right]$ on a : \\
 $F_{Y}(x) = 1 - P\left(\Ev{ \min ( X, 1- X) \geq x }\right)\left(\Ev{
X, 1- X}\right) \geq x ) = 1 - \Prob\left(\Ev{ X \geq x, 1- X \geq
x}\right) = 1 - \Prob\left(\Ev{ X \geq x, X \leq 1-x}\right) = 1 -
\Prob\left(\Ev{ x \leq X \leq 1-x}\right) = 1 - (1-x-x) = 2x$. \\
 On en déduit que $f_{Y}(x) = 2$ si $0 \leq x \leq \frac{1}{2}$ et 0
sinon est une densité de $Y$, donc $Y$ suit la loi uniforme sur $\left[
0 ; \frac{1}{2} \right]$, et $\E(Y) = \frac{1}{4}$. \\

 \item On peut utiliser la loi du max et refaire le même genre de
raisonnement qu'à la question précédente. \\
 On peut aussi remarquer que $Z = 1 - Y$ donc $Z (\Omega) = \left[ \
\frac{1}{2} ; 1\right]$ et pour tout $x \in \left[ \ \frac{1}{2} ;
1\right]$, $\Prob\left(\Ev{\Ev{ Z \leq x}}\right) = \Prob\left(\Ev{ 1 -
Y \leq x}\right) = \Prob\left(\Ev{ Y \geq 1 - x}\right) = 1 -
\Prob\left(\Ev{\Ev{ Y \leq 1-x}}\right) = 1 - 2 (1-x) = 2x - 1$, et
$\Prob\left(\Ev{\Ev{ Z \leq x}}\right) = 0$ sinon. \\
 On en déduit que $f_{Z}(x) = 2$ si $\frac{1}{2} \leq x \leq 1$ et 0
sinon est une densité de $Z$, qui suit la loi uniforme sur $\left[ \
\frac{1}{2} ; 1\right]$ et vérifie donc $\E(Z) = \frac{3}{4}$. \\

 \item Comme $Y$ et $Z$ sont positives, et $Z \geq Y$, $ \frac{Y}{Z}
(\Omega) \subset [ 0 ; 1]$. \\
 Pour $x < 0$, $\Prob\left(\Ev{\Ev{ R \leq x}}\right) = 0$, et pour $x
> 1$, $\Prob\left(\Ev{\Ev{ R \leq x}}\right) = 1$. \\
 De plus on a pour tout $x \in [ 0 ; 1]$ : \\
 $\Prob\left(\Ev{\Ev{ R \leq x}}\right) = \Prob\left(\Ev{\Ev{ Y \leq x
Z}}\right) = \Prob\left(\Ev{ Y \leq x - x Y}\right) = P \left(\Ev{ Y
\leq \frac{x}{1 + x}}\right) = 2 \frac{x}{1 + x}$. \\
\\
 On en déduit que la fonction $f_{R}(x) = \frac{2(1 + x) - 2x}{(1 +
x)^{2}} = \frac{2}{(1 + x)^{2}}$ si $x \in [ 0 ; 1]$ et 0 sinon est une
densité de $R$. \\
 Enfin $\E(R) = \dint{0}{1} 2 \frac{x}{ (1 + x)^{2}}\ dx = 2
\dint{1}{2} \frac{ x-1}{x^{2}}\ dx = 2 \left[ \ \ln x + \frac{1}{x}
\right]_{1}{2} = 2 \ln 2 -1$. \\

 \item var x, y, z : real ; \\
 begin ; \\
 randomize ; \\
 x : = random; \\
 if x < = 1-x then begin 
 y : = x ; z : = 1-x ; end ;\\
 else begin y : = 1-x ; z : = x ; end; \\
 end. \\
 \end{noliste}

 

 \noindent \textbf{\underline{Exercice sans préparation}}
 \\
\\
 On considère l'endomorphisme $f$ de $\R^{4}$ dont la matrice
 dans la base canonique de $\R^{4}$ est la matrice 
 
\[
 M = \begin{smatrix}
 -1 & -1 & -1 & 1 \\
 -1 & -1 & 1 & -1 \\
 -1 & 1 & -1 & -1 \\
 1 & -1 & -1 & -1
\end{smatrix}
\]

 \begin{noliste}{1.}
 \setlength{\itemsep}{4mm}
 \item $M$ est symétrique donc diagonalisable, et $f$ est donc
diagonalisable. \\

 \item Il reste à calculer les sous-espaces propres associés aux
valeurs propres 2 et $-2$ (amusez-vous bien!) pour obtenir $P$ et $D$,
puis on écrit $M^{n} = P D^{n} P^{-1}$ et on ne va au bout des calculs
(calcul de $P^{-1}$, calcul explicite très lourd de $M^{n}$) que si
c'est demandé par l'examinateur. \\
 \end{noliste}
 \end{exercice}

 \newpage

 \begin{exercice} \indent \\
 \textbf{Question de cours.} Soit $(A_{i})_{i \in I}$ un système
complet d'évènements, c'est-à-dire une famille d'évènements deux à deux
incompatibles, dont la réunion est $\Omega$ l'univers de l'expérience
et dont aucun n'est négligeable (ils ont des probabilités non nulles).
\\
 Alors pour tout évènement $B \in \Omega$, $\Prob\left(\Ev{B}\right) =
\Sum{i \in I} \Prob( B \cap A_{i}) = \Sum{i \in I}
\Prob\left(\Ev{A_{i}}\right) P_{A_{i}} (B)$.
 \\
\\
 La formule de Bayes, utilisée ci-dessus, donne $P_{A}(B) = \frac{
\Prob( A \cap B) }{\Prob\left(\Ev{A}\right)}$ pour tous évènements $A$
et $B$ tels que $\Prob\left(\Ev{A}\right) \neq 0$. \\
\\
 Une coccinelle se déplace sur un tétraèdre régulier PQRS
 (une pyramide) en longeant les arêtes. Elle est placée à
 l'instant $n = 0$ sur le sommet $P$. On suppose que, si. elle se,
trouve sur
 un sommet à l'instant $n$, elle sera sur l'un des trois autres sommets

 à l'instant $n + 1$ de, façon équiprobable.
 \\
 Pour tout $n\in \N$ on note $p_{n}$ (respectivement $q_{n},\ r_{n}$ et
$s_{n}$) la probabilité que la coccinelle se trouve sur le sommet $P$
 (respectivement $Q$, $R$ et $S$) à l'instant $n$.

 \begin{noliste}{1.}
 \setlength{\itemsep}{4mm}
 \item On définit la matrice colonne $X_{n}$ par 


%%%LALALA 
 
\[
 X_{n} = \left( 
\begin{array}{c}
 p_{n} \\
 q_{n} \\
 r_{n} \\
 s_{n}
\end{array}
\right) 
\]
 \\
 $A = \frac{1}{3} \begin{smatrix}
0 & 1 & 1 & 1 \\
1 & 0 & 1 & 1 \\
1 & 1 & 0 & 1 \\
1 & 1 & 1 & 0 \\
\end{smatrix}
$. \\


%%%LALALA 

 \item Elle est symétrique donc diagonalisable. \\

 \item Pour simplifier le calcul des valeurs propres, on cherche un
polynôme annulateur en calculant $A^{2}$ : \\
 On obtient $A^{2} = \frac{1}{3} ( 2A + I)$ donc le polynôme $3 X^{2} -
2 X - 1$, est annulateur, et les seules valeurs propres possibles sont
ses racines $\frac{2-4}{6} = - \frac{1}{3}$ et $\frac{2 + 4}{6} = 1$.
\\
 On résout alors les équations $A X = X$ et $A X = \frac{1}{3} X$, on
en tire les vecteurs propres et les valeurs propres, puis $A = P D
P^{-1}$ et enfin $A^{n} = P D^{n} P^{-1}$ et enfin un calcul explicite
de $A^{n}$ (qui peut attendre éventuellement, on va voir pourquoi). \\
 On obtient ensuite par récurrence que $X_{n} = A^{n} X_{0} = A^{n}
\begin{smatrix}
1 \\
0 \\
0 \\
0 \\
\end{smatrix}
$ qui est la première colonne de $A^{n}$, qu'il faut donc calculer
(attention, si on ne veut calculer que la première colonne de la
matrice intermédiaire, il faut faire le calcul dans le bon ordre). \\

 \item $T (\Omega) = \llb 2 ; + \infty \llb$. \\
 De plus à l'instant 0 on est en P, et à l'instant 1 on est sur un des
trois autres sommets, donc on a visité deux sommets de manière
certaine. \\
 A partir de la, la coccinelle fait des allers retours entre $P$ et le
deuxième sommet, jusqu'au moment ou elle change. \\
 On a donc $\Prob\left(\Ev{\Ev{ T = k}}\right) = 1 \times \left(
\frac{1}{3} \right)^{k-2} \times \frac{2}{3} = \frac{2}{3^{k-1}}$. \\
 On a ensuite $\E(T) = \Sum{k = 2}{+ \infty} \frac{2k}{3^{k-1}} = 2
\left( \Sum{k = 1}{+ \infty} \frac{k}{3^{k-1}} -1 \right) = 2 \left( 1
- \frac{1}{ \left( \frac{2}{3} \right)^{2}} \right) = \frac{10}{4} =
\frac{5}{2}$. \\

 \item On a $U (\Omega) = \llb 3 ; + \infty \llb$ et : \\
 $\Prob\left(\Ev{\Ev{ U = l }}\right) = \Sum{k = 2}{l-1}
\Prob\left(\Ev{\Ev{ T = k}}\right) P_{\Ev{T = k}} \Ev{ U = l} = \Sum{k
= 2}{l-1} \frac{2}{3^{k-1}} \left(\frac{2}{3} \right)^{l-k-1}
\frac{1}{3} = \Sum{k = 2}{l-1} \frac{ 2^{l-k} }{3^{k-1 + l-k-1 + 1} } =
\frac{2^{l}}{3^{l-1} } \Sum{k = 2}{l-1} \frac{ 1 }{2^{k}} =
\frac{2^{l}}{3^{l-1} } \frac{1}{4} \frac{1 - \left( \frac{1}{2}
\right)^{l-2}}{ \frac{1}{2} } = \frac{2^{l-1}}{3^{l-1} } -
\frac{2}{3^{l-1}} = \frac{2^{l-1} - 2}{3^{l-1}}$. \\

 \item $\E(U) = \Sum{l = 3}{+ \infty} l \left( \frac{2}{3}
\right)^{l-1} - 2 \Sum{l = 3}{+ \infty} l \left( \frac{1}{3}
\right)^{l-1} = \frac{1}{ \left( \frac{1}{3} \right)^{2}} - 1 -
\frac{4}{3} - 2 \frac{1}{ \left( \frac{2}{3} \right)^{2}} + 2 +
\frac{4}{3} = 9 - \frac{9}{2} + 1 = \frac{11}{2}$.
 \end{noliste}

%%%LALALA 

 \newpage

 \noindent \textbf{\underline{Exercice sans préparation} }
 \\
\\
 Un agriculteur souhaite améliorer le rendement de son exploitation en
 utilisant de l'engrais. Une étude a montré que le rendement, en
 tonnes par hectare, pour la variété de blé cultivée est donn é par 
\[
 f\left( B,N\right) = 120B-8B^{2} + 4BN-2N^{2}
\]
 $B$ désigne la, quantité de semences de blé utilisée, $N$ la
 quantité d'engrais utilisée.

 \begin{noliste}{1.}
 \setlength{\itemsep}{4mm}
 \item $f$ est de classe $C^{2}$, et on a : \\
 $f_{B}'(B,B) = 120 - 16 B + 4N$, $f_{N}'(B,N) = 4 B - 4 N$,
$f_{B,B}''(B,N) = -16$, $f_{B,N}''(B,N) = 4$ et $f_{N,N}'' (B,N) = -4$.
\\
\\
 Les points vérifient $120 - 16 B + 4 N = 0$ et $4B - 4 N = 0$, donc $B
= N$ et $120-12B = 0$, et enfin $B = N = 10$. \\
 Au point $(10,10)$ on a $r = -16$, $s = 4$ et $t = -4$ donc $rt -
s^{2} = 64 - 16 = 48 > 0$ et $r = -16 < 0$, c'est donc un maximum
local. \\
 De plus on peut écrire $f(B,N) = 120 B - 6 B^{2} -2 ( B^{2} - 2 B N +
N^{2}) = 120 B - 6 B^{2} - 2 ( B- N)^{2} = -6 [ (B-10)^{2} -100]- 2
(B-N)^{2} = -6 (B-10)^{2} - 2 (B-N)^{2} + 600 = -6 (B-10)^{2} - 2
(B-N)^{2} + f (10,10)$ donc $(10,10)$ est un maximum global de $f$. \\

 \item On traduit la contrainte : on a $B = 23 - 2N$ donc le rendement
est donné par $g(N) = f( 23 - 2N, N) = 120 (23 - 2N) - 8 (23 - 2 N)^{2}
+ 4 (23 - 2N) N - 2 N^{2} = N^{2} ( -32 -8 - 2) + N ( - 240 + 736 + 92)
+ (120 - 184) \times 23 = -42 N^{2} + 588 N - 64 \times 23$. \\
 On dérive : $g'(N) = - 84 N + 588$, qui s'annule pour $N =
\frac{588}{84} = \frac{294}{42} = \frac{147}{21} = \frac{21}{3} = 7$ et
$B = 23 - 14 = 9$ donc le rendement optimum vaut $f( 9, 7) = 1080 - 648
+ 252 - 98 = 432 + 154 = 586$.
 \end{noliste}
 \end{exercice}

 \newpage

 \begin{exercice} \indent \\
 \textbf{Question de cours :} Une matrice carrée d'ordre $n$ est
diagonalisable si et seulement si la somme des dimensions de ses
sous-espaces propres vaut $n$.
 \\
\\
 On considère l'endomorphisme de $\R^{4}$ dont la matrice dans la
 base canonique de $\R^{4}$ est la matrice 
 
\[
 M = \begin{smatrix}
 -1 & -1 & -1 & 1 \\
 -1 & -1 & 1 & -1 \\
 -1 & 1 & -1 & -1 \\
 1 & -1 & -1 & -1
\end{smatrix}
\]

%%%LALALA 

 \begin{noliste}{1.}
 \setlength{\itemsep}{4mm}
 \item $M$ est symétrique donc diagonalisable, donc $f$ aussi. \\

 \item $M + 2 I$ est de rang 1 (ses quatre colonnes sont colinéaires)
donc $\ker ( f + 2 \id)$ est de dimension 3. \\
 $-2$ est donc valeur propre et $E_{-2} (f)$ est de dimension 3. \\

 \item $f\left( 1,-1,-1,1\right) = 2 ( 1, -1, -1, 1)$ donc $2$ est
valeur propre de $M$ et de $f$. \\

 \item Pour $-2$, on trouve facilement que $(e_{1} + e_{2}, e_{1} +
e_{3}, e_{1} - e_{4})$ est une base du sous-espace propre; pour $2$ on
prend $(1, 1, -1, 1)$ et on obtient une base de vecteurs propres, donc
une base dans laquelle la matrice de $f$ est diagonale, égale à
$\operatorname{diag} (-2, -2, -2, 2)$ donc $M'^{2} =
\operatorname{diag}( 4, 4, 4,,4 ) = 4 I$. \\

 \item Pour tout $n$, $M^{n} = P M'^{n} P^{-1}$. \\
 Si $n$ est pair, égal à $2k$, cela donne $M^{2k} = P \left(\Ev{ M'^{2}
}\right)^{k} ) P^{-1} = P \left(\Ev{ 4^{k} I }\right) P^{-1} = 4^{k} I
= 2^{2k} I = 2^{n} I$. \\
 Si $n$ est impair égal à $2m + 1$, $M^{2k + 1} = M^{2k} M = 4^{k} I M
= 2^{2k} M = 2^{n-1} M$. \\

 \item En posant $X_{n}$ le vecteur colonnes constitué des quatre
suites, on a $X_{n + 1} = \frac{1}{4} M X_{n}$, donc $X_{n} =
\frac{1}{4^{n}} M^{n} X_{0} = \frac{1}{2^{n}} X_{0}$ si $n$ est pair et
$\frac{1}{2^{n + 1}} M X_{0}$ si $n$ est impair. \\
 On voit que ces suites convergent toutes vers 0. \\
 \end{noliste}
 

 \noindent \textbf{\underline{Exercice sans préparation }}
 \\
 \\
 Soient $A$, $B$, $C$, des événements de même probabilité $p$ et tels
 que $\Prob\left( A\cap B\cap C\right) = 0$

 \begin{noliste}{1.}
 \setlength{\itemsep}{4mm}
 \item On a $\Prob( \overline{ A \cap B \cap C} ) = \Prob( \overline{A}
\cup \overline{B} \cup \overline{C} ) = 1$ et $ \Prob( \overline{A}
\cup \overline{B} \cup \overline{C} ) \leq \Prob\left(\Ev{
\overline{A}}\right) + \Prob\left(\Ev{\overline{B}}\right) +
\Prob\left(\Ev{ \overline{C}}\right) = 3(1-p)$. \\
 On en déduit que $1 \leq 3(1-p)$, $1-p \geq \frac{1}{3}$ et enfin $p
\leq \frac{2}{3}$. \\

 \item Il faut pour cela que les trois évènements $\overline{A},
\overline{B}$ et $\overline{C}$ soient incompatibles et de probabilité
$\frac{1}{3}$, formant un système complet d'évènement. \\
 Par exemple en prenant $X \suit \mathcal{U} ( \{ 0 ; 1 ; 2 \} )$, les
évènements $A = (X \neq 0)$, $B = (X \neq 1)$ et $C = (X \neq 2)$
vérifient $\Prob\left(\Ev{A}\right) = \Prob\left(\Ev{B}\right) =
\Prob\left(\Ev{C}\right) = \frac{2}{3}$ et $P ( A \cap B \cap C) = 0$.
\\

 \item On utilise la formule du crible pour faire apparaître $A \cap
B$, $A \cap C$ et $B \cap C$ : \\
 $\Prob(A \cup B \cup C) = P \left(\Ev{ A}\right) + \Prob\left(\Ev{
B}\right) + \Prob\left(\Ev{C}\right) - \Prob(A \cap B) - \Prob( A \cap
C) - \Prob( B \cap C) + \Prob(A \cap B \cap C)$. \\
 D'où $P ( A \cup B \cup C) = 3 p - 3 p^{2} = 3p (1-p)$. \\
 Or la fonction $f(p) = 3 p (1-p)$ atteint son maximum en $p =
\frac{1}{2}$, où elle vaut $\frac{1}{4}$. \\
 On en déduit que $P ( A \cup B \cup C) \leq \frac{3}{4}$. \\
 Pour réutiliser l'hypothèse, on passe au contraire pour faire
apparaître des intersections : \\
\\
 $\Prob( \overline{ A \cup B \cup C}) = P ( \overline{A} \cap
\overline{B} \cap \overline{C} ) \geq \frac{1}{4}$. \\
\\
 On a de plus $( \overline{A} \cap \overline{B} \cap \overline{C} )
\subset ( \overline{A} \cap \overline{B} )$, donc : \\
 $\frac{1}{4} \leq P ( \overline{A} \cap \overline{B} \cap \overline{C}
) \leq P ( \overline{A} \cap \overline{B} ) = (1-p)^{2}$. \\
\\
 Enfin cela donne $1-p \geq \frac{1}{2}$, et $p \leq \frac{1}{2}$. \\
\item Il faut avoir $\Prob\left(\Ev{ A}\right) = \Prob\left(\Ev{
      B}\right) = \Prob\left(\Ev{ C}\right)$, $\Prob( A \cap B) =
  \Prob( A \cap C) = \Prob( B \cap C) = \frac{1}{4}$, ces trois
  derniers étant
  deux à deux incompatibles (faire un dessin !!!!). \\
  \\
  Il faut séparer $\Omega$ en quatre évènements de probabilité
  $\frac{1}{4}$ : le premier est $A \cap B$, le deuxième $A \cap C$ et
  le
  troisième $B \cap C$ : \\
  Exemple : $X \suit \mathcal{U} ( \{ 1 ; 2 ; 3 ; 4 \} )$. \\
  On pose $A = ( X \in \{ 1 ; 2 \} )$, $B = ( X \in \{ 1 ; 3\})$ et $C
  =
  ( X \in \{ 2 ; 3\} )$. \\
  On a $\Prob\left(\Ev{ A}\right) = P \left(\Ev{B}\right) =
  \Prob\left(\Ev{ C}\right) = \frac{1}{2}$, $\Prob( A \cap B \cap C) =
  0$ et $\Prob( A \cap B) = \Prob( A \cap C) = \Prob( B \cap C) =
  \frac{1}{4}$ donc les évènements $A$, $B$ et $C$ sont deux à deux
  indépendants.
\end{noliste}
\end{exercice}


\end{document}
