\documentclass[11pt]{article}%
\usepackage{geometry}%
\geometry{a4paper,
  lmargin=2cm,rmargin=2cm,tmargin=2.5cm,bmargin=2.5cm}

\input{../../../macros.tex}

% \renewcommand{\thesection}{\Roman{section}.\hspace{-.3cm}}
% \renewcommand{\thesubsection}{\Alph{subsection}.\hspace{-.2cm}}

\pagestyle{fancy} %
\lhead{ECE2 \hfill Mathématiques \\} %
\chead{\hrule} %
\rhead{} %
\lfoot{} %
\cfoot{} %
\rfoot{\thepage} %

\renewcommand{\headrulewidth}{0pt}% : Trace un trait de séparation
                                    % de largeur 0,4 point. Mettre 0pt
                                    % pour supprimer le trait.

\renewcommand{\footrulewidth}{0.4pt}% : Trace un trait de séparation
                                    % de largeur 0,4 point. Mettre 0pt
                                    % pour supprimer le trait.

\setlength{\headheight}{14pt}

\title{\bf \vspace{-1.6cm} HEC 2007} %
\author{} %
\date{} %
\begin{document}
\maketitle %
\vspace{-1.2cm}\hrule %
\thispagestyle{fancy}

\vspace*{.2cm}

%%DEBUT

\begin{exerciceAP}~\\
  Question de cours : Définition d'un estimateur; définitions du biais
  et du risque quadratique d'un estimateur.
  \\[.2cm]
  On considère $n$ ($n>2$) variables aléatoires réelles indépendantes
  $X_{1}$, ... , $X_{n}$ de même loi de densité :
  \[
  f_{0} (x) = \left\{ 
    \begin{array}{cR{2cm}}
      \dfrac{3x^{2}}{\theta ^{3}} & si $x\in \left[ 0,\theta \right]$ \nl
      \nl[-.2cm]
      0 & sinon
    \end{array}
  \right.
  \]
  où $\theta $ est un paramètre strictement positif. On pose :
  \[
  S=X_{1}+\cdots +X_{n}\text{\quad et\quad }T=\max \left( X_{1},\cdots
    ,X_{n}\right)
  \]
  
  \begin{noliste}{1.}
    \setlength{\itemsep}{2mm}
  \item Calculer $\E\left( S\right) $ et $\V\left( S\right) .$

  \item Calculer $\Prob\left( T\leq t\right) $. En déduire une
    densité de $T$ puis $\E\left( T\right) $ et $\V\left( T\right) $.

  \item On suppose maintenant que $\theta $ est un paramètre inconnu
    qu'on se propose d'estimer.

    \begin{noliste}{a)}
    \setlength{\itemsep}{2mm}
    \item Montrer qu'il existe des constantes réelles $a$ et $b$
      telles que $S^{\prime }=aS$ et $T^{\prime }=bT$ soient des
      estimateurs sans biais de $\theta $ \ Calculer $\V\left(
        S^{\prime }\right) $ et $\V\left( T^{\prime }\right) $

    \item Entre les deux estimateurs de $S^{\prime }$ et $T^{\prime }$
      lequel doit-on préférer?
    \end{noliste}
  \end{noliste}
\end{exerciceAP}


\begin{exerciceSP}~\\
  Un agriculteur souhaite améliorer le rendement de son exploitation
  en utilisant de l'engrais. Une étude a montré que le rendement, en
  tonnes par hectare, pour la variété de blé cultivée est donné par :
  \[
  f\left( B,N\right) =120B-8B^{2}+4BN-2N^{2}
  \]
  $B$ désigne la, quantité de semences de blé utilisée, $N$ la
  quantité d'engrais utilisée.
  \begin{noliste}{1.}
    \setlength{\itemsep}{2mm}
  \item Déterminer les extrema de la fonction $f$ ; donner leur
    nature.
  \item Si on suppose de plus qu'une contrainte de budget impose
    $B+2N=23$ déterminer l'optimum de rendement.
  \end{noliste}
\end{exerciceSP}


%\newpage


\begin{exerciceAP}~\\
  Soit $E$ un $\R$-espace vectoriel de dimension finie.\\
  On note pour tout endomorphisme $u$ de $E$ et pour tout $\in
  \N^{\ast },$ $u^{0}= \mathrm{Id}_{E}$ et : \
  $u^{r} = \underset{r\text{ termes}}{\underbrace{u\circ \cdots \circ
      u}}$.\\[.2cm]
  On commence par considérer un endomorphisme non nul de $E$ tel que
  pour tout $x\in E$ il existe $r\left( x\right) \in \N^{\ast }$ tel
  que $u^{r\left( x\right) }(x) = 0$.
  \begin{noliste}{1.}
    \setlength{\itemsep}{2mm}
  \item Donner deux conditions suffisantes de diagonalisabilité d'une
    matrice réelle (en toute généralité).

  \item Montrer qu'il existe $r\in \N^{\ast }$ tel que $u^{r}$ soit
    l'application nulle et que $u^{r-1}$ ne soit pas l'application
    nulle.

  \item Déterminer les valeurs propres de l'endomorphisme $u$; est-il
    diagonalisable ?

  \item On pose : \ $v=\Sum{k=0}{r-1}\frac{u^{k}}{k!}$.\\
    Montrer que $v$ est un isomorphisme de $E$ sur $E$. Exprimer
    l'inverse de $v$ en fonction de $u$

  \item Donner une relation simple entre $\ker \left( u\right) $ et
    $\ker \left( v-\mathrm{Id}\right)$.

  \item Déterminer les valeurs propres de $v$.
  \end{noliste}
\end{exerciceAP}


\begin{exerciceSP}~\\
  Soient $n$ et $N$ des entiers non nuls. \\
  Une urne contient $n$ jetons numérotés de 1 à $n$. On effectue
  $N$ tirages avec remise dans cette urne.
  \begin{noliste}{1.}
    \setlength{\itemsep}{2mm}
  \item Soit $F_{i}$ 1a, variable aléatoire égale au nombre de fois où
    le jeton $i$ a été tiré.\\
    Déterminer la loi de $F_{i}$.\\[.2cm]
    On pose : $F = \Sum{i=1}{n}F_{i}$.\\
    Déterminer la loi de $F$, son espérance et sa variance.\\
    Les variables aléatoires $F_{i}$ sont-elles deux à deux
    indépendantes ?

  \item Pour tout $i\in \llb 1,n \rrb$ on note
    $X_{i}$ la variable aléatoire égale à $0$ si le numéro $i$ n'a
    pas été tiré et égale a 1 s'il été tiré au moins une fois.

    Déterminer la loi de $X_{i}$, son espérance et sa variance.

    Pour tout $\left( i,j\right) \in \llb 0,1\rrb^{2}$, déterminer la
    probabilité : $\Prob_{\Ev{X_{i}=0}} \left( X_{j}=0\right)$.\\
    Les variables $X_{i}$, et $X_{j}$ sont-elles indépendantes ?
  \end{noliste}
\end{exerciceSP}


%\newpage


\begin{exerciceAP}~\\
  On considère un type de composants électroniques, dont la durée de
  vie $X$, exprimée en heures, est une variable aléatoire de densité
  $f$ telle que :
  \[
  f(t) = \left\{
    \begin{array}{cR{2cm}}
      \dfrac{c}{t^{2}} & si $t\geq 10$ \nl
      \nl[-.2cm]
      0 & sinon
    \end{array}
  \right.
  \]

  \begin{noliste}{1.}
    \setlength{\itemsep}{2mm}
  \item Rappeler les qualités d'une densité de probabilité: en déduire
    la valeur du réel $c$.

  \item Déterminer les réels $m$ pour lesquels $\Prob\left( X\leq
      m\right) =\Prob\left( X>m\right) .$

  \item Quelle est la probabilité que. sur un lot de $5$ composants du
    type précédent, $3$ au moins fonctionnent durant au moins $15$
    heures ? \\\
    Deux machines $A$ et $B$ sont équipées de composants du type préc%
    édent. Plus précisément :
    \begin{noliste}{$\stimes$}
    \item $A$ contient deux composants et cesse de fonctionner dès que
      l'un de ces composants est défectueux,
      
    \item $B$ contient également deux composants mais un seul de ces
      composants suffit à la faire fonctionner
    \end{noliste}
    On note $T_{A}$, $T_{B}$ les durées de fonctionnement de ces
    machines.

  \item Déterminer une densité pour chacune des variables $T_{A}$ et $%
    T_{B}$.
    
  \item Pour chacune des variables $T_{A}$ et $T_{B}$ indiquer si elle
    possède une espérance. et le cas échéant, la calculer.
  \end{noliste}



\end{exerciceAP}


\begin{exerciceSP}~
  \begin{noliste}{1.}
    \setlength{\itemsep}{2mm}
  \item Donner s'il en existe un exemple de trois vecteurs de $\R^{2}$
    tels que deux quelconques d'entre eux soient linéairement indépendants.\\
    Existe-t-il un endomorphisme de $\R^{2}$ dont ces trois vecteurs
    soient vecteurs propres ?

  \item Soit $f$ un endomorphisme de $\R^{n}$ et $\mathcal{F}$ une
    famille de $n+1$ vecteurs propres de $f$ s'il en existe.
    \begin{noliste}{a)}
    \setlength{\itemsep}{2mm}
    \item $\mathcal{F}$ peut-elle être une famille libre ?
    \item On suppose que toute sous-famille de $n$ vecteurs de
      $\mathcal{F}$ est libre. \\
      Démontrer que les $n+1$ valeurs propres associées respectivement
      aux $n+1$ vecteurs de $\mathcal{F}$ sont égales.\\
      Que peut-on en conclure pour $f$ ?
    \end{noliste}
  \end{noliste}
\end{exerciceSP}


%\newpage


\begin{exerciceAP}~\\
  \textbf{Question de cours.} Rappeler la formule des probabilités
  totales.\\[.2cm]
  On lance deux pièces truquées : La pièce 1 donne pile avec une
  probabilité $p_{1}$ et la pièce 2 donne pile avec une probabilité%
  , $p_{2}$. \\
  On effectue les lancers de la fa\c{c}on suivante : on choisit une
  pièce uniformément au hasard et on lance la pièce choisie. Si on
  obtient pile, on relance la même pièce et ainsi de suite jusqu'a ce
  que l'on obtienne face; à ce moment on change de pièce - plus
  généralement, dès que l'on obtient face, on change de pièce. On
  suppose que $p_{1}$ et $p_{2}$ sont dans $\left] 0,1\right[ $
  \begin{noliste}{1.}
    \setlength{\itemsep}{2mm}
  \item Quelle est la probabilité de lancer la pièce 1 au $n$-ième
    lancer?
    
  \item Quelle est la probabilité, notée $r_{n}$, d'obtenir pile au
    $\eme{n}$ lancer?
    
  \item Calculer :$L=\dlim{n\rightarrow +\infty }r_{n}$.
    
  \item Dans cette question on suppose $p_{1}=1/3$ et $p_{2}=1/6$.\\
    écrire en langage Pascal l'expression d'une fonction permettant de
    calculer la valeur d'un rang $n_{0}$ à partir duquel :%
    \begin{equation*}
      \left \vert r_{n}-L\right \vert \leq 10^{-6}
    \end{equation*}
  \end{noliste}



\end{exerciceAP}


\begin{exerciceSP}~\\
  Soit $u$ et $v$ deux endomorphismes de $\R^{2}$ dont les matrices
  respectives dans la base canonique $\left( e_{1},e_{2}\right) $ de
  $\R^{2}$ sont notées $A$ et $B$. On suppose que $AB=%
  \begin{smatrix}
    0 & 0 \\ 
    0 & 0%
  \end{smatrix}%
  $ et $BA=%
  \begin{smatrix}
    0 & 1 \\ 
    0 & 0%
  \end{smatrix}%
  $

  \begin{noliste}{1.}
    \setlength{\itemsep}{2mm}
  \item $v$ peut-il être bijectif ? Déterminer $\im \left( v\right) $.

  \item Déterminer $\ker \left( u\right) $

  \item Donner la forme des matrices $A$ et $B$.
  \end{noliste}
\end{exerciceSP}


%\newpage


\begin{exerciceAP}~\\
  Soit $\alpha >0,$ $x_{0}>0$ et $f$ la fonction de $\R$ dans
  $\mathbb{R}$ définie par :%
  \[
  f\left( x\right) =\frac{\alpha }{x_{0}}\left( \frac{x_{0}}{x}\right)
  ^{\alpha +1}\text{ si }x\geq x_{0} \text{\quad et\quad \ } f\left(
    x\right) = 0 \text{ sinon}
  \]

  \begin{noliste}{1.}
    \setlength{\itemsep}{2mm}
  \item 
    \begin{noliste}{a)}
    \setlength{\itemsep}{2mm}
    \item Donner la définition d'une variable aléatoire à densité et
      vérifier que la fonction $f$ est bien la densité de probabilité
      d'une. variable aléatoire réelle $X$.\\
      Déterminer la fonction de répartition de $X$ et donner une allure de
      son graphe.\\
      On dit, que $X$ suit la loi de Pareto de paramètres $x_{0}$ et
      $\alpha $.

    \item Pour quelles valeurs de $\alpha $ 1a variable $X$ admet-elle une
      espérance et une variance ? Calculer l'espérance de $X$ lorsqu'elle
      existe.

    \item On suppose $\alpha >1$ et on pose, pour tout $x>x_{0}$%
      \begin{equation*}
        M_{X}\left( x\right) =\frac{\dint{x}{+\infty }tf\left( t\right) dt}{%
          \dint{x}{+\infty }f\left( t\right) dt}
      \end{equation*}%
      Calculer $M_{X}\left( x\right) $
    \end{noliste}

  \item On se propose d'établir une réciproque de la propriété
    précédente. Soit $x_{0}>0$ et $Y$ une variable aléatoire à
    valeurs dans $\left[ x_{0},+\infty \right[ $ de densité $h$ continue, 
    à valeurs strictement positives, admettant une espérance et telle
    qu'il existe un réel $k>1$ vérifiant : 
    \begin{equation*}
      \forall x>x_{0},\quad M_{Y}\left( x\right) =\frac{\dint{x}{+\infty
        }th\left( t\right) dt}{\dint{x}{+\infty }h\left( t\right) dt}=kx
    \end{equation*}

    \begin{noliste}{a)}
    \setlength{\itemsep}{2mm}
    \item On pose, pour tout $x>x_{0}$%
      \begin{equation*}
        G\left( x\right) =\dint{x}{+\infty }h\left( t\right) dt
      \end{equation*}%
      Montrer que
      \begin{equation*}
        G\left( x\right) =\frac{1-k}{k}xG^{\prime }\left( x\right)
      \end{equation*}

    \item En calculant, pour tout $x>x_{0}$, la dérivée de la fonction $%
      x\rightarrow x^{\frac{k}{k-1}}G\left( x\right) $, déterminer la valeur
      de $G\left( x\right) $, puis la fonction de répartition de $Y$.\\
      Quelle loi retrouve-t-on ?
    \end{noliste}
  \end{noliste}



\end{exerciceAP}


\begin{exerciceSP}~\\
  Soient $X$ et $Y$ deux variables aléatoires binomiales de paramètres
  $\left( n,1/2\right) $ indépendantes.\\
  Calculer $\Prob\left( X=Y\right) .$
\end{exerciceSP}

%\newpage

\begin{exerciceAP}~
  \begin{noliste}{1.}
    \setlength{\itemsep}{2mm}
  \item Rappeler la définition d'une famille génératrice et d'une
    famille libre.\\[.2cm]
    Dans l'espace vectoriel $C^{O}\left( \R\right) $ des fonctions
    continues de $\R$ dans $\R$, on considère les trois
    fonctions :
    \[
    f_{1}:x\rightarrow 1\quad f_{2}:x\rightarrow x\quad \text{et}
    \quad f_{0}:x\rightarrow \left \{
      \begin{array}{cc}
        x\ln \left \vert x\right \vert & \text{si }x\neq 0 \\ 
        0 & \text{si }x=0%
      \end{array}%
    \right. 
    \]
    Soit $E$ le sous-espace vectoriel de $C^{0}\left( \R%
    \right) $ engendré par $\left( f_{1},f_{2},f_{3}\right) $

  \item Prouver que la famille $\left( f_{1},f_{2},f_{3}\right) $ est
    une base de $E$.

  \item À toute fonction $f$ de $E$ on associe la fonction $\Phi
    \left( f\right) $ définie par $\Phi \left( f\right) =\left(
      xf\right) ^{\prime
    },$ dérivée de la fonction $x\rightarrow x~f\left( x\right) $\\
    Montrer que l'on définit ainsi un endomorphisme de $E$ dont on
    donnera la matrice $M$ dans la base $\left(
      f_{1},f_{2},f_{3}\right) $

    \begin{noliste}{a)}
      \setlength{\itemsep}{2mm}
    \item Montrer que $M$ est une matrice inversible et déterminer son
      inverse (éviter le plus possible les calculs).
      
    \item Résoudre dans $E$ l'équation :%
      \[
      \Phi \left( f\right) =a+bx+x\ln \left \vert x\right \vert
      \]
      Déterminer en particulier une primitive de $g:x\rightarrow x\ln
      \left \vert x\right \vert $ sur $\R$.
    \end{noliste}
  \end{noliste}
\end{exerciceAP}


\begin{exerciceSP}~\\
  Soit $X$ et $Y$ deux variables aléatoires binomiales de paramètres $
  \left( n,1/2\right)$.\\
  Calculer $\Prob( \Ev{X = Y} )$.
\end{exerciceSP}

%\newpage

\begin{exerciceAP}~\\
  \textbf{Question de cours :} Définition d'une loi uniforme sur
  l'intervalle $\left[ a,b\right] $ ($a<b$ )\\
  Donner une densité, la fonction de répartition d'une telle loi.
  \\
  Soit $X$ une variable aléatoire suivant une loi uniforme sur $\left[ 0,1%
  \right] $. On pose $Y=\min \left( X,1-X\right) $ et $Z=\left( X,1-X\right) $.

  \begin{noliste}{1.}
    \setlength{\itemsep}{2mm}
  \item Expliciter la fonction de répartition de $Y$. En déduire une
    densité , puis $\E\left( Y\right) $ si elle existe

  \item Expliciter la fonction de répartition de $Z$. En déduire une
    densité, puis $\E\left( Z\right) $ si elle existe. 

  \item Mêmes questions pour $R=Y/Z$

  \item Écrire un programme en Pascal permettant d'obtenir une simulation des
    lois de $X,$ $Y$ et $Z$
  \end{noliste}
\end{exerciceAP}


\begin{exerciceSP}~\\
  On considère l'endomorphisme $f$ de $\R^{4}$ dont la matrice dans la
  base canonique de $\R^{4}$ est la matrice
  \[
  M =
  \begin{smatrix}
    -1 & -1 & -1 & 1 \\
    -1 & -1 & 1 & -1 \\
    -1 & 1 & -1 & -1 \\
    1 & -1 & -1 & -1%
  \end{smatrix}%
  \]

  \begin{noliste}{1.}
    \setlength{\itemsep}{2mm}
  \item Montrer que $f$ est diagonalisable
    
  \item On admet que les valeurs propres de $A$ sont $2$ et $-2.$
    
    Calculer $M^{n}$ pour tout $n$ entier naturel.
  \end{noliste}
\end{exerciceSP}


%\newpage


\begin{exerciceAP}~\\
  \textbf{Question de cours} : énoncer la formule des probabilités
  totales et la formule de Bayes.\\[.2cm]
  Une coccinelle se déplace sur un tétraèdre régulier PQRS (une
  pyramide) en longeant les arêtes. Elle est placée à l'instant $n=0$
  sur le sommet $P$. On suppose que, si. elle se, trouve sur un sommet
  à l'instant $n$, elle sera sur l'un des trois autres sommets à
  l'instant $n+1$ de, fa\c{c}on équiprobable.
  \\
  Pour tout $n\in \N$ on note $p_{n}$ (respectivement $q_{n},\ r_{n}$
  et $s_{n}$) la probabilité que la coccinelle se trouve sur le sommet
  $P$ (respectivement $Q$, $R$ et $S$) à l'instant $n$.

  \begin{noliste}{1.}
    \setlength{\itemsep}{2mm}
  \item On définit la matrice colonne $X_{n}$ par 
    \begin{equation*}
      X_{n}=\left( 
        \begin{array}{c}
          p_{n} \\ 
          q_{n} \\ 
          r_{n} \\ 
          s_{n}%
        \end{array}%
      \right) 
    \end{equation*}%
    \\
    Déterminer une matrice, carrée $A\in \mathcal{M}_{4}\left( \R%
    \right) $, indépendante de $n$, telle que $X_{n+1}=AX_{n}$ pour
    tout $n\in \N^{\ast }$

  \item Montrer que cette matrice est diagonalisable.

  \item Calculer $A^{n}$ pour tout $n\in \N^{\ast }$ ; donner une
    expression de $p_{n},$ $q_{n},$ $r_{n}$ et $s_{n}$ en fonction de $n$.

  \item On note $T$ le temps nécessaire à la coccinelle pour visiter
    au moins 3 sommets distincts du tétraèdre. Déterminer la loi de $%
    T$. Calculer son espérance.

  \item On note $U$ le temps nécessaire à la coccinelle pour visiter
    tous les sommets du tétraèdre. Montrer que pour tout entier $\ell $
    supérieur on égal à 3%
    \begin{equation*}
      \Prob\left( U=\ell \right) =\frac{2^{\ell -1}-2}{3^{\ell -1}}
    \end{equation*}

  \item Déterminer l'espérance de $U$ si elle existe.
  \end{noliste}  
\end{exerciceAP}


\begin{exerciceSP}~\\
  Un agriculteur souhaite améliorer le rendement de son exploitation
  en utilisant de l'engrais. Une étude a montré que le rendement, en
  tonnes par hectare, pour la variété de blé cultivée est donné par :
  \[
  f\left( B,N\right) =120B-8B^{2}+4BN-2N^{2}
  \]
  $B$ désigne la, quantité de semences de blé utilisée, $N$ la
  quantité d'engrais utilisée.
  \begin{noliste}{1.}
    \setlength{\itemsep}{2mm}
  \item Déterminer les extrema de la fonction $f$ ; donner leur nature.
    
  \item Si on suppose de plus qu'une contrainte de budget impose
    $B+2N=23$ déterminer l'optimum de rendement.
  \end{noliste}
\end{exerciceSP}


%\newpage


\begin{exerciceAP}~\\
  \textbf{Question de cours :} Donner une condition nécessaire ci
  suffisante de diagonalisabilité d'une matrice.
  \\[.2cm]
  On considère l'endomorphisme de $\R^{4}$ dont la matrice dans la
  base canonique de $\R^{4}$ est la matrice 
  \begin{equation*}
    M=%
    \begin{smatrix}
      -1 & -1 & -1 & 1 \\ 
      -1 & -1 & 1 & -1 \\ 
      -1 & 1 & -1 & -1 \\ 
      1 & -1 & -1 & -1%
    \end{smatrix}%
  \end{equation*}

  \begin{noliste}{1.}
    \setlength{\itemsep}{2mm}
  \item Justifiez que $f$ est diagonalisable.

  \item Montrer que $-2$ est valeur propre de $_{.}f$ et déterminer la
    dimension du sous espace propre $E_{-2}$ associé

  \item Calculer $f\left( 1,-1,-1,1\right) $. En déduire une autre valeur
    propre ainsi que le sous-espace propre associé. 

  \item Déterminer une  base $\mathcal{B}$ de $\R^{4}$ dans
    laquelle la matrice $M^{\prime }$ de $f$ est diagonale. Soit $P$ la matrice
    de passage de la base canonique de $\R^{4}$ à la base $\mathcal{B%
    }$. Calculer $M^{\prime 2}$.

  \item Calculer $M^{n}$ pour tout entier naturel $n$.

  \item On considère les suites $\left( u_{n}\right) ,\left( v_{n}\right)
    ,\left( w_{n}\right) ,$ $\left( t_{n}\right) $ définies par : $u_{0},\
    v_{0},$ $w_{0}$, $t_{0}$ et 
    \begin{equation*}
      \left \{ 
        \begin{array}{c}
          u_{n+1}=\frac{1}{4}\left( -u_{n}-v_{n}-w_{n}+t_{n}\right)  \\ 
          v_{n+1}=\frac{1}{4}\left( -u_{n}-v_{n}+w_{n}-t_{n}\right)  \\ 
          w_{n+1}=\frac{1}{4}\left( -u_{n}+v_{n}-w_{n}-t_{n}\right)  \\ 
          t_{n+1}=\frac{1}{4}\left( u_{n}-v_{n}-w_{n}-t_{n}\right) 
        \end{array}%
      \right. 
    \end{equation*}%
    Calculer $u_{n},\ v_{n},\ w_{n}$ et  $t_{n}$ en fonction de $u_{0},\ v_{0},$ 
    $w_{0}$, $t_{0}$ et de $n$\\
    Que dire de leurs limites respectives ?
  \end{noliste}

\end{exerciceAP}


\begin{exerciceSP}~\\
  Soient $A$, $B$, $C$, des événements de même probabilité $p$ et tels
  que $\Prob\left( A\cap B\cap C\right) =0$

  \begin{noliste}{1.}
    \setlength{\itemsep}{2mm}
  \item Prouver que $p\leq \dfrac{2}{3}$

  \item $p$ peut-il prendre la valeur $\dfrac{2}{3}\ ?$

  \item On suppose en outre que $A$, $B$ et $C$ sont indépendants deux 
    à deux. Prouver l'inégalité :%
    \begin{equation*}
      p\leq \frac{1}{2}
    \end{equation*}

  \item $p$ peut-il prendre la, valeur $\dfrac{1}{2}$ ?
  \end{noliste}
\end{exerciceSP}

%%FIN

\end{document}
