\documentclass[11pt]{article}%
\usepackage{geometry}%
\geometry{a4paper,
 lmargin = 2cm,rmargin = 2cm,tmargin = 2.5cm,bmargin = 2.5cm}

\usepackage{array}
\usepackage{paralist}

\usepackage[svgnames, usenames, dvipsnames]{xcolor}
\xdefinecolor{RecColor}{named}{Aqua}
\xdefinecolor{IncColor}{named}{Aqua}
\xdefinecolor{ImpColor}{named}{PaleGreen}

% \usepackage{frcursive}

\usepackage{adjustbox}

%%%%%%%%%%%
\newcommand{\cRB}[1]{{\color{Red} \pmb{#1}}} %
\newcommand{\cR}[1]{{\color{Red} {#1}}} %
\newcommand{\cBB}[1]{{\color{Blue} \pmb{#1}}}
\newcommand{\cB}[1]{{\color{Blue} {#1}}}
\newcommand{\cGB}[1]{{\color{LimeGreen} \pmb{#1}}}
\newcommand{\cG}[1]{{\color{LimeGreen} {#1}}}

%%%%%%%%%%

\usepackage{diagbox} %
\usepackage{colortbl} %
\usepackage{multirow} %
\usepackage{pgf} %
\usepackage{environ} %
\usepackage{fancybox} %
\usepackage{textcomp} %
\usepackage{marvosym} %

%%%%%%%%%% pour pouvoir faire des dashedline dans les tableaux
\usepackage{arydshln}

%%%%%%%%%% pour qu'une cellcolor ne recouvre pas le trait du tableau
\usepackage{hhline}%

\usepackage{pgfplots}
\pgfplotsset{compat=1.10}
\usepgfplotslibrary{patchplots}
\usepgfplotslibrary{fillbetween}
\usepackage{tikz,tkz-tab}
\usepackage{ifthen}
\usepackage{calc}
\usetikzlibrary{calc,decorations.pathreplacing,arrows,positioning} 
\usetikzlibrary{fit,shapes,backgrounds}
% \usepackage[nomessages]{fp}% http://ctan.org/pkg/fp

\usetikzlibrary{matrix,arrows,decorations.pathmorphing,
  decorations.pathreplacing} 

\newcommand{\myunit}{1 cm}
\tikzset{
    node style sp/.style={draw,circle,minimum size=\myunit},
    node style ge/.style={circle,minimum size=\myunit},
    arrow style mul/.style={draw,sloped,midway,fill=white},
    arrow style plus/.style={midway,sloped,fill=white},
}

%%%%%%%%%%%%%%
%%%%% écrire des inférieur égal ou supérieur égal avec typographie
%%%%% francaise
%%%%%%%%%%%%%

\renewcommand{\geq}{\geqslant}
\renewcommand{\leq}{\leqslant}
\renewcommand{\emptyset}{\varnothing}

\newcommand{\Leq}{\leqslant}
\newcommand{\Geq}{\geqslant}

%%%%%%%%%%%%%%
%%%%% Macro Celia
%%%%%%%%%%%%%

\newcommand{\ff}[2]{\left[#1, #2\right]} %
\newcommand{\fo}[2]{\left[#1, #2\right[} %
\newcommand{\of}[2]{\left]#1, #2\right]} %
\newcommand{\soo}[2]{\left]#1, #2\right[} %
\newcommand{\abs}[1]{\left|#1\right|} %
\newcommand{\Ent}[1]{\left\lfloor #1 \right\rfloor} %


%%%%%%%%%%%%%%
%%%%% tikz : comment dessiner un "oeil"
%%%%%%%%%%%%%

\newcommand{\eye}[4]% size, x, y, rotation
{ \draw[rotate around={#4:(#2,#3)}] (#2,#3) -- ++(-.5*55:#1) (#2,#3)
  -- ++(.5*55:#1); \draw (#2,#3) ++(#4+55:.75*#1) arc
  (#4+55:#4-55:.75*#1);
  % IRIS
  \draw[fill=gray] (#2,#3) ++(#4+55/3:.75*#1) arc
  (#4+180-55:#4+180+55:.28*#1);
  % PUPIL, a filled arc
  \draw[fill=black] (#2,#3) ++(#4+55/3:.75*#1) arc
  (#4+55/3:#4-55/3:.75*#1);%
}


%%%%%%%%%%
%% discontinuité fonction
\newcommand\pointg[2]{%
  \draw[color = red, very thick] (#1+0.15, #2-.04)--(#1, #2-.04)--(#1,
  #2+.04)--(#1+0.15, #2+.04);%
}%

\newcommand\pointd[2]{%
  \draw[color = red, very thick] (#1-0.15, #2+.04)--(#1, #2+.04)--(#1,
  #2-.04)--(#1-0.15, #2-.04);%
}%

%%%%%%%%%%
%%% 1 : position abscisse, 2 : position ordonnée, 3 : taille, 4 : couleur
%%%%%%%%%%
% \newcommand\pointG[4]{%
%   \draw[color = #4, very thick] (#1+#3, #2-(#3/3.75))--(#1,
%   #2-(#3/3.75))--(#1, #2+(#3/3.75))--(#1+#3, #2+(#3/3.75)) %
% }%

\newcommand\pointG[4]{%
  \draw[color = #4, very thick] ({#1+#3/3.75}, {#2-#3})--(#1,
  {#2-#3})--(#1, {#2+#3})--({#1+#3/3.75}, {#2+#3}) %
}%

\newcommand\pointD[4]{%
  \draw[color = #4, very thick] ({#1-#3/3.75}, {#2+#3})--(#1,
  {#2+#3})--(#1, {#2-#3})--({#1-#3/3.75}, {#2-#3}) %
}%

\newcommand\spointG[4]{%
  \draw[color = #4, very thick] ({#1+#3/1.75}, {#2-#3})--(#1,
  {#2-#3})--(#1, {#2+#3})--({#1+#3/1.75}, {#2+#3}) %
}%

\newcommand\spointD[4]{%
  \draw[color = #4, very thick] ({#1-#3/2}, {#2+#3})--(#1,
  {#2+#3})--(#1, {#2-#3})--({#1-#3/2}, {#2-#3}) %
}%

%%%%%%%%%%

\newcommand{\Pb}{\mathtt{P}}

%%%%%%%%%%%%%%%
%%% Pour citer un précédent item
%%%%%%%%%%%%%%%
\newcommand{\itbf}[1]{{\small \bf \textit{#1}}}


%%%%%%%%%%%%%%%
%%% Quelques couleurs
%%%%%%%%%%%%%%%

\xdefinecolor{cancelcolor}{named}{Red}
\xdefinecolor{intI}{named}{ProcessBlue}
\xdefinecolor{intJ}{named}{ForestGreen}

%%%%%%%%%%%%%%%
%%%%%%%%%%%%%%%
% barrer du texte
\usetikzlibrary{shapes.misc}

\makeatletter
% \definecolor{cancelcolor}{rgb}{0.127,0.372,0.987}
\newcommand{\tikz@bcancel}[1]{%
  \begin{tikzpicture}[baseline=(textbox.base),inner sep=0pt]
  \node[strike out,draw] (textbox) {#1}[thick, color=cancelcolor];
  \useasboundingbox (textbox);
  \end{tikzpicture}%
}
\newcommand{\bcancel}[1]{%
  \relax\ifmmode
    \mathchoice{\tikz@bcancel{$\displaystyle#1$}}
               {\tikz@bcancel{$\textstyle#1$}}
               {\tikz@bcancel{$\scriptstyle#1$}}
               {\tikz@bcancel{$\scriptscriptstyle#1$}}
  \else
    \tikz@bcancel{\strut#1}%
  \fi
}
\newcommand{\tikz@xcancel}[1]{%
  \begin{tikzpicture}[baseline=(textbox.base),inner sep=0pt]
  \node[cross out,draw] (textbox) {#1}[thick, color=cancelcolor];
  \useasboundingbox (textbox);
  \end{tikzpicture}%
}
\newcommand{\xcancel}[1]{%
  \relax\ifmmode
    \mathchoice{\tikz@xcancel{$\displaystyle#1$}}
               {\tikz@xcancel{$\textstyle#1$}}
               {\tikz@xcancel{$\scriptstyle#1$}}
               {\tikz@xcancel{$\scriptscriptstyle#1$}}
  \else
    \tikz@xcancel{\strut#1}%
  \fi
}
\makeatother

\newcommand{\xcancelRA}{\xcancel{\rule[-.15cm]{0cm}{.5cm} \Rightarrow
    \rule[-.15cm]{0cm}{.5cm}}}

%%%%%%%%%%%%%%%%%%%%%%%%%%%%%%%%%%%
%%%%%%%%%%%%%%%%%%%%%%%%%%%%%%%%%%%

\newcommand{\vide}{\multicolumn{1}{c}{}}

%%%%%%%%%%%%%%%%%%%%%%%%%%%%%%%%%%%
%%%%%%%%%%%%%%%%%%%%%%%%%%%%%%%%%%%


\usepackage{multicol}
% \usepackage[latin1]{inputenc}
% \usepackage[T1]{fontenc}
\usepackage[utf8]{inputenc}
\usepackage[T1]{fontenc}
\usepackage[normalem]{ulem}
\usepackage[french]{babel}

\usepackage{url}    
\usepackage{hyperref}
\hypersetup{
  backref=true,
  pagebackref=true,
  hyperindex=true,
  colorlinks=true,
  breaklinks=true,
  urlcolor=blue,
  linkcolor=red,
  bookmarks=true,
  bookmarksopen=true
}

%%%%%%%%%%%%%%%%%%%%%%%%%%%%%%%%%%%%%%%%%%%
%% Pour faire des traits diagonaux dans les tableaux
%% Nécessite slashbox.sty
%\usepackage{slashbox}

\usepackage{tipa}
\usepackage{verbatim,listings}
\usepackage{graphicx}
\usepackage{fancyhdr}
\usepackage{mathrsfs}
\usepackage{pifont}
\usepackage{tablists}
\usepackage{dsfont,amsfonts,amssymb,amsmath,amsthm,stmaryrd,upgreek,manfnt}
\usepackage{enumerate}

%\newcolumntype{M}[1]{p{#1}}
\newcolumntype{C}[1]{>{\centering}m{#1}}
\newcolumntype{R}[1]{>{\raggedright}m{#1}}
\newcolumntype{L}[1]{>{\raggedleft}m{#1}}
\newcolumntype{P}[1]{>{\raggedright}p{#1}}
\newcolumntype{B}[1]{>{\raggedright}b{#1}}
\newcolumntype{Q}[1]{>{\raggedright}t{#1}}

\newcommand{\alias}[2]{
\providecommand{#1}{}
\renewcommand{#1}{#2}
}
\alias{\R}{\mathbb{R}}
\alias{\N}{\mathbb{N}}
\alias{\Z}{\mathbb{Z}}
\alias{\Q}{\mathbb{Q}}
\alias{\C}{\mathbb{C}}
\alias{\K}{\mathbb{K}}

%%%%%%%%%%%%
%% rendre +infty et -infty plus petits
%%%%%%%%%%%%
\newcommand{\sinfty}{{\scriptstyle \infty}}

%%%%%%%%%%%%%%%%%%%%%%%%%%%%%%%
%%%%% macros TP Scilab %%%%%%%%
\newcommand{\Scilab}{\textbf{Scilab}} %
\newcommand{\Scinotes}{\textbf{SciNotes}} %
\newcommand{\faire}{\noindent $\blacktriangleright$ } %
\newcommand{\fitem}{\scalebox{.8}{$\blacktriangleright$}} %
\newcommand{\entree}{{\small\texttt{ENTRÉE}}} %
\newcommand{\tab}{{\small\texttt{TAB}}} %
\newcommand{\mt}[1]{\mathtt{#1}} %
% guillemets droits

\newcommand{\ttq}{\textquotesingle} %

\newcommand{\reponse}[1]{\longboxed{
    \begin{array}{C{0.9\textwidth}}
      \nl[#1]
    \end{array}
  }} %

\newcommand{\reponseL}[2]{\longboxed{
    \begin{array}{C{#2\textwidth}}
      \nl[#1]
    \end{array}
  }} %

\newcommand{\reponseR}[1]{\longboxed{
    \begin{array}{R{0.9\textwidth}}
      #1
    \end{array}
  }} %

\newcommand{\reponseRL}[2]{\longboxed{
    \begin{array}{R{#2\textwidth}}
      #1
    \end{array}
  }} %

\newcommand{\reponseC}[1]{\longboxed{
    \begin{array}{C{0.9\textwidth}}
      #1
    \end{array}
  }} %

\colorlet{pyfunction}{Blue}
\colorlet{pyCle}{Magenta}
\colorlet{pycomment}{LimeGreen}
\colorlet{pydoc}{Cyan}
% \colorlet{SansCo}{white}
% \colorlet{AvecCo}{black}

\newcommand{\visible}[1]{{\color{ASCo}\colorlet{pydoc}{pyDo}\colorlet{pycomment}{pyCo}\colorlet{pyfunction}{pyF}\colorlet{pyCle}{pyC}\colorlet{function}{sciFun}\colorlet{var}{sciVar}\colorlet{if}{sciIf}\colorlet{comment}{sciComment}#1}} %

%%%% à changer ????
\newcommand{\invisible}[1]{{\color{ASCo}\colorlet{pydoc}{pyDo}\colorlet{pycomment}{pyCo}\colorlet{pyfunction}{pyF}\colorlet{pyCle}{pyC}\colorlet{function}{sciFun}\colorlet{var}{sciVar}\colorlet{if}{sciIf}\colorlet{comment}{sciComment}#1}} %

\newcommand{\invisibleCol}[2]{{\color{#1}#2}} %

\NewEnviron{solution} %
{ %
  \Boxed{
    \begin{array}{>{\color{ASCo}} R{0.9\textwidth}}
      \colorlet{pycomment}{pyCo}
      \colorlet{pydoc}{pyDo}
      \colorlet{pyfunction}{pyF}
      \colorlet{pyCle}{pyC}
      \colorlet{function}{sciFun}
      \colorlet{var}{sciVar}
      \colorlet{if}{sciIf}
      \colorlet{comment}{sciComment}
      \BODY
    \end{array}
  } %
} %

\NewEnviron{solutionC} %
{ %
  \Boxed{
    \begin{array}{>{\color{ASCo}} C{0.9\textwidth}}
      \colorlet{pycomment}{pyCo}
      \colorlet{pydoc}{pyDo}
      \colorlet{pyfunction}{pyF}
      \colorlet{pyCle}{pyC}
      \colorlet{function}{sciFun}
      \colorlet{var}{sciVar}
      \colorlet{if}{sciIf}
      \colorlet{comment}{sciComment}
      \BODY
    \end{array}
  } %
} %

\newcommand{\invite}{--\!\!>} %

%%%%% nouvel environnement tabular pour retour console %%%%
\colorlet{ConsoleColor}{Blue!15}
\colorlet{function}{Red}
\colorlet{var}{Maroon}
\colorlet{if}{Magenta}
\colorlet{comment}{LimeGreen}

\newcommand{\tcVar}[1]{\textcolor{var}{\bf \small #1}} %
\newcommand{\tcFun}[1]{\textcolor{function}{#1}} %
\newcommand{\tcIf}[1]{\textcolor{if}{#1}} %
\newcommand{\tcFor}[1]{\textcolor{if}{#1}} %

\newcommand{\moins}{\!\!\!\!\!\!- }
\newcommand{\espn}{\!\!\!\!\!\!}

\usepackage{booktabs,varwidth} \newsavebox\TBox
\newenvironment{console}
{\begin{lrbox}{\TBox}\varwidth{\linewidth}
    \tabular{>{\tt\small}R{0.84\textwidth}}
    \nl[-.4cm]} {\endtabular\endvarwidth\end{lrbox}%
  \fboxsep=1pt\colorbox{ConsoleColor}{\usebox\TBox}}

\newcommand{\lInv}[1]{%
  $\invite$ #1} %

\newcommand{\lAns}[1]{%
  \qquad ans \ = \nl %
  \qquad \qquad #1} %

\newcommand{\lVar}[2]{%
  \qquad #1 \ = \nl %
  \qquad \qquad #2} %

\newcommand{\lDisp}[1]{%
  #1 %
} %

\newcommand{\ligne}[1]{\underline{\small \tt #1}} %

\newcommand{\ligneAns}[2]{%
  $\invite$ #1 \nl %
  \qquad ans \ = \nl %
  \qquad \qquad #2} %

\newcommand{\ligneVar}[3]{%
  $\invite$ #1 \nl %
  \qquad #2 \ = \nl %
  \qquad \qquad #3} %

\newcommand{\ligneErr}[3]{%
  $\invite$ #1 \nl %
  \quad !-{-}error #2 \nl %
  #3} %
%%%%%%%%%%%%%%%%%%%%%% 

\newcommand{\bs}[1]{\boldsymbol{#1}} %
\newcommand{\nll}{\nl[.4cm]} %
\newcommand{\nle}{\nl[.2cm]} %
%% opérateur puissance copiant l'affichage Scilab
%\newcommand{\puis}{\!\!\!~^{\scriptscriptstyle\pmb{\wedge}}}
\newcommand{\puis}{\mbox{$\hspace{-.1cm}~^{\scriptscriptstyle\pmb{\wedge}}
    \hspace{0.05cm}$}} %
\newcommand{\pointpuis}{.\mbox{$\hspace{-.15cm}~^{\scriptscriptstyle\pmb{\wedge}}$}} %
\newcommand{\Sfois}{\mbox{$\mt{\star}$}} %

%%%%% nouvel environnement tabular pour les encadrés Scilab %%%%
\newenvironment{encadre}
{\begin{lrbox}{\TBox}\varwidth{\linewidth}
    \tabular{>{\tt\small}C{0.1\textwidth}>{\small}R{0.7\textwidth}}}
  {\endtabular\endvarwidth\end{lrbox}%
  \fboxsep=1pt\longboxed{\usebox\TBox}}

\newenvironment{encadreL}
{\begin{lrbox}{\TBox}\varwidth{\linewidth}
    \tabular{>{\tt\small}C{0.25\textwidth}>{\small}R{0.6\textwidth}}}
  {\endtabular\endvarwidth\end{lrbox}%
  \fboxsep=1pt\longboxed{\usebox\TBox}}

\newenvironment{encadreF}
{\begin{lrbox}{\TBox}\varwidth{\linewidth}
    \tabular{>{\tt\small}C{0.2\textwidth}>{\small}R{0.70\textwidth}}}
  {\endtabular\endvarwidth\end{lrbox}%
  \fboxsep=1pt\longboxed{\usebox\TBox}}

\newenvironment{encadreLL}[2]
{\begin{lrbox}{\TBox}\varwidth{\linewidth}
    \tabular{>{\tt\small}C{#1\textwidth}>{\small}R{#2\textwidth}}}
  {\endtabular\endvarwidth\end{lrbox}%
  \fboxsep=1pt\longboxed{\usebox\TBox}}

%%%%% nouvel environnement tabular pour les script et fonctions %%%%
\newcommand{\commentaireDL}[1]{\multicolumn{1}{l}{\it
    \textcolor{comment}{$\slash\slash$ #1}}}

\newcommand{\commentaire}[1]{{\textcolor{comment}{$\slash\slash$ #1}}}

\newcounter{cptcol}

\newcommand{\nocount}{\multicolumn{1}{c}{}}

\newcommand{\sciNo}[1]{{\small \underbar #1}}

\NewEnviron{scilab}{ %
  \setcounter{cptcol}{0}
  \begin{center}
    \longboxed{
      \begin{tabular}{>{\stepcounter{cptcol}{\tiny \underbar
              \thecptcol}}c>{\tt}l}
        \BODY
      \end{tabular}
    }
  \end{center}
}

\NewEnviron{scilabNC}{ %
  \begin{center}
    \longboxed{
      \begin{tabular}{>{\tt}l} %
          \BODY
      \end{tabular}
    }
  \end{center}
}

\NewEnviron{scilabC}[1]{ %
  \setcounter{cptcol}{#1}
  \begin{center}
    \longboxed{
      \begin{tabular}{>{\stepcounter{cptcol}{\tiny \underbar
              \thecptcol}}c>{\tt}l}
        \BODY
      \end{tabular}
    }
  \end{center}
}

\newcommand{\scisol}[1]{ %
  \setcounter{cptcol}{0}
  \longboxed{
    \begin{tabular}{>{\stepcounter{cptcol}{\tiny \underbar
            \thecptcol}}c>{\tt}l}
      #1
    \end{tabular}
  }
}

\newcommand{\scisolNC}[1]{ %
  \longboxed{
    \begin{tabular}{>{\tt}l}
      #1
    \end{tabular}
  }
}

\newcommand{\scisolC}[2]{ %
  \setcounter{cptcol}{#1}
  \longboxed{
    \begin{tabular}{>{\stepcounter{cptcol}{\tiny \underbar
            \thecptcol}}c>{\tt}l}
      #2
    \end{tabular}
  }
}

\NewEnviron{syntaxe}{ %
  % \fcolorbox{black}{Yellow!20}{\setlength{\fboxsep}{3mm}
  \shadowbox{
    \setlength{\fboxsep}{3mm}
    \begin{tabular}{>{\tt}l}
      \BODY
    \end{tabular}
  }
}

%%%%% fin macros TP Scilab %%%%%%%%
%%%%%%%%%%%%%%%%%%%%%%%%%%%%%%%%%%%

%%%%%%%%%%%%%%%%%%%%%%%%%%%%%%%%%%%
%%%%% TP Python - listings %%%%%%%%
%%%%%%%%%%%%%%%%%%%%%%%%%%%%%%%%%%%
\newcommand{\Python}{\textbf{Python}} %

\lstset{% general command to set parameter(s)
basicstyle=\ttfamily\small, % print whole listing small
keywordstyle=\color{blue}\bfseries\underbar,
%% underlined bold black keywords
frame=lines,
xleftmargin=10mm,
numbers=left,
numberstyle=\tiny\underbar,
numbersep=10pt,
%identifierstyle=, % nothing happens
commentstyle=\color{green}, % white comments
%%stringstyle=\ttfamily, % typewriter type for strings
showstringspaces=false}

\newcommand{\pysolCpt}[2]{
  \setcounter{cptcol}{#1}
  \longboxed{
    \begin{tabular}{>{\stepcounter{cptcol}{\tiny \underbar
            \thecptcol}}c>{\tt}l}
        #2
      \end{tabular}
    }
} %

\newcommand{\pysol}[1]{
  \setcounter{cptcol}{0}
  \longboxed{
    \begin{tabular}{>{\stepcounter{cptcol}{\tiny \underbar
            \thecptcol}}c>{\tt}l}
        #1
      \end{tabular}
    }
} %

% \usepackage[labelsep=endash]{caption}

% avec un caption
\NewEnviron{pythonCap}[1]{ %
  \renewcommand{\tablename}{Programme}
  \setcounter{cptcol}{0}
  \begin{center}
    \longboxed{
      \begin{tabular}{>{\stepcounter{cptcol}{\tiny \underbar
              \thecptcol}}c>{\tt}l}
        \BODY
      \end{tabular}
    }
    \captionof{table}{#1}
  \end{center}
}

\NewEnviron{python}{ %
  \setcounter{cptcol}{0}
  \begin{center}
    \longboxed{
      \begin{tabular}{>{\stepcounter{cptcol}{\tiny \underbar
              \thecptcol}}c>{\tt}l}
        \BODY
      \end{tabular}
    }
  \end{center}
}

\newcommand{\pyVar}[1]{\textcolor{var}{\bf \small #1}} %
\newcommand{\pyFun}[1]{\textcolor{pyfunction}{#1}} %
\newcommand{\pyCle}[1]{\textcolor{pyCle}{#1}} %
\newcommand{\pyImp}[1]{{\bf #1}} %

%%%%% commentaire python %%%%
\newcommand{\pyComDL}[1]{\multicolumn{1}{l}{\textcolor{pycomment}{\#
      #1}}}

\newcommand{\pyCom}[1]{{\textcolor{pycomment}{\# #1}}}
\newcommand{\pyDoc}[1]{{\textcolor{pydoc}{#1}}}

\newcommand{\pyNo}[1]{{\small \underbar #1}}

%%%%%%%%%%%%%%%%%%%%%%%%%%%%%%%%%%%
%%%%%% Système linéaire paramétré : écrire les opérations au-dessus
%%%%%% d'un symbole équivalent
%%%%%%%%%%%%%%%%%%%%%%%%%%%%%%%%%%%

\usepackage{systeme}

\NewEnviron{arrayEq}{ %
  \stackrel{\scalebox{.6}{$
      \begin{array}{l} 
        \BODY \\[.1cm]
      \end{array}$}
  }{\Longleftrightarrow}
}

\NewEnviron{arrayEg}{ %
  \stackrel{\scalebox{.6}{$
      \begin{array}{l} 
        \BODY \\[.1cm]
      \end{array}$}
  }{=}
}

\NewEnviron{operationEq}{ %
  \scalebox{.6}{$
    \begin{array}{l} 
      \scalebox{1.6}{$\mbox{Opérations :}$} \\[.2cm]
      \BODY \\[.1cm]
    \end{array}$}
}

% \NewEnviron{arraySys}[1]{ %
%   \sysdelim\{.\systeme[#1]{ %
%     \BODY %
%   } %
% }

%%%%%

%%%%%%%%%%
%%%%%%%%%% ESSAI
\newlength\fboxseph
\newlength\fboxsepva
\newlength\fboxsepvb

\setlength\fboxsepva{0.2cm}
\setlength\fboxsepvb{0.2cm}
\setlength\fboxseph{0.2cm}

\makeatletter

\def\longboxed#1{\leavevmode\setbox\@tempboxa\hbox{\color@begingroup%
\kern\fboxseph{\m@th$\displaystyle #1 $}\kern\fboxseph%
\color@endgroup }\my@frameb@x\relax}

\def\my@frameb@x#1{%
  \@tempdima\fboxrule \advance\@tempdima \fboxsepva \advance\@tempdima
  \dp\@tempboxa\hbox {%
    \lower \@tempdima \hbox {%
      \vbox {\hrule\@height\fboxrule \hbox{\vrule\@width\fboxrule #1
          \vbox{%
            \vskip\fboxsepva \box\@tempboxa \vskip\fboxsepvb}#1
          \vrule\@width\fboxrule }%
        \hrule \@height \fboxrule }}}}

\newcommand{\boxedhv}[3]{\setlength\fboxseph{#1cm}
  \setlength\fboxsepva{#2cm}\setlength\fboxsepvb{#2cm}\longboxed{#3}}

\newcommand{\boxedhvv}[4]{\setlength\fboxseph{#1cm}
  \setlength\fboxsepva{#2cm}\setlength\fboxsepvb{#3cm}\longboxed{#4}}

\newcommand{\Boxed}[1]{{\setlength\fboxseph{0.2cm}
  \setlength\fboxsepva{0.2cm}\setlength\fboxsepvb{0.2cm}\longboxed{#1}}}

\newcommand{\mBoxed}[1]{{\setlength\fboxseph{0.2cm}
  \setlength\fboxsepva{0.2cm}\setlength\fboxsepvb{0.2cm}\longboxed{\mbox{#1}}}}

\newcommand{\mboxed}[1]{{\setlength\fboxseph{0.2cm}
  \setlength\fboxsepva{0.2cm}\setlength\fboxsepvb{0.2cm}\boxed{\mbox{#1}}}}

\newsavebox{\fmbox}
\newenvironment{fmpage}[1]
     {\begin{lrbox}{\fmbox}\begin{minipage}{#1}}
     {\end{minipage}\end{lrbox}\fbox{\usebox{\fmbox}}}

%%%%%%%%%%
%%%%%%%%%%

\DeclareMathOperator{\ch}{ch}
\DeclareMathOperator{\sh}{sh}

%%%%%%%%%%
%%%%%%%%%%

\newcommand{\norme}[1]{\Vert #1 \Vert}

%\newcommand*\widefbox[1]{\fbox{\hspace{2em}#1\hspace{2em}}}

\newcommand{\nl}{\tabularnewline}

\newcommand{\hand}{\noindent\ding{43}\ }
\newcommand{\ie}{\textit{i.e. }}
\newcommand{\cf}{\textit{cf }}

\newcommand{\Card}{\operatorname{Card}}

\newcommand{\aire}{\mathcal{A}}

\newcommand{\LL}[1]{\mathscr{L}(#1)} %
\newcommand{\B}{\mathscr{B}} %
\newcommand{\Bc}[1]{B_{#1}} %
\newcommand{\M}[1]{\mathscr{M}_{#1}(\mathbb{R})}

\DeclareMathOperator{\im}{Im}
\DeclareMathOperator{\kr}{Ker}
\DeclareMathOperator{\rg}{rg}
\DeclareMathOperator{\tr}{tr}
\DeclareMathOperator{\spc}{Sp}
\DeclareMathOperator{\sgn}{sgn}
\DeclareMathOperator{\supp}{Supp}

\newcommand{\Mat}{{\rm{Mat}}}
\newcommand{\Vect}[1]{{\rm{Vect}}\left(#1\right)}

\newenvironment{smatrix}{%
  \begin{adjustbox}{width=.9\width}
    $
    \begin{pmatrix}
    }{%      
    \end{pmatrix}
    $
  \end{adjustbox}
}

\newenvironment{sarray}[1]{%
  \begin{adjustbox}{width=.9\width}
    $
    \begin{array}{#1}
    }{%      
    \end{array}
    $
  \end{adjustbox}
}

\newcommand{\vd}[2]{
  \scalebox{.8}{
    $\left(\!
      \begin{array}{c}
        #1 \\
        #2
      \end{array}
    \!\right)$
    }}

\newcommand{\vt}[3]{
  \scalebox{.8}{
    $\left(\!
      \begin{array}{c}
        #1 \\
        #2 \\
        #3 
      \end{array}
    \!\right)$
    }}

\newcommand{\vq}[4]{
  \scalebox{.8}{
    $\left(\!
      \begin{array}{c}
        #1 \\
        #2 \\
        #3 \\
        #4 
      \end{array}
    \!\right)$
    }}

\newcommand{\vc}[5]{
  \scalebox{.8}{
    $\left(\!
      \begin{array}{c}
        #1 \\
        #2 \\
        #3 \\
        #4 \\
        #5 
      \end{array}
    \!\right)$
    }}

\newcommand{\ee}{\text{e}}

\newcommand{\dd}{\text{d}}

%%% Ensemble de définition
\newcommand{\Df}{\mathscr{D}}
\newcommand{\Cf}{\mathscr{C}}
\newcommand{\Ef}{\mathscr{C}}

\newcommand{\rond}[1]{\,\overset{\scriptscriptstyle \circ}{\!#1}}

\newcommand{\df}[2]{\dfrac{\partial #1}{\partial #2}} %
\newcommand{\dfn}[2]{\partial_{#2}(#1)} %
\newcommand{\ddfn}[2]{\partial^2_{#2}(#1)} %
\newcommand{\ddf}[2]{\dfrac{\partial^2 #1}{\partial #2^2}} %
\newcommand{\ddfr}[3]{\dfrac{\partial^2 #1}{\partial #2 \partial
    #3}} %


\newcommand{\dlim}[1]{{\displaystyle \lim_{#1} \ }}
\newcommand{\dlimPlus}[2]{
  \dlim{
    \scalebox{.6}{
      $
      \begin{array}{l}
        #1 \rightarrow #2\\
        #1 > #2
      \end{array}
      $}}}
\newcommand{\dlimMoins}[2]{
  \dlim{
    \scalebox{.6}{
      $
      \begin{array}{l}
        #1 \rightarrow #2\\
        #1 < #2
      \end{array}
      $}}}

%%%%%%%%%%%%%%
%% petit o, développement limité
%%%%%%%%%%%%%%

\newcommand{\oo}[2]{{\underset {{\overset {#1\rightarrow #2}{}}}{o}}} %
\newcommand{\oox}[1]{{\underset {{\overset {x\rightarrow #1}{}}}{o}}} %
\newcommand{\oon}{{\underset {{\overset {n\rightarrow +\infty}{}}}{o}}} %
\newcommand{\po}[1]{{\underset {{\overset {#1}{}}}{o}}} %
\newcommand{\neqx}[1]{{\ \underset {{\overset {x \to #1}{}}}{\not\sim}\ }} %
\newcommand{\eqx}[1]{{\ \underset {{\overset {x \to #1}{}}}{\sim}\ }} %
\newcommand{\eqn}{{\ \underset {{\overset {n \to +\infty}{}}}{\sim}\ }} %
\newcommand{\eq}[2]{{\ \underset {{\overset {#1 \to #2}{}}}{\sim}\ }} %
\newcommand{\DL}[1]{{\rm{DL}}_1 (#1)} %
\newcommand{\DLL}[1]{{\rm{DL}}_2 (#1)} %

\newcommand{\negl}{<<}

\newcommand{\neglP}[1]{\begin{array}{c}
    \vspace{-.2cm}\\
    << \\
    \vspace{-.7cm}\\
    {\scriptstyle #1}
  \end{array}}

%%%%%%%%%%%%%%
%% borne sup, inf, max, min
%%%%%%%%%%%%%%
\newcommand{\dsup}[1]{\displaystyle \sup_{#1} \ }
\newcommand{\dinf}[1]{\displaystyle \inf_{#1} \ }
\newcommand{\dmax}[1]{\max\limits_{#1} \ }
\newcommand{\dmin}[1]{\min\limits_{#1} \ }

\newcommand{\dcup}[2]{{\textstyle\bigcup\limits_{#1}^{#2}}\hspace{.1cm}}
%\displaystyle \bigcup_{#1}^{#2}}
\newcommand{\dcap}[2]{{\textstyle\bigcap\limits_{#1}^{#2}}\hspace{.1cm}}
% \displaystyle \bigcap_{#1}^{#2}
%%%%%%%%%%%%%%
%% opérateurs logiques
%%%%%%%%%%%%%%
\newcommand{\NON}[1]{\mathop{\small \tt{NON}} (#1)}
\newcommand{\ET}{\mathrel{\mathop{\small \mathtt{ET}}}}
\newcommand{\OU}{\mathrel{\mathop{\small \tt{OU}}}}
\newcommand{\XOR}{\mathrel{\mathop{\small \tt{XOR}}}}

\newcommand{\id}{{\rm{id}}}

\newcommand{\sbullet}{\scriptstyle \bullet}
\newcommand{\stimes}{\scriptstyle \times}

%%%%%%%%%%%%%%%%%%
%% Probabilités
%%%%%%%%%%%%%%%%%%
\newcommand{\Prob}{\mathbb{P}}
\newcommand{\Ev}[1]{\left[ {#1} \right]}
\newcommand{\E}{\mathbb{E}}
\newcommand{\V}{\mathbb{V}}
\newcommand{\Cov}{{\rm{Cov}}}
\newcommand{\U}[2]{\mathcal{U}(\llb #1, #2\rrb)}
\newcommand{\Uc}[2]{\mathcal{U}([#1, #2])}
\newcommand{\Ucof}[2]{\mathcal{U}(]#1, #2])}
\newcommand{\Ucoo}[2]{\mathcal{U}(]#1, #2[)}
\newcommand{\Ucfo}[2]{\mathcal{U}([#1, #2[)}
\newcommand{\Bern}[1]{\mathcal{B}\left(#1\right)}
\newcommand{\Bin}[2]{\mathcal{B}\left(#1, #2\right)}
\newcommand{\G}[1]{\mathcal{G}\left(#1\right)}
\newcommand{\Pois}[1]{\mathcal{P}\left(#1\right)}
\newcommand{\HG}[3]{\mathcal{H}\left(#1, #2, #3\right)}
\newcommand{\Exp}[1]{\mathcal{E}\left(#1\right)}
\newcommand{\Norm}[2]{\mathcal{N}\left(#1, #2\right)}

\DeclareMathOperator{\cov}{Cov}

\newcommand{\var}{v.a.r. }
\newcommand{\suit}{\hookrightarrow}

\newcommand{\flecheR}[1]{\rotatebox{90}{\scalebox{#1}{\color{red}
      $\curvearrowleft$}}}


\newcommand{\partie}[1]{\mathcal{P}(#1)}
\newcommand{\Cont}[1]{\mathcal{C}^{#1}}
\newcommand{\Contm}[1]{\mathcal{C}^{#1}_m}

\newcommand{\llb}{\llbracket}
\newcommand{\rrb}{\rrbracket}

%\newcommand{\im}[1]{{\rm{Im}}(#1)}
\newcommand{\imrec}[1]{#1^{- \mathds{1}}}

\newcommand{\unq}{\mathds{1}}

\newcommand{\Hyp}{\mathtt{H}}

\newcommand{\eme}[1]{#1^{\scriptsize \mbox{ème}}}
\newcommand{\er}[1]{#1^{\scriptsize \mbox{er}}}
\newcommand{\ere}[1]{#1^{\scriptsize \mbox{ère}}}
\newcommand{\nd}[1]{#1^{\scriptsize \mbox{nd}}}
\newcommand{\nde}[1]{#1^{\scriptsize \mbox{nde}}}

\newcommand{\truc}{\mathop{\top}}
\newcommand{\fois}{\mathop{\ast}}

\newcommand{\f}[1]{\overrightarrow{#1}}

\newcommand{\checked}{\textcolor{green}{\checkmark}}

\def\restriction#1#2{\mathchoice
              {\setbox1\hbox{${\displaystyle #1}_{\scriptstyle #2}$}
              \restrictionaux{#1}{#2}}
              {\setbox1\hbox{${\textstyle #1}_{\scriptstyle #2}$}
              \restrictionaux{#1}{#2}}
              {\setbox1\hbox{${\scriptstyle #1}_{\scriptscriptstyle #2}$}
              \restrictionaux{#1}{#2}}
              {\setbox1\hbox{${\scriptscriptstyle #1}_{\scriptscriptstyle #2}$}
              \restrictionaux{#1}{#2}}}
\def\restrictionaux#1#2{{#1\,\smash{\vrule height .8\ht1 depth .85\dp1}}_{\,#2}}

\makeatletter
\newcommand*{\da@rightarrow}{\mathchar"0\hexnumber@\symAMSa 4B }
\newcommand*{\da@leftarrow}{\mathchar"0\hexnumber@\symAMSa 4C }
\newcommand*{\xdashrightarrow}[2][]{%
  \mathrel{%
    \mathpalette{\da@xarrow{#1}{#2}{}\da@rightarrow{\,}{}}{}%
  }%
}
\newcommand{\xdashleftarrow}[2][]{%
  \mathrel{%
    \mathpalette{\da@xarrow{#1}{#2}\da@leftarrow{}{}{\,}}{}%
  }%
}
\newcommand*{\da@xarrow}[7]{%
  % #1: below
  % #2: above
  % #3: arrow left
  % #4: arrow right
  % #5: space left 
  % #6: space right
  % #7: math style 
  \sbox0{$\ifx#7\scriptstyle\scriptscriptstyle\else\scriptstyle\fi#5#1#6\m@th$}%
  \sbox2{$\ifx#7\scriptstyle\scriptscriptstyle\else\scriptstyle\fi#5#2#6\m@th$}%
  \sbox4{$#7\dabar@\m@th$}%
  \dimen@=\wd0 %
  \ifdim\wd2 >\dimen@
    \dimen@=\wd2 %   
  \fi
  \count@=2 %
  \def\da@bars{\dabar@\dabar@}%
  \@whiledim\count@\wd4<\dimen@\do{%
    \advance\count@\@ne
    \expandafter\def\expandafter\da@bars\expandafter{%
      \da@bars
      \dabar@ 
    }%
  }%  
  \mathrel{#3}%
  \mathrel{%   
    \mathop{\da@bars}\limits
    \ifx\\#1\\%
    \else
      _{\copy0}%
    \fi
    \ifx\\#2\\%
    \else
      ^{\copy2}%
    \fi
  }%   
  \mathrel{#4}%
}
\makeatother



\newcount\depth

\newcount\depth
\newcount\totaldepth

\makeatletter
\newcommand{\labelsymbol}{%
      \ifnum\depth=0
        %
      \else
        \rlap{\,$\bullet$}%
      \fi
}

\newcommand*\bernoulliTree[1]{%
    \depth=#1\relax            
    \totaldepth=#1\relax
    \draw node(root)[bernoulli/root] {\labelsymbol}[grow=right] \draw@bernoulli@tree;
    \draw \label@bernoulli@tree{root};                                   
}                                                                        

\def\draw@bernoulli@tree{%
    \ifnum\depth>0 
      child[parent anchor=east] foreach \type/\label in {left child/$E$,right child/$S$} {%
          node[bernoulli/\type] {\label\strut\labelsymbol} \draw@bernoulli@tree
      }
      coordinate[bernoulli/increment] (dummy)
   \fi%
}

\def\label@bernoulli@tree#1{%
    \ifnum\depth>0
      ($(#1)!0.5!(#1-1)$) node[fill=white,bernoulli/decrement] {\tiny$p$}
      \label@bernoulli@tree{#1-1}
      ($(#1)!0.5!(#1-2)$) node[fill=white] {\tiny$q$}
      \label@bernoulli@tree{#1-2}
      coordinate[bernoulli/increment] (dummy)
   \fi%
}

\makeatother

\tikzset{bernoulli/.cd,
         root/.style={},
         decrement/.code=\global\advance\depth by-1\relax,
         increment/.code=\global\advance\depth by 1\relax,
         left child/.style={bernoulli/decrement},
         right child/.style={}}


\newcommand{\eps}{\varepsilon}

% \newcommand{\tendi}[1]{\xrightarrow[\footnotesize #1 \rightarrow
%   +\infty]{}}%

\newcommand{\tend}{\rightarrow}%
\newcommand{\tendn}{\underset{n\to +\infty}{\longrightarrow}} %
\newcommand{\ntendn}{\underset{n\to
    +\infty}{\not\hspace{-.15cm}\longrightarrow}} %
% \newcommand{\tendn}{\xrightarrow[\footnotesize n \rightarrow
%   +\infty]{}}%
\newcommand{\Tendx}[1]{\xrightarrow[\footnotesize x \rightarrow
  #1]{}}%
\newcommand{\tendx}[1]{\underset{x\to #1}{\longrightarrow}}%
\newcommand{\ntendx}[1]{\underset{x\to #1}{\not\!\!\longrightarrow}}%
\newcommand{\tendd}[2]{\underset{#1\to #2}{\longrightarrow}}%
% \newcommand{\tendd}[2]{\xrightarrow[\footnotesize #1 \rightarrow
%   #2]{}}%
\newcommand{\tendash}[1]{\xdashrightarrow[\footnotesize #1 \rightarrow
  +\infty]{}}%
\newcommand{\tendashx}[1]{\xdashrightarrow[\footnotesize x \rightarrow
  #1]{}}%
\newcommand{\tendb}[1]{\underset{#1 \to +\infty}{\longrightarrow}}%
\newcommand{\tendL}{\overset{\mathscr L}{\underset{n \to
      +\infty}{\longrightarrow}}}%
\newcommand{\tendP}{\overset{\Prob}{\underset{n \to
      +\infty}{\longrightarrow}}}%
\newcommand{\tenddL}[1]{\overset{\mathscr L}{\underset{#1 \to
      +\infty}{\longrightarrow}}}%

\NewEnviron{attention}{ %
  ~\\[-.2cm]\noindent
  \begin{minipage}{\linewidth}
  \setlength{\fboxsep}{3mm}%
  \ \ \dbend \ \ %
  \fbox{\parbox[t]{.88\linewidth}{\BODY}} %
  \end{minipage}\\
}

\NewEnviron{sattention}[1]{ %
  ~\\[-.2cm]\noindent
  \begin{minipage}{#1\linewidth}
  \setlength{\fboxsep}{3mm}%
  \ \ \dbend \ \ %
  \fbox{\parbox[t]{.88\linewidth}{\BODY}} %
  \end{minipage}\\
}

%%%%% OBSOLETE %%%%%%

% \newcommand{\attention}[1]{
%   \noindent
%   \begin{tabular}{@{}l|p{11.5cm}|}
%     \cline{2-2}
%     \vspace{-.2cm} 
%     & \nl
%     \dbend & #1 \nl
%     \cline{2-2}
%   \end{tabular}
% }

% \newcommand{\attentionv}[2]{
%   \noindent
%   \begin{tabular}{@{}l|p{11.5cm}|}
%     \cline{2-2}
%     \vspace{-.2cm} 
%     & \nl
%     \dbend & #2 \nl[#1 cm]
%     \cline{2-2}
%   \end{tabular}
% }

\newcommand{\explainvb}[2]{
  \noindent
  \begin{tabular}{@{}l|p{11.5cm}|}
    \cline{2-2}
    \vspace{-.2cm} 
    & \nl
    \hand & #2 \nl [#1 cm]
    \cline{2-2}
  \end{tabular}
}


% \noindent
% \begin{tabular}{@{}l|lp{11cm}|}
%   \cline{3-3} 
%   \multicolumn{1}{@{}l@{\dbend}}{} & & #1 \nl
%   \multicolumn{1}{l}{} & & \nl [-.8cm]
%   & & #2 \nl
%   \cline{2-3}
% \end{tabular}

% \newcommand{\attention}[1]{
%   \noindent
%   \begin{tabular}{@{}@{}cp{11cm}}
%     \dbend & #1 \nl
%   \end{tabular}
% }

\newcommand{\PP}[1]{\mathcal{P}(#1)}
\newcommand{\FF}[1]{\mathcal{F}(#1)}

\newcommand{\DSum}[2]{\displaystyle\sum\limits_{#1}^{#2}\hspace{.1cm}}
\newcommand{\Sum}[2]{{\textstyle\sum\limits_{#1}^{#2}}\hspace{.1cm}}
\newcommand{\Serie}{\textstyle\sum\ }
\newcommand{\Prod}[2]{\textstyle\prod\limits_{#1}^{#2}}

\newcommand{\Prim}[3]{\left[\ {#1} \ \right]_{\scriptscriptstyle
   \hspace{-.15cm} ~_{#2}\, }^ {\scriptscriptstyle \hspace{-.15cm} ~^{#3}\, }}

% \newcommand{\Prim}[3]{\left[\ {#1} \ \right]_{\scriptscriptstyle
%     \!\!~_{#2}}^ {\scriptscriptstyle \!\!~^{#3}}}

\newcommand{\dint}[2]{\displaystyle \int_{#1}^{#2}\ }
\newcommand{\Int}[2]{{\rm{Int}}_{\scriptscriptstyle #1, #2}}
\newcommand{\dt}{\ dt}
\newcommand{\dx}{\ dx}

\newcommand{\llpar}[1]{\left(\!\!\!
    \begin{array}{c}
      \rule{0pt}{#1}
    \end{array}
  \!\!\!\right.}

\newcommand{\rrpar}[1]{\left.\!\!\!
    \begin{array}{c}
      \rule{0pt}{#1}
    \end{array}
  \!\!\!\right)}

\newcommand{\llacc}[1]{\left\{\!\!\!
    \begin{array}{c}
      \rule{0pt}{#1}
    \end{array}
  \!\!\!\right.}

\newcommand{\rracc}[1]{\left.\!\!\!
    \begin{array}{c}
      \rule{0pt}{#1}
    \end{array}
  \!\!\!\right\}}

\newcommand{\ttacc}[1]{\mbox{\rotatebox{-90}{\hspace{-.7cm}$\llacc{#1}$}}}
\newcommand{\bbacc}[1]{\mbox{\rotatebox{90}{\hspace{-.5cm}$\llacc{#1}$}}}

\newcommand{\comp}[1]{\overline{#1}}%

\newcommand{\dcomp}[2]{\stackrel{\mbox{\ \ \----}{\scriptscriptstyle
      #2}}{#1}}%

% \newcommand{\Comp}[2]{\stackrel{\mbox{\ \
%       \-------}{\scriptscriptstyle #2}}{#1}}

% \newcommand{\dcomp}[2]{\stackrel{\mbox{\ \
%       \-------}{\scriptscriptstyle #2}}{#1}}

\newcommand{\A}{\mathscr{A}}

\newcommand{\conc}[1]{
  \begin{center}
    \fbox{
      \begin{tabular}{c}
        #1
      \end{tabular}
    }
  \end{center}
}

\newcommand{\concC}[1]{
  \begin{center}
    \fbox{
    \begin{tabular}{C{10cm}}
      \quad #1 \quad
    \end{tabular}
    }
  \end{center}
}

\newcommand{\concL}[2]{
  \begin{center}
    \fbox{
    \begin{tabular}{C{#2cm}}
      \quad #1 \quad
    \end{tabular}
    }
  \end{center}
}

% \newcommand{\lims}[2]{\prod\limits_{#1}^{#2}}

\newtheorem{theorem}{Théorème}[]
\newtheorem{lemma}{Lemme}[]
\newtheorem{proposition}{Proposition}[]
\newtheorem{corollary}{Corollaire}[]

% \newenvironment{proof}[1][Démonstration]{\begin{trivlist}
% \item[\hskip \labelsep {\bfseries #1}]}{\end{trivlist}}
\newenvironment{definition}[1][Définition]{\begin{trivlist}
\item[\hskip \labelsep {\bfseries #1}]}{\end{trivlist}}
\newenvironment{example}[1][Exemple]{\begin{trivlist}
\item[\hskip \labelsep {\bfseries #1}]}{\end{trivlist}}
\newenvironment{examples}[1][Exemples]{\begin{trivlist}
\item[\hskip \labelsep {\bfseries #1}]}{\end{trivlist}}
\newenvironment{notation}[1][Notation]{\begin{trivlist}
\item[\hskip \labelsep {\bfseries #1}]}{\end{trivlist}}
\newenvironment{propriete}[1][Propriété]{\begin{trivlist}
\item[\hskip \labelsep {\bfseries #1}]}{\end{trivlist}}
\newenvironment{proprietes}[1][Propriétés]{\begin{trivlist}
\item[\hskip \labelsep {\bfseries #1}]}{\end{trivlist}}
\newenvironment{remarkSC}[1][Remarque]{\begin{trivlist}
\item[\hskip \labelsep {\bfseries #1}]}{\end{trivlist}}
\newenvironment{application}[1][Application]{\begin{trivlist}
\item[\hskip \labelsep {\bfseries #1}]}{\end{trivlist}}

% Environnement pour les réponses des DS
\newenvironment{answer}{\par\emph{Réponse :}\par{}}
{\vspace{-.6cm}\hspace{\stretch{1}}\rule{1ex}{1ex}\vspace{.3cm}}

\newenvironment{answerTD}{\vspace{.2cm}\par\emph{Réponse :}\par{}}
{\hspace{\stretch{1}}\rule{1ex}{1ex}\vspace{.3cm}}

\newenvironment{answerCours}{\noindent\emph{Réponse :}}
{\rule{1ex}{1ex}}%\vspace{.3cm}}


% footnote in footer
\newcommand{\fancyfootnotetext}[2]{%
  \fancypagestyle{dingens}{%
    \fancyfoot[LO,RE]{\parbox{0.95\textwidth}{\footnotemark[#1]\footnotesize
        #2}}%
  }%
  \thispagestyle{dingens}%
}

%%% définit le style (arabic : 1,2,3...) et place des parenthèses
%%% autour de la numérotation
\renewcommand*{\thefootnote}{(\arabic{footnote})}
% http://www.tuteurs.ens.fr/logiciels/latex/footnote.html

%%%%%%%% tikz axis
% \pgfplotsset{every axis/.append style={
%                     axis x line=middle,    % put the x axis in the middle
%                     axis y line=middle,    % put the y axis in the middle
%                     axis line style={<->,color=blue}, % arrows on the axis
%                     xlabel={$x$},          % default put x on x-axis
%                     ylabel={$y$},          % default put y on y-axis
%             }}

%%%% s'utilise comme suit

% \begin{axis}[
%   xmin=-8,xmax=4,
%   ymin=-8,ymax=4,
%   grid=both,
%   ]
%   \addplot [domain=-3:3,samples=50]({x^3-3*x},{3*x^2-9}); 
% \end{axis}

%%%%%%%%



%%%%%%%%%%%% Pour avoir des numéros de section qui correspondent à
%%%%%%%%%%%% ceux du tableau
\renewcommand{\thesection}{\Roman{section}.\hspace{-.3cm}}
\renewcommand{\thesubsection}{\Roman{section}.\arabic{subsection}.\hspace{-.2cm}}
\renewcommand{\thesubsubsection}{\Roman{section}.\arabic{subsection}.\alph{subsubsection})\hspace{-.2cm}}
%%%%%%%%%%%% 

%%% Changer le nom des figures : Fig. au lieu de Figure
\usepackage[font=small,labelfont=bf,labelsep=space]{caption}
\captionsetup{%
  figurename=Fig.,
  tablename=Tab.
}
% \renewcommand{\thesection}{\Roman{section}.\hspace{-.2cm}}
% \renewcommand{\thesubsection}{\Roman{section}
%   .\hspace{.2cm}\arabic{subsection}\ .\hspace{-.3cm}}
% \renewcommand{\thesubsubsection}{\alph{subsection})}

\newenvironment{tabliste}[1]
{\begin{tabenum}[\bfseries\small\itshape #1]}{\end{tabenum}} 

\newenvironment{noliste}[1] %
{\begin{enumerate}[\bfseries\small\itshape #1]} %
  {\end{enumerate}}

\newenvironment{nonoliste}[1] %
{\begin{enumerate}[\hspace{-12pt}\bfseries\small\itshape #1]} %
  {\end{enumerate}}

\newenvironment{arrayliste}[1]{ 
  % List with minimal white space to fit in small areas, e.g. table
  % cell
  \begin{minipage}[t]{\linewidth} %
    \begin{enumerate}[\bfseries\small\itshape #1] %
      {\leftmargin=0.5em \rightmargin=0em
        \topsep=0em \parskip=0em \parsep=0em
        \listparindent=0em \partopsep=0em \itemsep=0pt
        \itemindent=0em \labelwidth=\leftmargin\labelsep+0.25em}
    }{
    \end{enumerate}\end{minipage}
}

\newenvironment{nolistes}[2]
{\begin{enumerate}[\bfseries\small\itshape
    #1]\setlength{\itemsep}{#2 mm}}{\end{enumerate}}

\newenvironment{liste}[1]
{\begin{enumerate}[\hspace{12pt}\bfseries\small\itshape #1]}{\end{enumerate}} 



%%%%%%%% Pour les programmes de colle %%%%%%%

\newcommand{\cours}{{\small \tt (COURS)}} %
\newcommand{\poly}{{\small \tt (POLY)}} %
\newcommand{\exo}{{\small \tt (EXO)}} %
\newcommand{\culture}{{\small \tt (CULTURE)}} %
\newcommand{\methodo}{{\small \tt (MÉTHODO)}} %
\newcommand{\methodob}{\Boxed{\mbox{\tt MÉTHODO}}} %

%%%%%%%% Pour les TD %%%%%%%
\newtheoremstyle{exostyle} {\topsep} % espace avant
{.6cm} % espace apres
{} % Police utilisee par le style de thm
{} % Indentation (vide = aucune, \parindent = indentation paragraphe)
{\bfseries} % Police du titre de thm
{} % Signe de ponctuation apres le titre du thm
{ } % Espace apres le titre du thm (\newline = linebreak)
{\thmname{#1}\thmnumber{ #2}\thmnote{.
    \normalfont{\textit{#3}}}} % composants du titre du thm : \thmname
                               % = nom du thm, \thmnumber = numéro du
                               % thm, \thmnote = sous-titre du thm
 
\theoremstyle{exostyle}
\newtheorem{exercice}{Exercice}
\newtheorem{exerciceAP}{Exercice avec préparation}
\newtheorem{exerciceSP}{Exercice sans préparation}
\newtheorem*{exoCours}{Exercice}

%%%%%%%% Pour des théorèmes Sans Espaces APRÈS %%%%%%%
\newtheoremstyle{exostyleSE} {\topsep} % espace avant
{} % espace apres
{} % Police utilisee par le style de thm
{} % Indentation (vide = aucune, \parindent = indentation paragraphe)
{\bfseries} % Police du titre de thm
{} % Signe de ponctuation apres le titre du thm
{ } % Espace apres le titre du thm (\newline = linebreak)
{\thmname{#1}\thmnumber{ #2}\thmnote{.
    \normalfont{\textit{#3}}}} % composants du titre du thm : \thmname
                               % = nom du thm, \thmnumber = numéro du
                               % thm, \thmnote = sous-titre du thm
 
\theoremstyle{exostyleSE}
\newtheorem{exerciceSE}{Exercice}
\newtheorem*{exoCoursSE}{Exercice}

% \newcommand{\lims}[2]{\prod\limits_{#1}^{#2}}

\newtheorem{theoremSE}{Théorème}[]
\newtheorem{lemmaSE}{Lemme}[]
\newtheorem{propositionSE}{Proposition}[]
\newtheorem{corollarySE}{Corollaire}[]

% \newenvironment{proofSE}[1][Démonstration]{\begin{trivlist}
% \item[\hskip \labelsep {\bfseries #1}]}{\end{trivlist}}
\newenvironment{definitionSE}[1][Définition]{\begin{trivlist}
  \item[\hskip \labelsep {\bfseries #1}]}{\end{trivlist}}
\newenvironment{exampleSE}[1][Exemple]{\begin{trivlist} 
  \item[\hskip \labelsep {\bfseries #1}]}{\end{trivlist}}
\newenvironment{examplesSE}[1][Exemples]{\begin{trivlist}
\item[\hskip \labelsep {\bfseries #1}]}{\end{trivlist}}
\newenvironment{notationSE}[1][Notation]{\begin{trivlist}
\item[\hskip \labelsep {\bfseries #1}]}{\end{trivlist}}
\newenvironment{proprieteSE}[1][Propriété]{\begin{trivlist}
\item[\hskip \labelsep {\bfseries #1}]}{\end{trivlist}}
\newenvironment{proprietesSE}[1][Propriétés]{\begin{trivlist}
\item[\hskip \labelsep {\bfseries #1}]}{\end{trivlist}}
\newenvironment{remarkSE}[1][Remarque]{\begin{trivlist}
\item[\hskip \labelsep {\bfseries #1}]}{\end{trivlist}}
\newenvironment{applicationSE}[1][Application]{\begin{trivlist}
\item[\hskip \labelsep {\bfseries #1}]}{\end{trivlist}}

%%%%%%%%%%% Obtenir les étoiles sans charger le package MnSymbol
%%%%%%%%%%%
\DeclareFontFamily{U} {MnSymbolC}{}
\DeclareFontShape{U}{MnSymbolC}{m}{n}{
  <-6> MnSymbolC5
  <6-7> MnSymbolC6
  <7-8> MnSymbolC7
  <8-9> MnSymbolC8
  <9-10> MnSymbolC9
  <10-12> MnSymbolC10
  <12-> MnSymbolC12}{}
\DeclareFontShape{U}{MnSymbolC}{b}{n}{
  <-6> MnSymbolC-Bold5
  <6-7> MnSymbolC-Bold6
  <7-8> MnSymbolC-Bold7
  <8-9> MnSymbolC-Bold8
  <9-10> MnSymbolC-Bold9
  <10-12> MnSymbolC-Bold10
  <12-> MnSymbolC-Bold12}{}

\DeclareSymbolFont{MnSyC} {U} {MnSymbolC}{m}{n}

\DeclareMathSymbol{\filledlargestar}{\mathrel}{MnSyC}{205}
\DeclareMathSymbol{\largestar}{\mathrel}{MnSyC}{131}

\newcommand{\facile}{\rm{(}$\scriptstyle\largestar$\rm{)}} %
\newcommand{\moyen}{\rm{(}$\scriptstyle\filledlargestar$\rm{)}} %
\newcommand{\dur}{\rm{(}$\scriptstyle\filledlargestar\filledlargestar$\rm{)}} %
\newcommand{\costaud}{\rm{(}$\scriptstyle\filledlargestar\filledlargestar\filledlargestar$\rm{)}}

%%%%%%%%%%%%%%%%%%%%%%%%%

%%%%%%%%%%%%%%%%%%%%%%%%%
%%%%%%%% Fin de la partie TD

%%%%%%%%%%%%%%%%
%%%%%%%%%%%%%%%%
\makeatletter %
\newenvironment{myitemize}{%
  \setlength{\topsep}{0pt} %
  \setlength{\partopsep}{0pt} %
  \renewcommand*{\@listi}{\leftmargin\leftmargini \parsep\z@
    \topsep\z@ \itemsep\z@} \let\@listI\@listi %
  \itemize %
}{\enditemize} %
\makeatother
%%%%%%%%%%%%%%%%
%%%%%%%%%%%%%%%%

%% Commentaires dans la correction du livre

\newcommand{\Com}[1]{
% Define box and box title style
\tikzstyle{mybox} = [draw=black!50,
very thick,
    rectangle, rounded corners, inner sep=10pt, inner ysep=8pt]
\tikzstyle{fancytitle} =[rounded corners, fill=black!80, text=white]
\tikzstyle{fancylogo} =[ text=white]
\begin{center}

\begin{tikzpicture}
\node [mybox] (box){%

    \begin{minipage}{0.90\linewidth}
\vspace{6pt}  #1
    \end{minipage}
};
\node[fancytitle, right=10pt] at (box.north west) 
{\bfseries\normalsize{Commentaire}};

\end{tikzpicture}%

\end{center}
%
}

% \NewEnviron{remark}{%
%   % Define box and box title style
%   \tikzstyle{mybox} = [draw=black!50, very thick, rectangle, rounded
%   corners, inner sep=10pt, inner ysep=8pt] %
%   \tikzstyle{fancytitle} =[rounded corners, fill=black!80,
%   text=white] %
%   \tikzstyle{fancylogo} =[ text=white]
%   \begin{center}
%     \begin{tikzpicture}
%       \node [mybox] (box){%
%         \begin{minipage}{0.90\linewidth}
%           \vspace{6pt} \BODY
%         \end{minipage}
%       }; %
%       \node[fancytitle, right=10pt] at (box.north west) %
%       {\bfseries\normalsize{Commentaire}}; %
%     \end{tikzpicture}%
%   \end{center}
% }
% \makeatother


%%%
% Remarques comme dans le livre
%%%

\NewEnviron{remark}{%
  % Define box and box title style
  \tikzstyle{mybox} = [draw=black!50, very thick, rectangle, rounded
  corners, inner sep=10pt, inner ysep=8pt] %
  \tikzstyle{fancytitle} = [rounded corners , fill=black!80,
  text=white] %
  \tikzstyle{fancylogo} =[ text=white]
  \begin{center}
    \begin{tikzpicture}
      \node [mybox] (box){%
        \begin{minipage}{0.90\linewidth}
          \vspace{6pt} \BODY
        \end{minipage}
      }; %
      \node[fancytitle, right=10pt] at (box.north west) %
      {\bfseries\normalsize{Commentaire}}; %
    \end{tikzpicture}%
  \end{center}
}

\NewEnviron{remarkST}{%
  % Define box and box title style
  \tikzstyle{mybox} = [draw=black!50, very thick, rectangle, rounded
  corners, inner sep=10pt, inner ysep=8pt] %
  \tikzstyle{fancytitle} = [rounded corners , fill=black!80,
  text=white] %
  \tikzstyle{fancylogo} =[ text=white]
  \begin{center}
    \begin{tikzpicture}
      \node [mybox] (box){%
        \begin{minipage}{0.90\linewidth}
          \vspace{6pt} \BODY
        \end{minipage}
      }; %
      % \node[fancytitle, right=10pt] at (box.north west) %
%       {\bfseries\normalsize{Commentaire}}; %
    \end{tikzpicture}%
  \end{center}
}

\NewEnviron{remarkL}[1]{%
  % Define box and box title style
  \tikzstyle{mybox} = [draw=black!50, very thick, rectangle, rounded
  corners, inner sep=10pt, inner ysep=8pt] %
  \tikzstyle{fancytitle} =[rounded corners, fill=black!80,
  text=white] %
  \tikzstyle{fancylogo} =[ text=white]
  \begin{center}
    \begin{tikzpicture}
      \node [mybox] (box){%
        \begin{minipage}{#1\linewidth}
          \vspace{6pt} \BODY
        \end{minipage}
      }; %
      \node[fancytitle, right=10pt] at (box.north west) %
      {\bfseries\normalsize{Commentaire}}; %
    \end{tikzpicture}%
  \end{center}
}

\NewEnviron{remarkSTL}[1]{%
  % Define box and box title style
  \tikzstyle{mybox} = [draw=black!50, very thick, rectangle, rounded
  corners, inner sep=10pt, inner ysep=8pt] %
  \tikzstyle{fancytitle} =[rounded corners, fill=black!80,
  text=white] %
  \tikzstyle{fancylogo} =[ text=white]
  \begin{center}
    \begin{tikzpicture}
      \node [mybox] (box){%
        \begin{minipage}{#1\linewidth}
          \vspace{6pt} \BODY
        \end{minipage}
      }; %
%       \node[fancytitle, right=10pt] at (box.north west) %
%       {\bfseries\normalsize{Commentaire}}; %
    \end{tikzpicture}%
  \end{center}
}

\NewEnviron{titre} %
{ %
  ~\\[-1.8cm]
  \begin{center}
    \bf \LARGE \BODY
  \end{center}
  ~\\[-.6cm]
  \hrule %
  \vspace*{.2cm}
} %

\NewEnviron{titreL}[2] %
{ %
  ~\\[-#1cm]
  \begin{center}
    \bf \LARGE \BODY
  \end{center}
  ~\\[-#2cm]
  \hrule %
  \vspace*{.2cm}
} %




\pagestyle{fancy} %
\lhead{ECE2 \hfill septembre 2017 \\
 Mathématiques\\[.2cm]} %
\chead{\hrule} %
\rhead{} %
\lfoot{} %
\cfoot{} %
\rfoot{\thepage} %

\renewcommand{\headrulewidth}{0pt}% : Trace un trait de séparation
 % de largeur 0,4 point. Mettre 0pt
 % pour supprimer le trait.

\renewcommand{\footrulewidth}{0.4pt}% : Trace un trait de séparation
 % de largeur 0,4 point. Mettre 0pt
 % pour supprimer le trait.

\setlength{\headheight}{14pt}

\title{\bf \vspace{-1cm} ORAUX HEC 2011} %
\author{} %
\date{} %

\begin{document}

\maketitle %
\vspace{-1.2cm}\hrule %
\thispagestyle{fancy}

\vspace*{.4cm}

% DEBUT DU DOC À MODIFIER : tout virer jusqu'au début de l'exo

\section{Annales 2011}
 %\setcounter{exercice}{0}

\begin{exercice}{\it (Exercice avec préparation)}~\\
  Soit $f$ la fonction définie sur $\R$ par :
  \[
  \forall x \in \R,\ \ f(x) = \frac{e^{- | x |} }{2}.
  \]
  \begin{noliste}{1.}
    \setlength{\itemsep}{4mm}
  \item C'est une fonction positive, continue sauf en un nombre fini
    de points et telle que $\dint{-\infty}{+ \infty} f$ converge et
    vaut 1.
  \item $f$ est positive par positivité de l'exponentielle, continue
    comme composée de fonctions continues (la valeur absolue est bien
    continue!) et paire donc $\dint{-\infty}{+ \infty} f$ converge
    absolument et vaut 1 si et seulement si $\dint{0}{+ \infty} f$
    converge et vaut $\frac{1}{2}$. \\
    Or pour $x > 0$, (à préciser pour pouvoir remplacer la valeur
    absolue!), $\dint{0}{x} f(t)\ dt = \frac{1}{2} \dint{0}{x} e^{-t}
    = \frac{1}{2}$ en primitivant ou en utilisant la loi exponentielle
    de paramètre $\lambda = 1$.\\
    Soit $X$ une variable aléatoire définie sur un espace probabilisé
    $(\Omega, \mathcal{A}, P)$ dont $f$ est une densité de probabilité. \\
  \item 
    \begin{noliste}{a)} 
      \setlength{\itemsep}{2mm}
    \item L'absolue convergence est équivalente à la convergence sur
      $[0 ; + \infty[$ par positivité de $t \rightarrow t f(t)$ et sur
      $]-\infty ; 0]$ par négativité de $t \rightarrow t f(t)$ donc
      sur $]-\infty ; + \infty[$.\\
      De plus la fonction $t \rightarrow t f(t)$ est impaire donc si
      $\dint{0}{+ \infty} t f(t)\ dt$ converge, alors
      $\dint{-\infty}{+ \infty} t f(t)\ dt$ converge et vaut 0. \\
      Or on a $ t^{2} \times t e^{-t} = t^{3} e^{-t} \xrightarrow[t
      \rightarrow + \infty]{} 0$ donc $ t e^{-t} = o \left(
        \frac{1}{t^{2}} \right)$ et par théorème de comparaison des
      intégrales de fonctions positives (on compare ici à une
      intégrale de Riemann convergente), $\dint{0}{+ \infty} t f(t)\
      dt$ converge, donc $X$ admet une espérance et $\E(X) = 0$.
    \item Un peu de calcul ici : il faut calculer la fonction de
      répartition de $f$ pour faire des calculs : \\
      En traitant bien à part les cas $x < 0$ et $x >0$ on trouve :
      \[
      F(x) = \left\{
        \begin{array}{cl}
          \frac{1}{2} e^{x} \text{ si } x <0 \\
          1 - \frac{1}{2} e^{-x} \text{ si } x \geq 0 \\
        \end{array}
      \right.
      \]
      On peut alors calculer $ = \Prob\left(\Ev{X > t - s} \right) =
      \frac{1}{2} \ee^{s - t}$ et $P_{[X > s]} \Ev{X > t} = \frac{
        \Prob( [X > s] \cap [X > t] )}{P \left(\Ev{X > s}\right)} =
      \frac{\Prob\left(\Ev{X > t} \right) }{\Prob\left( \Ev{ X > s}
        \right)} = \frac{ \frac{1}{2} e^{-t} }{ \frac{1}{2} e^{ -s} }
      = e^{s-t} $ si $s \geq 0$ (car alors $t \geq 0$), ce qui
      contredit le résultat.
    \end{noliste}

  \item Il faut prouver que $H_{n}$ est croissante, continue à
    droite et tend vers 0 en $-\infty$ et $1$ en $ + \infty$.\\
    Cependant comme cela ressemble à une variable à densité, on
    considère plutôt $h_{n} (t) = f(t) ( 1 + t e^{- n | t |})$ et on
    prouver que c'est une densité de probabilité : $H_{n}$, fonction
    de répartition
    associée, sera bien une fonction de répartition. \\
    La fonction est continue sur $\R$ par théorèmes généraux sur les
    fonctions continues. \\
    Pour tout $t \in \R$, $f(t) \geq 0$ donc il faut prouver que $ 1 +
    t e^{- n | t |}$ sur $\R$; sur $\R_+ $, c'est évident comme somme
    de deux quantités positives; sur $\R_-$, étudions la fonction
    $g_{n}(t) = 1 + t e^{n t}$ : elle est dérivable et on a $g_{n}'(t)
    = n t e^{n t } + e^{nt} = ( 1 + nt) e^{nt} $ qui s'annule en $t =
    - \frac{1}{n }$ qui est le minimum de la fonction (vérifier les
    signes éventuellement mais cela paraît évident). Enfin on a $g_{n}
    \left( - \frac{1}{n} \right) =
    1 - \frac{1}{n} e^{-1} = 1 - \frac{1}{n e} \geq 0$ car $ ne \geq 1$. \\
    La fonction $g_{n}$ est donc positive sur $\R_-$, et $h_{n}$ est
    positive sur $\R$.  \\
    Ensuite on a $h_{n} (t) = f(t) + t f(t) e^{-n | t |}$. \\
    La première fonction vérifie $\dint{-\infty}{+ \infty} f(t)\ dt $
    converge absolument et vaut 1. \\
    La deuxième est impaire donc comme tout à l'heure si on obtient la
    convergence sur $[ 0 ; + \infty[$, $\dint{-\infty}{+ \infty} t
    f(t)
    e^{-n | t |}$ convergera absolument et vaudra 0. \\
    Or on a $t f(t) = o \left( \frac{ 1}{t^{2}} \right)$ en $ +
    \infty$ et $e^{-n | t |} \xrightarrow[ t \rightarrow + \infty]{}
    0$ donc est négligeable devant 1 donc le produit vérifie $t f(t)
    e^{-n | t |} = o \left( \frac{ 1}{t^{2}} \right)$ et par théorème
    de comparaison des
    intégrales de fonctions positives, l'intégrale est convergente. \\
    finalement on obtient bien par somme que $\dint{-\infty}{+ \infty}
    h_{n} (t)\ dt$ converge absolument et vaut 1. \\
  \item Ce sont des variables aléatoires à densité donc la fonction
    de répartition de $X$ est continue sur $\R$; \\
    il faut donc prouver que pour tout $x \in \R$, $\dint{-\infty}{x}
    f(t) \left( 1 + t e^{ - n | t |} \right)\ dt$ converge vers
    $\dint{-\infty}{x} f(t)\ dt$, donc que $\dint{-\infty}{x} t f(t)
    e^{- n | t |}\ dt \rightarrow0$, et enfin cela équivaut à
    $\dint{-\infty}{x} t
    e^{ - (n + 1) | t |} \rightarrow 0.$ \\
    \\
    Ici il faut être précis dans les calculs : on le prouve pour $x
    \leq
    0$ par intégration par parties et calcul de l'intégrale. \\
    Ensuite pour $x \geq 0$ on a $\dint{-\infty}{0} t e^{ - (n + 1) |
      t |} \rightarrow 0$ donc il suffit de prouver $\dint{0}{x} t e^{
      - (n + 1) | t |} \rightarrow 0$, qu'on prouve également avec une
    intégration par parties (pas la même, la fonction n'a pas la même
    expression!) et
    calcul de l'intégrale. \\
  \end{noliste}
  \noindent \textbf{\underline{Exercice sans préparation}} \\
  \\
  Soit $n$ un entier supérieur ou égal à 2 et $(a_{1}, a_{2},\ \dots\,
  a_{n}) \in \R^{n} - \{ 0,\ \dots\,0) \}$. \\
  On considère la matrice colonne $X = \begin{smatrix}
    a_{1} \\
    a_{2} \\
    \vdots \\
    a_{n} \\
  \end{smatrix}
  \in \mathcal{M}_{n,1} (\R)$. \\
  On pose $B = X\ {}{t}X $ et $A = \ {}{t}X\ X$. \\
  On désigne par $u$ l'endomorphisme de $\R^{n}$ canoniquement associé
  à $B$.
  \begin{noliste}{1.}
    \setlength{\itemsep}{4mm}
  \item $A = \Sum{i = 1}{n} a_{i}{2}$ est un réel et $B = ( b_{i,j}
    )_{ 1 \leq i,j \leq n}$ est une matrice de $\mathcal{M}_{n} (\R)$
    avec $b_{i,j} = a_{i} a_{j}$.
  \item $u$ est de rang 1 car les colonnes de $B$ sont toutes
    multiples de $X$ et au moins une est non nulle (car un au moins
    des $a_{i}$ est non nul et le terme $a_{i}{2}$ correspondant est
    alors non nul donc la colonne correspondante est non nulle).
  \item $B$ est diagonalisable car elle est symétrique ($b_{i,j} =
    b_{j,i} = a_{i} a_{j}$). \\
  \item $B^{k} = (X {}{t} X ) (X {}{t} X ) \dots (X {}{t} X ) = X ({}{t}
    X X) \dots ({}{t} X X) {}{t} X = \left( \Sum{i = 1}{n} a_{i}{2}
    \right)^{k-1} X {}{t} X = \left( \Sum{i = 1}{n} a_{i}{2} \right)^{k-1}
    B$.
  \end{noliste}
\end{exercice}


\newpage


\begin{exercice}{\it (Exercice avec préparation)}~
  \begin{noliste}{1.}
    \setlength{\itemsep}{4mm}
  \item Toute suite croissante converge si et seulement si elle est
    majorée. Sinon elle diverge vers $ + \infty$. \\
    Toute suite décroissante converge si et seulement si elle est
    minorée.  Sinon elle diverge vers $-\infty$.

  \item Dans cette question seulement, on suppose $\alpha = 1$ et
    $\beta = 2$.
 \begin{noliste}{a)}
   \setlength{\itemsep}{2mm}
 \item $f'(x) = \frac{1 + x}{1 + 2x} + x \frac{1 + 2x - 2(1 + x) }{(1
     + 2x)^{2}} = \frac{(1 + x) (1 + 2x) -x}{(1 + 2x)^{2}} =
   \frac{2x^{2} + 2x + 1}{(1 + 2x)^{2}} >0$ (Au numérateur le
   discriminant est égal à $-4$ donc le trinôme est du signe de son
   coefficient dominant, donc positif
   et le dénominateur est un carré donc toujours positif.) \\
   On ne déduit que $f$ est strictement croissante sur $\R_+ $, avec
   $f(0) = 0$ et $\dlim{+ \infty} f = + \infty$. \\
 \item L'intervalle $\R_+ $ est stable par $f$ et $u_{0} \in \R_+ $
donc $u_{n} \in \R_+ $ pour tout $n$. \\
 Avec la croissance de $f$ on peut regarder le signe de $u_{0} -
u_{1}$; mais comme $u_{0}$ est quelconque, autant regarder directement
le signe de $f(x) - x$ : \\
 $f(x) - x = x \left( \frac{1 + x}{1 + 2x} - \frac{1 + 2x}{1 + 2x}
\right) = x \frac{-x}{1 + 2x} = \frac{-x^{2}}{1 + 2x} < 0$ donc la
suite est décroissante ($u_{n + 1} - u_{n} = f(u_{n}) - u_{n} < 0$) et
minorée par 0 donc converge. \\
 De plus elle ne peut converger que vers un point fixe de $f$,
vérifiant donc $f(x) = x \Leftrightarrow \frac{-x^{2}}{1 + 2x} = 0
\Leftrightarrow x = 0$ donc $(u_{n})$ converge vers 0. \\
 \item Question toute simple : je vous laisse faire ce programme
vous-même. \\
 \end{noliste}
 \item On peut reprendre la structure des questions précédentes.
Essayons de varier un peu : \\
 Pour montrer que $u_{n} >0$, on fait une récurrence immédiate avec en
hérédité : \\
 $u_{n} >0$ donc $ 1 + \alpha u_{n} > 0$ et $1 + \beta u_{n} > 0$ donc
$\frac{1 + \alpha u_{n}}{1 + \beta u_{n}} > 0$ et enfin $u_{n + 1} =
u_{n} \frac{1 + \alpha u_{n}}{1 + \beta u_{n}} > 0$. \\
 Ensuite on étudie $u_{n + 1} - u_{n} = u_{n} \frac{1 + \alpha u_{n} -
1 - \beta u_{n}}{1 + \beta u_{n}} = \frac{ (\alpha - \beta )
u_{n}{2}}{1 + \beta u_{n}} < 0$ car $\alpha - \beta <0$, $u_{n}{2} >0$
et $1 + \beta u_{n} >0$ donc $(u_{n})$ est strictement décroissante, et
minorée par 0 donc convergente vers un point fixe donc $l$ vérifie
$\frac{(\alpha - \beta) l^{2}}{1 + \beta l} = 0$ et enfin $l = 0$, donc
$(u_{n})$ converge vers 0. \\

 \item $v_{n + 1} - v_{n} = \frac{1}{u_{n + 1} } - \frac{1}{u_{n}} =
\frac{1}{u_{n}} \left( \frac{ 1 + \beta u_{n}}{1 + \alpha u_{n}} -
\frac{1 + \alpha u_{n}}{1 + \alpha u_{n}} \right) = \frac{ (\beta -
\alpha) u_{n} }{ (1 + \alpha u_{n}) u_{n}} = \frac{ \beta - \alpha }{
(1 + \alpha u_{n})} \xrightarrow[n \rightarrow + \infty]{} \frac{ \beta
- \alpha}{ 1 + \alpha \times 0} = \beta - \alpha$. \\

 \item On en déduit (on pose $w_{n} = v_{n + 1} - v_{n}$) que la suite
$W_{n} = \frac{1}{n} \left( w_{0} + w_{1} + \dots + u_{n-1} \right)$
converge vers $\beta - \alpha$. \\
 Or $W_{n} = \frac{1}{n} \left( v_{n} - v_{0} \right) = \frac{1}{n}
\left( \frac{1}{u_{n}} - \frac{1}{u_{0}} \right)$. \\
 Or on a $\frac{1}{u_{n}} \rightarrow + \infty$ et $\frac{1}{u_{0}}$
est une constante donc $ \left( \frac{1}{u_{n}} - \frac{1}{u_{0}}
\right) \sim \frac{1}{u_{n}}$ et enfin $W_{n} \sim \frac{1}{ n u_{n}}
\sim (\beta - \alpha)$ et enfin $u_{n} \sim \frac{1}{n (\beta -
\alpha)}$. \\
 \end{noliste}
 \noindent \textbf{\underline{Exercice sans préparation}} \\
\\
 $n$ souris (minimum 3) sont lâchées en direction de 3 cages, chaque
cage pouvant contenir les $n$ souris et chaque souris allant dans une
cage au hasard. 
 \begin{noliste}{1.}
 \setlength{\itemsep}{4mm}
 \item On pose $Y_{i}$ le nombre de souris dans la cage $i$, on a
$Y_{i} \suit \mathcal{B} \left( n, \frac{1}{3} \right)$ donc
$\Prob\left(\Ev{\Ev{Y_{i} = 0}}\right) = \left( \frac{2}{3}
\right)^{n}$. \\
 La probabilité cherchée vaut $\Prob( [ Y_{1} = 0] \cup [Y_{2} = 0]
\cup [Y_{3} = 0] ) = \Prob\left(\Ev{\Ev{Y_{1} = 0}}\right) +
\Prob\left(\Ev{\Ev{ Y_{2} = 0}}\right) + \Prob\left(\Ev{\Ev{ Y_{3} =
0}}\right) - \Prob( [Y_{1} = 0] \cap [Y_{2} = 0]) - \Prob([ Y_{1} = 0]
\cap [ Y_{3} = 0]) - \Prob( [ Y_{2} = 0] \cap [Y_{3} = 0]) + \Prob(
[Y_{1} = 0] \cap [Y_{2} = 0] \cap [Y_{3} = 0])$. \\
 Or on a $\Prob( [Y_{1} = 0] \cap [Y_{2} = 0] \cap [Y_{3} = 0]) = 0$
(les trois cages ne peuvent être vides en même temps, où seraient passé
les souris ?) \\
 D'autre part pour calculer $P ([Y_ i = 0] \cap [Y_{j} = 0])$ on pose
$Z_{i,j}$ le nombre de souris dans les cages $i$ et $j$, $Z_{i,j} \suit
\mathcal{B} \left( n, \frac{2}{3} \right)$ donc $P ( [Y_ i = 0] \cap
[Y_{j} = 0]) = \Prob\left(\Ev{ Z_{i,j} = 0}\right) = \left( \frac{1}{3}
\right)^{n}$. \\
 Enfin on obtient $\Prob( [ Y_{1} = 0] \cup [Y_{2} = 0] \cup [Y_{3} =
0] ) = 3 \frac{2^{n} - 1}{3^{n}} = \frac{2^{n} - 1}{3^{n-1}}$. \\

 \item On pose $X_{i}$ la variable aléatoire égale à 1 si la cage $i$
reste vide, et 0 sinon, on a $X_{i} \suit \mathcal{B} \left( \left(
\frac{2}{3} \right)^{n} \right)$ et on a $X = X_{1} + X_{2} + X_{3} $
donc $\E(X) = 3 \E(X_{1})$ (les trois variables suivent la même loi et
ont donc la même espérance). \\
 Enfin $\E(X) = \frac{2^{n}}{3^{n-1}}$.
 \end{noliste}
 \end{exercice}

 \newpage

 \begin{exercice}{\it (Exercice avec préparation)}~
 \begin{noliste}{1.}
 \setlength{\itemsep}{4mm}
 \item Une variable aléatoire $X$ est à densité s'il existe une
fonction $f$ positive, continue sauf en un nombre finie de points,
telle que pour tout $x \in \R$, $F_{X} ( x) = \dint{-\infty}{x} f(t)\
dt$. \\
 Toute fonction de répartition est croissante, continue à droite en
tout point et de limites $0$ en $-\infty$ et $1$ en $ + \infty$. \\
 La variable est à densité si et seulement si $F_{X}$ est de plus
continue sur $\R$, de classe $C^{1}$ sauf en un nombre fini de points.
\\
 \item $F$ continue sur $\R$ donc admet des primitives, et donc une
unique primitive sur $\R$ s'annulant en 0, notée $H_{f}$. \\
 De plus $H_{f}$ est dérivable sur $\R$, de dérivée $F$ continue donc
$H_{f}$ est de classe $C^{1}$ sur $\R$. \\
 \item Donner $H_{f}$ dans les cas suivants : 
 \begin{noliste}{a)}
 \setlength{\itemsep}{2mm} 
 \item On a alors $F(x) = 0$ si $x \leq 0$ et $F(x) = 1 - e^{-x}$ si $x
>0$, puis $H_{f}(x) = 0$ si $x \leq 0$ et $H_{f}(x) = x + e^{-x} - 1$
si $x > 0$, d'asymptote oblique $y = x -1$ en $ + \infty$. \\
 \item On a alors $F(x) = 0$ si $x \leq 0$ et $F(x) = 1 - \frac{1}{1 +
x}$ si $x >0$, puis $H_{f}(x) = 0$ si $x \leq 0$ et $H_{f}(x) = x - \ln
(1 + x)$ si $x > 0$, de direction asymptotique $y = x$ en $ + \infty$,
mais qui n'a pas d'asymptote en $ + \infty$. \\
 \item On a alors $F(x) = 0$ si $x \leq 0$ et $F(x) = 1 -
\frac{1}{\sqrt{1 + x} }$ si $x > 0$ puis $H_{f}(x) = 0$ si $x \leq 0$
et $H_{f}(x) = x - 2 \sqrt{1 + x} + 2$ si $x > 0$, de direction
asymptotique $y = x$, mais qui n'a pas d'asymptote en$ + \infty$. \\
 \end{noliste}
 \item On suppose que $X$ admet une espérance $l$. \begin{noliste}{a)}
 \setlength{\itemsep}{2mm}
 \item On intègre par parties avec $u = t$ et $v = F(t)$ de classe
$C^{1}$ sur $[ 0 ; x]$ et on a : \\
 $\dint{0}{x} t f(t)\ dt = [ t F(t) ]_{0}{x} - \dint{0}{x} F(t)\ dt = x
F(x) - H_{f} (x) + H_{f}(0) = x F(x) - H_{f} (x)$. \\
 Comme $X$ admet une espérance, on a $\dint{0}{x} t f(t)\ dt
\rightarrow \E(X)$ donc $H_{f} (x) = x F(x) - \dint{0}{x} t f(t)\ dt =
x \left( F(x) - \frac{\dint{0}{x} t f(t)\ dt }{x} \right) \sim x F(x)
\sim x $ car $\dlim{+ \infty} F(x) = 1$.
 \\
\\
 On en déduit que $\frac{H_{f}(x) }{x } \sim 1 \xrightarrow[ x
\rightarrow + \infty]{} 1$ donc on a une direction asymptotique $y =
x$. \\
 \item Difficile de répondre : les cas de la question 3 ne sont pas
concluants (les deux cas où il n'y a pas d'asymptote proviennent de
variables sans espérance). Il est probable qu'avec une rédaction
similaire, on montrer que si $X$ admet une variance, il y a bien une
asymptote (en intégrant par parties l'intégrale menant au moment
d'ordre 2). \\
 Je ne vois pas comment répondre à la question posée intégralement (ce
n'est peut-être pas le but recherché) : il faudrait arriver à obtenir
un développement asymptotique de $x (F(x) - 1)$. \\
 \end{noliste}
 \end{noliste}
 \noindent \textbf{\underline{Exercice sans préparation}} \\
\\
 Soit $E$ l'ensemble des matrices $M_{a,b} = \begin{smatrix}
a & b & b \\
b & a & b \\
b & b & a \\
\end{smatrix}
$ où $(a,b)$ prend toute valeur de $\R^{2}$. \begin{noliste}{1.}
 \setlength{\itemsep}{4mm}
 \item Evident. \\

 \item On peut calculer les premières puissances pour essayer de voir
une relation simple : cela échoue. \\
 La matrice est symétrique donc diagonalisable, on va la diagonaliser.
\\
 Pour simplifier on utilise le fait que $M(a,b) = a I + b A$ et comme
$I = P I P^{-1}$ pour tout $P$, si on diagonalise $A$ on aura $A = P D
P^{-1}$ puis $M(a,b) = a P I P^{-1} + b P D P^{-1} = P \left(\Ev{ a I +
b D}\right) P^{-1}$ et on aura la diagonalisation de $M(a,b)$. \\
 Enfin l'étude des valeurs propres et des sous-espaces propres de $A$
donne $A = P D P^{-1}$ avec \\
$P = \begin{smatrix}
1 & 1 & 1 \\
-1 & 0 & 1 \\
0 & -1 & 1 \\
\end{smatrix}
$ et $D = \begin{smatrix}
-1 & 0 & 0 \\
0 & -1 & 0 \\
0 & 0 & 2 \\
\end{smatrix}
$ donc $M(a,b) = P \begin{smatrix}
a -b & 0 & 0 \\
0 & a-b & 0 \\
0 & 0 & a + 2b \\
\end{smatrix}
$ donc $M(a,b)^{n} = P \begin{smatrix}
(a -b)^{n} & 0 & 0 \\
0 & (a-b)^{n} & 0 \\
0 & 0 & (a + 2b)^{n} \\
\end{smatrix}
P^{-1}$.

 \end{noliste}
 \end{exercice}

 \newpage


 \begin{exercice}{\it (Exercice avec préparation)}~\\
 Toutes les variables aléatoires de cet exercice sont définies sur
 un espace probabilisé $(\Omega, \mathcal{A}, P)$. Soit $p \in \ ]0
 ; 1[$ et $q = 1-p$.
 \begin{noliste}{1.}
 \setlength{\itemsep}{4mm}
 \item $n$ variables discrètes $(X_{1},\ \dots\, X_{n}$ sont
mutuellement indépendantes ou indépendantes dans leur ensemble si pour
tout $(x_{1},\ \dots\, x_{n}) \in \R^{n}$, $P \left(\Ev{
\bigcap\limits_{i = 1}{n} X_{i} = x_{i}}\right) = \prod\limits_{i =
1}{n} \Prob\left(\Ev{\Ev{ X = x_{i}}}\right)$. \\
 Bien évidemment il suffit de le vérifier pour des $x_{i}$ toujours
dans $X_{i} (\Omega)$ pour tout $i$. \\

 \item 
 \begin{noliste}{a)}
 \setlength{\itemsep}{2mm} 
 \item $X_{1}$ et $X_{2}$ suivent des lois géométriques de paramètre
$p$ donc on a $\Prob\left(\Ev{\Ev{X_{1} = 0}}\right) =
\Prob\left(\Ev{\Ev{ X_{2} = 0}}\right) = 0$. \\
 \item Déjà fait; l'indépendance des deux lois est obtenue par
indépendance des lancers pairs et des lancers impairs. \\
 \item $Y(\Omega) = \N$, $\Prob\left(\Ev{\Ev{ Y = 0}}\right) = 0$ et
pour tout $k \geq 0$ : \\
 $\Prob\left(\Ev{\Ev{ Y > k}}\right) = \Prob( [X_{1} > k] \cap [X_{2} >
k] ) = \Prob\left(\Ev{\Ev{ X_{1} > k}}\right)^{2}$ par indépendance et
même loi. \\
 D'où $\Prob\left(\Ev{\Ev{Y > k}}\right) = ( q^{k})^{2}$ et
$\Prob\left(\Ev{\Ev{ Y \leq k}}\right) = 1 - (q^{2})^{k}$. \\
 Enfin pour $k \geq 1$, $\Prob\left(\Ev{\Ev{Y = k}}\right) =
\Prob\left(\Ev{\Ev{ Y \leq k}}\right) - \Prob\left(\Ev{\Ev{ Y \leq
k-1}}\right) = (q^{2})^{k-1} - (q^{2})^{k} = (q^{2})^{k-1} ( 1 -
q^{2})$ et $Y$ suit la loi géométrique de paramètre $1 - q^{2}$. \\
 \end{noliste}
 \item Soit $X$ une variable aléatoire suivant une loi géométrique de
paramètre $p$. 
 \begin{noliste}{a)}
 \setlength{\itemsep}{2mm}
 \item On a $Y (\Omega) = \N^*$ : $Y \subset \N^*$ est évident et pour
$k \neq 0$, $Y = k$ est atteint pour $X = 2k$ donc on a bien $\N^*
\subset Y(\Omega)$. \\
 Ensuite on a plus précisément $\Ev{ Y = k} = \Ev{ X = 2k} \cup (X = 2k
- 1)$ qui sont incompatibles donc $\Prob\left(\Ev{\Ev{ Y = k}}\right) =
P \Ev{X = 2k} + \Prob\left(\Ev{\Ev{ X = 2k-1}}\right) = p \left(
q^{2k-1} + q^{2k-2} \right) = p q^{2k-2} ( q + 1) = [ (1 + q) (1-q) ]
(q^{2})^{k-1} = (1-q^{2}) (q^{2})^{k-1}$
 donc $Y \suit \mathcal{G} ( 1 - q^{2})$. \\
\item L'étude de la partie entière montre que $ (2Y - X) (\Omega) = \{
  0 ; 1 \}$ et on a :\\
  $\Prob\left(\Ev{ 2Y - X = 0} \right) = P \left(\Ev{X \mbox{pair}
    }\right) = \Sum{k = 1}{+ \infty} \Prob\left(\Ev{\Ev{ X =
        2k}}\right) = p q^{-1} \Sum{k = 1}{+ \infty} (q^{2})^{k} =
  \frac{ p q^{2}}{q ( 1 -q^{2}) } = \frac{ q}{1 + q}$.\\
  $\Prob\left(\Ev{ 2Y - X = 1}\right) = P \left(\Ev{X
      \mbox{impair}}\right) = \Sum{k = 0}{+ \infty}
  \Prob\left(\Ev{\Ev{ X = 2k + 1}}\right) = p \Sum{k = 0}{+ \infty}
  (q^{2})^{k} = \frac{p}{1-q^{2}} = \frac{ 1}{ 1 + q }$.\\
  Enfin on a $ P( \Ev{Y = k } \cap (2 Y - X = 0) ) =
  \Prob\left(\Ev{\Ev{ X = 2k}}\right) = p q^{2k-1}$ et
  $\Prob\left(\Ev{\Ev{ Y = k}}\right) \Prob\left(\Ev{ 2 Y - X =
      0}\right) = (1-q^{2}) (q^{2})^{k-1} \frac{q}{1 + q} = (1-q)
  q^{2k-1} = p q^{2k-1}$.\\
  De même on a $ P( \Ev{Y = k } \cap (2 Y - X = 1) ) =
  \Prob\left(\Ev{\Ev{ X = 2k-1}}\right) = p q^{2k-2}$ et
  $\Prob\left(\Ev{\Ev{ Y = k}}\right) \Prob\left(\Ev{ 2 Y - X =
      1}\right) = (1-q^{2}) (q^{2})^{k-1} \frac{1}{1 + q} = (1-q)
  q^{2k-2} = p q^{2k-2}$, et les variables sont indépendantes.
\end{noliste}
\end{noliste}
\noindent \textbf{\underline{Exercice sans préparation}} \\
\\
On note $E_{4}$ l'espace vectoriel des fonctions polynomiales de degré
inférieur ou égal à 4 et on considère l'application $\Delta$ qui à un
polynôme $P$ de $E_{4}$ associe le polynôme $Q = \Delta (P)$ défini
par : $Q(x) = P\left(\Ev{x + 2}\right) - P\left(\Ev{x}\right)$.
\begin{noliste}{1.}
 \setlength{\itemsep}{4mm}
 \item La linéarité est évidente. \\
 De plus on a $\deg P\left(\Ev{x + 2}\right) = \deg P \times \deg (X +
2) = \deg P$ donc $\deg \Delta (P) \leq \min (\deg P, \deg P) = \deg P
\leq 4$ donc $\Delta (P) \in E_{4}$ et $\Delta$ est un endomorphisme.
\\
\\
 Avec le binôme de Newton on trouve : $Mat_{\mathcal{B} } ( \Delta) =
\begin{smatrix}
0 & 2 & 4 & 8 & 16 \\
0 & 0 & 4 & 12 & 32 \\
0 & 0 & 0 & 6 & 24 \\
0 & 0 & 0 & 0 & 8 \\
0 & 0 & 0 & 0 & 0 \\
\end{smatrix}
$.
 \item Il est beaucoup plus simple d'utiliser la matrice : on trouve
$\ker \Delta = \Vect{ e_{0}} = \R_{0} [X]$. \\
 Sinon avec l'indication on suppose que $P\left(\Ev{x + 2}\right) =
P\left(\Ev{x}\right)$, alors on a $P\left(\Ev{2}\right) =
P\left(\Ev{0}\right)$, puis $P\left(\Ev{4}\right) =
P\left(\Ev{2}\right) = P\left(\Ev{0}\right)$ et par une récurrence
simple, pour tout $n$, $P\left(\Ev{2n}\right) = 0$ donc
$P\left(\Ev{x}\right) - P\left(\Ev{0}\right)$ a une infinité de
racines, donc $P\left(\Ev{x}\right) - P\left(\Ev{0}\right) = 0$ et
enfin $P\left(\Ev{x}\right) = P\left(\Ev{0}\right)$ est une constante.
D'où $\ker \Delta \subset \R_{0} [X]$. \\
 Enfin on trouve facilement que $\R_{0}[X] \subset \ker \Delta$ en
prenant un polynôme constant, qui vérifie trivialement $P\left(\Ev{x +
2}\right) = P\left(\Ev{x}\right)$ et on obtient le résultat. \\
 \item La matrice de $\Delta$ est triangulaire, on obtient que 0 est
l'unique valeur propre. \\
 Si $\Delta$ était diagonalisable, il existerait $P$ inversible telle
que $M = P 0 P^{-1} = 0$, ce qui est absurde. \\
 \item Soit $Q$ un polynôme admettant un antécédent, on a $\Delta (P) =
Q$. \\
 Alors l'équation $\Delta (R) = Q$ est équivalente à $\Delta (R) -
\Delta (P) = \Delta (R - P) = 0$ donc tout polynôme de la forme $R = P
+ cste$ est solution, et la réponse à la question est non.

 \end{noliste}
 \end{exercice}

 \newpage

 \begin{exercice}{\it (Exercice avec préparation)}~\\
 Dans tout l'exercice, $n$ désigne un entier naturel non nul et
$\R_{n}[X]$ l'espace vectoriel des polynômes à coefficients réels, de
degré inférieur ou égal à $n$. On note $M (m_{i,j})_{1 \leq i,j \leq n
+ 1}$ la matrice de $\mathcal{M}_{n + 1} (\R)$ de terme général : 
 
\[
 m_{i,j} = \left\{ 
\begin{array}{cc}
 i & \text{ si } j = i + 1 \\
n + 1 - j & \text{ si } i = j + 1 \\
0 & \text{ dans tous les autres cas } \\
\end{array}
\right.
\]
 et $u$ l'endomorphisme de $\R_{n}[X]$ dont la matrice dans la base
canonique $(1, X,\ \dots\, X^{n})$ est égale à $M$.
 \begin{noliste}{1.}
 \setlength{\itemsep}{4mm}
 \item Soit $u$ un endomorphisme d'un espace vectoriel $E$, on appelle
vecteur propre de $u$ tout vecteur $X$ non nul tel qu'il existe un réel
$\lambda$ vérifiant $u(X) = \lambda X$. On dit alors que $\lambda$ est
une valeur propre de $u$ et $X$ un vecteur propre de $u$ associé à la
valeur propre $\lambda$. \\
 Une famille de vecteurs propres associés à des valeurs propres
distinctes est libre. \\
 \item 
 \begin{noliste}{a)}
 \setlength{\itemsep}{2mm} 
 \item D'après la matrice, $u(X^{k}) = k X^{k-1} + (n + 1 - (k + 1))
X^{k + 1} = k X^{k-1} + (n-k) X^{k + 1}$ si $n-1 \geq k \geq 1$, $u(1)
= n X$ et $u(X^{n}) = n X^{n-1}$. \\
 \item On voit qu'en posant $e_{k} = X^{k}$, on a $u (e_{k}) = (1 -
X^{2}) e_{k}' + n X e_{k}$. \\
 Comme c'est vrai sur une base de $\R_{n} [x]$ on a pour tout $P \in
\R_{n} [X]$, $u(P) = (1 - X^{2}) P' + n X P$. \\
 \end{noliste}
 \item Pour $k \in \llb 0 ; n \rrb$, on pose $P_{k} (X) = (X-1)^{k} (X
+ 1)^{n-k}$.
 \begin{noliste}{a)}
 \setlength{\itemsep}{2mm} \item $u(P_{k}) = (1-X^{2}) \left[ k
(X-1)^{k-1} (X + 1)^{n-k} + (n-k) (X-1)^{k} (X + 1)^{n-k-1} \right] + n
X (X-1)^{k} (X + 1)^{n-k} = (X-1)^{k-1} (X + 1)^{n-k-1} \left[
(1-X^{2}) [ k (X + 1) + (n-k) (X-1) ] + n X (X-1) (X + 1) \right] \\
\\ = (X-1)^{k-1} (X + 1)^{n-k-1} \left[ (1-X^{2}) [ n X + 2k - n ] + n
X (X-1) (X + 1) \right] \\
\\ = (X-1)^{k-1} (X + 1)^{n-k-1} \left[ n X + 2k -n - n X^{3} + (n -
2k) X^{2} + n X^{3} - n X \right] \\
\\ = (n - 2k) (X-1)^{k-1} (X + 1)^{n-k-1} (X^{2} - 1) = (n-2k)
(X-1)^{k} (X + 1)^{n-k} = (n - 2k) P_{k}$. \\
 \item C'est une famille de vecteurs propres de $u$ associés à des
valeurs propres distinctes donc elle est libre; de plus elle est de
cardinal $n + 1$ donc c'est une base. \\
 \item Il existe une base de vecteurs propres de $u$ donc $u$ est
diagonalisable. \\
 \end{noliste}
 \item Dans cette question, on suppose que $n = 3$.
 \begin{noliste}{a)}
 \setlength{\itemsep}{2mm} \item $M = \begin{smatrix}
0 & 1 & 0 & 0 \\
3 & 0 & 2 & 0 \\
0 & 2 & 0 & 3 \\
0 & 0 & 1 & 0 \\
\end{smatrix}
$ et la base de vecteur propres est : \\
 $(X + 1)^{3} = X^{3} + 3 X^{2} + 3 X + 1$ associé à la valeur propre
$3$, \\
 $(X-1)(X + 1)^{2} = X^{3} + X^{2} - X - 1$ associé à la valeur propre
$1$, \\
 $(X-1)^{2} (X + 1) = X^{3} - X^{2} - X + 1$ associé à la valeur propre
$-1$, \\
 $(X-1)^{3} = X^{3} - 3 X^{2} + 3 X - 1$ associé à la valeur propre
$-3$, \\
 donc $P = \begin{smatrix}
1 & -1 & 1 & -1 \\
3 & -1 & -1 & 3 \\
3 & 1 & -1 & -3 \\
1 & 1 & 1 & 1 \\
\end{smatrix}
$ et $D = \begin{smatrix}
3 & 0 & 0 & 0 \\
0 & 1 & 0 & 0 \\
0 & 0 & -1 & 0 \\
0 & 0 & 0 & -3 \\
\end{smatrix}
$.
 \item $ \begin{smatrix}
a & b & c & d \\
e & f & g & h \\
i & j & k & l \\
m & n & o & p \\
\end{smatrix}
D = D \begin{smatrix}
a & b & c & d \\
e & f & g & h \\
i & j & k & l \\
m & n & o & p \\
\end{smatrix}
\Leftrightarrow \begin{smatrix}
3a & b & -c & -3d \\
3e & f & -g & -3h \\
3i & j & -k & -3l \\
3m & n & -o & -3p \\
\end{smatrix}
 = \begin{smatrix}
3a & 3b & 3c & 3d \\
e & f & g & h \\
- i & -j & -k & - l \\
-3 m & -3n & -3o & -3p \\
\end{smatrix}
\Leftrightarrow b = c = d = z = g = h = i = j = l = m = n = o = 0$ donc
les matrices commutant avec $D$ sont les matrices diagonales. \\
 \item On se place dans la base de vecteurs propres, on appelle $N$ la
matrice de $v$. \\
 Alors $v \circ v = u \Leftrightarrow N^{2} = D$. On a alors $N D = N
N^{2} = N^{3} = N^{2} N = D N$ donc $N$ commute avec $D$, et elle est
diagonale. \\
 De plus $\begin{smatrix}
a & 0 & 0 & 0 \\
0 & b & 0 & 0 \\
0 & 0 & c & 0 \\
0 & 0 & 0 & d \\
\end{smatrix}
^{2} = \begin{smatrix}
3 & 0 & 0 & 0 \\
0 & 1 & 0 & 0 \\
0 & 0 & -1 & 0 \\
0 & 0 & 0 & -3 \\
\end{smatrix}
\Leftrightarrow a^{2} = 3,\ b^{2} = 1,\ c^{2} = -1$ et $d^{2} = -3$ et
les deux dernières équations sont impossibles donc il n'y a pas de
solution. \\
 \end{noliste} 
 \end{noliste}
 \noindent \textbf{\underline{Exercice sans préparation}} \\
\\
 Soient $X$ et $Y$ deux variables aléatoires définies sur un espace
probabilisé $(\Omega, \mathcal{A}, P)$ à valeurs dans $\N^*$,
indépendantes et telles que : 
 
\[
 \forall i \in \N^*,\ \Prob\left(\Ev{\Ev{X = i}}\right) =
\Prob\left(\Ev{\Ev{Y = i}}\right) = \frac{1}{2^{i}}
\]
 \begin{noliste}{1.}
 \setlength{\itemsep}{4mm}
 \item On reconnaît la loi géométrique de paramètre $\frac{1}{2}$. \\
 \item $Z(\Omega) = \llb 2 ; + \infty \llb $ et avec la formule des
probabilités totales (système complet d'évènements $\Ev{X = i}_{i \in
\N^*}$ ) on a pour tout $k \geq 2$ : \\
 $\Prob\left(\Ev{\Ev{ Z = k}}\right) = \Sum{i = 1}{+ \infty} P ( [X =
i] \cap [Z = k] ) = \Sum{i = 1}{+ \infty} P ( [X = i] \cap [X + Y = k]
) = \Sum{i = 1}{+ \infty} P ( [X = i] \cap [Y = k-i] ) = \Sum{i = 1}{+
\infty} P \Ev{ [X = i] } \Prob\left(\Ev{\Ev{ [Y = k-i] }}\right) =
\Sum{i = 1}{k-1} \frac{1}{2^{i}} \times \frac{1}{2^{k-i}} = \Sum{i =
1}{k-1} \frac{1}{2^{k} } = (k-1) \left( \frac{1}{2} \right)^{k}$. \\
\\
 D'autre part pour tout $i \geq k$ on a $P_{\Ev{X + Y = k}} \Ev{X = i}
= 0$ et pour $1 < i < k-1$ on a : \\
 $P_{\Ev{X + Y = k}} \Ev{X = i} = \frac{ \Prob( [X + Y = k] \cap [X =
i] ) }{P \Ev{X + Y = k}} = \frac{ \Prob( [ X = i] \cap [Y = k-i] )
}{(k-1) \left( \frac{1}{2} \right)^{k}} = \frac{1}{k-1}$. \\
 \item $\Prob\left(\Ev{\Ev{ X = Y}}\right) = \Sum{i = 1}{+ \infty}
\Prob( [X = i] \cap [Y = i] ) = \Sum{i = 1}{+ \infty} \left(
\frac{1}{4} \right)^{i} = \frac{1}{4} \times \frac{1}{ 1 -\frac{1}{4} }
= \frac{1}{3}$. \\
\\
 Par symétrie $\Prob\left(\Ev{\Ev{X< Y}}\right) = P \Ev{X > Y}$ et
$\Prob\left(\Ev{\Ev{X<Y}}\right) + \Prob\left(\Ev{\Ev{X > Y}}\right) +
\Prob\left(\Ev{\Ev{X = Y}}\right) = 1$ (somme des probabilités sur un
système complet) donc $\Prob\left(\Ev{\Ev{X < Y}}\right) =
\Prob\left(\Ev{\Ev{X > Y}}\right) = \frac{1 - \frac{1}{3} }{2} =
\frac{1}{3}$. \\
 \item $\Prob\left(\Ev{\Ev{ X \geq 2 Y}}\right) = \Sum{i = 1}{+ \infty}
\Prob( [Y = i] \cap [X > 2i - 1] ) = \Sum{i = 1}{+ \infty}
\frac{1}{2^{i}} \frac{1}{2^{2i - 1}} = \Sum{i = 1}{+ \infty}
\frac{1}{2^{3i -1} } = 2 \Sum{i = 1}{+ \infty} \left( \frac{1}{8}
\right)^{i} = 2 \frac{1}{8} \frac{1 }{1 - \frac{1}{8} } = \frac{2}{7}$.
\\
\\
 Enfin $P_{[X \geq Y]} \Ev{ X \geq 2 Y} = \frac{ P ([X \geq Y] \cap [X
\geq 2 Y])}{\Prob\left(\Ev{\Ev{ X \geq Y}}\right)} = \frac{
\Prob\left(\Ev{\Ev{ X \geq 2 Y}}\right)}{P \Ev{X \geq Y}} = \frac{
\frac{2}{7} }{ \frac{2}{3} } = \frac{3}{7}$.

 \end{noliste}
 \end{exercice}

 \newpage

 \begin{exercice}{\it (Exercice avec préparation)}~\\
 Soit $(X_{n})_{n \in \N}$ une suite de variables aléatoires
indépendantes définie sur un espace probabilisé $(\Omega, \mathcal{A},
P)$ telles que, pour tout $n \in \N^*$, $X_{n}$ suit la loi
exponentielle de paramètre $\frac{1}{n}$ (d'espérance $n$). \\
 Pour tout $x$ réel on note $\lfloor x \rfloor$ sa partie entière. \\
 Pour $n \in \N^*$ soient :
 
\[
 Y_{n} = \lfloor X_{n} \rfloor \ \ \ \text{ et } \ \ \ Z_{n} = X_{n} -
\lfloor X_{n} \rfloor 
\]
 \begin{noliste}{1.}
 \setlength{\itemsep}{4mm}
 \item Une suite $(X_{n})$ de variables aléatoires converge en loi vers
$X$ si pour tout $x$ telle que $F_{X}$ est continue en $x$, $\dlim{n
\rightarrow + \infty} F_{X_{n}} (x) = F_{X}(x)$. \\
 \item $Y_{n} (\Omega) = \lfloor \R_+ \rfloor = \N$. \\
 Pour tout $k \in \N$, $\Prob\left(\Ev{\Ev{Y_{n} = k}}\right) =
\Prob\left(\Ev{ k \leq X_{n} < k + 1}\right) = F_{X_{n}} (k + 1) -
F_{X_{n}} (k) = 1 - e^{ - \frac{k + 1}{n} } - 1 + e^{ - \frac{k}{n} } =
\left( e^{- \frac{1}{n} } \right)^{k} ( 1 - e^{ - \frac{1}{n} })$. \\
\\
 On remarque que pour $k \in \N^*$, $\Prob\left(\Ev{\Ev{ Y + 1 =
k}}\right) = \Prob\left(\Ev{\Ev{ Y = k-1}}\right) = \left( e^{-
\frac{1}{n} } \right)^{k-1} ( 1 - e^{ - \frac{1}{n} })$ et $Y + 1$ suit
la loi géométrique de paramètre $ 1 - e^{ - \frac{1}{n} }$, d'espérance
$ \frac{1}{ 1 - e^{ - \frac{1}{n} }}$ et enfin $\E(Y) = \E(Y + 1-1) =
\E(Y + 1) - 1 = \frac{1}{ 1 - e^{ - \frac{1}{n} }} - 1 = \frac{1 - ( 1
- e^{ - \frac{1}{n} }) }{ 1 - e^{ - \frac{1}{n} }} = \frac{ e^{ -
\frac{1}{n} } }{ 1 - e^{ - \frac{1}{n} }}$. \\
 \item On $Y_{n} \leq X_{n} < Y_{n} + 1$ donc $0 \leq Z_{n} < 1$ et
$Z_{n} (\Omega) = [ 0 ; 1[$. \\
 Avec le système complet $\Ev{Y_{n} = k}_{k \in \N}$ on a
$\Prob\left(\Ev{\Ev{Z_{n} \leq t}}\right) = \Sum{k = 0}{+ \infty} P (
\Ev{Y_{n} = k} \cap \Ev{Z_{n} \leq t} ) = \Sum{k = 0}{+ \infty} P
\left(\Ev{k \leq X_{n} \leq k + t}\right) = \Sum{k = 0}{+ \infty}
F_{X_{n}} (k + t) - F_{X_{n}} (k) = \Sum{k = 0}{+ \infty} e^{-
\frac{k}{n} } - e^{ - \frac{ k + t}{n} } = (1 - e^{- \frac{t}{n} })
\Sum{k = 0}{+ \infty} \left( e^{ - \frac{1}{n} } \right)^{k} = \frac{1
- e^{- \frac{t}{n} } }{1 - e^{- \frac{1}{n} }} $. \\
 \item Pour tout $t \in [ 0 ; 1]$, on obtient $F_{Z_{n}} (t) \sim
\frac{ \frac{-t}{n} }{ \frac{-1}{n} } = t \xrightarrow[ n \rightarrow +
\infty]{} t$. \\
 De plus on a pour $t \leq 0$, $F_{Z_{n}} (t) = 0 \xrightarrow[ n
\rightarrow + \infty]{} 0$ et pour $t \geq 1$, $F_{Z_{n}} (t) = 1
\xrightarrow[ n \rightarrow + \infty]{} 1$; donc $(Z_{n})$ converge en
loi vers une variable aléatoire $Z$ suivant la loi uniforme sur $[ 0 ;
1]$. \\
 \item Soit $n \in \N^*$ et $N_{n}$ la variable aléatoire définie par :
 
\[
 N_{n} = \Card \left\{ k \in \llb 1 ; n \rrb \text{ tel que } X_{k}
\leq \frac{k}{n} \right\}
\]
 où $\Card (A)$ désigne le nombre d'éléments de l'ensemble fini $A$.
 \begin{noliste}{a)}
 \setlength{\itemsep}{2mm} 
 \item On a $N_{n} (\Omega) = \llb 0 ; n \rrb$ et $N_{n}$ compte le
nombre de succès dans une succession de $n$ épreuves de Bernouilli
indépendantes de même paramètre $P \left(\Ev{ X_{n} \leq
\frac{k}{n}}\right) = 1 - e^{ - \frac{ \frac{k}{n} }{k} } = 1 - e^{ -
\frac{1}{n} }$ (qui est bien indépendant de $k$) donc $N_{n} \suit
\mathcal{B} \left( n, 1 - e^{ - \frac{1}{n} } \right)$. \\
 \item Pour tout $i \in \N$, dès que $n \geq i$ on a : \\
 $\Prob\left(\Ev{\Ev{ N_{n} = i}}\right) = \binom{n}{i} \left( 1 - e^{-
\frac{1}{n} } \right)^{i} \left( e^{ - \frac{1}{n} } \right)^{n-i} \sim
\binom{n}{i} \left( \frac{1}{n} \right)^{i} \left( e^{ - \frac{1}{n} }
\right)^{n-i} = \frac{n}{n} \times \dots \times \frac{n-i + 1}{n}
\times \frac{1}{i!} \times e^{- \frac{n-i}{n} } \\
\Prob\left(\Ev{\Ev{ N_{n} = i}}\right) \sim \frac{e^{ -1}}{i!} =
\frac{1^{i} e^{-1} }{i!}$ et on reconnaît une loi de Poisson de
paramètre 1. \\
 D'où $(N_{n})$ converge en loi vers une variable aléatoire $N$ qui
suit la loi de Poisson de paramètre 1. \\
 \end{noliste} 
 \end{noliste}
 \noindent \textbf{\underline{Exercice sans préparation}} \\
\\
 Soit $E$ l'ensemble des matrices $M_{a,b} = \begin{smatrix}
a & b & b \\
b & a & b \\
b & b & a \\
\end{smatrix}
$ où $(a,b)$ prend toute valeur de $\R^{2}$. \begin{noliste}{1.}
 \setlength{\itemsep}{4mm}
 \item Evident. \\
 \item Si $a = b$, $M_{a,b}$ a trois colonnes égales donc n'est pas
inversible. \\
 Si $a = -2b$ on a $C_{1} + C_{2} + C_{3} = 0$ donc $M$ n'est pas
inversible. \\
 Sinon on a avec la méthode du pivot complet $M_{a,b}$ est inversible
et : \\
 $M_{a,b}{-1} = \frac{1}{(b-a) (a + 2b) } \begin{smatrix}
- (a + b) & b & b \\
b & -(a + b) & b \\
b & b & -(a + b) \\
\end{smatrix}
\in E$. \\
 \item On peut essayer les premières puissances; on n'obtient rien de
probant. \\
 Il faut alors diagonaliser; les valeurs propres peuvent être déduites
de la deuxième question : \\
 En effet $M_{a,b} - \lambda I = M_{a -\lambda, b}$ n'est pas
inversible si et seulement si $a - \lambda = b \Leftrightarrow \lambda
= a-b$ ou $a - \lambda = -2b \Leftrightarrow \lambda = a + 2b$. \\
 Pour chacune de ces valeurs on cherche les sous-espaces propres : \\
 $(M_{a,b} - (a-b) I)X = 0 \Leftrightarrow \begin{smatrix}
b & b & b \\
b & b & b \\
b & b & b \\
\end{smatrix}
\begin{smatrix}
x \\
y \\
z \\
\end{smatrix}
 = 0 \Leftrightarrow b (x + y + z) = 0 \Leftrightarrow b = 0$ ou $x =
-y-z$. \\
\\
 1er cas : $b = 0$, alors on a en fait $M_{a,b} = aI$ donc $M_{a,b}{n}
= a^{n} I$. \\
 2er cas : $b \neq 0$, on obtient alors un sous-espace propre de
dimension 2, engendré par $[ ( -1, 1, 0), (-1, 0, 1) ]$. \\
\\
 $(M_{a,b} - (a + 2b) I)X = 0 \Leftrightarrow \begin{smatrix}
-2b & b & b \\
b & -2b & b \\
b & b & -2b \\
\end{smatrix}
\begin{smatrix}
x \\
y \\
z \\
\end{smatrix}
 = 0 \Leftrightarrow \left\{ 
\begin{array}{l}
 - 2 x + y + z = 0 \\
x - 2y + z = 0 \\
x + y -2z = 0 \\
\end{array}
\right.$ (car $b \neq 0$) $\Leftrightarrow \left\{ 
\begin{array}{l}
 x + y -2z = 0 \\
- 3y + 3z = 0 \\
-2x + y + z = 0 \\
\end{array}
\right. \Leftrightarrow \left\{ 
\begin{array}{l}
 x = y \\
y = z \\
2y = 2x \\
\end{array}
\right. \Leftrightarrow (x,y,z) = x (1,1,1) $ ce qui donne une base du
sous-espace propre. \\
\\
 On obtient avec $P = \begin{smatrix}
-1 & -1 & 1 \\
1 & 0 & 1 \\
0 & 1 & 1 \\
\end{smatrix}
$ et $D = \begin{smatrix}
a-b & 0 & 0 \\
0 & a-b & 0 \\
0 & 0 & a + 2b \\
\end{smatrix}
$ que $M_{a,b} = P D P^{-1}$ puis $M_{a,b} = P D^{n} P^{-1}$ avec
$D^{n} = \begin{smatrix}
(a-b)^{n} & 0 & 0 \\
0 & (a-b)^{n} & 0 \\
0 & 0 & (a + 2b)^{n} \\
\end{smatrix}
$.

 \end{noliste}
 \end{exercice}

 \newpage

 \begin{exercice}{\it (Exercice avec préparation)}~
 \begin{noliste}{1.}
 \setlength{\itemsep}{4mm}
 \item Un estimateur d'un paramètre $\theta$ de la loi $P_{X}$ d'une
variable aléatoire $X$ dont on dispose d'un échantillon $(X_{n})$ est
une suite de variables aléatoires $(T_{n})$ où pour tout $n$, $T_{n}$
est une fonction des variables $X_{1},\ \dots\, X_{n})$. \\
\\
 Soient $a,\ b$ et $c$ trois réels strictement positifs et soit $f$ la
fonction définie sur $\R$ par : 
 
\[
 f(x) = 0 \text{ si } x < 0,\ \ \ f(x) = c \text{ si } x \in [ 0 ; a[,\
\ \ f(x) = \frac{b}{x^{4}} \text{ si } x \in [ a ; + \infty[.
\]
 \item La fonction est continue sauf éventuellement en $0$ et en $a$ et
positive, il reste à vérifier $\dint{-\infty}{+ \infty} f = 1$. \\
 Or $\dint{-\infty}{+ \infty} f = \dint{-\infty}{0} 0\ dx + \dint{0}{a}
c\ dx + \dint{a}{+ \infty} \frac{b}{x^{4}}\ dx = a c + b \dint{a}{+
\infty} \frac{1}{x^{4}}\ dx$. \\
 Cette dernière intégrale est convergente (intégrale de Riemann) et
$\dint{a}{y} \frac{1}{x^{4}}\ dx = \left[ \ \frac{-1}{3x^{3}}
\right]_{a}{y} \xrightarrow[y \rightarrow + \infty]{}
\frac{1}{3a^{3}}$. \\
 Il faut donc avoir $ a c + \frac{ b}{3 a^{3}} = 1$. \\
 De plus pour la continuité sur $\R_+ $ il faut la continuité en $a$,
qui donne $c = \frac{b}{a^{4}}$. \\
 On injecte dans la première égalité : $ \frac{b}{a^{3}} + \frac{b}{3
a^{3}} = 1$ donc $\frac{4b }{3a^{3}} = 1$, $b = \frac{3a^{3}}{4}$ et $c
= \frac{3}{4a}$. \\
\\
 On prend $a = 1$, la courbe est constante égale à 0 jusqu'à $x = 0$,
constante égale à $\frac{3}{4}$ sur $[0;1[$ et décroissante et convexe
de $\frac{3}{4}$ à 0 sur $[ 1 ; + \infty[$. \\
 \item Toutes les autres intégrales étant clairement absolument
convergentes (intégrale de 0 ou intégrale sur un segment), il reste à
vérifier que $ \frac{x^{k}}{x^{4}}$ est intégrable en $ + \infty$, ce
qui est vrai si et seulement si $4-k >1$, donc si et seulement si $k <
3$, c'est-à-dire $k \leq 2$. \\
 \item $\E(X) = c \frac{a^{2}}{2} + \frac{b}{2a^{2}} = \frac{3a}{8} +
\frac{3a}{8} = \frac{3a}{4}$. \\
 De même $\E(X^{2}) = c \frac{a^{3}}{3} + \frac{b}{a} = \frac{a^{2}}{4}
+ \frac{3 a^{2}}{4} = a^{2}$. \\
 Enfin $\V(X) = a^{2} - \frac{9}{16} a^{2} = \frac{7}{16} a^{2}$. \\
 \item Soit $(X_{n})$ une suite de variables aléatoires indépendantes
de même loi que $X$. On pose 
 
\[
 T_{n} = \frac{1}{n} \Sum{i = 1}{n} X_{i} 
\]
 \begin{noliste}{a)}
 \setlength{\itemsep}{2mm} \item $(X_{i})$ est un échantillon de la loi
de $X$ dont $a$ est un paramètre et pour tout $n$, $T_{n}$ est une
fonction de $X_{1},\ \dots\, X_{n}$ donc $(T_{n})$ est un estimateur de
$a$. \\
 \item On a $\E(T_{n}) = \E(X) = \frac{3}{4} a$ par linéarité de
l'espérance donc en posant $S_{n} = \frac{4}{3} T_{n}$, la suite
$(S_{n})$ est un estimateur sans biais de $a$. \\
 \item $R_{a} (S_{n}) = \V(S_{n})$ car $(S_{n})$ est sans biais, donc
$R_{a} (S_{n}) = \frac{16}{9} \V(T_{n}) = \frac{16}{9 n^{2}} V \left(
\Sum{i = 1}{n} X_{i} \right) = \frac{16}{9n^{2}} \Sum{i = 1}{n}
\V(X_{i})$ par indépendance des $X_{i}$, en enfin : \\
 $R_{a} (S_{n}) = \frac{16}{9 n^{2}} \times n \V(X) = \frac{16}{9 n}
\times \frac{7}{16} a^{2} = \frac{7 a^{2}}{9n}$. \\
 \end{noliste}
 \end{noliste}
 \noindent \textbf{\underline{Exercice sans préparation}} \\
\\
 Soit $A = \begin{smatrix}
1 & 2 & -2 \\
2 & 1 & -2 \\
2 & 2 & -3 \\
\end{smatrix}
$. \begin{noliste}{1.}
 \setlength{\itemsep}{4mm}
 \item $A^{2} - I = 0$.
 \item $X^{2} - 1 = (X-1) (X + 1)$ est annulateur de $A$ donc $\spc A
\subset \{ -1 ; 1\}$. \\
 $(A - I) X = 0 \Leftrightarrow X \in \Vect [ ( 1, 1, 1) ]$ et $(A + I)
X = 0 \Leftrightarrow X \in \Vect [ (1, 0, 1), ( 1, -1, 0) ]$ donc la
somme des dimensions des sous-espaces propres vaut 3, et $A$ est
diagonalisable. \\
 De plus on a $A = P D P^{-1}$ avec $P = \begin{smatrix}
1 & 1 & 1 \\
1 & 0 & -1 \\
1 & 1 & 0 \\
\end{smatrix}
$ et $D = \begin{smatrix}
1 & 0 & 0 \\
0 & -1 & 0 \\
0 & 0 & -1 \\
\end{smatrix}
$.

 \end{noliste}
 \end{exercice}

 \newpage

 

\end{document}