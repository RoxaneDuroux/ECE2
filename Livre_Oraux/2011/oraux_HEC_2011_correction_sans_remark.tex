\documentclass[11pt]{article}%
\usepackage{geometry}%
\geometry{a4paper,
  lmargin=2cm,rmargin=2cm,tmargin=2.5cm,bmargin=2.5cm}

\input{../../../macros.tex}

% \renewcommand{\thesection}{\Roman{section}.\hspace{-.3cm}}
% \renewcommand{\thesubsection}{\Alph{subsection}.\hspace{-.2cm}}

\pagestyle{fancy} %
\lhead{ECE2 \hfill Mathématiques \\} %
\chead{\hrule} %
\rhead{} %
\lfoot{} %
\cfoot{} %
\rfoot{\thepage} %

\renewcommand{\headrulewidth}{0pt}% : Trace un trait de séparation
                                    % de largeur 0,4 point. Mettre 0pt
                                    % pour supprimer le trait.

\renewcommand{\footrulewidth}{0.4pt}% : Trace un trait de séparation
                                    % de largeur 0,4 point. Mettre 0pt
                                    % pour supprimer le trait.

\setlength{\headheight}{14pt}

\title{\bf \vspace{-1.6cm} HEC 2011} %
\author{} %
\date{} %
\begin{document}
\maketitle %
\vspace{-1.2cm}\hrule %
\thispagestyle{fancy}

\vspace*{.2cm}

%%DEBUT

%%% EPR %%% HEC;
% type : oralAP; % 
% sujet : E 1; %
% annee : 2011; % 
% theme : ; %

\begin{exerciceAP}~\\
  Soit $f$ la fonction définie sur $\R$ par :
  \[
  \forall x \in \R,\ \ f(x) = \frac{e^{- \vert x \vert} }{2}.
  \]
  \begin{noliste}{1.}
    \setlength{\itemsep}{2mm}
  \item Question de cours : Rappeler la définition d'une densité de probabilité.
  \item Vérifier que $f$ est une densité de probabilité. \\ \\
    Soit $X$ une variable aléatoire définie sur un espace probabilisé
    $(\Omega , \A , P)$ dont $f$ est une densité de probabilité.
  \item 
    \begin{noliste}{a)}
    \setlength{\itemsep}{2mm} 
    \item Déterminer l'espérance $\E(X)$ de $X$.
    \item A-t-on, pour tout réel $s$, pour tout réel $t$ tel que $t \geq s$, 
      \[
      P_{[X > s]} ( [X > t]) = \Prob( [X > t - s])?
      \]
    \end{noliste}
  \item Pour tout entier $n \geq 1$ et tout réel $x$, on pose : 
    \[
    H_n(x) = \int_{-\infty}^x f(t) (1 + t e^{- n \vert t \vert} )\ dt.
    \]
    Montrer que $H_n$ est une fonction de répartition.
  \item Soit $X_n$ une variable aléatoire définie sur $(\Omega , \A ,
    P)$ de fonction de répartition $H_n$. Montrer que la suite
    $(X_n)_{n \in \N^*}$ converge en loi vers $X$.
  \end{noliste}
\end{exerciceAP}

%%% EPR %%% HEC;
% type : oralSP; % 
% sujet : E 1; %
% annee : 2011; % 
% theme : ; %

\begin{exerciceSP}~\\
  Soit $n$ un entier supérieur ou égal à 2 et $(a_1 , a_2,\ \dots\ ,
  a_n) \in \R^n - \{ 0 ,\ \dots\ ,0) \}$.\\
  On considère la matrice colonne $X = 
  \begin{smatrix} 
    a_1 \\ 
    a_2 \\
    \vdots \\ 
    a_n 
  \end{smatrix} \in \mathcal{M}_{n,1} (\R)$.\\
  On pose $B = X\ {}^tX $ et $A=\ {}^tX\ X$. \\
  On désigne par $u$ l'endomorphisme de $\R^n$ canoniquement associé à
  $B$.
  \begin{noliste}{1.}
    \setlength{\itemsep}{2mm}
  \item Expliciter la matrice $B$ et la matrice $A$. 
  \item Quel est le rang de $u$? Déterminer son noyau.
  \item $B$ est-elle diagonalisable?
  \item Calculer $B^k$ pour tout $k \in \N^*$.
  \end{noliste}
\end{exerciceSP}


%\newpage


%%% EPR %%% HEC;
% type : oralAP; % 
% sujet : E 2; %
% annee : 2011; % 
% theme : ; %

\begin{exerciceAP}~\\
  On admet la propriété $(\Prob)$ suivante : \\
  Si la suite réelle $(u_n)_{n \in \N}$ converge vers le nombre réel
  $L$, alors la suite $(V_n)_{n \in \N}$ définie par :
  \[
  \forall n \in \N, V_n = \frac{1}{n} (u_0 + u_1 + \dots u_{n-1} )
  \]
  converge aussi vers $L$. \\
  On se donne deux nombres réels $\alpha$ et $\beta$ tels que $0 <
  \alpha < \beta$. \\
  Soit $(u_n)_{n \in \N}$ la suite réelle définie par :
  \[
  u_0 > 0 \text{ et } \forall n \in \N,\ u_{n+1} = u_n \frac{1 +
    \alpha u_n}{1 + \beta u_n}
  \]
  \begin{noliste}{1.}
    \setlength{\itemsep}{2mm}
  \item Question de cours : Convergence et divergence des suites
    réelles monotones.
  \item Dans cette question seulement, on suppose $\alpha = 1$ et
    $\beta = 2$.
    \begin{noliste}{a)}
    \setlength{\itemsep}{2mm} 
    \item Étudier les variations de la fonction $f$ définie sur
      $\R_{+}$ par :
      \[
      f(x) = x \frac{1+x}{1+2x} 
      \]
    \item Étudier la convergence de la suite $(u_n)$.
    \item Écrire un programme en Pascal permettant le calcul de $u_{10}$.
    \end{noliste}
  \item Dans le cas général, prouver que la suite $(u_n)_{n \in \N}$
    converge et donner sa limite.
  \item On pose, pour tout $n \in \N$, $v_n = \frac{1}{u_n}$. Prouver
    que la suite $(v_{n+1} - v_n)_{n \in \N}$ converge vers $\beta -
    \alpha$.
  \item En utilisant la propriété $\Prob$, déduire du résultat
    précédent un équivalent de $u_n$ de la forme $\frac{1}{q n}$
    lorsque $n$ tend vers $+\infty$, où $q$ est un réel strictement
    positif.
  \end{noliste}
\end{exerciceAP}

%%% EPR %%% HEC;
% type : oralSP; % 
% sujet : E 2; %
% annee : 2011; % 
% theme : ; %

\begin{exerciceSP}~\\
  $n$ souris (minimum 3) sont lâchées en direction de 3 cages, chaque
  cage pouvant contenir les $n$ souris et chaque souris allant dans
  une cage au hasard. 
  \begin{noliste}{1.}
    \setlength{\itemsep}{2mm}
  \item Calculer la probabilité pour qu'une cage au moins reste vide.
  \item Soit $X$ la variable aléatoire égale au nombre de cages
    restées vides. Calculer l'espérance de $X$. 
  \end{noliste}
\end{exerciceSP}


%\newpage


%%% EPR %%% HEC;
% type : oralAP; % 
% sujet : E 3; %
% annee : 2011; % 
% theme : ; %

\begin{exerciceAP}~
  \begin{noliste}{1.}
    \setlength{\itemsep}{2mm}
  \item Question de cours : Variable aléatoire à densité. Propriétés
    de sa fonction de répartition.\\[.2cm]
    On considère une densité de probabilité $f$, nulle sur $\R_{-}$,
    continue sur $\R_+^*$, associée à une variable aléatoire
    $X$. On suppose que $X$ est définie sur un espace probabilisé
    $(\Omega , \A , P)$ et on note $F$ la fonction de
    répartition de $X$.\\
  \item Montrer que $F$ possède une unique primitive s'annulant en
    0. On note $H_f$ cette fonction. \\
    Montrer que $H_f$ est de classe $C^1$.
  \item Donner $H_f$ dans les cas suivants :
    \begin{noliste}{a)}
    \setlength{\itemsep}{2mm} 
    \item $f(x) = 0$ si $x \leq 0$ et $f(x) = e^{-x}$ si $x > 0$.
    \item $f(x)= 0$ si $x \leq 0$ et $f(x) = \frac{1}{(1+x)^2}$ si $x
      > 0$.
    \item $f(x) = 0$ si $x \leq 0$ et $f(x) = \frac{1}{2(1+x)^{3
          \slash 2} }$ si $x > 0$.
    \end{noliste}
    Dans chacune des cas, étudier l'existence d'une direction
    asymptotique et d'une asymptote oblique pour la courbe
    représentative de $H_f$ lorsque $x$ tend vers $+\infty$.
  \item On suppose que $X$ admet une espérance $l$.
    \begin{noliste}{a)}
    \setlength{\itemsep}{2mm}
    \item En intégrant par parties $\dint{0}{x} t f(t)\ dt$, montrer
      que $H_f(x) \sim x$ au voisinage de $+\infty$.\\
      En déduire que la courbe représentative de $H_f$ admet une
      direction asymptotique en $+\infty$.
    \item A-t-on toujours une asymptote ? 
    \end{noliste}
  \end{noliste}
\end{exerciceAP}

%%% EPR %%% HEC;
% type : oralSP; % 
% sujet : E 3; %
% annee : 2011; % 
% theme : ; %

\begin{exerciceSP}~\\
  Soit $E$ l'ensemble des matrices $M_{a,b} = 
  \begin{smatrix} 
    a & b & b \\ 
    b & a & b \\ 
    b & b & a
  \end{smatrix}$ où $(a,b)$ prend toute valeur de $\R^2$.
  \begin{noliste}{1.}
    \setlength{\itemsep}{2mm}
  \item Montrer que $E$ est un espace vectoriel réel de dimension 2. \\
    Calculer le produit $M_{a,b} M_{a',b'}$ pou $(a,b,a',b') \in \R^4$. \\
    Vérifier que ce produit appartient à $E$.
  \item Calculer $M_{a,b}^n$ pour $n \in \N^*$.
  \end{noliste}
\end{exerciceSP}


%\newpage


%%% EPR %%% HEC;
% type : oralAP; %
% sujet : E 4; %
% annee : 2011; %
% theme : vaPartieEntiere, vaGeometrique, probaLimMonotone, vaIndep; %

\begin{exerciceAP}~\\
  Toutes les variables aléatoires de cet exercice sont définies sur un
  espace probabilisé $(\Omega , \A , \Prob)$. Soit $p \in \ ]0, 1[$ et
  $q = 1-p$.
  \begin{noliste}{1.}
    \setlength{\itemsep}{2mm}
  \item Question de cours : Indépendance de $n$ variables aléatoires
    discrètes $(n \in \N^*)$.
  \item On effectue des lancers successifs et indépendants d'une pièce
    de monnaie. On suppose qu'à chaque lancer la probabilité d'obtenir
    Pile est égale à $p$. On notera $P$ et $F$ les évènements \og
    Obtenir Pile \fg{} et \og Obtenir Face \fg{}.\\
    On définit les variables aléatoires $X_1$ et $X_2$ de la façon
    suivante :
    \begin{noliste}{$\stimes$}
    \item $X_1$ vaut $k$ si le premier Pile de rang impair s'obtient
      au rang $2k-1$ (entier qui représente le $\eme{k}$ nombre impair
      de $\N^*$),
    \item $X_2$ vaut $k$ si le premier Pile de rang pair s'obtient au
      rang $2k$ (entier qui représente le $k$-ième nombre pair de
      $\N^*$.
    \end{noliste}
    Par exemple si l'on obtient $(F , P , F , F , F , P , P)$ alors
    $X_1$ prend la valeur 4 et $X_2$ prend la valeur 1.\\
    On posera $X_1 = 0$ (respectivement $X_2=0$) si Pile n'apparaît à
    aucun rang impair (respectivement à aucun rang pair).
    \begin{noliste}{a)}
      \setlength{\itemsep}{2mm}
    \item Prouver que $\Prob(\Ev{X_1 = 0}) = \Prob(\Ev{X_2 = 0}) = 0$.
    \item Calculer $\Prob(\Ev{X_1 = 1})$ et $\Prob(\Ev{X_2 =
        1})$. Déterminer les lois de $X_1$ et de $X_2$. Les variables
      aléatoires $X_1$ et $X_2$ sont-elles indépendantes ?
    \item Déterminer la loi de la variable aléatoire $Y$ égale au
      minimum de $X_1$ et de $X_2$.
    \end{noliste}
  \item Soit $X$ une variable aléatoire suivant une loi géométrique de
    paramètre $p$.
    \begin{noliste}{a)}
      \setlength{\itemsep}{2mm}
    \item Montrer que la variable aléatoire $Y = \left\lfloor
        \dfrac{X+1}{2} \right\rfloor$ suit une loi géométrique
      ($\lfloor x \rfloor$ désigne la partie entière du nombre $x$).
    \item Montrer que les variables aléatoires $Y$ et $2Y - X$ sont
      indépendantes.
    \end{noliste}
  \end{noliste}
\end{exerciceAP}
  

%%% EPR %%% HEC;
% type : oralSP; % 
% sujet : E 4; %
% annee : 2011; % 
% theme : ; %

\begin{exerciceSP}~\\
  On note $E_4$ l'espace vectoriel des fonctions polynomiales de degré
  inférieur ou égal à 4 et on considère l'application $\Delta$ qui à
  un polynôme $P$ de $E_4$ associe le polynôme $Q = \Delta (P)$ défini
  par :
  \[
  Q(x) = P(x+2) - P(x)
  \]
  \begin{noliste}{1.}
    \setlength{\itemsep}{2mm}
  \item Vérifier que l'application $\Delta$ est un endomorphisme de $E_4$. \\
    Expliciter la matrice de $\Delta$ dans la base canonique de $E_4$.
  \item Déterminer le noyau de $\Delta$. On pourra prouver que si $P
    \in \ker(\Delta)$, alors $P(x) - P(0)$ a une infinité de racines.
  \item L'endomorphisme $\Delta$ est-il diagonalisable?
  \item Existe-t-il un polynôme $Q$ appartenant à $E_4$ ayant un
    unique antécédent par $\Delta$?
  \end{noliste}
\end{exerciceSP}


%\newpage


%%% EPR %%% HEC;
% type : oralAP; % 
% sujet : E 5; %
% annee : 2011; % 
% theme : ; %

\begin{exerciceAP}~\\
  Dans tout l'exercice, $n$ désigne un entier naturel non nul et
  $\R_n[X]$ l'espace vectoriel des polynômes à coefficients réels, de
  degré inférieur ou égal à $n$. On note $M (m_{i,j})_{1 \leq i,j \leq
    n+1}$ la matrice de $\mathcal{M}_{n+1} (\R)$ de terme général :
  \[
  m_{i,j} = \left\{ 
    \begin{array}{cc} 
      i & \text{ si } j = i+1 
      \\[.2cm] 
      n + 1 - j & \text{ si } i = j + 1 
      \\[.2cm]
      0 & \text{ dans tous les autres cas } 
    \end{array} 
  \right. 
  \]
  et $u$ l'endomorphisme de $\R_n[X]$ dont la matrice dans la base
  canonique $(1 , X ,\ \dots\ , X^n)$ est égale à $M$.
  \begin{noliste}{1.}
    \setlength{\itemsep}{2mm}
  \item Question de cours : Rappeler la définition d'un vecteur propre
    d'un endomorphisme. \\
    Enoncer la propriété relative à une famille de vecteurs propres
    d'un endomorphisme, associés à des valeurs propres distinctes.
  \item
    \begin{noliste}{a)}
    \setlength{\itemsep}{2mm} \item Calculer $u(X^k)$ pour $k \in \llb 0 , n
      \rrb$.
    \item En déduire l'expression de $u(P)$ pour $P \in E$ en fonction
      notamment de $P$ et de $P'$.
    \end{noliste}
  \item Pour $k \in \llb 0 , n \rrb$, on pose $P_k (X) = (X-1)^k
    (X+1)^{n-k}$.
    \begin{noliste}{a)}
    \setlength{\itemsep}{2mm} \item Calculer $u(P_k)$.
    \item En déduire que $P_0 ,\ \dots\ , P_n)$ est une base de
      $\R^n[X]$.
    \item L'endomorphisme $u$ est-il diagonalisable? Préciser ses
      valeurs propres et les espaces propres associés.
    \end{noliste}
  \item Dans cette question, on suppose que $n = 3$.
    \begin{noliste}{a)}
    \setlength{\itemsep}{2mm} 
    \item Expliciter $M$ et déterminer une matrice diagonale $D$ et
      une matrice inversible $P$ telles que $P^{-1} M P = D$.
    \item Déterminer les matrices commutant avec $D$.
    \item Existe-t-il un endomorphisme $v$ de $\R_3[X]$ tel que $v
      \circ v = u$ ?
    \end{noliste}
  \end{noliste}
\end{exerciceAP}

%%% EPR %%% HEC;
% type : oralSP; %
% sujet : E 5; %
% annee : 2011; %
% theme : vaGeometrique, vaSomme, FPT, proba2va; %

\begin{exerciceSP}~\\
  Soient $X$ et $Y$ deux variables aléatoires définies sur un espace
  probabilisé $(\Omega , \A , P)$ à valeurs dans $\N^*$,
  indépendantes et telles que :
  \[
  \forall i \in \N^*, \ \Prob(\Ev{X=i}) = \Prob(\Ev{Y=i}) =
  \dfrac{1}{2^i}
  \] 
  \begin{noliste}{1.}
    \setlength{\itemsep}{2mm}
  \item Reconnaître la loi de $X$ et de $Y$.
  \item Déterminer la loi de la variable aléatoire $Z = X+Y$ et la loi
    de $X$ conditionnellement à $\Ev{X + Y = k}$, $k$ étant un entier
    supérieur ou égal à 2 fixé.
  \item Calculer $\Prob(\Ev{X = Y})$ et $\Prob(\Ev{X > Y})$.
  \item Calculer $\Prob(\Ev{X \geq 2 Y})$ et $\Prob_{\Ev{X \geq Y}}
    (\Ev{X \geq 2 Y})$.
  \end{noliste}
\end{exerciceSP}


%\newpage



%%% EPR %%% HEC;
% type : oralAP; % 
% sujet : E 6; %
% annee : 2011; % 
% theme : ; %

\begin{exerciceAP}~\\
  Soit $(X_n)_{n \in \N}$ une suite de variables aléatoires
  indépendantes définie sur un espace probabilisé $(\Omega ,
  \A , P)$ telles que, pour tout $n \in \N^*$, $X_n$
  suit la loi exponentielle de paramètre $\frac{1}{n}$ (d'espérance
  $n$).\\
  Pour tout $x$ réel on note $\lfloor x \rfloor$ sa partie entière. \\
  Pour $n \in \N^*$ soient :
  \[
  Y_n = \lfloor X_n \rfloor \ \ \ \text{ et } \ \ \ Z_n = X_n -
  \lfloor X_n \rfloor
  \]
  \begin{noliste}{1.}
    \setlength{\itemsep}{2mm}
  \item Question de cours : Définition de la convergence en loi d'une
    suite de variables aléatoires.
  \item Déterminer la loi de $Y_n$ et son espérance.
  \item Déterminer $Z_n (\Omega)$ et montrer que,
    \[
    \forall t \in [ 0 , 1]\ :\ \ \ \Prob(Z_n \leq t) = \frac{ 1 -
      e^{\frac{-t}{n} } }{1 - e^{\frac{-1}{n}}}.
    \]
  \item Montrer que la suite de variables aléatoires $(Z_n)_{ n \in
      \N}$ converge en loi vers une variable aléatoire $Z$ dont on
    précisera la loi.
  \item Soit $n \in \N^*$ et $N_n$ la variable aléatoire définie
    par :
    \[
    N_n = \Card \left\{ k \in \llb 1 , n \rrb \text{ tel que } X_k
      \leq \frac{k}{n} \right\}
    \]
    où $\Card (A)$ désigne le nombre d'éléments de l'ensemble fini $A$.
    \begin{noliste}{a)}
    \setlength{\itemsep}{2mm} \item Reconnaître la loi de $N_n$ et donner
      son espérance et sa variance.
    \item Montrer que la suite de variables aléatoires $(N_n)_{n \in
        \N}$ converge en loi vers une variable aléatoire $N$ dont on
      précisera la loi.
    \end{noliste} 
  \end{noliste}
\end{exerciceAP}

%%% EPR %%% HEC;
% type : oralSP; % 
% sujet : E 6; %
% annee : 2011; % 
% theme : ; %

\begin{exerciceSP}~\\
  Soit $E$ l'ensemble des matrices $M_{a,b} = 
  \begin{smatrix} 
    a & b & b \\ 
    b & a & b \\ 
    b & b & a 
  \end{smatrix}$ où $(a,b)$ prend toute valeur de $\R^2$.
  \begin{noliste}{1.}
    \setlength{\itemsep}{2mm}
  \item Montrer que $E$ est un espace vectoriel de dimension 2.
  \item Dans le cas où soit $a = b$, soit $a = -2b$, prouver que
    $M_{a,b}$ n'est pas inversible. \\
    Dans le cas contraire, calculer son inverse et montrer qu'il
    appartient à $E$.
  \item Calculer $M_{a,b}^n$ pour $n \in \N^*$.
  \end{noliste}
\end{exerciceSP}


%\newpage


%%% EPR %%% HEC;
% type : oralAP; % 
% sujet : E 7; %
% annee : 2011; % 
% theme : ; %

\begin{exerciceAP}~
  \begin{noliste}{1.}
    \setlength{\itemsep}{2mm}
  \item Question de cours : Estimateur, biais, risque quadratique. \\[.2cm]
    Soient $a,\ b$ et $c$ trois réels strictement positifs et soit $f$
    la fonction définie sur $\R$ par :
    \[
    f(x) = 0 \text{ si } x < 0,\ \ \ f(x) = c \text{ si } x \in [ 0 ,
    a[,\ \ \ f(x) = \frac{b}{x^4} \text{ si } x \in [ a , +\infty[.
    \]
  \item Déterminer $b$ et $c$ en fonction de $a$ pour que $f$ soit une
    densité de probabilité continue sur $\R_+$.\\
    On suppose que $b$ et $c$ sont ainsi définis dans la suite de
    l'exercice et $X$ est une variable aléatoire définie sur un espace
    probabilisé $(\Omega , \A , P)$ de densité $f$. \\ 
    Donner une allure de la représentation graphique de $f$.
  \item Pour quelles valeurs $k \in \N^*$, $X$ admet-elle un moment d'ordre $k$?
  \item Déterminer l'espérance et la variance de $X$ si elles existent.
  \item Soit $(X_n)$ une suite de variables aléatoires indépendantes
    de même loi que $X$. On pose :
    \[ 
    T_n = \dfrac{1}{n} \ \Sum{i=1}{n} X_i
    \]
    \begin{noliste}{a)}
    \setlength{\itemsep}{2mm} 
    \item Montrer que $(T_n)$ est un estimateur de $a$.
    \item Construire à partir de $(T_n)$ un estimateur $(S_n)$ sans
      biais de $a$.
    \item Quel est le risque quadratique de $S_n$?
    \end{noliste}
  \end{noliste}
\end{exerciceAP}

%%% EPR %%% HEC;
% type : oralSP; % 
% sujet : E 7; %
% annee : 2011; % 
% theme : ; %

\begin{exerciceSP}~\\
  Soit $A = 
  \begin{smatrix} 
    1 & 2 & -2 \\ 
    2 & 1 & -2 \\ 
    2 & 2 & -3
  \end{smatrix}$.
  \begin{noliste}{1.}
    \setlength{\itemsep}{2mm}
  \item Calculer $A^2 - I$.
  \item $A$ est-elle diagonalisable? Si oui, la diagonaliser.

  \end{noliste}
\end{exerciceSP}

%%FIN

\end{document}
