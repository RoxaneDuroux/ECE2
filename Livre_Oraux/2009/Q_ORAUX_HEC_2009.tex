\documentclass[11pt]{article}%
\usepackage{geometry}%
\geometry{a4paper,
 lmargin = 2cm,rmargin = 2cm,tmargin = 2.5cm,bmargin = 2.5cm}

\usepackage{array}
\usepackage{paralist}

\usepackage[svgnames, usenames, dvipsnames]{xcolor}
\xdefinecolor{RecColor}{named}{Aqua}
\xdefinecolor{IncColor}{named}{Aqua}
\xdefinecolor{ImpColor}{named}{PaleGreen}

% \usepackage{frcursive}

\usepackage{adjustbox}

%%%%%%%%%%%
\newcommand{\cRB}[1]{{\color{Red} \pmb{#1}}} %
\newcommand{\cR}[1]{{\color{Red} {#1}}} %
\newcommand{\cBB}[1]{{\color{Blue} \pmb{#1}}}
\newcommand{\cB}[1]{{\color{Blue} {#1}}}
\newcommand{\cGB}[1]{{\color{LimeGreen} \pmb{#1}}}
\newcommand{\cG}[1]{{\color{LimeGreen} {#1}}}

%%%%%%%%%%

\usepackage{diagbox} %
\usepackage{colortbl} %
\usepackage{multirow} %
\usepackage{pgf} %
\usepackage{environ} %
\usepackage{fancybox} %
\usepackage{textcomp} %
\usepackage{marvosym} %

%%%%%%%%%% pour pouvoir faire des dashedline dans les tableaux
\usepackage{arydshln}

%%%%%%%%%% pour qu'une cellcolor ne recouvre pas le trait du tableau
\usepackage{hhline}%

\usepackage{pgfplots}
\pgfplotsset{compat=1.10}
\usepgfplotslibrary{patchplots}
\usepgfplotslibrary{fillbetween}
\usepackage{tikz,tkz-tab}
\usepackage{ifthen}
\usepackage{calc}
\usetikzlibrary{calc,decorations.pathreplacing,arrows,positioning} 
\usetikzlibrary{fit,shapes,backgrounds}
% \usepackage[nomessages]{fp}% http://ctan.org/pkg/fp

\usetikzlibrary{matrix,arrows,decorations.pathmorphing,
  decorations.pathreplacing} 

\newcommand{\myunit}{1 cm}
\tikzset{
    node style sp/.style={draw,circle,minimum size=\myunit},
    node style ge/.style={circle,minimum size=\myunit},
    arrow style mul/.style={draw,sloped,midway,fill=white},
    arrow style plus/.style={midway,sloped,fill=white},
}

%%%%%%%%%%%%%%
%%%%% écrire des inférieur égal ou supérieur égal avec typographie
%%%%% francaise
%%%%%%%%%%%%%

\renewcommand{\geq}{\geqslant}
\renewcommand{\leq}{\leqslant}
\renewcommand{\emptyset}{\varnothing}

\newcommand{\Leq}{\leqslant}
\newcommand{\Geq}{\geqslant}

%%%%%%%%%%%%%%
%%%%% Macro Celia
%%%%%%%%%%%%%

\newcommand{\ff}[2]{\left[#1, #2\right]} %
\newcommand{\fo}[2]{\left[#1, #2\right[} %
\newcommand{\of}[2]{\left]#1, #2\right]} %
\newcommand{\soo}[2]{\left]#1, #2\right[} %
\newcommand{\abs}[1]{\left|#1\right|} %
\newcommand{\Ent}[1]{\left\lfloor #1 \right\rfloor} %


%%%%%%%%%%%%%%
%%%%% tikz : comment dessiner un "oeil"
%%%%%%%%%%%%%

\newcommand{\eye}[4]% size, x, y, rotation
{ \draw[rotate around={#4:(#2,#3)}] (#2,#3) -- ++(-.5*55:#1) (#2,#3)
  -- ++(.5*55:#1); \draw (#2,#3) ++(#4+55:.75*#1) arc
  (#4+55:#4-55:.75*#1);
  % IRIS
  \draw[fill=gray] (#2,#3) ++(#4+55/3:.75*#1) arc
  (#4+180-55:#4+180+55:.28*#1);
  % PUPIL, a filled arc
  \draw[fill=black] (#2,#3) ++(#4+55/3:.75*#1) arc
  (#4+55/3:#4-55/3:.75*#1);%
}


%%%%%%%%%%
%% discontinuité fonction
\newcommand\pointg[2]{%
  \draw[color = red, very thick] (#1+0.15, #2-.04)--(#1, #2-.04)--(#1,
  #2+.04)--(#1+0.15, #2+.04);%
}%

\newcommand\pointd[2]{%
  \draw[color = red, very thick] (#1-0.15, #2+.04)--(#1, #2+.04)--(#1,
  #2-.04)--(#1-0.15, #2-.04);%
}%

%%%%%%%%%%
%%% 1 : position abscisse, 2 : position ordonnée, 3 : taille, 4 : couleur
%%%%%%%%%%
% \newcommand\pointG[4]{%
%   \draw[color = #4, very thick] (#1+#3, #2-(#3/3.75))--(#1,
%   #2-(#3/3.75))--(#1, #2+(#3/3.75))--(#1+#3, #2+(#3/3.75)) %
% }%

\newcommand\pointG[4]{%
  \draw[color = #4, very thick] ({#1+#3/3.75}, {#2-#3})--(#1,
  {#2-#3})--(#1, {#2+#3})--({#1+#3/3.75}, {#2+#3}) %
}%

\newcommand\pointD[4]{%
  \draw[color = #4, very thick] ({#1-#3/3.75}, {#2+#3})--(#1,
  {#2+#3})--(#1, {#2-#3})--({#1-#3/3.75}, {#2-#3}) %
}%

\newcommand\spointG[4]{%
  \draw[color = #4, very thick] ({#1+#3/1.75}, {#2-#3})--(#1,
  {#2-#3})--(#1, {#2+#3})--({#1+#3/1.75}, {#2+#3}) %
}%

\newcommand\spointD[4]{%
  \draw[color = #4, very thick] ({#1-#3/2}, {#2+#3})--(#1,
  {#2+#3})--(#1, {#2-#3})--({#1-#3/2}, {#2-#3}) %
}%

%%%%%%%%%%

\newcommand{\Pb}{\mathtt{P}}

%%%%%%%%%%%%%%%
%%% Pour citer un précédent item
%%%%%%%%%%%%%%%
\newcommand{\itbf}[1]{{\small \bf \textit{#1}}}


%%%%%%%%%%%%%%%
%%% Quelques couleurs
%%%%%%%%%%%%%%%

\xdefinecolor{cancelcolor}{named}{Red}
\xdefinecolor{intI}{named}{ProcessBlue}
\xdefinecolor{intJ}{named}{ForestGreen}

%%%%%%%%%%%%%%%
%%%%%%%%%%%%%%%
% barrer du texte
\usetikzlibrary{shapes.misc}

\makeatletter
% \definecolor{cancelcolor}{rgb}{0.127,0.372,0.987}
\newcommand{\tikz@bcancel}[1]{%
  \begin{tikzpicture}[baseline=(textbox.base),inner sep=0pt]
  \node[strike out,draw] (textbox) {#1}[thick, color=cancelcolor];
  \useasboundingbox (textbox);
  \end{tikzpicture}%
}
\newcommand{\bcancel}[1]{%
  \relax\ifmmode
    \mathchoice{\tikz@bcancel{$\displaystyle#1$}}
               {\tikz@bcancel{$\textstyle#1$}}
               {\tikz@bcancel{$\scriptstyle#1$}}
               {\tikz@bcancel{$\scriptscriptstyle#1$}}
  \else
    \tikz@bcancel{\strut#1}%
  \fi
}
\newcommand{\tikz@xcancel}[1]{%
  \begin{tikzpicture}[baseline=(textbox.base),inner sep=0pt]
  \node[cross out,draw] (textbox) {#1}[thick, color=cancelcolor];
  \useasboundingbox (textbox);
  \end{tikzpicture}%
}
\newcommand{\xcancel}[1]{%
  \relax\ifmmode
    \mathchoice{\tikz@xcancel{$\displaystyle#1$}}
               {\tikz@xcancel{$\textstyle#1$}}
               {\tikz@xcancel{$\scriptstyle#1$}}
               {\tikz@xcancel{$\scriptscriptstyle#1$}}
  \else
    \tikz@xcancel{\strut#1}%
  \fi
}
\makeatother

\newcommand{\xcancelRA}{\xcancel{\rule[-.15cm]{0cm}{.5cm} \Rightarrow
    \rule[-.15cm]{0cm}{.5cm}}}

%%%%%%%%%%%%%%%%%%%%%%%%%%%%%%%%%%%
%%%%%%%%%%%%%%%%%%%%%%%%%%%%%%%%%%%

\newcommand{\vide}{\multicolumn{1}{c}{}}

%%%%%%%%%%%%%%%%%%%%%%%%%%%%%%%%%%%
%%%%%%%%%%%%%%%%%%%%%%%%%%%%%%%%%%%


\usepackage{multicol}
% \usepackage[latin1]{inputenc}
% \usepackage[T1]{fontenc}
\usepackage[utf8]{inputenc}
\usepackage[T1]{fontenc}
\usepackage[normalem]{ulem}
\usepackage[french]{babel}

\usepackage{url}    
\usepackage{hyperref}
\hypersetup{
  backref=true,
  pagebackref=true,
  hyperindex=true,
  colorlinks=true,
  breaklinks=true,
  urlcolor=blue,
  linkcolor=red,
  bookmarks=true,
  bookmarksopen=true
}

%%%%%%%%%%%%%%%%%%%%%%%%%%%%%%%%%%%%%%%%%%%
%% Pour faire des traits diagonaux dans les tableaux
%% Nécessite slashbox.sty
%\usepackage{slashbox}

\usepackage{tipa}
\usepackage{verbatim,listings}
\usepackage{graphicx}
\usepackage{fancyhdr}
\usepackage{mathrsfs}
\usepackage{pifont}
\usepackage{tablists}
\usepackage{dsfont,amsfonts,amssymb,amsmath,amsthm,stmaryrd,upgreek,manfnt}
\usepackage{enumerate}

%\newcolumntype{M}[1]{p{#1}}
\newcolumntype{C}[1]{>{\centering}m{#1}}
\newcolumntype{R}[1]{>{\raggedright}m{#1}}
\newcolumntype{L}[1]{>{\raggedleft}m{#1}}
\newcolumntype{P}[1]{>{\raggedright}p{#1}}
\newcolumntype{B}[1]{>{\raggedright}b{#1}}
\newcolumntype{Q}[1]{>{\raggedright}t{#1}}

\newcommand{\alias}[2]{
\providecommand{#1}{}
\renewcommand{#1}{#2}
}
\alias{\R}{\mathbb{R}}
\alias{\N}{\mathbb{N}}
\alias{\Z}{\mathbb{Z}}
\alias{\Q}{\mathbb{Q}}
\alias{\C}{\mathbb{C}}
\alias{\K}{\mathbb{K}}

%%%%%%%%%%%%
%% rendre +infty et -infty plus petits
%%%%%%%%%%%%
\newcommand{\sinfty}{{\scriptstyle \infty}}

%%%%%%%%%%%%%%%%%%%%%%%%%%%%%%%
%%%%% macros TP Scilab %%%%%%%%
\newcommand{\Scilab}{\textbf{Scilab}} %
\newcommand{\Scinotes}{\textbf{SciNotes}} %
\newcommand{\faire}{\noindent $\blacktriangleright$ } %
\newcommand{\fitem}{\scalebox{.8}{$\blacktriangleright$}} %
\newcommand{\entree}{{\small\texttt{ENTRÉE}}} %
\newcommand{\tab}{{\small\texttt{TAB}}} %
\newcommand{\mt}[1]{\mathtt{#1}} %
% guillemets droits

\newcommand{\ttq}{\textquotesingle} %

\newcommand{\reponse}[1]{\longboxed{
    \begin{array}{C{0.9\textwidth}}
      \nl[#1]
    \end{array}
  }} %

\newcommand{\reponseL}[2]{\longboxed{
    \begin{array}{C{#2\textwidth}}
      \nl[#1]
    \end{array}
  }} %

\newcommand{\reponseR}[1]{\longboxed{
    \begin{array}{R{0.9\textwidth}}
      #1
    \end{array}
  }} %

\newcommand{\reponseRL}[2]{\longboxed{
    \begin{array}{R{#2\textwidth}}
      #1
    \end{array}
  }} %

\newcommand{\reponseC}[1]{\longboxed{
    \begin{array}{C{0.9\textwidth}}
      #1
    \end{array}
  }} %

\colorlet{pyfunction}{Blue}
\colorlet{pyCle}{Magenta}
\colorlet{pycomment}{LimeGreen}
\colorlet{pydoc}{Cyan}
% \colorlet{SansCo}{white}
% \colorlet{AvecCo}{black}

\newcommand{\visible}[1]{{\color{ASCo}\colorlet{pydoc}{pyDo}\colorlet{pycomment}{pyCo}\colorlet{pyfunction}{pyF}\colorlet{pyCle}{pyC}\colorlet{function}{sciFun}\colorlet{var}{sciVar}\colorlet{if}{sciIf}\colorlet{comment}{sciComment}#1}} %

%%%% à changer ????
\newcommand{\invisible}[1]{{\color{ASCo}\colorlet{pydoc}{pyDo}\colorlet{pycomment}{pyCo}\colorlet{pyfunction}{pyF}\colorlet{pyCle}{pyC}\colorlet{function}{sciFun}\colorlet{var}{sciVar}\colorlet{if}{sciIf}\colorlet{comment}{sciComment}#1}} %

\newcommand{\invisibleCol}[2]{{\color{#1}#2}} %

\NewEnviron{solution} %
{ %
  \Boxed{
    \begin{array}{>{\color{ASCo}} R{0.9\textwidth}}
      \colorlet{pycomment}{pyCo}
      \colorlet{pydoc}{pyDo}
      \colorlet{pyfunction}{pyF}
      \colorlet{pyCle}{pyC}
      \colorlet{function}{sciFun}
      \colorlet{var}{sciVar}
      \colorlet{if}{sciIf}
      \colorlet{comment}{sciComment}
      \BODY
    \end{array}
  } %
} %

\NewEnviron{solutionC} %
{ %
  \Boxed{
    \begin{array}{>{\color{ASCo}} C{0.9\textwidth}}
      \colorlet{pycomment}{pyCo}
      \colorlet{pydoc}{pyDo}
      \colorlet{pyfunction}{pyF}
      \colorlet{pyCle}{pyC}
      \colorlet{function}{sciFun}
      \colorlet{var}{sciVar}
      \colorlet{if}{sciIf}
      \colorlet{comment}{sciComment}
      \BODY
    \end{array}
  } %
} %

\newcommand{\invite}{--\!\!>} %

%%%%% nouvel environnement tabular pour retour console %%%%
\colorlet{ConsoleColor}{Blue!15}
\colorlet{function}{Red}
\colorlet{var}{Maroon}
\colorlet{if}{Magenta}
\colorlet{comment}{LimeGreen}

\newcommand{\tcVar}[1]{\textcolor{var}{\bf \small #1}} %
\newcommand{\tcFun}[1]{\textcolor{function}{#1}} %
\newcommand{\tcIf}[1]{\textcolor{if}{#1}} %
\newcommand{\tcFor}[1]{\textcolor{if}{#1}} %

\newcommand{\moins}{\!\!\!\!\!\!- }
\newcommand{\espn}{\!\!\!\!\!\!}

\usepackage{booktabs,varwidth} \newsavebox\TBox
\newenvironment{console}
{\begin{lrbox}{\TBox}\varwidth{\linewidth}
    \tabular{>{\tt\small}R{0.84\textwidth}}
    \nl[-.4cm]} {\endtabular\endvarwidth\end{lrbox}%
  \fboxsep=1pt\colorbox{ConsoleColor}{\usebox\TBox}}

\newcommand{\lInv}[1]{%
  $\invite$ #1} %

\newcommand{\lAns}[1]{%
  \qquad ans \ = \nl %
  \qquad \qquad #1} %

\newcommand{\lVar}[2]{%
  \qquad #1 \ = \nl %
  \qquad \qquad #2} %

\newcommand{\lDisp}[1]{%
  #1 %
} %

\newcommand{\ligne}[1]{\underline{\small \tt #1}} %

\newcommand{\ligneAns}[2]{%
  $\invite$ #1 \nl %
  \qquad ans \ = \nl %
  \qquad \qquad #2} %

\newcommand{\ligneVar}[3]{%
  $\invite$ #1 \nl %
  \qquad #2 \ = \nl %
  \qquad \qquad #3} %

\newcommand{\ligneErr}[3]{%
  $\invite$ #1 \nl %
  \quad !-{-}error #2 \nl %
  #3} %
%%%%%%%%%%%%%%%%%%%%%% 

\newcommand{\bs}[1]{\boldsymbol{#1}} %
\newcommand{\nll}{\nl[.4cm]} %
\newcommand{\nle}{\nl[.2cm]} %
%% opérateur puissance copiant l'affichage Scilab
%\newcommand{\puis}{\!\!\!~^{\scriptscriptstyle\pmb{\wedge}}}
\newcommand{\puis}{\mbox{$\hspace{-.1cm}~^{\scriptscriptstyle\pmb{\wedge}}
    \hspace{0.05cm}$}} %
\newcommand{\pointpuis}{.\mbox{$\hspace{-.15cm}~^{\scriptscriptstyle\pmb{\wedge}}$}} %
\newcommand{\Sfois}{\mbox{$\mt{\star}$}} %

%%%%% nouvel environnement tabular pour les encadrés Scilab %%%%
\newenvironment{encadre}
{\begin{lrbox}{\TBox}\varwidth{\linewidth}
    \tabular{>{\tt\small}C{0.1\textwidth}>{\small}R{0.7\textwidth}}}
  {\endtabular\endvarwidth\end{lrbox}%
  \fboxsep=1pt\longboxed{\usebox\TBox}}

\newenvironment{encadreL}
{\begin{lrbox}{\TBox}\varwidth{\linewidth}
    \tabular{>{\tt\small}C{0.25\textwidth}>{\small}R{0.6\textwidth}}}
  {\endtabular\endvarwidth\end{lrbox}%
  \fboxsep=1pt\longboxed{\usebox\TBox}}

\newenvironment{encadreF}
{\begin{lrbox}{\TBox}\varwidth{\linewidth}
    \tabular{>{\tt\small}C{0.2\textwidth}>{\small}R{0.70\textwidth}}}
  {\endtabular\endvarwidth\end{lrbox}%
  \fboxsep=1pt\longboxed{\usebox\TBox}}

\newenvironment{encadreLL}[2]
{\begin{lrbox}{\TBox}\varwidth{\linewidth}
    \tabular{>{\tt\small}C{#1\textwidth}>{\small}R{#2\textwidth}}}
  {\endtabular\endvarwidth\end{lrbox}%
  \fboxsep=1pt\longboxed{\usebox\TBox}}

%%%%% nouvel environnement tabular pour les script et fonctions %%%%
\newcommand{\commentaireDL}[1]{\multicolumn{1}{l}{\it
    \textcolor{comment}{$\slash\slash$ #1}}}

\newcommand{\commentaire}[1]{{\textcolor{comment}{$\slash\slash$ #1}}}

\newcounter{cptcol}

\newcommand{\nocount}{\multicolumn{1}{c}{}}

\newcommand{\sciNo}[1]{{\small \underbar #1}}

\NewEnviron{scilab}{ %
  \setcounter{cptcol}{0}
  \begin{center}
    \longboxed{
      \begin{tabular}{>{\stepcounter{cptcol}{\tiny \underbar
              \thecptcol}}c>{\tt}l}
        \BODY
      \end{tabular}
    }
  \end{center}
}

\NewEnviron{scilabNC}{ %
  \begin{center}
    \longboxed{
      \begin{tabular}{>{\tt}l} %
          \BODY
      \end{tabular}
    }
  \end{center}
}

\NewEnviron{scilabC}[1]{ %
  \setcounter{cptcol}{#1}
  \begin{center}
    \longboxed{
      \begin{tabular}{>{\stepcounter{cptcol}{\tiny \underbar
              \thecptcol}}c>{\tt}l}
        \BODY
      \end{tabular}
    }
  \end{center}
}

\newcommand{\scisol}[1]{ %
  \setcounter{cptcol}{0}
  \longboxed{
    \begin{tabular}{>{\stepcounter{cptcol}{\tiny \underbar
            \thecptcol}}c>{\tt}l}
      #1
    \end{tabular}
  }
}

\newcommand{\scisolNC}[1]{ %
  \longboxed{
    \begin{tabular}{>{\tt}l}
      #1
    \end{tabular}
  }
}

\newcommand{\scisolC}[2]{ %
  \setcounter{cptcol}{#1}
  \longboxed{
    \begin{tabular}{>{\stepcounter{cptcol}{\tiny \underbar
            \thecptcol}}c>{\tt}l}
      #2
    \end{tabular}
  }
}

\NewEnviron{syntaxe}{ %
  % \fcolorbox{black}{Yellow!20}{\setlength{\fboxsep}{3mm}
  \shadowbox{
    \setlength{\fboxsep}{3mm}
    \begin{tabular}{>{\tt}l}
      \BODY
    \end{tabular}
  }
}

%%%%% fin macros TP Scilab %%%%%%%%
%%%%%%%%%%%%%%%%%%%%%%%%%%%%%%%%%%%

%%%%%%%%%%%%%%%%%%%%%%%%%%%%%%%%%%%
%%%%% TP Python - listings %%%%%%%%
%%%%%%%%%%%%%%%%%%%%%%%%%%%%%%%%%%%
\newcommand{\Python}{\textbf{Python}} %

\lstset{% general command to set parameter(s)
basicstyle=\ttfamily\small, % print whole listing small
keywordstyle=\color{blue}\bfseries\underbar,
%% underlined bold black keywords
frame=lines,
xleftmargin=10mm,
numbers=left,
numberstyle=\tiny\underbar,
numbersep=10pt,
%identifierstyle=, % nothing happens
commentstyle=\color{green}, % white comments
%%stringstyle=\ttfamily, % typewriter type for strings
showstringspaces=false}

\newcommand{\pysolCpt}[2]{
  \setcounter{cptcol}{#1}
  \longboxed{
    \begin{tabular}{>{\stepcounter{cptcol}{\tiny \underbar
            \thecptcol}}c>{\tt}l}
        #2
      \end{tabular}
    }
} %

\newcommand{\pysol}[1]{
  \setcounter{cptcol}{0}
  \longboxed{
    \begin{tabular}{>{\stepcounter{cptcol}{\tiny \underbar
            \thecptcol}}c>{\tt}l}
        #1
      \end{tabular}
    }
} %

% \usepackage[labelsep=endash]{caption}

% avec un caption
\NewEnviron{pythonCap}[1]{ %
  \renewcommand{\tablename}{Programme}
  \setcounter{cptcol}{0}
  \begin{center}
    \longboxed{
      \begin{tabular}{>{\stepcounter{cptcol}{\tiny \underbar
              \thecptcol}}c>{\tt}l}
        \BODY
      \end{tabular}
    }
    \captionof{table}{#1}
  \end{center}
}

\NewEnviron{python}{ %
  \setcounter{cptcol}{0}
  \begin{center}
    \longboxed{
      \begin{tabular}{>{\stepcounter{cptcol}{\tiny \underbar
              \thecptcol}}c>{\tt}l}
        \BODY
      \end{tabular}
    }
  \end{center}
}

\newcommand{\pyVar}[1]{\textcolor{var}{\bf \small #1}} %
\newcommand{\pyFun}[1]{\textcolor{pyfunction}{#1}} %
\newcommand{\pyCle}[1]{\textcolor{pyCle}{#1}} %
\newcommand{\pyImp}[1]{{\bf #1}} %

%%%%% commentaire python %%%%
\newcommand{\pyComDL}[1]{\multicolumn{1}{l}{\textcolor{pycomment}{\#
      #1}}}

\newcommand{\pyCom}[1]{{\textcolor{pycomment}{\# #1}}}
\newcommand{\pyDoc}[1]{{\textcolor{pydoc}{#1}}}

\newcommand{\pyNo}[1]{{\small \underbar #1}}

%%%%%%%%%%%%%%%%%%%%%%%%%%%%%%%%%%%
%%%%%% Système linéaire paramétré : écrire les opérations au-dessus
%%%%%% d'un symbole équivalent
%%%%%%%%%%%%%%%%%%%%%%%%%%%%%%%%%%%

\usepackage{systeme}

\NewEnviron{arrayEq}{ %
  \stackrel{\scalebox{.6}{$
      \begin{array}{l} 
        \BODY \\[.1cm]
      \end{array}$}
  }{\Longleftrightarrow}
}

\NewEnviron{arrayEg}{ %
  \stackrel{\scalebox{.6}{$
      \begin{array}{l} 
        \BODY \\[.1cm]
      \end{array}$}
  }{=}
}

\NewEnviron{operationEq}{ %
  \scalebox{.6}{$
    \begin{array}{l} 
      \scalebox{1.6}{$\mbox{Opérations :}$} \\[.2cm]
      \BODY \\[.1cm]
    \end{array}$}
}

% \NewEnviron{arraySys}[1]{ %
%   \sysdelim\{.\systeme[#1]{ %
%     \BODY %
%   } %
% }

%%%%%

%%%%%%%%%%
%%%%%%%%%% ESSAI
\newlength\fboxseph
\newlength\fboxsepva
\newlength\fboxsepvb

\setlength\fboxsepva{0.2cm}
\setlength\fboxsepvb{0.2cm}
\setlength\fboxseph{0.2cm}

\makeatletter

\def\longboxed#1{\leavevmode\setbox\@tempboxa\hbox{\color@begingroup%
\kern\fboxseph{\m@th$\displaystyle #1 $}\kern\fboxseph%
\color@endgroup }\my@frameb@x\relax}

\def\my@frameb@x#1{%
  \@tempdima\fboxrule \advance\@tempdima \fboxsepva \advance\@tempdima
  \dp\@tempboxa\hbox {%
    \lower \@tempdima \hbox {%
      \vbox {\hrule\@height\fboxrule \hbox{\vrule\@width\fboxrule #1
          \vbox{%
            \vskip\fboxsepva \box\@tempboxa \vskip\fboxsepvb}#1
          \vrule\@width\fboxrule }%
        \hrule \@height \fboxrule }}}}

\newcommand{\boxedhv}[3]{\setlength\fboxseph{#1cm}
  \setlength\fboxsepva{#2cm}\setlength\fboxsepvb{#2cm}\longboxed{#3}}

\newcommand{\boxedhvv}[4]{\setlength\fboxseph{#1cm}
  \setlength\fboxsepva{#2cm}\setlength\fboxsepvb{#3cm}\longboxed{#4}}

\newcommand{\Boxed}[1]{{\setlength\fboxseph{0.2cm}
  \setlength\fboxsepva{0.2cm}\setlength\fboxsepvb{0.2cm}\longboxed{#1}}}

\newcommand{\mBoxed}[1]{{\setlength\fboxseph{0.2cm}
  \setlength\fboxsepva{0.2cm}\setlength\fboxsepvb{0.2cm}\longboxed{\mbox{#1}}}}

\newcommand{\mboxed}[1]{{\setlength\fboxseph{0.2cm}
  \setlength\fboxsepva{0.2cm}\setlength\fboxsepvb{0.2cm}\boxed{\mbox{#1}}}}

\newsavebox{\fmbox}
\newenvironment{fmpage}[1]
     {\begin{lrbox}{\fmbox}\begin{minipage}{#1}}
     {\end{minipage}\end{lrbox}\fbox{\usebox{\fmbox}}}

%%%%%%%%%%
%%%%%%%%%%

\DeclareMathOperator{\ch}{ch}
\DeclareMathOperator{\sh}{sh}

%%%%%%%%%%
%%%%%%%%%%

\newcommand{\norme}[1]{\Vert #1 \Vert}

%\newcommand*\widefbox[1]{\fbox{\hspace{2em}#1\hspace{2em}}}

\newcommand{\nl}{\tabularnewline}

\newcommand{\hand}{\noindent\ding{43}\ }
\newcommand{\ie}{\textit{i.e. }}
\newcommand{\cf}{\textit{cf }}

\newcommand{\Card}{\operatorname{Card}}

\newcommand{\aire}{\mathcal{A}}

\newcommand{\LL}[1]{\mathscr{L}(#1)} %
\newcommand{\B}{\mathscr{B}} %
\newcommand{\Bc}[1]{B_{#1}} %
\newcommand{\M}[1]{\mathscr{M}_{#1}(\mathbb{R})}

\DeclareMathOperator{\im}{Im}
\DeclareMathOperator{\kr}{Ker}
\DeclareMathOperator{\rg}{rg}
\DeclareMathOperator{\tr}{tr}
\DeclareMathOperator{\spc}{Sp}
\DeclareMathOperator{\sgn}{sgn}
\DeclareMathOperator{\supp}{Supp}

\newcommand{\Mat}{{\rm{Mat}}}
\newcommand{\Vect}[1]{{\rm{Vect}}\left(#1\right)}

\newenvironment{smatrix}{%
  \begin{adjustbox}{width=.9\width}
    $
    \begin{pmatrix}
    }{%      
    \end{pmatrix}
    $
  \end{adjustbox}
}

\newenvironment{sarray}[1]{%
  \begin{adjustbox}{width=.9\width}
    $
    \begin{array}{#1}
    }{%      
    \end{array}
    $
  \end{adjustbox}
}

\newcommand{\vd}[2]{
  \scalebox{.8}{
    $\left(\!
      \begin{array}{c}
        #1 \\
        #2
      \end{array}
    \!\right)$
    }}

\newcommand{\vt}[3]{
  \scalebox{.8}{
    $\left(\!
      \begin{array}{c}
        #1 \\
        #2 \\
        #3 
      \end{array}
    \!\right)$
    }}

\newcommand{\vq}[4]{
  \scalebox{.8}{
    $\left(\!
      \begin{array}{c}
        #1 \\
        #2 \\
        #3 \\
        #4 
      \end{array}
    \!\right)$
    }}

\newcommand{\vc}[5]{
  \scalebox{.8}{
    $\left(\!
      \begin{array}{c}
        #1 \\
        #2 \\
        #3 \\
        #4 \\
        #5 
      \end{array}
    \!\right)$
    }}

\newcommand{\ee}{\text{e}}

\newcommand{\dd}{\text{d}}

%%% Ensemble de définition
\newcommand{\Df}{\mathscr{D}}
\newcommand{\Cf}{\mathscr{C}}
\newcommand{\Ef}{\mathscr{C}}

\newcommand{\rond}[1]{\,\overset{\scriptscriptstyle \circ}{\!#1}}

\newcommand{\df}[2]{\dfrac{\partial #1}{\partial #2}} %
\newcommand{\dfn}[2]{\partial_{#2}(#1)} %
\newcommand{\ddfn}[2]{\partial^2_{#2}(#1)} %
\newcommand{\ddf}[2]{\dfrac{\partial^2 #1}{\partial #2^2}} %
\newcommand{\ddfr}[3]{\dfrac{\partial^2 #1}{\partial #2 \partial
    #3}} %


\newcommand{\dlim}[1]{{\displaystyle \lim_{#1} \ }}
\newcommand{\dlimPlus}[2]{
  \dlim{
    \scalebox{.6}{
      $
      \begin{array}{l}
        #1 \rightarrow #2\\
        #1 > #2
      \end{array}
      $}}}
\newcommand{\dlimMoins}[2]{
  \dlim{
    \scalebox{.6}{
      $
      \begin{array}{l}
        #1 \rightarrow #2\\
        #1 < #2
      \end{array}
      $}}}

%%%%%%%%%%%%%%
%% petit o, développement limité
%%%%%%%%%%%%%%

\newcommand{\oo}[2]{{\underset {{\overset {#1\rightarrow #2}{}}}{o}}} %
\newcommand{\oox}[1]{{\underset {{\overset {x\rightarrow #1}{}}}{o}}} %
\newcommand{\oon}{{\underset {{\overset {n\rightarrow +\infty}{}}}{o}}} %
\newcommand{\po}[1]{{\underset {{\overset {#1}{}}}{o}}} %
\newcommand{\neqx}[1]{{\ \underset {{\overset {x \to #1}{}}}{\not\sim}\ }} %
\newcommand{\eqx}[1]{{\ \underset {{\overset {x \to #1}{}}}{\sim}\ }} %
\newcommand{\eqn}{{\ \underset {{\overset {n \to +\infty}{}}}{\sim}\ }} %
\newcommand{\eq}[2]{{\ \underset {{\overset {#1 \to #2}{}}}{\sim}\ }} %
\newcommand{\DL}[1]{{\rm{DL}}_1 (#1)} %
\newcommand{\DLL}[1]{{\rm{DL}}_2 (#1)} %

\newcommand{\negl}{<<}

\newcommand{\neglP}[1]{\begin{array}{c}
    \vspace{-.2cm}\\
    << \\
    \vspace{-.7cm}\\
    {\scriptstyle #1}
  \end{array}}

%%%%%%%%%%%%%%
%% borne sup, inf, max, min
%%%%%%%%%%%%%%
\newcommand{\dsup}[1]{\displaystyle \sup_{#1} \ }
\newcommand{\dinf}[1]{\displaystyle \inf_{#1} \ }
\newcommand{\dmax}[1]{\max\limits_{#1} \ }
\newcommand{\dmin}[1]{\min\limits_{#1} \ }

\newcommand{\dcup}[2]{{\textstyle\bigcup\limits_{#1}^{#2}}\hspace{.1cm}}
%\displaystyle \bigcup_{#1}^{#2}}
\newcommand{\dcap}[2]{{\textstyle\bigcap\limits_{#1}^{#2}}\hspace{.1cm}}
% \displaystyle \bigcap_{#1}^{#2}
%%%%%%%%%%%%%%
%% opérateurs logiques
%%%%%%%%%%%%%%
\newcommand{\NON}[1]{\mathop{\small \tt{NON}} (#1)}
\newcommand{\ET}{\mathrel{\mathop{\small \mathtt{ET}}}}
\newcommand{\OU}{\mathrel{\mathop{\small \tt{OU}}}}
\newcommand{\XOR}{\mathrel{\mathop{\small \tt{XOR}}}}

\newcommand{\id}{{\rm{id}}}

\newcommand{\sbullet}{\scriptstyle \bullet}
\newcommand{\stimes}{\scriptstyle \times}

%%%%%%%%%%%%%%%%%%
%% Probabilités
%%%%%%%%%%%%%%%%%%
\newcommand{\Prob}{\mathbb{P}}
\newcommand{\Ev}[1]{\left[ {#1} \right]}
\newcommand{\E}{\mathbb{E}}
\newcommand{\V}{\mathbb{V}}
\newcommand{\Cov}{{\rm{Cov}}}
\newcommand{\U}[2]{\mathcal{U}(\llb #1, #2\rrb)}
\newcommand{\Uc}[2]{\mathcal{U}([#1, #2])}
\newcommand{\Ucof}[2]{\mathcal{U}(]#1, #2])}
\newcommand{\Ucoo}[2]{\mathcal{U}(]#1, #2[)}
\newcommand{\Ucfo}[2]{\mathcal{U}([#1, #2[)}
\newcommand{\Bern}[1]{\mathcal{B}\left(#1\right)}
\newcommand{\Bin}[2]{\mathcal{B}\left(#1, #2\right)}
\newcommand{\G}[1]{\mathcal{G}\left(#1\right)}
\newcommand{\Pois}[1]{\mathcal{P}\left(#1\right)}
\newcommand{\HG}[3]{\mathcal{H}\left(#1, #2, #3\right)}
\newcommand{\Exp}[1]{\mathcal{E}\left(#1\right)}
\newcommand{\Norm}[2]{\mathcal{N}\left(#1, #2\right)}

\DeclareMathOperator{\cov}{Cov}

\newcommand{\var}{v.a.r. }
\newcommand{\suit}{\hookrightarrow}

\newcommand{\flecheR}[1]{\rotatebox{90}{\scalebox{#1}{\color{red}
      $\curvearrowleft$}}}


\newcommand{\partie}[1]{\mathcal{P}(#1)}
\newcommand{\Cont}[1]{\mathcal{C}^{#1}}
\newcommand{\Contm}[1]{\mathcal{C}^{#1}_m}

\newcommand{\llb}{\llbracket}
\newcommand{\rrb}{\rrbracket}

%\newcommand{\im}[1]{{\rm{Im}}(#1)}
\newcommand{\imrec}[1]{#1^{- \mathds{1}}}

\newcommand{\unq}{\mathds{1}}

\newcommand{\Hyp}{\mathtt{H}}

\newcommand{\eme}[1]{#1^{\scriptsize \mbox{ème}}}
\newcommand{\er}[1]{#1^{\scriptsize \mbox{er}}}
\newcommand{\ere}[1]{#1^{\scriptsize \mbox{ère}}}
\newcommand{\nd}[1]{#1^{\scriptsize \mbox{nd}}}
\newcommand{\nde}[1]{#1^{\scriptsize \mbox{nde}}}

\newcommand{\truc}{\mathop{\top}}
\newcommand{\fois}{\mathop{\ast}}

\newcommand{\f}[1]{\overrightarrow{#1}}

\newcommand{\checked}{\textcolor{green}{\checkmark}}

\def\restriction#1#2{\mathchoice
              {\setbox1\hbox{${\displaystyle #1}_{\scriptstyle #2}$}
              \restrictionaux{#1}{#2}}
              {\setbox1\hbox{${\textstyle #1}_{\scriptstyle #2}$}
              \restrictionaux{#1}{#2}}
              {\setbox1\hbox{${\scriptstyle #1}_{\scriptscriptstyle #2}$}
              \restrictionaux{#1}{#2}}
              {\setbox1\hbox{${\scriptscriptstyle #1}_{\scriptscriptstyle #2}$}
              \restrictionaux{#1}{#2}}}
\def\restrictionaux#1#2{{#1\,\smash{\vrule height .8\ht1 depth .85\dp1}}_{\,#2}}

\makeatletter
\newcommand*{\da@rightarrow}{\mathchar"0\hexnumber@\symAMSa 4B }
\newcommand*{\da@leftarrow}{\mathchar"0\hexnumber@\symAMSa 4C }
\newcommand*{\xdashrightarrow}[2][]{%
  \mathrel{%
    \mathpalette{\da@xarrow{#1}{#2}{}\da@rightarrow{\,}{}}{}%
  }%
}
\newcommand{\xdashleftarrow}[2][]{%
  \mathrel{%
    \mathpalette{\da@xarrow{#1}{#2}\da@leftarrow{}{}{\,}}{}%
  }%
}
\newcommand*{\da@xarrow}[7]{%
  % #1: below
  % #2: above
  % #3: arrow left
  % #4: arrow right
  % #5: space left 
  % #6: space right
  % #7: math style 
  \sbox0{$\ifx#7\scriptstyle\scriptscriptstyle\else\scriptstyle\fi#5#1#6\m@th$}%
  \sbox2{$\ifx#7\scriptstyle\scriptscriptstyle\else\scriptstyle\fi#5#2#6\m@th$}%
  \sbox4{$#7\dabar@\m@th$}%
  \dimen@=\wd0 %
  \ifdim\wd2 >\dimen@
    \dimen@=\wd2 %   
  \fi
  \count@=2 %
  \def\da@bars{\dabar@\dabar@}%
  \@whiledim\count@\wd4<\dimen@\do{%
    \advance\count@\@ne
    \expandafter\def\expandafter\da@bars\expandafter{%
      \da@bars
      \dabar@ 
    }%
  }%  
  \mathrel{#3}%
  \mathrel{%   
    \mathop{\da@bars}\limits
    \ifx\\#1\\%
    \else
      _{\copy0}%
    \fi
    \ifx\\#2\\%
    \else
      ^{\copy2}%
    \fi
  }%   
  \mathrel{#4}%
}
\makeatother



\newcount\depth

\newcount\depth
\newcount\totaldepth

\makeatletter
\newcommand{\labelsymbol}{%
      \ifnum\depth=0
        %
      \else
        \rlap{\,$\bullet$}%
      \fi
}

\newcommand*\bernoulliTree[1]{%
    \depth=#1\relax            
    \totaldepth=#1\relax
    \draw node(root)[bernoulli/root] {\labelsymbol}[grow=right] \draw@bernoulli@tree;
    \draw \label@bernoulli@tree{root};                                   
}                                                                        

\def\draw@bernoulli@tree{%
    \ifnum\depth>0 
      child[parent anchor=east] foreach \type/\label in {left child/$E$,right child/$S$} {%
          node[bernoulli/\type] {\label\strut\labelsymbol} \draw@bernoulli@tree
      }
      coordinate[bernoulli/increment] (dummy)
   \fi%
}

\def\label@bernoulli@tree#1{%
    \ifnum\depth>0
      ($(#1)!0.5!(#1-1)$) node[fill=white,bernoulli/decrement] {\tiny$p$}
      \label@bernoulli@tree{#1-1}
      ($(#1)!0.5!(#1-2)$) node[fill=white] {\tiny$q$}
      \label@bernoulli@tree{#1-2}
      coordinate[bernoulli/increment] (dummy)
   \fi%
}

\makeatother

\tikzset{bernoulli/.cd,
         root/.style={},
         decrement/.code=\global\advance\depth by-1\relax,
         increment/.code=\global\advance\depth by 1\relax,
         left child/.style={bernoulli/decrement},
         right child/.style={}}


\newcommand{\eps}{\varepsilon}

% \newcommand{\tendi}[1]{\xrightarrow[\footnotesize #1 \rightarrow
%   +\infty]{}}%

\newcommand{\tend}{\rightarrow}%
\newcommand{\tendn}{\underset{n\to +\infty}{\longrightarrow}} %
\newcommand{\ntendn}{\underset{n\to
    +\infty}{\not\hspace{-.15cm}\longrightarrow}} %
% \newcommand{\tendn}{\xrightarrow[\footnotesize n \rightarrow
%   +\infty]{}}%
\newcommand{\Tendx}[1]{\xrightarrow[\footnotesize x \rightarrow
  #1]{}}%
\newcommand{\tendx}[1]{\underset{x\to #1}{\longrightarrow}}%
\newcommand{\ntendx}[1]{\underset{x\to #1}{\not\!\!\longrightarrow}}%
\newcommand{\tendd}[2]{\underset{#1\to #2}{\longrightarrow}}%
% \newcommand{\tendd}[2]{\xrightarrow[\footnotesize #1 \rightarrow
%   #2]{}}%
\newcommand{\tendash}[1]{\xdashrightarrow[\footnotesize #1 \rightarrow
  +\infty]{}}%
\newcommand{\tendashx}[1]{\xdashrightarrow[\footnotesize x \rightarrow
  #1]{}}%
\newcommand{\tendb}[1]{\underset{#1 \to +\infty}{\longrightarrow}}%
\newcommand{\tendL}{\overset{\mathscr L}{\underset{n \to
      +\infty}{\longrightarrow}}}%
\newcommand{\tendP}{\overset{\Prob}{\underset{n \to
      +\infty}{\longrightarrow}}}%
\newcommand{\tenddL}[1]{\overset{\mathscr L}{\underset{#1 \to
      +\infty}{\longrightarrow}}}%

\NewEnviron{attention}{ %
  ~\\[-.2cm]\noindent
  \begin{minipage}{\linewidth}
  \setlength{\fboxsep}{3mm}%
  \ \ \dbend \ \ %
  \fbox{\parbox[t]{.88\linewidth}{\BODY}} %
  \end{minipage}\\
}

\NewEnviron{sattention}[1]{ %
  ~\\[-.2cm]\noindent
  \begin{minipage}{#1\linewidth}
  \setlength{\fboxsep}{3mm}%
  \ \ \dbend \ \ %
  \fbox{\parbox[t]{.88\linewidth}{\BODY}} %
  \end{minipage}\\
}

%%%%% OBSOLETE %%%%%%

% \newcommand{\attention}[1]{
%   \noindent
%   \begin{tabular}{@{}l|p{11.5cm}|}
%     \cline{2-2}
%     \vspace{-.2cm} 
%     & \nl
%     \dbend & #1 \nl
%     \cline{2-2}
%   \end{tabular}
% }

% \newcommand{\attentionv}[2]{
%   \noindent
%   \begin{tabular}{@{}l|p{11.5cm}|}
%     \cline{2-2}
%     \vspace{-.2cm} 
%     & \nl
%     \dbend & #2 \nl[#1 cm]
%     \cline{2-2}
%   \end{tabular}
% }

\newcommand{\explainvb}[2]{
  \noindent
  \begin{tabular}{@{}l|p{11.5cm}|}
    \cline{2-2}
    \vspace{-.2cm} 
    & \nl
    \hand & #2 \nl [#1 cm]
    \cline{2-2}
  \end{tabular}
}


% \noindent
% \begin{tabular}{@{}l|lp{11cm}|}
%   \cline{3-3} 
%   \multicolumn{1}{@{}l@{\dbend}}{} & & #1 \nl
%   \multicolumn{1}{l}{} & & \nl [-.8cm]
%   & & #2 \nl
%   \cline{2-3}
% \end{tabular}

% \newcommand{\attention}[1]{
%   \noindent
%   \begin{tabular}{@{}@{}cp{11cm}}
%     \dbend & #1 \nl
%   \end{tabular}
% }

\newcommand{\PP}[1]{\mathcal{P}(#1)}
\newcommand{\FF}[1]{\mathcal{F}(#1)}

\newcommand{\DSum}[2]{\displaystyle\sum\limits_{#1}^{#2}\hspace{.1cm}}
\newcommand{\Sum}[2]{{\textstyle\sum\limits_{#1}^{#2}}\hspace{.1cm}}
\newcommand{\Serie}{\textstyle\sum\ }
\newcommand{\Prod}[2]{\textstyle\prod\limits_{#1}^{#2}}

\newcommand{\Prim}[3]{\left[\ {#1} \ \right]_{\scriptscriptstyle
   \hspace{-.15cm} ~_{#2}\, }^ {\scriptscriptstyle \hspace{-.15cm} ~^{#3}\, }}

% \newcommand{\Prim}[3]{\left[\ {#1} \ \right]_{\scriptscriptstyle
%     \!\!~_{#2}}^ {\scriptscriptstyle \!\!~^{#3}}}

\newcommand{\dint}[2]{\displaystyle \int_{#1}^{#2}\ }
\newcommand{\Int}[2]{{\rm{Int}}_{\scriptscriptstyle #1, #2}}
\newcommand{\dt}{\ dt}
\newcommand{\dx}{\ dx}

\newcommand{\llpar}[1]{\left(\!\!\!
    \begin{array}{c}
      \rule{0pt}{#1}
    \end{array}
  \!\!\!\right.}

\newcommand{\rrpar}[1]{\left.\!\!\!
    \begin{array}{c}
      \rule{0pt}{#1}
    \end{array}
  \!\!\!\right)}

\newcommand{\llacc}[1]{\left\{\!\!\!
    \begin{array}{c}
      \rule{0pt}{#1}
    \end{array}
  \!\!\!\right.}

\newcommand{\rracc}[1]{\left.\!\!\!
    \begin{array}{c}
      \rule{0pt}{#1}
    \end{array}
  \!\!\!\right\}}

\newcommand{\ttacc}[1]{\mbox{\rotatebox{-90}{\hspace{-.7cm}$\llacc{#1}$}}}
\newcommand{\bbacc}[1]{\mbox{\rotatebox{90}{\hspace{-.5cm}$\llacc{#1}$}}}

\newcommand{\comp}[1]{\overline{#1}}%

\newcommand{\dcomp}[2]{\stackrel{\mbox{\ \ \----}{\scriptscriptstyle
      #2}}{#1}}%

% \newcommand{\Comp}[2]{\stackrel{\mbox{\ \
%       \-------}{\scriptscriptstyle #2}}{#1}}

% \newcommand{\dcomp}[2]{\stackrel{\mbox{\ \
%       \-------}{\scriptscriptstyle #2}}{#1}}

\newcommand{\A}{\mathscr{A}}

\newcommand{\conc}[1]{
  \begin{center}
    \fbox{
      \begin{tabular}{c}
        #1
      \end{tabular}
    }
  \end{center}
}

\newcommand{\concC}[1]{
  \begin{center}
    \fbox{
    \begin{tabular}{C{10cm}}
      \quad #1 \quad
    \end{tabular}
    }
  \end{center}
}

\newcommand{\concL}[2]{
  \begin{center}
    \fbox{
    \begin{tabular}{C{#2cm}}
      \quad #1 \quad
    \end{tabular}
    }
  \end{center}
}

% \newcommand{\lims}[2]{\prod\limits_{#1}^{#2}}

\newtheorem{theorem}{Théorème}[]
\newtheorem{lemma}{Lemme}[]
\newtheorem{proposition}{Proposition}[]
\newtheorem{corollary}{Corollaire}[]

% \newenvironment{proof}[1][Démonstration]{\begin{trivlist}
% \item[\hskip \labelsep {\bfseries #1}]}{\end{trivlist}}
\newenvironment{definition}[1][Définition]{\begin{trivlist}
\item[\hskip \labelsep {\bfseries #1}]}{\end{trivlist}}
\newenvironment{example}[1][Exemple]{\begin{trivlist}
\item[\hskip \labelsep {\bfseries #1}]}{\end{trivlist}}
\newenvironment{examples}[1][Exemples]{\begin{trivlist}
\item[\hskip \labelsep {\bfseries #1}]}{\end{trivlist}}
\newenvironment{notation}[1][Notation]{\begin{trivlist}
\item[\hskip \labelsep {\bfseries #1}]}{\end{trivlist}}
\newenvironment{propriete}[1][Propriété]{\begin{trivlist}
\item[\hskip \labelsep {\bfseries #1}]}{\end{trivlist}}
\newenvironment{proprietes}[1][Propriétés]{\begin{trivlist}
\item[\hskip \labelsep {\bfseries #1}]}{\end{trivlist}}
\newenvironment{remarkSC}[1][Remarque]{\begin{trivlist}
\item[\hskip \labelsep {\bfseries #1}]}{\end{trivlist}}
\newenvironment{application}[1][Application]{\begin{trivlist}
\item[\hskip \labelsep {\bfseries #1}]}{\end{trivlist}}

% Environnement pour les réponses des DS
\newenvironment{answer}{\par\emph{Réponse :}\par{}}
{\vspace{-.6cm}\hspace{\stretch{1}}\rule{1ex}{1ex}\vspace{.3cm}}

\newenvironment{answerTD}{\vspace{.2cm}\par\emph{Réponse :}\par{}}
{\hspace{\stretch{1}}\rule{1ex}{1ex}\vspace{.3cm}}

\newenvironment{answerCours}{\noindent\emph{Réponse :}}
{\rule{1ex}{1ex}}%\vspace{.3cm}}


% footnote in footer
\newcommand{\fancyfootnotetext}[2]{%
  \fancypagestyle{dingens}{%
    \fancyfoot[LO,RE]{\parbox{0.95\textwidth}{\footnotemark[#1]\footnotesize
        #2}}%
  }%
  \thispagestyle{dingens}%
}

%%% définit le style (arabic : 1,2,3...) et place des parenthèses
%%% autour de la numérotation
\renewcommand*{\thefootnote}{(\arabic{footnote})}
% http://www.tuteurs.ens.fr/logiciels/latex/footnote.html

%%%%%%%% tikz axis
% \pgfplotsset{every axis/.append style={
%                     axis x line=middle,    % put the x axis in the middle
%                     axis y line=middle,    % put the y axis in the middle
%                     axis line style={<->,color=blue}, % arrows on the axis
%                     xlabel={$x$},          % default put x on x-axis
%                     ylabel={$y$},          % default put y on y-axis
%             }}

%%%% s'utilise comme suit

% \begin{axis}[
%   xmin=-8,xmax=4,
%   ymin=-8,ymax=4,
%   grid=both,
%   ]
%   \addplot [domain=-3:3,samples=50]({x^3-3*x},{3*x^2-9}); 
% \end{axis}

%%%%%%%%



%%%%%%%%%%%% Pour avoir des numéros de section qui correspondent à
%%%%%%%%%%%% ceux du tableau
\renewcommand{\thesection}{\Roman{section}.\hspace{-.3cm}}
\renewcommand{\thesubsection}{\Roman{section}.\arabic{subsection}.\hspace{-.2cm}}
\renewcommand{\thesubsubsection}{\Roman{section}.\arabic{subsection}.\alph{subsubsection})\hspace{-.2cm}}
%%%%%%%%%%%% 

%%% Changer le nom des figures : Fig. au lieu de Figure
\usepackage[font=small,labelfont=bf,labelsep=space]{caption}
\captionsetup{%
  figurename=Fig.,
  tablename=Tab.
}
% \renewcommand{\thesection}{\Roman{section}.\hspace{-.2cm}}
% \renewcommand{\thesubsection}{\Roman{section}
%   .\hspace{.2cm}\arabic{subsection}\ .\hspace{-.3cm}}
% \renewcommand{\thesubsubsection}{\alph{subsection})}

\newenvironment{tabliste}[1]
{\begin{tabenum}[\bfseries\small\itshape #1]}{\end{tabenum}} 

\newenvironment{noliste}[1] %
{\begin{enumerate}[\bfseries\small\itshape #1]} %
  {\end{enumerate}}

\newenvironment{nonoliste}[1] %
{\begin{enumerate}[\hspace{-12pt}\bfseries\small\itshape #1]} %
  {\end{enumerate}}

\newenvironment{arrayliste}[1]{ 
  % List with minimal white space to fit in small areas, e.g. table
  % cell
  \begin{minipage}[t]{\linewidth} %
    \begin{enumerate}[\bfseries\small\itshape #1] %
      {\leftmargin=0.5em \rightmargin=0em
        \topsep=0em \parskip=0em \parsep=0em
        \listparindent=0em \partopsep=0em \itemsep=0pt
        \itemindent=0em \labelwidth=\leftmargin\labelsep+0.25em}
    }{
    \end{enumerate}\end{minipage}
}

\newenvironment{nolistes}[2]
{\begin{enumerate}[\bfseries\small\itshape
    #1]\setlength{\itemsep}{#2 mm}}{\end{enumerate}}

\newenvironment{liste}[1]
{\begin{enumerate}[\hspace{12pt}\bfseries\small\itshape #1]}{\end{enumerate}} 



%%%%%%%% Pour les programmes de colle %%%%%%%

\newcommand{\cours}{{\small \tt (COURS)}} %
\newcommand{\poly}{{\small \tt (POLY)}} %
\newcommand{\exo}{{\small \tt (EXO)}} %
\newcommand{\culture}{{\small \tt (CULTURE)}} %
\newcommand{\methodo}{{\small \tt (MÉTHODO)}} %
\newcommand{\methodob}{\Boxed{\mbox{\tt MÉTHODO}}} %

%%%%%%%% Pour les TD %%%%%%%
\newtheoremstyle{exostyle} {\topsep} % espace avant
{.6cm} % espace apres
{} % Police utilisee par le style de thm
{} % Indentation (vide = aucune, \parindent = indentation paragraphe)
{\bfseries} % Police du titre de thm
{} % Signe de ponctuation apres le titre du thm
{ } % Espace apres le titre du thm (\newline = linebreak)
{\thmname{#1}\thmnumber{ #2}\thmnote{.
    \normalfont{\textit{#3}}}} % composants du titre du thm : \thmname
                               % = nom du thm, \thmnumber = numéro du
                               % thm, \thmnote = sous-titre du thm
 
\theoremstyle{exostyle}
\newtheorem{exercice}{Exercice}
\newtheorem{exerciceAP}{Exercice avec préparation}
\newtheorem{exerciceSP}{Exercice sans préparation}
\newtheorem*{exoCours}{Exercice}

%%%%%%%% Pour des théorèmes Sans Espaces APRÈS %%%%%%%
\newtheoremstyle{exostyleSE} {\topsep} % espace avant
{} % espace apres
{} % Police utilisee par le style de thm
{} % Indentation (vide = aucune, \parindent = indentation paragraphe)
{\bfseries} % Police du titre de thm
{} % Signe de ponctuation apres le titre du thm
{ } % Espace apres le titre du thm (\newline = linebreak)
{\thmname{#1}\thmnumber{ #2}\thmnote{.
    \normalfont{\textit{#3}}}} % composants du titre du thm : \thmname
                               % = nom du thm, \thmnumber = numéro du
                               % thm, \thmnote = sous-titre du thm
 
\theoremstyle{exostyleSE}
\newtheorem{exerciceSE}{Exercice}
\newtheorem*{exoCoursSE}{Exercice}

% \newcommand{\lims}[2]{\prod\limits_{#1}^{#2}}

\newtheorem{theoremSE}{Théorème}[]
\newtheorem{lemmaSE}{Lemme}[]
\newtheorem{propositionSE}{Proposition}[]
\newtheorem{corollarySE}{Corollaire}[]

% \newenvironment{proofSE}[1][Démonstration]{\begin{trivlist}
% \item[\hskip \labelsep {\bfseries #1}]}{\end{trivlist}}
\newenvironment{definitionSE}[1][Définition]{\begin{trivlist}
  \item[\hskip \labelsep {\bfseries #1}]}{\end{trivlist}}
\newenvironment{exampleSE}[1][Exemple]{\begin{trivlist} 
  \item[\hskip \labelsep {\bfseries #1}]}{\end{trivlist}}
\newenvironment{examplesSE}[1][Exemples]{\begin{trivlist}
\item[\hskip \labelsep {\bfseries #1}]}{\end{trivlist}}
\newenvironment{notationSE}[1][Notation]{\begin{trivlist}
\item[\hskip \labelsep {\bfseries #1}]}{\end{trivlist}}
\newenvironment{proprieteSE}[1][Propriété]{\begin{trivlist}
\item[\hskip \labelsep {\bfseries #1}]}{\end{trivlist}}
\newenvironment{proprietesSE}[1][Propriétés]{\begin{trivlist}
\item[\hskip \labelsep {\bfseries #1}]}{\end{trivlist}}
\newenvironment{remarkSE}[1][Remarque]{\begin{trivlist}
\item[\hskip \labelsep {\bfseries #1}]}{\end{trivlist}}
\newenvironment{applicationSE}[1][Application]{\begin{trivlist}
\item[\hskip \labelsep {\bfseries #1}]}{\end{trivlist}}

%%%%%%%%%%% Obtenir les étoiles sans charger le package MnSymbol
%%%%%%%%%%%
\DeclareFontFamily{U} {MnSymbolC}{}
\DeclareFontShape{U}{MnSymbolC}{m}{n}{
  <-6> MnSymbolC5
  <6-7> MnSymbolC6
  <7-8> MnSymbolC7
  <8-9> MnSymbolC8
  <9-10> MnSymbolC9
  <10-12> MnSymbolC10
  <12-> MnSymbolC12}{}
\DeclareFontShape{U}{MnSymbolC}{b}{n}{
  <-6> MnSymbolC-Bold5
  <6-7> MnSymbolC-Bold6
  <7-8> MnSymbolC-Bold7
  <8-9> MnSymbolC-Bold8
  <9-10> MnSymbolC-Bold9
  <10-12> MnSymbolC-Bold10
  <12-> MnSymbolC-Bold12}{}

\DeclareSymbolFont{MnSyC} {U} {MnSymbolC}{m}{n}

\DeclareMathSymbol{\filledlargestar}{\mathrel}{MnSyC}{205}
\DeclareMathSymbol{\largestar}{\mathrel}{MnSyC}{131}

\newcommand{\facile}{\rm{(}$\scriptstyle\largestar$\rm{)}} %
\newcommand{\moyen}{\rm{(}$\scriptstyle\filledlargestar$\rm{)}} %
\newcommand{\dur}{\rm{(}$\scriptstyle\filledlargestar\filledlargestar$\rm{)}} %
\newcommand{\costaud}{\rm{(}$\scriptstyle\filledlargestar\filledlargestar\filledlargestar$\rm{)}}

%%%%%%%%%%%%%%%%%%%%%%%%%

%%%%%%%%%%%%%%%%%%%%%%%%%
%%%%%%%% Fin de la partie TD

%%%%%%%%%%%%%%%%
%%%%%%%%%%%%%%%%
\makeatletter %
\newenvironment{myitemize}{%
  \setlength{\topsep}{0pt} %
  \setlength{\partopsep}{0pt} %
  \renewcommand*{\@listi}{\leftmargin\leftmargini \parsep\z@
    \topsep\z@ \itemsep\z@} \let\@listI\@listi %
  \itemize %
}{\enditemize} %
\makeatother
%%%%%%%%%%%%%%%%
%%%%%%%%%%%%%%%%

%% Commentaires dans la correction du livre

\newcommand{\Com}[1]{
% Define box and box title style
\tikzstyle{mybox} = [draw=black!50,
very thick,
    rectangle, rounded corners, inner sep=10pt, inner ysep=8pt]
\tikzstyle{fancytitle} =[rounded corners, fill=black!80, text=white]
\tikzstyle{fancylogo} =[ text=white]
\begin{center}

\begin{tikzpicture}
\node [mybox] (box){%

    \begin{minipage}{0.90\linewidth}
\vspace{6pt}  #1
    \end{minipage}
};
\node[fancytitle, right=10pt] at (box.north west) 
{\bfseries\normalsize{Commentaire}};

\end{tikzpicture}%

\end{center}
%
}

% \NewEnviron{remark}{%
%   % Define box and box title style
%   \tikzstyle{mybox} = [draw=black!50, very thick, rectangle, rounded
%   corners, inner sep=10pt, inner ysep=8pt] %
%   \tikzstyle{fancytitle} =[rounded corners, fill=black!80,
%   text=white] %
%   \tikzstyle{fancylogo} =[ text=white]
%   \begin{center}
%     \begin{tikzpicture}
%       \node [mybox] (box){%
%         \begin{minipage}{0.90\linewidth}
%           \vspace{6pt} \BODY
%         \end{minipage}
%       }; %
%       \node[fancytitle, right=10pt] at (box.north west) %
%       {\bfseries\normalsize{Commentaire}}; %
%     \end{tikzpicture}%
%   \end{center}
% }
% \makeatother


%%%
% Remarques comme dans le livre
%%%

\NewEnviron{remark}{%
  % Define box and box title style
  \tikzstyle{mybox} = [draw=black!50, very thick, rectangle, rounded
  corners, inner sep=10pt, inner ysep=8pt] %
  \tikzstyle{fancytitle} = [rounded corners , fill=black!80,
  text=white] %
  \tikzstyle{fancylogo} =[ text=white]
  \begin{center}
    \begin{tikzpicture}
      \node [mybox] (box){%
        \begin{minipage}{0.90\linewidth}
          \vspace{6pt} \BODY
        \end{minipage}
      }; %
      \node[fancytitle, right=10pt] at (box.north west) %
      {\bfseries\normalsize{Commentaire}}; %
    \end{tikzpicture}%
  \end{center}
}

\NewEnviron{remarkST}{%
  % Define box and box title style
  \tikzstyle{mybox} = [draw=black!50, very thick, rectangle, rounded
  corners, inner sep=10pt, inner ysep=8pt] %
  \tikzstyle{fancytitle} = [rounded corners , fill=black!80,
  text=white] %
  \tikzstyle{fancylogo} =[ text=white]
  \begin{center}
    \begin{tikzpicture}
      \node [mybox] (box){%
        \begin{minipage}{0.90\linewidth}
          \vspace{6pt} \BODY
        \end{minipage}
      }; %
      % \node[fancytitle, right=10pt] at (box.north west) %
%       {\bfseries\normalsize{Commentaire}}; %
    \end{tikzpicture}%
  \end{center}
}

\NewEnviron{remarkL}[1]{%
  % Define box and box title style
  \tikzstyle{mybox} = [draw=black!50, very thick, rectangle, rounded
  corners, inner sep=10pt, inner ysep=8pt] %
  \tikzstyle{fancytitle} =[rounded corners, fill=black!80,
  text=white] %
  \tikzstyle{fancylogo} =[ text=white]
  \begin{center}
    \begin{tikzpicture}
      \node [mybox] (box){%
        \begin{minipage}{#1\linewidth}
          \vspace{6pt} \BODY
        \end{minipage}
      }; %
      \node[fancytitle, right=10pt] at (box.north west) %
      {\bfseries\normalsize{Commentaire}}; %
    \end{tikzpicture}%
  \end{center}
}

\NewEnviron{remarkSTL}[1]{%
  % Define box and box title style
  \tikzstyle{mybox} = [draw=black!50, very thick, rectangle, rounded
  corners, inner sep=10pt, inner ysep=8pt] %
  \tikzstyle{fancytitle} =[rounded corners, fill=black!80,
  text=white] %
  \tikzstyle{fancylogo} =[ text=white]
  \begin{center}
    \begin{tikzpicture}
      \node [mybox] (box){%
        \begin{minipage}{#1\linewidth}
          \vspace{6pt} \BODY
        \end{minipage}
      }; %
%       \node[fancytitle, right=10pt] at (box.north west) %
%       {\bfseries\normalsize{Commentaire}}; %
    \end{tikzpicture}%
  \end{center}
}

\NewEnviron{titre} %
{ %
  ~\\[-1.8cm]
  \begin{center}
    \bf \LARGE \BODY
  \end{center}
  ~\\[-.6cm]
  \hrule %
  \vspace*{.2cm}
} %

\NewEnviron{titreL}[2] %
{ %
  ~\\[-#1cm]
  \begin{center}
    \bf \LARGE \BODY
  \end{center}
  ~\\[-#2cm]
  \hrule %
  \vspace*{.2cm}
} %




\pagestyle{fancy} %
\lhead{ECE2 \hfill septembre 2017 \\
 Mathématiques\\[.2cm]} %
\chead{\hrule} %
\rhead{} %
\lfoot{} %
\cfoot{} %
\rfoot{\thepage} %

\renewcommand{\headrulewidth}{0pt}% : Trace un trait de séparation
 % de largeur 0,4 point. Mettre 0pt
 % pour supprimer le trait.

\renewcommand{\footrulewidth}{0.4pt}% : Trace un trait de séparation
 % de largeur 0,4 point. Mettre 0pt
 % pour supprimer le trait.

\setlength{\headheight}{14pt}

\title{\bf \vspace{-1cm} ORAUX HEC 2009} %
\author{} %
\date{} %

\begin{document}

\maketitle %
\vspace{-1.2cm}\hrule %
\thispagestyle{fancy}

\vspace*{.4cm}

% DEBUT DU DOC À MODIFIER : tout virer jusqu'au début de l'exo

\section{Annales 2009}

 %\setcounter{exercice}{0}

 \begin{exercice}{\it (Exercice avec préparation)}~
 \begin{noliste}{1.}
 \setlength{\itemsep}{4mm}
 \item Soit $(A_{i})_{i \in I}$ un système complet d'évènements,
c'est-à-dire vérifiant que $A_{i} \cap A_{j} = \emptyset$ si $i \neq
j$, $ \dcup{i \in I} A_{i} = \Omega$ et $\Prob\left(\Ev{A_{i}}\right)
\neq 0$ pour tout $i$. \\
 Alors pour tout évènement $E$, on a $\Prob\left(\Ev{E}\right) = \Sum{i
\in I} \Prob( E \cap A_{i}) = \Sum{i \in I}
\Prob\left(\Ev{A_{i}}\right) P_{A_{i}} (E)$. \\
 \item $p_{1} = q_{1} = r_{1} = \frac{1}{3}$ (on choisit au hasard). \\
 \item $p_{n} + q_{n} + r_{n} = 1$ car les évènements associés forment
un système complet d'évènements. \\
 \item On a $p_{n + 1} = \frac{1}{3} p_{n} + \frac{1}{12} r_{n}$. \\
 $q_{n + 1} = \frac{1}{3} p_{n} + q_{n} + \frac{7}{12} r_{n}$. \\
 $r_{n + 1} = \frac{1}{3} p_{n} + \frac{1}{3} r_{n}$. \\
 \item $p_{n} = \frac{1}{3} p_{n-1} + \frac{ 1}{12} r_{n-1} = \left(
\frac{1}{3} p_{n-1} + \frac{1}{3} r_{n-1} \right) - \frac{1}{3} r_{n-1}
+ \frac{1}{12} r_{n-1} = r_{n} - \frac{1}{4} r_{n-1}$. \\
 \item $r_{n + 1} = \frac{1}{3} p_{n} + \frac{1}{3} r_{n} = \frac{1}{3}
r_{n} - \frac{1}{12} r_{n-1} + \frac{1}{3} r_{n} = \frac{2}{3} r_{n} -
\frac{1}{12} r_{n-1}$ et en multipliant par 12 on a : $12 r_{n + 1} - 8
r_{n} + r_{n-1} = 0$, qui est bien une relation de récurrence linéaire
double. \\
\\
 On étudie l'équation caractéristique : $12 r^{2} - 8 r + 1 = 0$ a pour
discriminant $\Delta = 64 - 48 = 16$ donc les racines sont $r_{1} =
\frac{8 -4}{24} = \frac{1}{6}$ et $r_{2} = \frac{8 + 4}{24} =
\frac{1}{2}$. \\
\\
 On obtient $r_{n} = a \left( \frac{1}{6} \right)^{n} + b \left(
\frac{1}{2} \right)^{n}$, puis on étudie les premières valeurs : \\
 $r_{1} = \frac{1}{3}$ donc $\frac{1}{6} a + \frac{1}{2} b =
\frac{1}{3}$, ou encore $\frac{a + 3b - 2}{6} = 0$. \\
 $r_{2} = \frac{1}{3} r_{n} + \frac{1}{3} p_{1} = \frac{2}{9}$ donc $
\frac{1}{36} a + \frac{1}{4} b = \frac{2}{9}$, ou encore $\frac{ a + 9
b - 8}{36} = 0$. \\
 D'où $a = 2 - 3b$, puis $2-3b + 9 b - 8 = 0$, $b = 1$ et $a = -1$. \\
 Enfin $r_{n} = - \left( \frac{1}{6} \right)^{n} + \left( \frac{1}{2}
\right)^{n}$. \\
\\
 Ensuite on a $p_{n} = r_{n} - \frac{1}{4} r_{n-1} = - \left(
\frac{1}{6} \right)^{n} + \left( \frac{1}{2} \right)^{n} + \frac{1}{4}
\left( \frac{1}{6} \right)^{n-1} - \frac{1}{4} \left( \frac{1}{2}
\right)^{n-1} = \left( \frac{1}{6} \right)^{n} \left[ - 1 + \frac{6}{4}
\right] + \left( \frac{1}{2} \right)^{n} \left[ 1 - \frac{2}{4} \right]
= \frac{1}{2} \left[ \left( \frac{1}{6} \right)^{n} + \left(
\frac{1}{2} \right)^{n} \right]$. \\
\\
 Enfin on a $q_{n} = 1 - r_{n} - p_{n} = 1 + \frac{1}{2} \left(
\frac{1}{6} \right)^{n} - \frac{3}{2} \left( \frac{1}{2} \right)^{n}$.
\\
 \item On a $p_{n} \rightarrow 0$, $r_{n} \rightarrow 0$ et $q_{n}
\rightarrow 1$. \\
 \end{noliste}
 \noindent \textbf{\underline{Exercice sans préparation}} \\
\\
 Une matrice symétrique vérifie $M = {}{t} M$ donc $(I, M, {}{t} M)$
sont bien liées. \\
 Si une telle matrice est diagonalisable : \\
 1er cas : la relation de dépendance ne concerne pas la transposée,
alors $M = \lambda I = {}{t} M$ est bien diagonalisable. \\
 2e cas : on a ${}{t} M = \lambda I$, alors ${}{t} M = M = \lambda I$
est bien diagonalisable. \\
 3e cas : On a ${}{t} M = \alpha M$, alors on a $m_{i,j} = \alpha
m_{j,i} = \alpha^{2} m_{i,j}$ donc $\alpha^{2} = 1$, et $M$ est
symétrique (donc diagonalisable) ou antisymétrique (et on ne sait rien
dire). \\
 4e cas : on a $M = \alpha I + \beta {}{t} M$. \\
 $M$ est diagonalisable et s'écrit $P D P^{-1}$, et $I = P I P^{-1}$
donc on obtient $\beta {}{t} M = P \left(\Ev{ D - \alpha I}\right)
P^{-1}$ et enfin ${}{t} M = P \left(\Ev{ \frac{1}{\beta} ( D - \alpha
I)}\right) P^{-1} = P D' P^{-1}$ est diagonalisable dans la même base
que $M$ car $D'$ est diagonale. \\
 On pourrait encore dire bien des choses avec la théorie des matrices
orthogonales, mais comme elle est hors programme...
 \end{exercice}

 \newpage

 \begin{exercice}{\it (Exercice avec préparation)}~
 \begin{noliste}{1.}
 \setlength{\itemsep}{4mm}
 \item La loi géométrique est la loi d'attente du premier succès dans
une succession illimitée d'épreuves de Bernouilli identiques et
indépendantes, de paramètre $p$. \\
 Si $X \suit \mathcal{G} (p)$, on a $X(\Omega) = \N^*$ et pour tout $k
\in \N^*$, $\Prob\left(\Ev{\Ev{X = k}}\right) = (1-p)^{k-1} p$. \\
 De plus $X$ admet une espérance et une variance, et $\E(X) =
\frac{1}{p}$ et $\V(X) = \frac{1-p}{p^{2}}$. \\
 \item Soit $x$ un réel de $]0 ; 1[$.
 \begin{noliste}{a)}
 \setlength{\itemsep}{2mm} 
 \item La somme des termes d'une suite géométrique donne $\Sum{k =
1}{n} x^{k-1} = \frac{1 - t^{n}}{1-t}$. \\
 On intègre sur $[ 0 ; x]$ ces fonctions continues : \\
 $\dint{0}{x} \frac{1-t^{n}}{1-t}\ dt = \Sum{k = 1}{n} \dint{0}{x}
t^{k-1}\ dt = \Sum{k = 1}{n}\left[ \ \frac{t^{k}}{k} \right]_{0}{x} =
\Sum{k = 1}{n}\frac{x^{k}}{k}$. \\
 \item On a $ 0 \leq 1-x < 1-t \leq 1$ sur $[ 0 ; 1[$, donc $0 \leq
\frac{1}{1-t} \leq \frac{1}{1-x}$ et $0 \leq \frac{t^{x}}{1-t} \leq
\frac{t^{n}}{1-x}$. \\
 On intègre sur $[ 0 ; x]$ : $ 0 \leq \dint{0}{x} \frac{t^{n}}{1-t}\ dt
\leq \frac{1}{1-x} \dint{0}{x} t^{n}\ dt = \frac{x^{n + 1} }{(n + 1)
(1-x)} \xrightarrow[ n \rightarrow + \infty]{} 0$ donc par théorème de
comparaison, $\dlim{n \rightarrow + \infty} \dint{0}{x}
\frac{t^{n}}{1-t}\ dt = 0$. \\
 \item On en déduit que pour tout $n \in \N$, $S_{n} = \Sum{k = 1}{n}
\frac{x^{k}}{k} = \dint{0}{x} \frac{1}{1-t}\ dt - \dint{0}{x}
\frac{t^{n}}{1-t}\ dt \xrightarrow[ n \rightarrow + \infty]{}
\dint{0}{x} \frac{1}{1-t}\ dt = \left[ - \ln (| 1-t | ) \right]_{0}{x}
= - \ln (1-x) $. \\
 La suite des sommes partielles converge vers $- \ln (1-x)$ donc la
série converge et $\Sum{k = 1}{+ \infty} \frac{x^{k}}{k} = - \ln
(1-x)$. \\
 \end{noliste}
 \item \begin{noliste}{a)}
 \setlength{\itemsep}{2mm}
 \item $X(\Omega) = \N^*$ donc $Y(\Omega) = \left\{ \frac{1}{n}\ \big|\
n \in \N^* \right\}$. \\
 De plus pour tout $n \in \N^*$, $P \left(\Ev{ Y = \frac{1}{n}}\right)
= \Prob\left(\Ev{\Ev{ X = n}}\right) = q^{n-1} p$, où $q = 1-p$. \\
 \item Le moment d'ordre $r$ de $Y$ existe si et seulement si la série
$\Sum{n = 1}{+ \infty} \left( \frac{1}{n} \right)^{r} q^{n-1} p$
converge absolument, ce qui est équivalent à la convergence si la série
est à termes positifs. \\
 Or on a pour tout $n \geq 1$, pour tout $r \geq 1$, $n^{r} \geq 1$
donc $0 \leq \frac{1}{n^{r}} \leq 1$ et enfin $0 \leq \frac{ p q^{n-1}
}{n^{r}} \leq q^{n-1} p$. \\
 Or la série de terme général $q^{n-1} p$ converge (série géométrique
avec $ | q | 1$, donc par théorème de comparaison des séries à termes
positifs, la série converge et converge absolument car elle est à
termes positifs. \\
 \item $\E(Y) = \Sum{n = 1}{+ \infty} \frac{1}{n} p q^{n-1} =
\frac{p}{q} \Sum{n = 1}{+ \infty} \frac{q^{n}}{n} = \frac{ - p \ln
(1-q) }{q} = \frac{ - p \ln p}{1-p}$. \\
 \end{noliste}
 \end{noliste}
 \noindent \textbf{\underline{Exercice sans préparation}} \\
\\
 Soit la matrice $A = \begin{smatrix}
1 & 2 \\
3 & 4 \\
-1 & 4 \\
\end{smatrix}
$.\begin{noliste}{1.}
 \setlength{\itemsep}{4mm}
 \item On prend $B$ quelconque, on a $A B = \begin{smatrix}
1 & 2 \\
3 & 4 \\
-1 & 4 \\
\end{smatrix}
\begin{smatrix}
a & b & c \\
d & e & f \\
\end{smatrix}
 = \begin{smatrix}
a + 2d & b + 2e & c + 2f \\
3 a + 4d & 3b + 4e & 3c + 4f \\
-a + 4d & -b + 4e & -c + 4f \\
\end{smatrix}
$. \\
 En identifiant les coefficients, on obtient trois systèmes sur
$(a,d)$, $(b, e)$ et $(c,f)$, et la résolution du premier donne $a = d
= 0$ et $a + 2d = 1$ (absurde). \\
 Il n'y a donc pas de solution. \\
 \item On prend $c$ quelconque, on a $C A = \begin{smatrix}
a & b & c \\
d & e & f \\
\end{smatrix}
\begin{smatrix}
1 & 2 \\
3 & 4 \\
-1 & 4 \\
\end{smatrix}
 = \begin{smatrix}
a + 3b-c & 2a + 4b + 4c \\
d + 3e-f & 2d + 4e + 4f \\
\end{smatrix}
$. \\
 En identifiant les coefficients, on obtient deux systèmes sur
$(a,b,c)$ et $(d,e,f)$ dont la résolution donne $a = -2b-2c$ et $b = c
+ 1$ donc $a = -4c - 2$ qui a une infinité de solutions (une pour
chaque valeur réelle de $c$). \\
 De même le deuxième donne une infinité de solutions, et il y a donc
une infinité de solutions à l'équation matricielle $C A = I_{2}$.

 \end{noliste}
 \end{exercice}

 \newpage

 \begin{exercice}{\it (Exercice avec préparation)}~\\
 On considère la suite réelle $(u_{n})_{n \in \N}$ définie par : 
 
\[
 u_{0} = 3,\ u_{1} = \frac{29}{9} \text{ et } \forall n \in \N,\ u_{n +
2} = 9 - \frac{26}{u_{n + 1}} + \frac{24}{u_{n} u_{n + 1}}.
 
\]
 \begin{noliste}{1.}
 \setlength{\itemsep}{4mm}
 \item L'ensemble des suites réelles vérifiant $a u_{n + 2} + b u_{n +
1} + c u_{n}$ est obtenu à l'aide de l'étude de l'équation
caractéristique $a r^{2} + b r + c$ et de son discriminant $\Delta$. \\
 Si $\Delta = 0$, il y a une unique solution $r_{0}$ à l'équation, et
les suites solutions sont les $(u_{n})$ telles que pour tout $n$,
$u_{n} = a r_{0}{n} + b n r_{0}{n}$. \\
 Si $\Delta > 0$, il y a deux solutions distinctes $r_{1}$ et $r_{2}$ à
l'équation, et les suites solutions sont les $(u_{n})$ telles que pour
tout $n$, $u_{n} = a r_{1}{n} + b r_{2}{n}$. \\
 Enfin si $\Delta < 0$, on ne connaît pas les solutions (elles
s'écrivent avec des nombres complexes ou des sinus et cosinus). \\
 \item Il faut garder les valeurs de $u_{n}$ et $u_{n-1}$ pour obtenir
celle de $u_{n + 1}$, donc il faut déjà deux variables $u$ et $v$. \\
 De plus quand on donne la valeur de $u_{n + 1}$, on efface la valeur
de $u_{n}$, qui était dans $u$, et qu'on doit pourtant redonner à $v$
pour l'étape suivante (vous me suivez ?? ?) donc il faut une variable
auxiliaire pour ne pas la perdre. \\
 Cela donne : \\
 var u, v, aux : real ; k, n : integer; \\
 readln (n) \\
 u : = 29/9 ; v : = 3 ; \\
 for k : = 2 to n do \\
 begin \\
 aux : = u; u : = 9 - 26 / u + 24 / ( u $\ast$ v) ; v : = aux ; \\
 end; \\
 writeln (u ) ; \\
\\
 On peut rajouter quelques fioritures pour rendre le programme plus
facile à l'utilisateur ( writeln ('n ?') au début, writeln( ' la valeur
du terme d'ordre n de la suite est ') avant le résultat). \\
 \item Il existe une unique suite vérifiant $a_{0} = 3$ et $a_{n + 1} =
a_{n} \times u_{n}$ (définition par récurrence des suites). \\
 Il faut alors vérifier que $a_{n + 3} = 9 a_{n + 2} - 26 a_{n + 1} +
24 a_{n}$, et que $(a_{n})$ est à valeurs dans $\N^*$. \\
 Pour la relation de récurrence linéaire, on écrit : \\
 $a_{n + 3} = u_{n + 2} a_{n + 2} = \left( 9 - \frac{26}{u_{n + 1}} +
\frac{24}{u_{n} u_{n + 1}} \right) a_{n + 2} = 9 a_{n + 2} - 26 \frac{
a_{n + 2}}{u_{n + 1}} + 24 \frac{ a_{n + 2}}{u_{n} u_{n + 1}}$. \\
\\
 Or $a_{n + 2}{u_{n + 1}} = \frac{ a_{n + 1} u_{n + 1} }{ u_{n + 1} } =
a_{n + 1}$ et $ \frac{a_{n + 2} }{u_{n} u_{n + 1} } = \frac{a_{n + 1}
}{u_{n}} = \frac{ a_{n} u_{n} }{u_{n}} = a_{n}$ donc on a bien : \\
 $a_{n + 3} = 9 a_{n + 2} - 26 a_{n + 1} + 24 a_{n}$. \\
\\
 On montre par récurrence que pour tout $n$, $a_{n} \in \Z$. \\
 Initialisation : $a_{0} = 3 \in \N^*$, $a_{1} = u_{0} a_{0} = 9$ et
$a_{2} = u_{1} a_{1} = 9 \frac{29}{9} = 29$ donc c'est bon. \\
 Hérédité : on suppose qu'il existe $n \geq 2$ tel que $a_{n}$,
$a_{n-1}$ et $a_{n-2}$ soient des entiers naturels non nuls. \\
 Alors $9 a_{n + 2}$, $26 a_{n + 1}$ et $24 a_{n}$ sont des entiers, et
par somme $a_{n + 3}$ est un entier. \\
\\
 Je ne vois pas du tout comment montrer que la suite $(a_{n})$ est à
termes positifs, ce qui est pourtant essentiel pour prouver que la
suite $(u_{n})$ est bien définie. \\
 La question suivante permet de le prouver, mais a priori ici il
faudrait le montrer sans s'en servir.... \\
 \item Récurrence pas trop difficile : \\
 Initialisation : $a_{0} = 3$, et $2^{0} + 3^{0} + 4^{0} = 1 + 1 + 1 =
3$. \\
 $2^{1} + 3^{1} + 4^{1} = 9 = a_{1}$, et $2^{2} + 3^{2} + 4^{2} = 29 =
a_{2}$. \\
 Hérédité : on suppose qu'il existe $n \geq 0$ tel que $a_{k} = 2^{k} +
3^{k} + 4^{k}$ pour $ 0 \leq k \leq n + 2$. \\
 On a alors $a_{n + 3} = 9 ( 2^{n + 2} + 3^{n + 2} + 4^{n + 2} ) - 26 (
2^{n + 1} + 3^{n + 1} + 4^{n + 1} ) + 24 ( 2^{n} + 3^{n} + 4^{n}) =
2^{n + 2} ( 9 - 13 + 6) + 3^{n + 1} ( 27 - 26 + 8) + 4^{n + 1} ( 36 -
26 + 6) = 2 \times 2^{n + 2} + 3^{2} \times 3^{n + 1} + 4^{2} \times
4^{n + 1} = 2^{n + 3} + 3^{n + 3} + 4^{n + 3}$. \\
\\
 Cela prouve que $a_{n} > 0$, puis que $u_{n} = \frac{a_{n +
1}}{a_{n}}$ est bien définie. \\
 \item On a alors $u_{n} = \frac{ a_{n + 1}}{a_{n}} = \frac{2^{n + 1} +
3^{n + 1} + 4^{n + 1} }{2^{n} + 3^{n} + 4^{n}}$. \\
 Le numérateur s'écrit $4^{n + 1} \left( \frac{1}{2^{n + 1} } + \left(
\frac{3}{4} \right)^{n + 1} + 1 \right) \sim 4^{n + 1}$. \\
 De même $2^{n} + 3^{n} + 4^{n} \sim 4^{n}$, donc $u_{n} \sim
\frac{4^{n + 1} }{4^{n}} = 4$ donc $\dlim{n \rightarrow + \infty} u_{n}
= 4$. \\
 \end{noliste}
 \noindent \textbf{\underline{Exercice sans préparation}} \\
\\
 Soient $X_{1}$ et $X_{2}$ deux variables aléatoires définies sur un
espace probabilisé $(\Omega, \mathcal{A}, P)$, indépendantes et de lois
géométriques de paramètres $p_{1}$ et $p_{2}$ respectivement ($p_{i}
\in \ ]0 ; 1[$ pour $i = 1,2$). \\
 On pose $U = X_{1} + X_{2}$ et $T = X_{1} - X_{2}$.
\begin{noliste}{1.}
 \setlength{\itemsep}{4mm}
 \item Commençons avec le calcul de la covariance. \\
 $\Cov ( U, T) = \Cov( X_{1} + X_{2}, X_{1} - X_{2}) = \V(X_{1}) -
\V(X_{2}) = \frac{1-p_{1}}{p_{1}{2}} - \frac{1- p_{2}}{p_{2}{2}} =
\frac{ p_{2}{2} - p_{1} p_{2}{2} - p_{1}{2} + p_{1}{2} p_{2} }{p_{1}{2}
p_{2}{2}} = \frac{ (p_{2} - p_{1}) (p_{1} + p_{2}) + p_{1} p_{2} (
p_{1} - p_{2})}{(p_{1} p_{2} )^{2}} = \frac{p_{1} - p_{2}}{(p_{1} p_{2}
)^{2}} ( p_{1} p_{2} - p_{1} - p_{2})$. \\
 Ce dernier facteur est non nul car $0 < p_{1} p_{2} < p_{2}$ donc
$-p_{2} - p_{1} < p_{1} p_{2} - p_{2} - p_{1} < -p_{1} < 0$. \\
 Enfin on a $p_{1} \neq p_{2}$, donc $p_{1} - p_{2} \neq 0$, et donc
$\Cov ( U,T) \neq 0$, donc $U$ et $T$ ne sont pas indépendantes. \\
 \item Cette fois on a $\Cov (U,V) = \V(X_{1}) - \V(X_{2}) = 0$ ce qui
ne permet pas de conclure. \\
 On a $U (\Omega) = \llb 2 ; + \infty \llb$ et $T(\Omega) = \Z$. \\
 Pour tout $n \geq 2$, pour tout $k \in \Z$, $\Prob( [U = n] \cap [ T =
k] ) = P \left( \left[X_{1} = \frac{n + k}{2}\right] \cap \left[ X_{2}
= \frac{n-k}{2} \right] \right)$. \\
 Prenons une valeur nulle, par exemple avec $n + k$ impair, donc $n =
2$ et $k = 1$. \\
 $\Prob( [U = 2] \cap [T = 1] ) = 0$, et $\Prob\left(\Ev{\Ev{ U =
2}}\right) = \Prob( [ X_{1} = 1] \cap [X_{2} = 1] ) =
\Prob\left(\Ev{\Ev{ X_{1} = 1}}\right) \Prob\left(\Ev{\Ev{X_{2} =
1}}\right) \neq 0$. \\
 De même $ [ \ \Ev{ X_{1} = 2} \cap \Ev{X_{2} = 1}] \subset \Ev{T = 1}$
donc $\Prob\left(\Ev{\Ev{ T = 1}}\right) \geq P [ \ \Ev{ X_{1} = 2}
\cap \Ev{X_{2} = 1}] = \Prob\left(\Ev{\Ev{ X_{1} = 2}}\right)
\Prob\left(\Ev{\Ev{ X_{2} = 1}}\right) \neq 0$. \\
\\
 Les variables $U$ et $T$ ne sont donc pas indépendantes.
 \end{noliste}
 \end{exercice}

 \newpage


 \begin{exercice}{\it (Exercice avec préparation)}~\\
 Soient $(a,b,c) \in \R^{3}$ et $A = \begin{smatrix}
1 & a & b \\
0 & 1 & c \\
0 & 0 & 1 \\
\end{smatrix}
$. \\
 On pose $N = A- I$ et $M = N^{2} - N$ (où $I$ désigne la matrice
identité de $\M{3}$. \\
\\
 Soient $u$ et $v$ les endomorphismes de $\R^{3}$ canoniquement
associés aux matrices $M$ et $N$.
 \begin{noliste}{1.}
 \setlength{\itemsep}{4mm}
 \item Deux matrices $M$ et $N$ sont semblables s'il existe $P$
inversible telle que $M = P N P^{-1}$. \\
 On prouve que deux matrices sont semblables si et seulement si ce sont
deux matrices d'un même endomorphisme dans des bases distinctes, et on
en déduit que les dimensions de leurs noyaux et leurs images sont
égales, que leurs valeurs propres sont égales, et que les sous-espaces
propres associés sont de même dimension. \\
 Enfin on obtient que l'une est inversible si et seulement si l'autre
l'est, et que l'une est diagonalisable siet seulement si l'autre l'est.
\\
 Enfin on sait que si $M$ et $N$ sont semblables, pour tout $n \in \N$
(et même $n \in \Z$ si elles sont inversibles), $M^{n}$ et $N^{n}$ sont
semblables et plus précisément : $M^{n} = P N^{n} P^{-1}$. \\
 \item $A$ est triangulaire donc son unique valeur propre est 1. On en
déduit par l'absurde que $A$ est diagonalisable si et seulement si $A =
I$, donc si et seulement si $a = b = c = 0$. \\
 \item $0 \notin \spc A$ donc $A$ est inversible. \\
 $A$ étant triangulaire, on peut déterminer son inverse avec un pivot
qui sera très rapide : on obtient $A^{-1} = \begin{smatrix}
1 & -a & -b + ac \\
0 & 1 & -c \\
0 & 0 & 1 \\
\end{smatrix}
 = I + M$ en calculant $M = \begin{smatrix}
0 & -a & -b + ac \\
0 & 0 & -c \\
0 & 0 & 0 \\
\end{smatrix}
$. \\
 \item On suppose dans cette question que le rang de $u$ est égal à 2. 
 \begin{noliste}{a)}
 \setlength{\itemsep}{2mm}
 \item $N$ est nilpotente, de la forme $\begin{smatrix}
0 & a & b & \\
0 & 0 & c \\
0 & 0 & 0 \\
\end{smatrix}
$, avec $a \neq 0$ et $c \neq 0$, sinon les trois colonnes seraient
colinéaires et $u$ serait de rang 1. \\
 On a alors $N^{2} = \begin{smatrix}
0 & 0 & a c & \\
0 & 0 & 0 \\
0 & 0 & 0 \\
\end{smatrix}
\neq 0$, et $N^{3} = 0$. \\
 D'où $u^{2} \neq 0$, on prend $x$ tel que $u^{2} (x) \neq 0$. \\
 Soient alors $a,\ b$ et $c$ tels que $a x + b u(x) + c u^{2} (x) = 0$.
\\
 On compose par $u^{2}$, on a $a u^{2} (x) + b u^{3} (x) + c u^{4} (x)
= a u^{2} (x) = u^{2} (0) = 0$, et $u^{2} (x) \neq 0$ donc $a = 0$. \\
 Ensuite on a $b u(x) + cu^{2} (x) = 0$, on compose par $u$ et on a $b
u^{2} (x) + c u^{3} (x) = b u^{2} (x) = u ( 0) = 0$ donc $b = 0$, et
enfin $c u^{2} (x) = 0$ donc $c = 0$. \\
 La famille $(u^{2} (x), u(x), x)$ est donc libre et de cardinal 3 donc
c'est une base.
 \\
\\
 Dans cette base on a $ u( u^{2} (x)) = u ( u^{2} (x)) = u^{3} (x) =
0$. \\
 Ensuite $u( u(x) ) = u^{2} (x)$ et enfin on a $u(x) = u(x)$ donc la
matrice de $u$ est $\begin{smatrix}
0 & 1 & 0 \\
0 & 0 & 1 \\
0 & 0 & 0 \\
\end{smatrix}
$, et $N$ est bien semblable à cette matrice, car elles sont deux
matrices associées au même endomorphisme. \\
 \item $Mat_{ \mathcal{B} } (v) = \begin{smatrix}
0 & 1 & 0 \\
0 & 0 & 1 \\
0 & 0 & 0 \\
\end{smatrix}
^{2} - \begin{smatrix}
0 & 1 & 0 \\
0 & 0 & 1 \\
0 & 0 & 0 \\
\end{smatrix}
 = \begin{smatrix}
0 & -1 & 1 \\
0 & 0 & -1 \\
0 & 0 & 0 \\
\end{smatrix}
$ vérifie les conditions de la question précédente : nilpotente d'ordre
3, triangulaire et de rang 2 : en appliquant cette question, on montre
que cette matrice, et donc $M$, sont semblables à la matrice
$\begin{smatrix}
0 & 1 & 0 \\
0 & 0 & 1 \\
0 & 0 & 0 \\
\end{smatrix}
$, elle-même semblable à $N$. \\
 Finalement $M$ et $N$ sont bien semblables. \\
 \item Il existe alors $P$ inversible telle que $M = P N P^{-1}$. \\
 On en déduit que $A^{-1} = I + P N P^{-1} = P I P^{-1} + P N P^{-1} =
P \left(\Ev{ I + N }\right) P^{-1} = P A P^{-1}$ et $A$ et $A^{-1}$
sont donc semblables. \\
 \end{noliste}
 \end{noliste}
 \noindent \textbf{\underline{Exercice sans préparation}} \\
\\
 Soit $X$ une variable aléatoire que suit la loi de Poisson de
paramètre $\lambda > 0$. \\
 On désigne l'espérance par $E$. \begin{noliste}{1.}
 \setlength{\itemsep}{4mm}
 \item Le théorème de transfert assure que c'est équivalent à la
convergence absolue, et par positivité à la convergence, de la série
$\Sum{n = 0}{+ \infty} \frac{1}{1 + n} \times \frac{\lambda^{n}
e^{-\lambda} }{n!}$. \\
 Enfin on a $ 0 \leq \frac{1}{1 + n} \times \frac{\lambda^{n}
e^{-\lambda} }{n!} \leq \frac{\lambda^{n} e^{-\lambda} }{n!}$ donc par
comparaison des séries à termes positifs (cette dernière suite est le
terme général d'une série exponentielle convergente), la série converge
et l'espérance de $\frac{1}{1 + X}$ existe bien. \\
 \item La majoration précédente montre que $\E\left( \frac{1}{1 + X}
\right) \leq \Sum{n = 0}{+ \infty} \frac{\lambda^{n} e^{-\lambda} }{n!}
= 1$ (somme des probabilités de la loi de Poisson de paramètre
$\lambda$. \\
 Ensuite on écrit que $ 1 + n \geq n$ donc $\frac{1}{1 + n} \leq
\frac{1}{n}$, et $\frac{1}{1 + n} \frac{\lambda^{n} e^{-\lambda}}{n!}
\leq \frac{1}{\lambda} \frac{ \lambda^{n + 1} e^{- \lambda} }{(n +
1)!}$. \\
 On obtient alors $\E\left( \frac{1}{1 + X} \right) \leq
\frac{1}{\lambda} \Sum{n = 0}{+ \infty} \frac{ \lambda^{n + 1} e^{-
\lambda} }{(n + 1)!} \leq \frac{1}{\lambda} \Sum{n = 1}{+ \infty}
\frac{ \lambda^{n} e^{- \lambda} }{n!} \leq \frac{1}{\lambda} \Sum{n =
0}{+ \infty} \frac{ \lambda^{n} e^{- \lambda} }{n!} \leq
\frac{1}{\lambda}$. \\
\\
 On obtient bien que $\E\left( \frac{1}{1 + X} \right) \leq \min \left(
1, \frac{1}{\lambda} \right)$.
 \end{noliste}
 \end{exercice}

 \newpage

 \begin{exercice}{\it (Exercice avec préparation)}~\\
 Une urne contient des boules blanches, noires et rouges. Les
proportions respectives de ces boules sont $b$ pour les blanches, $n$
pour les noires et $r$ pour les rouges ($b + n + r = 1$). \\
 On effectue dans cette urne des tirages successifs indépendants avec
remise. Les proportions de boules restent ainsi les mêmes au cours de
l'expérience. \\
 On modélise l'expérience par un espace probabilisé $(\Omega,
\mathcal{A}, P)$.
 \begin{noliste}{1.}
 \setlength{\itemsep}{4mm}
 \item La loi d'un couple $(X,Y)$ de variables aléatoires discrètes est
la donnée de $(X,Y) (\Omega)$, ensemble des couples $(i,j)$ de valeurs
telles que l'évènement $\Ev{X = i} \cap \Ev{Y = j}$ est possible. \\
 Les lois marginales sont les lois des variables $X$ et $Y$, obtenues à
partir du couple et de la formule des probabilités totales. \\
 \item Pour $k \in \N^*$, on note $Z_{k}$ la variable aléatoire qui
prend la valeur $ + 1$ si une boule blanche est tirée au $k$-ième
tirage. On note $S_{k} = Z_{1} + \dots + Z_{k}$.
 \begin{noliste}{a)}
 \setlength{\itemsep}{2mm} 
 \item $S_{1} = Z_{1}$ vérifie $Z_{1} (\Omega) = \{ -1 ; 0 ; 1 \}$,
avec $\Prob\left(\Ev{\Ev{ Z_{1} = -1}}\right) = n$,
$\Prob\left(\Ev{\Ev{ Z_{1} = 0}}\right) = r$ et $\Prob\left(\Ev{\Ev{
Z_{1} = 1}}\right) = b$, d'espérance $\E(Z_{1}) = b - n$ et de variance
: \\
 $\V(Z_{1}) = b ( b-n-1)^{2} + r (b-n)^{2} + n (b-n + 1)^{2} =
(b-n)^{2} ( b + r + n) + b + n + 2 ( n (b-n) - b (b-n) ) \\
\V(Z_{1}) = (b-n)^{2} + b + n - 2 n^{2} - 2 b^{2} = -n^{2} - b^{2} - 2
b n + b + n \\
\V(Z_{1}) = - (n + b)^{2} + n + b = - (1-r)^{2} + 1-r = (1-r) ( 1 - 1 +
r) = r (1-r)$. \\
\\
 D'où par linéarité de l'espérance et indépendance des tirages pour la
variance $S_{k}$ a pour espérance $k (b-n)$ et pour variance $k r
(1-r)$. \\
 \item $g_{k}(t) = E \left(t^{S_{k}} \right) = E \left( t^{ \Sum{i =
1}{k} Z_{i}} \right) = E \left( \prod\limits_{i = 1}{k} t^{ Z_{i}}
\right) = \prod\limits_{i = 1}{k} E \left( t^{ Z_{i}} \right)$ par
indépendance des variables $Z_{i}$, qui donnent l'indépendance des
variables $t^{Z_{i}}$. \\
 De plus comme elles suivent toutes la même loi, on a $g_{k}(t) =
\left[ E \left( t^{ Z_{1}} \right) \right]^{k}$. \\
\\
 Enfin on a par théorème de transfert, $\E\left( t^{Z_{1}} \right) =
t^{-1} \times n + t^{0} \times r + t^{1} \times b = \frac{n}{t} + r + b
t = \frac{ n + r t + b t^{2}}{t}$. \\
\\
 Enfin on a $g_{k}(t) = \left( \frac{ n + r t + b t^{2}}{t}
\right)^{k}$. \\
 \item $g_{k}(t) = \Sum{a \in S_{k} (\Omega) } t^{a}
\Prob\left(\Ev{\Ev{S_{k} = a}}\right)$ donc $g_{k}'(t) = \Sum{a \in
S_{k} (\Omega) } a t^{k-1} \Prob\left(\Ev{\Ev{ S_{k} = a}}\right)$ et
enfin $g_{k}'(1) = \Sum{a \in S_{k} (\Omega) } a 1^{k-1}
\Prob\left(\Ev{\Ev{ S_{k} = a}}\right) = \Sum{a \in S_{k} (\Omega) } a
\Prob\left(\Ev{\Ev{ S_{k} = a}}\right) = \E( S_{k})$. \\
\\
 On calcule $g_{k}'(t) = k \times \frac{ (r + 2bt) t - n -rt - b
t^{2}}{t^{2}} \times \left( \frac{ n + r t + b t^{2}}{t} \right)^{k-1}
= k \times \frac{b t^{2} - n}{t^{2}} \times \left( \frac{ n + r t + b
t^{2}}{t} \right)^{k-1}$ et enfin $\E( S_{k}) = g_{k}'(1) = k( b -n) (n
+ r + b)^{k-1} = k(b-n) $ car $n + r + b = 1$. \\
 \end{noliste}
 \item 
 \begin{noliste}{a)}
 \setlength{\itemsep}{2mm} 
 \item $X_{1} \suit \mathcal{G} (b)$, $\E(X_{1}) = \frac{1}{b}$ et
$\V(X_{1}) = \frac{1-b}{b^{2}}$. \\
 \item Si $X_{1} = k$, on sait qu'on a pas tiré de boule blanche; on
est donc dans la situation de deux couleurs, noir de proportion
$\frac{n}{n + r}$ et rouge de proportion $\frac{r}{n + r}$. \\
 On a alors $P_{\Ev{X_{1} = k}} ( R_{1} \cap \dots \cap R_{k-1} ) =
\left( \frac{r}{n + r} \right)^{k-1}$. \\
 \item C'est la loi binomiale de paramètres $k-1$ et $\frac{r}{n + r}$.
\\
 \item Avec le système complet d'évènements $\Ev{X_{1} = k}_{k \in
\N^*}$ on a pour tout $i \in \N$ :\\
 $\Prob\left(\Ev{\Ev{W = i}}\right) = \Sum{k = 1}{+ \infty}
\Prob\left(\Ev{\Ev{X_{1} = k}}\right) P_{\Ev{X_{1} = k}} \Ev{W = i} =
\Sum{k = i + 1}{+ \infty} b (r + n)^{k-1} \binom{k}{i} \left(
\frac{r}{r + n} \right)^{i} \left( \frac{n}{r + n} \right)^{k-1-i} =
\Sum{k = i + 1}{+ \infty} b \binom{k}{i} r^{i} n^{k-1-i}$. \\
 \end{noliste}
 \item On note $Y_{1}$ la variable représentant le numéro du tirage
auquel une boule noire sort pour la première fois.
 \begin{noliste}{a)}
 \setlength{\itemsep}{2mm} 
 \item Si $k = l$, la probabilité est nulle (on ne peut pas tirer en
même temps une boule noire et une blanche. \\
 Si $k < l$ on a des rouges en position 1 à $k-1$, une blanche en
position $k$, une blanche ou une rouge en positions $k + 1$ à $l-1$ et
une noire en position $l$. \\
 D'où $P \left(\Ev{ X_{1} = k, Y_{1} = l}\right) = r^{k-1} b (r +
b)^{l-k-1} n$. \\
 Par symétrie si $k >l$, $\Prob\left(\Ev{X_{1} = k, Y_{1} = l}\right) =
r^{l-1} n (r + n)^{k-l-1} b$. \\
\\
 $\Prob\left(\Ev{\Ev{X_{1} = 1}}\right) \Prob\left(\Ev{\Ev{Y_{1} =
1}}\right) \neq 0 = \Prob\left(\Ev{ X_{1} = 1, Y_{1} = l}\right)$ donc
$X_{1}$ et $Y_{1}$ ne sont pas indépendantes. \\
 \item L'une des deux variables vaut toujours 1, et l'autre vaut une
valeur $k \geq 2$. \\
 D'où $X_{1} Y_{1} (\Omega) = \llb 2 ; + \infty \llb$ et
$\Prob\left(\Ev{ X_{1} Y_{1} = k}\right) = \Prob\left(\Ev{ X_{1} = 1,
Y_{1} = k}\right) + \Prob\left(\Ev{ X_{1} = k, Y_{1} = 1}\right) =
b^{k-1} n + n^{k-1} b $. \\
 On obtient en sommant $\E(X_{1} Y_{1}) = n \left( \frac{1}{(1-b)^{2}}
- 1 \right) + n \left( \frac{1}{(1-b)^{2}} - 1 \right) = \frac{1}{n} +
\frac{1}{b} - n-b = \frac{1}{n} + \frac{1}{b} - 1$. \\
\\
 On a ensuite $\Cov (X_{1}, Y_{1}) = E ( X_{1} Y_{1}) - \E(X_{1})
\E(Y_{1}) = \frac{1}{n} + \frac{1}{b} - 1 - \frac{1}{n} - \frac{1}{b} =
-1$. \\
 \end{noliste} 
 \end{noliste}
 \noindent \textbf{\underline{Exercice sans préparation}} \\
\\
 Soient $n \geq 2$ et $(x_{1}, x_{2},\ \dots\, x_{n}) \in \R^{n} -
\{(0,\ \dots\,0) \}$. \\
 On pose $X = \begin{smatrix}
x_{1} \\
x_{2} \\
\vdots \\
x_{n} \\
\end{smatrix}
$, puis $B = {}{t}X X$ et $A = X {}{t} X$. \begin{noliste}{1.}
 \setlength{\itemsep}{4mm}
 \item $B = \Sum{k = 1}{n} x_{k}{2}$ est un réel. \\
 \item $A$ est symétrique donc diagonalisable, on doit trouver la somme
des dimensions égale à $n$. \\
 On étudie l'équation $A Y = \lambda Y \Leftrightarrow (X {}{t} X) Y =
\lambda Y \Leftrightarrow X ({}{t} X Y) = \lambda Y \Leftrightarrow
({}{t} X Y) X = \lambda Y \Leftrightarrow \left( \Sum{k = 1}{n} x_{k}
y_{k} \right) X = \lambda Y$. \\
 Deux possibilités : soit $\lambda = 0$, et on obtient $\Sum{k = 1}{n}
x_{k} y_{k} = 0$. \\
 Soit $\lambda \neq 0$; alors $Y = \frac{ \left( \Sum{k = 1}{n} x_{k}
y_{k} \right)}{\lambda} X$ est colinéaire à $X$. \\
 Calculons alors $ X {}{t} X X = \left( \Sum{k = 1}{n} x_{k}{2} \right)
X$ donc $\lambda_{0} = \left( \Sum{k = 1}{n} x_{k}{2} \right) \neq 0$
est valeur propre et $X$ un vecteur propre associé, et ce qui précède
donne $E_{\lambda_{0}} (A) = \Vect{ X}$. \\
 On a obtenu de plus $\spc ( A) = \{ \lambda_{0}, 0 \}$. \\
\\
 Il reste à déterminer $\ker A$, qui doit être de dimension $n-1$. \\
 Les colonnes de $A$ sont toutes égales à $x_{k} X$, et il existe $i$
tel que $x_{i} \neq 0$. \\
 On a alors $C_{k} = x_{k} X = \frac{x_{k}}{x_{i}} \times x_{i} X =
\frac{x_{k}}{x_{i}} C_{i}$ et $C_{k} - \frac{x_{k}}{x_{i}} C_{1} = 0$.
\\
 On trouve que pour tout $k \neq i$, $e_{k} - \frac{x_{k}}{x_{i}} e_{i}
\in \ker A$. \\
 La famille obtenue est libre car échelonnée et admet $n-1$ vecteurs
donc c'est une base de $\ker A$. \\
\\
 Cet exercice ne peut être résolu qu'en écrivant la matrice $A$
explicitement, afin de comprendre à la vue de la matrice comment
trouver cette base de $\ker A$ (qui semble sortir de nulle part dans
cette correction!!).
 \end{noliste}
 \end{exercice}

 \newpage

 

\end{document}