\documentclass[11pt]{article}%
\usepackage{geometry}%
\geometry{a4paper,
  lmargin=2cm,rmargin=2cm,tmargin=2.5cm,bmargin=2.5cm}

\input{../../../macros.tex}

% \renewcommand{\thesection}{\Roman{section}.\hspace{-.3cm}}
% \renewcommand{\thesubsection}{\Alph{subsection}.\hspace{-.2cm}}

\pagestyle{fancy} %
\lhead{ECE2 \hfill Mathématiques \\} %
\chead{\hrule} %
\rhead{} %
\lfoot{} %
\cfoot{} %
\rfoot{\thepage} %

\renewcommand{\headrulewidth}{0pt}% : Trace un trait de séparation
                                    % de largeur 0,4 point. Mettre 0pt
                                    % pour supprimer le trait.

\renewcommand{\footrulewidth}{0.4pt}% : Trace un trait de séparation
                                    % de largeur 0,4 point. Mettre 0pt
                                    % pour supprimer le trait.

\setlength{\headheight}{14pt}

\title{\bf \vspace{-1.6cm} HEC 2009} %
\author{} %
\date{} %
\begin{document}
\maketitle %
\vspace{-1.2cm}\hrule %
\thispagestyle{fancy}

\vspace*{.2cm}

%%DEBUT

\begin{exerciceAP}~\\
  On étudie la vente d'un certain type de produit sur internet sur
  trois sites A, B, C et on fait les constatations suivantes :
  \begin{itemize}
  \item si un client choisit le site A pour un achat, il choisit
    indifféremment A, B ou C pour l'achat suivant,
  \item si un client fait un achat auprès du site B, il fait l'achat
    suivant sur le même site B,
  \item si un client fait un achat sur le site C, il choisira pour
    l'achat suivant le site A avec une probabilité 1/12, le site B
    avec une probabilité 7/12 et le site C avec une probabilité 1/3.
  \end{itemize}
  Au départ le client choisit au hasard l'un des trois sites. \\
  On suppose que l'expérience est modélisée par un espace probabilisé
  $(\Omega , \A , P)$. \\
  Pour $n \in \N^*$, on désigne par $p_n$, $q_n$ et $r_n$ les
  probabilités pour que, au $n$-ième achat, le client se fournisse
  respectivement auprès de A, B et C.
  \begin{noliste}{1.}
    \setlength{\itemsep}{2mm}
  \item Question de cours : Enoncer la formule des probabilités
    totales.
  \item Quelles sont les valeurs de $p_1$, $q_1$ et $r_1$?
  \item Pour tout $n \in \N^*$, donner une relation entre $p_n$, $q_n$ et $r_n$.
  \item Exprimer respectivement $p_{n+1}$, $q_{n+1}$ et $r_{n+1}$ en
    fonction des trois réels $p_n$, $q_n$ et $r_n$.
  \item Pour $n \geq 2$, exprimer $p_n$ en fonction de $r_n$ et
    $r_{n-1}$.
  \item Prouver que la suite $(r_n)_{n \in \N^*}$ est une suite
    récurrente linéaire. Donner l'expression de $r_n$, puis $p_n$ et
    $q_n$ en fonction de $n$.
  \item Étudier la convergence des trois suites $(r_n)$, $(p_n)$ et $(q_n)$. \\
  \end{noliste}
\end{exerciceAP}


\begin{exerciceSP}~\\
  Donner un exemple de matrice $M$ non nulle telle que $(I , M , {}^t
  M)$ soit une famille liée. \\
  Dans quel cas de telles matrices sont diagonalisables?
\end{exerciceSP}

%\newpage

\begin{exerciceAP}~
  \begin{noliste}{1.}
    \setlength{\itemsep}{2mm}
  \item Question de cours : Loi géométrique, espérance et variance.
  \item Soit $x$ un réel de $]0 ; 1[$.
    \begin{noliste}{a)}
    \setlength{\itemsep}{2mm} 
    \item Établir, pour tout $n \in \N^*$, l'égalité :
      \[
      \dint{0}{x} \frac{1-t^n}{1-t} = \Sum{k=1}{n} \frac{x^k}{k}
      \]
    \item Montrer que $\dlim{n \rightarrow +\infty} \dint{0}{x}
      \frac{t^n}{1-t}\ dt = 0$.
    \item En déduire la convergence de la série de terme général
      $\frac{x^k}{k}$ ainsi que l'égalité :
      \[
      \Sum{k=1}{+\infty} \frac{x^k}{k} = - \ln (1-x).
      \]
    \end{noliste}
  \item Soit $X$ une variable aléatoire réelle définie sur un espace
    probabilisé $(\Omega , \A , P)$ qui suit une loi
    géométrique de paramètre $p$ ($0<p<1$). \\
    On pose $Y = \frac{1}{X}$.
    \begin{noliste}{a)}
    \setlength{\itemsep}{2mm}
    \item Déterminer $Y(\Omega)$ et la loi de probabilité de $Y$.
    \item Établir, pour tout entier $m$ de $\N^*$, l'existence du
      moment d'ordre $m$, $\E(Y^m)$, de $Y$.
    \item Calculer $\E(Y)$ en fonction de $p$.
    \end{noliste}
  \end{noliste}
\end{exerciceAP}


\begin{exerciceSP}~\\
  Soit la matrice $A = 
  \begin{smatrix} 
    1 & 2 \\ 
    3 & 4 \\ 
    -1 & 4
  \end{smatrix}$.
  \begin{noliste}{1.}
    \setlength{\itemsep}{2mm}
  \item Existe-t-il $B \in \mathcal{M}_{2,3} (\R)$ telle que $AB =
    I_3$ ?
  \item Existe-t-il $C \in \mathcal{M}_{2,3} (\R)$ telle que $CA =
    I_2$ ?
  \end{noliste}
\end{exerciceSP}


%\newpage


\begin{exerciceAP}~\\
  On considère la suite réelle $(u_n)_{n \in \N}$ définie par : 
  \[
  u_0 = 3,\ u_1 = \frac{29}{9} \text{ et } \forall n \in \N,\ u_{n+2}
  = 9 - \frac{26}{u_{n+1}} + \frac{24}{u_n u_{n+1}}.
  \]
  \begin{noliste}{1.}
    \setlength{\itemsep}{2mm}
  \item Question de cours : Enoncer les résultats concernant les
    suites récurrentes linéaires d'ordre 2.

  \item Écrire une fonction en Pascal permettant de calculer la valeur
    du terme $u_n$ pour tout $n \in \N$ entré par l'utilisateur.

  \item Montrer qu'il existe une unique suite réelle $(a_n)_{n \in
      \N}$ à valeurs dans $\N^*$ telles que :
    \[
    \left\{ \begin{array}{c}
        a_0 = 3 \\
        \forall n \in \N,\ u_n = \frac{a_{n+1}}{a_n} \\
        \forall n \in \N,\ a_{n+3} = 9 a_{n+2} - 26 a_{n+1} + 24 a_n
      \end{array} 
    \right.
    \]
  \item Prouver que pour tout $n \in \N$, $a_n = 2^n + 3^n + 4^n$.
  \item Expliciter $u_n$ en fonction de $n$, puis $\dlim{n \rightarrow
      +\infty} u_n$.
  \end{noliste}
\end{exerciceAP}


\begin{exerciceSP}~\\
  Soient $X_1$ et $X_2$ deux variables aléatoires définies sur un
  espace probabilisé $(\Omega , \A , P)$, indépendantes et de lois
  géométriques de paramètres $p_1$ et $p_2$ respectivement ($p_i \in
  ]0 ; 1[$ pour $i = 1,2$). \\
  On pose $U = X_1 + X_2$ et $T = X_1 - X_2$.
  \begin{noliste}{a)}
    \setlength{\itemsep}{2mm}
  \item On suppose $p_1 \neq p_2$. Les variables aléatoires $U$ et $T$
    sont-elles indépendantes?
  \item On suppose $p_1 = p_2 = p$. Les variables aléatoires $U$ et
    $T$ sont-elles indépendantes?
  \end{noliste}
\end{exerciceSP}


%\newpage


\begin{exerciceAP}~\\
  Soient $(a,b,c) \in \R^3$ et $A = 
  \begin{smatrix} 
    1 & a & b \\ 
    0 & 1 & c \\ 
    0 & 0 & 1 
  \end{smatrix}$.\\
  On pose $N = A- I$ et $M = N^2 - N$ (où $I$ désigne la matrice
  identité de $\mathcal{M}_3(\R)$.
  \\[.2cm]
  Soient $u$ et $v$ les endomorphismes de $\R^3$ canoniquement
  associés aux matrices $N$ et $M$.
  \begin{noliste}{1.}
    \setlength{\itemsep}{2mm}
  \item Question de cours : Matrices semblables, définition et propriétés.
  \item Étudier la diagonalisabilité de $A$.
  \item Montrer que $A$ est inversible et exprimer $A^{-1}$ en
    fonction de $I$ et de $M$.
  \item On suppose dans cette question que le rang de $u$ est égal à 2. 
    \begin{noliste}{a)}
    \setlength{\itemsep}{2mm}
    \item Montrer l'existence d'un vecteur $x$ de $\R^3$ tel que
      $\mathcal{B} = (u^2(x),u(x),x)$ soit une base de $\R^3$.
      \\[.2cm]
      En déduire que $N$ est semblable à $
      \begin{smatrix}
        0 & 1 & 0 \\ 
        0 & 0 & 1 \\ 
        0 & 0 & 0 
      \end{smatrix}$.
    \item Exprimer la matrice de $v$ dans la base $\mathcal{B}$ et en
      déduire que $M$ et $N$ sont semblables.
    \item Conclure que $A$ et $A^{-1}$ sont aussi semblables.
    \end{noliste}
  \end{noliste}
\end{exerciceAP}


\begin{exerciceSP}~\\
  Soit $X$ une variable aléatoire que suit la loi de Poisson de
  paramètre $\lambda > 0$. \\
  On désigne l'espérance par $E$.
  \begin{noliste}{1.}
    \setlength{\itemsep}{2mm}
  \item Établir l'existence de $E \left( \frac{1}{1+X} \right)$.
  \item Montrer que $E \left( \frac{1}{1+X} \right) \leq \min \left( 1
      , \frac{1}{\lambda} \right)$.
  \end{noliste}
\end{exerciceSP}


%\newpage


\begin{exerciceAP}~\\
  Une urne contient des boules blanches, noires et rouges. Les
  proportions respectives de ces boules sont $b$ pour les blanches,
  $n$ pour les noires et $r$ pour les rouges ($b + n + r = 1$).\\
  On effectue dans cette urne des tirages successifs indépendants avec
  remise. Les proportions de boules restent ainsi les mêmes au cours
  de l'expérience. \\
  On modélise l'expérience par un espace probabilisé $(\Omega , \A ,
  P)$.

  \begin{noliste}{1.}
    \setlength{\itemsep}{2mm}
  \item Question de cours : Loi d'un couple de variables aléatoires
    discrètes. Lois marginales.
  \item Pour $k \in \N^*$, on note $Z_k$ la variable aléatoire qui
    prend la valeur $+1$ si une boule blanche est tirée au $k$-ième
    tirag, $-1$ si une boule noire est tirée au $k$-ième tirage et 0
    si une boule rouge est tirée au $k$-ième tirage. On note $S_k =
    Z_1 + \dots + Z_k$.
    \begin{noliste}{a)}
    \setlength{\itemsep}{2mm} 
    \item Trouver la loi de probabilité de $S_1$. Calculer son
      espérance et sa variance. En déduire l'espérance et la variance
      de $S_k$.
    \item Pour tout réel $t$ strictement positif et pour tout $k$ de
      $\N^*$, on pose $g_k(t) = E \left(t^{S_k} \right)$. \\
      Expliciter $g_k(t)$ en fonction de $t$ et de $k$.
    \item Montrer que $g_k'(1) = \E(S_k)$ et retrouver le résultat de
      la question $(a)$.
    \end{noliste}

  \item 
    \begin{noliste}{a)}
    \setlength{\itemsep}{2mm}
    \item On note $X_1$ la variable aléatoire représentant le numéro
      du tirage auquel une boule blanche sort pour la première
      fois. Trouver la loi de probabilité de $X_1$. Calculer son
      espérance et sa variance.
    \item Sachant que $X_1 = k$, quelle est la probabilité de tirer
      une boule rouge à chacun des $k-1$ premiers tirages?
    \item On note $W$ la variable aléatoire représentant le nombre de
      boules rouges tirées avant l'obtention de la première boule
      blanche. Quelle est la loi conditionnelle de $W$ sachant $X_1 =
      k$?
    \item En déduire la loi de $W$ (sous forme d'une somme qu'on ne
      cherchera pas à calculer).
    \end{noliste}

  \item On note $Y_1$ la variable représentant le numéro du tirage
    auquel une boule noire sort pour la première fois.
    \begin{noliste}{a)}
    \setlength{\itemsep}{2mm}
    \item Trouver, pour tout couple d'entiers strictement positifs
      $(k,l)$, la probabilité de l'évènement $[ X_1 = k\ ,\ Y_1 = l]$
      (on pourra distinguer selon que $k > l,\ k=l$ ou $k < l$). \\
      Les variables aléatoires $X_1$ et $Y_1$ sont-elles
      indépendantes?
    \item On se place, pour cette question, dans le cas particulier où
      $r=0$ (c'est-à-dire qu'il n'y a pas de boule rouge). Calculer
      alors la covariance de $X_1$ et $Y_1$.
    \end{noliste} 
  \end{noliste}
\end{exerciceAP}


\begin{exerciceSP}~\\
  Soient $n \geq 2$ et $(x_1 , x_2,\ \dots\ , x_n) \in \R^n - \{(0,\
  \dots\ ,0) \}$. \\
  On pose $X = 
  \begin{smatrix} 
    x_1 \\
    x_2 \\ 
    \vdots \\
    x_n
  \end{smatrix}$, puis $B = {}^tX X$ et $A = X {}^t X$.
  \begin{noliste}{1.}
    \setlength{\itemsep}{2mm}
  \item Écrire la matrice $B$.
  \item Déterminer les vecteurs propres et les sous-espaces propres de
    la matrice $A$.
  \end{noliste}
\end{exerciceSP}

%%FIN

\end{document}
