%\documentclass[11pt]{article}%
%\usepackage{geometry}%
%\geometry{a4paper,
% lmargin = 2cm,rmargin = 2cm,tmargin = 2.5cm,bmargin = 2.5cm}


\documentclass[11pt]{article}%
\usepackage{geometry}%
%\geometry{a4paper,
%  lmargin=2cm,rmargin=2cm,tmargin=2.5cm,bmargin=2.5cm}
\geometry{a4paper,twocolumn,landscape,
  lmargin=2cm,rmargin=2cm,tmargin=2cm,bmargin=2cm}

\setlength{\columnsep}{50pt}

\input{../../macros.tex}

\pagestyle{fancy} %
\lhead{ECE2 \hfill Mathématiques\\
} %
\chead{\hrule} %
\rhead{} %
\lfoot{} %
\cfoot{} %
\rfoot{\thepage} %

\renewcommand{\headrulewidth}{0pt}% : Trace un trait de séparation
 % de largeur 0,4 point. Mettre 0pt
 % pour supprimer le trait.

\renewcommand{\footrulewidth}{0.4pt}% : Trace un trait de séparation
 % de largeur 0,4 point. Mettre 0pt
 % pour supprimer le trait.

\setlength{\headheight}{14pt}

\title{\bf \vspace{- 2cm} Oraux HEC 2007 - 2018 :\\ questions de cours} %
\author{} %
\date{} %
\begin{document}

\maketitle %
\vspace{- 1.4cm}\hrule %
\thispagestyle{fancy}

\vspace*{.2cm}

\section*{Analyse}

\subsection*{Suites}

\begin{noliste}{1.}
\item Convergence et divergence des suites réelles monotones.
\item Énoncer les résultats concernant les suites récurrentes
  linéaires d'ordre~2.
\end{noliste}

\subsection*{Séries}

\begin{noliste}{1.}
\item Définition de la convergence d'une série numérique (à termes réels).
\item Définition d'une série convergente. Pour quels réels $x>0$ la
  série de terme général $(\ln x)^n$ est-elle convergente ? Donner sa
  somme en cas de convergence.
\item Donner des critères de convergence de séries à termes positifs.
\end{noliste}


\subsection*{Intégrales}

\begin{noliste}{1.}
\item Définition de la convergence d'une intégrale impropre.
\item Donner des critères de convergence d'une intégrale impropre.

  Préciser la nature de l'intégrale $\dint{a}{+\infty}
  \dfrac{\dt}{t^\alpha}$, où $a>0$ et $\alpha \in \R$.
\item Soit $I$ un intervalle de $\R$, $a\in I$, et $f:I\to \R$
  continue sur $I$. Donner les propriétés de l'application $x\in I \
  \mapsto \ \dint{a}{x} f(t)\dt.$
\end{noliste}


\subsection*{Fonctions}

\begin{noliste}{1.}
\item Définir le fait que deux fonctions soient équivalentes au
  voisinage de~$+\infty$.
\item Définition et représentation graphique de la fonction partie entière.
\item Définition de la continuité en un point d'une fonction réelle
  d'une variable réelle.
\item Rappeler la définition d'une bijection. Que peut-on dire de la
  composée de deux bijections ?
\item Quel est le lien entre la continuité d'une fonction et sa
  dérivabilité ?
\item Définition et propriétés des fonctions de classe de
  $\mathcal{C}^p$ ($p\in \N$).
\item Convexité d'une fonction définie sur un intervalle de $\R$.
\item Soit $f:\R \to \R$ et $a\in \R$. Que signifie graphiquement le
  fait que $a$ soit un point d'inflexion de la courbe représentative
  de $f$ ? Quelles sont les méthodes pour le calculer ?
\item Formule de Taylor-Young.
\item Soit $f$ une fonction à valeurs réelles de classe
  $\mathcal{C}^2$ définie sur une partie de $\R^2$. Rappeler la
  définition d'un point critique de $f$, et donner une condition
  suffisante pour que $f$ possède un extremum local en ce point.
\end{noliste}

\section*{Probabilités}

\subsection*{Lois usuelles}

\begin{noliste}{1.}
\item Définition et propriétés de la loi uniforme sur $[a,b]$.
\item Définition et propriétés de la loi exponentielle.
\item Définition et propriétés de la loi géométrique.
\item Définition d'un schéma binomial.
\item Définition et propriétés de la loi de Bernoulli et de la loi
  binomiale.
\item Donner l'allure du graphe de la \underline{fonction de
    répartition} de la loi normale centrée réduite.
\end{noliste}

\subsection*{Formules classiques}

\begin{noliste}{1.}
\item Formule des probabilité composées.
\item Formule des probabilités totales.
\item Définition de la loi d'un couple de variables aléatoires
  discrètes. Définitions des lois marginales et conditionnelles.
\item Formule de Bayes.
\end{noliste}

\subsection*{Indépendance}

\begin{noliste}{1.}
\item Définition de l'indépendance de deux variables aléatoires finies.
\item Définition de l'indépendance de deux variables aléatoires
  discrètes. Lien entre indépendance et covariance ?
\item Définition de l'indépendance de $n$ variables aléatoires discrètes.
\end{noliste}


\subsection*{Fonction de répartition et fonction de densité}

\begin{noliste}{1.}
\item Définition et propriétés de la fonction de répartition d'une
  variable aléatoire à densité.
\item Définition d'une variable aléatoire à densité et propriétés de
  sa fonction de répartition.
\item Définition d'une densité de probabilité.
\end{noliste}

\subsection*{Espérance, moments et variance}


\begin{noliste}{1.}
\item Espérance et variance d'une variable aléatoire discrète finie :
  définition et interprétation.
\item Moment d'ordre $r$ d'une variable aléatoire à densité :
  définition et existence.
\item Formule de Koenig-Huygens.
\item Définition et propriétés de la covariance de deux variables
  aléatoires discrètes.
\end{noliste}


\subsection*{Convergence, approximation et estimateurs}

\begin{noliste}{1.}
\item Définition de la convergence en loi d'une suite de variables aléatoires.
\item Étant donnée une variable aléatoire $X$ suivant la loi normale
  centrée réduite, écrire sous forme d'intégrale la probabilité que
  $X$ appartienne à un segment $[a,b]$ donné. Dans quel théorème cette
  probabilité apparait-elle comme une limite ?
\item Énoncé de l'inégalité de Markov.
\item Loi faible des grands nombres.
\item Définition d'un estimateur sans biais d'un paramètre inconnu.
\item Estimateur, biais et risque quadratique.
\end{noliste}

\section*{Algèbre linéaire}

\subsection*{Généralités}

\begin{noliste}{1.}
\item Que peut-on dire du degré de la somme et du produit de deux polynômes ?
\item Qu'appelle-t-on système de Cramer ?
\item Définition d'un isomorphisme d'espaces vectoriels.
\item Définition de la dimension d'un espace vectoriel.
\item Soit $E$ un espace vectoriel de dimension finie $n$. Définir ce
  qu'est une famille génératrice de $E$. Que peut-on dire de son
  cardinal ?
\item Soit $E$ un espace vectoriel de dimension finie $n$. Définir ce
  qu'est une famille libre de $E$. Que peut-on dire de son cardinal ?
\item Théorème du rang.
\end{noliste}

\subsection*{Réduction}

\begin{noliste}{1.}
\item Définition d'un endomorphisme diagonalisable.
\item Définition d'une matrice diagonalisable.
\item Matrices semblables : définition et propriétés.
\item Donner la définition d'une valeur propre pour un endomorphisme.
\item Rappeler la définition d'un vecteur propre d'un
  endomorphisme. Énoncer la propriété relative à une famille de
  vecteurs propres d'un endomorphisme associés à des valeurs propres
  distinctes.
\item Définition d'un polynôme annulateur d'une matrice. Lien avec les
  valeurs propres ?
\item Condition nécessaire et suffisante pour qu'un endomorphisme soit
  diagonalisable.
\item \'Enoncer deux conditions suffisantes de diagonalisabilité d'une matrice.
\end{noliste}







\end{document}

