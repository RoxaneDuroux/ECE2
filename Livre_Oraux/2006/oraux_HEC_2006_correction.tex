\documentclass[11pt]{article}%
\usepackage{geometry}%
\geometry{a4paper,
  lmargin=2cm,rmargin=2cm,tmargin=2.5cm,bmargin=2.5cm}

\input{../../../macros.tex}

% \renewcommand{\thesection}{\Roman{section}.\hspace{-.3cm}}
% \renewcommand{\thesubsection}{\Alph{subsection}.\hspace{-.2cm}}

\pagestyle{fancy} %
\lhead{ECE2 \hfill Mathématiques \\} %
\chead{\hrule} %
\rhead{} %
\lfoot{} %
\cfoot{} %
\rfoot{\thepage} %

\renewcommand{\headrulewidth}{0pt}% : Trace un trait de séparation
                                    % de largeur 0,4 point. Mettre 0pt
                                    % pour supprimer le trait.

\renewcommand{\footrulewidth}{0.4pt}% : Trace un trait de séparation
                                    % de largeur 0,4 point. Mettre 0pt
                                    % pour supprimer le trait.

\setlength{\headheight}{14pt}

\title{\bf \vspace{-1.6cm} HEC 2006} %
\author{} %
\date{} %
\begin{document}
\maketitle %
\vspace{-1.2cm}\hrule %
\thispagestyle{fancy}

\vspace*{.2cm}

%%DEBUT

\begin{exercice}{\it (Exercice avec préparation)}~\\
  Soient $X$ et $Y$ deux variables aléatoires indépendantes à
  densité continues.\\
  Soit $U=\min \left( X,Y\right) $ et $V=\max \left( X,Y\right)$.\\
  Soient $F_{X}$, $F_{Y}$, $F_{U}$, $F_{V}$ les fonctions de répartition
  de $X$, $Y$, $U$ et $V$ respectivement.

  \begin{noliste}{1.}
  \item Définition et propriétés de la fonction de ré%
    partition d'une variable aléatoire.

  \item Montrer que $\forall t\in \R,\mathbb{\ F}_{U}\left( t\right)
    =1-\left( 1-F_{X}\left( t\right) \right) \left( 1-F_{Y}\left( t\right)
    \right) $.

  \item Établir une relation analogue entre $F_{V},\ F_{X}$ et
    $F_{Y}$.

  \item On suppose à présent que $X$ et $Y$ suivent la loi
    exponentielle de paramètre 1.

    \begin{noliste}{a)}
    \item Quelle est la loi de $U$ ? Que vaut $\Prob\left( U=X\right) $ ?

    \item Montrer que $V$ a même loi que $Z=X+Y$. \\
      En déduire l'espérance et la variance de $V$.
    \end{noliste}
  \end{noliste}



\end{exercice}

\addtocounter{exercice}{-1}
\begin{exercice}{\it (Exercice sans préparation)}~\\
  Soit $A$ la matrice de $\mathcal{M}_{3}\left( \R\right) $ définie
  par : $A=%
  \begin{smatrix}
    0 & 1 & 1 \\
    1 & 0 & 1 \\
    0 & 0 & 1%
  \end{smatrix}%
  $.
  \begin{noliste}{1.}
  \item La matrice $A$ est-elle diagonalisable ? La matrice $A$
    est-elle inversible ?
  \item Déterminer tous les entiers naturels $p$ et $q$ tels que $%
    A^{2p+1}=A^{2q}$
  \item Existe-t-il un entier $n$ tel que $M^{n}=A$ si
    \begin{noliste}{a)}
    \item $M=\left(
        \begin{array}{ccc}
          0 & 1 & 2 \\
          1 & 1 & 1 \\
          1 & 2 & 3%
        \end{array}%
      \right) $
    \item $M_{2}=\left(
        \begin{array}{ccc}
          0 & 1 & 0 \\
          1 & -1 & 1 \\
          0 & 1 & 1%
        \end{array}%
      \right) $
    \end{noliste}    
  \end{noliste}
\end{exercice}


\newpage


\begin{exercice}{\it (Exercice avec préparation)}~\\
  Soit $f$ l'endomorphisme de $\R^{3}$ dont la matrice dans la base
  canonique s'écrit : $A=%
  \begin{smatrix}
    -1 & 0 & 1 \\
    0 & 0 & 0 \\
    1 & 0 & -1%
  \end{smatrix}%
  $
  \begin{noliste}{1.}
  \item Définition et propriétés des matrices de passage.

  \item Donner une base et la dimension de $\ker \left( f\right)$.

  \item Donner une base et la dimension de $\im \left( f\right)$.

  \item Donner les valeurs propres et les sous espaces propres de $f$.

  \item $f$ est-il diagonalisable ?

  \item Calculer $A^{n}$ pour $n\in \N^{\ast }.$

  \item On note $I$ la matrice identité dans la base canonique.\\
    Déterminer les réels $a$ tels que $\left( A-aI\right) ^{2}=I.$
  \end{noliste}
\end{exercice}

\addtocounter{exercice}{-1}
\begin{exercice}{\it (Exercice sans préparation)}~\\
  Soient $X_{1},\cdots ,X_{n}$, des variables aléatoires indépendantes
  de même loi définie :
  \[
  \Prob\left( X_{i}=-1\right) =\Prob\left( X_{i}=0\right)
  =\Prob \left( X_{i}=1\right) =\frac{1}{3}
  \]
  On définit alors des variables aléatoires $\left( Y_{i}\right)
  _{1\leq i\leq n}$ et $\left( Z_{i}\right) _{1\leq i\leq n}$ par
  :\\
  $Y_{i}=1$ si $X_{i}=1$ et $Y_{i}=0$ sinon.\\
  $Z_{i}=1$ si $X_{i}=0$ et $Z_{i}=0$ sinon.\\
  On pose $T_{1}=\Sum{i=1}{n}Y_{i},\quad T_{2}=\Sum{i=1}{n}Z_{i}$ et
  $U=T_{1}+T_{2}$.\\
  Déterminer $\Prob_{\left( T_{1}=t_{1}\right) \cap \left(
      T_{2}=t_{2}\right) }\left( X_{i}=1\right) $.\\
  Déterminer $\Prob_{\left( U=k\right) }\left( T_{1}=t_{1}\right)
  $ (avec $0\leq t_{1}\leq k\leq n$ et expliquer ce résultat.
\end{exercice}


\newpage


\begin{exercice}{\it (Exercice avec préparation)}~\\
  Soit $E$ l'ensemble des fonctions $f$ de $\R^{+}$ dans $\R$ de
  classe $C^{2}$ telles que
  \[
  \forall x\in \R^{+},\quad f^{\prime \prime }\left( x\right) -\left(
    1+x^{4}\right) f\left( x\right) =0
  \]
  On admet que $E$ contient une unique fonction $f_{0}$ vérifiant $%
  f_{0}\left( 0\right) =f^{\prime }\left( 0\right) =1$

  \begin{noliste}{1.}
  \item Rappeler la définition et les propriétés des fonctions
    convexes et montrer que $f_{0}^{2}$ est convexe.

  \item Montrer que $\forall t\in \R^{+},\ f_{0}\left( t\right) \geq 1$

  \item Montrer l'existence de l'intégrale $\dint{0}{+\infty
    }\dfrac{1}{ f_{0}^{2}\left( t\right) }dt$.\\[.2cm]
    On définit $f_{1}$par $\forall x\in \R^{+}:f_{1}\left( x\right)
    =f_{0}\left( x\right) \dint{x}{+\infty }\dfrac{1}{f_{0}^{2}\left(
        t\right) }dt$.

  \item Montrer que $f_{1}\in E$ et que $f_{1}$ est bornée.
  \end{noliste}
\end{exercice}

\addtocounter{exercice}{-1}
\begin{exercice}{\it (Exercice sans préparation)}~\\
  Soit $\left( X_{n}\right) $ une suite de variables aléatoires
  indépendantes suivant une même loi de Bernoulli de paramètre $p$,
  $0<p<1$.\\
  On pose $Y_{n}=X_{n}X_{n+1}$ et \ $U_{n}=Y_{1}+\cdots +Y_{n}$.
  \begin{noliste}{1.}
  \item Déterminer la loi de $Y_{n}$.
  \item Les variables $Y_{i}$, sont-elles deux à eux indépendantes
  \item Calculer $\E\left( U_n\right) $ et $\V\left( U_{n}\right) $
  \item Étudier la convergence de la suite $\left(
      \frac{U_{n}}{n}\right) $.
  \end{noliste}
\end{exercice}


\newpage


\begin{exercice}{\it (Exercice avec préparation)}~\\
  Soient $\theta \in \left[ -2;2\right] $ et $X$ une variable
  aléatoire à densité $f_{\theta }$ définie par $f_{\theta }\left(
    x\right) =\theta x-\frac{\theta }{2}+1$ si $x\in \left[
    0,1\right]$ et $0$ sinon.

  \begin{noliste}{1.}
  \item Donner la définition et des exemples d'esimateurs.

  \item Montrer que $X$ admet une espérance et une variance que l'on
    calculera.

  \item On admet que $\forall k\in \N,\mathbb{\ }\dint{0}{+\infty
    }u^{k}e^{-u}du=k!$.\\
    Montrer que la variable aléatoire $Y=-\ln \left( X\right) $ admet
    une espérance et une variance que l'on calculera (on pourra
    effectuer le changement de variable défini par la fonction
    $x\rightarrow -\ln \left( x\right) $ sur un intervalle adéquat)

    On considère un échantillon de $n$ variables aléatoires
    indépendantes de même loi que $X$ et on pose $\widehat{X_{n}} =
    \dfrac{X_{1}+\cdots +X_{n}}{n}$ et $T_{n} = 12\left(
      \widehat{X_{n}}-\dfrac{1}{2}\right) $.

  \item Montrer que $T_{n}$ est un estimateur sans biais de $\theta $.
  \end{noliste}
\end{exercice}

\addtocounter{exercice}{-1}
\begin{exercice}{\it (Exercice sans préparation)}~\\
  Soient $a>0$ et $f$ définie sur $\left( \R_{+}^{\ast }\right) ^{2}$
  par $f\left( x,y\right) =x^{2}+y^{2}+\dfrac{a}{xy}$.\\
  Déterminer les extrema de $f$.
\end{exercice}


\newpage


\begin{center} 
  Petits exercices à utiliser en cas de besoin. \\
  La forme de l'oral a changé depuis, ils ne sont pas tellement
  caractéristiques dans leur forme \\
  mais intéressants sur le fond. 
\end{center}

\rule{15cm}{0.1cm}

On dispose d'urnes $U_{1},\ U_{2},\ \cdots ,U_{n},\cdots .$ La premi%
ère $U_{1}$ contient une boule noire, une boule blanche et une boule
$B$ de couleur. inconnue. Les suivantes $U_{2},\ \cdots ,U_{n},\cdots$
contiennent une boule blanche et une boule noire.

On tire une première boule de l'urne $U_{1}$ qu'on remet dans
$U_{2}$..  Puis on tire une deuxième boule de $U_{2}$ qu'on remet
clans $U_{3}$ etc.

On désigne par $p_{n}$ \ la probabilité que la $n^{i\grave{e}me}$ \
boule tirée (de $U_{n}$) soit blanche.

\begin{noliste}{1.}
\item Dans cette question, on suppose que $p_{1000}=\frac{1}{2}-\left( \frac{%
      1}{3}\right) ^{999}$. \\
  La boule $B$ était-elle blanche ?

\item Dans cette question, on suppose que. pour tout $n$ supérieur
  à $1000$ on a l'égalité $p_{n}=\dfrac{2}{3}-\dfrac{1}{2}\dfrac{%
    1-\left( \frac{1}{3}\right) ^{n}}{1+\left( \frac{1}{3}\right)
    ^{n}}$

  La boule $B$ était-elle noire.
\end{noliste}

\rule{15cm}{0.1cm}

On dispose de $7$ Euros. Chaque semaine a lieu une loterie de 100
billets dont 10 sont gagnants. Chaque billet co\^{u}te 1 Euro.

\begin{noliste}{1.}
\item Dans cette question on veut, maximiser la chance de gagner au moins
  une fois. Discuter l'affirmation "On a intérêt, à acheter sept
  billets la première semaine plut\^{o}t que d'acheter un billet pendant
  sept semaines".

\item On veut maximiser le nombre de billets gagnants achetés. Discuter
  l'affirmation "On a intérêt à acheter sept billets la premiè%
  re semaine plut\^{o}t que d'acheter un billet pendant sept semaines".
\end{noliste}

\rule{15cm}{0.1cm}

Vous disposez d'une trousse contenant 10 stylos dont un seul
fonctionne.

\begin{noliste}{1.}
\item Vous en essayez un (au hasard), puis s'il y a échec, un deuxiè%
  me, s'il y a échec, vous remettez le premier et vous tirez \ (au hasard)
  le troisième puis, s'il y a échec, vous remettez le deuxième et
  vous tirer (au hasard) le quatrième, puis...\\
  Combien devrez vous effectuer d'essais de stylo en moyenne pour trouver le
  bon ?

\item Vous les essayez l'un après l'autre jusqu'à trouver celui qui
  fonctionne. \\
  Combien devrez vous effectuer d'essais de stylo en moyenne ?

\item Même question si suppose qu'à chaque essai infructueux. vous
  remettez (à tort ) le stylo et que vous tirez au hasard à, nouveau.
\end{noliste}

\rule{15cm}{0.1cm}

Dans un programme de calcul, l'opérateur décide d'utiliser $J$
chiffres significatifs après la virgule et d'arrondir tous les ré%
sultats d'opérations à cette configuration (donc à $0,5\cdot
10^{-J}$ près).

On suppose qu'il effectue $10^{6}$ opérations élémentaires
successives, que les erreurs commises pour chacune sont indépendantes,
de loi uniforme sur $\left[ -0,5\cdot 10^{-J};0,5\cdot 10^{-J}\right] $ et
que l'erreur sur le résultat final est la somme des erreurs commises sur
chaque opération.

Déterminer une valeur approchée de la \ probabilité \ pour que
l'erreurs finale soit inférieure ou égale, en valeur absolue à $%
0,5\cdot 10^{-J+3}$\\
On \ donne $2F\left( \sqrt{3}\right) -1\sim 0,92$ où $F$ est la
fonction de répartition de la loi normale centrée réduite.


\newpage


\rule{15cm}{0.1cm}


\begin{noliste}{1.}
\item Montrer que la fonction $g_{n}$ définie sur $\R$ par $%
  g_{n}\left( x\right) =\left( \Sum{k=0}{n}\dfrac{x^{k}}{k!}\right) e^{-x}$
  est dérivable et calculer sa dérivée.

\item Montrer que, pour tout $n$ de $\N$ l'équation $%
  \Sum{k=0}{n}\dfrac{x^{k}}{k!}=\dfrac{e^{x}}{2}\ $admet une solution et une
  seule dans $\R^{+}$

  \hspace{-1cm}Dans la suite on note $a_{n}$ cette solution.

\item Écrire un programme Turbo-pascal permettant de calculer le plus
  \ "économiquement" possible la valeur de
  $\Sum{k=0}{n}\dfrac{x^{k}}{k!}$ pour $x$ donné.

\item La suite de terme général $a_{n}$ est-elle monotone ?
\end{noliste}

%%FIN

\end{document}
