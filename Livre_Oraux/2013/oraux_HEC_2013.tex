\documentclass[11pt]{article}%
\usepackage{geometry}%
\geometry{a4paper,
  lmargin=2cm,rmargin=2cm,tmargin=2.5cm,bmargin=2.5cm}

\input{../../../macros.tex}

% \renewcommand{\thesection}{\Roman{section}.\hspace{-.3cm}}
% \renewcommand{\thesubsection}{\Alph{subsection}.\hspace{-.2cm}}

\pagestyle{fancy} %
\lhead{ECE2 \hfill Mathématiques \\} %
\chead{\hrule} %
\rhead{} %
\lfoot{} %
\cfoot{} %
\rfoot{\thepage} %

\renewcommand{\headrulewidth}{0pt}% : Trace un trait de séparation
                                    % de largeur 0,4 point. Mettre 0pt
                                    % pour supprimer le trait.

\renewcommand{\footrulewidth}{0.4pt}% : Trace un trait de séparation
                                    % de largeur 0,4 point. Mettre 0pt
                                    % pour supprimer le trait.

\setlength{\headheight}{14pt}

\title{\bf \vspace{-1.6cm} HEC 2013} %
\author{} %
\date{} %
\begin{document}
\maketitle %
\vspace{-1.2cm}\hrule %
\thispagestyle{fancy}

\vspace*{.2cm}

%%DEBUT

%%% EPR %%% HEC;
% type : oralAP; % 
% sujet : E 20; %
% annee : 2013; % 
% theme : vaDiscrete, vaBernoulli, vaBinomiale, FPT; % 

\begin{exerciceAP}~
  \begin{noliste}{1.}
    \setlength{\itemsep}{2mm}
  \item Question de cours : Le schéma binomial.
  \item Soit $(X_n)_{ n \in \N^* }$ une suite de variables aléatoires
    définies sur un espace probabilisé $(\Omega , \A , P )$,
    indépendantes et de même loi de Bernoulli de paramètre
    $\dfrac{1}{2}$.\\
    On pose, pour tout $n \in \N^*$, $W_n = \Sum{k=1}{n} k X_k$ et
    $s_n = \dfrac{n(n+1)}{2}$.
    \begin{noliste}{a)}
      \setlength{\itemsep}{2mm}
    \item Calculer l'espérance $\E(W_n)$ et la variance $\V(W_n)$ de
      la variable aléatoire $W_n$.
    \item Calculer les probabilités $\Prob(\Ev{W_n = 0})$ et
      $\Prob(\Ev{W_n = s_n})$.
    \item Calculer, selon les valeurs de $n$, la probabilité
      $\Prob(\Ev{W_n = 3})$.
    \end{noliste}

  \item Montrer que pour tout $k \in \llb 0, s_n \rrb$, on a :
    $\Prob(\Ev{W_n = k}) = \Prob(\Ev{W_n = s_n - k})$.

  \item 
    \begin{noliste}{a)}
      \setlength{\itemsep}{2mm}
    \item Déterminer pour tout $j \in \llb 0, s_n \rrb$ la loi de
      probabilité conditionnelle de $W_{n+1}$ sachant $(W_n = j)$.
    \item En déduire les relations :
      \[
      \Prob(\Ev{W_{n+1} = k}) = \left\{
        \begin{array}{ll} 
          \frac{ 1 }{ 2 } \ \Prob(\Ev{W_n = k}) & \text{ si } k \leq n \\ 
          \\ 
          \frac{ 1 }{ 2 } \ \Prob(\Ev{W_n = k}) + \frac{1}{2}
          \Prob(\Ev{W_n = k-n-1}) & \text{ si } n+1 \leq k \leq s_n \\ 
          \\
          \frac{1}{2} \Prob(\Ev{W_n = k-n-1}) & \text{ si } s_n + 1 \leq
          k \leq s_{n+1} 
        \end{array} 
      \right.
      \]
    \end{noliste}
  \end{noliste}
\end{exerciceAP}

%%% EPR %%% HEC;
% type : oralSP; % 
% sujet : E 20; %
% annee : 2013; % 
% theme : ; % 

\begin{exerciceSP}~\\
  On pose pour tout $n \in \N^*$ : $S_n = \Sum{k=1}{n} k^2
  \ln \left( \frac{ k }{ n } \right)$.
  \begin{noliste}{1.}
    \setlength{\itemsep}{2mm}
  \item Déterminer $\dlim{n \rightarrow + \infty} \dfrac{S_n}{ n^3}$.

  \item En déduire la limite lorsque $n$ tend vers $+\infty$ de
    $\frac{1}{n^3} \Sum{k=1}{n} k^2 \ln \left( \frac{ k+1 }{ n }
    \right)$.

  \end{noliste}
\end{exerciceSP}


%\newpage


% EPR% HEC;
% type : oralAP; %
% sujet : E 24; %
% annee : 2013; %
% theme : vaDiscrete, vaPoisson, vaCouple, vaBinomiale, FPT,
% estimBiais, estimation; %

\begin{exerciceAP}~
  \begin{noliste}{1.}
    \setlength{\itemsep}{2mm}
  \item Question de cours : Définition de l'indépendance de deux
    variables aléatoires discrètes.

  \item Soit $n$ un entier supérieur ou égal à 1. On jette $n$ fois de
    suite un dé pipé dont les 6 faces ne comportent que les nombres 1,
    2 et 3, et on suppose que les résultats des lancers sont
    indépendants.\\[.2cm]
    À chaque lancer, la probabilité d'obtenir 1 est $p$, celle
    d'obtenir 2 est $q$, et celle d'obtenir 3 est $1-p-q$, où $p$ et
    $q$ sont deux paramètres réels strictement positifs vérifiant
    $p+q<1$.\\

    Soit $X$ (resp. $Y$) la variable aléatoire égale au nombre de 1
    (resp. 2) obtenus en $n$ lancers consécutifs. 
    \begin{noliste}{a)}
    \setlength{\itemsep}{2mm}
    \item Quelles sont les lois respectives de $X$ et $Y$?
    \item Déterminer la loi du couple $(X,Y)$.
    \item Les variables aléatoires $X$ et $Y$ sont-elles indépendantes?
    \item Déterminer le biais et le risque quadratique de l'estimateur
      $T_n = \frac{ X }{ n+1 }$ du paramètre $p$.
    \end{noliste}

  \item On suppose dans cette question que le nombre de lancers
    effectués avec ce dé est une variable aléatoire $N$ suivant la loi
    de Poisson de paramètre $\lambda > 0$.\\

    Soit $X$ (resp. $Y$) la variable aléatoire égale au nombre de 1
    (resp. 2) obtenus en $N$ lancers consécutifs. 
    \begin{noliste}{a)}
    \setlength{\itemsep}{2mm}
    \item Déterminer les lois de $X$ et $Y$ respectivement.
    \item Vérifier que $X$ et $Y$ sont indépendantes.
    \item $T = \dfrac{ X }{ N + 1 }$ est-il un estimateur sans biais du
      paramètre $p$ ?
    \end{noliste}
  \end{noliste}
\end{exerciceAP}

%%% EPR %%% HEC;
% type : oralSP; % 
% sujet : E 24; %
% annee : 2013; % 
% theme : ; % 

\begin{exerciceSP}~\\
  Soit $A$ une matrice carrée de $\mathcal{M}_3 ( \R ) $.
  \begin{noliste}{1.}
    \setlength{\itemsep}{2mm}
  \item Montrer que si $A$ est diagonalisable, $A^3$ l'est aussi.
  \item On suppose dans cette question que $A = 
    \begin{smatrix} 
      0 & 0 & 1 \\ 
      1 & 0 & 0 \\ 
      0 & 1 & 0 
    \end{smatrix}$. 
    \begin{noliste}{a)}
    \setlength{\itemsep}{2mm}
    \item Calculer $A^3$.
    \item La matrice $A$ est-elle diagonalisable?
    \end{noliste}
  \end{noliste}
\end{exerciceSP}


%\newpage


%%% EPR %%% HEC;
% type : oralAP; % 
% sujet : E 28; %
% annee : 2013; % 
% theme : ; % 

\begin{exerciceAP}~\\
  \begin{noliste}{1.}
    \setlength{\itemsep}{2mm}
  \item Question de cours : Écrire, sous forme d'intégrale, la
    probabilité qu'une variable aléatoire suivant la loi normale
    centrée réduite appartienne à un segment $[a;b]$. Dans quelle
    théorème cette probabilité apparaît-elle comme une limite ?\\

    Soit $X$ une variable aléatoire définie sur un espace probabilisé
    $(\Omega , \A , P)$ suivant la loi normale centrée réduite. On
    note $\Phi$ la fonction de répartition de de $X$. On pose $Y =
    \vert X \vert$ (valeur absolue de $X$).

  \item 
    \begin{noliste}{a)}
    \setlength{\itemsep}{2mm}
    \item Montrer que $Y$ admet une espérance et une variance et les calculer.
    \item Calculer $E ( X Y )$.
    \end{noliste}

  \item On pose $Z = X + Y$. 
    \begin{noliste}{a)}
    \setlength{\itemsep}{2mm}
    \item Calculer $P ( Z=0 )$.
    \item Exprimer la fonction de répartition de $Z$ à l'aide de
      $\Phi$ et indiquer l'allure de sa représentation graphique.
    \item La variable aléatoire $Z$ admet-elle une densité? Est-elle discrète?
    \end{noliste}

  \item Soit $y \in \R$. 
    \begin{noliste}{a)}
    \setlength{\itemsep}{2mm}
    \item Exprimer à l'aide de $\Phi$, selon les valeurs de $y$, la
      probabilité $\Prob( [X \leq 1] \cap [Y \leq y] )$.
    \item Pour quelle valeur de $y$ les évènements $(X \leq 1)$ et $(Y
      \leq y)$ sont-ils indépendants?
    \end{noliste}
  \end{noliste}
\end{exerciceAP}

%%% EPR %%% HEC;
% type : oralSP; % 
% sujet : E 28; %
% annee : 2013; % 
% theme : ; % 

\begin{exerciceSP}~\\
  Soit $A$ une matrice de $\mathcal{M}_2 (\R)$ telle que $A^3=0$.
  \begin{noliste}{1.}
    \setlength{\itemsep}{2mm}
  \item Montrer que $A^2=0$.
  \item Montrer que l'ensemble des matrices $M \in \mathcal{M}_2 ( \R
    )$ telles que $AM = MA$ est un espace vectoriel. Quelle est sa
    dimension ?
  \end{noliste}
\end{exerciceSP}


%\newpage


%%% EPR %%% HEC;
% type : oralAP; % 
% sujet : E 29; %
% annee : 2013; % 
% theme : ; % 

\begin{exerciceAP}~
  \begin{noliste}{1.}
    \setlength{\itemsep}{2mm}
  \item Question de cours : Définition de la convergence en loi d'une
    suite de variables aléatoires.\\

    Soit $(X_n)_{ n \in \N }$ une suite de variables aléatoires
    indépendantes définies sur un espace probabilité $(\Omega , \A ,
    P)$, suivant toutes la loi de Bernoulli de paramètre
    $\frac{1}{2}$.\\

    On définit la suite de variables aléatoires $(Z_n)_{ n \in \N }$
    par les relations :
    \[
    Z_0 = \frac{ X_0 }{ 2 } \ \text{ et } \ \forall n \in \N^* \ , \
    Z_n = \frac{ Z_{n-1} + X_n }{ 2 } .
    \]

  \item 
    \begin{noliste}{a)}
    \setlength{\itemsep}{2mm}
    \item Pour tout $n \in \N^*$, exprimer $Z_n$ en fonction des
      variables aléatoires $X_0 , X_1 , \dots , X_n$.
    \item Les variables aléatoires $Z_{n-1}$ et $X_n$ sont-elles
      indépendantes ?
    \item Pour tout $n \in \N$, calculer $E (Z_n)$ et $V (Z_n)$.
    \end{noliste}

  \item Montrer que pour tout $n \in \N$ la variable aléatoire
    $2^{n+1} Z_n$ suit la loi uniforme discrète sur $\llb 0 ; 2^{n+1}
    - 1 \rrb$.

  \item Montrer que la suite de variables aléatoires $(Z_n)_{ n \in \N
    }$ converge en loi vers une variable à densité dont on précisera
    la loi.
  \end{noliste}
\end{exerciceAP}

%%% EPR %%% HEC;
% type : oralSP; % 
% sujet : E 29; %
% annee : 2013; % 
% theme : ; % 

\begin{exerciceSP}~
  \begin{noliste}{1.}
    \setlength{\itemsep}{2mm}
  \item Justifier, pour tout $n \in \N^*$, l'existence de l'intégrale
    $\dint{0}{1} \frac{ x^n \ln x }{ x^n - 1 } \ dx$.
  \item On pose pour tout $n \in \N^*$ : $u_n = \dint{0}{1} \frac{ x^n
      \ln x }{ x^n - 1 } \ dx$.\\
    Étudier la nature (convergence ou divergence) de la suite $(u_n)_{
      n \in \N^* }$.
  \end{noliste}
\end{exerciceSP}


%\newpage


%%% EPR %%% HEC;
% type : oralAP; % 
% sujet : E 32; %
% annee : 2013; % 
% theme : ; % 

\begin{exerciceAP}~
  \begin{noliste}{1.}
    \setlength{\itemsep}{2mm}
  \item Question de cours : Écrire une formule de Taylor à l'ordre $p$
    avec reste intégral, applicable à une fonction définie sur
    $[0;1]$, de classe $C^{p+1}$ sur cet intervalle $(p \in \N)$.

  \item Soit $x$ un réel de l'intervalle $[0;1[$. 
    \begin{noliste}{a)}
    \setlength{\itemsep}{2mm}
    \item Justifier pour tout $t \in [0;x]$, l'encadrement : $0 \leq
      \frac{ x - t }{ 1 - t } \leq x $.
    \item Démontrer l'égalité : $\ln (1-x) = - \Sum{n=1}{+\infty}
      \frac{ x^n }{ n }$.
    \end{noliste}

  \item Soit $X$ une variable aléatoire discrète définie sur un espace
    probabilisé $(\Omega , \A , P)$ telle que pour tout $n \in \N^*$,
    on a : $\Prob(X=n) = \frac{ 1 }{ n ( n+1) }$.
    \begin{noliste}{a)}
    \setlength{\itemsep}{2mm}
    \item Montrer que $\Prob(X \in \N^*) = 1$.
    \item Étudier l'existence des moments de $X$.
    \item Montrer que pour tout $s \in [0;1]$, la variable aléatoire
      $s^X$ admet une espérance, que l'on note $E (s^X)$, et vérifier
      que si $s \in ]0;1[$, on a :
      \[
      E ( s^X ) = \frac{ s + (1-s) \ln (1-s) }{ s } . 
      \]

    \item Pour tout $s \in [0;1]$, on pose $\phi(s) = E
      (s^X)$. Montrer que la fonction $\phi$ est continue sur le
      segment $[0;1]$. Est-elle dérivable sur cet intervalle?

    \item Calculer, lorsqu'elles existent, l'espérance et la variance
      de $X s^X$.
    \end{noliste}
  \end{noliste}
\end{exerciceAP}

%%% EPR %%% HEC;
% type : oralSP; % 
% sujet : E 32; %
% annee : 2013; % 
% theme : ; % 

\begin{exerciceSP}~
  \begin{noliste}{1.}
    \setlength{\itemsep}{2mm}
  \item Montrer que l'application $f : x \mapsto x^3 + x^2 + x$ de
    $\R$ dans $\R$ est bijective.
  \item Quelles sont les fonctions polynômes surjectives?
  \item Quelles sont les fonctions polynômes injectives?
  \end{noliste}
\end{exerciceSP}


%\newpage


%%% EPR %%% HEC;
% type : oralAP; %
% sujet : E 33; %
% annee : 2013; %
% theme : vaDiscrete, FPT, FPC, reductionMat, chaineMarkov, sciLFGN,
% sciSimuVaDiscrete; %

\begin{exerciceAP}~
  \begin{noliste}{1.}
    \setlength{\itemsep}{2mm}
  \item Question de cours : Formule des probabilités totales.\\
    Soit $p$ et $q$ deux réels vérifiant $0<p<1$ et $p+2q=1$. On note
    $\Delta$ la matrice de $\mathcal{M}_3 ( \R )$ définie par :
    \[
    \Delta = 
    \begin{smatrix} 
      p & q & q \\ 
      q & p & q \\ 
      q & q & p 
    \end{smatrix}
    \]

  \item Justifier que $\Delta$ est une matrice diagonalisable.

  \item Soit $D$ la matrice diagonale de $\mathcal{M}_3 ( \R )$
    semblable à $\Delta$ dont les éléments diagonaux sont écrits dans
    l'ordre croissant. Que peut-on dire de la limite des coefficients
    de $D^n$ lorsque $n$ tend vers $+\infty$.\\

    Un village possède trois restaurants $R_1$, $R_2$ et $R_3$. Un
    couple se rend dans un de ces trois restaurants chaque dimanche. A
    l'instant $n=1$ (c'est-à-dire le premier dimanche) il choisit le
    restaurant $R_1$, puis tous les dimanches suivants (instants
    $n=2$, $n=3$, etc.) il choisit le même restaurant que le dimanche
    précédent avec la probabilité $p$ ou change de restaurant avec la
    probabilité $2q$, chacun des deux autres restaurants étant choisis
    avec la même probabilité.\\

    On suppose que l'expérience est modélisée par un espace
    probabilisé $(\Omega , \A , P)$.

  \item Calculer la probabilité que le couple déjeune dans le
    restaurant $R_1$, respectivement $R_2$, respectivement $R_3$, le
    $n$-ième dimanche ($n \geq 2$).

  \item Soit $T$ la variable aléatoire égale au rang du premier
    dimanche où le couple retourne au restaurant $R_1$, s'il y
    retourne, et 0 sinon. 

    \begin{noliste}{a)}
    \setlength{\itemsep}{2mm}
    \item Déterminer la loi de $T$.
    \item Établir l'existence de l'espérance et de la variance de $T$
      et les calculer.
    \end{noliste}

  \item Écrire une procédure scilab permettant de calculer la
    fréquence de visite du restaurant $R_1$ par le couple en 52
    dimanches.
  \end{noliste}
\end{exerciceAP}

%%% EPR %%% HEC;
% type : oralSP; % 
% sujet : E 33; %
% annee : 2013; % 
% theme : ; % 

\begin{exerciceSP}~\\
  Soit $n \in \N^*$. On définit la fonction réelle $f_n$ par :
  $\forall x \in \R$, $f_n (x) = x + 1 - \frac{ e^x }{ n }$.
  \begin{noliste}{1.}
    \setlength{\itemsep}{2mm}
  \item Montrer que pour tout $n \in \N^*$, il existe un unique nombre
    réel négatif $x_n$ tel que $f_n ( x_n ) = 0$.
  \item 
    \begin{noliste}{a)}
    \setlength{\itemsep}{2mm}
    \item Montrer que la suite $(x_n)_{ n \in \N^* }$ est décroissante
      et convergente.
    \item Calculer la limite $\ell$ de la suite $(x_n)_{ n \in \N^* }$.
    \end{noliste}
  \item On pose $y_n = x_n - \ell$. Déterminer un équivalent de $y_n$
    lorsque $n$ tend vers $+\infty$.
  \end{noliste}
\end{exerciceSP}


%\newpage


%%% EPR %%% HEC;
% type : oralAP; % 
% sujet : E 34; %
% annee : 2013; % 
% theme : ; % 

\begin{exerciceAP}~
  \begin{noliste}{1.}
    \setlength{\itemsep}{2mm}
  \item Question de cours : Condition suffisante de diagonalisabilité
    d'une matrice.\\

    Soit $A$ la matrice de $\mathcal{M}_3 (\R)$ définie par : $A
    = 
    \begin{smatrix} 
      0 & 1 & 0 \\ 
      0 & 0 & 1 \\ 
      -2 & 1 & 2
    \end{smatrix}$.

  \item 
    \begin{noliste}{a)}
    \setlength{\itemsep}{2mm}
    \item Soit $\lambda \in \R$. Montrer que le système $A X = \lambda
      X$ d'inconnue $X \in \mathcal{M}_{3,1} (\R)$ possède des
      solutions non nulles si et seulement si $(\lambda^2-1) (\lambda
      -2) = 0$. Donner alors les solutions de ce système.

    \item En déduire une matrice inversible $P$ et une matrice
      diagonale $D$ telles que $A = P D P^{-1}$.
    \end{noliste}

  \item Soit $(x_n)_{ n \in \N}$ une suite réelle définie par : pour
    tout $n \in \N$, $x_{n+3} = 2 x_{n+2} + x_{n+1} - 2 x_n$.\\

    On pose pour tout $n \in \N$ : $X_n = 
    \begin{smatrix} 
      x_n \\
      x_{n+1} \\ 
      x_{n+2} 
    \end{smatrix}$ et $Y_n = P^{-1} X_n$.
    \begin{noliste}{a)}
    \setlength{\itemsep}{2mm}
    \item Quelle relation a-t-on entre $X_{n+1}$, $X_n$ et $A$?
    \item En déduire l'expression de $Y_n$ en fonction de $n$, $D$ et
      $Y_0$.
    \item Donner une condition nécessaire et suffisante sur $x_0$,
      $x_1$ et $x_2$ pour que la suite $(x_n)_{n \in \N}$ soit
      convergente (respectivement, pour que la série $\sum\limits_{ n
        \geq 0 } x_n$ soit convergente).
    \end{noliste}

  \item On pose $B = 
    \begin{smatrix} 
      5 & 0 & -2 \\
      4 & 3 & -4 \\
      8 & 0 & -5
    \end{smatrix}$ et pour tout $(a,b) \in \R^2$, $M(a, b) = 
    \begin{smatrix} 
      5b & a & -2b \\
      4 b & 3b & a - 4 b \\
      -2 a + 8 b & a & 2a - 5 b 
    \end{smatrix}$.

    \begin{noliste}{a)}
    \setlength{\itemsep}{2mm}
    \item Montrer que tout vecteur propre de $A$ est vecteur propre de
      $B$. La réciproque est-elle vraie?
    \item En déduire que $M(a,b)$ est diagonalisable et préciser ses
      valeurs propres.
    \item Déterminer les couples $(a,b) \in \R^2$ pour lesquelles la
      suite $( M(a,b)^n )_{ n \in \N }$ converge vers la matrice
      nulle, c'est-à-dire que chacun de ses neuf coefficients est le
      terme général d'une suite convergeant vers 0.
    \end{noliste}
  \end{noliste}
\end{exerciceAP}

%%% EPR %%% HEC;
% type : oralSP; % 
% sujet : E 34; %
% annee : 2013; % 
% theme : ; % 

\begin{exerciceSP}~\\
  Soit $p \in ]0;1[$. Soit $(X_n)_{ n \in \N^* }$ une suite de
  variables aléatoires définies sur un espace probabilisé $(\Omega ,
  \A , P )$ indépendantes et de même loi donnée par :
  \[
  \forall n \in \N^* , \ P (X_n = -1) = p \ \ \text{ et } \ \ P (X_n =
  1) = 1 - p
  \]
  On pose pour tout $n \in \N^*$, $Z_n = \prod\limits_{i=1}^n X_i$.
  \begin{noliste}{1.}
    \setlength{\itemsep}{2mm}
  \item Calculer l'espérance $E (Z_n) $ de $Z_n$ et $\dlim{ n
      \rightarrow + \infty } E (Z_n)$.
  \item Quelle est la loi de $Z_n$?
  \item Pour quelles valeurs de $p$ les variables aléatoires $Z_1$ et
    $Z_2$ sont-elles indépendantes?
  \end{noliste}
\end{exerciceSP}


%\newpage


%%% EPR %%% HEC;
% type : oralAP; %
% sujet : E 40; %
% annee : 2013; %
% theme : vaDiscrete, denombrement, fct2var, reductionMat; %

\begin{exerciceAP}~
  \begin{noliste}{1.}
    \setlength{\itemsep}{2mm}
  \item Question de cours : Soit $f$ une fonction de classe $C^2$
    définie sur une partie de $\R^2$ à valeurs réelles. Rappeler la
    définition d'un point critique et la condition suffisante
    d'extremum local en un point.\\

    Soit $X$ une variable aléatoire discrète finie définie sur un
    espace probabilisé $(\Omega , \A , P)$.\\
    On pose pour tout $n \in \N^*$ : $X ( \Omega ) = \{ x_1 , \dots ,
    x_n \} \subset \R$ et on suppose que $\forall i \in \llb 1 ; n
    \rrb$, $P (X=x_i) \neq 0$.\\

    On définit l'entropie de $X$ par : $ H(X) = - \frac{ 1 }{ \ln 2 }
    \Sum{i=1}{n} P (X=x_i) \ln \big( \Prob(X=x_i) \big)$.

  \item Soient $x_1 , x_2 , x_3 , x_4$ quatre réels distincts. On
    considère un jeu de 32 cartes dont on tire une carte au
    hasard. Soit $X$ la variable aléatoire prenant les valeurs
    suivantes : 
    \begin{noliste}{$\stimes$}
    \item $x_1$ si la carte tirée est rouge (coeur ou carreau),
    \item $x_2$ si la carte tirée est un pique,
    \item $x_3$ si la carte tirée est le valet, la dame, le roi ou
      l'as de trèfle,
    \item $x_4$ dans les autres cas.
    \end{noliste}

    On tire une carte notée $C$ et un enfant décide de déterminer la
    valeur $X(C)$ en posant dans l'ordre les questions suivantes
    auxquelles il lui est répondu par "oui" ou par "non". LA carte $C$
    est-elle rouge? La carte $C$ est-elle un pique ? La carte $C$
    est-elle le valet, la dame, le roi ou l'as de trèfle ? \\
    Soit $N$ la variable aléatoire égale au nombre de questions posées
    (l'enfant cesse de poser des questions dès qu'il a obtenu une
    réponse "oui"). 

    \begin{noliste}{a)}
    \setlength{\itemsep}{2mm}
    \item Calculer l'entropie $H(X)$ de $X$.
    \item Déterminer la loi et l'espérance $E (N)$ de $N$. Comparer
      $\E(N)$ et $H(X)$.
    \end{noliste}

  \item Soit $f$ la fonction définie sur $\R^2$ à valeurs réelles
    telle que : $f(x,y) = x \ln x + y \ln y + (1-x-y) \ln
    (1-x-y)$. 
    \begin{noliste}{a)}
    \setlength{\itemsep}{2mm}
    \item Préciser le domaine de définition de $f$. Dessiner ce
      domaine dans le plan rapporté à un repère orthonormé.
    \item Montrer que $f$ ne possède qu'un seul point critique et
      qu'en ce point, $f$ admet un extremum local.
    \item Soit $X$ une variable aléatoire réelle prenant les valeurs
      $x_1$, $x_2$ et $x_3$ avec les probabilités non nulles $p_1$,
      $p_2$ et $p_3$ respectivement.\\

      Calculer $H(X)$ et montrer que $H(X)$ est maximale lorsque
      $p_1=p_2=p_3 = \frac{1}{3}$.
    \end{noliste}   
  \end{noliste}
\end{exerciceAP}

%%% EPR %%% HEC;
% type : oralSP; % 
% sujet : E 40; %
% annee : 2013; % 
% theme : ; % 

\begin{exerciceSP}~\\
  On rappelle l'identité remarquable $a^3 + b^3 = (a+b) (a^2 - a b +
  b^2)$.\\
  Soit $n \in \N^*$ et $A$ et $B$ deux matrices de $\mathcal{M}_n ( \R
  )$ vérifiant $A^3 = 0$, $A B = B A$ et $B$ inversible. \\
  Montrer que $A + B$ est inversible.
\end{exerciceSP}


%\newpage


%%% EPR %%% HEC;
% type : oralAP; % 
% sujet : E 9; %
% annee : 2013; % 
% theme : ; % 

\begin{exerciceAP}~
  \begin{noliste}{1.}
    \setlength{\itemsep}{2mm}
  \item Question de cours : Critères de convergence d'une intégrale
    sur un intervalle du type $[a ; +\infty[$ ($a \in \R$).

  \item Soit $x \in \R_+^*$. 
    \begin{noliste}{a)}
    \setlength{\itemsep}{2mm}
    \item Établir la convergence de l'intégrale $\dint{0}{+\infty}
      \frac{ e^{ -t } }{ x + t } \ dt$. On pose alors $f(x) =
      \dint{0}{+\infty} \frac{ e^{ -t } }{ x + t } \ dt$.

    \item Montrer que $f$ est monotone sur $\R_+^*$.

    \end{noliste}

  \item Soit $g$ et $h$ les fonctions définies sur $\R_+^*$ à valeurs
    réelles telles que :
    \[
    g(x) = \dint{0}{1} \frac{ e^{ -t } - 1 }{ x + t } \ dt \ \ \
    \text{ et } \ \ \ h(x) = \dint{1}{+\infty} \frac{ e^{ -t } }{ x +
      t } \ dt .
    \]

    \begin{noliste}{a)}
      \setlength{\itemsep}{2mm}
    \item Soit $\varphi$ la fonction définie sur $[0;1]$ par :
      $\varphi (t) = \left\{
        \begin{array}{ll} 
          \frac{ e^{ -t } - 1 }{t} & \text{ si } t \in ] 0 ; 1] \\ 
          \\ 
          -1 & \text{ si } t = 0 
        \end{array} \right.$ \\
      Montrer que $\varphi$ est continue sur le segment $[0;1]$.

    \item En déduire que la fonction $g$ est bornée sur $\R_+^*$.

    \item Montrer de même que la fonction $h$ est bornée sur $\R_+^*$.

    \item Montrer que pour tout $x > 0$, on a : $f(x) = \ln (x+1) -
      \ln x + g(x) + h(x)$. En déduire un équivalent de $f(x)$ lorsque
      $x$ tend vers 0.

    \end{noliste}

  \item À l'aide de l'encadrement $0 \leq \frac{ 1 }{ x } - \frac{ 1
    }{ x + t } \leq \frac{ t }{ x^2 }$ valable pour tout $x > 0$ et
    pour tout $t \geq 0$, montrer que $f(x)$ est équivalent à $\frac{
      1 }{ x } $ lorsque $x$ tend vers $+\infty$.
  \end{noliste}
\end{exerciceAP}

%%% EPR %%% HEC;
% type : oralSP; % 
% sujet : E 9; %
% annee : 2013; % 
% theme : ; % 

\begin{exerciceSP}~\\
  Les variables aléatoires sont définies sur un espace probabilisé
  $(\Omega , \A , P)$. \\ 
  Soit $X$ une variable aléatoire qui suit la loi de Poisson de
  paramètre $\lambda > 0$ et soit $Y$ une variable aléatoire
  indépendante de $X$ telle que : $ Y ( \Omega ) = \{ 1 ; 2 \} , P ( Y
  = 1 ) = P ( Y = 2 ) = \frac{ 1 }{ 2 }$. On pose $Z = X Y$.
  \begin{noliste}{1.}
    \setlength{\itemsep}{2mm}

  \item Déterminer la loi de $Z$.

  \item On admet que : $\Sum{ k = 0 }{ +\infty } \frac{ \lambda^{ 2k }
    }{ (2k)! } = \frac{ e^{ \lambda } + e^{ - \lambda } }{ 2 }
    $. Quelle est la probabilité que $Z$ prenne une valeur paire?

  \end{noliste}
\end{exerciceSP}

%%FIN

\end{document}
