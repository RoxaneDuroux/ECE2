\documentclass[11pt]{article}%
\usepackage{geometry}%
\geometry{a4paper,
 lmargin = 2cm,rmargin = 2cm,tmargin = 2.5cm,bmargin = 2.5cm}

\usepackage{array}
\usepackage{paralist}

\usepackage[svgnames, usenames, dvipsnames]{xcolor}
\xdefinecolor{RecColor}{named}{Aqua}
\xdefinecolor{IncColor}{named}{Aqua}
\xdefinecolor{ImpColor}{named}{PaleGreen}

% \usepackage{frcursive}

\usepackage{adjustbox}

%%%%%%%%%%%
\newcommand{\cRB}[1]{{\color{Red} \pmb{#1}}} %
\newcommand{\cR}[1]{{\color{Red} {#1}}} %
\newcommand{\cBB}[1]{{\color{Blue} \pmb{#1}}}
\newcommand{\cB}[1]{{\color{Blue} {#1}}}
\newcommand{\cGB}[1]{{\color{LimeGreen} \pmb{#1}}}
\newcommand{\cG}[1]{{\color{LimeGreen} {#1}}}

%%%%%%%%%%

\usepackage{diagbox} %
\usepackage{colortbl} %
\usepackage{multirow} %
\usepackage{pgf} %
\usepackage{environ} %
\usepackage{fancybox} %
\usepackage{textcomp} %
\usepackage{marvosym} %

%%%%%%%%%% pour pouvoir faire des dashedline dans les tableaux
\usepackage{arydshln}

%%%%%%%%%% pour qu'une cellcolor ne recouvre pas le trait du tableau
\usepackage{hhline}%

\usepackage{pgfplots}
\pgfplotsset{compat=1.10}
\usepgfplotslibrary{patchplots}
\usepgfplotslibrary{fillbetween}
\usepackage{tikz,tkz-tab}
\usepackage{ifthen}
\usepackage{calc}
\usetikzlibrary{calc,decorations.pathreplacing,arrows,positioning} 
\usetikzlibrary{fit,shapes,backgrounds}
% \usepackage[nomessages]{fp}% http://ctan.org/pkg/fp

\usetikzlibrary{matrix,arrows,decorations.pathmorphing,
  decorations.pathreplacing} 

\newcommand{\myunit}{1 cm}
\tikzset{
    node style sp/.style={draw,circle,minimum size=\myunit},
    node style ge/.style={circle,minimum size=\myunit},
    arrow style mul/.style={draw,sloped,midway,fill=white},
    arrow style plus/.style={midway,sloped,fill=white},
}

%%%%%%%%%%%%%%
%%%%% écrire des inférieur égal ou supérieur égal avec typographie
%%%%% francaise
%%%%%%%%%%%%%

\renewcommand{\geq}{\geqslant}
\renewcommand{\leq}{\leqslant}
\renewcommand{\emptyset}{\varnothing}

\newcommand{\Leq}{\leqslant}
\newcommand{\Geq}{\geqslant}

%%%%%%%%%%%%%%
%%%%% Macro Celia
%%%%%%%%%%%%%

\newcommand{\ff}[2]{\left[#1, #2\right]} %
\newcommand{\fo}[2]{\left[#1, #2\right[} %
\newcommand{\of}[2]{\left]#1, #2\right]} %
\newcommand{\soo}[2]{\left]#1, #2\right[} %
\newcommand{\abs}[1]{\left|#1\right|} %
\newcommand{\Ent}[1]{\left\lfloor #1 \right\rfloor} %


%%%%%%%%%%%%%%
%%%%% tikz : comment dessiner un "oeil"
%%%%%%%%%%%%%

\newcommand{\eye}[4]% size, x, y, rotation
{ \draw[rotate around={#4:(#2,#3)}] (#2,#3) -- ++(-.5*55:#1) (#2,#3)
  -- ++(.5*55:#1); \draw (#2,#3) ++(#4+55:.75*#1) arc
  (#4+55:#4-55:.75*#1);
  % IRIS
  \draw[fill=gray] (#2,#3) ++(#4+55/3:.75*#1) arc
  (#4+180-55:#4+180+55:.28*#1);
  % PUPIL, a filled arc
  \draw[fill=black] (#2,#3) ++(#4+55/3:.75*#1) arc
  (#4+55/3:#4-55/3:.75*#1);%
}


%%%%%%%%%%
%% discontinuité fonction
\newcommand\pointg[2]{%
  \draw[color = red, very thick] (#1+0.15, #2-.04)--(#1, #2-.04)--(#1,
  #2+.04)--(#1+0.15, #2+.04);%
}%

\newcommand\pointd[2]{%
  \draw[color = red, very thick] (#1-0.15, #2+.04)--(#1, #2+.04)--(#1,
  #2-.04)--(#1-0.15, #2-.04);%
}%

%%%%%%%%%%
%%% 1 : position abscisse, 2 : position ordonnée, 3 : taille, 4 : couleur
%%%%%%%%%%
% \newcommand\pointG[4]{%
%   \draw[color = #4, very thick] (#1+#3, #2-(#3/3.75))--(#1,
%   #2-(#3/3.75))--(#1, #2+(#3/3.75))--(#1+#3, #2+(#3/3.75)) %
% }%

\newcommand\pointG[4]{%
  \draw[color = #4, very thick] ({#1+#3/3.75}, {#2-#3})--(#1,
  {#2-#3})--(#1, {#2+#3})--({#1+#3/3.75}, {#2+#3}) %
}%

\newcommand\pointD[4]{%
  \draw[color = #4, very thick] ({#1-#3/3.75}, {#2+#3})--(#1,
  {#2+#3})--(#1, {#2-#3})--({#1-#3/3.75}, {#2-#3}) %
}%

\newcommand\spointG[4]{%
  \draw[color = #4, very thick] ({#1+#3/1.75}, {#2-#3})--(#1,
  {#2-#3})--(#1, {#2+#3})--({#1+#3/1.75}, {#2+#3}) %
}%

\newcommand\spointD[4]{%
  \draw[color = #4, very thick] ({#1-#3/2}, {#2+#3})--(#1,
  {#2+#3})--(#1, {#2-#3})--({#1-#3/2}, {#2-#3}) %
}%

%%%%%%%%%%

\newcommand{\Pb}{\mathtt{P}}

%%%%%%%%%%%%%%%
%%% Pour citer un précédent item
%%%%%%%%%%%%%%%
\newcommand{\itbf}[1]{{\small \bf \textit{#1}}}


%%%%%%%%%%%%%%%
%%% Quelques couleurs
%%%%%%%%%%%%%%%

\xdefinecolor{cancelcolor}{named}{Red}
\xdefinecolor{intI}{named}{ProcessBlue}
\xdefinecolor{intJ}{named}{ForestGreen}

%%%%%%%%%%%%%%%
%%%%%%%%%%%%%%%
% barrer du texte
\usetikzlibrary{shapes.misc}

\makeatletter
% \definecolor{cancelcolor}{rgb}{0.127,0.372,0.987}
\newcommand{\tikz@bcancel}[1]{%
  \begin{tikzpicture}[baseline=(textbox.base),inner sep=0pt]
  \node[strike out,draw] (textbox) {#1}[thick, color=cancelcolor];
  \useasboundingbox (textbox);
  \end{tikzpicture}%
}
\newcommand{\bcancel}[1]{%
  \relax\ifmmode
    \mathchoice{\tikz@bcancel{$\displaystyle#1$}}
               {\tikz@bcancel{$\textstyle#1$}}
               {\tikz@bcancel{$\scriptstyle#1$}}
               {\tikz@bcancel{$\scriptscriptstyle#1$}}
  \else
    \tikz@bcancel{\strut#1}%
  \fi
}
\newcommand{\tikz@xcancel}[1]{%
  \begin{tikzpicture}[baseline=(textbox.base),inner sep=0pt]
  \node[cross out,draw] (textbox) {#1}[thick, color=cancelcolor];
  \useasboundingbox (textbox);
  \end{tikzpicture}%
}
\newcommand{\xcancel}[1]{%
  \relax\ifmmode
    \mathchoice{\tikz@xcancel{$\displaystyle#1$}}
               {\tikz@xcancel{$\textstyle#1$}}
               {\tikz@xcancel{$\scriptstyle#1$}}
               {\tikz@xcancel{$\scriptscriptstyle#1$}}
  \else
    \tikz@xcancel{\strut#1}%
  \fi
}
\makeatother

\newcommand{\xcancelRA}{\xcancel{\rule[-.15cm]{0cm}{.5cm} \Rightarrow
    \rule[-.15cm]{0cm}{.5cm}}}

%%%%%%%%%%%%%%%%%%%%%%%%%%%%%%%%%%%
%%%%%%%%%%%%%%%%%%%%%%%%%%%%%%%%%%%

\newcommand{\vide}{\multicolumn{1}{c}{}}

%%%%%%%%%%%%%%%%%%%%%%%%%%%%%%%%%%%
%%%%%%%%%%%%%%%%%%%%%%%%%%%%%%%%%%%


\usepackage{multicol}
% \usepackage[latin1]{inputenc}
% \usepackage[T1]{fontenc}
\usepackage[utf8]{inputenc}
\usepackage[T1]{fontenc}
\usepackage[normalem]{ulem}
\usepackage[french]{babel}

\usepackage{url}    
\usepackage{hyperref}
\hypersetup{
  backref=true,
  pagebackref=true,
  hyperindex=true,
  colorlinks=true,
  breaklinks=true,
  urlcolor=blue,
  linkcolor=red,
  bookmarks=true,
  bookmarksopen=true
}

%%%%%%%%%%%%%%%%%%%%%%%%%%%%%%%%%%%%%%%%%%%
%% Pour faire des traits diagonaux dans les tableaux
%% Nécessite slashbox.sty
%\usepackage{slashbox}

\usepackage{tipa}
\usepackage{verbatim,listings}
\usepackage{graphicx}
\usepackage{fancyhdr}
\usepackage{mathrsfs}
\usepackage{pifont}
\usepackage{tablists}
\usepackage{dsfont,amsfonts,amssymb,amsmath,amsthm,stmaryrd,upgreek,manfnt}
\usepackage{enumerate}

%\newcolumntype{M}[1]{p{#1}}
\newcolumntype{C}[1]{>{\centering}m{#1}}
\newcolumntype{R}[1]{>{\raggedright}m{#1}}
\newcolumntype{L}[1]{>{\raggedleft}m{#1}}
\newcolumntype{P}[1]{>{\raggedright}p{#1}}
\newcolumntype{B}[1]{>{\raggedright}b{#1}}
\newcolumntype{Q}[1]{>{\raggedright}t{#1}}

\newcommand{\alias}[2]{
\providecommand{#1}{}
\renewcommand{#1}{#2}
}
\alias{\R}{\mathbb{R}}
\alias{\N}{\mathbb{N}}
\alias{\Z}{\mathbb{Z}}
\alias{\Q}{\mathbb{Q}}
\alias{\C}{\mathbb{C}}
\alias{\K}{\mathbb{K}}

%%%%%%%%%%%%
%% rendre +infty et -infty plus petits
%%%%%%%%%%%%
\newcommand{\sinfty}{{\scriptstyle \infty}}

%%%%%%%%%%%%%%%%%%%%%%%%%%%%%%%
%%%%% macros TP Scilab %%%%%%%%
\newcommand{\Scilab}{\textbf{Scilab}} %
\newcommand{\Scinotes}{\textbf{SciNotes}} %
\newcommand{\faire}{\noindent $\blacktriangleright$ } %
\newcommand{\fitem}{\scalebox{.8}{$\blacktriangleright$}} %
\newcommand{\entree}{{\small\texttt{ENTRÉE}}} %
\newcommand{\tab}{{\small\texttt{TAB}}} %
\newcommand{\mt}[1]{\mathtt{#1}} %
% guillemets droits

\newcommand{\ttq}{\textquotesingle} %

\newcommand{\reponse}[1]{\longboxed{
    \begin{array}{C{0.9\textwidth}}
      \nl[#1]
    \end{array}
  }} %

\newcommand{\reponseL}[2]{\longboxed{
    \begin{array}{C{#2\textwidth}}
      \nl[#1]
    \end{array}
  }} %

\newcommand{\reponseR}[1]{\longboxed{
    \begin{array}{R{0.9\textwidth}}
      #1
    \end{array}
  }} %

\newcommand{\reponseRL}[2]{\longboxed{
    \begin{array}{R{#2\textwidth}}
      #1
    \end{array}
  }} %

\newcommand{\reponseC}[1]{\longboxed{
    \begin{array}{C{0.9\textwidth}}
      #1
    \end{array}
  }} %

\colorlet{pyfunction}{Blue}
\colorlet{pyCle}{Magenta}
\colorlet{pycomment}{LimeGreen}
\colorlet{pydoc}{Cyan}
% \colorlet{SansCo}{white}
% \colorlet{AvecCo}{black}

\newcommand{\visible}[1]{{\color{ASCo}\colorlet{pydoc}{pyDo}\colorlet{pycomment}{pyCo}\colorlet{pyfunction}{pyF}\colorlet{pyCle}{pyC}\colorlet{function}{sciFun}\colorlet{var}{sciVar}\colorlet{if}{sciIf}\colorlet{comment}{sciComment}#1}} %

%%%% à changer ????
\newcommand{\invisible}[1]{{\color{ASCo}\colorlet{pydoc}{pyDo}\colorlet{pycomment}{pyCo}\colorlet{pyfunction}{pyF}\colorlet{pyCle}{pyC}\colorlet{function}{sciFun}\colorlet{var}{sciVar}\colorlet{if}{sciIf}\colorlet{comment}{sciComment}#1}} %

\newcommand{\invisibleCol}[2]{{\color{#1}#2}} %

\NewEnviron{solution} %
{ %
  \Boxed{
    \begin{array}{>{\color{ASCo}} R{0.9\textwidth}}
      \colorlet{pycomment}{pyCo}
      \colorlet{pydoc}{pyDo}
      \colorlet{pyfunction}{pyF}
      \colorlet{pyCle}{pyC}
      \colorlet{function}{sciFun}
      \colorlet{var}{sciVar}
      \colorlet{if}{sciIf}
      \colorlet{comment}{sciComment}
      \BODY
    \end{array}
  } %
} %

\NewEnviron{solutionC} %
{ %
  \Boxed{
    \begin{array}{>{\color{ASCo}} C{0.9\textwidth}}
      \colorlet{pycomment}{pyCo}
      \colorlet{pydoc}{pyDo}
      \colorlet{pyfunction}{pyF}
      \colorlet{pyCle}{pyC}
      \colorlet{function}{sciFun}
      \colorlet{var}{sciVar}
      \colorlet{if}{sciIf}
      \colorlet{comment}{sciComment}
      \BODY
    \end{array}
  } %
} %

\newcommand{\invite}{--\!\!>} %

%%%%% nouvel environnement tabular pour retour console %%%%
\colorlet{ConsoleColor}{Blue!15}
\colorlet{function}{Red}
\colorlet{var}{Maroon}
\colorlet{if}{Magenta}
\colorlet{comment}{LimeGreen}

\newcommand{\tcVar}[1]{\textcolor{var}{\bf \small #1}} %
\newcommand{\tcFun}[1]{\textcolor{function}{#1}} %
\newcommand{\tcIf}[1]{\textcolor{if}{#1}} %
\newcommand{\tcFor}[1]{\textcolor{if}{#1}} %

\newcommand{\moins}{\!\!\!\!\!\!- }
\newcommand{\espn}{\!\!\!\!\!\!}

\usepackage{booktabs,varwidth} \newsavebox\TBox
\newenvironment{console}
{\begin{lrbox}{\TBox}\varwidth{\linewidth}
    \tabular{>{\tt\small}R{0.84\textwidth}}
    \nl[-.4cm]} {\endtabular\endvarwidth\end{lrbox}%
  \fboxsep=1pt\colorbox{ConsoleColor}{\usebox\TBox}}

\newcommand{\lInv}[1]{%
  $\invite$ #1} %

\newcommand{\lAns}[1]{%
  \qquad ans \ = \nl %
  \qquad \qquad #1} %

\newcommand{\lVar}[2]{%
  \qquad #1 \ = \nl %
  \qquad \qquad #2} %

\newcommand{\lDisp}[1]{%
  #1 %
} %

\newcommand{\ligne}[1]{\underline{\small \tt #1}} %

\newcommand{\ligneAns}[2]{%
  $\invite$ #1 \nl %
  \qquad ans \ = \nl %
  \qquad \qquad #2} %

\newcommand{\ligneVar}[3]{%
  $\invite$ #1 \nl %
  \qquad #2 \ = \nl %
  \qquad \qquad #3} %

\newcommand{\ligneErr}[3]{%
  $\invite$ #1 \nl %
  \quad !-{-}error #2 \nl %
  #3} %
%%%%%%%%%%%%%%%%%%%%%% 

\newcommand{\bs}[1]{\boldsymbol{#1}} %
\newcommand{\nll}{\nl[.4cm]} %
\newcommand{\nle}{\nl[.2cm]} %
%% opérateur puissance copiant l'affichage Scilab
%\newcommand{\puis}{\!\!\!~^{\scriptscriptstyle\pmb{\wedge}}}
\newcommand{\puis}{\mbox{$\hspace{-.1cm}~^{\scriptscriptstyle\pmb{\wedge}}
    \hspace{0.05cm}$}} %
\newcommand{\pointpuis}{.\mbox{$\hspace{-.15cm}~^{\scriptscriptstyle\pmb{\wedge}}$}} %
\newcommand{\Sfois}{\mbox{$\mt{\star}$}} %

%%%%% nouvel environnement tabular pour les encadrés Scilab %%%%
\newenvironment{encadre}
{\begin{lrbox}{\TBox}\varwidth{\linewidth}
    \tabular{>{\tt\small}C{0.1\textwidth}>{\small}R{0.7\textwidth}}}
  {\endtabular\endvarwidth\end{lrbox}%
  \fboxsep=1pt\longboxed{\usebox\TBox}}

\newenvironment{encadreL}
{\begin{lrbox}{\TBox}\varwidth{\linewidth}
    \tabular{>{\tt\small}C{0.25\textwidth}>{\small}R{0.6\textwidth}}}
  {\endtabular\endvarwidth\end{lrbox}%
  \fboxsep=1pt\longboxed{\usebox\TBox}}

\newenvironment{encadreF}
{\begin{lrbox}{\TBox}\varwidth{\linewidth}
    \tabular{>{\tt\small}C{0.2\textwidth}>{\small}R{0.70\textwidth}}}
  {\endtabular\endvarwidth\end{lrbox}%
  \fboxsep=1pt\longboxed{\usebox\TBox}}

\newenvironment{encadreLL}[2]
{\begin{lrbox}{\TBox}\varwidth{\linewidth}
    \tabular{>{\tt\small}C{#1\textwidth}>{\small}R{#2\textwidth}}}
  {\endtabular\endvarwidth\end{lrbox}%
  \fboxsep=1pt\longboxed{\usebox\TBox}}

%%%%% nouvel environnement tabular pour les script et fonctions %%%%
\newcommand{\commentaireDL}[1]{\multicolumn{1}{l}{\it
    \textcolor{comment}{$\slash\slash$ #1}}}

\newcommand{\commentaire}[1]{{\textcolor{comment}{$\slash\slash$ #1}}}

\newcounter{cptcol}

\newcommand{\nocount}{\multicolumn{1}{c}{}}

\newcommand{\sciNo}[1]{{\small \underbar #1}}

\NewEnviron{scilab}{ %
  \setcounter{cptcol}{0}
  \begin{center}
    \longboxed{
      \begin{tabular}{>{\stepcounter{cptcol}{\tiny \underbar
              \thecptcol}}c>{\tt}l}
        \BODY
      \end{tabular}
    }
  \end{center}
}

\NewEnviron{scilabNC}{ %
  \begin{center}
    \longboxed{
      \begin{tabular}{>{\tt}l} %
          \BODY
      \end{tabular}
    }
  \end{center}
}

\NewEnviron{scilabC}[1]{ %
  \setcounter{cptcol}{#1}
  \begin{center}
    \longboxed{
      \begin{tabular}{>{\stepcounter{cptcol}{\tiny \underbar
              \thecptcol}}c>{\tt}l}
        \BODY
      \end{tabular}
    }
  \end{center}
}

\newcommand{\scisol}[1]{ %
  \setcounter{cptcol}{0}
  \longboxed{
    \begin{tabular}{>{\stepcounter{cptcol}{\tiny \underbar
            \thecptcol}}c>{\tt}l}
      #1
    \end{tabular}
  }
}

\newcommand{\scisolNC}[1]{ %
  \longboxed{
    \begin{tabular}{>{\tt}l}
      #1
    \end{tabular}
  }
}

\newcommand{\scisolC}[2]{ %
  \setcounter{cptcol}{#1}
  \longboxed{
    \begin{tabular}{>{\stepcounter{cptcol}{\tiny \underbar
            \thecptcol}}c>{\tt}l}
      #2
    \end{tabular}
  }
}

\NewEnviron{syntaxe}{ %
  % \fcolorbox{black}{Yellow!20}{\setlength{\fboxsep}{3mm}
  \shadowbox{
    \setlength{\fboxsep}{3mm}
    \begin{tabular}{>{\tt}l}
      \BODY
    \end{tabular}
  }
}

%%%%% fin macros TP Scilab %%%%%%%%
%%%%%%%%%%%%%%%%%%%%%%%%%%%%%%%%%%%

%%%%%%%%%%%%%%%%%%%%%%%%%%%%%%%%%%%
%%%%% TP Python - listings %%%%%%%%
%%%%%%%%%%%%%%%%%%%%%%%%%%%%%%%%%%%
\newcommand{\Python}{\textbf{Python}} %

\lstset{% general command to set parameter(s)
basicstyle=\ttfamily\small, % print whole listing small
keywordstyle=\color{blue}\bfseries\underbar,
%% underlined bold black keywords
frame=lines,
xleftmargin=10mm,
numbers=left,
numberstyle=\tiny\underbar,
numbersep=10pt,
%identifierstyle=, % nothing happens
commentstyle=\color{green}, % white comments
%%stringstyle=\ttfamily, % typewriter type for strings
showstringspaces=false}

\newcommand{\pysolCpt}[2]{
  \setcounter{cptcol}{#1}
  \longboxed{
    \begin{tabular}{>{\stepcounter{cptcol}{\tiny \underbar
            \thecptcol}}c>{\tt}l}
        #2
      \end{tabular}
    }
} %

\newcommand{\pysol}[1]{
  \setcounter{cptcol}{0}
  \longboxed{
    \begin{tabular}{>{\stepcounter{cptcol}{\tiny \underbar
            \thecptcol}}c>{\tt}l}
        #1
      \end{tabular}
    }
} %

% \usepackage[labelsep=endash]{caption}

% avec un caption
\NewEnviron{pythonCap}[1]{ %
  \renewcommand{\tablename}{Programme}
  \setcounter{cptcol}{0}
  \begin{center}
    \longboxed{
      \begin{tabular}{>{\stepcounter{cptcol}{\tiny \underbar
              \thecptcol}}c>{\tt}l}
        \BODY
      \end{tabular}
    }
    \captionof{table}{#1}
  \end{center}
}

\NewEnviron{python}{ %
  \setcounter{cptcol}{0}
  \begin{center}
    \longboxed{
      \begin{tabular}{>{\stepcounter{cptcol}{\tiny \underbar
              \thecptcol}}c>{\tt}l}
        \BODY
      \end{tabular}
    }
  \end{center}
}

\newcommand{\pyVar}[1]{\textcolor{var}{\bf \small #1}} %
\newcommand{\pyFun}[1]{\textcolor{pyfunction}{#1}} %
\newcommand{\pyCle}[1]{\textcolor{pyCle}{#1}} %
\newcommand{\pyImp}[1]{{\bf #1}} %

%%%%% commentaire python %%%%
\newcommand{\pyComDL}[1]{\multicolumn{1}{l}{\textcolor{pycomment}{\#
      #1}}}

\newcommand{\pyCom}[1]{{\textcolor{pycomment}{\# #1}}}
\newcommand{\pyDoc}[1]{{\textcolor{pydoc}{#1}}}

\newcommand{\pyNo}[1]{{\small \underbar #1}}

%%%%%%%%%%%%%%%%%%%%%%%%%%%%%%%%%%%
%%%%%% Système linéaire paramétré : écrire les opérations au-dessus
%%%%%% d'un symbole équivalent
%%%%%%%%%%%%%%%%%%%%%%%%%%%%%%%%%%%

\usepackage{systeme}

\NewEnviron{arrayEq}{ %
  \stackrel{\scalebox{.6}{$
      \begin{array}{l} 
        \BODY \\[.1cm]
      \end{array}$}
  }{\Longleftrightarrow}
}

\NewEnviron{arrayEg}{ %
  \stackrel{\scalebox{.6}{$
      \begin{array}{l} 
        \BODY \\[.1cm]
      \end{array}$}
  }{=}
}

\NewEnviron{operationEq}{ %
  \scalebox{.6}{$
    \begin{array}{l} 
      \scalebox{1.6}{$\mbox{Opérations :}$} \\[.2cm]
      \BODY \\[.1cm]
    \end{array}$}
}

% \NewEnviron{arraySys}[1]{ %
%   \sysdelim\{.\systeme[#1]{ %
%     \BODY %
%   } %
% }

%%%%%

%%%%%%%%%%
%%%%%%%%%% ESSAI
\newlength\fboxseph
\newlength\fboxsepva
\newlength\fboxsepvb

\setlength\fboxsepva{0.2cm}
\setlength\fboxsepvb{0.2cm}
\setlength\fboxseph{0.2cm}

\makeatletter

\def\longboxed#1{\leavevmode\setbox\@tempboxa\hbox{\color@begingroup%
\kern\fboxseph{\m@th$\displaystyle #1 $}\kern\fboxseph%
\color@endgroup }\my@frameb@x\relax}

\def\my@frameb@x#1{%
  \@tempdima\fboxrule \advance\@tempdima \fboxsepva \advance\@tempdima
  \dp\@tempboxa\hbox {%
    \lower \@tempdima \hbox {%
      \vbox {\hrule\@height\fboxrule \hbox{\vrule\@width\fboxrule #1
          \vbox{%
            \vskip\fboxsepva \box\@tempboxa \vskip\fboxsepvb}#1
          \vrule\@width\fboxrule }%
        \hrule \@height \fboxrule }}}}

\newcommand{\boxedhv}[3]{\setlength\fboxseph{#1cm}
  \setlength\fboxsepva{#2cm}\setlength\fboxsepvb{#2cm}\longboxed{#3}}

\newcommand{\boxedhvv}[4]{\setlength\fboxseph{#1cm}
  \setlength\fboxsepva{#2cm}\setlength\fboxsepvb{#3cm}\longboxed{#4}}

\newcommand{\Boxed}[1]{{\setlength\fboxseph{0.2cm}
  \setlength\fboxsepva{0.2cm}\setlength\fboxsepvb{0.2cm}\longboxed{#1}}}

\newcommand{\mBoxed}[1]{{\setlength\fboxseph{0.2cm}
  \setlength\fboxsepva{0.2cm}\setlength\fboxsepvb{0.2cm}\longboxed{\mbox{#1}}}}

\newcommand{\mboxed}[1]{{\setlength\fboxseph{0.2cm}
  \setlength\fboxsepva{0.2cm}\setlength\fboxsepvb{0.2cm}\boxed{\mbox{#1}}}}

\newsavebox{\fmbox}
\newenvironment{fmpage}[1]
     {\begin{lrbox}{\fmbox}\begin{minipage}{#1}}
     {\end{minipage}\end{lrbox}\fbox{\usebox{\fmbox}}}

%%%%%%%%%%
%%%%%%%%%%

\DeclareMathOperator{\ch}{ch}
\DeclareMathOperator{\sh}{sh}

%%%%%%%%%%
%%%%%%%%%%

\newcommand{\norme}[1]{\Vert #1 \Vert}

%\newcommand*\widefbox[1]{\fbox{\hspace{2em}#1\hspace{2em}}}

\newcommand{\nl}{\tabularnewline}

\newcommand{\hand}{\noindent\ding{43}\ }
\newcommand{\ie}{\textit{i.e. }}
\newcommand{\cf}{\textit{cf }}

\newcommand{\Card}{\operatorname{Card}}

\newcommand{\aire}{\mathcal{A}}

\newcommand{\LL}[1]{\mathscr{L}(#1)} %
\newcommand{\B}{\mathscr{B}} %
\newcommand{\Bc}[1]{B_{#1}} %
\newcommand{\M}[1]{\mathscr{M}_{#1}(\mathbb{R})}

\DeclareMathOperator{\im}{Im}
\DeclareMathOperator{\kr}{Ker}
\DeclareMathOperator{\rg}{rg}
\DeclareMathOperator{\tr}{tr}
\DeclareMathOperator{\spc}{Sp}
\DeclareMathOperator{\sgn}{sgn}
\DeclareMathOperator{\supp}{Supp}

\newcommand{\Mat}{{\rm{Mat}}}
\newcommand{\Vect}[1]{{\rm{Vect}}\left(#1\right)}

\newenvironment{smatrix}{%
  \begin{adjustbox}{width=.9\width}
    $
    \begin{pmatrix}
    }{%      
    \end{pmatrix}
    $
  \end{adjustbox}
}

\newenvironment{sarray}[1]{%
  \begin{adjustbox}{width=.9\width}
    $
    \begin{array}{#1}
    }{%      
    \end{array}
    $
  \end{adjustbox}
}

\newcommand{\vd}[2]{
  \scalebox{.8}{
    $\left(\!
      \begin{array}{c}
        #1 \\
        #2
      \end{array}
    \!\right)$
    }}

\newcommand{\vt}[3]{
  \scalebox{.8}{
    $\left(\!
      \begin{array}{c}
        #1 \\
        #2 \\
        #3 
      \end{array}
    \!\right)$
    }}

\newcommand{\vq}[4]{
  \scalebox{.8}{
    $\left(\!
      \begin{array}{c}
        #1 \\
        #2 \\
        #3 \\
        #4 
      \end{array}
    \!\right)$
    }}

\newcommand{\vc}[5]{
  \scalebox{.8}{
    $\left(\!
      \begin{array}{c}
        #1 \\
        #2 \\
        #3 \\
        #4 \\
        #5 
      \end{array}
    \!\right)$
    }}

\newcommand{\ee}{\text{e}}

\newcommand{\dd}{\text{d}}

%%% Ensemble de définition
\newcommand{\Df}{\mathscr{D}}
\newcommand{\Cf}{\mathscr{C}}
\newcommand{\Ef}{\mathscr{C}}

\newcommand{\rond}[1]{\,\overset{\scriptscriptstyle \circ}{\!#1}}

\newcommand{\df}[2]{\dfrac{\partial #1}{\partial #2}} %
\newcommand{\dfn}[2]{\partial_{#2}(#1)} %
\newcommand{\ddfn}[2]{\partial^2_{#2}(#1)} %
\newcommand{\ddf}[2]{\dfrac{\partial^2 #1}{\partial #2^2}} %
\newcommand{\ddfr}[3]{\dfrac{\partial^2 #1}{\partial #2 \partial
    #3}} %


\newcommand{\dlim}[1]{{\displaystyle \lim_{#1} \ }}
\newcommand{\dlimPlus}[2]{
  \dlim{
    \scalebox{.6}{
      $
      \begin{array}{l}
        #1 \rightarrow #2\\
        #1 > #2
      \end{array}
      $}}}
\newcommand{\dlimMoins}[2]{
  \dlim{
    \scalebox{.6}{
      $
      \begin{array}{l}
        #1 \rightarrow #2\\
        #1 < #2
      \end{array}
      $}}}

%%%%%%%%%%%%%%
%% petit o, développement limité
%%%%%%%%%%%%%%

\newcommand{\oo}[2]{{\underset {{\overset {#1\rightarrow #2}{}}}{o}}} %
\newcommand{\oox}[1]{{\underset {{\overset {x\rightarrow #1}{}}}{o}}} %
\newcommand{\oon}{{\underset {{\overset {n\rightarrow +\infty}{}}}{o}}} %
\newcommand{\po}[1]{{\underset {{\overset {#1}{}}}{o}}} %
\newcommand{\neqx}[1]{{\ \underset {{\overset {x \to #1}{}}}{\not\sim}\ }} %
\newcommand{\eqx}[1]{{\ \underset {{\overset {x \to #1}{}}}{\sim}\ }} %
\newcommand{\eqn}{{\ \underset {{\overset {n \to +\infty}{}}}{\sim}\ }} %
\newcommand{\eq}[2]{{\ \underset {{\overset {#1 \to #2}{}}}{\sim}\ }} %
\newcommand{\DL}[1]{{\rm{DL}}_1 (#1)} %
\newcommand{\DLL}[1]{{\rm{DL}}_2 (#1)} %

\newcommand{\negl}{<<}

\newcommand{\neglP}[1]{\begin{array}{c}
    \vspace{-.2cm}\\
    << \\
    \vspace{-.7cm}\\
    {\scriptstyle #1}
  \end{array}}

%%%%%%%%%%%%%%
%% borne sup, inf, max, min
%%%%%%%%%%%%%%
\newcommand{\dsup}[1]{\displaystyle \sup_{#1} \ }
\newcommand{\dinf}[1]{\displaystyle \inf_{#1} \ }
\newcommand{\dmax}[1]{\max\limits_{#1} \ }
\newcommand{\dmin}[1]{\min\limits_{#1} \ }

\newcommand{\dcup}[2]{{\textstyle\bigcup\limits_{#1}^{#2}}\hspace{.1cm}}
%\displaystyle \bigcup_{#1}^{#2}}
\newcommand{\dcap}[2]{{\textstyle\bigcap\limits_{#1}^{#2}}\hspace{.1cm}}
% \displaystyle \bigcap_{#1}^{#2}
%%%%%%%%%%%%%%
%% opérateurs logiques
%%%%%%%%%%%%%%
\newcommand{\NON}[1]{\mathop{\small \tt{NON}} (#1)}
\newcommand{\ET}{\mathrel{\mathop{\small \mathtt{ET}}}}
\newcommand{\OU}{\mathrel{\mathop{\small \tt{OU}}}}
\newcommand{\XOR}{\mathrel{\mathop{\small \tt{XOR}}}}

\newcommand{\id}{{\rm{id}}}

\newcommand{\sbullet}{\scriptstyle \bullet}
\newcommand{\stimes}{\scriptstyle \times}

%%%%%%%%%%%%%%%%%%
%% Probabilités
%%%%%%%%%%%%%%%%%%
\newcommand{\Prob}{\mathbb{P}}
\newcommand{\Ev}[1]{\left[ {#1} \right]}
\newcommand{\E}{\mathbb{E}}
\newcommand{\V}{\mathbb{V}}
\newcommand{\Cov}{{\rm{Cov}}}
\newcommand{\U}[2]{\mathcal{U}(\llb #1, #2\rrb)}
\newcommand{\Uc}[2]{\mathcal{U}([#1, #2])}
\newcommand{\Ucof}[2]{\mathcal{U}(]#1, #2])}
\newcommand{\Ucoo}[2]{\mathcal{U}(]#1, #2[)}
\newcommand{\Ucfo}[2]{\mathcal{U}([#1, #2[)}
\newcommand{\Bern}[1]{\mathcal{B}\left(#1\right)}
\newcommand{\Bin}[2]{\mathcal{B}\left(#1, #2\right)}
\newcommand{\G}[1]{\mathcal{G}\left(#1\right)}
\newcommand{\Pois}[1]{\mathcal{P}\left(#1\right)}
\newcommand{\HG}[3]{\mathcal{H}\left(#1, #2, #3\right)}
\newcommand{\Exp}[1]{\mathcal{E}\left(#1\right)}
\newcommand{\Norm}[2]{\mathcal{N}\left(#1, #2\right)}

\DeclareMathOperator{\cov}{Cov}

\newcommand{\var}{v.a.r. }
\newcommand{\suit}{\hookrightarrow}

\newcommand{\flecheR}[1]{\rotatebox{90}{\scalebox{#1}{\color{red}
      $\curvearrowleft$}}}


\newcommand{\partie}[1]{\mathcal{P}(#1)}
\newcommand{\Cont}[1]{\mathcal{C}^{#1}}
\newcommand{\Contm}[1]{\mathcal{C}^{#1}_m}

\newcommand{\llb}{\llbracket}
\newcommand{\rrb}{\rrbracket}

%\newcommand{\im}[1]{{\rm{Im}}(#1)}
\newcommand{\imrec}[1]{#1^{- \mathds{1}}}

\newcommand{\unq}{\mathds{1}}

\newcommand{\Hyp}{\mathtt{H}}

\newcommand{\eme}[1]{#1^{\scriptsize \mbox{ème}}}
\newcommand{\er}[1]{#1^{\scriptsize \mbox{er}}}
\newcommand{\ere}[1]{#1^{\scriptsize \mbox{ère}}}
\newcommand{\nd}[1]{#1^{\scriptsize \mbox{nd}}}
\newcommand{\nde}[1]{#1^{\scriptsize \mbox{nde}}}

\newcommand{\truc}{\mathop{\top}}
\newcommand{\fois}{\mathop{\ast}}

\newcommand{\f}[1]{\overrightarrow{#1}}

\newcommand{\checked}{\textcolor{green}{\checkmark}}

\def\restriction#1#2{\mathchoice
              {\setbox1\hbox{${\displaystyle #1}_{\scriptstyle #2}$}
              \restrictionaux{#1}{#2}}
              {\setbox1\hbox{${\textstyle #1}_{\scriptstyle #2}$}
              \restrictionaux{#1}{#2}}
              {\setbox1\hbox{${\scriptstyle #1}_{\scriptscriptstyle #2}$}
              \restrictionaux{#1}{#2}}
              {\setbox1\hbox{${\scriptscriptstyle #1}_{\scriptscriptstyle #2}$}
              \restrictionaux{#1}{#2}}}
\def\restrictionaux#1#2{{#1\,\smash{\vrule height .8\ht1 depth .85\dp1}}_{\,#2}}

\makeatletter
\newcommand*{\da@rightarrow}{\mathchar"0\hexnumber@\symAMSa 4B }
\newcommand*{\da@leftarrow}{\mathchar"0\hexnumber@\symAMSa 4C }
\newcommand*{\xdashrightarrow}[2][]{%
  \mathrel{%
    \mathpalette{\da@xarrow{#1}{#2}{}\da@rightarrow{\,}{}}{}%
  }%
}
\newcommand{\xdashleftarrow}[2][]{%
  \mathrel{%
    \mathpalette{\da@xarrow{#1}{#2}\da@leftarrow{}{}{\,}}{}%
  }%
}
\newcommand*{\da@xarrow}[7]{%
  % #1: below
  % #2: above
  % #3: arrow left
  % #4: arrow right
  % #5: space left 
  % #6: space right
  % #7: math style 
  \sbox0{$\ifx#7\scriptstyle\scriptscriptstyle\else\scriptstyle\fi#5#1#6\m@th$}%
  \sbox2{$\ifx#7\scriptstyle\scriptscriptstyle\else\scriptstyle\fi#5#2#6\m@th$}%
  \sbox4{$#7\dabar@\m@th$}%
  \dimen@=\wd0 %
  \ifdim\wd2 >\dimen@
    \dimen@=\wd2 %   
  \fi
  \count@=2 %
  \def\da@bars{\dabar@\dabar@}%
  \@whiledim\count@\wd4<\dimen@\do{%
    \advance\count@\@ne
    \expandafter\def\expandafter\da@bars\expandafter{%
      \da@bars
      \dabar@ 
    }%
  }%  
  \mathrel{#3}%
  \mathrel{%   
    \mathop{\da@bars}\limits
    \ifx\\#1\\%
    \else
      _{\copy0}%
    \fi
    \ifx\\#2\\%
    \else
      ^{\copy2}%
    \fi
  }%   
  \mathrel{#4}%
}
\makeatother



\newcount\depth

\newcount\depth
\newcount\totaldepth

\makeatletter
\newcommand{\labelsymbol}{%
      \ifnum\depth=0
        %
      \else
        \rlap{\,$\bullet$}%
      \fi
}

\newcommand*\bernoulliTree[1]{%
    \depth=#1\relax            
    \totaldepth=#1\relax
    \draw node(root)[bernoulli/root] {\labelsymbol}[grow=right] \draw@bernoulli@tree;
    \draw \label@bernoulli@tree{root};                                   
}                                                                        

\def\draw@bernoulli@tree{%
    \ifnum\depth>0 
      child[parent anchor=east] foreach \type/\label in {left child/$E$,right child/$S$} {%
          node[bernoulli/\type] {\label\strut\labelsymbol} \draw@bernoulli@tree
      }
      coordinate[bernoulli/increment] (dummy)
   \fi%
}

\def\label@bernoulli@tree#1{%
    \ifnum\depth>0
      ($(#1)!0.5!(#1-1)$) node[fill=white,bernoulli/decrement] {\tiny$p$}
      \label@bernoulli@tree{#1-1}
      ($(#1)!0.5!(#1-2)$) node[fill=white] {\tiny$q$}
      \label@bernoulli@tree{#1-2}
      coordinate[bernoulli/increment] (dummy)
   \fi%
}

\makeatother

\tikzset{bernoulli/.cd,
         root/.style={},
         decrement/.code=\global\advance\depth by-1\relax,
         increment/.code=\global\advance\depth by 1\relax,
         left child/.style={bernoulli/decrement},
         right child/.style={}}


\newcommand{\eps}{\varepsilon}

% \newcommand{\tendi}[1]{\xrightarrow[\footnotesize #1 \rightarrow
%   +\infty]{}}%

\newcommand{\tend}{\rightarrow}%
\newcommand{\tendn}{\underset{n\to +\infty}{\longrightarrow}} %
\newcommand{\ntendn}{\underset{n\to
    +\infty}{\not\hspace{-.15cm}\longrightarrow}} %
% \newcommand{\tendn}{\xrightarrow[\footnotesize n \rightarrow
%   +\infty]{}}%
\newcommand{\Tendx}[1]{\xrightarrow[\footnotesize x \rightarrow
  #1]{}}%
\newcommand{\tendx}[1]{\underset{x\to #1}{\longrightarrow}}%
\newcommand{\ntendx}[1]{\underset{x\to #1}{\not\!\!\longrightarrow}}%
\newcommand{\tendd}[2]{\underset{#1\to #2}{\longrightarrow}}%
% \newcommand{\tendd}[2]{\xrightarrow[\footnotesize #1 \rightarrow
%   #2]{}}%
\newcommand{\tendash}[1]{\xdashrightarrow[\footnotesize #1 \rightarrow
  +\infty]{}}%
\newcommand{\tendashx}[1]{\xdashrightarrow[\footnotesize x \rightarrow
  #1]{}}%
\newcommand{\tendb}[1]{\underset{#1 \to +\infty}{\longrightarrow}}%
\newcommand{\tendL}{\overset{\mathscr L}{\underset{n \to
      +\infty}{\longrightarrow}}}%
\newcommand{\tendP}{\overset{\Prob}{\underset{n \to
      +\infty}{\longrightarrow}}}%
\newcommand{\tenddL}[1]{\overset{\mathscr L}{\underset{#1 \to
      +\infty}{\longrightarrow}}}%

\NewEnviron{attention}{ %
  ~\\[-.2cm]\noindent
  \begin{minipage}{\linewidth}
  \setlength{\fboxsep}{3mm}%
  \ \ \dbend \ \ %
  \fbox{\parbox[t]{.88\linewidth}{\BODY}} %
  \end{minipage}\\
}

\NewEnviron{sattention}[1]{ %
  ~\\[-.2cm]\noindent
  \begin{minipage}{#1\linewidth}
  \setlength{\fboxsep}{3mm}%
  \ \ \dbend \ \ %
  \fbox{\parbox[t]{.88\linewidth}{\BODY}} %
  \end{minipage}\\
}

%%%%% OBSOLETE %%%%%%

% \newcommand{\attention}[1]{
%   \noindent
%   \begin{tabular}{@{}l|p{11.5cm}|}
%     \cline{2-2}
%     \vspace{-.2cm} 
%     & \nl
%     \dbend & #1 \nl
%     \cline{2-2}
%   \end{tabular}
% }

% \newcommand{\attentionv}[2]{
%   \noindent
%   \begin{tabular}{@{}l|p{11.5cm}|}
%     \cline{2-2}
%     \vspace{-.2cm} 
%     & \nl
%     \dbend & #2 \nl[#1 cm]
%     \cline{2-2}
%   \end{tabular}
% }

\newcommand{\explainvb}[2]{
  \noindent
  \begin{tabular}{@{}l|p{11.5cm}|}
    \cline{2-2}
    \vspace{-.2cm} 
    & \nl
    \hand & #2 \nl [#1 cm]
    \cline{2-2}
  \end{tabular}
}


% \noindent
% \begin{tabular}{@{}l|lp{11cm}|}
%   \cline{3-3} 
%   \multicolumn{1}{@{}l@{\dbend}}{} & & #1 \nl
%   \multicolumn{1}{l}{} & & \nl [-.8cm]
%   & & #2 \nl
%   \cline{2-3}
% \end{tabular}

% \newcommand{\attention}[1]{
%   \noindent
%   \begin{tabular}{@{}@{}cp{11cm}}
%     \dbend & #1 \nl
%   \end{tabular}
% }

\newcommand{\PP}[1]{\mathcal{P}(#1)}
\newcommand{\FF}[1]{\mathcal{F}(#1)}

\newcommand{\DSum}[2]{\displaystyle\sum\limits_{#1}^{#2}\hspace{.1cm}}
\newcommand{\Sum}[2]{{\textstyle\sum\limits_{#1}^{#2}}\hspace{.1cm}}
\newcommand{\Serie}{\textstyle\sum\ }
\newcommand{\Prod}[2]{\textstyle\prod\limits_{#1}^{#2}}

\newcommand{\Prim}[3]{\left[\ {#1} \ \right]_{\scriptscriptstyle
   \hspace{-.15cm} ~_{#2}\, }^ {\scriptscriptstyle \hspace{-.15cm} ~^{#3}\, }}

% \newcommand{\Prim}[3]{\left[\ {#1} \ \right]_{\scriptscriptstyle
%     \!\!~_{#2}}^ {\scriptscriptstyle \!\!~^{#3}}}

\newcommand{\dint}[2]{\displaystyle \int_{#1}^{#2}\ }
\newcommand{\Int}[2]{{\rm{Int}}_{\scriptscriptstyle #1, #2}}
\newcommand{\dt}{\ dt}
\newcommand{\dx}{\ dx}

\newcommand{\llpar}[1]{\left(\!\!\!
    \begin{array}{c}
      \rule{0pt}{#1}
    \end{array}
  \!\!\!\right.}

\newcommand{\rrpar}[1]{\left.\!\!\!
    \begin{array}{c}
      \rule{0pt}{#1}
    \end{array}
  \!\!\!\right)}

\newcommand{\llacc}[1]{\left\{\!\!\!
    \begin{array}{c}
      \rule{0pt}{#1}
    \end{array}
  \!\!\!\right.}

\newcommand{\rracc}[1]{\left.\!\!\!
    \begin{array}{c}
      \rule{0pt}{#1}
    \end{array}
  \!\!\!\right\}}

\newcommand{\ttacc}[1]{\mbox{\rotatebox{-90}{\hspace{-.7cm}$\llacc{#1}$}}}
\newcommand{\bbacc}[1]{\mbox{\rotatebox{90}{\hspace{-.5cm}$\llacc{#1}$}}}

\newcommand{\comp}[1]{\overline{#1}}%

\newcommand{\dcomp}[2]{\stackrel{\mbox{\ \ \----}{\scriptscriptstyle
      #2}}{#1}}%

% \newcommand{\Comp}[2]{\stackrel{\mbox{\ \
%       \-------}{\scriptscriptstyle #2}}{#1}}

% \newcommand{\dcomp}[2]{\stackrel{\mbox{\ \
%       \-------}{\scriptscriptstyle #2}}{#1}}

\newcommand{\A}{\mathscr{A}}

\newcommand{\conc}[1]{
  \begin{center}
    \fbox{
      \begin{tabular}{c}
        #1
      \end{tabular}
    }
  \end{center}
}

\newcommand{\concC}[1]{
  \begin{center}
    \fbox{
    \begin{tabular}{C{10cm}}
      \quad #1 \quad
    \end{tabular}
    }
  \end{center}
}

\newcommand{\concL}[2]{
  \begin{center}
    \fbox{
    \begin{tabular}{C{#2cm}}
      \quad #1 \quad
    \end{tabular}
    }
  \end{center}
}

% \newcommand{\lims}[2]{\prod\limits_{#1}^{#2}}

\newtheorem{theorem}{Théorème}[]
\newtheorem{lemma}{Lemme}[]
\newtheorem{proposition}{Proposition}[]
\newtheorem{corollary}{Corollaire}[]

% \newenvironment{proof}[1][Démonstration]{\begin{trivlist}
% \item[\hskip \labelsep {\bfseries #1}]}{\end{trivlist}}
\newenvironment{definition}[1][Définition]{\begin{trivlist}
\item[\hskip \labelsep {\bfseries #1}]}{\end{trivlist}}
\newenvironment{example}[1][Exemple]{\begin{trivlist}
\item[\hskip \labelsep {\bfseries #1}]}{\end{trivlist}}
\newenvironment{examples}[1][Exemples]{\begin{trivlist}
\item[\hskip \labelsep {\bfseries #1}]}{\end{trivlist}}
\newenvironment{notation}[1][Notation]{\begin{trivlist}
\item[\hskip \labelsep {\bfseries #1}]}{\end{trivlist}}
\newenvironment{propriete}[1][Propriété]{\begin{trivlist}
\item[\hskip \labelsep {\bfseries #1}]}{\end{trivlist}}
\newenvironment{proprietes}[1][Propriétés]{\begin{trivlist}
\item[\hskip \labelsep {\bfseries #1}]}{\end{trivlist}}
\newenvironment{remarkSC}[1][Remarque]{\begin{trivlist}
\item[\hskip \labelsep {\bfseries #1}]}{\end{trivlist}}
\newenvironment{application}[1][Application]{\begin{trivlist}
\item[\hskip \labelsep {\bfseries #1}]}{\end{trivlist}}

% Environnement pour les réponses des DS
\newenvironment{answer}{\par\emph{Réponse :}\par{}}
{\vspace{-.6cm}\hspace{\stretch{1}}\rule{1ex}{1ex}\vspace{.3cm}}

\newenvironment{answerTD}{\vspace{.2cm}\par\emph{Réponse :}\par{}}
{\hspace{\stretch{1}}\rule{1ex}{1ex}\vspace{.3cm}}

\newenvironment{answerCours}{\noindent\emph{Réponse :}}
{\rule{1ex}{1ex}}%\vspace{.3cm}}


% footnote in footer
\newcommand{\fancyfootnotetext}[2]{%
  \fancypagestyle{dingens}{%
    \fancyfoot[LO,RE]{\parbox{0.95\textwidth}{\footnotemark[#1]\footnotesize
        #2}}%
  }%
  \thispagestyle{dingens}%
}

%%% définit le style (arabic : 1,2,3...) et place des parenthèses
%%% autour de la numérotation
\renewcommand*{\thefootnote}{(\arabic{footnote})}
% http://www.tuteurs.ens.fr/logiciels/latex/footnote.html

%%%%%%%% tikz axis
% \pgfplotsset{every axis/.append style={
%                     axis x line=middle,    % put the x axis in the middle
%                     axis y line=middle,    % put the y axis in the middle
%                     axis line style={<->,color=blue}, % arrows on the axis
%                     xlabel={$x$},          % default put x on x-axis
%                     ylabel={$y$},          % default put y on y-axis
%             }}

%%%% s'utilise comme suit

% \begin{axis}[
%   xmin=-8,xmax=4,
%   ymin=-8,ymax=4,
%   grid=both,
%   ]
%   \addplot [domain=-3:3,samples=50]({x^3-3*x},{3*x^2-9}); 
% \end{axis}

%%%%%%%%



%%%%%%%%%%%% Pour avoir des numéros de section qui correspondent à
%%%%%%%%%%%% ceux du tableau
\renewcommand{\thesection}{\Roman{section}.\hspace{-.3cm}}
\renewcommand{\thesubsection}{\Roman{section}.\arabic{subsection}.\hspace{-.2cm}}
\renewcommand{\thesubsubsection}{\Roman{section}.\arabic{subsection}.\alph{subsubsection})\hspace{-.2cm}}
%%%%%%%%%%%% 

%%% Changer le nom des figures : Fig. au lieu de Figure
\usepackage[font=small,labelfont=bf,labelsep=space]{caption}
\captionsetup{%
  figurename=Fig.,
  tablename=Tab.
}
% \renewcommand{\thesection}{\Roman{section}.\hspace{-.2cm}}
% \renewcommand{\thesubsection}{\Roman{section}
%   .\hspace{.2cm}\arabic{subsection}\ .\hspace{-.3cm}}
% \renewcommand{\thesubsubsection}{\alph{subsection})}

\newenvironment{tabliste}[1]
{\begin{tabenum}[\bfseries\small\itshape #1]}{\end{tabenum}} 

\newenvironment{noliste}[1] %
{\begin{enumerate}[\bfseries\small\itshape #1]} %
  {\end{enumerate}}

\newenvironment{nonoliste}[1] %
{\begin{enumerate}[\hspace{-12pt}\bfseries\small\itshape #1]} %
  {\end{enumerate}}

\newenvironment{arrayliste}[1]{ 
  % List with minimal white space to fit in small areas, e.g. table
  % cell
  \begin{minipage}[t]{\linewidth} %
    \begin{enumerate}[\bfseries\small\itshape #1] %
      {\leftmargin=0.5em \rightmargin=0em
        \topsep=0em \parskip=0em \parsep=0em
        \listparindent=0em \partopsep=0em \itemsep=0pt
        \itemindent=0em \labelwidth=\leftmargin\labelsep+0.25em}
    }{
    \end{enumerate}\end{minipage}
}

\newenvironment{nolistes}[2]
{\begin{enumerate}[\bfseries\small\itshape
    #1]\setlength{\itemsep}{#2 mm}}{\end{enumerate}}

\newenvironment{liste}[1]
{\begin{enumerate}[\hspace{12pt}\bfseries\small\itshape #1]}{\end{enumerate}} 



%%%%%%%% Pour les programmes de colle %%%%%%%

\newcommand{\cours}{{\small \tt (COURS)}} %
\newcommand{\poly}{{\small \tt (POLY)}} %
\newcommand{\exo}{{\small \tt (EXO)}} %
\newcommand{\culture}{{\small \tt (CULTURE)}} %
\newcommand{\methodo}{{\small \tt (MÉTHODO)}} %
\newcommand{\methodob}{\Boxed{\mbox{\tt MÉTHODO}}} %

%%%%%%%% Pour les TD %%%%%%%
\newtheoremstyle{exostyle} {\topsep} % espace avant
{.6cm} % espace apres
{} % Police utilisee par le style de thm
{} % Indentation (vide = aucune, \parindent = indentation paragraphe)
{\bfseries} % Police du titre de thm
{} % Signe de ponctuation apres le titre du thm
{ } % Espace apres le titre du thm (\newline = linebreak)
{\thmname{#1}\thmnumber{ #2}\thmnote{.
    \normalfont{\textit{#3}}}} % composants du titre du thm : \thmname
                               % = nom du thm, \thmnumber = numéro du
                               % thm, \thmnote = sous-titre du thm
 
\theoremstyle{exostyle}
\newtheorem{exercice}{Exercice}
\newtheorem{exerciceAP}{Exercice avec préparation}
\newtheorem{exerciceSP}{Exercice sans préparation}
\newtheorem*{exoCours}{Exercice}

%%%%%%%% Pour des théorèmes Sans Espaces APRÈS %%%%%%%
\newtheoremstyle{exostyleSE} {\topsep} % espace avant
{} % espace apres
{} % Police utilisee par le style de thm
{} % Indentation (vide = aucune, \parindent = indentation paragraphe)
{\bfseries} % Police du titre de thm
{} % Signe de ponctuation apres le titre du thm
{ } % Espace apres le titre du thm (\newline = linebreak)
{\thmname{#1}\thmnumber{ #2}\thmnote{.
    \normalfont{\textit{#3}}}} % composants du titre du thm : \thmname
                               % = nom du thm, \thmnumber = numéro du
                               % thm, \thmnote = sous-titre du thm
 
\theoremstyle{exostyleSE}
\newtheorem{exerciceSE}{Exercice}
\newtheorem*{exoCoursSE}{Exercice}

% \newcommand{\lims}[2]{\prod\limits_{#1}^{#2}}

\newtheorem{theoremSE}{Théorème}[]
\newtheorem{lemmaSE}{Lemme}[]
\newtheorem{propositionSE}{Proposition}[]
\newtheorem{corollarySE}{Corollaire}[]

% \newenvironment{proofSE}[1][Démonstration]{\begin{trivlist}
% \item[\hskip \labelsep {\bfseries #1}]}{\end{trivlist}}
\newenvironment{definitionSE}[1][Définition]{\begin{trivlist}
  \item[\hskip \labelsep {\bfseries #1}]}{\end{trivlist}}
\newenvironment{exampleSE}[1][Exemple]{\begin{trivlist} 
  \item[\hskip \labelsep {\bfseries #1}]}{\end{trivlist}}
\newenvironment{examplesSE}[1][Exemples]{\begin{trivlist}
\item[\hskip \labelsep {\bfseries #1}]}{\end{trivlist}}
\newenvironment{notationSE}[1][Notation]{\begin{trivlist}
\item[\hskip \labelsep {\bfseries #1}]}{\end{trivlist}}
\newenvironment{proprieteSE}[1][Propriété]{\begin{trivlist}
\item[\hskip \labelsep {\bfseries #1}]}{\end{trivlist}}
\newenvironment{proprietesSE}[1][Propriétés]{\begin{trivlist}
\item[\hskip \labelsep {\bfseries #1}]}{\end{trivlist}}
\newenvironment{remarkSE}[1][Remarque]{\begin{trivlist}
\item[\hskip \labelsep {\bfseries #1}]}{\end{trivlist}}
\newenvironment{applicationSE}[1][Application]{\begin{trivlist}
\item[\hskip \labelsep {\bfseries #1}]}{\end{trivlist}}

%%%%%%%%%%% Obtenir les étoiles sans charger le package MnSymbol
%%%%%%%%%%%
\DeclareFontFamily{U} {MnSymbolC}{}
\DeclareFontShape{U}{MnSymbolC}{m}{n}{
  <-6> MnSymbolC5
  <6-7> MnSymbolC6
  <7-8> MnSymbolC7
  <8-9> MnSymbolC8
  <9-10> MnSymbolC9
  <10-12> MnSymbolC10
  <12-> MnSymbolC12}{}
\DeclareFontShape{U}{MnSymbolC}{b}{n}{
  <-6> MnSymbolC-Bold5
  <6-7> MnSymbolC-Bold6
  <7-8> MnSymbolC-Bold7
  <8-9> MnSymbolC-Bold8
  <9-10> MnSymbolC-Bold9
  <10-12> MnSymbolC-Bold10
  <12-> MnSymbolC-Bold12}{}

\DeclareSymbolFont{MnSyC} {U} {MnSymbolC}{m}{n}

\DeclareMathSymbol{\filledlargestar}{\mathrel}{MnSyC}{205}
\DeclareMathSymbol{\largestar}{\mathrel}{MnSyC}{131}

\newcommand{\facile}{\rm{(}$\scriptstyle\largestar$\rm{)}} %
\newcommand{\moyen}{\rm{(}$\scriptstyle\filledlargestar$\rm{)}} %
\newcommand{\dur}{\rm{(}$\scriptstyle\filledlargestar\filledlargestar$\rm{)}} %
\newcommand{\costaud}{\rm{(}$\scriptstyle\filledlargestar\filledlargestar\filledlargestar$\rm{)}}

%%%%%%%%%%%%%%%%%%%%%%%%%

%%%%%%%%%%%%%%%%%%%%%%%%%
%%%%%%%% Fin de la partie TD

%%%%%%%%%%%%%%%%
%%%%%%%%%%%%%%%%
\makeatletter %
\newenvironment{myitemize}{%
  \setlength{\topsep}{0pt} %
  \setlength{\partopsep}{0pt} %
  \renewcommand*{\@listi}{\leftmargin\leftmargini \parsep\z@
    \topsep\z@ \itemsep\z@} \let\@listI\@listi %
  \itemize %
}{\enditemize} %
\makeatother
%%%%%%%%%%%%%%%%
%%%%%%%%%%%%%%%%

%% Commentaires dans la correction du livre

\newcommand{\Com}[1]{
% Define box and box title style
\tikzstyle{mybox} = [draw=black!50,
very thick,
    rectangle, rounded corners, inner sep=10pt, inner ysep=8pt]
\tikzstyle{fancytitle} =[rounded corners, fill=black!80, text=white]
\tikzstyle{fancylogo} =[ text=white]
\begin{center}

\begin{tikzpicture}
\node [mybox] (box){%

    \begin{minipage}{0.90\linewidth}
\vspace{6pt}  #1
    \end{minipage}
};
\node[fancytitle, right=10pt] at (box.north west) 
{\bfseries\normalsize{Commentaire}};

\end{tikzpicture}%

\end{center}
%
}

% \NewEnviron{remark}{%
%   % Define box and box title style
%   \tikzstyle{mybox} = [draw=black!50, very thick, rectangle, rounded
%   corners, inner sep=10pt, inner ysep=8pt] %
%   \tikzstyle{fancytitle} =[rounded corners, fill=black!80,
%   text=white] %
%   \tikzstyle{fancylogo} =[ text=white]
%   \begin{center}
%     \begin{tikzpicture}
%       \node [mybox] (box){%
%         \begin{minipage}{0.90\linewidth}
%           \vspace{6pt} \BODY
%         \end{minipage}
%       }; %
%       \node[fancytitle, right=10pt] at (box.north west) %
%       {\bfseries\normalsize{Commentaire}}; %
%     \end{tikzpicture}%
%   \end{center}
% }
% \makeatother


%%%
% Remarques comme dans le livre
%%%

\NewEnviron{remark}{%
  % Define box and box title style
  \tikzstyle{mybox} = [draw=black!50, very thick, rectangle, rounded
  corners, inner sep=10pt, inner ysep=8pt] %
  \tikzstyle{fancytitle} = [rounded corners , fill=black!80,
  text=white] %
  \tikzstyle{fancylogo} =[ text=white]
  \begin{center}
    \begin{tikzpicture}
      \node [mybox] (box){%
        \begin{minipage}{0.90\linewidth}
          \vspace{6pt} \BODY
        \end{minipage}
      }; %
      \node[fancytitle, right=10pt] at (box.north west) %
      {\bfseries\normalsize{Commentaire}}; %
    \end{tikzpicture}%
  \end{center}
}

\NewEnviron{remarkST}{%
  % Define box and box title style
  \tikzstyle{mybox} = [draw=black!50, very thick, rectangle, rounded
  corners, inner sep=10pt, inner ysep=8pt] %
  \tikzstyle{fancytitle} = [rounded corners , fill=black!80,
  text=white] %
  \tikzstyle{fancylogo} =[ text=white]
  \begin{center}
    \begin{tikzpicture}
      \node [mybox] (box){%
        \begin{minipage}{0.90\linewidth}
          \vspace{6pt} \BODY
        \end{minipage}
      }; %
      % \node[fancytitle, right=10pt] at (box.north west) %
%       {\bfseries\normalsize{Commentaire}}; %
    \end{tikzpicture}%
  \end{center}
}

\NewEnviron{remarkL}[1]{%
  % Define box and box title style
  \tikzstyle{mybox} = [draw=black!50, very thick, rectangle, rounded
  corners, inner sep=10pt, inner ysep=8pt] %
  \tikzstyle{fancytitle} =[rounded corners, fill=black!80,
  text=white] %
  \tikzstyle{fancylogo} =[ text=white]
  \begin{center}
    \begin{tikzpicture}
      \node [mybox] (box){%
        \begin{minipage}{#1\linewidth}
          \vspace{6pt} \BODY
        \end{minipage}
      }; %
      \node[fancytitle, right=10pt] at (box.north west) %
      {\bfseries\normalsize{Commentaire}}; %
    \end{tikzpicture}%
  \end{center}
}

\NewEnviron{remarkSTL}[1]{%
  % Define box and box title style
  \tikzstyle{mybox} = [draw=black!50, very thick, rectangle, rounded
  corners, inner sep=10pt, inner ysep=8pt] %
  \tikzstyle{fancytitle} =[rounded corners, fill=black!80,
  text=white] %
  \tikzstyle{fancylogo} =[ text=white]
  \begin{center}
    \begin{tikzpicture}
      \node [mybox] (box){%
        \begin{minipage}{#1\linewidth}
          \vspace{6pt} \BODY
        \end{minipage}
      }; %
%       \node[fancytitle, right=10pt] at (box.north west) %
%       {\bfseries\normalsize{Commentaire}}; %
    \end{tikzpicture}%
  \end{center}
}

\NewEnviron{titre} %
{ %
  ~\\[-1.8cm]
  \begin{center}
    \bf \LARGE \BODY
  \end{center}
  ~\\[-.6cm]
  \hrule %
  \vspace*{.2cm}
} %

\NewEnviron{titreL}[2] %
{ %
  ~\\[-#1cm]
  \begin{center}
    \bf \LARGE \BODY
  \end{center}
  ~\\[-#2cm]
  \hrule %
  \vspace*{.2cm}
} %




\pagestyle{fancy} %
\lhead{ECE2 \hfill septembre 2017 \\
 Mathématiques\\[.2cm]} %
\chead{\hrule} %
\rhead{} %
\lfoot{} %
\cfoot{} %
\rfoot{\thepage} %

\renewcommand{\headrulewidth}{0pt}% : Trace un trait de séparation
 % de largeur 0,4 point. Mettre 0pt
 % pour supprimer le trait.

\renewcommand{\footrulewidth}{0.4pt}% : Trace un trait de séparation
 % de largeur 0,4 point. Mettre 0pt
 % pour supprimer le trait.

\setlength{\headheight}{14pt}

\title{\bf \vspace{-1cm} ORAUX HEC 2013} %
\author{} %
\date{} %

\begin{document}

\maketitle %
\vspace{-1.2cm}\hrule %
\thispagestyle{fancy}

\vspace*{.4cm}

% DEBUT DU DOC À MODIFIER : tout virer jusqu'au début de l'exo

\section{Annales 2013}

% \setcounter{exercice}{0}
\begin{exercice}{\it (Exercice avec préparation)}~
 \begin{noliste}{1.}
 \setlength{\itemsep}{4mm}
 \item Soit $n$ un entier supérieur ou égal à 1. On dit qu'une
 expérience aléatoire suit le schéma binomial si elle consiste en
 $n$ répétitions indépendantes de la même épreuve, pouvant mener à
 2 résultat : succès ou échecs.\\
 On sait alors que la variable aléatoire comptant le nombre de
 succès pendant l'expérience suit une loi binomiale de paramètres
 $n$ et $p$, où $p$ est la probabilité du succès lors d'une
 épreuve. On sait de même que le nombre d'échecs suit la loi
 binomiale de paramètres $n$ et $q = 1-p$.
 
 \item Soit $(X_{n})_{ n \in \N^* }$ une suite de variables aléatoires
 définies sur un espace probabilisé $(\Omega, \mathcal{A}, P )$,
 indépendantes et de même loi de Bernouilli de paramètre
 $\frac{1}{2}$.
 
 On pose, pour tout $n \in \N^*$, $W_{n} = \Sum{k = 1}{n} k X_{k}$ et
 $s_{n} = \frac{ n ( n + 1 ) }{ 2 }$.
 
 \begin{noliste}{a)}
 \setlength{\itemsep}{2mm}

 \item Par linéarité de l'espérance, $W_{n}$ admet une espérance et :
 
\[
 E ( W_{n} ) = \Sum{k = 1}{n} k E ( X_{k} ) = \Sum{k = 1}{n} \frac{ 1
}{
 2 } k = \frac{ 1 }{ 2} \Sum{k = 1}{n} k = \frac{ 1 }{ 2} \times
 \frac{ n (n + 1 ) }{ 2 } = \frac{ 1 }{ 2 } s_{n}.
\]
 
 Par indépendance des variables $X_{i}$ puis par quadracité de la
 variance, $W_{n}$ admet une variance et :
 
\[
 V ( X_{n} ) = \Sum{k = 1}{n} V ( k X_{k} ) = \Sum{k = 1}{n} k^{2} V (
X_{k}
 ) = \Sum{k = 1}{n} \frac{ 1 }{ 4 } k^{2} = \frac{ 1 }{ 4 }
 \Sum{k = 1}{n} k^{2} = \frac{ 1 }{ 4 } \times \frac{ n (n + 1 ) (2n +
1
 ) }{ 6 } = \frac{ 2n + 1 }{ 12 } s_{n}.
\]

 \item Comme une somme de termes positifs n'est nulle que si tous
 les termes sont nuls,
 
\[
 \Ev{W_{n} = 0} = \bigcap\limits_{i = 1}{n} \Ev{X_{i} = 0}
\]
 et par indépendances des $X_{i}$,
 
\[
 P \Ev{ W_{n} = 0 } = \prod\limits_{i = 1}{n} P \Ev{X_{i} = 0 } = 
 \prod\limits_{i = 1}{n} \frac{1}{2} = \left( \frac{ 1 }{ 2 }
 \right)^{n}.
\]
 D'autre par on remarque $W_{n}$ atteint la valeur $s_{n}$ lorsque
 tous les $X_{i}$ valent 1. De plus si un ou plus des $X_{i}$ ne vaut
 pas 1, la somme sera alors inférieure ou égale à $s_{n} - 1$, donc
 ne pourra pas valoir $s_{n}$. On en déduit l'écriture d'évènements
 suivante, puis par indépendance :
 
\[
 \Ev{W_{n} = s_{n} } = \bigcap\limits_{i = 1}{n} \Ev{X_{i} = 1} \ \ \
 \text{ et } \ \ \ P \Ev{ X_{n} = 0 } = \prod\limits_{i = 1}{n} P
 \Ev{X_{i} = 1} = \prod\limits_{i = 1}{n} \frac{1}{2} = \left( \frac{
 1 }{ 2 } \right)^{n}.
\]
 
 \item $W_{n} = 3$ peut être atteinte sous les formes suivantes :
 
\[
 3 = 0 + 0 + 3 + 0 + \dots + 0 \ \ \text{ ou } \ \ 3 = 1 + 2 + 0
 + \dots + 0
\]
 qui existent ou non selon que $n$ dépasse 2, puis 3. On a donc : 
 \begin{noliste}{$\sbullet$}
 \item Pour $n = 1$,
 
\[
 \Ev{W_{1} = 3} = \emptyset \ \ \ \text{ et } \ \ \ P \Ev{W_{1} = 3 } =
0. 
\]
 
 \item Pour $n = 2$,
 
\[
 \Ev{W_{2} = 3 } = \Ev{X_{1} = 1 } \cap \Ev{X_{2} = 1 } \ \ \ \text{
 et } \ \ \ P \Ev{ W_{2} = 3 } = \frac{ 1 }{ 4 }.
\]

 \item Pour $n \geq 3$,
 
\[
 \Ev{W_{n} = 3 } = \left[ \ \Ev{X_{1} = 1 } \cap \Ev{X_{2} = 1 } \cap
 \left( \bigcap\limits_{i = 3}{n} \Ev{X_{i} = 0} \right) \right]
 \cup \left[ \ \Ev{X_{1} = 0 } \cap \Ev{X_{2} = 0} \cap \Ev{X_{3} = 1}
 \cap \left( \bigcap\limits_{i = 4}{n} \Ev{X_{i} = 0 } \right)
 \right]
\]
 puis par incompatibilité de la réunion et indépendance des
 $X_{i}$,
 
\[
 P \Ev{ W_{n} = 3 } = \left( \frac{ 1 }{ 2 } \right)^{n} + \left(
 \frac{ 1 }{ 2 } \right)^{n} = \frac{ 2 }{ 2^{n} } = \left(
 \frac{ 1 }{ 2 } \right)^{ n-1 }.
\]
 \end{noliste}
 \end{noliste}
 
 \item Question difficile. Il faut comprendre que pour chaque
 situation menant à $W_{n} = k$, la situation "inverse" (tous les
 échecs transformés en succès et tous les succès en échecs) donnera
 $W_{n} = s_{n} - k$. Pour formaliser l'égalité d'évènements, on écrit
:
 
\[
 (W_{n} = s_{n} - k ) = ( s_{n} - W_{n} = k ) = \left( \Sum{i = 1}{n} i
-
 \Sum{i = 1}{n} i X_{i} = k \right) = \left( \Sum{i = 1}{n} i (1-X_{i})
 = k \right) = \Ev{ Z_{n} = k }
\]
 où $Z_{n} = \Sum{i = 1}{n} i (1 - X_{i} )$ avec les variables $Y_{i} =
1
 - X_{i}$ qui sont indépendantes et qui suivent la loi de Bernouilli
 de paramètre $\frac{1}{2}$, donc $Z_{n}$ suit la même loi que
 $W_{n}$. On en déduit bien que :
 
\[
 P \Ev{ W_{n} = k } = P \Ev{ Z_{n} = k } = P \left(\Ev{ W_{n} = s_{n} -
k }\right).
\]
 
 \item
 \begin{noliste}{a)}
 \setlength{\itemsep}{2mm}
 \item Sachant $\Ev{W_{n} = j}$, on a alors :
 
\[
 W_{n + 1} = \Sum{k = 1}{n + 1} k X_{k} = \Sum{k = 1}{n} k X_{k} + (n +
1)
 X_{n + 1} = W_{n} + (n + 1) W_{n + 1} = j + (n + 1) X_{n + 1}
\]
 avec $X_{n + 1}$ qui vaut 0 ou 1, donc :
 
\[
 W_{n + 1} \backslash \Ev{W_{n} = j } ( \Omega ) = \{ j ; j + n + 1
 \} \ \, \ \ P_{ \Ev{W_{n} = j } } ( W_{n + 1} = j ) = P_{ \Ev{W_{n}
 = j } } ( X_{n + 1} = 0 ) = \frac{ 1 }{ 2 }
\]
 et
 
\[
 P_{ \Ev{W_{n} = j } } ( W_{n + 1} = j + n + 1 ) = P_{ \Ev{W_{n} = j }
 } ( X_{n + 1} = 1 ) = \frac{ 1 }{ 2 }.
\]
 
 \item De plus $\Ev{W_{n} = j}_{ j \in \llb 0 ; s_{n} \rrb }$ est un
 système complet d'évènements donc par formule des probabilités
 totales, pour tout $k \in \llb 0 ; s_{n + 1} \rrb$ :
 
\[
 ( W_{n + 1} = k ) = \dcup{j = 0}{ s_{n} } [ \ \Ev{W_{n} = j }
 \cap (W_{n + 1} = k ) ]
\]
 donc par incompatibilité de la réunion et probabilités
 composées :
 
\[
 P \left(\Ev{ W_{n + 1} = k }\right) = \Sum{j = 0}{ s_{n} } P \Ev{
W_{n} = j } P_{
 \Ev{W_{n} = j} } ( W_{n + 1} = k )
\]
 et cette probabilité conditionnelle est non nulle nulle si et
 seulement $k \in \{ j ; j + n + 1 \}$ donc pour tout $j$
 vérifiant :
 
\[
 j = k \ \ \text{ ou } \ \ j + n + 1 = k \Longleftrightarrow j = k -
 n - 1.
\]
 
 On sépare alors selon que ces valeur existent ou non :
 \begin{noliste}{$\sbullet$}
 \item Si $ k \leq n $, $k-n-1 \leq -1$, $k$ est bien dans le
 support de $W_{n}$ mais $k-n-1$ n'y est pas donc :
 
\[
 P \left(\Ev{ W_{n + 1} = k }\right) = \Sum{j = 0}{k-1} 0 + \frac{ 1 }{
2}
 \Prob\left(\Ev{\Ev{ W_{n} = k }}\right) + \Sum{j = k + 1}{s_{n}} 0.
\]
 
 \item Si $n + 1 \leq j \leq s_{n}$, $k$ est bien dans le support de
 $W_{n}$ et $0 \leq k-n-1 \leq s_{n} - n - 1 \leq s_{n}$ donc $k-n-1$
 y est aussi et :
 
\[
 P \left(\Ev{ W_{n + 1} = k }\right) = \Sum{j = 0}{k-n-2} 0 +
\frac{1}{2} P \left(\Ev{ W_{n}
 = k-n-1 }\right) + \Sum{j = k-n}{k-1} 0 + \frac{ 1 }{ 2 }
\Prob\left(\Ev{\Ev{
 W_{n} = k }}\right) + \Sum{j = k + 1}{s_{n}} 0
\]
 
 \item Si $s_{n} + 1 \leq k \leq s_{n + 1}$, alors $k$ n'est pas dans
 le support de $W_{n}$ (trop grand) mais $k-n-1$ y est bien (en
 effet $s_{n + 1} = s_{n} + n + 1$ donc $k-n-1$ est inférieur à
 $s_{n}$, et supérieur à $s_{n} - n = s_{n-1} $ qui est
 positif). On obtient alors :
 
\[
 P \left(\Ev{ W_{n + 1} = k }\right) = \Sum{j = 0}{k-n-2} 0 +
\frac{1}{2} P \left(\Ev{ W_{n} = 
 k-n-1 }\right) + \Sum{j = n-k}{s_{n}} 0
\]
 \end{noliste}
 et en rassemblant :
 
\[
 P \left(\Ev{ W_{n + 1} = k }\right) = \left\{ 
\begin{array}{ll}
 \frac{ 1 }{ 2 } P \Ev{ W_{n} = k } & \text{ si
 } k \leq n \\
\\\
frac{ 1 }{ 2 } P \Ev{W_{n} = k} + \frac{ 1
 }{ 2 } P \left(\Ev{ W_{n} = k-n-1}\right) & \text{ si } n + 1 \leq k
\leq s_{n} \\
 \\
\frac{ 1 }{ 2 } P \left(\Ev{ W_{n} = k-n-1}\right) & \text{ si } s_{n}
+ 1
 \leq k \leq s_{n + 1}
 \\
\end{array}
 \right.
\]
 \end{noliste}
 \end{noliste}
\end{exercice}

\begin{exercice}{\it (Exercice sans préparation)}~\\
 On pose pour tout $n \in \N^*$ : $S_{n} = \Sum{k = 1}{n} k^{2} \ln
\left(
 \frac{ k }{ n } \right)$.
 \begin{noliste}{1.}
 \setlength{\itemsep}{4mm}
 \item On fait apparaître une somme de Riemann :
 
\[
 \frac{ S_{n} }{ n^{3} } = \frac{ 1 }{ n } \Sum{k = 1}{n} \frac{ k^{2}
}{
 n^{2} } \ln \left( \frac{ k }{ n } \right) = \frac{ 1 }{ n }
 \Sum{k = 1}{n} f \left( \frac{ k }{ n } \right)
\]
 et posant la fonction $f(x) = x^{2} \ln (x)$ qui est continue sur
 $]0;1]$, et qui se prolonge par continuité en 0 en posant $f(0) = 
 0$ (par croissances comparées). Les sommes de Riemann donnent
 alors :
 
\[
 \dlim{ n \rightarrow + \infty } \frac{ S_{n} }{ n^{3} } = \dint{0}{1}
f(x) \ dx 
\]
 
 qu'on calcule en revenant à l'intégrale partielle (car $\ln$ n'est
 pas définie en 0) : avec une IPP évidente,
 \begin{eqnarray*}
 \dint{y}{1} f(x) \ dx & = & \dint{y}{1} x^{2} \ln x \ dx = \left[ \
\frac{
 x^{3} }{ 3 } \ln x \right]_{y}{1} - \dint{y}{1} \frac{ x^{3} }{ 3 }
 \times \frac{ 1 }{ x } \ dx \\
 & = & \frac{ 1 }{ 3 } \ln 1 - \frac{ y^{3} \ln y }{ 3 } - \frac{ 1
 }{ 3 } \dint{y}{1} x^{2} \ dx = - \frac{ y^{3} \ln y }{ 3 } - \frac{ 1
 }{ 3 } \left( \frac{ 1 }{ 3 } - \frac{ y^{3} }{ 3 } \right) \\
 & \xrightarrow[ y \rightarrow 0 ]{} & 0 - \frac{ 1 }{ 3 } \left(
 \frac{ 1 }{ 3 } - 0 \right) = - \frac{ 1 }{ 9 } 
 \end{eqnarray*} 
 donc :
 
\[
 \dlim{ n \rightarrow + \infty } \frac{ S_{n} }{ n^{3} } = \dint{0}{1}
 f(x) \ dx = - \frac{ 1 }{ 9 }.
\]
 
 \item On fait apparaître la somme précédente :
 \begin{eqnarray*}
 \frac{1}{n^{3}} \Sum{k = 1}{n} k^{2} \ln \left( \frac{ k + 1 }{ n }
 \right) & = & \frac{1}{n^{3}} \Sum{k = 1}{n} k^{2} \ln \left( \frac{ k
 \left( 1 + \frac{ 1 }{ k } \right) }{ n } \right) = 
 \frac{1}{n^{3}} \Sum{k = 1}{n} k^{2} \left[ \ln \left( \frac{ k }{ n
 } \right) + \ln \left( 1 + \frac{ 1 }{ k } \right) \right] \\
 & = & \frac{1}{n^{3}} \Sum{k = 1}{n} k^{2} \ln \left( \frac{ k }{ n }
 \right) + \frac{1}{n^{3}} \Sum{k = 1}{n} k^{2} \ln \left( 1 + \frac{
 1 }{ k } \right) 
 \end{eqnarray*} 
 La première somme a pour limite $- \frac{1}{9}$, la deuxième n'est
 plus une somme de Riemann, on ne peut plus la calculer. On va
 chercher sa limite par encadrement. \\
 Comme $ 1 + \frac{ 1 }{ k } \geq 1$, cette somme est à termes
 positifs donc elle est positive. Pour la minorer on minore le
 $\ln$, avec l'inégalité classique :
 
\[
 \ln ( 1 + x ) \leq x \ \text{ sur } ]-1 ; + \infty[ 
\]
 (par étude de la fonction $g(x) = \ln (1 + x) - x$, ou par
 concavité de $\ln$ et en remarquant que $y = x$ est la tangente au
 point $a = 1$ de $ h(x) = \ln (1 + x)$). On obtient alors :
 
\[
 0 \leq \frac{1}{n^{3}} \Sum{k = 1}{n} k^{2} \ln \left( 1 + \frac{ 1 }{
k
 } \right) \leq \frac{1}{n^{3}} \Sum{k = 1}{n} k^{2} \times \frac{ 1 }{
 k } = \frac{ 1 }{ n^{3} } \Sum{k = 1}{n} k = \frac{ n (n + 1) }{ 2
n^{3}
 } = \frac{ 1 + \frac{ 1 }{ n } }{ n } \xrightarrow[ n \rightarrow
 + \infty ]{} 0.
\]
 Par théorème d'encadrement, cette deuxième somme tend vers 0 puis
 la somme de départ tend vers $- \frac{ 1 }{ 9 }$.\\
 Deuxième méthode pour faire apparaître $S_{n}$ : changement
 d'indice : on pose $k' = k + 1$ et on obtient :
 \begin{eqnarray*}
 \frac{1}{n^{3}} \Sum{k = 1}{n} k^{2} \ln \left( \frac{ k + 1 }{ n }
 \right) & = & \frac{1}{n^{3}} \Sum{k = 2}{n + 1} (k-1)^{2} \ln \left(
 \frac{ k }{ n } \right) = \frac{1}{n^{3}} \Sum{k = 2}{n + 1} (k^{2} -
 2 k + 1 ) \ln \left( \frac{ k }{ n } \right) \\
 & = & \frac{1}{n^{3}} \Sum{k = 2}{n + 1} k^{2} \ln \left( \frac{ k }{
n
 } \right) - 2 \frac{1}{n^{3}} \Sum{k = 2}{n + 1} k \ln \left( \frac{
 k }{ n } \right) + \frac{1}{n^{3}} \Sum{k = 2}{n + 1} \ln \left(
 \frac{ k }{ n } \right) \\
 & = & \frac{1}{n} \Sum{k = 2}{n + 1} \frac{ k^{2} }{ n^{2} } \ln
\left(
 \frac{ k }{ n } \right) - \frac{ 2 }{ n } \times \frac{ 1 }{
 n } \Sum{k = 2}{n + 1} \frac{ k }{ n } \ln \left( \frac{ k }{ n }
 \right) + \frac{1}{n^{2}} \times \frac{ 1 }{ n } \Sum{k = 2}{n + 1}
 \ln \left( \frac{ k }{ n } \right) 
 \end{eqnarray*}
 La première somme de Riemann tend à nouveau vers $- \frac{ 1 }{
 9}$ avec les sommes de Riemann. La deuxième tend vers $\dint{0}{1}
 g(x) \ dx $ en posant $g(x) = x \ln x$ qui se prolonge à nouveau
 par continuité en 0, donc cette intégrale est une constante finie
 et le deuxième terme tend vers 0 (on multiplie par une quantité
 qui tend vers 0).\\
 Enfin le troisième terme ne peut pas s'écrire comme une somme de
 Riemann (car $h(x) = \ln x$ ne se prolonge pas par continuité en
 0). Par contre on peut encadrer ce terme entre $ \frac{ 1 }{ n^{3} }
 \ln \left( 1 + \frac{ 1 }{ n } \right)$ (majorant, avec tous les
 autres termes négatifs) qui tend vers 0 (DL) et avec $k \leq n$ :
 
\[
 \frac{ 1 }{ n^{2}} \Sum{k = 2}{n} \frac{ k }{ n } \ln \left( \frac{ k
 }{ n } \right) + \frac{ 1 }{ n^{3} } \ln \left( 1 + \frac{ 1 }{ n
 } \right)
\]
 (minorant, qui tend aussi vers 0 pour les mêmes raisons que le
 second terme) et conclure par encadrement.
 \end{noliste}
\end{exercice}


\newpage


\begin{exercice}{\it (Exercice avec préparation)}~
 \begin{noliste}{1.}
 \setlength{\itemsep}{4mm}
 \item 
 \begin{noliste}{}
 \item Soit $(\Omega, \A, \Prob)$ un espace probabilisé.
 \item Soient $X$ et $Y$ deux \var discrètes.
 \end{noliste}
 \begin{noliste}{$\sbullet$}
 \item Les {\bf \var $X$ et $Y$ sont indépendantes} (pour la
 probabilité $\Prob$) si :
 
\[
 \forall x \in X(\Omega), \forall y \in Y(\Omega), \ \Prob(\Ev{X
 = x} \, \cap \, \Ev{Y = y}) = \Prob\left(\Ev{\Ev{X = x}}\right) \times
 \Prob\left(\Ev{\Ev{Y = y}}\right)
\]
 \item Autrement dit, les \var $X$ et $Y$ sont indépendantes si pour
 tout $x \in X(\Omega)$ et tout $y \in Y(\Omega)$, les événements
 $\Ev{X = x}$ et $\Ev{Y = y}$ sont indépendants.
 \end{noliste}
 
 \item 
 \begin{noliste}{a)}
 \setlength{\itemsep}{2mm}
 \item $X$ représente les nombres de succès (obtention de 1) lors
 de la succession de $n$ épreuves de Bernoulli (lancers du dé)
 indépendantes et de même paramètre $p$. Ainsi : $X \suit
 \Bin{n}{p}$.\\
 En considérant l'obtention de $2$ comme succès, on obtient : $Y
 \suit \Bin{n}{q}$.
 
 \item D'après la question précédente, $X(\Omega) = Y (\Omega) = 
 \llb 0, n \rrb$.\\
 Soit $(k, j) \in \llb 0, n \rrb^{2}$. On procède alors par
 disjonciton de cas.
 \begin{noliste}{$\sbullet$}
 \item \dashuline{Si $k + j > n$} alors $\Ev{X = k} \cap \Ev{Y = j} = 
 \varnothing$.\\
 En effet, on ne peut obtenir plus de résultats que de
 lancers. Ainsi :
 
\[
 \Prob(\Ev{X = k} \cap \Ev{Y = j}) = 0
\]
 
 \item \dashuline{Si $0 \leq k + j \leq n$} alors un tirage qui
 réalise $\Ev{X = k} \cap \Ev{Y = j}$ contient exactement $k$ fois
 le nombre 1, $j$ fois le nombre 2 et donc $n-k-j$ fois le
 nombre 3.\\[.2cm]
 Ainsi, chaque tirage réalisant $\Ev{X = k} \cap \Ev{Y = j}$ est
 entièrement déterminé par :
 \begin{noliste}{$\stimes$}
 \item la position des $k$ nombres 1 : $\dbinom{n}{k}$ possibilités.
 \item la position des $j$ nombres 2 dans les places restantes :
 $\dbinom{n-k}{j}$ possibilités.
 \item la position des $n-k-j$ nombres 3 dans les places
 restantes : $\dbinom{n-k-j}{n-k-j} = 1$ possibilité.
 \end{noliste}
 Il y a donc : 
 
\[
 \dbinom{n}{k} \ \dbinom{n-k}{j} = \dbinom{n}{k} \binom{n-k}{j}
 \times 1 = \dfrac{n!}{k! \ \bcancel{(n-k)!}} \
 \dfrac{\bcancel{(n-k)!}}{j!(n-k-j)!} = \dfrac{n!}{k! \ j! \
 (n-k-j)!}
\]
 tels tirages.\\[.2cm]
 De plus, la probabilité d'un tirage ne dépend que du nombre de
 1, 2 et 3 qu'il contient. Ainsi, les tirages qui réalisent
 $\Ev{X = k} \cap \Ev{Y = j}$ ont tous la même probabilité
 d'apparaître, qui est la probabilité d'apparition du tirage :
 
\[
\begin{array}{cccccccccccc}
 1 & \ldots & \ldots & \ldots & 1 & 2 & \quad \ldots \quad & 2 & 3 & 
 \quad \ldots \quad & \quad \ldots \quad & 3 \\[-.4cm] 
 \multicolumn{5}{c}{\bbacc{3cm}} & 
 \multicolumn{3}{c}{\bbacc{2cm}} & 
 \multicolumn{4}{c}{\bbacc{3.2cm}} \\[.4cm]
 \multicolumn{5}{c}{\text{$k$ occurrences}} & 
 \multicolumn{3}{c}{\text{$j$ occurrences}} & 
 \multicolumn{4}{c}{\text{$n-k-j$ occurrences}}
\end{array}
\]
 Ainsi, la probabilité d'apparition d'un tel tirage est : $p^{k}
 \ q^{j} \ (1-p-q)^{n-k-j}$.\\
 On en déduit que :
 
\[
 \Prob(\Ev{X = k} \cap \Ev{Y = j}) = \dfrac{n!}{k! \ j! \ (n-k-j)!}
 \ p^{k} \ q^{j} \ (1-p-q)^{n-k-j}
\]
 \end{noliste}
 

 \newpage


 \item On remarque tout d'abord que $\Ev{X = n} \cap \Ev{Y = n} = 
 \varnothing$ car on ne peut obtenir à la fois $n$ fois le nombre
 1 et $n$ fois le nombre 2 en seulement $n$ lancers. Ainsi :
\[
 \Prob(\Ev{X = n} \cap \Ev{Y = n}) = 0 \neq \Prob\left(\Ev{\Ev{X =
n}}\right) \times
 \Prob\left(\Ev{\Ev{Y = n}}\right)
\]
 et $X$ et $Y$ ne sont donc pas indépendantes. 

 \item La \var $T_{n}$ admet une espérance (resp. une variance) car
 la \var $X$ en admet une.\\
 De plus, par linéarité de l'espérance et propriété de la
 variance :
\[
 \E_{p}(T_{n}) = \dfrac{1}{n + 1} \ \E(X) = \dfrac{n p}{n + 1} \qquad
 \text{ et } \qquad \V_{p}(T_{n}) = \dfrac{1}{(n + 1)^{2}} \ \V(X) = 
 \dfrac{n p (1-p)}{(n + 1)^{2}}
\]
 Comme $T_{n}$ admet une espérance, $T_{n}$ admet un biais :
\[
\begin{array}{rcl}
 b_{p}(T_{n}) & = & \E_{p}(T_{n} - p) \ = \ \E_{p}(T_{n}) - p \\[.2cm]
 & = & \dfrac{n p}{n + 1} - p \ = \ \left( \dfrac{n}{n + 1} - 1
 \right) \ p \ = \ \dfrac{n - (n + 1)}{n + 1} \ p \ = \ -
 \dfrac{p}{n + 1}
\end{array}
\]
 Comme $T_{n}$ admet une variance, $T_{n}$ admet un risque
 quadratique. Par la formule de décomposition biais variance, on
 obtient : 
\[
\begin{array}{rcl}
 r_{p}(T_{n}) & = & \V_{p}(T_{n}) + (b_{p}(T_{n}))^{2} %\\[.2cm]
 \ = \ \dfrac{n p (1-p)}{(n + 1)^{2}} + \left(
 -\dfrac{p}{n + 1}\right)^{2} \\[.4cm]
 & = & \dfrac{n p (1-p)}{(n + 1)^{2}} + \dfrac{p^{2}}{(n + 1)^{2}} \ =
\
 \dfrac{p \ (n(1-p) + p)}{(n + 1)^{2}} 
\end{array}
\]
 \end{noliste}

 \item 
 \begin{noliste}{a)}
 \setlength{\itemsep}{2mm}
 \item Pour tout $n \in \N$, si $N = n$ alors la \var $X$ prend
 ses valeurs dans $\llb 0, n \rrb$. On en déduit que : 
\[
 X(\Omega) = \dcup{n = 0}{+ \infty} \llb 0, n \rrb = \N
\]
 La famille $(\Ev{N = n})_{n \in \N}$ forme un système complet
 d'événements.\\
 On en déduit, par la formule des probabilités totales que, pour
 tout $k \in \N$ :
 \[
 \begin{array}{rcl@{\quad}>{\it}R{5cm}}
 \Prob\left(\Ev{\Ev{X = k}}\right) & = & \Sum{n = 0}{+ \infty}
\Prob(\Ev{N = n} \cap
 \Ev{X = k}) \\[.4cm]
 & = & \Sum{n = k}{+ \infty} \Prob(\Ev{N = n} \cap
 \Ev{X = k}) & (car si $k > n$ alors \\
$\Ev{N = n} \cap \Ev{X = k}
 = \varnothing$) \nl
 \\[-.2cm]
 & = & \Sum{n = k}{+ \infty} \Prob\left(\Ev{\Ev{N = n}}\right) \
 \Prob_{\Ev{N = n}}(\Ev{X = k}) & (car $\Prob\left(\Ev{\Ev{N =
n}}\right) \neq 0$)
 \nl
 \\[-.2cm]
 & = & \Sum{n = k}{+ \infty} \dfrac{\lambda^{n} \ \ee^{-
 \lambda}}{\bcancel{n!}} \times \dfrac{\bcancel{n!}}{k!
 (n-k)!} \ p^{k} (1-p)^{n-k} 
 \\[.6cm]
 & = & \dfrac{p^{k} \ \ee^{-\lambda}}{k!} \ \Sum{n = k}{+ \infty}
 \dfrac{\lambda^{n}}{(n-k)!} (1-p)^{n-k} 
 \\[.6cm]
 & = & \dfrac{p^{k} \ \ee^{-\lambda}}{k!} \ \Sum{n = 0}{+ \infty}
 \dfrac{\lambda^{n + k}}{n!} (1-p)^{n} \ = \ \dfrac{\lambda^{k} \ p^{k}
\
 \ee^{-\lambda}}{k!} \ \Sum{n = 0}{+ \infty}
 \dfrac{\lambda^{n}}{n!} (1-p)^{n} & (par décalage d'indice) 
 \nl
 \\[-.2cm] 
 & = & \dfrac{\lambda^{k} \ p^{k} \ \ee^{ - \lambda }}{k!} \ \ee^{
 \lambda (1-p) } \ = \ \dfrac{(\lambda p)^{k} \ \ee^{- \lambda + 
 \lambda - \lambda p}}{k!} \ = \ \dfrac{(\lambda p)^{k} \ 
 \ee^{- \lambda p}}{k!}
\end{array}
\]
 On en déduit que : $X \suit \Pois{\lambda p}$.\\
 En raisonnant de même pour $Y$, on obtient : $Y \suit
 \Pois{\lambda p}$.

 \item Soit $(k, j) \in \N^{2}$. La famille $(\Ev{N = n})_{n \in \N}$
 forme un système complet d'événements.\\
 On en déduit, par la formule des probabilités totales que :
\[
 \begin{array}{cl@{\quad}>{\it}R{5cm}}
 & \Prob(\Ev{X = k} \cap \Ev{Y = j}) \\[.2cm]
 = & \Sum{n = 0}{+ \infty} \Prob(\Ev{N = n} \cap \Ev{X = k} \cap \Ev{Y
= j}) 
 \\[.4cm]
 = & \Sum{n = j + k}{+ \infty} \Prob(\Ev{N = n} \cap \Ev{X = k} \cap
 \Ev{Y = j}) & (car si $j + k > n$ alors \\
$\Ev{N = n} \cap \Ev{X = k} \cap
 \Ev{Y = j}$) 
 \nl
 \\[-.2cm]
 = & \Sum{n = j + k}{+ \infty} \Prob\left(\Ev{\Ev{N = n}}\right) \
 \Prob_{\Ev{N = n}}(\Ev{X = k} \cap \Ev{Y = j}) & (car
 $\Prob\left(\Ev{\Ev{N = n}}\right) \neq 0$) 
 \nl
 \\[-.2cm]
 = & \Sum{n = j + k}{+ \infty} \dfrac{\lambda^{n} \ \ee^{-
 \lambda}}{\bcancel{n!}} \times \dfrac{\bcancel{n!}}{j! \
 k! \ (n-k-j)!} \ p^{k} q^{j} (1-p-q)^{n-j-k} & (d'après la
 question \itbf{2.b)})
 \nl
 \\[-.2cm]
 = & \dfrac{p^{k} \ q^{j} \ \ee^{ - \lambda}}{j! \ k!} \
 \Sum{n = j + k}{+ \infty} \lambda^{n} \
\dfrac{(1-p-q)^{n-j-k}}{(n-k-j)!} 
 \\[.6cm]
 = & \dfrac{p^{k} \ q^{j} \ \ee^{ - \lambda}}{j! \ k!} \
 \Sum{n = 0}{+ \infty} \lambda^{n + j + k} \ \dfrac{(1-p-q)^{n}}{n!}
 & (par décalage d'indice) \nl
 \\[-.2cm]
 = & \dfrac{\lambda^{j + k} \ p^{k} \ q^{j} \ \ee^{-\lambda}}{j! \ k!}
\
 \Sum{n = 0}{+ \infty} \lambda^{n} \ \dfrac{(1-p-q)^{n}}{n!}
 \\[.6cm]
 = & \dfrac{\lambda^{j + k} \ p^{k} \ q^{j} \ \ee^{- \lambda}}{j! \
 k!} \ \ee^{\lambda (1-p-q)} \ = \ \dfrac{\lambda^{j + k} \
 p^{k} \ q^{j} \ \bcancel{\ee^{- \lambda}}}{j! \ k!} \
 \bcancel{\ee^{\lambda}} \ \ee^{-\lambda p} \ \ee^{- \lambda q}
 \\[.6cm]
 = & \dfrac{(\lambda p)^{k} \ \ee^{- \lambda p}}{k!} \times
 \dfrac{(\lambda q)^{j} \ \ee^{- \lambda q}}{j!} 
 \\[.6cm]
 = & \Prob\left(\Ev{\Ev{X = k}}\right) \ \Prob\left(\Ev{\Ev{Y =
j}}\right)
\end{array}
\]
 On en conclut que $X$ et $Y$ sont indépendantes.

 \item Sous réserve de convergence, le théorème de transfert
 permet d'affirmer :
\[
 \begin{array}{rcl@{\quad}>{\it}R{5cm}}
 \E\left( \dfrac{X}{N + 1} \right) & = & \Sum{(n, k) \in
 N(\Omega) \times X(\Omega)}{} \dfrac{k}{n + 1} \ \Prob(\Ev{N
 = n} \cap \Ev{X = k})
 \\[.6cm]
 & = & \Sum{n = 0}{+ \infty} \Sum{k = 0}{n} \dfrac{k}{n + 1}
 \dfrac{\lambda^{n} \ \ee^{ - \lambda}}{n!} \times \dbinom{n}{k}
 p^{k} \ (1-p)^{n-k} 
 \\[.6cm] 
 & = & \Sum{n = 0}{+ \infty} \dfrac{\lambda^{n} \ \ee^{- \lambda}}{
 (n + 1)!} \ \Sum{k = 0}{n} k \dbinom{n}{k} p^{k} q^{n-k} 
 \\[.6cm] 
 & = & \Sum{n = 0}{+ \infty} \dfrac{\lambda^{n} \ \ee^{- \lambda}}{
 (n + 1)!} \ \E(W) & (si $W$ est une \var \\
telle que $T \suit
 \Bin{n}{p}$) 
 \nl
 \\[-.2cm] 
 & = & \Sum{n = 0}{+ \infty} \dfrac{\lambda^{n} \ \ee^{- \lambda}}{
 (n + 1)!} \ n p 
 \\[.6cm]
 & = & p \ \ee^{-\lambda} \ \Sum{n = 0}{+ \infty}
 \dfrac{\lambda^{n}}{(n + 1)!} \ n 
\end{array}
\]


\newpage


\noindent
Enfin :
\[
 \dfrac{n}{(n + 1)!} \ = \ \dfrac{(n + 1 - 1)}{(n + 1)!} \ = \
 \dfrac{n + 1}{(n + 1)!} - \dfrac{1}{(n + 1)!} = \dfrac{1}{n!} -
 \dfrac{1}{(n + 1)!}
\]
Et ainsi :
\[
\begin{array}{rcl}
 \E\left( \dfrac{X}{N + 1} \right) & = & p \ \ee^{ - \lambda}
 \ \left( \Sum{n = 0}{+ \infty} \dfrac{\lambda^{n}}{n!} -
 \Sum{n = 0}{+ \infty} \dfrac{\lambda^{n}}{(n + 1)!} \right) 
 \\[.6cm]
 & = & p \ \ee^{- \lambda} \left( \ee^{\lambda} -
 \Sum{n = 1}{+ \infty} \dfrac{\lambda^{n-1}}{n!} \right) \ = \ 
 p \ \ee^{- \lambda} \left( \ee^{\lambda} - \dfrac{1}{\lambda}
 \left( \Sum{n = 0}{+ \infty} \dfrac{\lambda^{n}}{n!} - 1
 \right) \right) 
 \\[.6cm]
 & = & p \ \ee^{- \lambda} \left(\ee^{\lambda} -
 \dfrac{\ee^{\lambda} - 1}{\lambda} \right) = p \ \left( 1 - 
 \dfrac{1 - \ee^{ - \lambda}}{\lambda} \right) 
\end{array}
\]
 La réserve est ainsi levée et $\E\left( \dfrac{X}{n + 1}
 \right) = p \ \left( 1 - \dfrac{1 - \ee^{ - \lambda}}{\lambda}
 \right)$.\\[.2cm]
 Comme la \var $T$ admet une espérance, elle admet aussi un
 biais :
 
\[
\begin{array}{rcl}
 b_{p}(T) & = & \E_{p}(T - p) \ = \ \E_{p}(T_{n}) - p \\[.2cm]
 & = & p \ \left(1 - \dfrac{1 - \ee^{- \lambda}}{\lambda}
 \right) - p \ = \ p \dfrac{\ee^{- \lambda} - 1}{\lambda} 
 \ \neq \ 0
\end{array}
\]
 donc $T$ n'est pas un estimateur sans biais de $p$.
 \end{noliste}
 \begin{remark}~
 \begin{noliste}{$\sbullet$}
 \item La question précédente était 
 
 \item 
 Question extrêmement difficile. On peut répondre très
 rapidement dans le cas $p = q = \frac{1}{3}$ par une astuce : en
 posant $Z$ le nombre de 3 obtenus, on a :
 
\[
 X + Y + Z = N
\]
 
 donc par linéarité :
 
\[
 E \left( \frac{ X + Y + Z }{ N + 1 } \right) = E \left(
 \frac{ X }{ N + 1 } \right) + E \left( \frac{ Y }{ N + 1 }
 \right) + E \left( \frac{ Z }{ N + 1 } \right) = E \left(
 \frac{ N }{ N + 1 } \right) = 1 - E \left( \frac{ 1 }{ N + 
 1 } \right).
\]
 
 Or dans ce cas, $X$, $Y$ et $Z$ suivent la même loi de
 Poisson de paramètre $\frac{ \lambda }{ 3 }$ donc on en
 déduit que :
 
\[
 E \left( \frac{ X }{ N + 1 } \right) = \frac{ 1 }{ 3 } - E
 \left( \frac{ 1 }{ 3 ( N + 1 ) } \right)
\]
 
 et avec $N + 1 > 0$, on obtient $\E\left( \frac{ 1 }{ 3 (N + 1)
 } \right) > 0 $ donc $\E\left( \frac{ X }{ N + 1 } \right)
 \neq \frac{ 1 }{ 3 }$ : on en déduit que $\E\left( \frac{ X
 }{ N + 1 } \right) $ ne peut être égale à $p$ pour tout $p$,
 donc que $\frac{ X }{ N + 1 }$ n'est pas un estimateur sans
 biais de $p$.
 \end{noliste}
 \end{remark}
 \end{noliste}
 \end{exercice}
 

\newpage


\addtocounter{exercice}{-1}
\begin{exercice}{\it (Exercice sans préparation)}~\\
 Soit $A$ une matrice carrée de $\mathcal{M}_{3} ( \R ) $.
 \begin{noliste}{1.}
 \setlength{\itemsep}{4mm}
 \item On suppose que $A$ est diagonalisable, alors il existe $P$
 inversible et $D$ diagonale telles que :
 \[
 A = P D P^{-1} 
 \]
 On obtient alors sans difficulté :
 
\[
 A^{3} = P D P^{-1} P D P^{-1} P D P^{-1} = P D I D I D P^{-1} = P
D^{3} P^{-1 } 
\]
 
 avec $P$ inversible et $D^{3}$ diagonale, comme puissance d'une
 matrice diagonale, donc $A^{3}$ est bien diagonalisable.

 \item On suppose dans cette question que $A = \begin{smatrix}
0 & 0
 & 1 \\
1 & 0 & 0 \\
0 & 1 & 0
 \\
\end{smatrix}
$. 
 \begin{noliste}{a)}
 \setlength{\itemsep}{2mm}
 \item On obtient sans difficulté :
 
\[
 A^{3} = I
\]
 
 \item On en déduit que le polynôme $X^{3} - 1$ est annulateur de
 $A$, et on a :
 
\[
 X^{3} - 1 = 0 \Longleftrightarrow X^{3} = 1 \Longleftrightarrow X = 
 1
\]
 
 donc 1 est la seule valeur propre possible de $A$. \\

 Si $A$ était diagonalisable, 1 serait forcément valeur propre,
 et il existerait $P$ inversible et $D$ diagonale telle que $ A = 
 P D P^{-1}$.

 Enfin $D$ ne peut comporter que des valeurs propres de $A$ sur
 sa diagonale, donc $D = I$ et enfin :
 
\[
 A = P I P^{-1} = I 
\]
 
 qui est absurde, donc $A$ n'est pas diagonalisable.
 
 \end{noliste}
 \end{noliste}
\end{exercice}


 \newpage


 \begin{exercice}{\it (Exercice avec préparation)}~
 \begin{noliste}{1.}
 \setlength{\itemsep}{4mm}
 \item On peut écrire, avec $\varphi$ la densité de la loi normale
centrée réduite : 
 
\[
 P \left(\Ev{ X \in [a;b] }\right) = P \left(\Ev{ a \leq X \leq b
}\right) = \dint{a}{b} \varphi (t) \ dt = \frac{ 1 }{ \sqrt{ 2 \pi } }
\dint{a}{b} e^{ - \frac{ t^{2} }{ 2 } } \ dt. 
\]

 Cette probabilité apparaît comme limite dans le théorème central
limite : si $X_{n}$ est une suite de variables aléatoires indépendantes
et identiquement distribuées, admettant une espérance $m$ et une
variance $\sigma^{2}$ commune, alors la somme centrée réduite des
$X_{i}$ converge en loi vers la loi normale centrée réduite, donc : 
 
\[
 P \left(\Ev{ \frac{ \Sum{i = 1}{n} X_{i} - n m }{ \sigma \sqrt{n} }
\in [a ; b ]}\right) \xrightarrow[ n \rightarrow + \infty ]{} \frac{ 1
}{ \sqrt{ 2 \pi } } \dint{a}{b} e^{ - \frac{ t^{2} }{ 2 } } \ dt. 
\]

 Soit $X$ une variable aléatoire définie sur un espace probabilisé
$(\Omega, \mathcal{A}, P)$ suivant la loi normale centrée réduite. On
note $\Phi$ la fonction de répartition de de $X$. On pose $Y = | X |$
(valeur absolue de $X$).

 \item \begin{noliste}{a)}
 \setlength{\itemsep}{2mm}

 \item Par théorème de transfert, on s'intéresse à l'intégrale : 
 
\[
 \dint{-\infty}{+ \infty} | t | \varphi (t) \ dt = 2 \dint{0}{+ \infty}
| t | \varphi (t) \ dt = \frac{ 2 }{ \sqrt{ 2 \pi } } \dint{0}{+
\infty} t e^{ - \frac{ t^{2} }{ 2 } } \ dt 
\]

 par parité de $\varphi$ et de la valeur absolue. On passe alors par
l'intégrale partielle : 
 
\[
 \dint{0}{A} t e^{ - \frac{ t^{2} }{ 2 } } \ dt = \left[ - e^{ - \frac{
t^{2} }{ 2 } } \right]_{0}{A} = 1 - e^{ - \frac{ A^{2} }{ 2 } }
\xrightarrow[ A \rightarrow + \infty ]{} 1. 
\]

 Cette intégrale est donc convergente, et comme la fonction intégrée
est positive, elle converge absolument. On en déduit que $Y$ admet une
espérance et que : 
 
\[
 E ( Y ) = \frac{ 2 }{ \sqrt{ 2 \pi } } \times 1 = \sqrt{ \frac{ 2 }{
\pi } }. 
\]

 Pour la variance, on s'intéresse au moment d'ordre 2 de $Y$, donc à : 
 
\[
 E ( Y^{2} ) = E [ \ \vert X |^{2} ] = E ( X^{2} ) = V (X) + [ \ \E(X)
]^{2} = 1 + 0^{2}. 
\]

 On en déduit que $Y$ admet un moment d'ordre deux et donc une variance
(car $x$ admet un moment d'ordre 2) et que : 
 
\[
 V ( Y ) = 1 - \left( \sqrt{ \frac{ 2 }{ \pi } } \right)^{2} = 1 -
\frac{ 2 }{ \pi } = \frac{ \pi - 2 }{ \pi }. 
\]

 \item De nouveau avec le théorème de transfert, avec $X Y = X | X |$,
on s'intéresse à l'intégrale :
 
\[
 \dint{-\infty}{+ \infty} t \times | t | \varphi (t) \ dt 
\]

 La fonction intégrée est clairement impaire : si l'intégrale est
convergente, alors elle vaudra 0. De plus elle converge si et seulement
si l'intégrale suivante converge :
 
\[
 \dint{0}{ + \infty } t | t | \varphi (t) \ dt = \dint{0}{+ \infty}
t^{2} \varphi (t) \ dt 
\]

 qui converge puisque la loi normale centrée réduite admet un moment
d'ordre deux. On en déduit que $X Y$ admet une espérance et que : 
 
\[
 E ( X Y ) = 0. 
\]

 \end{noliste}

 \item On pose $Z = X + Y$. \begin{noliste}{a)}
 \setlength{\itemsep}{2mm}

 \item On remarque que 
 
\[
 Z = X + | X | = \left\{\begin{array}{cl}
 X - X & \text{ si } X \leq 0 \\
X + X & \text{ si } X \geq 0 \\
\end{array}
\right. = \left\{\begin{array}{cl}
 0 & \text{ si } X \leq 0 \\
X + X & \text{ si } X \geq 0 \\
\end{array}
\right. 
\]

 donc on obtient : 
 
\[
 Z = 0 \Longleftrightarrow X \leq 0 \ \ \text{ ou } \ \ X \geq 0 \text{
et } 2 X = 0 \Longleftrightarrow X = 0 
\]

 et finalement : 
 
\[
 \Ev{Z = 0 } = \Ev{X \leq 0 } \ \ \text{ donc } \ \ P \Ev{ Z = 0 } =
\Phi ( 0 ) = \frac{ 1 }{ 2 }. 
\]

 \item D'après ce qu'on a vu précédemment, $Z$ est nulle quand $X$ est
négative, et vaut $2X$ quand $X$ est positive, donc prend toutes les
valeurs de $\R_+ $ dans ce cas. On en déduit que $Z ( \Omega ) = \R_+
$, donc :
 \begin{noliste}{$\sbullet$}

 \item $ \forall x < 0 $ (avant le support), 
 
\[
 F_{Z} ( x ) = P \Ev{ Z \leq x } = 0 
\]

 \item Pour $x = 0$,
 
\[
 F_{Z} (0) = P \Ev{ Z \leq 0 } = P \Ev{ Z < 0 } + P \Ev{ Z = 0 } = 0 +
\frac{ 1 }{ 2 } = \frac{ 1 }{ 2 }. 
\]

 \item Pour $x > 0$, en utilisant le sce $\Ev{X \leq 0}, \Ev{X > 0}$ : 
 \begin{eqnarray*}
 \Ev{ Z \leq x } & = & [ \ \Ev{X \leq 0} \cap \Ev{Z \leq x } ] \cup [ \
\Ev{X > 0} \cap \Ev{Z \leq x } ] = [ \ \Ev{X \leq 0 } \cap \Ev{ 0 \leq
x } ] \cup [ \ \Ev{X > 0 } \cap \Ev{ 2 X \leq x } ] \\
\\
 & = & [ X \leq 0 ] \cup \left[ 0 < X \leq \frac{ x }{ 2 } \right] =
\left( X \leq \frac{ x }{ 2 } \right)
 \end{eqnarray*}

 ce qui donne : 
 
\[
 F_{Z} ( x ) = \Prob\left(\Ev{\Ev{ Z \leq x }}\right) = P \left(\Ev{ X
\leq \frac{ x }{ 2 }}\right) = \Phi \left( \frac{ x }{ 2 } \right) 
\]

 \end{noliste}

 et en rassemblant : 
 
\[
 F_{Z} ( x ) = \left\{\begin{array}{cl}
 0 & \text{ si } x < 0 \\
\frac{1}{2} & \text{ si } x = 0 \\
\Phi \left( \frac{ x }{ 2 } \right) & \text{ si } x > 0 \\
\end{array}
\right. 
\]


 \item $Z$ n'est pas à densité puisqu'il existe une valeur (0) pour
laquelle la probabilité de tomber "pile" est non nulle. Elle n'est pas
discrète car son support ($\R_+ $) est un intervalle. \\

 \end{noliste}

 \item Soit $y \in \R$. \begin{noliste}{a)}
 \setlength{\itemsep}{2mm}

 \item Lorsque $y< 0$, l'évènement $\Ev{ Y \leq y} $ est impossible
donc :
 
\[
 P [ \ \Ev{X \leq 1 } \cap \Ev{Y \leq y } ] = 0. 
\]

 Lorsque $y \geq 0$, on a : 
 
\[
 [ \ \Ev{X \leq 1 } \cap \Ev{Y \leq y} ] = [ \ \Ev{ X \leq 1 } \cap ( |
x | \leq y ) ] = [ \ \Ev{ X \leq 1 } \cap ( -y \leq X \leq y ) ] = [ -y
\leq X \leq \min (1, y) ]. 
\]

 On en déduit que pour $0 \leq y \leq 1$ :
 
\[
 P [ \ \Ev{X \leq 1 } \cap \Ev{Y \leq y} ] = P [ -y \leq X \leq y ] =
\Phi ( y) - \Phi ( -y) = \Phi (y) - [ 1 - \Phi (y) ] = 2 \Phi ( y) - 1.
\]
 Enfin pour $y > 1$ on a :
 
\[
 P [ \ \Ev{X \leq 1 } \cap \Ev{Y \leq y} ] = P [ -y \leq X \leq 1 ] =
\Phi ( 1) - \Phi ( -y) = \Phi (1) - [ 1 - \Phi (y) ] = \Phi (1) + \Phi
( y) - 1. 
\]
 \item On calcule dans les 3 cas : \begin{noliste}{$\sbullet$}

 \item Si $y< 0$,
 
\[
 P \Ev{ X \leq 1 } P \Ev{ Y \leq y } = \Phi (1) \times 0 = 0 = P [ \
\Ev{X \leq 1 } \cap \Ev{Y \leq y } ] 
\]

 donc les évènements $\Ev{X \leq 1}$ et $\Ev{Y \leq y}$ dont
indépendants. \\

 \item Si $0 \leq y \leq 1$,
 
\[
 P \Ev{ X \leq 1 } P \Ev{ Y \leq y } = \Phi ( 1 ) P \left(\Ev{ - y \leq
X \leq y }\right) = \Phi (1) \times \left[ 2 \Phi (y) - 1 \right] \neq
P [ \ \Ev{X \leq 1 } \cap \Ev{Y \leq y } ] 
\]

 (sauf si $ 2 \Phi ( y) - 1 = 0$) car $\Phi (1) < 1$ ($\varphi$ n'est
pas nulle après 1, donc la probabilité d'être supérieure à 1 est non
nulle), et les évènements $\Ev{X \leq 1}$ et $\Ev{Y \leq y}$ ne sont
pas indépendants, sauf pour 
 
\[
 2 \Phi (y) - 1 = 0 \Longleftrightarrow \Phi (y) = \frac{ 1 }{ 2 }
\Longleftrightarrow y = 0. 
\]

 \item Si $y > 1$,
 
\[
 P \Ev{ X \leq 1 } P \Ev{ Y \leq y } = \Phi ( 1 ) P \left(\Ev{ - y \leq
X \leq y }\right) = \Phi (1) \times \left[ 2 \Phi (y) - 1 \right] 
\]

 donc
 \begin{eqnarray*}
 P \Ev{ X \leq 1 } P \Ev{ Y \leq y } = P [ \ \Ev{X \leq 1 } \cap \Ev{Y
\leq y } ] & \Longleftrightarrow & \Phi (1) \times \left[ 2 \Phi (y) -
1 \right] = \Phi (1) + \Phi ( y) - 1 \\
\\
 & \Longleftrightarrow & \Phi (y) \left( 2 \Phi (1) - 1 \right) = 2
\Phi (1) - 1 \Longleftrightarrow \Phi (y) = 1 
 \end{eqnarray*}

 qui est absurde car $\Phi$ est strictement croissante donc n'atteint
jamais 1, qui est sa limite en $ + \infty$.

 \end{noliste}

 Finalement les évènements $\Ev{X \leq 1}$ et $\Ev{Y \leq y}$ sont
indépendants si et seulement si $y \leq 0$.

 \end{noliste}

 \end{noliste}

%%% LALALAL

 \indent

 \noindent \textbf{\underline{Exercice sans préparation}} \\
\\
 Soit $A$ une matrice de $\mathcal{M}_{2} (\R)$ telle que $A^{3} = 0$.
 \begin{noliste}{1.}
 \setlength{\itemsep}{4mm}

 \item Question très difficile. On suppose que $A^{2} \neq 0$, alors en
posant $u$ l'application linéaire de $\R^{2}$ dont la matrice dans la
base canonique est $A$, on a $u^{2} \neq 0$ et $u^{3} = 0$. \\

 On en déduit alors qu'il existe un vecteur $x$ dans $\R^{2}$ tel que
$u^{2} (x) \neq 0$, et on étudie la liberté de la famille $(x, u (x),
u^{2} (x) )$ : on considère $a$, $b$ et $c$ tels que 
 
\[
 a x + b u(x) + c u^{2} (x) = 0 
\]

 alors en composant par $u$ et par $u^{2}$, on obtient (avec $u(0) =
u^{2} (0) = 0$ car ce sont des applications linéaires) : 
 
\[
 a u(x) + b u^{2} (x) = 0 \ \ \text{ et } \ \ a u^{2} (x) = 0. 
\]

 De plus $u^{2} (x) \neq 0$, donc on obtient $a = 0$. On en déduit
alors que : 
 
\[
 b u(x) + c u^{2} (x) = 0 \ \ \text{ et } \ \ b u^{2} (x) = 0 
\]

 On obtient de même $b = 0$, puis $c u^{2} (x) = 0$, et enfin $c = 0$,
donc la famille $(x, u(x), u^{2} (x) )$ est libre; or elle comporte
trois vecteurs dans un espace de dimension 2, elle ne peut être libre :
c'est absurde et on obtient donc : 
 
\[
 A^{2} = 0. 
\]

 \item On peut conclure à la première partie par caractérisation des
sous-espaces vectoriels. Mais comme la dimension est demandée, il
faudra de toute manière expliciter l'espace. On sépare deux cas :
\begin{noliste}{$\sbullet$}

 \item si $A = 0$, alors $A M = 0 = M A$ est vrai pour toute matrice
$M$, donc l'ensemble considéré est $\mathcal{M}_{2} ( \R )$, qui est
bien un espace vectoriel, de dimension 4. \\

 \item si $A \neq 0$, un raisonnement analogue à la question 1 montre
qu'il existe $x$ dans $\R^{2}$ tel que la famille $(x, u(x) )$ est une
base de $\R^{2}$, et dans cette base la matrice de $u$ est : 
 
\[
 N = \begin{smatrix}
0 & 0 \\
1 & 0 \\
\end{smatrix}. 
\]

 On en déduit par formule de changement de base qu'il existe une
matrice $P$ inversible (constituée des coordonnées de $x$ et $u(x)$, en
colonnes, dans la base canonique) telle que :
 
\[
 A = P N P^{-1}. 
\]

 On montre alors que l'équation recherchée se ramène à : 
 \begin{eqnarray*}
 A M = M A & \Longleftrightarrow & P N P^{-1} M = M P N P^{-1}
\Longleftrightarrow P^{-1} [ P N P^{-1} M ] P = P^{-1} [ M P N P^{-1} ]
P \\
\\
 & \Longleftrightarrow & N ( P ^{-1} M P ) = ( P^{-1} M P ) N. 
 \end{eqnarray*}

 On pose alors $P^{-1} M P = \begin{smatrix}
a & b \\
c & d \\
\end{smatrix}
$ et on obtient : 
 \begin{eqnarray*}
 A M = M A & \Longleftrightarrow & \begin{smatrix}
0 & 0 \\
1 & 0 \\
\end{smatrix}
\begin{smatrix}
a & b \\
c & d \\
\end{smatrix}
 = \begin{smatrix}
a & b \\
c & d \\
\end{smatrix}
\begin{smatrix}
0 & 0 \\
1 & 0 \\
\end{smatrix}
\Longleftrightarrow \begin{smatrix}
0 & 0 \\
a & b \\
\end{smatrix}
 = \begin{smatrix}
b & 0 \\
d & 0 \\
\end{smatrix}
\\
\\
 & \Longleftrightarrow & \left\{\begin{array}{c}
 b = 0 \\
a = d \\
\end{array}
\right. \Longleftrightarrow P^{-1} M P = \begin{smatrix}
a & 0 \\
c & a \\
\end{smatrix}
 = a I + c N \\
\\
 & \Longleftrightarrow & M = P \left(\Ev{ a I + c N }\right) P^{-1} = a
P I P^{-1} + c P N P^{-1} = a I + c A.
 \end{eqnarray*}

 On en déduit que l'ensemble cherché est $\Vect{ I, A }$ qui est un
espace vectoriel, et la famille $(I,A)$ en est génératrice. Elle est de
plus libre si et seulement si $A$ n'est pas colinéaire à $I$. Si
c'était le cas, on aurait : 
 
\[
 A = \lambda I \ \ \text{ donc } \ \ A^{3} = \lambda^{3} I = 0 \ \
\text{ donc } \ \ \lambda^{3} = 0 \ \ \text{ donc } \ \ \lambda = 0 
\]

 et enfin $A = 0 I = 0$ ce qui est absurde. Cette famille est donc
libre, c'est une base de l'espace vectoriel donc celui-ci est de
dimension 2. \\

 Remarque : on pouvait éviter ce raisonnement par l'absurde en écrivant
: 
 
\[
 a I + b A = 0 \Longleftrightarrow P^{-1} ( a P I P^{-1} + b P N P^{-1}
) P = 0 \Longleftrightarrow a I + b N = 0 
\]

 qui donne immédiatement $a = c = 0$.

 \end{noliste}

 \end{noliste}
 \end{exercice}

 \newpage

 \begin{exercice}{\it (Exercice avec préparation)}~
 \begin{noliste}{1.}
 \setlength{\itemsep}{4mm}
 \item Une suite de variables aléatoires $(X_{n})_{ n \in \N }$
converge en loi vers une variable $X$ si pour tout $x$ réel tel que
$F_{X}$ est continue en $x$, on a : 
 
\[
 \dlim{ n \rightarrow + \infty } F_{ X_{n} } (x) = F_{X} ( x). 
\]

 Cas particulier des variables discrètes (la définition précédente est
toujours valable, mais il y a une caractérisation beaucoup plus
simple). S'il existe un ensemble \textbf{discret} I tel que les
supports de tous les $X_{n}$ et de $X$ sont inclus dans I (en général
on trouve I comme "limite" des supports des $X_{n}$), $(X_{n})$
converge en loi vers $X$ si et seulement si : 
 
\[
 \forall i \in I, \ \dlim{ n \rightarrow + \infty } P \Ev{ X_{n} = i }
= P \Ev{ X = i }. 
\]

 Soit $(X_{n})_{ n \in \N }$ une suite de variables aléatoires
indépendantes définies sur un espace probabilité $(\Omega, \mathcal{A},
P)$, suivant toutes la loi de Bernouilli de paramètre $\frac{1}{2}$. \\

 On définit la suite de variables aléatoires $(Z_{n})_{ n \in \N }$ par
les relations : 
 
\[
 Z_{0} = \frac{ X_{0} }{ 2 } \ \text{ et } \ \forall n \in \N^* \, \
Z_{n} = \frac{ Z_{n-1} + X_{n} }{ 2 }. 
\]

 \item \begin{noliste}{a)}
 \setlength{\itemsep}{2mm}

 \item On itère la relation de récurrence (on ne peut procéder par
récurrence puisque le résultat n'est pas donné) : 
 
\[
 Z_{n} = \frac{ Z_{n-1} }{ 2 } + \frac{ X_{n} }{ 2 } = \frac{ \frac{
Z_{n-2} }{ 2 } + \frac{ X_{n-1} }{ 2 } }{ 2 } + \frac{ X_{n} }{ 2 } =
\frac{ Z_{n-2} }{ 4 } + \frac{ X_{n-1} }{ 4 } + \frac{ X_{n} }{ 2 }. 
\]

 A l'ordre suivant on obtient alors : 
 
\[
 Z_{n} = \frac{ Z_{n-3} }{ 8 } + \frac{ X_{n-2} }{ 8 } + \frac{ X_{n-1}
}{ 4 } + \frac{ X_{n} }{ 2 } = \frac{ Z_{n-3} }{ 2^{3} } + \frac{
X_{n-2} }{ 2^{3} } + \frac{ X_{n-1} }{ 2^{2} } + \frac{ X_{n} }{ 2 }. 
\]

 En itérant $k$ fois on obtiendra alors :
 
\[
 Z_{n} = \frac{ Z_{n-k} }{ 2^{k} } + \frac{ X_{n- (k-1)} }{ 2^{k} } +
\frac{ X_{n- (k-2) } }{ 2^{k-1} } + \dots + \frac{ X_{n} }{ 2 }. 
\]

 Enfin lorsqu'on arrive à $k = n$ on obtiendra : 
 
\[
 Z_{n} = \frac{ Z_{0} }{ 2^{n} } + \frac{ X_{1} }{ 2^{n} } + \frac{
X_{2} }{ 2^{n-1 } } + \dots + \frac{ X_{n} }{ 2 } = \frac{ X_{0} }{
2^{n + 1} } + \frac{ X_{1} }{ 2^{n} } + \dots + \frac{ X_{n-1} }{ 2^{2}
} + \frac{ X_{n} }{ 2 }. 
\]

 Enfin cette somme se réécrit avec le symbole $\sum$ : en posant $i$
l'indice sur les variables $X$ on remarque que : 
 
\[
 Z_{n} = \Sum{ i = 0 }{n} \frac{ X_{i} }{ 2^{ n + 1 - i } } = \Sum{ i =
0 }{n} \left( \frac{ 1 }{ 2 } \right)^{ n + 1 - i } X_{i}. 
\]

 \item On en déduit que
 
\[
 Z_{n-1} = \Sum{i = 0}{n-1} \left( \frac{ 1 }{ 2 } \right)^{ n - i }
X_{i} 
\]

 ne dépend que des variables $X_{0}, \dots, X_{n-1}$. De plus les
$(X_{i})$ sont indépendantes, donc par lemme des coalitions, $Z_{n-1}$
et $X_{n}$ sont indépendantes. \\

 \item Par linéarité de l'espérance d'une part, indépendance des
$X_{i}$ et quadracité de la variance d'autre part, $Z_{n}$ admet une
espérance et une variance et : 
 \begin{eqnarray*}
 E ( Z_{n} ) & = & \Sum{i = 0}{n} \left( \frac{ 1 }{ 2 } \right)^{ n +
1 - i } E ( X_{i} ) = \Sum{i = 0}{n} \left( \frac{ 1 }{ 2 } \right)^{ n
+ 1 - i } \times \frac{1}{2} = \Sum{i = 0}{n} \left( \frac{ 1 }{ 2 }
\right)^{ n + 2 - i } \\
\\
 & = & \left( \frac{ 1 }{ 2 } \right)^{n + 2} \Sum{i = 0}{n} 2^{i} =
\left( \frac{ 1 }{ 2 } \right)^{ n + 2 } \times \frac{ 1 - 2^{n + 1} }{
1 - 2 } = \left( \frac{ 1 }{ 2 } \right)^{ n + 2 } (2^{n + 1} - 1 ) \\
\\
 & = & \frac{ 1 }{ 2 } - \left( \frac{ 1 }{ 2 } \right)^{ n + 2 } =
\frac{ 1 }{ 2 } \left( 1 - \left( \frac{ 1 }{ 2 } \right)^{n + 1}
\right). 
 \end{eqnarray*}

 et
 \begin{eqnarray*}
 V ( Z_{n} ) & = & \Sum{i = 0}{n} V \left[ \ \left( \frac{ 1 }{ 2 }
\right)^{ n + 1 - i } X_{i} \right] = \Sum{i = 0}{n} \left( \frac{ 1 }{
2 } \right)^{ 2n + 2 -2 i } V ( X_{i} ) = \Sum{i = 0}{n} \left( \frac{
1 }{ 2 } \right)^{ 2n + 2 -2 i } \times \frac{1}{4} = \Sum{i = 0}{n}
\left( \frac{ 1 }{ 2 } \right)^{ 2n + 4 -2 i } \\
\\
 & = & \left( \frac{ 1 }{ 2 } \right)^{2n + 4} \Sum{i = 0}{n} 4^{i} =
\left( \frac{ 1 }{ 2 } \right)^{ 2n + 4 } \times \frac{ 1 - 4^{n + 1}
}{ 1 - 4 } = \left( \frac{ 1 }{ 2 } \right)^{ 2n + 4 } \times \frac{
2^{2n + 2} - 1 }{ 3 } \\
\\
 & = & \frac{ \frac{ 1 }{ 4 } - \left( \frac{ 1 }{ 2 } \right)^{ 2n + 4
} }{ 3 } = \frac{ 1 }{ 12 } \left( 1 - \left( \frac{ 1 }{ 2 }
\right)^{2n + 2} \right)
 \end{eqnarray*}

 \end{noliste}

 \item On va montrer ce résultat par récurrence sur $n$, puisque c'est
ainsi qu'est définie la suite $(Z_{n})$ : \begin{noliste}{$\sbullet$}

 \item \textbf{Initialisation} : $2^{0 + 1} Z_{0} = 2 E_{0} = X_{0}$
suit la loi de Bernouilli de paramètre $\frac{ 1 }{ 2 }$, qui est bien
la loi uniforme sur $\llb 0 ; 1 \rrb$, avec $2^{0 + 1} - 1 = 2^{1} - 1
= 2 - 1 = 1 $. \\

 \item \textbf{Hérédité} : on suppose qu'il existe $n \in \N$ fixé tel
que $ 2^{n + 1} Z_{n}$ suit la loi uniforme discrète sur $\llb 0 ; 2^{n
+ 1} - 1 \rrb$, alors : 
 
\[
 2^{n + 2} Z_{n + 1} = 2^{n + 2} \times \frac{ Z_{n} + X_{n + 1} }{ 2 }
= 2^{ n + 1 } ( Z_{n} + X_{n + 1} ) = 2^{n + 1} Z_{n} + 2^{n + 1} X_{n
+ 1}. 
\]

 La valeur minimale possible est obtenue avec la valeur minimale de
$2^{n + 1} Z_{n}$, qui vaut 0, et de $X_{n + 1}$, qui vaut 0 : c'est
donc 0. \\

 La valeur maximale possible est obtenue avec la valeur maximale de
$2^{n + 1} Z_{n}$, qui vaut $2^{n + 1}-1$, et de $X_{n + 1}$, qui vaut
1 : c'est donc :
 
\[
 2^{n + 1} - 1 + 2^{n + 1} = 2 \times 2^{n + 1} - 1 = 2^{n + 2} - 1. 
\]

 Vérifions que toutes les valeurs entières sont obtenues (il est
immédiat que toutes les valeurs sont bien entières comme somme de deux
entiers) : \begin{noliste}{$\sbullet$}

 \item Avec $X_{n + 1} = 0$, $Z_{n + 1}$ peut prendre toutes les
valeurs de $( 2^{n + 1} Z_{n} ) ( \Omega)$, donc tous les entiers de 0
à $2^{n + 1} - 1$. \\

 \item Avec $X_{n + 1} = 1$, $Z_{n + 1}$ donne un entier entre 0 et
$2^{n + 1} - 1$, auquel on ajoute $2^{n + 1}$ : cela donne tous les
entiers de 
 
\[
 0 + 2^{n + 1} = 2^{n + 1} \ \ \text { à } \ \ 2^{n + 1} -1 + 2^{n + 1}
= 2^{n + 2} - 1. 
\]

 \end{noliste}

 On obtient bien que $( 2^{n + 2} Z_{n + 1} ) (\Omega) = \llb 0 ; 2^{n
+ 2 - 1 } \rrb$. Montrons à présent que toutes les probabilités de la
loi sont égales : on sait que toutes celles de la loi de $2^{n + 1}
Z_{n}$ le sont donc il existe $a \in \ ]0 ;1[$ tel que : 
 
\[
 \forall k \in \llb 0 ; 2^{n + 1} - 1 \rrb, \ P \left(\Ev{ 2^{n + 1}
Z_{n} = k }\right) = a. 
\]

 On a vu dans la recherche du support que : 
 
\[
 \forall k \in \llb 0 ; 2^{n + 1} - 1 \rrb, \ ( 2^{n + 2} Z_{n + 1} = k
) = ( 2^{n + 1} Z_{n} = k ) \cap (X_{n + 1} = 0 ) 
\]

 et par indépendance (question 2b), 
 
\[
 P \left(\Ev{ 2^{n + 2} Z_{n + 1} = k }\right) = \frac{ 1 }{ 2 } a. 
\]

 De même on a vu que (et par indépendance pour la probabilité) :
 
\[
 \forall k \in \llb 2^{n + 1} ; 2^{n + 2} - 1 \rrb, \ ( 2^{n + 2} Z_{n
+ 1} = k ) = ( 2^{n + 1} Z_{n} = k - 2^{n + 1} ) \cap (X_{n + 1} = 1 )
\ \ \ \text{ donc } \ \ \ P \left(\Ev{ 2^{n + 2} Z_{n + 1} = k }\right)
= \frac{ 1 }{ 2 } a. 
\]

 On en déduit bien que toutes les probabilités de la loi sont égales,
c'est une loi uniforme et $2^{n + 2} Z_{n + 1}$ suit bien la loi
uniforme discrète sur $\llb 0 ; 2^{n + 2} - 1 \rrb$, la propriété est
encore vraie au rang $n + 1$. \\

 \item \textbf{Conclusion} : pour tout $n \in \N$, $ 2^{n + 1} Z_{n}$
suit la loi uniforme discrète sur $\llb 0 ; 2^{n + 1} - 1 \rrb$. \\

 \end{noliste}

 \item Comme la variable aléatoire d'arrivée est à densité, sa fonction
de répartition est continue sur $\R$, et il faut donc prouver la
convergence de $F_{ Z_{n} } (x)$ vers $F_{Z} (x)$ pour tout $x \in \R$.
On commence donc par chercher la fonction de répartition de $Z_{n}$. \\

 D'après la question précédente, on remarque que $Z_{n}$ suit la loi
uniforme discrète sur l'ensemble : 
 
\[
 Z_{n} ( \Omega ) = \left\{ \frac{ k }{ 2^{n + 1} } \text{ tq } k \in
\llb 0 ; 2^{n + 1} - 1 \rrb \right\} = \left\{ 0 ; \frac{ 1 }{ 2^{n +
1} } ; \dots ; \frac{ 2^{n + 1} - 1 }{ 2^{ n + 1 } } \right\} = \left\{
0 ; \frac{ 1 }{ 2^{n + 1} } ; \dots ; 1 - \frac{ 1 }{ 2^{ n + 1 } }
\right\} 
\]

 On en déduit que pour tout $x < 0$ (avant le support) on a : 
 
\[
 F_{ Z_{n} } (x) = 0 \xrightarrow[ n \rightarrow + \infty ]{} 0. 
\]

 De même pour tout $x \geq 1$ (après le support pour tout $n$, puisque
$ 1 - \frac{ 1 }{ 2^{n + 1} }$ tend en croissant vers 1 : 
 
\[
 F_{ Z_{n} } (x) = 1 \xrightarrow[ n \rightarrow + \infty ]{} 1. 
\]

 Considérons à présent un $x \in [ 0 ; 1 [$, on a : 
 
\[
 F_{ Z_{n} } (x) = \Sum{ k = 0 }{ j (x) } \Prob\Ev{ Z_{n} = k } 
\]

 avec $j(x)$ le plus grand entier vérifiant : 
 
\[
 \frac{ j }{ 2^{n + 1} } \leq x. 
\]

 On en déduit d'une part que : 
 
\[
 F_{ Z_{n} } (x) = \Sum{ k = 0 }{ j (x) } \frac{ 1 }{ \Card ( \llb 0 ;
2^{n + 1} - 1 \rrb ) } = \Sum{ k = 0 }{ j (x) } \frac{ 1 }{ 2^{n + 1} }
= \frac{ j(x) + 1 }{ 2^{ n + 1 } } 
\]

 et d'autre part pour exprimer $j(x)$, que 
 
\[
 \frac{ j(x) }{ 2^{ n + 1 } } \leq x < \frac{ j(x) + 1 }{ 2^{n + 1} }
\Longleftrightarrow j(x) \leq 2^{n + 1} x < j(x) + 1
\Longleftrightarrow j(x) = \lfloor 2^{ n + 1 } x \rfloor. 
\]

 Finalement on en déduit que : 
 
\[
 F_{ Z_{n} } (x) = \frac{ \lfloor 2^{ n + 1 } x \rfloor + 1 }{ 2^{ n +
1 } }. 
\]

 Pour déterminer la limite de cette quantité, il faut encadrer la
partie entière et utiliser le théorème d'encadrement. On sait que : 
 
\[
 \lfloor 2^{ n + 1 } x \rfloor \leq 2^{n + 1} x < \lfloor 2^{ n + 1 } x
\rfloor + 1 \ \ \text{ donc } \ \ 2^{ n + 1 } x - 1 < \lfloor 2^{ n + 1
} x \rfloor \leq 2^{n + 1} x. 
\]

 On en déduit (on rajoute un et on divise par un nombre strictement
positif, opérations strictement croissantes) que : 
 
\[
 \frac{ 2^{n + 1} x }{ 2^{n + 1} } < F_{ Z_{n} } (x) \leq \frac{ 2^{n +
1} x + 1 }{ 2^{ n + 1 } } \Longleftrightarrow x < F_{ Z_{n} } (x) \leq
x + \frac{ 1 }{ 2^{ n + 1 } } 
\]

 et comme les termes de gauche et droite ont pour limite immédiate $x$
lorsque $n$ tend vers $ + \infty$, on obtient finalement : 
 
\[
 \dlim{ n \rightarrow + \infty } F_{ Z_{n} } (x) = x. 
\]

 Enfin en rassemblant tous les cas, on a pour $x \in \R$ : 
 
\[
 F_{ Z_{n} } (x) \xrightarrow[ n \rightarrow + \infty ]{} \left\{\begin{array}{cl}
 0 & \text{ si } x < 0 \\
x & \text{ si } 0 \leq x < 1 \\
1 & \text{ si } x \geq 1 \\
\end{array}
\right. 
\]

 On reconnaît exactement la fonction de répartition de la loi uniforme
sur $[0 ; 1]$ (à densité), donc si $Z$ est une variable aléatoire qui
suit cette loi, $(Z_{n})$ converge en loi vers $Z$.

 \end{noliste}

 \noindent \textbf{\underline{Exercice sans préparation}} \\

 \begin{noliste}{1.}
 \setlength{\itemsep}{4mm}

 \item Cette intégrale est généralisée en 0 (car $\ln$ n'y est pas
défini) et en 1 (le dénominateur vaut alors 0). \\
\begin{noliste}{$\sbullet$}

 \item Au voisinage de 0, par croissances comparées, on a 
 
\[
 \dlim{ x \rightarrow 0 } \frac{ x^{n} \ln x }{ x^{n} - 1 } = \frac{ 0
}{ -1 } = 0 
\]

 donc l'intégrale est faussement généralisée (la fonction intégrée est
prolongeable par continuité en 0), elle converge donc. \\

 \item Au voisinage de 1, la limite donne une forme indéterminée. On
cherche un équivalent de chaque facteur au point 1.
\begin{noliste}{$\sbullet$}

 \item Par produit immédiat, $x^{n}$ converge vers 1 donc 
 
\[
 x^{n} \underset{ x \rightarrow 1 }{ \sim } 1. 
\]

 \item Pour le $\ln$, on pose $ x = 1 + y \Longleftrightarrow y = x -
1$, avec $y$ qui tend vers 0, on a par DL : 
 
\[
 \ln (1 + y ) = y + o (y) \underset{ y \rightarrow 0 }{ \sim } y 
\]

 donc
 
\[
 \ln x \underset{ x \rightarrow 1 }{ \sim } x - 1. 
\]

 \item Enfin avec le même changement de variable et par DL : 
 
\[
 (1 + y)^{n} - 1 = 1 + n y + o(y) - 1 = n y + o(y) \underset{ y
\rightarrow 0 }{ \sim } n y 
\]

 donc
 
\[
 x^{n} - 1 \underset{ x \rightarrow 1 }{ \sim } n (x-1 ). 
\]

 \end{noliste}

 On obtient finalement : 
 
\[
 \frac{ x^{n} \ln x }{ x^{n} - 1 } \underset{ x \rightarrow 1 }{ \sim }
\frac{ 1 \times (x-1 ) }{ n (x-1 ) } = \frac{ 1 }{ n } \xrightarrow[ x
\rightarrow 1 ]{} \frac{1}{n}. 
\]

%%% LALALAL

 De nouveau la fonction est prolongeable par continuité, et l'intégrale
est faussement généralisée donc converge au voisinage de 1. \\

 \end{noliste}

 Finalement l'intégrale $\dint{0}{1} \frac{ x^{n} \ln x }{ x^{n} - 1 }
$ converge, elle existe bien. \\

\item La question appelle clairement le théorème de convergence
  monotone, puisque la limite n'est pas demandée. On étudie la
  monotonie de $(u_{n})$, en cherchant le signe de :
  \begin{eqnarray*}
    u_{n + 1} - u_{n} & = & \dint{0}{1} \ln x \left( \frac{ x^{n + 1} }{
        x^{n + 1} - 1 } - \frac{ x^{n} }{ x^{n} - 1 } \right) \ dx =
    \dint{0}{1} \ln x \times \frac{ x^{n + 1} (x^{n} - 1 ) - x^{n} (x^{n +
        1} - 1 ) }{ (x^{n + 1} - 1 ) ( x^{n} - 1 ) } \ dx \\
    \\
    & = & \dint{0}{1} \frac{ \ln x }{ (x^{n + 1} - 1 ) ( x^{n} - 1 ) }
    \times x^{n} \times \left( x ( x^{n} - 1 ) - x^{n +
        1} + 1 \right) \ dx \\
    \\
    & = & \dint{0}{1} \frac{x^{n} \ln x }{ (x^{n + 1} - 1 ) ( x^{n} - 1 )
    } \left( x^{ n + 1 } - x - x^{n + 1} + 1 \right) \ dx
    = \dint{0}{1} \frac{ x^{n} (1-x) \ln x }{ (x^{n + 1} - 1 ) ( x^{n} - 1
      ) \dx}
 \end{eqnarray*}

%%% LALALAL

 Or pour tout $x \in \ ] 0 ; 1[$, on a : 
 
\[
 x^{n} \geq 0 \ \, \ \ 1-x \geq 0 \ \, \ \ \ln x \leq 0 \ \, \ \ x^{n}
\leq 1 \text{ donc } x^{n} - 1 \leq 0 \ \, \ \ x^{n + 1} \leq 1 \text{
donc } x^{n + 1} - 1 \leq 0 
\]

 et par produit et quotient de deux facteurs positifs et trois
négatifs, la fonction intégrée est négative sur $]0;1[$. Comme les
bornes sont dans l'ordre croissant, on en déduit que pour tout $n \in
\N$ : 
 
\[
 u_{n + 1} - u_{n} \leq 0 
\]

 et la suite $(u_{n})$ est décroissante. \\

 Enfin la suite est positive car la fonction intégrée est positive sur
$]0 ; 1[$ avec un facteur positif et deux négatifs et les bornes sont
dans l'ordre croissant, donc elle décroît et elle est minorée par 0 :
c'est une suite convergente.

 \end{noliste}
 \end{exercice}

 \newpage

 \begin{exercice}{\it (Exercice avec préparation)}~
 \begin{noliste}{1.}
 \setlength{\itemsep}{4mm}
 \item Soit $f$ une fonction de classe $C^{p + 1}$ sur $[0;1]$, alors
pour tous $a$ et $b$ de $[0;1]$, on a :
 \begin{eqnarray*}
 f(b) & = & f(a) + f'(a) (b-a) + \dots + f^{(p)} (a) \frac{ (b-a)^{p}
}{ p! } + \dint{a}{b} \frac{ (b-t)^{p} }{ p ! } f^{ (p + 1) } (t) \ dt
\\
\\
 & = & \Sum{k = 0}{p} f^{ (k) } (a) \frac{ (b-a)^{k} }{ k! } +
\dint{a}{b} \frac{ (b-t)^{p} }{ p ! } f^{ (p + 1) } (t) \ dt 
 \end{eqnarray*}

 \item Soit $x$ un réel de l'intervalle $[0;1[$. \begin{noliste}{a)}
 \setlength{\itemsep}{2mm}

 \item Comme $t \in [0;x]$ et $x \in [0;1[$ on a $0 \leq t \leq x < 1$,
et donc : 
 
\[
 x - t \geq 0 \ \ \text{ et } \ \ 1-t > 0 \ \ \text{ donc par quotient
} \ \ \frac{ x-t }{ 1-t } \geq 0. 
\]

 Pour l'autre côté, on considère la différence : 
 
\[
 \frac{ x-t }{ 1-t } - x = \frac{ x - t - x (1-t ) }{ 1 - t } = \frac{
x - t - x + x t }{ 1 - t } = \frac{ t (x-1 ) }{ 1 - t } 
\]

 et comme $t \geq 0$, $x-1 \leq 0$ et $1-t > 0$, cette quantité est
négative donc on a bien : 
 
\[
 \forall t \in [0 ; x ], \ 0 \leq \frac{ x - t }{ 1 - t } \leq x. 
\]

 \item On reconnaît à droite une intégration de la série géométrique.
On part de la somme géométrique finie : pour tout $p \geq 1$, on a : 
 
\[
 \Sum{n = 1}{p} x^{n-1} = \Sum{n = 0}{p-1} x^{n} = \frac{ 1 - x^{p} }{
1 - x }. 
\]

 On intègre cette inégalité entre 0 et $x$ (en la réécrivant pour tout
$t$ au lieu de $x$) : 
 
\[
 \Sum{n = 1}{p} \dint{0}{x} t^{n-1} \ dt = \dint{0}{x} \frac{ 1 - t^{p}
}{ 1 - t } \ dt = \dint{0}{x} \frac{ 1 }{ 1-t } \ dt - \dint{0}{x}
\frac{ t^{p} }{ 1-t } \ dt. 
\]

 On calcule les intégrales "faciles" : 
 
\[
 \Sum{n = 1}{p} \dint{0}{x} t^{n-1} \ dt = \Sum{n = 1}{p} \frac{ x^{n}
}{ n } 
\]


%%% LALALAL
 et
 
\[
 \dint{0}{x} \frac{ 1 }{ 1-t } \ dt = \left[ - \ln (1 - t)
\right]_{0}{x} = - \ln (1-x ) 
\]

 donc :
 
\[
 \ln (1-x ) = \dint{0}{x} \frac{ t^{p} }{ 1-t } \ dt - \Sum{n = 1}{p}
\frac{ x^{n} }{ n }. 
\]

 Enfin l'intégrale restante ne peut pas se calculer, mais on va montrer
que sa limite lorsque $p$ tend vers $ + \infty$ est nulle : pour cela
on encadre l'intégrale en écrivant : 
 
\[
 0 \leq t \leq x \ \ \text{ donc } \ \ - x \leq -t \leq 0 \ \ \text{
donc } \ \ 1-x \leq 1-t \leq 1 
\]

 et comme tous les termes sont strictement positifs (avec $x < 1$), par
stricte décroissance de l'inverse sur $\R_+^*$ : 
 
\[
 1 \leq \frac{ 1 }{ 1-t } \leq \frac{ 1 }{ 1-x } \ \ \text{ et enfin }
\ \ t^{p} \leq \frac{ t^{p} }{ 1-t } \leq \frac{ t^{p} }{ 1 - x } 
\]

 avec $t^{p} \geq 0$. On intègre avec les bornes dans l'ordre croissant
: 
 
\[
 \dint{0}{x} t^{p} \ dt \leq \dint{0}{x} \frac{ t^{p} }{ 1-t } \ dt
\leq \frac{1}{1-x} \dint{0}{x} t^{p} \ dt 
\]

 et en calculant les termes de gauche et droite : 
 
\[
 \frac{ x^{p + 1} }{ p + 1 } \leq \dint{0}{x} \frac{ t^{p} }{ 1-t } \
dt \leq \frac{ x^{p + 1} }{ (1-x) (p + 1) }. 
\]

 Enfin comme $ | x | < 1$, par quotient de limites, les deux termes
extrémaux tendent vers 0 lorsque $p$ tend vers $ + \infty$, et par
encadrement le terme central aussi. \\

 Enfin on peut passer à la limite l'égalité obtenue précédemment sur
$\ln (1-x)$, avec tous les termes qui convergent : 
 
\[
 \ln (1-x ) = - \Sum{n = 1}{+ \infty} \frac{ x^{n} }{ n }. 
\]

 \end{noliste}

 \item Soit $X$ une variable aléatoire discrète définie sur un espace
probabilisé $(\Omega, \mathcal{A}, P)$ telle que pour tout $n \in
\N^*$, on a : $\Prob\left(\Ev{\Ev{X = n}}\right) = \frac{ 1 }{ n ( n +
1) }$.

 \begin{noliste}{a)}
 \setlength{\itemsep}{2mm}

 \item Difficile. Comme ce n'est pas une série usuelle, il faut penser
à faire apparaître un télescopage en remarquant que : 
 
\[
 \frac{ 1 }{ n (n + 1) } = \frac{ (n + 1) - n }{ n (n + 1) } = \frac{ 1
}{ n } - \frac{ 1 }{ n + 1 }. 
\]

 On peut alors écrire : 
 
\[
 P \left(\Ev{ X \in \N^* }\right) = \Sum{n = 1}{+ \infty} P \Ev{ X = n
} = \dlim{ p \rightarrow + \infty} \left( \Sum{n = 1}{p} \frac{ 1 }{ n
} - \frac{ 1 }{ n + 1 } \right) = \dlim{ p \rightarrow + \infty} \left(
1 - \frac{ 1 }{ p + 1} \right) = 1. 
\]

 \item Soit $r \geq 1$, on étudie l'existence du moment d'ordre $r$ de
$X$, en considérant la série de terme général : 
 
\[
 n^{r} \Prob\Ev{ X = n } = \frac{ n^{r} }{ n (n + 1) } = \frac{ n^{ r-1
} }{ n \left( 1 + \frac{1}{n} \right) } \underset{ + \infty }{ \sim }
n^{ r- 2 } = \frac{ 1 }{ n^{ 2 - r } }. 
\]

 Les deux termes généraux sont positifs donc par théorème de
comparaison $X$ admet un moment d'ordre $r$ si et seulement si la série
de terme général $\frac{ 1 }{ n^{ 2-r } }$ est convergente. Cette série
de Riemann converge si et seulement si : 
 
\[
 2 - r > 1 \Longleftrightarrow r < 1 \Longleftrightarrow r \leq 0 
\]

 puisque $r$ est un entier : la variable $X$ n'admet donc aucun moment
(à part celui d'ordre 0, qui correspond à la somme des probabilités et
qui doit obligatoirement exister et valoir 1 pour $x$ soit une variable
aléatoire). \\

 \item Par théorème de transfert, on s'intéresse à la série (sous
réserve de convergence absolue) : 
 \begin{eqnarray*}
 E ( s^{X} ) & = & \Sum{n = 1}{+ \infty} s^{n} P \Ev{ X = n } = \Sum{ n
= 1 }{ + \infty } \frac{ s^{n} }{ n (n + 1) }. 
 \end{eqnarray*}

 La convergence absolue de cette série s'obtient immédiatement, avec
$s^{n} \leq 1$ donc $\frac{ s^{n} }{ n (n + 1) } \leq \frac{ 1 }{ n (n
+ 1) }$ qui est le terme général d'une série convergente (3a), et avec
les deux termes généraux positifs le théorème de comparaison donne la
convergence puis la convergence absolue (puisque le terme général est
positif) de la série considérée. \\

 Pour la calculer (seulement valable pour $s \in \ ]0 ; 1[$), on pense
à utiliser la question 2, et pour ne plus avoir que $n$ au dénominateur
on pense à la transformation de la question 3a : 
 \begin{eqnarray*}
 E ( s^{X} ) & = & \Sum{ n = 1 }{ + \infty } \frac{ s^{n} }{ n } -
\Sum{ n = 1 }{ + \infty } \frac{ s^{n} }{ n + 1 } = - \ln (1-s) - \Sum{
n = 2 }{ + \infty } \frac{ s^{n-1} }{ n } \\
\\
 & = & - \ln (1-s) - \frac{1}{s} \left( \Sum{ n = 1 }{ + \infty }
\frac{ s^{n} }{ n } - \frac{ s }{ 1 } \right) = - \ln (1-s) -
\frac{1}{s} \left( - \ln (1-s) - s \right) \\
\\
 & = & - \ln (1-s) + \frac{ \ln (1-s) }{ s} + 1 = \frac{ - s \ln (1-s)
+ \ln (1-s) + s }{ s } = \frac{ s + (1-s) \ln (1-s) }{ s }. 
 \end{eqnarray*}

 \item On calcule les valeurs en $s = 0$ et $s = 1$ : 
 
\[
 E ( 0^{X} ) = E (0 ) = 0 \ \ \text{ et } \ \ E ( 1^{X} ) = E (1) = 1 
\]


%%% LALALAL

 donc : 
 
\[
 \phi (s) = \left\{\begin{array}{cl}
 0 & \text{ si } s = 0 \\
\frac{ s + (1-s) \ln (1-s) }{ s } & \text{ si } 0 < s < 1 \\
1 & \text{ si } s = 1 \\
\end{array}
\right. 
\]

 Cette fonction est continue (et dérivable) sur $]0;1[$ comme somme,
produit, quotient, composée de fonctions dérivables, avec $s \neq 0$ et
$1-s > 0$. \\

 Etudions la continuité en 0 et 1 : \\
\begin{noliste}{$\sbullet$}

 \item Au voisinage de 0, par DL de $\ln (1-s)$, on a : 
 
\[
 \phi (s) = 1 + \frac{ (1-s) \left( -s - \frac{ s^{2} }{ 2 } + o (
s^{2} ) \right) }{ s } = 1 + (1-s ) \left(- 1 - \frac{ s }{ 2 } + o(s)
\right) \xrightarrow[ s \rightarrow 0]{} 1 + 1 \times (-1) = 0 = \phi
(0) 
\]

 et $\phi$ est continue en 0. \\

 \item Au voisinage de 1, par croissances comparées (avec $y = 1-s$ qui
tend vers 0), on obtient : 
 
\[
 \dlim{ s \rightarrow 1 } \phi (s) = \frac{ 1 + 0 }{ 1 } = 1 = \phi (1)
\]


 et $\phi$ est continue en 1.

 \end{noliste}

 On en déduit que $\phi$ est continue sur $[0;1]$, on étudie à présent
sa dérivabilité en 0 et 1 : \begin{noliste}{$\sbullet$}

 \item Au voisinage de 0, on a : 
 \begin{eqnarray*}
 \frac{ \phi (s) - \phi (0) }{ s - 0 } & = & \frac{ \phi (s) }{ s } =
\frac{ s + (1-s) \ln (1-s) }{ s^{2} } = \frac{ s + (1-s) \left( - s -
\frac{ s^{2} }{ 2 } + o(s^{2} ) \right) }{ s^{2} } \\
\\
 & = & \frac{ 1 + (1-s) \left( - 1 - \frac{ s }{ 2 } + o(s ) \right) }{
s } = \frac{ 1 - 1 - \frac{ s }{ 2 } + s + o(s) }{ s } = \frac{ 1 }{ 2
} + o(1 ) \\
\\
 & \xrightarrow[ s \rightarrow 0]{} \frac{ 1 }{ 2 } 
 \end{eqnarray*}

 donc $\phi$ est dérivable en 0. \\

 \item Au voisinage de 1, on a :
 \begin{eqnarray*}
 \frac{ \phi (s) - \phi (1) }{ s - 1 } & = & \frac{ \phi (s) - 1 }{ s -
1 } = \frac{ \frac{ s + (1-s) \ln (1-s) }{ s } - \frac{ s }{ s } }{ s -
1 } = \frac{ s + (1-s) \ln (1-s) - s }{ s (s-1) } \\
\\
 & = & \frac{ (1-s) \ln (1-s) }{ s (s-1) } = - \frac{ \ln (1-s) }{ s }
\xrightarrow[ s \rightarrow 1 ]{} - \frac{ - \infty }{ 1 } = + \infty 
 \end{eqnarray*}

 donc $\phi$ n'est pas dérivable en 1.

 \end{noliste}

 On en déduit que $\phi$ est continue sur $[0;1]$, dérivable sur
$[0;1[$, mais pas dérivable en 1. \\

 \item Par théorème de transfert, on s'intéresse à la série :
 
\[
 \Sum{n = 1}{+ \infty} n s^{n} P \Ev{ X = n } = \Sum{n = 1}{+ \infty}
\frac{ s^{n} }{ n + 1 }. \]


 Cette série est à termes positifs; pour $0 \leq s < 1$, le terme
général est négligeable devant $s^{n}$ qui est le terme général d'une
série convergente, donc l'espérance existe bien. \\

 Pour $s = 1$, on reconnaît la série harmonique, la série diverge et
l'espérance n'existe pas. \\

 Pour $s > 1$, le terme général diverge par croissances comparées, la
série diverge donc grossièrement et l'espérance n'existe pas. Enfin,
pour $0 \leq s < 1$ : 
 
\[
 E ( X s^{X} ) = \Sum{n = 1}{+ \infty} \frac{ s^{n} }{ n + 1 } = \Sum{n
= 2}{+ \infty} \frac{ s^{n-1} }{ n } = \frac{1}{s} \Sum{n = 2}{+
\infty} \frac{ s^{n} }{ n } = \frac{ - \ln (1-s) - s }{ s }. 
\]

 Pour le moment d'ordre deux, toujours par transfert, on s'intéresse à
: 
 
\[
 \Sum{n = 1}{+ \infty} n^{2} s^{2n} P \Ev{ X = n } = \Sum{n = 1}{+
\infty} \frac{ n s^{2n} }{ n + 1 }. 
\]

 Cette série est à termes positifs, et le terme général est équivalent
en $ + \infty$ à $s^{2n} = (s^{2})^{n}$, donc la série converge
absolument si et seulement si : 
 
\[
 s^{2} < 1 \Longleftrightarrow 0\leq s < 1 
\]


%%% LALALAL

 Enfin pour ces valeurs de $s$, on a : 
 \begin{eqnarray*}
 E ( X^{2} s^{ 2X } ) & = & \Sum{n = 1}{+ \infty} \frac{ (n + 1 - 1 )
(s^{2})^{n} }{ n + 1 } = \Sum{n = 1}{+ \infty} (s^{2})^{n} - \Sum{n =
1}{+ \infty} \frac{ (s^{2})^{n} }{ n + 1 } \\
\\
 & = & \frac{ 1 }{ 1 - s^{2} } - 1 - E ( X s^{X} ) = \frac{ s^{2} }{ 1
- s^{2} } + \frac{ \ln (1-s) + s }{ s }. 
 \end{eqnarray*}

 Enfin $X s^{X}$ admet une variance si et seulement si $0 \leq s < 1$,
et : 
 
\[
 V ( X s^{X} ) = \frac{ s^{2} }{ 1 - s^{2} } + \frac{ \ln (1-s) + s }{
s } - \left( \frac{ \ln (1-s) + s }{ s } \right)^{2}. 
\]

 \end{noliste}

 \end{noliste}

 \noindent \textbf{\underline{Exercice sans préparation}} \\

 \begin{noliste}{1.}
 \setlength{\itemsep}{4mm}

 \item On étudie les variations de $f$ : $f$ est dérivable et pour tout
$x \in \R$,
 
\[
 f'(x) = 3 x^{2} + 2 x + 1 
\]

 et ce polynôme du second degré a pour discriminant $\Delta = 4 - 12 =
-8 < 0$, donc avec $3 > 0$, $f'(x)$ est strictement positive sur $\R$.
\\

 On en déduit que $f$ est strictement croissante sur $\R$, avec pour
limites $ + \infty$ et $-\infty$ en $ + \infty$ et $- \infty$ (terme
prépondérant $x^{3}$ à chaque fois), donc par théorème de bijection,
$f$ est bijective de $\R$ dans $\R$. \\

 \item Pour qu'un polynôme soit surjectif, il faut qu'il prenne toutes
les valeurs de $-\infty$ à $ + \infty$. Pour cela, comme il est
continue, il faut et il suffit qu'il ait une limite égale à $-\infty$
et une égale à $ + \infty$, et comme les seules limites sont en $ +
\infty$ et $-\infty$, l'une doit être égale à $ + \infty$, l'autre à
$-\infty$. \\

 Or un polynôme est équivalent en $\pm \infty$, à son terme de plus
haut degré, qui sera de la forme $a_{n} x^{n}$, avec $a_{n} \neq 0$ (où
le polynôme est de degré $n$). La limite en $ + \infty$ sera décidée
par le signe de $a_{n}$, et celle en $-\infty$ par le signe de $a_{n}$
\text{ et la parité de n}. \\

 Si $n$ est pair, les deux limites sont les mêmes (car $x^{n}$ a la
même limite en $ + \infty$et $-\infty$), si $n$ est impaire elle seront
opposées, et dans tous les cas elles seront infinies. \\

 On en déduit qu'un polynôme est surjectif de $\R$ dans $\R$ si et
seulement si il est de degré impair. \\

 \item Pour être injective, une fonction continue à valeurs réelles
doit être strictement monotone, donc sa dérivée (tout polynôme est
dérivable) ne doit jamais s'annuler. \\

 on en déduit que la dérivée $f'$ de $f$ n'est pas surjective, donc par
2. $f'$ est de degré pair, et $f$ est de degré impair. \\

 Je ne vois comment aller plus loin sans la décomposition en facteurs
premiers des polynômes réels, qui n'est pas au programme, et qui
permettrait de prouver que $f'$ doit s'écrire : 
 
\[
 f'(x) = \prod\limits_{k = 1}{p} P_{k}(x) 
\]

 avec, pour tout $k$, $P_{k}$ un polynôme du second degré dont le
discriminant est strictement négatif.

 \end{noliste}
 \end{exercice}

 \newpage

 \begin{exercice}{\it (Exercice avec préparation)}~
 \begin{noliste}{1.}
 \setlength{\itemsep}{4mm}
 \item Soit $(A_{i})_{ i \in I }$ un système complet d'évènements, avec
$I$ un ensemble discret (fini ou indexé par $\N$) et $B$ un évènement,
alors : 
 
\[
 B = \dcup{ i \in I } ( A_{i} \cap B ) 
\]

 avec l'union incompatible car les $(A_{i})$ le sont, et les $A_{i}$ de
probabilité non nulles donc par probabilités composées on obtient : 
 
\[
 P \left(\Ev{ B }\right) = \Sum{i \in I} P ( A_{i} \cap B ) = \Sum{ i
\in I } P \left(\Ev{A_{i} }\right) P_{ A_{i} } ( B ). 
\]

 Soit $p$ et $q$ deux réels vérifiant $0<p<1$ et $p + 2q = 1$. On note
$\Delta$ la matrice de $\mathcal{M}_{3} ( \R )$ définie par :
 
\[
 \Delta = \begin{smatrix}
p & q & q \\
q & p & q \\
q & q & p \\
\end{smatrix}
\]

 \item $\Delta$ est symétrique donc diagonalisable. \\

 \item Il faut donc déterminer les valeurs propres de $\Delta$. On
remarque que
 
\[
 \Delta = p I + q A 
\]

 avec $A = \begin{smatrix}
0 & 1 & 1 \\
1 & 0 & 1 \\
1 & 1 & 0 \\
\end{smatrix}
$ qui est symétrique donc diagonalisable, donc en diagonalisant $A$ on
obtiendra : 
 
\[
 \Delta = p I + q P D' P^{-1} = p P I P^{-1} + q P D' P^{-1} = P
\left(\Ev{ p I + q D' }\right) P^{-1} 
\]

 et les valeurs propres de $\Delta$ sont donc données par : 
 
\[
 \spc ( \Delta ) = \{ p + q \lambda \text{ tq } \lambda \in \spc ( A )
\}. 
\]


%%% LALALAL

 On cherche donc les valeurs propres de $A$, c'est-à-dire les $\lambda$
réels tels que $A - \lambda I$ n'est pas inversible : 
 \begin{eqnarray*}
 A - \lambda I = \begin{smatrix}
- \lambda & 1 & 1 \\
1 & -\lambda & 1 \\
1 & 1 & - \lambda \\
\end{smatrix}
 & \Longleftrightarrow L_{1} \leftrightarrow L_{3} & \begin{smatrix}
1 & 1 & - \lambda \\
1 & -\lambda & 1 \\
- \lambda & 1 & 1 \\
\end{smatrix}
\\
\\
 & \Longleftrightarrow\begin{array}{c}
 \\
L_{2} \leftarrow L_{2} - L_{1} \\
L_{3} \leftarrow L_{3} + \lambda L_{1} \\
\end{array}
 & \begin{smatrix}
1 & 1 & - \lambda \\
0 & -\lambda - 1 & 1 + \lambda \\
0 & 1 + \lambda & 1 - \lambda^{2} \\
\end{smatrix}
\\
\\
 & \Longleftrightarrow\begin{array}{c}
 \\
\\L_{3} \leftarrow L_{3} + L_{2} \\
\end{array}
 & \begin{smatrix}
1 & 1 & - \lambda \\
0 & -\lambda - 1 & 1 + \lambda \\
0 & 0 & 2 + \lambda - \lambda^{2} \\
\end{smatrix}
 \end{eqnarray*}

 et les valeurs propres de $A$ sont les solution de $- \lambda - 1 = 0
\Longleftrightarrow \lambda = -1$ et de $- \lambda^{2} + \lambda + 2 =
0$, qui a pour discriminant et racines : 
 
\[
 \Delta = 1 + 8 = 9 \ \, \ \ \lambda_{1} = \frac{ - 1 - 3 }{ -2 } = 2 \
\, \ \ \lambda_{2} = \frac{ -1 + 3 }{ -2 } = - 1 
\]

 On en déduit que
 
\[
 D' = \begin{smatrix}
-1 & 0 & 0 \\
0 & -1 & 0 \\
0 & 0 & 2 \\
\end{smatrix}
\ \ \text{ puis } \ \ D = \begin{smatrix}
p - q & 0 & 0 \\
0 & p - q & 0 \\
0 & 0 & p + 2 q \\
\end{smatrix}
 = \begin{smatrix}
p - q & 0 & 0 \\
0 & p-q & 0 \\
0 & 0 & 1 \\
\end{smatrix}. 
\]

 On ne déduit que 
 
\[
 D^{n} = \begin{smatrix}
(p - q)^{n} & 0 & 0 \\
0 & (p-q)^{n} & 0 \\
0 & 0 & 1 \\
\end{smatrix}
\]

 et on remarque par inégalité triangulaire que
 
\[
 | p - q | = | p + (-q) | \leq | p | + | - q | = p + q = p + 2q - q = 1
- q < 1 
\]

 donc on peut conclure que : 
 
\[
 D^{n} \xrightarrow[ n \rightarrow + \infty ]{} \begin{smatrix}
0 & 0 & 0 \\
0 & 0 & 0 \\
0 & 0 & 1 \\
\end{smatrix}. 
\]


%%% LALALAL

 Un village possède trois restaurants $R_{1}$, $R_{2}$ et $R_{3}$. Un
couple se rend dans un de ces trois restaurants chaque dimanche. A
l'instant $n = 1$ (c'est-à-dire le premier dimanche) il choisit le
restaurant $R_{1}$, puis tous les dimanches suivants (instants $n = 2$,
$n = 3$, etc.) il choisit le même restaurant que le dimanche précédent
avec la probabilité $p$ ou change de restaurant avec la probabilité
$2q$, chacun des deux autres restaurants étant choisis avec la même
probabilité. \\

 On suppose que l'expérience est modélisée par un espace probabilisé
$(\Omega, \mathcal{A}, P)$.

 \item On note $A_{n}$, $B_{n}$ et $C_{n}$ les trois évènements. La
formule des probabilités totales avec le sce $(A_{n}, B_{n}, C_{n})$
donne : 
 
\[
 A_{n + 1} = (A_{n} \cap A_{n + 1} ) \cup ( B_{n} \cap A_{n + 1} ) \cup
( C_{n} \cap (A_{n + 1} ) 
\]

 et de même pour $B_{n + 1}$ et $C_{n + 1}$; par incompatibilité de la
réunion et probabilité composées, on obtient : 
 
\[
 \left\{\begin{array}{c}
 P \left(\Ev{ A_{n + 1} }\right) = p P \left(\Ev{A_{n} }\right) + q P
\left(\Ev{ B_{n} }\right) + q P \left(\Ev{C_{n} }\right) \\
P\left(\Ev{ B_{n + 1} }\right) = q P \left(\Ev{ A_{n} }\right) + p P
\left(\Ev{ B_{n} }\right) + q P \left(\Ev{C_{n}}\right) \\
P \left(\Ev{ C_{n + 1} }\right) = q P\left(\Ev{A_{n}}\right) + q P
\left(\Ev{ B_{n} }\right) + p P \left(\Ev{C_{n} }\right)
\end{array}
\right. \ \ \ \text{ donc } \ \ \ \begin{smatrix}
P \left(\Ev{ A_{n + 1} }\right) \\
P \left(\Ev{ B_{n + 1} }\right) \\
P \left(\Ev{ C_{n + 1} }\right) \\
\end{smatrix}
 = \Delta \begin{smatrix}
P\left(\Ev{ A_{n} }\right) \\
P \left(\Ev{B_{n} }\right) \\
P \left(\Ev{ C_{n} }\right) \\
\end{smatrix}
\]

 Une récurrence immédiate donne alors : 
 
\[
 \begin{smatrix}
P\left(\Ev{ A_{n} }\right) \\
P \left(\Ev{B_{n} }\right) \\
P \left(\Ev{ C_{n} }\right) \\
\end{smatrix}
 = \Delta^{n-1} \begin{smatrix}
P\left(\Ev{ A_{1} }\right) \\
P \left(\Ev{B_{1} }\right) \\
P \left(\Ev{ C_{1} }\right) \\
\end{smatrix}
 = P D^{n-1} P^{-1} \begin{smatrix}
1 \\
0 \\
0 \\
\end{smatrix}
\]

 Pour aller plus loin il faut la matrice $P$ et $P^{-1}$ : en calculant
les sous-espaces propres de $A$ puis en inversant la matrice $P$ on
obtient : 
 
\[
 P = \begin{smatrix}
-1 & -1 & 1 \\
1 & 0 & 1 \\
0 & 1 & 1 \\
\end{smatrix}
\ \ \ \text{ puis } \ \ \ P^{-1} = \frac{1}{3} \begin{smatrix}
-1 & 2 & -1 \\
-1 & -1 & 2 \\
1 & 1 & 1 \\
\end{smatrix}
\]

 qui donne finalement : 
 
\[
 \begin{smatrix}
P\left(\Ev{ A_{n} }\right) \\
P \left(\Ev{B_{n} }\right) \\
P \left(\Ev{ C_{n} }\right) \\
\end{smatrix}
 = \frac{1}{3} P D^{n-1} \begin{smatrix}
-1 \\
-1 \\
1 \\
\end{smatrix}
 = \frac{1}{3} P \begin{smatrix}
- (p-q)^{n-1} \\
- (p-q)^{n-1} \\
1 \\
\end{smatrix}
 = \frac{1}{3} \begin{smatrix}
1 + 2 (p-q)^{n-1} \\
1 - (p-q)^{n-1} \\
1 - (p-q)^{n-1} \\
\end{smatrix}
\]

 \item Soit $T$ la variable aléatoire égale au rang du premier dimanche
où le couple retourne au restaurant $R_{1}$, s'il y retourne, et 0
sinon. \begin{noliste}{a)}
 \setlength{\itemsep}{2mm}

 \item Le premier retour au restaurant se fait au minimum lors du 2e
jour, et la valeur 0 est rajoutée artificiellement donc : 
 
\[
 T ( \Omega ) = \llb 2 ; + \infty \llb \cup \{ 0 \}. 
\]

 Pour $k = 2$, on a :
 
\[
 \Ev{T = 2 } = A_{1} \cap A_{2} \ \ \text{ donc } \ \ P \Ev{ T = 2 } =
1 \times p = p. 
\]

 Pour tout $k \geq 3$, en a de plus : 
 
\[
 \Ev{T = k } = A_{1} \cap \overline{A_{2} } \cap \dots \cap \overline{
A_{k-1} } \cap A_{k}. 
\]

 Or lorsqu'on est à un restaurant autre que le premier, la probabilité
d'aller au premier à l'instant suivant est $q$, donc de ne pas y aller
est $1-q = p + q$, ce qui donne : 
 
\[
 P \Ev{ T = k } = 1 \times (2q ) \times (p + q ) \times \dots \times (p
+ q) \times q = 2q (p + q)^{k-3} q = 2 q^{2} (p + q)^{k-3}. 
\]

 Enfin on calcule la dernière probabilité à partir des autres : 
 \begin{eqnarray*}
 P \Ev{ T = 0 } & = & 1 - \Sum{k = 2}{+ \infty} P \Ev{ T = k } = 1 - p
- 2 q^{2} \Sum{k = 3}{+ \infty} (p + q)^{k-3} = 1 - p - 2q^{2} \Sum{k =
0}{+ \infty } (p + q)^{k} \\
\\
 & = & 1 - p - 2 q^{2} \times \frac{ 1 }{ 1 - (p + q) } = 1 - p - 2
q^{2} \times \frac{ 1 }{ q } = 1 - p - 2q = 1 - ( p + 2q ) = 1 - 1 = 0.
 \end{eqnarray*}

 \item On a vu que la valeur 0 n'arrive qu'avec une probabilité 0, on
peut donc la retirer. On considère les séries : 
 
\[
 \Sum{k = 2}{+ \infty} k P \Ev{ T = k } = 2 p + 2 q^{2} \Sum{k = 3}{+
\infty} k (p + q)^{k-3} 
\]

 et
 
\[
 \Sum{k = 2}{+ \infty} k^{2} P \Ev{ T = k } = 4 p + 2 q^{2} \Sum{k =
3}{+ \infty} k^{2} (p + q)^{k-3} 
\]

 Comme $0 < p + q = 1 - q < 1$, ces séries géométriques dérivées et
dérivées secondes convergent absolument donc $T$ admet une espérance,
un moment d'ordre deux, et enfin une variance et : 
 \begin{eqnarray*}
 E ( T ) & = & 2 p + \frac{ 2 q^{2} }{ (p + q)^{2} } \left( \Sum{k =
1}{+ \infty} k (p + q)^{k-1} - 1 - 2 (p + q) \right) = 2 p + \frac{ 2
q^{2} }{ (1-q)^{2} } \left( \frac{ 1 }{ [ 1 - (p + q) ]^{2} } - 1 - 2
(1-q) \right) \\
\\
 & = & 2 p + \frac{ 2 q^{2} }{ (1-q)^{2} } \left( \frac{ 1 }{ q^{2} } -
1 - 2 (1-q) \right) = 2 p + \frac{ 2 q^{2} }{ (1-q)^{2} } \times \frac{
1 - q^{2} - 2 q^{2} (1-q) }{ q^{2} } \\
\\
 & = & 2 p + 2 \times \frac{ (1-q) (1 + q) - 2 q^{2} (1-q) }{ (1-q)^{2}
} = 2 \left( p + \frac{ 1 + q - 2 q^{2} }{ 1- q } \right).
 \end{eqnarray*}

 puis
 \begin{eqnarray*}
 E ( T^{2} ) & = & 4 p + 2 q^{2} \Sum{k = 3}{+ \infty} [ k (k-1) + k ]
(p + q)^{k-3} = 2 p + 2 q^{2} \Sum{k = 3}{+ \infty} k (k-1) (p +
q)^{k-3} + 2 q^{2} \Sum{k = 3}{+ \infty} k (p + q)^{k-3} \\
\\
 & = & 4 p + 2 \times \frac{ 1 + q - 2 q^{2} }{ 1 - q } + \frac{ 2
q^{2} }{ p + q } \left( \Sum{k = 2}{+ \infty} k (k-1) (p + q)^{k-2} - 2
\right) \\
\\
 & = & 4 p + 2 \times \frac{ 1 + q - 2 q^{2} }{ 1 - q } + \frac{ 2
q^{2} }{ 1-q } \left( \frac{ 2 }{ [ 1 -(p + q) ]^{3} } - 2 \right) \\
\\
 & = & 4 p + 2 \times \frac{ 1 + q - 2 q^{2} }{ 1 - q } + \frac{ 2
q^{2} }{ 1-q } \left( \frac{ 2 }{ q^{3} } - 2 \right) = 4 p + 2 \times
\frac{ 1 + q - 2 q^{2} }{ 1 - q } + 4 \frac{ 1 - q^{3} ) }{ (1-q) q }
 \end{eqnarray*}

 et enfin :
 
\[
 \V( T ) = 4 p + 2 \times \frac{ 1 + q - 2 q^{2} }{ 1 - q } + 4 \frac{
1 - q^{3} ) }{ (1-q) q } - 4 \left( p + \frac{ 1 + q - 2 q^{2} }{ 1- q
} \right)^{2} 
\]
 

 \end{noliste}

 \item On réalise l'expérience sur 52 dimanches, en rajoutant un
compteur qui devra être incrémenté ) chaque visite de $R_{1}$ et qui
sera divisé par 52 à la fin de la boucle pour avoir la fréquence de
visite (on rassemble les restaurants 1 et 2 en un seul restaurant, noté
2) : 

\begin{verbatim}
x = 1
c = 1
for k = 2 :52 do
 if x = 1 then 
 if rand()<p then c = c + 1
 else x = 2
 end
 else if rand()<q then
 x = 1, c = c + 1
 end
 end
end
f = c/52
\end{verbatim}

 \end{noliste}

 \noindent \textbf{\underline{Exercice sans préparation}} \\
\\
 Soit $n \in \N^*$. On définit la fonction réelle $f_{n}$ par :
$\forall x \in \R$, $f_{n} (x) = x + 1 - \frac{ e^{x} }{ n }$.
 \begin{noliste}{1.}
 \setlength{\itemsep}{4mm}

 \item On étudie la fonction $f_{n}$ sur $\R_-$ : elle y est dérivable,
avec
 
\[
 f_{n} '(x) = 1 - \frac{ e^{x} }{ n } = \frac{ n - e^{x} }{ n } > 0 
\]

 puisque $x \leq 0$, donc $e^{x} \leq 1$, et enfin $ n - e^{x} \geq n -
1 \geq 0$ avec $n \geq 1$. On en déduit que $f_{n}$ est continue et
strictement croissante sur $\R_-$, avec par opérations élémentaires
 
\[
 \dlim{ x \rightarrow - \infty } f_{n} (x) = - \infty \ \ \text{ et } \
\ f_{n} (0 ) = 1 - \frac{ 1 }{ n } = \frac{ n - 1 }{ n } \geq 0 
\]

 On en déduit que $f_{n}$ réalise une bijection de $\R_-$ dans $ \left]
- \infty ; \frac{n-1}{n} \right]$, avec $0 \in \left] - \infty ;
\frac{n-1}{n} \right]$, donc il existe un unique $x_{n} \in \R_-$ tel
que $f_{n} (x_{n} ) = 0$. \\

 \item \begin{noliste}{a)}
 \setlength{\itemsep}{2mm}

 \item Pour comparer $x_{n}$ et $x_{n + 1}$, on va comparer leurs
images par $f_{n}$ : on sait que $f_{n} (x_{n} ) = 0$ et on a : 
 
\[
 f_{n} (x_{n + 1} ) = x_{n + 1} - 1 - \frac{ e^{ x_{n + 1} } }{ n } 
\]

 Or on sait que $f_{n + 1} (x_{n + 1} ) = x_{n + 1} + 1 - \frac{ e^{
x_{n + 1} } }{ n + 1 } = 0$ donc on obtient : 
 
\[
 f_{n} ( x_{n + 1} ) = \frac{ e^{ x_{n + 1} } }{ n + 1 } - \frac{ e^{
x_{n + 1} } }{ n } = e^{ x_{ n + 1 } } \left( \frac{ 1 }{ n + 1 } -
\frac{ 1 }{ n} \right) = - \frac{ e^{ x_{ n + 1 } } }{ n (n + 1) } < 0 
\]

 et on en déduit par croissance de $f_{n}$ sur $\R_-$ que :
 
\[
 f_{n} (x_{n + 1} ) < 0 = f_{n} (x_{n} ) \ \ \text{ donc } \ \ x_{n +
1} < x_{n} 
\]

 et ce pour tout $n \in \N$, donc la suite $(x_{n})$ est décroissante.
\\

 On cherche alors l'existence d'un minorant à la suite. 0 n'a aucune
chance d'être un minorant puisque la suite est négative, essayons $-1$.
Pour comparer $-1$ et $x_{n}$, on compare les images par $f_{n}$ : 
 
\[
 f_{n} (-1) = -1 + 1 - \frac{ e^{-1} }{ n } = - \frac{ 1 }{ n e } < 0 =
f_{n} ( x_{n} ) 
\]

 donc par croissance de $f_{n}$, pour tout $n \in \N$, $x_{n} \geq -1$
et la suite $(x_{n})$ est décroissante et minorée par $-1$ : elle
converge. \\

 \item On sait que pour tout $n \in \N$ : 
 
\[
 f_{n} (x_{n} ) = x_{n} + 1 - \frac{ e^{ x_{n} } }{ n } = 0 
\]

 donc en passant à la limite cette égalité, on obtient immédiatement : 
 
\[
 \ell + 1 - \frac{ e^{ \ell } }{ + \infty } = 0 \ \ \text{ donc } \ \
\ell + 1 = 0 \ \ \text{ et enfin } \ \ \ell = -1. 
\]

 \end{noliste}

 \item On pa donc $y_{n} = x_{n} + 1$, et d'après la définition de
$x_{n}$ on a : 
 
\[
 f_{n} (x_{n} ) = y_{n} - \frac{ e^{ x_{n} } }{ n } = 0 \ \ \text{ donc
} \ \ y_{n} = \frac{ e^{ x_{n} } }{ n }. 
\]

 Or le numérateur converge par composition vers $\frac{ 1 }{ e }$, qui
n'est pas nul donc c'est un équivalent de ce numérateur. Par quotient
on en déduit que : 
 
\[
 y_{n} \underset{ + \infty }{ \sim } \frac{ 1 }{ n e }. 
\]

 \end{noliste}
 \end{exercice}

 \newpage

 \begin{exercice}{\it (Exercice avec préparation)}~
 \begin{noliste}{1.}
 \setlength{\itemsep}{4mm}
 \item Une matrice est diagonalisable lorsque l'une des conditions
suivantes est vérifiées : \begin{noliste}{$\sbullet$}

 \item Elle est symétrique (condition suffisante).

 \item Elle est d'ordre $n$ et admet $n$ valeurs propres distinctes
(condition suffisante).

 \item Elle est d'ordre $n$ et la somme des dimensions de ses
sous-espaces propres vaut $n$ (condition nécessaire et suffisante).

 \item Il existe une base de $\mathcal{M}_{n,1} ( \R )$ constituée de
vecteurs propres de cette matrice (nécessaire et suffisante).

 \end{noliste}


 Soit $A$ la matrice de $\mathcal{M}_{3} (\R)$ définie par : $A =
\begin{smatrix}
0 & 1 & 0 \\
0 & 0 & 1 \\
-2 & 1 & 2 \\
\end{smatrix}
$.

 \item \begin{noliste}{a)}
 \setlength{\itemsep}{2mm}

 \item $A X = \lambda X$ possède des solutions non nulles si et
seulement si $A - \lambda I$ n'est pas inversible, et on en cherche une
réduite triangulaire : 
 \begin{eqnarray*}
 \begin{smatrix}
-\lambda & 1 & 0 \\
0 & - \lambda & 1 \\
-2 & 1 & 2 - \lambda \\
\end{smatrix}
 & \Longleftrightarrow L_{1} \leftrightarrow L_{3} & \begin{smatrix}
-2 & 1 & 2 - \lambda \\
0 & - \lambda & 1 \\
-\lambda & 1 & 0 \\
\end{smatrix}
\\
\\
 & \Longleftrightarrow L_{3} \leftrightarrow 2 L_{3} - \lambda L_{1} &
\begin{smatrix}
-2 & 1 & 2 - \lambda \\
0 & - \lambda & 1 \\
0 & 2 - \lambda & \lambda ( \lambda - 2 ) \\
\end{smatrix}
\\
\\
 & \Longleftrightarrow L_{2} \leftrightarrow L_{2} - L_{3} &
\begin{smatrix}
-2 & 1 & 2 - \lambda \\
0 & -2 & 1 - \lambda ( \lambda - 2 ) \\
0 & 2 - \lambda & \lambda ( \lambda - 2 ) \\
\end{smatrix}
\\
\\
 & \Longleftrightarrow L_{3} \leftrightarrow 2 L_{3} + (2- \lambda)
L_{2} & \begin{smatrix}
-2 & 1 & 2 - \lambda \\
0 & -2 & 1 - \lambda ( \lambda - 2 ) \\
0 & 0 & P \left(\Ev{\lambda}\right) \\
\end{smatrix}
 \end{eqnarray*}

 avec 
 
\[
 P \left(\Ev{ \lambda }\right) = 2 \lambda ( \lambda - 2 ) + (2 -
\lambda ) [ 1 - \lambda ( \lambda - 2 ) ] 
\]

 On simplifie alors $P \left(\Ev{ \lambda}\right)$, en espérant faire
apparaître le polynôme annoncé : 
 \begin{eqnarray*}
 P \left(\Ev{ \lambda }\right) & = & ( \lambda - 2 ) \left[ \
\rule{0cm}{0.4cm} 2 \lambda - 1 + \lambda ( \lambda - 2 ) \right] = (
\lambda - 2 ) \left[ \ \rule{0cm}{0.4cm} 2 \lambda - 1 + \lambda^{2} -
2 \lambda \right] \\
\\
 & = & ( \lambda - 2 ) ( \lambda^{2} - 1 ).
 \end{eqnarray*}

 On en déduit sans difficulté que la valeurs de $\lambda$
correspondantes sont $-1, 1$ et 2, qui sont donc les valeurs propres de
$A$. On cherche alors les sous-espaces propres associés en se servant
de la réduite triangulaire précédente : \\
\begin{noliste}{$\sbullet$}

 \item Pour $\lambda = -1$, avec $X = \begin{smatrix}
x \\
y \\
z \\
\end{smatrix}
$, on obtient : 
 \begin{eqnarray*}
 A X = - X & \Longleftrightarrow & ( A + I ) X = 0 \Longleftrightarrow
\left\{\begin{array}{r}
 -2x + y + 3 z = 0 \\
-2 y -2 z = 0 \\
0 = 0 \\
\end{array}
\right. \Longleftrightarrow \left\{\begin{array}{r}
 x = z \\
y = - z \\
\end{array}
\right. \\
\\
 & \Longleftrightarrow & X = z \begin{smatrix}
1 \\
-1 \\
1 \\
\end{smatrix}
 \end{eqnarray*}

 donc $E_{ -1 } ( A ) = \Vect { \begin{smatrix}
1 \\
-1 \\
1 \\
\end{smatrix}
}$. 

 \item Pour $\lambda = 1$, avec $X = \begin{smatrix}
x \\
y \\
z \\
\end{smatrix}
$, on obtient : 
 \begin{eqnarray*}
 A X = X & \Longleftrightarrow & ( A - I ) X = 0 \Longleftrightarrow
\left\{\begin{array}{r}
 -2x + y + z = 0 \\
-2 y + 2 z = 0 \\
0 = 0 \\
\end{array}
\right. \Longleftrightarrow \left\{\begin{array}{r}
 x = z \\
y = z \\
\end{array}
\right. \\
\\
 & \Longleftrightarrow & X = z \begin{smatrix}
1 \\
1 \\
1 \\
\end{smatrix}
 \end{eqnarray*}

 donc $E_{1} ( A ) = \Vect { \begin{smatrix}
1 \\
1 \\
1 \\
\end{smatrix}
}$.

 \item Pour $\lambda = 2$, avec $X = \begin{smatrix}
x \\
y \\
z \\
\end{smatrix}
$, on obtient : 
 \begin{eqnarray*}
 A X = 2 X & \Longleftrightarrow & ( A - 2I ) X = 0 \Longleftrightarrow
\left\{\begin{array}{r}
 -2x + y \ \ \ \ = 0 \\
-2 y + z = 0 \\
0 = 0 \\
\end{array}
\right. \Longleftrightarrow \left\{\begin{array}{r}
 x = 1/4 z \\
y = 1/2 z \\
\end{array}
\right. \\
\\
 & \Longleftrightarrow & X = z \begin{smatrix}
1/4 \\
1/2 \\
1 \\
\end{smatrix}
 \end{eqnarray*}

 donc $E_{2} ( A ) = \Vect { \begin{smatrix}
1/4 \\
1/2 \\
1 \\
\end{smatrix}
} = \Vect { \begin{smatrix}
1 \\
2 \\
4 \\
\end{smatrix}
} $.

 \end{noliste}

 \item $A$ est diagonalisable car elle est d'ordre 3 et admet 3 valeurs
propres distinctes. On en déduit que la concaténation des bases des
sous-espaces propres forme une base de vecteurs propres de $A$, donc en
posant : 
 
\[
 P = \begin{smatrix}
1 & 1 & 1 \\
-1 & 1 & 2 \\
1 & 1 & 4 \\
\end{smatrix}
\ \ \ \text{ et } \ \ \ D = \begin{smatrix}
-1 & 0 & 0 \\
0 & 1 & 0 \\
0 & 0 & 2 \\
\end{smatrix}
\]

 $P$ est inversible et la formule de changement de base donne $A = P D
P^{-1}$. \\

 \end{noliste}

 \item Soit $(x_{n})_{ n \in \N}$ une suite réelle définie par : pour
tout $n \in \N$, $x_{n + 3} = 2 x_{n + 2} + x_{n + 1} - 2 x_{n}$. \\

 On pose pour tout $n \in \N$ : $X_{n} = \begin{smatrix}
x_{n} \\
x_{n + 1} \\
x_{n + 2} \\
\end{smatrix}
$ et $Y_{n} = P^{-1} X_{n}$. \begin{noliste}{a)}
 \setlength{\itemsep}{2mm}

 \item On obtient sans difficulté que :
 
\[
 X_{ n + 1 } = A X_{n}. 
\]

 \item On en déduit que 
 
\[
 Y_{ n + 1 } = P^{-1 } A X_{n} = P^{-1} P D P^{-1} X_{n} = D P^{-1}
X_{n} = D Y_{n} 
\]

 donc par une récurrence immédiate ou par itération de la relation,
 
\[
 Y_{n} = D^{n} Y_{0}. 
\]

 \item On va calculer $x_{n}$ pour tout $n$ : on commence par calculer
$Y_{n}$ : il faut pour cela calculer $Y_{0} = P^{-1} X_{0}$, et donc
$P^{-1}$. Par méthode de Gauss-Jordan on obtient sans difficulté : 
 
\[
 P^{-1} = \frac{1}{6} \begin{smatrix}
2 & - 3 & 1 \\
6 & 3 & -3 \\
-2 & 0 & 2 \\
\end{smatrix}
\ \ \text{ donc } \ \ Y_{0} = \frac{1}{6} \begin{smatrix}
2 x_{0} - 3 x_{1} + x_{2} \\
6 x_{0} + 3 x_{1} - 3 x_{2} \\
-2 x_{0} + 2 x_{2} \\
\end{smatrix}
\]

 puis
 
\[
 Y_{n} = \begin{smatrix}
(-1)^{n} & 0 & 0 \\
0 & 1 & 0 \\
0 & 0 & 2^{n} \\
\end{smatrix}
Y_{0} = \frac{1}{6} \begin{smatrix}
(-1)^{n} [ 2x_{0} - 3 x_{1} + x_{2} ] \\
3 ( 2 x_{0} + x_{1} - x_{2} ) \\
2^{n + 1} ( x_{2} - x_{0} ) \\
\end{smatrix}
\]

 et enfin $x_{n}$ est la première ligne de $X_{n} = P Y_{n}$, donc : 
 
\[
 x_{n} = \frac{1}{6} \left[ (-1)^{n} [ 2x_{0} - 3 x_{1} + x_{2} ] + 3 (
2 x_{0} + x_{1} - x_{2} ) + 2^{n + 1} ( x_{2} - x_{0} )
\rule{0cm}{0.4cm} \right] 
\]

 Pour que cette suite converge, il faut que les coefficients devant les
suites géométriques divergentes soient nuls. En effet si $x_{2} - x_{0}
\neq 0$, on montre que : 
 
\[
 x_{n} = \frac{1}{6} (x_{2} - x_{0} ) 2^{ n + 1} \left( 1 + o (1)
\right) \underset{ + \infty }{ \sim } (x_{2} - x_{0} ) 2^{n + 1} 
\]

 diverge, donc il faut que $x_{2} - x_{0} = 0$. Ensuite puisque l'autre
partie est constante, il faut que $[ (-1)^{n} ( 2 x_{0} - 3 x_{1} +
x_{2} ) ]$ converge, ce qui n'est possible que si la constante est
nulle puisque $[ (-1)^{n}]$ diverge. On en déduit que $(x_{n})_{ n \in
\N}$ converge si et seulement si : 
 
\[
 \left\{\begin{array}{c}
 2 x_{0} + - 3 x_{1} + x_{2} = 0 \\
x_{0} = x_{2} \\
\end{array}
\right. \Longleftrightarrow x_{0} = x_{1} = x_{2}. 
\]

 Enfin pour que la série de terme général $(x_{n})$ converge, il faut
que la suite $x_{n}$ converge (vers 0) donc que les conditions
précédentes soient vérifiées. On obtient alors : 
 
\[
 \forall n \in \N, \ x_{n} = \frac{ 3 }{ 6 } ( 2x_{0} + x_{1} - x_{2} )
= x_{0} 
\]

 est une suite constante, et la série de terme général $(x_{n})$ ne
peut converger que si $x_{n} = x_{0} = 0$, donc si et seulement si : 
 
\[
 x_{0} = x_{1} = x_{2} = 0. 
\]

 \end{noliste}

 \item On pose $B = \begin{smatrix}
5 & 0 & -2 \\
4 & 3 & -4 \\
8 & 0 & -5 \\
\end{smatrix}
$ et pour tout $(a,b) \in \R^{2}$, $M( a, b ) = \begin{smatrix}
5b & a & -2b \\
4 b & 3b & a - 4 b \\
-2 a + 8 b & a & 2a - 5 b \\
\end{smatrix}
$.\begin{noliste}{a)}
 \setlength{\itemsep}{2mm}

 \item On a montré que les vecteurs propres de $A$ sont les vecteurs de
la forme : 
 
\[
 \lambda \begin{smatrix}
1 \\
-1 \\
1 \\
\end{smatrix}
\ \, \ \ \lambda \begin{smatrix}
1 \\
1 \\
1 \\
\end{smatrix}
\ \, \ \ \lambda \begin{smatrix}
1 \\
2 \\
4 \\
\end{smatrix}
\]

 et on les teste tous : 
 
\[
 B \lambda \begin{smatrix}
1 \\
-1 \\
1 \\
\end{smatrix}
 = \lambda \begin{smatrix}
3 \\
-3 \\
3 \\
\end{smatrix}
 = 3 \lambda \begin{smatrix}
1 \\
-1 \\
1 \\
\end{smatrix}
\]

 donc les vecteurs propres de $A$ associé à la valeur propre $-1$ sont
vecteurs propres de $B$ associés à la valeur propre 3. De même,
 
\[
 B \lambda \begin{smatrix}
1 \\
1 \\
1 \\
\end{smatrix}
 = \lambda \begin{smatrix}
3 \\
3 \\
3 \\
\end{smatrix}
 = 3 \lambda \begin{smatrix}
1 \\
1 \\
1 \\
\end{smatrix}
\]

 donc les vecteurs propres de $A$ associé à la valeur propre $1$ sont
vecteurs propres de $B$ associés à la valeur propre 3. Enfin
 
\[
 B \lambda \begin{smatrix}
1 \\
2 \\
4 \\
\end{smatrix}
 = \lambda \begin{smatrix}
-3 \\
-6 \\
-12 \\
\end{smatrix}
 = -3 \lambda \begin{smatrix}
1 \\
2 \\
4 \\
\end{smatrix}
\]

 donc les vecteurs propres de $A$ associé à la valeur propre $2$ sont
vecteurs propres de $B$ associés à la valeur propre $-3$. \\

 La réciproque n'est pas vraie, car on a ici testé tous les vecteurs
propres de $A$. Or, comme on a trouvé deux fois la même valeur propre
pour $B$, on peut construire par combinaison d'autres vecteurs propres
de $B$. Par exemple,
 
\[
 \begin{smatrix}
1 \\
-1 \\
1 \\
\end{smatrix}
 + \begin{smatrix}
1 \\
1 \\
1 \\
\end{smatrix}
 = \begin{smatrix}
2 \\
0 \\
2 \\
\end{smatrix}
\]

 est vecteur propre de $B$ (associé à la valeur propre 3) mais pas
vecteur propre de $A$ : 
 
\[
 B \begin{smatrix}
2 \\
0 \\
2 \\
\end{smatrix}
 = \begin{smatrix}
6 \\
0 \\
6 \\
\end{smatrix}
 = 3 \begin{smatrix}
2 \\
0 \\
2 \\
\end{smatrix}
\ \ \ \text{ mais } \ \ \ A \begin{smatrix}
2 \\
0 \\
2 \\
\end{smatrix}
 = \begin{smatrix}
0 \\
2 \\
0 \\
\end{smatrix}
\]

 qui n'est pas colinéaire à $\begin{smatrix}
2 \\
0 \\
2 \\
\end{smatrix}
$. \\

 \item On en déduit que la base de vecteurs propres de $A$ précédente
est aussi base de vecteurs propres de $B$ : avec la même matrice $P$
que précédemment et avec $D' = \begin{smatrix}
3 & 0 & 0 \\
0 & 3 & 0 \\
0 & 0 & -3 \\
\end{smatrix}
$, la formule de changement de base donne $B = P D' P^{-1}$ puis : 
 
\[
 M (a,b) = a A + b B = a P D P^{-1} + b P D' P^{-1} = P \left(\Ev{ a D
+ b D' }\right) P^{-1} = P \begin{smatrix}
3b - a & 0 & 0 \\
0 & a + 3 b & 0 \\
0 & 0 & 2 a - 3 b \\
\end{smatrix}
P^{-1} 
\]

 On en déduit que $M(a,b)$ est diagonalisable et que ses valeurs
propres sont $3b-a, a + 3b$ et $2a-3b$. \\

 \item On en déduit que 
 
\[
 M( a,b)^{n} = P \begin{smatrix}
(3b - a)^{n} & 0 & 0 \\
0 & (a + 3 b)^{n} & 0 \\
0 & 0 & (2 a - 3 b)^{n} \\
\end{smatrix}
P^{-1} 
\]

 qui aura pour limite $P D_{\infty} P^{-1}$, où $D_{ \infty}$ est la
limite de $D^{n}$, si elle existe (et si elle n'existe pas,
$M(a,b)^{n}$ n'a pas de limite. Comme $P$ et $P^{-1}$ sont inversibles,
ce produit est nul si et seulement si $D_{ \infty } = 0$, et donc si et
seulement si les trois suites géométriques convergent vers 0, donc si
et seulement si : 
 
\[
 -1 < 3 b - a < 0 \ \ \text{ et } \ \ -1 < a + 3 b < 1 \ \ \text{ et }
\ \ -1 < 2a - 3 b < 1. 
\]

 \end{noliste}

 \end{noliste}

 \indent

 \noindent \textbf{\underline{Exercice sans préparation}} \\
\\
 Soit $p \in \ ]0;1[$. Soit $(X_{n})_{ n \in \N^* }$ une suite de
variables aléatoires définies sur un espace probabilisé $(\Omega,
\mathcal{A}, P )$ indépendantes et de même loi donnée par : 
 
\[
 \forall n \in \N^*, \ P \Ev{X_{n} = -1} = p \ \ \text{ et } \ \ P
\Ev{X_{n} = 1} = 1 - p. 
\]

 \noindent On pose pour tout $n \in \N^*$, $Z_{n} = \prod\limits_{i =
1}{n} X_{i}$.
 \begin{noliste}{1.}
 \setlength{\itemsep}{4mm}

 \item Les $(X_{i})$ sont indépendantes donc : 
 
\[
 E ( Z_{n} ) = E \left( \prod\limits_{i = 1}{n} X_{i} \right) =
\prod\limits_{i = 1}{n} E ( X_{i} ) = \prod\limits_{i = 1}{n} \left( 1
\times (1-p) - 1 \times p \right) = ( 1 - 2 p)^{n}. 
\]

 Or on sait que
 
\[
 0 < p < 1 \ \ \text{ donc } \ \ -2 < -2p < 0 \ \ \text{ et enfin } \ \
-1 < 1 - 2p < 1 
\]

 et donc $\E( Z_{n})$ converge vers 0. \\

 \item $Z_{n}$ est un produit de nombres qui valent 1 ou $-1$, elle
vaut donc 1 et ou $-1$ : 
 
\[
 Z_{n} ( \Omega ) = \{ -1 ; 1 \} 
\]

 De plus $\Ev{Z_{n} = 1}$ signifie que le nombre de valeurs $X_{i}$ qui
valent $-1$ est pair, donc en posant $S$ la variable égale au nombre de
$X_{i}$ qui valent $-1$ (qui suit une loi binomiale de paramètres $n$
et $p$), on a : 
 
\[
 \Ev{Z_{n} = 1 } = ( S \text{ est paire } ) = \dcup{ k = 0 }{ \lfloor
\frac{n}{2} \rfloor } \Ev{ S = 2 k }. 
\]

 On en déduit par incompatibilité que : 
 
\[
 P \Ev{ Z_{n} = 1 } = \Sum{k = 0}{ \lfloor \frac{ n }{ 2 } \rfloor }
\binom{n}{2k} p^{2k} (1-p)^{n-2k}. 
\]

 Cependant on ne sait pas calculer cette somme. On va alors trouver la
valeur des probabilités de manière beaucoup plus astucieuse, en se
servant de l'espérance précédemment calculée. Posons $q$ la probabilité
de $\Ev{Z_{n} = 1}$, alors l'autre probabilité vaut $1-q$, et on a :
 
\[
 E ( Z_{n} ) = 1 \times q - 1 \times (1-2q) = 2 q -1 
\]

 ce qui donne : 
 
\[
 2 q - 1 = ( 1 - 2 p)^{n} \ \ \text{ et } \ \ q = \frac{ (1-2p)^{n} + 1
}{ 2 } 
\]

 donc on obtient finalement : 
 
\[
 P \Ev{ Z_{n} = 1 } = \frac{ (1-2p)^{n} + 1 }{ 2 } \ \ \text{ et } \ \
P \Ev{ Z_{n} = -1 } = 1 - q = \frac{ 1 - (1-2p)^{n} }{ 2 }. 
\]

 \item On connaît les lois de $Z_{1}$ et $Z_{2}$, on cherche la loi du
couple : 
 
\[
 \Ev{Z_{1} = 1 } \cap \Ev{Z_{2} = 1 } = \Ev{X_{1} = 1 } \cap \Ev{X_{2}
= 1 } \ \ \text{ donc } \ \ P [ \ \Ev{Z_{1} = 1 } \cap \Ev{Z_{2} = 1 }
= P \Ev{X_{1} = 1 } \cap \Ev{X_{2} = 1 } ] = (1-p)^{2}. 
\]

 
\[
 \Ev{Z_{1} = 1 } \cap \Ev{Z_{2} = -1 } = \Ev{X_{1} = 1 } \cap \Ev{X_{2}
= -1 } \ \ \text{ donc } \ \ P [ \ \Ev{Z_{1} = 1 } \cap \Ev{Z_{2} = -1
} = P \Ev{X_{1} = 1 } \cap \Ev{X_{2} = -1 } ] = p (1-p). 
\]

 
\[
 \Ev{Z_{1} = -1 } \cap \Ev{Z_{2} = 1 } = \Ev{X_{1} = -1 } \cap
\Ev{X_{2} = -1 } \ \ \text{ donc } \ \ P [ \ \Ev{Z_{1} = -1 } \cap
\Ev{Z_{2} = 1 } = P \Ev{X_{1} = -1 } \cap \Ev{X_{2} = -1 } ] = p^{2}. 
\]

 
\[
 \Ev{Z_{1} = -1 } \cap \Ev{Z_{2} = -1 } = \Ev{X_{1} = -1 } \cap
\Ev{X_{2} = 1 } \ \ \text{ donc } \ \ P [ \ \Ev{Z_{1} = -1 } \cap
\Ev{Z_{2} = -1 } = P \Ev{X_{1} = -1 } \cap \Ev{X_{2} = 1 } ] = p (1-p).
\]

 Enfin $Z_{1}$ et $Z_{2}$ sont indépendantes si on a : 
 
\[
 \left\{\begin{array}{c}
 (1-p)^{2} = (1-p) \times \frac{ (1-2p)^{2} + 1 }{ 2 } \\
p (1-p) = (1-p) \times \frac{ 1 - (1-2p)^{2} }{ 2 } \\
p^{2} = p \times \frac{ (1-2p)^{2} + 1 }{ 2 } \\
p (1-p) = p \times \frac{ 1 - (1-2p)^{2} }{ 2 } \\
\end{array}
\right. \Longleftrightarrow \left\{\begin{array}{c}
 2 (1-p) = (1-2p)^{2} + 1 \\
2 p = 1 - (1-2p)^{2} \\
2p = (1-2p)^{2} + 1 \\
2 (1-p) = 1 - (1-2p)^{2} \\
\end{array}
\right. 
\]

 Ceci impose que $p = 1-p$ donc que $p = 1/2$, on vérifie alors les
valeurs : 
 
\[
 (1-2 \times 1/2 )^{2} + 1 = 0^{2} + 1 = 1 = 2 p = 2 (1-p) \ \ \ \text{
et } \ \ \ 1 - (1 - 2 \times 1/2)^{2} = 1 - 0^{2} = 1 = 2p = 2 (1-p) 
\]

 et les quatre égalités sont bien vérifiées. On en déduit que $Z_{1}$
et $Z_{2}$ sont indépendantes si et seulement si $p = \frac{1}{2}$.

 \end{noliste}
 \end{exercice}

 \newpage

 \begin{exercice}{\it (Exercice avec préparation)}~
 \begin{noliste}{1.}
 \setlength{\itemsep}{4mm}
 \item Un point critique de $f$ est un couple $(a,b)$ tel que
$\partial_{1} (f) (a,b) = \partial_{2} (f) (a,b) = 0$. \\

 De plus on sait que lorsque la matrice Hessienne au point $(a,b)$,
$\nabla^{2} (f) (a,b) = \begin{smatrix}
\partial_{1,1}{2} (f) (a,b) & \partial_{1,2}{2} (f) (a,b) \\
\partial_{2,1}{2} (f) (a,b) & \partial_{2,2}{2} (f) (a,b) \\
\end{smatrix}
$ admet des valeurs propres de même signe (strict), $f$ admet un
extremum local au point $(a,b)$. \\

 Soit $X$ une variable aléatoire discrète finie définie sur un espace
probabilisé $(\Omega, \mathcal{A}, P)$. \\
 On pose pour tout $n \in \N^*$ : $X ( \Omega ) = \{ x_{1}, \dots,
x_{n} \} \subset \R$ et on suppose que $\forall i \in \llb 1 ; n \rrb$,
$P \Ev{X = x_{i}} \neq 0$. \\

 On définit l'entropie de $X$ par : $ H(X) = - \frac{ 1 }{ \ln 2 }
\Sum{i = 1}{n} P \Ev{X = x_{i}} \ln \big( \Prob\left(\Ev{\Ev{X =
x_{i}}}\right) \big)$.

 \item Soient $x_{1}, x_{2}, x_{3}, x_{4}$ quatre réels distincts. On
considère un jeu de 32 cartes dont on tire une carte au hasard. Soit
$X$ la variable aléatoire prenant les valeurs suivantes :
\begin{noliste}{$\sbullet$}

 \item $x_{1}$ si la carte tirée est rouge (coeur ou carreau),

 \item $x_{2}$ si la carte tirée est un pique,

 \item $x_{3}$ si la carte tirée est le valet, la dame, le roi ou l'as
de trèfle,

 \item $x_{4}$ dans les autres cas.

 \end{noliste}

 On tire une carte notée $C$ et un enfant décide de déterminer la
valeur $X(C)$ en posant dans l'ordre les questions suivantes auxquelles
il lui est répondu par "oui" ou par "non". LA carte $C$ est-elle rouge
? La carte $C$ est-elle un pique ? La carte $C$ est-elle le valet, la
dame, le roi ou l'as de trèfle ? \\
 Soit $N$ la variable aléatoire égale au nombre de questions posées
(l'enfant cesse de poser des questions dès qu'il a obtenu une réponse
"oui"). \begin{noliste}{a)}
 \setlength{\itemsep}{2mm}

 \item On obtient sans difficulté que $X ( \Omega ) = \{ x_{1} ; x_{2}
; x_{3} ; x_{4} \}$ et : 
 
\[
P \Ev{ X = x_{1} } = \frac{1}{2} \ \, \ \ P \left(\Ev{ X = x_{2} =
    \frac{1}{4}}\right) \ \, \ \ P \Ev{ X = x_{3} } = P \Ev{ X = x_{4} } =
  \frac{ 1 }{ 8 }.
\] 

%%% LALALA

On peut alors calculer :
 \begin{eqnarray*}
 H ( X ) & = & - \frac{ 1 }{ \ln 2 } [ 1/2 \ln (1/2) + 1/4 \ln (1/4) +
2 \times 1/8 \ln (1/8) ] = - \frac{ 1 }{ 4 \ln 2 } [ - 2 \ln 2 - \ln 4
- \ln 8 ] \\
\\
 & = & \frac{ 2 \ln 2 + \ln 2^{2} + \ln 2^{3} }{ 4 \ln 2 } = \frac{ (2
+ 2 + 3) \ln 2 }{ 4 \ln 2 } = \frac{ 7 }{ 4 }. 
 \end{eqnarray*}

 \item Il faut au minimum 1 question, et au maximum trois questions
pour déterminer $X ( C )$. De plus on remarque que : 
 
\[
 \Ev{N = 1 } = \Ev{ X = x_{1} } \ \, \ \ \Ev{ N = 2 } = \Ev{ X = x_{2}}
\ \, \ \ \Ev{ N = 3 } = \Ev{ X = x_{3} } \cup ( \Ev{X = X_{4} } 
\]

 donc : 


%%% LALALA

\[
 P \Ev{ N = 1 } = \frac{ 1 }{ 2 } \ \ \text{ et } \ \ P \Ev{ N = 2} = P
\Ev{ N = 3 } = \frac{ 1 }{ 4 }. 
\]

 Enfin on calcule : 
 
\[
 E ( N ) = \frac{ 1 }{ 2 } + \frac{ 2 }{ 4 } + \frac{ 3 }{ 4 } = \frac{
7 }{ 4 } = H ( X ). 
\]

 \end{noliste}

 \item Soit $f$ la fonction définie sur $\R^{2}$ à valeurs réelles
telle que : $f(x,y) = x \ln x + y \ln y + (1-x-y) \ln (1-x-y)$.
\begin{noliste}{a)}
 \setlength{\itemsep}{2mm}

 \item $f$ est bien définie si et seulement si les $\ln$ le sont, il
faut donc : 
 
\[
 x > 0 \ \, \ \ y > 0 \ \ \text{ et } \ \ 1 - x - y > 0
\Longleftrightarrow x + y < 1 \Longleftrightarrow y < 1 - x. 
\]

 On trace alors dans un repère orthonormé la droite d'équation
 $y = 1-x$, et on hachure la partie du plan au-dessus de l'axe
 des abscisses ($y > 0$), à droite de l'axe des ordonnées ($x >
 0$) et en-dessous de la droite tracée ($y < 1 - x $).

 \item $f$ est de classe $C^{2}$ sur son en semble de définition et : 
 
\[
 \partial_{1} (f) (x,y) = \ln x + 1 - \ln (1-x-y) - 1 = \ln x - \ln
(1-x-y) \ \ \text{ et } \ \ \partial_{2} (f) (x,y) = \ln y - \ln
(1-x-y). 
\]

 On résout alors le système, avec $\ln$ bijective : 
 \begin{eqnarray*}
 \left\{\begin{array}{c}
 \ln x - \ln (1-x-y) = 0 \\
\ln y - \ln (1-x-y) = 0 \\
\end{array}
\right. & \Longleftrightarrow & \left\{\begin{array}{c}
 \ln x = \ln (1-x-y) \\
\ln y - \ln (1-x-y) = 0 \\
\end{array}
\right. \Longleftrightarrow \left\{\begin{array}{c}
 x = 1-x-y \\
\ln y - \ln (1-x-y) = 0 \\
\end{array}
\right. \\
\\
 & \Longleftrightarrow & \left\{\begin{array}{c}
 y = 1- 2 x \\
\ln (1-2x) - \ln (1-x-1 + 2x) = 0 \\
\end{array}
\right. \Longleftrightarrow \left\{\begin{array}{c}
 y = 1- 2 x \\
\ln (1-2x) = \ln (x) \\
\end{array}
\right. \\
\\
 & \Longleftrightarrow & \left\{\begin{array}{c}
 y = 1- 2 x \\
1-2x = x \\
\end{array}
\right. \Longleftrightarrow \left\{\begin{array}{c}
 y = 1- 2 x \\
3x = 1 \\
\end{array}
\right. \Longleftrightarrow \left\{\begin{array}{c}
 y = 1/3 \\
x = 1/3 \\
\end{array}
\right.
 \end{eqnarray*}

 donc le seul point critique de $f$ est $(1/3 ; 1/3)$. On calcule alors
les dérivées partielles secondes puis la matrice Hessienne au point
$(1/3 ; 1/3)$ : 
 
\[
 \partial_{1,1}{2} (f ) ( x,y) = \frac{1}{x} + \frac{ 1 }{ 1 - x - y }
\ \, \ \ \partial_{1,2}{2} (f) (x,y) = \partial_{2,1}{2} (f) (x,y) =
\frac{ 1 }{ 1 - x - y } \ \, \ \ \partial_{2,2}{2} (f ) ( x,y) =
\frac{1}{y} + \frac{ 1 }{ 1 - x - y } 
\]

 puis au point $(1/3 ; 1/3)$ : 
 
\[
 \partial_{1,1}{2} (f ) ( 1/3,1/3) = 3 + 3 = 6 \ \, \ \
\partial_{1,2}{2} (f) (1/3,1/3) = \partial_{2,1}{2} (f) (1/3,1/3) = 3 \
\, \ \ \partial_{2,2}{2} (f ) ( 1/3,1/3) = 6 
\]

 et on cherche les valeurs propres de la Hessienne : 
 \begin{eqnarray*}
 \begin{smatrix}
6 - \lambda & 3 \\
3 & 6 - \lambda \\
\end{smatrix}
 & \Longleftrightarrow L_{1} \leftrightarrow L_{2} & \begin{smatrix}
3 & 6 - \lambda \\
6 - \lambda & 3 \\
\end{smatrix}
\Longleftrightarrow L_{2} \leftarrow 3 L_{2} - ( 6 - \lambda) L_{1}
\begin{smatrix}
3 & 6 - \lambda \\
0 & 9 - (6 - \lambda)^{2} \\
\end{smatrix}
 \end{eqnarray*}

 donc les valeurs propres sont les solutions de l'équation : 
 
\[
 9 - ( 6 - \lambda)^{2} = 0 \Longleftrightarrow ( 3 + 6 - \lambda) ( 3
- 6 + \lambda ) = 0 \Longleftrightarrow ( 9 - \lambda ) ( \lambda - 3 )
= 0 
\]

 donc 3 et 9, qui sont strictement positives donc $f$ admet un minimum
local au point $(1/3,1/3)$. \\

 \item On calcule sans difficulté : 
 
\[
 H ( X ) = - \frac{ 1 }{ \ln 2 } [ p_{1} \ln (p_{1} ) + p_{2} \ln
(p_{2} ) + p_{3} \ln (p_{3} ) ] = - \frac{ 1 }{ \ln 2 } f( p_{1}, p_{2}
) 
\]

 car on sait que $p_{1} + p_{2} + p_{3} = 1$, donc $p_{3} = 1 - p_{1} -
p_{2}$. On en déduit avec $ - \frac{ 1 }{ \ln 2 } $ que $f$ et $H(X)$
atteignent leurs extrema locaux aux mêmes points, mais que les natures
sont inversées. \\

 Finalement $H(X)$ admet un maximum local au point $p_{1} = p_{2} =
\frac{1}{3}$, donc $p_ 3 = 1 - p_{1} - p_{2} = \frac{1}{3}$. 

 \end{noliste}
 
 \end{noliste}

 \indent

 \noindent \textbf{\underline{Exercice sans préparation}} \\
\\
 On rappelle l'identité remarquable $a^{3} + b^{3} = (a + b) (a^{2} - a
b + b^{2})$. \\
 \\
 On vérifie à l'aide de la commutativité de $A$ et $B$ que l'identité
remarquable est toujours valable sur les matrices : 
 
\[
 (A + B) ( A^{2} - A B + B^{2} ) = A^{3} - A^{2} B + A B^{2} + B A^{2}
- B A B + B^{3} = A^{3} - A^{2} B + A B^{2} + A^{2} B - A B^{2} + B^{3}
= A^{3} + B^{3} 
\]

 \noindent et puisque $A^{3} = 0$, on en déduit que
 
\[
 B^{3} = A^{3} + B^{3} = (A + B ) ( A^{2} - A B + B^{2} ). 
\]

 \noindent De plus puisque $B$ est inversible on peut multiplier par
$B^{-1}$ (on le fait trois fois, à droite obligatoirement pour obtenir
$(A + B) \times...$) : 
 
\[
 B^{3} (B^{-1} )^{3} = (A + B ) (A^{2} - A B + B^{2} ) (B^{-1} )^{3} \
\ \text{ et enfin } \ \ ( A + B ) [ (A^{2} - A B + B^{2} ) (B^{-1})^{3}
] = I 
\]

 \noindent donc $A + B$ est inversible, et son inverse est la matrice
qui la multiplie pour donner $I$.

 \end{exercice}

 \newpage

 \begin{exercice}{\it (Exercice avec préparation)}~
 \begin{noliste}{1.}
 \setlength{\itemsep}{4mm}
 \item Dans le cas général, cette intégrale est dite convergente si
l'intégrale $\dint{a}{x} f(t) \ dt $ converge lorsque $x$ tend vers $ +
\infty$. \\

 Si la fonction intégrée est positive, on a alors 4 critères possibles
pour obtenir la convergence : \\
\begin{noliste}{$\sbullet$}

 \item Si l'intégrale partielle est bornée (condition nécessaire et
suffisante). 

 \item Si la fonction intégrée est majorée par une fonction dont
l'intégrale converge (condition suffisante).

 \item Si la fonction intégrée est négligeable en l'infini devant une
fonction dont l'intégrale converge (condition suffisante).

 \item Si la fonction intégrée est équivalente en l'infini à une
fonction dont l'intégrale converge (condition nécessaire et
suffisante).

 \end{noliste}

 \item Soit $x \in \R_+^*$. \begin{noliste}{a)}
 \setlength{\itemsep}{2mm}

 \item Cette intégrale n'est généralisée qu'en $ + \infty$, et comme $
\frac{ 1 }{ t }$ tend vers 0 on a : 
 
\[
 e^{ -t }{ x + t } = \frac{ e^{-t} }{ t \left( 1 + \frac{ x }{ t }
\right) } \underset{ + \infty }{ \sim } \frac{ e^{ - t } }{ t } = o_{ +
\infty } ( e^{ - t } ) 
\]

 Or les deux fonctions sont à termes positifs, et l'intégrale de 0 à $
+ \infty$ de $e^{-t}$ converge (densité de la loi exponentielle), donc
par théorème de comparaison l'intégrale de la question converge. On
pose alors 
 
\[
 f(x) = \dint{0}{ + \infty } \frac{ e^{ - t } }{ x + t } \ dt. 
\]

 \item Question difficile par manque d'habitude. Cette fonction n'a
rien à voir avec les fonctions intégrales habituellement étudiées, car
la dépendance en $x$ ne se trouve pas sur la borne : on ne peut pas
poser une primitive de la fonction à l'intérieur fixée (elle devrait
dépendre de $x$). \\

 Il faut alors s'inspirer les suites intégrales, qui, sur le même
modère que $f$, dépendent en général de $n$ avec le $n$ à l'intérieur
de l'intégrale et pas sur la borne, et on revient à la définition de la
monotonie : \\

 Soient $x$ et $y$ deux réels strictement positifs, tels que $x < y$.
On obtient alors pour tout $t \in \R_+ $ : 
 
\[
 0 < x + t < y + t \ \ \text{ donc (inverse strictement décroissante
sur } \R +^*) \ \ \frac{ 1 }{ y + t } < \frac{ 1 }{ x + t } 
\]

 On multiplie par $e^{-t} > 0$ et on intègre avec des bornes dans
l'ordre croissant, on obtient : 
 
\[
 x < y \Longrightarrow f(y) \leq f(x) 
\]

 et la fonction $f$ est décroissante.

 \end{noliste}

 \item Soit $g$ et $h$ les fonctions définies sur $\R_+^*$ à valeurs
réelles telles que : 
 
\[
 g(x) = \dint{0}{1} \frac{ e^{ -t } - 1 }{ x + t } \ dt \ \ \ \text{ et
} \ \ \ h(x) = \dint{1}{+ \infty} \frac{ e^{ -t } }{ x + t } \ dt. 
\]

 \begin{noliste}{a)}
 \setlength{\itemsep}{2mm}

 \item $\varphi$ est continue sur $] 0 ; 1]$ par opérations
élémentaires, et en 0 on a $-t$ qui tend vers 0 donc par DL : 
 
\[
 \varphi (t) = \frac{ e^{ -t } - 1 }{ t } = \frac{ 1 - t + o(t) - 1 }{
t } = -1 + o(1) \xrightarrow[ t \rightarrow 0]{} -1 = \varphi (0) 
\]

 donc $\varphi$ est continue en 0, et donc sur $[0 ; 1]$. \\

 \item Pour tout $x > 0$ et $t \in \ ] 0 ; 1]$, on a : 
 
\[
 x + t > t > 0 \ \ \text{ donc } \ \ 0 < \frac{1}{ x + t } < \frac{ 1
}{ t } 
\]

 puis en multipliant par $e^{ - t } - 1 < 0 $ (avec $-t < 0$ donc $e^{
- t } < 1 $) : 
 
\[
 \varphi (t) \leq \frac{ e^{ - t } - 1 }{ x + t } \leq 0 
\]

 et cette inégalité reste vraie en $t = 0$, car $\frac{ e^{0} - 1 }{ x
+ 0 } = 0$, donc en intégrant avec des bornes dans l'ordre croissant
(avec des intégrales qui convergent car aucune n'est généralisée) on
obtient : 
 
\[
 \dint{0}{1} \varphi (t) \ dt \leq g(x) \leq 0 
\]

 et la fonction $g$ est bien bornée (l'intégrale à gauche est une
constante, elle ne dépend pas de $x$). \\

 \item Avec une preuve strictement identique à la fonction $f$ (sauf
que cette fois-ci la fonction $h$ est définie en $x = 0$, et qu'on
intègre entre 1 et $ + \infty$), la fonction $h$ est décroissante et
vérifie : 
 
\[
 h(x) \leq h(0) = \dint{1}{+ \infty} \frac{ e^{ - t } }{ t } \ dt 
\]

 avec l'intégrale de droite qui est constante (ne dépend pas de $x$)
donc $h$ est majorée. \\

 De plus $h$ est minorée par 0 car c'est l'itnégrale d'une fonction
positive avec des bornes dans l'ordre croissant, elle est donc
positive. On en déduit finalement que $h$ est bornée. \\

 \item On sépare l'intégrale avec la relation de Chasles pour faire
apparaître $h(x)$ : 
 
\[
 f(x) = \dint{0}{1} \frac{ e^{ - t } }{ x + t } \ dt + \dint{1}{+
\infty} \frac{ e^{ -t } }{ x + t } \ dt = \dint{0}{1} \frac{ e^{ - t }
}{ x + t } \ dt + h(x). 
\]

 Ensuite on fait apparaître $g(x)$ : 
 
\[
 f(x) = \dint{0}{1} \frac{ e^{-t} - 1 }{ x + t } \ dt + \dint{0}{1}
\frac{ 1 }{ x + t } \ dt + h(x) = g(x) + h(x) + \dint{0}{1} \frac{ 1 }{
x + t } \ dt. 
\]

 Enfin on calcule l'intégrale restante : 
 
\[
 f(x) = g(x) + h(x) + \left[ \ \rule{0cm}{0.4cm} \ln (x + t)
\right]_{0}{1} = g(x) + h(x) + \ln (1 + x ) - \ln (x). 
\]

 Au voisinage de 0, les fonctions $g$ et $h$ sont bornées et $\ln (1 +
x)$ tend vers 0 donc le terme prépondérant est $- \ln x$ qui tend vers
$-\infty$ : 
 
\[
 f(x) = - \ln x + o ( \ln x ) \underset{ x \rightarrow 0^+ }{ \sim } -
\ln x. 
\]

 \end{noliste}

 \item Cet encadrement s'obtient immédiatement en mettant les deux
fractions au même dénominateur. On va alors s'en servir pour faire
apparaître un équivalent de $f(x)$, donc en obtenant un encadrement
concernant $f(x)$. \\

 On multiplie donc par $e^{-t} > 0$ et on intègre avec des bornes dans
l'ordre croissant (toutes les intégrales sont convergentes) : 
 
\[
 0 \leq \frac{1}{x} \dint{0}{ + \infty} e^{ - t } \ dt - f(x) \leq
\frac{1}{x^{2}} \dint{0}{+ \infty} t e^{-t} \ dt. 
\]

 On reconnaît la densité et l'espérance d'une loi exponentielle, on
obtient donc : 
 
\[
 0 \leq \frac{ 1 }{ x } - f(x) \leq \frac{ 1 }{ x^{2} } 
\]

 puis on encadre $f$ : 
 
\[
 - \frac{1}{x} \leq - f(x) \leq - \frac{1}{x} + \frac{1}{x^{2}} \ \
\text{ donc } \ \ \frac{ 1 }{ x } - \frac{ 1 }{ x^{2} } \leq f(x) \leq
\frac{1}{ x } 
\]

 Enfin en multipliant par $x > 0$ : 
 
\[
 1 - \frac{ 1 }{ x } \leq \frac{ f(x) }{ \frac{ 1 }{ x } } \leq 1 
\]

 et par encadrement (les termes de gauche et droite convergent
facilement vers 1), on obtient : 
 
\[
 \frac{ f(x) }{ \frac{ 1 }{ x } } \xrightarrow[ x \rightarrow + \infty
]{} 1 \ \ \text{ et enfin } \ \ f(x) \underset{ x \rightarrow + \infty
}{ \sim } \frac{ 1 }{ x }. 
\]

 \end{noliste}

 \indent

 \noindent \textbf{\underline{Exercice sans préparation}} \\
\\
 Les variables aléatoires sont définies sur un espace probabilisé
$(\Omega, \mathcal{A}, P)$. \\
 Soit $X$ une variable aléatoire qui suit la loi de Poisson de
paramètre $\lambda > 0$ et soit $Y$ une variable aléatoire indépendante
de $X$ telle que : $ Y ( \Omega ) = \{ 1 ; 2 \}, P \Ev{ Y = 1 } = P
\Ev{ Y = 2 } = \frac{ 1 }{ 2 }$. On pose $Z = X Y$.
 \begin{noliste}{1.}
 \setlength{\itemsep}{4mm}

 \item Lorsque $Y = 1$, $Z = X Y$ prend toutes les valeurs de $\N$;
lorsque $Y = 2$, $Z = X Y$ prend toutes les valeurs paires de $\N$.
Finalement on obtient : 
 
\[
 Z ( \Omega ) = \N. 
\]

 Si $k = 2 j + 1$ est impair, il ne peut être atteint qu'avec $Y = 1$
donc (avec $X$ et $Y$ indépendantes) : 
 
\[
 \forall j \in \N, \ ( Z = 2 j + 1 ) = (X = 2 j + 1 ) \cap \Ev{Y = 1 }
\ \ \text{ et } \ \ P \left(\Ev{ Z = 2 j + 1 }\right) = \frac{ 1 }{ 2 }
P \left(\Ev{ X = 2 j + 1 }\right) = \frac{ e^{ - \lambda } \lambda^{ 2j
+ 1 } }{ 2 (2j + 1)! } 
\]

 Si $k = 2j$ est pair, il peut être atteint avec $Y = 1$ ou 2, donc : 
 
\[
 \forall j \in \N, \ \Ev{ Z = 2 j } = [ \ \Ev{ X = 2j } \cap \Ev{ Y = 1
} ] \cup [ \ \Ev{ X = j } \cap \Ev{ Y = 2 } ] 
\]

 et
 
\[
 P \Ev{ Z = 2j } = \frac{ P \Ev{ X = 2j } + P \Ev{ X = j } }{ 2 } =
\frac{ e^{ - \lambda } }{ 2 } \left( \frac{ \lambda^{ 2j } }{ (2j)! } +
\frac{ \lambda^{j} }{ j! } \right). 
\]

 \item On décompose : 
 
\[
 (Z \text{ est paire } ) = \dcup{j = 0}{+ \infty} \Ev{Z = 2j } 
\]

 avec une union incompatible donc : 
 \begin{eqnarray*}
 P \left(\Ev{ Z \text{ est paire } }\right) & = & \Sum{j = 0}{+ \infty}
\left[ \ \frac{ e^{ - \lambda } }{ 2 } \left( \frac{ \lambda^{ 2j } }{
(2j)! } + \frac{ \lambda^{j} }{ j! } \right) \right] \\
\\
 & = & \frac{ e^{ - \lambda } }{ 2 } \left[ \ \Sum{j = 0}{+ \infty}
\frac{ \lambda^{ 2j } }{ (2j)! } + \Sum{j = 0}{+ \infty} \frac{
\lambda^{j} }{ j! } \right] \\
\\
 & = & \frac{ e^{ - \lambda } }{ 2 } \left[ \ \frac{ e^{ \lambda } +
e^{ - \lambda } }{ 2 } + e^{ \lambda } \right] = \frac{ 1 + e^{ - 2
\lambda } + 2 }{ 4 } = \frac{ 3 + e^{ - 2 \lambda } }{ 4 }. 
 \end{eqnarray*}

 \end{noliste}
 \end{exercice}

 \newpage
 
 

\end{document}