\documentclass[11pt]{article}%
\usepackage{geometry}%
\geometry{a4paper,
  lmargin=2cm,rmargin=2cm,tmargin=2.5cm,bmargin=2.5cm}

\input{../../../macros.tex}

% \renewcommand{\thesection}{\Roman{section}.\hspace{-.3cm}}
% \renewcommand{\thesubsection}{\Alph{subsection}.\hspace{-.2cm}}

\pagestyle{fancy} %
\lhead{ECE2 \hfill Mathématiques \\} %
\chead{\hrule} %
\rhead{} %
\lfoot{} %
\cfoot{} %
\rfoot{\thepage} %

\renewcommand{\headrulewidth}{0pt}% : Trace un trait de séparation
                                    % de largeur 0,4 point. Mettre 0pt
                                    % pour supprimer le trait.

\renewcommand{\footrulewidth}{0.4pt}% : Trace un trait de séparation
                                    % de largeur 0,4 point. Mettre 0pt
                                    % pour supprimer le trait.

\setlength{\headheight}{14pt}

\title{\bf \vspace{-1.6cm} HEC 2019} %
\author{} %
\date{} %
\begin{document}
\maketitle %
\vspace{-1.2cm}\hrule %
\thispagestyle{fancy}

\vspace*{.2cm}

%%DEBUT

\subsection*{Sujet Mathieu}

\begin{exerciceAP}~
  \begin{noliste}{1.}
  \item
    \begin{noliste}{a)}
    \item Formule du binôme de Newton.

      \begin{proof}~
        \conc{$\forall (a,b) \in \R^2$, $(a+b)^n = \Sum{k=0}{n}
          \dbinom{n}{k} a^{n-k} \, b^k$}~\\[-1cm]
      \end{proof}
      
    \item Soit $n \in \N^*$. En utilisant l'égalité suivante :
      \[
        2^n \ = \ (1+1)^n + (1-1)^n
      \]
      prouver : $\Sum{k=0}{\lfloor \frac{n}{2} \rfloor} \dbinom{n}{2k}
      \ = \ 2^{n-1}$.

      \begin{proof}~\\
        Soit $n \in \N$.
        \[
          \begin{array}{rcl@{\quad}>{\it}R{4cm}}
            2^n
            & = & (1+1)^n + (1-1)^n
            \\[.2cm]
            & = & \Sum{k=0}{n} \dbinom{n}{k} 1^{n-k} \, 1^k +
                  \Sum{k=0}{n} \dbinom{n}{k} 1^{n-k} \, (-1)^k
            & (d'après la formule du binôme de Newton)
              \nl
              \nl[-.2cm]
            & = & \Sum{k=0}{n} \dbinom{n}{k} + \Sum{k=0}{n}
                  \dbinom{n}{k} (-1)^k
            \\[.6cm]
            & = & \multicolumn{2}{l}{\left(\Sum{
                \scalebox{.6}{$
                  \begin{array}{c}
                    k=0 \\
                    \text{$k$ pair}
                  \end{array}$}
              }{n}
            \dbinom{n}{k}
            + \Sum{
                \scalebox{.6}{$
                  \begin{array}{c}
                    k=0 \\
                    \text{$k$ impair}
                  \end{array}$}
              }{n}
            \dbinom{n}{k} \right)
            + \left(\Sum{
                \scalebox{.6}{$
                  \begin{array}{c}
                    k = 0 \\
                    \text{$k$ pair}
                  \end{array}$}
              }{n}
            \dbinom{n}{k} (-1)^k
            + \Sum{
                \scalebox{.6}{$
                  \begin{array}{c}
                    k=0 \\
                    \text{$k$ impair}
                  \end{array}$}
              }{n}
            \dbinom{n}{k} (-1)^k \right)}
            \\[1cm]
            & = & \multicolumn{2}{l}{\Sum{j=0}{\lfloor \frac{n}{2} \rfloor}
                  \dbinom{n}{2j} + \Sum{j=0}{\lfloor \frac{n-1}{2} \rfloor }
                  \dbinom{n}{2j+1} + \Sum{j=0}{\lfloor \frac{n}{2} \rfloor}
                  \dbinom{n}{2j} (-1)^{2j} + \Sum{k=0}{\lfloor \frac{n-1}{2} \rfloor}
                  \dbinom{n}{2j+1} (-1)^{2j+1}}
            \\[.6cm]
            & = & \Sum{j=0}{\lfloor \frac{n}{2} \rfloor}
                  \dbinom{n}{2j} + \bcancel{\Sum{j=0}{\lfloor \frac{n-1}{2} \rfloor}
                  \dbinom{n}{2j+1}} + \Sum{j=0}{\lfloor \frac{n}{2} \rfloor}
                  \dbinom{n}{2j} - \bcancel{\Sum{j=0}{\lfloor \frac{n-1}{2} \rfloor}
                  \dbinom{n}{2j+1}}
            \\[.6cm]
            & = & 2 \, \Sum{j=0}{\lfloor \frac{n}{2} \rfloor} \dbinom{n}{2j}
          \end{array}
        \]
        \conc{On en déduit : $2^{n-1} = \Sum{k=0}{\lfloor \frac{n}{2}
            \rfloor} \dbinom{n}{2k}$.}~\\[-1cm]
      \end{proof}
    \end{noliste}


    \newpage
    
    
  \item
    \begin{noliste}{a)}
    \item Soit $n \in \N$. Montrer qu'il existe un unique polynôme
      $P_n$ tel que :
      \[
        \forall x \in \R \setminus \ ]-1,1[, \ P_n(x) = \left(x + \sqrt{x^2-1}\right)^n +
        \left(x - \sqrt{x^2-1}\right)^n
      \]

      \begin{proof}~\\
        Soit $x \in \R \setminus \ ]-1,1[$.
        \[
          \begin{array}{cl@{\quad}>{\it}R{4cm}}
            & \left(x + \sqrt{x^2-1}\right)^n + \left(x - \sqrt{x^2
              -1}\right)^n
            \\[.4cm]
            = & \Sum{k=0}{n} \dbinom{n}{k} x^{n-k} \, \left(\sqrt{x^2-1}\right)^k +
                  \Sum{k=0}{n} \dbinom{n}{k} x^{n-k} \, \left(- \sqrt{x^2-1} \right)^k
            & (d'après la formule du binôme de Newton)
              \nl
              \nl[-.2cm]
            = & \left(\Sum{
                \scalebox{.6}{$
                  \begin{array}{l}
                    k=0 \\
                    \text{$k$ pair}
                  \end{array}$}
              }{n}
            \dbinom{n}{k} x^{n-k} \, \left(\sqrt{x^2-1}\right)^k
            + \Sum{
                \scalebox{.6}{$
                  \begin{array}{l}
                    k=0 \\
                    \text{$k$ impair}
                  \end{array}$}
              }{n}
            \dbinom{n}{k} x^{n-k} \,
            \left(\sqrt{x^2-1}\right)^k\right)
            \\[1cm]
            & + \ \left(\Sum{
                \scalebox{.6}{$
                  \begin{array}{l}
                    k=0 \\
                    \text{$k$ pair}
                  \end{array}$}
              }{n}
            \dbinom{n}{k} x^{n-k} \, \left(-\sqrt{x^2-1}\right)^k
            + \Sum{
                \scalebox{.6}{$
                  \begin{array}{l}
                    k=0 \\
                    \text{$k$ impair}
                  \end{array}$}
              }{n}
            \dbinom{n}{k} x^{n-k} \, \left(-\sqrt{x^2-1}\right)^k \right)
            \\[1cm]
            = & \multicolumn{2}{l}{\left(\Sum{j=0}{\lfloor \frac{n}{2} \rfloor}
            \dbinom{n}{2j} x^{n-2j} \, \left(\sqrt{x^2-1}\right)^{2j}
            + \Sum{j=0}{\lfloor \frac{n-1}{2} \rfloor}
            \dbinom{n}{2j+1} x^{n-(2j+1)} \,
            \left(\sqrt{x^2-1}\right)^{2j+1}\right)}
            \\[1cm]
            & \multicolumn{2}{l}{+ \ \left(\Sum{j=0}{\lfloor \frac{n}{2} \rfloor}
            \dbinom{n}{2j} x^{n-2j} \, \left(-\sqrt{x^2-1}\right)^{2j}
            + \Sum{j=0}{\lfloor \frac{n-1}{2} \rfloor}
            \dbinom{n}{2j+1} x^{n-(2j+1)} \, \left(-\sqrt{x^2-1}\right)^{2j+1} \right)}
            \\[1cm]
            = & \Sum{j=0}{\lfloor \frac{n}{2} \rfloor}
            \dbinom{n}{2j} x^{n-2j} \, \left(x^2-1\right)^{j}
            + \bcancel{\Sum{j=0}{\lfloor \frac{n-1}{2} \rfloor}
            \dbinom{n}{2j+1} x^{n-(2j+1)} \,
            \left(\sqrt{x^2-1}\right)^{2j+1}}
            \\[1cm]
            & + \ \Sum{j=0}{\lfloor \frac{n}{2} \rfloor}
            \dbinom{n}{2j} x^{n-2j} \, \left(x^2-1\right)^{j}
            - \bcancel{\Sum{j=0}{\lfloor \frac{n-1}{2} \rfloor}
            \dbinom{n}{2j+1} x^{n-(2j+1)} \, \left(\sqrt{x^2-1}\right)^{2j+1}}
            \\[1cm]
            = & 2 \, \Sum{j=0}{\lfloor \frac{n}{2} \rfloor}
                \dbinom{n}{2j} x^{n-2j} \, \left(x^2-1\right)^{j}
          \end{array}
        \]
        De plus :
        \begin{noliste}{$\sbullet$}
        \item pour tout $j \in \llb 0, \left\lfloor \dfrac{n}{2}
          \right\rfloor \rrb$,
          $x \mapsto x^{n-2j}$ est bien une fonction polynomiale, car
          $n-2j \geq 0$,
          
        \item $x \mapsto (x^2-1)^j$ est bien une fonction polynomiale
          en tant que produit de fonctions polynomiales.
        \end{noliste}
        Finalement, la fonction $x \mapsto 2 \ \Sum{j=0}{\lfloor \frac{n}{2} \rfloor}
        \dbinom{n}{2j} x^{n-2j} \, \left(x^2-1\right)^{j}$ est
        bien une fonction polynomiale.
        \conc{Il existe un unique polynôme $P_n$ qui coincide\\ avec la
          fonction $x \mapsto \left(x + \sqrt{x^2-1}\right)^n +
          \left(x - \sqrt{x^2-1}\right)^n$ sur $\R \setminus \ ]-1,1[$
        :\\ le polynôme défini par $P_n(X) = 2 \ \Sum{j=0}{\lfloor \frac{n}{2} \rfloor}
                \dbinom{n}{2j} X^{n-2j} \, \left(X^2-1\right)^{j}$.}~\\[-1.4cm]
      \end{proof}


      \newpage

      
    \item Quel est le coefficient de $X^n$ dans l'expression de
      $P_n(X)$ ?

      \begin{proof}~\\
        Soit $n \in \N^*$.
        \begin{noliste}{$\sbullet$}
        \item Soit $j \in \llb 0, \left\lfloor \dfrac{n}{2}
          \right\rfloor \rrb$. On note $Q_j(X) = X^{n-2j} \left( X^2-1
          \right)^j$.
          \begin{noliste}{$\stimes$}
          \item Le polynôme $Q_j$ est de degré $n$. En effet :
            \[
              \deg(Q_j) \ = \ \deg\left( X^{n-2j}
                \left(X^2-1\right)^j\right) \ = \
              \deg\left(X^{n-2j}\right) + j \deg\left(X^2-1\right)
              \ = \ (n-2j) + j \times 2 \ = \ n
            \]
            
          \item Son coefficient dominant (le coefficient de $X^n$) est $1$.
          \end{noliste}
          On a ainsi démontré que le polynôme $Q_j$ est de la forme :
          $ Q_j(X) \ = \ X^n + \Sum{k=0}{n-1} a_{j,k} \, X^k$
          
        \item On en déduit :
          \[
            \begin{array}{rcl@{\quad}>{\it}R{3cm}}
              P_n(X) 
              & = & 2 \ \Sum{j=0}{\lfloor \frac{n}{2} \rfloor}
                    \dbinom{n}{2j} \left(X^n + \Sum{k=0}{n-1} a_{j,k}
                    X^k\right)
              \\[.6cm]
              & = & \left(2 \ \Sum{j=0}{\lfloor \frac{n}{2} \rfloor}
                    \dbinom{n}{2j} \right) X^n + 2 \ \Sum{j=0}{\lfloor
                    \frac{n}{2} \rfloor} \Sum{k=0}{n-1} a_{k,j} X^k
              \\[.6cm]
              & = & \left(2 \times 2^{n-1}\right) X^n + \Sum{k=0}{n-1}
                    \left(2 \ \Sum{j=0}{\lfloor \frac{n}{2} \rfloor}
                    a_{k,j} \right) X^k
              & (d'après la question \itbf{1.b)}, car $n \in \N^*$)
                \nl
                \nl[-.2cm]
              & = & 2^n \, X^n + R_{n-1}(X)
            \end{array}
          \]
          où $R_{n-1}$ est un polynôme de degré au plus $n-1$.
          
        \item Soit $x \in \R \setminus \ ]-1,1[$.
            \[
              P_0(x) \ = \ \left(x + \sqrt{x^2-1} \right)^0 + \left(x
                - \sqrt{x^2-1} \right)^0 \ = \ 2
            \]
            On obtient : $\forall x \in \R \setminus \ ]-1,1[$,
            $P_0(x)-2 = 0$.\\
            En particulier, le polynôme $P_0(X)-2$ admet une infinité
            de racines.\\
            C'est donc le polynôme nul.
            \conc{D'où : $P_0(X) = 2$.}
          \conc{On en déduit que, si $n \in \N^*$, le coefficient dominant de $P_n$ est
            $2^n$,\\ et le coefficient dominant de $P_0$ est $2$.}~\\[-1.4cm]
        \end{noliste}
      \end{proof}
    \end{noliste}
    
  \item
    \begin{noliste}{a)}
    \item Justifier la relation de récurrence suivante :
      \[
        \forall n \in \N, \quad P_{n+2}(X) \ = \ 2 \, X \, P_{n+1}(X) -
        P_n(X)
      \]

      \begin{proof}~\\
        Soit $n \in \N$.
        \begin{noliste}{$\sbullet$}
        \item Soit $x \in \R \setminus \ ]-1,1[$.
          \[
            \begin{array}{cl}
              & 2x \, P_{n+1}(x) - P_n(x)
              \\[.2cm]
              = & 2 \, x \left(\left(x + \sqrt{x^2-1}\right)^{n+1} +
                    \left(x - \sqrt{x^2-1}\right)^{n+1} \right) -
                    \left( \left(x+ \sqrt{x^2-1} \right)^n + \left(x -
                    \sqrt{x^2-1} \right)^n\right)
              \\[.6cm]
              = & \left(x + \sqrt{x^2-1}\right)^n \left(2x \left(x +
                  \sqrt{x^2-1} \right) -1\right) + \left(x +
                  \sqrt{x^2-1} \right)^n \left(2x \left(x -
                  \sqrt{x^2-1}\right) -1 \right)
              \\[.6cm]
              = & \left(x + \sqrt{x^2-1}\right)^n \left(2x^2 + 2x \,
                  \sqrt{x^2-1} -1\right) + \left(x +
                  \sqrt{x^2-1} \right)^n \left(2x^2 - 2x \,
                  \sqrt{x^2-1} -1 \right)
            \end{array}
          \]
          Or :
          \[
            \begin{array}{cl}
              & P_{n+2}(x)
              \\[.2cm]
              = & \left(x + \sqrt{x^2-1} \right)^{n+2} + \left(x -
                    \sqrt{x^2-1} \right)^{n+2}
              \\[.6cm]
              = & \left(x + \sqrt{x^2-1} \right)^{n} \left(x + \sqrt{x^2-1}\right)^2
                  + \left(x - \sqrt{x^2-1} \right)^{n} \left(x -
                  \sqrt{x^2-1}\right)^2
              \\[.6cm]
              = & \left(x + \sqrt{x^2-1} \right)^{n} \left(x^2 + 2x \,
                  \sqrt{x^2-1} + (x^2-1) \right)
                  + \left(x - \sqrt{x^2-1} \right)^{n} \left(x^2 - 2x \,
                  \sqrt{x^2-1} + (x^2-1)\right)
              \\[.6cm]
              = & \left(x + \sqrt{x^2-1} \right)^{n} \left(2x^2 + 2x \,
                  \sqrt{x^2-1} -1 \right)
                  + \left(x - \sqrt{x^2-1} \right)^{n} \left(2x^2 - 2x \,
                  \sqrt{x^2-1} -1\right)
            \end{array}
          \]
          Finalement : $\forall x \in \R \setminus \ ]-1,1[$,
          $P_{n+2}(x) = 2x \, P_{n+1}(x) - P_n(x)$.
          
        \item On en déduit : $\forall x \in \R \setminus \ ]-1,1[$,
          $P_{n+2}(x) - 2x \, P_{n+1}(x) + P_n(x) = 0$.\\
          En particulier, le polynôme $P_{n+2}(X) - 2X \, P_{n+1}(X) +
          P_n(X)$ admet une infinité de racines. C'est donc le
          polynôme nul.
          \conc{$P_{n+2}(X) \ = \ 2X \, P_{n+1}(X) - P_n(X)$}~\\[-1.4cm]
        \end{noliste}
      \end{proof}
      
    \item ???
    \end{noliste}
    
  \item
    \begin{noliste}{a)}
    \item Proposer deux fonctions \Scilab{} :
      \begin{noliste}{$\stimes$}
      \item l'une prenant en entrée deux vecteurs {\tt P} et {\tt Q}
        de tailles différentes et permettant d'en faire la somme. On supposera
        que la taille du vecteur {\tt Q} est supérieure à celle du
        vecteur {\tt P}.
        
      \item l'autre prenant en entrée un vecteur {\tt P} et permettant
        de concaténer au vecteur {\tt P} un $0$ à sa gauche.
      \end{noliste}
      {\it (\'Enoncé déduit de souvenirs)}

      \begin{proof}~
        \begin{noliste}{$\sbullet$}
        \item Pour la première fonction, on concatène au vecteur {\tt
            P} (le plus petit des deux vecteurs) plusieurs $0$ à sa
          droite pour qu'il soit de même taille que le vecteur {\tt Q}
          pour rendre la somme licite. On obtient la fonction suivante
          :
          \begin{scilab}
            & \tcFun{function} \tcVar{S} = Somme(\tcVar{P}, \tcVar{Q}) \nl %
            & \quad R = [\tcVar{P}, zeros(1, length(\tcVar{Q}) -
            length(\tcVar{P}))] \nl %
            & \quad \tcVar{S} = R + \tcVar{Q} \nl %
            & \tcFun{endfunction}
          \end{scilab}
          
        \item On propose la fonction suivante :
          \begin{scilab}
            & \tcFun{function} \tcVar{T} = AjoutZero(\tcVar{P}) \nl %
            & \quad \tcVar{T} = [0, \tcVar{P}] \nl %
            & \tcFun{endfunction}
          \end{scilab}
        \end{noliste}
      \end{proof}


      \newpage
      
      
    \item Proposer une fonction \Scilab{} prenant en entrée un
      paramètre $n$ et permettant de calculer le
      polynôme $P_n$. On pourra pour cela utiliser la représentation
      matriciel des polynômes en présence dans la base
      canonique de $\R[X]$.\\
      {\it (\'Enoncé extrapolé)}

      \begin{proof}~
        \begin{noliste}{$\sbullet$}
        \item La suite de polynômes $(P_n)$ est définie par la
          relation de récurrence de la question \itbf{3.a)}.\\
          On cherche alors d'abord à déterminer $P_0$ et $P_1$.
          \begin{noliste}{$\stimes$}
          \item D'après la question \itbf{2.a)}, on a : $P_0(X) = 2$.
            
          \item Soit $x \in \R \setminus \ ]-1,1[$.
            \[
              P_1(x) \ = \ \left(x + \sqrt{x^2-1} \right)^1 + \left(x
                - \sqrt{x^2-1} \right)^1 \ = \ x + \bcancel{\sqrt{x^2
                  -1}} +x - \bcancel{\sqrt{x^2-1}} \ = \ 2 \, x
            \]
            On obtient donc : $\forall x \in \R \setminus \ ]-1,1[$,
            $P_1(x) - 2x =0$.\\
            En particulier, le polynôme $P_1(X) -2X$ admet une
            infinité de racines.\\
            C'est donc le polynôme nul.
            \conc{D'où : $P_1(X) = 2 \, X$.}
          \end{noliste}
          
        \item On propose la fonction \Scilab{} suivante :
          \begin{scilab}
            & \tcFun{function} \tcVar{P} = Suite(\tcVar{n}) \nl %
            & \quad Q = [2] \nl %
            & \quad R = [0, 2] \nl %
            & \quad \tcFor{for} k = 1:\tcVar{n} \nl %
            & \quad \quad S = 2 \Sfois{} AjoutZero(R) \nl %
            & \quad \quad T = Somme(-Q, S) \nl %
            & \quad \quad Q = R \nl %
            & \quad \quad R = T \nl %
            & \quad \tcFor{end} \nl %
            & \quad \tcVar{P} = Q \nl %
            & \tcFun{endfunction}
          \end{scilab}
          
          \item L'énoncé suggère en effet d'utiliser la représentation
          matricielle des polynômes dans la base canonique de
          $\R[X]$.\\
          On note $\B_n$ la base canonique de $\R_n[X]$. On obtient :
          \begin{noliste}{$\stimes$}
          \item $\Mat_{\B_0}(P_0) = (2)$ et $\Mat_{\B_1}(P_1) =
            \begin{smatrix}
              0\\
              2
            \end{smatrix}$.\\
            On initialise donc la suite $(P_n)$ avec les lignes
            suivantes :
            \begin{scilabC}{1}
              & \quad Q = [2] \nl %
              & \quad R = [0, 2]
            \end{scilabC}
          \end{noliste}
            
          \item On rappelle que $P_{n+1} \in \R_{n+1}[X]$. \\
            Si $\Mat_{\B_{n+1}}(P_{n+1}) =
            \begin{smatrix}
              a_0\\
              \vdots\\
              a_{n+1}
            \end{smatrix}$, alors, en notant $S_{n+2}(X) = X P_{n+1}(X)
            \in \R_{n+2}[X]$,
            on a : $\Mat_{\B_{n+2}}(S_{n+2}) =
            \begin{smatrix}
              0\\
              a_0\\
              \vdots\\
              a_{n+1}
            \end{smatrix}$.\\
            Pour obtenir le polynôme $2X \, P_{n+1}(X)$, dont la
            représentation matricielle est stockée dans la variable
            {\tt S}, on utilise
            donc la commande suivante :
            \begin{scilabC}{4}
              & \quad \quad S = 2 \Sfois{} AjoutZero(R)
            \end{scilabC}
            

            \newpage

          
          \item On doit ensuite sommer les polynômes $2X \,
            P_{n+1}(X)$ et $-P_n(X)$ pour obtenir $P_{n+2}$.\\
            Dans la fonction {\tt Somme} le premier argument est le
            polynôme de plus petit degré. Pour respecter cet ordre, on
            détermine la représentation matricielle de $P_{n+2}$ avec
            la commande suivante :
            \begin{scilabC}{5}
              & \quad \quad T = Somme(-Q, S)
            \end{scilabC}
          
          \item Enfin, dans ce programme :
            \begin{noliste}{$\stimes$}
            \item la variable {\tt Q} contient la représentation
              matricielle du polynôme $P_n$ dans la base $\B_n$,
              
            \item la variable {\tt R} contient la représentation
              matricielle du polynôme $P_{n+1}$ dans la base
              $B_{n+1}$,
              
            \item la variable {\tt S} contient la représentation
              matricielle du polynôme $P_{n+2}$ dans la base $\B_{n+2}$.
            \end{noliste}
            Il faut donc renvoyer la variable {\tt Q}, ce qu'on
            effectue avec la commande :
            \begin{scilabC}{9}
              & \tcVar{P} = Q
            \end{scilabC}~\\[-1.4cm]
        \end{noliste}
      \end{proof}
    \end{noliste}
  \end{noliste}
\end{exerciceAP}


\begin{exerciceSP}~\\
  On considère une \var $Z$ de loi normale centrée réduite. On note
  $f$ une densité de $Z$.
  \begin{noliste}{1.}
  \item Justifier que l'intégrale $\dint{x}{+\infty} \dfrac{f(t)}{t^2}
    \dt$ converge si $x>0$.\\
    Est-ce toujours le cas si $x=0$ ?

    \begin{proof}~\\
      Soit $x>0$.
      \begin{noliste}{$\sbullet$}
      \item La fonction $t \mapsto \dfrac{f(t)}{t^2}$ est continue sur
        $[x, +\infty[$ en tant que quotient de fonctions continues sur
        $[x, +\infty[$ dont le dénominateur ne s'annule pas sur cet intervalle.
        
      \item Par définition de la fonction $f$ (on rappelle que : $f :
        t \mapsto \dfrac{1}{\sqrt{2 \pi}} \, \ee^{-\frac{t^2}{2}}$),
        on a : 
        \[
          \forall t \in \R, \ 0 \ \leq \ f(t) \ \leq \ f(0) = \dfrac{1}{\sqrt{2 \pi}}
        \]
        Ainsi, pour tout $t \geq x$ :
        \[
          0 \ \leq \ \dfrac{f(t)}{t^2} \ \leq \ \dfrac{1}{\sqrt{2 \pi}
            \ t^2}
        \]
        
      \item Ainsi : 
        \begin{noliste}{$\stimes$}
        \item $\forall t \in [x, +\infty[$, $0 \ \leq \
          \dfrac{f(t)}{t^2} \ \leq \ \dfrac{1}{\sqrt{2 \pi} \ t^2}$
          
        \item l'intégrale $\dint{x}{+\infty} \dfrac{1}{t^2} \dt$ est
          une intégrale de Riemann, impropre en $+\infty$, d'exposant
          $2$ ($2>1$). C'est donc une intégrale convergente. D'où
          l'intégrale $\dint{x}{+\infty} \dfrac{1}{\sqrt{2 \pi} \ t^2}
          \dt$ est convergente.
        \end{noliste}
        \conc{Par critère de comparaison d'intégrales généralisées de
          fonctions continues positives,\\ l'intégrale
          $\dint{x}{+\infty} \dfrac{f(t)}{t^2} \dt$ est convergente.}


        \newpage

      
      \item On remarque : 
        \[
          \dfrac{f(t)}{t^2} \ \eq{t \to 0} \ \dfrac{f(0)}{t^2} \ = \
          \dfrac{1}{\sqrt{2 \pi} \ t^2}
        \]
        Ainsi : 
        \begin{noliste}{$\stimes$}
        \item $\dfrac{f(t)}{t^2} \ \eq{t \to 0} \ 
          \dfrac{1}{\sqrt{2 \pi} \ t^2} \ (\geq 0)$.
          
        \item l'intégrale $\dint{0}{1} \dfrac{1}{t^2} \dt$ est
          une intégrale de Riemann, impropre en $0$, d'exposant
          $2$ ($2\bcancel{<}1$). C'est donc une intégrale divergente. D'où
          l'intégrale $\dint{0}{+\infty} \dfrac{1}{\sqrt{2 \pi} \ t^2}
          \dt$ est divergente.
        \end{noliste}
        \conc{Par critère d'équivalence d'intégrales généralisées de
          fonctions continues positives,\\ l'intégrale
          $\dint{0}{+\infty} \dfrac{f(t)}{t^2} \dt$ est divergente.}
      \end{noliste}
    \end{proof}
    
  \item ???
  \end{noliste}
\end{exerciceSP}


%%FIN


\end{document}
