\documentclass[11pt]{article}%
\usepackage{geometry}%
\geometry{a4paper,
  lmargin=2cm,rmargin=2cm,tmargin=2.5cm,bmargin=2.5cm}

\input{../../../macros.tex}

% \renewcommand{\thesection}{\Roman{section}.\hspace{-.3cm}}
% \renewcommand{\thesubsection}{\Alph{subsection}.\hspace{-.2cm}}

\pagestyle{fancy} %
\lhead{ECE2 \hfill Mathématiques \\} %
\chead{\hrule} %
\rhead{} %
\lfoot{} %
\cfoot{} %
\rfoot{\thepage} %

\renewcommand{\headrulewidth}{0pt}% : Trace un trait de séparation
                                    % de largeur 0,4 point. Mettre 0pt
                                    % pour supprimer le trait.

\renewcommand{\footrulewidth}{0.4pt}% : Trace un trait de séparation
                                    % de largeur 0,4 point. Mettre 0pt
                                    % pour supprimer le trait.

\setlength{\headheight}{14pt}

\title{\bf \vspace{-1.6cm} HEC 2019} %
\author{} %
\date{} %
\begin{document}
\maketitle %
\vspace{-1.2cm}\hrule %
\thispagestyle{fancy}

\vspace*{.2cm}

%%DEBUT

\subsection*{Sujet Live}

\begin{exerciceAP}~
  \begin{noliste}{1.}
  \item
    \begin{noliste}{a)}
    \item Formule du binôme de Newton.
      
    \item Soit $n \in \N^*$. En utilisant l'égalité suivante :
      \[
        2^n \ = \ (1+1)^n + (1-1)^n
      \]
      prouver : $\Sum{k=0}{\lfloor \frac{n}{2} \rfloor} \dbinom{n}{2k}
      \ = \ 2^{n-1}$.
    \end{noliste}
    
  \item
    \begin{noliste}{a)}
    \item Soit $n \in \N$. Montrer qu'il existe un unique polynôme
      $P_n$ tel que :
      \[
        \forall x \in \R \setminus \ ]-1,1[, \ P_n(x) = \left(x +
        \sqrt{x^2-1}\right)^n +
        \left(x - \sqrt{x^2-1}\right)^n
      \]
      
    \item Quel est le coefficient de $X^n$ dans l'expression de
      $P_n(X)$ ?
    \end{noliste}
    
  \item
    \begin{noliste}{a)}
    \item Justifier la relation de récurrence suivante :
      \[
        \forall n \in \N, \quad P_{n+2}(X) \ = \ 2 \, X \, P_{n+1}(X) -
        P_n(X)
      \]
      
    \item ???
    \end{noliste}
    
  \item
    \begin{noliste}{a)}
    \item Proposer deux fonctions \Scilab{} :
      \begin{noliste}{$\stimes$}
      \item l'une prenant en entrée deux vecteurs {\tt P} et {\tt Q}
        de tailles différentes et permettant d'en faire la somme. On supposera
        que la taille du vecteur {\tt Q} est supérieure à celle du
        vecteur {\tt P}.
        
      \item l'autre prenant en entrée un vecteur {\tt P} et permettant
        de concaténer au vecteur {\tt P} un $0$ à sa gauche.
      \end{noliste}
      {\it (\'Enoncé déduit de souvenirs)}
      
    \item Proposer une fonction \Scilab{} prenant en entrée un
      paramètre $n$ et permettant de calculer le
      polynôme $P_n$. On pourra pour cela utiliser la représentation
      matriciel des polynômes en présence dans la base
      canonique de $\R[X]$.\\
      {\it (\'Enoncé extrapolé)}
    \end{noliste}
  \end{noliste}
\end{exerciceAP}


\begin{exerciceSP}~\\
  On considère une \var $Z$ de loi normale centrée réduite. On note
  $f$ une densité de $Z$.
  \begin{noliste}{1.}
  \item Justifier que l'intégrale $\dint{x}{+\infty} \dfrac{f(t)}{t^2}
    \dt$ converge si $x>0$.\\
    Est-ce toujours le cas si $x=0$ ?
    
  \item ???
  \end{noliste}
\end{exerciceSP}


%%FIN


\end{document}
