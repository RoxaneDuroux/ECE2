\documentclass[11pt]{article}%
\usepackage{geometry}%
\geometry{a4paper,
 lmargin = 2cm,rmargin = 2cm,tmargin = 2.5cm,bmargin = 2.5cm}

\usepackage{array}
\usepackage{paralist}

\usepackage[svgnames, usenames, dvipsnames]{xcolor}
\xdefinecolor{RecColor}{named}{Aqua}
\xdefinecolor{IncColor}{named}{Aqua}
\xdefinecolor{ImpColor}{named}{PaleGreen}

% \usepackage{frcursive}

\usepackage{adjustbox}

%%%%%%%%%%%
\newcommand{\cRB}[1]{{\color{Red} \pmb{#1}}} %
\newcommand{\cR}[1]{{\color{Red} {#1}}} %
\newcommand{\cBB}[1]{{\color{Blue} \pmb{#1}}}
\newcommand{\cB}[1]{{\color{Blue} {#1}}}
\newcommand{\cGB}[1]{{\color{LimeGreen} \pmb{#1}}}
\newcommand{\cG}[1]{{\color{LimeGreen} {#1}}}

%%%%%%%%%%

\usepackage{diagbox} %
\usepackage{colortbl} %
\usepackage{multirow} %
\usepackage{pgf} %
\usepackage{environ} %
\usepackage{fancybox} %
\usepackage{textcomp} %
\usepackage{marvosym} %

%%%%%%%%%% pour pouvoir faire des dashedline dans les tableaux
\usepackage{arydshln}

%%%%%%%%%% pour qu'une cellcolor ne recouvre pas le trait du tableau
\usepackage{hhline}%

\usepackage{pgfplots}
\pgfplotsset{compat=1.10}
\usepgfplotslibrary{patchplots}
\usepgfplotslibrary{fillbetween}
\usepackage{tikz,tkz-tab}
\usepackage{ifthen}
\usepackage{calc}
\usetikzlibrary{calc,decorations.pathreplacing,arrows,positioning} 
\usetikzlibrary{fit,shapes,backgrounds}
% \usepackage[nomessages]{fp}% http://ctan.org/pkg/fp

\usetikzlibrary{matrix,arrows,decorations.pathmorphing,
  decorations.pathreplacing} 

\newcommand{\myunit}{1 cm}
\tikzset{
    node style sp/.style={draw,circle,minimum size=\myunit},
    node style ge/.style={circle,minimum size=\myunit},
    arrow style mul/.style={draw,sloped,midway,fill=white},
    arrow style plus/.style={midway,sloped,fill=white},
}

%%%%%%%%%%%%%%
%%%%% écrire des inférieur égal ou supérieur égal avec typographie
%%%%% francaise
%%%%%%%%%%%%%

\renewcommand{\geq}{\geqslant}
\renewcommand{\leq}{\leqslant}
\renewcommand{\emptyset}{\varnothing}

\newcommand{\Leq}{\leqslant}
\newcommand{\Geq}{\geqslant}

%%%%%%%%%%%%%%
%%%%% Macro Celia
%%%%%%%%%%%%%

\newcommand{\ff}[2]{\left[#1, #2\right]} %
\newcommand{\fo}[2]{\left[#1, #2\right[} %
\newcommand{\of}[2]{\left]#1, #2\right]} %
\newcommand{\soo}[2]{\left]#1, #2\right[} %
\newcommand{\abs}[1]{\left|#1\right|} %
\newcommand{\Ent}[1]{\left\lfloor #1 \right\rfloor} %


%%%%%%%%%%%%%%
%%%%% tikz : comment dessiner un "oeil"
%%%%%%%%%%%%%

\newcommand{\eye}[4]% size, x, y, rotation
{ \draw[rotate around={#4:(#2,#3)}] (#2,#3) -- ++(-.5*55:#1) (#2,#3)
  -- ++(.5*55:#1); \draw (#2,#3) ++(#4+55:.75*#1) arc
  (#4+55:#4-55:.75*#1);
  % IRIS
  \draw[fill=gray] (#2,#3) ++(#4+55/3:.75*#1) arc
  (#4+180-55:#4+180+55:.28*#1);
  % PUPIL, a filled arc
  \draw[fill=black] (#2,#3) ++(#4+55/3:.75*#1) arc
  (#4+55/3:#4-55/3:.75*#1);%
}


%%%%%%%%%%
%% discontinuité fonction
\newcommand\pointg[2]{%
  \draw[color = red, very thick] (#1+0.15, #2-.04)--(#1, #2-.04)--(#1,
  #2+.04)--(#1+0.15, #2+.04);%
}%

\newcommand\pointd[2]{%
  \draw[color = red, very thick] (#1-0.15, #2+.04)--(#1, #2+.04)--(#1,
  #2-.04)--(#1-0.15, #2-.04);%
}%

%%%%%%%%%%
%%% 1 : position abscisse, 2 : position ordonnée, 3 : taille, 4 : couleur
%%%%%%%%%%
% \newcommand\pointG[4]{%
%   \draw[color = #4, very thick] (#1+#3, #2-(#3/3.75))--(#1,
%   #2-(#3/3.75))--(#1, #2+(#3/3.75))--(#1+#3, #2+(#3/3.75)) %
% }%

\newcommand\pointG[4]{%
  \draw[color = #4, very thick] ({#1+#3/3.75}, {#2-#3})--(#1,
  {#2-#3})--(#1, {#2+#3})--({#1+#3/3.75}, {#2+#3}) %
}%

\newcommand\pointD[4]{%
  \draw[color = #4, very thick] ({#1-#3/3.75}, {#2+#3})--(#1,
  {#2+#3})--(#1, {#2-#3})--({#1-#3/3.75}, {#2-#3}) %
}%

\newcommand\spointG[4]{%
  \draw[color = #4, very thick] ({#1+#3/1.75}, {#2-#3})--(#1,
  {#2-#3})--(#1, {#2+#3})--({#1+#3/1.75}, {#2+#3}) %
}%

\newcommand\spointD[4]{%
  \draw[color = #4, very thick] ({#1-#3/2}, {#2+#3})--(#1,
  {#2+#3})--(#1, {#2-#3})--({#1-#3/2}, {#2-#3}) %
}%

%%%%%%%%%%

\newcommand{\Pb}{\mathtt{P}}

%%%%%%%%%%%%%%%
%%% Pour citer un précédent item
%%%%%%%%%%%%%%%
\newcommand{\itbf}[1]{{\small \bf \textit{#1}}}


%%%%%%%%%%%%%%%
%%% Quelques couleurs
%%%%%%%%%%%%%%%

\xdefinecolor{cancelcolor}{named}{Red}
\xdefinecolor{intI}{named}{ProcessBlue}
\xdefinecolor{intJ}{named}{ForestGreen}

%%%%%%%%%%%%%%%
%%%%%%%%%%%%%%%
% barrer du texte
\usetikzlibrary{shapes.misc}

\makeatletter
% \definecolor{cancelcolor}{rgb}{0.127,0.372,0.987}
\newcommand{\tikz@bcancel}[1]{%
  \begin{tikzpicture}[baseline=(textbox.base),inner sep=0pt]
  \node[strike out,draw] (textbox) {#1}[thick, color=cancelcolor];
  \useasboundingbox (textbox);
  \end{tikzpicture}%
}
\newcommand{\bcancel}[1]{%
  \relax\ifmmode
    \mathchoice{\tikz@bcancel{$\displaystyle#1$}}
               {\tikz@bcancel{$\textstyle#1$}}
               {\tikz@bcancel{$\scriptstyle#1$}}
               {\tikz@bcancel{$\scriptscriptstyle#1$}}
  \else
    \tikz@bcancel{\strut#1}%
  \fi
}
\newcommand{\tikz@xcancel}[1]{%
  \begin{tikzpicture}[baseline=(textbox.base),inner sep=0pt]
  \node[cross out,draw] (textbox) {#1}[thick, color=cancelcolor];
  \useasboundingbox (textbox);
  \end{tikzpicture}%
}
\newcommand{\xcancel}[1]{%
  \relax\ifmmode
    \mathchoice{\tikz@xcancel{$\displaystyle#1$}}
               {\tikz@xcancel{$\textstyle#1$}}
               {\tikz@xcancel{$\scriptstyle#1$}}
               {\tikz@xcancel{$\scriptscriptstyle#1$}}
  \else
    \tikz@xcancel{\strut#1}%
  \fi
}
\makeatother

\newcommand{\xcancelRA}{\xcancel{\rule[-.15cm]{0cm}{.5cm} \Rightarrow
    \rule[-.15cm]{0cm}{.5cm}}}

%%%%%%%%%%%%%%%%%%%%%%%%%%%%%%%%%%%
%%%%%%%%%%%%%%%%%%%%%%%%%%%%%%%%%%%

\newcommand{\vide}{\multicolumn{1}{c}{}}

%%%%%%%%%%%%%%%%%%%%%%%%%%%%%%%%%%%
%%%%%%%%%%%%%%%%%%%%%%%%%%%%%%%%%%%


\usepackage{multicol}
% \usepackage[latin1]{inputenc}
% \usepackage[T1]{fontenc}
\usepackage[utf8]{inputenc}
\usepackage[T1]{fontenc}
\usepackage[normalem]{ulem}
\usepackage[french]{babel}

\usepackage{url}    
\usepackage{hyperref}
\hypersetup{
  backref=true,
  pagebackref=true,
  hyperindex=true,
  colorlinks=true,
  breaklinks=true,
  urlcolor=blue,
  linkcolor=red,
  bookmarks=true,
  bookmarksopen=true
}

%%%%%%%%%%%%%%%%%%%%%%%%%%%%%%%%%%%%%%%%%%%
%% Pour faire des traits diagonaux dans les tableaux
%% Nécessite slashbox.sty
%\usepackage{slashbox}

\usepackage{tipa}
\usepackage{verbatim,listings}
\usepackage{graphicx}
\usepackage{fancyhdr}
\usepackage{mathrsfs}
\usepackage{pifont}
\usepackage{tablists}
\usepackage{dsfont,amsfonts,amssymb,amsmath,amsthm,stmaryrd,upgreek,manfnt}
\usepackage{enumerate}

%\newcolumntype{M}[1]{p{#1}}
\newcolumntype{C}[1]{>{\centering}m{#1}}
\newcolumntype{R}[1]{>{\raggedright}m{#1}}
\newcolumntype{L}[1]{>{\raggedleft}m{#1}}
\newcolumntype{P}[1]{>{\raggedright}p{#1}}
\newcolumntype{B}[1]{>{\raggedright}b{#1}}
\newcolumntype{Q}[1]{>{\raggedright}t{#1}}

\newcommand{\alias}[2]{
\providecommand{#1}{}
\renewcommand{#1}{#2}
}
\alias{\R}{\mathbb{R}}
\alias{\N}{\mathbb{N}}
\alias{\Z}{\mathbb{Z}}
\alias{\Q}{\mathbb{Q}}
\alias{\C}{\mathbb{C}}
\alias{\K}{\mathbb{K}}

%%%%%%%%%%%%
%% rendre +infty et -infty plus petits
%%%%%%%%%%%%
\newcommand{\sinfty}{{\scriptstyle \infty}}

%%%%%%%%%%%%%%%%%%%%%%%%%%%%%%%
%%%%% macros TP Scilab %%%%%%%%
\newcommand{\Scilab}{\textbf{Scilab}} %
\newcommand{\Scinotes}{\textbf{SciNotes}} %
\newcommand{\faire}{\noindent $\blacktriangleright$ } %
\newcommand{\fitem}{\scalebox{.8}{$\blacktriangleright$}} %
\newcommand{\entree}{{\small\texttt{ENTRÉE}}} %
\newcommand{\tab}{{\small\texttt{TAB}}} %
\newcommand{\mt}[1]{\mathtt{#1}} %
% guillemets droits

\newcommand{\ttq}{\textquotesingle} %

\newcommand{\reponse}[1]{\longboxed{
    \begin{array}{C{0.9\textwidth}}
      \nl[#1]
    \end{array}
  }} %

\newcommand{\reponseL}[2]{\longboxed{
    \begin{array}{C{#2\textwidth}}
      \nl[#1]
    \end{array}
  }} %

\newcommand{\reponseR}[1]{\longboxed{
    \begin{array}{R{0.9\textwidth}}
      #1
    \end{array}
  }} %

\newcommand{\reponseRL}[2]{\longboxed{
    \begin{array}{R{#2\textwidth}}
      #1
    \end{array}
  }} %

\newcommand{\reponseC}[1]{\longboxed{
    \begin{array}{C{0.9\textwidth}}
      #1
    \end{array}
  }} %

\colorlet{pyfunction}{Blue}
\colorlet{pyCle}{Magenta}
\colorlet{pycomment}{LimeGreen}
\colorlet{pydoc}{Cyan}
% \colorlet{SansCo}{white}
% \colorlet{AvecCo}{black}

\newcommand{\visible}[1]{{\color{ASCo}\colorlet{pydoc}{pyDo}\colorlet{pycomment}{pyCo}\colorlet{pyfunction}{pyF}\colorlet{pyCle}{pyC}\colorlet{function}{sciFun}\colorlet{var}{sciVar}\colorlet{if}{sciIf}\colorlet{comment}{sciComment}#1}} %

%%%% à changer ????
\newcommand{\invisible}[1]{{\color{ASCo}\colorlet{pydoc}{pyDo}\colorlet{pycomment}{pyCo}\colorlet{pyfunction}{pyF}\colorlet{pyCle}{pyC}\colorlet{function}{sciFun}\colorlet{var}{sciVar}\colorlet{if}{sciIf}\colorlet{comment}{sciComment}#1}} %

\newcommand{\invisibleCol}[2]{{\color{#1}#2}} %

\NewEnviron{solution} %
{ %
  \Boxed{
    \begin{array}{>{\color{ASCo}} R{0.9\textwidth}}
      \colorlet{pycomment}{pyCo}
      \colorlet{pydoc}{pyDo}
      \colorlet{pyfunction}{pyF}
      \colorlet{pyCle}{pyC}
      \colorlet{function}{sciFun}
      \colorlet{var}{sciVar}
      \colorlet{if}{sciIf}
      \colorlet{comment}{sciComment}
      \BODY
    \end{array}
  } %
} %

\NewEnviron{solutionC} %
{ %
  \Boxed{
    \begin{array}{>{\color{ASCo}} C{0.9\textwidth}}
      \colorlet{pycomment}{pyCo}
      \colorlet{pydoc}{pyDo}
      \colorlet{pyfunction}{pyF}
      \colorlet{pyCle}{pyC}
      \colorlet{function}{sciFun}
      \colorlet{var}{sciVar}
      \colorlet{if}{sciIf}
      \colorlet{comment}{sciComment}
      \BODY
    \end{array}
  } %
} %

\newcommand{\invite}{--\!\!>} %

%%%%% nouvel environnement tabular pour retour console %%%%
\colorlet{ConsoleColor}{Blue!15}
\colorlet{function}{Red}
\colorlet{var}{Maroon}
\colorlet{if}{Magenta}
\colorlet{comment}{LimeGreen}

\newcommand{\tcVar}[1]{\textcolor{var}{\bf \small #1}} %
\newcommand{\tcFun}[1]{\textcolor{function}{#1}} %
\newcommand{\tcIf}[1]{\textcolor{if}{#1}} %
\newcommand{\tcFor}[1]{\textcolor{if}{#1}} %

\newcommand{\moins}{\!\!\!\!\!\!- }
\newcommand{\espn}{\!\!\!\!\!\!}

\usepackage{booktabs,varwidth} \newsavebox\TBox
\newenvironment{console}
{\begin{lrbox}{\TBox}\varwidth{\linewidth}
    \tabular{>{\tt\small}R{0.84\textwidth}}
    \nl[-.4cm]} {\endtabular\endvarwidth\end{lrbox}%
  \fboxsep=1pt\colorbox{ConsoleColor}{\usebox\TBox}}

\newcommand{\lInv}[1]{%
  $\invite$ #1} %

\newcommand{\lAns}[1]{%
  \qquad ans \ = \nl %
  \qquad \qquad #1} %

\newcommand{\lVar}[2]{%
  \qquad #1 \ = \nl %
  \qquad \qquad #2} %

\newcommand{\lDisp}[1]{%
  #1 %
} %

\newcommand{\ligne}[1]{\underline{\small \tt #1}} %

\newcommand{\ligneAns}[2]{%
  $\invite$ #1 \nl %
  \qquad ans \ = \nl %
  \qquad \qquad #2} %

\newcommand{\ligneVar}[3]{%
  $\invite$ #1 \nl %
  \qquad #2 \ = \nl %
  \qquad \qquad #3} %

\newcommand{\ligneErr}[3]{%
  $\invite$ #1 \nl %
  \quad !-{-}error #2 \nl %
  #3} %
%%%%%%%%%%%%%%%%%%%%%% 

\newcommand{\bs}[1]{\boldsymbol{#1}} %
\newcommand{\nll}{\nl[.4cm]} %
\newcommand{\nle}{\nl[.2cm]} %
%% opérateur puissance copiant l'affichage Scilab
%\newcommand{\puis}{\!\!\!~^{\scriptscriptstyle\pmb{\wedge}}}
\newcommand{\puis}{\mbox{$\hspace{-.1cm}~^{\scriptscriptstyle\pmb{\wedge}}
    \hspace{0.05cm}$}} %
\newcommand{\pointpuis}{.\mbox{$\hspace{-.15cm}~^{\scriptscriptstyle\pmb{\wedge}}$}} %
\newcommand{\Sfois}{\mbox{$\mt{\star}$}} %

%%%%% nouvel environnement tabular pour les encadrés Scilab %%%%
\newenvironment{encadre}
{\begin{lrbox}{\TBox}\varwidth{\linewidth}
    \tabular{>{\tt\small}C{0.1\textwidth}>{\small}R{0.7\textwidth}}}
  {\endtabular\endvarwidth\end{lrbox}%
  \fboxsep=1pt\longboxed{\usebox\TBox}}

\newenvironment{encadreL}
{\begin{lrbox}{\TBox}\varwidth{\linewidth}
    \tabular{>{\tt\small}C{0.25\textwidth}>{\small}R{0.6\textwidth}}}
  {\endtabular\endvarwidth\end{lrbox}%
  \fboxsep=1pt\longboxed{\usebox\TBox}}

\newenvironment{encadreF}
{\begin{lrbox}{\TBox}\varwidth{\linewidth}
    \tabular{>{\tt\small}C{0.2\textwidth}>{\small}R{0.70\textwidth}}}
  {\endtabular\endvarwidth\end{lrbox}%
  \fboxsep=1pt\longboxed{\usebox\TBox}}

\newenvironment{encadreLL}[2]
{\begin{lrbox}{\TBox}\varwidth{\linewidth}
    \tabular{>{\tt\small}C{#1\textwidth}>{\small}R{#2\textwidth}}}
  {\endtabular\endvarwidth\end{lrbox}%
  \fboxsep=1pt\longboxed{\usebox\TBox}}

%%%%% nouvel environnement tabular pour les script et fonctions %%%%
\newcommand{\commentaireDL}[1]{\multicolumn{1}{l}{\it
    \textcolor{comment}{$\slash\slash$ #1}}}

\newcommand{\commentaire}[1]{{\textcolor{comment}{$\slash\slash$ #1}}}

\newcounter{cptcol}

\newcommand{\nocount}{\multicolumn{1}{c}{}}

\newcommand{\sciNo}[1]{{\small \underbar #1}}

\NewEnviron{scilab}{ %
  \setcounter{cptcol}{0}
  \begin{center}
    \longboxed{
      \begin{tabular}{>{\stepcounter{cptcol}{\tiny \underbar
              \thecptcol}}c>{\tt}l}
        \BODY
      \end{tabular}
    }
  \end{center}
}

\NewEnviron{scilabNC}{ %
  \begin{center}
    \longboxed{
      \begin{tabular}{>{\tt}l} %
          \BODY
      \end{tabular}
    }
  \end{center}
}

\NewEnviron{scilabC}[1]{ %
  \setcounter{cptcol}{#1}
  \begin{center}
    \longboxed{
      \begin{tabular}{>{\stepcounter{cptcol}{\tiny \underbar
              \thecptcol}}c>{\tt}l}
        \BODY
      \end{tabular}
    }
  \end{center}
}

\newcommand{\scisol}[1]{ %
  \setcounter{cptcol}{0}
  \longboxed{
    \begin{tabular}{>{\stepcounter{cptcol}{\tiny \underbar
            \thecptcol}}c>{\tt}l}
      #1
    \end{tabular}
  }
}

\newcommand{\scisolNC}[1]{ %
  \longboxed{
    \begin{tabular}{>{\tt}l}
      #1
    \end{tabular}
  }
}

\newcommand{\scisolC}[2]{ %
  \setcounter{cptcol}{#1}
  \longboxed{
    \begin{tabular}{>{\stepcounter{cptcol}{\tiny \underbar
            \thecptcol}}c>{\tt}l}
      #2
    \end{tabular}
  }
}

\NewEnviron{syntaxe}{ %
  % \fcolorbox{black}{Yellow!20}{\setlength{\fboxsep}{3mm}
  \shadowbox{
    \setlength{\fboxsep}{3mm}
    \begin{tabular}{>{\tt}l}
      \BODY
    \end{tabular}
  }
}

%%%%% fin macros TP Scilab %%%%%%%%
%%%%%%%%%%%%%%%%%%%%%%%%%%%%%%%%%%%

%%%%%%%%%%%%%%%%%%%%%%%%%%%%%%%%%%%
%%%%% TP Python - listings %%%%%%%%
%%%%%%%%%%%%%%%%%%%%%%%%%%%%%%%%%%%
\newcommand{\Python}{\textbf{Python}} %

\lstset{% general command to set parameter(s)
basicstyle=\ttfamily\small, % print whole listing small
keywordstyle=\color{blue}\bfseries\underbar,
%% underlined bold black keywords
frame=lines,
xleftmargin=10mm,
numbers=left,
numberstyle=\tiny\underbar,
numbersep=10pt,
%identifierstyle=, % nothing happens
commentstyle=\color{green}, % white comments
%%stringstyle=\ttfamily, % typewriter type for strings
showstringspaces=false}

\newcommand{\pysolCpt}[2]{
  \setcounter{cptcol}{#1}
  \longboxed{
    \begin{tabular}{>{\stepcounter{cptcol}{\tiny \underbar
            \thecptcol}}c>{\tt}l}
        #2
      \end{tabular}
    }
} %

\newcommand{\pysol}[1]{
  \setcounter{cptcol}{0}
  \longboxed{
    \begin{tabular}{>{\stepcounter{cptcol}{\tiny \underbar
            \thecptcol}}c>{\tt}l}
        #1
      \end{tabular}
    }
} %

% \usepackage[labelsep=endash]{caption}

% avec un caption
\NewEnviron{pythonCap}[1]{ %
  \renewcommand{\tablename}{Programme}
  \setcounter{cptcol}{0}
  \begin{center}
    \longboxed{
      \begin{tabular}{>{\stepcounter{cptcol}{\tiny \underbar
              \thecptcol}}c>{\tt}l}
        \BODY
      \end{tabular}
    }
    \captionof{table}{#1}
  \end{center}
}

\NewEnviron{python}{ %
  \setcounter{cptcol}{0}
  \begin{center}
    \longboxed{
      \begin{tabular}{>{\stepcounter{cptcol}{\tiny \underbar
              \thecptcol}}c>{\tt}l}
        \BODY
      \end{tabular}
    }
  \end{center}
}

\newcommand{\pyVar}[1]{\textcolor{var}{\bf \small #1}} %
\newcommand{\pyFun}[1]{\textcolor{pyfunction}{#1}} %
\newcommand{\pyCle}[1]{\textcolor{pyCle}{#1}} %
\newcommand{\pyImp}[1]{{\bf #1}} %

%%%%% commentaire python %%%%
\newcommand{\pyComDL}[1]{\multicolumn{1}{l}{\textcolor{pycomment}{\#
      #1}}}

\newcommand{\pyCom}[1]{{\textcolor{pycomment}{\# #1}}}
\newcommand{\pyDoc}[1]{{\textcolor{pydoc}{#1}}}

\newcommand{\pyNo}[1]{{\small \underbar #1}}

%%%%%%%%%%%%%%%%%%%%%%%%%%%%%%%%%%%
%%%%%% Système linéaire paramétré : écrire les opérations au-dessus
%%%%%% d'un symbole équivalent
%%%%%%%%%%%%%%%%%%%%%%%%%%%%%%%%%%%

\usepackage{systeme}

\NewEnviron{arrayEq}{ %
  \stackrel{\scalebox{.6}{$
      \begin{array}{l} 
        \BODY \\[.1cm]
      \end{array}$}
  }{\Longleftrightarrow}
}

\NewEnviron{arrayEg}{ %
  \stackrel{\scalebox{.6}{$
      \begin{array}{l} 
        \BODY \\[.1cm]
      \end{array}$}
  }{=}
}

\NewEnviron{operationEq}{ %
  \scalebox{.6}{$
    \begin{array}{l} 
      \scalebox{1.6}{$\mbox{Opérations :}$} \\[.2cm]
      \BODY \\[.1cm]
    \end{array}$}
}

% \NewEnviron{arraySys}[1]{ %
%   \sysdelim\{.\systeme[#1]{ %
%     \BODY %
%   } %
% }

%%%%%

%%%%%%%%%%
%%%%%%%%%% ESSAI
\newlength\fboxseph
\newlength\fboxsepva
\newlength\fboxsepvb

\setlength\fboxsepva{0.2cm}
\setlength\fboxsepvb{0.2cm}
\setlength\fboxseph{0.2cm}

\makeatletter

\def\longboxed#1{\leavevmode\setbox\@tempboxa\hbox{\color@begingroup%
\kern\fboxseph{\m@th$\displaystyle #1 $}\kern\fboxseph%
\color@endgroup }\my@frameb@x\relax}

\def\my@frameb@x#1{%
  \@tempdima\fboxrule \advance\@tempdima \fboxsepva \advance\@tempdima
  \dp\@tempboxa\hbox {%
    \lower \@tempdima \hbox {%
      \vbox {\hrule\@height\fboxrule \hbox{\vrule\@width\fboxrule #1
          \vbox{%
            \vskip\fboxsepva \box\@tempboxa \vskip\fboxsepvb}#1
          \vrule\@width\fboxrule }%
        \hrule \@height \fboxrule }}}}

\newcommand{\boxedhv}[3]{\setlength\fboxseph{#1cm}
  \setlength\fboxsepva{#2cm}\setlength\fboxsepvb{#2cm}\longboxed{#3}}

\newcommand{\boxedhvv}[4]{\setlength\fboxseph{#1cm}
  \setlength\fboxsepva{#2cm}\setlength\fboxsepvb{#3cm}\longboxed{#4}}

\newcommand{\Boxed}[1]{{\setlength\fboxseph{0.2cm}
  \setlength\fboxsepva{0.2cm}\setlength\fboxsepvb{0.2cm}\longboxed{#1}}}

\newcommand{\mBoxed}[1]{{\setlength\fboxseph{0.2cm}
  \setlength\fboxsepva{0.2cm}\setlength\fboxsepvb{0.2cm}\longboxed{\mbox{#1}}}}

\newcommand{\mboxed}[1]{{\setlength\fboxseph{0.2cm}
  \setlength\fboxsepva{0.2cm}\setlength\fboxsepvb{0.2cm}\boxed{\mbox{#1}}}}

\newsavebox{\fmbox}
\newenvironment{fmpage}[1]
     {\begin{lrbox}{\fmbox}\begin{minipage}{#1}}
     {\end{minipage}\end{lrbox}\fbox{\usebox{\fmbox}}}

%%%%%%%%%%
%%%%%%%%%%

\DeclareMathOperator{\ch}{ch}
\DeclareMathOperator{\sh}{sh}

%%%%%%%%%%
%%%%%%%%%%

\newcommand{\norme}[1]{\Vert #1 \Vert}

%\newcommand*\widefbox[1]{\fbox{\hspace{2em}#1\hspace{2em}}}

\newcommand{\nl}{\tabularnewline}

\newcommand{\hand}{\noindent\ding{43}\ }
\newcommand{\ie}{\textit{i.e. }}
\newcommand{\cf}{\textit{cf }}

\newcommand{\Card}{\operatorname{Card}}

\newcommand{\aire}{\mathcal{A}}

\newcommand{\LL}[1]{\mathscr{L}(#1)} %
\newcommand{\B}{\mathscr{B}} %
\newcommand{\Bc}[1]{B_{#1}} %
\newcommand{\M}[1]{\mathscr{M}_{#1}(\mathbb{R})}

\DeclareMathOperator{\im}{Im}
\DeclareMathOperator{\kr}{Ker}
\DeclareMathOperator{\rg}{rg}
\DeclareMathOperator{\tr}{tr}
\DeclareMathOperator{\spc}{Sp}
\DeclareMathOperator{\sgn}{sgn}
\DeclareMathOperator{\supp}{Supp}

\newcommand{\Mat}{{\rm{Mat}}}
\newcommand{\Vect}[1]{{\rm{Vect}}\left(#1\right)}

\newenvironment{smatrix}{%
  \begin{adjustbox}{width=.9\width}
    $
    \begin{pmatrix}
    }{%      
    \end{pmatrix}
    $
  \end{adjustbox}
}

\newenvironment{sarray}[1]{%
  \begin{adjustbox}{width=.9\width}
    $
    \begin{array}{#1}
    }{%      
    \end{array}
    $
  \end{adjustbox}
}

\newcommand{\vd}[2]{
  \scalebox{.8}{
    $\left(\!
      \begin{array}{c}
        #1 \\
        #2
      \end{array}
    \!\right)$
    }}

\newcommand{\vt}[3]{
  \scalebox{.8}{
    $\left(\!
      \begin{array}{c}
        #1 \\
        #2 \\
        #3 
      \end{array}
    \!\right)$
    }}

\newcommand{\vq}[4]{
  \scalebox{.8}{
    $\left(\!
      \begin{array}{c}
        #1 \\
        #2 \\
        #3 \\
        #4 
      \end{array}
    \!\right)$
    }}

\newcommand{\vc}[5]{
  \scalebox{.8}{
    $\left(\!
      \begin{array}{c}
        #1 \\
        #2 \\
        #3 \\
        #4 \\
        #5 
      \end{array}
    \!\right)$
    }}

\newcommand{\ee}{\text{e}}

\newcommand{\dd}{\text{d}}

%%% Ensemble de définition
\newcommand{\Df}{\mathscr{D}}
\newcommand{\Cf}{\mathscr{C}}
\newcommand{\Ef}{\mathscr{C}}

\newcommand{\rond}[1]{\,\overset{\scriptscriptstyle \circ}{\!#1}}

\newcommand{\df}[2]{\dfrac{\partial #1}{\partial #2}} %
\newcommand{\dfn}[2]{\partial_{#2}(#1)} %
\newcommand{\ddfn}[2]{\partial^2_{#2}(#1)} %
\newcommand{\ddf}[2]{\dfrac{\partial^2 #1}{\partial #2^2}} %
\newcommand{\ddfr}[3]{\dfrac{\partial^2 #1}{\partial #2 \partial
    #3}} %


\newcommand{\dlim}[1]{{\displaystyle \lim_{#1} \ }}
\newcommand{\dlimPlus}[2]{
  \dlim{
    \scalebox{.6}{
      $
      \begin{array}{l}
        #1 \rightarrow #2\\
        #1 > #2
      \end{array}
      $}}}
\newcommand{\dlimMoins}[2]{
  \dlim{
    \scalebox{.6}{
      $
      \begin{array}{l}
        #1 \rightarrow #2\\
        #1 < #2
      \end{array}
      $}}}

%%%%%%%%%%%%%%
%% petit o, développement limité
%%%%%%%%%%%%%%

\newcommand{\oo}[2]{{\underset {{\overset {#1\rightarrow #2}{}}}{o}}} %
\newcommand{\oox}[1]{{\underset {{\overset {x\rightarrow #1}{}}}{o}}} %
\newcommand{\oon}{{\underset {{\overset {n\rightarrow +\infty}{}}}{o}}} %
\newcommand{\po}[1]{{\underset {{\overset {#1}{}}}{o}}} %
\newcommand{\neqx}[1]{{\ \underset {{\overset {x \to #1}{}}}{\not\sim}\ }} %
\newcommand{\eqx}[1]{{\ \underset {{\overset {x \to #1}{}}}{\sim}\ }} %
\newcommand{\eqn}{{\ \underset {{\overset {n \to +\infty}{}}}{\sim}\ }} %
\newcommand{\eq}[2]{{\ \underset {{\overset {#1 \to #2}{}}}{\sim}\ }} %
\newcommand{\DL}[1]{{\rm{DL}}_1 (#1)} %
\newcommand{\DLL}[1]{{\rm{DL}}_2 (#1)} %

\newcommand{\negl}{<<}

\newcommand{\neglP}[1]{\begin{array}{c}
    \vspace{-.2cm}\\
    << \\
    \vspace{-.7cm}\\
    {\scriptstyle #1}
  \end{array}}

%%%%%%%%%%%%%%
%% borne sup, inf, max, min
%%%%%%%%%%%%%%
\newcommand{\dsup}[1]{\displaystyle \sup_{#1} \ }
\newcommand{\dinf}[1]{\displaystyle \inf_{#1} \ }
\newcommand{\dmax}[1]{\max\limits_{#1} \ }
\newcommand{\dmin}[1]{\min\limits_{#1} \ }

\newcommand{\dcup}[2]{{\textstyle\bigcup\limits_{#1}^{#2}}\hspace{.1cm}}
%\displaystyle \bigcup_{#1}^{#2}}
\newcommand{\dcap}[2]{{\textstyle\bigcap\limits_{#1}^{#2}}\hspace{.1cm}}
% \displaystyle \bigcap_{#1}^{#2}
%%%%%%%%%%%%%%
%% opérateurs logiques
%%%%%%%%%%%%%%
\newcommand{\NON}[1]{\mathop{\small \tt{NON}} (#1)}
\newcommand{\ET}{\mathrel{\mathop{\small \mathtt{ET}}}}
\newcommand{\OU}{\mathrel{\mathop{\small \tt{OU}}}}
\newcommand{\XOR}{\mathrel{\mathop{\small \tt{XOR}}}}

\newcommand{\id}{{\rm{id}}}

\newcommand{\sbullet}{\scriptstyle \bullet}
\newcommand{\stimes}{\scriptstyle \times}

%%%%%%%%%%%%%%%%%%
%% Probabilités
%%%%%%%%%%%%%%%%%%
\newcommand{\Prob}{\mathbb{P}}
\newcommand{\Ev}[1]{\left[ {#1} \right]}
\newcommand{\E}{\mathbb{E}}
\newcommand{\V}{\mathbb{V}}
\newcommand{\Cov}{{\rm{Cov}}}
\newcommand{\U}[2]{\mathcal{U}(\llb #1, #2\rrb)}
\newcommand{\Uc}[2]{\mathcal{U}([#1, #2])}
\newcommand{\Ucof}[2]{\mathcal{U}(]#1, #2])}
\newcommand{\Ucoo}[2]{\mathcal{U}(]#1, #2[)}
\newcommand{\Ucfo}[2]{\mathcal{U}([#1, #2[)}
\newcommand{\Bern}[1]{\mathcal{B}\left(#1\right)}
\newcommand{\Bin}[2]{\mathcal{B}\left(#1, #2\right)}
\newcommand{\G}[1]{\mathcal{G}\left(#1\right)}
\newcommand{\Pois}[1]{\mathcal{P}\left(#1\right)}
\newcommand{\HG}[3]{\mathcal{H}\left(#1, #2, #3\right)}
\newcommand{\Exp}[1]{\mathcal{E}\left(#1\right)}
\newcommand{\Norm}[2]{\mathcal{N}\left(#1, #2\right)}

\DeclareMathOperator{\cov}{Cov}

\newcommand{\var}{v.a.r. }
\newcommand{\suit}{\hookrightarrow}

\newcommand{\flecheR}[1]{\rotatebox{90}{\scalebox{#1}{\color{red}
      $\curvearrowleft$}}}


\newcommand{\partie}[1]{\mathcal{P}(#1)}
\newcommand{\Cont}[1]{\mathcal{C}^{#1}}
\newcommand{\Contm}[1]{\mathcal{C}^{#1}_m}

\newcommand{\llb}{\llbracket}
\newcommand{\rrb}{\rrbracket}

%\newcommand{\im}[1]{{\rm{Im}}(#1)}
\newcommand{\imrec}[1]{#1^{- \mathds{1}}}

\newcommand{\unq}{\mathds{1}}

\newcommand{\Hyp}{\mathtt{H}}

\newcommand{\eme}[1]{#1^{\scriptsize \mbox{ème}}}
\newcommand{\er}[1]{#1^{\scriptsize \mbox{er}}}
\newcommand{\ere}[1]{#1^{\scriptsize \mbox{ère}}}
\newcommand{\nd}[1]{#1^{\scriptsize \mbox{nd}}}
\newcommand{\nde}[1]{#1^{\scriptsize \mbox{nde}}}

\newcommand{\truc}{\mathop{\top}}
\newcommand{\fois}{\mathop{\ast}}

\newcommand{\f}[1]{\overrightarrow{#1}}

\newcommand{\checked}{\textcolor{green}{\checkmark}}

\def\restriction#1#2{\mathchoice
              {\setbox1\hbox{${\displaystyle #1}_{\scriptstyle #2}$}
              \restrictionaux{#1}{#2}}
              {\setbox1\hbox{${\textstyle #1}_{\scriptstyle #2}$}
              \restrictionaux{#1}{#2}}
              {\setbox1\hbox{${\scriptstyle #1}_{\scriptscriptstyle #2}$}
              \restrictionaux{#1}{#2}}
              {\setbox1\hbox{${\scriptscriptstyle #1}_{\scriptscriptstyle #2}$}
              \restrictionaux{#1}{#2}}}
\def\restrictionaux#1#2{{#1\,\smash{\vrule height .8\ht1 depth .85\dp1}}_{\,#2}}

\makeatletter
\newcommand*{\da@rightarrow}{\mathchar"0\hexnumber@\symAMSa 4B }
\newcommand*{\da@leftarrow}{\mathchar"0\hexnumber@\symAMSa 4C }
\newcommand*{\xdashrightarrow}[2][]{%
  \mathrel{%
    \mathpalette{\da@xarrow{#1}{#2}{}\da@rightarrow{\,}{}}{}%
  }%
}
\newcommand{\xdashleftarrow}[2][]{%
  \mathrel{%
    \mathpalette{\da@xarrow{#1}{#2}\da@leftarrow{}{}{\,}}{}%
  }%
}
\newcommand*{\da@xarrow}[7]{%
  % #1: below
  % #2: above
  % #3: arrow left
  % #4: arrow right
  % #5: space left 
  % #6: space right
  % #7: math style 
  \sbox0{$\ifx#7\scriptstyle\scriptscriptstyle\else\scriptstyle\fi#5#1#6\m@th$}%
  \sbox2{$\ifx#7\scriptstyle\scriptscriptstyle\else\scriptstyle\fi#5#2#6\m@th$}%
  \sbox4{$#7\dabar@\m@th$}%
  \dimen@=\wd0 %
  \ifdim\wd2 >\dimen@
    \dimen@=\wd2 %   
  \fi
  \count@=2 %
  \def\da@bars{\dabar@\dabar@}%
  \@whiledim\count@\wd4<\dimen@\do{%
    \advance\count@\@ne
    \expandafter\def\expandafter\da@bars\expandafter{%
      \da@bars
      \dabar@ 
    }%
  }%  
  \mathrel{#3}%
  \mathrel{%   
    \mathop{\da@bars}\limits
    \ifx\\#1\\%
    \else
      _{\copy0}%
    \fi
    \ifx\\#2\\%
    \else
      ^{\copy2}%
    \fi
  }%   
  \mathrel{#4}%
}
\makeatother



\newcount\depth

\newcount\depth
\newcount\totaldepth

\makeatletter
\newcommand{\labelsymbol}{%
      \ifnum\depth=0
        %
      \else
        \rlap{\,$\bullet$}%
      \fi
}

\newcommand*\bernoulliTree[1]{%
    \depth=#1\relax            
    \totaldepth=#1\relax
    \draw node(root)[bernoulli/root] {\labelsymbol}[grow=right] \draw@bernoulli@tree;
    \draw \label@bernoulli@tree{root};                                   
}                                                                        

\def\draw@bernoulli@tree{%
    \ifnum\depth>0 
      child[parent anchor=east] foreach \type/\label in {left child/$E$,right child/$S$} {%
          node[bernoulli/\type] {\label\strut\labelsymbol} \draw@bernoulli@tree
      }
      coordinate[bernoulli/increment] (dummy)
   \fi%
}

\def\label@bernoulli@tree#1{%
    \ifnum\depth>0
      ($(#1)!0.5!(#1-1)$) node[fill=white,bernoulli/decrement] {\tiny$p$}
      \label@bernoulli@tree{#1-1}
      ($(#1)!0.5!(#1-2)$) node[fill=white] {\tiny$q$}
      \label@bernoulli@tree{#1-2}
      coordinate[bernoulli/increment] (dummy)
   \fi%
}

\makeatother

\tikzset{bernoulli/.cd,
         root/.style={},
         decrement/.code=\global\advance\depth by-1\relax,
         increment/.code=\global\advance\depth by 1\relax,
         left child/.style={bernoulli/decrement},
         right child/.style={}}


\newcommand{\eps}{\varepsilon}

% \newcommand{\tendi}[1]{\xrightarrow[\footnotesize #1 \rightarrow
%   +\infty]{}}%

\newcommand{\tend}{\rightarrow}%
\newcommand{\tendn}{\underset{n\to +\infty}{\longrightarrow}} %
\newcommand{\ntendn}{\underset{n\to
    +\infty}{\not\hspace{-.15cm}\longrightarrow}} %
% \newcommand{\tendn}{\xrightarrow[\footnotesize n \rightarrow
%   +\infty]{}}%
\newcommand{\Tendx}[1]{\xrightarrow[\footnotesize x \rightarrow
  #1]{}}%
\newcommand{\tendx}[1]{\underset{x\to #1}{\longrightarrow}}%
\newcommand{\ntendx}[1]{\underset{x\to #1}{\not\!\!\longrightarrow}}%
\newcommand{\tendd}[2]{\underset{#1\to #2}{\longrightarrow}}%
% \newcommand{\tendd}[2]{\xrightarrow[\footnotesize #1 \rightarrow
%   #2]{}}%
\newcommand{\tendash}[1]{\xdashrightarrow[\footnotesize #1 \rightarrow
  +\infty]{}}%
\newcommand{\tendashx}[1]{\xdashrightarrow[\footnotesize x \rightarrow
  #1]{}}%
\newcommand{\tendb}[1]{\underset{#1 \to +\infty}{\longrightarrow}}%
\newcommand{\tendL}{\overset{\mathscr L}{\underset{n \to
      +\infty}{\longrightarrow}}}%
\newcommand{\tendP}{\overset{\Prob}{\underset{n \to
      +\infty}{\longrightarrow}}}%
\newcommand{\tenddL}[1]{\overset{\mathscr L}{\underset{#1 \to
      +\infty}{\longrightarrow}}}%

\NewEnviron{attention}{ %
  ~\\[-.2cm]\noindent
  \begin{minipage}{\linewidth}
  \setlength{\fboxsep}{3mm}%
  \ \ \dbend \ \ %
  \fbox{\parbox[t]{.88\linewidth}{\BODY}} %
  \end{minipage}\\
}

\NewEnviron{sattention}[1]{ %
  ~\\[-.2cm]\noindent
  \begin{minipage}{#1\linewidth}
  \setlength{\fboxsep}{3mm}%
  \ \ \dbend \ \ %
  \fbox{\parbox[t]{.88\linewidth}{\BODY}} %
  \end{minipage}\\
}

%%%%% OBSOLETE %%%%%%

% \newcommand{\attention}[1]{
%   \noindent
%   \begin{tabular}{@{}l|p{11.5cm}|}
%     \cline{2-2}
%     \vspace{-.2cm} 
%     & \nl
%     \dbend & #1 \nl
%     \cline{2-2}
%   \end{tabular}
% }

% \newcommand{\attentionv}[2]{
%   \noindent
%   \begin{tabular}{@{}l|p{11.5cm}|}
%     \cline{2-2}
%     \vspace{-.2cm} 
%     & \nl
%     \dbend & #2 \nl[#1 cm]
%     \cline{2-2}
%   \end{tabular}
% }

\newcommand{\explainvb}[2]{
  \noindent
  \begin{tabular}{@{}l|p{11.5cm}|}
    \cline{2-2}
    \vspace{-.2cm} 
    & \nl
    \hand & #2 \nl [#1 cm]
    \cline{2-2}
  \end{tabular}
}


% \noindent
% \begin{tabular}{@{}l|lp{11cm}|}
%   \cline{3-3} 
%   \multicolumn{1}{@{}l@{\dbend}}{} & & #1 \nl
%   \multicolumn{1}{l}{} & & \nl [-.8cm]
%   & & #2 \nl
%   \cline{2-3}
% \end{tabular}

% \newcommand{\attention}[1]{
%   \noindent
%   \begin{tabular}{@{}@{}cp{11cm}}
%     \dbend & #1 \nl
%   \end{tabular}
% }

\newcommand{\PP}[1]{\mathcal{P}(#1)}
\newcommand{\FF}[1]{\mathcal{F}(#1)}

\newcommand{\DSum}[2]{\displaystyle\sum\limits_{#1}^{#2}\hspace{.1cm}}
\newcommand{\Sum}[2]{{\textstyle\sum\limits_{#1}^{#2}}\hspace{.1cm}}
\newcommand{\Serie}{\textstyle\sum\ }
\newcommand{\Prod}[2]{\textstyle\prod\limits_{#1}^{#2}}

\newcommand{\Prim}[3]{\left[\ {#1} \ \right]_{\scriptscriptstyle
   \hspace{-.15cm} ~_{#2}\, }^ {\scriptscriptstyle \hspace{-.15cm} ~^{#3}\, }}

% \newcommand{\Prim}[3]{\left[\ {#1} \ \right]_{\scriptscriptstyle
%     \!\!~_{#2}}^ {\scriptscriptstyle \!\!~^{#3}}}

\newcommand{\dint}[2]{\displaystyle \int_{#1}^{#2}\ }
\newcommand{\Int}[2]{{\rm{Int}}_{\scriptscriptstyle #1, #2}}
\newcommand{\dt}{\ dt}
\newcommand{\dx}{\ dx}

\newcommand{\llpar}[1]{\left(\!\!\!
    \begin{array}{c}
      \rule{0pt}{#1}
    \end{array}
  \!\!\!\right.}

\newcommand{\rrpar}[1]{\left.\!\!\!
    \begin{array}{c}
      \rule{0pt}{#1}
    \end{array}
  \!\!\!\right)}

\newcommand{\llacc}[1]{\left\{\!\!\!
    \begin{array}{c}
      \rule{0pt}{#1}
    \end{array}
  \!\!\!\right.}

\newcommand{\rracc}[1]{\left.\!\!\!
    \begin{array}{c}
      \rule{0pt}{#1}
    \end{array}
  \!\!\!\right\}}

\newcommand{\ttacc}[1]{\mbox{\rotatebox{-90}{\hspace{-.7cm}$\llacc{#1}$}}}
\newcommand{\bbacc}[1]{\mbox{\rotatebox{90}{\hspace{-.5cm}$\llacc{#1}$}}}

\newcommand{\comp}[1]{\overline{#1}}%

\newcommand{\dcomp}[2]{\stackrel{\mbox{\ \ \----}{\scriptscriptstyle
      #2}}{#1}}%

% \newcommand{\Comp}[2]{\stackrel{\mbox{\ \
%       \-------}{\scriptscriptstyle #2}}{#1}}

% \newcommand{\dcomp}[2]{\stackrel{\mbox{\ \
%       \-------}{\scriptscriptstyle #2}}{#1}}

\newcommand{\A}{\mathscr{A}}

\newcommand{\conc}[1]{
  \begin{center}
    \fbox{
      \begin{tabular}{c}
        #1
      \end{tabular}
    }
  \end{center}
}

\newcommand{\concC}[1]{
  \begin{center}
    \fbox{
    \begin{tabular}{C{10cm}}
      \quad #1 \quad
    \end{tabular}
    }
  \end{center}
}

\newcommand{\concL}[2]{
  \begin{center}
    \fbox{
    \begin{tabular}{C{#2cm}}
      \quad #1 \quad
    \end{tabular}
    }
  \end{center}
}

% \newcommand{\lims}[2]{\prod\limits_{#1}^{#2}}

\newtheorem{theorem}{Théorème}[]
\newtheorem{lemma}{Lemme}[]
\newtheorem{proposition}{Proposition}[]
\newtheorem{corollary}{Corollaire}[]

% \newenvironment{proof}[1][Démonstration]{\begin{trivlist}
% \item[\hskip \labelsep {\bfseries #1}]}{\end{trivlist}}
\newenvironment{definition}[1][Définition]{\begin{trivlist}
\item[\hskip \labelsep {\bfseries #1}]}{\end{trivlist}}
\newenvironment{example}[1][Exemple]{\begin{trivlist}
\item[\hskip \labelsep {\bfseries #1}]}{\end{trivlist}}
\newenvironment{examples}[1][Exemples]{\begin{trivlist}
\item[\hskip \labelsep {\bfseries #1}]}{\end{trivlist}}
\newenvironment{notation}[1][Notation]{\begin{trivlist}
\item[\hskip \labelsep {\bfseries #1}]}{\end{trivlist}}
\newenvironment{propriete}[1][Propriété]{\begin{trivlist}
\item[\hskip \labelsep {\bfseries #1}]}{\end{trivlist}}
\newenvironment{proprietes}[1][Propriétés]{\begin{trivlist}
\item[\hskip \labelsep {\bfseries #1}]}{\end{trivlist}}
\newenvironment{remarkSC}[1][Remarque]{\begin{trivlist}
\item[\hskip \labelsep {\bfseries #1}]}{\end{trivlist}}
\newenvironment{application}[1][Application]{\begin{trivlist}
\item[\hskip \labelsep {\bfseries #1}]}{\end{trivlist}}

% Environnement pour les réponses des DS
\newenvironment{answer}{\par\emph{Réponse :}\par{}}
{\vspace{-.6cm}\hspace{\stretch{1}}\rule{1ex}{1ex}\vspace{.3cm}}

\newenvironment{answerTD}{\vspace{.2cm}\par\emph{Réponse :}\par{}}
{\hspace{\stretch{1}}\rule{1ex}{1ex}\vspace{.3cm}}

\newenvironment{answerCours}{\noindent\emph{Réponse :}}
{\rule{1ex}{1ex}}%\vspace{.3cm}}


% footnote in footer
\newcommand{\fancyfootnotetext}[2]{%
  \fancypagestyle{dingens}{%
    \fancyfoot[LO,RE]{\parbox{0.95\textwidth}{\footnotemark[#1]\footnotesize
        #2}}%
  }%
  \thispagestyle{dingens}%
}

%%% définit le style (arabic : 1,2,3...) et place des parenthèses
%%% autour de la numérotation
\renewcommand*{\thefootnote}{(\arabic{footnote})}
% http://www.tuteurs.ens.fr/logiciels/latex/footnote.html

%%%%%%%% tikz axis
% \pgfplotsset{every axis/.append style={
%                     axis x line=middle,    % put the x axis in the middle
%                     axis y line=middle,    % put the y axis in the middle
%                     axis line style={<->,color=blue}, % arrows on the axis
%                     xlabel={$x$},          % default put x on x-axis
%                     ylabel={$y$},          % default put y on y-axis
%             }}

%%%% s'utilise comme suit

% \begin{axis}[
%   xmin=-8,xmax=4,
%   ymin=-8,ymax=4,
%   grid=both,
%   ]
%   \addplot [domain=-3:3,samples=50]({x^3-3*x},{3*x^2-9}); 
% \end{axis}

%%%%%%%%



%%%%%%%%%%%% Pour avoir des numéros de section qui correspondent à
%%%%%%%%%%%% ceux du tableau
\renewcommand{\thesection}{\Roman{section}.\hspace{-.3cm}}
\renewcommand{\thesubsection}{\Roman{section}.\arabic{subsection}.\hspace{-.2cm}}
\renewcommand{\thesubsubsection}{\Roman{section}.\arabic{subsection}.\alph{subsubsection})\hspace{-.2cm}}
%%%%%%%%%%%% 

%%% Changer le nom des figures : Fig. au lieu de Figure
\usepackage[font=small,labelfont=bf,labelsep=space]{caption}
\captionsetup{%
  figurename=Fig.,
  tablename=Tab.
}
% \renewcommand{\thesection}{\Roman{section}.\hspace{-.2cm}}
% \renewcommand{\thesubsection}{\Roman{section}
%   .\hspace{.2cm}\arabic{subsection}\ .\hspace{-.3cm}}
% \renewcommand{\thesubsubsection}{\alph{subsection})}

\newenvironment{tabliste}[1]
{\begin{tabenum}[\bfseries\small\itshape #1]}{\end{tabenum}} 

\newenvironment{noliste}[1] %
{\begin{enumerate}[\bfseries\small\itshape #1]} %
  {\end{enumerate}}

\newenvironment{nonoliste}[1] %
{\begin{enumerate}[\hspace{-12pt}\bfseries\small\itshape #1]} %
  {\end{enumerate}}

\newenvironment{arrayliste}[1]{ 
  % List with minimal white space to fit in small areas, e.g. table
  % cell
  \begin{minipage}[t]{\linewidth} %
    \begin{enumerate}[\bfseries\small\itshape #1] %
      {\leftmargin=0.5em \rightmargin=0em
        \topsep=0em \parskip=0em \parsep=0em
        \listparindent=0em \partopsep=0em \itemsep=0pt
        \itemindent=0em \labelwidth=\leftmargin\labelsep+0.25em}
    }{
    \end{enumerate}\end{minipage}
}

\newenvironment{nolistes}[2]
{\begin{enumerate}[\bfseries\small\itshape
    #1]\setlength{\itemsep}{#2 mm}}{\end{enumerate}}

\newenvironment{liste}[1]
{\begin{enumerate}[\hspace{12pt}\bfseries\small\itshape #1]}{\end{enumerate}} 



%%%%%%%% Pour les programmes de colle %%%%%%%

\newcommand{\cours}{{\small \tt (COURS)}} %
\newcommand{\poly}{{\small \tt (POLY)}} %
\newcommand{\exo}{{\small \tt (EXO)}} %
\newcommand{\culture}{{\small \tt (CULTURE)}} %
\newcommand{\methodo}{{\small \tt (MÉTHODO)}} %
\newcommand{\methodob}{\Boxed{\mbox{\tt MÉTHODO}}} %

%%%%%%%% Pour les TD %%%%%%%
\newtheoremstyle{exostyle} {\topsep} % espace avant
{.6cm} % espace apres
{} % Police utilisee par le style de thm
{} % Indentation (vide = aucune, \parindent = indentation paragraphe)
{\bfseries} % Police du titre de thm
{} % Signe de ponctuation apres le titre du thm
{ } % Espace apres le titre du thm (\newline = linebreak)
{\thmname{#1}\thmnumber{ #2}\thmnote{.
    \normalfont{\textit{#3}}}} % composants du titre du thm : \thmname
                               % = nom du thm, \thmnumber = numéro du
                               % thm, \thmnote = sous-titre du thm
 
\theoremstyle{exostyle}
\newtheorem{exercice}{Exercice}
\newtheorem{exerciceAP}{Exercice avec préparation}
\newtheorem{exerciceSP}{Exercice sans préparation}
\newtheorem*{exoCours}{Exercice}

%%%%%%%% Pour des théorèmes Sans Espaces APRÈS %%%%%%%
\newtheoremstyle{exostyleSE} {\topsep} % espace avant
{} % espace apres
{} % Police utilisee par le style de thm
{} % Indentation (vide = aucune, \parindent = indentation paragraphe)
{\bfseries} % Police du titre de thm
{} % Signe de ponctuation apres le titre du thm
{ } % Espace apres le titre du thm (\newline = linebreak)
{\thmname{#1}\thmnumber{ #2}\thmnote{.
    \normalfont{\textit{#3}}}} % composants du titre du thm : \thmname
                               % = nom du thm, \thmnumber = numéro du
                               % thm, \thmnote = sous-titre du thm
 
\theoremstyle{exostyleSE}
\newtheorem{exerciceSE}{Exercice}
\newtheorem*{exoCoursSE}{Exercice}

% \newcommand{\lims}[2]{\prod\limits_{#1}^{#2}}

\newtheorem{theoremSE}{Théorème}[]
\newtheorem{lemmaSE}{Lemme}[]
\newtheorem{propositionSE}{Proposition}[]
\newtheorem{corollarySE}{Corollaire}[]

% \newenvironment{proofSE}[1][Démonstration]{\begin{trivlist}
% \item[\hskip \labelsep {\bfseries #1}]}{\end{trivlist}}
\newenvironment{definitionSE}[1][Définition]{\begin{trivlist}
  \item[\hskip \labelsep {\bfseries #1}]}{\end{trivlist}}
\newenvironment{exampleSE}[1][Exemple]{\begin{trivlist} 
  \item[\hskip \labelsep {\bfseries #1}]}{\end{trivlist}}
\newenvironment{examplesSE}[1][Exemples]{\begin{trivlist}
\item[\hskip \labelsep {\bfseries #1}]}{\end{trivlist}}
\newenvironment{notationSE}[1][Notation]{\begin{trivlist}
\item[\hskip \labelsep {\bfseries #1}]}{\end{trivlist}}
\newenvironment{proprieteSE}[1][Propriété]{\begin{trivlist}
\item[\hskip \labelsep {\bfseries #1}]}{\end{trivlist}}
\newenvironment{proprietesSE}[1][Propriétés]{\begin{trivlist}
\item[\hskip \labelsep {\bfseries #1}]}{\end{trivlist}}
\newenvironment{remarkSE}[1][Remarque]{\begin{trivlist}
\item[\hskip \labelsep {\bfseries #1}]}{\end{trivlist}}
\newenvironment{applicationSE}[1][Application]{\begin{trivlist}
\item[\hskip \labelsep {\bfseries #1}]}{\end{trivlist}}

%%%%%%%%%%% Obtenir les étoiles sans charger le package MnSymbol
%%%%%%%%%%%
\DeclareFontFamily{U} {MnSymbolC}{}
\DeclareFontShape{U}{MnSymbolC}{m}{n}{
  <-6> MnSymbolC5
  <6-7> MnSymbolC6
  <7-8> MnSymbolC7
  <8-9> MnSymbolC8
  <9-10> MnSymbolC9
  <10-12> MnSymbolC10
  <12-> MnSymbolC12}{}
\DeclareFontShape{U}{MnSymbolC}{b}{n}{
  <-6> MnSymbolC-Bold5
  <6-7> MnSymbolC-Bold6
  <7-8> MnSymbolC-Bold7
  <8-9> MnSymbolC-Bold8
  <9-10> MnSymbolC-Bold9
  <10-12> MnSymbolC-Bold10
  <12-> MnSymbolC-Bold12}{}

\DeclareSymbolFont{MnSyC} {U} {MnSymbolC}{m}{n}

\DeclareMathSymbol{\filledlargestar}{\mathrel}{MnSyC}{205}
\DeclareMathSymbol{\largestar}{\mathrel}{MnSyC}{131}

\newcommand{\facile}{\rm{(}$\scriptstyle\largestar$\rm{)}} %
\newcommand{\moyen}{\rm{(}$\scriptstyle\filledlargestar$\rm{)}} %
\newcommand{\dur}{\rm{(}$\scriptstyle\filledlargestar\filledlargestar$\rm{)}} %
\newcommand{\costaud}{\rm{(}$\scriptstyle\filledlargestar\filledlargestar\filledlargestar$\rm{)}}

%%%%%%%%%%%%%%%%%%%%%%%%%

%%%%%%%%%%%%%%%%%%%%%%%%%
%%%%%%%% Fin de la partie TD

%%%%%%%%%%%%%%%%
%%%%%%%%%%%%%%%%
\makeatletter %
\newenvironment{myitemize}{%
  \setlength{\topsep}{0pt} %
  \setlength{\partopsep}{0pt} %
  \renewcommand*{\@listi}{\leftmargin\leftmargini \parsep\z@
    \topsep\z@ \itemsep\z@} \let\@listI\@listi %
  \itemize %
}{\enditemize} %
\makeatother
%%%%%%%%%%%%%%%%
%%%%%%%%%%%%%%%%

%% Commentaires dans la correction du livre

\newcommand{\Com}[1]{
% Define box and box title style
\tikzstyle{mybox} = [draw=black!50,
very thick,
    rectangle, rounded corners, inner sep=10pt, inner ysep=8pt]
\tikzstyle{fancytitle} =[rounded corners, fill=black!80, text=white]
\tikzstyle{fancylogo} =[ text=white]
\begin{center}

\begin{tikzpicture}
\node [mybox] (box){%

    \begin{minipage}{0.90\linewidth}
\vspace{6pt}  #1
    \end{minipage}
};
\node[fancytitle, right=10pt] at (box.north west) 
{\bfseries\normalsize{Commentaire}};

\end{tikzpicture}%

\end{center}
%
}

% \NewEnviron{remark}{%
%   % Define box and box title style
%   \tikzstyle{mybox} = [draw=black!50, very thick, rectangle, rounded
%   corners, inner sep=10pt, inner ysep=8pt] %
%   \tikzstyle{fancytitle} =[rounded corners, fill=black!80,
%   text=white] %
%   \tikzstyle{fancylogo} =[ text=white]
%   \begin{center}
%     \begin{tikzpicture}
%       \node [mybox] (box){%
%         \begin{minipage}{0.90\linewidth}
%           \vspace{6pt} \BODY
%         \end{minipage}
%       }; %
%       \node[fancytitle, right=10pt] at (box.north west) %
%       {\bfseries\normalsize{Commentaire}}; %
%     \end{tikzpicture}%
%   \end{center}
% }
% \makeatother


%%%
% Remarques comme dans le livre
%%%

\NewEnviron{remark}{%
  % Define box and box title style
  \tikzstyle{mybox} = [draw=black!50, very thick, rectangle, rounded
  corners, inner sep=10pt, inner ysep=8pt] %
  \tikzstyle{fancytitle} = [rounded corners , fill=black!80,
  text=white] %
  \tikzstyle{fancylogo} =[ text=white]
  \begin{center}
    \begin{tikzpicture}
      \node [mybox] (box){%
        \begin{minipage}{0.90\linewidth}
          \vspace{6pt} \BODY
        \end{minipage}
      }; %
      \node[fancytitle, right=10pt] at (box.north west) %
      {\bfseries\normalsize{Commentaire}}; %
    \end{tikzpicture}%
  \end{center}
}

\NewEnviron{remarkST}{%
  % Define box and box title style
  \tikzstyle{mybox} = [draw=black!50, very thick, rectangle, rounded
  corners, inner sep=10pt, inner ysep=8pt] %
  \tikzstyle{fancytitle} = [rounded corners , fill=black!80,
  text=white] %
  \tikzstyle{fancylogo} =[ text=white]
  \begin{center}
    \begin{tikzpicture}
      \node [mybox] (box){%
        \begin{minipage}{0.90\linewidth}
          \vspace{6pt} \BODY
        \end{minipage}
      }; %
      % \node[fancytitle, right=10pt] at (box.north west) %
%       {\bfseries\normalsize{Commentaire}}; %
    \end{tikzpicture}%
  \end{center}
}

\NewEnviron{remarkL}[1]{%
  % Define box and box title style
  \tikzstyle{mybox} = [draw=black!50, very thick, rectangle, rounded
  corners, inner sep=10pt, inner ysep=8pt] %
  \tikzstyle{fancytitle} =[rounded corners, fill=black!80,
  text=white] %
  \tikzstyle{fancylogo} =[ text=white]
  \begin{center}
    \begin{tikzpicture}
      \node [mybox] (box){%
        \begin{minipage}{#1\linewidth}
          \vspace{6pt} \BODY
        \end{minipage}
      }; %
      \node[fancytitle, right=10pt] at (box.north west) %
      {\bfseries\normalsize{Commentaire}}; %
    \end{tikzpicture}%
  \end{center}
}

\NewEnviron{remarkSTL}[1]{%
  % Define box and box title style
  \tikzstyle{mybox} = [draw=black!50, very thick, rectangle, rounded
  corners, inner sep=10pt, inner ysep=8pt] %
  \tikzstyle{fancytitle} =[rounded corners, fill=black!80,
  text=white] %
  \tikzstyle{fancylogo} =[ text=white]
  \begin{center}
    \begin{tikzpicture}
      \node [mybox] (box){%
        \begin{minipage}{#1\linewidth}
          \vspace{6pt} \BODY
        \end{minipage}
      }; %
%       \node[fancytitle, right=10pt] at (box.north west) %
%       {\bfseries\normalsize{Commentaire}}; %
    \end{tikzpicture}%
  \end{center}
}

\NewEnviron{titre} %
{ %
  ~\\[-1.8cm]
  \begin{center}
    \bf \LARGE \BODY
  \end{center}
  ~\\[-.6cm]
  \hrule %
  \vspace*{.2cm}
} %

\NewEnviron{titreL}[2] %
{ %
  ~\\[-#1cm]
  \begin{center}
    \bf \LARGE \BODY
  \end{center}
  ~\\[-#2cm]
  \hrule %
  \vspace*{.2cm}
} %




\pagestyle{fancy} %
\lhead{ECE2 \hfill septembre 2017 \\
 Mathématiques\\[.2cm]} %
\chead{\hrule} %
\rhead{} %
\lfoot{} %
\cfoot{} %
\rfoot{\thepage} %

\renewcommand{\headrulewidth}{0pt}% : Trace un trait de séparation
 % de largeur 0,4 point. Mettre 0pt
 % pour supprimer le trait.

\renewcommand{\footrulewidth}{0.4pt}% : Trace un trait de séparation
 % de largeur 0,4 point. Mettre 0pt
 % pour supprimer le trait.

\setlength{\headheight}{14pt}

\title{\bf \vspace{-1cm} ORAUX HEC 2010} %
\author{} %
\date{} %

\begin{document}

\maketitle %
\vspace{-1.2cm}\hrule %
\thispagestyle{fancy}

\vspace*{.4cm}

% DEBUT DU DOC À MODIFIER : tout virer jusqu'au début de l'exo

\section{Annales 2010}

 %\setcounter{exercice}{0}

 \begin{exercice} \indent \\
 \textbf{\underline{Exercice avec préparation}} \\
 \begin{noliste}{1.}
 \setlength{\itemsep}{4mm}
 \item La loi géométrique est la loi d'attente du premier succès dans
une succession illimitée d'épreuves de Bernouilli identiques et
indépendantes, de paramètre $p$. \\
 Si $X \suit \mathcal{G} (p)$, on a $X(\Omega) = \N^*$ et pour tout $k
\in \N^*$, $\Prob\left(\Ev{\Ev{X = k}}\right) = (1-p)^{k-1} p$. \\
 De plus $X$ admet une espérance et une variance, et $\E(X) =
\frac{1}{p}$ et $\V(X) = \frac{1-p}{p^{2}}$. \\
 \item Pour tout $h \in \llb 1 ; n \rrb$, on définit la variable
aléatoire $T_{h}$ égale au nombre d'années nécessaires pour que le
$h$-ième bulbe fleurisse. \begin{noliste}{a)}
 \setlength{\itemsep}{2mm} 
 \item $T_{h} \suit \mathcal{G} (p)$. \\
 \item $T = \max (T_{1}, T_{2},\ \dots\, T_{n})$ donc $T(\Omega) =
\N^*$ et pour tout $k \geq 1$ : \\
 $\Prob\left(\Ev{\Ev{ T \leq k}}\right) = P \left(\Ev{
\bigcap\limits_{h = 1}{n} [T_{h} \leq k]}\right) = \prod\limits_{h =
1}{n} \Prob\left(\Ev{\Ev{ T_{h} \leq k}}\right) = \left( 1 - q^{k}
\right)^{n}$, qui est valable aussi pour $k = 0$ (cela donne bien 0).
\\
 On en déduit que $\Prob\left(\Ev{\Ev{ T = k}}\right) =
\Prob\left(\Ev{\Ev{ T \leq k}}\right) - \Prob\left(\Ev{\Ev{ T \leq
k-1}}\right) = \left( 1 - q^{k} \right)^{n} - \left( 1 - q^{k-1}
\right)^{n}$. \\
 \end{noliste}
 \item \begin{noliste}{a)}
 \setlength{\itemsep}{2mm} 
 \item $\dlim{N \rightarrow + \infty} \Sum{k = 1}{n} \binom{n}{k}
(-1)^{k} N (q^{k})^{N} = 0$ car c'est une somme finie de termes tendant
vers 0 par croissances comparées de la suite $u_{N} = N$ et de la suite
géométrique $v_{N} = (q^{k})^{N}$. \\
 \item $ \Sum{k = 1}{n} (-1)^{k} \binom{n}{k} \Sum{j = 1}{N}
(q^{k})^{j-1} = \Sum{k = 1}{n} (-1)^{k} \binom{n}{k} \times \frac{1 -
q^{k N} }{1 - q^{k}} \xrightarrow[ N \rightarrow + \infty]{} \Sum{k =
1}{n} (-1)^{k} \binom{n}{k} \times \frac{1 }{1 - q^{k}}$.
 \\
 \item On étudie $ \Sum{j = 1}{N} j \Prob\left(\Ev{\Ev{ T = j}}\right)
= \Sum{j = 1}{N} j \left[ \ \Sum{k = 0}{n} \binom{n}{k} (-1)^{k}
(q^{j})^{k} - \Sum{k = 0}{n} \binom{n}{k} (-1)^{k} (q^{j-1})^{k}
\right] \\
\\ = \Sum{k = 0}{n} \binom{n}{k} (-1)^{k} \Sum{j = 1}{N} j (q^{k})^{j}
- \Sum{k = 0}{n} \binom{n}{k} (-1)^{k} \Sum{j = 1}{N} j (q^{k})^{j-1}
\\
\\ = \binom{n}{0} (-1)^{0} \Sum{j = 1}{N} j (1)^{j} + \Sum{k = 1}{n}
\binom{n}{k} (-1)^{k} \Sum{j = 1}{N} j (q^{k})^{j} - \binom{n}{0}
(-1)^{0} \Sum{j = 1}{N} j (1)^{j-1} - \Sum{k = 1}{n} \binom{n}{k}
(-1)^{k} \Sum{j = 1}{N} j (q^{k})^{j-1} \\
\\ = \frac{N (N + 1)}{2} + \Sum{k = 1}{n} \binom{n}{k} (-1)^{k} q^{k}
\Sum{j = 1}{N} j (q^{k})^{j-1} - \frac{N (N + 1)}{2} - \Sum{k = 1}{n}
\binom{n}{k} (-1)^{k} \Sum{j = 1}{N} j (q^{k})^{j-1} \\
\\\
xrightarrow[ N \rightarrow + \infty]{} \Sum{k = 1}{n} \binom{n}{k}
(-1)^{k} q^{k} \frac{1}{(1-q^{k})^{2}} - \Sum{k = 1}{n} \binom{n}{k}
(-1)^{k} \frac{1}{(1-q^{k})^{2}} = \Sum{k = 1}{n} \binom{n}{k} (-1)^{k}
\frac{q^{k} + 1 }{(1-q^{k}) (1 + q^{k})} = \Sum{k = 1}{n} \binom{n}{k}
(-1)^{k} \frac{ 1 }{1-q^{k}} $ \\
\\\
\
 On retrouve le même résultat qu'à partir de la question b, mais je ne
m'en suis pas du tout servi. On doit probablement pouvoir l'utiliser
d'une manière ou d'une autre, peut-être en regroupant les deux sommes
en $k$ tout de suite au lieu de le faire à la fin. \\
 \end{noliste}
 \end{noliste}
 \noindent \textbf{\underline{Exercice sans préparation}} \\
\\
 $f$ est diagonalisable, on étudier la question dans une base
$\mathcal{B}$ où la matrice de $f$ est diagonale, égale à $D =
\begin{smatrix}
a_{1} & 0 & \dots & 0 \\
0 & a_{2} & & 0 \\
\vdots & & \ddots & 0 \\
0 & 0 & \dots & a_{n} \\
\end{smatrix}
$. \\
Pour avoir $\im(f) \subset \ker f$ il faut avoir $f^{2} = 0$, donc
$D^{2} = 0$, donc $a_{i}{2} = \Leftrightarrow a_{i} = 0$ pour tout
$i$,
et $D = 0$. \\
La seule solution est donc l'endomorphisme nul, $f = 0_{\mathcal{L}
  (E)}$. \\
\\\
\ Autre solution : $f^{2} = 0$ donc 0 est la seule valeur propre
possible, et $f$ diagonalisable donc la matrice diagonale contenant 0
uniquement sur la diagonale, donc nulle, est une matrice de $f$, et $f
= 0$.
 \end{exercice}

 \newpage

 \begin{exercice} \indent \\
 \textbf{\underline{Exercice avec préparation}} \\
 \begin{noliste}{1.}
 \setlength{\itemsep}{4mm}
 \item $f = o(g)$ si et seulement si $ f = g \times \epsilon$ avec
$\epsilon (x) \xrightarrow[x \rightarrow + \infty] 0$ puis on définit
$f \sim g$ si et seulement si $f = g + o(g)$, ou $g = f + o(f)$, ou
encore $f = g \times u$, avec $u(x) \xrightarrow[x \rightarrow +
\infty]{} 1$. \\
\\
 Croissances comparées des fonctions usuelles : \\
 Pour tout $\alpha \in \R$, pour tout $\lambda > 0$, $x^{\alpha} = o
(e^{\lambda x} ) $ donc $e^{-\lambda x} = o ( x^{\alpha})$ en $ +
\infty$. \\
 D'autre part pour tout $\alpha \in \R$, pour tout $\lambda > 0$,
$x^{\alpha} = o (e^{-\lambda x} ) $ et $e^{\lambda x} = o (
x^{\alpha})$ en $-\infty$. \\
 Pour tous $\alpha >0$ et $\beta >0$, $( \ln x)^{\beta} = o (
x^{\alpha})$ en $ + \infty$. \\
 \item Soit $g$ la fonction définie sur $\R_+^*$ à valeurs réelles,
telle que :
 
\[
 \forall x \in \ ] 0 ; + \infty[,\ \ \ g(x) = x \ln^{2} (x). 
\]
 \begin{noliste}{a)}
 \setlength{\itemsep}{2mm} 
 \item $g$ est dérivable sur $]1 ; + \infty[$ et $g'(x) = \ln^{2} (x) +
x \times \frac{1}{x} \times 2 \ln (x) = \ln (x) ( \ln x + 2) >0$ sur $]
1 ; + \infty[$ donc $g$ est strictement croissante et continue donc
réalise une bijection de $]1 ; + \infty[$ dans $]\dlim{1} g ; \dlim{+
\infty} g [$. \\
\\
 Or par continuité de $g$, $\dlim{1} g = g(1) = 0$ et par produit de
limites, $\dlim{+ \infty} g = + \infty$, donc $g$ réalise une bijection
de $] 1 ; + \infty [$ dans $] 0 ; + \infty[$. \\
\\
 Soit $h$ la bijection réciproque de la restriction de $g$ à
l'intervalle $]1 ; + \infty[$. 
 \item \begin{nonoliste}{(i)}
 \item On écrit $g( h(x) ) = h(x) \ln^{2} (h(x) ) = x$ et on compose
par $\ln$ : \\
 On obtient $\ln (h(x) ) + 2 \ln ( \ln (h(x) ) ) = \ln (x)$. \\
 \item On a $\ln x = \ln (h(x) ) \left( 1 + 2 \frac{ \ln ( \ln (h (x) )
) }{ \ln ( h(x) )} \right)$. \\
 On pose $X = \ln (h(x)) \xrightarrow[ x \rightarrow + \infty]{} +
\infty$ par composée de limites, on a alors par croissances comparées,
$ \frac{ \ln X }{X} \xrightarrow[X \rightarrow + \infty]{} 0$ donc par
composition, $\dlim{x \rightarrow + \infty} 1 + 2 \frac{ \ln ( \ln (h
(x) ) ) }{ \ln ( h(x) )} = 1$, et enfin $\ln x \sim \ln (h(x) )$ en $ +
\infty$. \\
\\
 ATTENTION : cela ne donne pas $h(x) \sim x$ !!!! \\
\\
 On écrit alors $[ \ \ln (h (x) ) ]^{2} = \ln ( h(x) ) \times \ln (
h(x) ) \sim \ln (x) \times \ln (x) = \ln^{2} (x)$ et on réutilise $h(x)
\ln^{2} (h(x)) = x \sim h(x) \ln^{2} (x)$ donc $h(x) \sim
\frac{x}{\ln^{2} (x)}$. \\
 \end{nonoliste}
 \end{noliste}
 \item Soit $X$ une variable aléatoire de densité $f$ définie par :
 
\[
 f(x) = \left\{ 
\begin{array}{ll}
 \frac{1}{2 g (| x |)} & \text{ si } | x | < \frac{1}{e} \text{ et } x
\neq 0 \\
0 & \text{sinon} \\
\end{array}
\right. 
\]
 \begin{noliste}{a)}
 \setlength{\itemsep}{2mm}
 \item $f$ est continue sauf éventuellement en 0, $\frac{1}{e}$ et
$\frac{-1}{e}$ et positive par positivité de $g$, il reste à prouver
que $\dint{-\infty}{+ \infty} f $ est convergente et vaut 1. \\
 Par parité il suffit de prouver que $\dint{0}{+ \infty} f $ converge
vers $\frac{1}{2}$. \\
 On a $\dint{\frac{1}{e} }{+ \infty} f$ converge et vaut 0 comme
intégrale de la fonction nulle. \\
 Enfin on étudie $\dint{x}{\frac{1}{e} } \frac{1}{2 t \ln^{2} t}\ dt$
car l'intégrale n'est généralisée qu'en 0. \\
 $\dint{x}{\frac{1}{e} } \frac{1}{2t \ln^{2} t}\ dt = \left[ -
\frac{1}{2\ln t} \right]_{x}{\frac{1}{e} } = - \frac{1}{-2 \ln e } +
\frac{1}{ 2 \ln x} \xrightarrow[ x \rightarrow 0]{} \frac{1}{2}$ donc
l'intégrale converge et vaut $\frac{1}{2}$ et $f$ est bien une densité
de probabilité. \\
 \item L'imparité puis la positivité permette de se restreindre à
l'étude de la convergence de \\
$\dint{0}{\frac{1}{e} } t \times \frac{1}{2 t \ln^{2} t}\ dt =
\dint{0}{ \frac{1}{e} } \frac{dt}{2 \ln^{2} t}$. \\
 Pour la convergence on peut écrire $ \frac{1}{2 \ln^{2} t} = o \left(
\frac{1}{t \ln^{2} (t)} \right)$ et on a prouvé la convergence de
l'intégrale de cette fonction : on utilise alors le théorème de
comparaison des intégrales de fonctions positives. \\
 Ensuite l'imparité permet de donner $\E(X) = 0$ (sans réaliser le
calcul d'intégrale, qui est impossible!) \\
 \item Sur le même principe on écrit $ \frac{t}{ 2 \ln^{2} t} = o
\left( \frac{1}{t \ln^{2} (t)} \right)$ pour prouver l'existence du
moment d'ordre deux et donc de la variance. \\
 Par contre le calcul du moment d'ordre deux est celui de l'intégrale
d'une fonction paire, et il faudrait réaliser le calcul de l'intégrale,
qu'on ne sait pas faire. \\
 \end{noliste}
 \end{noliste}
 \noindent \textbf{\underline{Exercice sans préparation}} \\
\\
 On note $\mathcal{B} = (e_{1}, e_{2}, e_{3})$ la base canonique de
$\R^{3}$. \\
 Soit $f$ l'endomorphisme de $\R^{3}$ dont la matrice dans la base
$\mathcal{B}$ est $M = \begin{smatrix}
1 & -1 & 1 \\
1 & 0 & 0 \\
0 & 1 & 0 \\
\end{smatrix}
$. \begin{noliste}{1.}
 \setlength{\itemsep}{4mm}
 \item $f(e_{1} + e_{2} + e_{3}) = M \begin{smatrix}
1 \\
1 \\
1
\end{smatrix}
 = \begin{smatrix}
1 \\
1 \\
1
\end{smatrix}
$, $f( e_{2} ) = M \begin{smatrix}
0 \\
1 \\
0
\end{smatrix}
 = \begin{smatrix}
0 \\
0 \\
1
\end{smatrix}
$ et $f(- e_{1} + e_{3}) = M \begin{smatrix}
-1 \\
0 \\
1
\end{smatrix}
 = \begin{smatrix}
0 \\
-1 \\
0
\end{smatrix}
$. \\
 \item On prouve que la famille au-dessus est une base, et on a $M' =
Mat_{\mathcal{B'}} (f)$ est semblable à $M = Mat_{\mathcal{B} } (f)$.
\\
 \item $M$ est diagonalisable si et seulement si $M'$ l'est. \\
 L'étude des valeurs propres de $M'$ donne $\spc M' = \{ 1 \}$; si $M'$
était diagonalisable, on aurait $M' = P I P^{-1} = I$ qui est absurde,
donc $M'$ et $M$ ne sont pas diagonalisables.
 \end{noliste}
 \end{exercice}

 \newpage

 \begin{exercice}{\it (Exercice avec préparation)}~\\
 Soit $X$ une variable aléatoire définie sur un espace probabilisé
$(\Omega, \mathcal{A}, P)$, qui suit la loi binomiale $\mathcal{B} (n,
p)$, avec $n \geq 2$ et $0 < p < 1$. \\
\\
 On définit sur $(\Omega, \mathcal{A}, P)$ une variable aléatoire $Y$
de la façon suivante : \\
 \begin{noliste}{$\sbullet$}
 \item pour tout $k \in \llb 1 ; n \rrb$, la réalisation de l'évènement
$[X = k]$ entraîne celle de l'évènement $[Y = k]$;
 \item la loi conditionnelle de $Y$ sachant $[X = 0]$ est la loi
uniforme sur $\llb 1 ; n \rrb$.
 \end{noliste}
 \begin{noliste}{1.}
 \setlength{\itemsep}{4mm}
 \item On réalise une succession de $n$ épreuves de Bernouilli
indépendantes et identiques de paramètre $p$, et on compte le nombre de
succès. \\
 La variable $X$ associée vérifie alors $X(\Omega) = \llb 0 ; b \rrb$.
\\
 Pour calculer $\Prob\left(\Ev{\Ev{X = k}}\right)$, on compte le nombre
de possibilités amenant à ce résultat et la probabilité de chacune. \\
 Il faut obtenir $k$ succès et $n-k$ échecs : on place les $k$ succès
parmi les $n$ épreuves pour obtenir toutes les possibilités : il y en a
donc $\binom{n}{k}$. \\
 Dans chacun de ces cas, on obtient de manière indépendante $k$ succès
et $n$ échecs avec une probabilité $p^{k} q^{n-k}$. \\
 On obtient alors $\Prob\left(\Ev{\Ev{X = k}}\right) = \binom{n}{k}
p^{k} q^{n-k}$. \\
 L'espérance s'obtient en écrivant les $n$ $X_{i}$ variables de
Bernouilli, avec la linéarité de l'espérance : $\E(X) = n p$. \\
 De même grâce à l'indépendance on calcule facilement la variance de la
somme : $\V(X) = n p (1-p)$. \\
 \item On a $Y (\Omega ) = \llb 1 ; n \rrb$ et pour tout $k \in \N$, en
utilisant le système complet $\Ev{X = i}_{0 \leq i \leq n}$ on a : \\
 $\Prob\left(\Ev{\Ev{ Y = k}}\right) = \Sum{i = 0}{n}
\Prob\left(\Ev{\Ev{ X = i}}\right) P_{\Ev{X = i} } \Ev{Y = k} =
\Prob\left(\Ev{\Ev{ X = k}}\right) \times 1 + \Prob\left(\Ev{\Ev{X =
0}}\right) \times \frac{1}{n} = \binom{n}{k} p^{k} q^{n-k} +
\frac{q^{n}}{n}$. \\
 \item $\E(Y) = \Sum{k = 1}{n} k \binom{n}{k} p^{k} q^{n-k} + \Sum{k =
1}{n} \frac{q^{n}}{n} = \Sum{k = 0}{n} k \binom{n}{k} p^{k} q^{n-k} - 0
+ \frac{q^{n}}{n} \Sum{k = 1}{n} 1 = \E(X) + \frac{q^{n}}{n} \times n =
n p + q^{n}$. \\
 \item 
 \begin{noliste}{a)}
 \setlength{\itemsep}{2mm} 
 \item Pour tout $k \in \llb 1 ; n \rrb$ on a $P_{X \neq 0} \Ev{ Y = k}
= \frac{ \Prob( [Y = k] \cap [X \neq 0])}{\Prob\left(\Ev{ X \neq
0}\right)} = \frac{P \Ev{X = k}}{1 - \Prob\left(\Ev{\Ev{X = 0}}\right)
} = \binom{n}{k} \frac{ p^{k} q^{n-k}}{1 - q^{n}}$. \\
 \item $\E(Y \slash X \neq 0) = \Sum{k = 1}{n} k \binom{n}{k} \frac{
p^{k} q^{n-k}}{1 - q^{n}} = \frac{1}{1 - q^{n}} \Sum{k = 1}{n} k
\binom{n}{k} p^{k} q^{n-k} = \frac{1}{1 - q^{n}} \left( \Sum{k = 0}{n}
k \binom{n}{k} p^{k} q^{n-k} - 0 \right) = \frac{\E(X)}{1 - q^{n}} =
\frac{n p}{1 - q^{n}}$. \\
 \end{noliste}
 \end{noliste}
 \noindent \textbf{\underline{Exercice sans préparation}} \\
\\
 Soit $A$ une matrice symétrique réelle d'ordre $n$ ($n \in \N^*$) et
vérifiant $A^{k} = I_{n}$. \\
 Que peut-on dire dans les cas suivants : \begin{noliste}{1.}
 \setlength{\itemsep}{4mm}
 \item $A$ est symétrique donc diagonalisable, et le polynôme
$P\left(\Ev{x}\right) = x^{k} -1$ est annulateur de $A$. \\
 1 est une racine évidente de ce polynôme, étudions l'existence
d'autres racines. \\
 $P'(x) = k x^{k-1} > 0$ sur $\R^*$ et nul en 0 car $k-1$ est pair,
donc $P$ est strictement croissante et l'équation $P\left(\Ev{x}\right)
= 1$ admet une unique solution, qui est donc $x = 1$. \\
 On en déduit que $\spc A \subset \{ 1 \}$ et comme $A$ est
diagonalisable elle admet au moins une valeur propre : d'où 1 est la
seule valeur propre de $A$, il existe $P$ inversible tel que $A = P I
P^{-1} = I$, donc $A$ est forcément la matrice identité. \\
 \item Ici l'étude de $P'(x)$ donne $P'(x) < 0$ sur $\R_-^*$, nul en 0
et strictement positif en 0. \\
 De plus $P\left(\Ev{0}\right) = -1 <0$ et $\dlim{+ \infty} f =
\dlim{-\infty} f = + \infty$ donc le théorème de la bijection donne
exactement deux solutions, et celles-ci sont évidentes : 1 et $-1$. \\
 Il y a donc 3 possibilités : \\
 - $\spc A = \{ 1 \}$ qui donne $A = I$. \\
 - $\spc A = \{ -1 \}$ qui donne $A = -I$. \\
 - $\spc A = \{ -1 ; 1 \}$ qui donne $A = P D P^{-1}$ avec $D$
contenant 1 et $-1$ sur sa diagonale.
 \end{noliste}
 \end{exercice}

 \newpage


 \begin{exercice}{\it (Exercice avec préparation)}~\\
 Toutes les variables aléatoires de cet exercice sont définies sur un
espace probabilisé $(\Omega, \mathcal{A}, P)$.
 \begin{noliste}{1.}
 \setlength{\itemsep}{4mm}
 \item Soit $X$ une variable discrète finie telle que $ X(\Omega) = \{
x_{i}\ \big |\ 1 \leq i \leq n \}$. \\
 On pose alors $\E(X) = \Sum{i = 1}{n} x_{i} \Prob\left(\Ev{\Ev{X =
x_{i}}}\right)$ qui est la valeur moyenne obtenue lorsqu'on réalise un
grand nombre de fois l'expérience. \\
 $\V(X) = \Sum{i = 1}{n} ( x_{i} - \E(X) )^{2} \Prob\left(\Ev{\Ev{ X =
x_{i}}}\right)$ qui est la moyenne des écarts au carré, et qui permet
de mesurer la dispersion des valeurs de $X$ (pondérées par leurs
probabilités) autour de son espérance (qui est la valeur moyenne). \\
 \item Soient $a$ et $b$ deux réels tels que $a < b$. On considère une
variable aléatoire $X$ (discrète ou possédant une densité) prenant
toutes ses valeurs dans l'intervalle $[a ; b]$ et ayant un moment
d'ordre 2.
% LALALA 
 \begin{noliste}{a)}
 \setlength{\itemsep}{2mm} 
 \item On a $\V(X) = E ( [X - \E(X)]^{2} ) = E ( [ (X - \lambda ) +
(\lambda - \E(X) )^{2} ) \\
 = E( [X - \lambda]^{2} + 2 [ \ \lambda - \E(X) ] [X - \lambda] + [ \
\lambda - \E(X) ]^{2} ) \\
 = E ( [X - \lambda]^{2} ) + 2 (\lambda - \E(X) ) [ \ \E(X - \lambda) ]
+ [ \ \lambda - \E(X) ]^{2} \\
 = \E( [X - \lambda]^{2}) - [ \ \lambda - \E(X) ]^{2} \leq \E( [X -
\lambda]^{2})$ car un carré est toujours positif. \\
 \item On se place en $\lambda = \frac{a + b}{2}$ la moyenne de $a$ et
$b$ et on obtient : \\
 $ 0 \leq \left( X - \frac{a + b}{2} \right)^{2} \leq \left( \frac{
b-a}{2} \right)^{2}$ donc $0 \leq E \left[ \left( X - \frac{a + b}{2}
\right)^{2} \right] \leq \left( \frac{ b-a}{2} \right)^{2}$ et donc
$\V(X) \leq E \left[ \left( X - \frac{a + b}{2} \right)^{2} \right]
\leq \frac{ (b-a)^{2}}{4} $. \\

% LALALA 
 \end{noliste}
 \item 
 \begin{noliste}{a)}
 \setlength{\itemsep}{2mm} 
 \item On a alors $ \E(X) = \frac{1}{2} a + \frac{1}{2} b = \frac{a +
b}{2}$ puis $\V(X) = \frac{1}{2} \left[ \ \left( a - \frac{a + b}{2}
\right)^{2} + \left( b - \frac{a + b}{2} \right)^{2} \right] =
\frac{1}{2} \times 2 \times \left( \frac{b-a}{2} \right)^{2} = \frac{
(b-a)^{2}}{4}$. \\
 \item Étude d'une réciproque : on suppose que $\V(X) =
\frac{(b-a)^{2}}{4}$. \\
 On a alors $\V(X) = E \left[ \left( X - \frac{a + b}{2} \right)^{2}
\right] = \frac{ (b-a)^{2}}{4} $
 En reprenant la question 2.a., on voit que pour avoir $\V(X) = E
\left[ \left( X - \frac{a + b}{2} \right)^{2} \right] $, il faut avoir
$\left( \frac{a + b}{2} - \E(X) \right)^{2} = 0$ donc $\E(X) = \frac{a
+ b}{2}$. \\

% LALALA 
Ensuite on reprend la question 2.b. : si il existe $c \in \ ] a ;b [$
tel que $\Prob\left(\Ev{ X = c} \right) \neq 0$ on a :\\
$ E \left[ \left( X - \frac{a + b}{2} \right)^{2} \right] = \left( c -
  \frac{a + b}{2} \right)^{2} \Prob\left(\Ev{X = c} \right) + \Sum{x
  \neq c}{} \left( x - \frac{a + b}{2} \right)^{2} \Prob\left(\Ev{ X =
    x} \right) \leq \left( c - \frac{a + b}{2} \right)^{2}
\Prob\left(\Ev{X = c} \right) + \frac{ (b-a)^{2}}{4} \Sum{x \neq c}
\Prob\left(\Ev{ X = x} \right) = \left( c - \frac{a + b}{2}
\right)^{2} \Prob\left(\Ev{X = c} \right) + \frac{ (b-a)^{2}}{4} ( 1 -
P \Ev{X = c} ) = \frac{ (b-a)^{2}}{4} + \Prob\left(\Ev{ X = c} \right)
\left( \left( c - \frac{a + b}{2} \right)^{2} - \frac{ (b-a)^{2}}{4}
\right) < \frac{ (b-a)^{2}}{4}$.\\
\\
% LALALA 
 On en déduit que $X (\Omega) = \{ a ; b \}$, puis que $\E(X) = \frac{a
+ b}{2} = a \Prob\left(\Ev{\Ev{ X = a}}\right) + b ( 1 -
\Prob\left(\Ev{\Ev{ X = a}}\right) )$ donc $\Prob\left(\Ev{\Ev{X = a
}}\right) = \frac{ \frac{a + b}{2} - b }{a - b} = \frac{ \frac{ a-b}{2}
}{a-b} = \frac{1}{2}$ et $\Prob\left(\Ev{\Ev{ X = b}}\right) = 1 -
\Prob\left(\Ev{\Ev{ X = a }}\right) = \frac{1}{2}$. \\
 \end{noliste}
 \item Cela signifie que la variance est maximale lorsque la dispersion
est maximale et symétrique : il faut que les deux seules valeurs prises
soient symétriques autour de l'espérance, et aient la même probabilité
d'être atteintes. \\
 \end{noliste}
% LALALA 
 \noindent \textbf{\underline{Exercice sans préparation}} \\
\begin{noliste}{1.}
 \setlength{\itemsep}{4mm}
 \item $M$ est triangulaire donc elle est inversible si et seulement si
$X $ et $Y$ sont non nuls. \\
 D'où $P \left(\Ev{ M \text{ inversible } }\right) = P ( [ X \neq 0]
\cap [ Y \neq 0] ) = 1$ car $ 0 \notin X(\Omega) = Y (\Omega)$. \\
 \item On a $\spc M = \{ X ; Y \}$ et comme $M$ n'est pas diagonale on
a : si il y a une unique valeur propre, $M$ n'est pas diagonalisable et
si il y a deux valeurs propres distinctes, elle l'est. \\
 D'où $\Prob\left(\Ev{ M \text{ diagonalisable } }\right) = P
\left(\Ev{X \neq Y}\right) = 1 - P \Ev{ X = Y} = 1 - \Sum{k = 1}{+
\infty} p^{2} (q^{2})^{k-1} = 1 - \frac{p^{2}}{1 - q^{2}} = 1 -
\frac{p}{1 + q} = \frac{1 + q-p}{1 + q} = \frac{2q}{1 + q}$.

 \end{noliste}
 \end{exercice}

% LALALA 

 \newpage

 \begin{exercice}{\it (Exercice avec préparation)}~\\
 Soit $n$ un entier supérieur ou égal à 2, et $p$ et $q$ deux réels de
$]0;1[$ tels que $p + q = 1$. On considère deux variables aléatoires
discrètes $X$ et $Y$ définies sur une espace probabilisé $(\Omega,
\mathcal{A}, P)$. \\
 La loi du couple $(X, Y)$ est donnée par : \\
 pour tout $(j,k)$ tels que $0 \leq j \leq n$ et $1 \leq k \leq n$,
 
\[
 \Prob( [X = j] \cap [Y = k] ) = \left\{ 
\begin{array}{cc}
 \binom{n}{k} p^{k} q^{n-k} & \text{ si } k = j,\ j \neq 0 \\
\\\
frac{q^{n}}{n} & \text{ si } j = 0 \\
\\0 & \text{ si } k \neq j \text{ et } j \neq 0 \\
\end{array}
\right.
\]
 \begin{noliste}{1.}
 \setlength{\itemsep}{4mm}
 \item La loi d'un couple $(X,Y)$ de variables aléatoires discrètes est
la donnée de $(X,Y) (\Omega)$, ensemble des couples $(i,j)$ de valeurs
telles que l'évènement $\Ev{X = i} \cap \Ev{Y = j}$ est possible. \\
 Les lois marginales sont les lois des variables $X$ et $Y$, obtenues à
partir du couple et de la formule des probabilités totales. \\
 Les lois conditionnelles sont les lois d'une variable sachant que
l'autre a donné un résultat précis, c'est-à-dire les lois du type
$P_{\Ev{X = i_{0}}} \Ev{Y = j}$ ou $P_{\Ev{Y = j_{0}}} \Ev{X = i}$. \\
 \item 
 \begin{noliste}{a)}
 \setlength{\itemsep}{2mm} 
 \item On a $X(\Omega) = \llb 0 ; n \rrb$ et pour tout $j \in \llb 1 ;
n \rrb$, avec le système complet $\Ev{Y = k}_{1 \leq k \leq n}$ on a :
\\
 $\Prob\left(\Ev{\Ev{X = j}}\right) = \binom{n}{j} p^{j} q^{n-j}$. \\
 Enfin pour $j = 0$ on a $\Prob\left(\Ev{\Ev{X = 0}}\right) = \Sum{k =
1}{n} \frac{q^{n}}{n} = q^{n}$. \\
 Finalement on voit que $X \suit \mathcal{B} (n, p)$. \\
\\
 D'autre part on a $Y (\Omega) = \llb 1 ; n \rrb$ et avec le système
complet $\Ev{X = j}_{0 \leq k \leq n}$ on a : \\
 $\Prob\left(\Ev{\Ev{Y = k}}\right) = \binom{n}{k} p^{k} q^{n-k} +
\frac{q^{n}}{n}$. \\
 \item $\E(Y) = \Sum{k = 1}{n} k \binom{n}{k} p^{k} q^{n-k} + \Sum{k =
1}{n} \frac{q^{n}}{n} = \Sum{k = 0}{n} k \binom{n}{k} p^{k} q^{n-k} - 0
+ \frac{q^{n}}{n} \Sum{k = 1}{n} 1 = \E(X) + \frac{q^{n}}{n} \times n =
n p + q^{n}$. \\
 \end{noliste}
 \item Soit $j$ un entier tel que $0 \leq j \leq n$.
 \begin{noliste}{a)}
 \setlength{\itemsep}{2mm} 
 \item Pour tout $ j \geq 1$, pour tout $1 \leq k \leq n$ on a
$P_{\Ev{X = j}} \Ev{Y = k} = \frac{ P ( [ X = j] \cap [Y = k] )}{P
\Ev{X = j}} = 0$ si $j \neq k$ et $1$ si $j = k$. \\
 Pour $j = 0$, on a pour tout $1 \leq k \leq n$, $P_{\Ev{X = 0}} \Ev{Y
= k} = \frac{ \frac{q^{n}}{n} }{ q^{n}} = \frac{1}{n}$. \\
 \item Si $j > 0$, on a $\E( Y \slash X = j) = j$ (c'est une variable
certaine). \\
 Pour $j = 0$, on a $\E( Y \slash X = 0) = \Sum{k = 1}{n} k \times
\frac{1}{n} = \frac{1}{n} \times \frac{n (n + 1)}{2} = \frac{n +
1}{2}$. \\
 \end{noliste}
 \item
 \begin{noliste}{a)}
 \setlength{\itemsep}{2mm} 
 \item $\Prob( [X = 1] \cap [Y = 1] ) = n p q^{n-1}$ et
$\Prob\left(\Ev{\Ev{X = 1}}\right) \Prob\left(\Ev{\Ev{Y = 1}}\right) =
n p q^{n-1} \times \left( n p q^{n-1} + \frac{q^{n}}{n} \right)$. \\
 Il faut donc prouver que $ \left( n p q^{n-1} + \frac{q^{n}}{n}
\right) = n q^{n-1} - n q^{n} + \frac{q^{n}}{n} \neq 1$. \\
 On étudie la fonction $f(q) = n q^{n-1} - n q^{n} + \frac{q^{n}}{n}$
qui est dérivable : \\
 $f'(q) = n (n-1) q^{n-2} - n^{2} q^{n-1} + q^{n-1} = q^{n-2} \big[ n
(n-1) - (n^{2} - 1) q \big] = q^{n-2} \big[ n ( n-1) - (n-1) (n + 1) q
\big] = (n-1) q^{n-2} ( n - (n + 1) q )$ qui s'annule en $q =
\frac{n}{n + 1}$ en étant positive avant et négative après; en
rajoutant $f(0) = 0$ et $f(1) = \frac{1}{n}$ et $ f \left( \frac{n}{n +
1} \right) = \frac{n^{n}}{(n + 1)^{n}} + \frac{n^{n-1} }{(n + 1)^{n}} =
\frac{ n^{n} + n^{n-1} }{(n + 1)^{n}} $. \\
\\
 De plus on a $(n + 1)^{n} = \Sum{k = 0}{n} \binom{n}{k} n^{k} \geq
n^{n} + n n^{n-1} > n^{n} + n^{n-1}$ et on obtient que $f(q) < 1$ sur
$]0 ; 1[$. \\
 On ne déduit que les variables $X$ et $Y$ ne sont pas indépendantes.
\\
 \item On utilise le théorème de transfert : \\
 $\E(XY) = \Sum{j = 0}{n} \Sum{k = 1}{n} j k \Prob([ X = j] \cap [Y =
k] ) = \Sum{j = 1}{n} j^{2} \Prob([ X = j] \cap [Y = j] ) = \Sum{j =
1}{n} j^{2} \binom{n}{j} p^{j} q^{n-j} = \Sum{j = 0}{n} j^{2}
\binom{n}{j} p^{j} q^{n-j} = \E(X^{2}) = \V(X) + [ \ \E(X)]^{2} = n p
(1-p) + n^{2} p^{2} = n p ( 1 - p + np) = n p [(n-1) p + 1 ]$. \\
 Enfin on a $\Cov (X,Y) = \E(XY) - \E(X) \E(Y) = n p [(n-1) p + 1 ] - n
p ( n p + q^{n}) = np [ (n-1) p + 1 - np - q^{n}] = n p [ 1 - p -
q^{n}] = n p q ( 1 - q^{n-1} ) \neq 0$ sauf si $q^{n-1} = 1$ qui donne
$q = 1$. \\
 \item On retrouve que $X$ et $Y$ ne sont pas indépendantes, quelle que
soit la valeur de $q$. \\
 \end{noliste} 
 \end{noliste}
 \noindent \textbf{\underline{Exercice sans préparation}} \\
\\
 Étudier la convergence de la suite $(u_{n})_{n \in \N^* }$ définie par
:
 
\[
 \forall n \in \N^*,\ \ \ u_{n} = \frac{1}{n^{\alpha}} \Sum{k = 1}{n} k
\ln \left( 1 + \frac{k}{n} \right)
\]
 où $\alpha$ est un nombre réel, \begin{noliste}{1.}
 \setlength{\itemsep}{4mm}
 \item Si $\alpha = 2$ on a $u_{n} = \frac{1}{n} \Sum{k = 1}{n}
\frac{k}{n} \ln \left( 1 + \frac{k}{n} \right) \xrightarrow[ n
\rightarrow + \infty]{} \dint{0}{1} x \ln ( 1 + x )\ dx = \dint{1}{2}
(x-1) \ln x\ dx = \frac{ \ln 2 }{2} + \frac{1}{4} $ avec une
intégration par parties (somme de Riemann). \\
 \item Si $\alpha < 2$ on a $u_{n} = \frac{1}{n} \Sum{k = 1}{n}
\frac{k}{n} \ln \left( 1 + \frac{k}{n} \right) \times n^{2- \alpha}
\xrightarrow[ n \rightarrow + \infty]{} + \infty$. \\
 Si $\alpha > 2$ on a $u_{n} = \frac{1}{n} \Sum{k = 1}{n} \frac{k}{n}
\ln \left( 1 + \frac{k}{n} \right) \times n^{2- \alpha} \xrightarrow[ n
\rightarrow + \infty]{} 0$.
 \end{noliste}
 \end{exercice}

 \newpage

 \begin{exercice}{\it (Exercice avec préparation)}~
 \begin{noliste}{1.}
 \setlength{\itemsep}{4mm}
 \item Soit $X$ une variable à densité, de densité $f$. On dit que
 $X$ admet un moment d'ordre $r$ si $X^{r}$ admet une espérance et
 on pose $\E(X^{r})$ le moment d'ordre $r$ de $X$. \\
 Avec le théorème de transfert, $X$ admet un moment d'ordre $r$ si
 et seulement si l'intégrale $\dint{-\infty}{+ \infty} t^{r} f(t)\ \ dt
$ converge absolument (équivalent à la convergence grâce à la
 décomposition en une fonction de signe constant sur $[ 0 ;
 + \infty[$ et sur $] - \infty ; 0]$). \\
 Enfin si $X$ admet un moment d'ordre $r$, elle admet un moment
 d'ordre $k$ pour tout $1 \leq k \leq r$. \\
 \item On met au même dénominateur, on identifie et on obtient
 $\frac{1}{x (x + 1)} = \frac{ 1}{x} - \frac{1}{x + 1}$. 
 \item On pose :
 
\[
 f(x) = \left\{ 
\begin{array}{cc}
 \frac{k}{x(x + 1)} & \text{ si } x \geq 1 \\
0 & \text{ sinon } \\
\end{array}
\right.
\]
 où $k$ est un paramètre réel. 
 \begin{noliste}{a)}
 \setlength{\itemsep}{2mm}
 \item $f$ est positive et continue sauf en 1, vérifions
$\dint{-\infty}{+ \infty} f = \dint{1}{+ \infty}\left( \frac{k}{x} -
\frac{k}{x + 1} \right)\ dx = 1$. \\
 On a $\dint{1}{t} \left( \frac{k}{x} - \frac{k}{x + 1} \right)\ dx = k
\left[ \ \ln x - \ln (x + 1) \right]_{1}{t} = k \ln \left( \frac{t}{t +
1} \right) + k \ln 2 \xrightarrow[ t \rightarrow + \infty]{} k \ln 2$
donc l'intégrale converge et vaut $k \ln 2$. \\
 D'où $f$ est une densité de probabilité si et seulement si $k =
\frac{1}{\ln 2}$. \\
 \item L'intégrale $\dint{1}{+ \infty} x f(x)\ dx = \dint{1}{+ \infty}
\frac{1}{(x + 1) \ln 2}\ dx$ diverge comme intégrale de Riemann avec
$\alpha = 1$ donc $X$ n'admet pas d'espérance. \\
 \end{noliste}
 \item \begin{noliste}{a)}
 \setlength{\itemsep}{2mm} 
 \item On a $T (\Omega) = \N^*$ puis $\Prob\left(\Ev{\Ev{T = k}}\right)
= \Prob\left(\Ev{ k \leq X < k + 1}\right) = \dint{k}{k + 1} f(t)\ dt =
\left[ \ \ln \left( \frac{x}{x + 1} \right) \right]_{k}{k + 1} = \ln
\left( \frac{ (k + 1)^{2}}{k (k + 2)} \right) = \ln \left( \frac{ k^{2}
+ 2k + 1}{k (k + 2)} \right) = \ln \left( \frac{ k( k + 2) + 1}{k (k +
2)} \right) = \ln \left( 1 + \frac{ 1}{k (k + 2)} \right) $. \\
 \item On en déduit que $\Sum{n = 1}{+ \infty} \ln \left( 1 +
\frac{1}{n(n + 2) } \right) = 1$ (c'est la somme des probabilités sur
un système complet d'évènements). \\
 \end{noliste}
 \item $Z(\Omega) = \ ] 0 ; 1]$ et on a pour tout $x \leq 0$, $F_{Z}(x)
= 0$;\\
 Pour $0 < x \leq 1$, $F_{Z}(x) = \Prob\left(\Ev{ 0 <Z \leq x}\right) =
P \left(\Ev{ X \geq \frac{1}{x}}\right) = 1 - F_{X} \left( \frac{1}{x}
\right) = \frac{1}{\ln 2} \ln \left( \frac{ \frac{1}{x} + 1 }{
\frac{1}{x} } \right) = \frac{1}{\ln 2} \ln ( 1 + x )$; \\
 Enfin pour $x \geq 1$, $F_{Z}(x) = 1$. \\
 Enfin on obtient $f_{Z}(x) = 0$ si $x \leq 0$ ou $x \geq 1$ et
$f_{Z}(x) = \frac{1}{(1 + x) \ln 2}$ si $x \in \ ]0 ; 1[$. \\
 \item \begin{noliste}{a)}
 \setlength{\itemsep}{2mm}
 \item On a $Y (\Omega) = [ 0 ; 1[$ donc $F_{Y}(x) = 0$ si $x \leq 0$
et 1 si $x \geq 1$. \\
 Enfin si $0 < x < 1$ on a $F_{Y}(x) = \Sum{k = 1}{+ \infty}
\Prob\left(\Ev{ k \leq X \leq k + x}\right) = \frac{1}{\ln 2} \Sum{k =
1}{+ \infty} \ln \left( \frac{ (k + x) (k + 1)}{k (k + x + 1)} \right)
= \frac{1}{\ln 2} \Sum{k = 1}{+ \infty} \ln \left( \frac{ (k + x)}{k }
- \frac{k + x + 1}{k + 1} \right) = \dlim{n \rightarrow + \infty}
\frac{1}{\ln 2} \ln( 1 + x) - \frac{1}{\ln 2} \ln \left( \frac{ n + x +
1}{n + 1} \right) = \frac{1}{\ln 2} \ln (1 + x) $ (télescopage). \\
 On reconnaît la même loi que $Z$. \\
 \item Les deux autres intégrales étant trivialement convergentes vers
0, il faut prouver que $\dint{0}{1} \frac{ x^{r} }{(1 + x)\ln 2}$
converge, ce qui ne pose aucun problème puisque la fonction est
continue sur $[ 0 ; 1]$ et l'intégrale n'estg pas généralisée. \\
\item $\E(Y) = \frac{1}{\ln 2} \dint{0}{1} \frac{x}{(1 + x)}\ dx =
  \frac{1}{\ln 2} \dint{1}{2} \frac{x-1}{(x)}\ dx = \frac{1}{\ln 2}
  \dint{1}{} \dx - \frac{1}{\ln 2} \dint{1}{2} \frac{1}{x}\ dx =
  \frac{1}{\ln 2} - \frac{1}{\ln 2} \ln 2 = \frac{1}{\ln 2} - 1$. \\
 \end{noliste} 
 \end{noliste}
 \noindent \textbf{\underline{Exercice sans préparation}} \\
\\
 Soit $n \geq 2$ et $A = \begin{smatrix}
0 & 1 & \dots & 1 \\
1 & 0 & \dots & 1 \\
\vdots & & \ddots & \vdots \\
1 & 1 & \dots & 0 \\
\end{smatrix}
\in \mathcal{M}_{n} (\R)$. \\
\\
 Avec des pivots habiles et des calculs très rigoureux, on trouve
$A^{-1} = \frac{1}{n-1} ( A + (2-n) I)$. \\
 Méthode beaucoup plus simple : chercher un polynôme annulateur en
calculant $A^{2}$ : \\
 On trouve $A^{2} = (n-2) A + (n-1) I$ donc $A ( A + (2-n) I ) = (n-1)
I$ et enfin $A^{-1} = \frac{1}{n-1} ( A + (2-n) I )$.
 \end{exercice}

 \newpage

 \begin{exercice}{\it (Exercice avec préparation)}~
 \begin{noliste}{1.}
 \setlength{\itemsep}{4mm}
 \item Un estimateur d'un paramètre $\theta$ de la loi $P_{X}$ d'une
 variable aléatoire $X$ dont on dispose d'un échantillon $(X_{n})$
 est une suite de variables aléatoires $(T_{n})$ où pour tout $n$,
 $T_{n}$ est une fonction des variables $X_{1},\ \dots\, X_{n})$. On
 dit qu'il est sans biais si pour tout $n$, $\E(T_{n}) = \theta$. \\
 \\
 Soit $Z$ une variable aléatoire discrète d'espérance $\E(Z) = 
 \theta$ ($\theta \in \R^*$) et de variance $\V(Z) = 1$. \\
\\
 Pour $n$ entier de $\N^*$, on dispose d'un $n$-échantillon $(Z_{1},
Z_{2},\ \dots\, Z_{n})$ de variables aléatoires indépendantes et
 de même loi que $Z$, définies sur un espace probabilisé $(\Omega,
\mathcal{A}, P)$. \\
\\
 On pose $\overline{Z_{n}} = \frac{1}{n} \Sum{j = 1}{n} Z_{j}$. On
 suppose que $\theta$ est inconnu.
 \item 
 \begin{noliste}{a)}
 \setlength{\itemsep}{2mm} 
 \item Oui. Le calcul est évident. \\
 \item $R_{\theta} = \V(\overline{Z_{n}}) = \frac{1}{n^{2}} \times n
 \V(Z) = \frac{1}{n}$. \\
 \end{noliste}

 \item Soient $\beta_{1}, \beta_{2},\ \dots\, \beta_{n}$ des réels non
 nuls et $Y_{n} = \Sum{j = 1}{n} \beta_{j} Z_{j}$. 
 \begin{noliste}{a)}
 \setlength{\itemsep}{2mm}
 \item Il faut $\Sum{j = 1}{n} \beta_{j} = 1$. \\
\\
 On suppose que cette condition est vérifiée. \\
 \item On a vu $\V( \overline{Z_{n}} ) = \frac{1}{n}$. \\
 On a $\V( \overline{Z_{n}} + Y_{n} ) = V \left( \Sum{j = 1}{n} \left(
 \beta_{j} - \frac{1}{n} \right) Z_{j} \right) = \Sum{j = 1}{n}
 \left( \beta_{j} - \frac{1}{n} \right)^{2} \V(Z_{j}) = \Sum{j = 1}{n}
 \left( \beta_{j} - \frac{1}{n} \right)^{2}$. \\
 On en déduit que $\V(\overline{Z_{n}} ) + \V(Y_{n}) + 2 \Cov (
 \overline{Z_{n}}, Y_{n}) = \Sum{j = 1}{n} \left( \beta_{j} -
 \frac{1}{n} \right)^{2}$. \\
 D'où $\Cov ( \overline{Z_{n}}, Y_{n}) = \frac{1}{2} \left(
 \Sum{j = 1}{n} \left( \beta_{j} - \frac{1}{n} \right)^{2} -
 \beta_{j}{2} - \frac{1}{n^{2}} \right) = \Sum{j = 1}{n}
 \frac{\beta_{j}}{n} = \frac{ \Sum{j = 1}{n} \beta_{j} }{n} = 
 \frac{1}{n}$. \\
 On a alors $0 \leq \V(Y_{n} - \overline{Z_{n}} ) = \V(Y_{n}) + 
 V(\overline{Z_{n}} - 2 \Cov ( U_{n}, \overline{Z_{n}} ) = \V(Y_{n}) + 
 \frac{1}{n} - \frac{2}{n} = \V(Y_{n}) - \frac{1}{n} = \V(Y_{n}) - \V(
 \overline{Z_{n}} )$ donc $\V(Y_{n}) \geq \V( \overline{Z_{n}})$. \\
 Interprétation : 
 \end{noliste}
 \item Soient $\alpha_{1}, \alpha_{2},\ \dots\, \alpha_{n}$ des réels
non
 nuls. \\
 On définit la variable aléatoire $U_{n}$ par : $U_{n} = \Sum{j = 1}{n}
 \alpha_{j} Z_{j}$, \\
 et on suppose que $\E(U_{n}) = \theta$ et $\V(U_{n}) = \frac{1}{n}$.
\\
 On a alors $\V( U_{n} - \overline{Z_{n}} ) = \V( U_{n}) + \V(
 \overline{Z_{n}}) - 2 \Cov (U_{n}, \overline{Z_{n}}) = \frac{2}{n} -
 \frac{2}{n} = 0$. \\
 On en déduit que la variable $U_{n} - \overline{Z_{n}}$ est
 quasi-certaine : si elle pouvait prendre deux valeurs distinctes
 avec une probabilité non nulle, l'une des deux au moins serait
 différente de son espérance et la variance serait donc
 strictement positive. \\
 On en déduit qu'il existe $a$ tel que $\Prob\left(\Ev{ U_{n} -
\overline{Z_{n}} = 
 a }\right) = 1$. \\
 On a alors $\E( U_{n} - \overline{Z_{n}} ) = a$; or on sait que
 $\E(U_{n} - \overline{Z_{n}} ) = \theta - \theta = 0$, donc $a = 0$ et
 $U_{n} = \overline{Z_{n}}$ avec une probabilité égale à 1 (presque
 sûrement). 
 \end{noliste}
 \noindent \textbf{\underline{Exercice sans préparation}} \\
 Soit $f$ la fonction définie sur $\R_+^* \times \R_+^*$, à valeurs
 réelles, par : 
 
\[
 f(x,y) = \frac{x^{2} + xy + \sqrt{y} }{x \sqrt{y} }. 
\]
 \begin{noliste}{1.}
 \setlength{\itemsep}{4mm}
 \item On a des produits, sommes et quotient dont le dénominateur ne
 s'annule pas des fonctions $(x,y) \rightarrow x$, $(,y)
 \rightarrow y$ et $(x,y) \rightarrow \sqrt{y}$ qui sont de classe
 $C^{2}$ sur $\R_+^* \times \R_+^*$. 
 \item $f_{x}'(x,y) = \frac{(2x + y ) x \sqrt{y} - \sqrt{y} (x^{2} + xy
 + \sqrt{y} ) }{x^{2} y} = \frac{x^{2} - \sqrt{y} }{x^{2}
 \sqrt{y}}$ après simplifications. \\
 $f_{y}'(x,y) = \frac{\left(x + \frac{1}{2 \sqrt{y} } \right) x
 \sqrt{y} - \frac{x}{2 \sqrt{y} } ( x^{2} + xy + \sqrt{y} ) }{x^{2}
 y} = \frac{ y - x }{2 y \sqrt{y}}$ après simplifications. \\
 On en déduit que $f_{x}'(x,y) = f_{y}'(x,y)$ si et seulement si $x = 
 y$ et $x^{2} = \sqrt{y}$, qui donne $x^{2} = \sqrt{x}$, $x^{4} = x$,
 $x^{3} = 1$ car $x \neq 0$, et enfin $x = y = 1$. \\
 \item On a $f_{x,x}'' (x,y) = \frac{ 2x^{3} \sqrt{y} - 2 x \sqrt{y}
 (x^{2} - \sqrt{y}) }{ x^{4} y} = \frac{ 2 }{ x^{3}}$ donc $r = 
 2$. \\
 $f_{x,y}'' (x,y) = -\frac{ 1}{2 y \sqrt{y} }$ donc $s = -
\frac{1}{2}$. \\
 $f_{y,y}'' = \frac{ 2 y \sqrt{y} - (y-x) 3 \sqrt{y} }{4 y^{3}} = 
 \frac{3x - y}{4 y^{2} \sqrt{y}} $ donc $t = \frac{2}{4} = 
 \frac{1}{2}$. \\
 On a alors $rt - s^{2} = 1 - \frac{1}{4} = \frac{3}{4} > 0$ donc
 $f$ admet un extremum local en $(1,1)$, et $r = 2 > 0$ donc c'est
 un minimum local.
 \end{noliste}
 \end{exercice}

 \newpage

 

\end{document}