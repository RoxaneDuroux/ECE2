\documentclass[11pt]{article}%
\usepackage{geometry}%
\geometry{a4paper,
  lmargin=2cm,rmargin=2cm,tmargin=2.5cm,bmargin=2.5cm}

\input{../../../macros.tex}

% \renewcommand{\thesection}{\Roman{section}.\hspace{-.3cm}}
% \renewcommand{\thesubsection}{\Alph{subsection}.\hspace{-.2cm}}

\pagestyle{fancy} %
\lhead{ECE2 \hfill Mathématiques \\} %
\chead{\hrule} %
\rhead{} %
\lfoot{} %
\cfoot{} %
\rfoot{\thepage} %

\renewcommand{\headrulewidth}{0pt}% : Trace un trait de séparation
                                    % de largeur 0,4 point. Mettre 0pt
                                    % pour supprimer le trait.

\renewcommand{\footrulewidth}{0.4pt}% : Trace un trait de séparation
                                    % de largeur 0,4 point. Mettre 0pt
                                    % pour supprimer le trait.

\setlength{\headheight}{14pt}

\title{\bf \vspace{-1.6cm} HEC 2018} %
\author{} %
\date{} %
\begin{document}
\maketitle %
\vspace{-1.2cm}\hrule %
\thispagestyle{fancy}

\vspace*{.2cm}

%%DEBUT

\subsection*{Sujet E 1}

\begin{exerciceAP}~
  \begin{noliste}{1.}
    \item Question de cours : Définition d'une famille génératrice
    d'un espace vectoriel $E$ de dimension $n$.\\
    Relation entre cardinal d'une famille génératrice et dimension 
    de $E$.
    
    \item On note ${\cal C}(f) = \{g \in \LL{E} \ | \ f \circ g = 
    g \circ f \}$.
    \begin{noliste}{a)}
      \item Montrer que ${\cal C}(f)$ est un espace vectoriel.
      
      \item Quelle peut être la dimension maximale de ${\cal C}(f)$ ?
    \end{noliste}
    
    \item On note $J = 
    \begin{smatrix}
      0 & 1\\
      0 & 0
    \end{smatrix}$.
    \begin{noliste}{a)}
      \item Déterminer $\{ M \in \M{2} \ | \ MJ = JM\}$.
      
      \item On note $j$ l'endomorphisme de $E$ dont une matrice 
      représentative est $J$. \\
      Quelle est la dimension de ${\cal C}(j)$ ?
    \end{noliste}
    
  \item Soit $a \in E \setminus \{0_E\}$ et soit $f \in \LL{E}$.\\
    On suppose qu'il existe $n_0 \in \N^*$ tel que la famille $(a,
    f(a), f^2(a), \ldots, f^{n_0}(a))$ est génératrice de $E$.
    \begin{noliste}{a)}
      \item Montrer qu'il existe un plus grand entier $p\in \N^*$ tel 
      que la famille $(a, f(a), \ldots, f^{p-1}(a))$ est libre.
      
    \item Montrer que $(a,f(a), \ldots, f^{p-1}(a))$ est une base de
      $E$.
      
    \item Que vaut $p$ ?
      
    \item Montrer que, pour tout $g \in {\cal C}[f)$, il existe
      $(\alpha_0, \ldots, \alpha_{n-1}) \in \R^{n}$ tel que $g =
      \Sum{i=0}{n-1} \alpha_i \, f^i$.
      
    \item En déduire $\dim({\cal C}(f))$.
    \end{noliste}
  \end{noliste}
\end{exerciceAP}



\begin{exerciceSP}~\\
  On considère une urne contenant initialement une boule rouge et une 
  boule verte.\\
  On effectue des tirages dans cette urne de la manière suivante : 
  si on tire une boule d'une couleur, on la remet dans l'urne et on 
  ajoute une boule de la couleur opposée.\\
  On note $X$ la \var associée au rang d'obtention de la première 
  boule rouge.\\
  Montrer que $X$ admet une espérance et la calculer.
\end{exerciceSP}




\newpage




\subsection*{Sujet E 2}

\begin{exerciceAP}~
  \begin{noliste}{1.}
    \item Question de cours : Formule de Taylor-Young.
    
    \item On note $f: t \mapsto \ln(1+t) -t$.
    \begin{noliste}{a)}
      \item Quel est l'ensemble de définition de $f$ ?
      
      \item Donner un équivalent de $f$ en $0$.
      
      \item Montrer que la série de terme général $f\left( \frac{1}{n}
        \right)$ est convergente.
    \end{noliste}
    
  \item Soit $(u_n)_{n \in \N^*}$ une suite de réels strictement
    positifs et soit $a>0$. \\
    On définit une suite $(w_n)_{n \in \N^*}$ de la manière suivante :
    \[
      \forall n \in \N^*, \ w_n = \dfrac{u_{n+1}}{u_n} -1 + 
      \dfrac{a}{n}
    \]
    On suppose que la série $\Sum{n \geq 1}{} w_n$ converge 
    absolument.
    \begin{noliste}{a)}
      \item Montrer que la série $\Sum{n \geq 1}{} \dfrac{w_n}{n}$
      est convergente.
      
      \item Montrer que la série $\Sum{n \geq 1}{} w_n^2$ est 
      convergente.
    \end{noliste}
    
  \item On définit la suite $(l_n)_{n \in \N^*}$ par : $\forall n \in
    \N^*$, $l_n = \ln (n^a \, u_n)$.
    \begin{noliste}{a)}
      \item Montrer que, pour tout $n \in \N^*$ :
      \[
        l_{n+1} - l_n \ = \ ...
      \]
      
      \item Montrer que la série $\Sum{n \geq 1}{} (l_{n+1} - l_n)$
      est convergente.
    \end{noliste}
    
    \item 
    \begin{noliste}{a)}
      \item Montrer que la suite $(l_n)_{n \in \N^*}$ converge.
      
      \item En déduire que : $u_n \eqn{} \dfrac{A}{n^a}$, où $A>0$.
    \end{noliste}
  \end{noliste}
\end{exerciceAP}



\begin{exerciceSP}~
  \begin{noliste}{1.}
    \item Tracer la fonction de répartition d'une loi normale 
    $\Norm{0}{1}$.
    
    \item 
    
    \item Que fait le programme suivant ?
    \begin{scilab}
      & N = 1000 \nl %
      & X = zeros(N,1) \nl %
      & \tcFor{for} i = 1:N \nl %
      & \qquad X(i,1) = grand(1,1,\ttq{}norm\ttq{},0,1) \Sfois{}
      grand(1,1,\ttq{}bin\ttq{},1,1/2) \nl %
      & \tcFor{end} \nl %
      & p = length(find(X < 1,96)) / N
    \end{scilab}
  \end{noliste}
\end{exerciceSP}




\newpage



\subsection*{Sujet E 3}

\begin{exerciceAP}~
  \begin{noliste}{1.}
    \item Question de cours : énoncé de l'inégalité de Markov.\\
    Citer une conséquence de cette inégalité.
    
  \item Soit $X_1$, $\ldots$, $X_n$ des \var mutuellement
    indépendantes, de même loi définie par :
    \[
      \begin{array}{l}
        X_1(\Omega) = \{-1,1\} \\[.2cm]
        \Prob(\Ev{X_1 = -1}) = \dfrac{1}{2} \quad \text{et} \quad
        \Prob(\Ev{X_1=1}) = \dfrac{1}{2}
      \end{array}
    \]
    On note : $S_n = \Sum{k=1}{n} X_k$.
    \begin{noliste}{a)}
      \item Calculer $\E(S_n)$.
      
      \item Calculer $\V(S_n)$.
    \end{noliste}
    
    \item Déterminer $\E(S_n^4)$.
    
    \item Soit $\alpha >0$.\\
      Montrer que $\Prob\left(\Ev{ \dfrac{\vert S_n \vert}{n} \geq
          \dfrac{1}{n^\alpha}}\right) \leq ...$.
  \end{noliste}
\end{exerciceAP}




\begin{exerciceSP}~\\
  Soit $f \in \LL{E}$.
  \begin{noliste}{1.}
  \item Quel est le lien entre le spectre de $f$ et un polynôme
    annulateur de $f$ ?
    
  \item Supposons que l'endomorphisme $f$ est diagonalisable et
    que : $\im(f) \subset \kr(f)$.\\
    Que dire de l'endomorphisme $f$ ?
  \end{noliste}
\end{exerciceSP}




%%FIN


\end{document}
