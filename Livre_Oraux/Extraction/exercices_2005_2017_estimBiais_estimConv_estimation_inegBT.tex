\subsection*{Exercices oraux HEC 2017}

%%% EPR %%% HEC;
% type : oralAP; %
% sujet : E 91; %
% annee : 2017; %
% theme : vaDiscrete, vaUniformeDiscrete, vaMax, estimBiais, 
% estimation; %

\begin{exerciceAP}~
  \begin{noliste}{1.}
    \setlength{\itemsep}{2mm}
    \item Question de cours : définition d'un estimateur sans biais
    d'un paramètre inconnu.
  \end{noliste}
  
  \noindent
  Soit $N$ un entier supérieur ou égal à $2$.\\
  Soit $(X_n)_{n\geq 1}$ une suite de variables aléatoires 
  indépendantes, définies sur le même espace probabilisé 
  $(\Omega, \A, \Prob)$, et suivant chacune la loi uniforme discrète
  sur $\llb 1,N \rrb$.
  \begin{noliste}{1.}
    \setlength{\itemsep}{2mm}
    \setcounter{enumi}{1}
    \item Pour tout entier $n\geq 1$, on pose : $s_n(N) = 
    \Sum{k=1}{N-1} \left( \dfrac{k}{N} \right)^n$.
    \begin{noliste}{a)}
    \setlength{\itemsep}{2mm}
      \item Montrer que la suite $(s_n(N))_{n\geq 1}$ est 
      strictement monotone et convergente.
      
      \item Trouver sa limite.
    \end{noliste}
    
    \item Pour tout entier $n\geq 1$, on pose : $T_n = \max(X_1,
    X_2, \ldots, X_n)$.
    \begin{noliste}{a)}
    \setlength{\itemsep}{2mm}
      \item Calculer pour tout $k\in \llb 1,N \rrb$, $\Prob(
      \Ev{T_n=k})$.
      
      \item Montrer que $\E(T_n) = N - s_n(N)$.
    \end{noliste}
    
    \item 
    \begin{noliste}{a)}
    \setlength{\itemsep}{2mm}
      \item Justifier que $\Prob(\Ev{ \vert T_N - N \vert \geq 1})$
      tend vers $0$ quand $n$ tend vers l'infini.
      
      \item En déduire, pour tout $\eps>0$, la limite de 
      $\Prob(\Ev{\vert T_n - N \vert \geq \eps})$ quand $n$ tend 
      vers l'infini.
    \end{noliste}
    
    \item On suppose dans cette question que $N$ est un paramètre 
    inconnu.
    \begin{noliste}{a)}
    \setlength{\itemsep}{2mm}
      \item Expliquer pourquoi on ne peut pas dire que $T_n + s_n(N)$
      est un estimateur sans biais de $N$.
      
      \item Trouver une suite convergente et asymptotiquement sans 
      biais d'estimateurs de $N$.
    \end{noliste}
  \end{noliste}
\end{exerciceAP} 


