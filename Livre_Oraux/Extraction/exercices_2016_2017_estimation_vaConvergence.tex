\subsection*{Exercices oraux HEC 2016}

%EPR% HEC;
% type : oralSP; % 
% sujet : E 83; %
% annee : 2016; % 
% theme : vaDensite, vaUniformeDensite, vaDiscrete, vaUniformeDiscrete,
% vaConvergence; % 

\begin{exerciceSP}~\\
Les variables aléatoires de cet exercice sont supposées définies sur un 
espace probabilisé $(\Omega,\A,\Prob)$.\\
Soit $Z$ une variable aléatoire qui suit la loi uniforme sur 
l'intervalle $[0,1]$ et pour tout entier $n\geq 1$, on note $Y_n$ une 
variable aléatoire à valeurs dans 
$\left\{0,\dfrac{1}{n},\dfrac{2}{n},\hdots, \dfrac{n-1}{n}\right\}$ 
telle que : 
\[
  \forall k\in\llb 0,n-1\rrb, \ 
  \Prob\left(\Ev{Y_n=\dfrac{k}{n}}\right)=\dfrac{1}{n}
\]
Soit $f$ une fonction définie et continue sur $[0,1]$. Montrer que 
$\dlim{n\to+\infty} \E(f(Y_n))=\E(f(Z))$.
\end{exerciceSP} 


%EPR% HEC;
% type : oralAP; % 
% sujet : E 88; %
% annee : 2016; % 
% theme : vaConvergence, vaPartieEntiere, vaDensite, vaExponentielle,
% vaDiscrete, vaGeometrique, vaUniformeDensite; % 

\begin{exerciceAP}~\\
Toutes les variables aléatoires qui interviennent dans l'exercice sont 
supposées définies sur le même espace probabilisé $(\Omega, 
\A,\Prob)$.
\begin{noliste}{1.}
    \setlength{\itemsep}{2mm}
  \item Question de cours : Convergence en loi d'une suite de variables 
  aléatoires.\\
  Dans tout l'exercice, $X$ désigne une variable aléatoire suivant la 
  loi exponentielle de paramètre $\lambda>0$.
  
  \item 
  \begin{noliste}{a)}
    \setlength{\itemsep}{2mm}
    \item On pose : $T=\lfloor X\rfloor$ (partie entière de $X$). 
    Montrer que la loi de $T$ est donnée par :
    \[
      \forall k \in\N, \ 
      \Prob(\Ev{T=k}) = \Big(1 - \ee^{-\lambda} \Big) \, 
      \Big(\ee^{-\lambda}\Big)^k
    \]
    
    \item Quelle est la loi de $T+1$ ? En déduire l'espérance et la 
    variance de $T$.
  \end{noliste}
  
  \item On pose : $Z=X-\lfloor X\rfloor$.\\
  Montrer que $Z$ est une variable aléatoire à densité et déterminer 
  une densité de $Z$.

  \item Soit $(X_n)_{n\geq 1}$ une suite de variables aléatoires 
  indépendantes telles que, pour tout $n\in\N^*$, $X_n$ suit une loi 
  exponentielle de paramètre $\dfrac{\lambda}{n}$. On pose pour tout 
  $n\in\N^*$ : $Z_n=X_n-\lfloor X_n\rfloor$.\\
  Montrer que la suite de variables aléatoires $(Z_n)_{n\geq 1}$ 
  converge en loi vers une variable aléatoire dont on précisera la loi.
\end{noliste}
\end{exerciceAP} 


\subsection*{Exercices oraux HEC 2017}

%EPR% HEC;
% type : oralAP; %
% sujet : E 91; %
% annee : 2017; %
% theme : vaDiscrete, vaUniformeDiscrete, vaMax, estimBiais, 
% estimation; %

\begin{exerciceAP}~
  \begin{noliste}{1.}
    \setlength{\itemsep}{2mm}
    \item Question de cours : définition d'un estimateur sans biais
    d'un paramètre inconnu.
  \end{noliste}
  
  \noindent
  Soit $N$ un entier supérieur ou égal à $2$.\\
  Soit $(X_n)_{n\geq 1}$ une suite de variables aléatoires 
  indépendantes, définies sur le même espace probabilisé 
  $(\Omega, \A, \Prob)$, et suivant chacune la loi uniforme discrète
  sur $\llb 1,N \rrb$.
  \begin{noliste}{1.}
    \setlength{\itemsep}{2mm}
    \setcounter{enumi}{1}
    \item Pour tout entier $n\geq 1$, on pose : $s_n(N) = 
    \Sum{k=1}{N-1} \left( \dfrac{k}{N} \right)^n$.
    \begin{noliste}{a)}
    \setlength{\itemsep}{2mm}
      \item Montrer que la suite $(s_n(N))_{n\geq 1}$ est 
      strictement monotone et convergente.
      
      \item Trouver sa limite.
    \end{noliste}
    
    \item Pour tout entier $n\geq 1$, on pose : $T_n = \max(X_1,
    X_2, \ldots, X_n)$.
    \begin{noliste}{a)}
    \setlength{\itemsep}{2mm}
      \item Calculer pour tout $k\in \llb 1,N \rrb$, $\Prob(
      \Ev{T_n=k})$.
      
      \item Montrer que $\E(T_n) = N - s_n(N)$.
    \end{noliste}
    
    \item 
    \begin{noliste}{a)}
    \setlength{\itemsep}{2mm}
      \item Justifier que $\Prob(\Ev{ \vert T_N - N \vert \geq 1})$
      tend vers $0$ quand $n$ tend vers l'infini.
      
      \item En déduire, pour tout $\eps>0$, la limite de 
      $\Prob(\Ev{\vert T_n - N \vert \geq \eps})$ quand $n$ tend 
      vers l'infini.
    \end{noliste}
    
    \item On suppose dans cette question que $N$ est un paramètre 
    inconnu.
    \begin{noliste}{a)}
    \setlength{\itemsep}{2mm}
      \item Expliquer pourquoi on ne peut pas dire que $T_n + s_n(N)$
      est un estimateur sans biais de $N$.
      
      \item Trouver une suite convergente et asymptotiquement sans 
      biais d'estimateurs de $N$.
    \end{noliste}
  \end{noliste}
\end{exerciceAP} 


