\subsection*{Exercices oraux HEC 2014}

%%% EPR %%% HEC;
% type : oralAP; %
% sujet : E 26; %
% annee : 2014; %
% theme : vaCouple, vaDiscrete, vaFctGeneratrice, vaCovariance,
% denombrement, fct2var; %

\begin{exerciceAP}~
  \begin{noliste}{1.}
    \setlength{\itemsep}{2mm}
  \item Question de cours : Définition de l'indépendance de deux
    variables aléatoires discrètes. Lien entre indépendance et
    covariance.\\
    Soit $X$ et $Y$ deux variables aléatoires discrètes finies à
    valeurs dans $\N$, définies sur un espace probabilisé
    $(\Omega,\A,P)$. On suppose que $X(\Omega)\subset \llb0,n\rrb$ et
    $Y(\Omega) \subset \llb 0, m \rrb$, où $n$ et $m$ sont deux
    entiers de $\N^*$.\\
    Pour tout couple $(i,j) \in \llb 0, n \rrb \times \llb 0, m \rrb$,
    on pose : $p_{i,j} = \Prob(\Ev{X=i} \cap \Ev{Y=j})$.\\
    Soit $F_X$ et $F_Y$ les deux fonctions de $\R$ dans $\R$ définies
    par : $F_X(x)=\Sum{i=0}{n} \Prob(\Ev{X=i})x^i$ et \\
    $F_Y(x)=\Sum{j=0}{m} \Prob(\Ev{Y=j})x^j$.\\
    Soit $Z=(X,Y)$ et $G_Z$ la fonction de $\R^2$ dans $\R$ définie
    par : $G_Z(x,y)=\Sum{i=0}{n}\Sum{j=0}{m} p_{i,j}x^iy^j$.

  \item Donner la valeur de $G_Z(1,1)$ et exprimer les espérances de
    $X$, $Y$ et $XY$, puis la covariance de $(X,Y)$ à l'aide des
    dérivées partielles premières et secondes de $G_Z$ au point
    $(1,1)$.

  \item Soit $f$ une fonction polynomiale de deux variables définies
    sur $\R^2$ par : $f(x,y) = \Sum{i=0}{n} \Sum{j=0}{m}
    a_{i,j}x^iy^j$ avec $a_{i,j} \in \R$.\\
    On suppose que pour tout couple $(x,y)\in\R^2$, on a $f(x,y)=0$.
    \begin{noliste}{a)}
    \setlength{\itemsep}{2mm}
  \item Montrer que pour tout $(i,j) \in \llb0, n \rrb \times \llb 0,
    m \rrb$, on a $a_{i,j}=0$.
    \item En déduire que $X$ et $Y$ sont indépendantes, si et
      seulement si, pour tout $(x,y)\in\R^2$,\\
      $G_Z(x,y)=F_X(x)F_Y(y)$. (on pourra poser : $a_{i,j}=p_{i,j} -
      \Prob(\Ev{X=i})\Prob(\Ev{Y=j})$).
    \end{noliste}

  \item Une urne contient des jetons portant chacun une des lettres
    $A$, $B$ ou $C$. La proportion des jetons portant la lettre $A$
    est $p$, celle des jetons portant la lettre $B$ est $q$ et celle
    des jetons portant la lettre $C$ est $r$, où $p$, $q$ et $r$ sont
    trois réels strictement positifs vérifiant $p+q+r=1$.\\
    Soit $n\in\N^*$. On effectue $n$ tirages d'un jeton avec remise
    dans cette urne. On note $X$ (resp. $Y$) la variable aléatoire
    égale au nombre de jetons tirés portant la lettre $A$ (resp. $B$)
    à l'issue de ces $n$ tirages.
    \begin{noliste}{a)}
    \setlength{\itemsep}{2mm}
    \item Quelles sont les lois de $X$ et $Y$ respectivement ?
      Déterminer $F_X$ et $F_Y$.
    \item Déterminer la loi de $Z$. En déduire $G_Z$.
    \item Les variables aléatoires $X$ et $Y$ sont-elles indépendantes
      ?
    \item Calculer la covariance de $(X,Y)$. Le signe de cette
      covariance était-il prévisible ?
    \end{noliste}
  \end{noliste}
\end{exerciceAP} 


%%% EPR %%% HEC;
% type : oralAP; %
% sujet : E 39; %
% annee : 2014; %
% theme : vaDiscrete, vaPoisson, vaCouple, vaCovariance, vaSomme, FPT; %

\begin{exerciceAP}~
  \begin{noliste}{1.}
    \setlength{\itemsep}{2mm}
  \item Question de cours : \\
    Loi d'un couple de variables aléatoires discrètes. Lois
    marginales. Lois conditionnelles.
  \end{noliste}
  Soit $c$ un réel strictement positif et soit $X$ et $Y$ deux
  variables aléatoires à valeurs dans $\N$ définies sur un espace
  probabilisé $(\Omega, \A, \Prob)$, telles que :
  \[
  \forall (i,j)\in\N^2, \ \Prob(\Ev{X=i}\cap\Ev{Y=j}) =
  c\dfrac{i+j}{i!j!}
  \]
  \begin{noliste}{1.}
    \setcounter{enumi}{1}
  \item 
    \begin{noliste}{a)}
    \setlength{\itemsep}{2mm}
  \item Montrer que pour tout $i\in\N$, on a : $\Prob(\Ev{X=i}) = c
    \dfrac{(i+1)}{i!} \ \ee$. \\
    En déduire la valeur de $c$.
    \item Montrer que $X$ admet une espérance et une variance et les
      calculer.
    \item Les variables aléatoires $X$ et $Y$ sont-elles indépendantes
      ?
    \end{noliste}
  \item
    \begin{noliste}{a)}
    \setlength{\itemsep}{2mm}
    \item Déterminer la loi de $X+Y-1$.
    \item En déduire la variance de $X+Y$.
    \item Calculer la covariance de $X$ et de $X+5Y$.\\
      Les variables aléatoires $X$ et $X+5Y$ sont-elles indépendantes
      ?
    \end{noliste}

  \item On pose : $Z=\dfrac{1}{X+1}$.
    \begin{noliste}{a)}
    \setlength{\itemsep}{2mm}
    \item Montrer que $Z$ admet une espérance et la calculer.
    \item Déterminer pour $i\in\N$, la loi conditionnelle de $Y$
      sachant $\Ev{X=i}$.
    \item Pour $A\in\A$, on pose : $g_A(Y)=\Sum{k=0}{+\infty} k
      P_A(\Ev{Y=k})$.\\
      Établir l'existence d'une fonction affine $f$ telle que, pour
      tout $\omega\in\Omega$, on a : 
      \[
      g_{\Ev{X=X(\omega)}}(Y) = f(Z(\omega))
      \]
    \end{noliste}
  \end{noliste}
\end{exerciceAP} 


\subsection*{Exercices oraux HEC 2016}

%%% EPR %%% HEC;
% type : oralAP; % 
% sujet : E 85; %
% annee : 2016; % 
% theme : vaCouple, vaCovariance, vaDiscrete, vaMoments; % 

\begin{exerciceAP}~\\
Toutes les variables aléatoires qui interviennent dans l'exercice sont 
supposées définies sur le même espace probabilisé 
$(\Omega,\A,\Prob)$.
\begin{noliste}{1.}
    \setlength{\itemsep}{2mm}
  \item Question de cours : Définition et propriétés de la covariance 
  de deux variables aléatoires discrètes.\\
  Soit $p$, $q$ et $r$ des réels fixés de l'intervalle $]0,1[$ tels que 
  $p+q+r=1$. \\
  Soit $(X_n)_{n\geq 1}$ une suite de variables aléatoires à 
  valeurs dans $\{-1,0,1\}$, indépendantes et de même loi donnée par :
  \[
    \forall n \in\N^*, \ \Prob(\Ev{X_n=1})=p, \ \Prob(\Ev{X_n=-1})=q, \ 
    \Prob(\Ev{X_n=0})=r.
  \]
  On pose pour tout entier $n\geq 1$ : $Y_n=\Prod{k=1}{n} X_k$.
  
  \item 
  \begin{noliste}{a)}
    \setlength{\itemsep}{2mm}
    \item Pour tout entier $n\geq 1$, préciser $Y_n(\Omega)$ et 
    calculer $\Prob(\Ev{Y_n=0})$.
    
    \item Pour tout entier $n\geq 1$, calculer $\E(X_n)$ et $\E(Y_n)$.
  \end{noliste}
  
  \item On pose pour tout entier $n\geq 1$, on a : 
  $p_n=\Prob(\Ev{Y_n=1})$.
  \begin{noliste}{a)}
    \setlength{\itemsep}{2mm}
    \item Calculer $p_1$ et $p_2$.
    
    \item Établir une relation de récurrence entre $p_{n+1}$ et 
    $p_n$.
    
    \item En déduire que pour tout entier $n\geq 1$, on a : 
    $p_n=\dfrac{(p+q)^n +(p-q)^n}{2}$.
    
    \item Pouvait-on à l'aide de la question $2$, trouver 
    directement la loi de $Y_n$ ?
  \end{noliste}
  
  \item 
  \begin{noliste}{a)}
    \setlength{\itemsep}{2mm}
    \item Établir l'inégalité : $(p+q)^n > (p-q)^{2n}$. Calculer 
    $\V(Y_n)$.
    
    \item Calculer la covariance $\cov(Y_n,Y_{n+1})$ des deux 
    variables aléatoires $Y_n$ et $Y_{n+1}$.
  \end{noliste}
\end{noliste}
\end{exerciceAP} 


\subsection*{Exercices oraux HEC 2017}

%%% EPR %%% HEC;
% type : oralAP;
% sujet : E 81;
% annee : 2017;
% theme : vaDiscrete, vaGeometrique, sciSimuVaDiscrete, denombrement,
% vaCouple; %

\begin{exerciceAP}~
  \begin{noliste}{1.}
    \setlength{\itemsep}{2mm}
    \item Question de cours
    \begin{noliste}{a)}
    \setlength{\itemsep}{2mm}
      \item Définition et propriétés de la loi géométrique.
      
      \item Compléter la ligne de code \Scilab{} contenant des points
      d'interrogation pour que la fonction {\tt geo} suivante 
      fournisse une simulation de la loi géométrique dont le 
      paramètre est égal à l'argument $p$ de la fonction.
      \begin{scilab}
        & \tcFun{function} \tcVar{x} = geo(\tcVar{p}) \nl %
        & \qquad \tcVar{x} = 1; \nl %
        & \qquad \tcFor{while} rand() ??? \nl %
        & \qquad \qquad \tcVar{x} = \tcVar{x} + 1; \nl %
        & \qquad \tcFor{end}; \nl %
        & \tcFun{endfunction}
      \end{scilab}~
    \end{noliste}
   \end{noliste}
    Une urne contient trois jetons numérotés $1$, $2$ et $3$. On 
    effectue dans cette urne, une suite de tirages d'un jeton avec 
    remise.
    \begin{noliste}{1.}
    \setlength{\itemsep}{2mm}
      \setcounter{enumi}{1}
      \item On note $Y$ la variable aléatoire égale au nombre de 
      tirages nécessaires pour obtenir, pour la première fois, deux
      numéros successifs distincts.
      \begin{noliste}{a)}
    \setlength{\itemsep}{2mm}
	\item Reconnaître la loi de la variable aléatoire $Y-1$.
	
	\item Déterminer l'espérance $\E(Y)$ et la variance $\V(Y)$
	de la variable aléatoire $Y$.
      \end{noliste}
      
      \item On note $Z$ la variable aléatoire égale au nombre de 
      tirages nécessaires pour obtenir, pour la première fois, les 
      trois numéros.
      \begin{noliste}{a)}
    \setlength{\itemsep}{2mm}
	\item Soit deux entiers $k \geq 2$ et $\ell \geq 3$.\\
	Calculer $\Prob(\Ev{Y=k} \cap \Ev{Z=\ell})$ selon les 
	valeurs de $k$ et $\ell$.
	
	\item En déduire que, pour tout entier $\ell \geq 3$, on a :
	$\Prob(\Ev{Z = \ell}) = \dfrac{2}{3} \, \left( 
	\dfrac{2^{\ell-2} -1}{3^{\ell -2}}\right)$.
	
	\item Calculer $\E(Z)$.
      \end{noliste}
      
      \item D'une manière plus générale, calculer l'espérance de la 
      variable aléatoire égale au nombre de tirages nécessaires 
      pour obtenir, pour la première fois, tous les numéros, dans 
      l'hypothèse où l'urne contient au départ $n$ jetons, numérotés
      de $1$ à $n$.
    \end{noliste}
\end{exerciceAP} 


