\subsection*{Exercices oraux HEC 2013}

%%% EPR %%% HEC;
% type : oralAP; % 
% sujet : E 20; %
% annee : 2013; % 
% theme : vaDiscrete, vaBernoulli, vaBinomiale, FPT; % 

\begin{exerciceAP}~
  \begin{noliste}{1.}
    \setlength{\itemsep}{2mm}
  \item Question de cours : Le schéma binomial.
  \item Soit $(X_n)_{ n \in \N^* }$ une suite de variables aléatoires
    définies sur un espace probabilisé $(\Omega , \A , P )$,
    indépendantes et de même loi de Bernoulli de paramètre
    $\dfrac{1}{2}$.\\
    On pose, pour tout $n \in \N^*$, $W_n = \Sum{k=1}{n} k X_k$ et
    $s_n = \dfrac{n(n+1)}{2}$.
    \begin{noliste}{a)}
      \setlength{\itemsep}{2mm}
    \item Calculer l'espérance $\E(W_n)$ et la variance $\V(W_n)$ de
      la variable aléatoire $W_n$.
    \item Calculer les probabilités $\Prob(\Ev{W_n = 0})$ et
      $\Prob(\Ev{W_n = s_n})$.
    \item Calculer, selon les valeurs de $n$, la probabilité
      $\Prob(\Ev{W_n = 3})$.
    \end{noliste}

  \item Montrer que pour tout $k \in \llb 0, s_n \rrb$, on a :
    $\Prob(\Ev{W_n = k}) = \Prob(\Ev{W_n = s_n - k})$.

  \item 
    \begin{noliste}{a)}
      \setlength{\itemsep}{2mm}
    \item Déterminer pour tout $j \in \llb 0, s_n \rrb$ la loi de
      probabilité conditionnelle de $W_{n+1}$ sachant $(W_n = j)$.
    \item En déduire les relations :
      \[
      \Prob(\Ev{W_{n+1} = k}) = \left\{
        \begin{array}{ll} 
          \frac{ 1 }{ 2 } \ \Prob(\Ev{W_n = k}) & \text{ si } k \leq n \\ 
          \\ 
          \frac{ 1 }{ 2 } \ \Prob(\Ev{W_n = k}) + \frac{1}{2}
          \Prob(\Ev{W_n = k-n-1}) & \text{ si } n+1 \leq k \leq s_n \\ 
          \\
          \frac{1}{2} \Prob(\Ev{W_n = k-n-1}) & \text{ si } s_n + 1 \leq
          k \leq s_{n+1} 
        \end{array} 
      \right.
      \]
    \end{noliste}
  \end{noliste}
\end{exerciceAP} 


%%% EPR %%% HEC;
% type : oralAP; %
% sujet : E 33; %
% annee : 2013; %
% theme : vaDiscrete, FPT, FPC, reductionMat, chaineMarkov, sciLFGN,
% sciSimuVaDiscrete; %

\begin{exerciceAP}~
  \begin{noliste}{1.}
    \setlength{\itemsep}{2mm}
  \item Question de cours : Formule des probabilités totales.\\
    Soit $p$ et $q$ deux réels vérifiant $0<p<1$ et $p+2q=1$. On note
    $\Delta$ la matrice de $\mathcal{M}_3 ( \R )$ définie par :
    \[
    \Delta = 
    \begin{smatrix} 
      p & q & q \\ 
      q & p & q \\ 
      q & q & p 
    \end{smatrix}
    \]

  \item Justifier que $\Delta$ est une matrice diagonalisable.

  \item Soit $D$ la matrice diagonale de $\mathcal{M}_3 ( \R )$
    semblable à $\Delta$ dont les éléments diagonaux sont écrits dans
    l'ordre croissant. Que peut-on dire de la limite des coefficients
    de $D^n$ lorsque $n$ tend vers $+\infty$.\\

    Un village possède trois restaurants $R_1$, $R_2$ et $R_3$. Un
    couple se rend dans un de ces trois restaurants chaque dimanche. A
    l'instant $n=1$ (c'est-à-dire le premier dimanche) il choisit le
    restaurant $R_1$, puis tous les dimanches suivants (instants
    $n=2$, $n=3$, etc.) il choisit le même restaurant que le dimanche
    précédent avec la probabilité $p$ ou change de restaurant avec la
    probabilité $2q$, chacun des deux autres restaurants étant choisis
    avec la même probabilité.\\

    On suppose que l'expérience est modélisée par un espace
    probabilisé $(\Omega , \A , P)$.

  \item Calculer la probabilité que le couple déjeune dans le
    restaurant $R_1$, respectivement $R_2$, respectivement $R_3$, le
    $n$-ième dimanche ($n \geq 2$).

  \item Soit $T$ la variable aléatoire égale au rang du premier
    dimanche où le couple retourne au restaurant $R_1$, s'il y
    retourne, et 0 sinon. 

    \begin{noliste}{a)}
    \setlength{\itemsep}{2mm}
    \item Déterminer la loi de $T$.
    \item Établir l'existence de l'espérance et de la variance de $T$
      et les calculer.
    \end{noliste}

  \item Écrire une procédure scilab permettant de calculer la
    fréquence de visite du restaurant $R_1$ par le couple en 52
    dimanches.
  \end{noliste}
\end{exerciceAP} 


%%% EPR %%% HEC;
% type : oralAP; %
% sujet : E 40; %
% annee : 2013; %
% theme : vaDiscrete, denombrement, fct2var, reductionMat; %

\begin{exerciceAP}~
  \begin{noliste}{1.}
    \setlength{\itemsep}{2mm}
  \item Question de cours : Soit $f$ une fonction de classe $C^2$
    définie sur une partie de $\R^2$ à valeurs réelles. Rappeler la
    définition d'un point critique et la condition suffisante
    d'extremum local en un point.\\

    Soit $X$ une variable aléatoire discrète finie définie sur un
    espace probabilisé $(\Omega , \A , P)$.\\
    On pose pour tout $n \in \N^*$ : $X ( \Omega ) = \{ x_1 , \dots ,
    x_n \} \subset \R$ et on suppose que $\forall i \in \llb 1 ; n
    \rrb$, $P (X=x_i) \neq 0$.\\

    On définit l'entropie de $X$ par : $ H(X) = - \frac{ 1 }{ \ln 2 }
    \Sum{i=1}{n} P (X=x_i) \ln \big( \Prob(X=x_i) \big)$.

  \item Soient $x_1 , x_2 , x_3 , x_4$ quatre réels distincts. On
    considère un jeu de 32 cartes dont on tire une carte au
    hasard. Soit $X$ la variable aléatoire prenant les valeurs
    suivantes : 
    \begin{noliste}{$\stimes$}
    \item $x_1$ si la carte tirée est rouge (coeur ou carreau),
    \item $x_2$ si la carte tirée est un pique,
    \item $x_3$ si la carte tirée est le valet, la dame, le roi ou
      l'as de trèfle,
    \item $x_4$ dans les autres cas.
    \end{noliste}

    On tire une carte notée $C$ et un enfant décide de déterminer la
    valeur $X(C)$ en posant dans l'ordre les questions suivantes
    auxquelles il lui est répondu par "oui" ou par "non". LA carte $C$
    est-elle rouge? La carte $C$ est-elle un pique ? La carte $C$
    est-elle le valet, la dame, le roi ou l'as de trèfle ? \\
    Soit $N$ la variable aléatoire égale au nombre de questions posées
    (l'enfant cesse de poser des questions dès qu'il a obtenu une
    réponse "oui"). 

    \begin{noliste}{a)}
    \setlength{\itemsep}{2mm}
    \item Calculer l'entropie $H(X)$ de $X$.
    \item Déterminer la loi et l'espérance $E (N)$ de $N$. Comparer
      $\E(N)$ et $H(X)$.
    \end{noliste}

  \item Soit $f$ la fonction définie sur $\R^2$ à valeurs réelles
    telle que : $f(x,y) = x \ln x + y \ln y + (1-x-y) \ln
    (1-x-y)$. 
    \begin{noliste}{a)}
    \setlength{\itemsep}{2mm}
    \item Préciser le domaine de définition de $f$. Dessiner ce
      domaine dans le plan rapporté à un repère orthonormé.
    \item Montrer que $f$ ne possède qu'un seul point critique et
      qu'en ce point, $f$ admet un extremum local.
    \item Soit $X$ une variable aléatoire réelle prenant les valeurs
      $x_1$, $x_2$ et $x_3$ avec les probabilités non nulles $p_1$,
      $p_2$ et $p_3$ respectivement.\\

      Calculer $H(X)$ et montrer que $H(X)$ est maximale lorsque
      $p_1=p_2=p_3 = \frac{1}{3}$.
    \end{noliste}   
  \end{noliste}
\end{exerciceAP} 


\subsection*{Exercices oraux HEC 2014}

%%% EPR %%% HEC;
% type : oralAP; %
% sujet : E 26; %
% annee : 2014; %
% theme : vaCouple, vaDiscrete, vaFctGeneratrice, vaCovariance,
% denombrement, fct2var; %

\begin{exerciceAP}~
  \begin{noliste}{1.}
    \setlength{\itemsep}{2mm}
  \item Question de cours : Définition de l'indépendance de deux
    variables aléatoires discrètes. Lien entre indépendance et
    covariance.\\
    Soit $X$ et $Y$ deux variables aléatoires discrètes finies à
    valeurs dans $\N$, définies sur un espace probabilisé
    $(\Omega,\A,P)$. On suppose que $X(\Omega)\subset \llb0,n\rrb$ et
    $Y(\Omega) \subset \llb 0, m \rrb$, où $n$ et $m$ sont deux
    entiers de $\N^*$.\\
    Pour tout couple $(i,j) \in \llb 0, n \rrb \times \llb 0, m \rrb$,
    on pose : $p_{i,j} = \Prob(\Ev{X=i} \cap \Ev{Y=j})$.\\
    Soit $F_X$ et $F_Y$ les deux fonctions de $\R$ dans $\R$ définies
    par : $F_X(x)=\Sum{i=0}{n} \Prob(\Ev{X=i})x^i$ et \\
    $F_Y(x)=\Sum{j=0}{m} \Prob(\Ev{Y=j})x^j$.\\
    Soit $Z=(X,Y)$ et $G_Z$ la fonction de $\R^2$ dans $\R$ définie
    par : $G_Z(x,y)=\Sum{i=0}{n}\Sum{j=0}{m} p_{i,j}x^iy^j$.

  \item Donner la valeur de $G_Z(1,1)$ et exprimer les espérances de
    $X$, $Y$ et $XY$, puis la covariance de $(X,Y)$ à l'aide des
    dérivées partielles premières et secondes de $G_Z$ au point
    $(1,1)$.

  \item Soit $f$ une fonction polynomiale de deux variables définies
    sur $\R^2$ par : $f(x,y) = \Sum{i=0}{n} \Sum{j=0}{m}
    a_{i,j}x^iy^j$ avec $a_{i,j} \in \R$.\\
    On suppose que pour tout couple $(x,y)\in\R^2$, on a $f(x,y)=0$.
    \begin{noliste}{a)}
    \setlength{\itemsep}{2mm}
  \item Montrer que pour tout $(i,j) \in \llb0, n \rrb \times \llb 0,
    m \rrb$, on a $a_{i,j}=0$.
    \item En déduire que $X$ et $Y$ sont indépendantes, si et
      seulement si, pour tout $(x,y)\in\R^2$,\\
      $G_Z(x,y)=F_X(x)F_Y(y)$. (on pourra poser : $a_{i,j}=p_{i,j} -
      \Prob(\Ev{X=i})\Prob(\Ev{Y=j})$).
    \end{noliste}

  \item Une urne contient des jetons portant chacun une des lettres
    $A$, $B$ ou $C$. La proportion des jetons portant la lettre $A$
    est $p$, celle des jetons portant la lettre $B$ est $q$ et celle
    des jetons portant la lettre $C$ est $r$, où $p$, $q$ et $r$ sont
    trois réels strictement positifs vérifiant $p+q+r=1$.\\
    Soit $n\in\N^*$. On effectue $n$ tirages d'un jeton avec remise
    dans cette urne. On note $X$ (resp. $Y$) la variable aléatoire
    égale au nombre de jetons tirés portant la lettre $A$ (resp. $B$)
    à l'issue de ces $n$ tirages.
    \begin{noliste}{a)}
    \setlength{\itemsep}{2mm}
    \item Quelles sont les lois de $X$ et $Y$ respectivement ?
      Déterminer $F_X$ et $F_Y$.
    \item Déterminer la loi de $Z$. En déduire $G_Z$.
    \item Les variables aléatoires $X$ et $Y$ sont-elles indépendantes
      ?
    \item Calculer la covariance de $(X,Y)$. Le signe de cette
      covariance était-il prévisible ?
    \end{noliste}
  \end{noliste}
\end{exerciceAP} 


%%% EPR %%% HEC;
% type : oralAP; %
% sujet : E 39; %
% annee : 2014; %
% theme : vaDiscrete, vaPoisson, vaCouple, vaCovariance, vaSomme, FPT; %

\begin{exerciceAP}~
  \begin{noliste}{1.}
    \setlength{\itemsep}{2mm}
  \item Question de cours : \\
    Loi d'un couple de variables aléatoires discrètes. Lois
    marginales. Lois conditionnelles.
  \end{noliste}
  Soit $c$ un réel strictement positif et soit $X$ et $Y$ deux
  variables aléatoires à valeurs dans $\N$ définies sur un espace
  probabilisé $(\Omega, \A, \Prob)$, telles que :
  \[
  \forall (i,j)\in\N^2, \ \Prob(\Ev{X=i}\cap\Ev{Y=j}) =
  c\dfrac{i+j}{i!j!}
  \]
  \begin{noliste}{1.}
    \setcounter{enumi}{1}
  \item 
    \begin{noliste}{a)}
    \setlength{\itemsep}{2mm}
  \item Montrer que pour tout $i\in\N$, on a : $\Prob(\Ev{X=i}) = c
    \dfrac{(i+1)}{i!} \ \ee$. \\
    En déduire la valeur de $c$.
    \item Montrer que $X$ admet une espérance et une variance et les
      calculer.
    \item Les variables aléatoires $X$ et $Y$ sont-elles indépendantes
      ?
    \end{noliste}
  \item
    \begin{noliste}{a)}
    \setlength{\itemsep}{2mm}
    \item Déterminer la loi de $X+Y-1$.
    \item En déduire la variance de $X+Y$.
    \item Calculer la covariance de $X$ et de $X+5Y$.\\
      Les variables aléatoires $X$ et $X+5Y$ sont-elles indépendantes
      ?
    \end{noliste}

  \item On pose : $Z=\dfrac{1}{X+1}$.
    \begin{noliste}{a)}
    \setlength{\itemsep}{2mm}
    \item Montrer que $Z$ admet une espérance et la calculer.
    \item Déterminer pour $i\in\N$, la loi conditionnelle de $Y$
      sachant $\Ev{X=i}$.
    \item Pour $A\in\A$, on pose : $g_A(Y)=\Sum{k=0}{+\infty} k
      P_A(\Ev{Y=k})$.\\
      Établir l'existence d'une fonction affine $f$ telle que, pour
      tout $\omega\in\Omega$, on a : 
      \[
      g_{\Ev{X=X(\omega)}}(Y) = f(Z(\omega))
      \]
    \end{noliste}
  \end{noliste}
\end{exerciceAP} 


\subsection*{Exercices oraux HEC 2015}

%%% EPR %%% HEC;
% type : oralSP; % 
% sujet : E 70; %
% annee : 2015; % 
% theme : vaBernoulli, vaIndep, vaCovariance, vaDiscrete, ; % 

\begin{exerciceSP}~\\
  Soit $(X_n)_{n\in\N^*}$ une suite de variables aléatoires
  indépendantes définies sur un espace probabilisé $(\Omega,\A,P)$, de
  même loi de Bernoulli de paramètre $\dfrac{1}{2}$. On pose pour tout
  $n\in\N^*$ : $W_n = \Sum{k=1}{n} k X_k$.
  \begin{noliste}{1.}
    \setlength{\itemsep}{2mm}
  \item Calculer $\E(W_n)$ et $\V(W_n)$.
  \item Les variables $W_n$ et $W_{n+1}$ sont-elles indépendantes ?
  \end{noliste}
\end{exerciceSP} 


%%% EPR %%% HEC;
% type : oralSP; % 
% sujet : E 71; %
% annee : 2015; % 
% theme : vaBernoulli, vaDiscrete, vaCovariance, inegCauchySchwarz; % 

\begin{exerciceSP}~\\
  Soit $X_1,X_2,\hdots,X_n$ $n$ variables aléatoires telles que pour
  tout $k\in\llb 1,n\rrb$, $X_k$ suit la loi de Bernoulli de paramètre
  $p_k$ avec $0 < p_k < 1$.\\
  On pose : $Y = \Sum{k=1}{n} X_k$. Montrer que $\V(Y)\leq
  \dfrac{n^2}{4}$.
\end{exerciceSP} 


%%% EPR %%% HEC;
% type : oralAP; % 
% sujet : E 73; %
% annee : 2015; % 
% theme : vaDiscrete, FPT; % 

\begin{exerciceAP}~
  \begin{noliste}{1.}
    \setlength{\itemsep}{2mm}
  \item Question de cours : Formule des probabilités totales.\\
    Soit $n$ un entier naturel supérieur ou égal à $2$.\\
    On considère $n$ urnes numérotées de $1$ à $n$ et $N$ un entier
    naturel multiple de $2^n$.\\
    Pour tout $k\in\llb 1,n\rrb$, la $k$-ième urne contient $N$ boules
    dont $\dfrac{N}{2^k}$ boules blanches, les autres étant noires.\\
    On tire dans l'urne $1$ une boule au l'on place dans l'urne $2$,
    puis on tire dans l'urne $2$ une boule que l'on place dans l'urne
    $3$ et ainsi de suite jusqu'à tirer dans l'urne $n-1$ une boule
    que l'on place dans l'urne $n$, puis on tire une boule dans l'urne
    $n$.\\
    L'expérience est modélisée par un espace probabilisé
    $(\Omega,\A,P)$.
  \item Pour tout $k\in\llb 1,n\rrb$, soit $p_k$ la probabilité que la
    boule tirée dans l'urne $k$ soit blanche.\\
    Trouver une relation de récurrence entre $p_{k+1}$ et $p_k$
    ($1\leq k\leq n-1$).
  \item 
    \begin{noliste}{a)}
    \setlength{\itemsep}{2mm}
    \item Calculer $p_n$ en fonction de $n$ et $N$.
    \item Pour $n$ fixé, calculer $\dlim{N\to+\infty}
      p_n$. Interpréter cette limite.
    \end{noliste}
  \item Soit $i\in\llb 1,n-1\rrb$. Calculer la probabilité
    conditionnelle que la $n$-ième boule tirée soit blanche sachant
    que la boule tirée dans l'urne $i$ est blanche.
  \end{noliste}
\end{exerciceAP} 


%%% EPR %%% HEC;
% type : oralAP; % 
% sujet : E 79; %
% annee : 2015; % 
% theme : vaFDR, vaDensite, vaDiscrete, vaPartieEntiere, ; % 

\begin{exerciceAP}~\\
  {\it Toutes les variables aléatoires de cet exercice sont définies
    sur un espace probabilisé $(\Omega,\A,P)$}.
  \begin{noliste}{1.}
    \setlength{\itemsep}{2mm}
  \item Question de cours : Définition et propriétés de la fonction de
    répartition d'une variable aléatoire à densité.\\
    Soit $a$ un paramètre réel et $F$ la fonction définie sur $\R$, à
    valeurs réelles, telle que :
    \[
    F(x)=\left\{
      \begin{array}{cl}
        1+\ln\left(\dfrac{x}{x+1}\right) & \mbox{ si $x\geq a$}\\
        0 & \mbox{ si $x<a$}
      \end{array}
    \right.
    \]
  \item 
    \begin{noliste}{a)}
    \setlength{\itemsep}{2mm}
    \item Montrer que $F$ est continue sur $\R$ si et seulement si
      $a=\dfrac{1}{\ee -1}$.

    \item Étudier les variations de $F$ et tracer l'allure de sa
      courbe représentative dans un repère orthogonal du plan.
    \end{noliste}

  \item
    \begin{noliste}{a)}
    \setlength{\itemsep}{2mm}
    \item Montrer que $F$ est la fonction de répartition d'une
      variable aléatoire $X$ à densité.
    \item La variable aléatoire $X$ admet-elle une espérance ?
    \end{noliste}

  \item Soit $Y$ la variable aléatoire à valeurs dans $\N$ définie par
    $Y=\lfloor X\rfloor$ (partie entière de $X$). On pose : $Z=X-Y$.
    \begin{noliste}{a)}
    \setlength{\itemsep}{2mm}
  \item Calculer $\Prob(Y=0)$ et montrer que pour tout $n\in\N^*$, on
    a : $\Prob(Y=n) = \ln\left(1 + \dfrac{1}{n(n + 2)}\right)$.
    \item Déterminer la fonction de répartition et une densité de $Z$.
    \item Établir l'existence de l'espérance $\E(Z)$ de $Z$. Calculer
      $\E(Z)$.
    \end{noliste}
  \end{noliste}
\end{exerciceAP} 


\subsection*{Exercices oraux HEC 2016}

%%% EPR %%% HEC;
% type : oralAP; % 
% sujet : E 82; %
% annee : 2016; % 
% theme : vaDensite, vaUniformeDensite, partieEntiere, vaDiscrete, 
% vaPartieEntiere; % 

\begin{exerciceAP}~\\
On suppose que toutes les variables aléatoires qui interviennent dans 
l'exercice sont définies sur un même espace probabilisé $(\Omega, 
\A,\Prob)$.
\begin{noliste}{1.}
    \setlength{\itemsep}{2mm}
  \item Question de cours : Loi uniforme sur un intervalle $[a,b]$;
    définition, propriétés.
  
  \item Pour tout $x$ réel, on note $\lfloor x \rfloor$ la partie 
  entière de $x$.
  \begin{noliste}{a)}
    \setlength{\itemsep}{2mm}
    \item Pour $n$ entier de $\N^*$, montrer que pour tout $x$ réel, 
    on a : $\dlim{n\to+\infty} \dfrac{\lfloor nx\rfloor}{n}=x$.
    
    \item Établir pour tout $(x,y)\in\R^2$ l'équivalence suivante 
    : $\lfloor y \rfloor \leq x \ \Leftrightarrow \ y < \lfloor x 
    \rfloor +1$.
    
    \item Soit $\alpha$ et $\beta$ deux réels vérifiant $0\leq 
    \alpha\leq \beta\leq 1$ et soit $N_n(\alpha,\beta)$ le nombre 
    d'entiers $k$ qui vérifient $\alpha <\dfrac{k}{n}\leq \beta$. 
    Exprimer $N_n(\alpha,\beta)$ en fonction de $\lfloor n\alpha 
    \rfloor$ et $\lfloor n\beta \rfloor$.
  \end{noliste}
  
  \item Pour tout entier $n\geq 1$, on note $Y_n$ la variable aléatoire 
  discrète dont la loi est donnée par :
  \[
    \forall k\in\llb 0,n-1\rrb, \ 
    \Prob\left(\Ev{Y_n=\frac{k}{n}}\right)=\frac{1}{n}.
  \]
  Soit $Z$ une variable aléatoire suivant la loi uniforme sur 
  l'intervalle $[0,1]$. Pour tout entier $n\geq 1$, on définit la 
  variable aléatoire $Z_n$ par : $Z_n=\dfrac{\lfloor nZ\rfloor}{n}$. 
  Soit $\alpha$ et $\beta$ deux réels vérifiant $0\leq \alpha\leq 
  \beta\leq 1$.
  \begin{noliste}{a)}
    \setlength{\itemsep}{2mm}
    \item Montrer que $\dlim{n\to+\infty} \Prob(\Ev{\alpha <Y_n\leq 
    \beta})=\beta-\alpha$.
    
    \item Comparer les fonctions de répartition respectives de $Y_n$ 
    et $Z_n$. Conclusion.
  \end{noliste}
\end{noliste}
\end{exerciceAP} 


%%% EPR %%% HEC;
% type : oralSP; % 
% sujet : E 83; %
% annee : 2016; % 
% theme : vaDensite, vaUniformeDensite, vaDiscrete, vaUniformeDiscrete,
% vaConvergence; % 

\begin{exerciceSP}~\\
Les variables aléatoires de cet exercice sont supposées définies sur un 
espace probabilisé $(\Omega,\A,\Prob)$.\\
Soit $Z$ une variable aléatoire qui suit la loi uniforme sur 
l'intervalle $[0,1]$ et pour tout entier $n\geq 1$, on note $Y_n$ une 
variable aléatoire à valeurs dans 
$\left\{0,\dfrac{1}{n},\dfrac{2}{n},\hdots, \dfrac{n-1}{n}\right\}$ 
telle que : 
\[
  \forall k\in\llb 0,n-1\rrb, \ 
  \Prob\left(\Ev{Y_n=\dfrac{k}{n}}\right)=\dfrac{1}{n}
\]
Soit $f$ une fonction définie et continue sur $[0,1]$. Montrer que 
$\dlim{n\to+\infty} \E(f(Y_n))=\E(f(Z))$.
\end{exerciceSP} 


%%% EPR %%% HEC;
% type : oralAP; % 
% sujet : E 85; %
% annee : 2016; % 
% theme : vaCouple, vaCovariance, vaDiscrete, vaMoments; % 

\begin{exerciceAP}~\\
Toutes les variables aléatoires qui interviennent dans l'exercice sont 
supposées définies sur le même espace probabilisé 
$(\Omega,\A,\Prob)$.
\begin{noliste}{1.}
    \setlength{\itemsep}{2mm}
  \item Question de cours : Définition et propriétés de la covariance 
  de deux variables aléatoires discrètes.\\
  Soit $p$, $q$ et $r$ des réels fixés de l'intervalle $]0,1[$ tels que 
  $p+q+r=1$. \\
  Soit $(X_n)_{n\geq 1}$ une suite de variables aléatoires à 
  valeurs dans $\{-1,0,1\}$, indépendantes et de même loi donnée par :
  \[
    \forall n \in\N^*, \ \Prob(\Ev{X_n=1})=p, \ \Prob(\Ev{X_n=-1})=q, \ 
    \Prob(\Ev{X_n=0})=r.
  \]
  On pose pour tout entier $n\geq 1$ : $Y_n=\Prod{k=1}{n} X_k$.
  
  \item 
  \begin{noliste}{a)}
    \setlength{\itemsep}{2mm}
    \item Pour tout entier $n\geq 1$, préciser $Y_n(\Omega)$ et 
    calculer $\Prob(\Ev{Y_n=0})$.
    
    \item Pour tout entier $n\geq 1$, calculer $\E(X_n)$ et $\E(Y_n)$.
  \end{noliste}
  
  \item On pose pour tout entier $n\geq 1$, on a : 
  $p_n=\Prob(\Ev{Y_n=1})$.
  \begin{noliste}{a)}
    \setlength{\itemsep}{2mm}
    \item Calculer $p_1$ et $p_2$.
    
    \item Établir une relation de récurrence entre $p_{n+1}$ et 
    $p_n$.
    
    \item En déduire que pour tout entier $n\geq 1$, on a : 
    $p_n=\dfrac{(p+q)^n +(p-q)^n}{2}$.
    
    \item Pouvait-on à l'aide de la question $2$, trouver 
    directement la loi de $Y_n$ ?
  \end{noliste}
  
  \item 
  \begin{noliste}{a)}
    \setlength{\itemsep}{2mm}
    \item Établir l'inégalité : $(p+q)^n > (p-q)^{2n}$. Calculer 
    $\V(Y_n)$.
    
    \item Calculer la covariance $\cov(Y_n,Y_{n+1})$ des deux 
    variables aléatoires $Y_n$ et $Y_{n+1}$.
  \end{noliste}
\end{noliste}
\end{exerciceAP} 


%%% EPR %%% HEC;
% type : oralAP; % 
% sujet : E 88; %
% annee : 2016; % 
% theme : vaConvergence, vaPartieEntiere, vaDensite, vaExponentielle,
% vaDiscrete, vaGeometrique, vaUniformeDensite; % 

\begin{exerciceAP}~\\
Toutes les variables aléatoires qui interviennent dans l'exercice sont 
supposées définies sur le même espace probabilisé $(\Omega, 
\A,\Prob)$.
\begin{noliste}{1.}
    \setlength{\itemsep}{2mm}
  \item Question de cours : Convergence en loi d'une suite de variables 
  aléatoires.\\
  Dans tout l'exercice, $X$ désigne une variable aléatoire suivant la 
  loi exponentielle de paramètre $\lambda>0$.
  
  \item 
  \begin{noliste}{a)}
    \setlength{\itemsep}{2mm}
    \item On pose : $T=\lfloor X\rfloor$ (partie entière de $X$). 
    Montrer que la loi de $T$ est donnée par :
    \[
      \forall k \in\N, \ 
      \Prob(\Ev{T=k}) = \Big(1 - \ee^{-\lambda} \Big) \, 
      \Big(\ee^{-\lambda}\Big)^k
    \]
    
    \item Quelle est la loi de $T+1$ ? En déduire l'espérance et la 
    variance de $T$.
  \end{noliste}
  
  \item On pose : $Z=X-\lfloor X\rfloor$.\\
  Montrer que $Z$ est une variable aléatoire à densité et déterminer 
  une densité de $Z$.

  \item Soit $(X_n)_{n\geq 1}$ une suite de variables aléatoires 
  indépendantes telles que, pour tout $n\in\N^*$, $X_n$ suit une loi 
  exponentielle de paramètre $\dfrac{\lambda}{n}$. On pose pour tout 
  $n\in\N^*$ : $Z_n=X_n-\lfloor X_n\rfloor$.\\
  Montrer que la suite de variables aléatoires $(Z_n)_{n\geq 1}$ 
  converge en loi vers une variable aléatoire dont on précisera la loi.
\end{noliste}
\end{exerciceAP} 


\subsection*{Exercices oraux HEC 2017}

%%% EPR %%% HEC;
% type : oralSP; %
% sujet : E 42; %
% annee : 2017; %
% theme : sciSimuVaDiscrete, sciLfGN, vaDiscrete; %

\begin{exerciceSP}~\\
  On considère une urne contenant $b$ boules blanches et $r$ 
  boules rouges.
  \begin{noliste}{1.}
    \setlength{\itemsep}{2mm}
    \item La fonction \Scilab{} suivante permet de simuler des 
    tirages dans cette urne.
    \begin{scilab}
      & \tcFun{function} \tcVar{y} = X(\tcVar{b}, \tcVar{r}) \nl %
      & \qquad V = \%F \commentaire{le booléen "faux"} \nl %
      & \qquad \tcFor{for} k = 1:3 \nl %
      & \qquad \qquad V = V|grand(1, 1, \ttq{}uin\ttq{}, 1, 
      \tcVar{b} + \tcVar{r}) <= \tcVar{b}; \nl %
      & \qquad \tcFor{end}; \nl %
      & \qquad \tcIf{if} V \tcIf{then} \tcVar{y} = 1; \tcIf{else} 
      \tcVar{y} = 2; \nl %
      & \qquad \tcIf{end}; \nl %
      & \tcFun{endfunction}
    \end{scilab}
    
    Que retourne la fonction $X$ et quelle loi simule-t-elle ?
    
    \item De quelle valeur théorique la valeur affichée après 
    l'exécution des instructions suivantes fournit-elle une 
    approximation ?
    \begin{scilab}
      & R = []; \nl %
      & \tcFor{for} k = 1:10000 \nl %
      & \qquad R = [R, X(5,5)]; \nl %
      & \tcFor{end}; \nl %
      & disp(mean(R))
    \end{scilab}
  \end{noliste}
\end{exerciceSP} 


%%% EPR %%% HEC;
% type : oralAP;
% sujet : E 81;
% annee : 2017;
% theme : vaDiscrete, vaGeometrique, sciSimuVaDiscrete, denombrement,
% vaCouple; %

\begin{exerciceAP}~
  \begin{noliste}{1.}
    \setlength{\itemsep}{2mm}
    \item Question de cours
    \begin{noliste}{a)}
    \setlength{\itemsep}{2mm}
      \item Définition et propriétés de la loi géométrique.
      
      \item Compléter la ligne de code \Scilab{} contenant des points
      d'interrogation pour que la fonction {\tt geo} suivante 
      fournisse une simulation de la loi géométrique dont le 
      paramètre est égal à l'argument $p$ de la fonction.
      \begin{scilab}
        & \tcFun{function} \tcVar{x} = geo(\tcVar{p}) \nl %
        & \qquad \tcVar{x} = 1; \nl %
        & \qquad \tcFor{while} rand() ??? \nl %
        & \qquad \qquad \tcVar{x} = \tcVar{x} + 1; \nl %
        & \qquad \tcFor{end}; \nl %
        & \tcFun{endfunction}
      \end{scilab}~
    \end{noliste}
   \end{noliste}
    Une urne contient trois jetons numérotés $1$, $2$ et $3$. On 
    effectue dans cette urne, une suite de tirages d'un jeton avec 
    remise.
    \begin{noliste}{1.}
    \setlength{\itemsep}{2mm}
      \setcounter{enumi}{1}
      \item On note $Y$ la variable aléatoire égale au nombre de 
      tirages nécessaires pour obtenir, pour la première fois, deux
      numéros successifs distincts.
      \begin{noliste}{a)}
    \setlength{\itemsep}{2mm}
	\item Reconnaître la loi de la variable aléatoire $Y-1$.
	
	\item Déterminer l'espérance $\E(Y)$ et la variance $\V(Y)$
	de la variable aléatoire $Y$.
      \end{noliste}
      
      \item On note $Z$ la variable aléatoire égale au nombre de 
      tirages nécessaires pour obtenir, pour la première fois, les 
      trois numéros.
      \begin{noliste}{a)}
    \setlength{\itemsep}{2mm}
	\item Soit deux entiers $k \geq 2$ et $\ell \geq 3$.\\
	Calculer $\Prob(\Ev{Y=k} \cap \Ev{Z=\ell})$ selon les 
	valeurs de $k$ et $\ell$.
	
	\item En déduire que, pour tout entier $\ell \geq 3$, on a :
	$\Prob(\Ev{Z = \ell}) = \dfrac{2}{3} \, \left( 
	\dfrac{2^{\ell-2} -1}{3^{\ell -2}}\right)$.
	
	\item Calculer $\E(Z)$.
      \end{noliste}
      
      \item D'une manière plus générale, calculer l'espérance de la 
      variable aléatoire égale au nombre de tirages nécessaires 
      pour obtenir, pour la première fois, tous les numéros, dans 
      l'hypothèse où l'urne contient au départ $n$ jetons, numérotés
      de $1$ à $n$.
    \end{noliste}
\end{exerciceAP} 


%%% EPR %%% HEC;
% type : oralAP; %
% sujet : E 91; %
% annee : 2017; %
% theme : vaDiscrete, vaUniformeDiscrete, vaMax, estimBiais, 
% estimation; %

\begin{exerciceAP}~
  \begin{noliste}{1.}
    \setlength{\itemsep}{2mm}
    \item Question de cours : définition d'un estimateur sans biais
    d'un paramètre inconnu.
  \end{noliste}
  
  \noindent
  Soit $N$ un entier supérieur ou égal à $2$.\\
  Soit $(X_n)_{n\geq 1}$ une suite de variables aléatoires 
  indépendantes, définies sur le même espace probabilisé 
  $(\Omega, \A, \Prob)$, et suivant chacune la loi uniforme discrète
  sur $\llb 1,N \rrb$.
  \begin{noliste}{1.}
    \setlength{\itemsep}{2mm}
    \setcounter{enumi}{1}
    \item Pour tout entier $n\geq 1$, on pose : $s_n(N) = 
    \Sum{k=1}{N-1} \left( \dfrac{k}{N} \right)^n$.
    \begin{noliste}{a)}
    \setlength{\itemsep}{2mm}
      \item Montrer que la suite $(s_n(N))_{n\geq 1}$ est 
      strictement monotone et convergente.
      
      \item Trouver sa limite.
    \end{noliste}
    
    \item Pour tout entier $n\geq 1$, on pose : $T_n = \max(X_1,
    X_2, \ldots, X_n)$.
    \begin{noliste}{a)}
    \setlength{\itemsep}{2mm}
      \item Calculer pour tout $k\in \llb 1,N \rrb$, $\Prob(
      \Ev{T_n=k})$.
      
      \item Montrer que $\E(T_n) = N - s_n(N)$.
    \end{noliste}
    
    \item 
    \begin{noliste}{a)}
    \setlength{\itemsep}{2mm}
      \item Justifier que $\Prob(\Ev{ \vert T_N - N \vert \geq 1})$
      tend vers $0$ quand $n$ tend vers l'infini.
      
      \item En déduire, pour tout $\eps>0$, la limite de 
      $\Prob(\Ev{\vert T_n - N \vert \geq \eps})$ quand $n$ tend 
      vers l'infini.
    \end{noliste}
    
    \item On suppose dans cette question que $N$ est un paramètre 
    inconnu.
    \begin{noliste}{a)}
    \setlength{\itemsep}{2mm}
      \item Expliquer pourquoi on ne peut pas dire que $T_n + s_n(N)$
      est un estimateur sans biais de $N$.
      
      \item Trouver une suite convergente et asymptotiquement sans 
      biais d'estimateurs de $N$.
    \end{noliste}
  \end{noliste}
\end{exerciceAP} 


%%% EPR %%% HEC;
% type : oralAP; %
% sujet : E 106; %
% annee : 2017; %
% theme : vaMoments, vaDiscrete, partieEntiere; %

\begin{exerciceAP}~\\
  Dans tout l'exercice, $(\Omega, \A, \Prob)$ désigne un espace 
  probabilisé, et toutes les variables considérées sont définies sur 
  cet espace probabilisé et $X$ désigne une variable aléatoire 
  \underline{à valeurs dans $\N$}.
  \begin{noliste}{1.}
    \setlength{\itemsep}{2mm}
    \item Question de cours : rappeler la formule de Koenig-Huygens.
    
    \item Démontrer que, si $X$ admet un moment d'ordre deux, alors 
    on a :
    \[
      \forall c \in \R, \ \V(X) \leq \E((X-c)^2)
    \]
  \end{noliste}
  
  \item Dans cette question, $n$ est un entier strictement positif et 
  $X$ une variable aléatore telle que :
  \[
    X(\Omega) \subset \llb 0, 2n \rrb
  \]
  \begin{noliste}{a)}
    \setlength{\itemsep}{2mm}
    \item En utilisant une des inégalités prouvées en \itbf{1.b)}, 
    démontrer que la variance de $X$ est inférieure ou égale à $n^2$.
    
    \item Démontrer que, si $\E(X) = n$, alors la variance de $X$ est 
    égale à $n^2$ si, et seulement si, $\Prob(\Ev{X=0}) = 
    \Prob(\Ev{X=2n}) = \dfrac{1}{2}$.
    
    \item Quelle est la plus petite valeur possible de $\V(X)$ 
    lorsque $\E(X)=n$ ?
  \end{noliste}
  
  \noindent
  Dans toute la suite de l'exercice, $c$ désigne un nombre réel positif
  qui n'est \underline{pas entier} et $\lfloor c \rfloor$ sa partie 
  entière.
  \begin{noliste}{1.}
    \setlength{\itemsep}{2mm}
    \setcounter{enumi}{2}
    \item Soit $X_0$ une variable aléatoire vérifiant : $\left\{
    \begin{array}{l}
      X_0(\Omega) = \{ \lfloor c \rfloor, \ \lfloor c \rfloor 
      +1\}\\[.1cm]
      \E(X_0) = c
    \end{array}
    \right.$
    \begin{noliste}{a)}
    \setlength{\itemsep}{2mm}
      \item Vérifier que : $\Prob(\Ev{X_0 = \lfloor c \rfloor }) =
      \lfloor c \rfloor +1-c$.
      
      \item En déduire que la variance de $X_0$ est égale à 
      $(c- \lfloor c \rfloor )(\lfloor c \rfloor +1-c)$.
    \end{noliste}
    
    \item Dans cette question et la suivante, $X$ est une variable 
    aléatoire à valeurs dans $\N$ qui admet une espérance et une 
    variance, et vérifie : $\E(X)=c$.\\
    On note $A= \Ev{X \leq c}$ et $p=\Prob(A)$.
    \begin{noliste}{a)}
    \setlength{\itemsep}{2mm}
      \item Justifier que $p$ est strictement compris $0$ et $1$.
      
      \item Justifier la ocnvergence de la série $\Sum{k\geq 0}{} 
      k \, \Prob_{\bar{A}}(\Ev{X=k})$, où $\bar{A}$ désigne le 
      complémentaire de l'événement $A$.
    \end{noliste}
    
    \item On note : $c_0 = \Sum{k=0}{\lfloor c \rfloor} 
    k \, \Prob_A(\Ev{X=k})$ et $c_1 = \Sum{k= \lfloor c \rfloor +1}
    {+\infty} k \, \Prob_{\bar{A}}(\Ev{X=k})$.\\[.1cm]
    Soit $Y$ une variable aléatoire telle que $\Prob(\Ev{Y=c_0})
    =p$ et $\Prob(\Ev{Y=c_1})=1-p$.
    \begin{noliste}{a)}
    \setlength{\itemsep}{2mm}
      \item Vérifier que les variables aléatoires $X$ et $Y$ ont la même
      espérance.
      
      \item Prouver l'égalité : $\V(Y) = (c-c_0)(c_1-c)$.
      
      \item Démonter l'inégalité :
      \[
        \V(X) \geq \V(Y)
      \]
      
      \item En déduire que $\V(X_0)$ est la plus petite valeur possible 
      de $\V(X))$.
    \end{noliste}
  \end{noliste}
\end{exerciceAP} 


