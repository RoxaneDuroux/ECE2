\documentclass[11pt]{article}%
\usepackage{geometry}%
\geometry{a4paper,
 lmargin = 2cm,rmargin = 2cm,tmargin = 2.5cm,bmargin = 2.5cm}

\usepackage{array}
\usepackage{paralist}

\usepackage[svgnames, usenames, dvipsnames]{xcolor}
\xdefinecolor{RecColor}{named}{Aqua}
\xdefinecolor{IncColor}{named}{Aqua}
\xdefinecolor{ImpColor}{named}{PaleGreen}

% \usepackage{frcursive}

\usepackage{adjustbox}

%%%%%%%%%%%
\newcommand{\cRB}[1]{{\color{Red} \pmb{#1}}} %
\newcommand{\cR}[1]{{\color{Red} {#1}}} %
\newcommand{\cBB}[1]{{\color{Blue} \pmb{#1}}}
\newcommand{\cB}[1]{{\color{Blue} {#1}}}
\newcommand{\cGB}[1]{{\color{LimeGreen} \pmb{#1}}}
\newcommand{\cG}[1]{{\color{LimeGreen} {#1}}}

%%%%%%%%%%

\usepackage{diagbox} %
\usepackage{colortbl} %
\usepackage{multirow} %
\usepackage{pgf} %
\usepackage{environ} %
\usepackage{fancybox} %
\usepackage{textcomp} %
\usepackage{marvosym} %

%%%%%%%%%% pour pouvoir faire des dashedline dans les tableaux
\usepackage{arydshln}

%%%%%%%%%% pour qu'une cellcolor ne recouvre pas le trait du tableau
\usepackage{hhline}%

\usepackage{pgfplots}
\pgfplotsset{compat=1.10}
\usepgfplotslibrary{patchplots}
\usepgfplotslibrary{fillbetween}
\usepackage{tikz,tkz-tab}
\usepackage{ifthen}
\usepackage{calc}
\usetikzlibrary{calc,decorations.pathreplacing,arrows,positioning} 
\usetikzlibrary{fit,shapes,backgrounds}
% \usepackage[nomessages]{fp}% http://ctan.org/pkg/fp

\usetikzlibrary{matrix,arrows,decorations.pathmorphing,
  decorations.pathreplacing} 

\newcommand{\myunit}{1 cm}
\tikzset{
    node style sp/.style={draw,circle,minimum size=\myunit},
    node style ge/.style={circle,minimum size=\myunit},
    arrow style mul/.style={draw,sloped,midway,fill=white},
    arrow style plus/.style={midway,sloped,fill=white},
}

%%%%%%%%%%%%%%
%%%%% écrire des inférieur égal ou supérieur égal avec typographie
%%%%% francaise
%%%%%%%%%%%%%

\renewcommand{\geq}{\geqslant}
\renewcommand{\leq}{\leqslant}
\renewcommand{\emptyset}{\varnothing}

\newcommand{\Leq}{\leqslant}
\newcommand{\Geq}{\geqslant}

%%%%%%%%%%%%%%
%%%%% Macro Celia
%%%%%%%%%%%%%

\newcommand{\ff}[2]{\left[#1, #2\right]} %
\newcommand{\fo}[2]{\left[#1, #2\right[} %
\newcommand{\of}[2]{\left]#1, #2\right]} %
\newcommand{\soo}[2]{\left]#1, #2\right[} %
\newcommand{\abs}[1]{\left|#1\right|} %
\newcommand{\Ent}[1]{\left\lfloor #1 \right\rfloor} %


%%%%%%%%%%%%%%
%%%%% tikz : comment dessiner un "oeil"
%%%%%%%%%%%%%

\newcommand{\eye}[4]% size, x, y, rotation
{ \draw[rotate around={#4:(#2,#3)}] (#2,#3) -- ++(-.5*55:#1) (#2,#3)
  -- ++(.5*55:#1); \draw (#2,#3) ++(#4+55:.75*#1) arc
  (#4+55:#4-55:.75*#1);
  % IRIS
  \draw[fill=gray] (#2,#3) ++(#4+55/3:.75*#1) arc
  (#4+180-55:#4+180+55:.28*#1);
  % PUPIL, a filled arc
  \draw[fill=black] (#2,#3) ++(#4+55/3:.75*#1) arc
  (#4+55/3:#4-55/3:.75*#1);%
}


%%%%%%%%%%
%% discontinuité fonction
\newcommand\pointg[2]{%
  \draw[color = red, very thick] (#1+0.15, #2-.04)--(#1, #2-.04)--(#1,
  #2+.04)--(#1+0.15, #2+.04);%
}%

\newcommand\pointd[2]{%
  \draw[color = red, very thick] (#1-0.15, #2+.04)--(#1, #2+.04)--(#1,
  #2-.04)--(#1-0.15, #2-.04);%
}%

%%%%%%%%%%
%%% 1 : position abscisse, 2 : position ordonnée, 3 : taille, 4 : couleur
%%%%%%%%%%
% \newcommand\pointG[4]{%
%   \draw[color = #4, very thick] (#1+#3, #2-(#3/3.75))--(#1,
%   #2-(#3/3.75))--(#1, #2+(#3/3.75))--(#1+#3, #2+(#3/3.75)) %
% }%

\newcommand\pointG[4]{%
  \draw[color = #4, very thick] ({#1+#3/3.75}, {#2-#3})--(#1,
  {#2-#3})--(#1, {#2+#3})--({#1+#3/3.75}, {#2+#3}) %
}%

\newcommand\pointD[4]{%
  \draw[color = #4, very thick] ({#1-#3/3.75}, {#2+#3})--(#1,
  {#2+#3})--(#1, {#2-#3})--({#1-#3/3.75}, {#2-#3}) %
}%

\newcommand\spointG[4]{%
  \draw[color = #4, very thick] ({#1+#3/1.75}, {#2-#3})--(#1,
  {#2-#3})--(#1, {#2+#3})--({#1+#3/1.75}, {#2+#3}) %
}%

\newcommand\spointD[4]{%
  \draw[color = #4, very thick] ({#1-#3/2}, {#2+#3})--(#1,
  {#2+#3})--(#1, {#2-#3})--({#1-#3/2}, {#2-#3}) %
}%

%%%%%%%%%%

\newcommand{\Pb}{\mathtt{P}}

%%%%%%%%%%%%%%%
%%% Pour citer un précédent item
%%%%%%%%%%%%%%%
\newcommand{\itbf}[1]{{\small \bf \textit{#1}}}


%%%%%%%%%%%%%%%
%%% Quelques couleurs
%%%%%%%%%%%%%%%

\xdefinecolor{cancelcolor}{named}{Red}
\xdefinecolor{intI}{named}{ProcessBlue}
\xdefinecolor{intJ}{named}{ForestGreen}

%%%%%%%%%%%%%%%
%%%%%%%%%%%%%%%
% barrer du texte
\usetikzlibrary{shapes.misc}

\makeatletter
% \definecolor{cancelcolor}{rgb}{0.127,0.372,0.987}
\newcommand{\tikz@bcancel}[1]{%
  \begin{tikzpicture}[baseline=(textbox.base),inner sep=0pt]
  \node[strike out,draw] (textbox) {#1}[thick, color=cancelcolor];
  \useasboundingbox (textbox);
  \end{tikzpicture}%
}
\newcommand{\bcancel}[1]{%
  \relax\ifmmode
    \mathchoice{\tikz@bcancel{$\displaystyle#1$}}
               {\tikz@bcancel{$\textstyle#1$}}
               {\tikz@bcancel{$\scriptstyle#1$}}
               {\tikz@bcancel{$\scriptscriptstyle#1$}}
  \else
    \tikz@bcancel{\strut#1}%
  \fi
}
\newcommand{\tikz@xcancel}[1]{%
  \begin{tikzpicture}[baseline=(textbox.base),inner sep=0pt]
  \node[cross out,draw] (textbox) {#1}[thick, color=cancelcolor];
  \useasboundingbox (textbox);
  \end{tikzpicture}%
}
\newcommand{\xcancel}[1]{%
  \relax\ifmmode
    \mathchoice{\tikz@xcancel{$\displaystyle#1$}}
               {\tikz@xcancel{$\textstyle#1$}}
               {\tikz@xcancel{$\scriptstyle#1$}}
               {\tikz@xcancel{$\scriptscriptstyle#1$}}
  \else
    \tikz@xcancel{\strut#1}%
  \fi
}
\makeatother

\newcommand{\xcancelRA}{\xcancel{\rule[-.15cm]{0cm}{.5cm} \Rightarrow
    \rule[-.15cm]{0cm}{.5cm}}}

%%%%%%%%%%%%%%%%%%%%%%%%%%%%%%%%%%%
%%%%%%%%%%%%%%%%%%%%%%%%%%%%%%%%%%%

\newcommand{\vide}{\multicolumn{1}{c}{}}

%%%%%%%%%%%%%%%%%%%%%%%%%%%%%%%%%%%
%%%%%%%%%%%%%%%%%%%%%%%%%%%%%%%%%%%


\usepackage{multicol}
% \usepackage[latin1]{inputenc}
% \usepackage[T1]{fontenc}
\usepackage[utf8]{inputenc}
\usepackage[T1]{fontenc}
\usepackage[normalem]{ulem}
\usepackage[french]{babel}

\usepackage{url}    
\usepackage{hyperref}
\hypersetup{
  backref=true,
  pagebackref=true,
  hyperindex=true,
  colorlinks=true,
  breaklinks=true,
  urlcolor=blue,
  linkcolor=red,
  bookmarks=true,
  bookmarksopen=true
}

%%%%%%%%%%%%%%%%%%%%%%%%%%%%%%%%%%%%%%%%%%%
%% Pour faire des traits diagonaux dans les tableaux
%% Nécessite slashbox.sty
%\usepackage{slashbox}

\usepackage{tipa}
\usepackage{verbatim,listings}
\usepackage{graphicx}
\usepackage{fancyhdr}
\usepackage{mathrsfs}
\usepackage{pifont}
\usepackage{tablists}
\usepackage{dsfont,amsfonts,amssymb,amsmath,amsthm,stmaryrd,upgreek,manfnt}
\usepackage{enumerate}

%\newcolumntype{M}[1]{p{#1}}
\newcolumntype{C}[1]{>{\centering}m{#1}}
\newcolumntype{R}[1]{>{\raggedright}m{#1}}
\newcolumntype{L}[1]{>{\raggedleft}m{#1}}
\newcolumntype{P}[1]{>{\raggedright}p{#1}}
\newcolumntype{B}[1]{>{\raggedright}b{#1}}
\newcolumntype{Q}[1]{>{\raggedright}t{#1}}

\newcommand{\alias}[2]{
\providecommand{#1}{}
\renewcommand{#1}{#2}
}
\alias{\R}{\mathbb{R}}
\alias{\N}{\mathbb{N}}
\alias{\Z}{\mathbb{Z}}
\alias{\Q}{\mathbb{Q}}
\alias{\C}{\mathbb{C}}
\alias{\K}{\mathbb{K}}

%%%%%%%%%%%%
%% rendre +infty et -infty plus petits
%%%%%%%%%%%%
\newcommand{\sinfty}{{\scriptstyle \infty}}

%%%%%%%%%%%%%%%%%%%%%%%%%%%%%%%
%%%%% macros TP Scilab %%%%%%%%
\newcommand{\Scilab}{\textbf{Scilab}} %
\newcommand{\Scinotes}{\textbf{SciNotes}} %
\newcommand{\faire}{\noindent $\blacktriangleright$ } %
\newcommand{\fitem}{\scalebox{.8}{$\blacktriangleright$}} %
\newcommand{\entree}{{\small\texttt{ENTRÉE}}} %
\newcommand{\tab}{{\small\texttt{TAB}}} %
\newcommand{\mt}[1]{\mathtt{#1}} %
% guillemets droits

\newcommand{\ttq}{\textquotesingle} %

\newcommand{\reponse}[1]{\longboxed{
    \begin{array}{C{0.9\textwidth}}
      \nl[#1]
    \end{array}
  }} %

\newcommand{\reponseL}[2]{\longboxed{
    \begin{array}{C{#2\textwidth}}
      \nl[#1]
    \end{array}
  }} %

\newcommand{\reponseR}[1]{\longboxed{
    \begin{array}{R{0.9\textwidth}}
      #1
    \end{array}
  }} %

\newcommand{\reponseRL}[2]{\longboxed{
    \begin{array}{R{#2\textwidth}}
      #1
    \end{array}
  }} %

\newcommand{\reponseC}[1]{\longboxed{
    \begin{array}{C{0.9\textwidth}}
      #1
    \end{array}
  }} %

\colorlet{pyfunction}{Blue}
\colorlet{pyCle}{Magenta}
\colorlet{pycomment}{LimeGreen}
\colorlet{pydoc}{Cyan}
% \colorlet{SansCo}{white}
% \colorlet{AvecCo}{black}

\newcommand{\visible}[1]{{\color{ASCo}\colorlet{pydoc}{pyDo}\colorlet{pycomment}{pyCo}\colorlet{pyfunction}{pyF}\colorlet{pyCle}{pyC}\colorlet{function}{sciFun}\colorlet{var}{sciVar}\colorlet{if}{sciIf}\colorlet{comment}{sciComment}#1}} %

%%%% à changer ????
\newcommand{\invisible}[1]{{\color{ASCo}\colorlet{pydoc}{pyDo}\colorlet{pycomment}{pyCo}\colorlet{pyfunction}{pyF}\colorlet{pyCle}{pyC}\colorlet{function}{sciFun}\colorlet{var}{sciVar}\colorlet{if}{sciIf}\colorlet{comment}{sciComment}#1}} %

\newcommand{\invisibleCol}[2]{{\color{#1}#2}} %

\NewEnviron{solution} %
{ %
  \Boxed{
    \begin{array}{>{\color{ASCo}} R{0.9\textwidth}}
      \colorlet{pycomment}{pyCo}
      \colorlet{pydoc}{pyDo}
      \colorlet{pyfunction}{pyF}
      \colorlet{pyCle}{pyC}
      \colorlet{function}{sciFun}
      \colorlet{var}{sciVar}
      \colorlet{if}{sciIf}
      \colorlet{comment}{sciComment}
      \BODY
    \end{array}
  } %
} %

\NewEnviron{solutionC} %
{ %
  \Boxed{
    \begin{array}{>{\color{ASCo}} C{0.9\textwidth}}
      \colorlet{pycomment}{pyCo}
      \colorlet{pydoc}{pyDo}
      \colorlet{pyfunction}{pyF}
      \colorlet{pyCle}{pyC}
      \colorlet{function}{sciFun}
      \colorlet{var}{sciVar}
      \colorlet{if}{sciIf}
      \colorlet{comment}{sciComment}
      \BODY
    \end{array}
  } %
} %

\newcommand{\invite}{--\!\!>} %

%%%%% nouvel environnement tabular pour retour console %%%%
\colorlet{ConsoleColor}{Blue!15}
\colorlet{function}{Red}
\colorlet{var}{Maroon}
\colorlet{if}{Magenta}
\colorlet{comment}{LimeGreen}

\newcommand{\tcVar}[1]{\textcolor{var}{\bf \small #1}} %
\newcommand{\tcFun}[1]{\textcolor{function}{#1}} %
\newcommand{\tcIf}[1]{\textcolor{if}{#1}} %
\newcommand{\tcFor}[1]{\textcolor{if}{#1}} %

\newcommand{\moins}{\!\!\!\!\!\!- }
\newcommand{\espn}{\!\!\!\!\!\!}

\usepackage{booktabs,varwidth} \newsavebox\TBox
\newenvironment{console}
{\begin{lrbox}{\TBox}\varwidth{\linewidth}
    \tabular{>{\tt\small}R{0.84\textwidth}}
    \nl[-.4cm]} {\endtabular\endvarwidth\end{lrbox}%
  \fboxsep=1pt\colorbox{ConsoleColor}{\usebox\TBox}}

\newcommand{\lInv}[1]{%
  $\invite$ #1} %

\newcommand{\lAns}[1]{%
  \qquad ans \ = \nl %
  \qquad \qquad #1} %

\newcommand{\lVar}[2]{%
  \qquad #1 \ = \nl %
  \qquad \qquad #2} %

\newcommand{\lDisp}[1]{%
  #1 %
} %

\newcommand{\ligne}[1]{\underline{\small \tt #1}} %

\newcommand{\ligneAns}[2]{%
  $\invite$ #1 \nl %
  \qquad ans \ = \nl %
  \qquad \qquad #2} %

\newcommand{\ligneVar}[3]{%
  $\invite$ #1 \nl %
  \qquad #2 \ = \nl %
  \qquad \qquad #3} %

\newcommand{\ligneErr}[3]{%
  $\invite$ #1 \nl %
  \quad !-{-}error #2 \nl %
  #3} %
%%%%%%%%%%%%%%%%%%%%%% 

\newcommand{\bs}[1]{\boldsymbol{#1}} %
\newcommand{\nll}{\nl[.4cm]} %
\newcommand{\nle}{\nl[.2cm]} %
%% opérateur puissance copiant l'affichage Scilab
%\newcommand{\puis}{\!\!\!~^{\scriptscriptstyle\pmb{\wedge}}}
\newcommand{\puis}{\mbox{$\hspace{-.1cm}~^{\scriptscriptstyle\pmb{\wedge}}
    \hspace{0.05cm}$}} %
\newcommand{\pointpuis}{.\mbox{$\hspace{-.15cm}~^{\scriptscriptstyle\pmb{\wedge}}$}} %
\newcommand{\Sfois}{\mbox{$\mt{\star}$}} %

%%%%% nouvel environnement tabular pour les encadrés Scilab %%%%
\newenvironment{encadre}
{\begin{lrbox}{\TBox}\varwidth{\linewidth}
    \tabular{>{\tt\small}C{0.1\textwidth}>{\small}R{0.7\textwidth}}}
  {\endtabular\endvarwidth\end{lrbox}%
  \fboxsep=1pt\longboxed{\usebox\TBox}}

\newenvironment{encadreL}
{\begin{lrbox}{\TBox}\varwidth{\linewidth}
    \tabular{>{\tt\small}C{0.25\textwidth}>{\small}R{0.6\textwidth}}}
  {\endtabular\endvarwidth\end{lrbox}%
  \fboxsep=1pt\longboxed{\usebox\TBox}}

\newenvironment{encadreF}
{\begin{lrbox}{\TBox}\varwidth{\linewidth}
    \tabular{>{\tt\small}C{0.2\textwidth}>{\small}R{0.70\textwidth}}}
  {\endtabular\endvarwidth\end{lrbox}%
  \fboxsep=1pt\longboxed{\usebox\TBox}}

\newenvironment{encadreLL}[2]
{\begin{lrbox}{\TBox}\varwidth{\linewidth}
    \tabular{>{\tt\small}C{#1\textwidth}>{\small}R{#2\textwidth}}}
  {\endtabular\endvarwidth\end{lrbox}%
  \fboxsep=1pt\longboxed{\usebox\TBox}}

%%%%% nouvel environnement tabular pour les script et fonctions %%%%
\newcommand{\commentaireDL}[1]{\multicolumn{1}{l}{\it
    \textcolor{comment}{$\slash\slash$ #1}}}

\newcommand{\commentaire}[1]{{\textcolor{comment}{$\slash\slash$ #1}}}

\newcounter{cptcol}

\newcommand{\nocount}{\multicolumn{1}{c}{}}

\newcommand{\sciNo}[1]{{\small \underbar #1}}

\NewEnviron{scilab}{ %
  \setcounter{cptcol}{0}
  \begin{center}
    \longboxed{
      \begin{tabular}{>{\stepcounter{cptcol}{\tiny \underbar
              \thecptcol}}c>{\tt}l}
        \BODY
      \end{tabular}
    }
  \end{center}
}

\NewEnviron{scilabNC}{ %
  \begin{center}
    \longboxed{
      \begin{tabular}{>{\tt}l} %
          \BODY
      \end{tabular}
    }
  \end{center}
}

\NewEnviron{scilabC}[1]{ %
  \setcounter{cptcol}{#1}
  \begin{center}
    \longboxed{
      \begin{tabular}{>{\stepcounter{cptcol}{\tiny \underbar
              \thecptcol}}c>{\tt}l}
        \BODY
      \end{tabular}
    }
  \end{center}
}

\newcommand{\scisol}[1]{ %
  \setcounter{cptcol}{0}
  \longboxed{
    \begin{tabular}{>{\stepcounter{cptcol}{\tiny \underbar
            \thecptcol}}c>{\tt}l}
      #1
    \end{tabular}
  }
}

\newcommand{\scisolNC}[1]{ %
  \longboxed{
    \begin{tabular}{>{\tt}l}
      #1
    \end{tabular}
  }
}

\newcommand{\scisolC}[2]{ %
  \setcounter{cptcol}{#1}
  \longboxed{
    \begin{tabular}{>{\stepcounter{cptcol}{\tiny \underbar
            \thecptcol}}c>{\tt}l}
      #2
    \end{tabular}
  }
}

\NewEnviron{syntaxe}{ %
  % \fcolorbox{black}{Yellow!20}{\setlength{\fboxsep}{3mm}
  \shadowbox{
    \setlength{\fboxsep}{3mm}
    \begin{tabular}{>{\tt}l}
      \BODY
    \end{tabular}
  }
}

%%%%% fin macros TP Scilab %%%%%%%%
%%%%%%%%%%%%%%%%%%%%%%%%%%%%%%%%%%%

%%%%%%%%%%%%%%%%%%%%%%%%%%%%%%%%%%%
%%%%% TP Python - listings %%%%%%%%
%%%%%%%%%%%%%%%%%%%%%%%%%%%%%%%%%%%
\newcommand{\Python}{\textbf{Python}} %

\lstset{% general command to set parameter(s)
basicstyle=\ttfamily\small, % print whole listing small
keywordstyle=\color{blue}\bfseries\underbar,
%% underlined bold black keywords
frame=lines,
xleftmargin=10mm,
numbers=left,
numberstyle=\tiny\underbar,
numbersep=10pt,
%identifierstyle=, % nothing happens
commentstyle=\color{green}, % white comments
%%stringstyle=\ttfamily, % typewriter type for strings
showstringspaces=false}

\newcommand{\pysolCpt}[2]{
  \setcounter{cptcol}{#1}
  \longboxed{
    \begin{tabular}{>{\stepcounter{cptcol}{\tiny \underbar
            \thecptcol}}c>{\tt}l}
        #2
      \end{tabular}
    }
} %

\newcommand{\pysol}[1]{
  \setcounter{cptcol}{0}
  \longboxed{
    \begin{tabular}{>{\stepcounter{cptcol}{\tiny \underbar
            \thecptcol}}c>{\tt}l}
        #1
      \end{tabular}
    }
} %

% \usepackage[labelsep=endash]{caption}

% avec un caption
\NewEnviron{pythonCap}[1]{ %
  \renewcommand{\tablename}{Programme}
  \setcounter{cptcol}{0}
  \begin{center}
    \longboxed{
      \begin{tabular}{>{\stepcounter{cptcol}{\tiny \underbar
              \thecptcol}}c>{\tt}l}
        \BODY
      \end{tabular}
    }
    \captionof{table}{#1}
  \end{center}
}

\NewEnviron{python}{ %
  \setcounter{cptcol}{0}
  \begin{center}
    \longboxed{
      \begin{tabular}{>{\stepcounter{cptcol}{\tiny \underbar
              \thecptcol}}c>{\tt}l}
        \BODY
      \end{tabular}
    }
  \end{center}
}

\newcommand{\pyVar}[1]{\textcolor{var}{\bf \small #1}} %
\newcommand{\pyFun}[1]{\textcolor{pyfunction}{#1}} %
\newcommand{\pyCle}[1]{\textcolor{pyCle}{#1}} %
\newcommand{\pyImp}[1]{{\bf #1}} %

%%%%% commentaire python %%%%
\newcommand{\pyComDL}[1]{\multicolumn{1}{l}{\textcolor{pycomment}{\#
      #1}}}

\newcommand{\pyCom}[1]{{\textcolor{pycomment}{\# #1}}}
\newcommand{\pyDoc}[1]{{\textcolor{pydoc}{#1}}}

\newcommand{\pyNo}[1]{{\small \underbar #1}}

%%%%%%%%%%%%%%%%%%%%%%%%%%%%%%%%%%%
%%%%%% Système linéaire paramétré : écrire les opérations au-dessus
%%%%%% d'un symbole équivalent
%%%%%%%%%%%%%%%%%%%%%%%%%%%%%%%%%%%

\usepackage{systeme}

\NewEnviron{arrayEq}{ %
  \stackrel{\scalebox{.6}{$
      \begin{array}{l} 
        \BODY \\[.1cm]
      \end{array}$}
  }{\Longleftrightarrow}
}

\NewEnviron{arrayEg}{ %
  \stackrel{\scalebox{.6}{$
      \begin{array}{l} 
        \BODY \\[.1cm]
      \end{array}$}
  }{=}
}

\NewEnviron{operationEq}{ %
  \scalebox{.6}{$
    \begin{array}{l} 
      \scalebox{1.6}{$\mbox{Opérations :}$} \\[.2cm]
      \BODY \\[.1cm]
    \end{array}$}
}

% \NewEnviron{arraySys}[1]{ %
%   \sysdelim\{.\systeme[#1]{ %
%     \BODY %
%   } %
% }

%%%%%

%%%%%%%%%%
%%%%%%%%%% ESSAI
\newlength\fboxseph
\newlength\fboxsepva
\newlength\fboxsepvb

\setlength\fboxsepva{0.2cm}
\setlength\fboxsepvb{0.2cm}
\setlength\fboxseph{0.2cm}

\makeatletter

\def\longboxed#1{\leavevmode\setbox\@tempboxa\hbox{\color@begingroup%
\kern\fboxseph{\m@th$\displaystyle #1 $}\kern\fboxseph%
\color@endgroup }\my@frameb@x\relax}

\def\my@frameb@x#1{%
  \@tempdima\fboxrule \advance\@tempdima \fboxsepva \advance\@tempdima
  \dp\@tempboxa\hbox {%
    \lower \@tempdima \hbox {%
      \vbox {\hrule\@height\fboxrule \hbox{\vrule\@width\fboxrule #1
          \vbox{%
            \vskip\fboxsepva \box\@tempboxa \vskip\fboxsepvb}#1
          \vrule\@width\fboxrule }%
        \hrule \@height \fboxrule }}}}

\newcommand{\boxedhv}[3]{\setlength\fboxseph{#1cm}
  \setlength\fboxsepva{#2cm}\setlength\fboxsepvb{#2cm}\longboxed{#3}}

\newcommand{\boxedhvv}[4]{\setlength\fboxseph{#1cm}
  \setlength\fboxsepva{#2cm}\setlength\fboxsepvb{#3cm}\longboxed{#4}}

\newcommand{\Boxed}[1]{{\setlength\fboxseph{0.2cm}
  \setlength\fboxsepva{0.2cm}\setlength\fboxsepvb{0.2cm}\longboxed{#1}}}

\newcommand{\mBoxed}[1]{{\setlength\fboxseph{0.2cm}
  \setlength\fboxsepva{0.2cm}\setlength\fboxsepvb{0.2cm}\longboxed{\mbox{#1}}}}

\newcommand{\mboxed}[1]{{\setlength\fboxseph{0.2cm}
  \setlength\fboxsepva{0.2cm}\setlength\fboxsepvb{0.2cm}\boxed{\mbox{#1}}}}

\newsavebox{\fmbox}
\newenvironment{fmpage}[1]
     {\begin{lrbox}{\fmbox}\begin{minipage}{#1}}
     {\end{minipage}\end{lrbox}\fbox{\usebox{\fmbox}}}

%%%%%%%%%%
%%%%%%%%%%

\DeclareMathOperator{\ch}{ch}
\DeclareMathOperator{\sh}{sh}

%%%%%%%%%%
%%%%%%%%%%

\newcommand{\norme}[1]{\Vert #1 \Vert}

%\newcommand*\widefbox[1]{\fbox{\hspace{2em}#1\hspace{2em}}}

\newcommand{\nl}{\tabularnewline}

\newcommand{\hand}{\noindent\ding{43}\ }
\newcommand{\ie}{\textit{i.e. }}
\newcommand{\cf}{\textit{cf }}

\newcommand{\Card}{\operatorname{Card}}

\newcommand{\aire}{\mathcal{A}}

\newcommand{\LL}[1]{\mathscr{L}(#1)} %
\newcommand{\B}{\mathscr{B}} %
\newcommand{\Bc}[1]{B_{#1}} %
\newcommand{\M}[1]{\mathscr{M}_{#1}(\mathbb{R})}

\DeclareMathOperator{\im}{Im}
\DeclareMathOperator{\kr}{Ker}
\DeclareMathOperator{\rg}{rg}
\DeclareMathOperator{\tr}{tr}
\DeclareMathOperator{\spc}{Sp}
\DeclareMathOperator{\sgn}{sgn}
\DeclareMathOperator{\supp}{Supp}

\newcommand{\Mat}{{\rm{Mat}}}
\newcommand{\Vect}[1]{{\rm{Vect}}\left(#1\right)}

\newenvironment{smatrix}{%
  \begin{adjustbox}{width=.9\width}
    $
    \begin{pmatrix}
    }{%      
    \end{pmatrix}
    $
  \end{adjustbox}
}

\newenvironment{sarray}[1]{%
  \begin{adjustbox}{width=.9\width}
    $
    \begin{array}{#1}
    }{%      
    \end{array}
    $
  \end{adjustbox}
}

\newcommand{\vd}[2]{
  \scalebox{.8}{
    $\left(\!
      \begin{array}{c}
        #1 \\
        #2
      \end{array}
    \!\right)$
    }}

\newcommand{\vt}[3]{
  \scalebox{.8}{
    $\left(\!
      \begin{array}{c}
        #1 \\
        #2 \\
        #3 
      \end{array}
    \!\right)$
    }}

\newcommand{\vq}[4]{
  \scalebox{.8}{
    $\left(\!
      \begin{array}{c}
        #1 \\
        #2 \\
        #3 \\
        #4 
      \end{array}
    \!\right)$
    }}

\newcommand{\vc}[5]{
  \scalebox{.8}{
    $\left(\!
      \begin{array}{c}
        #1 \\
        #2 \\
        #3 \\
        #4 \\
        #5 
      \end{array}
    \!\right)$
    }}

\newcommand{\ee}{\text{e}}

\newcommand{\dd}{\text{d}}

%%% Ensemble de définition
\newcommand{\Df}{\mathscr{D}}
\newcommand{\Cf}{\mathscr{C}}
\newcommand{\Ef}{\mathscr{C}}

\newcommand{\rond}[1]{\,\overset{\scriptscriptstyle \circ}{\!#1}}

\newcommand{\df}[2]{\dfrac{\partial #1}{\partial #2}} %
\newcommand{\dfn}[2]{\partial_{#2}(#1)} %
\newcommand{\ddfn}[2]{\partial^2_{#2}(#1)} %
\newcommand{\ddf}[2]{\dfrac{\partial^2 #1}{\partial #2^2}} %
\newcommand{\ddfr}[3]{\dfrac{\partial^2 #1}{\partial #2 \partial
    #3}} %


\newcommand{\dlim}[1]{{\displaystyle \lim_{#1} \ }}
\newcommand{\dlimPlus}[2]{
  \dlim{
    \scalebox{.6}{
      $
      \begin{array}{l}
        #1 \rightarrow #2\\
        #1 > #2
      \end{array}
      $}}}
\newcommand{\dlimMoins}[2]{
  \dlim{
    \scalebox{.6}{
      $
      \begin{array}{l}
        #1 \rightarrow #2\\
        #1 < #2
      \end{array}
      $}}}

%%%%%%%%%%%%%%
%% petit o, développement limité
%%%%%%%%%%%%%%

\newcommand{\oo}[2]{{\underset {{\overset {#1\rightarrow #2}{}}}{o}}} %
\newcommand{\oox}[1]{{\underset {{\overset {x\rightarrow #1}{}}}{o}}} %
\newcommand{\oon}{{\underset {{\overset {n\rightarrow +\infty}{}}}{o}}} %
\newcommand{\po}[1]{{\underset {{\overset {#1}{}}}{o}}} %
\newcommand{\neqx}[1]{{\ \underset {{\overset {x \to #1}{}}}{\not\sim}\ }} %
\newcommand{\eqx}[1]{{\ \underset {{\overset {x \to #1}{}}}{\sim}\ }} %
\newcommand{\eqn}{{\ \underset {{\overset {n \to +\infty}{}}}{\sim}\ }} %
\newcommand{\eq}[2]{{\ \underset {{\overset {#1 \to #2}{}}}{\sim}\ }} %
\newcommand{\DL}[1]{{\rm{DL}}_1 (#1)} %
\newcommand{\DLL}[1]{{\rm{DL}}_2 (#1)} %

\newcommand{\negl}{<<}

\newcommand{\neglP}[1]{\begin{array}{c}
    \vspace{-.2cm}\\
    << \\
    \vspace{-.7cm}\\
    {\scriptstyle #1}
  \end{array}}

%%%%%%%%%%%%%%
%% borne sup, inf, max, min
%%%%%%%%%%%%%%
\newcommand{\dsup}[1]{\displaystyle \sup_{#1} \ }
\newcommand{\dinf}[1]{\displaystyle \inf_{#1} \ }
\newcommand{\dmax}[1]{\max\limits_{#1} \ }
\newcommand{\dmin}[1]{\min\limits_{#1} \ }

\newcommand{\dcup}[2]{{\textstyle\bigcup\limits_{#1}^{#2}}\hspace{.1cm}}
%\displaystyle \bigcup_{#1}^{#2}}
\newcommand{\dcap}[2]{{\textstyle\bigcap\limits_{#1}^{#2}}\hspace{.1cm}}
% \displaystyle \bigcap_{#1}^{#2}
%%%%%%%%%%%%%%
%% opérateurs logiques
%%%%%%%%%%%%%%
\newcommand{\NON}[1]{\mathop{\small \tt{NON}} (#1)}
\newcommand{\ET}{\mathrel{\mathop{\small \mathtt{ET}}}}
\newcommand{\OU}{\mathrel{\mathop{\small \tt{OU}}}}
\newcommand{\XOR}{\mathrel{\mathop{\small \tt{XOR}}}}

\newcommand{\id}{{\rm{id}}}

\newcommand{\sbullet}{\scriptstyle \bullet}
\newcommand{\stimes}{\scriptstyle \times}

%%%%%%%%%%%%%%%%%%
%% Probabilités
%%%%%%%%%%%%%%%%%%
\newcommand{\Prob}{\mathbb{P}}
\newcommand{\Ev}[1]{\left[ {#1} \right]}
\newcommand{\E}{\mathbb{E}}
\newcommand{\V}{\mathbb{V}}
\newcommand{\Cov}{{\rm{Cov}}}
\newcommand{\U}[2]{\mathcal{U}(\llb #1, #2\rrb)}
\newcommand{\Uc}[2]{\mathcal{U}([#1, #2])}
\newcommand{\Ucof}[2]{\mathcal{U}(]#1, #2])}
\newcommand{\Ucoo}[2]{\mathcal{U}(]#1, #2[)}
\newcommand{\Ucfo}[2]{\mathcal{U}([#1, #2[)}
\newcommand{\Bern}[1]{\mathcal{B}\left(#1\right)}
\newcommand{\Bin}[2]{\mathcal{B}\left(#1, #2\right)}
\newcommand{\G}[1]{\mathcal{G}\left(#1\right)}
\newcommand{\Pois}[1]{\mathcal{P}\left(#1\right)}
\newcommand{\HG}[3]{\mathcal{H}\left(#1, #2, #3\right)}
\newcommand{\Exp}[1]{\mathcal{E}\left(#1\right)}
\newcommand{\Norm}[2]{\mathcal{N}\left(#1, #2\right)}

\DeclareMathOperator{\cov}{Cov}

\newcommand{\var}{v.a.r. }
\newcommand{\suit}{\hookrightarrow}

\newcommand{\flecheR}[1]{\rotatebox{90}{\scalebox{#1}{\color{red}
      $\curvearrowleft$}}}


\newcommand{\partie}[1]{\mathcal{P}(#1)}
\newcommand{\Cont}[1]{\mathcal{C}^{#1}}
\newcommand{\Contm}[1]{\mathcal{C}^{#1}_m}

\newcommand{\llb}{\llbracket}
\newcommand{\rrb}{\rrbracket}

%\newcommand{\im}[1]{{\rm{Im}}(#1)}
\newcommand{\imrec}[1]{#1^{- \mathds{1}}}

\newcommand{\unq}{\mathds{1}}

\newcommand{\Hyp}{\mathtt{H}}

\newcommand{\eme}[1]{#1^{\scriptsize \mbox{ème}}}
\newcommand{\er}[1]{#1^{\scriptsize \mbox{er}}}
\newcommand{\ere}[1]{#1^{\scriptsize \mbox{ère}}}
\newcommand{\nd}[1]{#1^{\scriptsize \mbox{nd}}}
\newcommand{\nde}[1]{#1^{\scriptsize \mbox{nde}}}

\newcommand{\truc}{\mathop{\top}}
\newcommand{\fois}{\mathop{\ast}}

\newcommand{\f}[1]{\overrightarrow{#1}}

\newcommand{\checked}{\textcolor{green}{\checkmark}}

\def\restriction#1#2{\mathchoice
              {\setbox1\hbox{${\displaystyle #1}_{\scriptstyle #2}$}
              \restrictionaux{#1}{#2}}
              {\setbox1\hbox{${\textstyle #1}_{\scriptstyle #2}$}
              \restrictionaux{#1}{#2}}
              {\setbox1\hbox{${\scriptstyle #1}_{\scriptscriptstyle #2}$}
              \restrictionaux{#1}{#2}}
              {\setbox1\hbox{${\scriptscriptstyle #1}_{\scriptscriptstyle #2}$}
              \restrictionaux{#1}{#2}}}
\def\restrictionaux#1#2{{#1\,\smash{\vrule height .8\ht1 depth .85\dp1}}_{\,#2}}

\makeatletter
\newcommand*{\da@rightarrow}{\mathchar"0\hexnumber@\symAMSa 4B }
\newcommand*{\da@leftarrow}{\mathchar"0\hexnumber@\symAMSa 4C }
\newcommand*{\xdashrightarrow}[2][]{%
  \mathrel{%
    \mathpalette{\da@xarrow{#1}{#2}{}\da@rightarrow{\,}{}}{}%
  }%
}
\newcommand{\xdashleftarrow}[2][]{%
  \mathrel{%
    \mathpalette{\da@xarrow{#1}{#2}\da@leftarrow{}{}{\,}}{}%
  }%
}
\newcommand*{\da@xarrow}[7]{%
  % #1: below
  % #2: above
  % #3: arrow left
  % #4: arrow right
  % #5: space left 
  % #6: space right
  % #7: math style 
  \sbox0{$\ifx#7\scriptstyle\scriptscriptstyle\else\scriptstyle\fi#5#1#6\m@th$}%
  \sbox2{$\ifx#7\scriptstyle\scriptscriptstyle\else\scriptstyle\fi#5#2#6\m@th$}%
  \sbox4{$#7\dabar@\m@th$}%
  \dimen@=\wd0 %
  \ifdim\wd2 >\dimen@
    \dimen@=\wd2 %   
  \fi
  \count@=2 %
  \def\da@bars{\dabar@\dabar@}%
  \@whiledim\count@\wd4<\dimen@\do{%
    \advance\count@\@ne
    \expandafter\def\expandafter\da@bars\expandafter{%
      \da@bars
      \dabar@ 
    }%
  }%  
  \mathrel{#3}%
  \mathrel{%   
    \mathop{\da@bars}\limits
    \ifx\\#1\\%
    \else
      _{\copy0}%
    \fi
    \ifx\\#2\\%
    \else
      ^{\copy2}%
    \fi
  }%   
  \mathrel{#4}%
}
\makeatother



\newcount\depth

\newcount\depth
\newcount\totaldepth

\makeatletter
\newcommand{\labelsymbol}{%
      \ifnum\depth=0
        %
      \else
        \rlap{\,$\bullet$}%
      \fi
}

\newcommand*\bernoulliTree[1]{%
    \depth=#1\relax            
    \totaldepth=#1\relax
    \draw node(root)[bernoulli/root] {\labelsymbol}[grow=right] \draw@bernoulli@tree;
    \draw \label@bernoulli@tree{root};                                   
}                                                                        

\def\draw@bernoulli@tree{%
    \ifnum\depth>0 
      child[parent anchor=east] foreach \type/\label in {left child/$E$,right child/$S$} {%
          node[bernoulli/\type] {\label\strut\labelsymbol} \draw@bernoulli@tree
      }
      coordinate[bernoulli/increment] (dummy)
   \fi%
}

\def\label@bernoulli@tree#1{%
    \ifnum\depth>0
      ($(#1)!0.5!(#1-1)$) node[fill=white,bernoulli/decrement] {\tiny$p$}
      \label@bernoulli@tree{#1-1}
      ($(#1)!0.5!(#1-2)$) node[fill=white] {\tiny$q$}
      \label@bernoulli@tree{#1-2}
      coordinate[bernoulli/increment] (dummy)
   \fi%
}

\makeatother

\tikzset{bernoulli/.cd,
         root/.style={},
         decrement/.code=\global\advance\depth by-1\relax,
         increment/.code=\global\advance\depth by 1\relax,
         left child/.style={bernoulli/decrement},
         right child/.style={}}


\newcommand{\eps}{\varepsilon}

% \newcommand{\tendi}[1]{\xrightarrow[\footnotesize #1 \rightarrow
%   +\infty]{}}%

\newcommand{\tend}{\rightarrow}%
\newcommand{\tendn}{\underset{n\to +\infty}{\longrightarrow}} %
\newcommand{\ntendn}{\underset{n\to
    +\infty}{\not\hspace{-.15cm}\longrightarrow}} %
% \newcommand{\tendn}{\xrightarrow[\footnotesize n \rightarrow
%   +\infty]{}}%
\newcommand{\Tendx}[1]{\xrightarrow[\footnotesize x \rightarrow
  #1]{}}%
\newcommand{\tendx}[1]{\underset{x\to #1}{\longrightarrow}}%
\newcommand{\ntendx}[1]{\underset{x\to #1}{\not\!\!\longrightarrow}}%
\newcommand{\tendd}[2]{\underset{#1\to #2}{\longrightarrow}}%
% \newcommand{\tendd}[2]{\xrightarrow[\footnotesize #1 \rightarrow
%   #2]{}}%
\newcommand{\tendash}[1]{\xdashrightarrow[\footnotesize #1 \rightarrow
  +\infty]{}}%
\newcommand{\tendashx}[1]{\xdashrightarrow[\footnotesize x \rightarrow
  #1]{}}%
\newcommand{\tendb}[1]{\underset{#1 \to +\infty}{\longrightarrow}}%
\newcommand{\tendL}{\overset{\mathscr L}{\underset{n \to
      +\infty}{\longrightarrow}}}%
\newcommand{\tendP}{\overset{\Prob}{\underset{n \to
      +\infty}{\longrightarrow}}}%
\newcommand{\tenddL}[1]{\overset{\mathscr L}{\underset{#1 \to
      +\infty}{\longrightarrow}}}%

\NewEnviron{attention}{ %
  ~\\[-.2cm]\noindent
  \begin{minipage}{\linewidth}
  \setlength{\fboxsep}{3mm}%
  \ \ \dbend \ \ %
  \fbox{\parbox[t]{.88\linewidth}{\BODY}} %
  \end{minipage}\\
}

\NewEnviron{sattention}[1]{ %
  ~\\[-.2cm]\noindent
  \begin{minipage}{#1\linewidth}
  \setlength{\fboxsep}{3mm}%
  \ \ \dbend \ \ %
  \fbox{\parbox[t]{.88\linewidth}{\BODY}} %
  \end{minipage}\\
}

%%%%% OBSOLETE %%%%%%

% \newcommand{\attention}[1]{
%   \noindent
%   \begin{tabular}{@{}l|p{11.5cm}|}
%     \cline{2-2}
%     \vspace{-.2cm} 
%     & \nl
%     \dbend & #1 \nl
%     \cline{2-2}
%   \end{tabular}
% }

% \newcommand{\attentionv}[2]{
%   \noindent
%   \begin{tabular}{@{}l|p{11.5cm}|}
%     \cline{2-2}
%     \vspace{-.2cm} 
%     & \nl
%     \dbend & #2 \nl[#1 cm]
%     \cline{2-2}
%   \end{tabular}
% }

\newcommand{\explainvb}[2]{
  \noindent
  \begin{tabular}{@{}l|p{11.5cm}|}
    \cline{2-2}
    \vspace{-.2cm} 
    & \nl
    \hand & #2 \nl [#1 cm]
    \cline{2-2}
  \end{tabular}
}


% \noindent
% \begin{tabular}{@{}l|lp{11cm}|}
%   \cline{3-3} 
%   \multicolumn{1}{@{}l@{\dbend}}{} & & #1 \nl
%   \multicolumn{1}{l}{} & & \nl [-.8cm]
%   & & #2 \nl
%   \cline{2-3}
% \end{tabular}

% \newcommand{\attention}[1]{
%   \noindent
%   \begin{tabular}{@{}@{}cp{11cm}}
%     \dbend & #1 \nl
%   \end{tabular}
% }

\newcommand{\PP}[1]{\mathcal{P}(#1)}
\newcommand{\FF}[1]{\mathcal{F}(#1)}

\newcommand{\DSum}[2]{\displaystyle\sum\limits_{#1}^{#2}\hspace{.1cm}}
\newcommand{\Sum}[2]{{\textstyle\sum\limits_{#1}^{#2}}\hspace{.1cm}}
\newcommand{\Serie}{\textstyle\sum\ }
\newcommand{\Prod}[2]{\textstyle\prod\limits_{#1}^{#2}}

\newcommand{\Prim}[3]{\left[\ {#1} \ \right]_{\scriptscriptstyle
   \hspace{-.15cm} ~_{#2}\, }^ {\scriptscriptstyle \hspace{-.15cm} ~^{#3}\, }}

% \newcommand{\Prim}[3]{\left[\ {#1} \ \right]_{\scriptscriptstyle
%     \!\!~_{#2}}^ {\scriptscriptstyle \!\!~^{#3}}}

\newcommand{\dint}[2]{\displaystyle \int_{#1}^{#2}\ }
\newcommand{\Int}[2]{{\rm{Int}}_{\scriptscriptstyle #1, #2}}
\newcommand{\dt}{\ dt}
\newcommand{\dx}{\ dx}

\newcommand{\llpar}[1]{\left(\!\!\!
    \begin{array}{c}
      \rule{0pt}{#1}
    \end{array}
  \!\!\!\right.}

\newcommand{\rrpar}[1]{\left.\!\!\!
    \begin{array}{c}
      \rule{0pt}{#1}
    \end{array}
  \!\!\!\right)}

\newcommand{\llacc}[1]{\left\{\!\!\!
    \begin{array}{c}
      \rule{0pt}{#1}
    \end{array}
  \!\!\!\right.}

\newcommand{\rracc}[1]{\left.\!\!\!
    \begin{array}{c}
      \rule{0pt}{#1}
    \end{array}
  \!\!\!\right\}}

\newcommand{\ttacc}[1]{\mbox{\rotatebox{-90}{\hspace{-.7cm}$\llacc{#1}$}}}
\newcommand{\bbacc}[1]{\mbox{\rotatebox{90}{\hspace{-.5cm}$\llacc{#1}$}}}

\newcommand{\comp}[1]{\overline{#1}}%

\newcommand{\dcomp}[2]{\stackrel{\mbox{\ \ \----}{\scriptscriptstyle
      #2}}{#1}}%

% \newcommand{\Comp}[2]{\stackrel{\mbox{\ \
%       \-------}{\scriptscriptstyle #2}}{#1}}

% \newcommand{\dcomp}[2]{\stackrel{\mbox{\ \
%       \-------}{\scriptscriptstyle #2}}{#1}}

\newcommand{\A}{\mathscr{A}}

\newcommand{\conc}[1]{
  \begin{center}
    \fbox{
      \begin{tabular}{c}
        #1
      \end{tabular}
    }
  \end{center}
}

\newcommand{\concC}[1]{
  \begin{center}
    \fbox{
    \begin{tabular}{C{10cm}}
      \quad #1 \quad
    \end{tabular}
    }
  \end{center}
}

\newcommand{\concL}[2]{
  \begin{center}
    \fbox{
    \begin{tabular}{C{#2cm}}
      \quad #1 \quad
    \end{tabular}
    }
  \end{center}
}

% \newcommand{\lims}[2]{\prod\limits_{#1}^{#2}}

\newtheorem{theorem}{Théorème}[]
\newtheorem{lemma}{Lemme}[]
\newtheorem{proposition}{Proposition}[]
\newtheorem{corollary}{Corollaire}[]

% \newenvironment{proof}[1][Démonstration]{\begin{trivlist}
% \item[\hskip \labelsep {\bfseries #1}]}{\end{trivlist}}
\newenvironment{definition}[1][Définition]{\begin{trivlist}
\item[\hskip \labelsep {\bfseries #1}]}{\end{trivlist}}
\newenvironment{example}[1][Exemple]{\begin{trivlist}
\item[\hskip \labelsep {\bfseries #1}]}{\end{trivlist}}
\newenvironment{examples}[1][Exemples]{\begin{trivlist}
\item[\hskip \labelsep {\bfseries #1}]}{\end{trivlist}}
\newenvironment{notation}[1][Notation]{\begin{trivlist}
\item[\hskip \labelsep {\bfseries #1}]}{\end{trivlist}}
\newenvironment{propriete}[1][Propriété]{\begin{trivlist}
\item[\hskip \labelsep {\bfseries #1}]}{\end{trivlist}}
\newenvironment{proprietes}[1][Propriétés]{\begin{trivlist}
\item[\hskip \labelsep {\bfseries #1}]}{\end{trivlist}}
\newenvironment{remarkSC}[1][Remarque]{\begin{trivlist}
\item[\hskip \labelsep {\bfseries #1}]}{\end{trivlist}}
\newenvironment{application}[1][Application]{\begin{trivlist}
\item[\hskip \labelsep {\bfseries #1}]}{\end{trivlist}}

% Environnement pour les réponses des DS
\newenvironment{answer}{\par\emph{Réponse :}\par{}}
{\vspace{-.6cm}\hspace{\stretch{1}}\rule{1ex}{1ex}\vspace{.3cm}}

\newenvironment{answerTD}{\vspace{.2cm}\par\emph{Réponse :}\par{}}
{\hspace{\stretch{1}}\rule{1ex}{1ex}\vspace{.3cm}}

\newenvironment{answerCours}{\noindent\emph{Réponse :}}
{\rule{1ex}{1ex}}%\vspace{.3cm}}


% footnote in footer
\newcommand{\fancyfootnotetext}[2]{%
  \fancypagestyle{dingens}{%
    \fancyfoot[LO,RE]{\parbox{0.95\textwidth}{\footnotemark[#1]\footnotesize
        #2}}%
  }%
  \thispagestyle{dingens}%
}

%%% définit le style (arabic : 1,2,3...) et place des parenthèses
%%% autour de la numérotation
\renewcommand*{\thefootnote}{(\arabic{footnote})}
% http://www.tuteurs.ens.fr/logiciels/latex/footnote.html

%%%%%%%% tikz axis
% \pgfplotsset{every axis/.append style={
%                     axis x line=middle,    % put the x axis in the middle
%                     axis y line=middle,    % put the y axis in the middle
%                     axis line style={<->,color=blue}, % arrows on the axis
%                     xlabel={$x$},          % default put x on x-axis
%                     ylabel={$y$},          % default put y on y-axis
%             }}

%%%% s'utilise comme suit

% \begin{axis}[
%   xmin=-8,xmax=4,
%   ymin=-8,ymax=4,
%   grid=both,
%   ]
%   \addplot [domain=-3:3,samples=50]({x^3-3*x},{3*x^2-9}); 
% \end{axis}

%%%%%%%%



%%%%%%%%%%%% Pour avoir des numéros de section qui correspondent à
%%%%%%%%%%%% ceux du tableau
\renewcommand{\thesection}{\Roman{section}.\hspace{-.3cm}}
\renewcommand{\thesubsection}{\Roman{section}.\arabic{subsection}.\hspace{-.2cm}}
\renewcommand{\thesubsubsection}{\Roman{section}.\arabic{subsection}.\alph{subsubsection})\hspace{-.2cm}}
%%%%%%%%%%%% 

%%% Changer le nom des figures : Fig. au lieu de Figure
\usepackage[font=small,labelfont=bf,labelsep=space]{caption}
\captionsetup{%
  figurename=Fig.,
  tablename=Tab.
}
% \renewcommand{\thesection}{\Roman{section}.\hspace{-.2cm}}
% \renewcommand{\thesubsection}{\Roman{section}
%   .\hspace{.2cm}\arabic{subsection}\ .\hspace{-.3cm}}
% \renewcommand{\thesubsubsection}{\alph{subsection})}

\newenvironment{tabliste}[1]
{\begin{tabenum}[\bfseries\small\itshape #1]}{\end{tabenum}} 

\newenvironment{noliste}[1] %
{\begin{enumerate}[\bfseries\small\itshape #1]} %
  {\end{enumerate}}

\newenvironment{nonoliste}[1] %
{\begin{enumerate}[\hspace{-12pt}\bfseries\small\itshape #1]} %
  {\end{enumerate}}

\newenvironment{arrayliste}[1]{ 
  % List with minimal white space to fit in small areas, e.g. table
  % cell
  \begin{minipage}[t]{\linewidth} %
    \begin{enumerate}[\bfseries\small\itshape #1] %
      {\leftmargin=0.5em \rightmargin=0em
        \topsep=0em \parskip=0em \parsep=0em
        \listparindent=0em \partopsep=0em \itemsep=0pt
        \itemindent=0em \labelwidth=\leftmargin\labelsep+0.25em}
    }{
    \end{enumerate}\end{minipage}
}

\newenvironment{nolistes}[2]
{\begin{enumerate}[\bfseries\small\itshape
    #1]\setlength{\itemsep}{#2 mm}}{\end{enumerate}}

\newenvironment{liste}[1]
{\begin{enumerate}[\hspace{12pt}\bfseries\small\itshape #1]}{\end{enumerate}} 



%%%%%%%% Pour les programmes de colle %%%%%%%

\newcommand{\cours}{{\small \tt (COURS)}} %
\newcommand{\poly}{{\small \tt (POLY)}} %
\newcommand{\exo}{{\small \tt (EXO)}} %
\newcommand{\culture}{{\small \tt (CULTURE)}} %
\newcommand{\methodo}{{\small \tt (MÉTHODO)}} %
\newcommand{\methodob}{\Boxed{\mbox{\tt MÉTHODO}}} %

%%%%%%%% Pour les TD %%%%%%%
\newtheoremstyle{exostyle} {\topsep} % espace avant
{.6cm} % espace apres
{} % Police utilisee par le style de thm
{} % Indentation (vide = aucune, \parindent = indentation paragraphe)
{\bfseries} % Police du titre de thm
{} % Signe de ponctuation apres le titre du thm
{ } % Espace apres le titre du thm (\newline = linebreak)
{\thmname{#1}\thmnumber{ #2}\thmnote{.
    \normalfont{\textit{#3}}}} % composants du titre du thm : \thmname
                               % = nom du thm, \thmnumber = numéro du
                               % thm, \thmnote = sous-titre du thm
 
\theoremstyle{exostyle}
\newtheorem{exercice}{Exercice}
\newtheorem{exerciceAP}{Exercice avec préparation}
\newtheorem{exerciceSP}{Exercice sans préparation}
\newtheorem*{exoCours}{Exercice}

%%%%%%%% Pour des théorèmes Sans Espaces APRÈS %%%%%%%
\newtheoremstyle{exostyleSE} {\topsep} % espace avant
{} % espace apres
{} % Police utilisee par le style de thm
{} % Indentation (vide = aucune, \parindent = indentation paragraphe)
{\bfseries} % Police du titre de thm
{} % Signe de ponctuation apres le titre du thm
{ } % Espace apres le titre du thm (\newline = linebreak)
{\thmname{#1}\thmnumber{ #2}\thmnote{.
    \normalfont{\textit{#3}}}} % composants du titre du thm : \thmname
                               % = nom du thm, \thmnumber = numéro du
                               % thm, \thmnote = sous-titre du thm
 
\theoremstyle{exostyleSE}
\newtheorem{exerciceSE}{Exercice}
\newtheorem*{exoCoursSE}{Exercice}

% \newcommand{\lims}[2]{\prod\limits_{#1}^{#2}}

\newtheorem{theoremSE}{Théorème}[]
\newtheorem{lemmaSE}{Lemme}[]
\newtheorem{propositionSE}{Proposition}[]
\newtheorem{corollarySE}{Corollaire}[]

% \newenvironment{proofSE}[1][Démonstration]{\begin{trivlist}
% \item[\hskip \labelsep {\bfseries #1}]}{\end{trivlist}}
\newenvironment{definitionSE}[1][Définition]{\begin{trivlist}
  \item[\hskip \labelsep {\bfseries #1}]}{\end{trivlist}}
\newenvironment{exampleSE}[1][Exemple]{\begin{trivlist} 
  \item[\hskip \labelsep {\bfseries #1}]}{\end{trivlist}}
\newenvironment{examplesSE}[1][Exemples]{\begin{trivlist}
\item[\hskip \labelsep {\bfseries #1}]}{\end{trivlist}}
\newenvironment{notationSE}[1][Notation]{\begin{trivlist}
\item[\hskip \labelsep {\bfseries #1}]}{\end{trivlist}}
\newenvironment{proprieteSE}[1][Propriété]{\begin{trivlist}
\item[\hskip \labelsep {\bfseries #1}]}{\end{trivlist}}
\newenvironment{proprietesSE}[1][Propriétés]{\begin{trivlist}
\item[\hskip \labelsep {\bfseries #1}]}{\end{trivlist}}
\newenvironment{remarkSE}[1][Remarque]{\begin{trivlist}
\item[\hskip \labelsep {\bfseries #1}]}{\end{trivlist}}
\newenvironment{applicationSE}[1][Application]{\begin{trivlist}
\item[\hskip \labelsep {\bfseries #1}]}{\end{trivlist}}

%%%%%%%%%%% Obtenir les étoiles sans charger le package MnSymbol
%%%%%%%%%%%
\DeclareFontFamily{U} {MnSymbolC}{}
\DeclareFontShape{U}{MnSymbolC}{m}{n}{
  <-6> MnSymbolC5
  <6-7> MnSymbolC6
  <7-8> MnSymbolC7
  <8-9> MnSymbolC8
  <9-10> MnSymbolC9
  <10-12> MnSymbolC10
  <12-> MnSymbolC12}{}
\DeclareFontShape{U}{MnSymbolC}{b}{n}{
  <-6> MnSymbolC-Bold5
  <6-7> MnSymbolC-Bold6
  <7-8> MnSymbolC-Bold7
  <8-9> MnSymbolC-Bold8
  <9-10> MnSymbolC-Bold9
  <10-12> MnSymbolC-Bold10
  <12-> MnSymbolC-Bold12}{}

\DeclareSymbolFont{MnSyC} {U} {MnSymbolC}{m}{n}

\DeclareMathSymbol{\filledlargestar}{\mathrel}{MnSyC}{205}
\DeclareMathSymbol{\largestar}{\mathrel}{MnSyC}{131}

\newcommand{\facile}{\rm{(}$\scriptstyle\largestar$\rm{)}} %
\newcommand{\moyen}{\rm{(}$\scriptstyle\filledlargestar$\rm{)}} %
\newcommand{\dur}{\rm{(}$\scriptstyle\filledlargestar\filledlargestar$\rm{)}} %
\newcommand{\costaud}{\rm{(}$\scriptstyle\filledlargestar\filledlargestar\filledlargestar$\rm{)}}

%%%%%%%%%%%%%%%%%%%%%%%%%

%%%%%%%%%%%%%%%%%%%%%%%%%
%%%%%%%% Fin de la partie TD

%%%%%%%%%%%%%%%%
%%%%%%%%%%%%%%%%
\makeatletter %
\newenvironment{myitemize}{%
  \setlength{\topsep}{0pt} %
  \setlength{\partopsep}{0pt} %
  \renewcommand*{\@listi}{\leftmargin\leftmargini \parsep\z@
    \topsep\z@ \itemsep\z@} \let\@listI\@listi %
  \itemize %
}{\enditemize} %
\makeatother
%%%%%%%%%%%%%%%%
%%%%%%%%%%%%%%%%

%% Commentaires dans la correction du livre

\newcommand{\Com}[1]{
% Define box and box title style
\tikzstyle{mybox} = [draw=black!50,
very thick,
    rectangle, rounded corners, inner sep=10pt, inner ysep=8pt]
\tikzstyle{fancytitle} =[rounded corners, fill=black!80, text=white]
\tikzstyle{fancylogo} =[ text=white]
\begin{center}

\begin{tikzpicture}
\node [mybox] (box){%

    \begin{minipage}{0.90\linewidth}
\vspace{6pt}  #1
    \end{minipage}
};
\node[fancytitle, right=10pt] at (box.north west) 
{\bfseries\normalsize{Commentaire}};

\end{tikzpicture}%

\end{center}
%
}

% \NewEnviron{remark}{%
%   % Define box and box title style
%   \tikzstyle{mybox} = [draw=black!50, very thick, rectangle, rounded
%   corners, inner sep=10pt, inner ysep=8pt] %
%   \tikzstyle{fancytitle} =[rounded corners, fill=black!80,
%   text=white] %
%   \tikzstyle{fancylogo} =[ text=white]
%   \begin{center}
%     \begin{tikzpicture}
%       \node [mybox] (box){%
%         \begin{minipage}{0.90\linewidth}
%           \vspace{6pt} \BODY
%         \end{minipage}
%       }; %
%       \node[fancytitle, right=10pt] at (box.north west) %
%       {\bfseries\normalsize{Commentaire}}; %
%     \end{tikzpicture}%
%   \end{center}
% }
% \makeatother


%%%
% Remarques comme dans le livre
%%%

\NewEnviron{remark}{%
  % Define box and box title style
  \tikzstyle{mybox} = [draw=black!50, very thick, rectangle, rounded
  corners, inner sep=10pt, inner ysep=8pt] %
  \tikzstyle{fancytitle} = [rounded corners , fill=black!80,
  text=white] %
  \tikzstyle{fancylogo} =[ text=white]
  \begin{center}
    \begin{tikzpicture}
      \node [mybox] (box){%
        \begin{minipage}{0.90\linewidth}
          \vspace{6pt} \BODY
        \end{minipage}
      }; %
      \node[fancytitle, right=10pt] at (box.north west) %
      {\bfseries\normalsize{Commentaire}}; %
    \end{tikzpicture}%
  \end{center}
}

\NewEnviron{remarkST}{%
  % Define box and box title style
  \tikzstyle{mybox} = [draw=black!50, very thick, rectangle, rounded
  corners, inner sep=10pt, inner ysep=8pt] %
  \tikzstyle{fancytitle} = [rounded corners , fill=black!80,
  text=white] %
  \tikzstyle{fancylogo} =[ text=white]
  \begin{center}
    \begin{tikzpicture}
      \node [mybox] (box){%
        \begin{minipage}{0.90\linewidth}
          \vspace{6pt} \BODY
        \end{minipage}
      }; %
      % \node[fancytitle, right=10pt] at (box.north west) %
%       {\bfseries\normalsize{Commentaire}}; %
    \end{tikzpicture}%
  \end{center}
}

\NewEnviron{remarkL}[1]{%
  % Define box and box title style
  \tikzstyle{mybox} = [draw=black!50, very thick, rectangle, rounded
  corners, inner sep=10pt, inner ysep=8pt] %
  \tikzstyle{fancytitle} =[rounded corners, fill=black!80,
  text=white] %
  \tikzstyle{fancylogo} =[ text=white]
  \begin{center}
    \begin{tikzpicture}
      \node [mybox] (box){%
        \begin{minipage}{#1\linewidth}
          \vspace{6pt} \BODY
        \end{minipage}
      }; %
      \node[fancytitle, right=10pt] at (box.north west) %
      {\bfseries\normalsize{Commentaire}}; %
    \end{tikzpicture}%
  \end{center}
}

\NewEnviron{remarkSTL}[1]{%
  % Define box and box title style
  \tikzstyle{mybox} = [draw=black!50, very thick, rectangle, rounded
  corners, inner sep=10pt, inner ysep=8pt] %
  \tikzstyle{fancytitle} =[rounded corners, fill=black!80,
  text=white] %
  \tikzstyle{fancylogo} =[ text=white]
  \begin{center}
    \begin{tikzpicture}
      \node [mybox] (box){%
        \begin{minipage}{#1\linewidth}
          \vspace{6pt} \BODY
        \end{minipage}
      }; %
%       \node[fancytitle, right=10pt] at (box.north west) %
%       {\bfseries\normalsize{Commentaire}}; %
    \end{tikzpicture}%
  \end{center}
}

\NewEnviron{titre} %
{ %
  ~\\[-1.8cm]
  \begin{center}
    \bf \LARGE \BODY
  \end{center}
  ~\\[-.6cm]
  \hrule %
  \vspace*{.2cm}
} %

\NewEnviron{titreL}[2] %
{ %
  ~\\[-#1cm]
  \begin{center}
    \bf \LARGE \BODY
  \end{center}
  ~\\[-#2cm]
  \hrule %
  \vspace*{.2cm}
} %




\pagestyle{fancy} %
\lhead{ECE2 \hfill septembre 2017 \\
 Mathématiques\\[.2cm]} %
\chead{\hrule} %
\rhead{} %
\lfoot{} %
\cfoot{} %
\rfoot{\thepage} %

\renewcommand{\headrulewidth}{0pt}% : Trace un trait de séparation
 % de largeur 0,4 point. Mettre 0pt
 % pour supprimer le trait.

\renewcommand{\footrulewidth}{0.4pt}% : Trace un trait de séparation
 % de largeur 0,4 point. Mettre 0pt
 % pour supprimer le trait.

\setlength{\headheight}{14pt}

\title{\bf \vspace{-1cm} ORAUX HEC 2012} %
\author{} %
\date{} %

\begin{document}

\maketitle %
\vspace{-1.2cm}\hrule %
\thispagestyle{fancy}

\vspace*{.4cm}

% DEBUT DU DOC À MODIFIER : tout virer jusqu'au début de l'exo

\section{Annales 2012}
 
 %\setcounter{exercice}{0}

 \begin{exercice}{\it (Exercice avec préparation)}~
 \begin{noliste}{1.}
 \setlength{\itemsep}{4mm}
 \item La série de terme général $(u_{n})$ converge si la suite
$(S_{n})$ des sommes partielles : $S_{n} = \sum \limits_{k = 1}{n}
u_{k}$ admet une limite finie lorsque $n$ tend vers $ + \infty$.\\
 La série de terme général $(\ln x)^{n}$ est géométrique donc converge
si et seulement si $|\ln x|<1 \Leftrightarrow -1< \ln x < 1
\Leftrightarrow \frac{1}{e} < x < e$ par stricte croissance de la
fonction exponentielle. On a alors : \\
 
\[
 \Sum{n = 0}{+ \infty} (\ln x)^{n} = \frac{1 }{1-\ln x}
\] 

 \item 
 \begin{noliste}{a)}
 \setlength{\itemsep}{2mm}
 \item $f_{n}$ est de classe $\mathcal{C}{2}$ sur $]0, + \infty[$ comme
somme de produits de fonctions usuelles de classe $\mathcal{C}{2}$ sur
$]0, + \infty[$.\\
 $\forall x > 0$, $f_{n}{'}(x) = \frac{n}{x}(\ln x)^{n-1}-1$.\\
 Ainsi, si $n = 1$, $f_{1}{'}(x) = \frac{1}{x}-1$ donc $f_{1}{''}(x) =
-\frac{1}{x^{2}}$; \\
 si $n \geq 2$, $f_{n}{''}(x) = -\frac{1}{x^{2}} n (\ln x) ^{n-1} +
\frac{1}{x^{2}} n (n-1) (\ln x)^{n-2} = \frac{n}{x^{2}} (\ln
x)^{n-2}[-\ln x + n-1]$.
\item \noindent

  \begin{center}
    \begin{tikzpicture}[scale=.8, transform shape] %
      \tkzTabInit[lgt=4,espcl=4] %
      {$x$ / 0.8, %
        signe de $f_{1}{''}(x)$ /1, %
        variations de $f_{1}{'}$ /2, %
        signe de $f_{1}{'}(x)$ /1, %
        variations de $f_{1}$ /2} %
      {$0$, $1$, $ + \infty$} %
      \tkzTabLine{d,-,t,-, } %
      \tkzTabVar{D+/, R, -/} %
      \tkzTabIma{1}{3}{2}{0} %
      \tkzTabLine{d, +, z, -, } %
      \tkzTabVar{D-/, + /$-1$, -/} %
    \end{tikzpicture} 
  \end{center}

%%% LALALA
 \vspace{0.3cm}
 \item $\lim \limits_{ x \to 0} f_{2} = -\infty$ et $f_{2}(1) = -1$
donc d'après le théorème des valeurs intermédiaires ($f_{2}$ est
continue) il existe $a \in \ ]0,1[$ tel que $f_{2}(a) = 0$.
 \end{noliste}
 \item
 \begin{noliste}{a)}
 \setlength{\itemsep}{2mm}
 \item $\frac{n}{x^{2}} (\ln x)^{n-2}$ est strictement positif sur $]1,
+ \infty[$ donc $f_{n}{''}(x)$ est du signe de $-\ln x + n-1$.\\
 $-\ln(x) + n-1 \geq 0 \Leftrightarrow \ln x \leq n-1 \leftrightarrow x
\leq e^{n-1}$ par croissance de exp.
 \begin{center} \begin{tikzpicture}
 \tkzTabInit{$x$/0.8,signe de $f_{n}{''}(x)$/1,variations de
$f_{n}{'}$/2}{$1$, $e^{n-1}$, $ + \infty$}
 \tkzTabLine{, +,z,-, }
 \tkzTabVar{-/ $-1$, + / $>0$, -/ $-1$}
 \tkzTabVal{1}{2}{0.5}{$\alpha_{n}$}{0}
 \tkzTabVal{2}{3}{0.5}{$\beta_{n}$}{0}
 \end{tikzpicture} \end{center} 
 On a $f_{n}{'}(e^{n-1}) >0$, en effet : $f_{n}{'}(e^{n-1}) = n
\left(\frac{n-1}{e} \right)^{n-1}-1$ avec : si $n \geq 4$, $n-1 \geq e$
donc $\frac{n-1}{e} >1$ donc $n \frac{n-1}{e} >1$ d'où
$f_{n}{'}(e^{n-1}) >0$ et si $n = 3$, $f_{3}{'}(e^{2}) =
3\frac{2^{2}}{e^{2}}-1 = \frac{12}{e^{2}}-1 >0$. \\
 On cherche maintenant à étudier le signe de $f_{n}'$ : \\
 $f_{n}{'}$ est continue et strictement croissante sur $]1, e^{n-1}]$
donc définit une bijection de $]1, e^{n-1}]$ dans $]-1, f_{n}(e^{n-1})]
$ qui contient 0 donc il existe un unique $\alpha_{n} \in \ ]1,
e^{n-1}[$ tel que $f_{n}{'}(\alpha_{n}) = 0$.\\
 De même, il existe un unique $\beta_{n}$ sur $]e^{n-1}, + \infty[$ tel
que $f_{n}{'}(\beta_{n}) = 0$.\\
 \begin{center} \begin{tikzpicture}
 \tkzTabInit{$x$/0.8,signe de $f_{n}{'}(x)$/1,variations de
$f_{n}$/2}{$1$, $\alpha_{n}$, $\beta_{n}$, $ + \infty$}
 \tkzTabLine{,-,z, +,z,-, }
 \tkzTabVar{+ / $-1$, -/ $<-1$, + / $>0$, -/ $-\infty$}
 \tkzTabVal{2}{3}{0.5}{$u_{n}$}{0}
 \tkzTabVal{3}{4}{0.5}{$v_{n}$}{0}
 \tkzTabVal{2}{3}{0.75}{$e^{n-1}$}{}
 \end{tikzpicture} \end{center} 
 En effet, $f_{n}(\beta_{n}) > f_{n}(e^{n-1})$ avec $f_{n}(e^{n-1}) =
(n-1)^{n}-e^{n-1} > (n-1)^{n-1}-e^{n-1} >0$ si $n \geq 4 $ car $n-1 >e$
et la fonction puissance est strictement croissante sur $R_{+}$. Et on
vérifie que $f_{3}(e^{2}) >0$.\\
\\
 $f_{n}$ est décroissante donc majorée par $-1$ sur $]1, \alpha_{n}]$
donc ne s'annule pas sur cet intervalle.\\
 Par deux théorèmes de la bijection sur $[ \ \alpha_{n}, \beta_{n}]$ et
$]\beta_{n}, + \infty[$ on démontre l'existence de deux racines $u_{n}$
et $v_{n}$ sur $]1, + \infty[$.
 \item $v_{n} > \beta_{n} > e^{n-1}$ or $\lim \limits_{n \to + \infty}
e^{n-1} = + \infty$ donc par comparaison, $\lim \limits_{n \to +
\infty} v_{n} = + \infty$.
 \end{noliste}
 \item Cette question est une vraie question de recherche. Il faut
utiliser les méthodes habituelles d'étude des suites implicites : \\
 $\bullet$ \textbf{On étudie les variations de la suite $u_{n}$ en
comparant $f_{n}(u_{n})$ et $f_{n}(u_{n + 1})$ :} \\
 $f_{n}(u_{n + 1}) = ( \ln u_{n + 1})^{n}-u_{n + 1} $ or $f_{n +
1}(u_{n + 1}) = 0$ donc $u_{n + 1} = (\ln u_{n + 1})^{n + 1}$ d'où : \\
 $f_{n}(u_{n + 1}) = ( \ln u_{n + 1})^{n}- ( \ln u_{n + 1})^{n + 1} = (
\ln u_{n + 1})^{n}[ 1-\ln(u_{n + 1})]$. Avec $u_{n + 1} \geq 1$ donc
$\ln(u_{n + 1}) \geq 0$.\\
 Etudions le signe de $1-\ln u_{n + 1} $ : pour cela, comparons $u_{n +
1}$ et $e$ : \\
 $\forall n \in \N$, $f_{n}(e) = ( \ln e)^{n} - e = 1-e<0 $ donc
d'après le tableau de variations de $f_{n}$, $e \in \ ]1, u_{n}[$ ou $e
\in \ ]v_{n}, + \infty[$ or $v_{n} > e^{n-1} > e$ donc $\forall n \in
\N, \ e < u_{n}$.\\
 Ainsi, $e < u_{n + 1}$ donc $1-\ln u_{n + 1} <0$ \\
 Donc $f_{n}(u_{n + 1}) <0 = f_{n}(u_{n})$, ainsi $u_{n + 1} \in \
]-\infty, u_{n}[$. 
 

%%% LALALA
\[
 \boxed{\text{La suite $(u_{n})$ est décroissante}}
\]
 $\bullet$ \textbf{On conclut que la suite $(u_{n})$ converge vers un
certain réel l } : \\
 $(u_{n})$ est décroissante et minorée par 1 donc converge vers un réel
$l \geq 1$ \\
 $\bullet$ \textbf{On trouve $l$ en passant à la limite dans la
relation $f_{n}(u_{n}) = 0$} : \\
 $u_{n} = (\ln u_{n})^{n}$ : $u_{n}{\frac{1}{n}} = \ln u_{n}$ donc
$e^{\frac{1}{n} \ln u_{n}} = \ln u_{n}$. \\
 avec $\ln u_{n} \to \ln l$ donc $\frac{1}{n} \ln u_{n} \to 0$ donc
$e^{\frac{1}{n} \ln u_{n}} \to 1$ \\
En passant à la limite on obtient donc $1 = \ln l$ c-a-d $l = e$.\\
\\
 \end{noliste}
 \textbf{\underline{Exercice sans préparation}} \\
\\
 \begin{noliste}{1.}
 \setlength{\itemsep}{4mm}
 \item
 \begin{noliste}{a)}
 \setlength{\itemsep}{2mm}
 \item $cov(Y_{k}, Y_{k + 1}) = \frac{1}{2} \left(\V(Y_{k} + Y_{k +
1})-\V(Y_{k})-\V(Y_{k + 1}) \right)$.\\
 Or $\V(Y_{k} + Y_{k + 1}) = \V(X_{k} + 2X_{k + 1} + X_{k + 2}) =
\V(X_{k}) + 4\V(X_{k + 1}) + \V(X_{k + 2})$ par indépendance des
$X_{k}$; \\
 $\V(Y_{k} + Y_{k + 1}) = 6\V(X_{1})$. et, de même, $\V(Y_{k}) =
2\V(X_{1}) = \V(Y_{k + 1})$ d'où : \\
 $cov(Y_{k}, Y_{k + 1}) = \V(X_{1}) = pq$. 
 \item On étudie $p \to p(1-p)$ sur $]0,1[$. 
 \end{noliste}
 \item si $ l \geq k + 2$ alors $Y_{k}$ et $Y_{l}$ sont fonctions de
 variables $X_{n}$ distinctes : c'est variables étant indépendantes,
 $Y_{k}$ et $Y_{l}$ sont indépendantes donc $cov(Y_{k}, Y_{l}) = 0$ si
 $l = k + 1$ alors $cov(Y_{k}, Y_{l}) = pq$ d'après 1.a).
 \item On note $Y = \frac{1}{n} \sum \limits_{k = 1}{n} Y_{k}$ alors
 d'après l'inégalité de Bienaymé-Tchebychev : $P\left(\Ev{|Y-\E(Y)| >
 \varepsilon}\right)\left(\Ev{Y}\right)| >
 \varepsilon) \leq \frac{\V(Y)}{ \varepsilon^{2}}$ c-a-d $\mathcal{P}
 \left( \left[ \ \left| \frac{1}{n} \sum \limits_{k = 1}{n} Y_{k}-2p
 \right|> \varepsilon\right] \right) \leq V \left( \frac{1}{n}
 \sum \limits_{k = 1}{n} Y_{k} \right) / \varepsilon^{2}$.\\
 Avec $\V\left( \frac{1}{n} \sum \limits_{k = 1}{n} Y_{k} \right) = 
 \frac{1}{n^{2}} V \left(\sum \limits_{k = 1}{n} Y_{k}
 \right) = \frac{1}{n^{2}} [ \ \sum \limits_{k = 1}{n} \V(Y_{k}) + 2
\sum
 \limits_{k = 1}{n} \sum \limits_{l = k + 1}{n} cov(Y_{k},
 Y_{l})] = \frac{1}{n^{2}} [ \ \sum \limits_{k = 1}{n} \V(Y_{k}) + 2
\sum
 \limits_{k = 1}{n-1} cov(Y_{k}, Y_{k + 1})] = \frac{4n-2}{n^{2}}
\V(X_{1})
 \to 0$.
 \end{noliste}
 \end{exercice}

 \begin{exercice}{\it (Exercice avec préparation)}~
 \begin{noliste}{1.}
 \setlength{\itemsep}{4mm}
 \item Soit $(A_{i})_{i \in I}$ un système complet d'évènements (c-a-d
deux à deux incompatibles et dont la réunion fait l'univers) \textbf{de
probabilités non nulles}, alors pour tout évènement $B$ : \\
 
\[
 \Prob\left(\Ev{B}\right) = \sum \limits_{i \in I} \Prob(B \cap A_{i})
= \sum \limits_{i \in I} P_{A_{i}}(B ) \Prob\left(\Ev{A_{i}}\right)
\]
 \item
 \begin{noliste}{a)}
 \setlength{\itemsep}{2mm}
\item $I_{n,0} = \dint{0}{1} x^{n} \dx = \frac{1}{n + 1}$.
 \item On pose $u$ et $v$ les fonctions définies sur $[0,1]$ par $u(x)
= \frac{x^{n + 1}}{n + 1}$ et $v(x) = (1-x)^{p + 1}$. $u$ et $v$ sont
de classe $\mathcal{C}{1}$ sur $[0,1]$ de dérivées, $u'(x) = x^{n}$ et
$v'(x) = -(p + 1)(1-x)^{p}$ \\
 Par intégration par parties, on obtient : \\
 
\[
\dint{0}{1} x^{n} (1-x)^{p + 1} \dx = [ \ \frac{x^{n + 1}}{n +
  1}(1-x)^{p + 1}]_{0}{1} + \dint{0}{1} \frac{x^{n + 1}}{n + 1}(p +
1)(1-x)^{p} \dx
\]
 soit : 
 
\[
 I_{n,p + 1} = \frac{p + 1}{n + 1} I_{n + 1,p}
\]
 \item On connait $I_{n,0}$, on part donc de $I_{n,p}$ et on se ramène
par récurrence à un $I_{m,0}$ : \\
 On démontrer par récurrence que $I_{n,p} = \frac{p! n!}{(n + p)!} I_{n
+ p,0} = \frac{p! n!}{(n + p + 1)!}$.
 \end{noliste}
 \item On note $U_{k}$ l'évènement "choisir l'urne $U_{k}$". Il existe
$\lambda$ tel que pour tout $k\in \llb 1,N \rrb$,
$\Prob\left(\Ev{U_{k}}\right) = \lambda k$ (probabilité proportionnelle
au nombre de boules rouges. \\
 Or $\sum \limits_{k = 1}{N} \Prob\left(\Ev{U_{k}}\right) = 1$ (les
$(U_{k})_{1 \leq k \leq N}$ forment un s.c.e) donc $\lambda \sum
\limits_{k = 1}{N} k = 1$, $\lambda = \frac{2}{N(N + 1)}$.
 
\[
 \boxed{\forall k \in \llb, 1,N \rrb, \ \Prob\left(\Ev{U_{k}}\right) =
\frac{2k}{N(N + 1)}}
\]
 \item
 \begin{noliste}{a)}
 \setlength{\itemsep}{2mm}
 \item La probabilité d'obtenir une boule rouge dépend de l'urne
choisie, on décompose donc sur le système complet d'évènements
$(U_{k})_{ 1 \leq k \leq N}$ : \\
 
\[
 \Prob\left(\Ev{E_{n}}\right) = \sum \limits_{k = 1}{N}
\Prob\left(\Ev{U_{k}}\right) P_{U_{k}}(E_{n})
\]
 Sachant que l'on tire dans l'urne $U_{k}$, Les tirages étant
indépendants, le nombre de boules rouges obtenues au cours de 2n
tirages suit une loi binomiale de paramètres 2n et $p = \frac{k}{N}$
donc $P_{U_{k}}(E_{n}) = \binom{2n}{n} \left(\frac{k}{N}
\frac{N-k}{N}\right)^{n}$. Ainsi, 
 
\[
 \Prob\left(\Ev{E_{n}}\right) = \sum \limits_{k = 1}{N} \frac{2k}{N(N +
1)} \binom{2n}{n}
 \frac{k^{n}(N-k)^{n}}{N^{2n}}
\]
 \item $P_{E_{n}}(R_{2n + 1}) = \frac{\Prob(E_{n} \cap R_{2n +
1})}{\Prob\left(\Ev{E_{n}}\right)}$ avec \\
 $\Prob(E_{n} \cap R_{2n + 1}) = \sum \limits_{k = 1}{N}
\Prob\left(\Ev{U_{k}}\right) P_{U_{k}}(E_{n}
 \cap R_{2n + 1}) = \sum \limits_{k = 1}{N}
\Prob\left(\Ev{U_{k}}\right) P_{U_{k}}(E_{n}
 )P_{U_{k}}(R_{2n + 1}) = \sum \limits_{k = 1}{N} \frac{2k}{N(N + 1)}
 \binom{2n}{n} \left(\frac{k}{N} \right)^{n} \left(\frac{N-k}{N}
 \right)^{n}\frac{k}{N}$.\\
 Ainsi, 
 
\[
 \Prob(E_{n} \cap R_{2n + 1}) = \frac{\sum \limits_{k = 1}{N} \left(
 \frac{k}{N}\right)^{n + 2} \left(1-\frac{k}{N} \right)^{n}
 }{\sum \limits_{k = 1}{N} \left( \frac{k}{N}\right)^{n + 1}
 \left(1-\frac{k}{N} \right)^{n}}
\]
 
\[
 \Prob(E_{n} \cap R_{2n + 1}) = \frac{\frac{1}{N}\sum \limits_{k =
1}{N}
 \left( \frac{k}{N}\right)^{n + 2} \left(1-\frac{k}{N} \right)^{n}
 }{\frac{1}{N}\sum \limits_{k = 1}{N} \left(
 \frac{k}{N}\right)^{n + 1} \left(1-\frac{k}{N} \right)^{n}}
\]
On reconnait les sommes de Riemann des fonctions $x \mapsto x^{n + 2}
(1-x)^{n}$ et $x \mapsto x^{n + 1}(1-x)^{n} $ continues sur
$[0,1]$ donc : \\
$\bullet$ $\frac{1}{N}\sum \limits_{k = 1}{N} \left(
  \frac{k}{N}\right)^{n + 2} \left(1-\frac{k}{N} \right)^{n} \to
\dint{0}{1} x^{n + 2} (1-x)^{n?} \dx$\\
$\bullet$ $\frac{1}{N}\sum \limits_{k = 1}{N} \left(
  \frac{k}{N}\right)^{n + 1} \left(1-\frac{k}{N} \right)^{n} \to
\dint{0}{1} x^{n + 1} (1-x)^{n?} \dx$\\
$\Prob(E_{n} \cap R_{2n + 1}) \to \frac{I_{n + 2,n}}{I_{n + 1,n}} =
\frac{n!(n + 2)!}{(2n + 3)!} \frac{(2n + 2)!}{n!(n + 1)!} = \frac{n +
  2}{2n + 3}$.
 \end{noliste}
 \item
 \end{noliste}
 \textbf{\underline{Exercice sans préparation}} \\
\\
 \begin{noliste}{1.}
 \setlength{\itemsep}{4mm}
 \item On donne la base canonique de $\M{2}$ : $\left( \begin{smatrix}
1 & 0 \\
0 & 0
\end{smatrix}, \begin{smatrix}
0 & 1 \\
0 & 0
\end{smatrix}, \begin{smatrix}
0 & 0 \\
1 & 0
\end{smatrix}, \begin{smatrix}
0 & 0 \\
0 & 1
\end{smatrix}
\right)$
 \item on prend des matrices inversibles (colonnes non colinéaires) les
plus simples possibles en faisant attention qu'elles forment une
famille libre : $\left( \begin{smatrix}
1 & 0 \\
0 & 1
\end{smatrix}, \begin{smatrix}
1 & 0 \\
0 & -1
\end{smatrix}, \begin{smatrix}
0 & 1 \\
1 & 0
\end{smatrix}, \begin{smatrix}
0 & 1 \\
-1 & 0
\end{smatrix}
\right)$
 \item On prend une base des matrices symétriques : $\left(
\begin{smatrix}
1 & 0 \\
0 & 0
\end{smatrix}, \begin{smatrix}
0 & 1 \\
1 & 0
\end{smatrix}, \begin{smatrix}
0 & 0 \\
0 & 1
\end{smatrix}
\right)$. On la complète en une base en prenant une matrice
triangulaire à deux valeurs propres distinctes, exemple : $
\begin{smatrix}
1 & 1 \\
0 & 2
\end{smatrix}
$.
 \end{noliste}
 \end{exercice}


\newpage


 \begin{exercice}{\it (Exercice avec préparation)}~
 \begin{noliste}{1.}
 \setlength{\itemsep}{4mm}
 \item Un estimateur $T$ de $\theta$ admettant une espérance est dit
sans biais si $\E(T) = \theta$. \\
 Si $T$ admet un moment d'ordre 2, son risque quadratique est défini
par $r(T) = E((T- \theta)^{2})$.
 \item $X$ est une variable bornée donc admet des moments de tous
ordres\\
 $\E(X) = \dint{-\infty}{+ \infty} x f(x \dx = \dint{0}{\theta}
2\frac{x^{2}}{\theta^{2}}dx = \frac{2}{3} \theta$.\\
 $\E(X^{2}) = \dint{-\infty}{+ \infty} x^{2}f(x)dx = \dint{0}{\theta}
2\frac{x^{3}}{\theta^{2}}dx = \frac{1}{2} \theta^{2}$. Ainsi, d'après
Koenig-Huygens, $\V(X) = \E(X^{2})-(\E(X))^{2} = \frac{1}{18}
\theta^{2}$. 
 \item 
 \begin{noliste}{a)}
 \setlength{\itemsep}{2mm}
 \item $\forall x \in \R$, $F(x) = \Prob\left(\Ev{\Ev{X \leq x
}}\right) = \dint{-\infty}{x} f(t)dt$.\\
 $\bullet$ si $x < 0$, $F(x) = 0$ \\
 $\bullet$ si $ 0 \leq x \leq \theta$, $F(x) = \dint{0}{x} f(t)dt =
\frac{x^{2}}{\theta^{2}}$.\\
 $\bullet$ si $x > \theta$, $F(x) = 1$ 
 \item
 \end{noliste}
 \item \begin{noliste}{a)}
 \setlength{\itemsep}{2mm}
 \item $\E(\overline{X_{n}}) = \frac{1}{n} \Sum{k = 1}{n} \E(X_{k}) =
\E(X) = \frac{2}{3} \theta$ (linéarité de l'espérance). \\
 Soit alors $T_{n} = \frac{3}{2} \overline{X_{n}}$. Par linéarité de
l'espérance, $T_{n}$ est un estimateur sans biais de $\theta$.
 \item $r(T_{n}) = \V(T_{n}) = \frac{c^{2}}{n^{2}} \V(\sum \limits_{k =
1}{n} X_{k}) = \frac{c^{2}}{n^{2}} \sum \limits_{k = 1}{n} \V(X_{k}) =
\frac{c^{2}}{n} \V(X)$ par indépendance des $X_{k}$. \\
 $r(T_{n}) = \frac{\theta^{2}}{8n} $.\\
 $r(\overline{X_{n}}) = \V(\overline{X_{n}}) + \left(
b(\overline{X_{n}})\right)^{2} = \frac{(2n + 1) \theta^{2}}{18n} $ 
 \end{noliste}
 \item \begin{noliste}{a)}
 \setlength{\itemsep}{2mm}
 \item $\forall x \in \R$, $G_{n}(x) = \Prob\left(\Ev{\Ev{M_{n} \leq x
}}\right) = P( \Ev{X_{1} \leq x } \cap \Ev{X_{2} \leq x} \cap... \cap
X_{n} \leq x) = \Prob\left(\Ev{\Ev{X_{1} \leq x}}\right)
\Prob\left(\Ev{\Ev{X_{2} \leq x }}\right)...\Prob\left(\Ev{\Ev{X_{n}
\leq x}}\right) = \Prob\left(\Ev{\Ev{X \leq x}}\right)^{n} = \left\{
\begin{array}{lr}
 0 & \text{ si } x < 0\\
\frac{x^{2n}}{\theta^{2n}} & \text{ si } 0 \leq x \leq \theta\\
1 & \text{ si } x > \theta
\end{array}
\right.$.\\
 $G_{n}$ est de classe $\mathcal{C}_{1}$ sur $\R$ sauf en $0$ et
$\theta$, on vérifie aisément qu'elle est continue en 0 et $\theta$
donc $M_{n}$ est une variable à densité de densité : $g_{n}(x) =
\left\{
\begin{array}{lr}
 0 & \text{ si } x < 0\\
\frac{2n x^{2n-1}}{\theta^{2n}} & \text{ si } 0 \leq x \leq \theta\\
0 & \text{ si } x > \theta
\end{array}
\right.$
 \item $M_{n}$ est une variable finie donc admet des moments de tous
ordres : \\
 $\E(M_{n}) = \dint{0}{\theta} 2n x^{2n}{\theta^{2n}}dx = \frac{2n}{2n
+ 1} \theta$. On choisit donc $W_{n} = \frac{2n + 1}{2n} M_{n}$ comme
estimateur sans biais de $\theta$
 \item Comparons leur risque quadratique : calculer $r(W_{n}) =
\V(W_{n}) = \frac{(2n + 1)^{2}}{(2n)^{2}} \V(M_{n})$ avec,\\
 $\E(M_{n}{2}) = \frac{n}{n + 1} \theta^{2}$ donc $\V(M_{n}) =
\frac{n}{(n + 1)(2n + 1)^{2}} \theta^{2}$ d'où $r(W_{n}) =
\frac{\theta^{2}}{4n(n + 1)}$. \\
 $W_{n}$ est meilleur estimateur que $T_{n}$ car son risque quadratique
est en $\frac{1}{n^{2}}$ au voisinage de $ + \infty$ alors que celui de
$T_{n}$ est en $\frac{1}{n}$ (son risque quadratique tend plus vite
vers 0).
 \end{noliste}
 \item
 \begin{noliste}{a)}
 \setlength{\itemsep}{2mm}
 \item On résout l'équation $G_{n}(a \theta) = \frac{\alpha}{2}
\Leftrightarrow a = \left( \frac{\alpha}{2}\right)^{\frac{1}{2n}}$ et
on résout l'équation $G_{n}(\theta)- G_{n}(b \theta) = \frac{\alpha}{2}
\leftrightarrow b = \left( 1-\frac{\alpha}{2}\right)^{\frac{1}{2n}}$ 
 \item On isole $\theta$ dans les inégalité en renversant ces
inégalités : \\
 On a $\Prob\left(\Ev{M_{n} \leq a \theta}\right) = \frac{\alpha}{2}
\Leftrightarrow \Prob\left(\Ev{ \theta \geq \frac{M_{n}}{a}}\right) =
\frac{\alpha}{2}$. \\
 $\Prob\left(\Ev{ b \theta \leq M_{n}}\right) = \Prob\left(\Ev{ b
\theta \leq M_{n} \leq \theta}\right) = \frac{\alpha}{2}$ donc
$\Prob\left(\Ev{ \theta \leq \frac{M_{n}}{b}}\right) =
\frac{\alpha}{2}$.\\
 Ainsi, $\Prob\left(\Ev{ \frac{M_{n}}{b} \leq \theta \leq
\frac{M_{n}}{a}}\right) = 1- P(\left(\Ev{\theta \leq \frac{M_{n}}{b}
}\right) \cup (\theta \geq \frac{M_{n}}{a}) ) = 1-\Prob\left(\Ev{\theta
\leq \frac{M_{n}}{b} }\right)- P \left(\Ev{\theta \geq
\frac{M_{n}}{a}}\right) = 1-\alpha.$
 \end{noliste}
 \end{noliste}
 \textbf{\underline{Exercice sans préparation}} \\
\\
 \begin{noliste}{1.}
 \setlength{\itemsep}{4mm}
 \item $A(A^{2} + A + I) = 0$. Si $A$ est inversible, on peut
simplifier par $A$ en multipliant par $A^{-1}$ à gauche, on obtient :
$A^{2} + A + I = 0 $ soit $A(-A-I) = (-A-I)A = I$ donc $A^{-1} = -A-I$.
 \item Si $A$ est symétrique alors elle est diagonalisable c'est-à-dire
il existe $P$ inversible et $D$ diagonale tels que $A = PDP^{-1}$ où
$D$ a sur sa diagonale des valeurs propres de $A$.\\
 Or $X^{3} + X^{2} + X$ est un polynôme annulateur de $A$ donc si
$\lambda$ est valeur propre de $A$ alors $\lambda^{3} + \lambda^{2} +
\lambda = 0$ donc $\lambda( \lambda^{2} + \lambda + 1) = 0$ or
l'équation de degré 2 n'a pas de solutions donc $\lambda = 0$ donc $D =
0$ d'où $A = P0P^{-1} = 0$.
 \end{noliste}
 \end{exercice}

 \begin{exercice}{\it (Exercice avec préparation)}~
 \begin{noliste}{1.}
 \setlength{\itemsep}{4mm}
 \item $X$ et $Y$ sont indépendantes si $\forall i \in X(\Omega)$,
$\forall j \in Y(\Omega)$, $\Prob([X = i] \cap [Y = j]) =
\Prob\left(\Ev{\Ev{[X = i]}}\right) \Prob\left(\Ev{\Ev{[Y =
j]}}\right)$.
 \item \begin{noliste}{a)}
 \setlength{\itemsep}{2mm}
 \item faire un graphique pour représenter les différents déplacements
possibles. On obtient : \\
 $T_{n}(\Omega) = \{-1,0,1\}$, $\Prob\left(\Ev{\Ev{[T_{n} =
-1]}}\right) = \Prob\left(\Ev{\Ev{[T_{n} = 1]}}\right) = \frac{1}{4}$
et $\Prob\left(\Ev{\Ev{[T_{n} = 0]}}\right) = \frac{1}{2} $
(équiprobabilité des quatre déplacements)
 $\E(T_{n}) = -1 \Prob\left(\Ev{\Ev{[T_{n} = -1]}}\right) +
0\Prob\left(\Ev{\Ev{[T_{n} = 0]}}\right) + 1\Prob\left(\Ev{\Ev{[T_{n} =
1]}}\right) = 0$.\\
 $\E(T_{n}{2}) = (-1)^{2} \Prob\left(\Ev{\Ev{ [T_{n} = -1]}}\right) +
1^{2} \Prob\left(\Ev{\Ev{[T_{n} = 1]}}\right) = \frac{1}{2}$ donc par
K-H $\V(T_{n}) = \E(T_{n}{2}) = \frac{1}{2}$. 
 \item $\sum \limits_{k = 1}{n} T_{k} = \sum \limits_{k = 1}{n}
(X_{k}-X_{k-1}) = X_{n}-X_{0} = X_{n}$ (télescopage).
 \item $\E(T_{k}) = -1 \Prob\left(\Ev{\Ev{[T_{k} = -1]}}\right) +
0\Prob\left(\Ev{\Ev{[T_{k} = 0]}}\right) + 1\Prob\left(\Ev{\Ev{[T_{k} =
1]}}\right) = 0$ et donc par linéarité de l'espérance $\E(X_{n}) = \sum
\limits_{k = 1}{n} \E(T_{k}) = 0$.
 \item Commençons par calculer $\V(X_{n})$ facile à calculer car somme
de variables indépendantes donc $\V(X_{n}) = \sum \limits_{k = 1}{n}
\V(T_{k}) = \frac{n}{2}$. Et par K-H : $\E(x_{n}{2}) = \V(X_{n}) +
\E(X_{n})^{2}$ donc :
 
\[
 \E(X_{n}{2}) = \frac{n}{2}
\]
 \end{noliste}
 \item \begin{noliste}{a)}
 \setlength{\itemsep}{2mm}
 \item $\Prob([X_{n} = n] \cap [Y_{n} = n]) = 0$ car à chaque
déplacement, l'abscisse et l'ordonnée ne se modifient pas simultanément
donc il faudrait 2n déplacements et non n déplacements pour avoir cette
configuration.
 \item Il faut utiliser l'inégalité $\E(Z_{n})^{2} \leq \E(Z_{n}{2})$
vraie pour toute variable aléatoire : en effet, $\V(Z_{n}) \geq 0 $
donc $\E(Z_{n}{2})-\E(Z_{n})^{2} \geq0$. \\
 Ainsi, $\E(Z_{n})^{2} \leq \E(Z_{n}{2})$ avec $Z_{n}{2} = X_{n}{2} +
Y_{n}{2}$ donc par linéarité de l'espérance $\E(Z_{n}{2}) =
\E(X_{n}{2}) + \E(Y_{n}{2}) = n$ par symétrie de $X_{n}$ et $Y_{n}$. \\
 Ainsi, $\E(Z_{n})^{2} \leq n $ donc $\E(Z_{n}) \leq \sqrt{n}$ ($Z_{n}$
distance donc positive )
 \end{noliste}
 \item \begin{noliste}{a)}
 \setlength{\itemsep}{2mm}
 \item Soit $k$ le nombre total de déplacements à l'est et $l$ le
nombre total de déplacements au nord. Pour revenir à l'origine la puce
doit faire $k$ déplacements à l'ouest et $l$ déplacements au sud soit
au total $2(k + l)$ déplacements (nombre pair de déplacements) ainsi,
si $n$ impair, $p_{n} = 0$ 
 \item D'après l'explication précédente, si la puce fait $2m$
déplacements alors la somme des déplacements à l'ouest et au nord vaut
$m$. \\
 Notons $N$ le nombre de déplacements au nord, $E$ le nombre de
déplacements à l'est, $O$ le nombre de déplacements à l'ouest et $S$ le
nombre de déplacements au sud.\\
 $\Ev{[N = k]}_{0 \leq k \leq m}$ est un système complet d'évènements
donc d'après la formule des probas totales : 
 
\[
 \Prob\left(\Ev{\Ev{M_{n} = 0}}\right) = \sum \limits_{k = 0}{n}
\Prob([M_{n} = 0] \cap [N = k]) = \sum \limits_{k = 0}{n} \Prob( [N =
k]\cap [S = k] \cap [O = m-k] \cap [E = m-k]) \text{ (d' après
l'explication précédente.)}
\]
 \underline{ Calculons $\Prob( [N = k]\cap [S = k] \cap [O = m-k] \cap
[E = m-k])$} :\\
 -on choisit la place des déplacements vers le nord au cours de $2m$
déplacements : on a $\binom{2m}{k}$ choix.\\
 - ces emplacements étant choisis, on choisit la place des déplacements
vers l'est parmi les places restantes : on a $\binom{2m-k}{m-k}$
choix.\\
 - on choisit l'emplacement des déplacements vers le sud parmi les
places restantes : $\binom{m}{k}$ choix. et il ne reste plus de choix
pour les déplacements vers l'est. \\
 La probabilité de chacun de ces déplacements possibles, par
indépendance des déplacements est $\frac{1}{4^{2m}}$, on a donc au
final : 
 
\[
 \Prob( [N = k]\cap [S = k] \cap [O = m-k] \cap [E = m-k]) =
\binom{2m}{k} \binom{2m-k}{m-k} \binom{m}{k} \frac{1}{4^{2m}}
\]
 Or on montre facilement que $\binom{2m}{k} \binom{2m-k}{m-k} =
\binom{2m}{m} \binom{m}{k}$ donc 
 
\[
 \Prob( [N = k]\cap [S = k] \cap [O = m-k] \cap [E = m-k]) =
\binom{2m}{m} \binom{m}{k}{2} \frac{1}{4^{2m}}
\]
 Ainsi, $\Prob\left(\Ev{\Ev{M_{n} = 0}}\right) = \binom{2m}{m}
\frac{1}{4^{2m}} \sum \limits_{k = 0}{n} \binom{m}{k}{2}$.
 \end{noliste}
 \end{noliste}
 \textbf{\underline{Exercice sans préparation}} \\
\\
 \begin{noliste}{1.}
 \setlength{\itemsep}{4mm}
 \item On se sait rien faire lorsque le premier indice tend vers
l'infini, on va donc se ramener à ce que l'on connait (les séries) : \\
 Soit $M >n$, $\sum \limits_{k = n}{M} \frac{1}{k^{3}} = \sum
\limits_{k = 1}{M} \frac{1}{k^{3}}-\sum \limits_{k = 1}{n}
\frac{1}{k^{3}}$ or la première somme converge lorsque $M$ tend vers $
+ \infty$ comme série de Riemann avec $\alpha = 3 >1$ donc $v_{n}$
existe bien et vaut : 
 
\[
 v_{n} = \sum \limits_{k = n}{+ \infty} \frac{1}{k^{3}} = \sum
\limits_{k = 1}{+ \infty} \frac{1}{k^{3}}-\sum \limits_{k = 1}{n}
\frac{1}{k^{3}}
\]
 On fait tendre $n$ vers $ + \infty$ dans la seconde expression, on
obtient : 
 
\[
 \lim \limits_{n \to + \infty} v_{n} = \sum \limits_{k = 1}{+ \infty}
\frac{1}{k^{3}}-\sum \limits_{k = 1}{+ \infty} \frac{1}{k^{3}} = 0 
\]
 \item On est clairement dans un exercice de comparaison
série-intégrale.
 \begin{noliste}{a)}
 \setlength{\itemsep}{2mm}
 \item La fonction $x \mapsto \frac{1}{x^{3}}$ est décroissante sur
$\R_{+}^*$ donc $\forall k \geq 1$, soit $ x \in [k, k + 1]$, alors :
$\frac{1}{(k + 1)^{3}}\leq \frac{1}{x^{3}} \leq \frac{1}{k^{3}}$. La
fonction étant de plus continue, on peut intégrer l'inégalité sur $[k,
k + 1]$. (bornes croissantes ) :
 
\[
 \dint{k}{k + 1} \frac{1}{(k + 1)^{3}}dx\leq \dint{k}{k +
1}\frac{1}{x^{3}}dx \leq\dint{k}{k + 1} \frac{1}{k^{3}}dx 
\]
 
\[
 \frac{1}{(k + 1)^{3}} \leq \dint{k}{k + 1}\frac{1}{x^{3}}dx \leq
\frac{1}{k^{3}}
\]
 On somme alors l'encadrement pour $k$ allant de $n$ à $n + m$ et on
obtient le résultat.
 \item Encadrons $v_{n}$ grâce à la question précédente : 
 $\bullet$ on a $\sum \limits_{k = n}{n + m} \frac{1}{k^{3}} \geq
\dint{n}{n + m + 1}\frac{1}{x^{3}}dx = \frac{1}{2n^{2}}-\frac{1}{2(n +
m + 1)^{2}}$, on fait tendre $m$ vers $ + \infty$, on obtient : 
 
\[
 v_{n} \geq \frac{1}{2n^{2}}
\]
 $\bullet$ La seconde inégalité est : $ \sum \limits_{k = n}{n + m}
\frac{1}{(k + 1)^{3}} \leq \dint{n}{n + m + 1}\frac{1}{x^{3}}dx$ or par
changement d'indice on a : 
 $ \sum \limits_{k = n}{n + m} \frac{1}{(k + 1)^{3}} = \sum \limits_{k
= n + 1}{n + m + 1} \frac{1}{k^{3}} = \sum \limits_{k = n}{n + m + 1}
\frac{1}{k^{3}} - \frac{1}{n^{3}}$.\\
 On passe alors à la limite dans l'inégalité, on obtient : 
 
\[
 v_{n} - \frac{1}{n^{3}} \leq \frac{1}{2n^{2}} \Leftrightarrow v_{n}
\leq \frac{1}{2n^{2}} + \frac{1}{n^{3}} 
\]
 D'où 
 
\[
 \frac{1}{2n^{2}} \leq v_{n} \leq \frac{1}{2n^{2}} + \frac{1}{n^{3}}
\]
 Grâce à cet encadrement, on montre par le théorème d'encadrement que
$\frac{v_{n}}{\frac{1}{2n^{2}}} \to 1$ donc $v_{n} \sim
\frac{1}{2n^{2}}$. 
 \end{noliste}
 \end{noliste}
 \end{exercice}

 \begin{exercice}{\it (Exercice avec préparation)}~
 \begin{noliste}{1.}
 \setlength{\itemsep}{4mm}
 \item Deux matrices $A$ et $B$ de $\M{n}$ sont semblables s'il existe
une matrice $P$ de $\M{n}$ inversible telle que $A = PBP^{-1}$. 
 \item
 \begin{noliste}{a)}
 \setlength{\itemsep}{2mm}
 \item $f^{2}(i) = f (f(i)) = f(i-j + k) = f(i)-f(j) + f(k) = 2j-2k$
(linéarité de $f$) donc $(f^{2}-2f + 2Id)(i) = f^{2}(i)-2f(i) + 2i = 0$
\\
 ainsi, $(2Id-f)((f^{2}-2f + 2Id)(i)) = (2Id-f)(0) = 0$ par linéarité
de $2Id-f$. \\
 De même, on trouve : $(f^{2}-2f + 2Id)(j) = i + j + k$ et
$(2Id-f)((f^{2}-2f + 2Id)(j)) = (2Id-f)(i + j + k) = 0$ et $(f^{2}-2f +
2Id)(k) = i + j + k$ et $(2Id-f)((f^{2}-2f + 2Id)(k)) = (2Id-f)(i + j +
k) = 0$.
 \item Montrons que $f$ est surjective $\Leftrightarrow Im(f) =
\R^{3}$.\\
 $Im(f) = Vect(f(i), f(j), f(k)) = Vect(i-j + k, i + 2j, j + k)$.
Vérifions la liberté de la famille $(i-j + k, i + 2j, j + k)$ : \\
 $a(i-j + k) + b(i + 2j) + c(j + k) = 0 \Leftrightarrow (a + b)i + (-a
+ 2b + c)j + (a + c)k = 0$ $\Leftrightarrow \left\{
\begin{array}{ccccc}
 a + & b & & = & 0\\
-a + & 2b + & c & = & 0\\
 & -b + & c & = & 0
\end{array}
\right.$ car $(i,j,k)$ est une base de $E$ donc une famille libre.\\
 On résout le système par pivots, on obtient :$ a = b = c = 0$. La
famille est libre et génératrice de $Im(f)$ donc c'est une base de
$Im(f)$ et $dim(Im(f) = 3$.\\
 $Im(f)$ est un sous-espace de $E$ de dimension 3 donc $Im(f) = E$.\\
 $f$ est une endomorphisme de $E$ ( $E$ de dim finie) surjectif donc
bijectif.
 

%%% LALALA

\[
 \text{$f$ est un automorphisme de $E$}
\]
 \item La matrice de $f$ dans la base $(ij,k)$ est $A = \begin{smatrix}
1 & 1 & 0\\
-1 & 2 & 1\\
1 & 0 & 1
\end{smatrix}
$. 
 \item D'après la question 2.a) $(2-X)(X^{2}-2X + 2)$ est un polynôme
annulateur de $f$. Donc si $\lambda$ est valeur propre de $f$ alors
$(2-\lambda)(\lambda^{2}-2\lambda + 2) = 0$ donc $\lambda = 2$.\\
 On résout l'équation $(A-2I)X = 0$ on obtient $E_{2}(f) = Vect(i + j +
k)$. 
 \end{noliste}
 \item La somme des dimensions des sous-espaces propres de $f$ est
égale à 1 et non à 3 donc $f$ n'est pas diagonalisable.
 \begin{noliste}{a)}
 \setlength{\itemsep}{2mm}
 \item On résout l'équation $\alpha(-b,a,0) + \beta (0,c,-b) + \gamma
(-c,0,a) = (0,0,0)$ en distinguant les cas $b\neq 0$ et $b = 0$ (alors
$a$ ou $c \neq 0$), et on obtient dans tous les cas que la famille est
liée or les vecteurs ne sont pas colinéaires deux à deux donc
$Vect(U,V,W) = Vect(U,V)$ où $(U,V)$ est une base de $Vect(U,V,W)$. 
 \item On remarque que $U$, $V$ et $W$ appartiennent à $P$ car ils
vérifient l'équation de $P$ donc comme $P$ espace vectoriel alors
$Vect(U,V,W) \subset P$. On vérifie facilement en trouvant une base de
$P$ que $P$ est de dimension 2 donc : 
 
\[
 P = Vect(U,V,W)
\]
 \underline{Calculons $f(U)$, $f(V)$ et $f(W)$ :}\\
 $f(U) = f(-bi + aj) = -bf(i) + af(j) = -b(i-j + k) + a(i + 2j) =
(a-b)i + (b + 2a)j-bk$.\\
 $f(V) = ci + (2c-b)j-bk$ et $f(W) = -ci + (c + a)j + (a-c)k$. comme
$f(P)$ est un espace vectoriel alors $f(P) \subset P$ si et seulement
si $f(U)$, $f(V)$ et $f(W) \in P$, si et seulement si $\left\{ 
\begin{array}{c}
a(a-b) + b(b + 2a)-cb = 0\\
ac + b(2c-b)-cb = 0\\
-ac + b(c + a) + c(a-c) = 0
\end{array}
\right.$.\\
 Il reste à résoudre ce système d'équations (bonne chance...)
 \end{noliste}
 \end{noliste}
 \textbf{\underline{Exercice sans préparation}} \\
\\
 \begin{noliste}{1.}
 \setlength{\itemsep}{4mm}
 \item $x (1-\Phi(ax)) = x\Prob\left(\Ev{\Ev{X >ax}}\right) (clairement
positif) = x \dint{ax}{+ \infty} \frac{1}{\sqrt{2 \pi}} e^{-t^{2}}{2}dt
= \frac{1}{a} \dint{ax}{+ \infty} \frac{ax}{\sqrt{2 \pi}}
e^{-t^{2}}{2}dt \leq \frac{1}{a} \dint{ax}{+ \infty} \frac{t}{\sqrt{2
\pi}} e^{-t^{2}}{2}dt = \frac{1}{\sqrt{2 \pi a}}e^{-a^{2}x2/2} \leq
\sqrt{\frac{2}{\pi}} e^\frac{-ax^{2}}{2}$. 
 \item La densité d'une variable centrée de variance $\frac{1}{a}$ est
$f(x) = \sqrt{\frac{a}{2 \pi}}e^{-ax^{2}/2}$. Cette densité étant
paire, on a : \\
 
\[
 \dint{0}{+ \infty}\sqrt{\frac{a}{2 \pi}}e^{-ax^{2}/2}dx = \frac{1}{2}
\Leftrightarrow \dint{0}{+ \infty} \sqrt{\frac{2}{\pi}}
e^\frac{-ax^{2}}{2}dx = \frac{1}{\sqrt{a}} 
\]
 Ainsi, par croissance des bornes : \\
 
\[
 0 \leq \dint{0}{+ \infty}x (1-\Phi(ax))dx \leq \frac{1}{\sqrt{a}}
\]
 Et par le théorème d'encadrement, on obtient : $\lim \limits_{ a \to +
\infty} \dint{0}{+ \infty}x (1-\Phi(ax))dx = 0$. 
 \end{noliste}
 \end{exercice}

 \begin{exercice}{\it (Exercice avec préparation)}~
 \begin{noliste}{1.}
 \setlength{\itemsep}{4mm}
 \item Une fonction $f$ définie sur $\R$ est convexe si elle est au
 dessus de toutes ses tangentes.\\
 Caractérisation : une fonction $f$ définie sur $\R$ est convexe si et
seulement si elle est au dessous de toutes ses cordes.\\
 Si $f$ est de classe $\mathcal{C}{1}$ sur $\R$ alors elle est convexe
si et seulement si sa dérivée $f'$ est croissante.\\
 Si $f$ est de classe $\mathcal{C}{2}$ sur $\R$ alors elle est convexe
si et seulement si sa dérivée seconde $f''$ est positive.\\
 \item \begin{noliste}{a)}
 \setlength{\itemsep}{2mm}
 \item Si $x \geq 0$, $t \mapsto e^{t^{2}}$ est continue sur $[0,x]$
donc $\dint{0}{x} e^{t^{2}}dt$ existe; \\
 Si $x \leq 0$, $t \mapsto e^{t^{2}}$ est continue sur $[x,0]$ donc
$\dint{0}{x} e^{t^{2}}dt$ existe.
\item $g : t \mapsto e^{t^{2}}$ est continue sur $\R$ donc admet une
  primitive $G$ sur cet intervalle. Ainsi, $\forall x \in \R$, $f(x) =
  G(x)-G(0)$ or $G$ est de classe $\mathcal{C}{2}$ sur $\R$ en tant
  que primitive de $g$, fonction de classe $\mathcal{C}{1}$ sur $\R$
  donc $f$
  est aussi de classe $\mathcal{C}_{2}$ sur $\R$. \\
  $\bullet$ $\R$ est centré en 0.\\
  $\bullet$ Soit $x \in \R$, $f(-x) = \dint{0}{-x} e^{t^{2}} \ dt$. On
  pose le changement de variable $u = -t$ de classe $\mathcal{C}{1}$
  sur $[0,x]$ ($t \mapsto -t$ affine), alors $f(-x) = \dint{0}{x}
  e^{(-u)^{2}} -du = -\dint{0}{x} e^{u^{2}} du = -f(x)$ \\
 
\[
 \boxed{\text{$f$ est impaire}}
\].\\
 $\forall x \in \R$, $f'(x) = g(x) = e^{x^{2}}$ et $f''(x) = 2x
e^{x^{2}}$ est du signe de $2x$ car $\exp>0$ donc $f'' >0$ sur
$]-\infty, 0[$ et $f''>0$ sur $]0, + \infty[$. \\
 
\[
 \boxed{\text{$f$ est concave sur $ ]-\infty,0]$ et convexe sur $[0, +
\infty[$}}
\]
 \item $f' >0$ sur $\R$ donc $f$ est strictement croissante sur $\R$. 
 \end{noliste}
 \item \begin{noliste}{a)}
 \setlength{\itemsep}{2mm}
 \item si $t >1$ alors $t^{2} >t$ donc $\exp(t^{2}) > \exp(t)$ donc
$\dint{1}{x} e^{t^{2}}dt > \dint{1}{x} e_{t} = e^{x} - e \to + \infty $
lorsque $x \to + \infty$.\\
Ainsi, $\lim \limits_{x \to + \infty } f(x) = + \infty$ et par
imparité, $\lim \limits_{ x \to -\infty } f(x) = -\infty$.\\
 $f$ est continue et strictement croissante sur $\R$ donc réalise une
bijection de $\R$ dans $f(\R) = \R$ ainsi,pour tout $ n \in \N^*$, il
existe un unique réel $u_{n}$ tel que $f(u_{n}) = \frac{1}{n}$. 
 \item $f(u_{n}) = \frac{1}{n}$ et $f(u_{n + 1}) = \frac{1}{n + 1}$ or
$\frac{1}{n + 1} < \frac{1}{n}$ par stricte décroissante de la fonction
inverse sur $\R_{+}{*}$ donc $f(u_{n + 1}) < f(u_{n})$ d'où $u_{n + 1}
< u_{n}$ en composant par $f^{-1}$ strictement croissante.
 
\[
 \boxed{\text{$(u_{n})$ est décroissante}}
\]
 De plus, $f(x) < f(0) = 0$ si $x <0$ donc $u_{n} >0$. \\
 La suite $(u_{n})$ est décroissante et minorée (par 0) donc
convergente vers un certain réel $l$. 
 \item On passe à la limite lorsque $n \to + \infty$ dans l'égalité
$f(u_{n}) = \frac{1}{n}$. Par continuité de $f$ sur $\R$ donc en $l$,
on obtient $f(l) = 0$. Or l'unique antécédent par $f$ de 0 est 0 donc
$l = 0$.
 \end{noliste}
 \item \begin{noliste}{a)}
 \setlength{\itemsep}{2mm}
 \item La fonction $\exp$ est convexe sur $\R$ donc au dessus de sa
tangente en $0$ donc $\forall u\in \R$, $\exp(u) \geq 1 + u$. \\
 Pour démontrer l'autre inégalité, on étudie la fonction $u \mapsto 1 +
2u - e^{u}$ sur $[0, \ln(2)]$.
 \item Soit $t \in [0, \sqrt{\ln(2)}]$, alors $t^{2} \in [0, \ln(2)]$
donc en appliquant l'inégalité précédente pour $u = t^{2}$, on obtient
$ 1 + t^{2} \leq e^{t^{2}} \leq 1 + 2t^{2}$. \\

 Pour pouvoir alors intégrer l'encadrement sur $[0, u_{n}]$, il faut
que $u_{n} \leq \sqrt{\ln(2)}$ or $\lim \limits_{n \to + \infty } u_{n}
= 0$ donc en prenant la définition de la limite pour $ \epsilon =
\sqrt{\ln(2)}$, on sait qu'il existe $n_{0}$ tel que pour tout $n \geq
n_{0}$, $|u_{n}| = u_{n} \leq \sqrt{\ln(2)}$. \\
 Pour $n \geq n_{0}$, par croissance des bornes on obtient : 
 \[
 \dint{0}{u_{n}} (1 + t^{2})dt \leq \dint{0}{u_{n}} e^{t^{2}} \ dt =
 f(u_{n}) = \frac{1}{n} \leq \dint{0}{u_{n}} (1 + 2t^{2})dt
\]
 \item On calcule les intégrales de cette inégalité, on obtient :
$u_{n} + \frac{u_{n}{3}}{3} \leq \frac{1}{n} \leq u_{n} +
\frac{2u_{n}{3}}{3}$. \\
 c-a-d $nu_{n} + \frac{nu_{n}{3}}{3} \leq 1 \leq nu_{n} + \frac{2n
u_{n}{3}}{3}$.\\
 $\bullet$ En divisant par $n u_{n}$, on obtient déjà que $ n u_{n}
\sim 1$ car par le théorème d'encadrement $\lim \limits_{n \to +
\infty} \frac{1}{nu_{n}} = 1$ \\
 Ainsi, $\lim \limits_{n \to + \infty} nu_{n} = 1$. Au passage, on a
montré que $u_{n} \sim \frac{1}{n}$.\\
 On renverse alors l'encadrement pour encadrer $n u_{n}{3}$ : 
 
\[
 \frac{3}{2}(1-n u_{n}) \leq n u_{n}{3} \leq 3(1-n u_{n})
\]
 Par encadrement, on obtient $ n u_{n}{3} \to 0$.\\
\\
 \end{noliste}
 \end{noliste}
 \textbf{\underline{Exercice sans préparation}} \\
\\
 $\bullet$ $Y$ prend la valeur 0 et lorsque $X$ prend toutes les
valeurs paires supérieures à 2 alors $\frac{X}{2}$ prend toutes les
valeurs entières supérieures à 1 donc 
 
\[
 \boxed{Y(\Omega) = \N}
\]
$\bullet$ Soit $k \geq 1$, $\Prob\left(\Ev{\Ev{Y = k}}\right) =
\Prob\left(\Ev{ \frac{X}{2} = k}\right) = \Prob\left(\Ev{\Ev{X =
      2k}}\right) = p q^{2k-1}$.\\
$\Prob\left(\Ev{\Ev{Y = 0}}\right) = \Prob\left(\Ev{ \bigcup
    \limits_{k = 0}{+ \infty} [X = 2k + 1]}\right) = \sum \limits_{k =
  0}{+ \infty} \Prob\left(\Ev{\Ev{[X = 2k + 1]}}\right) $
(incompatibilité deux à deux des évènements de l'union) $ = \sum
\limits_{k = 0}{+
  \infty}pq^{2k} = \frac{p}{1-q^{2}} = \frac{1}{1 + q}$. \\
Pour tout $k \in \N$, $k\Prob\left(\Ev{\Ev{[Y = k]}}\right) \geq 0$
donc la série de terme général $k \Prob\left(\Ev{\Ev{[Y = k]}}\right)$
converge absolument si et seulement si elle converge. \\
Sous réserve de convergence, on a : \\
$\E(Y) = 0 \Prob\left(\Ev{\Ev{ [Y = 0]}}\right) + \sum \limits_{k =
  1}{+ \infty} k \Prob\left(\Ev{\Ev{[Y = k]}}\right) = \sum \limits_{k
  = 1}{+ \infty} k pq^{2k-1} = pq \sum \limits_{k = 1}{+ \infty}
k(q^{2})^{k-1}$. On reconnaît une série géométrique dérivée de raison
$q^{2}$ avec $|q^{2}| = q^{2} <1$ donc la série converge et
 
\[
 \E(Y) = \frac{pq}{(1-q^{2})^{2}} = \frac{q}{(1 + q)^{2}(1-q)} 
\]
 \end{exercice}

 \begin{exercice}{\it (Exercice avec préparation)}~
 \begin{noliste}{1.}
 \setlength{\itemsep}{4mm}
 \item Soient $X$ et $Y$ deux variables aléatoires discrètes. La loi du
couple $(X,Y)$ est déterminée par : \\
 $ \bullet$ La donnée de $(X,Y)(\Omega)$ : l'ensemble des valeurs
prises par le couple.\\
 $\bullet$ La donnée, pour tout $(i,j) \in (X,Y)(\Omega)$ de $\Prob([X
= i] \cap [Y = j])$.\\
 Les lois marginales sont les lois de $X$ et de $Y$. On déterminer la
loi d'une variable en utilisant la formule des probas totales sur le
sce des valeurs prises par l'autre variable : 
 
\[
 \Prob\left(\Ev{\Ev{[X = i]}}\right) = \sum \limits_{j \in Y(\Omega)}
\Prob([X = i] \cap [Y = j])
\]

%%% LALALALA
\[
 \Prob\left(\Ev{\Ev{[Y = j]}}\right) = \sum \limits_{i \in X(\Omega)}
\Prob([X = i] \cap [Y = j]) 
\]
 Pour tout $j \in Y(\Omega)$, la loi de $X$ sachant $[Y = j]$ est
définie par : \\
 Pour tout $i \in X(\Omega)$, $P_{[Y = j]} \Ev{[X = i]} =
\frac{\Prob([X = i] \cap [Y = j])} {\Prob\left(\Ev{\Ev{[Y =
j]}}\right)}$

\item $\bullet$ $(X,Y)( \Omega) = \{(i,j) \in \N^{2} | j \leq i\}$.\\
  $\bullet$ $\forall (i,j) \in (X,Y)(\Omega)$, $\Prob([X = i] \cap [Y
  = j]) = \Prob\left(\Ev{\Ev{[X = i]}}\right)P_{[X = i]}\Ev{ [Y = j]}
  = e^{-\lambda} \frac{\lambda^{i}}{i!} \binom{i}{j} p^{j} (1-p)^{i-j}
  = e^{-\lambda} \frac{1}{j!(i-j)!}(\lambda p)^{j} (1-p)^{i-j} $

\item $\forall j \in \N$, $\Prob\left(\Ev{[Y = j}\right) = \sum
  \limits_{i = 0}{+ \infty} \Prob([X = i] \cap [Y = j]) = \sum
  \limits_{i = j}{+ \infty} \Prob([X = i] \cap [Y = j]) = \sum
  \limits_{i = j}{+ \infty} e^{-\lambda} \frac{1}{j!(i-j)!}(\lambda
  p)^{j} (1-p)^{i-j} = e^{-\lambda} \frac{(\lambda p)^{j}}{j!} \sum
  \limits_{i = j}{+ \infty} \frac{(1-p)^{i-j }}{(i-j)!}$. On pose le
  changement d'indice $k = i-j$ : $\Prob\left(\Ev{[Y = j]}\right) =
  e^{-\lambda} \frac{(\lambda p)^{j}}{j!} \sum \limits_{k = 0}{+
    \infty} \frac{(1-p)^{k }}{k!} = e^{-\lambda} \frac{(\lambda
    p)^{j}}{j!} e^{\lambda(1-p)} = e^{-\lambda
    p} \frac{(\lambda p)^{j}}{j!}$.\\
  $Y$ suit une loi de Poisson de paramètre $\lambda p$.
 \item $\bullet$ $(X-Y)(\Omega) = \N$ ($X$ peut prendre n'importe
quelle valeur entière et de nombre de succès de la binomiale a tjrs une
proba non nulle de valoir 0). \\
 $\bullet$ Soit $n \in \N$, $\Prob\left(\Ev{\Ev{[X-Y] = n}}\right) =
\sum \limits_{i = 0}{+ \infty} \Prob([X = i] \cap [Y = i-n]) = \sum
\limits_{i = n}{+ \infty} \Prob([X = i] \cap [Y = i-n]) = \sum
\limits_{i = n}{+ \infty}e^{-\lambda} \frac{1}{(i-n)!}{n!}(\lambda
p)^{i-n}(1-p)^{n} = e^{-\lambda}\frac{(1-p)^{n}}{n!}\sum \limits_{i =
n}{+ \infty} \frac{1}{(i-n)!}(\lambda p)^{i-n}$. On pose le changement
d'indice $k = i-n$ : \\
 $\Prob\left(\Ev{\Ev{[X-Y] = n}}\right) =
e^{-\lambda}\frac{(1-p)^{n}}{n!}\sum \limits_{k = 0}{+ \infty}
\frac{1}{k!}(\lambda p)^{k} = e^{\lambda(1-p)}\frac{(1-p)^{n}}{n!}$

%%% LALALALA
\item
  \begin{noliste}{a)}
 \setlength{\itemsep}{2mm}
 \item $\forall (j,n) \in \N^{2}$, $\Prob([Y = j ] \cap [X-Y = n]) =
\Prob([Y = j] \cap [X = j + n]) = e^{-\lambda} \frac{1}{j!n!} (\lambda
p)^{j}(1-p)^{n} = e^{-\lambda p} \frac{(\lambda p)^{j}}{j!} \times
e^{\lambda(1-p)} \frac{(1-p)^{n}}{n!} = \Prob\left(\Ev{\Ev{[Y =
j]}}\right)\Prob\left(\Ev{\Ev{[X-Y = n]}}\right)$.\\
 Les variables $Y$ et $X-Y$ sont indépendantes.
 \item $Y$ et $X-Y$ étant indépendantes, on a $cov(Y, X-Y) = 0$ cad
$cov(Y,X)- Cov(Y,Y) = Cov(X,Y)-\V(Y) = 0$ (linéarité à droite et
symétrie de la covariance). \\
 Ainsi, $Cov(X,Y) = \V(Y) $ donc $\frac{Cov(X,Y)}{\sigma(X) \sigma(Y)}
= \frac{ \sigma(Y)}{\sigma(X)} = \sqrt{p}$. 
 \end{noliste}
 \end{noliste}
 \textbf{\underline{Exercice sans préparation}} \\
\\
 $A$ est diagonalisable donc il existe $P \in \M{n}$ inversible et
$D\in \M{n} $ diagonale constituée de valeurs propres de $A$ telles que
$A = PDP^{-1}$. \\
 Or $X^{k}-1$ est un polynôme annulateur de $A$ donc si $\lambda$ est
valeur propre de $A$ alors $\lambda^{k} = 1$. \\
 $\bullet$ Si $k$ est impair alors $\lambda = 1$ donc $D = I_{n}$ donc
$A = PI_{n} P^{-1} = I_{n}$ donc $A^{2} = I_{n}$.\\
 $\bullet$ Si $k$ est pair alors $\lambda = \pm 1$. Donc $D$ est
diagonale de coefficients diagonaux tous égaux à $\pm 1$ donc $D^{2} =
I_{n}$.\\
 Ainsi, $A^{2} = PDP^{-1} PDP^{-1} = PD^{2} P^{-1} = PI_{n} P^{-1} =
I_{n}$. 
 \end{exercice}

 \begin{exercice}{\it (Exercice avec préparation)}~
 \begin{noliste}{1.}
 \setlength{\itemsep}{4mm}
 \item Une fonction est de classe $\mathcal{C}{p}$ sur un intervalle
$I$ si elle est $p$ fois dérivable et que sa dérivée $p$-ième est
continue. \\
 Toute fonction de classe $\mathcal{C}{p}$ admet une formule de Taylor
à l'ordre $p-1$ cad pour tout $a$, $b$ $\in I$, $f(b) = \Sum{k =
0}{(p-1)} \frac{(b-a)^{k}}{k!} f^{(k)}(a) + \dint{a}{b} (b-t)^{p-1}
f^{(p)}(t)dt$.
 \item 
 \begin{noliste}{a)}
 \setlength{\itemsep}{2mm}
 \item Montrons que $(f_{1},f_{2})$ est libre : \\
 Soit $a$ et $b$ deux réels tels que $\forall x \in \R$, $a f_{1}(x) +
b f_{2}(x) = 0$. Alors, en prenant des valeurs particulières ($x = 0$
et $x = 1$) on obtient $a = 0$ et $a + b = 0$ cad $a = b = 0$. \\
 La famille est libre et génératrice de $E$ donc est une base de $E$.
 \item La dérivation est linéaire donc $\Delta$ est linéaire.\\
 $\forall x \in \R$, $\Delta(f_{1})(x) = f_{1}'(x) = \alpha f_{1}(x)$
donc $\delta(f_{1}) = \alpha f_{1} \in E$.\\
 $\forall x \in \R$, $\Delta(f_{2})(x) = e^{\alpha x} + \alpha x
e^{\alpha(x)}$ donc $\Delta(f_{2}) = f_{1} + \alpha f_{2} \in E$. Par
linéarité der $\Delta$ et par stabilité de $E$ par combinaisons
linéaires, on obtient, $\Delta(E) \subset E$. \\
 
\[
 \boxed{\text{$\Delta$ est un endomorphisme de $E$}}
\]

%%% LALALALA
 
 La matrice de $\Delta$ dans la base $(f_{1},f_{2})$ est $A =
\begin{smatrix}
\alpha & 1\\
0 & \alpha
\end{smatrix}
$
 \item $A$ est inversible car trig sup avec des coeff diag non nuls
donc $\Delta$ est bijective et comme $A$ est triangulaire, son unique
valeur propre est $\alpha$. \\
 Si $A$ était diagonalisable alors il existerait $P$ inversible telle
que $A = P \alpha I_{n} P^{-1} = \alpha I_{n}$ : contradiction donc
$\Delta$ n'est pas diagonalisable.
 \end{noliste}
 \item $A^{-1} = \begin{smatrix}
\frac{1}{\alpha} & -\frac{1}{\alpha^{2}}\\
0 & \frac{1}{\alpha}\end{smatrix}
$.\\
 Soit $f \in E$, si $\Delta^{-1}(f) = g$ alors $\Delta(g) = f$ cad $g'
= f$ cad $g$ est une primitive de $f$. $\Delta^{-1}$ associe donc à $f$
une primitive de $f$ (celle qui est dans $E$). \\
 $A^{-1} \begin{smatrix}
-3\\
2
\end{smatrix}
 = \begin{smatrix}
-\frac{3
 \alpha + 2}{\alpha^{2}} \\
\frac{2}{\alpha}\end{smatrix}
$ donc $\Delta^{-1}(-3f_{1} + 2f_{2}) = -\frac{3
 \alpha + 2}{\alpha^{2}}f_{1} + \frac{2}{\alpha}$. \\
 Les primitive de $f$ sont donc les fonctions de la forme $-\frac{3
\alpha + 2}{\alpha^{2}}f_{1} + \frac{2}{\alpha} + constante$
 \item\begin{noliste}{a)}
 \setlength{\itemsep}{2mm}
 \item On calcule les premières puissances, on trouve $A^{2} =
\begin{smatrix}
\alpha^{2} & 2 \alpha\\
0 & \alpha^{2}
\end{smatrix}
$, $A^{3} = \begin{smatrix}
\alpha^{3} & 3 \alpha^{2}\\
0 & \alpha^{3}
\end{smatrix}
$, $A^{4} = \begin{smatrix}
\alpha^{4} & 4 \alpha^{3}\\
0 & \alpha^{4}
\end{smatrix}
$. On fait alors l'hypothèse que $\forall n \in \N$, $A^{n} =
\begin{smatrix}
\alpha^{n} & n \alpha^{n-1}\\
0 & \alpha^{n}
\end{smatrix}
$ et on démontrer cette relation par récurrence. 
 \item $A^{n} \begin{smatrix}
-3\\
2
\end{smatrix}
 = \begin{smatrix}
-3 \alpha^{n} + 2n \alpha^{n-1}\\
2\alpha^{n}
\end{smatrix}
$ donc $\Delta^{n}(f) = \alpha^{n-1}[ (-3\alpha + 2n)f_{1} + 2 \alpha
f_{2}]$. \\
 \end{noliste}
 \end{noliste}
 \textbf{\underline{Exercice sans préparation}} \\
\\
 \begin{noliste}{1.}
 \setlength{\itemsep}{4mm}
 \item si $X$ pair alors $(-1)^{X} = 1$ et si $X$ impair alors
$(-1)^{X} = -1$ donc $Y(\Omega) = \{-1,1\}$.\\
 D'après le théorème de transfert, sous réserve de convergence absolue,
on a : 
 
\[
 \E(Y) = \sum \limits_{k = 0}{+ \infty} (-1)^{k} \Prob\left(\Ev{\Ev{X =
k}}\right) 
\]
 Montrons la convergence absolue de cette série : la série de terme
général $|(-1)^{k} \Prob\left(\Ev{\Ev{[X = k]}}\right)| =
\Prob\left(\Ev{\Ev{X = k}}\right)$ converge (sce) donc la série
converge bien absolument. \\
 
\[
 \E(Y) = \sum \limits_{k = 0}{+ \infty} (-1)^{k}
e^{-\lambda}\frac{\lambda^{k}}{k!} = e^{-\lambda} \sum \limits_{k =
0}{+ \infty}\frac{(-\lambda)^{k}}{k!} = e^{-2\lambda}
\]
 \item on a $\E(Y) = 1 \Prob\left(\Ev{\Ev{ [Y =
1]}}\right)-1\Prob\left(\Ev{\Ev{[Y = -1]}}\right) = e^{-2\lambda}$ et
$\Prob\left(\Ev{\Ev{[Y = 1]}}\right) + \Prob\left(\Ev{\Ev{[Y =
-1]}}\right) = 1$. On résout ce système de deux équations à deux
inconnues, on obtient : $\Prob\left(\Ev{\Ev{[Y = 1]}}\right) = \frac{
e^{-2\lambda} + 1}{2}$ et $\Prob\left(\Ev{\Ev{[Y = -1]}}\right) =
\frac{1-e^{-2\lambda}}{2 }$
 \end{noliste}
 \end{exercice}

 \begin{exercice}{\it (Exercice avec préparation)}~
 \begin{noliste}{1.}
 \setlength{\itemsep}{4mm}
 \item Si $\dint{a}{b} f(t)dt$ converge absolument, alors elle
converge.\\
 Théorèmes de comparaison si $f$ est positif et continue par morceaux
sur $]a,b]$ : \\
 $\bullet$ Si $f \leq g$ avec $\dint{a}{b} g(t)dt$ converge alors
$\dint{a}{b} f(t \ dt $ converge.\\
 $\bullet$ Si $f = o_{a}( g)$ avec $\dint{a}{b} g(t)dt$ converge alors
$\dint{a}{b} f(t \ dt $ converge.\\
 $\bullet$ Si $f \underset{a}{\sim} g$alors $\dint{a}{b} g(t)dt$
converge si et seulement si $\dint{a}{b} f(t \ dt $ converge.
 \item Si $x \leq 0$ alors $t \to t^{-x} \sqrt{1 + t}$ est continue sur
$[0,1]$ donc l'intégrale converge. \\
 Si $x>0$,la fonction intérieure est positive et l'intégrale est
impropre en 0 or $\frac{1}{t^{x}} \sqrt{1 + t} \underset{t \to 0}{sim}
\frac{1}{t^{x}}$ avec $\dint{0}{1} \frac{1}{t^{x}}dt$ converge si et
seulement si $x<1$ (intégrale de Riemann) donc d'après les théorèmes de
comparaison, $f(x)$ existe si et seulement si $x <1$.
 \item On ne sait pas dériver $f$ donc on revient à la définition des
variations d'une fonction : \\
 soient $a<b <1$, alors $-a > -b $ donc pour tout $t \in \ ]0,1[$, $-a
\ln(t) < -b \ln(t) $ car $\ln(t) < 0$ donc $e^{-a \ln(t) } < e^{-b
\ln(t) }$ par stricte croissance de l'exponentielle.\\
 On a donc $t^{-a} < t^{-b}$ donc $t^{-a} \sqrt{1 + t} < t^{-b} \sqrt{1
+ t}$. On intègre l'inégalité sur $]0,1[$, par croissance des bornes,
on obtient $f(a) < f(b)$.\\
 $f$ est strictement croissante sur $D$.
 \item\begin{noliste}{a)}
 \setlength{\itemsep}{2mm}
 \item $\forall t \in \ ]0,1]$, $1 \leq 1 + t \leq 2$ et $1 \leq
\sqrt{1 + t} \leq \sqrt{2}$ par croissance de la fct racine sur $\R_+
$. d'où $t^{-x} \leq t^{-x} \sqrt{1 + t} \leq \sqrt{2} t^{-x}$. \\
 En intégrant cet encadrement sur $t \in \ ]0,1]$, on obtient
(croissance des bornes) : 
 
\[
 \frac{1}{1-x} \leq f(x) \leq\frac{\sqrt{2}}{1-x} 
\]
 \item $\lim \limits_{x \to -\infty} \frac{1}{1-x} = 0$ donc d'après le
théorème d'encadrement, $\lim \limits_{x \to -\infty } f(x) = 0$.\\
 $\lim \limits_{x \to 1^-} \frac{1}{1-x} = + \infty$ donc d'après le
théorème de comparaison, $\lim \limits_{x \to 1^-} f(x) = + \infty$.
 \end{noliste}
 \item \begin{noliste}{a)}
 \setlength{\itemsep}{2mm}
 \item $f(0) = \dint{0}{1} t^{0} \sqrt{1 + t}dt = \dint{0}{1} (1 +
t)^{\frac{1}{2}}dt = \frac{2}{3} (2^\frac{3}{2}-1)$.
 \item $f(x) = \dint{0}{1} t^{-x} \sqrt{1 + t}dt$. \\
 On pose $u$ et $v$ définies sur $[0,1]$ par $u(t) = t^{-x}$ et $v(t) =
\frac{2}{3}(1 + t)\sqrt{1 + t}$. $u$ et $v$ sont de classe
$\mathcal{C}{1}$ sur $]0,1]$ de dérivées $u'(t) = x t^{-(x + 1)}$ et
$v'(t) = \sqrt{1 + t}$. Ainsi, pour tout $0 < M < 1$, on a, par I.P.P.
: \\
 
\[
 \dint{M}{1} t^{-x} \sqrt{1 + t}dt = [ \ \frac{2}{3}t^{-x}(1 +
t)\sqrt{1 + t}]_{M}{1} + x\frac{2}{3} \dint{M}{1} t^{-(x + 1)}(1 +
t)\sqrt{1 + t}dt 
\]
 avec $\dint{M}{1}t^{-(x + 1)}(1 + t)\sqrt{1 + t}dt = \dint{M}{1}
(t^{-x} + t^{-(x + 1)})\sqrt{1 + t}dt = \dint{M}{1} t^{-x}\sqrt{1 +
t}dt + \dint{M}{1} t^{-(x + 1)}\sqrt{1 + t}dt $.\\
 On fait tendre $M$ vers $0$, on obtient : 
 
\[
 f(x) = \frac{4 \sqrt{2}}{3} + \frac{2}{3}x(f(x) + f(x + 1))
\]
 On isole $f(x + 1)$, on obtient : 
 
\[
 f(x + 1) = \frac{1}{x}[f(x)(\frac{3}{2}-x)-2\sqrt{2}]
\]
 \item On obtient $f(x + 1)\underset{x \to 0}{\sim}
\frac{1}{x}[f(0)frac{3}{2}-2\sqrt{2}] = -\frac{1}{x}$ à condition que
$f$ soit continue en 0 ($\ast$].\\
 Alors $f(X) \underset{X \to 1}{\sim} \frac{1}{1-X}$ en posant $X = x +
1$. \\
\\
 $(\ast) $ continuité de $f$ en 0 : \\
 $|f(x)-f(0)| = \left| \dint{0}{1} (t^{-x}-1) \sqrt{1 + t}dt\right|
\leq \dint{0}{1} |t^{-x}-1|\sqrt{1 + t}dt \leq \sqrt{2}
\dint{0}{1}|t^{-x}-1|dt = \sqrt{2} | \frac{1}{1-x}-1| \to 0$ lorsque
$x$ tend vers 0 d'où $f$ est continue en 0.
 \end{noliste}
 \item
 \end{noliste}
 \textbf{\underline{Exercice sans préparation}} \\
\\
 \begin{noliste}{1.}
 \setlength{\itemsep}{4mm}
 \item L'univers est l'ensemble des $n$-listes d'urnes, avec
répétitions donc $card(\Omega) = n^{n}$. \\
 On note $A$ l'évènement "chaque urne reçoit exactement 1 boule", alors
$A$ est constitué de l'ensemble des permutations des $n$ urnes donc
$card(A) = n!$. \\
 Par équiprobabilité des choix, on : \\
 
\[
 p_{n} = \Prob\left(\Ev{A}\right) = \frac{n!}{n^{n}}
\]. 
 \item $\forall n \in \N^*$, $\frac{p_{n + 1}}{p_{n}} = \frac{(n +
1)!n^{n}}{n! (n + 1)^{n + 1}} = \frac{(n + 1)!}{n!} \frac{n^{n}}{(n +
1)^{n + 1}} = (n + 1) \frac{n^{n}}{(n + 1)^{n + 1}} = \frac{n^{n}}{(n +
1)^{n}} = \left(\frac{n}{n + 1} \right)^{n}<1$ \\
 La suite $(o_{n})$ est donc décroissante et minorée (par 0 car chaque
terme est une proba ) donc convergente.
 \item $p_{n} = \frac{n}{n} \frac{n-1}{n} \frac{n-2}{n}....\frac{1}{n}
< \frac{1}{n} $ donc $0 \leq p_{n} \leq \frac{1}{n}$ et par le théorème
d'encadrement, on trouve $p_{n} \to 0$. 
 \end{noliste}
 \end{exercice}

 

\end{document}