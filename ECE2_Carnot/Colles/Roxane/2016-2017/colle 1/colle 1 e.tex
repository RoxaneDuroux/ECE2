\title{Colle 1,e}

\date{$27/09$}

\documentclass{article}

\usepackage{amsmath}
\usepackage{amssymb}

\begin{document}
\maketitle


\textbf{Question de cours:}
Donner les principales \'egalit\'es et in\'egalit\'es pour les nombres complexes.

\section{Exercice 1}
Racines carr\'ees d'un complexe:
\\ R\'esoudre l\'equation $z^2=1+i$ de deux mani\'eres diff\'erentes.



\section{Exercice 2}
R\'esoudre dans $\mathbb{C}$ l'\'equation: $z^2-(1+i)z+4+8i=0$


\section{Exercice 3}
Soit n un entier naturel non nul. Pour tout r\'eel t, calculer la somme:
\\ $ D_n(t)=\frac{1}{2} +cos(t)+cos(2t)+..+cos(nt)$


\section{Exercice 4}
Soient a,b,c trois nombres complexes de module 1.
\\1) Comparer $\bar{a}$ et $a^{-1}$ .
\\2) On pose $z_1=\frac{a+b}{a-b}$ et $z_2=\frac{a+b}{1-ab}$. Pouver que $z_1$ et $z_2$ sont des imaginaires purs.
\\3) On note $\alpha$ l'argument de a et $\beta$ l'argument de b. R\'eecrire $z_1$ et $z_2$ en fonction de $\alpha$ et $\beta$.
\\4) Comparer les modules de $ a+b+c$ et $ ab+bc+ca$.
\\5) Montrer que le complexe: $\frac{c+ab\bar{c}-(a+b)}{a-b}$ est un imaginaire pur.

\bibliographystyle{abbrv}


\end{document}
This is never printed
  