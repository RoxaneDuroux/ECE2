\title{Colle 1,d}

\date{$27/09$}

\documentclass{article}

\usepackage{amsmath}
\usepackage{amssymb}

\begin{document}
\maketitle


\textbf{Question de cours:}
D\'efinition et propri\'et\'es de la partie enti\`ere.


\section{Exercice 1 }
Montrer que: $\forall x\in \mathbb{R}$, et $\forall n \in \mathbb{N}^*$ :
\\
\\1)  $0\leq E(nx)-nE(x)\leq n-1 $
\\
\\2) $E( \frac{1}{n}E(nx))=E(x)$
\\
\\3) $\sum_{k=0}^{n-1} E(x+\frac{k}{n})=E(nx) $


\section{Exercice 2 }
Calculer: $E(\frac{m+n}{2})+E(\frac{n-m+1}{2})=$ pour tout $(m,n)\in (\mathbb{Z}^*)^2$ .
\\ 




\bibliographystyle{abbrv}


\end{document}
This is never printed
  