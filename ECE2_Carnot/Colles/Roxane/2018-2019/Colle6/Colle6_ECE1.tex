\documentclass[11pt]{article}%
\usepackage{geometry}%
\geometry{a4paper,
  lmargin=2cm,rmargin=2cm,tmargin=1.5cm,bmargin=1.5cm}
  
\input{../../../macros.tex}



\begin{document}
\begin{flushleft}
ECE1 \\
Mathématiques
\end{flushleft}

\begin{center}
\textbf{\Large{Colles - Semaine 6}}
\end{center}

\hrule

\vspace*{0,2cm}

\section{Série 1}

\subsection*{Exercice 1}
\noindent
  Soient $(u_n)$ et $(v_n)$ les suites définies par $u_0 = 1$, $v_0 =
  2$, et :
  \[
  \left\{
  \begin{array}{l}
    \forall n \in \N, \ u_{n+1} = 3 u_n + 2 v_n \\
    \forall n \in \N, \ v_{n+1} = 2 u_n + 3 v_n 
  \end{array}
  \right.
  \]
  \begin{noliste}{a.}
  \item Montrer que la suite $(u_n-v_n)$ est constante.
  \item En déduire que $(u_n)$ est arithmético-géométrique.
  \item Calculer $u_n$ et $v_n$.
  \item Sans utiliser le résultat de la question précédente,
    déterminer la nature de la suite $(u_n + v_n)$.\\
    En déduire, en utilisant une autre méthode, le calcul de $u_n$ et
    $v_n$.
  \end{noliste}

  
\subsection*{Exercice 2}
\noindent
Résoudre l'inéquation suivante : $\sqrt{x+5} \geq \sqrt{x^{2}-4}$



\section{Série 2}

\subsection*{Exercice 1}
\noindent
  On considère la suite $(u_n)$ définie par : $
  \left\{
  \begin{array}{l}
    u_0 \in \R \\
    u_1 \in \R \\
    \forall n \Geq 2, \ u_n = 4(u_{n-1} - u_{n-2})
  \end{array}
  \right.
  $
  \begin{noliste}{1)}
  \item Donner les valeurs de $u_2$, $u_3$ et $u_4$ en fonciton de
    $u_0$ et $u_1$.
  \item On note $(v_n)$ la suite définie par : $\forall n \in \N, \
    u_n = 2^n v_n$.
    \begin{noliste}{a.}
    \item Montrer que $(v_n)$ vérifie la relation : $\forall n \Geq 2,
      \ v_{n} - v_{n-1} = v_{n-1} - v_{n-2}$.
    \item Quelle est la nature de la suite $(v_n)$ ?
    \item Exprimer $(v_n)$ en fonction de $n$, $u_0$ et $u_1$.
    \item En déduire l'expression de $u_n$.
    \end{noliste}
  \item Déterminer la suite $(u_n)$ dans le cas où : $u_0 = 1$ et $u_1
    = 8$.
  \item
    \begin{noliste}{a.}
    \item Déterminer la suite $(u_n)$ dans le cas où : $u_0 = 1$ et
      $u_1 = 2$.
    \item Que peut-on dire dans ce cas des suites $(v_n)$ et $(u_n)$ ?
    \item Calculer alors la somme : $S_n = \Sum{k=1}{n} u_k$.
    \end{noliste}
  \end{noliste}


\subsection*{Exercice 2}
\noindent
Résoudre l'inéquation suivante : $\ln(3x+1) \leq \ln (2x-1)$


\newpage


\section{Série 3}

\subsection*{Exercice 1}

  \begin{noliste}{1)}
  \item Dresser le tableau de variation de la fonction $f$ définie sur
    $\R^+$ par $f(x) = \ln (1+x)$.\\
    En déduire le signe de $f$.
  \item Soit $a$ un réel strictement positif et $(u_n)$ la suite
    définie par :
    \[
    \left\{
      \begin{array}{l}
        u_0 = a \\
        \forall n \in \N, \ u_{n+1} = \ln (1+u_n)
      \end{array}
    \right.
    \]
    \begin{noliste}{a.}
    \item Démontrer que pour tout $n \in \N$, $u_n$ est défini et $u_n 
> 
0$.
    \item Quel est le sens de variation de $(u_n)$ ?
    \end{noliste}
  \end{noliste}

\subsection*{Exercice 2}
\noindent
  Résoudre l'inéquation suivante : $5\left( \dfrac{1}{3}\right) ^{x}\leq 
10^{-10}$


\end{document}
