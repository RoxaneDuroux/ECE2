\documentclass[11pt]{article}%
\usepackage{geometry}%
\geometry{a4paper,
  lmargin=2cm,rmargin=2cm,tmargin=2.5cm,bmargin=2.5cm}
  
\input{../../../macros.tex}



\begin{document}
\begin{flushleft}
ECE2 \\
Mathématiques
\end{flushleft}


\vspace{0.1cm}

\begin{center}
\textbf{\Large{Programme de colle - Semaine 2}}
\end{center}

\hrule

\vspace*{0,2cm}

\section*{Notation}

On adoptera les principes suivants pour noter les étudiants :
\begin{noliste}{$\stimes$}
\item si l'étudiant sait répondre à la question de cours, il 
aura une note $> 8$.
\item si l'étudiant ne sait pas répondre à la question de 
cours ou s'il y a trop d'hésitations, il aura une note $\leq 8$.
\end{noliste}

\section*{Questions de cours}

On choisira pour chaque étudiant une question de cours parmi les 
suivantes :
\begin{noliste}{$\sbullet$}
  \item 
  {\bf Proposition 1} :\\
  Toute suite convergente est bornée :
  \[
  \left( u_n \right) \mbox{ convergente } \Rightarrow \ \left( u_n 
  \right) \mbox{ bornée }.
  \]
  
  \begin{proof}[Preuve]~\\
  Soit $\left( u_n \right)$ une suite qui converge vers $\ell \in \R$, 
  donc, soit $\eps>0$,
  \[
    \exists n_0 \in \N, \mbox{ tel que } \forall n \geq n_0, \vert u_n 
    - \ell \vert < \eps,
  \]
  c'est-à-dire pour tout $ n \geq n_0$, $\ell - \eps < u_n < \ell + 
  \eps$.\\
  On note
  \[
    M= \max_{i \in \llbracket 0, n_0 \rrbracket} \vert u_i \vert.
  \]
  $M$ existe car $\Card{(\llbracket 0, n_0 \rrbracket)}$ est fini. On a 
  donc
  \[
    \forall n \in \N, \min (M, \ell - \eps) \leq u_n \leq \max(M, \ell 
    + \eps),
  \]
  \ie la suite $\left( u_n \right)$ est bornée.
\end{proof}

  \item {\bf Proposition 2} : \\
         Toute suite croissante non majorée diverge.

  \begin{proof}~\\
    Soit $\left( u_n \right)$ une suite réelle croissante et non 
    majorée. Traduisons ces propositions \og avec des $\eps$ \fg{}.\\
    $\left( u_n \right)$ est croissante, c'est-à-dire
    \[
      \forall n \in \N, \ u_n \leq u_{n+1}.
    \]
    $\left( u_n \right)$ n'est pas majorée, c'est-à-dire 
    \texttt{NON}$(\exists A >0, \ \forall n \in \N, \ u_n \leq A)$. Donc
    \[
      \forall A >0, \ \exists N \in \N, \ u_N > A.
    \]
    Traduisons maintenant ce que l'on veut obtenir : $\left( u_n 
    \right)$ diverge vers $+ \infty$ :
    \[
      \forall A >0, \exists N \in \N, \forall n \geq N, \ u_n >A.
    \]
    On peut remarquer qu'on y est déjà presque avec la définition de 
    \og non   majorée \fg{}.
    On sait donc que pour tout $A>0$, il existe $N \in \N$ tel que $u_N 
    >A$. \\
    Or la suite $\left( u_n \right)$ est croissante. Donc par 
    récurrence 
    immédiate, pour tout $n \geq N$, $u_n >A$, ce qui est exactement ce 
    qu'il fallait démontrer.
   \end{proof}
   
   \item {\bf Propriété de recouvrement} :\\
   Soit $\left( u_n \right)$ une suite à valeurs réelles.\\ 
   Si les 
   sous-suites $\left(u_{2n} \right)$ et $\left( u_{2n+1} \right)$ 
   convergent vers un même réel $\ell$, alors la suite $\left( u_n 
   \right)$ converge vers $\ell$.
   \begin{proof}[Preuve]~\\
    Soit $I$ un intervalle ouvert contenant $\ell$.
    \begin{noliste}{$\triangleright$}
    \item $(u_{2n}) \tendn \ell$ donc l'intervalle $I$ contient 
    tous les termes de la suite $(u_{2n})$ (\ie $I$ 
    contient tous les termes pairs de la suite $(u_{n})$) sauf un 
    nombre fini,
    \item $(u_{2n+1}) \tendn \ell$ donc l'intervalle $I$ contient 
    tous les termes de la suite $(u_{2n+1})$ (\ie $I$ 
    contient tous les termes impairs de la suite $(u_{n})$) sauf un 
    nombre fini.
    \end{noliste}
    Finalement, $I$ contient tous les termes de la suite $\left( u_n 
    \right)$ sauf un nombre fini.\\
    Ceci est valable pour tout intervalle ouvert contenant $\ell$. 
    Donc la suite $\left( u_n \right)$ 
    converge donc vers $\ell$.
    \end{proof}
\end{noliste}


\section*{Connaissances exigibles}

\begin{noliste}{$-$}
\item convergence de suites numériques (théorème de convergence 
monotone, théorème d'encadrement, etc.)
\item suites adjacentes
\item la proriété de recouvrement est connu mais est hors programme. 
Une démonstration est donc nécessaire à chaque utilisation.
\item étude de suites récurrentes (les élèves seront guidés dans 
le cheminement de ces études)
\item équivalents
\item négligeabilité
\item Les séries ne sont pas au programme de cette série de colles. 
On peut néanmoins demander la manipulation de sommes.
\end{noliste}

\end{document}
