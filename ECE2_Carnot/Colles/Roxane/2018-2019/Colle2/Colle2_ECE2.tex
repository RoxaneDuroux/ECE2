\documentclass[11pt]{article}%
\usepackage{geometry}%
\geometry{a4paper,
  lmargin=2cm,rmargin=2cm,tmargin=1.5cm,bmargin=1.5cm}
  
\input{../../../macros.tex}



\begin{document}
\begin{flushleft}
ECE2 \\
Mathématiques
\end{flushleft}

\begin{center}
\textbf{\Large{Colles - Semaine 2}}
\end{center}

\hrule

\vspace*{0,2cm}
 
\begin{exercice}~
 \begin{noliste}{1.}
 \item \begin{noliste}{a)}
 	\item Rappeler la définition de \og la suite $(u_n)$ converge 
 vers $a$ \fg{}.
 	\item Supposons que $(u_n)$ est une suite réelle convergente de 
 limite $a \in \R_+^*$.\\
 	Démontrer qu'il existe un entier $n_0 \in \N$ tel que : $ 
 \forall n \geq n_0, \ u_n \geq \dfrac{a}{2}$.
 	\end{noliste}
 \item On considère la fonction $f \ : \ x \mapsto x(1-x)$, et la 
suite 
 $(u_n)$ définie par : $ \left\{ 
 \begin{array}{l}
 u_0 \in ]0,1[
 \\[.1cm]
 \forall n \in \N, \ u_{n+1} =f(u_n)
 \end{array} 
 \right.$.
 \begin{noliste}{a)}
 \item \'Etudier les variations de $f$.
 \item \begin{noliste}{i.}
 	\item Montrer que : $ \forall n \in \N, \ 0<u_n< 
 \dfrac{1}{n+1}$.\\
 	En déduire la limite de la suite $(u_n)$.
 	\item Pour tout entier naturel $n$, on pose $v_n=nu_n$.\\
 	Montrer que la suite $(v_n)$ est croissante. En déduire qu'elle 
 converge et que sa limite $L$ vérifie $L \in \ ]0,1]$.
 	
 	\item Pour tout entier naturel $n$, on pose 
 $w_n=n(v_{n+1}-v_n)$.\\
 	Montrer que la suite $(w_n)$ est convergente et que sa limite 
 vaut $L(1-L)$.
 	\end{noliste}
 \end{noliste}
 \item On suppose que $L \neq 1$.\\
 Montrer en utilisant le préliminaire qu'il existe un entier naturel 
 $n_0$ tel que :
 \[
 \forall n \geq n_0, \ v_{n+1} -v_n \geq \dfrac{L(1-L)}{2n}.
 \]
 En déduire que $ \dlim{n \to + \infty} v_n = + \infty$.
 
 \item En déduire que $ u_n \underset{n \to + \infty}{\sim} 
 \dfrac{1}{n}$.
 \end{noliste}
 \end{exercice}
 
 
 \begin{exercice}~\\
 On considère la fonction $f \ : \ x \mapsto 2 \, x \, \ee^x$.
 \begin{noliste}{1.}
 \item \begin{noliste}{a)}
	\item Montrer que $f$ réalise une bijection de $[0,1]$ sur un 
 ensemble que l'on déterminera.
 	\item Donner les tableaux de variations de $f$ et de $f^{-1}$.
 	\end{noliste}
 \item \begin{noliste}{a)}
	\item Vérifier qu'il existe un et un seul réel $\alpha$ dans 
 $[0,1]$, tel que $\alpha \ee^\alpha =1$.
 	\item Montrer que $\alpha \neq 0$.
 	\end{noliste}
 \item On définit la suite $(u_n)$ par : $ \left\{ 
 \begin{array}{l}
 u_0=\alpha
 \\[.1cm]
 \forall n \in \N, \ u_{n+1}=f^{-1}(u_n)
 \end{array} 
 \right.$.\\
 Montrer que, pour tout entier naturel $n$, $u_n$ existe et $u_n \in 
 ]0,1]$.
 \item \begin{noliste}{a)}
 	\item Montrer que pour tout réel $x$ de $[0,1]$, $f(x)-x \geq 
 0$.
 	\item En déduite déduire que la suite $(u_n)$ est décroissante.
 	\item Montrer que la suite $(u_n)$ est convergente et qu'elle a 
 pour limite $0$.
	\end{noliste}
 \end{noliste}
 \end{exercice}
 
 \newpage
 
 \begin{exercice}~
 \begin{noliste}{1.}
\item \'Etudier les variations de la fonction $f$ définie sur $[0, + 
 \infty[$ par :
 \[
 \forall x \in [0, + \infty[, \ f(x)=x+ \ee^x.
 \]
 \item Montrer que pour tout $n \in \N^*$, l'équation $x+ \ee^x =n$ 
admet 
 une unique solution dans $\R_+$ notée $u_n$. Préciser la valeur de 
 $u_1$.
 \item \begin{noliste}{a)}
 	\item Démontrer que la suite $(u_n)$ est strictement croissante.
 	\item La suite $(u_n)$ est-elle majorée ? En déduire la limite 
 de $(u_n)$.
 	\end{noliste}
 \item \begin{noliste}{a)}
 	\item Montrer que : $\forall n \in \N^*, \ n- \ln(n) \leq 
 \ee^{u_n} \leq n$.
 	\item en déduire que : $ u_n \underset{n \to + \infty}{\sim} 
 \ln(n)$.
 	\end{noliste}
 \item Pour tout $n \in \N^*$, on note $v_n=u_n- \ln(n)$. 
 \begin{noliste}{a)}
 	\item démontrer que : $ \forall n \in \N^*, \ \ee^{v_n} = 1- 
 \dfrac{u_n}{n}$.
 	\item En déduire un équivalent simple de $v_n$.
 	\end{noliste}
 \end{noliste}
 \end{exercice}
  
 
 
 
 
 
\end{document}
