\documentclass[11pt]{article}%
\usepackage{geometry}%
\geometry{a4paper,
  lmargin=2cm,rmargin=2cm,tmargin=1.5cm,bmargin=1.5cm}
  
\input{../../../macros.tex}



\begin{document}
\begin{flushleft}
ECE2 \\
Mathématiques
\end{flushleft}

\begin{center}
\textbf{\Large{Colles - Semaine 4}}
\end{center}

\hrule

\vspace*{0,2cm}


\begin{exercice}~\\
Soient $F$ et $G$ deux sous-ensembles de $\R^3$ définis par :
\[
F= \{ (x,y,z) \in \R^3 \ / \ x-y-z=0\} \ \mbox{ et } \ G=\{ (x,y,z)\in 
\R^3 \ / \ 2x+y+z=0\}
\]
\begin{noliste}{1.}
\item Montrer que $F$ et $G$ sont deux espaces vectoriels.
\item Déterminer une base de $F$ et une base de $G$.
\item Déterminer $F \cap G$.
\item Soit un vecteur $(a,b,c) \in \R^3$.
	\begin{noliste}{a)}
	\item Montrer qu'il existe $u$ dans $F$ et $v$ dans $G$ tels 
que $(a,b,c)=u+v$.
	\item Les vecteurs $u$ et $v$ sont-ils uniques ?
	\end{noliste}
\end{noliste}
\end{exercice}


\begin{exercice}~\\
Soit $F=\{(x,y,z) \in \R^3 \ / \ x=0\}$ et $G=\{(x,y,z) \in \R^3 \ / \ 
x+y-3z=0 \}$.
\begin{noliste}{1.}
\item Montrer que $F$ et $G$ sont deux espaces vectoriels réels.
\item Déterminer une base de $F$ et une base de $G$.
\item La famille obtenue en réunissant les vecteurs de la base de $F$ 
et ceux de la base de $G$ obtenues à la question précédente est-elle 
une famille libre ?
\item Déterminer l'espace vectoriel $F \cap G$.
\end{noliste}
\end{exercice}


\begin{exercice}~\\
On considère les sous-ensembles $F=\{(x,y,z) \in \R^3 \ / \ x+y+z=0\}$ 
et 
 $G=\{ (x,y,z) \in \R^3 \ / \ x=y=z\}$.
 \begin{noliste}{1.}
 \item Montrer que $F$ et $G$ sont deux sous-espaces vectoriels de 
$\R^3$.
 \item Déterminer la dimension de $F$ et celle de $G$.
 \item Déterminer $F \cap G$.
 \item Montrer que tout vecteur de $\R^3$ peut s'écrire de manière 
unique comme somme d'un vecteur de $F$ et d'un vecteur de $G$.
 \item Que peut-on en déduire de $F+G$ ? En déduire une base de $\R^3$ 
différente de la base canonique.
 \end{noliste}
\end{exercice}

\end{document}
