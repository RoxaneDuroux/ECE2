\documentclass[11pt]{article}%
\usepackage{geometry}%
\geometry{a4paper,
  lmargin=2cm,rmargin=2cm,tmargin=2.5cm,bmargin=2.5cm}
  
\input{../../../macros.tex}



\begin{document}
\begin{flushleft}
ECE2 \\
Mathématiques
\end{flushleft}


\vspace{0.1cm}

\begin{center}
\textbf{\Large{Programme de colle - Semaine 4}}
\end{center}

\hrule

\vspace*{0,2cm}

\section*{Notation}

On adoptera les principes suivants pour noter les étudiants :
\begin{noliste}{$\stimes$}
\item si l'étudiant sait répondre à la question de cours, il 
aura une note $>8$.
\item si l'étudiant ne sait pas répondre à la question de 
cours ou s'il y a trop d'hésitations, il aura une note $\leq 8$.
\end{noliste}

\section*{Questions de cours}

\begin{noliste}{$\sbullet$}
  \item {\bf Intersection de sev}\\
  Soit $E$ un espace vectoriel et soient $F$ et $G$ deux sous-espaces 
  vectoriels de $E$. Alors $F \cap G$ est un sous-espace vectoriel.

  \begin{proof}[Preuve]~
  \begin{noliste}{$\sbullet$}
  \item On a $F \cap G \subset F \subset E$. 
  \item $0_E \in F \cap G$. En effet : $0_E \in F$ et $0_E \in G$. 
  \item Soit $(\lambda, \mu) \in \R^2$. Soit $(u,v) \in (F \cap 
  G)^2$.
  \begin{noliste}{$\stimes$}
    \item Comme $(u,v) \in (F\cap G)^2$, alors $(u,v) \in F^2$.\\ 
    Or $F$ est un sous-espace vectoriel de $E$, on a alors : $u 
    + \lambda v \in F$.
    
    \item Comme $(u,v) \in (F\cap G)^2$, alors $(u,v) \in G^2$.\\ 
    Or $G$ est un sous-espace vectoriel de $E$, on a alors : $u 
    + \lambda v \in G$.
  \end{noliste}
  On en déduit que $u+\lambda v \in F \cap G$.\\[-1cm]
  \end{noliste}
  \end{proof}
  
  \item {\bf Stabilité d'un sev engendré}\\
  Soit $E$ un $\R$-ev. Soit $(u_1,\hdots,u_m)\in E^m$. On a :
  \[
   u_{m+1} \in\Vect{u_1,\hdots, u_m} \ \Rightarrow \ \Vect{u_1,
   \hdots, u_m,u_{m+1}} = \Vect{u_1,\hdots, u_m}
  \]
  
  \begin{proof}[Preuve]~\\
  Supposons $u_{m+1}\in\Vect{u_1,\hdots, u_m}$.\\
  Démontrons que $\Vect{u_1,\hdots, u_m,u_{m+1}}=\Vect{u_1,
  \hdots,u_m}$.
  \begin{noliste}{$\stimes$}
  \item $(\supset)$ Évident.
  
  \item $(\subset)$ Comme $u\in \Vect{u_1,\hdots,u_m,u_{m+1}}$, alors 
  le 
  vecteur $u$ s'écrit $u=\Sum{i=1}{m+1} \lambda_i \cdot u_i$.\\
  Or $u_{m+1}\in\Vect{u_1,\hdots,u_m}$, donc $u_{m+1}=\Sum{i=1}{m} 
  \mu_i \cdot u_i$. Ainsi :
  \[
    \begin{array}{rclcc}
    u &=& \left(\Sum{i=1}{m} \lambda_i \cdot u_i\right) & + & 
    \lambda_{m+1} \cdot u_{m+1}
    \\[.6cm]
    &=& \left(\Sum{i=1}{m} \lambda_i \cdot u_i\right) & + & 
    \lambda_{m+1} \cdot \left( \Sum{i=1}{m} \mu_i \cdot u_i\right)
    \\[.6cm]
    &=& \Sum{i=1}{m} (\lambda_i+\lambda_{m+1} \times \mu_i) \cdot u_i
    \end{array}
  \]
  et donc $u\in\Vect{u_1,\hdots,u_m}$.
  \end{noliste}~\\[-1cm]
  \end{proof}
  
  
  \newpage
  
  
  \item {\bf Techniques de base}\\
  On choisira de demander au choix à l'étudiant de :
  \begin{noliste}{$\stimes$}
    \item montrer qu'un espace $F$ est un sous-espace 
    vectoriel d'un espace vectoriel $E$, sur un exemple dans $\R^n$, 
    $\M{n,p}$, $\R[X]$, $\R_n[X]$, $\R^\N$, $\R^\R$, etc.
    \item montrer qu'une famille de vecteurs est génératrice d'un 
    espace vectoriel donné, sur un exemple.
    \item montrer qu'une famille de vecteurs est libre dans un 
    espace vectoriel donné, sur un exemple.
  \end{noliste}
\end{noliste}


\section*{Connaissances exigibles}

\begin{noliste}{$\sbullet$}
\item convergence de suites numériques (théorème de convergence 
monotone, théorème d'encadrement, etc.)
\item suites adjacentes
\item étude de suites récurrentes (les élèves doivent être guidés dans 
le cheminement de ces études)
\item équivalents
\item négligeabilité
\item séries numériques, à termes positifs, séries usuelles, 
comparaison 
série / intégrale, comparaisons de séries par négligeabilité et 
équivalence.
\item les séries alternées sont hors programme mais les étudiants ont 
vu 
en exercice comment démontrer le critère de convergence des séries 
alternées.
\item toutes les techniques sont à connaître (sommation télescopique, 
calcul direct des sommes partielles - séries usuelles, comparaison 
séries / intégrales, critères sur les SATP...)
\item on insistera particulièrement en colle sur les rédactions 
classiques (notamment pour tous les critères sur les SATP).
\item espaces vectoriels, sous-espaces vectoriels
\item famille génératrice, famille libre, base
\item bases canoniques de $\R^n$, $\M{n,p}$ et $\R_n[X]$.
\item la notion de dimension a été introduite en cours 
mais aucune des propriétés y faisant référence.
\item La notion de rang n'est pas au programme de cette semaine de 
colle.
\end{noliste}





\end{document}
