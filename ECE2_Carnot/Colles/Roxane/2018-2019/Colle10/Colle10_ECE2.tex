\documentclass[11pt]{article}%
\usepackage{geometry}%
\geometry{a4paper,
  lmargin=2cm,rmargin=2cm,tmargin=1.5cm,bmargin=1.5cm}
  
\input{../../../macros.tex}



\begin{document}
\begin{flushleft}
ECE2 \\
Mathématiques
\end{flushleft}

\begin{center}
\textbf{\Large{Colles - Semaine 10}}
\end{center}

\hrule

\vspace*{0,2cm}

\begin{exercice}~\\
 On donne les matrices carrées d'ordre $3$ suivantes : 
\[
A=\begin{smatrix}
5 & 5 & -14\\
6 & 6 & -16\\
5 & 5 & -14    
\end{smatrix}
;\quad  B = 
\begin{smatrix}
8 & 4 & -16\\
0 & 4 & -8\\
4 & 4 & -12   
\end{smatrix}
; \quad P = 
\begin{smatrix}
1 & 1 & 1\\
2 & -1 & 1\\
1 & 0 & 1
\end{smatrix}
\]
Ainsi que les matrices colonnes :\quad $
V_1 = 
\begin{smatrix} 
1\\ 2\\ 1  
\end{smatrix}
; \quad 
V_2 = 
\begin{smatrix} 
1\\ -1\\ 0
\end{smatrix} 
; \quad
V_{3} = 
\begin{smatrix}
1\\ 1\\ 1
\end{smatrix}.$
\begin{noliste}{1.}
\item
Vérifier que $V_1, V_2,$ et $V_3$ sont des vecteurs propres de $A$. 
Quelles sont les valeurs propres associées ?
\item
\begin{noliste}{a)}
\item
Montrer que $P$ est inversible et calculer $P^{-1}$.
\item
Justifier la relation :\quad $P^{-1}AP = 
\begin{smatrix} 
1 & 0 & 0\\
0 & 0 & 0\\
0 & 0 & -4
\end{smatrix}$. On note $D$ cette matrice 
diagonale.

\item
Calculer la matrice $\Delta = P^{-1}BP$ et vérifier qu'elle est 
diagonale.
\end{noliste}
\item
On se propose de calculer les matrices colonne $X_n$ définies par les 
relations :
\[
X_0= 
\begin{smatrix}
1\\ 0\\ 1
\end{smatrix} 
, \quad  X_1 = 
\begin{smatrix}
0\\ -1\\ 1
\end{smatrix} 
,\quad \mathrm{et} \quad \forall n \in 
\N\quad ,\ \ X_{n + 2} = A \, X_{n + 1} + B \, X_n
\]
A cet effet, on définit, pour tout $n\in\N$ : \quad $Y_{n} = P^{ - 
1}X_{n} \ \text{ et on pose également }\ Y_{n} = \begin{pmatrix}
 {u_{n}}    \\
 {v_{n}}  \\
 {w_{n}}  \end{pmatrix} .$
\begin{noliste}{a)}
\item
Montrer que $Y_{0} = \begin{pmatrix}
 { - 1}   \\
 {0} \\
 {2}   
\end{pmatrix}$ et $Y_{1} = \begin{pmatrix} { - 3}  
\\-1\\4\end{pmatrix}$.

\item
Montrer que pour tout entier naturel $n$,  $Y_{n + 2} = DY_{n + 1} 
+ \Delta Y_{n}.$

\item
Montrer alors que pour tout entier naturel $n$ :
\[
\left\lbrace
\begin{array}{rcl}
 {u_{n + 2}}  & =&  {u_{n + 1}}  \\
 {v_{n + 2}}   & { =} & {4v_{n}}  \\
 {w_{n + 2}}& { =}   & { - 4w_{n + 1} - 4w_{n}} \end{array} \right.
\]
En déduire les expressions explicites de $u_{n}$, $v_{n}$ et $w_{n}$ 
en fonction de $n$.
\item
Donner finalement la matrice $X_{n}$, en fonction de $n$.
\end{noliste}
\end{noliste}
\end{exercice}


\begin{exercice}~\\
On note $\B=(e_1,e_2,e_3)$ la base canonique de $\R^3$ et on considère 
l'endomorphisme de $\R^3$ défini par :
\[
f(e_1)=\frac{1}{3}(e_2+e_3), \ \mbox{ et } \ 
f(e_2)=f(e_3)=\frac{2}{3}e_1
\]
\begin{noliste}{1.}
\item \'Ecrire la matrice de $f$ dans la base $\B$.
\item Déterminer la dimension de $\im (f)$ puis celle de $\kr(f)$.
\item En déduire une valeur propre de $f$ ainsi que le sous-espace 
propre associé.
\item Déterminer les autres valeurs propres de $f$ ainsi que les 
sous-espaces propres associés.
\item En déduire que $f$ est diagonalisable.
\end{noliste}
\end{exercice}

\newpage


\begin{exercice}~\\
L'application $f$ désigne un endomorphisme de $\R^n$. On munit $\R^n$ 
d'une base $(e_1,\cdots, e_n)$.
\begin{noliste}{1.}
\item On suppose que 
\[
\forall x\in\R^n, \ \exists \lambda_x \in\R \mbox{ tel que } 
f(x)=\lambda_x x
\]
\begin{noliste}{a)}
\item \'Ecrire de deux manières différentes le vecteur 
$f(e_1+\cdots+e_n)$.
\item En déduire qu'il existe un réel $\lambda$ tel que $f=\lambda 
\cdot \id$.
\end{noliste}
\item Soit $x$ un vecteur non nul de $\R^n$.\\
Justifier qu'il existe une base de $\R^n$ de la forme 
$(x,\eps_2,\hdots,\eps_n)$.\\
On note alors $p_x$ l'application de $\R^n$ dans $\R^n$ définie par :
\[
\forall (a,b_2,\hdots, b_n)\in\R^n, \ p_x\left( a\cdot x + \sum_{k=2}^n 
b_k\cdot\eps_k\right)=a\cdot x
\]
\begin{noliste}{a)}
\item Montrer que $p_x$ est un endomorphisme de $\R^n$.
\item Montrer que pour tout $z\in\R^n$, on a
\[
p_x(z)=z \ \Leftrightarrow \ z\in\Vect{x}
\]
\end{noliste}
\item Montrer l'équivalence suivante :
\[
\forall g\in\LL{\R^n}, \ f\circ g=g\circ f \ \Leftrightarrow \ 
\exists \lambda\in\R, \ f=\lambda \cdot \id
\]
\end{noliste}
\end{exercice}


\begin{exercice}~\\
On note $m$ un paramètre réel et on considère les matrices $H_m$ 
définies
par : $H_m=
\begin{smatrix}
-1-m & m & 2\\
-m & 1 & m\\
-2 & m & 3-m
\end{smatrix}.$\\
On note $h_{m}$ l'endomorphisme de $\R^3$ ayant pour matrice 
$H_m$ dans la base canonique de $\R^3$.

\begin{noliste}{1.}
\item 
On suppose dans cette question que $m=2$.

\begin{noliste}{a)}
\item
Écrire la matrice $H_2$.
\item 
Déterminer les valeurs propres de l'endomorphisme $h_{2}$ et les 
sous-espaces propres associés.

\item 
L'endomorphisme $h_2$ est-il diagonalisable? Si oui, donner une base de 
vecteurs propres de $h_2$.
\end{noliste}

\item 
Étudier de même les valeurs propres et les sous-espaces propres de 
$h_{0}$. Cet endomorphisme est-il diagonalisable?

\item 
\begin{noliste}{a)}
\item 
Montrer qu'il existe un réel $a$, qu'on déterminera, qui est valeur
propre de l'endomorphisme $h_m$ pour toutes les valeurs du paramètre 
$m$.

\item 
Déterminer, pour chaque valeur de $m$, le sous-espace propre de $h_m$ 
associé à
la valeur propre $a$. Montrer qu'on peut trouver un vecteur non nul 
$v_{1}$
appartenant à tous ces sous-espaces.
\end{noliste}

\item 
Soit $F$ le sous-espace de $\R^3$ engendré par les vecteurs 
$v_2=(1,0,1)$ et $v_3=(1,1,0)$ :\\ 
$F=\Vect{v_2,v_3}$.\\
Déterminer les vecteurs $h_m(v_2)$ et $h_m(v_3)$ et montrer que 
ces vecteurs appartiennent à $F$ pour tout $m$ réel.\\
En déduire que le $F$ est stable par $h_m$, c'est-à-dire que 
$h_m(F)\subset F$.

\item 
Montrer que $(v_1,v_2,v_3)$ est une base de $\R^3$.\\
\'Ecrire la matrice de $h_m$ dans la base $(v_1,v_2,v_3)$. En déduire 
les valeurs de $m$ pour lesquelles l'endomorphisme $h_m$ est 
diagonalisable.
\end{noliste}
\end{exercice}




\end{document}
