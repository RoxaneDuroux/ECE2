\documentclass[11pt]{article}%
\usepackage{geometry}%
\geometry{a4paper,
  lmargin=2cm,rmargin=2cm,tmargin=1.5cm,bmargin=1.5cm}
  
\input{../../../macros.tex}



\begin{document}
\begin{flushleft}
ECE2 \\
Mathématiques
\end{flushleft}

\begin{center}
\textbf{\Large{Colles - Semaine 11}}
\end{center}

\hrule

\vspace*{0,2cm}

\begin{exercice}~\\
Une urne contient $n$ boules noires (avec $n\in\N^*$) et deux boules 
blanches. On effectue dans cette urne des tirages successifs d'une 
boule, sans remise. On note :
\begin{noliste}{$\stimes$}
\item $X$ la \var égale au nombre de tirages nécessaires pour 
obtenir la première boule blanche.

\item $Y$ la \var égale au nombre de tirages nécessaires pour 
obtenir la seconde boule blanche.

\item Pour tout $i\in \llb 1,n+2\rrb$, $N_i$ (resp. $B_i$) 
l'événement \og le $i$ème tirage amène une boule noire (resp. blanche) 
\fg{}.
\end{noliste}
\begin{noliste}{1.}
\item \begin{noliste}{a)}
	\item Préciser $X(\Omega)$. Décrire, pour tout $k\in 
X(\Omega)$, l'événement $\Ev{X=k}$ à l'aide des événements $N_i$ et 
$B_i$.
	\item Montrer que pour tout $k\in X(\Omega)$, $ 
\Prob(\Ev{X=k})=\dfrac{2(n+2-k)}{(n+1)(n+2)}$.
	\item Calculer $\E(X)$.
	\end{noliste}
\item \begin{noliste}{a)}
	\item Déterminer $Y(\Omega)$.
	\item Déterminer la loi jointe du couple $(X,Y)$.
	\item En déduire la loi de $Y$.
	\item Calculer $\E(Y)$.
	\end{noliste}
\item Calculer $\cov(X,Y)$. Commenter son signe.
\end{noliste}
\end{exercice}


\begin{exercice}~\\
On considère une suite infinie de lancers d'une pièce équilibrée. Pour 
tout entier naturel non nul $n$, on désigne par $P_n$ l'événement \og 
Pile apparaît au $n$ème lancer \fg{} et par $F_n$ l'événement \og Face 
apparaît au $n$ème lancer \fg{}.\\
Soit $Y$ la v.a. désignant le rang du lancer où, pour la première fois, 
apparaît un Face précédé d'au moins deux Pile si cette configuration 
apparaît, et prenant la valeur $0$ si cette configuration n'apparaît 
jamais.\\
On suppose que l'expérience est modélisée par un espace probabilisé 
$(\Omega, \A,\Prob)$.\\
On pose $c_1=c_2=0$ et pour tout $n\geq 3$, $c_n=\Prob(\Ev{Y=n})$. On 
note également :
\[
\forall n \geq 3, \ B_n=P_{n-2}\cap P_{n-1}\cap F_n \ \mbox{ et } \ 
U_n=\dcup{i=3}{n} B_i
\]
On pose enfin $u_1=u_2=0$ et pour tout $n\geq 3$, $u_n=\Prob(U_n)$
\begin{noliste}{1.}
\item Montrer que la suite $(u_n)_{n\geq 3}$ est monotone et 
convergente.
\item \begin{noliste}{a)}
	\item Pour tout $n\geq 3$, calculer $\Prob(B_n)$.
	\item Montrer que, pour tout $n\geq 3$, les événements $B_n$, 
$B_{n+1}$ et $B_{n+2}$ sont deux à deux incompatibles.
	\item Calculer les valeurs de $u_3$, $u_4$ et $u_5$.
	\end{noliste}
\item Dans cette question, on suppose $n\geq 5$.
	\begin{noliste}{a)}
	\item Comparer les événements $U_n \cap B_{n+1}$ et $U_{n-2} 
\cap B_{n+1}$. Préciser leurs probabilités respectives.
	\item Montrer que pour tout $n\geq 3$, $u_{n+1} 
=u_n+\dfrac{1}{8}(1-u_{n-2})$.
	\item Déterminer la limite de la suite $(u_n)$.
	\item Calculer $\Prob(\Ev{Y=0})$.
	\end{noliste}
\item Pour tout $n\in \N^*$, on pose $v_n=1-u_n$.
\begin{noliste}{a)}
\item Trouver $(\beta,\gamma)\in\R^2$ tels que pour tout $n\in\N^*$, 
$v_n=\beta v_{n+2} +\gamma v_{n+3}$.
\item Montrer que la série de terme général $v_n$ est convergente et 
calculer $\Sum{n=0}{+\infty} v_n$.
\end{noliste}
\end{noliste}
\end{exercice}

\newpage


\begin{exercice}~\\ % ESCP 2017 (voie S)
 Soit $n$ un entier naturel tel que $n\geq 2$. On dispose d'un paquet 
de $n$ cartes $C_1$, $C_2$, $\hdots$, $C_n$ que l'on distribue 
intégralement, les unes après les autres entre $n$ joueurs $J_1$, 
$J_2$, $\hdots$, $J_n$ selon le protocole suivant :
\begin{noliste}{$\stimes$}
 \item la première carte $C_1$ est donnée à $J_1$ ;
 \item la deuxième carte $C_2$ est donnée de façon équiprobable entre 
$J_1$ et $J_2$ ;
 \item la troisième carte $C_3$ est donnée de façon équiprobable entre 
$J_1$, $J_2$ et $J_3$ ;
 \item et ainsi de suite, jusqu'à la dernière carte $C_n$ qui est donc 
distribuée de façon équiprobable entre les joueurs $J_1$, $\hdots$, 
$J_n$.
\end{noliste}
On suppose l'expérience modélisée sur un espace probabilisé $(\Omega, 
\A, \Prob)$.\\
On note $X_n$ la variable aléatoire égale au nombre de joueurs qui 
n'ont reçu aucune carte à la fin de la distribution.
\begin{noliste}{1.}
 \item Déterminer $X_n(\Omega)$ et calculer $\Prob(\Ev{X_n=0})$ et 
$\Prob(\Ev{X_n=n-1})$.

 \item Pour tout $i$ de $\llb 1,n\rrb$, on note $B_i$ la \var qui vaut 
$1$ si $J_i$ n'a reçu aucune carte à la fin de la distribution et vaut 
$0$ sinon.\\
Déterminer la loi de $B_i$. Exprimer la \var $X_n$ en fonction des \var 
$B_i$ et en déduire l'espérance de $X_n$.

 \item En faisant le moins de calculs possibles, donner la loi de $X_4$.
 
 \item 
 \begin{noliste}{a)}
  \item Montrer que pour $i$ et $j$ dans $\llb 1,n\rrb$ tels que $i<j$, 
on a :
\[
 \Prob(\Ev{B_i=1})\cap \Ev{B_j=1})=\dfrac{(i-1)(j-2)}{n(n-1)}
\]
En déduire la covariance des \var $B_j$ et $B_j$.

  \item Montrer que $\V(X_n)=\dfrac{n+1}{12}$.
 \end{noliste}
\end{noliste}
\end{exercice}





\end{document}
