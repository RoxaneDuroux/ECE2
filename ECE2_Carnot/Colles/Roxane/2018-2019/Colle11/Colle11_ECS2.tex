\documentclass[11pt]{article}%
\usepackage{geometry}%
\geometry{a4paper,
  lmargin=2cm,rmargin=2cm,tmargin=1.5cm,bmargin=1.5cm}
  
\input{../../../macros.tex}



\begin{document}
\begin{flushleft}
ECS2 \\
Mathématiques
\end{flushleft}

\begin{center}
\textbf{\Large{Colles - Semaine 11}}
\end{center}

\hrule

\vspace*{0,2cm}

\section*{Série 1}
\subsection*{Question de cours}
\noindent
Démontrer que tout endomorphisme d'un espace vectoriel de dimension 
finie admet un polynôme annulateur non nul, et son spectre est inclus 
dans les racines de ce polynôme annulateur.

\subsection*{Exercice 1} % EDHEC 2010
\noindent
Soit $f:(x,y,z)\in\R^3\mapsto \dfrac{x^2}{2}-y+z+xyz$.
\begin{noliste}{1.}
\item
Montrer que $f$ est de classe $\mathcal C^1$ sur $\R^3$ et calculer les 
dérivées partielles de $f$.
\item
Déterminer les points critiques de $f$.
\item
Calculer pour tout $(h,k,\ell)\in\R^3$, $f(1+h,-1+k,1+\ell)-f(1,-1,1)$.
\item
Montrer que $f$ n'admet aucun extremum sur $\R^3$.
%Rechercher les extrema de $f$ sur $\R^2$.
\end{noliste}


\subsection*{Exercice 2} % ESCP 2007
\noindent
Soit $E$ un $\K$-espace vectoriel où $\K=\R$ ou $\C$. On
considère deux projecteurs $p$ et $q$ de $E$ différents de
l'identité $\id_E$ et de l'application nulle ; on suppose en outre
que $p$ et $q$ commutent et que leur somme $f$ n'est pas égale
à l'identité.

\begin{noliste}{1.}
\item
Montrer que $p\circ q$ est un projecteur de $E$ et calculer
$f^3-3f^2+2f$.\\
On note $\spc(f)$ l'ensemble des valeurs propres de $f$.

\item
\begin{noliste}{a)}
\item
 Montrer que $0\in\spc(f)$ si et seulement si $\kr(p)\cap\kr(q) \neq 
 \{0_E\}$.

\item
Montrer que $2\in\spc(f)$ si et seulement si $\im(p)\cap \im(q)\neq 
\{0_E\}$.

\item
Montrer que $\kr(f)\oplus \kr(f- \id_E)\oplus\kr(f-2 \, \id_E)=E$.

\end{noliste}

\item
En déduire que $[2\notin\spc(f)$ ou $0\notin\spc (f)]$
entraîne
que $1\in\spc(f)$ et $\spc(f)\neq\{1\}$.
\end{noliste}



\newpage

\section*{Série 2}
\subsection*{Question de cours}
\noindent
Déterminer les valeurs propres de $A=
\begin{smatrix}
 3 & 0 & 1\\
 -1 & 2 & -1\\
 -2 & 0 & 0
\end{smatrix}$.

\subsection*{Exercice} % ESCP 2017
\noindent
Soit $E$ un $\K$-espace vectoriel de dimension finie (avec $\K=\R$ 
ou $\K=\C$). On note $\LL{E}$ l'espace vectoriel des endomorphismes de 
$E$.\\
Soit $p$ un projecteur de $E$ tel que $p\neq 0$ et $p\neq \id_E$. Pour 
tout $f\in \LL{E}$, on pose :
\[
 \varphi(f)=\dfrac{1}{2} \, (f\circ p +p\circ f)
\]
\begin{noliste}{1.}
 \item Montrer que $\varphi$ est un endomorphisme de $\LL{E}$.
 
 \item Calculer $(\varphi \circ \varphi)(f)$ et $(\varphi \circ \varphi 
 \circ \varphi)(f)$ ; en déduire les valeurs propres possibles de 
 $\varphi$.
 
 \item Pour tout sous-espace vectoriel $F$ de $E$, montrer que les 
 ensembles suivants sont des sous-espaces vectoriels de $\LL{E}$ :
 \[
  \mathcal{K}(F) = \{f\in \LL{E} \ / \ F\subset \kr(f)\} \quad 
  \mbox{ et } \quad \mathcal{I}(F) = \{f\in\LL{E} \ / \ \im(f) 
  \subset F\}
 \]
 
 \item Soient $f,g \in \LL{E}$.
 \begin{noliste}{a)}
  \item Calculer $f\circ g$ lorsque $f\in \mathcal{K}(\im(g))$ ou
  lorsque $g\in \mathcal{I}(\kr(f))$.
  
  \item Calculer $p\circ f$ lorsque $f\in \mathcal{I}(\im(p))$.
  
  \item Montrer que $f\circ p=f$ lorsque $f\in \mathcal{K}(\kr(p))$.
 \end{noliste}
 
 \item 
 \begin{noliste}{a)}
  \item Pour chacun des sous-espaces vectoriels suivants, montrer que 
  leurs éléments non nuls sont des vecteurs propres de $\varphi$ et 
  préciser les valeurs propres correspondantes :
  \[
   \mathcal{A} = \mathcal{K}(\im(p)) \cap \mathcal{I}(\kr(p)), \quad 
   \B = \mathcal{K}(\im(p)) \cap \mathcal{I}(\im(p)) \quad \mbox{ et }
   \quad \mathcal{C} = \mathcal{K}(\kr(p)) \cap \mathcal{I}(\im(p))
  \]
  
  \item Montrer que les sous-espaces $\mathcal{A}$, $\B$ et 
  $\mathcal{C}$ sont en somme directe.
  
  \item Quelles sont les valeurs propres de $\varphi$ ?
 \end{noliste}
\end{noliste}





\newpage
 
\section*{Série 3}
\subsection*{Question de cours}
\noindent
Démontrer la stabilité par somme des lois de Poisson.

\subsection*{Exercice 1}
\noindent
Soit $f$ la fonction définie pour tout couple $(x,y)$ de $\R^{2}$
par: $f(x,y) = 2 x^{2} +2 y^{2}+ 2xy - x - y$.
\begin{noliste}{1.}
\item
\begin{noliste}{a)}
\item 
Calculer les dérivées partielles premières de $f$.
\item  
En déduire que le seul point critique $A$ de $f$.%(1/6,1/6)
\end{noliste}
\item\label{min}
Montrer que $f$ présente un minimum global en $A$.
\item 
On considère la fonction $g$ définie pour tout couple $(x,y)$ de 
$\R^{2}$ par :
\[
 g(x,y) = 2 e^{2x} + 2 e^{2y} + 2 e^{x+y} - e^{x}- e^{y}
\]
\begin{noliste}{a)}
\item 
Utiliser la question \ref{min} pour établir que : \quad $\forall(x,y)\in 
\R^{2},\ g(x,y) \ge -\dfrac{1}{6}\cdotp$
\item 
En déduire que $g$ possède un minimum global sur $\R^{2}$ et préciser en 
quel point ce minimum est atteint.
\end{noliste}
\end{noliste}



\subsection*{Exercice 2} % ESC 2001
\noindent
On désigne pour tout entier naturel non nul $n$: $E_n=\R_n[X]$, espace
vectoriel des polynômes à coefficients réels qui sont soit le 
polynôme nul, soit de degré inférieur ou égal à $n$.\\
Pour tout polynôme $P$ de $E_n$, on note $P'$ le polynôme dérivé 
de $P$. On définit sur $E_n$ l'application $f$, qui à tout 
polynôme $P$ associe le polynôme $f(P)$ défini par:
\[
f(P)=(X^2-1)P'-(nX +1)P
\]
\begin{noliste}{1.}
\item Propriétés générales.
\begin{noliste}{a)}
\item Calculer $f(X^n),\; f(1)$. Calculer $f(P)$ pour $P = X^k,\; k\in
\llb 1,n-1 \rrb$ et $n\geq 2$.\\
Quelles sont les valeurs de $k\in\llb 0,n\rrb$ pour lesquelles le 
degré de $X^k$ est égal à celui $f(X^k)$?

\item Montrer que $f$ est un endomorphisme de $E_n$.

\item Écrire la matrice $A$ de $f$ dans la base canonique de $E_n$: 
$(1,X,X^2,\dots, X^n)$.
\end{noliste}

\item Étude pour des valeurs particulières de $n$.
\begin{noliste}{a)}
\item On suppose dans cette question seulement que $n=1$.\\
Trouver les valeurs propres de $A$.\\
Déterminer les vecteurs propres de l'endomorphisme $f$.

\item On suppose dans cette question seulement que $n=2$.\\
Trouver les valeurs propres de $A$.\\
Déterminer les vecteurs propres de l'endomorphisme $f$.
\end{noliste}

\item On suppose désormais que $n$ est un entier naturel non nul 
quelconque.
\begin{noliste}{a)}
\item Montrer que si un polynôme $P$ est vecteur propre de 
 $f$, alors $P$ est de degré $n$.

\item On considère les polynômes $(P_k)_{0\leq k\leq n}$ tels que 
pour tout $k\in\llb 0,n \rrb$ :
\[
P_k(X)=(X-1)^k(X+1)^{n-k}
\]
Montrer que pour tout $k\in\llb 0,n\rrb$, $f(P_k)=(2k-n-1)P_k$\\
En déduire les valeurs propres et vecteurs propres associés de 
l'endomorphisme $f$.\\
L'endomorphisme $f$ est-il diagonalisable ?\\
Pour quelles valeurs de $n$ est-il bijectif ? (on justifiera ses 
réponses)
\end{noliste}
\end{noliste}

\end{document}

