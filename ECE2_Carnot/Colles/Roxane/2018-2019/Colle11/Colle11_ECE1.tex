\documentclass[11pt]{article}%
\usepackage{geometry}%
\geometry{a4paper,
  lmargin=2cm,rmargin=2cm,tmargin=1.5cm,bmargin=1.5cm}
  
\input{../../../macros.tex}



\begin{document}
\begin{flushleft}
ECE1 \\
Mathématiques
\end{flushleft}

\begin{center}
\textbf{\Large{Colles - Semaine 13}}
\end{center}

\hrule

\vspace*{0,2cm}

\section{Série 1}

\subsection*{Exercice 1}
\noindent
Calculer $\Sum{k=1}{n} k \dbinom{n}{k}$ puis $\Sum{k=1}{n} 
k(k-1) \dbinom{n}{k}$.


\subsection*{Exercice 2}
\noindent
Soient $A$, $B$, $C$ trois ensembles.\\
  Soient $f : A \to B$, $g : B \to C$ et $h : C \to A$ trois
  applications telles que :
  \begin{liste}{$\sbullet$}
  \item $h \circ g \circ f$ et $f \circ h \circ g$ sont injectives
  \item $g \circ f \circ h$ est surjective
  \end{liste}
  Montrer que $f$, $g$ et $h$ sont bijectives.



\section{Série 2}

\subsection*{Exercice 1}
\noindent
Soient $n$ et $p$ deux entiers tels que $p \leq n$.
  \begin{noliste}{1.}
  \item En raisonnant par récurrence sur $n$ montrer la formule de \og
    Pascal généralisée \fg{} :
    \[
     \Sum{k=p}{n} \dbinom{k}{p} = \dbinom{n+1}{p+1}
    \]
  \item Justifier, que pour tout $k$, on a : $\dbinom{k}{p} =
    \dbinom{k+1}{p+1} - \dbinom{k}{p+1}$.
  \item En déduire une nouvelle démonstration de la formule de \og
    Pascal généralisée \fg{}.
  \end{noliste}


\subsection*{Exercice 2}
\noindent
\begin{noliste}{1)}
  \item On considère tout d'abord les applications $f$ et $g$
    suivantes.
    \[
    \begin{array}{ccc}
      \begin{array}{ccrcl}
        f & : & \N & \to & \N \\
        & & n & \mapsto & 2n \\[.7cm]
      \end{array}
      & \qquad \qquad &
      \begin{array}{ccrcl}
        g & : & \N & \to & \N \\
        & & n & \mapsto & \left\{
          \begin{array}{l@{\quad}R{3cm}}
            \dfrac{n}{2} & si $n$ est pair \nl
            n & si $n$ est impair
          \end{array}
        \right.
      \end{array}
    \end{array}
    \]
    \begin{noliste}{a.}
    \item Étudier le caractère injectif, surjectif, bijectif de ces
      applications.
    \item Calculer $g \circ f$. Cette application est-elle injective /
      surjective / bijective ?
    \item Calculer $f \circ g$. Cette application est-elle injective /
      surjective / bijective ?
    \end{noliste}

  \item On considère maintenant $f : E \to E$ et $g : E \to E$ deux
    applications.
    \begin{noliste}{a.}
    \item Démontrer que :
      \[
      g \circ f = id_E \quad \Rightarrow \quad \left\{
        \begin{array}{l}
          \mbox{$f$ injective}\\
          \mbox{$g$ surjective}
        \end{array}
      \right.
      \]
    \item Donner un tel exemple avec $f$ et $g$ non bijectives.
    \item La réciproque de l'implication est-elle vraie ?
    \item Que devient ce résultat lorsque $E$ est un ensemble fini ?
  \end{noliste}
  \end{noliste}

      
      \newpage
      

\section{Série 3}

\subsection*{Exercice 1}
\noindent
Montrer que : $\dbinom{n}{0} + \dbinom{n}{2} + \dbinom{n}{4} + \dots =
  \dbinom{n}{1} + \dbinom{n}{3} + \dbinom{n}{5} + \dots$ et trouver la
  valeur commune des deux sommes.



\subsection*{Exercice 2}
\noindent
Démontrer que : $\forall a \in \N, \forall b \in \N, \forall p \in
  \N, \ \Sum{k=0}{p} \dbinom{a}{k} \dbinom{b}{p-k} = \dbinom{a+b}{p}$.
  \begin{noliste}{1.}
  \item En calculant de deux manières $(1+x)^a (1+x)^b$.
  \item En cherchant le nombre de parties de cardinal $p$ dans $E \cup
    F$ où $E$ et $F$ sont des ensembles disjoints de cardinaux $a$ et 
    $b$.
  \item Application : soit $n, p, q$ des entiers naturels. Montrer que
    : $\Sum{k=0}{q} \dbinom{q}{k} \dbinom{n}{p+k} = 
    \dbinom{n+q}{p+q}$.
  \end{noliste}


\end{document}
