\documentclass[11pt]{article}%
\usepackage{geometry}%
\geometry{a4paper,
  lmargin=2cm,rmargin=2cm,tmargin=1.5cm,bmargin=1.5cm}
  
\input{../../../macros.tex}



\begin{document}
\begin{flushleft}
ECS2 \\
Mathématiques
\end{flushleft}

\begin{center}
\textbf{\Large{Colles - Semaine 15}}
\end{center}

\hrule

\vspace*{0,2cm}

\section*{Série 1}
\subsection*{Question de cours}
\noindent
Démontrer que les espaces vectoriels $F$ et $F^\perp$ sont 
supplémentaires dans $E$, et que $(F^\perp)^\perp=F$.

\subsection*{Exercice} % ESCP 2017
\noindent
On considère l'application $F$ définie sur $\R$ par : 
$F(t)=\dfrac{\ee^t}{1+\ee^t}$.
\begin{noliste}{1.}
 \item Montrer que $F$ est la fonction de répartition d'une variable 
 aléatoire à densité dont on déterminera une densité notée $f$.
 
 \item Soit $X$ une variable aléatoire de densité $f$. Montrer que $X$
 admet des moments d'ordre $n$ pour tout entier naturel $n$.\\[.2cm]
 On pose : $I=\dint{0}{+\infty} \dfrac{t}{1+\ee^t} \dt$.\\[.2cm]
 Calculer l'espérance $\E(X)$ de $X$ ainsi que sa variance $\V(X)$ en 
 fonction de $I$.
 
 \item On considère une suite de \var $(X_n)_{n\geq 1}$ définies sur le 
 même espace probabilisé $(\Omega, \A, \Prob)$, mutuellement 
 indépendantes et de densité $f$.\\
 Soit $(\overline{X}_n)_{n\geq 1}$ la suite de \var définie par :
 \[
  \forall n\geq 1, \ \overline{X}_n = \dfrac{1}{n} \, \Sum{i=1}{n} X_i
 \]
 \begin{noliste}{a)}
  \item Montrer que la suite $(\overline{X}_n)_{n\geq 1}$ converge en 
  probabilité vers $0$, puis déterminer une suite de réels 
  $(a_n)_{n\geq 1}$ telle que la suite $(a_n \, \overline{X}_n)_{n\geq 
  1}$ converge en loi vers une variable aléatoire suivant la loi 
  normale centrée réduite.
  
  \item On pose $S_n^2 = \dfrac{1}{n} \, \Sum{i=1}{n} X_i^2$.\\
  Construire à partir de $S_n^2$ un estimateur sans biais de $I$. 
  Montrer que cet estimateur est convergent.
 \end{noliste}
 
 \item Proposer en \Scilab{} une simulation de la loi associée à $f$.
\end{noliste}





\newpage

\section*{Série 2}
\subsection*{Question de cours}
\noindent
Déterminer le spectre de $A=
\begin{smatrix}
 3 & 0 & 1\\
 -1 & 2 & -1\\
 -2 & 0 & 0
\end{smatrix}$.

\subsection*{Exercice} % ESCP 2017
\noindent
Soit $n\in\N^*$ et $E=\R_{2n+1}[X]$ l'espace vectoriel des polynômes à 
coefficients de degré inférieur ou égal à $2n+1$.\\
On définit l'application $f$ qui à tout $P\in E$ associe le polynôme 
$f(P)$ défini par :
\[
 \forall x \in \R^*, \ f(P)(x) = x^{2n+1} \, P\left(\dfrac{1}{x}\right)
\]
\begin{noliste}{1.}
 \item Montrer que $f$ est un endomorphisme de $E$.
 
 \item 
 \begin{noliste}{a)}
  \item Déterminer $f \circ f$.
  
  \item En déduire que $f$ est diagonalisable (on pourra utiliser 
  l'application $p=\dfrac{1}{2} (f+ \id_E)$).
 \end{noliste}
 
 \item Soit $\varphi$ l'application définie sur $E \times E$ par :
 \[
  \mbox{pour } P(X)= \Sum{k=0}{2n+1} a_k \, X^k \mbox{ et } Q(X) =
  \Sum{k=0}{2n+1} b_k \, X^k, \quad \varphi(P,Q) = 
  \Sum{k=0}{2n+1} a_k \, b_k
 \]
 Montrer que $\varphi$ est un produit scalaire sur $E$.
 
 \item 
 \begin{noliste}{a)}
  \item Montrer que $f$ est un endomorphisme symétrique de 
  $(E,\varphi)$.
  
  \item En déduire que $\kr(f-\id_E)$ et $\kr(f+\id_E)$ sont 
  supplémentaires.
  
  \item Déterminer la dimension de chaque sous-espace propre de $f$.
 \end{noliste}
 
 \item Les résultats précédents restent-ils valables si $E=
 \R_{2n}[X]$ et :
 \[
  \forall x \in \R^*, \ f(P)(x) = x^{2n} \, P\left(\dfrac{1}{x}\right)
 \]
\end{noliste}





\newpage
 
\section*{Série 3}
\subsection*{Question de cours}
\noindent
Démontrer l'inégalité de Markov, puis celle de Bienaymé-Tchebychev.


\subsection*{Exercice} % ESCP 2017
\noindent
Soit $\alpha$ un réel strictement positif et $(Y_i)_{i\in\N^*}$ une 
suite de variables aléatoires indépendantes définies sur le même espace 
probabilisé $(\Omega, \A, \Prob)$. On suppose de plus que pour tout 
$i$, $Y_i$ suit la loi exponentielle de paramètre $i \, \alpha$.\\
Pour tout $n\in\N^*$, on pose $Z_n=\Sum{i=1}{n} Y_i$ et on note $g_n$ 
la densité de $Z_n$ nulle sur $\R_-$ et continue sur $\R_+^*$.
\begin{noliste}{1.}
 \item 
 \begin{noliste}{a)}
  \item Déterminer la fonction $g_2$.
  \item Montrer que pour $n\geq 1$ et $x>0$, on a : $g_n(x) =
  n \, \alpha \, \ee^{-\alpha \, x}\left(1-\ee^{-\alpha \, 
  x}\right)^{n-1}$
  \item Calculer l'espérance de $Z_n$ et en donner un équivalent simple 
  lorsque $n$ tend vers l'infini.
  \item Calculer la variance de $Z_n$ et montrer qu'elle admet une 
  limite finie lorsqie $n$ tend vers l'infini.
 \end{noliste}
 
 \item Pour $n\in\N^*$, on pose $U_n=\dfrac{1}{n} \, Z_n$.
 \begin{noliste}{a)}
  \item Déterminer la fonction de répartition $H_n$ de $U_n$.
  \item Montrer que la suite $(U_n)$ converge en loi et déterminer 
  la loi limite.
  \item Déterminer la limite quand $n$ tend vers l'infini de $\E(U_n)$ 
  et $\V(U_n)$.
 \end{noliste}
\end{noliste}



\end{document}
