\documentclass[11pt]{article}%
\usepackage{geometry}%
\geometry{a4paper,
  lmargin=2cm,rmargin=2cm,tmargin=2.5cm,bmargin=2.5cm}
  
\input{../../../macros.tex}



\begin{document}
\begin{flushleft}
ECE2 \\
Mathématiques
\end{flushleft}


\vspace{0.1cm}

\begin{center}
\textbf{\Large{Programme de colle - Semaine 15}}
\end{center}

\hrule

\vspace*{0,2cm}

\section*{Notation}

\noindent
On adoptera les principes suivants pour noter les étudiants :
\begin{noliste}{$\stimes$}
\item si l'étudiant sait répondre à la question de cours, il 
aura une note $>8$.
\item si l'étudiant ne sait pas répondre à la question de 
cours ou s'il y a trop d'hésitations, il aura une note $\leq 8$.
\end{noliste}


\section*{Questions de cours}

\noindent
On demandera à l'étudiant un ou plusieurs des points suivants :
\begin{noliste}{-}
  \item déterminer le DL à l'ordre 2 d'une fonction,
  
  \item démontrer la continuité d'une fonction ou calculer une limite à 
  l'aide d'un DL d'ordre 2
  
  \item déterminer la position d'une tangente par rapport à une courbe à
  l'aide d'un DL d'ordre 2
\end{noliste}




\section*{Connaissances exigibles}


\subsection*{Comparaison de fonctions et développements limités}

\begin{noliste}{-}
 \item Fonctions négligeables, équivalentes. Équivalents usuels
 \item Développements limités d'ordre 1 et 2. Formule de Taylor-Young. 
 Développements limités usuels.
\end{noliste}



\subsection*{Fonctions de deux variables}

\begin{noliste}{-}
 \item Définition et propriétés de la distance 
 $d(A,B)=\sqrt{(x_B-x_A)^2+(y_B-y_A)^2}$
 \item Continuité en un point, opérations sur les fonctions continues
 \item Dérivées partielles d'ordre $1$, gradient
 \item Classe $\Cont{1}$, opérations sur les fonctions de classe 
 $\Cont{1}$, DL à l'ordre $1$
 \item Dérivées partielles d'ordre $2$, matrice hessienne
 \item Classe $\Cont{2}$, opérations sur les fonctions de classe 
 $\Cont{2}$, théorème de Schwarz, DL d'ordre $2$
 \item Boule ouverte et fermée de $\R^2$, partie ouverte et fermée de 
 $\R^2$, partie bornée de $\R^2$
 \item Définition extremum local, point critique
 \item Condition nécessaire d'existence d'un extremum
 \item Condition suffisante d'existence d'un extremum (version avec les 
 valeurs propres de la matrice hessienne : la version avec le déterminant 
 de la matrice hessienne n'est pas au programme)
 \item Extremum global
 \item Toute fonction continue sur un fermé borné est bornée et atteint 
 ses bornes
\end{noliste}




\end{document}
