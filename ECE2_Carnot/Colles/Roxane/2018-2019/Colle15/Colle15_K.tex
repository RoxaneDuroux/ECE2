\documentclass[11pt]{article}%
\usepackage{geometry}%
\geometry{a4paper,
  lmargin=2cm,rmargin=2cm,tmargin=1.5cm,bmargin=1.5cm}
  
\input{../../../macros.tex}



\begin{document}
\begin{flushleft}
K \\
Mathématiques
\end{flushleft}

\begin{center}
\textbf{\Large{Colles - Semaine 15}}
\end{center}

\hrule

\vspace*{0,2cm}

\section*{Planche 1} % ESCP 2017

\noindent
\begin{noliste}{1.}
 \item Soit $F$ la fonction définie sur $\R$ par : $\forall x\in\R$, 
 $F(x)=\exp\left(-\ee^{-x}\right)$.
 \begin{noliste}{a)}
  \item Justifier que $F$ est une fonction de répartition.
  \item Soit $X$ une \var de fonction de répartition $F$. Déterminer 
  une densité $f$ de $X$.
 \end{noliste}
 {\it On suppose désormais que $X$ est une \var sur $(\Omega, \A, 
 \Prob)$ et que toutes les \var citées sont définies sur ce même 
 espace.}
 
 \item 
 \begin{noliste}{a)}
  \item Soit $Z=\ee^{-X}$. Justifier que $Z$ est une variable aléatoire 
  réelle sur $(\Omega, \A, \Prob)$ et déterminer sa loi.
  \item On rappelle que {\tt grand(1,1,\ttq{}exp\ttq{},1)} simule
    une variable aléatoire et suivant une loi
    exponentielle de paramètre $1$. Écrire une
    fonction {\tt Scilab} qui simule la variable
    aléatoire $X$.
  \item Soient $x$ et $y$ deux réels strictement positifs. Établir une 
  relation entre la probabilité conditionnelle 
  $\Prob_{\Ev{X\leq -\ln(X)}}(\Ev{X\leq -\ln(x+y)})$ et 
  $\Prob(\Ev{X\leq - \ln(y)})$.
 \end{noliste}
 
 \item Soit $(Y_i)_{i\in\N^*}$ une suite de \var définies sur $(\Omega, 
 \A, \Prob)$, mutuellement indépendantes et de même loi exponentielle 
 de paramètre $1$.\\
 Soit d'autre part $L$ une \var de loi de Poisson de paramètre $1$ 
 indépendante des variables aléatoires de la suite $(Y_i)_{i\in\N^*}$.\\
 On définit $S$ par :
 \begin{noliste}{$\stimes$}
  \item si $L(\omega)=0$, alors $S(\omega)=0$.
  \item si $L(\omega)=k$, avec $k\in\N^*$, alors $S(\omega) = 
  \max(Y_1(\omega), \hdots, Y_k(\omega))$.
 \end{noliste}
 \begin{noliste}{a)}
  \item Soit $k$ un entier naturel non nul.\\
  Déterminer la loi de la variable aléatoire $S_k=\max(Y_1,\hdots, 
  Y_k)$.
  \item Démontrer que pour tous réels $a$ et $b$ tels que $0<a<b$, 
  on a :
  \[
   \Prob(\Ev{a\leq S \leq b}) = \Prob(\Ev{a\leq X \leq b})
  \]
  \item Calculer $\Prob(\Ev{S=0})$.
 \end{noliste}
\end{noliste}



\newpage


\section*{Planche 2} % ESCP 2017

\noindent
Soit $\alpha$ un réel strictement positif et $(Y_i)_{i\in\N^*}$ une 
suite de variables aléatoires indépendantes définies sur le même espace 
probabilisé $(\Omega, \A, \Prob)$. On suppose de plus que pour tout 
$i$, $Y_i$ suit la loi exponentielle de paramètre $i \, \alpha$.\\
Pour tout $n\in\N^*$, on pose $Z_n=\Sum{i=1}{n} Y_i$ et on note $g_n$ 
la densité de $Z_n$ nulle sur $\R_-$ et continue sur $\R_+^*$.
\begin{noliste}{1.}
 \item 
 \begin{noliste}{a)}
  \item Déterminer la fonction $g_2$.
  \item Montrer que pour $n\geq 1$ et $x>0$, on a : $g_n(x) =
  n \, \alpha \, \ee^{-\alpha \, x}\left(1-\ee^{-\alpha \, 
  x}\right)^{n-1}$
  \item Calculer l'espérance de $Z_n$ et en donner un équivalent simple 
  lorsque $n$ tend vers l'infini.
  \item Calculer la variance de $Z_n$ et montrer qu'elle admet une 
  limite finie lorsqie $n$ tend vers l'infini.
 \end{noliste}
 
 \item Pour $n\in\N^*$, on pose $U_n=\dfrac{1}{n} \, Z_n$.
 \begin{noliste}{a)}
  \item Déterminer la fonction de répartition $H_n$ de $U_n$.
  \item Montrer que, pour tout $x\in\R$, la suite $(F_{U_n}(x))$ 
  converge vers un réel $F(x)$.\\
  Montrer que $F$ est la fonction de répartition d'une \var que l'on 
  précisera.
  \item Déterminer la limite quand $n$ tend vers l'infini de $\E(U_n)$ 
  et $\V(U_n)$.
 \end{noliste}
\end{noliste}


\newpage


\section*{Planche 3} % EDHEC 2010

\noindent
Dans cet exercice, $a$ désigne un réel strictement positif.\\
On considère deux variables aléatoires $X$ et $Y$, définies sur
un espace probabilisé $(\Omega,\mathcal{A},\Prob)$, indépendantes, et 
suivant toutes deux la loi uniforme sur $[0,a[$.\\
On pose $Z=|X-Y|$ et on admet que $-Y$, $X-Y$ et $Z$ sont des
variables aléatoires à  densité, elles aussi définies sur l'espace
probabilisé $(\Omega,\mathcal{A},\Prob)$.
\begin{noliste}{1.}
\item 
\begin{noliste}{a)}
\item Déterminer une densité de $-Y$.
\item En déduire que la variable aléatoire $X-Y$ admet pour densité la
fonction $g$ définie par :\vspace{-0.3cm}
$$
g(x)=\begin{cases}
\dfrac{a-|x|}{a^{2}} & \text{ si } x\in[-a,a]\\
0 & \text{ sinon }\end{cases}
\vspace{-0.3cm}$$
On note $G$ la fonction de répartition de $X-Y$.
\end{noliste}
\item 
\begin{noliste}{a)}
\item Exprimer la fonction de répartition $H$ de la variable $Z$ en 
fonction
de $G$.
\item En déduire qu'une densité de $Z$ est la fonction $h$ définie par :
$\quad 
h(x)=\begin{cases}
\dfrac{2(a-x)}{a^{2}} & \text{ si } x\in[0,a]\\
0 & \text{ sinon }\end{cases}
$
\end{noliste}
\item Montrer que $Z$ possède une espérance et une variance et les 
déterminer.
\item Simulation informatique.\\
On rappelle qu'en \Scilab{}, la commande \texttt{rand()} permet
de simuler la loi uniforme sur $[0,1[$.\\
Compléter la déclaration de fonction suivante pour qu'elle retourne
à  chaque appel un nombre réel choisi selon la loi de $Z$.

\begin{scilab}
  & \tcFun{function} \tcVar{v}=z(\tcVar{a}) \nl %
  & \quad x = ........... \nl %
  & \quad y = ........... \nl %
  & \quad \tcVar{v} = ........... \nl %
  & \tcFun{endfunction}
\end{scilab}
\end{noliste}



\end{document}
