\documentclass[11pt]{article}%
\usepackage{geometry}%
\geometry{a4paper,
  lmargin=2cm,rmargin=2cm,tmargin=1.5cm,bmargin=1.5cm}
  
\input{../../../macros.tex}



\begin{document}
\begin{flushleft}
ECE1 \\
Mathématiques
\end{flushleft}

\begin{center}
\textbf{\Large{Colles - Semaine 15}}
\end{center}

\hrule

\vspace*{0,2cm}


\section{Série 1}

\subsection*{Exercice 1}
\noindent
  Un professeur oublie fréquemment ses clés. Ces oublis vérifient le
  schéma suivant :
  \begin{noliste}{$\stimes$}
  \item si le jour $n$ il oublie ses clés, le jour suivant il les
    oublie avec la probabilité $\frac{1}{10}$,
  \item si le jour $n$ il n'oublie pas ses clés, le jour 
  suivant il les oublie avec la probabilité $\frac{4}{10}$.
  \end{noliste}
  Pour tout $n \in \N$, on note $E_n$ l'événement : \og le jour $n$,
  le professeur oublie ses clés \fg{} et $p_n = \Prob(E_n)$.
  \begin{noliste}{1.}
  \item Établir une relation entre $p_{n+1}$ et $p_n$.
  \item En déduire l'expression explicite de $p_n$.
  \end{noliste}
  On notera $p_1 = a \in \R$, la probabilité que le professeur oublie
  ses clés le premier jour.
  
  
\subsection*{Exercice 2}
\noindent
Un QCM comporte 10 questions. Pour chaque question, 4 réponses sont 
proposées dont une seule est exacte.
\begin{noliste}{1.}
 \item Combien y a-t-il de grilles réponses possibles ?
 \item Combien y a-t-il de grilles comportant au moins 6 réponses 
  correctes ?
\end{noliste}~\\[-1.4cm]



\section{Série 2}

\subsection*{Exercice 1}
\noindent
On considère une urne $U$ contenant $9$ boules blanches et $1$ boule
  noire, et une urne $V$ contenant $3$ boules blanches et $7$ boules
  noires. On lance un dé équilibré à $6$ faces numérotées de $1$ à
  $6$ :
  \begin{noliste}{$\stimes$}
  \item si on obtient $1$, on effectue deux tirages (avec 
  remise) dans l'urne $U$,
  \item si on n'obtient pas $1$, on effectue deux tirages 
  (avec remise) dans l'urne $V$.
  \end{noliste}
  On considère les événements $U$ : \og on tire dans l'urne U \fg{}, V
  : \og on tire dans l'urne $V$ \fg{}, $B_i$ : \og la $i$eme boule
  est blanche \fg{} et $N_i$ : \og la $i$eme boule est noire \fg{}
  pour $i \in \{1,2\}$.
  \begin{noliste}{1.}
  \item Les événements $B_1$ et $N_2$ sont-ils indépendants ?
  \item Sachant que l'on a obtenu une boule blanche puis une boule
    noire, de quelle urne est-il plus probable qu'on les ait tirées ?
  \end{noliste}
  
  
\subsection*{Exercice 2}
\noindent
On souhaite ranger sur une étagère 4 livres de maths distincts, 6 
livres de philosophie distincts et 2 livres de géographie distincts. De 
combien de façon peut-on effectuer ce rangement dans les cas suivants :
\begin{noliste}{1.}
 \item Les livres doivent être groupés par matière.
 \item Seuls les livres de maths doivent être groupés.
\end{noliste}~\\[-1.4cm]

     
\newpage
      

\section{Série 3}

\subsection*{Exercice 1}
\noindent
On a décelé dans une population une probabilité de $0,01$ pour qu'un
enfant soit atteint par une maladie M. La probabilité qu'un enfant
non atteint par M ait une réaction négative à un test T est de
0,9. S'il est atteint par M, la probabilité qu'il ait une réaction
positive au test est de $0,95$.
\begin{noliste}{1.}
 \item Quelle est la probabilité qu'un enfant % pris au hasard
 ait une réaction positive au test ?
 \item Quelle est la probabilité qu'un enfant % pris au hasard
 et ayant une réaction positive soit atteint par M ?
\end{noliste}


\subsection*{Exercice 2}
\noindent
\begin{noliste}{1.}
 \item Combien de nombres de 4 chiffres peut-on écrire avec des 
 chiffres compris entre 1 et 6 ?
 \item Et si les chiffres sont distincts ?
\end{noliste}






\end{document}
