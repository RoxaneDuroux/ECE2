\documentclass[11pt]{article}%
\usepackage{geometry}%
\geometry{a4paper,
  lmargin=2cm,rmargin=2cm,tmargin=1.5cm,bmargin=1.5cm}
  
\input{../../../macros.tex}



\begin{document}
\begin{flushleft}
K \\
Mathématiques
\end{flushleft}

\begin{center}
\textbf{\Large{Colles - Semaine 8}}
\end{center}

\hrule

\vspace*{0,2cm}

\section*{Planche 1}

% ESCP 2013

\noindent
Toutes les variables aléatoires de  cet exercice sont définies
sur un même espace probabilisé $(\Omega,{\cal A},\Prob)$.

\begin{noliste}{1.}
\item
Soit $Z$ une variable al\'{e}atoire \`{a} valeurs dans $\N$.\\ 
Montrer
que la variable al\'{e}atoire $2^{-Z}$ admet une esp\'{e}rance. On la
note $r(Z)$.\\
On suppose dans la suite de l'exercice   que pour tout $n\in \N$,
$\Prob(\Ev{Z=n})=  \left( \dfrac{1}{2} \right)^{n+1}$.

\item
\begin{noliste}{a)}
\item
Montrer que l'on d\'{e}finit ainsi une  loi de probabilit\'{e} et
calculer $r(Z)$.
\item
Montrer que pour tout $(n,q)\in \N^2$, $ \DSum{k=0}{n}
\dbinom{k+q}{ q}= \dbinom{n+q+1}{ q+1}$.
\item
Soit $(X_i)_{i\in\N^*}$ une suite de variables al\'{e}atoires
indépendantes de même loi que $Z$ et pour tout entier $q\geq 1$,
on pose
$S_q =
\DSum{i=1}{q} X_i$.\\
Montrer que la loi de $S_q$ est d\'{e}finie par :
\[
\forall\, n \in \N,  \quad \Prob(\Ev{S_q=n})= \dbinom{n+q-1}{q-1}\left(
\dfrac12\right)^{n+q}
\]
\item
Calculer $r(S_q)$. En d\'{e}duire que
\[
\DSum{n=0}{+\infty} \dbinom{n+q-1}{ q-1} \left(
\dfrac{1}{4}\right)^{n} = \left( \dfrac{4}{3}\right)^q
\]
\end{noliste}
\item
On suppose dans cette question  que $Z$ représente le nombre de
lionceaux devant naître en 2014 d'un couple de lions. Chaque
lionceau a la probabilité $\dfrac{1}{2}$ d'être mâle ou 
femelle,
indépendamment des autres. On note $F$ la variable aléatoire
représentant le nombre de femelles devant naître en 2014.\\
Déterminer la loi de $F$.
\end{noliste}


\newpage


\section*{Planche 2}

 % ESCP 2003
\noindent
Soit $a$ un nombre r\'eel tel que $ 0 < a < 1$ et $ b$ un nombre r\'eel 
strictement positif.\\
On consid\`ere un couple $(X, Y)$ de variables al\'eatoires \`a valeurs 
dans  $\N^2$,
dont la loi de probabilit\'e est donn\'ee par:
\[
\Prob(\Ev{X=i}\cap \Ev{Y=j})=\begin{cases}\hfill 0\hfill &\text{ si 
$i<j$}\cr
\dfrac{b^i\, \ee^{-b}a^j(1-a)^{i-j}}{j!(i-j)!}&\text{ si $i\geq 
j$}\end{cases}
\]

\begin{noliste}{1.}
\item
D\'eterminer la loi de probabilit\'e de $X$. D\'eterminer, si elles 
existent, son esp\'erance et sa variance.
\item
D\'eterminer la loi de probabilit\'e de $ Y$.
\item
Les variables $X$ et $Y$ sont-elles ind\'ependantes ?
\item
Soit $Z$ la variable al\'eatoire  $Z = X-Y$. D\'eterminer sa loi.
\item
Les variables $Y$ et $Z$ sont-elles ind\'ependantes ?
\end{noliste}


\newpage


\section*{Planche 3}


\noindent
Soit $E$ un $\R$ espace vectoriel et $u$ un endomorphisme de $E$. Si 
$F$ est un sous-espace vectoriel de $E$, on dit qu $F$ est stable par 
$u$ si : $\forall x \in F$, $u(x) \in F$. Pour tout entier $k \geq 1$, 
on note $u^k$ l'application $u \circ u \circ \cdots \circ u$ où $u$ 
apparaît $k$ fois.\\
Soient $n \geq 1$ un entier et $p$ un projecteur. On suppose que $u^n$ 
est l'application linéaire identité et que $\im(p)$ est stable par $u$. 
On pose :
\[
  q \ = \ \dfrac{1}{n} \ \Sum{k=1}{n} u^k \circ p \circ u^{n-k}
\]
\begin{noliste}{1.}
  \item Montrer que $\im(q) \subset \im(p)$ et que $p \circ q = q$.
  
  \item Montrer : $q \circ u = u \circ q$.
  
  \item Montrer : $q \circ p =p$.
  
  \item Montrer que $q$ est un projecteur.
  
  \item Montrer que $\kr(q)$ est un supplémentaire de $\im(p)$ stable 
  par $u$.
\end{noliste}





\end{document}
