\documentclass[11pt]{article}%
\usepackage{geometry}%
\geometry{a4paper,
  lmargin=2cm,rmargin=2cm,tmargin=2.5cm,bmargin=2.5cm}
  
\input{../../../macros.tex}



\begin{document}
\begin{flushleft}
ECE2 \\
Mathématiques
\end{flushleft}


\vspace{0.1cm}

\begin{center}
\textbf{\Large{Programme de colle - Semaine 7}}
\end{center}

\hrule

\vspace*{0,2cm}

\section*{Notation}

\noindent
On adoptera les principes suivants pour noter les étudiants :
\begin{noliste}{$\stimes$}
\item si l'étudiant sait répondre à la question de cours, il 
aura une note $>8$.
\item si l'étudiant ne sait pas répondre à la question de 
cours ou s'il y a trop d'hésitations, il aura une note $\leq 8$.
\end{noliste}

\section*{Questions de cours}

\begin{noliste}{$\sbullet$}
\item {\bf Le noyau et l'image sont des ev}\\
Soit $f \in \LL{E,F}$. Alors : 
\begin{noliste}{$\stimes$}
  \item $\kr(f)$ est un sous espace vectoriel de $E$,
  
  \item $\im(f)$ est un sous espace vectoriel de $F$.
\end{noliste}

\begin{proof}[Preuve]~
  \begin{noliste}{$\sbullet$}
    \item \dashuline{$\kr(f)$} : 
      \begin{noliste}{-}
        \item $\kr(f) \subset E$ par définition.
        
        \item $0_{E} \in \kr(f)$ car $f(0_{E}) = 0_{F}$ 
        
        \item Soit $(u_1,u_2) \in (\kr(f))^2$ et soit $(\lambda_1,
        \lambda_2) \in \R^2$, vérifions : 
        $\lambda_1 \cdot u_1 + \lambda_2 \cdot u_2 \in \kr(f)$, \ie 
        montrons : $f(\lambda_1 \cdot u_1 + \lambda_2 \cdot u_2)=
        0_{F}$.
        \[
          \begin{array}{rcl@{\quad}>{\it}R{4cm}}
            f(\lambda_1 \cdot u_1 + \lambda_2 \cdot u_2) & = & 
            \lambda_1 \cdot f(u_1) + \lambda_2 \cdot f(u_2) 
            & (car $f$ est linéaire) 
            \nl
            \nl[-.2cm]
            &=& \lambda_1 \cdot 0_F + \lambda_2 \cdot 0_F 
            & (car $u_1\in \kr(f)$ et $u_2\in\kr(f)$)
            \nl
            \nl[-.2cm]
            &=& 0_F
          \end{array}
        \]
        Donc $\lambda_1 \cdot u_1 + \lambda_2 \cdot u_2\in\kr(f)$.
      \end{noliste}
      $\kr(f)$ est donc un sous espace vectoriel de $E$.      
      
    \item \dashuline{$\im(f)$} :
      \begin{noliste}{-}
        \item $\im(f) \subset F$ par définition.
        
        \item $0_{F} \in \im(f)$ car $f(0_{E}) = 0_{F}$
        
        \item Soit $(v_1,v_2) \in (\im(f))^2$ et soit $(\lambda_1,
        \lambda_2)\in \R^2$, vérifions : 
        $v_3= \lambda_1 \cdot v_1 + \lambda_2 \cdot v_2 \in \im(f)$, 
        c'est-à-dire :
        \[
          \exists u_3 \in E, \ \ f(u_3) \ = \ v_3.
        \]
        Comme $v_1$ et $v_2$ appartiennent à $\im(f)$, on sait alors :
        \[
          \exists (u_1,u_2) \in E^2, \ f(u_1) = v_1 \ \text{ et } \ 
          f(u_2) =v_2
        \]
        Donc, comme $f$ est linéaire :
        \[
          v_3 \ = \ \lambda_1 \cdot v_1 + \lambda_2 \cdot v_2 \ = \ 
          \lambda_1 \cdot f(u_1) + \lambda_2 \cdot f(u_2) \ = \ 
          f(\lambda_1 \cdot u_1 + \lambda_2 \cdot u_2)
        \]
        Donc en posant $u_3 = \lambda_1 \cdot u_1 + \lambda_2 \cdot 
        u_2$, on a bien $f(u_3) = v_3$, donc $v_3 \in \im(f)$. 
      \end{noliste}
      On conclut que $\im(f)$ est un sous-espace vectoriel de $F$.
  \end{noliste}
\end{proof}


\newpage


\item {\bf Caractérisation de l'image}\\
Soit $f : E \to F$ une application linéaire et soit $(e_1,\ldots,e_n)$ 
une base de $E$. Alors
\[
\im(f) \ = \ \Vect{ f(e_1),\ldots,f(e_n)}.
\]

\begin{proof}[Preuve]~\\
Montrons que $\im(f) \ = \ \Vect{ f(e_1),\ldots,f(e_n)}$ par 
double inclusion.
\begin{noliste}{$\sbullet$}
  \item \dashuline{$\im(f) \ \subset \ \Vect{f(e_1),\ldots,f(e_n)}$ : } 
  \\[.2cm]
  Soit $v \in \im(f)$, alors il existe $u\in E$ tel que $v=f(u)$. De 
  plus, $u \in E$, donc il s'écrit comme une combinaison linéaire de la 
  base canonique, \ie il existe $(\lambda_1,\ldots,\lambda_n)\in\R^n$ 
  tel que :
  \[
    u = \Sum{k=1}{n} \lambda_k \cdot e_k
  \]
  Donc : 
  \[
    v = f(u) = f\left( \Sum{k=1}{n} \lambda_k \cdot e_k \right) \ 
    = \ \Sum{k=1}{n} \lambda_k \cdot f(e_k) .
  \]
  $v$ est donc une combinaison linéaires des $(f(e_1),\ldots,f(e_n))$, 
  donc $v \in \Vect{f(e_1),\ldots,f(e_n)}$.
  
  \item \dashuline{$ \Vect{f(e_1),\ldots,f(e_n)} \ \subset \ 
  \im(f)$ : } 
  \\[.2cm]
  Soit $v \in \Vect{f(e_1),\ldots,f(e_n)}$, donc $v$ s'écrit comme 
  combinaison linéaire des $(f(e_1),\ldots,f(e_n))$. Ainsi, il existe 
  $(\lambda_1,\ldots,\lambda_n)\in\R^n$ tel que :
  \[
    v = \Sum{k=1}{n} \lambda_k \cdot f(e_k) = f \left( \Sum{k=1}{n} 
    \lambda_k \cdot e_k\right)
  \]
  donc en posant $u = \Sum{k=1}{n} \lambda_k \cdot e_k$, on a bien 
  $f(u) = v$. On en déduit que $v \in \im(f)$. 
\end{noliste}
\end{proof}




\item {\bf Caractérisation des applications injectives / surjectives}\\
Soit $f\in\LL{E,F}$. Alors
\begin{noliste}{1.}
\item $f$ est injective si et seulement si $\kr(f) = \{0_{E}\}$ 
(\ie $\forall u \in E$, $f(u)=0_{F} \Rightarrow u = 0_{E}$).
\item $f$ est surjective si et seulement si $\im(f) = F$.
\end{noliste}


\begin{proof}[Preuve]~
  \begin{noliste}{1.}
    \item On raisonne par double implication.
    \begin{noliste}{}
      \item[\quad ($\Rightarrow$)] Supposons $f$ est injective alors, 
      par définition : 
      \[
        (u_1,u_2) \in E^2, \ f(u_1) = f(u_2) \Rightarrow u_1 = u_2
      \]
      Soit $u\in\kr(f)$.\\
      Alors $f(u)=0_F=f(0_E)$, \ie $f(u)=f(0_E)$.\\ 
      Donc, comme $f$ est injective, $u=0_E$.\\
      Donc $\kr(f) = \{0_{E}\}$.
      
      \item[\quad ($\Leftarrow$)] Supposons $\kr(f) = \{0_{E}\}$.\\ 
      Soient $(u_1,u_2) \in E^2$ tel que $f(u_1) = f(u_2)$, alors par 
      linéarité de $f$ :
      \[
        f(u_1) = f(u_2) \ \Leftrightarrow \ f(u_1) - f(u_2) = 0_{F} \ 
        \Leftrightarrow \ f(u_1-u_2)= 0_{F} \ \Leftrightarrow \ u_1-u_2 
        \in \kr(f)
      \]
    On en déduit que $u_1-u_2=0_{F}$ donc $u_1=u_2$, \ie $f$ est 
    injective.
  \end{noliste}
  
  
  \newpage
  
  
  \item On procède par double implication.
  \begin{noliste}{}
    \item[\quad ($\Rightarrow$)] Supposons $f$ est surjective, alors 
    pour tout $v \in F$, il existe $u \in E$ tel que $f(u) = v$.\\ 
    Donc $v \in \im(f)$. On en déduit : $\im(f) = F$.
    
    \item[\quad ($\Leftarrow$)] Supposons $\im(f) = F$.\\
    Soit $v \in F$. Alors $v \in \im(f)$.\\
    Donc, par définition de $\im(f)$, il existe $u \in E$ tel que :
    $v=f(u)$.\\
    On en déduit que $f$ est surjective.
  \end{noliste}
\end{noliste}
\end{proof}
\end{noliste}


\section*{Connaissances exigibles}

\subsection*{Algèbre linéaire}

\begin{noliste}{$\sbullet$}
  \item Application linéaire, endomorphisme, isomorphisme, 
  automorphisme.
\item Noyau, image, caractérisation des injections et surjections, 
caractérisation de $\im(f)$.
\item Rang, théorème du rang, caractérisation des isomorphismes.
\item Application linéaire associée à une matrice.
\item Matrice associée à une application linéaire.
\item {\bf Aucun résultat de réduction n'est au programme.}
\end{noliste}






\end{document}
