\documentclass[11pt]{article}%
\usepackage{geometry}%
\geometry{a4paper,
  lmargin=2cm,rmargin=2cm,tmargin=1.5cm,bmargin=1.5cm}
  
\input{../../../macros.tex}



\begin{document}
\begin{flushleft}
ECE1 \\
Mathématiques
\end{flushleft}

\begin{center}
\textbf{\Large{Colles - Semaine 8}}
\end{center}

\hrule

\vspace*{0,2cm}



\section{Série 1}

\subsection*{Exercice 1}
\noindent
  On considère les matrices $A =
  \begin{smatrix}
    2 & \frac{2}{3} \\[.2cm]
    -\frac{5}{2} & -\frac{2}{3}
  \end{smatrix}$
  et
  $P =
  \begin{smatrix}
    -2 & -2 \\
    3 & 5
  \end{smatrix}
  $.
  \begin{noliste}{1.}
  \item Les matrices $A$ et $P$ sont-elles inversibles ?
  \item On donne $P^{-1}=\dfrac{1}{4} 
  \begin{smatrix}
    -5 & -2\\
    3 & 2
  \end{smatrix}
  $. Calculer $B = P^{-1} A P$.
  \item En déduire $B^n$ puis $A^n$ pour tout $n\in \N$.
  \item Déterminer l'ensemble des suites réelles $(x_n)$ et $(y_n)$
    qui vérifient : $\forall n \in \N, \left\{
    \begin{array}{rcrrr}
      x_{n+1} & = & 2 x_n & + & \frac{2}{3} y_n \\[.2cm]
      y_{n+1} & = & -\frac{5}{2} x_n & - & \frac{2}{3} y_n
    \end{array}
    \right.
    $
  \end{noliste}

\subsection*{Exercice 2}
\noindent
On considère la matrice $A = 
      \begin{smatrix}
        1 & 1 & 1\\
        0 & 1 & 1\\
        0 & 0 & 1\\
      \end{smatrix}$. On pose $B = A - I_3$.
      \begin{noliste}{1.}
       \item Soit $n\in\N$. Calculer $B^n$.
       \item On donne le résultat suivant :\\
    Soit $(M,N)\in\left(\M{p}\right)^2$ tel que $MN=NM$, alors
    \[
     \forall n \in\N, \ (M+N)^n = \DSum{k=0}{n} \dbinom{n}{k} 
     M^k N^{n-k}
    \]
    Soit $n\in\N$. En déduire une expression de $A^n$.
      \end{noliste}

\newpage

\section{Série 2}

\subsection*{Exercice 1}
\noindent
On considère la matrice $A =
      \begin{smatrix}
        -1 & -2 \\
        3 & 4 \\
      \end{smatrix}$.
    \begin{noliste}{1.}
    \item Calculer $A^2 - 3A + 2I_2$. En déduire que $A$ est
      inversible et calculer son inverse.  
    \item On donne le résultat suivant :\\
    Soit $(M,N)\in\left(\M{p}\right)^2$ tel que $MN=NM$, alors
    \[
     \forall n \in\N, \ (M+N)^n = \DSum{k=0}{n} \dbinom{n}{k} 
     M^k N^{n-k}
    \]
    Prouver l'existence de deux suites $(u_n)$ et $(v_n)$ telles
      que : $\forall n \in \N, A^n = u_n A + v_n I_2$.
    \item Que peut-on dire de la suite $(u_n+v_n)$ ?
    \item En déduire l'expression de $(u_n)$, $(v_n)$ et $A^n$.
    \end{noliste}


\subsection*{Exercice 2}
\noindent
On considère la matrice $A$ définie par $A = 
\begin{smatrix} 1 & 0 & 0 \\ 6 & -5 & 6 \\ 3 & -3 & 4 \end{smatrix}$.
\begin{noliste}{1.}
\item Démontrer qu'il existe une suite $(a_n)_{n \in \N}$ telle que 
pour tout $n \in \N$, on ait:
\[
A^n = \begin{smatrix} 1 & 0 & 0 \\ 2a_n & 1 - 2a_n & 2a_n \\ a_n & 
-a_n & 1+a_n \end{smatrix}
\]
\item Montrer que la suite $(a_n)$ est arithmético-géométrique. En 
déduire $a_n$ en fonction de $n$, puis $A^n$ en fonction de $n$.
\end{noliste}

\newpage

\section{Série 3}

\subsection*{Exercice 1}
\noindent
Soit $M$ une matrice de $\M{n}$ telle que : $~^tM = -M$. Soit $X 
\in\M{n,1}$ et $U = ~^tXMX$.
  \begin{noliste}{1.}
  \item Quel est la taille de la matrice $U$ ? En déduire $~^tU$.
  \item En déduire que $U = 0$.
  \item Montrer que $I+M$ est inversible.\\
    \emph{(on pourra supposer $(I+M)X = 0$ et calculer $~^t(MX)(MX)$
      dans ce cas)}
  \item Soit $A = (I-M)(I+M)^{-1}$. Montrer que $~^tA =
    (I-M)^{-1}(I+M)$.
  \item À quelle condition a-t-on $~^tA = A^{-1}$ ? Montrer que cette
    condition est vérifiée.
  \end{noliste}

\subsection*{Exercice 2}
\noindent
\begin{noliste}{1.}
\item Soit $A = \begin{smatrix} 0 & 1 & 0   \\ -1& 2 & 0  \\ 1 & 0 & -1 
 \end{smatrix}$. 
 Montrer que $A^3 - A^2 - A + I = 0$. En déduire que $A$ est inversible 
et calculer $A^{-1}$.
 Montrer ensuite que $A^2$ est inversible et calculer $(A^2)^{-1}$.
\item Soit $C = \begin{smatrix} -2 & -1 & -1 \\ 3 &  2 & 1 \\  1 & 1 & 
0 \end{smatrix}$. Calculer $C^3$, en déduire que $C$ n'est pas 
inversible.
\end{noliste}




\end{document}
