\documentclass[11pt]{article}%
\usepackage{geometry}%
\geometry{a4paper,
  lmargin=2cm,rmargin=2cm,tmargin=1.5cm,bmargin=1.5cm}
  
\input{../../../macros.tex}



\begin{document}
\begin{flushleft}
ECE2 \\
Mathématiques
\end{flushleft}

\begin{center}
\textbf{\Large{Colles - Semaine 8}}
\end{center}

\hrule

\vspace*{0,2cm}


\begin{exercice}~\\
On considère l'espace vectoriel $\M{2}$ muni de sa base canonique 
$\B=(M_1,M_2,M_3,M_4)$ avec
\[
  M_1= 
  \begin{smatrix} 
    1 & 0\\ 
    0 & 0 
  \end{smatrix}
  , \ M_2 = 
  \begin{smatrix} 
    0 & 1\\
    0 & 0 
  \end{smatrix}
  , \ M_3= 
  \begin{smatrix} 
    0 & 0\\ 
    1 & 0 
  \end{smatrix}
  , \ M_4= 
  \begin{smatrix} 
    0 & 0\\ 
    0 & 1 
  \end{smatrix}.
\]
Soit $f: \left\{\begin{array}{ccc}
\M{2} & \rightarrow & \M{2}\\
M & \mapsto & ^t M
\end{array}\right.$ et $g: \left\{\begin{array}{ccc}
\M{2} & \rightarrow & \M{2}\\
M & \mapsto & M+ ^t M
\end{array}\right.$
\begin{noliste}{1.}
\item Montrer que $f \in \LL{\M{2}}$ et $g \in \LL{\M{2}}$.
\item Déterminer la matrice $A$ de $f$ relativement à la base $\B$.
\item En déduire sans calcul supplémentaire la matrice de $g$ 
relativement à la base $\B$.
\item Les applications $f$ et $g$ sont-elles des automorphismes ? Si 
oui, déterminer l'application réciproque.
\item Si $a$ et $b$ sont deux automorphismes, est-ce que $a+b$ est 
également un automorphisme ?
\end{noliste}
\end{exercice}


\begin{exercice}~\\
Soit $f \in \LL{\R^3}$ de matrice $A= \begin{smatrix} 0&1&-1\\ 0&-1&1\\ 0&-1&1 
\end{smatrix}$ dans la base canonique de $\R^3$.
\begin{noliste}{1.}
\item Montrer que $f \circ f =0$.
\item Sans calcul, déterminer si $f$ est bijectif.
\item Montrer que $\im(f)$ est inclus dans $\kr(f)$.
\item En déduire les dimensions de ces 2 espaces vectoriels.
\item Déterminer des bases de $\kr(f)$ et $\im(f)$.
\item Soit $u \notin \kr(f)$ et $v \in \kr(f)$.
\begin{noliste}{a)}
\item Montrer que la famille $(u,f(u))$ est libre dans $\R^3$.
\item Montrer que la famille $(u,f(u),v)$ est une base de $\R^3$.
\item Déterminer la matrice de $f$ dans la base $(u,f(u),v)$.
\end{noliste}
\end{noliste}
\end{exercice}


\begin{exercice}~\\
On note $f$ l'application définie sur $\R^3$ par : 
$f((x,y,z))=(y+z,y,x+y)$
\begin{noliste}{1.}
\item Montrer que $f$ est un endomorphisme de $\R^3$.
\item Déterminer le noyau de $f$. En déduire le rang de $f$.
\item Déterminer l'image de $f$.
\item Montrer que $f$ est un automorphisme de $\R^3$.
\item Montrer que $H=\{ u \in \R^3 \ / \ f(u)=u\}$ est un espace 
vectoriel réel.\\
Déterminer une base de $H$.
\item On note $F$ l'ensemble défini par $F=\{(x,y,z)\in\R^3 \ / \ 
y=0\}$.\\
Montrer que $f$ stabilise $F$, \ie $f(F)\subset F$.
\end{noliste}
\end{exercice}


\newpage


\begin{exercice}~\\
Soit $E$ un espace vectoriel de dimension finie.\\
On dit qu'un endomorphisme $f$ de $E$ est un projecteur si $f\circ 
f=f$.\\
On considère $p$ et $q$ deux projecteurs de $E$.
\begin{noliste}{1.}
\item Montrer que $p+q$ est un projecteur de $E$ si et seulement si 
$p\circ q = q \circ p=0$.
\item On suppose dans cette question que $p+q$ est un projecteur de 
$E$.\\
Montrer que
\[
  \kr(p+q)=\kr(p)\cap\kr(q)
\]
\[
  \im(p)\cap\im(q)=\{0\}
\]
\[
  \im(p+q)=\im(p)+\im(q)
\]
(si $A$ et $B$ sont deux espaces vectoriels, on note $A+B=\{a+b \ / \ 
a\in A \mbox{ et } b\in B\}$)
\end{noliste}
\end{exercice}


\begin{exercice}~\\
L'application $f$ désigne un endomorphisme de $\R^n$. On munit $\R^n$ 
d'une base $(e_1,\cdots, e_n)$.
\begin{noliste}{1.}
\item On suppose que 
\[
\forall x\in\R^n, \ \exists \lambda_x \in\R \mbox{ tel que } 
f(x)=\lambda_x x
\]
\begin{noliste}{a)}
\item \'Ecrire de deux manières différentes le vecteur 
$f(e_1+\cdots+e_n)$.
\item En déduire qu'il existe un réel $\lambda$ tel que $f=\lambda 
\cdot \id$.
\end{noliste}
\item Soit $x$ un vecteur non nul de $\R^n$.\\
Justifier qu'il existe une base de $\R^n$ de la forme 
$(x,\eps_2,\hdots,\eps_n)$.\\
On note alors $p_x$ l'application de $\R^n$ dans $\R^n$ définie par :
\[
\forall (a,b_2,\hdots, b_n)\in\R^n, \ p_x\left( a\cdot x + 
\Sum{k=2}{n} b_k\cdot\eps_k\right)=a\cdot x
\]
\begin{noliste}{a)}
\item Montrer que $p_x$ est un endomorphisme de $\R^n$.
\item Montrer que pour tout $z\in\R^n$, on a
\[
p_x(z)=z \ \Leftrightarrow \ z\in\Vect{x}
\]
\end{noliste}
\item Montrer l'équivalence suivante :
\[
\forall g\in\LL{\R^n}, \ f\circ g=g\circ f \ \Leftrightarrow \ 
\exists \lambda\in\R, \ f=\lambda \cdot \id
\]
\end{noliste}
\end{exercice}


\end{document}
