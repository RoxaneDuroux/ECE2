\documentclass[11pt]{article}%
\usepackage{geometry}%
\geometry{a4paper,
  lmargin=2cm,rmargin=2cm,tmargin=1.5cm,bmargin=1.5cm}
  
\input{../../../macros.tex}



\begin{document}
\begin{flushleft}
ECS2 \\
Mathématiques
\end{flushleft}

\begin{center}
\textbf{\Large{Colles - Semaine 9}}
\end{center}

\hrule

\vspace*{0,2cm}

\section*{Série 1}
\subsection*{Question de cours}
\noindent
Calculer $\dint{1}{+\infty} \dfrac{\ln(x)}{x(1+(\ln(x))^4} \dx$.

\subsection*{Exercice} % EDHEC 2010
\noindent
Dans cet exercice, $a$ désigne un réel strictement positif.\\
On considère deux variables aléatoires $X$ et $Y$, définies sur
un espace probabilisé $(\Omega,\mathcal{A},\Prob)$, indépendantes, et 
suivant toutes deux la loi uniforme sur $[0,a[$.\\
On pose $Z=|X-Y|$ et on admet que $-Y$, $X-Y$ et $Z$ sont des
variables aléatoires à  densité, elles aussi définies sur l'espace
probabilisé $(\Omega,\mathcal{A},\Prob)$.
\begin{noliste}{1.}
\item 
\begin{noliste}{a)}
\item Déterminer une densité de $-Y$.
\item En déduire que la variable aléatoire $X-Y$ admet pour densité la
fonction $g$ définie par :\vspace{-0.3cm}
$$
g(x)=\begin{cases}
\dfrac{a-|x|}{a^{2}} & \text{ si } x\in[-a,a]\\
0 & \text{ sinon }\end{cases}
\vspace{-0.3cm}$$
On note $G$ la fonction de répartition de $X-Y$.
\end{noliste}
\item 
\begin{noliste}{a)}
\item Exprimer la fonction de répartition $H$ de la variable $Z$ en 
fonction
de $G$.
\item En déduire qu'une densité de $Z$ est la fonction $h$ définie par :
$\quad 
h(x)=\begin{cases}
\dfrac{2(a-x)}{a^{2}} & \text{ si } x\in[0,a]\\
0 & \text{ sinon }\end{cases}
$
\end{noliste}
\item Montrer que $Z$ possède une espérance et une variance et les 
déterminer.
\item Simulation informatique.\\
On rappelle qu'en \Scilab{}, la commande \texttt{rand()} permet
de simuler la loi uniforme sur $[0,1[$.\\
Compléter la déclaration de fonction suivante pour qu'elle retourne
à  chaque appel un nombre réel choisi selon la loi de $Z$.

\begin{scilab}
  & \tcFun{function} \tcVar{v}=z(\tcVar{a}) \nl %
  & \quad x = ........... \nl %
  & \quad y = ........... \nl %
  & \quad \tcVar{v} = ........... \nl %
  & \tcFun{endfunction}
\end{scilab}
\end{noliste}




\newpage

\section*{Série 2}
\subsection*{Question de cours}
\noindent
Déterminer les valeurs propres de $A=
\begin{smatrix}
 3 & 0 & 1\\
 -1 & 2 & -1\\
 -2 & 0 & 0
\end{smatrix}$.

\subsection*{Exercice} % ESCP 2017
\noindent
Dans cette exercice, $E=\R_n[X]$ désigne l'espace vectoriel des 
polynômes à coefficients réels de degré inférieur ou égal à $n$, avec 
$n\in\N^*$, $D$ représente l'endomorphisme de dérivation $D:P\mapsto 
P'$.
\begin{noliste}{1.}
 \item Montrer que $\varphi : P \mapsto P\left(\dfrac{X}{2}\right)$ est 
 un automorphisme de $E$. Les endomorphismes $\varphi$ et $D$ 
 commutent-ils ?
 
 \item Soit $\Phi$ l'application définie sur $E$ par :
 \[
  \forall P\in E, \ \Phi(P) = \Sum{k=0}{+\infty} P^{(k)}\left( 
  \dfrac{X}{2}\right)
 \]
 \begin{noliste}{a)}
  \item Montrer que $\Phi$ est bien définie et appartient à $\LL{E}$.
  
  \item Montrer que $\varphi^{-1} \circ \Phi = (I-D)^{-1}$ et que $\Phi 
  \circ \varphi^{-1} = (I-2D)^{-1}$, où $I$ représente l'endomorphisme 
  identité de $E$.
  
  \item En déduire que $\Phi$ est un automorphisme de $E$.
 \end{noliste}
 
 \item 
 \begin{noliste}{a)}
  \item Déterminer les valeurs propres possibles de $\Phi$.
  
  \item Soit $(\lambda, P)\in \R\times E$ un couple propre de $\Phi$, 
  c'est-à-dire vérifiant $\Phi(P)=\lambda P$, avec $P\neq 0$. Montrer 
  que cette équation est équivalente à l'équation :
  \[
   \mu P(X) = P(2X) - 2P'(2X)
  \]
  où $\mu$ s'exprime en fonction de $\lambda$.
  
  \item Soit $P$ un polynôme propre unitaire (\ie de coefficient 
  dominant égal à $1$) de $\Phi$ de degré $n$. Déterminer l'expression 
  de ses coefficients en fonction de $n$.\\
  %Que peut-on en conclure ?
 \end{noliste}
\end{noliste}





\newpage
 
\section*{Série 3}
\subsection*{Question de cours}
\noindent
Soit $X$ une \var à densité. Soit $(a,b)\in\R^2$. Étudier la \var 
$Y=aX+b$.

\subsection*{Exercice} % ESCP 2017
\noindent
\begin{noliste}{1.}
 \item Soit $F$ la fonction définie sur $\R$ par : $\forall x\in\R$, 
 $F(x)=\exp\left(-\ee^{-x}\right)$.
 \begin{noliste}{a)}
  \item Justifier que $F$ est une fonction de répartition.
  \item Soit $X$ une \var de fonction de répartition $F$. Déterminer 
  une densité $f$ de $X$.
 \end{noliste}
 {\it On suppose désormais que $X$ est une \var sur $(\Omega, \A, 
 \Prob)$ et que toutes les \var citées sont définies sur ce même 
 espace.}
 
 \item 
 \begin{noliste}{a)}
  \item Soit $Z=\ee^{-X}$. Justifier que $Z$ est une variable aléatoire 
  réelle sur $(\Omega, \A, \Prob)$ et déterminer sa loi.
  \item On rappelle que {\tt grand(1,1,\ttq{}exp\ttq{},1)} simule
    une variable aléatoire et suivant une loi
    exponentielle de paramètre $1$. Écrire une
    fonction {\tt Scilab} qui simule la variable
    aléatoire $X$.
  \item Soient $x$ et $y$ deux réels strictement positifs. Établir une 
  relation entre la probabilité conditionnelle 
  $\Prob_{\Ev{X\leq -\ln(X)}}(\Ev{X\leq -\ln(x+y)})$ et 
  $\Prob(\Ev{X\leq - \ln(y)})$.
 \end{noliste}
 
 \item Soit $(Y_i)_{i\in\N^*}$ une suite de \var définies sur $(\Omega, 
 \A, \Prob)$, mutuellement indépendantes et de même loi exponentielle 
 de paramètre $1$.\\
 Soit d'autre part $L$ une \var de loi de Poisson de paramètre $1$ 
 indépendante des variables aléatoires de la suite $(Y_i)_{i\in\N^*}$.\\
 On définit $S$ par :
 \begin{noliste}{$\stimes$}
  \item si $L(\omega)=0$, alors $S(\omega)=0$.
  \item si $L(\omega)=k$, avec $k\in\N^*$, alors $S(\omega) = 
  \max(Y_1(\omega), \hdots, Y_k(\omega))$.
 \end{noliste}
 \begin{noliste}{a)}
  \item Soit $k$ un entier naturel non nul.\\
  Déterminer la loi de la variable aléatoire $S_k=\max(Y_1,\hdots, 
  Y_k)$.
  \item Démontrer que pour tous réels $a$ et $b$ tels que $0<a<b$, 
  on a :
  \[
   \Prob(\Ev{a\leq S \leq b}) = \Prob(\Ev{a\leq X \leq b})
  \]
  \item Calculer $\Prob(\Ev{S=0})$.
 \end{noliste}
\end{noliste}


\end{document}

