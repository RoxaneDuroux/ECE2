\documentclass[11pt]{article}%
\usepackage{geometry}%
\geometry{a4paper,
  lmargin=2cm,rmargin=2cm,tmargin=1.5cm,bmargin=1.5cm}
  
\input{../../../macros.tex}



\begin{document}
\begin{flushleft}
K \\
Mathématiques
\end{flushleft}

\begin{center}
\textbf{\Large{Colles - Semaine 9}}
\end{center}

\hrule

\vspace*{0,2cm}

\section*{Planche 1}

\subsection*{Question de cours}

\noindent
Intégrale de Riemann au voisinage de $+\infty$


\subsection*{Exercice}

\noindent
Pour tout $n\in\N^{*}$, on pose $
u_{n} = \dint{0}{+\infty}\dfrac{\ee^{-x}}{x+\frac{1}{n}}\dx$.

\begin{noliste}{1.}
\item Montrer que la suite $(u_{n})_{n\in\N^{*}}$ est bien définie.
\item Pour tout $n\in\N^{*}$, on pose alors $
v_{n}=\dint{0}{1}\dfrac{\ee^{-x}}{x+\frac{1}{n}}\dx$
et $w_{n}=\dint{1}{+\infty}\dfrac{\ee^{-x}}{x+\frac{1}{n}}\dx$.

\begin{noliste}{a)}
\item Montrer que: $\forall n\in\N^{*}, \ 0\leq w_{n} \leq \dfrac{1}{\ee}$.
\item Montrer que: $\forall n\in\N^{*},\ v_{n}\geq \dfrac{1}{\ee}\ln(n+1)$.
\item Donner la limite de la suite $(u_{n})$.
\end{noliste}
\item On se propose de déterminer un équivalent de $u_{n}$ lorsque $n$
est au voisinage de $+\infty$.

\begin{noliste}{a)}
\item Montrer que l'intégrale $I=\dint{0}{1}\dfrac{1-\ee^{-x}}{x} \dx$
est une intégrale convergente.
\item Établir que : $\forall n \in \N^{*}, \ 0 \leq \dint{0}{1} 
\dfrac{1-\ee^{-x}}{x+\frac{1}{n}} \dx \leq I$.
\item En déduire un encadrement de $v_{n}$ valable pour tout 
$n\in\N^{*}$.
\item Donner enfin, en utilisant cet encadrement, un équivalent simple 
de $u_{n}$.
\end{noliste}
\end{noliste}


\newpage


\section*{Planche 2}

\subsection*{Question de cours}

\noindent
Intégrale de Riemann au voisinage de $0$


\subsection*{Exercice}

\noindent
Pour tout $n\in\N^*$, on pose 
$I_n=\dint{-\infty}{+\infty}\frac{1}{(1+x^2)^n} \dx$ et 
$u_n=\sqrt{n}\,I_n$.

On admet que 
$\dint{-\infty}{+\infty}\ee^{-\frac{x^2}{2}}\dx=\sqrt{2\pi}$.
\begin{noliste}{1.}
\item Pour tout $n\in\N^*$, montrer que $I_n$ est convergente. 
\'Etablir une relation de récurrence entre $I_n$ et $I_{n+1}$.

\item Montrer que la suite $(u_n)_{n\in\N^*}$ est monotone et étudier 
sa convergence.

\item Calculer $I_n$, pour tout $n\geq 1$.

\item \begin{noliste}{a)}
	\item Montrer que, pour tout réel $x$, $\ln(1+x^2)\leq x^2$.\\ En 
déduire que pour tout $n\geq 1$, $I_n\geq 
\dint{-\infty}{+\infty}\ee^{-nx^2}\dx$.		
	
	\item Montrer que, pour tout $n\in\N^*$ : 
$\dint{-\infty}{+\infty}\ee^{-n x^2}dx=\frac{\sqrt\pi}{\sqrt{n}}$.
	
	\item En déduire une minoration de la suite $(u_n)_{n\in\N^*}$ 
et conclure que la suite $(u_n)_{n\in\N^*}$ ne tend pas vers $0$ 
lorsque $n$ tend vers $+\infty$.
	
	\end{noliste}
\item Montrer qu'il existe un réel $\alpha$ tel que $\dbinom{2n}{n} 
\eqn\dfrac{\alpha\,4^n}{\sqrt{n}}\cdot$
\end{noliste}



\newpage


\section*{Planche 3}

\subsection*{Question de cours}

\noindent
Théorème d'intégration par parties sur un segment


\subsection*{Exercice}

\noindent
On pose, pour tout $n\in\N$, 
$I_n=\dint{0}{1}\dfrac{x^n}{\sqrt{1-x}}\dx$.\\
\begin{noliste}{1.}
\item Montrer que $I_n$ existe, pour tout $n\in\N$.
\item Montrer que la suite $(I_n)_{n\in\N}$ est convergente.
\item \begin{noliste}{a)}
	\item Montrer que, pour tout $n\in\N^*$, 
$I_n=\dfrac{2n}{2n+1}\,I_{n-1}$.
	\item En déduire l'existence et la nature de la série de terme 
général $v_n=\ln(I_n)-\ln(I_{n-1})$, puis la limite de $(I_n)_{n\in\N}$.
	\end{noliste}
\item Pour tout $n\in\N$, on pose $J_n=\sqrt{n}\,I_n$ et 
$K_n=\sqrt{n+1}\,I_n$.
	\begin{noliste}{a)}
	\item Montrer que les suites $(J_n)_{n\in\N}$ et 
$(K_n)_{n\in\N}$ sont adjacentes.
	\item En déduire qu'il existe un réel $\alpha>0$ tel que 
$I_n\eqn \dfrac{\alpha}{\sqrt{n}}\cdot$
	\end{noliste}
\item \begin{noliste}{a)}
	\item Calculer $I_n$ en fonction de $n$.
	\item On admet la formule de Stirling : $n!
\eqn n^n\,\ee^{-n}\sqrt{2\pi n}$. Montrer que 
$I_n\eqn \ee \,\left(\dfrac{2n}{2n+1}
\right)^{2n+1}\frac{\sqrt{\pi}}{\sqrt{n}}\cdot$
	\item Déterminer la valeur de $\alpha$.
\end{noliste}
\end{noliste}




\end{document}
