\documentclass[11pt]{article}%
\usepackage{geometry}%
\geometry{a4paper,
  lmargin=2cm,rmargin=2cm,tmargin=2.5cm,bmargin=2.5cm}
  
\input{../../../macros.tex}



\begin{document}
\begin{flushleft}
ECE2 \\
Mathématiques
\end{flushleft}


\vspace{0.1cm}

\begin{center}
\textbf{\Large{Programme de colle - Semaine 9}}
\end{center}

\hrule

\vspace*{0,2cm}

\section*{Notation}

\noindent
On adoptera les principes suivants pour noter les étudiants :
\begin{noliste}{$\stimes$}
\item si l'étudiant sait répondre à la question de cours, il 
aura une note $>8$.
\item si l'étudiant ne sait pas répondre à la question de 
cours ou s'il y a trop d'hésitations, il aura une note $\leq 8$.
\end{noliste}

\section*{Questions de cours}


\noindent
La question de cours sera {\bf pour tous les élèves} le calcul d'une 
intégrale à vue (impropre ou non), une IPP ou un changement de variable.

\begin{examples}~
 \[
  \dint{1}{\ee} \dfrac{1}{t\, \ln(t)} \dt, \qquad
  \dint{2}{+\infty} \dfrac{1}{3^t} \dt, \qquad
  \dint{0}{+\infty} \dfrac{t}{(1+t^2)^3} \dt
 \]
\end{examples}


\section*{Connaissances exigibles}

\begin{noliste}{-}
 \item Techniques de calculs d'intégrales sur un segment : IPP, 
 changement de variables (les élèves doivent être capable d'identifier 
 un changement de variable affine et peuvent être guidés pour les 
 autres), intégration à vue.
 \item Utilisation de la définition de la convergence d'une intégrale 
 impropre
 \item Intégrales de Riemann impropres en $0$ ou en $+\infty$ 
 (encadrement, négligeabilité, équivalence).
 \item Critère de comparaison / équivalence / négligeabilité des 
 intégrales impropres de fonctions continues positives
 \item Convergence absolue d'une intégrale
 \item Comparaison série / intégrale
 \item Somme de Riemann (la colle ne doit pas se focaliser sur ce point)
\end{noliste}




\end{document}
