\documentclass[11pt]{article}%
\usepackage{geometry}%
\geometry{a4paper,
  lmargin=2cm,rmargin=2cm,tmargin=2.5cm,bmargin=2.5cm}
  
\input{../../../macros.tex}



\begin{document}
\begin{flushleft}
ECE2 \\
Mathématiques
\end{flushleft}


\vspace{0.1cm}

\begin{center}
\textbf{\Large{Programme de colle - Semaine 13}}
\end{center}

\hrule

\vspace*{0,2cm}

\section*{Notation}

\noindent
On adoptera les principes suivants pour noter les étudiants :
\begin{noliste}{$\stimes$}
\item si l'étudiant sait répondre à la question de cours, il 
aura une note $>8$.
\item si l'étudiant ne sait pas répondre à la question de 
cours ou s'il y a trop d'hésitations, il aura une note $\leq 8$.
\end{noliste}


\section*{Questions de cours}

\noindent
\begin{noliste}{-}
 \item {\bf Transformation affine d'une \var uniforme}\\
 {\it On demandera à l'étudiant de démontrer l'une des {\bf implications} 
 suivantes.}
 \begin{noliste}{1.}
    \item $ X \suit \Uc{0}{1} \quad \Leftrightarrow \quad 
    Y=(b-a)X+a \suit \Uc{a}{b}$
    
    \item $ X \suit \Uc{a}{b} \quad \Leftrightarrow \quad 
    Y=\dfrac{1}{b-a}(X-a) \suit \Uc{0}{1}$
  \end{noliste}
 
 \begin{proof}~
  \begin{noliste}{$\sbullet$}
      \item Notons $h : x \mapsto (b-a)x +a$, de telle sorte que $Y=h(X)$.\\
      Comme $X \suit \Uc{0}{1}$, alors $X(\Omega) = [0,1]$. On en déduit :
      \[
        \begin{array}{rcl@{\quad}>{\it}R{5.5cm}}
          Y(\Omega) &=& \big(h(X)\big)(\Omega) \ = \ h \big(X(\Omega)\big)
          \\[.2cm]
          &=& h([0,1])
          \\[.2cm]
          &=& \left[ h(0), h(1) \right]
          & (car $h$ est continue et strictement croissante sur $[0,1]$)
          \nl
          \nl[-.2cm]
          &=& [a,b]
        \end{array}
      \]
      Donc : $Y(\Omega) = [a,b]$.


      \item Déterminons la fonction de répartition de $Y$, $F_Y$.\\
      Soit $x\in\R$.
      \begin{noliste}{$\stimes$}
        \item Soit $x<a$, alors
        $\Ev{Y\leq x}=\varnothing$ car $Y(\Omega)=[a,b]$. Ainsi :
        \[
          F_Y(x) = \Prob(\Ev{Y \leq x})
          = \Prob(\varnothing) = 0
        \]
        
        \item Soit $x>b$, alors 
        $\Ev{Y\leq x}=\Omega$ car $Y(\Omega)=[a,b]$. Ainsi :
        \[
          F_Y(x) = \Prob(\Ev{Y \leq x}) = \Prob(\Omega) = 1
        \]
  
        \item Soit $x \in [a,b]$.
        \[
          \begin{array}{rcl@{\quad}>{\it}R{4cm}}
            F_Y(x) &=& \Prob(\Ev{Y \leq x})
            \ = \ \Prob(\Ev{(b-a)X+a \leq x})
            \\[.2cm]
            &=& \Prob \left(\Ev{X \leq \dfrac{x-a}{b-a}}\right)
            \ = \ F_X \left( \dfrac{x-a}{b-a} \right)
            \\[.4cm]
            &=& \dfrac{x-a}{b-a} & (par définition de $F_X$ et car $ 
            \dfrac{x-a}{b-a} \in [0,1]$)
          \end{array}
        \]
      \end{noliste}
      On reconnaît la fonction de répartition de la loi $\Uc{a}{b}$. Comme 
      la fonction de répartition caractérise la loi, on en déduit que $Y 
      \suit \Uc{a}{b}$.
    \end{noliste}
 \end{proof}
 
 
 \newpage
 
 
 \item {\bf Méthode d'inversion pour la loi exponentielle}\\
 Soit $\lambda \in \ ]0,+\infty[$.\\
 $X\suit \mathcal{U}([0,1[) \quad \Rightarrow \quad Y=-\dfrac{1}{\lambda} 
 \, \ln(1-X) \suit \Exp{\lambda}$
 
 \begin{proof}~
  \begin{noliste}{$\sbullet$}
    \item Notons $h : x \mapsto - \dfrac{1}{\lambda} \ \ln(1-x)$, de telle 
    sorte que $Y = h(X)$.\\
    Comme $X \suit \Uc{0}{1}$, alors $X(\Omega) = \ ]0,1[$. On en déduit :
    \[
      \begin{array}{rcl@{\quad}>{\it}R{6cm}}
        Y(\Omega) &=& \big(h(X)\big)(\Omega) \ = \ h\big( X(\Omega)\big)
        \\[.2cm]
        &=& h(]0,1[)
        \\[.2cm]
        &=& \left] \dlim{x\to 0} h(0), \dlim{x\to 1} h(x) \right[
        & (car $h$ est continue et strictement croissante sur $]0,1[$ $(*)$)
        \nl
        \nl[-.2cm]
        &=& ]0, +\infty[
      \end{array}
    \]
    Ainsi : $Y(\Omega) = \ ]0,+\infty[$.\\
    On peut démontrer $(*)$ par une rapide étude de fonction :
    \begin{noliste}{$\stimes$}
      \item la fonction $h$ est dérivable (donc continue) sur $]0,1[$ en tant   
      que composée de fonctions dérivables.
      
      \item soit $x \in \ ]0,1[$.
      \[
        h'(x) \ = \ - \dfrac{1}{\lambda} \ \left(- \dfrac{1}{1-x}\right)
        \ = \ \dfrac{1}{\lambda (1-x)} >0
      \]
      Donc la fonction $h$ est strictement croissante sur $]0,1[$.
    \end{noliste}


    \item Déterminons la fonction de répartition de $Y$.
    \begin{noliste}{$\stimes$}
      \item Soit $y\leq 0$, alors $\Ev{Y\leq x}=\varnothing$ car 
      $Y(\Omega)=[0,+\infty[$. Donc :
      \[
        F_Y(y) = \Prob(\Ev{Y\leq y}) = \Prob(\varnothing) = 0
      \]

      \item Soit $y\in \ ]0,+\infty[$.
      \[
        \begin{array}{rcl@{\quad}>{\it}R{5cm}}
          F_Y(y) &=& \Prob(\Ev{Y\leq y})
          \\[.2cm]
          &=& \Prob\left(\Ev{ -\dfrac{1}{\lambda} \ln(1-X) 
          \leq y}\right)
          \\[.6cm]
          &=& \Prob(\Ev{\ln(1-X) \geq -\lambda y}) & (car $-\lambda<0$)
          \nl
          \nl[-.2cm]
          &=& \Prob\left(\Ev{1-X \geq \ee^{-\lambda y}}\right) & (car la 
          fonction $\exp$ est strictement croissante sur $\R$)
          \nl
          \nl[-.2cm]
          &=& \Prob\left(\Ev{ X \leq  1 - \ee^{-\lambda y}}\right)
          \\[.2cm]
          &=& F_X\left( 1-\ee^{-\lambda y} \right)
        \end{array}
      \]
      Or on a les équivalences suivantes :
      \[
        \begin{array}{rcl@{\quad}>{\it}R{5cm}}
          0< y  & \Leftrightarrow & 0 > -\lambda > -\lambda \, y 
          & (car $-\lambda <0$)
          \nl
          \nl[-.2cm]
          & \Leftrightarrow & 1=\ee^0 > \ee^{-\lambda y} > 0 & (car la 
          fonction $\exp$ est strictement croissante sur $\R$)
          \nl
          \nl[-.2cm]
          & \Leftrightarrow & 0 < 1-\ee^{-\lambda y} < 1
        \end{array}
      \]
      De plus, comme $X \suit \Uc{0}{1}$ : $F_X : u \mapsto \left\{
      \begin{array}{l@{\quad}>{}R{2cm}}
        0 & si $u\leq 0$\nl
        u & si $u\in \ ]0,1[$\nl
        1 & si $u\geq 1$
      \end{array}
      \right.$,\\ 
      donc $F_X\left( 1-\ee^{-\lambda y} \right)= 1-\ee^{-\lambda y}$.
    \end{noliste}
  \end{noliste}
  
  
  \newpage
  
  
  \noindent
  Finalement : $F_Y : y \mapsto \left\{
  \begin{array}{l@{\quad}>{}R{1.5cm}}
    0 & si $y\leq 0$\nl
    1-\ee^{-\lambda y} & si $y > 0$
  \end{array}
  \right.$. \\[.2cm]
  On reconnaît la fonction de répartition de la loi exponentielle de 
  paramètre $\lambda$. Or la fonction de répartition caractérise la loi 
  d'une \var, donc $Y\suit \Exp{\lambda}$.
 \end{proof}
 
 
 \item {\bf Propriétés de $\Phi$, où $\Phi$ est la fonction de 
 répartition d'une loi $\Norm{0}{1}$}
 \begin{noliste}{$\sbullet$}
    \item $
      \phi(0) = \Prob(\Ev{X\leq 0})=\dfrac{1}{2}
    $
    
    \item Pour tout $x\in\R$ :
    \[
      \phi(-x) = 1-\phi(x)
    \]
    
    \item Pour tout $x\in\R$ :
    \[
      \Prob(\Ev{\vert X \vert \leq x})=2\phi(x)-1
    \]
  \end{noliste}
 
 \begin{proof}~
  \begin{noliste}{$\sbullet$}
    \item Comme $\varphi$ est une densité, $\dint{-\infty}{+\infty} 
    \varphi(t) \dt$ converge. De plus :
    \[
      \begin{array}{rcl@{\quad}>{\it}R{5cm}}
        \dint{-\infty}{+\infty} \varphi(t)\dt &=& 2 \dint{-\infty}{0} 
        \varphi(t)\dt & (car $\varphi$ est paire)
        \nl
        \nl[-.2cm]
        &=& 2 \, \phi(0)
        \\[.2cm]
        &=& 2 \, \Prob(\Ev{X\leq 0})
      \end{array}
    \]
    
    \item Soit $x\in\R$.
    \[
      \phi(-x) = \dint{-\infty}{-x} \varphi(t)\dt
    \]
    On effectue alors le changement de variable $\Boxed{u =\psi(t)}$, avec 
    la fonction $\psi$ de classe $\Cont{1}$ définie par $\psi : t\mapsto 
    -t$.
    \[
      \left|
      \begin{array}{P{8cm}}
        $u = -t$ \nl
        $\hookrightarrow$ $du = -\dt$ \quad et \quad $dt = - \ du$ \nl
        \vspace{-.4cm}
        \begin{noliste}{$\sbullet$}
          \item $t = -\infty \ \Rightarrow \ u = +\infty$
          \item $t = -x \ \Rightarrow \ u = x$ %
          \vspace{-.4cm}
        \end{noliste}
      \end{array}
      \right.
    \]
    On a donc :
    \[
      \begin{array}{rcl@{\quad}>{\it}R{5cm}}
        \dint{-\infty}{-x} \varphi(t)\dt &=& \dint{x}{+\infty} 
        \varphi(-u) \ du
        \\[.4cm]
        &=& \dint{x}{+\infty} \varphi(u) \ du & (car $\varphi$ est paire)
        \nl
        \nl[-.2cm]
        &=& \dint{-\infty}{+\infty} \varphi(u) \ du - \dint{-\infty}{x} 
        \varphi(u) \ du
        \\[.4cm]
        &=& 1 - \phi(x)
      \end{array}
    \]
    
    \item Soit $x\in\R$.
    \[
      \begin{array}{rcl@{\quad}>{\it}R{5cm}}
        \Prob(\Ev{\vert X \vert \leq x}) &=& \Prob(\Ev{-x \leq X \leq 
        x})
        \\[.2cm] 
        &=& \phi(x)-\phi(-x)
        \\[.2cm]
        &=& \phi(x) - (1-\phi(x))
        \\[.2cm]
        &=& 2 \, \phi(x)-1
      \end{array}
    \]
  \end{noliste}
 \end{proof}
\end{noliste}


\newpage


\section*{Connaissances exigibles}

\begin{noliste}{-}
 \item Définition v.a. à densité, caractérisation fonction de 
  répartition (fdr) et densité de probabilité, lien entre fdr et densité
  \item Formules de calcul de $\Prob(\Ev{a\leq X\leq b})$ et autres, 
  pour $X$ une v.a. à densité
  \item Définition, linéarité et croissance de l'espérance d'une v.a. à 
  densité
  \item Théorème de transfert
  \item Moments d'ordre $r$
  \item Définitions et propriétés de la variance et de l'écart-type
  \item Définition indépendance de deux v.a., lemme des coalitions, 
  $\E(XY)=\E(X)\E(Y)$ et\\ $\V(X+Y)=\V(X)+\V(Y)$ {\bf si $X$ et $Y$ 
  sont indépendantes}
  \item Ensemble image, densité, fonction de répartition, espérance, variance, 
  graphes : loi uniforme, loi exponentielle, lois normales
  \item Transformation affine d'une loi uniforme
  \item Propriétés de la fonction de répartition $\Phi$ d'une loi
  $\Norm{0}{1}$
  \item Transformée affine d'une loi gaussienne
  \item Lecture de la table de la loi $\Norm{0}{1}$
\end{noliste}



\end{document}
