\documentclass[11pt]{article}%
\usepackage{geometry}%
\geometry{a4paper,
  lmargin=2cm,rmargin=2cm,tmargin=1.5cm,bmargin=1.5cm}
  
\input{../../../macros.tex}



\begin{document}
\begin{flushleft}
ECS2 \\
Mathématiques
\end{flushleft}

\begin{center}
\textbf{\Large{Colles - Semaine 13}}
\end{center}

\hrule

\vspace*{0,2cm}

\section*{Série 1}
\subsection*{Question de cours}
\noindent
Énoncer et démontrer l'inégalité de Cauchy-Schwarz.

\subsection*{Exercice} % ESCP 2017
\noindent
\begin{noliste}{1.}
 \item Soit $F$ la fonction définie sur $\R$ par : $\forall x\in\R$, 
 $F(x)=\exp\left(-\ee^{-x}\right)$.
 \begin{noliste}{a)}
  \item Justifier que $F$ est une fonction de répartition.
  \item Soit $X$ une \var de fonction de répartition $F$. Déterminer 
  une densité $f$ de $X$.
 \end{noliste}
 {\it On suppose désormais que $X$ est une \var sur $(\Omega, \A, 
 \Prob)$ et que toutes les \var citées sont définies sur ce même 
 espace.}
 
 \item 
 \begin{noliste}{a)}
  \item Soit $Z=\ee^{-X}$. Justifier que $Z$ est une variable aléatoire 
  réelle sur $(\Omega, \A, \Prob)$ et déterminer sa loi.
  \item On rappelle que {\tt grand(1,1,\ttq{}exp\ttq{},1)} simule
    une variable aléatoire et suivant une loi
    exponentielle de paramètre $1$. Écrire une
    fonction {\tt Scilab} qui simule la variable
    aléatoire $X$.
  \item Soient $x$ et $y$ deux réels strictement positifs. Établir une 
  relation entre la probabilité conditionnelle 
  $\Prob_{\Ev{X\leq -\ln(X)}}(\Ev{X\leq -\ln(x+y)})$ et 
  $\Prob(\Ev{X\leq - \ln(y)})$.
 \end{noliste}
 
 \item Soit $(Y_i)_{i\in\N^*}$ une suite de \var définies sur $(\Omega, 
 \A, \Prob)$, mutuellement indépendantes et de même loi exponentielle 
 de paramètre $1$.\\
 Soit d'autre part $L$ une \var de loi de Poisson de paramètre $1$ 
 indépendante des variables aléatoires de la suite $(Y_i)_{i\in\N^*}$.\\
 On définit $S$ par :
 \begin{noliste}{$\stimes$}
  \item si $L(\omega)=0$, alors $S(\omega)=0$.
  \item si $L(\omega)=k$, avec $k\in\N^*$, alors $S(\omega) = 
  \max(Y_1(\omega), \hdots, Y_k(\omega))$.
 \end{noliste}
 \begin{noliste}{a)}
  \item Soit $k$ un entier naturel non nul.\\
  Déterminer la loi de la variable aléatoire $S_k=\max(Y_1,\hdots, 
  Y_k)$.
  \item Démontrer que pour tous réels $a$ et $b$ tels que $0<a<b$, 
  on a :
  \[
   \Prob(\Ev{a\leq S \leq b}) = \Prob(\Ev{a\leq X \leq b})
  \]
  \item Calculer $\Prob(\Ev{S=0})$.
 \end{noliste}
\end{noliste}




\newpage

\section*{Série 2}
\subsection*{Question de cours}
\noindent
Déterminer le spectre de $A=
\begin{smatrix}
 3 & 0 & 1\\
 -1 & 2 & -1\\
 -2 & 0 & 0
\end{smatrix}$.

\subsection*{Exercice} % ESCP 2017
\noindent
Soit $n\in\N^*$. On note $\langle \ \cdot \ , \ \cdot \ \rangle$ le 
produit scalaire canonique de $\R^n$.\\
On dit que deux familles $(u_1, \ldots, u_k)$ et $(v_1,\ldots, v_k)$ de 
vecteurs de $\R^n$ sont biorthogonales si l'on a :
\[
 \forall (i,j)\in \llb 1,k\rrb^2, \ \langle u_i,v_j\rangle =\left\{
 \begin{array}{ll}
  1 & \mbox{ si $i=j$}\\
  0 & \mbox{ sinon}
 \end{array}
 \right.
\]
\begin{noliste}{1.}
 \item 
 \begin{noliste}{a)}
  \item Montrer que si les familles $(u_1, \ldots, u_k)$ et $(v_1, 
  \ldots , v_k)$ de $\R^n$ sont biorthogonales, alors ces deux familles 
  sont libres. Que peut-on en déduire pour $k$ ?
  
  \item Montrer que si $\B'=(u_1,\ldots, u_n)$ est une base quelconque 
  de $\R^n$, alors il existe une unique base $\mathcal{C}'=(v_1,\ldots,
  v_n)$ telle que $\B'$ et $\mathcal{C}$ soient biorthogonales. La base
  $\mathcal{C}$ s'appelle la base biorthogonale de la base $\B'$.
 \end{noliste}
 
 \noindent
 Dans la suite de l'exercice, on confond tout vecteur de $\R^n$ avec la 
 matrice colonne canoniquement associée.
 
 \item Soit $A$ la matrice carrée d'ordre $n$ pour laquelle il existe 
 un entier naturel $r$, des familles $(u_1,\ldots, u_r)$ et 
 $(v_1,\ldots, v_r)$ de $\R^n$ biorthogonales et $r$ réels non nuls 
 $(\lambda_1, \ldots, \lambda_r)$ tels que :
 \[
  A=\Sum{i=1}{r} \lambda_i \, u_i \, {}^t v_i
 \]
 \begin{noliste}{a)}
  \item Montrer que $u_i$ est un vecteur propre de $A$.
  
  \item Montrer que $\kr(A)=\left(\Vect{v_1, \ldots, v_r}\right)^\perp$.
  En déduire le rang de $A$.
  
  \item Montrer que $A$ est diagonalisable.
  
  \item Réciproquement, montrer que si $A$ est diagonalisable de rang 
  $r$, alors il existe $(u_1, \ldots, u_r)$ et $(v_1,\ldots, v_r)$ de 
  $\R^n$ biorthogonales et des réels non nuls $\lambda_1, \ldots, 
  \lambda_r$ tels que :
  \[
   A = \Sum{i=1}{r} \lambda_i \, u_i \, {}^t v_i
  \]
 \end{noliste}
 
 \item Soit $A\in\M{n}$, symétrique de rang $r$. Montrer qu'il existe 
 une famille $(u_1,\ldots, u_r)$ et des réels non nuls $\lambda_1, 
 \ldots, \lambda_r)$ tels que :
 \[
  A = \Sum{i=1}{r} \lambda_i \, u_i \, {}^t u_i
 \]
 La réciproque est-elle vraie ?
\end{noliste}




\newpage
 
\section*{Série 3}
\subsection*{Question de cours}
\noindent
Démontrer la stabilité par somme des lois de Poisson.


\subsection*{Exercice} % ESCP 2017
\noindent
Soit $\alpha$ un réel strictement positif et $(Y_i)_{i\in\N^*}$ une 
suite de variables aléatoires indépendantes définies sur le même espace 
probabilisé $(\Omega, \A, \Prob)$. On suppose de plus que pour tout 
$i$, $Y_i$ suit la loi exponentielle de paramètre $i \, \alpha$.\\
Pour tout $n\in\N^*$, on pose $Z_n=\Sum{i=1}{n} Y_i$ et on note $g_n$ 
la densité de $Z_n$ nulle sur $\R_-$ et continue sur $\R_+^*$.
\begin{noliste}{1.}
 \item 
 \begin{noliste}{a)}
  \item Déterminer la fonction $g_2$.
  \item Montrer que pour $n\geq 1$ et $x>0$, on a : $g_n(x) =
  n \, \alpha \, \ee^{-\alpha \, x}\left(1-\ee^{-\alpha \, 
  x}\right)^{n-1}$
  \item Calculer l'espérance de $Z_n$ et en donner un équivalent simple 
  lorsque $n$ tend vers l'infini.
  \item Calculer la variance de $Z_n$ et montrer qu'elle admet une 
  limite finie lorsqie $n$ tend vers l'infini.
 \end{noliste}
 
 \item Pour $n\in\N^*$, on pose $U_n=\dfrac{1}{n} \, Z_n$.
 \begin{noliste}{a)}
  \item Déterminer la fonction de répartition $H_n$ de $U_n$.
  \item Montrer que, pour tout $x\in\R$, la suite $(F_{U_n}(x))$ 
  converge vers un réel $F(x)$.\\
  Montrer que $F$ est la fonction de répartition d'une \var que l'on 
  précisera.
  \item Déterminer la limite quand $n$ tend vers l'infini de $\E(U_n)$ 
  et $\V(U_n)$.
 \end{noliste}
\end{noliste}



\end{document}
