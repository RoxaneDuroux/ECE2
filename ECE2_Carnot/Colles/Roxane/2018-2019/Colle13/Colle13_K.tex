\documentclass[11pt]{article}%
\usepackage{geometry}%
\geometry{a4paper,
  lmargin=2cm,rmargin=2cm,tmargin=1.5cm,bmargin=1.5cm}
  
\input{../../../macros.tex}



\begin{document}
\begin{flushleft}
K \\
Mathématiques
\end{flushleft}

\begin{center}
\textbf{\Large{Colles - Semaine 13}}
\end{center}

\hrule

\vspace*{0,2cm}

\section*{Planche 1} 

\subsection*{Exercice 1}
\noindent
Soit $f$ la fonction définie pour tout couple $(x,y)$ de $\R^{2}$
par: $f(x,y) = 2 x^{2} +2 y^{2}+ 2xy - x - y$.
\begin{noliste}{1.}
\item
\begin{noliste}{a)}
\item 
Calculer les dérivées partielles premières de $f$.
\item  
En déduire que le seul point critique $A$ de $f$.%(1/6,1/6)
\end{noliste}
\item\label{min}
Montrer que $f$ présente un minimum global en $A$.
\item 
On considère la fonction $g$ définie pour tout couple $(x,y)$ de 
$\R^{2}$ par :
\[
 g(x,y) = 2 e^{2x} + 2 e^{2y} + 2 e^{x+y} - e^{x}- e^{y}
\]
\begin{noliste}{a)}
\item 
Utiliser la question \ref{min} pour établir que : \quad $\forall(x,y)\in 
\R^{2},\ g(x,y) \ge -\dfrac{1}{6}\cdotp$
\item 
En déduire que $g$ possède un minimum global sur $\R^{2}$ et préciser en 
quel point ce minimum est atteint.
\end{noliste}
\end{noliste}

\subsection*{Exercice 2} % ESCP 2007
\noindent
Soit $E$ un $\K$-espace vectoriel où $\K=\R$ ou $\C$. On
considère deux projecteurs $p$ et $q$ de $E$ différents de
l'identité $\id_E$ et de l'application nulle ; on suppose en outre
que $p$ et $q$ commutent et que leur somme $f$ n'est pas égale
à l'identité.

\begin{noliste}{1.}
\item
Montrer que $p\circ q$ est un projecteur de $E$ et calculer
$f^3-3f^2+2f$.\\
On note $\spc(f)$ l'ensemble des valeurs propres de $f$.

\item
\begin{noliste}{a)}
\item
 Montrer que $0\in\spc(f)$ si et seulement si $\kr(p)\cap\kr(q) \neq 
 \{0_E\}$.

\item
Montrer que $2\in\spc(f)$ si et seulement si $\im(p)\cap \im(q)\neq 
\{0_E\}$.

\item
Montrer que $\kr(f)\oplus \kr(f- \id_E)\oplus\kr(f-2 \, \id_E)=E$.

\end{noliste}

\item
En déduire que $[2\notin\spc(f)$ ou $0\notin\spc (f)]$
entraîne
que $1\in\spc(f)$ et $\spc(f)\neq\{1\}$.
\end{noliste}


\newpage


\section*{Planche 2} % ESCP 2017
\noindent
Soit $E$ un $\K$-espace vectoriel de dimension finie (avec $\K=\R$ 
ou $\K=\C$). On note $\LL{E}$ l'espace vectoriel des endomorphismes de 
$E$.\\
Soit $p$ un projecteur de $E$ tel que $p\neq 0$ et $p\neq \id_E$. Pour 
tout $f\in \LL{E}$, on pose :
\[
 \varphi(f)=\dfrac{1}{2} \, (f\circ p +p\circ f)
\]
\begin{noliste}{1.}
 \item Montrer que $\varphi$ est un endomorphisme de $\LL{E}$.
 
 \item Calculer $(\varphi \circ \varphi)(f)$ et $(\varphi \circ \varphi 
 \circ \varphi)(f)$ ; en déduire les valeurs propres possibles de 
 $\varphi$.
 
 \item Pour tout sous-espace vectoriel $F$ de $E$, montrer que les 
 ensembles suivants sont des sous-espaces vectoriels de $\LL{E}$ :
 \[
  \mathcal{K}(F) = \{f\in \LL{E} \ / \ F\subset \kr(f)\} \quad 
  \mbox{ et } \quad \mathcal{I}(F) = \{f\in\LL{E} \ / \ \im(f) 
  \subset F\}
 \]
 
 \item Soient $f,g \in \LL{E}$.
 \begin{noliste}{a)}
  \item Calculer $f\circ g$ lorsque $f\in \mathcal{K}(\im(g))$ ou
  lorsque $g\in \mathcal{I}(\kr(f))$.
  
  \item Calculer $p\circ f$ lorsque $f\in \mathcal{I}(\im(p))$.
  
  \item Montrer que $f\circ p=f$ lorsque $f\in \mathcal{K}(\kr(p))$.
 \end{noliste}
 
 \item 
 \begin{noliste}{a)}
  \item Pour chacun des sous-espaces vectoriels suivants, montrer que 
  leurs éléments non nuls sont des vecteurs propres de $\varphi$ et 
  préciser les valeurs propres correspondantes :
  \[
   \mathcal{A} = \mathcal{K}(\im(p)) \cap \mathcal{I}(\kr(p)), \quad 
   \B = \mathcal{K}(\im(p)) \cap \mathcal{I}(\im(p)) \quad \mbox{ et }
   \quad \mathcal{C} = \mathcal{K}(\kr(p)) \cap \mathcal{I}(\im(p))
  \]
  
  \item Montrer que les sous-espaces $\mathcal{A}$, $\B$ et 
  $\mathcal{C}$ sont en somme directe.
  
  \item Quelles sont les valeurs propres de $\varphi$ ?
 \end{noliste}
\end{noliste}


\newpage


\section*{Planche 3}
\noindent
On note $m$ un paramètre réel et on considère les matrices $H_m$ 
définies
par : $H_m=
\begin{smatrix}
-1-m & m & 2\\
-m & 1 & m\\
-2 & m & 3-m
\end{smatrix}.$\\
On note $h_{m}$ l'endomorphisme de $\R^3$ ayant pour matrice 
$H_m$ dans la base canonique de $\R^3$.

\begin{noliste}{1.}
\item 
On suppose dans cette question que $m=2$.

\begin{noliste}{a)}
\item
Écrire la matrice $H_2$.
\item 
Déterminer les valeurs propres de l'endomorphisme $h_{2}$ et les 
sous-espaces propres associés.

\item 
L'endomorphisme $h_2$ est-il diagonalisable? Si oui, donner une base de 
vecteurs propres de $h_2$.
\end{noliste}

\item 
Étudier de même les valeurs propres et les sous-espaces propres de 
$h_{0}$. Cet endomorphisme est-il diagonalisable?

\item 
\begin{noliste}{a)}
\item 
Montrer qu'il existe un réel $a$, qu'on déterminera, qui est valeur
propre de l'endomorphisme $h_m$ pour toutes les valeurs du paramètre 
$m$.

\item 
Déterminer, pour chaque valeur de $m$, le sous-espace propre de $h_m$ 
associé à
la valeur propre $a$. Montrer qu'on peut trouver un vecteur non nul 
$v_{1}$
appartenant à tous ces sous-espaces.
\end{noliste}

\item 
Soit $F$ le sous-espace de $\R^3$ engendré par les vecteurs 
$v_2=(1,0,1)$ et $v_3=(1,1,0)$ :\\ 
$F=\Vect{v_2,v_3}$.\\
Déterminer les vecteurs $h_m(v_2)$ et $h_m(v_3)$ et montrer que 
ces vecteurs appartiennent à $F$ pour tout $m$ réel.\\
En déduire que le $F$ est stable par $h_m$, c'est-à-dire que 
$h_m(F)\subset F$.

\item 
Montrer que $(v_1,v_2,v_3)$ est une base de $\R^3$.\\
\'Ecrire la matrice de $h_m$ dans la base $(v_1,v_2,v_3)$. En déduire 
les valeurs de $m$ pour lesquelles l'endomorphisme $h_m$ est 
diagonalisable.
\end{noliste}






\end{document}
