\documentclass[11pt]{article}%
\usepackage{geometry}%
\geometry{a4paper,
  lmargin=2cm,rmargin=2cm,tmargin=1.5cm,bmargin=1.5cm}
  
\input{../../../macros.tex}



\begin{document}
\begin{flushleft}
ECE2 \\
Mathématiques
\end{flushleft}

\begin{center}
\textbf{\Large{Colles - Semaine 3}}
\end{center}

\hrule

\vspace*{0,2cm}


\begin{exercice}~
\begin{noliste}{1.}
\item Déterminer que la fonction $F \ : \ x \mapsto x \ln(x) -x$ est 
une primitive de la fonction $f \ : \ x \mapsto \ln(x)$ sur $]0, + 
\infty[$.
\item Démontrer que : $ \forall k \geq 2, \ \dint{k-1}{k} \ln(t) \dt 
\leq \ln(k) \leq \dint{k}{k+1} \ln(t) \dt$.
\item En déduire que : $\forall n \geq 2, \ n \ln(n) -n \leq \ln(n!) 
\leq (n+1)\ln(n+1) - (n+1) +1$.
\item En déduire un équivalent simple de $\ln(n!)$.
\item La série $ \DSum{n \geq 1}{} \dfrac{\ln(n!)}{n^3}$ est-elle 
convergente ?
\end{noliste}
\end{exercice}

\begin{exercice}~\\
Soit $ k \in \N^*$ et $(u_n)$ une suite de réels positifs.\\
On définit la suite $(v_n)$ par : $\forall n \in \N, \ v_n = \ln(1 + 
u_n^k)$.
\begin{noliste}{1.}
\item Montrer que si la série $\Sum{}{} u_n$ converge, alors la série 
$\Sum{}{} v_n$ converge.
\item On se propose d'étudier la réciproque de l'implication précédente.
	\begin{noliste}{a)}
	\item On suppose que $k=1$. Montrer que si la série $\Sum{}{} 
v_n$ converge, alors la série $\Sum{}{} u_n$ converge.
	\item On suppose que $k>1$. Donner un exemple de suite $(u_n)$ 
telle que la série $\Sum{}{} v_n$ converge et la série $\Sum{}{} u_n$ 
diverge.
	\end{noliste}
\end{noliste}
\end{exercice}


\begin{exercice}~\\
On considère la suite $(u_n)$ définie par : $ \left\{ \begin{matrix}
u_0 >0\\
\forall n \in \N, \ u_{n+1} = u_n + u_n^2
\end{matrix} \right.$.
\begin{noliste}{1.}
\item \begin{noliste}{a)}
	\item Montrer que la suite $(u_n)$ est croissante.
	\item Montrer que la suite $(u_n)$ diverge vers $+ \infty$.
	\end{noliste}

\item On pose, pour tout entier naturel $n$, $ v_n = 
\dfrac{\ln(u_n)}{2^n}$.
	\begin{noliste}{1.}
	\item Montrer que pour tout $t>0$, $\ln(1+t) \leq t$.
	\item Montrer que, pour tout $n \in \N$ : $ 0 \leq v_{n+1} - 
v_n \leq \dfrac{1}{2^{n+1}u_n}$.
	\item Montrer que la série de terme général $v_{n+1} - v_n$ est 
convergente.
	\item En déduire que la suite $(v_n)$ converge. On note $\ell$ 
sa limite.
	\end{noliste}

\item \begin{noliste}{a)}
	\item Montrer, à l'aide de la question \itbf{2.b)}, que :
	\[
	\forall n \in \N, \ \forall p \in \N, \ 0 \leq v_{n+p+1} - v_n 
\leq \dfrac{1}{2^nu_n}.
	\]
	\item Montrer que, pour tout $n \in \N$ : $ 0 \leq \ell - 
v_n \leq \dfrac{1}{2^n u_n}$.
	\item En déduire que $ u_n \eqn 
\ee^{2^n \ell}$.
	\end{noliste}
\end{noliste}
\end{exercice}






\end{document}

