\documentclass[11pt]{article}%
\usepackage{geometry}%
\geometry{a4paper,
  lmargin=2cm,rmargin=2cm,tmargin=1.5cm,bmargin=1.5cm}
  
\input{../../../macros.tex}



\begin{document}
\begin{flushleft}
ECE1 \\
Mathématiques
\end{flushleft}

\begin{center}
\textbf{\Large{Colles - Semaine 2}}
\end{center}

\hrule

\vspace*{0,2cm}

\section{Série 1}

\subsection*{Exercice 1}
\noindent
  Résoudre dans $\R$ l'équation : $\sqrt{17x^2-16x} -2x = x+4$.


\subsection*{Exercice 2}
\noindent
Faire l'étude de la fonction $f:x\mapsto \sqrt{\dfrac{2-x}{1+x}}$.



\section{Série 2}

\subsection*{Exercice 1}
\noindent
Résoudre $\dfrac{x^3 + 2x^2 -x +1}{x-1} = 2 - x +x^2$



\subsection*{Exercice 2}
\noindent
Faire l'étude de la fonction $f:x\mapsto -2+x-\ln x$.



\section{Série 3}

\subsection*{Exercice 1}
\noindent
Résoudre $\dfrac{x}{x+1}+ \dfrac{1}{x(x-1)}\leq 1$ 



\subsection*{Exercice 2}
\noindent
  On considère la fonction $f : x \mapsto \dfrac{\ee^x}{\ee^{2x}+1}$.
  \begin{noliste}{1.}
  \item Montrer que $f$ est paire.
  \item Déterminer les limites de $f$ en $+\infty$ et $-\infty$.
  \item Étudier les variations de $f$ et tracer sa courbe
    représentative.
%   \item Montrer qu'il existe un unique réel $\ell$ tel que $f(\ell) =
%     \ell$. Justifier $0 \leq \ell \leq \dfrac{1}{2}$.
  \end{noliste}


\end{document}
