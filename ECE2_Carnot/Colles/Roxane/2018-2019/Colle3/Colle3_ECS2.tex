\documentclass[11pt]{article}%
\usepackage{geometry}%
\geometry{a4paper,
  lmargin=2cm,rmargin=2cm,tmargin=1.5cm,bmargin=1.5cm}
  
\input{../../../macros.tex}



\begin{document}
\begin{flushleft}
ECS2 \\
Mathématiques
\end{flushleft}

\begin{center}
\textbf{\Large{Colles - Semaine 3}}
\end{center}

\hrule

\vspace*{0,2cm}

\section*{Série 1}
\subsection*{Question de cours}
Quelle est la nature de la série $\Sum{}{} \ln\left( 
1+\dfrac{1}{n}\right)$.

\subsection*{Exercice}
\noindent
Si $a=(a_1,a_2,a_3,a_4)\in\R^4$, on pose $M(a)=\begin{pmatrix} 
0&0&0&a_1\\ 0&0&0&a_2\\ 0&0&0&a_3\\a_1&a_2&a_3&a_4\end{pmatrix}$. \\
On note $F$ le sous-espace vectoriel de $\M{4}$ constitué des 
matrices $M(a)$ lorsque $a $ parcourt $\R^4$. \\
On note $J=\begin{pmatrix} 1&0&0&0\\ 0&1&0&0\\ 
0&0&1&0\\0&0&0&0\end{pmatrix}$.
\begin{noliste}{1.}
\item
Soit $E$ un espace vectoriel de dimension finie égale à $n$.
\begin{noliste}{a)}
\item
\textsf{Question de cours :} Rappeler la définition du sous-espace 
vectoriel engendré par une famille $(x_1,\ldots,x_p)$ de vecteurs de 
$E$. Dans quel cas la famille $(x_1,\ldots,x_p)$ est-elle une base de 
$\text{Vect}(x_1,\ldots,x_p)$ ?
\item
Soit $E_1$ de base $(x_1,\ldots,x_p)$ et $E_2$ de base 
$(y_1,\ldots,y_q)$ deux sous-espaces vectoriels de $E$ tels que $E_1\cap 
E_2=\{0_E\}$. Montrer que la famille $(x_1,\ldots,x_p,y_1,\ldots,y_q)$ 
est libre. Qu'en déduit-on sur $p+q$ ?
\end{noliste}
\item
Montrer que $F$ est un sous-espace vectoriel de $\M{4}$ et en 
donner la dimension.
\item
Soit $(e_1,e_2,e_3,e_4)$ la base canonique de $\R^4$. Pour 
$i\in\llb 1,4\rrb$, on pose $M_i=M(e_i)$. Montrer que $\forall 
i\in\llb 1,4\rrb$, la matrice $M_i+J$ est inversible et que la famille 
$(M_i+J)_{1\le i\le 4}$ est libre.
\item
Soit $a\in\R^4$. Montrer que si pour tout réel $\theta$ non nul, la 
matrice $M(a)+\theta J$ est non inversible, alors $a=(0,0,0,0)$.
\item
Soit $G$ un sous-espace vectoriel de $\M{4}$ qui ne contient 
aucune matrice inversible et tel que $J\in G$.
\begin{noliste}{a)}
\item
Déterminer $G\cap F$ et en déduire que la dimension de $G$ est 
inférieure ou égale à $12.$
\item
Existe-t-il un sous-espace vectoriel de $\M{4}$ de dimension 
$12$ ne contenant aucune matrice inversible et contenant $J$ ?
\end{noliste}
\end{noliste}


\newpage

\section*{Série 2}
\subsection*{Question de cours}
Calculer $\DSum{k=0}{n} \dfrac{k^2}{(-5)^k}$.

\subsection*{Exercice}
\noindent
Un péage comporte 10 guichets numérotés de $1$ à $10$. Le nombre de 
voitures $N$, arrivant au péage en $1$ heure, suit une loi de Poisson 
de paramètre $\lambda > 0$. On suppose de plus que les conducteurs 
choisissent leur file au hasard et indépendamment des autres. Soit 
$X_1$ la variable aléatoire égale au nombre de voitures se présentant 
au guichet n\textdegree 1 en une heure.
\begin{noliste}{1.}
\item Déterminer le nombre moyen de voitures arrivant au péage en une 
heure.
\item Quelle est la proba qu'une voiture qui arrive au péage se dirige 
vers le guichet n°1 ? 
\item Calculer $\Prob_{\Ev{N=n}}(\Ev{X_1 =k})$ pour tout $0 \leq k \leq 
n$. Et pour $k>n$?
\item Justifier que $\Prob(\Ev{X_1 =k})= \DSum{n=k}{+\infty} 
\Prob_{\Ev{N=n}}(\Ev{X_1 =k}) \, \times \, \Prob(\Ev{N =n})$\\
puis montrer que $ \Prob(\Ev{X_1 =k}) = \ee^{- \lambda} 
\left(\dfrac{1}{10}\right)^k \dfrac{\lambda^k}{k!} \, \DSum{n=0}{+ 
\infty}   \left(\dfrac{9}{10}\right)^n \dfrac{\lambda^n}{n!}$.
\item En déduire la loi de $X_1$, son espérance et sa variance
\end{noliste}


\newpage

\section*{Série 3}
\subsection*{Question de cours}
Démonstration de \og Une application linéaire est un isomorphisme si et 
seulement si elle envoie bases sur bases \fg{}

\subsection*{Exercice}
\noindent
On lance successivement une pièce truquée dont la probabilité de faire 
face est de $p \in ]0,1[$. Pour $n \geq 1$, notons $F_n$ :\og Obtenir 
Face au $n$-ième lancer \fg{}, et\\
$P_n$ :\og Obtenir Pile au $n$-ième lancer \fg{}.\\
On note $T_n$ : \og le premier Pile est obtenu au $n$-ième lancer \fg{}.
 \begin{noliste}{1.}
\item Pour $n \geq 1$, exprimer l'événement $T_n$ en fonction des $F_i$ 
et $P_i$.
\item Donner $\Prob(T_n)$ en fonction de $p$ et $n$.
\item On lance la pièce une infinité de fois. \'Ecrire les événements 
suivants :
 \begin{noliste}{$-$}
  \item $A_n$ :\og obtenir au moins un pile au cours des $n$ premiers 
lancers. \fg{},
  \item $A$ : \og obtenir au moins un pile \fg{}.
 \end{noliste}
 \item Parmi les suites $(F_n)$, $(P_n)$, $(T_n)$ et $(A_n)$, 
lesquelles sont croissantes ?\\ décroissantes ?
 \item Parmi les suites $(F_n)$, $(P_n)$, $(T_n)$ et $(A_n)$, 
lesquelles sont constituées d'événements mutuellement indépendants?
 \item Donner la probabilité $\Prob(A)$. Que peut-on dire de 
l'événement $A$?
\end{noliste}


\section*{Exercice supplémentaire}
\begin{noliste}{1.}
\item Un tirage au Loto est une combinaison (l'ordre ne compte pas) de 
6 numéros distincts compris entre 1 et 49. Quelle est la probabilité 
(on donne : ${49\choose 6}=13 983 816$), notée $p$ dans la suite, de 
gagner (c-à-d d'avoir les 6 bons numéros) au Loto~?
\item On joue au loto indéfiniment et on définit un événement en 
posant~:
$A=\text{\og{} on gagne au moins une fois 
\fg{}}$, et  pour tout $n\in\N^*$~:
\begin{noliste}{$\sbullet$}
\item $A_n=\text{\og{}on gagne pour la premi\`ere fois au $\eme{n}$ 
tirage \fg{}}$,

\item $B_n=\text{\og{} on gagne au $n$-\`eme tirage \fg{}}$.
\end{noliste}
Exprimer $A_n$ à l'aide des $B_k$. En déduire $\Prob(A_n)$ en fonction 
de~$p$.
\item En déduire $\Prob(A)$ en exprimant $A$ à l'aide des $A_n$. 
Comment appelle-t-on un tel événement~$A$~?
\end{noliste}

\end{document}
