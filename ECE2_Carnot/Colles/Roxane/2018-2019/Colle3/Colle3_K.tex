\documentclass[11pt]{article}%
\usepackage{geometry}%
\geometry{a4paper,
  lmargin=2cm,rmargin=2cm,tmargin=1.5cm,bmargin=1.5cm}
  
\input{../../../macros.tex}



\begin{document}
\begin{flushleft}
K \\
Mathématiques
\end{flushleft}

\begin{center}
\textbf{\Large{Colles - Semaine 3}}
\end{center}

\hrule

\vspace*{0,2cm}

\section*{Planche 1}



\subsection*{Exercice} % Exo Gilles

\noindent
Pour tout entier $n\geq 1$, on définit la fonction polynomiale $P_n$ 
sur $\R$ par :
\[
 P_n(x) \ = \ (x-1)(2-x) + \dfrac{x^3}{n}
\]
\begin{noliste}{1.}
 \item Montrer que pour $n$ suffisamment grand, $P_n$ admet deux 
 extrema locaux, aux points $r_{1,n} \leq r_{2,n}$ avec $r_{1,n} 
 \tendn \ell$ et $r_{2,n} \eqn \gamma \, n$ pour deux réels $\ell$
 et $\gamma$ à préciser.
 
 \item Montrer que pour $n$ suffisamment grand, $P_n$ admet trois 
 racines distinctes notées $a_n < b_n < c_n$.
 
 \item Montrer que la suite $(a_n)$ ainsi définie est croissante, 
 de limite $\alpha$ à préciser.
 
 \item En déduire le développement limité :
 \[
  a_n \ = \ \alpha + \dfrac{\beta}{n} + \oon \left(\dfrac{1}{n}
  \right)
 \]
 pour un réel $\beta$ à préciser.
\end{noliste}






\newpage



\section*{Planche 2}


\subsection*{Exercice}

\noindent
Soit $f$ la fonction définie sur $[0,+\infty[$ par :
\[
 f \ : \ x \mapsto \left\{
 \begin{array}{cR{3cm}}
  x^{1+\frac{1}{x}}=\ee^{(1+\frac{1}{x})\ln (x)} & si $x>0$
  \nl
  \nl[-.2cm]
  0 & si $x=0$
 \end{array}
 \right.
\]

On désigne par $\mathcal{C}$ la courbe représentative de $f$ dans le 
plan muni d'un repère orthonormé.
\begin{noliste}{1.}
  \item
  \begin{noliste}{a)}
    \item Montrer que $f$ est continue en $0$.
    
    \item Étudier la dérivabilité de $f$ en $0$.
  \end{noliste}
  
  \item
  \begin{noliste}{a)}
    \item Montrer que, pour tout réel $x$ strictement positif : 
    $\ln (x) \leq x+1$.
    
    \item Calculer $f'(x)$ pour $x>0$ et déterminer son signe.\\ 
    Préciser le sens de variation de $f$.
  \end{noliste}
  
  \item
  \begin{noliste}{a)}
    \item Déterminer la limite de $f$ en $+\infty $.
    
    \item Déterminer un équivalent de $f(x)-x$ en $+\infty $.\\ 
    En déduire la nature de la branche infinie de $\mathcal{C}$ en $+
    \infty $.
  \end{noliste}
  
  \item On définit la suite $(u_n)$ par :
  \[
   \left\{
   \begin{array}{l}
    u_0 >0\\
    \forall n \in \N, \ u_{n+1} = f(u_n)
   \end{array}
   \right.
  \]
  \begin{noliste}{a)}
    \item Montrer que la suite $(u_n)$ est bien définie et : 
    $\forall n \in \N$, $u_n >0$.
    
    \item Que dire de la suite $(u_n)$ si $u_0=0$ ?
    
    \item On se place maintenant dans le cas : $u_0= \dfrac{1}{2}$.
    \begin{nonoliste}{(i)}
     \item Déterminer la monotonie de $(u_n)$.
     
     \item Montrer que la suite $(u_n)$ converge et 
     déterminer sa limite.
    \end{nonoliste}
  \end{noliste}
\end{noliste}



\newpage




\section*{Planche 3}



\subsection*{Exercice} % ENS 2017

\noindent
Soient $a>0$ et $\beta>1$. On considère une fonction continue $f : 
[0,1] \to [0,1]$ qui peut s'écrire comme suit au voisinage de $0$ :
\[
 f(x) \ = \ x - a \, x^\beta + \oox{0}(x^\beta)
\]
On fixe $u_0 \in [0,1]$, et on considère la suite $(u_n)_{n\geq 0}$ 
définie par : $\forall n \in \N$, $u_{n+1} = f(u_n)$.
\begin{noliste}{1.}
 \item Que vaut $f(0)$ ? Montrer qu'il existe $\eta >0$ tel que $f(x) 
 <x$ sur $]0,\eta]$. En déduire que si $u_0$ est suffisamment petit, 
 $(u_n)$ converge vers $0$.\\
 Dans la suite, on suppose que $u_n \tendn 0$.
 
 \item Soit $\gamma \in \R$. Donner un équivalent de $(f(x))^\gamma - 
 x^\gamma$ lorsque $x$ tend vers $0$. En déduire une valeur de $
 \gamma$ pour laquelle la suite $u_{n+1}^\gamma - u_n^\gamma$ 
 converge vers un réel strictement positif.
 
 \item Soit $(v_n)_{n\geq 1}$ une suite de nombres réels tels que 
 $v_{n+1}-v_n \tendn \ell$, où $\ell \in \R$. Montrer que $
 \dfrac{v_n}{n} \tendn \ell$.
 
 \item En déduire un équivalente de $(u_n)$.
\end{noliste}



\end{document}
