\documentclass[11pt]{article}%
\usepackage{geometry}%
\geometry{a4paper,
  lmargin=2cm,rmargin=2cm,tmargin=2.5cm,bmargin=2.5cm}
  
\input{../../../macros.tex}



\begin{document}
\begin{flushleft}
ECE2 \\
Mathématiques
\end{flushleft}


\vspace{0.1cm}

\begin{center}
\textbf{\Large{Programme de colle - Semaine 3}}
\end{center}

\hrule

\vspace*{0,2cm}

\section*{Notation}

\noindent
On adoptera les principes suivants pour noter les étudiants :
\begin{noliste}{$\stimes$}
\item si l'étudiant sait répondre à la question de cours, il 
aura une note $> 8$.
\item si l'étudiant ne sait pas répondre à la question de 
cours ou s'il y a trop d'hésitations, il aura une note $\leq 8$.
\end{noliste}

\section*{Questions de cours}

\begin{noliste}{$\sbullet$}
  
  \item {\bf Critère de convergence des séries télescopiques} :\\
   Soit $(u_n)_{n\in\N}$ une suite de réels.
  \[
    \begin{array}{rcl}
      \mbox{$(u_n)$ converge} & \Leftrightarrow & 
      \mbox{$\Sum{}{} (u_{n+1}-u_n)$ converge}
    \end{array}
  \]
  De plus, si $(u_n)$ converge vers $\ell \in \R$,
  \[
    \dlim{n \to +\infty} S_n = \Sum{k=0}{+\infty}(u_{k+1} - u_k) = 
    \ell - u_0.
  \]

\begin{proof}[Preuve]~\\
  Pour tout $n \geq 0$, on a :
  \[
    \begin{array}{rcl@{\quad}>{\it}R{4cm}}
      S_n = \Sum{k=0}{n} (u_{k+1} - u_k)
      & = & \Sum{k=0}{n} u_{k+1} - \Sum{k=0}{n} u_k 
      \\[.6cm]
      & = & \Sum{j=1}{n+1} u_{j} - \Sum{k=0}{n} u_k 
      & (avec le changement d'indice $j=k+1$) 
      \nl
      \nl[-.2cm]
      & = & \Sum{j=1}{n} u_{j} + u_{n+1} - \left( \Sum{k=1}{n} u_k + 
      u_0 \right) 
      \\[.4cm]
      & = & u_{n+1} - u_0
    \end{array}
  \]
  Donc $(S_n)$ converge si et seulement si $(u_n)$ converge.\\ 
  De plus, si $(u_n)$ converge vers $\ell \in \R$ : 
  \[
    \dlim{n \to +\infty} \Sum{k=0}{n}(u_{k+1} - u_k) = \dlim{n \to 
    \infty}  (u_{n+1} - u_0) = \ell - u_0.
  \]~\\[-1.2cm]
\end{proof}


\newpage
    
    
  \item {\bf Critère de convergence des séries de Riemann} :\\
  $\DSum{}{} \dfrac{1}{n^\alpha} \mbox{ converge} \ \Leftrightarrow \
  \alpha >1$
  
  \begin{proof}[Preuve]~\\
    On détaillera uniquement le cas $\alpha>1$.\\
    Soit $k \in \N$ et $x \in [k, k+1]$.\\ 
    Alors, par décroissante de la fonction 
    $x \mapsto \dfrac{1}{x^\alpha}$ sur $]0,+\infty[$ :
    \[
      \dfrac{1}{(k+1)^\alpha} \ \leq \ \dfrac{1}{x^\alpha} 
      \ \leq \ \dfrac{1}{k^\alpha},
    \]
    Donc, par croissance de l'intégrale, les bornes étant 
    dans l'ordre croissant ($k \leq k+1$) :
    \[
      \dint{k}{k+1} \dfrac{1}{(k+1)^\alpha} \dx 
      \ \leq \ \dint{k}{k+1} \dfrac{1}{x^\alpha} \dx
      \ \leq \ \dint{k}{k+1} \dfrac{1}{k^\alpha} \dx
    \]
    Comme $k^\alpha$ et $(k+1)^\alpha$ sont des constantes par 
    rapport à la variable d'intégration $x$, on obtient :
    \[
      \dfrac{1}{(k+1)^\alpha} \ \leq \ \dint{k}{k+1} 
      \dfrac{1}{x^\alpha} \dx \ \leq \ \dfrac{1}{k^\alpha}
    \]
    Soit $n \in \N^*$. En sommant l'encadrement précédent de $1$ à 
    $n-1$, on obtient :
    \[
      \Sum{k=1}{n-1} \dfrac{1}{(k+1)^\alpha} \ \leq \ 
      \dint{1}{n} \dfrac{1}{x^\alpha} \dx \ \leq \ \Sum{k=1}{n-1} 
      \dfrac{1}{k^\alpha}
    \]
    c'est-à-dire :
    \[
    \Sum{k=2}{n} \dfrac{1}{k^\alpha}
    \ \leq \ \dint{1}{n} \dfrac{1}{x^\alpha} \dx 
    \ \leq \ \Sum{k=1}{n-1} \dfrac{1}{k^\alpha}
    \]
    Or : $\dint{1}{n} \dfrac{1}{x^\alpha} \dx 
    \ = \ \Prim{ \dfrac{1}{1-\alpha} \, \dfrac{1}{x^{\alpha-1}}}{1}{n} 
    \ = \ \dfrac{1}{\alpha-1} \left( 1- \dfrac{1}{n^\alpha} \right)$. 
    Donc :
    \[
      \Sum{k=2}{n} \dfrac{1}{k^\alpha}
      \ \leq \ \dfrac{1}{\alpha-1} \left( 1- \dfrac{1}{n^\alpha} 
      \right)
      \ \leq \ \Sum{k=1}{n-1} \dfrac{1}{k^\alpha}.
    \]
    Si $\alpha >1$, alors : $0 \leq 1 - \dfrac{1}{n^\alpha} 
      \leq 1$ pour tout $n \in \N^*$.\\[.2cm]
      La suite $\left( \Sum{k=1}{n} \dfrac{1}{k^\alpha} \right)$ est 
      donc majorée par $\dfrac{1}{\alpha -1}$. Elle est de plus 
      croissante car c'est une somme de termes positifs. Donc, par 
      théorème de convergence monotone, elle converge.\\ 
      D'où la série $\Sum{}{} \dfrac{1}{n^\alpha}$ est convergente.
  \end{proof}
  
  
  \newpage
  
  
  \item {\bf Comparaison série / intégrale} :\\ 
  Soit $f$ une fonction définie sur $[0,+ \infty[$, continue, 
  positive et décroissante sur cet intervalle.\\
  Alors la série $\Sum{}{} f(n)$ et la suite $\left(
  \dint{0}{n} f(t) \dt\right)$ ont même nature.

\begin{proof}[Preuve]~\\
Pour tout $k \in \N$, comme $f$ est décroissante, on a :
\[
  \forall x \in [k,k+1], \ f(k+1) \leq f(x) \leq f(k)
\]
Par croissance de l'intégrale, les bornes étant bien ordonnées 
($k\leq k+1$) :
\[
  f(k+1) \ \leq \ \dint{k}{k+1} f(t) \dt \ \leq \ f(k)
\]
On somme l'encadrement précédent pour $k$ variant de $0$ à 
$n-1$. On obtient ainsi :
\[
  \Sum{k=0}{n-1} f(k+1) \ \leq \ \dint{0}{n} f(t) \dt
  \ \leq \ \Sum{k=0}{n-1} f(k)
\]
et, si on note $S_n$ la somme partielle d'indice $n$ de la série 
$\Sum{}{} f(n)$, cela s'écrit :
\[
  S_n-f(0) \ \leq \ \dint{0}{n}f(t) \dt \ \leq \ S_{n-1}
\]
En observant que les deux suites $(S_n)$ et $\left( \dint{0}{n} f(t) 
\dt \right)_{n\in\N}$ sont croissantes (car $f$ est positive), on 
peut affirmer grâce à la dernière égalité :
\begin{noliste}{$\stimes$}
  \item si $(S_n)$ converge vers $\ell$, alors $\left( \dint{0}{n} 
  f(t) \dt \right)_{n\in\N}$ est majorée par $\ell$, donc converge 
  aussi ;

  \item si $\left( \dint{0}{n} f(t) \dt \right)_{n\in\N}$ converge 
  vers $\ell'$, alors $(S_n)$ est majorée par $\ell' + f(0)$, donc 
  converge aussi.
\end{noliste}
\end{proof}

  \underline{{\it Remarque}} : On pourra demander à l'étudiant s'il 
  est possible d'amoindrir les hypothèses.
\end{noliste}


\section*{Connaissances exigibles}

\begin{noliste}{$\sbullet$}
\item convergence de suites numériques (théorème de convergence 
monotone, théorème d'encadrement, etc.)
\item suites adjacentes
\item étude de suites récurrentes (les élèves doivent être guidés dans 
le cheminement de ces études)
\item équivalents
\item négligeabilité
\item séries numériques, à termes positifs, séries usuelles, comparaison 
série / intégrale, comparaisons de séries par négligeabilité et 
équivalence.
\item les séries alternées sont hors programme mais les étudiants ont vu 
en exercice comment démontrer le critère de convergence des séries 
alternées.
\item toutes les techniques sont à connaître (sommation télescopique, 
calcul direct des sommes partielles - séries usuelles, comparaison 
séries / intégrales, critères sur les SATP...)
\item on insistera particulièrement en colle sur les rédactions 
classiques (notamment pour tous les critères sur les SATP).
\end{noliste}





\end{document}
