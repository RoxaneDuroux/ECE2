\documentclass[11pt]{article}%
\usepackage{geometry}%
\geometry{a4paper,
  lmargin=2cm,rmargin=2cm,tmargin=1.5cm,bmargin=1.5cm}
  
\input{../../../macros.tex}



\begin{document}
\begin{flushleft}
ECE2 \\
Mathématiques
\end{flushleft}

\begin{center}
\textbf{\Large{Colles - Semaine 14}}
\end{center}

\hrule

\vspace*{0,2cm}

\begin{exercice}~
  \begin{noliste}{1.}
  \item Montrer que l'intégrale $ \dint{0}{+\infty }\dfrac{1}{\left(
        1+x\right) ^{2}} \dx$ est convergente et donner sa valeur.
  \item On considère la fonction $f$ définie par : $\forall x\in
    \mathbb{R},\ f\left( x\right) =\dfrac{1}{2\left( 1+\left \vert
          x\right \vert \right) ^{2}}$
    \begin{noliste}{a)}
    \item Montrer que $f$ est paire.
    \item Montrer que $f$ peut être considérée comme une fonction
      densité de probabilité.
    \end{noliste}
  \end{noliste}
  Dans la suite, on considère une variable aléatoire $X$, définie sur
  un espace probabilisé $\left( \Omega , \A, \Prob \right) $ 
admettant
  $f$ comme densité.\\
  On note $F$ la fonction de répartition de $X$.
  \begin{noliste}{1.}
    \addtocounter{enumi}{+2}
  \item On pose $Y=\ln \left( 1+\left \vert X\right \vert \right) $ et
    on admet que $Y$ est une variable aléatoire, elle aussi définie
    sur l'espace probabilisé $\left( \Omega , \A, \Prob \right)$.
    \begin{noliste}{a)}
    \item Déterminer $Y\left( \Omega \right) $.
    \item Exprimer la fonction de répartition $G$ de $Y$ à l'aide de
      $F.$
    \item En déduire que $Y$ admet pour densité la fonction $g$
      définie par : 
      \[ 
      g(x) = %
      \left\{
        \begin{array}{cl}
          2 \, \ee^{x}f\left( \ee^{x}-1\right) & \mbox{ si $x \geq 0$} 
	  \\
          0 & \mbox{ si $x<0$} %
        \end{array}
      \right.
      \]
    \item Montrer enfin que $Y$ suit une loi exponentielle dont on
      déterminera le paramètre.
    \end{noliste}
  \end{noliste}
\end{exercice}


\begin{exercice}~\\
  Soit $X$ et $Y$ deux variables aléatoires indépendantes de loi
  uniforme sur $[0 ,1]$.\\
  On définit les variable aléatoires $U = \min (X, Y)$ et $V = \max
  (X, Y)$.
  \begin{noliste}{1.}
  \item Démontrer que :
    \[
    \Ev{U > t} = \Ev{X > t} \cap \Ev{Y > t} \qquad \mbox{et} \qquad 
    \Ev{V \leq t} = \Ev{X \leq t} \cap \Ev{Y \leq t}
    \]
  \item Déterminer la fonction de répartition $G$, puis une densité
    $g$ de $U$.
  \item Déterminer la fonction de répartition $H$, puis une densité
    $h$ de $V$.
  \item Calculer l'espérance de $U$.
  \item Exprimer $U + V$ en fonction de $X$ et $Y$.
    En déduire l'espérance de $V$.
  \end{noliste}
\end{exercice}


\begin{exercice}~\\
  Toutes les variables aléatoires qui interviennent dans ce problème
  sont considérées comme définies sur des espaces probabilisés non
  nécessairement identiques, mais qui, par souci de simplification,
  seront tous notés $\left( \Omega , \A, \Prob\right)$.
  \begin{noliste}{1.}
  \item On considère la fonction $g$ définie sur $\R$ par : $g(x) =
    \dfrac{1}{2} \times \ee^{-\left\vert x\right\vert }$.
    \begin{noliste}{a)}
    \item Montrer que les intégrales $ \dint{-\infty }{0}g\left(
        x\right) \dx$ et $ \dint{0}{+\infty }g\left( x\right) \dx$ 
	sont convergentes et de même valeur.
    \item Établir que $g$ est une densité de probabilité sur $%
      \R$.
    \end{noliste}
  \end{noliste}
  Soit $Y$ une variable aléatoire à valeurs réelles admettant $g$ pour
  densité.\\
  On dit alors que $Y$ suit la loi $\mathcal{L} \left( 0\right)$.
  \begin{noliste}{1.}
    \addtocounter{enumi}{+1}
  \item Étudier les variations de $g$ et tracer l'allure de sa
    représentation graphique dans le plan rapporté à un repère
    orthonormé.
  \item
    \begin{noliste}{a)}
    \item Pour $r \in \N$, montrer l'existence de $m_{r}(Y)$ (moment
      d'ordre $r$ de $Y$).
    \item Calculer, pour tout $r$ de $\N$, $m_{r}(Y)$ en fonction de
      $r$. Quelles sont les valeurs de l'espérance $\E(Y)$ et de la
      variance $\V(Y)$ de la \var $Y$ ?
    \end{noliste}
  \end{noliste}
\end{exercice}








\end{document}
