\documentclass[11pt]{article}%
\usepackage{geometry}%
\geometry{a4paper,
  lmargin=2cm,rmargin=2cm,tmargin=1.5cm,bmargin=1.5cm}
  
\input{../../../macros.tex}



\begin{document}
\begin{flushleft}
ECE2 \\
Mathématiques
\end{flushleft}

\begin{center}
\textbf{\Large{Colles - Semaine 16}}
\end{center}

\hrule

\vspace*{0,2cm}

\begin{exercice}[EML 2015]~\\
Dans cet exercice, on pourra utiliser l'encadrement $2<\ee<3$.
\begin{noliste}{1.}
\item On considère l'application $\varphi:\left\{
\begin{array}{ccc}
\R & \rightarrow & \R\\
x & \mapsto & x^2 \ee^x -1
\end{array}
\right.$
\begin{noliste}{a)}
\item Dresser le tableau de variations de $\varphi$, en précisant les 
limites de $\varphi$ en $-\infty$ et $+\infty$ et sa valeur en $0$.
\item Établir que l'équation $\ee^x=\dfrac{1}{x^2}$, d'inconnue 
$x\in \ ]0,+\infty[$, admet une solution et une seule, notée $\alpha$, 
et que $\alpha$ appartient à l'intervalle $\left] 
\dfrac{1}{2},1\right[$.
\end{noliste}
On considère l'ouvert $U= \ ]0,+\infty[\times \R$ de $\R^2$ et 
l'application $g$ de classe $\Cont{2}$ suivante :
\[
g:\left\{
\begin{array}{ccc}
U & \rightarrow & \R\\
(x,y) & \mapsto & \dfrac{1}{x} + \ee^x -y^2 \ee^y
\end{array}
\right.
\]
\item Représenter graphiquement l'ensemble $U$.
\item Calculer, pour tout $(x,y)$ de $U$, les dérivées partielles 
premières de $g$ en $(x,y)$.
\item Montrer que $g$ admet deux points critiques et deux seulement et 
que ceux-ci sont $(\alpha,0)$ et $(\alpha,-2)$, où $\alpha$ est le réel 
défini à la question $2$.
\item Est-ce que $g$ admet un extremum local en $(\alpha,0)$ ?
\item Est-ce que $g$ admet un extremum local en $(\alpha,-2)$ ?
\item Est-ce que $g$ admet un extremum global sur $U$ ?
\end{noliste}
\end{exercice}

\begin{exercice}[ESCP 2002]~\\
\noindent
Soit $a$ un paramètre réel et $F_a$ la fonction définie sur $\R^2$ par :
\[
  \begin{matrix}
    \forall (x,y)\in\R^2,\quad  F_a(x,y)=& \textbf{(}x & y & 
    a\textbf{)} \\ & & &
    \\
    & & & 
  \end{matrix}
  \;
  \begin{smatrix} 
    -3 & 1 & 1 \\
    1 & -3 & 1 \\ 
    1 & 1 & -3
  \end{smatrix}
  \; 
  \begin{smatrix} 
    x \\ 
    y \\ 
    a 
  \end{smatrix}
\]
\begin{noliste}{1.}
  \item Déterminer, pour tout $(x,y)\in\R^2$, l'expression de $F_a(x,y)$ en 
  fonction de $x,y$ et $a$.
  
  \item Vérifier que cette fonction est de classe $\Cont{1}$ sur $\R^2$ et 
  calculer ses dérivées partielles d'ordre $1$ en tout point $(x,y)$ de 
  $\R^2$.
  
  \item Montrer qu'il existe un unique point $(x_0,y_0)$ de $\R^2$, que 
  l'on précisera, en lequel les dérivées partielles d'ordre $1$ de $F_a$ sont
  nulles. Calculer $F_a(x_0,y_0)$.
  
  \item Calculer, pour tout couple $(x,y)$ de $\R^2$, le nombre :
  \[
    G_a(x,y)=F_a(x,y)+\frac{1}{3}(3x-y-a)^2+2a^2,
  \]
  et préciser son signe.	
  
  \item En déduire que la fonction $F_a$ admet un unique extremum sur $\R^2$. 
  Préciser s'il s'agit d'un minimum ou d'un maximum global et donner sa valeur 
  notée $M(a)$.
  
  \item Montrer que la fonction $M$ qui, à tout réel $a$ associe le nombre 
  $M(a)$, admet un unique extremum que l'on précisera. Que peut-on en conclure 
  ?
\end{noliste}
\end{exercice}


\newpage


\begin{exercice}[INSEEC 2002]~\\
\noindent
On considère la fonction $g$ définie sur $\R^3$ par :
\[ 
  \forall (x,y,z)\in\R^3,\ g(x,y,z) = 4x^2+4y^2+2z^2+4xz-4yz
\]
On définit la fonction $f:\R^2\longrightarrow \R$ par : $\forall (x,y)\in\R^2 
, \ f(x,y)=g(x,y,y^2)$.\\
On dit alors qu'on étudie la fonction $g$ \textbf{sous la contrainte} $z=y^2$.
\begin{noliste}{1.}
  \item Expliciter $f(x,y)$, et calculer $\dfn{f}{1}(x,y)$, $\dfn{f}{2}(x,y)$, 
  $\ddfn{f}{1,1}(x,y)$, $\ddfn{f}{1,2}(x,y)$ et $\ddfn{f}{2,2}(x,y)$.
  
  \item Déterminer les extrema éventuels de $f$ sur $\R^2$.
  
  \item Montrer, pour tout $(x,y,z)\in\R^3$ : 
  \[
    g(x,y,z) = 4 \, \left(x+\dfrac{1}{2}z\right)^2 + 4 \, 
    \left(y-\dfrac{1}{2}z\right)^2
  \]
  En déduire que $f$ admet un minimum global en $(0,0)$.
  
  \item Montrer que $f$ présente un minimum local en $(-2,2)$.
  
  \item Déterminer le développement limité d'ordre $2$ de $f$ en 
  $\left(-\dfrac{1}{2},1\right)$. En déduire le développement limité d'ordre 
  $2$ de $f\left(-\dfrac{1}{2}+h,1+h\right)$ et de 
  $f\left(-\dfrac{1}{2}+h,1-h\right)$, lorsque $h$ est au voisinage de $0$. En 
  déduire que $f$ ne présente pas d'extremum local 
  en $\left(-\dfrac{1}{2},1\right).$
\end{noliste}
\end{exercice}



\begin{exercice}[HEC 2017]~\\
  \noindent
  Soit $f$ la fonction de deux variables définies sur $\R^2$ par :
  \[
    f : (x,y) \mapsto x^3 + y^3 -9xy +1
  \]
  \begin{noliste}{1.}
    \item 
    \begin{noliste}{a)}
      \item Donner le développement limité à l'ordre $2$ de $f$ au voisinage 
      de $(0,0)$.
      
      \item En déduire que $(0,0)$ est un point col de $f$.
    \end{noliste}
    
    \item 
    \begin{noliste}{a)}
      \item Montrer que $f$ admet un extremum local.
      
      \item Cet extremum est-il global ?
    \end{noliste}
  \end{noliste}
\end{exercice}





\end{document}
