\documentclass[11pt]{article}%
\usepackage{geometry}%
\geometry{a4paper,
  lmargin=2cm,rmargin=2cm,tmargin=1.5cm,bmargin=1.5cm}
  
\input{../../../macros.tex}



\begin{document}
\begin{flushleft}
ECS2 \\
Mathématiques
\end{flushleft}

\begin{center}
\textbf{\Large{Colles - Semaine 7}}
\end{center}

\hrule

\vspace*{0,2cm}


\section*{Série 1}
\subsection*{Question de cours}
\noindent
Démontrer l'inégalité de Cauchy-Schwarz et son cas d'égalité.

\subsection*{Exercice} % ESCP 2003
\noindent
Soit $a$ un nombre r\'eel tel que $ 0 < a < 1$ et $ b$ un nombre r\'eel 
strictement positif.\\
On consid\`ere un couple $(X, Y)$ de variables al\'eatoires \`a valeurs 
dans  $\N^2$,
dont la loi de probabilit\'e est donn\'ee par:
\[
\Prob(\Ev{X=i}\cap \Ev{Y=j})=\begin{cases}\hfill 0\hfill &\text{ si 
$i<j$}\cr
\dfrac{b^i\, \ee^{-b}a^j(1-a)^{i-j}}{j!(i-j)!}&\text{ si $i\geq 
j$}\end{cases}
\]

\begin{noliste}{1.}
\item
D\'eterminer la loi de probabilit\'e de $X$. D\'eterminer, si elles 
existent, son esp\'erance et sa variance.
\item
D\'eterminer la loi de probabilit\'e de $ Y$.
\item
Les variables $X$ et $Y$ sont-elles ind\'ependantes ?
\item
Soit $Z$ la variable al\'eatoire  $Z = X-Y$. D\'eterminer sa loi.
\item
Les variables $Y$ et $Z$ sont-elles ind\'ependantes ?
\end{noliste}



\newpage

\section*{Série 2}
\subsection*{Question de cours}
\noindent
Démontrer que le produit scalaire canonique sur l'ensemble des matrices 
carrées est bien un produit scalaire

\subsection*{Exercice} % ESCP 2017
\noindent
Soit $n\in\N^*$ et $E$ l'espace vectoriel des polynômes à coefficients 
réels de degré inférieur ou égal à $n$.
\begin{noliste}{1.}
 \item Montrer qu'on définit sur $E$ un produit scalaire en posant :
 \[
  \forall (P,Q)\in E^2, \ \langle P,Q \rangle = \dint{0}{+\infty} 
  P(t)Q(t) \ee^{-t} \dt
 \]
 On note $\Vert . \Vert$ la norme associée.
 
 \item Montrer qu'il existe une base $\mathcal{Q}=(Q_0,Q_1, \hdots, 
  Q_n)$ orthonormée de $E$, pour ce produit scalaire, telle que pour 
  tout $k\in\llb 0,n\rrb$, $\deg(Q_k)=k$.
  
  \item 
  \begin{noliste}{a)}
    \item Que dire de l'ensemble $F=\{P\in E, \ P(0)=0\}$ ?
    
    \item En déterminer une base à l'aide des éléments de $\mathcal{Q}$.
  \end{noliste}
  
  \item Soit $U=Q_0 + \Sum{j=1}{n} Q_j(0) \ Q_j$.
  Justifier que : $\forall \ V \in F$, $\langle V,U\rangle =0$.
  
  \item Montrer que, pour tout $k\in\llb 1,n\rrb$, on a :
  $\langle Q_k, Q_k'\rangle =0$.
  
  \item Soit $k\in\llb 1,n\rrb$. En calculant de deux façons :
  \[
   I_k=\dint{0}{+\infty} \dfrac{d}{dt}\left[-\left(Q_k(t)\right)^2 
   \ee^{-t}\right] \dt
  \]
  déterminer la valeur de $\left(Q_k(0)\right)^2$.
  
  \item On note $F^\perp = \{ P \in E \ / \ \forall \ V \in F, \ 
  \langle P, V\rangle =0\}$.
  On admettra que : $F^\perp = \Vect{U}$.\\
  Calculer $\delta = \min\{\Vert Q_0-P\Vert, \ P \in F^\perp\}$ 
  et $d=\min\{\Vert Q_0-P\Vert, \ P\in F\}$.
\end{noliste}







\newpage
 
\section*{Série 3}
\subsection*{Question de cours}
\noindent
Soit $X$ une \var à densité. Soit $(a,b)\in\R^2$. Que vaut $\V(aX+b)$ 
?\\
Démontrer ce résultat.

\subsection*{Exercice} % ESCP 2013
\noindent
Toutes les variables aléatoires de  cet exercice sont définies
sur un même espace probabilisé $(\Omega,{\cal A},\Prob)$.

\begin{noliste}{1.}
\item
Soit $Z$ une variable al\'{e}atoire \`{a} valeurs dans $\N$.\\ 
Montrer
que la variable al\'{e}atoire $2^{-Z}$ admet une esp\'{e}rance. On la
note $r(Z)$.\\
On suppose dans la suite de l'exercice   que pour tout $n\in \N$,
$\Prob(\Ev{Z=n})=  \left( \dfrac{1}{2} \right)^{n+1}$.

\item
\begin{noliste}{a)}
\item
Montrer que l'on d\'{e}finit ainsi une  loi de probabilit\'{e} et
calculer $r(Z)$.
\item
Montrer que pour tout $(n,q)\in \N^2$, $ \DSum{k=0}{n}
\dbinom{k+q}{ q}= \dbinom{n+q+1}{ q+1}$.
\item
Soit $(X_i)_{i\in\N^*}$ une suite de variables al\'{e}atoires
indépendantes de même loi que $Z$ et pour tout entier $q\geq 1$,
on pose
$S_q =
\DSum{i=1}{q} X_i$.\\
Montrer que la loi de $S_q$ est d\'{e}finie par :
\[
\forall\, n \in \N,  \quad \Prob(\Ev{S_q=n})= \dbinom{n+q-1}{q-1}\left(
\dfrac12\right)^{n+q}
\]
\item
Calculer $r(S_q)$. En d\'{e}duire que
\[
\DSum{n=0}{+\infty} \dbinom{n+q-1}{ q-1} \left(
\dfrac{1}{4}\right)^{n} = \left( \dfrac{4}{3}\right)^q
\]
\end{noliste}
\item
On suppose dans cette question  que $Z$ représente le nombre de
lionceaux devant naître en 2014 d'un couple de lions. Chaque
lionceau a la probabilité $\dfrac{1}{2}$ d'être mâle ou 
femelle,
indépendamment des autres. On note $F$ la variable aléatoire
représentant le nombre de femelles devant naître en 2014.\\
Déterminer la loi de $F$.
\end{noliste}





\end{document}
