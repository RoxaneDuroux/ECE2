\documentclass[11pt]{article}%
\usepackage{geometry}%
\geometry{a4paper,
  lmargin=2cm,rmargin=2cm,tmargin=1.5cm,bmargin=1.5cm}
  
\input{../../../macros.tex}



\begin{document}
\begin{flushleft}
K \\
Mathématiques
\end{flushleft}

\begin{center}
\textbf{\Large{Colles - Semaine 7}}
\end{center}

\hrule

\vspace*{0,2cm}

\section*{Planche 1}

\noindent
  On considère $E$ un espace vectoriel de dimension finie $n \in
  \N^*$.\\
  On considère $f$ un endomorphisme de $\LL{E}$.
  \begin{noliste}{A)}
  \item \dashuline{Suite des noyaux itérés}
    \begin{noliste}{1.}
    \item Démontrer : $\forall i \in \N$, $\kr\big( f^i \big) \subset
      \kr\big( f^{i+1} \big)$.
    \item Dans cette question, on suppose qu'il existe $r \in \N$ tel
      que :
      \[
      \kr\big( f^{r} \big) = \kr\big( f^{r+1} \big)
      \]
      Démontrer : $\forall i \in \llb r, +\infty \llb$, $\kr\big( f^i
      \big) = \kr\big( f^{i+1} \big)$.
    \item Pour tout $i \in \N$, on note : $d_i = \dim\left(
        \kr\big(f^{i}\big) \right)$.
      \begin{noliste}{a)}
      \item Démontrer que la suite $(d_i)_{i\in\N}$ est monotone.
      \item En procédant par l'absurde, démontrer qu'il existe $r \in
        \llb 0, n \rrb$ tel que : $d_r = d_{r+1}$.
      \item En déduire que la suite $(d_i)_{i\in\N}$ est
        stationnaire.
      \end{noliste}
    \end{noliste}

  \item \dashuline{Suite des images itérées}
    \begin{noliste}{1.}
    \item Démontrer : $\forall j \in \N$, $\im\big( f^{j+1} \big)
      \subset \im\big( f^{j} \big)$.
    \item Dans cette question, on suppose qu'il existe $s \in \N$ tel
      que :
      \[
      \im\big( f^{s+1} \big) = \im\big( f^{s} \big)
      \]
      Démontrer : $\forall j \in \llb s, +\infty \llb$, $\im\big(
      f^{j+1} \big) = \im\big( f^{j} \big)$.
    \item Pour tout $j \in \N$, on note : $m_j = \dim\left(
        \im\big(f^{j}\big) \right)$.
      \begin{noliste}{a)}
      \item Démontrer que la suite $(m_j)_{j\in\N}$ est monotone.
      \item Démontrer : $m_{r+1} = m_{r}$ (où $r$ est l'entier défini
        en question \itbf{A.3.d)}).
      \item En déduire que la suite $(m_j)_{j\in\N}$ est stationnaire.
      % \item Démontrez enfin : $r = s$.
      \end{noliste}
    \end{noliste}
  \end{noliste}


\newpage


\section*{Planche 2}

\begin{exercice}~\\ % ENS 2018
Soit $E$ un espace vectoriel. Une {\it involution} de $\LL{E}$ est un 
endomorphisme $u \in \LL{E}$ tel que $u \circ u = \id_E$, où $\id_E$ 
désigne l'endomorphisme identité.
\begin{noliste}{1.}
  \item Soient $a$ et $b$ deux endomorphismes bijectifs de $\LL{E}$ 
  vérifiant :
  \[
    a \circ b \circ a =b \quad \text{et} \quad b \circ a \circ b = a
  \]
  Montrer que $a^2=b^2$ et que $a^2$ est une involution.
  
  \item Soient $a$ et $b$ deux involutions de $\LL{E}$.
  \begin{noliste}{a)}
    \item Montrer que $\im(a \circ b - b\circ a) \subset \im(a-b) 
    \cap \im(a+b)$.

    \item Montrer que $\im(a-b) \cap \im(a+b) \subset \im(a \circ b - b 
    \circ a)$.
  \end{noliste}
\end{noliste}
\end{exercice}

\begin{exercice}~\\ % ENS 16
  Soient $E$, $F$, $G$ trois espaces vectoriels, $f:E \to F$ et $g:F 
  \to G$ deux applications linéaires. Montrer  :
\begin{noliste}{1.}
  \item $\kr(g \circ f) = \kr(f) \ \Leftrightarrow \ \kr(g) \cap \im(f) 
  = \{0\}$
  
  \item $\im(g \circ f) = \im(g) \ \Leftrightarrow \ \kr(g) + \im(f) 
  = F$
\end{noliste}
\end{exercice}



\newpage


\section*{Planche 3}


\noindent
Soit $E$ un $\R$ espace vectoriel et $u$ un endomorphisme de $E$. Si 
$F$ est un sous-espace vectoriel de $E$, on dit qu $F$ est stable par 
$u$ si : $\forall x \in F$, $u(x) \in F$. Pour tout entier $k \geq 1$, 
on note $u^k$ l'application $u \circ u \circ \cdots \circ u$ où $u$ 
apparaît $k$ fois.\\
Soient $n \geq 1$ un entier et $p$ un projecteur. On suppose que $u^n$ 
est l'application linéaire identité et que $\im(p)$ est stable par $u$. 
On pose :
\[
  q \ = \ \dfrac{1}{n} \ \Sum{k=1}{n} u^k \circ p \circ u^{n-k}
\]
\begin{noliste}{1.}
  \item Montrer que $\im(q) \subset \im(p)$ et que $p \circ q = q$.
  
  \item Montrer : $q \circ u = u \circ q$.
  
  \item Montrer : $q \circ p =p$.
  
  \item Montrer que $q$ est un projecteur.
  
  \item Montrer que $\kr(q)$ est un supplémentaire de $\im(p)$ stable 
  par $u$.
\end{noliste}





\end{document}
