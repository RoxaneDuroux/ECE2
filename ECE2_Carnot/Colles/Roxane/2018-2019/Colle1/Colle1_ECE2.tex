\documentclass[11pt]{article}%
\usepackage{geometry}%
\geometry{a4paper,
  lmargin=2cm,rmargin=2cm,tmargin=1.5cm,bmargin=1.5cm}
  
\input{../../../macros.tex}



\begin{document}
\begin{flushleft}
ECE2 \\
Mathématiques
\end{flushleft}

\begin{center}
\textbf{\Large{Colles - Semaine 1}}
\end{center}

\hrule

\vspace*{0,2cm}
 
\begin{exercice}~
\begin{noliste}{1.}
\item \'Etudier la fonction $f$ définie sur $[0,+ \infty($, par $ 
f(x) = \dfrac{4}{3+x}$.
\item On considère la suite $(u_n)$ définie par $ \left\{ 
\begin{matrix}
u_0=a \in \R_+^*\\
\forall n \in \N, \ u_{n+1} = \dfrac{4}{3+u_n}
\end{matrix} \right.$
\begin{noliste}{a)}
\item Montrer que la suite $(u_n)$ est bien définie.
\item Déterminer la seule limite possible $\ell$ de la suite $(u_n)$.
\item Montrer que pour tout $ x \geq 0, \ \vert f'(x) \vert \leq
\dfrac{4}{9}$.
\item Montrer que pour tout $ n \in \N, \ \vert u_n - \ell \vert 
\leq \left( \dfrac{4}{9} \right)^n \vert a- \ell \vert$.
\item Conclure quant à la convergence de la suite $(u_n)$.
\end{noliste}
\end{noliste}
\end{exercice}

\begin{exercice}~\\
  On considère les fonctions $f_n : x \mapsto x^n + x - 1$ pour $n \in
  \N^*$.
  \begin{noliste}{1.}
  \item Soit $n\in\N^*$. Démontrer que l'équation $f_n(x) = 0$ admet
    une unique solution $x_n \in \ ]0,1[$.\\
    On s'intéresse maintenant à la suite $(x_n)$.
  \item Démontrer que, pour tout $n>0$ : $f_{n+1}(x_n) <
    f_{n+1}(x_{n+1})$.\\
    En déduire que : $\forall n>0, \ x_n < x_{n+1}$.
  \item Démontrer que $(x_n)$ converge et que sa limite $\ell$ est
    telle que $0 < \ell \leq 1$.
  \item Démontrer que : $\forall n>0, \ x_n \leq \ell$.
  \item En procédant par l'absurde, montrer que $\ell = 1$.
  \end{noliste}
\end{exercice}


\begin{exercice}~\\
  On considère la suite $(u_n)$ définie par :
  $
  \left\{
  \begin{array}{l}
    u_0 \in \R \\ 
    \forall n \in \N, \ u_{n+1} = \exp (u_n) - 1
  \end{array}
  \right.
  $\\[.2cm]
  On note $f$ la fonction définie par : $f(x) = \exp(x) - 1$.
\begin{noliste}{1.}
\item Montrer que l'équation $f (x) = x$ a une unique solution qui est
  $0$.\\
  Déterminer le signe de $f(x) - x$. Préciser le sens de variations de
  $f$.\\[.1cm]
  {\bf On suppose maintenant que $u_0 = 1$.}
\item Montrer que pour tout entier $n$, $1 \leq u_n \leq u_{n+1}$.
\item Montrer que $(u_n)$ n'est pas majorée et en déduire sa limite.
\item Monter que si $x \geq 1$ alors $f (x) \geq (e - 1) x$.
\item En déduire que pour tout entier $n$, $u_n \geq (e - 1)^n$ et
  retrouver la limite de la suite.\\[.1cm]
  {\bf On suppose maintenant que $u_0 < 0$.}
\item Montrer que pour tout entier $n$, $u_n < 0$.
\item En déduire que $(u_n)$ est croissante puis qu'elle converge vers 
$0$.
\end{noliste}
\end{exercice}

% \newpage
% 
% \begin{exercice}~
% \begin{noliste}{1.}
% \item \begin{noliste}{a)}
% 	\item Rappeler la définition de \og la suite $(u_n)$ converge 
% vers $a$ \fg{}.
% 	\item Supposons que $(u_n)$ est une suite réelle convergente de 
% limite $a \in \R_+^*$.\\
% 	Démontrer qu'il existe un entier $n_0 \in \N$ tel que : $ 
% \forall n \geq n_0, \ u_n \geq \dfrac{a}{2}$.
% 	\end{noliste}
% \item On considère la fonction $f \ : \ x \mapsto x(1-x)$, et la suite 
% $(u_n)$ définie par : $ \left\{ \begin{matrix}
% u_0 \in ]0,1[\\
% \forall n \in \N, \ u_{n+1} =f(u_n)
% \end{matrix} \right.$.
% \begin{noliste}{a)}
% \item \'Etudier les variations de $f$.
% \item \begin{noliste}{i.}
% 	\item Montrer que : $ \forall n \in \N, \ 0<u_n< 
% \dfrac{1}{n+1}$.\\
% 	En déduire la limite de la suite $(u_n)$.
% 	\item Pour tout entier naturel $n$, on pose $v_n=nu_n$.\\
% 	Montrer que la suite $(v_n)$ est croissante. En déduire qu'elle 
% converge et que sa limite $L$ vérifie $L \in ]0,1]$.
% 	
% 	\item Pour tout entier naturel $n$, on pose 
% $w_n=n(v_{n+1}-v_n)$.\\
% 	Montrer que la suite $(w_n)$ est convergente et que sa limite 
% vaut $L(1-L)$.
% 	\end{noliste}
% \end{noliste}
% \item On suppose que $L \neq 1$.\\
% Montrer en utilisant le préliminaire qu'il existe un entier naturel 
% $n_0$ tel que :
% \[
% \forall n \geq n_0, \ v_{n+1} -v_n \geq \dfrac{L(1-L)}{2n}.
% \]
% En déduire que $ \dlim{n \to + \infty} v_n = + \infty$.
% 
% \item En déduire que $ u_n \underset{n \to + \infty}{\sim} 
% \dfrac{1}{n}$.
% \end{noliste}
% \end{exercice}
% 
% 
% \begin{exercice}~\\
% On considère la fonction $f \ : \ x \mapsto 2xe^x$.
% \begin{noliste}{1.}
% \item \begin{noliste}{a)}
% 	\item Montrer que $f$ réalise une bijection de $[0,1]$ sur un 
% ensemble que l'on déterminera.
% 	\item Donner les tableaux de variations de $f$ et de $f^{-1}$.
% 	\end{noliste}
% \item \begin{noliste}{a)}
% 	\item Vérifier qu'il existe un et un seul réel $\alpha$ dans 
% $[0,1]$, tel que $\alpha e^\alpha =1$.
% 	\item Montrer que $\alpha \neq 0$.
% 	\end{noliste}
% \item On définit la suite $(u_n)$ par : $ \left\{ \begin{matrix}
% u_0=\alpha\\
% \forall n \in \N, \ u_{n+1}=f^{-1}(u_n)
% \end{matrix} \right.$.\\
% Montrer que, pour tout entier naturel $n$, $u_n$ existe et $u_n \in 
% ]0,1]$.
% \item \begin{noliste}{a)}
% 	\item Montrer que pour tout réel $x$ de $[0,1]$, $f(x)-x \geq 
% 0$.
% 	\item En déduite déduire que la suite $(u_n)$ est décroissante.
% 	\item Montrer que la suite $(u_n)$ est convergente et qu'elle a 
% pour limite $0$.
% 	\end{noliste}
% \end{noliste}
% \end{exercice}
% 
% \begin{exercice}~
% \begin{noliste}{1.}
% \item \'Etudier les variations de la fonction $f$ définie sur $[0, + 
% \infty[$ par :
% \[
% \forall x \in [0, + \infty[, \ f(x)=x+ e^x.
% \]
% \item Montrer que pour tout $n \in \N^*$, l'équation $x+ e^x =n$ admet 
% une unique solution dans $\R_+$ notée $u_n$. Préciser la valeur de 
% $u_1$.
% \item \begin{noliste}{a)}
% 	\item Démontrer que la suite $(u_n)$ est strictement croissante.
% 	\item La suite $(u_n)$ est-elle majorée ? En déduire la limite 
% de $(u_n)$.
% 	\end{noliste}
% \item \begin{noliste}{a)}
% 	\item Montrer que : $\forall n \in \N^*, \ n- \ln(n) \leq 
% e^{u_n} \leq n$.
% 	\item en déduire que : $ u_n \underset{n \to + \infty}{\sim} 
% \ln(n)$.
% 	\end{noliste}
% \item Pour tout $n \in \N^*$, on note $v_n=u_n- \ln(n)$. 
% \begin{noliste}{a)}
% 	\item démontrer que : $ \forall n \in \N^*, \ e^{v_n} = 1- 
% \dfrac{u_n}{n}$.
% 	\item En déduire un équivalent simple de $v_n$.
% 	\end{noliste}
% \end{noliste}
% \end{exercice}
%  
 
 
 
 
 
\end{document}
