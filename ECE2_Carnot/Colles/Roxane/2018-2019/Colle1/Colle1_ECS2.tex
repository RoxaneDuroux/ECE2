\documentclass[11pt]{article}%
\usepackage{geometry}%
\geometry{a4paper,
  lmargin=2cm,rmargin=2cm,tmargin=1.5cm,bmargin=1.5cm}
  
\input{../../../macros.tex}



\begin{document}
\begin{flushleft}
ECS2 \\
Mathématiques
\end{flushleft}

\begin{center}
\textbf{\Large{Colles - Semaine 1}}
\end{center}

\hrule

\vspace*{0,2cm}

\section*{Série 1}
\subsection*{Question de cours}
Développement limité en $0$ à l'ordre $6$ de $f(x)=\cos(x)\sin(3x)$.

\subsection*{Exercice 1}
\noindent
Montrer que $A(a)$, $B(b)$ et $C(c)$ sont alignés si et seulement si 
$a\bar{b} + b\bar{c} + c\bar{a} \in \R$.

\subsection*{Exercice 2}
\noindent
On considère la suite $(u_n)$ définie par 
\[
\left\{
\begin{array}{l}
u_0=1\\
\forall n \in \N, \ u_{n+1} = \dfrac{1}{u_n+n+1}
\end{array}
\right.
\]
\begin{noliste}{1.}
\item \begin{noliste}{a)}
\item Montrer que : $\forall n \in \N, u_n >0$. En déduire que $(u_n)$ 
est bien définie.
\item Montrer que : $  \forall n \in \N^*, \ u_n \leq \dfrac{1}{n}$.
\item Déterminer la limite de la suite $(u_n)$.
\end{noliste}
\item \begin{noliste}{a)}
\item Déterminer la limite de la suite $(nu_n)$.
\item En déduire un équivalent de $u_n$.
\end{noliste}
\item \begin{noliste}{1.}
\item Démontrer que $ \dlim{n \to + \infty} n^3 \left( u_n 
-\dfrac{1}{n} \right) = -1$.
\item En déduire que $ u_n = \dfrac{1}{n} - \dfrac{1}{n^3} + \oon 
\left( \dfrac{1}{n^3} \right)$.
\end{noliste}
\end{noliste}

\newpage

\section*{Série 2}
\subsection*{Question de cours}
Calculer $\dint{1}{+\infty} \dfrac{\ln(x)}{x(1+(\ln(x))^4} \dx$.

\subsection*{Exercice 1}
\noindent
Pour tous polynômes $P$ et $Q$ de $\C[X]$, on pose 
$[P,Q]=\bar{P}Q-P\bar{Q}$.
\begin{noliste}{1.}
 \item Discuter le degré de $[P,Q]$ si $\deg(P)=p$ et $\deg(Q)=q$.
 \item Montrer que pour tous polynômes $P$, $Q$ et $R$ :
 \[
  [[P,Q],R]+[[Q,R],P]+[[R,P],Q]=0
 \]
\end{noliste}


\subsection*{Exercice 2}
\noindent
On introduit la fonction $f$ définie par : $\forall x \in\R$, 
$f(x)=\ee^x -2$.\\
On note $(u_n)$ la suite telle que : $\left\{
\begin{array}{ll}
 u_0=-1\\
 \forall n\in\N, \ u_{n+1}=f(u_n)
\end{array}
\right.$
\begin{noliste}{1.}
 \item Montrer que l'équation $f(x)=x$ admet une unique solution 
 strictement négative.
 \item Montrer que : $\forall n \in\N$, $\alpha \leq u_n \leq -1$.
 \item Montrer que pour tout $n\in\N$, $0\leq u_{n+1} -\alpha \leq 
 \dfrac{1}{\ee}(u_n-\alpha)$.
 \item En déduire que 
 \[
  \forall n\in\N, \ 0\leq u_n - \alpha \leq 
  \left(\dfrac{1}{\ee}\right)^n
 \]
 \item Écrire un programme \Scilab{} permettant de déterminer une 
 valeur approchée de $\alpha$ à $10^{-4}$ près.
\end{noliste}

\newpage

\section*{Série 3}
\subsection*{Question de cours}
Déterminer l'expression explicite de la suite définie par $\left\{
\begin{array}{ll}
 u_0=1 \\
 \forall n\in\N, \ u_{n+1} = \dfrac{4}{5}u_n +1
\end{array}
\right.$

\subsection*{Exercice 1}
\noindent
Étudier la suite définie par $u_0=2017$ et pour tout $n\geq 0$ :
\[
 u_{n+1} = 1805 + \sqrt{u_n}
\]


\subsection*{Exercice 2}
\begin{noliste}{1.}
 \item Montrer que pour tous $a$, $b$ de $\C^*$, on a $\left\vert 
\dfrac{a}{\vert a \vert^2} - \dfrac{b}{\vert b\vert^2} \right\vert = 
\dfrac{\vert a-b\vert}{\vert a \vert \cdot \vert b \vert}$.
 \item Montrer que : $\forall (x,y,z)\in\C^3$, $\vert x\vert \cdot 
\vert y-z\vert \leq \vert y \vert \cdot \vert z-x\vert + \vert z \vert 
\cdot \vert x-y\vert$.
 \item Montrer l'inégalité dite de Ptolémée :
 \[
  \forall (x,y,z,w)\in\C^4, \ \vert x-y\vert \cdot \vert z-w\vert \leq 
\vert x-z\vert \cdot \vert y-w\vert + \vert x-w\vert \cdot \vert 
y-z\vert
 \]

\end{noliste}




\end{document}
