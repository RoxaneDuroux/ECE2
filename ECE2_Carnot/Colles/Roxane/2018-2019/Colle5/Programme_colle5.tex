\documentclass[11pt]{article}%
\usepackage{geometry}%
\geometry{a4paper,
  lmargin=2cm,rmargin=2cm,tmargin=2.5cm,bmargin=2.5cm}
  
\input{../../../macros.tex}



\begin{document}
\begin{flushleft}
ECE2 \\
Mathématiques
\end{flushleft}


\vspace{0.1cm}

\begin{center}
\textbf{\Large{Programme de colle - Semaine 5}}
\end{center}

\hrule

\vspace*{0,2cm}

\section*{Notation}

On adoptera les principes suivants pour noter les étudiants :
\begin{noliste}{$\stimes$}
\item si l'étudiant sait répondre à la question de cours, il 
aura une note $>8$.
\item si l'étudiant ne sait pas répondre à la question de 
cours ou s'il y a trop d'hésitations, il aura une note $\leq 8$.
\end{noliste}

\section*{Questions de cours}

\begin{noliste}{$\sbullet$}
  \item {\bf Stabilité d'un sev engendré}\\
  Soit $E$ un $\R$-ev. Soit $(u_1,\hdots,u_m)\in E^m$. On a :
  \[
   u_{m+1} \in\Vect{u_1,\hdots, u_m} \ \Rightarrow \ \Vect{u_1,
   \hdots, u_m,u_{m+1}} = \Vect{u_1,\hdots, u_m}
  \]
  
  \begin{proof}[Preuve]~\\
  Supposons $u_{m+1}\in\Vect{u_1,\hdots, u_m}$.\\
  Démontrons que $\Vect{u_1,\hdots, u_m,u_{m+1}}=\Vect{u_1,
  \hdots,u_m}$.
  \begin{noliste}{$\stimes$}
  \item $(\supset)$ Évident.
  
  \item $(\subset)$ Comme $u\in \Vect{u_1,\hdots,u_m,u_{m+1}}$, alors 
  le 
  vecteur $u$ s'écrit $u=\Sum{i=1}{m+1} \lambda_i \cdot u_i$.\\
  Or $u_{m+1}\in\Vect{u_1,\hdots,u_m}$, donc $u_{m+1}=\Sum{i=1}{m} 
  \mu_i \cdot u_i$. Ainsi :
  \[
    \begin{array}{rclcc}
    u &=& \left(\Sum{i=1}{m} \lambda_i \cdot u_i\right) & + & 
    \lambda_{m+1} \cdot u_{m+1}
    \\[.6cm]
    &=& \left(\Sum{i=1}{m} \lambda_i \cdot u_i\right) & + & 
    \lambda_{m+1} \cdot \left( \Sum{i=1}{m} \mu_i \cdot u_i\right)
    \\[.6cm]
    &=& \Sum{i=1}{m} (\lambda_i+\lambda_{m+1} \times \mu_i) \cdot u_i
    \end{array}
  \]
  et donc $u\in\Vect{u_1,\hdots,u_m}$.
  \end{noliste}~\\[-1cm]
  \end{proof}
  
  \item {\bf Techniques de base}\\
  On choisira de demander au choix à l'étudiant de :
  \begin{noliste}{$\stimes$}
    \item montrer qu'un espace $F$ est un sous-espace 
    vectoriel d'un espace vectoriel $E$, sur un exemple dans $\R^n$, 
    $\M{n,p}$, $\R[X]$, $\R_n[X]$, $\R^\N$, $\R^\R$, etc.
    \item montrer qu'une famille de vecteurs est génératrice d'un 
    espace vectoriel donné, sur un exemple.
    \item montrer qu'une famille de vecteurs est libre dans un 
    espace vectoriel donné, sur un exemple.
  \end{noliste}
  
  
  \newpage
  
  
  \item {\bf Propriétés d'une probabilité} 
  \underline{On choisira $3$ propriétés à démontrer parmi les 
  suivantes} :\\
  Soit $(\Omega,\A,\Prob)$ un espace probabilisé. Alors
  \begin{noliste}{1.}
    \item Pour tous événements $A$ et $B$ tel que $A \subset B$, 
    alors $\Prob(A) \leq \Prob(B)$.
    \item Pour tout événement $A \in \A$, $\Prob(\overline{A}) = 1 - 
    \Prob(A)$. En particulier, $\Prob(\varnothing) = 0$.
    \item $\Prob (B \setminus A) = \Prob (B) - \Prob (A \cap B)$
    \item $\Prob (A \cup B) = \Prob (A) + \Prob (B) - \Prob (A \cap 
    B)$
    \item $
      \begin{array}{rccl}
        \Prob (A \cup B \cup C) & = & & \Prob(A) \ + \ \Prob(B) \ + 
        \ \Prob(C) \\
        & & - & \Prob(A \cap B) \ - \ \Prob(A \cap C) \ - \ \Prob(B 
        \cap C) \\ 
        & & + & \Prob(A \cap B \cap C) \\ 
      \end{array}
      $\\[.1cm]
    {\em (formule du crible)}
\end{noliste}

\begin{proof}[Preuve]~
\begin{noliste}{1.}
\item Pour tous événements $A$ et $B$, les événements $A\cap B$ et 
$\overline{A} \cap B$ sont incompatibles. Si $A \subset B$ alors $A 
\cap B = A$. Ainsi
\[
\Prob(B) = \Prob\big((A \cap B) \cup (\overline{A} \cap B)\big) = 
\Prob(A \cap B) + \Prob(\overline{A} \cap B) = \Prob(A) + 
\underbrace{\Prob(\overline{A} \cap B)}_{\geq 0}.
\]  
Donc $\Prob(B) \geq \Prob(A)$.

\item $A$ et $\overline{A}$ forment un système complet d'événements 
donc 
$\Prob(\Omega) = \Prob(A \cup \overline{A}) = \Prob(A) + 
\Prob(\overline{A})$. Donc $\Prob(A) = 1 - \Prob(A)$.

\item On a : $(A \setminus B) \ \cup \ (A \cap B) = A$ (réunion
    disjointe).\\
    Ainsi, par $\sigma$-additivité :
    \[
    \Prob((A \setminus B) \ \cup \ (A \cap B)) = \Prob(A \setminus B)
    + \Prob(A \cap B) = \Prob(A)
    \]
\item On a : $A \cup B = A \cup (B \setminus A)$ (la deuxième
    réunion est disjointe).\\
    On en déduit, à l'aide du point \itbf{2)} que :
    \[
    \begin{array}{rcl}
      \Prob(A \cup B) & = & \Prob(A \cup (B \setminus A)) \\[.2cm]
      & = & \Prob(A) + \Prob(B \setminus A) \\[.2cm]
      & = & \Prob(A) + \Prob(B) - \Prob(A \cap B)
    \end{array}
    \]    

  \item Généralisation de la formule précédente :
    \[
    \begin{array}{rcl}
      \multicolumn{3}{l}{\Prob(A \cup B \cup C)} \\[.2cm]
      & = & \Prob(A \cup (B \cup C)) \\[.2cm]
      & = & \Prob(A) + \Prob(B \cup C) - \Prob(A \cap (B \cup C)) 
\\[.2cm]
      & = & \Prob(A) + \Prob(B) + \Prob(C) - \Prob(B \cap C) - 
      \Prob(A \cap (B \cup C)) \\[.2cm]
      & = & \Prob(A) + \Prob(B) + \Prob(C) - \Prob(B \cap C) - 
      \Prob((A \cap B) \cup (A \cap C)) \\[.2cm]
      & = & \Prob(A) + \Prob(B) + \Prob(C) - \Prob(B \cap C) \\[.1cm]
      & & - \left( \ \Prob(A \cap B) + \Prob(A \cap C) - \Prob((A \cap 
B) \cap 
        (A \cap C)) \ \right)\\[.2cm]
      & = & \Prob(A) + \Prob(B) + \Prob(C) \\[.1cm]
      & & - \Prob(B \cap C) - \Prob(A \cap B) - \Prob(A \cap C) \\[.1cm]
      & & + \Prob(A \cap B \cap C)
    \end{array}
    \]
\end{noliste}
\end{proof}
\end{noliste}


\section*{Connaissances exigibles}

\begin{noliste}{$\sbullet$}
\item espaces vectoriels, sous-espaces vectoriels
\item famille génératrice, famille libre, base
\item bases canoniques de $\R^n$, $\M{n,p}$ et $\R_n[X]$.
\item coordonnées dans une base
\item dimension d'un espace vectoriel
\item rang d'une famille de vecteurs, rang d'une matrice
\item \warn les élèves ne connaissent pas les endomorphismes (et donc 
pas le théorème du rang)
\item définition de tribu, probabilité
\item événements incompatibles, système complet d'événements, 
indépendance
\item probabilités conditionnelles, formule de Bayes
\item formule du crible, formule des probabilités totales, formule 
des probabilités composées
\item la notion de variable aléatoire n'a pas encore été revue.
\end{noliste}





\end{document}
