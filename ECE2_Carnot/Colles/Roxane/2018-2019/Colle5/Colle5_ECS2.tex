\documentclass[11pt]{article}%
\usepackage{geometry}%
\geometry{a4paper,
  lmargin=2cm,rmargin=2cm,tmargin=1.5cm,bmargin=1.5cm}
  
\input{../../../macros.tex}



\begin{document}
\begin{flushleft}
ECS2 \\
Mathématiques
\end{flushleft}

\begin{center}
\textbf{\Large{Colles - Semaine 5}}
\end{center}

\hrule

\vspace*{0,2cm}

\section*{Série 1}
\subsection*{Question de cours}
Montrer que $P=
\begin{smatrix}
 1 & 1 & 1\\
 -\sqrt{2} & 0 & \sqrt{2}\\
 1 & -1 & 1
\end{smatrix}$ est inversible et déterminer son inverse.


\subsection*{Exercice}
\noindent
Dans cet exercice, $x$ d\'esigne un r\'eel \'el\'ement de $[0,1[$ et $n$ 
un entier
sup\'erieur ou \'egal \`a $1$.
\begin{noliste}{1.}
\item
\begin{noliste}{a)}
\item
Montrer que: $\displaystyle \forall p\in\N^*,\quad {1\over p+1}\le
\int_p^{p+1}{1\over t}\dt\le {1\over p}$.
\item
En d\'eduire que: $\displaystyle \forall k\in\N^*,\quad 0\le
\sum_{p=1}^k{1\over p}-\ln(k)\le 1$.
\end{noliste}
\item
\begin{noliste}{a)}
\vspace{-0.2cm}\item
Montrer que : $\displaystyle \sum_{p=1}^n{x^p\over
p}=-\ln(1-x)-\int_0^x{t^n\over 1-t} \dt$.
\item

En d\'eduire que la s\'erie de terme g\'en\'eral $\displaystyle 
{x^n\over n}$ converge
et exprimer sa somme en fonction de $x$.
\end{noliste}

\item
\begin{noliste}{a)}
\item
Pour tout $x$ de $]0,1[$, calculer $\displaystyle
\lim_{n\to+\infty}\ln(n^2\ln(n)x^n)$.
\item
En d\'eduire que, pour tout $x$ de $[0,1[$, la s\'erie de terme 
g\'en\'eral $\ln(n)x^n$
est convergente.

\hskip -8mm On pose maintenant $\displaystyle 
S_n(x)=\sum_{k=1}^n\ln(k)x^k$ et
$\displaystyle S(x)=\sum_{k=1}^{+\infty}\ln(k)x^k$.
\end{noliste}
\item
Le but de cette question est de trouver un \'equivalent simple de $S(x)$ 
lorsque
$x$ est au voisinage de $1^-$.
\begin{noliste}{a)}
\item
Montrer, en utilisant la premi\`ere question, que :
$\quad  0\leq \DSum{k=1}{n}\DSum{p=1}{k}{x^k\over p}-S_n(x)\le 
\sum_{k=1}^nx^k$.
\item
En d\'eduire que:\quad$\displaystyle 0\le {1\over 
1-x}\left(\sum_{p=1}^n{x^p\over
p}-x^{n+1}\sum_{p=1}^n{1\over p}\right)-S_n(x)\le {x\over 1-x}$.
\item
Justifier que:\quad $\displaystyle 
\lim_{n\to+\infty}x^{n+1}\sum_{p=1}^n{1\over
p}=0$.
\item
En d\'eduire que: \quad $\displaystyle 
S(x)\eqx{1^-}-{\ln(1-x)\over 1-x}$.
\end{noliste}
\end{noliste}



\newpage

\section*{Série 2}
\subsection*{Question de cours}
\noindent
Montrer que la loi binomiale est stable par somme.

\subsection*{Exercice}
\noindent
Pour tout $n\in\N^{*}$, on pose ${\displaystyle 
u_{n}=\int_{0}^{+\infty}{\frac{\ee^{-x}}{x+\frac{1}{n}}dx}}$.

\begin{noliste}{1.}
\item Montrer que la suite $(u_{n})_{n\in\N^{*}}$ est bien définie.
\item Pour tout $n\in\N^{*}$, on pose alors ${\displaystyle 
v_{n}=\int_{0}^{1}{\frac{\ee^{-x}}{x+\frac{1}{n}}dx}}$
et ${\displaystyle 
w_{n}=\int_{1}^{+\infty}{\frac{\ee^{-x}}{x+\frac{1}{n}}dx}}$.

\begin{noliste}{a)}
\item Montrer que: ${\displaystyle \forall n\in\N^{*},\;0\le 
w_{n}\le\frac{1}{\ee}}$.
\item Montrer que: ${\displaystyle \forall n\in\N^{*},\; 
v_{n}\ge\frac{1}{\ee}\ln(n+1)}$.
\item Donner la limite de la suite $(u_{n})$.
\end{noliste}
\item On se propose de déterminer un équivalent de $u_{n}$ lorsque $n$
est au voisinage de $+\infty$.

\begin{noliste}{a)}
\item Montrer que l'intégrale ${\displaystyle 
I=\int_{0}^{1}{\frac{1-\ee^{-x}}{x}dx}}$
est une intégrale convergente.
\item Établir que: ${\displaystyle \forall 
n\in\N^{*},\;0\le\int_{0}^{1}{\frac{1-\ee^{-x}}{x+\frac{1}{n}}dx}\le I}$.
\item En déduire un encadrement de $v_{n}$ valable pour tout 
$n\in\N^{*}$.
\item Donner enfin, en utilisant cet encadrement, un équivalent simple 
de
$u_{n}$.
\end{noliste}
\end{noliste}


\newpage

\section*{Série 3}
\subsection*{Question de cours}
Calculer $\dint{1}{+\infty} \dfrac{\ln(x)}{x(1+(\ln(x))^4)} \dx$.

\subsection*{Exercice}
\noindent
On pose, pour tout $n\in\N$, 
$I_n=\dint{0}{1}\dfrac{x^n}{\sqrt{1-x}}\dx$.\\
\begin{noliste}{1.}
\item Montrer que $I_n$ existe, pour tout $n\in\N$.
\item Montrer que la suite $(I_n)_{n\in\N}$ est convergente.
\item \begin{noliste}{a)}
	\item Montrer que, pour tout $n\in\N^*$, 
$I_n=\dfrac{2n}{2n+1}\,I_{n-1}$.
	\item En déduire l'existence et la nature de la série de terme 
général $v_n=\ln(I_n)-\ln(I_{n-1})$, puis la limite de $(I_n)_{n\in\N}$.
	\end{noliste}
\item Pour tout $n\in\N$, on pose $J_n=\sqrt{n}\,I_n$ et 
$K_n=\sqrt{n+1}\,I_n$.
	\begin{noliste}{a)}
	\item Montrer que les suites $(J_n)_{n\in\N}$ et 
$(K_n)_{n\in\N}$ sont adjacentes.
	\item En déduire qu'il existe un réel $\alpha>0$ tel que 
$I_n\eqn \dfrac{\alpha}{\sqrt{n}}\cdot$
	\end{noliste}
\item \begin{noliste}{a)}
	\item Calculer $I_n$ en fonction de $n$.
	\item On admet la formule de Stirling : $n!
\eqn n^n\,\ee^{-n}\sqrt{2\pi n}$. Montrer que 
$I_n\eqn e\,\left(\dfrac{2n}{2n+1}
\right)^{2n+1}\frac{\sqrt{\pi}}{\sqrt{n}}\cdot$
	\item Déterminer la valeur de $\alpha$.
\end{noliste}
\end{noliste}



\end{document}
