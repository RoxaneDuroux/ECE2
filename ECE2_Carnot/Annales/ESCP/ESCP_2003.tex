\documentclass[11pt]{article}%
\usepackage{geometry}%
\geometry{a4paper,
 lmargin = 2cm,rmargin = 2cm,tmargin = 2.5cm,bmargin = 2.5cm}

\input{../../macros.tex}

\pagestyle{fancy} %
\lhead{ECE2 \hfill Mathématiques\\
} %
\chead{\hrule} %
\rhead{} %
\lfoot{} %
\cfoot{} %
\rfoot{\thepage} %

\renewcommand{\headrulewidth}{0pt}% : Trace un trait de séparation
 % de largeur 0,4 point. Mettre 0pt
 % pour supprimer le trait.

\renewcommand{\footrulewidth}{0.4pt}% : Trace un trait de séparation
 % de largeur 0,4 point. Mettre 0pt
 % pour supprimer le trait.

\setlength{\headheight}{14pt}

\title{\bf \vspace{-2cm} ESCP 2003} %
\author{} %
\date{} %
\begin{document}

\maketitle %
\vspace{-1.4cm}\hrule %
\thispagestyle{fancy}

\vspace*{.2cm}


% DEBUT DU DOC À MODIFIER : tout virer jusqu'au début de l'exo

%Définition et changement de valeurs de
compteurs%newcounter{cpt1}{section} compteur cpt1 remis à 0 à chaque
aumentation par stepcounter du compteur section%setcounter{cpt1}{3} on
met le compteur à 3%addtocounter{cpt1}{5} on ajoute 5 au compteur%
stepcounter{cpt1} on ajoute 1% ifthenelse{test}{alors}{sinon} (page
206) pour subordonner à une condition % whiledo{test}{commande} pour
faire une boucle (page 206 aussi) % value{cpt1} pour noter dans le
document la valeur de cpt1 
%Définition définitive d'opérateurs
mathématiques\newcommand{\ch}{\operatorname{ch}} 
\newcommand{\sh}{\operatorname{sh}}
\renewcommand{\tanh}{\operatorname{th}}
\renewcommand{\sinh}{\operatorname{sh}}
\renewcommand{\cosh}{\operatorname{ch}}
\newcommand{\argsh}{\operatorname{argsh}}
\newcommand{\argch}{\operatorname{argch}}
\newcommand{\argth}{\operatorname{argth}}
\newcommand{\ker}{\operatorname{Ker}}
\renewcommand{\im}{\operatorname{Im}}
\newcommand{\rg}{\operatorname{rg}}
\newcommand{\Id}{\operatorname{Id}}
\newcommand{\id}{\operatorname{id}}
\renewcommand{\leq}{\leq}
\renewcommand{\geq}{\geq }

%Définition de nouvelles couleurs : rgb(trois paramètres red green blue
entre 0 et 1); cmyk (quatre cyan magenta yellow black) entre 0 et 1;
gray (entre 0 et 1) et black, white, red, green, blue, cyan, magenta,
yellow% definecolor{0gris}{gray}{0.8} 
% Nouvelle commande pour encadrer le titre car shabox ne veut que d'une
seule ligne; ATTENTION A LA TAILLE; petite différence avec shadowbox ou
doublebox, voire fcolorbox ou colorbox (au lieu de shabox; laisser le
parbox tranquille sauf pour la taille de la boîte
\newcommand{\Tbox}[1]{\begin{center} \shabox{\parbox{0.6
\linewidth}{#1}} \end{center}} %[1] pour 1 paramètre ; #1 pour ce que
fait le 1er paramètre; entre accolades ce que fait la commande
%Mise en page en mode fancy : en-têtes et pieds de pages puis
définition des en-têtes et pieds de pages\pagestyle{fancy}
\lhead{ECE 2 - Mathématiques \\
Quentin Dunstetter - ENC-Bessières 2011$\backslash$2012}
\chead{}
\rhead{ESCP 2003}
\rfoot[ \ \thepage]{\thepage}
\cfoot{}
\lfoot{}
\thispagestyle{fancy} %Mise en page de la 1ère page en mode fancy
%Trait en bas et en haut de la page (entre en-tête et texte et texte et
pied de page)\renewcommand{\footrulewidth}{0.4pt}
\renewcommand{\headrulewidth}{0.4pt}


%DEBUT DU DOCUMENT\vspace*{3cm}

\begin{center}
{\LARG\E\textbf{BANQUE COMMUNE D'ÉPREUVES}}



{\large \textsc{CONCOURS D ADMISSION DE 2003}}



{\large \textbf{Concepteur : ESCP}}



\rule{2.39cm}{0.05cm}



{\Large \textbf{OPTION ÉCONOMIQUE}}



{\Large \textbf{MATHÉMATIQUES }}



{\Large Lundi 9 mai, de 14h à 18h}



\rule{2.39cm}{0.05cm}
\end{center}

\textit{La présentation, la lisibilité, l'orthographe, la qualité
de la rédaction, la clarté et la précision des raisonnements
entreront pour une part importante dans l'appréciation des copies.}

\textit{Les candidats sont invités à \textbf{encadrer} dans la mesure
du possible les résultats de leurs calculs.}

\textit{Ils ne doivent faire usage d'aucun document. L'utilisation de
toute
calculatrice et de tout matériel électronique est interdite. Seule
l'utilisation d'une règle graduée est autorisée.}

\textit{Si au cours de l'épreuve, un candidat repère ce qui lui semble
être une erreur d'énoncé, il la signalera sur sa copie et
poursuivra sa composition en expliquant les raisons des initiatives
qu'il sera
amené à prendre.}

\vspace*{3cm}

\section*{EXERCICE}

Soit $a,\;b$ deux entiers naturels non nuls et $s$ leur somme.\\
Une urne contient initialement $a$ boules noires et $b$ boules blanches
indiscernables au toucher.\\
On effectue dans cette urne une suite infinie de tirages au hasard
d'une
boule selon le protocole suivant :

\begin{noliste}{$\sbullet$}
\item si la boule tirée est blanche, elle est remise dans l'urne ;

\item si la boule tirée est noire, elle est remplacée dans l'urne par
une boule blanche prise dans une réserve annexe.
\end{noliste}

\noindent Avant chaque tirage, l'urne contient donc toujours $s$
boules.\\
On désigne par $(\Omega,\mathcal{B},\mathbf{P})$ un espace probabilisé
qui
modélise cette expérience et, pour tout entier naturel $n$ non nul, on
note :

\begin{noliste}{$\sbullet$}
\item $B_{n}$ l'évènement " la $n$-ième boule tirée est blanche ";

\item $X_{n}$ la variable aléatoire désignant le nombre de boules
blanches
tirées au cours des $n$ premiers tirages;

\item $u_{n}$ l'espérance de la variable aléatoire $X_{n}$,
c'est-à-dire $u_{n} = \mathbf{E}(X_{n})$.
\end{noliste}

\begin{noliste}{1.}
 \setlength{\itemsep}{4mm}
\item \textbf{Étude d'un ensemble de suites}\label{A}\\
Soit $A$ l'ensemble des suites $(x_{n})_{n\geq 1}$ de réels qui
vérifient : 
\[
\forall n\in \N^{\times },\quad s\,x_{n + 1} = (s-1)\,x_{n} + b + n
\]

\begin{noliste}{a)}
 \setlength{\itemsep}{2mm}
\item \label{vn} Soit $\alpha $ et $\beta $ deux réels et
$(v_{n})_{n\geq 1}$ la suite définie par : \quad $\forall n\in \N
$ $^{\times },\quad v_{n} = \alpha \,n + \beta $\.\\
Déterminer en fonction de $b$ et de $s$ les valeurs de $\alpha $ et
$\beta $
pour que la suite $(v_{n})_{n\geq 1}$ appartienne à $A$.

\item Soit $(x_{n})_{n\geq 1}$ une suite appartenant à $A$,
$(v_{n})_{n\geq 1}$ la suite déterminée à la question précédente et
$(y_{n})_{n\geq 1}$ la suite définie par : \ $\forall n\in \N$
$^{\times },\quad y_{n} = x_{n}-v_{n}$\.\\
Montrer que la suite $(y_{n})_{n\geq 1}$ est une suite géométrique et
expliciter, pour tout entier naturel $n$ non nul, $y_{n}$ puis $x_{n}$
en
fonction de $x_{1},\;b,\;s$ et $n$.
\end{noliste}

\item \textbf{Expression de la probabilité $\text{P}(B_{n + 1})$ à
l'aide de $u_{n}$}

\begin{noliste}{a)}
 \setlength{\itemsep}{2mm}
\item Donner, en fonction de $b$ et de $s$, les valeurs respectives de
la
probabilité $\mathbf{P}(B_{1})$ et du nombre $u_{1}$.

\item Calculer la probabilité $\mathbf{P}(B_{2})$ et vérifier l'égalité
: \ $\mathbf{P}(B_{2}) = \dfrac{b + 1-u_{1}}{s}\;.$

\item Soit $n$ un entier naturel vérifiant $1\leq n\leq a$.
Montrer que, pour tout entier $k$ de l'intervalle $[ \
\hspace{-0.15em}[0,n]\hspace{-0.13em}]$, la probabilité conditionnelle
$\mathbf{P}(B_{n + 1}/[X_{n} = k])$ est égale à $\dfrac{b + n-k}{s}\;.$
\\
En déduire l'égalité : \quad $\mathbf{P}(B_{n + 1}) = \dfrac{b +
n-u_{n}}{s}\;.$

\item Soit $n$ un entier naturel vérifiant $n>a$. \\
Si $k$ est un entier de l'intervalle $[ \
\hspace{-0.15em}[0,n-a-1]\hspace{-0.13em}]$, quel est l'évènement
$[X_{n} = k]$ ? \\
Si $k$ est un entier de l'intervalle $[ \
\hspace{-0.15em}[n-a,n]\hspace{-0.13em}]$, justifier l'égalité :
$\mathbf{P}(B_{n + 1}/[X_{n} = k]) = \dfrac{b + n-k}{s}\;. $ Montrer
enfin que l'égalité \ $\mathbf{P}(B_{n + 1}) = \dfrac{b + n-u_{n}}{s}
$\ est encore vérifiée.
\end{noliste}

\item \textbf{Calcul des nombres $u_{n}$ et $\text{P}(B_{n})$}

\begin{noliste}{a)}
 \setlength{\itemsep}{2mm}
\item Soit $n$ un entier naturel non nul. Établir, pour tout entier $k$
de
l'intervalle $[ \ \hspace{-0.15em}[n + 1-a,n]\hspace{-0.13em}]$,
l'égalité : 
\[
\mathbf{P}([X_{n + 1} = k]) = \dfrac{a-n + k}{s}\,\mathbf{P}([X_{n} =
k]) + \dfrac{b + n-k + 1}{s}\,\mathbf{P}([X_{n} = k-1])
\]
Vérifier cette égalité pour $k = n + 1,\;k = n-a$ et pour tout entier
$k$ de
l'intervalle $[ \ \hspace{-0.15em}[1,n-a-1]\hspace{-0.13em}]$\.

\item Calculer, pour tout entier naturel $n$ non nul, $u_{n + 1}$ en
fonction
de $u_{n}$ et de $n$. En déduire que la suite \ $(u_{n})_{n\geq 1}$\
appartient à l'ensemble $A$ étudié dans la question \ref{A}

\item Donner, pour tout entier naturel $n$ non nul, les valeurs de
$u_{n}$
et de $\mathbf{P}(B_{n + 1})$ en fonction de $b,\;s$ \ et $n$.

\item Quelles sont les limites des suites $(u_{n})_{n\geq 1}$\ et
$(\mathbf{P}(B_{n}))_{n\geq 1}$ ?
\end{noliste}
\end{noliste}

\section*{PROBLEME}

Dans tout le problème, on désigne par $\mathcal{C}$ l'espace vectoriel
des
applications continues de $\R.$ dans $\R.$.\\
à toute application $f$ de $\mathcal{C}$, on associe l'application
$D(f)$ de 
$\R.$ dans $\R.$ définie par : 
\[
\forall x\in \R.,\quad D(f)(x) = f(x + 1)-f(x)
\]
\textit{Les parties \textbf{A}, \textbf{B} et \textbf{C} sont
indépendantes.}
\\
\textbf{Question préliminaire : } $D$ est-il un endomorphisme de
$\mathcal{C}
$ ?

\subsection*{Partie A : Image par $D$ d'une fonction de répartition}

\begin{noliste}{1.}
 \setlength{\itemsep}{4mm}
\item \label{R} Soit $F$ une application de $\mathcal{C}$. Rappeler les
propriétés que doit posséder $F$ pour être considérée comme une
fonction de répartition.

\item Soit $F$ une application de $\mathcal{C}$ qui est une fonction de
répartition et $g$ l'application $D(F)$.

\begin{noliste}{a)}
 \setlength{\itemsep}{2mm}
\item Montrer que $g$ est positive.

\item Prouver, pour tout réel $x$, la double inégalité : \quad\
$F(x)\leq \dint{x}{x + 1}F(t)\,dt\leq F(x + 1)$.\\
En déduire que les limites \ $\dlim{x\rightarrow -\infty
}\dint{x}{x + 1}F(t)\,dt$\ et \ $\dlim{x\rightarrow + \infty
}\dint{x}{x + 1}F(t)\,dt$\ existent et préciser leurs valeurs.

\item Soit $A$ et $B$ deux réels vérifiant $A<0<B$ et $I(A,B)$
l'intégrale : 
$I(A,B) = \dint{A}{B}g(t)\,dt$. \\
Justifier l'égalité : \quad\ $I(A,B) = \dint{B}{B +
1}F(t)\,dt-\dint{A}{A + 1}F(t)\,dt$\.

\item Prouver alors soigneusement que $g$ est une densité de
probabilité.
\end{noliste}

\item \textbf{Un exemple} \\
On suppose, dans cette question, que $F$ est la fonction de répartition
d'une variable aléatoire qui suit la loi uniforme sur l'intervalle
$[0,1]$
et on pose : $g = D(F)$.\\
Déterminer $g(x)$ pour tout réel $x$, en distinguant les cas $x<-1$,\
$-1\leq x<0$, \ $0\leq x<1$\ et \ $1\leq x$. Représenter
graphiquement l'application $g$.
\end{noliste}

\subsection*{Partie B : Recherche des valeurs propres de $D$}

Si $\lambda $ est un réel, on dit que $\lambda $ est une \textit{valeur
propre} de $D$ s'il existe une application $f$ de $\mathcal{C}$,
distincte
de l'application nulle, vérifiant : \ $D(f) = \lambda \,f$\.

\begin{noliste}{1.}
 \setlength{\itemsep}{4mm}
\item Soit $a$ un réel. On note $g_{a}$ l'application de $\mathcal{C}$
définie par : \ $\forall x\in \R.$ $,g_{a}(x) = e^{ax}$.\\
Déterminer l'application $D(g_{a})$.

\item En déduire que tout réel $\lambda$ strictement supérieur à $-1$
est une valeur propre de $D$.

\item Soit $a$ un réel. On note $h_{a}$ l'application de $\mathcal{C}$
définie par : \ $\forall x\in \R.$ $,h_{a}(x) = \sin (\pi x)\,e^{ax}$. 
\\
Déterminer l'application $D(h_{a})$.

\item En déduire que tout réel $\lambda$ strictement inférieur à $-1$
est une valeur propre de $D$.

\item Le réel $-1$ est-il une valeur propre de $D$ ?
\end{noliste}

\subsection*{Partie C : Image par $D$ d'une application polynomiale}

Pour tout entier naturel $p$, on désigne par $E_{p}$ le sous-espace de
$\mathcal{C}$ dont les éléments sont les applications polynomiales de
degré
au plus $p$.\\
On note $X$ l'application $x\mapsto x$ et, pour tout entier naturel non
nul $k$, on note $X^{k}$ l'application $x\mapsto x^{k}$.\\
Soit $(H_{i})_{i\in \N}$ la suite d'applications polynomiales définie
par : 
\[
H_{0} = 1\qquad \text{et}\qquad \forall i\in \N^{\times },\quad H_{i} =
\dfrac{1}{i!}\prod_{k = 0}{i-1}(X-k)
\]

\begin{noliste}{1.}
 \setlength{\itemsep}{4mm}
\item Préciser $H_{1},\;H_{2},\;H_{3}$ et montrer que $\mathcal{U}_{3}
= (H_{0},\;H_{1},\; H_{2}, \; H_{3})$ est une base de $E_{3}$.

\item Soit $\mathcal{B}_{3} = (1,X,X^{2},X^{3})$ la base canonique de
$E_{3}$.

\begin{noliste}{a)}
 \setlength{\itemsep}{2mm}
\item Écrire la matrice de passage $P$ de la base $\mathcal{B}_{3}$ à
la
base $\mathcal{U}_{3}$ et calculer la matrice $P^{-1}$.

\item \label{X3} Soit\ $a_{0},\;a_{1},\;a_{2},\ :\ a_{3}$ des réels et
$Q$
l'application polynomiale $a_{0} + a_{1}X + a_{2}X^{2} + a_{3}X^{3}$.
\\
Quelles sont les coordonnées de $Q$ dans la base $\mathcal{U}_{3}$ ?\\
En particulier, vérifier l'égalité : \ $X^{3} = H_{1} + 6\,H_{2} +
6\,H_{3}$.
\end{noliste}

\item \label{Poisson}\textbf{Application : moment d'ordre 3 d'une
variable aléatoire de Poisson}\\
Soit $a$ un réel strictement positif et $Z$ une variable aléatoire
suivant
la loi de Poisson de paramètre $a$.

\begin{noliste}{a)}
 \setlength{\itemsep}{2mm}
\item \label{a} Pour tout entier naturel $n$ supérieur ou égal à $3$,
on
pose : \ $S_{n} = \Sum{k = 0}{n}\dfrac{k^{3}\,a^{k}}{k!}$. \\
Transformer $S_{n}$ à l'aide de la relation : \ $\forall k\in \N$
$,\quad k^{3} = H_{1}(k) + 6\,H_{2}(k) + 6\,H_{3}(k)$.\\
En déduire que la série de terme général \ $\dfrac{n^{3}\,a^{n}}{n!}$ \
est
convergente et préciser \ $\Sum{n = 0}{\infty
}\dfrac{n^{3}\,a^{n}}{n!}\;.$

\item \label{b} En déduire que la variable aléatoire $Z$ admet un
moment
d'ordre $3$ donné par : 
\[
\mathbf{E}(Z^{3}) = a + 3a^{2} + a^{3}
\]
\end{noliste}

\item Dans cette question, $p$ est un entier naturel non nul fixé.

\begin{noliste}{a)}
 \setlength{\itemsep}{2mm}
\item Montrer que, si $Q$ appartient à $E_{p}$, $D(Q)$ appartient aussi
à $E_{p}$. \\
On note alors $D_{p}$ l'endomorphisme de $E_{p}$ qui, à tout $Q$ de
$E_{p}$,
associe $D(Q)$.

\item Montrer que la famille \;$\mathcal{U}_{p} = (H_{0}, H_{1},
\ldots, H_{p})$ est
une base de $E_{p}$.

\item Déterminer $D_{p}(H_{0})$, $D_{p}(H_{1})$ et prouver, pour tout
entier 
$i$ vérifiant $0<i\leq p$, l'égalité :
\[
D_{p}(H_{i}) = H_{i-1}.
\]

\item Écrire la matrice $M_{p}$ représentative de $D_{p}$ dans la base
$\mathcal{U}_{p}$.

\item Préciser la ou les valeurs propres de $M_{p}$. Cette matrice
est-elle
diagonalisable ?
\end{noliste}

\item \textbf{Application : moment d'ordre $p$ d'une variable aléatoire
de
Poisson}\\
Soit $p$ un entier naturel non nul fixé et $b_{0},b_{1},\ldots,b_{p}$
les réels vérifiant 
\[
X^{p} = b_{0}H_{0} + b_{1}H_{1} + \cdots + b_{p}H_{p}
\]
Par une méthode analogue à celle de la question \ref{Poisson}, montrer
que
la variable aléatoire $Z$ définie dans la question \ref{Poisson} admet
un
moment d'ordre $p$ donné par \ $\mathbf{E}(Z^{p}) = \Sum{i =
0}{p}\dfrac{b_{i}\,a^{i}}{i!}\;.$

\item Dans cette question, $p$ est un entier naturel non nul et, pour
tout
entier $i$ vérifiant $0\leq i\leq p$, on considère l'application
$\varphi_{i}$ de $E_{p}$ dans $\R.$ qui, à tout élément $Q$ de $E_{p}
$, associe le réel : 
\[
\varphi_{i}(Q) = \Sum{k = 0}{i}(-1)^{i-k}\,C_{i}{k}\,Q(k)
\]
où $\text{C}_{i}{k}$ désigne le coefficient binomial d'indices $i$ et
$k$.

\begin{noliste}{a)}
 \setlength{\itemsep}{2mm}
\item Montrer que, pour tout entier $i$ vérifiant $0\leq i\leq p$,
l'application $\varphi_{i}$ est linéaire.

\item Soit $i$ et $j$ deux entiers vérifiant $0\leq i\leq p$ et $0\leq
j\leq p$; établir les égalités : 
\[
\varphi_{i}(H_{i}) = 1\qquad \text{et\ si}\quad j\neq i,\quad \varphi
_{i}(H_{j}) = 0
\]

\item En déduire, pour tout entier $i$ vérifiant $0\leq i\leq p$, la
relation : \ $b_{i} = \Sum{k = 0}{i}(-1)^{i-k}\,C_{i}{k}\,k^{p}$\.
\end{noliste}
\end{noliste}

\label{fin}

\end{document}

