\documentclass[11pt]{article}%
\usepackage{geometry}%
\geometry{a4paper,
 lmargin = 2cm,rmargin = 2cm,tmargin = 2.5cm,bmargin = 2.5cm}

\input{../../macros.tex}

\pagestyle{fancy} %
\lhead{ECE2 \hfill Mathématiques\\
} %
\chead{\hrule} %
\rhead{} %
\lfoot{} %
\cfoot{} %
\rfoot{\thepage} %

\renewcommand{\headrulewidth}{0pt}% : Trace un trait de séparation
 % de largeur 0,4 point. Mettre 0pt
 % pour supprimer le trait.

\renewcommand{\footrulewidth}{0.4pt}% : Trace un trait de séparation
 % de largeur 0,4 point. Mettre 0pt
 % pour supprimer le trait.

\setlength{\headheight}{14pt}

\title{\bf \vspace{-2cm} ESCP 1988} %
\author{} %
\date{} %
\begin{document}

\maketitle %
\vspace{-1.4cm}\hrule %
\thispagestyle{fancy}

\vspace*{.2cm}


% DEBUT DU DOC À MODIFIER : tout virer jusqu'au début de l'exo

%Définition et changement de valeurs de
compteurs%newcounter{cpt1}{section} compteur cpt1 remis à 0 à chaque
aumentation par stepcounter du compteur section%setcounter{cpt1}{3} on
met le compteur à 3%addtocounter{cpt1}{5} on ajoute 5 au compteur%
stepcounter{cpt1} on ajoute 1% ifthenelse{test}{alors}{sinon} (page
206) pour subordonner à une condition % whiledo{test}{commande} pour
faire une boucle (page 206 aussi) % value{cpt1} pour noter dans le
document la valeur de cpt1 
%Définition définitive d'opérateurs
mathématiques\newcommand{\ch}{\operatorname{ch}} 
\newcommand{\sh}{\operatorname{sh}}
\renewcommand{\tanh}{\operatorname{th}}
\renewcommand{\sinh}{\operatorname{sh}}
\renewcommand{\cosh}{\operatorname{ch}}
\newcommand{\argsh}{\operatorname{argsh}}
\newcommand{\argch}{\operatorname{argch}}
\newcommand{\argth}{\operatorname{argth}}
\newcommand{\ker}{\operatorname{Ker}}
\renewcommand{\im}{\operatorname{Im}}
\newcommand{\rg}{\operatorname{rg}}
\newcommand{\Id}{\operatorname{Id}}
\newcommand{\id}{\operatorname{id}}
\renewcommand{\leq}{\leq}
\renewcommand{\geq}{\geq }

%Définition de nouvelles couleurs : rgb(trois paramètres red green blue
entre 0 et 1); cmyk (quatre cyan magenta yellow black) entre 0 et 1;
gray (entre 0 et 1) et black, white, red, green, blue, cyan, magenta,
yellow% definecolor{0gris}{gray}{0.8} 
% Nouvelle commande pour encadrer le titre car shabox ne veut que d'une
seule ligne; ATTENTION A LA TAILLE; petite différence avec shadowbox ou
doublebox, voire fcolorbox ou colorbox (au lieu de shabox; laisser le
parbox tranquille sauf pour la taille de la boîte
\newcommand{\Tbox}[1]{\begin{center} \shabox{\parbox{0.6
\linewidth}{#1}} \end{center}} %[1] pour 1 paramètre ; #1 pour ce que
fait le 1er paramètre; entre accolades ce que fait la commande
%Mise en page en mode fancy : en-têtes et pieds de pages puis
définition des en-têtes et pieds de pages\pagestyle{fancy}
\lhead{ECE 2 - Mathématiques \\
Quentin Dunstetter - ENC-Bessières 2011$\backslash$2012}
\chead{}
\rhead{ESCP 1988}
\rfoot[ \ \thepage]{\thepage}
\cfoot{}
\lfoot{}
\thispagestyle{fancy} %Mise en page de la 1ère page en mode fancy
%Trait en bas et en haut de la page (entre en-tête et texte et texte et
pied de page)\renewcommand{\footrulewidth}{0.4pt}
\renewcommand{\headrulewidth}{0.4pt}


%DEBUT DU DOCUMENT\vspace*{3cm}

\begin{center}
{\LARG\E\textbf{BANQUE COMMUNE D'ÉPREUVES}}



{\large \textsc{CONCOURS D ADMISSION DE 1988}}



{\large \textbf{Concepteur : ESCP}}



\rule{2.39cm}{0.05cm}



{\Large \textbf{OPTION ÉCONOMIQUE}}



{\Large \textbf{MATHÉMATIQUES }}



{\Large Lundi 9 mai, de 14h à 18h}



\rule{2.39cm}{0.05cm}
\end{center}

\textit{La présentation, la lisibilité, l'orthographe, la qualité
de la rédaction, la clarté et la précision des raisonnements
entreront pour une part importante dans l'appréciation des copies.}

\textit{Les candidats sont invités à \textbf{encadrer} dans la mesure
du possible les résultats de leurs calculs.}

\textit{Ils ne doivent faire usage d'aucun document. L'utilisation de
toute
calculatrice et de tout matériel électronique est interdite. Seule
l'utilisation d'une règle graduée est autorisée.}

\textit{Si au cours de l'épreuve, un candidat repère ce qui lui semble
être une erreur d'énoncé, il la signalera sur sa copie et
poursuivra sa composition en expliquant les raisons des initiatives
qu'il sera
amené à prendre.}

\vspace*{3cm}

\section*{EXERCICE 1 }

On considère la matrice $A = \left( 
\begin{array}{cc}
1 & 3 \\
3 & 1
\end{array}
\right) $.

\begin{noliste}{1.}
 \setlength{\itemsep}{4mm}
\item Montrer que la matrice $A$ est diagonalisable et expliciter une
matrice diagonale $D$ semblable à $A$.\\
Soit $k$ un nombre entier naturel. Calculer $D^{k}$; en déduire
$A^{k}$.

\item On considère la suite $\left( u_{n}\right) $ définie par la
relation
de récurrence : \quad $u_{n + 1} = \dfrac{u_{n} + 3}{3u_{n} + 1}$, et
la condition
initiale $u_{0} = c$, où $c$ est un nombre réel strictement positif.\\
On considère en outre les suites $\left( v_{n}\right) $ et $\left(
w_{n}\right) $ définies par les relations de récurrence : $\left\{ 
\begin{array}{l}
v_{n + 1} = v_{n} + 3w_{n} \\
w_{n + 1} = 3v_{n} + w_{n}
\end{array}
\right. $ et les conditions initiales : $v_{0} = c$, $w_{0} = 1$.

\begin{noliste}{a)}
 \setlength{\itemsep}{2mm}
\item Pour tout nombre entier naturel $n$ exprimer $u_{n}$ à l'aide de
$v_{n}
$, et de $w_{n}$.

\item Pour tout nombre entier naturel $n$, exprimer $u_{n}$ en fonction
de $n
$ et de $c$.

\item Montrer que la suite $\left( u_{n}\right) $ converge et calculer
sa
limite.
\end{noliste}
\end{noliste}

\section*{EXERCICE 2 }

Soit $f$ la fonction définie par la relation : $f\left( x\right) = x\ln
x$.

\begin{noliste}{1.}
 \setlength{\itemsep}{4mm}
\item Étudier la variation de $f$.

\item Montrer qu'il existe une fonction réelle $g$ et une seule définie
sur
l'intervalle $\left[ -\dfrac{1}{e}; + \infty \right[ $ telle que, pour
tout
point $x$ de cet intervalle : $g\left( x\right) \,\ln g\left( x\right)
= x$.

\item Étudier la variation de la fonction $g$. En particulier,
déterminer la
limite de $g$ en $ + \infty $. Tracer sur une même figure les courbes
représentatives de $f$ et de $g$.

\item Montrer que $\underset{x\rightarrow + \infty }{\lim }\dfrac{\ln
g\left(
x\right) }{\ln x} = 1$. En déduire lorsque $x$ tend vers $ + \infty $
la limite
de $\dfrac{g\left( x\right) \ln x}{x}$.
\end{noliste}

\section*{EXERCICE 3}

\begin{noliste}{1.}
 \setlength{\itemsep}{4mm}
\item Une urne contient des boules rouges et des boules blanches. La
proportion de boules rouges est égale à $q$, où $0\leq q\leq 1$. On
effectue 
$n$ tirages indépendants. avec remise, dans cette urne et on note $N$
la
variable aléatoire représentant le nombre de boules rouges tirées.\\
Déterminer la loi de probabilité de $N$. Calculer son espérance et sa
variance.

\item Soit $r$ un nombre entier naturel non nul. On suppose qu'on a
$\left(
r + 1\right) $ urnes $U_{0},U_{1},...,U_{r}$, et que, pour tout entier
$j$ tel
que $0\leq j\leq r$, la proportion de boules rouges contenues dans
l'urne $U_{j}$, vaut $\dfrac{j}{r}$. On choisit une urne au hasard
(avec la même
probabilité pour chaque urne d'être choisie) et on effectue dans cette
urne $n$ tirages indépendants, avec remise. On note $N_{r}$, la
variable aléatoire
représentant le nombre de boules rouges ainsi tirées : pour tout entier
$k$
tel que $0\leq k\leq n$, on note $p_{r}\left( k\right) $ la probabilité
que $N_{r}$, prenne la valeur $k$.

\begin{noliste}{a)}
 \setlength{\itemsep}{2mm}
\item Trouver la loi de probabilité de $N_{r}$, et calculer l'espérance
de
cette variable aléatoire.

\item Montrer que : 
\[
\underset{r\rightarrow + \infty }{\lim }p_{r}\left( k\right) =
\dbinom{n}{k}\dint\limits_{0}{1}x^{k}\left( 1-x\right) ^{n-k}dx
\]
\end{noliste}

\item Soit $I_{k} = \dint\limits_{0}{1}x^{k}\left( 1-x\right)
^{n-k}dx$. A
l'aide d'intégrations par parties, calculer la valeur de $I_{k}$. En
déduire
que les nombres $\underset{r\rightarrow + \infty }{\lim }p_{r}\left(
k\right) 
$ ne dépendent pas de $k$.
\end{noliste}

\label{fin}

\end{document}

