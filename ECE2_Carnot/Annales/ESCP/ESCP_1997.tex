\documentclass[11pt]{article}%
\usepackage{geometry}%
\geometry{a4paper,
 lmargin = 2cm,rmargin = 2cm,tmargin = 2.5cm,bmargin = 2.5cm}

\input{../../macros.tex}

\pagestyle{fancy} %
\lhead{ECE2 \hfill Mathématiques\\
} %
\chead{\hrule} %
\rhead{} %
\lfoot{} %
\cfoot{} %
\rfoot{\thepage} %

\renewcommand{\headrulewidth}{0pt}% : Trace un trait de séparation
 % de largeur 0,4 point. Mettre 0pt
 % pour supprimer le trait.

\renewcommand{\footrulewidth}{0.4pt}% : Trace un trait de séparation
 % de largeur 0,4 point. Mettre 0pt
 % pour supprimer le trait.

\setlength{\headheight}{14pt}

\title{\bf \vspace{-2cm} ESCP 1997} %
\author{} %
\date{} %
\begin{document}

\maketitle %
\vspace{-1.4cm}\hrule %
\thispagestyle{fancy}

\vspace*{.2cm}


% DEBUT DU DOC À MODIFIER : tout virer jusqu'au début de l'exo

%Définition et changement de valeurs de
compteurs%newcounter{cpt1}{section} compteur cpt1 remis à 0 à chaque
aumentation par stepcounter du compteur section%setcounter{cpt1}{3} on
met le compteur à 3%addtocounter{cpt1}{5} on ajoute 5 au compteur%
stepcounter{cpt1} on ajoute 1% ifthenelse{test}{alors}{sinon} (page
206) pour subordonner à une condition % whiledo{test}{commande} pour
faire une boucle (page 206 aussi) % value{cpt1} pour noter dans le
document la valeur de cpt1 
%Définition définitive d'opérateurs
mathématiques\newcommand{\ch}{\operatorname{ch}} 
\newcommand{\sh}{\operatorname{sh}}
\renewcommand{\tanh}{\operatorname{th}}
\renewcommand{\sinh}{\operatorname{sh}}
\renewcommand{\cosh}{\operatorname{ch}}
\newcommand{\argsh}{\operatorname{argsh}}
\newcommand{\argch}{\operatorname{argch}}
\newcommand{\argth}{\operatorname{argth}}
\newcommand{\ker}{\operatorname{Ker}}
\renewcommand{\im}{\operatorname{Im}}
\newcommand{\rg}{\operatorname{rg}}
\newcommand{\Id}{\operatorname{Id}}
\newcommand{\id}{\operatorname{id}}
\renewcommand{\leq}{\leq}
\renewcommand{\geq}{\geq }

%Définition de nouvelles couleurs : rgb(trois paramètres red green blue
entre 0 et 1); cmyk (quatre cyan magenta yellow black) entre 0 et 1;
gray (entre 0 et 1) et black, white, red, green, blue, cyan, magenta,
yellow% definecolor{0gris}{gray}{0.8} 
% Nouvelle commande pour encadrer le titre car shabox ne veut que d'une
seule ligne; ATTENTION A LA TAILLE; petite différence avec shadowbox ou
doublebox, voire fcolorbox ou colorbox (au lieu de shabox; laisser le
parbox tranquille sauf pour la taille de la boîte
\newcommand{\Tbox}[1]{\begin{center} \shabox{\parbox{0.6
\linewidth}{#1}} \end{center}} %[1] pour 1 paramètre ; #1 pour ce que
fait le 1er paramètre; entre accolades ce que fait la commande
%Mise en page en mode fancy : en-têtes et pieds de pages puis
définition des en-têtes et pieds de pages\pagestyle{fancy}
\lhead{ECE 2 - Mathématiques \\
Quentin Dunstetter - ENC-Bessières 2011$\backslash$2012}
\chead{}
\rhead{ESCP 2005}
\rfoot[ \ \thepage]{\thepage}
\cfoot{}
\lfoot{}
\thispagestyle{fancy} %Mise en page de la 1ère page en mode fancy
%Trait en bas et en haut de la page (entre en-tête et texte et texte et
pied de page)\renewcommand{\footrulewidth}{0.4pt}
\renewcommand{\headrulewidth}{0.4pt}


%DEBUT DU DOCUMENT\vspace*{3cm}

\begin{center}
{\LARG\E\textbf{BANQUE COMMUNE D'ÉPREUVES}}



{\large \textsc{CONCOURS D ADMISSION DE 2005}}



{\large \textbf{Concepteur : ESCP}}



\rule{2.39cm}{0.05cm}



{\Large \textbf{OPTION ÉCONOMIQUE}}



{\Large \textbf{MATHÉMATIQUES }}



{\Large Lundi 9 mai, de 14h à 18h}



\rule{2.39cm}{0.05cm}
\end{center}

\textit{La présentation, la lisibilité, l'orthographe, la qualité
de la rédaction, la clarté et la précision des raisonnements
entreront pour une part importante dans l'appréciation des copies.}

\textit{Les candidats sont invités à \textbf{encadrer} dans la mesure
du possible les résultats de leurs calculs.}

\textit{Ils ne doivent faire usage d'aucun document. L'utilisation de
toute
calculatrice et de tout matériel électronique est interdite. Seule
l'utilisation d'une règle graduée est autorisée.}

\textit{Si au cours de l'épreuve, un candidat repère ce qui lui semble
être une erreur d'énoncé, il la signalera sur sa copie et
poursuivra sa composition en expliquant les raisons des initiatives
qu'il sera
amené à prendre.}

\vspace*{3cm}

\section*{EXERCICE I}

On note $\mathfrak{M}_{3}\left( \R\right) $ l'ensemble des matrices
carrées d'ordre 3 à coefficients réels, $u$ l'application identique de
l'espace vectoriel $\R^{3}$dans lui même et $I$ la matrice identité
de $\mathfrak{M}_{3}\left( \R\right) $ représentant $u$ dans la base
canonique de $\R^{3}$.\\
On considère la matrice 
\[
A = \left( 
\begin{array}{ccc}
2 & 0 & -1 \\
0 & 1 & 0 \\
-1 & 0 & 2
\end{array}
\right) 
\]
et on note $f$ l'endomorphisme de $\R^{3}$ représenté par $A$ dans
la base canonique de $\R^{3}$

\begin{noliste}{1.}
 \setlength{\itemsep}{4mm}
\item Déterminer les valeurs propres de $f$. L'endomorphisme $f$ est-il
diagonalisable ?

\item Etant donné un couple $\left( a,b\right) $ de réels, déterminer
les
valeurs propres de l'endomorphisme $af + bu$ de $\R^{3}.$ Pour quelles
valeurs de $\left( a,b\right) $ cet endomorphisme est-il diagonalisable
?

\item Quelles relations le couple $\left( a,b\right) $ doit-il vérifier
pour
que l'endomorphisme $af + bu$ soit inversible ? Montrer que l'inverse
de $af + bu$, quand il existe, est de la forme $\lambda f + \mu u$ où
$\lambda $ et $\mu $
sont des réels dont on donnera l'expression en fonction de $a$ et $b$.

On considère maintenant l'ensemble $\mathcal{E}$ des matrices $T$ de
$\mathfrak{M}_{3}\left( \R\right) $ qui commutent avec $A$ c'est à
dire qui vérifient $AT = TA$.

\item Montrer que $\mathcal{E}$ est un sous-espace vectoriel de
$\mathfrak{M}_{3}\left( \R\right) $

\item Pour une matrice $T$ de la forme $T = $ $\left( 
\begin{array}{ccc}
a & b & c \\
d & e & f \\
g & h & i
\end{array}
\right),$ calculer $AT-TA$. En déduire une base de $\mathcal{E}$ et sa
dimension.

\item Soit $\Phi $ l'application de $\mathfrak{M}_{3}\left( \R\right) $
dans lui même qui fait correspondre à la matrice $T$ la matrice
$AT-TA$. Montrer que $\Phi $ est un endomorphisme de
$\mathfrak{M}_{3}\left( 
\R\right) $. Donner une base du noyau et une base de l'image de $\Phi $
\end{noliste}

\section*{EXERCICE II}

Dans tout l'exercice $\lambda $ désignera un réel strictement positif
et $\
f_{\lambda }$ sera la fonction définie sur $\R$ par : $f_{\lambda
} = e^{-\lambda x^{2}}$, pour tout réel $x$.\\
Le but de l'exercice est l'étude de la suite $\left( u_{n}\right)_{n\in

\N}$ définie par $u_{0} = 0$ et $u_{n + 1} = $ $f_{\lambda }\left(
u_{n}\right) $ pour tout $n\in \N$

\begin{noliste}{1.}
 \setlength{\itemsep}{4mm}
\item 

\begin{noliste}{a)}
 \setlength{\itemsep}{2mm}
\item Montrer que l'équation $f_{\lambda }\left( x\right) $ $ = x$,
d'inconnue 
$x$, admet une seule racine dans $\R$ et que cette racine appartient 
à $]0,1[$. On note $\ell_{\lambda }$ cette racine.

\item Montrer que, si $\lambda >\dfrac{e}{2}$, alors $\ell_{\lambda
}>\dfrac{1}{\sqrt{2\lambda }}$
\end{noliste}

\item On suppose dans cette question que $\lambda $ $\leq
\dfrac{1}{2}$.

\begin{noliste}{a)}
 \setlength{\itemsep}{2mm}
\item Montrer que $\max\limits_{x\in \left[ 0,1\right] }\left|
f_{\lambda }{\prime }\left( x\right) \right| <1$

\item Montrer que la suite $\left( u_{n}\right)_{n\in \N}$ admet
pour limite $\ell_{\lambda }$.
\end{noliste}

On revient au cas général, c'est à dire $\lambda \in \R_{+}{\times
} $.

\item On pose $g_{\lambda } = f_{\lambda }\circ f_{\lambda }$.

\begin{noliste}{a)}
 \setlength{\itemsep}{2mm}
\item Montrer que $g_{\lambda }$ est strictement croissante sur $\left[
0, + \infty \right[ $.

\item Montrer que les deux suites $\left( u_{2n}\right)_{n\in \N}$
et $\left( u_{2n + 1}\right)_{n\in \N}$ sont monotones et
convergentes.
\end{noliste}

\item 

\begin{noliste}{a)}
 \setlength{\itemsep}{2mm}
\item Montrer que les racines éventuelles de l'équation $g_{\lambda
}\left(
x\right) = x$ appartiennent à $]0,1[$. Vérifier que $\ell_{\lambda }$
est
une racine de cette dernière équation.

\item Soit $x\in \left] 0,1\right[ $. Montrer que $g_{\lambda }\left(
x\right) = x$ si et seulement si $\ln \left( -\ln \left( x\right)
\right)
 + 2\lambda x^{2}-\ln \left( \lambda \right) = 0$

\item Pour tout $x\in \left] 0,1\right[ $, on pose $h_{\lambda }\left(
x\right) = \ln \left( -\ln \left( x\right) \right) + 2\lambda x^{2}-\ln
\left(
\lambda \right) $

Montrer que la fonction $h_{\lambda }$ est dérivable sur $]0,1[$ et que
$h_{\lambda }{\prime }\left( x\right) $ a le signe opposé de celui de
$1 + 4\lambda x^{2}\ln \left( x\right) $

\item Pour tout $x\in \left] 0,1\right[ $, on pose $k_{\lambda }\left(
x\right) = 1 + 4\lambda x^{2}\ln \left( x\right) $. Dresser le tableau
de
variation de la fonction $k_{\lambda }$.

\item On se place désormais dans le cas où $\lambda >\dfrac{e}{2}$

\begin{noliste}{$\sbullet$}
\item Montrer que, dans ce cas, $k_{\lambda }\left( \ell_{\lambda
}\right)
<0$

\item Dresser le tableau de variation de la fonction $h_{\lambda }$ et
en déduire que l'équation $h_{\lambda }\left( x\right) = x$ admet trois
racines $\mu_{\lambda },\;\ell_{\lambda },\,\nu_{\lambda }$ vérifiant
$0<\mu
_{\lambda }<\ell_{\lambda }<\nu_{\lambda }<1$

\item Montrer que les suites $\left( u_{2n}\right)_{n\in \N}$ et
$\left( u_{2n + 1}\right)_{n\in \N}$ convergent vers $\mu_{\lambda }$
et $\,\nu_{\lambda }$ respectivement.
\end{noliste}
\end{noliste}
\end{noliste}

\section*{EXERCICE III}

On note $\N$ l'ensemble des entiers naturels. Soit $a$ et $b$ deux
réels tels que $0<a<1$ et $0<b<1$.\\
On effectue une suite d'expériences aléatoires consistant à jeter
simultanément deux pièces de monnaie notées $A$ et $B$. On suppose que
ces expériences sont indépendantes et qu'à chaque expérience les
résultats des deux
pièces sont indépendants. On suppose que, lors d'une expérience, la
probabilité que la pièce $A$ donne pile est $a$, et que la probabilité
que
la pièce $B$ donne pile est $b$.

\begin{noliste}{1.}
 \setlength{\itemsep}{4mm}
\item 

\begin{noliste}{a)}
 \setlength{\itemsep}{2mm}
\item Pour tout entier naturel $n$, calculer la probabilité $\mu_{n}$,
que
la pièce $A$ donne $n$ fois pile et, à la $(n + 1)^{i\grave{e}me}$
expérience,
face pour la première fois. Calculer de même la probabilité $\nu_{n}$
que
la pièce $B$ donne $n$ piles et, à la $\left( n + i\right)
^{i\grave{e}me}$ expérience, face pour la première fois.

\item Montrer que les suites $\left( \mu_{n}\right)_{n\in \N}$ et
$\left( \nu_{n}\right)_{n\in \N}$ définissent des lois de probabilité
sur $\N$. Ces lois seront notées dorénavant respectivement $\mu $
et $\nu $.
\end{noliste}

\item On considère deux variables aléatoires $X$ et $Y$, définies sur
un même espace probabilisé \\
$\left( \Omega,\mathcal{A},\mathbf{P}\right) $, à valeurs dans $\N$,
indépendantes et dont les lois de probabilité sont respective\-ment
$\mu $
et $\nu $. (La variable aléatoire $X$ représente le nombre
d'expériences
qu'il faut réaliser avant que la pièce $A$ donne face pour la première
fois
et la variable aléatoire $Y$ représente le nombre d'expériences qu'il
faut réaliser avant que la pièce $B$ donne face pour la première fois).

\begin{noliste}{a)}
 \setlength{\itemsep}{2mm}
\item Calculer l'espérance $\E(X)$ et la variance $\V(X)$.

\item Trouver, pour tout entier naturel $k$, la valeur de
$P\left(\Ev{X>k}\right)$.

\item On s'intéresse au nombre d'expériences qu'il faut réaliser avant
que
l'une au moins des pièces donne face pour la première fois. Pour cela
on
note $M$ la variable aléatoire définie par $M = \min \left(
X,Y\right).$

Calculer, pour tout entier naturel $k$, la probabilité
$\mathbf{P}\left(
M\geq k\right) $. En déduire la loi de probabilité de $M$.

\item Déterminer la probabilité que la pièce $B$ ne donne pas face
avant la
pièce $A$, c'est-à-dire $\mathbf{P}(Y\geq X)$.
\end{noliste}

\item On note $U = X + Y.$

\begin{noliste}{a)}
 \setlength{\itemsep}{2mm}
\item Déterminer la loi de probabilité de $U$. (On distinguera les cas
$a = b$
et $a\ne b$).

\item Calculer, pour tout couple $(j,k)$ d'entiers naturels, \\
les probabilités conditionnelles $\mathbf{P}\left( Y = k/U = j\right) $
\end{noliste}

\item On suppose désormais que $a = b$. On note $V = Y-X$.

\begin{noliste}{a)}
 \setlength{\itemsep}{2mm}
\item Calculer, pour tout entier naturel $k$ et tout entier relatif
$r$, la
probabilité de l'évènement $\left( M = k\text{ et }V = r\right) $. (On
distinguera le cas $r\geq 0$ et le cas $r<0$ ).

\item Trouver la loi de probabilité de $V$. Les variables aléatoires
$M$ et $V$ sont-elles indépendantes ?
\end{noliste}
\end{noliste}

\label{fin}

\end{document}

