\documentclass[11pt]{article}%
\usepackage{geometry}%
\geometry{a4paper,
 lmargin = 2cm,rmargin = 2cm,tmargin = 2.5cm,bmargin = 2.5cm}

\input{../../macros.tex}

\pagestyle{fancy} %
\lhead{ECE2 \hfill Mathématiques\\
} %
\chead{\hrule} %
\rhead{} %
\lfoot{} %
\cfoot{} %
\rfoot{\thepage} %

\renewcommand{\headrulewidth}{0pt}% : Trace un trait de séparation
 % de largeur 0,4 point. Mettre 0pt
 % pour supprimer le trait.

\renewcommand{\footrulewidth}{0.4pt}% : Trace un trait de séparation
 % de largeur 0,4 point. Mettre 0pt
 % pour supprimer le trait.

\setlength{\headheight}{14pt}

\title{\bf \vspace{-2cm} ESCP 1993} %
\author{} %
\date{} %
\begin{document}

\maketitle %
\vspace{-1.4cm}\hrule %
\thispagestyle{fancy}

\vspace*{.2cm}


% DEBUT DU DOC À MODIFIER : tout virer jusqu'au début de l'exo

%Définition et changement de valeurs de
compteurs%newcounter{cpt1}{section} compteur cpt1 remis à 0 à chaque
aumentation par stepcounter du compteur section%setcounter{cpt1}{3} on
met le compteur à 3%addtocounter{cpt1}{5} on ajoute 5 au compteur%
stepcounter{cpt1} on ajoute 1% ifthenelse{test}{alors}{sinon} (page
206) pour subordonner à une condition % whiledo{test}{commande} pour
faire une boucle (page 206 aussi) % value{cpt1} pour noter dans le
document la valeur de cpt1 
%Définition définitive d'opérateurs
mathématiques\newcommand{\ch}{\operatorname{ch}} 
\newcommand{\sh}{\operatorname{sh}}
\renewcommand{\tanh}{\operatorname{th}}
\renewcommand{\sinh}{\operatorname{sh}}
\renewcommand{\cosh}{\operatorname{ch}}
\newcommand{\argsh}{\operatorname{argsh}}
\newcommand{\argch}{\operatorname{argch}}
\newcommand{\argth}{\operatorname{argth}}
\newcommand{\ker}{\operatorname{Ker}}
\renewcommand{\im}{\operatorname{Im}}
\newcommand{\rg}{\operatorname{rg}}
\newcommand{\Id}{\operatorname{Id}}
\newcommand{\id}{\operatorname{id}}
\renewcommand{\leq}{\leq}
\renewcommand{\geq}{\geq }

%Définition de nouvelles couleurs : rgb(trois paramètres red green blue
entre 0 et 1); cmyk (quatre cyan magenta yellow black) entre 0 et 1;
gray (entre 0 et 1) et black, white, red, green, blue, cyan, magenta,
yellow% definecolor{0gris}{gray}{0.8} 
% Nouvelle commande pour encadrer le titre car shabox ne veut que d'une
seule ligne; ATTENTION A LA TAILLE; petite différence avec shadowbox ou
doublebox, voire fcolorbox ou colorbox (au lieu de shabox; laisser le
parbox tranquille sauf pour la taille de la boîte
\newcommand{\Tbox}[1]{\begin{center} \shabox{\parbox{0.6
\linewidth}{#1}} \end{center}} %[1] pour 1 paramètre ; #1 pour ce que
fait le 1er paramètre; entre accolades ce que fait la commande
%Mise en page en mode fancy : en-têtes et pieds de pages puis
définition des en-têtes et pieds de pages\pagestyle{fancy}
\lhead{ECE 2 - Mathématiques \\
Quentin Dunstetter - ENC-Bessières 2011$\backslash$2012}
\chead{}
\rhead{ESCP 1993}
\rfoot[ \ \thepage]{\thepage}
\cfoot{}
\lfoot{}
\thispagestyle{fancy} %Mise en page de la 1ère page en mode fancy
%Trait en bas et en haut de la page (entre en-tête et texte et texte et
pied de page)\renewcommand{\footrulewidth}{0.4pt}
\renewcommand{\headrulewidth}{0.4pt}


%DEBUT DU DOCUMENT\vspace*{3cm}

\begin{center}
{\LARG\E\textbf{BANQUE COMMUNE D'ÉPREUVES}}



{\large \textsc{CONCOURS D ADMISSION DE 1993}}



{\large \textbf{Concepteur : ESCP}}



\rule{2.39cm}{0.05cm}



{\Large \textbf{OPTION ÉCONOMIQUE}}



{\Large \textbf{MATHÉMATIQUES }}



{\Large Lundi 9 mai, de 14h à 18h}



\rule{2.39cm}{0.05cm}
\end{center}

\textit{La présentation, la lisibilité, l'orthographe, la qualité
de la rédaction, la clarté et la précision des raisonnements
entreront pour une part importante dans l'appréciation des copies.}

\textit{Les candidats sont invités à \textbf{encadrer} dans la mesure
du possible les résultats de leurs calculs.}

\textit{Ils ne doivent faire usage d'aucun document. L'utilisation de
toute
calculatrice et de tout matériel électronique est interdite. Seule
l'utilisation d'une règle graduée est autorisée.}

\textit{Si au cours de l'épreuve, un candidat repère ce qui lui semble
être une erreur d'énoncé, il la signalera sur sa copie et
poursuivra sa composition en expliquant les raisons des initiatives
qu'il sera
amené à prendre.}

\vspace*{3cm}

\section*{Exercice 1.}

Dans l'espace vectoriel $\R^{4}$ muni de sa base canonique $\mathcal{B}
= (e_{1},e_{2},e_{3},e_{4})$ on considère les quatre vecteurs : 
\[
f_{1} = (1,1,1,1),\;\;f_{2} = (1,1,-1,-1),\;\;f_{3} =
(1,-1,1,-1),\;\;f_{4} = (1,-1,-1,1).
\]

\begin{noliste}{1.}
 \setlength{\itemsep}{4mm}
\item Montrer que $(f_{1},f_{2},f_{3},f_{4})$ est une base de $\R^{4}$.
On
notera désormais cette base $\mathcal{C}$.

\item On considère l'endomorphisme $u$ de $\R^{4}$ défini par les
relations : 
\[
u\left( e_{1}\right) = f_{1},\quad u\left( e_{2}\right) = f_{2},\quad
u\left(
e_{3}\right) = f_{3},\quad u\left( e_{4}\right) = f_{4}.
\]

\begin{noliste}{a)}
 \setlength{\itemsep}{2mm}
\item Montrer que $u$ est un automorphisme de $\R^{4}$. Expliciter
sa matrice associée dans la base canonique $\mathcal{B}$.

\item Montrer que l'endomorphisme $u^{2}$ est un endomorphisme simple,
que
l'on déterminera.

\item Déterminer la matrice associée à l'endomorphisme réciproque de
$u$
dans la base $\mathcal{B}$.

\item Déterminer la matrice associée à l'endomorphisme $u$ dans la base
$\mathcal{C}$.
\end{noliste}

\item On considère les quatre suites $\left( x_{n}\right)_{n\geq 0}$,
$\left( y_{n}\right)_{n\geq 0}$, $\left( z_{n}\right)_{n\geq 0}$, 
$\left( t_{n}\right)_{n\geq 0}$ de nombres réels définies par les
valeurs initiales $x_{0}$, $y_{0}$, $z_{0}$, $t_{0}$ et, pour tout
entier
naturel $n$, par les relations de récurrence : 
\[
\left\{
\begin{array}{cl}
x_{n + 1} = \dfrac{1}{4}\left( x_{n} + y_{n} + z_{n} + t_{n}\right) \\
y_{n + 1} = \dfrac{1}{4}\left( x_{n} + y_{n}-z_{n}-t_{n}\right) \\
z_{n + 1} = \dfrac{1}{4}\left( x_{n}-y_{n} + z_{n}-t_{n}\right) \\
t_{n + 1} = \dfrac{1}{4}\left( x_{n}-y_{n}-z_{n} + t_{n}\right)
\end{array}
\right.
\]
A l'aide de l'endomorphisme $u$, donner une interprétation vectorielle
de
ces relations. \\
En déduire que les quatre suites considérées ont pour limite $0$.

\item Déterminer les valeurs propres de $u$ et une base de chacun des
sous-espaces propres associés. \\
L'endomorphisme $u$ est-il diagonalisable ?
\end{noliste}

\section*{Exercice 2.}

\begin{noliste}{1.}
 \setlength{\itemsep}{4mm}
\item Pour tout nombre entier naturel non nul $n$, on pose : 
\[
I_{n} = \dint{0}{\dfrac{\pi }{4}}\tan ^{n}x\,dx.
\]

\begin{noliste}{a)}
 \setlength{\itemsep}{2mm}
\item Sans calculer l'intégrale $I_{n}$, montrer que la suite
$(I_{n})_{n\geq 1}$ est décroissante, puis qu'elle est convergente.

\item Calculer la dérivée de la fonction $x\longmapsto \tan ^{n + 1}x$.

\\
En déduire, pour tout nombre entier naturel non nul $n$, une expression
de $I_{n} + I_{n + 2}$.

\item Montrer que, pour tout nombre entier naturel non nul $n$ : 
\[
\frac{1}{2(n + 1)}\leq I_{n}\leq \frac{1}{n + 1}.
\]
En déduire la limite de la suite $(I_{n})_{n\geq 1}$.
\end{noliste}

\item Soit $t$ un nombre réel appartenant à l'intervalle $\left]
0,\dfrac{\pi }{4}\right[ $. \\
Pour tout nombre entier naturel non nul $n$, on pose : 
\[
I_{n}(t) = \dint{0}{t}\tan ^{n}x\,dx\;\text{ et }\;S_{n}(t) = I_{1}(t)
+ I_{2}(t) + \cdots + I_{n}(t).
\]

\begin{noliste}{a)}
 \setlength{\itemsep}{2mm}
\item Pour tout nombre entier naturel non nul $n$ et tout élément $x$
de $[0,t]$, vérifier que :

\begin{nonoliste}{(i)}
\item $\left| \tan x + \tan ^{2}x + \cdots + \tan ^{n}x-\dfrac{\tan
x}{1-\tan
x}\right| = \dfrac{\tan ^{n + 1}x}{1-\tan x}$.

\item $0\leq \dfrac{\tan ^{n}x}{1-\tan x}\leq \dfrac{\tan ^{n}t}{1-\tan
t}$.
\end{nonoliste}

\item Montrer que pour tout nombre entier naturel non nul $n$ : 
\[
\left| S_{n}(t)-\dint{0}{t}\frac{\tan x}{1-\tan x}dx\right|
\leq t\frac{\tan ^{n + 1}t}{1-\tan t}.
\]
En déduire que la suite $(S_{n}(t))_{n\geq 1}$ a une limite quand $n$
tend vers l'infini.

\item On se propose de trouver une expression de cette limite.

\begin{nonoliste}{(i)}
\item Soit $f$ la fonction définie sur l'intervalle $\left[
0,\dfrac{\pi }{4}\right[ $ par la relation : 
\[
f(x) = \ln \left( \cos x-\sin x\right).
\]
Montrer que $f$ est dérivable et calculer sa dérivée.

\item On pose : 
\[
J(t) = \dint{0}{t}\frac{\sin x}{\cos x-\sin x}dx\;\text{ et }\;K(t) =
\dint{0}{t}\frac{\cos x}{\cos x-\sin x}dx.
\]
Calculer $J(t)$ et $K(t)$ à partir de $K(t)-J(t)$ et $K(t) + J(t)$. En
déduire
la limite de $(S_{n}(t))_{n\geq 1}$ quand $n$ tend vers l'infini.
\end{nonoliste}
\end{noliste}
\end{noliste}

\section*{Exercice 3.}

Soit $f$ la fonction numérique définie sur $\R$ par les relations : 
\[
f(x) = \left\{
\begin{array}{cl}
6x(1-x) & \text{si $0\leq x\leq 1$ ;} \\
0 & \text{sinon.}
\end{array}
\right.
\]

\begin{noliste}{1.}
 \setlength{\itemsep}{4mm}
\item Montrer que $f$ est une densité de probabilité.

\item Soit $X$ une variable aléatoire à valeurs dans $[0,1]$, de
densité de
probabilité $f$.

\begin{noliste}{a)}
 \setlength{\itemsep}{2mm}
\item Déterminer la fonction de répartition $F$ de $X$.

\item Calculer l'espérance et la variance de $X$.
\end{noliste}

\item 

\begin{noliste}{a)}
 \setlength{\itemsep}{2mm}
\item Montrer qu'à tout élément $y$ de l'intervalle $[0,1]$ on peut
faire
correspondre un élément $u$ et un seul de $[0,1]$ tel que : 
\[
\dint{0}{u}f(x)\,dx = y.
\]
On définit alors l'application $\varphi $ sur $[0,1]$ en posant
$\varphi
(y) = u$.

\item Montrer que $\varphi $ est continue sur $[0,1]$ et dérivable sur
$]0,1[ $. \\
Trouver les valeurs de $y$ dans $]0,1[$ telles que $\varphi ^{\prime
}(y) = \dfrac{3}{4}$.

\item Déterminer la fonction de répartition de la variable aléatoire $Z
= 3X^{2}-2X^{3}$. \\
Quelle est sa loi de probabilité ?
\end{noliste}

\item On considère des variables aléatoires $X_{1}$ et $X_{2}$ de même
loi
que $X$. On suppose que les variables $X_{1}$ et $X_{2}$ sont
indépendantes
et on pose $T = \sup (X_{1},X_{2})$. (Autrement dit, $T$ est la
variable aléatoire qui prend la plus grande des valeurs prises par
$X_{1}$ et $X_{2}$.)

Trouver la fonction de répartition de $T$. En déduire une densité de
probabilité de $T$ et l'espérance de $T$.
\end{noliste}

\label{fin}

\end{document}

