\documentclass[11pt]{article}%
\usepackage{geometry}%
\geometry{a4paper,
 lmargin = 2cm,rmargin = 2cm,tmargin = 2.5cm,bmargin = 2.5cm}

\input{../../macros.tex}

\pagestyle{fancy} %
\lhead{ECE2 \hfill Mathématiques\\
} %
\chead{\hrule} %
\rhead{} %
\lfoot{} %
\cfoot{} %
\rfoot{\thepage} %

\renewcommand{\headrulewidth}{0pt}% : Trace un trait de séparation
 % de largeur 0,4 point. Mettre 0pt
 % pour supprimer le trait.

\renewcommand{\footrulewidth}{0.4pt}% : Trace un trait de séparation
 % de largeur 0,4 point. Mettre 0pt
 % pour supprimer le trait.

\setlength{\headheight}{14pt}

\title{\bf \vspace{-2cm} ESCP 2002} %
\author{} %
\date{} %
\begin{document}

\maketitle %
\vspace{-1.4cm}\hrule %
\thispagestyle{fancy}

\vspace*{.2cm}


% DEBUT DU DOC À MODIFIER : tout virer jusqu'au début de l'exo

%Définition et changement de valeurs de
compteurs%newcounter{cpt1}{section} compteur cpt1 remis à 0 à chaque
aumentation par stepcounter du compteur section%setcounter{cpt1}{3} on
met le compteur à 3%addtocounter{cpt1}{5} on ajoute 5 au compteur%
stepcounter{cpt1} on ajoute 1% ifthenelse{test}{alors}{sinon} (page
206) pour subordonner à une condition % whiledo{test}{commande} pour
faire une boucle (page 206 aussi) % value{cpt1} pour noter dans le
document la valeur de cpt1 
%Définition définitive d'opérateurs
mathématiques\newcommand{\ch}{\operatorname{ch}} 
\newcommand{\sh}{\operatorname{sh}}
\renewcommand{\tanh}{\operatorname{th}}
\renewcommand{\sinh}{\operatorname{sh}}
\renewcommand{\cosh}{\operatorname{ch}}
\newcommand{\argsh}{\operatorname{argsh}}
\newcommand{\argch}{\operatorname{argch}}
\newcommand{\argth}{\operatorname{argth}}
\newcommand{\ker}{\operatorname{Ker}}
\renewcommand{\im}{\operatorname{Im}}
\newcommand{\rg}{\operatorname{rg}}
\newcommand{\Id}{\operatorname{Id}}
\newcommand{\id}{\operatorname{id}}
\renewcommand{\leq}{\leq}
\renewcommand{\geq}{\geq }

%Définition de nouvelles couleurs : rgb(trois paramètres red green blue
entre 0 et 1); cmyk (quatre cyan magenta yellow black) entre 0 et 1;
gray (entre 0 et 1) et black, white, red, green, blue, cyan, magenta,
yellow% definecolor{0gris}{gray}{0.8} 
% Nouvelle commande pour encadrer le titre car shabox ne veut que d'une
seule ligne; ATTENTION A LA TAILLE; petite différence avec shadowbox ou
doublebox, voire fcolorbox ou colorbox (au lieu de shabox; laisser le
parbox tranquille sauf pour la taille de la boîte
\newcommand{\Tbox}[1]{\begin{center} \shabox{\parbox{0.6
\linewidth}{#1}} \end{center}} %[1] pour 1 paramètre ; #1 pour ce que
fait le 1er paramètre; entre accolades ce que fait la commande
%Mise en page en mode fancy : en-têtes et pieds de pages puis
définition des en-têtes et pieds de pages\pagestyle{fancy}
\lhead{ECE 2 - Mathématiques \\
Quentin Dunstetter - ENC-Bessières 2011$\backslash$2012}
\chead{}
\rhead{ESCP 2002}
\rfoot[ \ \thepage]{\thepage}
\cfoot{}
\lfoot{}
\thispagestyle{fancy} %Mise en page de la 1ère page en mode fancy
%Trait en bas et en haut de la page (entre en-tête et texte et texte et
pied de page)\renewcommand{\footrulewidth}{0.4pt}
\renewcommand{\headrulewidth}{0.4pt}


%DEBUT DU DOCUMENT\vspace*{3cm}

\begin{center}
{\LARG\E\textbf{BANQUE COMMUNE D'ÉPREUVES}}



{\large \textsc{CONCOURS D ADMISSION DE 2002}}



{\large \textbf{Concepteur : ESCP}}



\rule{2.39cm}{0.05cm}



{\Large \textbf{OPTION ÉCONOMIQUE}}



{\Large \textbf{MATHÉMATIQUES }}



{\Large Lundi 9 mai, de 14h à 18h}



\rule{2.39cm}{0.05cm}
\end{center}

\textit{La présentation, la lisibilité, l'orthographe, la qualité
de la rédaction, la clarté et la précision des raisonnements
entreront pour une part importante dans l'appréciation des copies.}

\textit{Les candidats sont invités à \textbf{encadrer} dans la mesure
du possible les résultats de leurs calculs.}

\textit{Ils ne doivent faire usage d'aucun document. L'utilisation de
toute
calculatrice et de tout matériel électronique est interdite. Seule
l'utilisation d'une règle graduée est autorisée.}

\textit{Si au cours de l'épreuve, un candidat repère ce qui lui semble
être une erreur d'énoncé, il la signalera sur sa copie et
poursuivra sa composition en expliquant les raisons des initiatives
qu'il sera
amené à prendre.}

\vspace*{3cm}

\section*{EXERCICE}

On désigne par $I$, $O$, $J$ et $A$ les matrices carrées d'ordre $3$
suivantes : 
\[
I = 
\begin{smatrix}
1 & 0 & 0 \\
0 & 1 & 0 \\
0 & 0 & 1
\end{smatrix},\qquad O = 
\begin{smatrix}
0 & 0 & 0 \\
0 & 0 & 0 \\
0 & 0 & 0
\end{smatrix},\qquad J = 
\begin{smatrix}
1 & 1 & 1 \\
1 & 1 & 1 \\
1 & 1 & 1
\end{smatrix},\qquad A = 
\begin{smatrix}
-3 & 1 & 1 \\
1 & -3 & 1 \\
1 & 1 & -3
\end{smatrix}
\]

\begin{noliste}{1.}
 \setlength{\itemsep}{4mm}
\item 

\begin{noliste}{a)}
 \setlength{\itemsep}{2mm}
\item Exprimer la matrice $A$ en fonction des matrices $I$ et $J$, puis
la
matrice $J$ en fonction des matrices $A$ et $I$.

\item Exprimer $J^{2}$ en fonction de $J$ et en déduire que la matrice
$A$ vérifie l'égalité $A^{2} + 5A + 4I = O$.

\item Montrer que la matrice $A$ est inversible et exprimer son inverse
$A^{-1}$ en fonction des matrices $I$ et $J$.
\end{noliste}

\item 

\begin{noliste}{a)}
 \setlength{\itemsep}{2mm}
\item Soit $U$ la matrice-colonne $\begin{smatrix}
1 \\
1 \\
1
\end{smatrix}
$. Calculer le produit matriciel $J\,U$. \\
En déduire une valeur propre de la matrice $J$.

\item Montrer que $0$ est valeur propre de $J$ et donner une base du
sous-espace propre associé.

\item La matrice $J$ est-elle inversible ? La matrice $J$ est-elle
diagonalisable ?

\item Soit $X$ une matrice-colonne non nulle à trois éléments et
$\lambda $
un réel vérifiant $J\,X = \lambda X$. Montrer qu'il existe un réel $\mu
$ que
l'on donnera en fonction de $\lambda $ vérifiant $AX = \mu X$.

\item En déduire que $A$ est diagonalisable et que ses valeurs propres
sont $-1$ et $-4$.

\item Sans expliciter la matrice $A^{-1}$, calculer ses valeurs propres
et
montrer qu'elle est diagonalisable.
\end{noliste}

\item Soit $a$ un paramètre réel et $F_{a}$ la fonction définie sur
$\R^{2}$ par : 
\[
\begin{matrix}
F_{a}(x,y) = & \mathbf{(}x & y & a\mathbf{)} \\
 & & & \\
 & & & 
\end{matrix}\begin{smatrix}
-3 & 1 & 1 \\
1 & -3 & 1 \\
1 & 1 & -3
\end{smatrix}
\begin{smatrix}
x \\
y \\
a
\end{smatrix}
\]

\begin{noliste}{a)}
 \setlength{\itemsep}{2mm}
\item Vérifier que cette fonction est de classe $C^{1}$ sur $\R^{2}$
et calculer ses dérivées partielles d'ordre $1$ en tout point $(x,y)$
de $\R^{2}$.

\item Montrer qu'il existe un unique point $(x_{0},y_{0})$ de $\R^{2}
$, que l'on précisera, en lequel les dérivées partielles d'ordre $1$ de

$F_{a}$ sont nulles. Calculer $F_{a}(x_{0},y_{0})$.

\item Calculer, pour tout couple $(x,y)$ de $\R^{2}$, le nombre :\
$G_{a}(x,y) = F_{a}(x,y) + \dfrac{1}{3}(3x-y-a)^{2} + 2a^{2}$\ et
préciser son
signe.

\item En déduire que la fonction $F_{a}$ admet un unique extremum sur
$\R^{2}$. Préciser s'il s'agit d'un minimum ou d'un maximum et donner
sa valeur notée $M(a)$.

\item Montrer que la fonction $M$ qui, à tout réel $a$ associe le
nombre $M(a)$, admet un unique extremum que l'on précisera. Que peut-on
en conclure ?
\end{noliste}
\end{noliste}

\section*{Problème\ }

Pour toutes suites numériques $u = (u_{n})_{n\in \N}$ et $v =
(v_{n})_{n\in \N}$, on définit la suite $u\times v = w$ par :
\[
\forall n\in \N,\;w_{n} = \Sum{k = 0}{n}u_{k}\,v_{n-k}
\]
\vspace{-8mm} \vspace{0.3cm} \vspace{0.2cm}

\subsection*{Partie A : Exemples}

\begin{noliste}{1.}
 \setlength{\itemsep}{4mm}
\item \textbf{Premiers exemples }\\
Pour tout entier naturel $n$, calculer $w_{n}$ en fonction de $n$ dans
chacun des cas suivants :

\begin{noliste}{a)}
 \setlength{\itemsep}{2mm}
\item pour tout entier naturel $n$, $u_{n} = 2$ et $v_{n} = 3$.

\item pour tout entier naturel $n$, $u_{n} = 2^{n}$ et $v_{n} = 3^{n}$.

\item \label{Poisson} pour tout entier naturel $n$, $u_{n} =
\dfrac{2^{n}}{n!}$
et $v_{n} = \dfrac{3^{n}}{n!}$
\end{noliste}

\item \textbf{Programmation }\\
Dans cette question, les suites $u$ et $v$ sont définies par :\quad\
$\forall n\in \N,\;u_{n} = \ln (n + 1)\quad \text{et}\quad v_{n} =
\dfrac{1}{n + 1}$

Écrire un programme en -\Scilab{} qui demande à l'utilisateur une
valeur
de l'entier naturel $n$, qui calcule et affiche les valeurs \
$w_{0},\;w_{1},\;\ldots,\;w_{n}$.

\item \label{q3}\textbf{Un résultat de convergence }\\
Dans cette question, la suite $u$ est définie par : \ $\forall n\in
\N,\;u_{n} = \left( \dfrac{1}{2}\right) ^{n}$ et $v$ est une suite de
réels
positifs, décroissante à partir du rang $1$ et de limite nulle.

\begin{noliste}{a)}
 \setlength{\itemsep}{2mm}
\item Établir, pour tout couple d'entiers naturels $(n,m)$ vérifiant
$n<m$,
l'inégalité :$\qquad \Sum{k = n + 1}{m}u_{k}\leq u_{n}$\.

\item Soit $n$ un entier strictement supérieur à $1$. Prouver les
inégalités : 
\[
w_{2n}\leq v_{0}\,u_{2n} + 2v_{n} + v_{1}\,u_{n}\quad \text{et}\quad
w_{2n + 1}\leq v_{0}\,u_{2n + 1} + 2v_{n + 1} + v_{1}\,u_{n}
\]

\item En déduire que les deux suites \ $(w_{2n})_{n\in \N}$
\ et \ $(w_{2n + 1})_{n\in \N}$ convergent vers $0$
ainsi que la suite $(w_{n})_{n\in \N}$.

\item Soit $u^{\prime }$ la suite définie par : \ $\forall n\in
\N,\;u_{n}{\prime } = \left( -\dfrac{1}{2}\right) ^{n}$. À l'aide de la
question précédente, montrer que la suite $u^{\prime }\times v$\ est
convergente et de limite nulle.
\end{noliste}
\end{noliste}

\subsection*{Partie B : Application à l'étude d'un ensemble de suites}

Dans cette partie, $A$ désigne l'ensemble des suites $a = (a_{n})_{n\in

\N}$ de réels positifs vérifiant : 
\[
\forall n\in \N^{\times },\quad a_{n + 1}\leq \dfrac{1}{2}(a_{n} +
a_{n-1})
\]

\begin{noliste}{1.}
 \setlength{\itemsep}{4mm}
\item Montrer que toute suite décroissante de réels positifs est
élément de $A$ et qu'une suite strictement croissante ne peut
appartenir à $A$.

\item Soit $z = (z_{n})_{n\in \N}$ une suite réelle vérifiant :
$\forall n\in \N^{\times },\;z_{n + 1} = \dfrac{1}{2}(z_{n} +
z_{n-1})$.

\begin{noliste}{a)}
 \setlength{\itemsep}{2mm}
\item Montrer qu'il existe deux constantes réelles $\alpha $ et $\beta
$
telles que l'on a : 
\[
\forall n\in \N,\quad z_{n} = \alpha + \beta \left(
-\dfrac{1}{2}\right) ^{n}
\]

\item En déduire qu'il existe des suites appartenant à $A$ et non
monotones.
\end{noliste}

\item Soit $a = (a_{n})_{n\in \N}$ un élément de $A$ et $b$ la suite
définie par : $\forall n\in \N,\;b_{n} = \left( -\dfrac{1}{2}\right)
^{n} $.

On définit alors la suite $c$ par : \ $c_{0} = a_{0}$ \ et \ $\forall
n\in 
\N^{\times },\;c_{n} = a_{n} + \dfrac{1}{2}a_{n-1}$. \ 

\begin{noliste}{a)}
 \setlength{\itemsep}{2mm}
\item Montrer que la suite $c$ est décroissante à partir du rang $1$ et
qu'elle converge vers un nombre $\ell $ que l'on ne cherchera pas à
calculer.

\item Pour tout entier naturel $n$, établir l'égalité :\quad\ $\Sum{k =
0}{n}\left( -\dfrac{1}{2}\right) ^{k}c_{n-k} = a_{n}$.

Que peut-on en déduire pour les suites \ $b\times c$ et $a$ ?

\item Soit $\varepsilon $ la suite définie par : $\forall n\in
\N,\;\varepsilon_{n} = c_{n}-\ell $ et $d$ la suite $b\times
\varepsilon $.

En utilisant le résultat de la question \textbf{3}. de la Partie
\textbf{1},
montrer que la suite $d$ converge vers $0$.

\item Pour tout entier naturel $n$, établir l'égalité : $d_{n} =
a_{n}-\dfrac{2}{3}\ell \left( 1-\left( -\dfrac{1}{2}\right) ^{n +
1}\right) $.

En déduire que la suite $a$ converge et préciser sa limite.
\end{noliste}
\end{noliste}

\subsection*{Partie C : Application aux variables aléatoires}

\begin{noliste}{1.}
 \setlength{\itemsep}{4mm}
\item \textbf{Résultats préliminaires }

On suppose que $X$ et $Y$ sont deux variables aléatoires indépendantes,
à
valeurs dans $\N$ et on désigne par $S$ leur somme.

\begin{noliste}{a)}
 \setlength{\itemsep}{2mm}
\item Pour tout entier naturel $n$, on pose : $u_{n} = P\left(\Ev{X =
n]}\right)$\ et \ $v_{n} = P\left(\Ev{Y = n]}\right)$. \\
Montrer que l'on a : \ $\forall n\in \N,\;P\left(\Ev{S = n]}\right) =
w_{n}$, \ ($w$ étant la suite définie à partir des suites $u$ et $v$ en
tête du problème).

\item Retrouver alors le résultat de la question \textbf{1}.c) de la
Partie 
\textbf{1} par un choix adéquat des lois de $X$ et de $Y$.

\item Pour toute variable aléatoire $Z$ à valeurs dans $\N$, on note
$2^{-Z}$\ la variable aléatoire prenant, pour tout
entier naturel $n$, la valeur $2^{-n}$\ si et seulement
si l'évènement $\lbrack Z = n]$\ est réalisé. Montrer que
la variable aléatoire $2^{-Z}$ admet une espérance donnée par : 
\[
\E\left( 2^{-Z}\right) = \Sum{n = 0}{\infty }P\left(\Ev{Z =
n]}\right)\left(\Ev{ \dfrac{1}{2}}\right) ^{n}
\]
On note $r(Z)$ cette espérance.

\item Que peut-on dire des variables aléatoires $2^{-X}$ et $2^{-Y}$ ? 
\\
En déduire l'égalité : \ $r(S) = r(X)\,r(Y)$.

\item On suppose que $(X_{n})_{n\in \N^{\times }}$ est une suite de
variables aléatoires indépendantes, à valeurs dans $\N$ et de même
loi. Pour tout entier naturel non nul $q$, on désigne par $S_{q}$ la
variable aléatoire définie par : \quad\ $S_{q} = \Sum{i = 1}{q}X_{i}$\.
Établir l'égalité : \ $r(S_{q}) = \left( r(X_{1})\right) ^{q}$.
\end{noliste}

\item \textbf{Une formule sommatoire }

\begin{noliste}{a)}
 \setlength{\itemsep}{2mm}
\item \ Montrer que les égalités :$\qquad \forall n\in
\N,\;P\left(\Ev{Z = n]}\right) = \left( \dfrac{1}{2}\right) ^{n + 1}$
définissent la loi de
probabilité d'une variable aléatoire $Z$ à valeurs dans $\N$.
Calculer alors le nombre $r(Z)$.

\item On suppose que $(X_{n})_{n\in \N^{\times }}$ est une suite de
variables aléatoires indépendantes, à valeurs dans $\N$, de même loi
que $Z$ et, pour tout entier naturel non nul $q$, on désigne encore par
$S_{q}$ la variable : 
\[
S_{q} = \Sum{i = 1}{q}X_{i}
\]
En admettant, pour tout entier naturel non nul $q$, l'égalité \ $\Sum{k
= 0}{n}C_{k + q}{q} = C_{n + q + 1}{q + 1}$, montrer par récurrence
que la loi de $S_{q}$ est donnée par : 
\[
\forall n\in \N,\;P\left(\Ev{S_{q} = n]}\right) = C_{n +
q-1}{q-1}\left( \dfrac{1}{2}\right) ^{n + q}
\]

\item \label{Q2d} Pour tout entier naturel non nul $q$, calculer le
nombre $r(S_{q})$ et en déduire la relation : 
\[
\Sum{n = 0}{\infty }C_{n + q-1}{q-1}\left( \dfrac{1}{4}\right) ^{n} =
\left( 
\dfrac{4}{3}\right) ^{q}
\]
\end{noliste}

\item \textbf{Un exemple concret }

On admet, dans cette question, que la variable aléatoire $Z$ définie à
la
question \textbf{2}.a) représente le nombre de petits devant naître en
2003
d'un couple de kangourous. Chaque petit kangourou a la même probabilité
$\dfrac{1}{2}$ d'être mâle ou femelle, indépendamment des autres. On
note $F$
la variable aléatoire égale au nombre de femelles devant naître en
$2003$.

\begin{noliste}{a)}
 \setlength{\itemsep}{2mm}
\item Préciser, pour tout entier naturel $n,$ la loi conditionnelle de
$F$
sachant $[Z = n]$.

\item À l'aide de la formule obtenue en \ref{Q2d}, montrer que la loi
de $F$
est donnée par : 
\[
\forall n\in \N,\;P\left(\Ev{F = n]}\right) = \dfrac{2}{3}\left(
\dfrac{1}{3}\right)
^{n}
\]

\item Justifier l'existence des espérances $\E(Z)$\ et $\E(F)$\ des
variables
aléatoires $Z$ et $F$, puis vérifier l'égalité : \ $\E(Z) = 2\E(F).$\ 
\end{noliste}
\end{noliste}

\label{fin}

\end{document}

