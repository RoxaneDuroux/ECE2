\documentclass[11pt]{article}%
\usepackage{geometry}%
\geometry{a4paper,
 lmargin = 2cm,rmargin = 2cm,tmargin = 2.5cm,bmargin = 2.5cm}

\input{../../macros.tex}

\pagestyle{fancy} %
\lhead{ECE2 \hfill Mathématiques\\
} %
\chead{\hrule} %
\rhead{} %
\lfoot{} %
\cfoot{} %
\rfoot{\thepage} %

\renewcommand{\headrulewidth}{0pt}% : Trace un trait de séparation
 % de largeur 0,4 point. Mettre 0pt
 % pour supprimer le trait.

\renewcommand{\footrulewidth}{0.4pt}% : Trace un trait de séparation
 % de largeur 0,4 point. Mettre 0pt
 % pour supprimer le trait.

\setlength{\headheight}{14pt}

\title{\bf \vspace{-2cm} ESCP 1999} %
\author{} %
\date{} %
\begin{document}

\maketitle %
\vspace{-1.4cm}\hrule %
\thispagestyle{fancy}

\vspace*{.2cm}


% DEBUT DU DOC À MODIFIER : tout virer jusqu'au début de l'exo

%Définition et changement de valeurs de
compteurs%newcounter{cpt1}{section} compteur cpt1 remis à 0 à chaque
aumentation par stepcounter du compteur section%setcounter{cpt1}{3} on
met le compteur à 3%addtocounter{cpt1}{5} on ajoute 5 au compteur%
stepcounter{cpt1} on ajoute 1% ifthenelse{test}{alors}{sinon} (page
206) pour subordonner à une condition % whiledo{test}{commande} pour
faire une boucle (page 206 aussi) % value{cpt1} pour noter dans le
document la valeur de cpt1 
%Définition définitive d'opérateurs
mathématiques\newcommand{\ch}{\operatorname{ch}} 
\newcommand{\sh}{\operatorname{sh}}
\renewcommand{\tanh}{\operatorname{th}}
\renewcommand{\sinh}{\operatorname{sh}}
\renewcommand{\cosh}{\operatorname{ch}}
\newcommand{\argsh}{\operatorname{argsh}}
\newcommand{\argch}{\operatorname{argch}}
\newcommand{\argth}{\operatorname{argth}}
\newcommand{\ker}{\operatorname{Ker}}
\renewcommand{\im}{\operatorname{Im}}
\newcommand{\rg}{\operatorname{rg}}
\newcommand{\Id}{\operatorname{Id}}
\newcommand{\id}{\operatorname{id}}
\renewcommand{\leq}{\leq}
\renewcommand{\geq}{\geq }

%Définition de nouvelles couleurs : rgb(trois paramètres red green blue
entre 0 et 1); cmyk (quatre cyan magenta yellow black) entre 0 et 1;
gray (entre 0 et 1) et black, white, red, green, blue, cyan, magenta,
yellow% definecolor{0gris}{gray}{0.8} 
% Nouvelle commande pour encadrer le titre car shabox ne veut que d'une
seule ligne; ATTENTION A LA TAILLE; petite différence avec shadowbox ou
doublebox, voire fcolorbox ou colorbox (au lieu de shabox; laisser le
parbox tranquille sauf pour la taille de la boîte
\newcommand{\Tbox}[1]{\begin{center} \shabox{\parbox{0.6
\linewidth}{#1}} \end{center}} %[1] pour 1 paramètre ; #1 pour ce que
fait le 1er paramètre; entre accolades ce que fait la commande
%Mise en page en mode fancy : en-têtes et pieds de pages puis
définition des en-têtes et pieds de pages\pagestyle{fancy}
\lhead{ECE 2 - Mathématiques \\
Quentin Dunstetter - ENC-Bessières 2011$\backslash$2012}
\chead{}
\rhead{ESCP 1999}
\rfoot[ \ \thepage]{\thepage}
\cfoot{}
\lfoot{}
\thispagestyle{fancy} %Mise en page de la 1ère page en mode fancy
%Trait en bas et en haut de la page (entre en-tête et texte et texte et
pied de page)\renewcommand{\footrulewidth}{0.4pt}
\renewcommand{\headrulewidth}{0.4pt}


%DEBUT DU DOCUMENT\vspace*{3cm}

\begin{center}
{\LARG\E\textbf{BANQUE COMMUNE D'ÉPREUVES}}



{\large \textsc{CONCOURS D ADMISSION DE 1999}}



{\large \textbf{Concepteur : ESCP}}



\rule{2.39cm}{0.05cm}



{\Large \textbf{OPTION ÉCONOMIQUE}}



{\Large \textbf{MATHÉMATIQUES }}



{\Large Lundi 9 mai, de 14h à 18h}



\rule{2.39cm}{0.05cm}
\end{center}

\textit{La présentation, la lisibilité, l'orthographe, la qualité
de la rédaction, la clarté et la précision des raisonnements
entreront pour une part importante dans l'appréciation des copies.}

\textit{Les candidats sont invités à \textbf{encadrer} dans la mesure
du possible les résultats de leurs calculs.}

\textit{Ils ne doivent faire usage d'aucun document. L'utilisation de
toute
calculatrice et de tout matériel électronique est interdite. Seule
l'utilisation d'une règle graduée est autorisée.}

\textit{Si au cours de l'épreuve, un candidat repère ce qui lui semble
être une erreur d'énoncé, il la signalera sur sa copie et
poursuivra sa composition en expliquant les raisons des initiatives
qu'il sera
amené à prendre.}

\vspace*{3cm}

\section*{EXERCICE I}

Soit $\mathfrak{M}_{2}\left( \R\right) $ l'ensemble des matrices
carr"es d'ordre 2 muni de sa structure d'espace vectoriel et soit $J$
la matrice 
\[
J = \left( 
\begin{array}{cc}
0 & 1 \\
1 & 0
\end{array}
\right) 
\]
On considère l'application $S$ de $\mathfrak{M}_{2}\left( \R\right) $
dans lui-même qui associe à tout élément $M$ de $\mathfrak{M}_{2}\left(

\R\right) $ l'élément $S(M) = J\,M\,J$.

\begin{noliste}{1.}
 \setlength{\itemsep}{4mm}
\item 

\begin{noliste}{a)}
 \setlength{\itemsep}{2mm}
\item Montrer que l'application $S$ ainsi définie est un automorphisme
de
l'espace vectoriel $\mathfrak{M}_{2}\left( \R\right).$ Quel est
l'automorphisme réciproque de $S$ ?

\item Montrer que si $M$ et $N$ sont deux éléments quelconques de
$\mathfrak{M}_{2}\left( \R\right) $, on a $S(MN) = S(M)S(N)$
\end{noliste}

\item On considère les éléments 
\[
I = \left( 
\begin{array}{cc}
1 & 0 \\
0 & 1
\end{array}
\right) \;J = \left( 
\begin{array}{cc}
0 & 1 \\
1 & 0
\end{array}
\right) \;K = \left( 
\begin{array}{cc}
1 & 0 \\
0 & -1
\end{array}
\right) \;L = \left( 
\begin{array}{cc}
0 & 1 \\
-1 & 0
\end{array}
\right)
\]
Montrer que $(I,J,K,L)$ forme une base de l'espace vectoriel
$\mathfrak{M}_{2}\left( \R\right) $.

\item Montrer que $I,\,J,\,K,\,L$ sont des vecteurs propres de $S$.
Déterminer la matrice représentant l'auto\-morphisme $S$ dans la base
$\left(
I,J,K,L\right) $.

\item Soit $\mathcal{F}$ l'ensemble des éléments $M$ de
$\mathfrak{M}_{2}\left( \R\right) $ vérifiant $S(M) = M$ et soit
$\mathcal{G}$
l'ensemble des éléments $M$ de $\mathfrak{M}_{2}\left( \R\right) $(R)
Vérifiant $S(M) = -M$. Montrer que $\mathcal{F}$ et $\mathcal{G}$ sont
des
sous-espaces vectoriels de $\mathfrak{M}_{2}\left( \R\right) $ et
que tout élément $M$ de $\mathfrak{M}_{2}\left( \R\right) $ peut
s'écrire d'une manière et d'une seule sous la forme $M = M_{+} + M_{-}$
avec $M_{+}\in \mathcal{F}$ et $M_{-}\in \mathcal{G}$.

A titre d'exemple, déterminer les matrices $A_{+}$ et $A_{-}$ lorsque
$A = \left( 
\begin{array}{cc}
3 & -1 \\
1 & -2
\end{array}
\right) $.

\item 

\begin{noliste}{a)}
 \setlength{\itemsep}{2mm}
\item Montrer que le produit de deux matrices appartenant à
$\mathcal{F}$
appartient aussi à $\mathcal{F}$. Que peut-on dire du produit de deux
éléments de $\mathcal{G}$ ?

\item Plus précisément, pour deux matrices $M$ et $N$ de
$\mathfrak{M}_{2}\left( \R\right) $, exprimer $(MN)_{+}$ et $(MN)_{-}$
en
fonction de $M_{+}$, $M_{-}$, $N_{+}$ et $N_{-}$.
\end{noliste}
\end{noliste}

\section*{EXERCICE II}

Pour tout entier $k$ supérieur ou égal à 2, soit $f_{k}$ la fonction
définie
sur $\left] 0, + \infty \right[ $ par :

\[
f_{k}(x) = \dfrac{\ln ^{k}\left( x\right) }{x-1}\text{ si }x>0\text{ et
}x\neq
1\;\text{et }f_{k}\left( 1\right) = 0
\]

\begin{noliste}{1.}
 \setlength{\itemsep}{4mm}
\item Étude des fonctions $f_{k}$.

\begin{noliste}{a)}
 \setlength{\itemsep}{2mm}
\item Soit $k$ un entier supérieur ou égal à 2.

Justifier la dérivabilité de la fonction $f_{k}$ sur $\left] 0,1\right[
\ \cup \left] 1, + \infty \right[ $ et préciser la valeur de la dérivée
$f_{k}{\prime }(x)$, pour tout $x$ appartenant à $\left] 0,1\right[ \
\cup \left] 1, + \infty \right[ $.

Montrer que $f_{k}$ est dérivable en 1 et donner, selon les valeurs de
$k $,
la valeur de $f_{k}{\prime }(1)$

\item On considère les fonctions auxiliaires $\varphi_{k}$ définies,
pour
tout $x>0$, par 
\[
\varphi_{k}(x) = k(x-1)-xln(x).
\]
Étudier, pour tout entier $k$ supérieur ou égal à 2, les variations de
la
fonction $\varphi_{k}$. Montrer que l'équation $\varphi_{k}\left(
x\right)
 = 0$ admet une racine unique dans l'intervalle $\left] 1, + \infty
\right[ $.
Dans la suite, on notera $a_{k}$ cette racine.

\item En distinguant les cas $k = 2$, $k$ pair supérieur ou égal à 4,
$k$
impair supérieur ou égal à 3, donner le tableau de variation de la
fonction $f_{k}$ (on précisera les limites aux bornes).
\end{noliste}

\item Étude asymptotique de la suite $\left( a_{k}\right)_{k\geq 2}$.

\begin{noliste}{a)}
 \setlength{\itemsep}{2mm}
\item Montrer que, pour tout entier $k$ supérieur ou égal à 2,
$e^{k-1}\leq a_{k}\leq e^{k}$

\item Pour tout entier $k$ supérieur ou égal à 2, on pose $a_{k} =
e^{k}\left(
1 + \delta_{k}\right) $. Montrer que le réel $\delta_{k}$ vérifie
l'équation 
\[
-ke^{-k} = \left( 1 + \delta_{k}\right) \ln \left( 1 +
\delta_{k}\right)
\]
Justifier l'inégalité : $\left| \ln \left( 1 + \delta_{k}\right)
\right| \leq ke^{1-k}$. En déduire que la suite $\left( \delta
_{k}\right)_{k\geq 2}$ a une limite nulle et, plus précisément, que
$\delta_{k}$ est équivalent à $-ke^{-k}$ quand $k$ tend vers l'infini.

\item Justifier, en conclusion, la relation 
\[
a_{k} = e^{k}-k + o\left( k\right) \text{ quand }k\rightarrow + \infty
\]
\end{noliste}

\item Calcul approché des nombres $a_{k}$.

Écrire un programme en -\Scilab{} donnant une valeur approchée à moins
de $10^{-4}$ près du nombre $a_{4}$.
\end{noliste}

\section*{EXERCICE III}

Une chaîne de fabrication produit des objets dont certains peuvent être
défectueux. Pour modéliser ce processus on considère une suite $\left(
X_{n}\right)_{n\geq 1}$ de variables aléatoires de Bernoulli
indépendantes, de paramètre p, $(0<p<1)$. La variable aléatoire $X_{n}$
prend la
valeur 1 si le $n^{i\grave{e}me}$ objet produit est défectueux et prend
la
valeur 0 s'il est de bonne qualité.\\
Pour contrôler la qualité des objets produits, on effectue des
prélèvements
aléatoires et on considère une suite $\left( Y_{n}\right)_{n\geq 1}$
de variables aléatoires de Bernoulli indépendantes, de paramètre
$p^{\prime }
$, ($0<p^{\prime }<1$), telle que $Y_{n}$ prend la valeur 1 si le
$n^{i\grave{e}me}$ objet produit est contrôlé et 0 s'il ne l'est pas.\\
Toutes les variables aléatoires $X_{n}$ et $Y_{n}$ sont définies sur un
même
espace probabilisé $\Omega $, muni d'une probabilité notée $p$ et sont
supposées toutes indépendantes entre elles.\\
La probabilité conditionnelle d'un évènement $A$ sachant un évènement
$B$
est notée $\mathbf{P}_{B}\left( A\right) $\\
Pour tout entier $n\geq 1$, on pose $Z_{n} = X_{n}Y_{n}$. La variable
aléatoire $Z_{n}$ ainsi définie vaut donc 1 si le $n^{i\grave{e}me}$
objet est à
la fois défectueux et contrôlé et 0 sinon.\\
L'objet de l'exercice est d'étudier le nombre d'objets défectueux
produits
par la chaîne avant qu'un objet défectueux n'ait été détecté.

\begin{noliste}{1.}
 \setlength{\itemsep}{4mm}
\item Déterminer, pour tout entier $n\geq 1$, la loi de la variable
aléatoire $Z_{n}$ et la covariance des variables $X_{n}$ et $Z_{n}$. En
déduire
que les variables $X_{n}$ et $Z_{n}$ ne sont pas indépendantes.

\textsl{Par contre, il résulte des hypothèses (et on ne demande pas de
le
justifier) que, pour tout entier }$n$\textsl{, la variable aléatoire
}$Z_{n}$\textsl{\ est indépendante des variables }$\left( X_{i},i\ne
n\right) $\textsl{\ et des variables }$\left( Y_{i},i\ne n\right)
$\textsl{, de même
que des variables }$\left( Z_{i},i\ne n\right) $\textsl{.}

\item Soit, pour tout entier $n\geq 1$, $A_{n}$ l'évènement :
\textquotedblleft le $n^{i\grave{e}me}$ objet fabriqué est le premier
qui
ait été contrôlé et trouvé défectueux\textquotedblleft.

\begin{noliste}{a)}
 \setlength{\itemsep}{2mm}
\item Exprimer $A_{n}$ à l'aide des variables aléatoires
$Z_{1},\,Z_{2},\dots,\,Z_{n}$ et déterminer $p\left( A_{n}\right) $.

\item Montrer qu'on finira, presque sûrement, par détecter un objet
défectueux.
\end{noliste}

\item Soit un entier $n\geq 2$.

\begin{noliste}{a)}
 \setlength{\itemsep}{2mm}
\item Pour tout entier $k$ vérifiant $1\leq k\leq n-1$, calculer
la probabilité des évènements \\
$\left( X_{k} = 1\right) \cap A_{n}$ et $\left( X_{k} = 1\right) \cap
\left(
Z_{k} = 0\right) $.

On note $B_{k}$ l'évènement $\left( Z_{k} = 0\right) $. Montrer
l'égalité des
probabilités conditionnelles

\[
\mathbf{P}_{A_{n}}\left( X_{k} = 1\right) = \mathbf{P}_{B_{k}}\left(
X_{k} = 1\right) = \dfrac{p-pp^{\prime }}{1-pp^{\prime }}
\]

\item Montrer que si $x_{1},\;x_{2},\dots,\;x_{n-1}$ est une suite
quelconque de nombres égaux à 0 ou à 1, on a 
\[
\mathbf{P}_{A_{n}}\left( X_{1} = x_{1},X_{2} = x_{2},\dots,X_{n-1} =
x_{n-1}\right) = \prod_{i = 1}{n-1}\mathbf{P}_{A_{n}}\left(
X_{i} = x_{i}\right)
\]

\item Soit $S_{n} = \Sum{j = 1}{n-1}X_{j}$ le nombre d'objets
défectueux fabriqués avant le $n^{i\grave{e}me}$ objet et soit un
entier $m$ vérifiant $0\leq m\leq n-1$. Calculer
$\mathbf{P}_{A_{n}}\left(
S_{n} = m\right) $

\item Déterminer l'espérance de $S_{n}$ pour la probabilité
conditionnelle
sachant $A_{n}$.
\end{noliste}
\end{noliste}

\label{fin}

\end{document}

