\documentclass[11pt]{article}%
\usepackage{geometry}%
\geometry{a4paper,
 lmargin = 2cm,rmargin = 2cm,tmargin = 2.5cm,bmargin = 2.5cm}

\input{../../macros.tex}

\pagestyle{fancy} %
\lhead{ECE2 \hfill Mathématiques\\
} %
\chead{\hrule} %
\rhead{} %
\lfoot{} %
\cfoot{} %
\rfoot{\thepage} %

\renewcommand{\headrulewidth}{0pt}% : Trace un trait de séparation
 % de largeur 0,4 point. Mettre 0pt
 % pour supprimer le trait.

\renewcommand{\footrulewidth}{0.4pt}% : Trace un trait de séparation
 % de largeur 0,4 point. Mettre 0pt
 % pour supprimer le trait.

\setlength{\headheight}{14pt}

\title{\bf \vspace{-2cm} ESCP 1998} %
\author{} %
\date{} %
\begin{document}

\maketitle %
\vspace{-1.4cm}\hrule %
\thispagestyle{fancy}

\vspace*{.2cm}


% DEBUT DU DOC À MODIFIER : tout virer jusqu'au début de l'exo

%Définition et changement de valeurs de
compteurs%newcounter{cpt1}{section} compteur cpt1 remis à 0 à chaque
aumentation par stepcounter du compteur section%setcounter{cpt1}{3} on
met le compteur à 3%addtocounter{cpt1}{5} on ajoute 5 au compteur%
stepcounter{cpt1} on ajoute 1% ifthenelse{test}{alors}{sinon} (page
206) pour subordonner à une condition % whiledo{test}{commande} pour
faire une boucle (page 206 aussi) % value{cpt1} pour noter dans le
document la valeur de cpt1 
%Définition définitive d'opérateurs
mathématiques\newcommand{\ch}{\operatorname{ch}} 
\newcommand{\sh}{\operatorname{sh}}
\renewcommand{\tanh}{\operatorname{th}}
\renewcommand{\sinh}{\operatorname{sh}}
\renewcommand{\cosh}{\operatorname{ch}}
\newcommand{\argsh}{\operatorname{argsh}}
\newcommand{\argch}{\operatorname{argch}}
\newcommand{\argth}{\operatorname{argth}}
\newcommand{\ker}{\operatorname{Ker}}
\renewcommand{\im}{\operatorname{Im}}
\newcommand{\rg}{\operatorname{rg}}
\newcommand{\Id}{\operatorname{Id}}
\newcommand{\id}{\operatorname{id}}
\renewcommand{\leq}{\leq}
\renewcommand{\geq}{\geq }

%Définition de nouvelles couleurs : rgb(trois paramètres red green blue
entre 0 et 1); cmyk (quatre cyan magenta yellow black) entre 0 et 1;
gray (entre 0 et 1) et black, white, red, green, blue, cyan, magenta,
yellow% definecolor{0gris}{gray}{0.8} 
% Nouvelle commande pour encadrer le titre car shabox ne veut que d'une
seule ligne; ATTENTION A LA TAILLE; petite différence avec shadowbox ou
doublebox, voire fcolorbox ou colorbox (au lieu de shabox; laisser le
parbox tranquille sauf pour la taille de la boîte
\newcommand{\Tbox}[1]{\begin{center} \shabox{\parbox{0.6
\linewidth}{#1}} \end{center}} %[1] pour 1 paramètre ; #1 pour ce que
fait le 1er paramètre; entre accolades ce que fait la commande
%Mise en page en mode fancy : en-têtes et pieds de pages puis
définition des en-têtes et pieds de pages\pagestyle{fancy}
\lhead{ECE 2 - Mathématiques \\
Quentin Dunstetter - ENC-Bessières 2011$\backslash$2012}
\chead{}
\rhead{ESCP 1998}
\rfoot[ \ \thepage]{\thepage}
\cfoot{}
\lfoot{}
\thispagestyle{fancy} %Mise en page de la 1ère page en mode fancy
%Trait en bas et en haut de la page (entre en-tête et texte et texte et
pied de page)\renewcommand{\footrulewidth}{0.4pt}
\renewcommand{\headrulewidth}{0.4pt}


%DEBUT DU DOCUMENT\vspace*{3cm}

\begin{center}
{\LARG\E\textbf{BANQUE COMMUNE D'ÉPREUVES}}



{\large \textsc{CONCOURS D ADMISSION DE 1998}}



{\large \textbf{Concepteur : ESCP}}



\rule{2.39cm}{0.05cm}



{\Large \textbf{OPTION ÉCONOMIQUE}}



{\Large \textbf{MATHÉMATIQUES }}



{\Large Lundi 9 mai, de 14h à 18h}



\rule{2.39cm}{0.05cm}
\end{center}

\textit{La présentation, la lisibilité, l'orthographe, la qualité
de la rédaction, la clarté et la précision des raisonnements
entreront pour une part importante dans l'appréciation des copies.}

\textit{Les candidats sont invités à \textbf{encadrer} dans la mesure
du possible les résultats de leurs calculs.}

\textit{Ils ne doivent faire usage d'aucun document. L'utilisation de
toute
calculatrice et de tout matériel électronique est interdite. Seule
l'utilisation d'une règle graduée est autorisée.}

\textit{Si au cours de l'épreuve, un candidat repère ce qui lui semble
être une erreur d'énoncé, il la signalera sur sa copie et
poursuivra sa composition en expliquant les raisons des initiatives
qu'il sera
amené à prendre.}

\vspace*{3cm}

\section*{\textbf{EXERCICE I}}

Dans tout l'exercice $n$ désignera un entier naturel supérieur ou égal
à $2$.

\begin{noliste}{1.}
 \setlength{\itemsep}{4mm}
\item 

\begin{noliste}{a)}
 \setlength{\itemsep}{2mm}
\item Étudier, suivant la parité de $n$, le tableau de variations de la
fonction $f_{n}$ définie sur $\R$ par $x\mapsto x^{n + 1} + x^{n}$.

\item Montrer que dans tous les cas $f_{n}\left( -\dfrac{n}{n +
1}\right) <2$.

\item En déduire, suivant la parité de $n$, le nombre de solutions de
l'équation d'inconnue $x$ : 
\[
x^{n + 1} + x^{n} = 2.
\]
\end{noliste}

\item On note $A$ la matrice $\begin{smatrix}
1 & 1 \\
1 & 1
\end{smatrix}
$. Montrer qu'il existe une matrice inversible $P\in
\mathfrak{M}_{2}(\R)$ telle que 
\[
A = P
\begin{smatrix}
0 & 0 \\
0 & 2
\end{smatrix}
P^{-1}.
\]

\item On considère l'équation matricielle d'inconnue $X\in
\mathfrak{M}_{2}(\R)$ : 
\[
(E_{n})\qquad X^{n + 1} + X^{n} = A.
\]

\begin{noliste}{a)}
 \setlength{\itemsep}{2mm}
\item Montrer que la résolution de cette équation peut se ramener à la
résolution de l'équation d'inconnue $Y\in \M{2}$ : 
\[
(E_{n}{\prime })\qquad Y^{n + 1} + Y^{n} = 
\begin{smatrix}
0 & 0 \\
0 & 2
\end{smatrix}.
\]

\item Soit $Y$ une solution de $(E_{n}{\prime })$. On pose $Y = 
\begin{smatrix}
a & b \\
c & d
\end{smatrix}
$ et $D = 
\begin{smatrix}
0 & 0 \\
0 & 2
\end{smatrix}
$.

\begin{nonoliste}{(i)}
\item Montrer que $DY = YD$.

\item En déduire que $b = c = 0$.

\item Quelles sont les valeurs possibles de $a$ ?

\item Discuter, suivant les valeurs de $n$, le nombre de solutions de
l'équation $(E_{n})$.
\end{nonoliste}

\item On note $\mu $ la solution négative de l'équation numérique
$x^{4} + x^{3} = 2$. Déterminer les solutions de l'équation $(E_{3})$ à
l'aide de 
$\mu $.
\end{noliste}
\end{noliste}

\section*{{\textbf{EXERCICE II}}}

On désigne par $\lambda $ un paramètre réel strictement supérieur à
$1$.
Soit $H$ l'ensemble des points $(x,y)$ de $\R^{2}$ tels que $x>0$ et
soit $D$ l'ensemble des points de $H$ tels que $\,y\neq 0$. L'objet de
l'exercice est l'étude des extremums de la fonction $f$ définie sur $H$
par 
\[
f(x,y) = x^{\lambda }\,y-y^{2}-y\,\ln (x + 1) + 1.
\]

\begin{noliste}{1.}
 \setlength{\itemsep}{4mm}
\item Soit $\phi$ la fonction définie sur $]0, + \infty[$ par $\phi(x)
= 
x^\lambda - \ln(x + 1)$ et $\phi^{\prime}$ sa dérivée.

\begin{noliste}{a)}
 \setlength{\itemsep}{2mm}
\item Montrer que l'équation $\phi^{\prime}(x) = 0$ admet une racine et
une seule dans $]0, + \infty[$.

On note $b$ cette racine et on pose $\phi(b) = 2c$. Montrer que $c <
0$.

\item Montrer que l'équation $\phi(x) = 0$ admet une racine et une
seule,
notée $a$, dans $]0, + \infty[$ et que $a > b$.
\end{noliste}

\item Calculer les dérivées partielles $\,f_{x}{\prime }$ et
$\,f_{y}{\prime }$ de la fonction $f$.

\item 

\begin{noliste}{a)}
 \setlength{\itemsep}{2mm}
\item Déterminer l'ensemble des points $(x,y)\in H$ vérifiant
$f_{x}{\prime
}(x,y) = 0$.

\item Déterminer les points $(x,y)\in H$ vérifiant $f_{x}{\prime }(x,y)
= 0$
et $f_{y}{\prime }(x,y) = 0$. On exprimera les solutions $(x,y)$
trouvées à
l'aide des nombres $\,a$, $\,b$ et $\,c$ définis à la question
\textbf{1}.
\end{noliste}

\item 

\begin{noliste}{a)}
 \setlength{\itemsep}{2mm}
\item Calculer les dérivées partielles secondes de $f$.

\item Montrer que $f$ admet dans $D$ un extremum en un unique point
$(x_{\lambda },y_{\lambda })$ que l'on précisera.
\end{noliste}
\end{noliste}

\section*{\textbf{EXERCICE III}}

Toutes les variables aléatoires considérées dans cet exercice sont
supposées
définies sur un même espace probabilisé, muni de la probabilité $P$.\\
Pour tout entier $n\geq 1$, soit $X_{n}$ une variable aléatoire réelle
vérifiant $P\left(\Ev{X_{n} = k}\right) = \dfrac{1}{n}$ pour tout
entier $k$ tel que $0\leq
k\leq n-1$. On pose $Y_{n} = \dfrac{X_{n}}{n}.$\\
D'autre part, soit $Z$ une variable aléatoire de loi uniforme sur
l'intervalle $[0,1]$.

\begin{noliste}{1.}
 \setlength{\itemsep}{4mm}
\item 

\begin{noliste}{a)}
 \setlength{\itemsep}{2mm}
\item Déterminer l'espérance $\E(Z)$ et la variance $\V(Z)$ de la
variable aléatoire $Z$.

\item Calculer, pour tout $n\geq 1$, l'espérance et la variance de
$Y_{n}$.

\noindent Déterminer les limites des suites $(\E(Y_{n}))_{n\geq 1}$ et
$(\V(Y_{n}))_{n\geq 1}$.

\item Montrer que, pour toute fonction $f$ de classe $\mathcal{C}{1}$
sur $[0,1]$, à valeurs réelles, strictement monotone, on a
$\,\dlim{n\rightarrow + \infty }\E(f(Y_{n})) = E(f(Z)).$
\end{noliste}

\item Pour tout réel $x$ on note $\mathrm{Ent}(x)$ la partie entière de
$x$,
c'est-à-dire le plus grand nombre entier relatif inférieur ou égal à
$x$.

\begin{noliste}{a)}
 \setlength{\itemsep}{2mm}
\item Montrer que, pour tout réel $x$, $\dlim{n\rightarrow + \infty
}\dfrac{\mathrm{Ent}(nx)}{n}\, = \,x.$

\item Soit $a$ et $b$ deux réels vérifiant $0\leq a\leq b\leq
1$ et soit $I_{n}(a,b)$ le nombre d'entiers $k$ vérifiant
$a<\dfrac{k}{n}\leq b$. Montrer que $\,I_{n}(a,b) =
\mathrm{Ent}(nb)-\mathrm{Ent}(na).$

\item Montrer que, si $0\leq a\leq b\leq 1$, $\dlim{n\rightarrow +
\infty }P\left(\Ev{a<Y_{n}\leq b}\right) = P\left(\Ev{a<Z\leq
b}\right).$
\end{noliste}

\item Pour tout entier $n\geq 1$ on note $Z_{n}$ la variable aléatoire
$\dfrac{1}{n}\mathrm{Ent}(n\,Z)$ et on pose $D_{n} = Z-Z_{n}$.

\begin{noliste}{a)}
 \setlength{\itemsep}{2mm}
\item Montrer $Z_{n}$ et $Y_{n}$ ont même loi de probabilité.

\item Trouver la fonction de répartition et une densité de $D_{n}$.

\item Pour un entier $k$ tel que $0\leq k\leq n-1$ et un réel $y$
tel que $0\leq y\leq \dfrac{1}{n}$, exprimer à l'aide de la
variable aléatoire $Z$ l'évènement $\{Z_{n} =
\dfrac{k}{n}\;$et$\;D_{n}\leq y\}$. En déduire la valeur de
$P\left(\Ev{Z_{n} = \dfrac{k}{n}\;$et$\;D_{n}\leq y}\right)$.

\item Montrer que les variables aléatoires $Z_{n}$ et $D_{n}$ sont
indépendantes.
\end{noliste}
\end{noliste}

\label{fin}

\end{document}

