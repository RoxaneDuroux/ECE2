\documentclass[11pt]{article}%
\usepackage{geometry}%
\geometry{a4paper,
 lmargin = 2cm,rmargin = 2cm,tmargin = 2.5cm,bmargin = 2.5cm}

\input{../../macros.tex}

\pagestyle{fancy} %
\lhead{ECE2 \hfill Mathématiques\\
} %
\chead{\hrule} %
\rhead{} %
\lfoot{} %
\cfoot{} %
\rfoot{\thepage} %

\renewcommand{\headrulewidth}{0pt}% : Trace un trait de séparation
 % de largeur 0,4 point. Mettre 0pt
 % pour supprimer le trait.

\renewcommand{\footrulewidth}{0.4pt}% : Trace un trait de séparation
 % de largeur 0,4 point. Mettre 0pt
 % pour supprimer le trait.

\setlength{\headheight}{14pt}

\title{\bf \vspace{-2cm} ESCP 1990} %
\author{} %
\date{} %
\begin{document}

\maketitle %
\vspace{-1.4cm}\hrule %
\thispagestyle{fancy}

\vspace*{.2cm}


% DEBUT DU DOC À MODIFIER : tout virer jusqu'au début de l'exo

%Définition et changement de valeurs de
compteurs%newcounter{cpt1}{section} compteur cpt1 remis à 0 à chaque
aumentation par stepcounter du compteur section%setcounter{cpt1}{3} on
met le compteur à 3%addtocounter{cpt1}{5} on ajoute 5 au compteur%
stepcounter{cpt1} on ajoute 1% ifthenelse{test}{alors}{sinon} (page
206) pour subordonner à une condition % whiledo{test}{commande} pour
faire une boucle (page 206 aussi) % value{cpt1} pour noter dans le
document la valeur de cpt1 
%Définition définitive d'opérateurs
mathématiques\newcommand{\ch}{\operatorname{ch}} 
\newcommand{\sh}{\operatorname{sh}}
\renewcommand{\tanh}{\operatorname{th}}
\renewcommand{\sinh}{\operatorname{sh}}
\renewcommand{\cosh}{\operatorname{ch}}
\newcommand{\argsh}{\operatorname{argsh}}
\newcommand{\argch}{\operatorname{argch}}
\newcommand{\argth}{\operatorname{argth}}
\newcommand{\ker}{\operatorname{Ker}}
\renewcommand{\im}{\operatorname{Im}}
\newcommand{\rg}{\operatorname{rg}}
\newcommand{\Id}{\operatorname{Id}}
\newcommand{\id}{\operatorname{id}}
\renewcommand{\leq}{\leq}
\renewcommand{\geq}{\geq }

%Définition de nouvelles couleurs : rgb(trois paramètres red green blue
entre 0 et 1); cmyk (quatre cyan magenta yellow black) entre 0 et 1;
gray (entre 0 et 1) et black, white, red, green, blue, cyan, magenta,
yellow% definecolor{0gris}{gray}{0.8} 
% Nouvelle commande pour encadrer le titre car shabox ne veut que d'une
seule ligne; ATTENTION A LA TAILLE; petite différence avec shadowbox ou
doublebox, voire fcolorbox ou colorbox (au lieu de shabox; laisser le
parbox tranquille sauf pour la taille de la boîte
\newcommand{\Tbox}[1]{\begin{center} \shabox{\parbox{0.6
\linewidth}{#1}} \end{center}} %[1] pour 1 paramètre ; #1 pour ce que
fait le 1er paramètre; entre accolades ce que fait la commande
%Mise en page en mode fancy : en-têtes et pieds de pages puis
définition des en-têtes et pieds de pages\pagestyle{fancy}
\lhead{ECE 2 - Mathématiques \\
Quentin Dunstetter - ENC-Bessières 2011$\backslash$2012}
\chead{}
\rhead{ESCP 1990}
\rfoot[ \ \thepage]{\thepage}
\cfoot{}
\lfoot{}
\thispagestyle{fancy} %Mise en page de la 1ère page en mode fancy
%Trait en bas et en haut de la page (entre en-tête et texte et texte et
pied de page)\renewcommand{\footrulewidth}{0.4pt}
\renewcommand{\headrulewidth}{0.4pt}


%DEBUT DU DOCUMENT\vspace*{3cm}

\begin{center}
{\LARG\E\textbf{BANQUE COMMUNE D'ÉPREUVES}}



{\large \textsc{CONCOURS D ADMISSION DE 1990}}



{\large \textbf{Concepteur : ESCP}}



\rule{2.39cm}{0.05cm}



{\Large \textbf{OPTION ÉCONOMIQUE}}



{\Large \textbf{MATHÉMATIQUES }}



{\Large Lundi 9 mai, de 14h à 18h}



\rule{2.39cm}{0.05cm}
\end{center}

\textit{La présentation, la lisibilité, l'orthographe, la qualité
de la rédaction, la clarté et la précision des raisonnements
entreront pour une part importante dans l'appréciation des copies.}

\textit{Les candidats sont invités à \textbf{encadrer} dans la mesure
du possible les résultats de leurs calculs.}

\textit{Ils ne doivent faire usage d'aucun document. L'utilisation de
toute
calculatrice et de tout matériel électronique est interdite. Seule
l'utilisation d'une règle graduée est autorisée.}

\textit{Si au cours de l'épreuve, un candidat repère ce qui lui semble
être une erreur d'énoncé, il la signalera sur sa copie et
poursuivra sa composition en expliquant les raisons des initiatives
qu'il sera
amené à prendre.}

\vspace*{3cm}

\section*{Exercice 1}

\begin{noliste}{1.}
 \setlength{\itemsep}{4mm}
\item Utiliser la méthode du pivot pour inverser la matrice : 
\[
A = 
\begin{smatrix}
5 & 3 & -1 \\
-1 & 5 & 3 \\
3 & -1 & 5
\end{smatrix}
\]

\item On considère le système $(S)$ de trois équations à trois
inconnues
suivant : 
\[
(S) :\left\{
\begin{array}{cl}
x^{2}-yz = 5 \\
y^{2}-zx = -1 \\
z^{2}-xy = 3
\end{array}
\right.
\]
Pour tout triplet $(x,y,z)$ de nombres réels, on pose : 
\[
M_{(x,y,z)} = 
\begin{smatrix}
x & y & z \\
z & x & y \\
y & z & x
\end{smatrix}
\]
Prouver que si $(x,y,z)$ est une solution de $(S)$, alors : 
\[
AM_{(x,y,z)} = (5x-y + 3z)I_{3}.
\]
En déduire que, dans ces conditions, il existe un nombre réel $k$ tel
que : 
\[
x = 2k,\quad y = -k,\quad z = k.
\]

\item Montrer que le système $(S)$ admet deux solutions que l'on
calculera.
\end{noliste}

\section*{Exercice 2}

Soit $f$ la fonction numérique définie sur $[0, + \infty \lbrack $ par
la
relation : 
\[
f(t) = \ln (1 + t) + \frac{t^{2}}{1 + t^{2}}.
\]

\begin{noliste}{1.}
 \setlength{\itemsep}{4mm}
\item 

\begin{noliste}{a)}
 \setlength{\itemsep}{2mm}
\item Étudier les variations de $f$.

\item Déterminer la limite du rapport $\dfrac{f(t)}{t}$ lorsque $t$
tend
vers $ + \infty $. Tracer la courbe représentative de $f$.
\end{noliste}

\item Soit $n$ un entier naturel non nul. On considère l'équation : 
\[
(E_{n}) :f(t) = \frac{1}{n}
\]

\begin{noliste}{a)}
 \setlength{\itemsep}{2mm}
\item Montrer que l'équation $(E_{n})$ admet une solution $\alpha_{n}$
et
une seule. Donner des valeurs approchées de $\alpha_{1}$ et de
$\alpha_{2}$
à $10^{-2}$ près.

\item Montrer que la fonction $f$ admet une fonction réciproque.
Dresser le
tableau de variation de $f^{-1}$ et tracer la courbe représentative de
cette
fonction.

En déduire le sens de variation et la limite de la suite $(\alpha
_{n})_{n\in \N^{\times }}$.

\item Déterminer la limite du rapport $\dfrac{f(t)}{t}$ lorsque $t$
tend
vers $0$ par valeurs strictement positives.

En déduire la limite de la suite $(n\alpha_{n})_{n\in \N^{\times }}$.
\end{noliste}
\end{noliste}

\section*{Exercice 3}

Pour tout nombre réel $p$ tel que $0<p<1$, on dit qu'une variable
aléatoire $X$ définie sur un espace probabilisé
$\left(\Omega,\mathcal{A},P\right)$ suit une loi de type $G$ de
paramètre $p$ si, pour tout
nombre entier naturel $n$, 
\[
P\left(\Ev{X = n}\right) = \left(\Ev{1-p}\right)^{n}p.
\]

\begin{noliste}{1.}
 \setlength{\itemsep}{4mm}
\item \label{def T} Soit $T$ une variable aléatoire définie sur $\left(
\Omega,\mathcal{A},P\right) $ représentant le nombre de jours pendant
lesquels une machine fonctionne avant de tomber en panne. On suppose,
que
pour tout nombre entier naturel $n$, $P\left(\Ev{T\geq n}\right)>0$ ;
on note $\theta
(n)$ la probabilité conditionnelle $P\left(\Ev{T = n/T\geq n}\right)$,
qu'on appelle 
\emph{taux de panne} de la machine au $n^{i\grave{e}me}$ jour.

\begin{noliste}{a)}
 \setlength{\itemsep}{2mm}
\item Montrer que :
\[
\theta (n) = 1-\frac{P\left(\Ev{T\geq n + 1}\right)}{P\left(\Ev{T\geq
n}\right)}.
\]

\item Exprimer $P\left(\Ev{T\geq n}\right)$ et $P\left(\Ev{T =
n}\right)$ à l'aide des nombres $\theta
(j) $, où $j\in \N$.

\item Montrer que la variable aléatoire $T$ suit une loi de type $G$ si
et
seulement si la suite $\left( \theta (n)\right)_{n\in \N}$ est
constante.
\end{noliste}

\item \label{def Ti} On suppose maintenant qu'on a affaire à un système
de
deux machines montées en série : chaque pièce passe successivement dans
les
deux machines, notées $M_{1}$ et $M_{2}$. \\
à chaque machine $M_{i}$, où $i\in \{1,2\}$, on associe une variable
aléatoire $T_{i}$ définie comme la variable $T$ de la question \ref{def
T}. 
\\
On note $T_{\ast }$ le nombre de jours pendant lesquels le système
fonctionne, c'est-à-dire pendant lesquels aucune des deux machines ne
tombe
en panne. \\
On suppose que $T_{1}$ et $T_{2}$ sont définies sur l'espace
probabilisé $\left( \Omega,\mathcal{A},P\right) $, indépendantes et de
loi $G$ de paramètres respectifs $p_{1}$ et $p_{2}$ ;\\
on pose $q_{1} = 1-p_{1}$ et $q_{2} = 1-p_{2}$.

\begin{noliste}{a)}
 \setlength{\itemsep}{2mm}
\item Exprimer $T_{\ast }$ à l'aide de $T_{1}$ et $T_{2}$. Calculer
$P\left(\Ev{T_{\times }\geq n}\right)$.

\item Exprimer le taux de panne $\theta_{\ast }(n)$ du système en
fonction
de $q_{1}$ et $q_{2}$.
\end{noliste}

\item Soit cette fois un système de deux machines en parallèle : chaque
pièce est fabriquée soit par $M_{1}$ soit par $M_{2}$. On note $T_{1}$
et $T_{2} $ les variables aléatoires associées respectivement à $M_{1}$
et à $M_{2}$ comme dans la question \ref{def Ti}. On suppose qu'elles
satisfont
aux mêmes hypothèses et on note $T^{\ast }$ le nombre de jours pendant
lesquels le système fonctionne, c'est-à-dire que l'une au moins des
deux
machines ne tombe pas en panne.

\begin{noliste}{a)}
 \setlength{\itemsep}{2mm}
\item Exprimer $T^{\ast }$ à l'aide de $T_{1}$ et $T_{2}$. Calculer
$P\left(\Ev{T^{\ast }\geq n}\right)$.

\item Exprimer le taux de panne $\theta ^{\ast }(n)$ du système.
\end{noliste}
\end{noliste}

\label{fin}

\end{document}

