\documentclass[11pt]{article}%
\usepackage{geometry}%
\geometry{a4paper,
  lmargin=2cm,rmargin=2cm,tmargin=2.5cm,bmargin=2.5cm}

\input{../../../../macros.tex}
%\input{../../../../../../macros.tex}

\pagestyle{fancy} %
\lhead{ECE2 \\
  Mathématiques\\[.2cm]
  \hrule} %
\chead{} %
\rhead{} %
\lfoot{} %
\cfoot{} %
\rfoot{\thepage} %

\renewcommand{\headrulewidth}{0pt}% : Trace un trait de séparation
                                    % de largeur 0,4 point. Mettre 0pt
                                    % pour supprimer le trait.

\renewcommand{\footrulewidth}{0.4pt}% : Trace un trait de séparation
                                    % de largeur 0,4 point. Mettre 0pt
                                    % pour supprimer le trait.

\setlength{\headheight}{14pt}

\title{\bf \vspace{-1cm} ESCP 2001} %
\author{} %
\date{} %
\begin{document}

\maketitle %
\vspace{-1.2cm}\hrule %
\thispagestyle{fancy}

\vspace*{.4cm}


\section*{Exercice 1}

\begin{noliste}{1.}
 \setlength{\itemsep}{4mm}
\item On considère la matrice $A$ définie par : $A = 
\begin{smatrix}
-1 & -1 & 2 \\
1 & 2 & -1 \\
-2 & -1 & 3
\end{smatrix}
$ et on note $\phi $ l'endomorphisme de $\R^{3}$ représenté par $A$
dans la base canonique.

\begin{noliste}{a)}
 \setlength{\itemsep}{2mm}
\item 

\begin{nonoliste}{i)}
\item Montrer que $A$ admet les valeurs propres $1$ et $2$ et n'en
admet pas
d'autre.\\
Déterminer les sous-espaces propres $E_{1}$ et $E_{2}$ associés à ces
valeurs propres

\item La matrice $A$ est-elle diagonalisable ?
\end{nonoliste}

\item Soit $V$ un vecteur propre de $A$ associé à la valeur propre $1$.
Trouver un vecteur $W$ de $\R^{3}$ tel que $\phi (W) = v + W$.

\item Soit $U$ un vecteur propre de $A$ associé à la valeur propre $2$.
Montrer que la famille $(U,V,W)$ est une base de $\R^{3}$.

\item Déterminer la matrice $B$ représentant l'endomorphisme $\phi $
dans la
base $(U,V,W)$ ainsi qu'une matrice inversible $P$ telle qu'on ait
l'égalité\ $B = P^{-1}AP$.
\end{noliste}

\item Etant données les matrices 
\[
I = 
\begin{smatrix}
1 & 0 & 0 \\
0 & 1 & 0 \\
0 & 0 & 1
\end{smatrix},\qquad H = 
\begin{smatrix}
1 & 0 & 0 \\
0 & 0 & 0 \\
0 & 0 & 0
\end{smatrix}
\qquad N = 
\begin{smatrix}
0 & 0 & 0 \\
0 & 0 & 1 \\
0 & 0 & 0
\end{smatrix},
\]
on associe à tout élément $(a,b,c)$ de $R^{3}$ la matrice $C_{(a,b,c)}$
définie par : 
\[
C_{(a,b,c)} = aI + bH + cN
\]
On note $M$ l'ensemble des matrices $C_{(a,b,c)}$ où $(a,b,c)$ décrit
$\R^{3}$.

\begin{noliste}{a)}
 \setlength{\itemsep}{2mm}
\item Montrer que $M$ est un sous-espace vectoriel de l'espace
vectoriel $\M{3}$ des matrices carrées d'ordre 3 et
déterminer
sa dimension.

\item Vérifier que la matrice $B$ définie en \itbf{4.b)} appartient à 
$M $.

\item Préciser les conditions que doivent vérifier $(a,b,c)$ pour que
$C_{(a,b,c)}$ soit inversible. Déterminer, quand elle existe, sa
matrice
inverse.

\item Déterminer les valeurs propres de $C_{(a,b,c)}$.\\
Montrer que cette matrice est diagonalisable si et seulement si $c$ est
nul.
\end{noliste}
\end{noliste}

\newpage

\section*{Exercice 2}

\begin{noliste}{1.}
 \setlength{\itemsep}{4mm}
\item On considère la fonction $G$ de deux variables réelles définie,
pour
tout $x$ et $y$ strictement positifs, par : 
\[
G(x,y) = \dfrac{x^{2}}{2y^{2}}-\ln x + y-\dfrac{3}{2}
\]

\begin{noliste}{a)}
 \setlength{\itemsep}{2mm}
\item Calculer les dérivées partielles d'ordre 1 et 2 de la fonction
$G$.

\item Rechercher les extrema éventuels de la fonction $G$ dans le
domaine $]0, + \infty \lbrack \times ]0, + \infty \lbrack $.
\end{noliste}

\item On considère maintenant la fonction $f$ définie, pour tout $x$
strictement positif, par : 
\[
f(x) = G(x,1) = \dfrac{x^{2}}{2}-\ln x-\dfrac{1}{2}
\]

\begin{noliste}{a)}
 \setlength{\itemsep}{2mm}
\item Étudier les variations de $f$. Montrer que c'est une fonction
convexe.
Donner sa représentation graphique.

\item 

\begin{nonoliste}{i)}
\item Calculer une primitive de la fonction $f$ sur l'intervalle $]0, +
\infty
\lbrack $.

\item En déduire que l'intégrale $\dint{0}{1}f(x)\,\text{d}x$ existe
et calculer sa valeur.
\end{nonoliste}

\item Soit $n$ un entier supérieur ou égal à 2. On pose $S_{n} =
\dfrac{1}{n}\DSum{j = 1}{n}f\left(\dfrac{j}{n}\right)$.

\begin{nonoliste}{i)}
\item Établir, pour tout entier $j$ vérifiant $1\leq j\leq n$, les
inégalités : 
\[
\dfrac{1}{n}f\left( \dfrac{j + 1}{n}\right) \leq
\dint{\frac{j}{n}}{\frac{j + 1}{n}}f(x)\,\text{d}x\leq
\dfrac{1}{n}f\left( \dfrac{j}{n}\right) 
\]

\item En déduire l'encadrement : 
\[
\dint{\frac{1}{n}}{1}f(x)\,\dx\leq S_{n}\leq \dfrac{1}{n}f\left(
\dfrac{1}{n}\right) + \dint{\frac{1}{n}}{1}f(x)\,\dx
\]

\item Montrer les inégalités : 
\[
0\leq \dfrac{1}{n}f\left( \dfrac{1}{n}\right) \leq
\dint{0}{\frac{1}{n}}f(x)\,\dx
\]
\end{nonoliste}

\item On considère la suite $(S_{n})_{n\geq 2}$ définie précédemment.
Montrer que cette suite converge et déterminer sa limite.

\item On rappelle que, pour tout entier naturel non nul, on a l'égalité
$\DSum{k = 1}{n}k^{2} = \dfrac{n(n + 1)(2n + 1)}{6}$.

\begin{nonoliste}{(i)}
\item Exprimer, pour tout entier naturel non nul, la somme $\DSum{j =
1}{n}f\left( \dfrac{j}{n}\right) $ en fonction de $n$.

\item En déduire la limite : $\dlim{n\rightarrow + \infty
}\dfrac{1}{n}\ln \left( \dfrac{n^{n}}{n!}\right) $.
\end{nonoliste}
\end{noliste}
\end{noliste}

\newpage

\section*{Exercice 3}

\begin{noliste}{1.}
 \setlength{\itemsep}{4mm}
\item \textbf{Préliminaire}

Montrer, pour tout entier naturel non nul $n$, l'égalité : $\DSum{k =
1}{n}k^{3} = \dfrac{n^{2}(n + 1)^{2}}{4}$.

\item Soit $N$ un entier supérieur ou égal à 2.\\
Une urne contient $N$ boules dont $N-2$ sont blanches et 2 sont
noires. On
tire au hasard, successivement et sans remise, les $N$ boules de cette
urne.\\
Les tirages étant numérotés de 1 à $N$, on note $X_1$ la variable
aléatoire égale au numéro du tirage qui a fourni, pour la première
fois, une boule
noire et $X_{2}$ la variable aléatoire égale au numéro du tirage qui a
fourni, pour la deuxième fois, une boule noire.

\begin{noliste}{a)}
 \setlength{\itemsep}{2mm}
\item Préciser l'espace probabilisé $(\Omega,\A,\Prob)$ que l'on peut
utiliser
pour modéliser cette expérience aléaoire.

\item Soit $i$ et $j$ deux entiers de l'intervale $\llb 1,N\rrb$. 
Montrer que
l'on
a : 
\[
\Prob\left(\Ev{X_{1} = i,\ X_{2} = j}\right) = \left\{ 
\begin{array}{rcl}
0 & \text{si} & 1\leq j\leq i\leq N \\
\dfrac{2}{N(N-1)} & \text{si} & 1\leq i<j\leq N
\end{array}
\right.
\]

\item Déterminer les lois de probabilité des variables $X_{1}$ et
$X_{2}$.
Ces variables sont-elles indépendantes ?

\item 

\begin{nonoliste}{i)}
\item Démontrer que la variable $N + 1-X_{2}$ a même loi que $X_{1}$.

\item Déterminer la loi de la variable $X_{2}-X_{1}$ et la comparer à
celle
de $X_{1}$.
\end{nonoliste}

\item À l'aide des résultats de la question 4 :

\begin{nonoliste}{i)}
\item Calculer les espérances $\E(X_{1})$ et $\E(X_{2})$.

\item Montrer l'égalité des variances $\V(X_{1})$ et $\V(X_{2})$.

\item établir la relation : $2 \cov(X_{1},X_{2}) = \V(X_{1})$ où
$\cov(X_{1},X_{2})
$ désigne la covariance des variables $X_{1}$ et $X_{2}$.
\end{nonoliste}

\item Calculer $\V(X_{1})$; en déduire $\V(X_{2})$ et
$\cov(X_{1},X_{2})$.
\end{noliste}

\item Dans cette partie, $N$ désigne encore un entier supérieur ou égal
à
deux.

\begin{noliste}{a)}
 \setlength{\itemsep}{2mm}
\item On considère le programme \Scilab{} suivant, où 
\texttt{grand(1,1,\ttq{}uin\ttq{},1,10)}
désigne
un nombre entier tiré au hasard par l'ordinateur dans l'intervalle
$[1,10]$ :

\begin{scilab}
  & a = grand(1,1,\ttq{}uin\ttq{},1,10) \nl %
  & b = grand(1,1,\ttq{}uin\ttq{},1,10) \nl %
  & \tcIf{if} a > b \tcIf{then} \nl %
  & \qquad c = a \nl %
  & \qquad a = b \nl %
  & \qquad b = c \nl %
  & \tcIf{end} \nl %
  & \tcIf{if} a < b \tcIf{then} \nl %
  & \qquad disp(\ttq{}a = \ttq{} + string(a)) \nl %
  & \qquad disp(\ttq{}b = \ttq{} + string(b)) \nl %
  & \tcIf{end}
\end{scilab}

\newpage

\begin{nonoliste}{i)}
\item Que fait l'ordinateur dans le cas où les variables \texttt{a} et 
\texttt{b}
contiennent
toutes les deux le même nombre ?

\item Qu'affiche l'ordinateur dans le cas où les variables \texttt{a} 
et \texttt{b}
contiennent respectivement les nombres 3 et 5 ?

\item Qu'affiche l'ordinateur dans le cas où les variables \texttt{a} 
et \texttt{b}
contiennent respectivement les nombres 10 et 1 ?\ 
\end{nonoliste}

\item On suppose que $A$ et $B$ sont deux variables aléatoires définies
sur
le même espace probabilisé $(\Omega,\A,\Prob)$, indépendantes, suivant 
la
même
loi uniforme sur l'ensemble $\{1,2,\ldots,N)$ et on désigne par $D$
l'évènement : "$A$ ne prend pas la même valeur que $B$".

\begin{nonoliste}{i)}
\item Montrer que la probabilité de l'évènement $D$ est
$\dfrac{N-1}{N}$.

\item Soit $Y_{1}$ et $Y_{2}$ les variables aléatoires définies par :
$\left\{ 
\begin{array}{l}
Y_{1} = \min (A,B) \\
Y_{2} = \max (A,B)
\end{array}
\right. $\\[.2cm]
Calculer, pour tout couple $(i,j)$ d'éléments de $\{1,2,\ldots,N\}$, la
probabilité conditionnelle $\Prob_D\left(\Ev{Y_{1} = i,Y_{2} = 
j}\right)$.
\end{nonoliste}

\item Expliquer pourquoi le programme de la question \itbf{3.a)} permet 
de
simuler les
variables aléatoires $X_{1}$ et $X_{2}$, dans le cas où
$N$
est égal à 10.
\end{noliste}
\end{noliste}











\end{document}
