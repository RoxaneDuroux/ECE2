\documentclass[11pt]{article}%
\usepackage{geometry}%
\geometry{a4paper,
 lmargin = 2cm,rmargin = 2cm,tmargin = 2.5cm,bmargin = 2.5cm}

\input{../../macros.tex}

\pagestyle{fancy} %
\lhead{ECE2 \hfill Mathématiques\\
} %
\chead{\hrule} %
\rhead{} %
\lfoot{} %
\cfoot{} %
\rfoot{\thepage} %

\renewcommand{\headrulewidth}{0pt}% : Trace un trait de séparation
 % de largeur 0,4 point. Mettre 0pt
 % pour supprimer le trait.

\renewcommand{\footrulewidth}{0.4pt}% : Trace un trait de séparation
 % de largeur 0,4 point. Mettre 0pt
 % pour supprimer le trait.

\setlength{\headheight}{14pt}

\title{\bf \vspace{-2cm} ESCP 2005} %
\author{} %
\date{} %
\begin{document}

\maketitle %
\vspace{-1.4cm}\hrule %
\thispagestyle{fancy}

\vspace*{.2cm}


% DEBUT DU DOC À MODIFIER : tout virer jusqu'au début de l'exo

%Définition et changement de valeurs de
compteurs%newcounter{cpt1}{section} compteur cpt1 remis à 0 à chaque
aumentation par stepcounter du compteur section%setcounter{cpt1}{3} on
met le compteur à 3%addtocounter{cpt1}{5} on ajoute 5 au compteur%
stepcounter{cpt1} on ajoute 1% ifthenelse{test}{alors}{sinon} (page
206) pour subordonner à une condition % whiledo{test}{commande} pour
faire une boucle (page 206 aussi) % value{cpt1} pour noter dans le
document la valeur de cpt1 
%Définition définitive d'opérateurs
mathématiques\newcommand{\ch}{\operatorname{ch}} 
\newcommand{\sh}{\operatorname{sh}}
\renewcommand{\tanh}{\operatorname{th}}
\renewcommand{\sinh}{\operatorname{sh}}
\renewcommand{\cosh}{\operatorname{ch}}
\newcommand{\argsh}{\operatorname{argsh}}
\newcommand{\argch}{\operatorname{argch}}
\newcommand{\argth}{\operatorname{argth}}
\newcommand{\ker}{\operatorname{Ker}}
\renewcommand{\im}{\operatorname{Im}}
\newcommand{\rg}{\operatorname{rg}}
\newcommand{\Id}{\operatorname{Id}}
\newcommand{\id}{\operatorname{id}}
\renewcommand{\leq}{\leq}
\renewcommand{\geq}{\geq }

%Définition de nouvelles couleurs : rgb(trois paramètres red green blue
entre 0 et 1); cmyk (quatre cyan magenta yellow black) entre 0 et 1;
gray (entre 0 et 1) et black, white, red, green, blue, cyan, magenta,
yellow% definecolor{0gris}{gray}{0.8} 
% Nouvelle commande pour encadrer le titre car shabox ne veut que d'une
seule ligne; ATTENTION A LA TAILLE; petite différence avec shadowbox ou
doublebox, voire fcolorbox ou colorbox (au lieu de shabox; laisser le
parbox tranquille sauf pour la taille de la boîte
\newcommand{\Tbox}[1]{\begin{center} \shabox{\parbox{0.6
\linewidth}{#1}} \end{center}} %[1] pour 1 paramètre ; #1 pour ce que
fait le 1er paramètre; entre accolades ce que fait la commande
%Mise en page en mode fancy : en-têtes et pieds de pages puis
définition des en-têtes et pieds de pages\pagestyle{fancy}
\lhead{ECE 2 - Mathématiques \\
Quentin Dunstetter - ENC-Bessières 2011$\backslash$2012}
\chead{}
\rhead{ESCP 2005}
\rfoot[ \ \thepage]{\thepage}
\cfoot{}
\lfoot{}
\thispagestyle{fancy} %Mise en page de la 1ère page en mode fancy
%Trait en bas et en haut de la page (entre en-tête et texte et texte et
pied de page)\renewcommand{\footrulewidth}{0.4pt}
\renewcommand{\headrulewidth}{0.4pt}


%DEBUT DU DOCUMENT\vspace*{3cm}

\begin{center}
{\LARG\E\textbf{BANQUE COMMUNE D'ÉPREUVES}}



{\large \textsc{CONCOURS D ADMISSION DE 2005}}



{\large \textbf{Concepteur : ESCP}}



\rule{2.39cm}{0.05cm}



{\Large \textbf{OPTION ÉCONOMIQUE}}



{\Large \textbf{MATHÉMATIQUES }}



{\Large Lundi 9 mai, de 14h à 18h}



\rule{2.39cm}{0.05cm}
\end{center}

\textit{La présentation, la lisibilité, l'orthographe, la qualité
de la rédaction, la clarté et la précision des raisonnements
entreront pour une part importante dans l'appréciation des copies.}

\textit{Les candidats sont invités à \textbf{encadrer} dans la mesure
du possible les résultats de leurs calculs.}

\textit{Ils ne doivent faire usage d'aucun document. L'utilisation de
toute
calculatrice et de tout matériel électronique est interdite. Seule
l'utilisation d'une règle graduée est autorisée.}

\textit{Si au cours de l'épreuve, un candidat repère ce qui lui semble
être une erreur d'énoncé, il la signalera sur sa copie et
poursuivra sa composition en expliquant les raisons des initiatives
qu'il sera
amené à prendre.}

\vspace*{3cm}

\section*{EXERCICE}

Dans tout l'exercice, $E$ désigne un espace vectoriel réel de dimension
$n$,
avec $n\geq 2$. Si $v$ est un endomorphisme de $E$, pour tout entier
naturel $k$, on note $v^{k}$ l'endomorphisme défini par récurrence par
$v^{0} = \operatorname{Id}$, où $\operatorname{Id}$ représente
l'endomorphisme identité, et $v^{k + 1} = v^{k}\circ v$.

Les parties A et B de cet exercice sont indépendantes.

\subsection*{Partie A.}

Dans cette partie, on suppose que l'entier $n$ est égal à 2, et on
considère
un endomorphisme $f$ vérifiant $f^{2} = 0$ et $f\neq 0$.

\begin{noliste}{1.}
 \setlength{\itemsep}{4mm}
\item Montrer qu'il existe un vecteur $x$ de $E$ tel que $(x,f(x))$
soit une
base de $E$.

\item En déduire que la matrice associée à $f$ dans, cette base est
$\begin{smatrix}
0 & 0 \\
1 & 0
\end{smatrix}
$.
\end{noliste}

\subsection*{Partie B.}

Dans cette partie, on suppose que $n = 4$ et on cherche à résoudre
l'équation $u^{2} = -\operatorname{Id}$, où $u$ est un endomorphisme de
$E$. Soit $f$ une solution
de cette équation.

\begin{noliste}{1.}
 \setlength{\itemsep}{4mm}
\item Montrer qu'il n'existe pas de scalaire $\lambda $ tel que
l'équation $f(x) = \lambda x$ d'inconnue $x\in E$, admette une solution
non nulle.

\item Soit $x$ un vecteur non nul de $E$. Montrer que la famille
$(x,f(x))$
est libre.\\
On note $F$ le sous-espace vectoriel engendré par cette famille. Quelle
est
la dimension de $F$ ?

\item 

\begin{noliste}{a)}
 \setlength{\itemsep}{2mm}
\item Montrer qu'il existe une base de $E$ de la forme
$(x,f(x),z_{1},z_{2})$.

\item Soit $G$ le sous-espace vectoriel de $E$ engendré par la famille
$(z_{1},z_{2})$; soit $y$ un vecteur non nul de $G$. Montrer que la
famille $(x,f(x),y,f(y))$ est libre.
\end{noliste}

\item Montrer que dans une base bien choisie, la matrice associée à $f$
s'écrit : $\begin{smatrix}
0 & -1 & 0 & 0 \\
1 & 0 & 0 & 0 \\
0 & 0 & 0 & -1 \\
0 & 0 & 1 & 0
\end{smatrix}
$
\end{noliste}

\subsection*{Partie C. }

On suppose dans cette partie, que $E$ désigne l'espace vectoriel
$\R_{3}[X]$ des polynômes à coefficients réels, de degré inférieur ou
égal à 3.\\
On définit sur $E$ l'application $f$ qui, à tout polynôme $P$ de $E$,
associe
$f(P)$ défini par 
\[
f(P)(X) = (1 + X^{2})P^{\prime \prime }(X)-2XP^{\prime }(X)
\]
où $P^{\prime }$ et $P^{\prime \prime }$\ sont respectivement les
dérivées
première et seconde de $P$.

\begin{noliste}{1.}
 \setlength{\itemsep}{4mm}
\item Montrer que $f$ est un endomorphisme de $E$.

\item 

\begin{noliste}{a)}
 \setlength{\itemsep}{2mm}
\item Écrire la matrice associée à $f$ dans la base canonique
$(1,X,X^{2},X^{3})$ de $E$.

\item En déduire que l'ensemble des valeurs propres de $f$ est
$\{0,-2\}$.\\
On note $E_{0}$ et $E_{-2}$ les sous-espaces propres associés
respectivement
aux valeurs propres $0$ et $-2$.

\item Déterminer une base de $E_{0}$ et une base de $E_{-2}$.

\item L'endomorphisme $f$ est-il diagonalisable ?
\end{noliste}

\item On veut résoudre dans cette question, l'équation $u^{2} = f$ dans
laquelle l'inconnue $u$ désigne un endomorphisme de $E$. Soit $g$ une
solution de cette équation.

\begin{noliste}{a)}
 \setlength{\itemsep}{2mm}
\item Montrer qu'il existe une base de $E$ dans laquelle la matrice
associée 
à $g^{2}$ s'écrit : $\begin{smatrix}
0 & 0 & 0 & 0 \\
0 & 0 & 0 & 0 \\
0 & 0 & -2 & 0 \\
0 & 0 & 0 & -2
\end{smatrix}
$

\item Montrer que $f$ et $g$ commutent, c'est-à-dire que pour tout $x$
de $E$, on a $(g\circ f)(x) = (f\circ g)(x)$.

\item On s'intéresse à la restriction de $g$ à $E_{0}$. Montrer que
l'on définit ainsi un endomorphisme de $E_{0}$ qu'on notera $g_{0}$.\\
Montrer de même que la restriction de $g$ à $E_{-2}$ définit un
endomorphisme de $E_{-2}$ qu'on notera $g_{-2}$.
\end{noliste}

\item En utilisant les résultats des parties précédentes, donner la
forme
d'une matrice associée à $g$.
\end{noliste}

\section*{PROBLEME}

\subsection*{Partie I}

Dans cette partie, $(a_{n})_{n\in \N}$ est une suite de réels
strictement positifs, décroissante et de limite nulle.\\
Pour tout entier naturel $n$, on pose : $u_{n} = \dsum\limits_{k =
0}{2n}(-1)^{k}a_{k}$, $v_{n} = \dsum\limits_{k = 0}{2n +
1}(-1)^{k}a_{k}$, $s_{n} = \dsum\limits_{k = 0}{n}(-1)^{k}a_{k}$.

\begin{noliste}{1.}
 \setlength{\itemsep}{4mm}
\item 

\begin{noliste}{a)}
 \setlength{\itemsep}{2mm}
\item Montrer que la suite $(u_{n})_{n\in \N}$ est décroissante, et
que la suite $(v_{n})_{n\in \N}$ est croissante.

\item Montrer que, pour tout $n$ de $\N$, $v_{n}\leq u_{n}$. En
déduire que la suite $(u_{n})_{n\in \N}$ admet une limite $s$, et
que la suite $(v_{n})_{n\in \N}$ admet la même limite $s$.

\item En déduire que la suite $(s_{n})_{n\in \N}$ converge vers $s$.
\end{noliste}

\item Montrer que la série de terme général $(-1)^{n}a_{n}$, est
convergente.

\item Montrer que la série de terme général $\dfrac{(-1)^{k}}{k +
1}$est
convergente. On note ${}\dsum\limits_{k = 0}{+ \infty
}\dfrac{(-1)^{k}}{k + 1}$
sa somme.

\item 

\begin{noliste}{a)}
 \setlength{\itemsep}{2mm}
\item Établir, pour tout réel $t$ positif et pour tout $n$ de
$\N^{\times }$, l'égalité :${}\dsum\limits_{k = 0}{n-1}(-1)^{k}t^{k} =
\dfrac{1}{1 + t}-(-1)^{n}\dfrac{t^{n}}{1 + t}.$

\item En déduire que pour tout $n$ de $\N^{\times }$ : $\dsum\limits_{k
= 0}{n}\dfrac{(-1)^{k}}{k + 1} = \ln
(2)-(-1)^{n}\dint\limits_{0}{1}\dfrac{t^{n}}{1 + t}dt$

\item En déduire la valeur de la somme $\dsum\limits_{k = 0}{+ \infty
}\dfrac{(-1)^{k}}{k + 1}$.
\end{noliste}
\end{noliste}

\subsection*{Partie II}

Deux amis, Pierre et Paul jouent au jeu suivant : ils possèdent une
machine
qui, à chaque sollicitation, leur donne aléatoirement un entier naturel
$n$;
chaque sollicitation constitue une manche de ce jeu et :

\begin{noliste}{$\sbullet$}
\item si cet entier $n$ est impair, Paul donne $n$ Euros à Pierre : on
considère que Pierre gagne et que son gain est égal à + $n$;

\item si cet entier $n$ est pair, Pierre donne $n$ Euros à Paul : on
considère que Pierre perd et que son gain est égal à - $n$;

\item si $n$ = 0, on considère que Pierre perd, et que son gain est
égal à 0.
\end{noliste}

\noindent On considère un espace probabilisé $(\Omega ;A;P)$ qui
modélise le
jeu.\\
Soit $X$ la variable aléatoire correspondant au nombre obtenu lors
d'une
sollicitation.

\begin{noliste}{1.}
 \setlength{\itemsep}{4mm}
\item On suppose, jusqu'à la fin de la question 5, que la loi de
probabilité
de $X$ est définie par 
\[
P\left(\Ev{X = 0}\right) = 0,\quad \text{et pour tout }n>1
:P\left(\Ev{X = n}\right) = \dfrac{\alpha }{n(n + 1)}
\]
où $\alpha $ est un réel strictement positif.

\begin{noliste}{a)}
 \setlength{\itemsep}{2mm}
\item Déterminer deux réels $a$ et $b$ tels que, pour tout $n$ de
$\N^{\times } :\dfrac{1}{n(n + 1)} = \dfrac{a}{n} + \dfrac{b}{n + 1}$.

\item En déduire la valeur de $\alpha $.
\end{noliste}

\item 

\begin{noliste}{a)}
 \setlength{\itemsep}{2mm}
\item Calculer la probabilité que Pierre gagne une manche quelconque.

\item Calculer l'espérance du gain de Pierre pour une manche.
\end{noliste}

\item Pierre et Paul effectuent deux manches consécutives. On suppose
que
les résultats de ces deux manches sont indépendants. On note $Y$ le
gain
cumulé de Pierre à l'issue de ces deux manches.\\
Calculer $P\left(\Ev{Y = 0}\right)$, $P\left(\Ev{Y = 2}\right)$ et
$P\left(\Ev{Y = -2}\right)$.

\item 

\begin{noliste}{a)}
 \setlength{\itemsep}{2mm}
\item Montrer que l'intégrale $\dint\limits_{0}{1}\dfrac{\ln
(x)}{1-x^{2}}dx
$ est convergente.

\item Établir, pour tout réel $x$ de $]0;1[$ et pour tout $n$ de
$\N^{\times }$, l'égalité : 
\[
\dfrac{\ln (x)}{1-x^{2}} = \dsum\limits_{k = 0}{n}x^{2k}\ln (x) +
\dfrac{x^{2n + 2}\ln (x)}{1-x^{\;2}}
\]

\item Montrer que pour tout entier naturel $k$, l'intégrale
$\dint\limits_{0}{1}x^{2k}\ln (x)dx$ converge.\\
Exprimer en fonction de $k$ la valeur de cette intégrale.

\item Montrer que la fonction $x\mapsto \dfrac{x^{2}\ln (x)}{1-x^{2}}$,
définie sur $]0;1[$, est prolongeable par continuité en 0 et en 1. En
déduire
qu'elle est bornée sur [0; 1]. Calculer $\underset{n\rightarrow +
\infty }{{\lim }}\dint\limits_{0}{1}\dfrac{x^{2n + 2}\ln
(x)}{1-x^{2}}dx.$

\item En déduire l'égalité : $\dint\limits_{0}{1}\dfrac{x^{2}\ln
(x)}{1-x^{2}}dx = -\dsum\limits_{k = 0}{+ \infty }\dfrac{1}{(2k +
1)^{2}}.$\\
On admet que : $\dsum\limits_{k = 1}{+ \infty }\dfrac{1}{k^{2}} =
\dfrac{\pi ^{2}}{6}$. Calculer la valeur de
$\dint\limits_{0}{1}\dfrac{\ln (x)}{1-x^{2}}dx$.
\end{noliste}

\item 

\begin{noliste}{a)}
 \setlength{\itemsep}{2mm}
\item Déterminer trois réels $a$, $b$ et $c$ tels que pour tout $n$ de
$\N^{\times }$, on ait l'égalité suivante :
\[
\dfrac{1}{n(n + 1)^{2}(n + 2)} = \dfrac{a}{n} + \dfrac{b}{(n + 1)^{2}}
+ \dfrac{c}{n + 2}
\]

\item Calculer $P\left(\Ev{Y = 1}\right)$.
\end{noliste}

\item On suppose dans cette question que $X$ suit une loi de Poisson de
paramètre $\lambda $, ($\lambda >0$).

\begin{noliste}{a)}
 \setlength{\itemsep}{2mm}
\item Calculer, en fonction de $\lambda $, la probabilité que Pierre
gagne
une manche.

\item Comparer la probabilité que Pierre gagne une manche à celle qu'il
perde une manche.

\item Calculer l'espérance du gain de Pierre pour une manche.
\end{noliste}
\end{noliste}

\label{fin}

\end{document}

