\documentclass[11pt]{article}%
\usepackage{geometry}%
\geometry{a4paper,
 lmargin = 2cm,rmargin = 2cm,tmargin = 2.5cm,bmargin = 2.5cm}

\input{../../macros.tex}

\pagestyle{fancy} %
\lhead{ECE2 \hfill Mathématiques\\
} %
\chead{\hrule} %
\rhead{} %
\lfoot{} %
\cfoot{} %
\rfoot{\thepage} %

\renewcommand{\headrulewidth}{0pt}% : Trace un trait de séparation
 % de largeur 0,4 point. Mettre 0pt
 % pour supprimer le trait.

\renewcommand{\footrulewidth}{0.4pt}% : Trace un trait de séparation
 % de largeur 0,4 point. Mettre 0pt
 % pour supprimer le trait.

\setlength{\headheight}{14pt}

\title{\bf \vspace{-2cm} ESCP 1996} %
\author{} %
\date{} %
\begin{document}

\maketitle %
\vspace{-1.4cm}\hrule %
\thispagestyle{fancy}

\vspace*{.2cm}


% DEBUT DU DOC À MODIFIER : tout virer jusqu'au début de l'exo

%Définition et changement de valeurs de
compteurs%newcounter{cpt1}{section} compteur cpt1 remis à 0 à chaque
aumentation par stepcounter du compteur section%setcounter{cpt1}{3} on
met le compteur à 3%addtocounter{cpt1}{5} on ajoute 5 au compteur%
stepcounter{cpt1} on ajoute 1% ifthenelse{test}{alors}{sinon} (page
206) pour subordonner à une condition % whiledo{test}{commande} pour
faire une boucle (page 206 aussi) % value{cpt1} pour noter dans le
document la valeur de cpt1 
%Définition définitive d'opérateurs
mathématiques\newcommand{\ch}{\operatorname{ch}} 
\newcommand{\sh}{\operatorname{sh}}
\renewcommand{\tanh}{\operatorname{th}}
\renewcommand{\sinh}{\operatorname{sh}}
\renewcommand{\cosh}{\operatorname{ch}}
\newcommand{\argsh}{\operatorname{argsh}}
\newcommand{\argch}{\operatorname{argch}}
\newcommand{\argth}{\operatorname{argth}}
\newcommand{\ker}{\operatorname{Ker}}
\renewcommand{\im}{\operatorname{Im}}
\newcommand{\rg}{\operatorname{rg}}
\newcommand{\Id}{\operatorname{Id}}
\newcommand{\id}{\operatorname{id}}
\renewcommand{\leq}{\leq}
\renewcommand{\geq}{\geq }

%Définition de nouvelles couleurs : rgb(trois paramètres red green blue
entre 0 et 1); cmyk (quatre cyan magenta yellow black) entre 0 et 1;
gray (entre 0 et 1) et black, white, red, green, blue, cyan, magenta,
yellow% definecolor{0gris}{gray}{0.8} 
% Nouvelle commande pour encadrer le titre car shabox ne veut que d'une
seule ligne; ATTENTION A LA TAILLE; petite différence avec shadowbox ou
doublebox, voire fcolorbox ou colorbox (au lieu de shabox; laisser le
parbox tranquille sauf pour la taille de la boîte
\newcommand{\Tbox}[1]{\begin{center} \shabox{\parbox{0.6
\linewidth}{#1}} \end{center}} %[1] pour 1 paramètre ; #1 pour ce que
fait le 1er paramètre; entre accolades ce que fait la commande
%Mise en page en mode fancy : en-têtes et pieds de pages puis
définition des en-têtes et pieds de pages\pagestyle{fancy}
\lhead{ECE 2 - Mathématiques \\
Quentin Dunstetter - ENC-Bessières 2011$\backslash$2012}
\chead{}
\rhead{ESCP 1996}
\rfoot[ \ \thepage]{\thepage}
\cfoot{}
\lfoot{}
\thispagestyle{fancy} %Mise en page de la 1ère page en mode fancy
%Trait en bas et en haut de la page (entre en-tête et texte et texte et
pied de page)\renewcommand{\footrulewidth}{0.4pt}
\renewcommand{\headrulewidth}{0.4pt}


%DEBUT DU DOCUMENT\vspace*{3cm}

\begin{center}
{\LARG\E\textbf{BANQUE COMMUNE D'ÉPREUVES}}



{\large \textsc{CONCOURS D ADMISSION DE 1996}}



{\large \textbf{Concepteur : ESCP}}



\rule{2.39cm}{0.05cm}



{\Large \textbf{OPTION ÉCONOMIQUE}}



{\Large \textbf{MATHÉMATIQUES }}



{\Large Lundi 9 mai, de 14h à 18h}



\rule{2.39cm}{0.05cm}
\end{center}

\textit{La présentation, la lisibilité, l'orthographe, la qualité
de la rédaction, la clarté et la précision des raisonnements
entreront pour une part importante dans l'appréciation des copies.}

\textit{Les candidats sont invités à \textbf{encadrer} dans la mesure
du possible les résultats de leurs calculs.}

\textit{Ils ne doivent faire usage d'aucun document. L'utilisation de
toute
calculatrice et de tout matériel électronique est interdite. Seule
l'utilisation d'une règle graduée est autorisée.}

\textit{Si au cours de l'épreuve, un candidat repère ce qui lui semble
être une erreur d'énoncé, il la signalera sur sa copie et
poursuivra sa composition en expliquant les raisons des initiatives
qu'il sera
amené à prendre.}

\vspace*{3cm}

\section*{Exercice I}

On note $\left( e_{1},e_{2},e_{3}\right) $ la base canonique de
l'espace
vectoriel $\R^{3}$. Dans tout l'exercice on associe à tout nombre réel
$a$ les vecteurs $U_{a},V_{a}$ et $W_{a}$ définis par : 
\begin{eqnarray*}
U_{a} & = & e_{1} + 2e_{2} + e_{3} \\
V_{a} & = & -e_{1} + (a-3)e_{2} + (a-1)e_{3} \\
W_{a} & = & -2e_{1}-4e_{2} + ae_{3}
\end{eqnarray*}

et l'endomorphisme $\Phi_{a}$ de $\R^{3}$ défini par les conditions : 
\[
\Phi_{a}\left( e_{1}\right) = U_{a},\;\;\Phi_{a}\left( e_{2}\right)
 = V_{a},\;\;\Phi_{a}\left( e_{3}\right) = W_{a}
\]

\begin{noliste}{1.}
 \setlength{\itemsep}{4mm}
\item Montrer que $\Phi_{0}$ est un automorphisme et déterminer la
matrice
de son automorphisme réciproque, dans la base canonique de $\R^{3}$

\item 

\begin{noliste}{a)}
 \setlength{\itemsep}{2mm}
\item Déterminer les valeurs du paramètre $a$ pour lesquelles
$\Phi_{a}$
est un automorphisme.

\item Déterminer, pour chaque valeur du paramètre $a$, le noyau de
$\Phi
_{a} $.

\item Déterminer les valeurs de $a$ pour lesquelles le vecteur
$e_{1}-e_{2}-e_{3}$ appartient à l'image de $\Phi_{a}$.
\end{noliste}

\item On considère maintenant les trois vecteurs :

$f_{1} = e_{1} + 3e_{2}-e_{3}$

$f_{2} = 2e_{1} + 4e_{2}-e_{3}$

$f_{3} = e_{3}.$

\begin{noliste}{a)}
 \setlength{\itemsep}{2mm}
\item Montrer que $\left( f_{1},f_{2},f_{3}\right) $ est; une base de
$\R^{3}$ et déterminer la matrice de $\Phi_{1}$ dans cette base.

\item L'endomorphisme $\Phi_{1}$ est-il diagonalisable ?
\end{noliste}

\item 

\begin{noliste}{a)}
 \setlength{\itemsep}{2mm}
\item Déterminer les valeurs propres de $\Phi_{-2}$.

\item Cet endomorphisme est-il diagonalisable ?
\end{noliste}
\end{noliste}

\section*{Exercice II}

On considère la fonction $f$ définie sur $\left] 0, + \infty \right[ $
par $f(1) = 1$ et 
\[
f(x) = \dfrac{x + 1}{x-1}\dfrac{\ln \left( x\right) }{2}\;\;\text{si\ \
}x\neq 1
\]

\begin{noliste}{1.}
 \setlength{\itemsep}{4mm}
\item Montrer que $f$ est une fonction continue sur $\left] 0, + \infty
\right[.$

\item Calculer la dérivée $f^{\prime }$ de $f$ sur les intervalles
$\left]
0,1\right[ $ et $\left] 1, + \infty \right[ $. Étudier son signe et en
déduire
que $f$ est monotone sur chacun de ces deux intervalles.

\item Montrer que pour tout $x$ strictement positif et différent de 1,
la dérivée $f^{\prime }$ de $f$ vérifie : 
\[
f^{\prime }\left( x\right) = \dfrac{\left( x-1\right) -\ln \left(
x\right) }{\left( x-1\right) ^{2}}-\dfrac{1}{2x}
\]
En déduire que $f$ est dérivable au point 1 et déterminer $f^{\prime
}\left(
1\right) $.

Montrer que $f^{\prime }$ est continue sur l'intervalle $\left] 0, +
\infty \right[.$

\item Montrer que, pour tout $x>1$, on a $\ln \left( x\right) <\left(
x-1\right) $. En déduire que, pour tout $x>1$, on a $f(x)<x.$

\item Donner la représentation graphique de la fonction $f$.

\item Soit $a$ un réel supérieur à 1.

\begin{noliste}{a)}
 \setlength{\itemsep}{2mm}
\item Montrer qu'il existe une suite $\left( x_{n}\right)_{n\geq 0}$ de
réels vérifiant $x_{0} = a$ et, pour tout entier $n\geq 0$, $x_{n + 1}
= f\left(
x_{n}\right) $

\item Montrer que cette suite est décroissante et qu'elle admet une
limite $\ell $ que l'on précisera.
\end{noliste}

\item On se propose d'étudier la vitesse avec laquelle la suite $\left(
x_{n}\right)_{n\geq 0}$ tend vers $\ell $.

\begin{noliste}{a)}
 \setlength{\itemsep}{2mm}
\item Montrer qu'il existe un entier $n_{0}$ tel que $\left| f\left(
x_{n}\right) -\ell \right| \leq \dfrac{1}{3}$ $\left| x_{n}-\ell
\right| $
pour tout $n\geq n_{0}$

\item En déduire que la suite $\left( x_{n}-\ell \right)_{0}$ est
négligeable devant la suite $\left( 1/2^{n}\right)_{n\geq 0}$
\end{noliste}
\end{noliste}

\section*{Exercice III}

On note $\N$ l'ensemble des entiers naturels et $\N^{\times }
$ l'ensemble des entiers strictement positifs\\
Une urne contient des boules blanches, noires et rouges. Les
proportions
respectives de ces boules sont $p$ pour les blanches, $q$ pour les
noires et 
$r$ pour les rouges ($p + q + r = 1$ )\\
On fait dan cette urne des tirages successifs indépendants numérotés 1,
2,... etc. Ces tirages sont faits avec relise de la boule tirée. Les
proportions des boules restent ainsi les mêmes au cours de
l'expérience.\\
Toutes les variables aléatoires sont définies dans un espace de
probabilité $\left( \Omega,\mathcal{A},P\right) $

\begin{noliste}{1.}
 \setlength{\itemsep}{4mm}
\item On note $X_{1}$ la variable aléatoire représentant le numéro du
tirage
auquel une boule blanche sort pour la prenrière fois. Trouver la loi de
probabilité de $X_{1}$ ; calculer son espérance et sa variance.

\item On note $X_{2}$ la variable aléatoire représentant le numéro du
deuxième tirage d'une boule blanche.

\begin{noliste}{a)}
 \setlength{\itemsep}{2mm}
\item Trouver, pour tout couple d'entiers strictement positifs $\left(
k,l\right) $, la probabilité de l'évènement.$\left\{
X_{1} = k\;,\;X_{2} = k + l\right\} $. En déduire la loi de probabilité
de $X_{2}$

\item Montrer que la variable $U_{2} = X_{2}-X_{1}$ est indépendante de
$X_{1}$
et qu'elle a la même loi de probabilité. En déduire l'espérance et la
variance de $X_{2}$.
\end{noliste}

\item On note $W$ la variable aléatoire représentant le nombre de
boules
rouges tirées avant l'obtention de la première boule blanche. Pour tout
couple $(k,l)$ de $\N^{\times }\times \N$, déterminer la
probabilité conditionnelle de l'évènement $\{W = l\}$ sachant que
$X_{1} = k$.
Quelle est la loi conditionnelle de $W$ sachant $X_{1} = k$ ?

\item On note $Y_{1}$ la variable aléatoire représentant le numéro du
tirage
auquel une boule noire sort pour la première fois.

\begin{noliste}{a)}
 \setlength{\itemsep}{2mm}
\item Trouver 1a loi de probabilité du couple $\left(
X_{1},Y_{1}\right) $.
Les variables aléatoires $X_{1}$ et $Y_{1}$sont elles indépendantes ?

\item On se place, pour cette question, dans le cas particulier où $r =
0$
(c'est à dire qu'il n'y a pas de boule rouge). Calculer alors la
covariance
de $X_{1}$ et $Y_{1}$
\end{noliste}

\item Soit, pour $n$ entier strictement positif, $Z_{n}$ la variable
aléatoire qui prend la valeur $ + 1$ si au $n^{i\grave{e}me}$ tirage
une boule
blanche est tirée, $-1$ si au $n^{i\grave{e}me}$ tirage une boule noire
est
tirée, 0 si au $n^{i\grave{e}me}$ tirage une boule rouge est tirée. On
note $S_{n} = Z_{1} + \dots + Z_{n}$

\begin{noliste}{a)}
 \setlength{\itemsep}{2mm}
\item Trouver la loi de probabilité de $S_{1}$. Calculer son espérance
et sa
variance ; en déduire l'espérance et la variance de $S_{n}$ pour tout
$n\ge
1.$

\item Soit $t$ un réel strictement positif. On pose $V_{n} =
t^{S_{n}}$.
Trouver la loi de probabilité de la variable $V_{1}$ et calculer son
espérance.

\item En déduire l'espérance de $V_{n}$.
\end{noliste}
\end{noliste}

\label{fin}

\end{document}

