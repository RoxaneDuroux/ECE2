\documentclass[11pt]{article}%
\usepackage{geometry}%
\geometry{a4paper,
 lmargin = 2cm,rmargin = 2cm,tmargin = 2.5cm,bmargin = 2.5cm}

\input{../../macros.tex}

\pagestyle{fancy} %
\lhead{ECE2 \hfill Mathématiques\\
} %
\chead{\hrule} %
\rhead{} %
\lfoot{} %
\cfoot{} %
\rfoot{\thepage} %

\renewcommand{\headrulewidth}{0pt}% : Trace un trait de séparation
 % de largeur 0,4 point. Mettre 0pt
 % pour supprimer le trait.

\renewcommand{\footrulewidth}{0.4pt}% : Trace un trait de séparation
 % de largeur 0,4 point. Mettre 0pt
 % pour supprimer le trait.

\setlength{\headheight}{14pt}

\title{\bf \vspace{-2cm} ESCP 2004} %
\author{} %
\date{} %
\begin{document}

\maketitle %
\vspace{-1.4cm}\hrule %
\thispagestyle{fancy}

\vspace*{.2cm}


% DEBUT DU DOC À MODIFIER : tout virer jusqu'au début de l'exo

%Définition et changement de valeurs de
compteurs%newcounter{cpt1}{section} compteur cpt1 remis à 0 à chaque
aumentation par stepcounter du compteur section%setcounter{cpt1}{3} on
met le compteur à 3%addtocounter{cpt1}{5} on ajoute 5 au compteur%
stepcounter{cpt1} on ajoute 1% ifthenelse{test}{alors}{sinon} (page
206) pour subordonner à une condition % whiledo{test}{commande} pour
faire une boucle (page 206 aussi) % value{cpt1} pour noter dans le
document la valeur de cpt1 
%Définition définitive d'opérateurs
mathématiques\newcommand{\ch}{\operatorname{ch}} 
\newcommand{\sh}{\operatorname{sh}}
\renewcommand{\tanh}{\operatorname{th}}
\renewcommand{\sinh}{\operatorname{sh}}
\renewcommand{\cosh}{\operatorname{ch}}
\newcommand{\argsh}{\operatorname{argsh}}
\newcommand{\argch}{\operatorname{argch}}
\newcommand{\argth}{\operatorname{argth}}
\newcommand{\ker}{\operatorname{Ker}}
\renewcommand{\im}{\operatorname{Im}}
\newcommand{\rg}{\operatorname{rg}}
\newcommand{\Id}{\operatorname{Id}}
\newcommand{\id}{\operatorname{id}}
\renewcommand{\leq}{\leq}
\renewcommand{\geq}{\geq }

%Définition de nouvelles couleurs : rgb(trois paramètres red green blue
entre 0 et 1); cmyk (quatre cyan magenta yellow black) entre 0 et 1;
gray (entre 0 et 1) et black, white, red, green, blue, cyan, magenta,
yellow% definecolor{0gris}{gray}{0.8} 
% Nouvelle commande pour encadrer le titre car shabox ne veut que d'une
seule ligne; ATTENTION A LA TAILLE; petite différence avec shadowbox ou
doublebox, voire fcolorbox ou colorbox (au lieu de shabox; laisser le
parbox tranquille sauf pour la taille de la boîte
\newcommand{\Tbox}[1]{\begin{center} \shabox{\parbox{0.6
\linewidth}{#1}} \end{center}} %[1] pour 1 paramètre ; #1 pour ce que
fait le 1er paramètre; entre accolades ce que fait la commande
%Mise en page en mode fancy : en-têtes et pieds de pages puis
définition des en-têtes et pieds de pages\pagestyle{fancy}
\lhead{ECE 2 - Mathématiques \\
Quentin Dunstetter - ENC-Bessières 2011$\backslash$2012}
\chead{}
\rhead{ESCP 2004}
\rfoot[ \ \thepage]{\thepage}
\cfoot{}
\lfoot{}
\thispagestyle{fancy} %Mise en page de la 1ère page en mode fancy
%Trait en bas et en haut de la page (entre en-tête et texte et texte et
pied de page)\renewcommand{\footrulewidth}{0.4pt}
\renewcommand{\headrulewidth}{0.4pt}


%DEBUT DU DOCUMENT\vspace*{3cm}

\begin{center}
{\LARG\E\textbf{BANQUE COMMUNE D'ÉPREUVES}}



{\large \textsc{CONCOURS D ADMISSION DE 2004}}



{\large \textbf{Concepteur : ESCP}}



\rule{2.39cm}{0.05cm}



{\Large \textbf{OPTION ÉCONOMIQUE}}



{\Large \textbf{MATHÉMATIQUES }}



{\Large Lundi 9 mai, de 14h à 18h}



\rule{2.39cm}{0.05cm}
\end{center}

\textit{La présentation, la lisibilité, l'orthographe, la qualité
de la rédaction, la clarté et la précision des raisonnements
entreront pour une part importante dans l'appréciation des copies.}

\textit{Les candidats sont invités à \textbf{encadrer} dans la mesure
du possible les résultats de leurs calculs.}

\textit{Ils ne doivent faire usage d'aucun document. L'utilisation de
toute
calculatrice et de tout matériel électronique est interdite. Seule
l'utilisation d'une règle graduée est autorisée.}

\textit{Si au cours de l'épreuve, un candidat repère ce qui lui semble
être une erreur d'énoncé, il la signalera sur sa copie et
poursuivra sa composition en expliquant les raisons des initiatives
qu'il sera
amené à prendre.}

\vspace*{3cm}

\section*{EXERCICE}

On désigne par $E$ l'espace vectoriel $\R^{6}$ et par $\mathcal{B}$
sa base canonique : \ $\mathcal{B} =
(e_{1},e_{2},e_{3},e_{4},e_{5},e_{6})$. 
\\
On pose \ $\mathcal{B}_{1} = (e_{1},e_{2},e_{3})$ et $\mathcal{B}_{2} =
(e_{4},e_{5},e_{6})$, et on désigne respectivement par $E_{1}$ et
$E_{2} $ les sous-espaces vectoriels de $E$ engendrés par
$\mathcal{B}_{1}$
et $\mathcal{B}_{2}$. \\
Enfin, $A$ est la matrice carrée d'ordre $3$ à coefficients réels
suivante : 
\[
\begin{smatrix}
0 & 2 & 1 \\
2 & 0 & 1 \\
-2 & 2 & -1
\end{smatrix}
\]

\begin{noliste}{1.}
 \setlength{\itemsep}{4mm}
\item Soit $u$ l'endomorphisme de $E_{1}$ dont la matrice dans la base
$\mathcal{B}_{1}$ est $A$.\\
Déterminer les valeurs propres de $u$ ainsi qu'une base de vecteurs
propres.

\item Soit $f$ l'application linéaire de $E_{1}$ vers $E_{2}$ définie
par :
\ $f(e_{1}) = e_{4},f(e_{2}) = e_{5}$\ et\ $f(e_{3}) = e_{6}$.\\
Montrer que $f$ est un isomorphisme et déterminer la matrice de son
isomorphisme réciproque $f^{-1}$ relativement aux bases
$\mathcal{B}_{2}$ et 
$\mathcal{B}_{1}$.

\item \label{q3}

\begin{noliste}{a)}
 \setlength{\itemsep}{2mm}
\item Montrer que, si $(x_{1},x_{2})$ est un élément de $E_{1}\times
E_{2}$ vérifiant l'égalité \ $x_{1} + x_{2} = 0$, les vecteurs $x_{1}$
et $x_{2}$ sont
nuls.

\item En déduire que, si $(x_{1},x_{2})$ et $(y_{1},y_{2})$ sont deux
éléments de $E_{1}\times E_{2}$ vérifiant l'égalité \\
$x_{1} + x_{2} = y_{1} + y_{2}$, alors on a : $x_{1} = y_{1}$ et $x_{2}
= y_{2}$.
\end{noliste}

\item Pour tout vecteur $x$ de $E$ dont les coordonnées dans la base
$\mathcal{B}$ sont $(\lambda_{1},\lambda_{2},\lambda_{3},\lambda
_{4},\lambda_{5},\lambda_{6})$, on pose : 
\[
\left\{ 
\begin{array}{c}
x_{1} = \lambda_{1}\,e_{1} + \lambda_{2}\,e_{2} + \lambda_{3}\,e_{3} \\
x_{2} = \lambda_{4}\,e_{4} + \lambda_{5}\,e_{5} + \lambda_{6}\,e_{6}
\end{array}
\right. \quad \text{et}\quad F(x) = u(x_{1}) + f(x_{1}) + f^{-1}(x_{2})
\]

\begin{noliste}{a)}
 \setlength{\itemsep}{2mm}
\item Prouver que l'application $F$ qui à tout vecteur $x$ de $E$
associe le
vecteur $F(x)$, est un endomorphisme de $E$.

\item Déterminer le noyau de $F$ et en déduire que $F$ est un
automorphisme.

\item Montrer que la matrice $M$ de $F$ dans la base $\mathcal{B}$ peut
s'écrire sous la forme : 
\[
M = 
\begin{smatrix}
0 & 2 & 1 & 1 & 0 & 0 \\
2 & 0 & 1 & 0 & 1 & 0 \\
-2 & 2 & -1 & 0 & 0 & 1 \\
1 & 0 & 0 & 0 & 0 & 0 \\
0 & 1 & 0 & 0 & 0 & 0 \\
0 & 0 & 1 & 0 & 0 & 0
\end{smatrix}
\]
\end{noliste}

\item On suppose, dans cette question, que $\mu $ est une valeur propre
de $F $ et que $x$ est un vecteur propre associé à $\mu $; on définit
les
vecteurs $x_{1}$ de $E_{1}$ et $x_{2}$ de $E_{2}$ comme dans la
question précédente.

\begin{noliste}{a)}
 \setlength{\itemsep}{2mm}
\item Justifier que la valeur propre $\mu $ n'est pas nulle.

\item Utiliser les résultats de la question \ref{q3} pour prouver que
les
vecteurs $x_{1}$ et $x_{2}$ sont tous les deux non nuls et que $x_{1}$
est
un vecteur propre de $u$ associé à la valeur propre $\mu -\dfrac{1}{\mu
}.$
\end{noliste}

\item Étudier la fonction $\varphi $ définie sur $\R^{\times }$ par
$\varphi (x) = x-\dfrac{1}{x}$ et en donner une représentation
graphique.

\item On suppose, dans cette question, que $\lambda $ est une valeur
propre
de $u$ et que $x_{1}$ est un vecteur propre de $u$ associé à $\lambda
$.

\begin{noliste}{a)}
 \setlength{\itemsep}{2mm}
\item Montrer que l'équation d'inconnue $\mu $ suivante : \ $\lambda =
\mu -\dfrac{1}{\mu }$ admet deux solutions distinctes $\mu_{1}$ et
$\mu_{2}$.

\item Montrer que $\mu_{1}$ et $\mu_{2}$ sont des valeurs propres de
$F$. 
\\
Donner, en fonction de $x_{1}$, un vecteur propre de $F$ associé à
$\mu_{1}$
et un vecteur propre de $F$ associé à $\mu_{2}$.
\end{noliste}

\item La matrice $M$ est-elle diagonalisable ?
\end{noliste}

\section*{PROBLEME}

Dans tout le problème, $r$ désigne un entier naturel vérifiant\ $1\leq
r\leq 10$. Une urne contient $10$ boules distinctes
$B_{1},B_{2},\ldots,B_{10}$. Une expérience aléatoire consiste à y
effectuer
une suite de tirages d'une boule \textbf{avec remise }, chaque boule
ayant
la même probabilité de sortir à chaque tirage. Cette expérience est
modélisée par un espace probabilisé $(\Omega,\mathcal{A},\mathbf{P})$.

\subsection*{Partie I : Étude du nombre de tirages nécessaires pour
obtenir
au moins une fois chacune des boules $B_{1},\ldots,B_{r}$}

On suppose que le nombre de tirages nécessaires pour obtenir au moins
une
fois chacune des boules $B_{1},\ldots,B_{r}$ définit une variable
aléatoire 
$Y_{r}$ sur $(\Omega,\mathcal{A},\mathbf{P})$.

\begin{noliste}{1.}
 \setlength{\itemsep}{4mm}
\item Cas particulier $r = 1$. \\
Montrer que la variable aléatoire $Y_{1}$ suit une loi géométrique;
préciser
son paramètre, son espérance et sa variance.

\item On suppose que $r$ est supérieur ou égal à $2$.

\begin{noliste}{a)}
 \setlength{\itemsep}{2mm}
\item Calculer la probabilité pour que les $r$ boules
$B_{1},B_{2},\ldots,B_{r}$ sortent dans cet ordre aux $r$ premiers
tirages.

\item En déduire la probabilité $\mathbf{P}([Y_{r} = r])$.

\item Préciser l'ensemble des valeurs que peut prendre la variable
aléatoire 
$Y_{r}$.
\end{noliste}

\item On suppose encore que $r$ est supérieur ou égal à $2$. Pour tout
entier $i$ vérifiant $1\leq i\leq r$, on désigne par $W_{i}$ la
variable aléatoire représentant le nombre de tirages nécessaires pour
que,
pour la première fois, $i$ boules distinctes parmi les boules
$B_{1},B_{2},\ldots,B_{r}$ soient sorties (en particulier, on a :
$W_{r} = Y_{r}$).\\
On pose : \quad $X_{1} = W_{1}$ \quad et, pour tout $i$ vérifiant
$2\leq
i\leq r$, \ $X_{i} = W_{i}-W_{i-1}$. \\
On admet que les variables aléatoires $X_{1},\ldots,X_{r}$ sont
indépendantes.

\begin{noliste}{a)}
 \setlength{\itemsep}{2mm}
\item Exprimer la variable aléatoire $Y_{r}$ à l'aide des variables
aléatoires $X_{1},\ldots,X_{r}$.

\item Interpréter concrètement la variable aléatoire $X_{i}$ pour tout
$i$ vérifiant $1\leq i\leq r$.

\item Montrer que, pour tout $i$ vérifiant $1\leq i\leq r$, la
variable aléatoire $X_{i}$ suit une loi géométrique; préciser son
espérance
et sa variance.

\item On pose : $S_{1}(r) = \Sum{k = 1}{r}\dfrac{1}{k}$ \quad et \quad
$S_{2}(r) = \Sum{k = 1}{r}\dfrac{1}{k^{2}}$\\
Exprimer l'espérance $\mathbf{E}(Y_{r})$ et la variance
$\mathbf{V}(Y_{r})$
de $Y_{r}$ à l'aide de $S_{1}(r)$ et de $S_{2}(r)$.
\end{noliste}

\item 

\begin{noliste}{a)}
 \setlength{\itemsep}{2mm}
\item Si $k$ est un entier naturel non nul, préciser le minimum et le
maximum de la fonction $t\mapsto \dfrac{1}{t}$ sur l'intervalle $[k,k +
1]$ et
en déduire un encadrement de l'intégrale $\dint{k}{k + 1}\dfrac{1}{t}dt
$.

\item Si $r$ est supérieur ou égal à $2$, donner un encadrement de
$S_{1}(r)$
et en déduire la double inégalité :
\[
10\ln (r + 1)\leq \mathbf{E}(Y_{r})\leq 10(\ln r + 1)
\]

\item Si $r$ supérieur ou égal à $2$, établir par une méthode analogue
à
celle de la question précédente, la double inégalité :
\[
1-\dfrac{1}{r + 1}\leq S_{2}(r)\leq 2-\dfrac{1}{r}
\]
En déduire un encadrement de $\mathbf{V}(Y_{r})$.
\end{noliste}
\end{noliste}

\subsection*{Partie II : Étude du nombre de boules distinctes parmi les
boules $B_{1},B_{2},\ldots,B_{r}$ tirées au moins une fois au cours des
$n$
premiers tirages}

Pour tout entier $n$ supérieur ou égal à $1$, on suppose que le nombre
de
boules distinctes parmi les boules $B_{1},B_{2},\ldots,B_{r}$ tirées au
moins une fois au cours des $n$ premiers tirages, définit une variable
aléatoire $Z_{n}$ sur $(\Omega,\mathcal{A},\mathbf{P})$; on note
$\mathbf{E}(Z_{n})$ l'espérance de $Z_{n}$ et on pose $Z_{0} = 0$.\\
Pour tout entier naturel $n$ non nul et pour tout entier naturel $k$,
on
note $p_{n,k}$ la probabilité de l'évènement $[Z_{n} = k]$ et on pose :
$p_{n,-1} = 0$.

\begin{noliste}{1.}
 \setlength{\itemsep}{4mm}
\item Étude des cas particuliers $n = 1$ et $n = 2$.

\begin{noliste}{a)}
 \setlength{\itemsep}{2mm}
\item Déterminer la loi de $Z_{1}$ et donner son espérance.

\item On suppose, dans cette question, que $r$ est supérieur ou égal à
$2$. 
\\
Déterminer la loi de $Z_{2}$ et montrer que son espérance est donnée
par :
\quad $\mathbf{E}(Z_{2}) = \dfrac{19\,r}{100}$
\end{noliste}

\item Établir, pour tout entier naturel $n$ non nul et pour tout entier
naturel $k$ au plus égal à $r$, l'égalité : 
\begin{equation}
10\,p_{n,k} = (10-r + k)p_{n-1,k} + (r + 1-k)p_{n-1,k-1}
\label{recurrence pn,k}
\end{equation}Vérifier que cette égalité reste vraie dans le cas où $k$
est supérieur ou égal à $r + 1$.

\item Pour tout entier naturel non nul $n$, on définit le polynôme
$Q_{n}$
par :\ pour tout réel $x$, 
\[
Q_{n}(x) = \Sum{k = 0}{n}p_{n,k}x^{k},\quad \text{et on pose}\quad \
Q_{0}(x) = 1.
\]

\begin{noliste}{a)}
 \setlength{\itemsep}{2mm}
\item Préciser les polynômes $Q_{1}$ et $Q_{2}$.

\item Calculer $Q_{n}(1)$ et exprimer $Q_{n}{\prime }(1)$ en fonction
de $\mathbf{E}(Z_{n})$ ($Q_{n}{\prime }$ désignant la dérivée du
polynôme $Q_{n} $).

\item En utilisant l'égalité (\ref{recurrence pn,k}), établir, pour
tout réel $x$ et pour tout entier naturel $n$ non nul, la relation
suivante : 
\begin{equation}
10Q_{n}(x) = (10-r + rx)\,Q_{n-1}(x) + x(1-x)Q_{n-1}{\prime }(x)
\label{recurrence Qn}
\end{equation}

\item En dérivant membre à membre l'égalité (\ref{recurrence Qn}),
former,
pour tout entier naturel $n$ non nul, une relation entre les espérances
$\mathbf{E}(Z_{n})$ et $\mathbf{E}(Z_{n-1})$.\\
En déduire, pour tout entier naturel $n$, la valeur de
$\mathbf{E}(Z_{n})$
en fonction de $n$ et de $r$.
\end{noliste}

\item 

\begin{noliste}{a)}
 \setlength{\itemsep}{2mm}
\item Pour tout entier naturel $n$, le polynôme $Q_{n}{\prime \prime }$
désigne la dérivée du polynôme $Q_{n}{\prime }$.\\
En utilisant une méthode semblable à celle de la question précédente,
trouver pour tout entier naturel $n$ non nul, une relation entre
$Q_{n}{\prime \prime }(1)$ et $Q_{n-1}{\prime \prime }(1)$.\\
En déduire que, pour tout entier naturel $n$ non nul, l'égalité
suivante : 
\[
Q_{n}{\prime \prime }(1) = r(r-1)\left[ 1 + \left( \frac{8}{10}\right)
^{n}-2\left( \frac{9}{10}\right) ^{n}\right]
\]

\item Calculer, pour tout entier naturel $n$, la variance de la
variable aléatoire $Z_{n}$ en fonction de $n$ et de $r$.
\end{noliste}
\end{noliste}

\label{fin}

\end{document}

