\documentclass[11pt]{article}%
\usepackage{geometry}%
\geometry{a4paper,
 lmargin = 2cm,rmargin = 2cm,tmargin = 2.5cm,bmargin = 2.5cm}

\input{../../macros.tex}

\pagestyle{fancy} %
\lhead{ECE2 \hfill Mathématiques\\
} %
\chead{\hrule} %
\rhead{} %
\lfoot{} %
\cfoot{} %
\rfoot{\thepage} %

\renewcommand{\headrulewidth}{0pt}% : Trace un trait de séparation
 % de largeur 0,4 point. Mettre 0pt
 % pour supprimer le trait.

\renewcommand{\footrulewidth}{0.4pt}% : Trace un trait de séparation
 % de largeur 0,4 point. Mettre 0pt
 % pour supprimer le trait.

\setlength{\headheight}{14pt}

\title{\bf \vspace{-2cm} ESCP 1992} %
\author{} %
\date{} %
\begin{document}

\maketitle %
\vspace{-1.4cm}\hrule %
\thispagestyle{fancy}

\vspace*{.2cm}


% DEBUT DU DOC À MODIFIER : tout virer jusqu'au début de l'exo
\begin{center}
\textbf{\Large ESCP}

\textsc{\large CONCOURS D'ADMISSION DE 1992}

\textbf{voie économique}


\end{center}

\hrule


\section*{Exercice~1}
On considère un élément $M$ de l'ensemble $\M{2}$ des matrices carrées
d'ordre $2$. On suppose que $M$ satisfait aux deux conditions
suivantes~ :
\begin{noliste}{1.}
 \setlength{\itemsep}{4mm}[i)]
\item $M^{2} + M = 2I$, où $I$ désigne la matrice unité de $\M{2}$ ;
\item $M$ n'est pas de la forme $cI$, où $c$ est un nombre réel.
\end{noliste}
On note $V_{M}$ l'ensemble des éléments $Z$ de $\M{2}$
de la forme $Z = xM + yI$, où $x$ et $y$ sont des nombres réels.
\begin{noliste}{1.}
 \setlength{\itemsep}{4mm}
\item Dans cette question, on suppose que $M$ est diagonalisable.
Déterminer
ses valeurs propres. Donner un exemple de matrice satisfaisant aux 
conditions i) et ii).
\item Soit $Z$ un élément de $V_{M}$. Montrer qu'il y a unicité du
couple
$(x,y)$ tel que $Z = xM + yI$. On dira désormais que ce couple est
associé
à $Z$.
\item Montrer que si $Z$ et $Z'$ appartiennent à $V_{M}$, il en est de
même
pour le produit $ZZ'$.
\item Montrer que la matrice $M$ possède une inverse qui appartient 
à $V_{M}$, et déterminer le couple $(x,y)$ associé à l'inverse de $M$.
\item 
\begin{noliste}{a)}
 \setlength{\itemsep}{2mm}
\item Soit $A = \alpha M + \beta I$ un élément de $V_{M}$.
Exprimer en fonction du couple $(\alpha, \beta )$ associé à $A$ le
couple
$(\alpha', \beta' )$ associé au produit $MA$. Montrer qu'il existe un
élément
$\Delta$ de $\M{2}$, que l'on déterminera, tel que~ :
\[
 \ \begin{smatrix}
 \alpha' \\
\beta'
\end{smatrix}
 = \Delta
\begin{smatrix}
 \alpha \\
\beta
\end{smatrix}.
\]
\item Déterminer les valeurs propres et une base de vecteurs propres de
la
matrice $\Delta$.
\item Soit $n$ un nombre entier naturel. Déduire de ce qui précède le
couple $(x_{n},y_{n})$ associé à $M^{n}$.
\end{noliste}
\item Trouver (en déterminant les couples de nombres réels associés)
toutes les matrices de $V_{M}$ égales à leur carré.
\end{noliste}

\subsection*{Exercice~2}
Pour tout nombre entier naturel $k$, on considère la fonction $f_{k}$
définie sur $\R_+ $
par la relation~ :
\[
f_{k}(x) = \dint{0}{1} t^{k} e^{-tx} \ \ dt.
\]
\begin{noliste}{1.}
 \setlength{\itemsep}{4mm}
 \item 
\begin{noliste}{a)}
 \setlength{\itemsep}{2mm}
 \item Montrer que, pour tout nombre entier naturel $k$, la fonction
$f_{k}$ est décroissante sur $\R_+ $. 
\item Étudier la suite $\left( f_{k}(0) \right)_{k \geq 0} $ de nombres
réels.
En déduire, pour tout nombre réel positif fixé $x$, la limite de la
suite 
$\left( f_{k}(x) \right)_{k \geq 0}$.
\end{noliste}
\item 
\begin{noliste}{a)}
 \setlength{\itemsep}{2mm}
\item \label{quesa} Soit $x$ un nombre réel strictement positif.
Établir une relation
entre $f_{k}(x)$ et $f_{k + 1} (x)$.
\item Expliciter les fonction $f_{0}$, $f_{1}$ et $f_{2}$.
\item Montrer que, lorsque $x$ tend vers $ + \infty$~ : 
\[
f_{0} (x) \sim \frac{1}{x}.
\]
\item À l'aide de la relation établie au \ref{quesa}), montrer que, 
pour tout nombre entier naturel $k$, lorsque $x$ tend vers
$ + \infty$ :
\[
f_{k} (x) \sim \frac{A_{k}}{x^{k + 1}}
\]
où $A_{k}$ est une constante que l'on déterminera.
\end{noliste}
\item 
\begin{noliste}{a)}
 \setlength{\itemsep}{2mm}
 \item Montrer que, pour tout nombre entier naturel $k$ et
pour tout réel strictement positif $x$~ :
\[
f_{k}(x) = \frac{1}{x^{k + 1}}\dint{0}{x} u^{k} e^{-u} \, du.
\]
En déduire que $f_{k}$ est dérivable sur $]0, + \infty[$ et calculer sa
dérivée.
\item Trouver une relation simple entre $f'_{k}$ et $f_{k + 1}$.
\end{noliste}
\item Montrer que pour tout nombre réel $y$ positif ou nul : 
\[
1 - e^{-y} \leq y.
\]
En déduire que, pour tout nombre entier naturel $k$, la fonction
$f_{k}$ est continue en $0$.
Est-elle dérivable à droite en ce point~ ?
\end{noliste}
%%%%%%%%%%%%%%%%%%%%%%%\section*{Exercice 3}
Le président d'un club de football de première division veut engager un
joueur J. Les données statistiques
recueillies lors des saisons précédentes permettent de formuler les
hypothèses suivantes :
\begin{noliste}{$\sbullet$}
	\item le nombre $N$ d'occasions de but qu'aura J pendant un match
donné est une variable aléatoire de Poisson
de paramètre $5$ ;
\item à chaque occasion de but qu'aura J, la probabilité pour qu'il
marque effectivement le but est $0{,}2$ ;
\item les variables aléatoires relatives à des matchs différents sont
indépendantes.
\end{noliste}
\begin{noliste}{1.}
 \setlength{\itemsep}{4mm}
	\item 
\begin{noliste}{a)}
 \setlength{\itemsep}{2mm}
	\item Trouver le nombre moyen d'occasions de but qu'aura J pendant un
match.
\item Pour tout nombre entier naturel $n$, on note $p_{n}$ la
probabilité que J ait $n$ occasions de but au cours
d'un match donné. 
Calculer le rapport $\frac{p_{n + 1}}{p_{n}}$.
En déduire les nombres les plus probables d'occasions de but
que J aura pendant un match donné.
\end{noliste}
\item On note $N$ la variable aléatoire égale au nombre de buts marqués
par J pendant un match. Trouver
la loi de $X$, donner son espérance et sa variance.
\item Pour le Championnat de France de première division, chacune des
$20$ équipes de cette division joue
deux matchs (aller-retour) contre chacune des autres équipes. On
suppose que J joue tous les matchs du
Championnat avec son équipe.
\begin{noliste}{a)}
 \setlength{\itemsep}{2mm}
	\item 	Déterminer le nombre de matchs que J jouera avec son équipe
dans le Championnat de France.
 \item On note $T$ la variable aléatoire égale an nombre de buts
marqués par J pendant le Championnat.
Trouver la loi de T ; donner son espérance et sa variance.
\item On admet que la loi de T peut être approchée par la loi de
Laplace-Gauss de moyenne $38$ et de
variance $38$. Donner une valeur approchée de la probabilité que J
marque au moins $50$ buts pendant le
Championnat. (On utilisera la table de la loi normale donnée en annexe,
page 4.)
\end{noliste}
\item L'équipe de J participe également à la Coupe de France, tournoi
mettant aux prises $64$ équipes engagées au départ. À chaque tour,
chaque équipe rencontre une autre équipe tirée au sort, et l'équipe
perdante
est éliminée.

	On suppose que J joue tous les matchs de la Coupe auxquels son équipe
participe.
	
	On suppose qu'à chaque tour l'équipe de J a la même probabilité, égale
à $\frac{2}{3}$, de gagner son match.
	\begin{noliste}{a)}
 \setlength{\itemsep}{2mm}
	\item Trouver la loi de la variable aléatoire $Y$ égale au nombre de
matchs disputés par J en Coupe de
France, et l'espérance de $Y$.
\item Calculer la probabilité que $J$ marque, en tout, deux buts en
Coupe de France.
\item Calculer la probabilité conditionnelle que l'équipe J gagne la
Coupe de France sachant que $J$ a marqué en tout deux buts pendant
cette Coupe.
\end{noliste}
\end{noliste}
\end{document}