\documentclass[11pt]{article}%
\usepackage{geometry}%
\geometry{a4paper,
 lmargin = 2cm,rmargin = 2cm,tmargin = 2.5cm,bmargin = 2.5cm}

\input{../../macros.tex}

\pagestyle{fancy} %
\lhead{ECE2 \hfill Mathématiques\\
} %
\chead{\hrule} %
\rhead{} %
\lfoot{} %
\cfoot{} %
\rfoot{\thepage} %

\renewcommand{\headrulewidth}{0pt}% : Trace un trait de séparation
 % de largeur 0,4 point. Mettre 0pt
 % pour supprimer le trait.

\renewcommand{\footrulewidth}{0.4pt}% : Trace un trait de séparation
 % de largeur 0,4 point. Mettre 0pt
 % pour supprimer le trait.

\setlength{\headheight}{14pt}

\title{\bf \vspace{-2cm} ESCP 2000} %
\author{} %
\date{} %
\begin{document}

\maketitle %
\vspace{-1.4cm}\hrule %
\thispagestyle{fancy}

\vspace*{.2cm}


% DEBUT DU DOC À MODIFIER : tout virer jusqu'au début de l'exo

%Définition et changement de valeurs de
compteurs%newcounter{cpt1}{section} compteur cpt1 remis à 0 à chaque
aumentation par stepcounter du compteur section%setcounter{cpt1}{3} on
met le compteur à 3%addtocounter{cpt1}{5} on ajoute 5 au compteur%
stepcounter{cpt1} on ajoute 1% ifthenelse{test}{alors}{sinon} (page
206) pour subordonner à une condition % whiledo{test}{commande} pour
faire une boucle (page 206 aussi) % value{cpt1} pour noter dans le
document la valeur de cpt1 
%Définition définitive d'opérateurs
mathématiques\newcommand{\ch}{\operatorname{ch}} 
\newcommand{\sh}{\operatorname{sh}}
\renewcommand{\tanh}{\operatorname{th}}
\renewcommand{\sinh}{\operatorname{sh}}
\renewcommand{\cosh}{\operatorname{ch}}
\newcommand{\argsh}{\operatorname{argsh}}
\newcommand{\argch}{\operatorname{argch}}
\newcommand{\argth}{\operatorname{argth}}
\newcommand{\ker}{\operatorname{Ker}}
\renewcommand{\im}{\operatorname{Im}}
\newcommand{\rg}{\operatorname{rg}}
\newcommand{\Id}{\operatorname{Id}}
\newcommand{\id}{\operatorname{id}}
\renewcommand{\leq}{\leq}
\renewcommand{\geq}{\geq }

%Définition de nouvelles couleurs : rgb(trois paramètres red green blue
entre 0 et 1); cmyk (quatre cyan magenta yellow black) entre 0 et 1;
gray (entre 0 et 1) et black, white, red, green, blue, cyan, magenta,
yellow% definecolor{0gris}{gray}{0.8} 
% Nouvelle commande pour encadrer le titre car shabox ne veut que d'une
seule ligne; ATTENTION A LA TAILLE; petite différence avec shadowbox ou
doublebox, voire fcolorbox ou colorbox (au lieu de shabox; laisser le
parbox tranquille sauf pour la taille de la boîte
\newcommand{\Tbox}[1]{\begin{center} \shabox{\parbox{0.6
\linewidth}{#1}} \end{center}} %[1] pour 1 paramètre ; #1 pour ce que
fait le 1er paramètre; entre accolades ce que fait la commande
%Mise en page en mode fancy : en-têtes et pieds de pages puis
définition des en-têtes et pieds de pages\pagestyle{fancy}
\lhead{ECE 2 - Mathématiques \\
Quentin Dunstetter - ENC-Bessières 2011$\backslash$2012}
\chead{}
\rhead{ESCP 2000}
\rfoot[ \ \thepage]{\thepage}
\cfoot{}
\lfoot{}
\thispagestyle{fancy} %Mise en page de la 1ère page en mode fancy
%Trait en bas et en haut de la page (entre en-tête et texte et texte et
pied de page)\renewcommand{\footrulewidth}{0.4pt}
\renewcommand{\headrulewidth}{0.4pt}


%DEBUT DU DOCUMENT\vspace*{3cm}

\begin{center}
{\LARG\E\textbf{BANQUE COMMUNE D'ÉPREUVES}}



{\large \textsc{CONCOURS D ADMISSION DE 2000}}



{\large \textbf{Concepteur : ESCP}}



\rule{2.39cm}{0.05cm}



{\Large \textbf{OPTION ÉCONOMIQUE}}



{\Large \textbf{MATHÉMATIQUES }}



{\Large Lundi 9 mai, de 14h à 18h}



\rule{2.39cm}{0.05cm}
\end{center}

\textit{La présentation, la lisibilité, l'orthographe, la qualité
de la rédaction, la clarté et la précision des raisonnements
entreront pour une part importante dans l'appréciation des copies.}

\textit{Les candidats sont invités à \textbf{encadrer} dans la mesure
du possible les résultats de leurs calculs.}

\textit{Ils ne doivent faire usage d'aucun document. L'utilisation de
toute
calculatrice et de tout matériel électronique est interdite. Seule
l'utilisation d'une règle graduée est autorisée.}

\textit{Si au cours de l'épreuve, un candidat repère ce qui lui semble
être une erreur d'énoncé, il la signalera sur sa copie et
poursuivra sa composition en expliquant les raisons des initiatives
qu'il sera
amené à prendre.}

\vspace*{3cm}

\section*{EXERCICE I}

Dans tout l'exercice, $\alpha $ désigne un paramètre réel. On considère
la
matrice $A_{\alpha } = 
\begin{smatrix}
-1 & 2-\alpha & -\alpha \\
-\alpha & 1 & -\alpha \\
2 & \alpha -2 & \alpha + 1
\end{smatrix}
$ et on note $\phi_{\alpha }$ l'endomorphisme de $\R^{3}$ représenté
par $A_{\alpha }$ dans la base canonique de $\R^{3.}$

\begin{noliste}{1.}
 \setlength{\itemsep}{4mm}
\item 

\begin{noliste}{a)}
 \setlength{\itemsep}{2mm}
\item Montrer que, quel que soit $\alpha $, l'endomorphisme
$\phi_{\alpha }$
admet la valeur propre $1$.

\item On note $E_{1}(\alpha )$ le sous-espace propre de $\phi_{\alpha
}$
associé à la valeur propre $1$. \\
Déterminer, suivant les valeurs de a, une base de $E_{1}(\alpha )$.
\end{noliste}

\item On considère les vecteurs $f_{1} = (1,1,-1)$ et $f_{2} =
(1,1,-2)$ et on
note $F_{1}$ le sous-espace de $\R^{3}$ engendré par $f_{1}$ et
$f_{2}$.

\begin{noliste}{a)}
 \setlength{\itemsep}{2mm}
\item Montrer que $(f_{l},f_{2})$ est une base de $F_{l}$.

\item Montrer que l'image par $\phi_{\alpha }$ de tout vecteur de
$F_{1}$
appartient à $F_{1}$.

\item Soit$\hat{\phi}_{\alpha }$ l'endomorphisme de $F_{1}$ induit par
$\phi
_{\alpha }$, c'est-à-dire vérifiant,.pour tout vecteur $V$ de $F_{1}$,
\[
\hat{\phi}_{\alpha }(V) = \phi_{\alpha }(V)
\]
Donner la matrice de $\hat{\phi}_{\alpha }$ dans la base
$(f_{l},f_{2})$de $F_{1}$.
\end{noliste}

\item Montrer que, pour tout réel $\alpha $, l'endomorphisme
$\phi_{\alpha
} $ admet la valeur propre $\alpha -1$ et qu'on peut trouver un vecteur
$f_{3}$de $\R^{3}$ ne dépendant pas de $\alpha $, qui soit, pour tout
réel $\alpha $, vecteur propre de $\phi_{\alpha }$ associé à la valeur
propre $\alpha -1$.

\begin{noliste}{a)}
 \setlength{\itemsep}{2mm}
\item Montrer que $(f_{l,}f_{2},f_{3})$ est une base de $\R^{3}$.
Donner la matrice de $\phi_{\alpha }$ dans cette base.

\item Pour quelles valeurs du paramètre $\alpha $ l'endomorphisme $\phi
_{\alpha }$ est-il diagonalisable ?
\end{noliste}
\end{noliste}

\section*{EXERCICE II}

\subsection*{I. Étude d'une suite vérifiant une relation de récurrence
linéaire}

Etant donné un paramètre réel $\alpha >0$, on note $E$ l'espace
vectoriel
des suites $U = (u_{n})_{n\geq 0}$ de réels qui vérifient, pour tout n
positif, la relation

\[
u_{n + 2} = \alpha (u_{n + 1} + u_{n})
\]

\begin{noliste}{1.}
 \setlength{\itemsep}{4mm}
\item Montrer qu'on peut trouver deux réels $r$ et $s$, avec $r<s$,
tels que
les suites $R = (r^{n})_{n\geq 0}$ et $S = (s^{n})_{n\geq 0}$ forment
une base de l'espace vectoriel $E.$ Exprimer $r$ et $s$ en fonction de
$\alpha $ et comparer $\left| r\right| $ et $\left| s\right|. $

\item Etant donné un élément $U = (u_{n})_{n\geq 0}$ de $E $ s'écrivant
$U = aR + bS$ avec $(a,b)\in \R^{2}$, donner l'expression de $a$ et $b$
en fonction de $u_{0}$ et $u_{1}.$

\item On suppose, dans cette question, que l'on a $0<a<\dfrac{1}{2}.$ 
\\
Soit $U = (u_{n})_{n\geq 0}$un élément de $E$.

\begin{noliste}{a)}
 \setlength{\itemsep}{2mm}
\item Montrer que la suite $U$ converge vers $0$.

\item Si $u_{1}-u_{0}r$ n'est pas nul, montrer qu'il existe un indice
$n_{0}$
tel que, pour $n>n_{0}$, $u_{n}$ ne s'annule pas et garde un signe
constant
et que l'on a :
\[
\underset{n\rightarrow + \infty }{\lim }\dfrac{\ln \left|
u_{n}\right| }{n} = \ln (s)
\]

\item Montrer que si, au contraire, $u_{1}-u_{0}r$ est nul et si la
suite $(u_{n})_{n\geq 0}$ n'est pas identiquement nulle, alors, pour
tout
entier $n$ positif, $u_{n}$ et $u_{n + l}$ sont de signes contraires.
Quel équivalent peut-on donner, dans ce cas, de$\ln \left| u_{n}\right|
$ ?
\end{noliste}

\item On suppose, dans cette question, que l'on a $\dfrac{1}{2}<a.$\\
A quelle condition sur $u_{0}$ et $u_{1}$ l'élément $U = (u_{n})_{n\geq

0} $ de $E$ est-il une suite bornée ? Montrer que les éléments de E qui
sont
des suites bornées forment un sous-espace vectoriel de E dont on
précisera
la dimension.
\end{noliste}

\subsection*{II. Étude d'une récurrence non linéaire}

Soit $\beta $ un réel strictement positif. On note $m = \min (l,\beta
)$ le
plus petit des nombres $1$ et $\beta $ et $M = \max (l,\beta )$ le plus
grand
de ces nombres.\\
On considère la suite $V = (v_{n})_{n\geq 0}$ vérifiant $v_{0} = 0$,
$v_{1} = \beta $ et, pour tout $n$ positif, la relation
\[
v_{n + 2} = \sqrt{v_{n + 1}} + \sqrt{v_{n}}
\]

\begin{noliste}{1.}
 \setlength{\itemsep}{4mm}
\item Montrer, pour tout $n$ strictement positif, l'inégalité $m\leq
v_{n}\leq 4M.$

\item Montrer que si la suite $V = (v_{n})_{n\geq 0}$ admet une limite,
cette limite est nécessairement égale à $4$.\\
On se propose de montrer que, pour tout $\beta $ strictement positif,
la
suite $V$ admet effectivement pour limite $4$.

\item Montrer, pour tout $n$ positif, l'inégalité
\[
\left| v_{n + 2}-4\right| \leq \dfrac{\left|
v_{n + 1}-4\right| }{\sqrt{v_{n + 1}} + 2} + \dfrac{\left|
v_{n}-4\right| }{\sqrt{v_{n}} + 2}
\]

\item On pose $\alpha = \dfrac{1}{\sqrt{m} + 2}$ et on considère la
suite U = (u$_{n})_{n\geq 0}$ vérifiant la relation de récurrence
linéaire 
\[
u_{n + 2} = \alpha (u_{n + 1} + u_{n})
\]
et les conditions initiales, $u_{0} = \left| v_{1}-4\right| $ et $u_{1}
= \left| v_{2}-4\right| $. \\
Montrer que, pour tout n strictement positif $\left| v_{n}-4\right|
\leq u_{n-1}$\textbf{.}

\item En conclusion, montrer à l'aide des résultats de la première
partie
que la suite $V$ converge vers $4$.

\item Écrire un programme -\Scilab{} qui lise un entier $N$ et un réel
$\beta $ et qui affiche, en sortie, les $N$ premiers termes de la suite
$V$.
\end{noliste}

\section*{EXERCICE III}

Sachant qu'un appareil a fonctionné correctement pendant une certaine
durée
x, on s'intéresse à la probabilité pour qu'il continue à bien
fonctionner
pendant encore au moins une durée y. Pour cela on convient de
représenter la
durée de vie de ce type d'appareil par une variable aléatoire réelle
$X$ définie sur un espace probabilisé dont on notera la probabilité P.
L'exercice
a pour objet l'étude de quelques fonctions liées à cette durée de vie.

\subsection*{I.}

On suppose d'abord que $X$ prend ses valeurs dans $\N^{\times }$ et
que, pour tout $n$ de $\N^{\times }$, $P\left(\Ev{X = n}\right)$ n'est
pas nul. On
pose; pour tout $n$ de $\N^{\times }$,
\[
p_{n} = P\left(\Ev{X = n}\right),\quad G_{n} = P\left(\Ev{X\geq
n}\right) = \Sum{k = n}{\infty }p_{k}\text{\quad et}\quad Z_{n} =
\dfrac{{p_{n}}}{{G_{n}}}
\]

\begin{noliste}{1.}
 \setlength{\itemsep}{4mm}
\item Justifier les inégalités $0<p_{n}<G_{n}\leq 1$ et $0<Z_{n}<1$.

\item Soit $n$ un entier naturel. Établir l'égalité $P\left(\Ev{X\geq 
n + 1/X\geq n}\right) = 1-Z_{n}$.

\begin{noliste}{a)}
 \setlength{\itemsep}{2mm}
\item Montrer que la suite $(P\left(\Ev{X\geq n + 1/X\geq
n}\right))_{n\in \N^{\times }}$ est constante si et seulement si la
suite $(Z_{n})_{n\in 
\N^{\times }}$. est constante.

\item Vérifier que les conditions précédentes sont réalisées dans le
cas où
la loi de $X$ est une loi géométrique.

\item Réciproquement, on suppose qu'il existe une constante p
appartenant à $]0,1[$ telle que la suite $(Z_{n})_{n\in \N^{\times }}$
soit la
suite constante égale à $p$. Montrer par récurrence que $X$ suit une
loi géométrique.
\end{noliste}

\item Montrer que si, pour tout entier $m$ de $\N^{\times }$, la
suite $\left( \dfrac{p_{n + m}}{p_{n}}\right)_{n\in \N^{\times }}$est
décroissante, alors la suite $(Z_{n})_{n\in \N^{\times }}$ est
croissante et la suite $(P\left(\Ev{X\geq n + 1/X\geq n}\right))_{n\in
\N^{\times }}$ est décroissante. (On dit alors qu'il y a vieillissement
de
l'appareil dont $X$ est la durée de vie.)
\end{noliste}

\subsection*{II.}

On suppose maintenant que la variable aléatoire $X$ prend ses valeurs
dans $\R$\textbf{*}$_{+}$ et admet une densité f continue et
strictement
positive sur $\R_{+}{\times }$, On pose, pour tout réel strictement
positif $x$,
\[
G(x) = \dint{x}{+ \infty }f(t)dt\quad \text{et\quad }Z(x) =
\dfrac{f(x)}{G(x)}
\]

\begin{noliste}{1.}
 \setlength{\itemsep}{4mm}
\item 

\begin{noliste}{a)}
 \setlength{\itemsep}{2mm}
\item Si x et y sont des réels strictement positifs, on pose$H(x,y) =
\dfrac{G(x + y)}{G(x)}$. \\
Montrer que l'on a alors, pour tout couple $(x,y)$ de $\R_{+}{\times
}$, l'égalité :
\[
\dfrac{\partial H}{\partial x}(x,y) = \dfrac{G(x + y)}{G(x)}(Z(x)-Z(x +
y))
\]

\item Montrer que la fonction $x\rightarrow Z(x)$ est une fonction
croissante sur $\R_{+}{\times }$ si et seulement si, pour tout réel 
$y$ strictement positif fixé, la fonction $x\rightarrow
P\left(\Ev{X\geq 
x + y/X\geq x}\right)$ est une fonction décroissante.
\end{noliste}

\item 

\begin{noliste}{a)}
 \setlength{\itemsep}{2mm}
\item Montrer que si la loi de $X$ est une loi exponentielle, alors la
fonction $Z$ est constante.

\item Réciproquement, montrer que si $Z$ est la fonction constante
égale au réel strictement positif $\lambda $, alors la fonction
$x\rightarrow
e^{\lambda x}G(x)$ est constante. Quelle est alors la loi de $X$ ?
\end{noliste}
\end{noliste}

\label{fin}

\end{document}

