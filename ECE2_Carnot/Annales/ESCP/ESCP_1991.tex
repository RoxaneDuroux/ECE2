\documentclass[11pt]{article}%
\usepackage{geometry}%
\geometry{a4paper,
 lmargin = 2cm,rmargin = 2cm,tmargin = 2.5cm,bmargin = 2.5cm}

\input{../../macros.tex}

\pagestyle{fancy} %
\lhead{ECE2 \hfill Mathématiques\\
} %
\chead{\hrule} %
\rhead{} %
\lfoot{} %
\cfoot{} %
\rfoot{\thepage} %

\renewcommand{\headrulewidth}{0pt}% : Trace un trait de séparation
 % de largeur 0,4 point. Mettre 0pt
 % pour supprimer le trait.

\renewcommand{\footrulewidth}{0.4pt}% : Trace un trait de séparation
 % de largeur 0,4 point. Mettre 0pt
 % pour supprimer le trait.

\setlength{\headheight}{14pt}

\title{\bf \vspace{-2cm} ESCP 1991} %
\author{} %
\date{} %
\begin{document}

\maketitle %
\vspace{-1.4cm}\hrule %
\thispagestyle{fancy}

\vspace*{.2cm}


% DEBUT DU DOC À MODIFIER : tout virer jusqu'au début de l'exo

%Définition et changement de valeurs de
compteurs%newcounter{cpt1}{section} compteur cpt1 remis à 0 à chaque
aumentation par stepcounter du compteur section%setcounter{cpt1}{3} on
met le compteur à 3%addtocounter{cpt1}{5} on ajoute 5 au compteur%
stepcounter{cpt1} on ajoute 1% ifthenelse{test}{alors}{sinon} (page
206) pour subordonner à une condition % whiledo{test}{commande} pour
faire une boucle (page 206 aussi) % value{cpt1} pour noter dans le
document la valeur de cpt1 
%Définition définitive d'opérateurs
mathématiques\newcommand{\ch}{\operatorname{ch}} 
\newcommand{\sh}{\operatorname{sh}}
\renewcommand{\tanh}{\operatorname{th}}
\renewcommand{\sinh}{\operatorname{sh}}
\renewcommand{\cosh}{\operatorname{ch}}
\newcommand{\argsh}{\operatorname{argsh}}
\newcommand{\argch}{\operatorname{argch}}
\newcommand{\argth}{\operatorname{argth}}
\newcommand{\ker}{\operatorname{Ker}}
\renewcommand{\im}{\operatorname{Im}}
\newcommand{\rg}{\operatorname{rg}}
\newcommand{\Id}{\operatorname{Id}}
\newcommand{\id}{\operatorname{id}}
\renewcommand{\leq}{\leq}
\renewcommand{\geq}{\geq }

%Définition de nouvelles couleurs : rgb(trois paramètres red green blue
entre 0 et 1); cmyk (quatre cyan magenta yellow black) entre 0 et 1;
gray (entre 0 et 1) et black, white, red, green, blue, cyan, magenta,
yellow% definecolor{0gris}{gray}{0.8} 
% Nouvelle commande pour encadrer le titre car shabox ne veut que d'une
seule ligne; ATTENTION A LA TAILLE; petite différence avec shadowbox ou
doublebox, voire fcolorbox ou colorbox (au lieu de shabox; laisser le
parbox tranquille sauf pour la taille de la boîte
\newcommand{\Tbox}[1]{\begin{center} \shabox{\parbox{0.6
\linewidth}{#1}} \end{center}} %[1] pour 1 paramètre ; #1 pour ce que
fait le 1er paramètre; entre accolades ce que fait la commande
%Mise en page en mode fancy : en-têtes et pieds de pages puis
définition des en-têtes et pieds de pages\pagestyle{fancy}
\lhead{ECE 2 - Mathématiques \\
Quentin Dunstetter - ENC-Bessières 2011$\backslash$2012}
\chead{}
\rhead{ESCP 1991}
\rfoot[ \ \thepage]{\thepage}
\cfoot{}
\lfoot{}
\thispagestyle{fancy} %Mise en page de la 1ère page en mode fancy
%Trait en bas et en haut de la page (entre en-tête et texte et texte et
pied de page)\renewcommand{\footrulewidth}{0.4pt}
\renewcommand{\headrulewidth}{0.4pt}


%DEBUT DU DOCUMENT\vspace*{3cm}

\begin{center}
{\LARG\E\textbf{BANQUE COMMUNE D'ÉPREUVES}}



{\large \textsc{CONCOURS D ADMISSION DE 1991}}



{\large \textbf{Concepteur : ESCP}}



\rule{2.39cm}{0.05cm}



{\Large \textbf{OPTION ÉCONOMIQUE}}



{\Large \textbf{MATHÉMATIQUES }}



{\Large Lundi 9 mai, de 14h à 18h}



\rule{2.39cm}{0.05cm}
\end{center}

\textit{La présentation, la lisibilité, l'orthographe, la qualité
de la rédaction, la clarté et la précision des raisonnements
entreront pour une part importante dans l'appréciation des copies.}

\textit{Les candidats sont invités à \textbf{encadrer} dans la mesure
du possible les résultats de leurs calculs.}

\textit{Ils ne doivent faire usage d'aucun document. L'utilisation de
toute
calculatrice et de tout matériel électronique est interdite. Seule
l'utilisation d'une règle graduée est autorisée.}

\textit{Si au cours de l'épreuve, un candidat repère ce qui lui semble
être une erreur d'énoncé, il la signalera sur sa copie et
poursuivra sa composition en expliquant les raisons des initiatives
qu'il sera
amené à prendre.}

\vspace*{3cm}

\section*{EXERCICE 1}

On considère les matrices carrées d'ordre 3 suivantes :
\[
A = \left( 
\begin{array}{rrr}
-1 & -1 & 2 \\
-2 & 0 & 2 \\
-2 & -1 & 3
\end{array}
\right) ;\quad B = \left( 
\begin{array}{rrr}
3 & 1 & -3 \\
0 & 1 & 0 \\
2 & 1 & -2
\end{array}
\right) ;\quad C = \left( 
\begin{array}{rrr}
7 & 3 & -8 \\
9 & 4 & -11 \\
9 & 4 & -11
\end{array}
\right) ;\quad P = \left( 
\begin{array}{rrr}
1 & -1 & 1 \\
0 & 2 & 1 \\
1 & 0 & 1
\end{array}
\right) 
\]
On se propose de résoudre le système d'équations : 
\[
\left( S\right) \qquad \left\{ 
\begin{array}{c}
XP = PX \\
AX-XB = C
\end{array}
\right. 
\]
où $X$ est un élément inconnu de $\mathfrak{M}_{3}(\R)$.

\begin{noliste}{1.}
 \setlength{\itemsep}{4mm}
\item 

\begin{noliste}{a)}
 \setlength{\itemsep}{2mm}
\item Appliquer la méthode du pivot, en explicitant les calculs, pour
montrer que la matrice $P$ est inversible et pour calculer son inverse.

\item La matrice $P^{-1}$ est-elle solution du système $(S)$ ?
\end{noliste}

\item 

\begin{noliste}{a)}
 \setlength{\itemsep}{2mm}
\item Déterminer les valeurs propres des matrices $A$ et $B$.

\item Montrer qu'il existe une base de vecteurs propres communs à $A$
et à $B
$ telle que la matrice de passage associée soit $P$.
\end{noliste}

\item Dans ce qui suit, on pose : $Y = X-P^{-1}$.

\begin{noliste}{a)}
 \setlength{\itemsep}{2mm}
\item Montrer que la matrice X est solution du système $(S)$ si et
seulement
si la matrice $Y$ vérifie le système :
\[
\left\{ 
\begin{array}{c}
PY = YP \\
\left( 
\begin{array}{rrr}
1 & 0 & 0 \\
0 & 1 & 0 \\
0 & 0 & 0
\end{array}
\right) Y-Y\left( 
\begin{array}{rrr}
0 & 0 & 0 \\
0 & 1 & 0 \\
0 & 0 & 1
\end{array}
\right) = O
\end{array}
\right. 
\]

\item En déduire l'ensemble des solutions du système $(S)$.
\end{noliste}
\end{noliste}

\section*{EXERCICE 2}

Pour tout nombre entier naturel $n$, on considère les fonctions $g_{n}$
et $G_{n}$ définies sur $\R_{+}$ par les relations :
\[
g_{n}(x) = x^{n}e^{-x/2};\quad G_{n}(x) = \dfrac{1}{2^{n +
1}n!}\dint{0}{x}g_{n}(t)dt.
\]

\begin{noliste}{1.}
 \setlength{\itemsep}{4mm}
\item Étudier les variations des fonctions $g_{0}$, $g_{1}$ et $g_{2}$.
Tracer les graphes de ces fonctions dans un même repère cartésien.

\item 

\begin{noliste}{a)}
 \setlength{\itemsep}{2mm}
\item À l'aide d'une intégration par parties, calculer $G_{n +
1}(x)-G_{n}(x)$

\item Calculer $G_{0}(x)$, $G_{1}(x)$ et $G_{2}(x)$.

\item Déterminer la limite de $G_{0}(x)$ lorsque $x$ tend vers $ +
\infty $.
En déduire la limite de $G_{n}(x)$ lorsque $x$ tend vers $ + \infty $
(l'entier $n$ étant fixé).
\end{noliste}

\item Soit $F_{n}$ la fonction définie sur $\R$ par : 
\[
\begin{tabular}{ll}
si $x\geq 0,$ & $F_{n}(x) = G_{n}(x).$ \\
si $x<0,$ & $F_{n}(x) = 0.$\end{tabular}
\]

\begin{noliste}{$\sbullet$}
\item Montrer que $F_{n}$ a les propriété d'une fonction de
répartition.\\
On considère une variable aléatoire $X_{n}$ de fonction de répartition
$F_{n}
$.

\item Déterminer une densité $f_{n}$ de $X_{n}$.

\item Calculer l'espérance $\E(X_{n})$ et la variance $\V(X_{n})$.
Vérifier
que le rapport $\dfrac{\E(X_{n})}{\V(X_{n})}$ est indépendant de $n$.
\end{noliste}

\item Montrer que, pour tout nombre entier naturel non nul $n$, la
fonction $f_{n}$ prend une valeur maximale $M_{n}$. Trouver la limite,
lorsque $n$
tend vers $ + \infty $, du rapport $\dfrac{M_{n + 1}}{M_{n}}.$

\item On suppose que le nombre réel strictement positif $x$ est fixé.
Soit $T
$ une variable aléatoire à valeurs dans $\N$ telle que, pour tout
nombre entier naturel non nul $n$, la probabilité pour que $T\geq n$
soit égale à $F_{n-1}(x)$. Déterminer la loi de probabilité de $T$.
Calculer
l'espérance et la variance de $T$.
\end{noliste}

\label{fin}

\end{document}

