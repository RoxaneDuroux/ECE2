\documentclass[11pt]{article}%
\usepackage{geometry}%
\geometry{a4paper,
 lmargin = 2cm,rmargin = 2cm,tmargin = 2.5cm,bmargin = 2.5cm}

\input{../../macros.tex}

\pagestyle{fancy} %
\lhead{ECE2 \hfill Mathématiques\\
} %
\chead{\hrule} %
\rhead{} %
\lfoot{} %
\cfoot{} %
\rfoot{\thepage} %

\renewcommand{\headrulewidth}{0pt}% : Trace un trait de séparation
 % de largeur 0,4 point. Mettre 0pt
 % pour supprimer le trait.

\renewcommand{\footrulewidth}{0.4pt}% : Trace un trait de séparation
 % de largeur 0,4 point. Mettre 0pt
 % pour supprimer le trait.

\setlength{\headheight}{14pt}

\title{\bf \vspace{-2cm} ESCP 1994} %
\author{} %
\date{} %
\begin{document}

\maketitle %
\vspace{-1.4cm}\hrule %
\thispagestyle{fancy}

\vspace*{.2cm}


% DEBUT DU DOC À MODIFIER : tout virer jusqu'au début de l'exo

%Définition et changement de valeurs de
compteurs%newcounter{cpt1}{section} compteur cpt1 remis à 0 à chaque
aumentation par stepcounter du compteur section%setcounter{cpt1}{3} on
met le compteur à 3%addtocounter{cpt1}{5} on ajoute 5 au compteur%
stepcounter{cpt1} on ajoute 1% ifthenelse{test}{alors}{sinon} (page
206) pour subordonner à une condition % whiledo{test}{commande} pour
faire une boucle (page 206 aussi) % value{cpt1} pour noter dans le
document la valeur de cpt1 
%Définition définitive d'opérateurs
mathématiques\newcommand{\ch}{\operatorname{ch}} 
\newcommand{\sh}{\operatorname{sh}}
\renewcommand{\tanh}{\operatorname{th}}
\renewcommand{\sinh}{\operatorname{sh}}
\renewcommand{\cosh}{\operatorname{ch}}
\newcommand{\argsh}{\operatorname{argsh}}
\newcommand{\argch}{\operatorname{argch}}
\newcommand{\argth}{\operatorname{argth}}
\newcommand{\ker}{\operatorname{Ker}}
\renewcommand{\im}{\operatorname{Im}}
\newcommand{\rg}{\operatorname{rg}}
\newcommand{\Id}{\operatorname{Id}}
\newcommand{\id}{\operatorname{id}}
\renewcommand{\leq}{\leq}
\renewcommand{\geq}{\geq }

%Définition de nouvelles couleurs : rgb(trois paramètres red green blue
entre 0 et 1); cmyk (quatre cyan magenta yellow black) entre 0 et 1;
gray (entre 0 et 1) et black, white, red, green, blue, cyan, magenta,
yellow% definecolor{0gris}{gray}{0.8} 
% Nouvelle commande pour encadrer le titre car shabox ne veut que d'une
seule ligne; ATTENTION A LA TAILLE; petite différence avec shadowbox ou
doublebox, voire fcolorbox ou colorbox (au lieu de shabox; laisser le
parbox tranquille sauf pour la taille de la boîte
\newcommand{\Tbox}[1]{\begin{center} \shabox{\parbox{0.6
\linewidth}{#1}} \end{center}} %[1] pour 1 paramètre ; #1 pour ce que
fait le 1er paramètre; entre accolades ce que fait la commande
%Mise en page en mode fancy : en-têtes et pieds de pages puis
définition des en-têtes et pieds de pages\pagestyle{fancy}
\lhead{ECE 2 - Mathématiques \\
Quentin Dunstetter - ENC-Bessières 2011$\backslash$2012}
\chead{}
\rhead{ESCP 1994}
\rfoot[ \ \thepage]{\thepage}
\cfoot{}
\lfoot{}
\thispagestyle{fancy} %Mise en page de la 1ère page en mode fancy
%Trait en bas et en haut de la page (entre en-tête et texte et texte et
pied de page)\renewcommand{\footrulewidth}{0.4pt}
\renewcommand{\headrulewidth}{0.4pt}


%DEBUT DU DOCUMENT\vspace*{3cm}

\begin{center}
{\LARG\E\textbf{BANQUE COMMUNE D'ÉPREUVES}}



{\large \textsc{CONCOURS D ADMISSION DE 1994}}



{\large \textbf{Concepteur : ESCP}}



\rule{2.39cm}{0.05cm}



{\Large \textbf{OPTION ÉCONOMIQUE}}



{\Large \textbf{MATHÉMATIQUES }}



{\Large Lundi 9 mai, de 14h à 18h}



\rule{2.39cm}{0.05cm}
\end{center}

\textit{La présentation, la lisibilité, l'orthographe, la qualité
de la rédaction, la clarté et la précision des raisonnements
entreront pour une part importante dans l'appréciation des copies.}

\textit{Les candidats sont invités à \textbf{encadrer} dans la mesure
du possible les résultats de leurs calculs.}

\textit{Ils ne doivent faire usage d'aucun document. L'utilisation de
toute
calculatrice et de tout matériel électronique est interdite. Seule
l'utilisation d'une règle graduée est autorisée.}

\textit{Si au cours de l'épreuve, un candidat repère ce qui lui semble
être une erreur d'énoncé, il la signalera sur sa copie et
poursuivra sa composition en expliquant les raisons des initiatives
qu'il sera
amené à prendre.}

\vspace*{3cm}

\section*{\protect\Large Exercice 1 :}

Pour tout entier naturel $n$, on pose : 
\[
u_{n} = \dint\limits_{0}{1}x^{n}e^{1-x}dx
\]

\begin{noliste}{1.}
 \setlength{\itemsep}{4mm}
\item Calculer $u_{0}$ et $u_{1}.$

\begin{noliste}{a)}
 \setlength{\itemsep}{2mm}
\item Montrer que, pour tout entier naturel $n$, $\dfrac{1}{n + 1}\leq
u_{n}\leq \dfrac{e}{n + 1}.$

\item Calculer la limite de la suite $\left( u_{n}\right)_{n\geq 0}.$
\end{noliste}

\begin{noliste}{a)}
 \setlength{\itemsep}{2mm}
\item Exprimer, pour $n\geq 1,u_{n}$ en fonction de $u_{n-1}$ et de
$n.$

\item En déduire que, pour tout entier $n\geq 0,u_{n} = n!\left(
e-\dsum\limits_{k = 0}{n}\dfrac{1}{k!}\right).$
\end{noliste}

\item Soit $a$ un nombre réel et soit $\left( v_{n}\right)_{n\geq 0}$
la suite définie par les conditions :
\[
v_{0} = a,\text{ et pour tout entier }n\geq 1,\text{ }v_{n} =
nv_{n-1}-1
\]
Montrer que si $a\neq u_{0}$, la suite $\left( v_{n}\right)_{n\geq 0}$
est divergente.

\begin{noliste}{a)}
 \setlength{\itemsep}{2mm}
\item Montrer que, pour tout entier naturel,
\[
u_{n} = \dfrac{1}{n + 1} + \dfrac{1}{\left( n + 1\right) \left( n +
2\right) } + \dfrac{u_{n + 2}}{\left( n + 1\right) \left( n + 2\right)
}
\]

\item En déduire qu'il existe deux constantes $c_{1},c_{2}$, que l'on
déterminera, telles qu'on ait :
\[
u_{n} = \dfrac{c_{1}}{n} + \dfrac{c_{2}}{n^{2}} + o\left(
\dfrac{1}{n^{2}}\right)
\]
lorsque $n$ tend vers l'infini.
\end{noliste}
\end{noliste}

\section*{\protect\Large Exercice 2 :}

Soit la matrice à coefficients réels :
\[
A = \left( 
\begin{array}{rrr}
-1 & -1 & -1 \\
-1 & 1 & -1 \\
-1 & -1 & 3
\end{array}
\right) 
\]

\begin{noliste}{1.}
 \setlength{\itemsep}{4mm}
\item Montrer que $\lambda $ est valeur propre de $A$ si et seulement
si : $\lambda ^{3}-3\lambda ^{2}-4\lambda + 8 = 0.$

\item En déduire que $A$ admet trois valeurs propres
$\lambda_{1},\lambda
_{2},\lambda_{3}$ qui vérifient :
\[
\lambda_{1}<1<\lambda_{2}<2<\lambda_{3}.
\]

\item On considère l'application $f$ de $\R$ dans $\R$ définie par :
\[
f\left( \lambda \right) = \dfrac{1}{6}\left( \lambda ^{3}-3\lambda
^{2} + 2\lambda + 8\right)
\]

\begin{noliste}{a)}
 \setlength{\itemsep}{2mm}
\item Montrer que $f\left( \lambda_{2}\right) = \lambda_{2}.$

\item Montrer que le segment $\left[ 1,2\right] $ est stable par $f$
(c'est-à-dire que $f\left( \left[ 1,2\right] \right) \subset \left[
1,2\right] $).

\item Montrer que, pour tout $\lambda $ de $\left[ 1,2\right],\left|
f\left( \lambda \right) -f\left( \lambda_{2}\right) \right| \leq 
\dfrac{1}{3}\left| \lambda -\lambda_{2}\right|.$
\end{noliste}

\item Soit $x_{0} = 1$ et pour tout entier $n\geq 1,x_{n} = f\left(
x_{n-1}\right).$

\begin{noliste}{a)}
 \setlength{\itemsep}{2mm}
\item Montrer que la suite $\left( x_{n}\right)_{n\geq 0}$ converge
vers $\lambda_{2}.$

\item Donner une valeur fractionnaire simple de $x_{1}.$
\end{noliste}

\begin{noliste}{a)}
 \setlength{\itemsep}{2mm}
\item En utilisant la factorisation :
\[
\lambda ^{3}-3\lambda ^{2}-4\lambda + 8 = \left( \lambda
-\lambda_{1}\right)
\left( \lambda -\lambda_{2}\right) \left( \lambda -\lambda_{3}\right)
\]
calculer $\lambda_{1} + \lambda_{3}$ et $\lambda_{1}\lambda_{3}$ en
fonction de $\lambda_{2}.$

\item On prend $x_{1}$ comme valeur approchée de $\lambda_{2}$ :\\
en déduire des valeurs approchées de $\lambda_{1}$ et $\lambda_{3}. $
\end{noliste}

\begin{noliste}{a)}
 \setlength{\itemsep}{2mm}
\item Montrer que la matrice $A$ est diagonalisable.

\item Montrer que l'on peut trouver dans $\R^{3}$ une base $\left(
V_{1},V_{2},V_{3}\right) $ de vecteurs propres de $A$ de la forme :
\[
V_{1} = \left( p\left( \lambda_{1}\right),\lambda_{1},q\left( \lambda
_{1}\right) \right),V_{2} = \left( p\left( \lambda_{2}\right),\lambda
_{2},q\left( \lambda_{2}\right) \right),V_{3} = \left( p\left( \lambda
_{3}\right),\lambda_{3},q\left( \lambda_{3}\right) \right),
\]
où $p$ et $q$ sont des polynômes qu'on déterminera.
\end{noliste}
\end{noliste}

\section*{\protect\Large Exercice 3 :}

On dispose d'un paquet de $m$ cartes, $m$ étant un entier supérieur ou
égal à
2. \\
Ces cartes sont numérotées de $1$ à $m.$\\
Un joueur $A$ propose à un joueur $B$ le jeu suivant, moyennant une
mise de
1 franc que lui verse $B$ à chaque partie :\\
$B$ tire une carte au hasard, montre le nombre $\beta $ qu'elle porte
et
remet la carte dans le paquet. \\
Puis $A$ tire une carte au hasard; quand celle-ci porte le nombre
$\alpha $ :

$\bullet $ Si $\alpha <\beta $, alors $A$ donne à $B$ la somme $\left(
\beta
-\alpha \right) $ francs : $B$ a donc gagné $\left( \beta -\alpha
-1\right) $
francs.

$\bullet $ Si $\alpha >\beta $, alors $B$ donne à $A$ la somme de 1
franc : $B$ a donc perdu 2 francs.

$\bullet $ Si $\alpha = \beta $, alors $B$ a simplement perdu 1 franc,
le
montant de la mise.

\begin{noliste}{1.}
 \setlength{\itemsep}{4mm}
\item On suppose dans cette question que $m = 6.$

\begin{noliste}{a)}
 \setlength{\itemsep}{2mm}
\item Dresser le tableau à double entrée donnant les gains (positifs ou
négatifs) de $B$ suivant les différentes valeurs du couple $\left(
\alpha,\beta \right).$

\item Soit $X$ la variable aléatoire représentant les gains de $B.$
Donner
la loi de probabilité de $X.$

\item Calculer l'espérance de $X$. Le jeu est-il équilibré ou
avantage-t-il
l'un des joueurs ?Calculer la variance de $X.$
\end{noliste}

\item On revient au cas général : $m\geq 2.$

\begin{noliste}{a)}
 \setlength{\itemsep}{2mm}
\item Établir, en préliminaire, les formules suivantes, pour tout
entier $N\geq 1$ :
\[
\dsum\limits_{k = 1}{n}k^{2} = \dfrac{N\left( N + 1\right) \left( 2N +
1\right) }{6},\dsum\limits_{k = 1}{n}k^{3} = \dfrac{N^{2}\left( N +
1\right) ^{2}}{4}
\]

\item Calculer, en fonction de $m,$ l'espérance $\E\left( X\right) $ de
la
variable aléatoire $X.$

\item Pour quelles valeurs de $m$ l'espérance est-elle positive ?

\item Calculer, en fonction de $m$, la variance de $X.$
\end{noliste}

\item On observe $n$ parties successives et on note $Y_{n}\left(
m\right) $
le nombre de parties où le gain de $B$ est strictement positif.\\
Donner la loi de probabilité de la variable aléatoire $Y_{n}\left(
m\right) $, son espérance et sa variance en fonction de $n$ et de $m.$
\end{noliste}

\label{fin}

\end{document}

