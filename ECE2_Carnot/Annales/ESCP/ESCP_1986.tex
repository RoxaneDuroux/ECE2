\documentclass[11pt]{article}%
\usepackage{geometry}%
\geometry{a4paper,
 lmargin = 2cm,rmargin = 2cm,tmargin = 2.5cm,bmargin = 2.5cm}

\input{../../macros.tex}

\pagestyle{fancy} %
\lhead{ECE2 \hfill Mathématiques\\
} %
\chead{\hrule} %
\rhead{} %
\lfoot{} %
\cfoot{} %
\rfoot{\thepage} %

\renewcommand{\headrulewidth}{0pt}% : Trace un trait de séparation
 % de largeur 0,4 point. Mettre 0pt
 % pour supprimer le trait.

\renewcommand{\footrulewidth}{0.4pt}% : Trace un trait de séparation
 % de largeur 0,4 point. Mettre 0pt
 % pour supprimer le trait.

\setlength{\headheight}{14pt}

\title{\bf \vspace{-2cm} ESCP 1986} %
\author{} %
\date{} %
\begin{document}

\maketitle %
\vspace{-1.4cm}\hrule %
\thispagestyle{fancy}

\vspace*{.2cm}


% DEBUT DU DOC À MODIFIER : tout virer jusqu'au début de l'exo

%Définition et changement de valeurs de
compteurs%newcounter{cpt1}{section} compteur cpt1 remis à 0 à chaque
aumentation par stepcounter du compteur section%setcounter{cpt1}{3} on
met le compteur à 3%addtocounter{cpt1}{5} on ajoute 5 au compteur%
stepcounter{cpt1} on ajoute 1% ifthenelse{test}{alors}{sinon} (page
206) pour subordonner à une condition % whiledo{test}{commande} pour
faire une boucle (page 206 aussi) % value{cpt1} pour noter dans le
document la valeur de cpt1 
%Définition définitive d'opérateurs
mathématiques\newcommand{\ch}{\operatorname{ch}} 
\newcommand{\sh}{\operatorname{sh}}
\renewcommand{\tanh}{\operatorname{th}}
\renewcommand{\sinh}{\operatorname{sh}}
\renewcommand{\cosh}{\operatorname{ch}}
\newcommand{\argsh}{\operatorname{argsh}}
\newcommand{\argch}{\operatorname{argch}}
\newcommand{\argth}{\operatorname{argth}}
\newcommand{\ker}{\operatorname{Ker}}
\renewcommand{\im}{\operatorname{Im}}
\newcommand{\rg}{\operatorname{rg}}
\newcommand{\Id}{\operatorname{Id}}
\newcommand{\id}{\operatorname{id}}
\renewcommand{\leq}{\leq}
\renewcommand{\geq}{\geq }

%Définition de nouvelles couleurs : rgb(trois paramètres red green blue
entre 0 et 1); cmyk (quatre cyan magenta yellow black) entre 0 et 1;
gray (entre 0 et 1) et black, white, red, green, blue, cyan, magenta,
yellow% definecolor{0gris}{gray}{0.8} 
% Nouvelle commande pour encadrer le titre car shabox ne veut que d'une
seule ligne; ATTENTION A LA TAILLE; petite différence avec shadowbox ou
doublebox, voire fcolorbox ou colorbox (au lieu de shabox; laisser le
parbox tranquille sauf pour la taille de la boîte
\newcommand{\Tbox}[1]{\begin{center} \shabox{\parbox{0.6
\linewidth}{#1}} \end{center}} %[1] pour 1 paramètre ; #1 pour ce que
fait le 1er paramètre; entre accolades ce que fait la commande
%Mise en page en mode fancy : en-têtes et pieds de pages puis
définition des en-têtes et pieds de pages\pagestyle{fancy}
\lhead{ECE 2 - Mathématiques \\
Quentin Dunstetter - ENC-Bessières 2011$\backslash$2012}
\chead{}
\rhead{ESCP 1986}
\rfoot[ \ \thepage]{\thepage}
\cfoot{}
\lfoot{}
\thispagestyle{fancy} %Mise en page de la 1ère page en mode fancy
%Trait en bas et en haut de la page (entre en-tête et texte et texte et
pied de page)\renewcommand{\footrulewidth}{0.4pt}
\renewcommand{\headrulewidth}{0.4pt}


%DEBUT DU DOCUMENT\vspace*{3cm}

\begin{center}
{\LARG\E\textbf{BANQUE COMMUNE D'ÉPREUVES}}



{\large \textsc{CONCOURS D ADMISSION DE 1986}}



{\large \textbf{Concepteur : ESCP}}



\rule{2.39cm}{0.05cm}



{\Large \textbf{OPTION ÉCONOMIQUE}}



{\Large \textbf{MATHÉMATIQUES }}



{\Large Lundi 9 mai, de 14h à 18h}



\rule{2.39cm}{0.05cm}
\end{center}

\textit{La présentation, la lisibilité, l'orthographe, la qualité
de la rédaction, la clarté et la précision des raisonnements
entreront pour une part importante dans l'appréciation des copies.}

\textit{Les candidats sont invités à \textbf{encadrer} dans la mesure
du possible les résultats de leurs calculs.}

\textit{Ils ne doivent faire usage d'aucun document. L'utilisation de
toute
calculatrice et de tout matériel électronique est interdite. Seule
l'utilisation d'une règle graduée est autorisée.}

\textit{Si au cours de l'épreuve, un candidat repère ce qui lui semble
être une erreur d'énoncé, il la signalera sur sa copie et
poursuivra sa composition en expliquant les raisons des initiatives
qu'il sera
amené à prendre.}

\vspace*{3cm}

\section*{EXERCICE 1}

Soit $n$ un nombre entier naturel. On note $E_{n},$l'espace vectoriel
des
polynômes à coefficients réels de degré inférieur ou égal à $n$.\\
Soit $u$ l'endomorphisme de $E_{n},$qui à tout polynôme $P$ associe le
polynôme $Q$ défini par la relation :
\[
Q(x) = P\left(\Ev{x-1}\right) + P\left(\Ev{x}\right).
\]

\begin{noliste}{1.}
 \setlength{\itemsep}{4mm}
\item 

\begin{noliste}{a)}
 \setlength{\itemsep}{2mm}
\item Prouver que si $P$ est un élément non nul de $E_{n}$ le degré de
$Q$
est égal a celui de $P.$

\item En déduire que $u$ est un automorphisme de $E_{n}.$
\end{noliste}

\item 

\begin{noliste}{a)}
 \setlength{\itemsep}{2mm}
\item Montrer qu'il existe un élément $P_{n}$ de $E_{n}$ et un seul tel
que
\[
(1)\qquad P_{n}(x-1) + P_{n}(x) = 2x^{n}.
\]

\item En dérivant l'équation (1), montrer que si $n\neq 0,\quad
P_{n}{\prime } = nP_{n-1}.$

\item Calculer $P_{0},P_{1}$, $P_{2}$, et $P_{3}$.
\end{noliste}

\item 

\begin{noliste}{a)}
 \setlength{\itemsep}{2mm}
\item Écrire la relation satisfaite par le polynôme $Q_{n}(x) =
P_{n}(-x-1).$

\item En déduire que : $P_{n}(-x-1) = (-1)^{n}P_{n}(x)$

\item Soit $C_{n}$ la courbe représentative de $P_{n}$. Trouver les
éléments de symétrie de $C_{n}$ suivant la parité de $n.$
\end{noliste}

\item 

\begin{noliste}{a)}
 \setlength{\itemsep}{2mm}
\item Prouver que si $n$ est pair et non nul, $P_{n}(0) = P_{n}(-1) =
0.$

\item Montrer que si $n$ est impair, $P_{n}(-1/2) = 0.$
\end{noliste}

\item Construire les courbes les courbes $C_{2}$, et $C_{3}$.
\end{noliste}

\section*{EXERCICE 2}

On considère la fonction numérique $f$ définie sur l'intervalle
$]-1;1[$ par
la relation : 
\[
f(x) = (1-x^{2})\ln \left( {\dfrac{{1 + x}}{{1-x}}}\right).
\]

\begin{noliste}{1.}
 \setlength{\itemsep}{4mm}
\item 

\begin{noliste}{a)}
 \setlength{\itemsep}{2mm}
\item Étudier la parité de $f$ et prouver que $f$ se prolonge en une
fonction continue $g$ sur $[-1;1].$

\item Calculer la dérivée de $f$ ; étudier la dérivabilité de $g$ aux
bornes
de l'intervalle.
\end{noliste}

\item Soit $h$ la fonction définie sur $]-1; + \infty \lbrack $ par la
relation $h(t) = \ln \left( {\dfrac{{t + 1}}{{t-1}}}\right) -t$.

\begin{noliste}{a)}
 \setlength{\itemsep}{2mm}
\item Étudier la variation de $h$.

\item Prouver que $h$ s'annule en un point $\alpha $ et un seul, et que
$1,5\leq \alpha \leq 1,6.$
\end{noliste}

\item 

\begin{noliste}{a)}
 \setlength{\itemsep}{2mm}
\item Étudier le signe de $f^{\prime }.$

\item Dresser le tableau de variation de $g$ et construire la courbe
représentative $C$ de $g$.\\
Exprimer en fonction de $\alpha $ les coordonnée des points de $C$ où
la
tangente est parallèle à l'axe des abscisses.
\end{noliste}

\item Calculer $\dint{0}{1}g(x)dx.$
\end{noliste}

\section*{EXERCICE 3}

Soit $E$ l'espace vectoriel des fonctions $f$ à valeurs réelles,
continues
sur l'intervalle $[-1;1]$ et telles que $f(-1) = f(1) = 0$.\\
On note $M(f)$ le maximum de $|f|$ sur $[-1;1]$ et on pose :
\[
I(f) = \dint{-1}{1}\left| f(t)\right\ dt.
\]
Soit enfin $B$ la partie de $E$ constituée des fonctions $f$ de classe
$C^{1}
$ sur les intervalles $[-1;0]$ et $[0;1]$ et telles que, pour tout
élément
non nul $t$ de $[-1;1],\quad \left| f^{\prime }(t)\right| \leq
1.$

\begin{noliste}{1.}
 \setlength{\itemsep}{4mm}
\item Soit $\varphi $ la fonction définie sur $[-1;1]$ par la relation
$\varphi (t) = 1-\left| t\right| $. \\
Prouver que $\varphi $ appartient à $B$ et construire sa courbe
représentative.

\item Soit $f$ un élément de $B.$Prouver que, pour tout point $t$ de
$[-1;1]$, $\quad \left| f(t)\right| \leq \varphi (t)$\\
(On pourra se placer d'abord sur $[0;1]$, puis sur $[-1;0]$.

\item Prouver que, pour tout élément $f$ de $B$, $I(f)\leq 1$;
déterminer les éléments $f$ de $B$ tels que $I(f) = l.$

\item 

\begin{noliste}{a)}
 \setlength{\itemsep}{2mm}
\item Prouver que, pour tout élément $f$ de $B$, $M(f)\leq 1.$

\item Soit $\psi $ un élément de $B$ tel que $M(\psi ) = 1.$ à l'aide
de la
question 2$,$montrer que $\left| \psi (0)\right| = 1.$\\
Lorsque $\psi (0) = 1$, prouver que $\psi = \varphi.$ (On utilisera la
méthode
indiquée dans la question 2.)

\item Déterminer les éléments $f$ de $B$ tels que $M(f) = 1.$
\end{noliste}
\end{noliste}

\section*{EXERCICE 4}

Soient $X$ et $Y$ deux variables aléatoires indépendantes, définies sur
un
espace probabilisé $(\Omega ;A;P).$On suppose que $X$ est une variable
de
Poisson de paramètre $\lambda $ et $Y$ une variable de Poisson de
paramètre $\mu $, où $\lambda $ et $\mu $ sont strictement positifs.
Pour tout nombre
entier naturel $n$, on notera
\[
p_{n} = P\left(\Ev{X = n}\right)\quad \text{et}\quad q_{n} =
P\left(\Ev{Y = n}\right)
\]

\begin{noliste}{1.}
 \setlength{\itemsep}{4mm}
\item Montrer que la variable $Z = X + Y$ est une variable de Poisson
dont on déterminera le paramètre.\\
Trouver l'espérance et la variance de $Z.$

\item Soit $k$ un nombre entier naturel. Déterminer la loi de
probabilité
conditionnelle de $X$ sachant que $Z = k.$

\item On pose $U = X-Y.$

\begin{noliste}{a)}
 \setlength{\itemsep}{2mm}
\item Trouver l'espérance et la variance de $U$.

\item Déterminer l'ensemble des valeurs que peut prendre $U$. La
variable $U$
est-elle une variable de Poisson ?

\item Trouver la loi conditionnelle de $U$ sachant que $Z = k$, où
$k\in N$
(on pourra utiliser le résultat de la question 2).
\end{noliste}
\end{noliste}

\label{fin}

\end{document}

