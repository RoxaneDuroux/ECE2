\documentclass[11pt]{article}%
\usepackage{geometry}%
\geometry{a4paper,
 lmargin = 2cm,rmargin = 2cm,tmargin = 2.5cm,bmargin = 2.5cm}

\input{../../macros.tex}

\pagestyle{fancy} %
\lhead{ECE2 \hfill Mathématiques\\
} %
\chead{\hrule} %
\rhead{} %
\lfoot{} %
\cfoot{} %
\rfoot{\thepage} %

\renewcommand{\headrulewidth}{0pt}% : Trace un trait de séparation
 % de largeur 0,4 point. Mettre 0pt
 % pour supprimer le trait.

\renewcommand{\footrulewidth}{0.4pt}% : Trace un trait de séparation
 % de largeur 0,4 point. Mettre 0pt
 % pour supprimer le trait.

\setlength{\headheight}{14pt}

\title{\bf \vspace{-2cm} ESCP 1989} %
\author{} %
\date{} %
\begin{document}

\maketitle %
\vspace{-1.4cm}\hrule %
\thispagestyle{fancy}

\vspace*{.2cm}


% DEBUT DU DOC À MODIFIER : tout virer jusqu'au début de l'exo

%Définition et changement de valeurs de
compteurs%newcounter{cpt1}{section} compteur cpt1 remis à 0 à chaque
aumentation par stepcounter du compteur section%setcounter{cpt1}{3} on
met le compteur à 3%addtocounter{cpt1}{5} on ajoute 5 au compteur%
stepcounter{cpt1} on ajoute 1% ifthenelse{test}{alors}{sinon} (page
206) pour subordonner à une condition % whiledo{test}{commande} pour
faire une boucle (page 206 aussi) % value{cpt1} pour noter dans le
document la valeur de cpt1 
%Définition définitive d'opérateurs
mathématiques\newcommand{\ch}{\operatorname{ch}} 
\newcommand{\sh}{\operatorname{sh}}
\renewcommand{\tanh}{\operatorname{th}}
\renewcommand{\sinh}{\operatorname{sh}}
\renewcommand{\cosh}{\operatorname{ch}}
\newcommand{\argsh}{\operatorname{argsh}}
\newcommand{\argch}{\operatorname{argch}}
\newcommand{\argth}{\operatorname{argth}}
\newcommand{\ker}{\operatorname{Ker}}
\renewcommand{\im}{\operatorname{Im}}
\newcommand{\rg}{\operatorname{rg}}
\newcommand{\Id}{\operatorname{Id}}
\newcommand{\id}{\operatorname{id}}
\renewcommand{\leq}{\leq}
\renewcommand{\geq}{\geq }

%Définition de nouvelles couleurs : rgb(trois paramètres red green blue
entre 0 et 1); cmyk (quatre cyan magenta yellow black) entre 0 et 1;
gray (entre 0 et 1) et black, white, red, green, blue, cyan, magenta,
yellow% definecolor{0gris}{gray}{0.8} 
% Nouvelle commande pour encadrer le titre car shabox ne veut que d'une
seule ligne; ATTENTION A LA TAILLE; petite différence avec shadowbox ou
doublebox, voire fcolorbox ou colorbox (au lieu de shabox; laisser le
parbox tranquille sauf pour la taille de la boîte
\newcommand{\Tbox}[1]{\begin{center} \shabox{\parbox{0.6
\linewidth}{#1}} \end{center}} %[1] pour 1 paramètre ; #1 pour ce que
fait le 1er paramètre; entre accolades ce que fait la commande
%Mise en page en mode fancy : en-têtes et pieds de pages puis
définition des en-têtes et pieds de pages\pagestyle{fancy}
\lhead{ECE 2 - Mathématiques \\
Quentin Dunstetter - ENC-Bessières 2011$\backslash$2012}
\chead{}
\rhead{ESCP 1989}
\rfoot[ \ \thepage]{\thepage}
\cfoot{}
\lfoot{}
\thispagestyle{fancy} %Mise en page de la 1ère page en mode fancy
%Trait en bas et en haut de la page (entre en-tête et texte et texte et
pied de page)\renewcommand{\footrulewidth}{0.4pt}
\renewcommand{\headrulewidth}{0.4pt}


%DEBUT DU DOCUMENT\vspace*{3cm}

\begin{center}
{\LARG\E\textbf{BANQUE COMMUNE D'ÉPREUVES}}



{\large \textsc{CONCOURS D ADMISSION DE 1989}}



{\large \textbf{Concepteur : ESCP}}



\rule{2.39cm}{0.05cm}



{\Large \textbf{OPTION ÉCONOMIQUE}}



{\Large \textbf{MATHÉMATIQUES }}



{\Large Lundi 9 mai, de 14h à 18h}



\rule{2.39cm}{0.05cm}
\end{center}

\textit{La présentation, la lisibilité, l'orthographe, la qualité
de la rédaction, la clarté et la précision des raisonnements
entreront pour une part importante dans l'appréciation des copies.}

\textit{Les candidats sont invités à \textbf{encadrer} dans la mesure
du possible les résultats de leurs calculs.}

\textit{Ils ne doivent faire usage d'aucun document. L'utilisation de
toute
calculatrice et de tout matériel électronique est interdite. Seule
l'utilisation d'une règle graduée est autorisée.}

\textit{Si au cours de l'épreuve, un candidat repère ce qui lui semble
être une erreur d'énoncé, il la signalera sur sa copie et
poursuivra sa composition en expliquant les raisons des initiatives
qu'il sera
amené à prendre.}

\vspace*{3cm}

\section*{EXERCICE 1}

Soit $f$ la fonction définie sur $\R$ par : 
\[
f(x) = \dint{x}{2x}e^{-t^{2}}\,dt.
\]

\begin{noliste}{1.}
 \setlength{\itemsep}{4mm}
\item 

\begin{noliste}{a)}
 \setlength{\itemsep}{2mm}
\item Étudier la parité de $f$.

\item Déterminer le signe de $f(x)$ suivant les valeurs de $x$.

\item Montrer que $f$ admet $0$ pour limite en $ + \infty $ et en
$-\infty $.
\end{noliste}

\item 

\begin{noliste}{a)}
 \setlength{\itemsep}{2mm}
\item Montrer que $f$ est dérivable et que : 
\[
f^{\prime }(x) = 2e^{-4x^{2}}-e^{-x^{2}}.
\]

\item Étudier la variation de $f$. Préciser les points où $f$ présente
un
extremum.

\item Calculer la dérivée seconde de $f$ et déterminer le signe de
$f^{\prime \prime }(x)$.

\item Construire la courbe représentative de $f$. (On admettra que le
maximum de $f$ est sensiblement égal à $0{.}3$.)
\end{noliste}

\item On considère une variable aléatoire réelle $X$ qui suit une loi
normale centrée d'écart-type $\sigma $. Pour tout nombre réel
strictement
positif $a$, on note $p(a)$ la probabilité de l'évènement : 
\[
a\leq X\leq 2a.
\]
Déterminer la valeur de $a$ pour laquelle $p(a)$ est maximal.
\end{noliste}

\section*{EXERCICE 2}

Soit $u = \left(u_{n}\right)$ la suite réelle définie par la relation
de
récurrence : 
\[
u_{n + 1} = u_{n} + \frac{2}{u_{n}}
\]
et la condition initiale $u_{0} = 1$.

\begin{noliste}{1.}
 \setlength{\itemsep}{4mm}
\item Montrer que la suite $u$ est bien définie.

\item Étudier la monotonie de la suite $u$. Montrer que : 
\[
\dlim{n\rightarrow + \infty} u_{n} = + \infty.
\]

\item Pour tout nombre entier naturel $k$, exprimer $u_{k +
1}{2}-u_{k}{2}$
en fonction de $u_{k}$. \\
En déduire la limite de $u_{k + 1}{2}-u_{k}{2}$ lorsque $k$ tend vers $
+ \infty $.

\item Pour tout nombre entier naturel $n$, on pose : 
\[
v_{n} = \frac{u_{n}{2}}{4}.
\]

\begin{noliste}{a)}
 \setlength{\itemsep}{2mm}
\item Montrer que pour tout nombre entier naturel $k$ : 
\[
v_{k + 1}-v_{k}\geq 1.
\]
En déduire que tout nombre entier naturel non nul $n$ : 
\[
v_{n}\geq n.
\]

\item \label{prem} Montrer que, pour tout nombre entier $k\geq 1$ : 
\[
v_{k + 1}-v_{k}\leq 1 + \frac{1}{k}.
\]

\item \label{deux} Prouver que, pour tout entier $k\geq 2$ : 
\[
\frac{1}{k}\leq \dint{k-1}{k}\frac{dt}{t}.
\]

\item Déduire des questions \ref{prem} et \ref{deux} que, pour tout
entier $n\geq 1$ : 
\[
v_{n}-v_{0}\leq n + \frac{1}{v_{0}} + 1 + \ln n.
\]
\end{noliste}

\item Déterminer la limite du rapport $\dfrac{v_{n}}{n}$ lorsque $n$
tend
vers $ + \infty $. \\
En déduire un équivalent de $u_{n}$.
\end{noliste}

\section*{EXERCICE 3}

On désigne par $n$ un nombre entier strictement supérieur à $1$.\\
Un sac contient des boules rouges et des boules blanches,
indiscernables si
ce n'est par la couleur. La proportion de boules rouges est $p$, où
$0<p<1$
; celle des boules blanches est $q = 1-p$. On effectue une suite de
tirages
d'une boule, avec remise de la boule tirée après chaque tirage, selon
la règle suivante :

\begin{noliste}{$\sbullet$}
\item dès qu'une boule rouge est tirée, on arrête les tirages ;

\item si les $n$ premières boules tirées sont blanches, on arrête les
tirages.
\end{noliste}

\begin{noliste}{1.}
 \setlength{\itemsep}{4mm}
\item Soit $\Omega $ l'ensemble des suites de couleurs de boules qu'on
peut
tirer selon cette règle. (Ces suites sont de longueur finie au plus
égale à $n$.)

\begin{noliste}{a)}
 \setlength{\itemsep}{2mm}
\item Montrer que $\Omega $ possède $n + 1$ éléments.

\item Déterminer la probabilité $P$ sur $\left(
\Omega,\mathcal{P}\left(
\Omega \right) \right) $ associé à cette expérience aléatoire (où
$\mathcal{P}\left( \Omega \right) $ désigne l'ensemble des parties de
$\Omega $).
\end{noliste}

\item On note $N$, $X_{1}$, $X_{2}$ les variables aléatoires
représentant
respectivement le nombre de tirages effectués, le nombre de boules
blanches
tirées et le nombre de boules rouges tirées.

\begin{noliste}{a)}
 \setlength{\itemsep}{2mm}
\item Établir une relation simple entre ces trois variables aléatoires.

\item Déterminer les lois de probabilité de ces trois variables et
exprimer
leur espérance en fonction de $n$ et $q$.
\end{noliste}

\item Trouver la loi de la variable $Z = X_{1}X_{2}$ et calculer son
espérance. En déduire l'expression de la covariance de $X_{1}$ et
$X_{2}$ en
fonction de $n$ et $q$.

\item Pour tout couple $(j,k)$ de nombres entiers naturels non nuls
tels que 
$j + k\leq n$, calculer la probabilité conditionnelle : 
\[
P\left(\Ev{N = j + k\,/N>k}\right).
\]
Quand vaut-elle $P\left(\Ev{N = j}\right)$ ?
\end{noliste}

\label{fin}

\end{document}

