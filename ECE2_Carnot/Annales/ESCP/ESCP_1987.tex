\documentclass[11pt]{article}%
\usepackage{geometry}%
\geometry{a4paper,
 lmargin = 2cm,rmargin = 2cm,tmargin = 2.5cm,bmargin = 2.5cm}

\input{../../macros.tex}

\pagestyle{fancy} %
\lhead{ECE2 \hfill Mathématiques\\
} %
\chead{\hrule} %
\rhead{} %
\lfoot{} %
\cfoot{} %
\rfoot{\thepage} %

\renewcommand{\headrulewidth}{0pt}% : Trace un trait de séparation
 % de largeur 0,4 point. Mettre 0pt
 % pour supprimer le trait.

\renewcommand{\footrulewidth}{0.4pt}% : Trace un trait de séparation
 % de largeur 0,4 point. Mettre 0pt
 % pour supprimer le trait.

\setlength{\headheight}{14pt}

\title{\bf \vspace{-2cm} ESCP 1987} %
\author{} %
\date{} %
\begin{document}

\maketitle %
\vspace{-1.4cm}\hrule %
\thispagestyle{fancy}

\vspace*{.2cm}


% DEBUT DU DOC À MODIFIER : tout virer jusqu'au début de l'exo

%Définition et changement de valeurs de
compteurs%newcounter{cpt1}{section} compteur cpt1 remis à 0 à chaque
aumentation par stepcounter du compteur section%setcounter{cpt1}{3} on
met le compteur à 3%addtocounter{cpt1}{5} on ajoute 5 au compteur%
stepcounter{cpt1} on ajoute 1% ifthenelse{test}{alors}{sinon} (page
206) pour subordonner à une condition % whiledo{test}{commande} pour
faire une boucle (page 206 aussi) % value{cpt1} pour noter dans le
document la valeur de cpt1 
%Définition définitive d'opérateurs
mathématiques\newcommand{\ch}{\operatorname{ch}} 
\newcommand{\sh}{\operatorname{sh}}
\renewcommand{\tanh}{\operatorname{th}}
\renewcommand{\sinh}{\operatorname{sh}}
\renewcommand{\cosh}{\operatorname{ch}}
\newcommand{\argsh}{\operatorname{argsh}}
\newcommand{\argch}{\operatorname{argch}}
\newcommand{\argth}{\operatorname{argth}}
\newcommand{\ker}{\operatorname{Ker}}
\renewcommand{\im}{\operatorname{Im}}
\newcommand{\rg}{\operatorname{rg}}
\newcommand{\Id}{\operatorname{Id}}
\newcommand{\id}{\operatorname{id}}
\renewcommand{\leq}{\leq}
\renewcommand{\geq}{\geq }

%Définition de nouvelles couleurs : rgb(trois paramètres red green blue
entre 0 et 1); cmyk (quatre cyan magenta yellow black) entre 0 et 1;
gray (entre 0 et 1) et black, white, red, green, blue, cyan, magenta,
yellow% definecolor{0gris}{gray}{0.8} 
% Nouvelle commande pour encadrer le titre car shabox ne veut que d'une
seule ligne; ATTENTION A LA TAILLE; petite différence avec shadowbox ou
doublebox, voire fcolorbox ou colorbox (au lieu de shabox; laisser le
parbox tranquille sauf pour la taille de la boîte
\newcommand{\Tbox}[1]{\begin{center} \shabox{\parbox{0.6
\linewidth}{#1}} \end{center}} %[1] pour 1 paramètre ; #1 pour ce que
fait le 1er paramètre; entre accolades ce que fait la commande
%Mise en page en mode fancy : en-têtes et pieds de pages puis
définition des en-têtes et pieds de pages\pagestyle{fancy}
\lhead{ECE 2 - Mathématiques \\
Quentin Dunstetter - ENC-Bessières 2011$\backslash$2012}
\chead{}
\rhead{ESCP 1987}
\rfoot[ \ \thepage]{\thepage}
\cfoot{}
\lfoot{}
\thispagestyle{fancy} %Mise en page de la 1ère page en mode fancy
%Trait en bas et en haut de la page (entre en-tête et texte et texte et
pied de page)\renewcommand{\footrulewidth}{0.4pt}
\renewcommand{\headrulewidth}{0.4pt}


%DEBUT DU DOCUMENT\vspace*{3cm}

\begin{center}
{\LARG\E\textbf{BANQUE COMMUNE D'ÉPREUVES}}



{\large \textsc{CONCOURS D ADMISSION DE 1987}}



{\large \textbf{Concepteur : ESCP}}



\rule{2.39cm}{0.05cm}



{\Large \textbf{OPTION ÉCONOMIQUE}}



{\Large \textbf{MATHÉMATIQUES }}



{\Large Lundi 9 mai, de 14h à 18h}



\rule{2.39cm}{0.05cm}
\end{center}

\textit{La présentation, la lisibilité, l'orthographe, la qualité
de la rédaction, la clarté et la précision des raisonnements
entreront pour une part importante dans l'appréciation des copies.}

\textit{Les candidats sont invités à \textbf{encadrer} dans la mesure
du possible les résultats de leurs calculs.}

\textit{Ils ne doivent faire usage d'aucun document. L'utilisation de
toute
calculatrice et de tout matériel électronique est interdite. Seule
l'utilisation d'une règle graduée est autorisée.}

\textit{Si au cours de l'épreuve, un candidat repère ce qui lui semble
être une erreur d'énoncé, il la signalera sur sa copie et
poursuivra sa composition en expliquant les raisons des initiatives
qu'il sera
amené à prendre.}

\vspace*{3cm}

\section*{EXERCICE 1}

On considère la fonction numérique $f$ définie par la relation : 
\[
f(x) = \dfrac{x^{2}-1}{4}-\dfrac{1}{2}\ln x
\]

\begin{noliste}{1.}
 \setlength{\itemsep}{4mm}
\item Étudier les variations de $f$ et construire la courbe
représentative
de cette fonction.

\item Déterminer la position de la courbe représentative de $f$ par
rapport à la parabole d'équation 
\[
y = \dfrac{1}{2}(x-1)^{2}
\]

\item Pour tout élément $x$ de l'intervalle $]0,1[$, calculer
$\dint{0}{1}f(t)dt$.\\
En déduire $\dlim{x\rightarrow {0^{+}}}\dint{x}{1}f(t)dt$

\item Soit $n$ un entier naturel supérieur ou égal à 3

\begin{noliste}{a)}
 \setlength{\itemsep}{2mm}
\item Calculer : 
\[
\dfrac{1}{n}\left[ f(\dfrac{1}{n}) + f(\dfrac{2}{n}) + \cdots +
f(\dfrac{n}{n})\right]
\]
On utilisera la relation : $\Sum{k = 1}{n}k^{2} = \dfrac{n(n + 1)(2n +
1)}{6}$

\item Prouver que : 
\[
\dint{1/n}{1}f(t)dt\leq \dfrac{1}{n}\Sum{k = 1}{n}f\left(
\dfrac{k}{n}\right) \leq \dfrac{1}{n}f\left( \dfrac{1}{n}\right)
 + \dint{1/n}{1}f(t)dt
\]
\end{noliste}

\item À l'aide des questions précédentes, montrer que : 
\[
\dlim{n\rightarrow + \infty }\dfrac{1}{n}\ln \dfrac{n!}{n^{n}} = -1
\]
\end{noliste}

\section*{EXERCICE 2}

\subsection*{I.}

On considère la matrice carrée d'ordre 3 : 
\[
A = \left( 
\begin{array}{ccc}
1 & 2 & 0 \\
2 & 1 & 0 \\
0 & 1 & 0
\end{array}
\right)
\]

\begin{noliste}{1.}
 \setlength{\itemsep}{4mm}
\item Trouver toutes les matrices carrées réelles $B$ d'ordre 3 telles
que $AB = 0$.\\
Déterminer la dimension de l'espace vectoriel formé par ces matrices
$B$.

\item Trouver toutes les matrices carrées réelles $B$ d'ordre 3 telles
que $AB = BA = 0$.\\
Déterminer la dimension de l'espace vectoriel formé par ces matrices
$B$.

\item Calculer les valeurs propres de $A$. La matrice $A$ est-elle
diagonalisable ? Si oui, expliciter une base de vecteurs propres.
\end{noliste}

\subsection*{II.}

Plus généralement, soit $E$ une espace vectoriel de dimension $n$ sur
$\R$, où $n$ est un nombre entier naturel non nul. On considère des
endomorphismes $u$ et $v$ de $E$.

\begin{noliste}{1.}
 \setlength{\itemsep}{4mm}
\item On suppose que : 
\[
vu = uv = 0
\]

\begin{noliste}{a)}
 \setlength{\itemsep}{2mm}
\item Montrer que $\operatorname{Im}(u)\subset \ker (v)$ et
$\operatorname{Im}(v)\subset
\ker (u)$.

\item Montrer que : 
\[
\dim \ker (u) + \dim \ker (v)\geq n
\]
\end{noliste}

\item On suppose que $\operatorname{Im}(u)\subset \ker (v)$ et
$\operatorname{Im}(v)\subset
\ker (u)$. Montrer que $uv = vu = 0$.

\item La propriété $uv = 0$ implique-t-elle la propriété $vu = 0$ ?
Donner éventuellement un contre-exemple.
\end{noliste}

\section*{EXERCICE 3}

Pour tout entier naturel $n$ tel que $n\geq 4$, on pose : 
\[
\begin{array}{ll}
f_{n}(x) = 0 & \text{si }x<0 \\
f_{n}(x) = \dfrac{a_{n}}{(x + 1)^{n}} & \text{si }x\geq 0
\end{array}
\]
où $a_{n}$ est un réel.

\begin{noliste}{1.}
 \setlength{\itemsep}{4mm}
\item Pour quelle valeur de $a_{n}$ la fonction $f_{n}$ est-elle une
densité
de probabilité ?

\item On prend désormais pour $a_{n}$ la valeur trouvée dans la
question 1).
On note $X_{n}$ une variable aléatoire réelle de densité $f_{n}$.

\begin{noliste}{a)}
 \setlength{\itemsep}{2mm}
\item Déterminer la fonction de répartition $F_{n}$ de $X_{n}$.

\item On pose $Y_{n} = X_{n} + 1$. Calculer l'espérance et la variance
de $Y_{n}$. En déduire l'espérance et la variance de $X_{n}$.
\end{noliste}

\item Soit $Z_{n} = -X_{n}$.

\begin{noliste}{a)}
 \setlength{\itemsep}{2mm}
\item Trouver la fonction de répartition $G_{n}$ et la densité de
probabilité
$g_{n}$ de $Z_{n}$.

\item Pour tout nombre réel $x$, on pose $G(x) = \dlim{n\rightarrow
 + \infty }G_{n}(x)$. Montrer que la fonction $G$ ainsi définie est une
fonction de répartition. Quelle est la loi de probabilité d'une
variable aléatoire $Z$ ayant $G$ pour fonction de répartition ?
\end{noliste}
\end{noliste}

\label{fin}

\end{document}

