\documentclass[11pt]{article}%
\usepackage{geometry}%
\geometry{a4paper,
 lmargin = 2cm,rmargin = 2cm,tmargin = 2.5cm,bmargin = 2.5cm}

\input{../../macros.tex}

\pagestyle{fancy} %
\lhead{ECE2 \hfill Mathématiques\\
} %
\chead{\hrule} %
\rhead{} %
\lfoot{} %
\cfoot{} %
\rfoot{\thepage} %

\renewcommand{\headrulewidth}{0pt}% : Trace un trait de séparation
 % de largeur 0,4 point. Mettre 0pt
 % pour supprimer le trait.

\renewcommand{\footrulewidth}{0.4pt}% : Trace un trait de séparation
 % de largeur 0,4 point. Mettre 0pt
 % pour supprimer le trait.

\setlength{\headheight}{14pt}

\title{\bf \vspace{-2cm} EDHEC 2010} %
\author{} %
\date{} %
\begin{document}

\maketitle %
\vspace{-1.4cm}\hrule %
\thispagestyle{fancy}

\vspace*{.2cm}


% DEBUT DU DOC À MODIFIER : tout virer jusqu'au début de l'exo

%Définition et changement de valeurs de
compteurs%newcounter{cpt1}{section} compteur cpt1 remis à 0 à chaque
aumentation par stepcounter du compteur section%setcounter{cpt1}{3} on
met le compteur à 3%addtocounter{cpt1}{5} on ajoute 5 au compteur%
stepcounter{cpt1} on ajoute 1% ifthenelse{test}{alors}{sinon} (page
206) pour subordonner à une condition % whiledo{test}{commande} pour
faire une boucle (page 206 aussi) % value{cpt1} pour noter dans le
document la valeur de cpt1 
%Définition définitive d'opérateurs
mathématiques\newcommand{\ch}{\operatorname{ch}} 
\newcommand{\sh}{\operatorname{sh}}
\renewcommand{\tanh}{\operatorname{th}}
\renewcommand{\sinh}{\operatorname{sh}}
\renewcommand{\cosh}{\operatorname{ch}}
\newcommand{\argsh}{\operatorname{argsh}}
\newcommand{\argch}{\operatorname{argch}}
\newcommand{\argth}{\operatorname{argth}}
\newcommand{\ker}{\operatorname{Ker}}
\renewcommand{\im}{\operatorname{Im}}
\newcommand{\rg}{\operatorname{rg}}
\newcommand{\Id}{\operatorname{Id}}
\newcommand{\id}{\operatorname{id}}
\renewcommand{\leq}{\leq}
\renewcommand{\geq}{\geq }

%Définition de nouvelles couleurs : rgb(trois paramètres red green blue
entre 0 et 1); cmyk (quatre cyan magenta yellow black) entre 0 et 1;
gray (entre 0 et 1) et black, white, red, green, blue, cyan, magenta,
yellow% definecolor{0gris}{gray}{0.8} 
% Nouvelle commande pour encadrer le titre car shabox ne veut que d'une
seule ligne; ATTENTION A LA TAILLE; petite différence avec shadowbox ou
doublebox, voire fcolorbox ou colorbox (au lieu de shabox; laisser le
parbox tranquille sauf pour la taille de la boîte
\newcommand{\Tbox}[1]{\begin{center} \shabox{\parbox{0.6
\linewidth}{#1}} \end{center}} %[1] pour 1 paramètre ; #1 pour ce que
fait le 1er paramètre; entre accolades ce que fait la commande
%Mise en page en mode fancy : en-têtes et pieds de pages puis
définition des en-têtes et pieds de pages\pagestyle{fancy}
\lhead{ECE 2 - Mathématiques \\
Quentin Dunstetter - ENC-Bessières 2011$\backslash$2012}
\chead{}
\rhead{Edhec 2010}
\rfoot[ \ \thepage]{\thepage}
\cfoot{}
\lfoot{}
\thispagestyle{fancy} %Mise en page de la 1ère page en mode fancy
%Trait en bas et en haut de la page (entre en-tête et texte et texte et
pied de page)\renewcommand{\footrulewidth}{0.4pt}
\renewcommand{\headrulewidth}{0.4pt}


%DEBUT DU DOCUMENT\vspace*{3cm}

\begin{center}
{\LARG\E\textbf{BANQUE COMMUNE D'ÉPREUVES}}



{\large \textsc{CONCOURS D ADMISSION DE 2010}}



{\large \textbf{Concepteur : Edhec}}



\rule{2.39cm}{0.05cm}



{\Large \textbf{OPTION ÉCONOMIQUE}}



{\Large \textbf{MATHÉMATIQUES }}



{\Large Lundi 9 mai, de 14h à 18h}



\rule{2.39cm}{0.05cm}
\end{center}

\textit{La présentation, la lisibilité, l'orthographe, la qualité
de la rédaction, la clarté et la précision des raisonnements
entreront pour une part importante dans l'appréciation des copies.}

\textit{Les candidats sont invités à \textbf{encadrer} dans la mesure
du possible les résultats de leurs calculs.}

\textit{Ils ne doivent faire usage d'aucun document. L'utilisation de
toute
calculatrice et de tout matériel électronique est interdite. Seule
l'utilisation d'une règle graduée est autorisée.}

\textit{Si au cours de l'épreuve, un candidat repère ce qui lui semble
être une erreur d'énoncé, il la signalera sur sa copie et
poursuivra sa composition en expliquant les raisons des initiatives
qu'il sera
amené à prendre.}

\vspace*{3cm}

\section*{Exercice 1}

On considère la fonction $f$ définie, pour tout couple $(x,y)$ de
l'ouvert $]0, + \infty[ \ \times]0, + \infty[$, par : 
\[
f(x,y) = (x + y)\left(\dfrac 1x + \dfrac 1y\right). 
\]

\begin{noliste}{1.}
 \setlength{\itemsep}{4mm}
\item Montrer que, pour tout couple $(x,y)$ de $]0, + \infty \lbrack
\times
]0, + \infty \lbrack $, on a : 
\[
f(x,y) = 2 + \dfrac{y}{x} + \dfrac{x}{y}\quad \text{ et }\quad f(x,y) =
\dfrac{(x + y)^{2}}{xy}.
\]

\item Montrer que $f$ est de classe $C^{2}$ sur $]0, + \infty \lbrack
\times
]0, + \infty \lbrack $.

\item Montrer que $f$ possède une infinité de points critiques et les
déterminer.

\item Déterminer les dérivées partielles secondes de $f$ et vérifier
que ces
dernières ne permettent pas de conclure à l'existence d'un extremum
local de 
$f$ sur $]0, + \infty \lbrack \times ]0, + \infty \lbrack $.

\item 
\begin{noliste}{a)}
 \setlength{\itemsep}{2mm}
\item Comparer les réels $(x + y)^{2}$ et $4xy$.

\item En déduire que $f$ admet sur $]0, + \infty \lbrack \times ]0, +
\infty
\lbrack $ un minimum global en tous ses points critiques et donner sa
valeur.
\end{noliste}

\item Soit $g$ la fonction définie pour tout $(x,y)$ de $]0, + \infty
\lbrack
\times ]0, + \infty \lbrack $, par : 
\[
g(x,y) = 2\ln (x + y)-\ln x-\ln y.
\]
Montrer que : $\forall (x,y)\in \ ]0, + \infty \lbrack \times ]0, +
\infty \lbrack,\,g(x,y)\geq 2\ln 2$.
\end{noliste}

\section*{Exercice 2}

Pour tout entier naturel $n$, on pose $u_{n} = \prod_{k = 0}{n}\left(1
+ \dfrac 1{2^{k}}\right) = (1 + 1)\left(1 + \dfrac
1{2}\right)\left(1 + \dfrac 1{4}\right)\cdots \left(1 + \dfrac
1{2^{n}}\right)$.

\begin{noliste}{1.}
 \setlength{\itemsep}{4mm}
\item Donner, sous forme d'entiers ou de fractions simplifiées, les
valeurs
de $u_{0},u_{1}$ et $u_{2}$.

\item 
\begin{noliste}{a)}
 \setlength{\itemsep}{2mm}
\item Montrer que, pour tout entier naturel $n$, on a : $u_{n}\geq 2$.

\item Exprimer $u_{n + 1}$ en fonction de $u_{n}$ puis en déduire les
variations de la suite $(u_{n})$.

\item 
Établir que, pour tout réel $x$ strictement supérieur à $-1$, on a :
$\ln
(1 + x)\leq x$.

\item En déduire, pour tout entier naturel $n$, un majorant de $\ln
(u_{n})$.
\end{noliste}

\item En utilisant les questions précédentes, montrer que la suite
$(u_{n})$
converge vers un réel $\ell$, élément de
$\left[2,\mathrm{e}{2}\right]$.

\item On se propose dans cette question de déterminer la nature de la
série
de terme général $(\ell -u_{n})$.

\begin{noliste}{a)}
 \setlength{\itemsep}{2mm}
\item Justifier que la suite $(\ln (u_{n}))_{n\in {\mathbf{N}}}$
converge et
que l'on a : $\ln (\ell ) = \Sum{k = 0}{+ \infty }\ln \left( 1 +
\dfrac{1}{2^{k}}\right) $.

\item Montrer que, pour tout $n$ de ${\mathbf{N}}$, on a $\ln \left(
\dfrac{\ell }{u_{n}}\right) = \Sum{k = n + 1}{+ \infty }\ln \left( 1 +
\dfrac{1}{2^{k}}\right) $.

\item Vérifier, en utilisant le résultat de la question 3a), que
$\forall
n\in {\mathbf{N}},\,0\leq \ln \left( \dfrac{\ell }{u_{n}}\right) \leq
\dfrac{1}{2^{n}}$.

\item Déduire de la question précédente que $\forall n\in
{\mathbf{N}}$, $0\leq \ell -u_{n}\leq \ell \left(
1-\mathrm{e}{-\frac{1}{2^{n}}}\right) $.

\item Justifier que, pour tout réel $x$, on a $1-\mathrm{e}{-x}\leq x$.
En déduire que $\forall n\in {\mathbf{N}}$, $0\leq \ell -u_{n}\leq
\dfrac{\ell }{2^{n}}$.

Conclure quant à la nature de la série de terme général $(\ell
-u_{n})$.
\end{noliste}
\end{noliste}

\section*{Exercice}

On considère la fonction $f$ définie sur ${\mathbf{R}}$ par : $f(x) =
\left\{
\begin{array}{cl}
\dfrac{1}{2x^{2}} & \text{si }x\leq -1\text{ ou }x\geq 1 \\
0 & \text{sinon}
\end{array}
\right.$.[1)]

\begin{noliste}{1.}
 \setlength{\itemsep}{4mm}
\item 
\begin{noliste}{a)}
 \setlength{\itemsep}{2mm}
\item Vérifier que $f$ est une fonction paire.

\item Montrer que $f$ peut être considérée comme une fonction densité
de
probabilité.
\end{noliste}

Dans la suite, on considère une variable aléatoire $X$ définie sur un
espace
probabilisé $(\Omega,\mathcal{A},P)$, admettant $f$ comme densité. On
note $F_{X}$ la fonction de répartition de $X$.

\item La variable aléatoire $X$ admet-elle une espérance ?

\item On pose $Y = \ln (|X|)$ et on admet que $Y$ est une variable
aléatoire,
elle aussi définie sur l'espace probabilisé $(\Omega,\mathcal{A},P)$.
On
note $F_{Y}$ sa fonction de répartition.

\begin{noliste}{a)}
 \setlength{\itemsep}{2mm}
\item Montrer que, pour tout réel $x$, on a : $F_{Y}(x) =
F_{X}(\mathrm{e}{x})-F_{X}(-\mathrm{e}{x})$.

\item Montrer, sans expliciter la fonction $F_{Y}$, que $Y$ est une
variable
aléatoire à densité, puis donner une densité de $Y$ et vérifier que $Y$
suit
une loi exponentielle dont on donnera le paramètre.

\item 
Montrer que, si $x$ est positif, alors $1-\mathrm{e}{-x}$ appartient à
$[0,1[$ et montrer que, si $x$ est strictement négatif, alors
$1-\mathrm{e}{-x}$ est strictement négatif.

\item On considère une variable aléatoire $U$ suivant la loi uniforme
sur $[0,1[$. Déterminer la fonction de répartition de la variable
aléatoire $Z = -\ln (1-U)$ et reconnaître la loi de $Z$.

\item Simulation informatique de la loi de $Y$. On rappelle qu'en 
\Scilab{}, la fonction \texttt{random} permet de simuler la loi
uniforme sur $[0,1[$. Compléter la déclaration de fonction suivante
pour qu'elle simule la
loi de $Y$.

\texttt{Function\ expo :real;}

\texttt{Begin}

\texttt{\hspace*{1cm}expo : = \ ---------\ ;}

\texttt{End;}
\end{noliste}
\end{noliste}

\section*{Problème}

On note $\mathcal{B} = (e_{1},e_{2},e_{3})$ la base canonique de
${\mathbf{R}}{3}$ et on considère l'endomorphisme de ${\mathbf{R}}{3}$
défini par les égalités suivantes : 
\[
f(e_{1}) = \dfrac{1}{3}(e_{2} + e_{3})\quad \text{ et }\quad f(e_{2}) =
f(e_{3}) = \dfrac{2}{3}e_{1}.
\]
\textbf{Partie 1 : Étude de \mathversion{bold}$f$.}

\begin{noliste}{1.}
 \setlength{\itemsep}{4mm}
\item 
\begin{noliste}{a)}
 \setlength{\itemsep}{2mm}
\item Écrire la matrice de $f$ dans la base $\mathcal{B}$.

\item Déterminer la dimension de Im$~f$ puis celle de Ker$~f$.

\item Donner alors une base de Ker$~f$, puis en déduire une valeur
propre de 
$f$ ainsi que le sous-espace propre associé.

\item Déterminer les autres valeurs propres de $f$ ainsi que les
sous-espaces propres associés.

\item En déduire que $f$ est diagonalisable.
\end{noliste}

\item On pose $P = 
\begin{smatrix}
2 & -2 & 0 \\
1 & 1 & 1 \\
1 & 1 & -1
\end{smatrix}
$, $Q = 
\begin{smatrix}
1 & 1 & 1 \\
-1 & 1 & 1 \\
0 & 2 & -2
\end{smatrix}
$ et $I = 
\begin{smatrix}
1 & 0 & 0 \\
0 & 1 & 0 \\
0 & 0 & 1
\end{smatrix}
$.

\begin{noliste}{a)}
 \setlength{\itemsep}{2mm}
\item Justifier sans calcul que $P$ est inversible, puis déterminer la
matrice $D$ diagonale telle que : $M = PDP^{-1}$.

\item Calculer $PQ$ puis en déduire $P^{-1}$.

\item Montrer par récurrence que, pour tout entier naturel $j$, on a
$M^{j} = PD^{j}P^{-1}$.

\item Écrire, pour tout entier naturel $j$ non nul, la première colonne
de
la matrice $M^{j}$. Vérifier que ce résultat reste valable si $j = 0$.
\end{noliste}
\end{noliste}

\ 

\textbf{Partie 2 : Étude d'une suite de variables aléatoires.}

Une urne contient trois boules numérotées de $1$ à $3$. Un tirage
consiste à
extraire au hasard une boule de l'urne puis à la remettre dans l'urne
pour
le tirage suivant.

On définit une suite de variables aléatoires
$(X_{k})_{k\in{\mathbf{N}}^*}$
de la manière suivante :

\begin{noliste}{1.}
 \setlength{\itemsep}{4mm}
\item 
\begin{noliste}{$\sbullet$}
\item Pour tout entier naturel $k$ non nul, $X_{k}$ est définie
\emph{après}
le $k^{\grave{e}me}$ tirage.

\item On procède au 1\textsuperscript{er} tirage et $X_{1}$ prend la
valeur
du numéro de la boule obtenue à ce tirage.

\item Après le $k^{\grave{e}me}$ tirage ($k\in {\mathbf{N}}{\ast }$)
:\\
Soit $X_{k}$ a pris la valeur $1$, dans ce cas on procède au $(k +
1)^{\grave{e}me}$ tirage et $X_{k + 1}$ prend la valeur du numéro
obtenu à ce $(k + 1)^{\grave{e}me}$ tirage.\\
Soit $X_{k}$ a pris la valeur $j$, différente de $1$, dans ce cas on
procède 
également au $(k + 1)^{\grave{e}me}$ tirage et $X_{k + 1}$ prend la
valeur $j$
si la boule tirée porte le numéro $j$ et la valeur $1$ sinon.
\end{noliste}

\item Reconnaître la loi de $X_{1}$.

\item Simulation informatique de l'expérience aléatoire décrite dans
cette
partie.

On rappelle que \texttt{random(n)} renvoie un entier compris entre $0$
et 
\texttt{n}$-1$.

Compléter le programme suivant pour qu'il simule l'expérience aléatoire
décrite dans cette partie et pour qu'il affiche la valeur de la
variable $X_{k}
$, l'entier $k$ étant entré au clavier par l'utilisateur.

\texttt{Program\ simul;}

\texttt{var\ i,k,X,tirage\ :\ integer;}

\texttt{Begin}

\texttt{\hspace*{1cm}Readln(k);\ X : = random(3) + 1;}

\texttt{\hspace*{1cm}For\ i : = 2\ to\ k\ do\ begin}

\texttt{\hspace*{2cm}tirage : = random(3) + 1;}

\texttt{\hspace*{2cm}If\ X = 1\ then\ X : = \ -------}

\texttt{\hspace*{2cm}else\ If\ tirage\ \ < > \ X\
then\ X : = \ -------;}

\texttt{\hspace*{1cm}end;}

\texttt{\hspace*{1cm}Writeln(X);}

\texttt{End.}

\item On note $U_{k}$ la matrice à $3$ lignes et une colonne dont
l'élément
de la $i^{\grave{e}me}$ ligne est $P\left(\Ev{X_{k} = i}\right)$.

\begin{noliste}{a)}
 \setlength{\itemsep}{2mm}
\item Déterminer les probabilités $P_{(X_{k} = j)}(X_{k + 1} = i)$,
pour tout
couple $(i,j)$ de $\{1,2,3\}\times \{1,2,3\}$.

\item On admet que $\left\{ (X_{k} = 1),(X_{k} = 2),(X_{k} = 3)\right\}
$ est un
système complet d'évènements.

Déterminer, grâce à la formule des probabilités totales, la matrice $A$
de $\M{3}$, telle que, pour tout entier naturel $k$ non
nul, on a $U_{k + 1} = AU_{k}$.

\item Montrer qu'en posant $U_{0} = 
\begin{smatrix}
1 \\
0 \\
0
\end{smatrix}
$, alors, pour tout $k$ de ${\mathbf{N}}$, on a : $U_{k} = AU_{0}$.

\item 
Vérifier que $A = M + \dfrac{1}{3}I$, puis établir que, pour tout $k$
de ${\mathbf{N}}$, on a : $A^{k} = \Sum{j = 0}{k}\binom{k}{j}\left( 
\dfrac{1}{3}\right) ^{k-j}M^{j}$.

\item En déduire les $3$ éléments de la première colonne de la matrice
$A^{k}
$, puis vérifier que la loi de $X_{k}$ est donnée par : 
\[
\forall k\in {\mathbf{N}}{\ast },\quad P\left(\Ev{X_{k} = 1}\right) =
\frac{1}{2}\left(
1 + \left( -\dfrac{1}{3}\right) ^{k}\right) \quad \text{et}\quad
P\left(\Ev{X_{k} = 2}\right) = P\left(\Ev{X_{k} = 3}\right) =
\frac{1}{4}\left( 1-\left( -\dfrac{1}{3}\right)
^{k}\right).
\]

\item Montrer que la suite $(X_{k})$ converge en loi vers une variable
aléatoire $X$ dont on donnera la loi.

\item 
Calculer l'espérance $\E(X_{k})$ de $X_{k}$.

\item Écrire une fonction \Scilab{}, notée \texttt{esp}, qui renvoie
$\E(X_{k})$ 
à l'appel de \texttt{esp(k)}.
\end{noliste}
\end{noliste}

\end{document}

