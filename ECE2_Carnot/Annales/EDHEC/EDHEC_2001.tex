\documentclass[11pt]{article}%
\usepackage{geometry}%
\geometry{a4paper,
 lmargin = 2cm,rmargin = 2cm,tmargin = 2.5cm,bmargin = 2.5cm}

\input{../../macros.tex}

\pagestyle{fancy} %
\lhead{ECE2 \hfill Mathématiques\\
} %
\chead{\hrule} %
\rhead{} %
\lfoot{} %
\cfoot{} %
\rfoot{\thepage} %

\renewcommand{\headrulewidth}{0pt}% : Trace un trait de séparation
 % de largeur 0,4 point. Mettre 0pt
 % pour supprimer le trait.

\renewcommand{\footrulewidth}{0.4pt}% : Trace un trait de séparation
 % de largeur 0,4 point. Mettre 0pt
 % pour supprimer le trait.

\setlength{\headheight}{14pt}

\title{\bf \vspace{-2cm} EDHEC 2001} %
\author{} %
\date{} %
\begin{document}

\maketitle %
\vspace{-1.4cm}\hrule %
\thispagestyle{fancy}

\vspace*{.2cm}


% DEBUT DU DOC À MODIFIER : tout virer jusqu'au début de l'exo

%Définition et changement de valeurs de
compteurs%newcounter{cpt1}{section} compteur cpt1 remis à 0 à chaque
aumentation par stepcounter du compteur section%setcounter{cpt1}{3} on
met le compteur à 3%addtocounter{cpt1}{5} on ajoute 5 au compteur%
stepcounter{cpt1} on ajoute 1% ifthenelse{test}{alors}{sinon} (page
206) pour subordonner à une condition % whiledo{test}{commande} pour
faire une boucle (page 206 aussi) % value{cpt1} pour noter dans le
document la valeur de cpt1 
%Définition définitive d'opérateurs
mathématiques\newcommand{\ch}{\operatorname{ch}} 
\newcommand{\sh}{\operatorname{sh}}
\renewcommand{\tanh}{\operatorname{th}}
\renewcommand{\sinh}{\operatorname{sh}}
\renewcommand{\cosh}{\operatorname{ch}}
\newcommand{\argsh}{\operatorname{argsh}}
\newcommand{\argch}{\operatorname{argch}}
\newcommand{\argth}{\operatorname{argth}}
\newcommand{\ker}{\operatorname{Ker}}
\renewcommand{\im}{\operatorname{Im}}
\newcommand{\rg}{\operatorname{rg}}
\newcommand{\Id}{\operatorname{Id}}
\newcommand{\id}{\operatorname{id}}
\renewcommand{\leq}{\leq}
\renewcommand{\geq}{\geq }

%Définition de nouvelles couleurs : rgb(trois paramètres red green blue
entre 0 et 1); cmyk (quatre cyan magenta yellow black) entre 0 et 1;
gray (entre 0 et 1) et black, white, red, green, blue, cyan, magenta,
yellow% definecolor{0gris}{gray}{0.8} 
% Nouvelle commande pour encadrer le titre car shabox ne veut que d'une
seule ligne; ATTENTION A LA TAILLE; petite différence avec shadowbox ou
doublebox, voire fcolorbox ou colorbox (au lieu de shabox; laisser le
parbox tranquille sauf pour la taille de la boîte
\newcommand{\Tbox}[1]{\begin{center} \shabox{\parbox{0.6
\linewidth}{#1}} \end{center}} %[1] pour 1 paramètre ; #1 pour ce que
fait le 1er paramètre; entre accolades ce que fait la commande
%Mise en page en mode fancy : en-têtes et pieds de pages puis
définition des en-têtes et pieds de pages\pagestyle{fancy}
\lhead{ECE 2 - Mathématiques \\
Quentin Dunstetter - ENC-Bessières 2011$\backslash$2012}
\chead{}
\rhead{Edhec 2001}
\rfoot[ \ \thepage]{\thepage}
\cfoot{}
\lfoot{}
\thispagestyle{fancy} %Mise en page de la 1ère page en mode fancy
%Trait en bas et en haut de la page (entre en-tête et texte et texte et
pied de page)\renewcommand{\footrulewidth}{0.4pt}
\renewcommand{\headrulewidth}{0.4pt}


%DEBUT DU DOCUMENT\vspace*{3cm}

\begin{center}
{\LARG\E\textbf{BANQUE COMMUNE D'ÉPREUVES}}



{\large \textsc{CONCOURS D ADMISSION DE 2001}}



{\large \textbf{Concepteur : Edhec}}



\rule{2.39cm}{0.05cm}



{\Large \textbf{OPTION ÉCONOMIQUE}}



{\Large \textbf{MATHÉMATIQUES }}



{\Large Lundi 9 mai, de 14h à 18h}



\rule{2.39cm}{0.05cm}
\end{center}

\textit{La présentation, la lisibilité, l'orthographe, la qualité
de la rédaction, la clarté et la précision des raisonnements
entreront pour une part importante dans l'appréciation des copies.}

\textit{Les candidats sont invités à \textbf{encadrer} dans la mesure
du possible les résultats de leurs calculs.}

\textit{Ils ne doivent faire usage d'aucun document. L'utilisation de
toute
calculatrice et de tout matériel électronique est interdite. Seule
l'utilisation d'une règle graduée est autorisée.}

\textit{Si au cours de l'épreuve, un candidat repère ce qui lui semble
être une erreur d'énoncé, il la signalera sur sa copie et
poursuivra sa composition en expliquant les raisons des initiatives
qu'il sera
amené à prendre.}

\vspace*{3cm}

\subsection*{Exercice 1}

$E$ désigne un espace vectoriel réel sur $\R,$ rapporté à sa base
$\mathcal{B} = (e_{1},e_{2},e_{3})$.\\
On désigne par $a$ un réel non nul et on considère l'endomorphisme
$f_{a}$
de E, défini par : 
\[
f_{a}(e_{2}) = 0\quad f_{a}(e_{1}) = f_{a}(e_{3}) = ae_{1} +
e_{2}-ae_{3}
\]

\begin{noliste}{1.}
 \setlength{\itemsep}{4mm}
\item 

\begin{noliste}{a)}
 \setlength{\itemsep}{2mm}
\item Écrire la matrice $A_{a}$ de $f_{a}$ relativement à la base B et
calculer $A_{a}{2}$.

\item Montrer que $0$ est la seule valeur propre de $A_{a}$.

\item $A_{a}$ est-elle diagonalisable ? Est-elle inversible ?
\end{noliste}

\item On pose $u_{1} = ae_{1} + e_{2}-ae_{3}$.

\begin{noliste}{a)}
 \setlength{\itemsep}{2mm}
\item Montrer que $\mathcal{B}{\prime } = (u_{1},e_{2},e_{3})$ est une
base
de $E$\\

\item Vérifier que la matrice de $f_{a}$ relativement à la base
$\mathcal{B}{\prime }$ est $K = 
\begin{smatrix}
0 & 0 & 1 \\
0 & 0 & 0 \\
0 & 0 & 0
\end{smatrix}
$.
\end{noliste}

Dans la suite, on cherche à caractériser les endomorphismes $g$ de E
tels
que $g\circ g = f_{a}$.

\item On suppose qu'un tel endomorphisme $g$ existe et on note M sa
matrice
dans $\mathcal{B}{\prime }$.

\begin{noliste}{a)}
 \setlength{\itemsep}{2mm}
\item Expliquer pourquoi $M^{2} = K$ puis montrer que $MK = KM$.

\item Déduire de ces deux relations que $M = 
\begin{smatrix}
0 & x & y \\
0 & 0 & z \\
0 & 0 & 0
\end{smatrix}
$, $x$, $y$ et $z$ étant 3 réels tels que $xz = 1$.
\end{noliste}

\item Réciproquement, vérifier que tout endomorphisme $g$ dont la
matrice
dans $\mathcal{B}{\prime }$ est du type ci-dessus est solution de
$g\circ
g = f_{a}$.
\end{noliste}

\section*{Exercice 2}

On désigne par n un entier naturel supérieur ou égal à 2.\\
On considère une épreuve aléatoire pouvant aboutir à 3 résultats
différents $R_{1},R_{2}$ et $R_{3}$ de probabilités respectives
$P_{1},P_{2}$ et $P_{3}$. On a donc $P_{1} + P_{2} + P_{3} = 1$ et on
admet que, pour tout $i$ de $\{1,2,3\}$, $0<P_{i}<1$.\\
On effectue $n$ épreuves indépendantes du type de celle décrite
ci-dessus.\\
Pour tout $i$ de $\{1,2,3\}$, on note $X_{i}$ la variable aléatoire qui
vaut 
$1$ si le résultat numéro $i$ n'est pas obtenu à l'issue des $n$
épreuves et 
$0$ sinon.\\
On désigne par $X$ la variable égale au nombre de résultats qui n'ont
pas été
obtenus à l'issue des $n$ épreuves.

\begin{noliste}{1.}
 \setlength{\itemsep}{4mm}
\item 

\begin{noliste}{a)}
 \setlength{\itemsep}{2mm}
\item Justifier soigneusement que $X = X_{1} + X_{2} + X_{3}.$

\item Donner la loi de $X_{i}$, pour tout $i$ de $\{1,2,3\}$.

\item En déduire l'espérance de $X$, notée $\E(X)$.
\end{noliste}

La suite de cet exercice consiste à rechercher les valeurs des réels
$P_{i}$
en lesquelles $\E(X)$ admet un minimum local. Pour ce faire, on note
$f$ la
fonction définie sur l'ouvert $]0,1[ \ \times ]0,1[$ de $\R^{2}$ par :
$f(x,y) = (1-x)^{n} + (1-y)^{n} + (x + y)^{n}$.

\item 

\begin{noliste}{a)}
 \setlength{\itemsep}{2mm}
\item On pose $P_{1} = x$ et $P_{2} = y$. Vérifier que $\E(X) =
f(x,y)$.

\item Montrer que $f$ est une fonction de classe $C^{2}$ sur $]0,1[ \
\times
]0,1[$.

\item Déterminer les dérivées partielles d'ordre $1$ de $f$.

\item En déduire que le seul point en lesquelles les dérivées
partielles
d'ordre $1$ de f s'annulent simultanément est le point $\left(
\dfrac{1}{3},\dfrac{1}{3}\right) $.

\item Démontrer que $f$ présente un minimum local en ce point.

\item Donner la valeur de $\E(X)$ correspondant à ce minimum.
\end{noliste}
\end{noliste}

\section*{Exercice 3}

Soit $f$ la fonction définie par : $\left\{ 
\begin{array}{l}
f(x) = 0\text{ si }x<0 \\
f(x) = xe^{-{\dfrac{x^{2}}{2}}}\text{ si }x\geq 0
\end{array}
\right. $

\begin{noliste}{1.}
 \setlength{\itemsep}{4mm}
\item Vérifier que f est une densité de probabilité.\\
La durée de vie d'un certain composant électronique est une variable
aléatoire $X$ dont une densité est $f$.

\item 

\begin{noliste}{a)}
 \setlength{\itemsep}{2mm}
\item Déterminer la fonction de répartition $F$ de $X$.

\item Calculer la médiane de $X$ c'est-à-dire le réel $\mu $ tel que
$p(X\leq \mu ) = \dfrac{1}{2}$.
\end{noliste}

\item On appelle mode de la variable $X$ tout réel x en lequel $f$
atteint
son maximum. Montrer que X a un seul mode, noté $M_{o}$, et le
déterminer.

\item 

\begin{noliste}{a)}
 \setlength{\itemsep}{2mm}
\item En utilisant un résultat connu concernant la loi normale, établir
que
X a une espérance et montrer que $\E(X) = \dfrac{\sqrt{2\pi }}{2}$.

\item Calculer, à l'aide d'une intégration par parties, la variance de
$X$.\vspace{0.5cm}
\end{noliste}
\end{noliste}

\section*{Problème}

\subsection*{Partie 1}

On pose, pour tout entier $n$ supérieur ou égal à $1$, $v_{n} = \Sum{k
= 1}{n}\dfrac{1}{k}$.

\begin{noliste}{1.}
 \setlength{\itemsep}{4mm}
\item Montrer que : $\forall k\in \N^{\ast },\quad \dfrac{1}{k + 1}\leq
\dint{k}{k + 1}\dfrac{dt}{t}$.

\item En déduire que : $\forall n\in \N^{\ast },\quad v_{n}\leq
\ln (n) + 1$.
\end{noliste}

\subsection*{Partie 2}

On considère une suite $(u_{n})$ définie par son premier terme $u_{0} =
1$ et
par la relation suivante, valable pour tout entier n : $u_{n + 1} =
u_{n} + \dfrac{1}{u_{n}}$.

\begin{noliste}{1.}
 \setlength{\itemsep}{4mm}
\item 

\begin{noliste}{a)}
 \setlength{\itemsep}{2mm}
\item Montrer par récurrence que chaque terme de cette suite est
parfaitement défini et strictement positif.

\item En déduire le sens de variation de la suite $(u_{n})$.
\end{noliste}

\item 

\begin{noliste}{a)}
 \setlength{\itemsep}{2mm}
\item Pour tout entier $k$, exprimer $u_{k + 1}{2}-u_{k}{2}$ en
fonction de $u_{k}{2}$.

\item En déduire que : $\forall n\in \N^{\ast },\quad
u_{n}{2} = 2n + 1 + \Sum{k = 0}{n-1}\dfrac{1}{u_{k}{2}}$.

\item Montrer que : $\forall n\in \N^{\ast },\quad
u_{n}{2}\geq 2n + 1$. En déduire la limite de la suite $(u_{n})$.
\end{noliste}

\item 

\begin{noliste}{a)}
 \setlength{\itemsep}{2mm}
\item À l'aide du résultat précédent, montrer que, pour tout entier $n$
supérieur ou égal à 2 : \\
$u_{n}\leq 2n + 2 + \dfrac{1}{2}v_{n-1}$.

\item En utilisant la partie 1., établir que, pour tout entier $n$
supérieur
ou égal à 2 : \\
$u_{n}{2}\leq 2n + \dfrac{5}{2} + \dfrac{\ln (n-1)}{2}$.

\item En déduire finalement que $u_{n}\sim \sqrt{2n}$ quand
$n\rightarrow
 + \infty $.
\end{noliste}
\end{noliste}

\subsection*{Partie 3}

\begin{noliste}{1.}
 \setlength{\itemsep}{4mm}
\item Écrire un programme en \Scilab{} permettant de calculer et
d'afficher $u_{n}$ lorsque l'utilisateur entre la valeur de $n$ au
clavier.

\item 

\begin{noliste}{a)}
 \setlength{\itemsep}{2mm}
\item Écrire un deuxième programme, toujours en \Scilab{}, qui permette
de déterminer et d'afficher le plus petit entier naturel $n$ pour
lequel $u_{n}\geq 100$.

\item On donne $ln2<0,70$ et $ln5<1,61$. En déduire un majorant de $\ln
5000$.

\item Montrer que l'entier $n$ trouvé en 2a) est compris entre $4995$
et $5000$.
\end{noliste}
\end{noliste}

\label{fin}

\end{document}

