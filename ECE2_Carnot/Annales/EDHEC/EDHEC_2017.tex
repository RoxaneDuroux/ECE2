\documentclass[11pt]{article}%
\usepackage{geometry}%
\geometry{a4paper,
 lmargin = 2cm,rmargin = 2cm,tmargin = 2.5cm,bmargin = 2.5cm}

\input{../../macros.tex}

\pagestyle{fancy} %
\lhead{ECE2 \hfill Mathématiques\\
} %
\chead{\hrule} %
\rhead{} %
\lfoot{} %
\cfoot{} %
\rfoot{\thepage} %

\renewcommand{\headrulewidth}{0pt}% : Trace un trait de séparation
 % de largeur 0,4 point. Mettre 0pt
 % pour supprimer le trait.

\renewcommand{\footrulewidth}{0.4pt}% : Trace un trait de séparation
 % de largeur 0,4 point. Mettre 0pt
 % pour supprimer le trait.

\setlength{\headheight}{14pt}

\title{\bf \vspace{-2cm} EDHEC 2017} %
\author{} %
\date{} %
\begin{document}

\maketitle %
\vspace{-1.4cm}\hrule %
\thispagestyle{fancy}

\vspace*{.2cm}

\section*{Exercice 1}

\noindent
On considère la fonction $f$ qui à tout couple $(x,y)$ de $\R^{2}$
associe le réel :
\[
f(x,y) = x^{4} + y^{4}-2(x-y)^{2}
\]
\begin{noliste}{1.}
  \setlength{\itemsep}{4mm}
\item Justifier que $f$ est de classe $C^{1}$ sur $\R^{2}$.

\item
  \begin{noliste}{a)}
    \setlength{\itemsep}{2mm}
  \item Calculer les dérivées partielles d'ordre $1$ de $f$.
  \item Montrer que le gradient de $f$ est nul si, et seulement si, on
    a : $ \left\{
      \begin{array}{rcl}
        x^{3}-x + y & = & 0 \\
        y^{3} + x-y & = & 0
      \end{array}
    \right.$.
  \item En déduire que $f$ possède trois points critiques : $(0,0)$,
    $(\sqrt{2},-\sqrt{2}), (-\sqrt{2},\sqrt{2})$.
  \end{noliste}

\item
  \begin{noliste}{a)}
    \setlength{\itemsep}{2mm}
  \item Calculer les dérivées partielles d'ordre 2 de $f$.
  \item Écrire la matrice hessienne de $f$ en chaque point critique.
  \item Déterminer les valeurs propres de chacune de ces trois
    matrices puis montrer que $f$ admet un minimum local en deux de
    ses points critiques. Donner la valeur de ce minimum.
  \item Déterminer les signes de $f(x,x)$ et $f(x,-x)$ au voisinage de
    $x = 0$. Conclure quant à l'existence d'un extremum en le troisième
    point critique de $f$.
  \end{noliste}

\item
  \begin{noliste}{a)}
    \setlength{\itemsep}{2mm}
  \item Pour tout $(x,y)$ de $\R^{2}$, calculer
    $f(x,y)-(x^{2}-2)^{2}-(y^{2}-2)^{2}-2(x + y)^{2}$.
  \item Que peut-on déduire de ce calcul quant au minimum de $f$ ?
  \end{noliste}

\item
  \begin{noliste}{a)}
    \setlength{\itemsep}{2mm}
  \item Compléter la deuxième ligne du script suivant afin de définir
    la fonction $f$.
    \begin{scilab}
      & \tcFun{function} \tcVar{z} = \underline{f}(\tVar{x},\tcVar{y}) \nl
      & \quad \tcVar{z} = ------ \nl
      & \tcFun{endfunction} \nl 
      & x = linspace(-2,2,101) \nl
      & y = x \nl
      & fplotd3d(x,y,f) \nl
    \end{scilab}
  \item Le script précédent, une fois complété, renvoie l'une des
    trois nappes suivantes. Laquelle ? \\
    Justifier la réponse.
    \begin{figure}[ + h]
      \includegraphics[width = 4cm,height = 4cm]{nappe1.pdf}
      \hfill{
        \includegraphics[width = 4cm,height = 4cm]{nappe2.pdf}
      }
      \hfill{
        \includegraphics[width = 4cm,height = 4cm]{nappe3.pdf}
      }
    \end{figure}\\
    \hfill{} \hfill{Nappe 1}\hfill{Nappe 2}\hfill{Nappe 3}\\
  \end{noliste}
\end{noliste}

\section*{Exercice 2}

\noindent
On note $E$ l'espace vectoriel des fonctions polynomiales de degré
inférieur ou égal à $2$ et on rappelle que la famille
$(e_{0},e_{1},e_{2})$ est une base de $E$, les fonctions $e_{0}$,
$e_{1}$ $e_{2}$ étant définies par :
\[
\forall t \in \R \quad e_{0}(t) = 1 \quad e_{1}(t) = t \quad e_{2}(t)
= t^{2}
\]
On considère l'application $\varphi$ qui, à toute fonction $P$ de $E$,
associe la fonction, notée $\varphi(P)$, définie par :
\[
\forall x \in \R \qquad \left(\varphi(P)\right)(x) = \dint{0}{1} P(x +
t) \ dt
\]
\begin{noliste}{1.}
 \setlength{\itemsep}{4mm}
\item
  \begin{noliste}{a)}
    \setlength{\itemsep}{2mm}
  \item Montrer que $\varphi$ est linéaire.
  \item Déterminer $\left(\varphi(e_{0})\right)(x)$,
    $\left(\varphi(e_{1})\right)(x)$ et
    $\left(\varphi(e_{2})\right)(x)$ en fonction de $x$, puis écrire
    $\varphi(e_{0})$, $\varphi(e_{1})$ et $\varphi(e_{2})$ comme
    combinaison linéaire de $e_{0}$, $e_{1}$ et $e_{2}$.
  \item Déduire des questions précédentes que $\varphi$ est un
    endomorphisme de $E$.
  \end{noliste}

\item
  \begin{noliste}{a)}
    \setlength{\itemsep}{2mm}
  \item Écrire la matrice $A$ de $\varphi$ dans la base
    $(e_{0},e_{1},e_{2})$. On vérifiera que la première ligne de $A$
    est :
    \[
    \begin{smatrix}
      1 & \dfrac{1}{2} & \dfrac{1}{3}
    \end{smatrix}
    \]
  \item Justifier que $\varphi$ est un automorphisme de $E$.
  \item L'endomorphisme $\varphi$ est-il diagonalisable ?
  \end{noliste}

\item Compléter les commandes {\tt Scilab} suivantes pour que soit
  affichée la matrice $A^{n}$ pour une valeur de $n$ entrée par
  l'utilisateur :
  \begin{scilab}
    & n = input(\ttq{}entrez une valeur pour n : \ttq{}) \nl %
    & A = [------] \nl %
    & disp(------)
  \end{scilab}
  
\item
  \begin{noliste}{a)}
    \setlength{\itemsep}{2mm}
  \item Montrer par récurrence que, pour tout entier naturel $n$, il
    existe un réel $u_{n}$ tel que l'on ait :
    \[
    A^{n} = 
    \begin{smatrix}
      1 & \dfrac{n}2 & u_{n}\\
      0 & 1 & n\\
      0 & 0 & 1
    \end{smatrix}
    \]
    Donner $u_{0}$ et établir que : \quad $\forall n \in \N \quad u_{n + 1}
    = u_{n} + \dfrac{1}{6}\left(3n + 2\right)$.
  \item En déduire, par sommation, l'expression de $u_{n}$ pour tout
    entier $n$.
  \item Écrire $A^{n}$ sous forme de tableau matriciel.
  \end{noliste}
\end{noliste}


\newpage


\section*{Exercice 3}

\noindent
Soit $V$ une variable aléatoire suivant la loi exponentielle de
paramètre $1$, dont la fonction de répartition est la fonction $F_{V}$
définie par : $F_{V}(x) = \left\{
\begin{array}{cl}
0 & \hbox{ si } x \leq 0\\
1-\ee^{-x} & \hbox{ si } x>0
\end{array}
\right.$.\\
On pose $W = -\ln(V)$ et on admet que $W$ est aussi une variable
aléatoire dont le fonction de répartition est notée $F_{W}$. On dit
que $W$ suit une loi de Gumbel.
\begin{noliste}{1.}
  \setlength{\itemsep}{4mm}
\item
  \begin{noliste}{a)}
    \setlength{\itemsep}{2mm}
  \item Montrer que : $\forall x \in \R, \ F_{W}(x) =
    \ee^{-\ee^{-x}}$.
  \item En déduire que $W$ est une variable à densité.
  \end{noliste}
\end{noliste}

\begin{noliste}{$\sbullet$}
\item On désigne par $n$ un entier naturel non nul et par $X_{1},
  \ldots, X_{n}$ des variables aléatoires définies sur le même espace
  probabilisé, indépendantes et suivant la même loi que $V$, c'est à
  dire la loi $\Exp{1}$.

\item On considère la variable aléatoire $Y_{n}$ définie par $Y_{n} =
  \max(X_{1},X_{2}, \ldots,X_{n})$, c'est à dire que pour tout
  $\omega$ de $\Omega$, on a : $Y_{n}(\omega) =
  \max(X_{1}(\omega),X_{2}(\omega), \ldots, X_n(\omega))$.\\
  On admet que $Y_{n}$ est une variable aléatoire à densité.
\end{noliste}

\begin{noliste}{1.}
  \setlength{\itemsep}{4mm} %
  \setcounter{enumi}{1}
\item
  \begin{noliste}{a)}
    \setlength{\itemsep}{2mm}
  \item Montrer que la fonction de répartition $F_{Y_{n}}$ de $Y_{n}$
    est définie par :
    \[
    F_{Y_{n}}(x) \left\{
      \begin{array}{cl}
        0 & \hbox{ si } x<0\\
        (1-\ee^{-x})^{n} & \hbox{ si } x \geq 0
      \end{array}
    \right.
    \]
  \item En déduire une densité $f_{Y_{n}}$ de $Y_{n}$.
  \end{noliste}
  
\item
  \begin{noliste}{a)}
    \setlength{\itemsep}{2mm}
  \item Donner un équivalent de $1-F_{Y_{n}}(t)$ lorsque $t$ est au
    voisinage de $ + \infty$, puis montrer que l'intégrale $
    \dint{0}{+ \infty} \left(1-F_{Y_{n}}(t) \right) \ dt$ est
    convergente.
  \item Établir l'égalité suivante :
    \[
    \forall x \in \R^+ \quad \dint{0}{x} (1-F_{Y_{n}}(t)) \ dt =
    x\left(1-F_{Y_{n}}(x)\right) + \dint{0}{x} t f_{Y_{n}}(t) \ dt
    \]
  \item Montrer que \quad $ \dlim{ x \to + \infty}
    x\left(1-F_{Y_{n}}(x)\right) = 0$.
  \item En déduire que $Y_{n}$ possède une espérance et prouver
    l'égalité :
    \[
    \E(Y_{n}) = \dint{0}{+ \infty} \left(1-F_{Y_{n}}\right) \ dt
    \]
  \end{noliste}
  
\item
  \begin{noliste}{a)}
    \setlength{\itemsep}{2mm}
  \item Montrer, grâce au changement de variable $u = 1-\ee^{-t}$, que
    l'on a :
    \[
    \forall x \in \R^+ \quad \dint{0}{x} \left(1-F_{Y_{n}}(t)\right) \
    dt = \dint{0}{1 - \ee^{-x}} \dfrac{1-u^{n}}{1-u} \ du
    \]
  \item En déduire que : $ \dint{0}{x} \left(1-F_{Y_{n}}(t)\right) \
    dt = \Sum{k = 1}{n} \dfrac{(1-\ee^{-x})^{k}}{k}$ puis donner
    $\E(Y_{n})$ sous forme de somme.
  \end{noliste}

\item On pose $Z_{n} = Y_{n}-\ln(n)$.
  \begin{noliste}{a)}
    \setlength{\itemsep}{2mm}
  \item On rappelle que {\tt grand(1,n,\ttq{}exp\ttq{},1)} simule
    $n$ variables aléatoires indépendantes et suivant toutes la loi
    exponentielle de paramètre $1$. Compléter la déclaration de
    fonction {\tt Scilab} suivante afin qu'elle simule la variable
    aléatoire $Z_{n}$.
    \begin{scilab}
      & \tcFun{function} \tcVar{Z} = \underline{f}(\tcVar{n}) \nl %
      & \qquad x = grand(1,\tcVar{n},\ttq{}exp\ttq{},1) \nl %
      & \qquad \tcVar{Z} = ------ \nl %
      & \tcFun{endfunction}
    \end{scilab}


\newpage


\item Voici deux scripts :\\
  \begin{minipage}{.45\linewidth}
    \begin{scilab}
      & V = grand(1,10000,'exp',1) \nl %
      & W = -log(V) \nl %
      & s = linspace(0,10,11) \nl %
      & histplot(s,W)
    \end{scilab}
    \begin{center}
      Script (1)
    \end{center}
  \end{minipage}
  \begin{minipage}{.45\linewidth}
    \begin{scilab}
      & n = input(\ttq{}entrez la valeur de n : \ttq{}) \nl %
      & Z = [] \commentaire{la matrice-ligne Z est vide} \nl %
      & for k = 1 :10000 \nl %
      & \qquad Z = [Z,f(n)] \nl %
      & end \nl %
      & s = linspace(0,10,11) \nl %
      & histplot(s,Z)
    \end{scilab}
    \begin{center}
      Script (2)
    \end{center}
  \end{minipage}~\\[.4cm]
  Chacun des scripts simule 10000 variables indépendantes, regroupe
  les valeurs renvoyées en 10 classes qui sont les intervalles
  $[0,1]$, $]1,2]$, $]2,3]$, \dots, $]9,10]$ et trace l'histogramme
  correspondant (la largeur de chaque rectangle est égale à $1$ et
  leur hauteur est proportionnelle à l'effectif de chaque classe).\\
  Le script (1) dans lequel les variables aléatoires suivent la loi de
  Gumbel (loi suivie par $W$), renvoie l'histogramme (1) ci-dessous,
  alors que le script (2) dans lequel les variables aléatoires suivent
  la même loi que $Z_{n}$, renvoie l'histogramme (2) ci-dessous, pour
  lequel on a choisi $n = 1000$.
  \begin{figure}[+h]
    \includegraphics[width = 6cm,height = 6cm]{exo3_{1}.pdf}
    \hfill{
      \includegraphics[width = 6cm,height = 6cm]{exo3_{2}.pdf}
    }
  \end{figure}\\
  Histogramme (1)\hfill{Histogramme(2) pour $n = 1000$}\\[.2cm]
  Quelle conjecture peut-on émettre quant au comportement de la suite
  des variables aléatoires $(Z_{n})$ ?
\end{noliste}

\item On note $F_{Z_{n}}$ la fonction de répartition de $Z_{n}$.
  \begin{noliste}{a)}
    \setlength{\itemsep}{2mm}
  \item Justifier que, pour tout réel $x$, on a : \quad $F_{Z_{n}}(x) =
    F_{Y_{n}}\left(x + \ln(n)\right)$.
  \item Déterminer explicitement $F_{Z_{n}}(x)$.
  \item Montrer que, pour tout réel $x$, on a : \quad $ \dlim{n \to
      + \infty} n \ln\left(1-\dfrac{\ee^{-x}}{n} \right) = -\ee^{-x}$.
  \item Démontrer le résultat conjecturé à la question \itbf{5)b}.
  \end{noliste}
\end{noliste}


\newpage


\section*{Problème}

\subsection*{Partie 1 : étude d'une variable aléatoire}}

\noindent
Les sommets d'un carré sont numérotés 1, 2, 3, et 4 de telle façon que
les côtés du carré relient le sommet 1 au sommet 2, le sommet 2 au
sommet 3, le sommet 3 au sommet 4 et le sommet 4 au sommet 1.\\
Un mobile se déplace aléatoirement sur les sommets de ce carré selon
le protocole suivant :
\begin{noliste}{$\sbullet$}
\item Au départ, c'est à dire à l'instant $0$, le mobile est sur le
  sommet 1.
\item Lorsque le mobile est à un instant donné sur un sommet, il se
  déplace à l'instant suivant sur l'un quelconque des trois autres
  sommets, et ceci de façon équiprobable.
\end{noliste}
Pour tout $n \in \N$, on note $X_{n}$ la variable aléatoire égale au
numéro du sommet sur lequel se situe le mobile à l'instant
$n$. D'après le premier des deux points précédents, on a donc $X_{0} =
1$.
\begin{noliste}{1.}
  \setlength{\itemsep}{4mm}
\item Donner la loi de $X_{1}$, ainsi que l'espérance $\E(X_{1})$ de
  la variable $X_{1}$.\\
  On admet pour la suite que la loi de $X_{2}$ est donnée par :
  \[
  P\left(\Ev{X_{2} = 1}\right) = \dfrac{1}{3} \quad P\left(\Ev{X_{2} =
      2}\right) = P\left(\Ev{X_{2} = 3}\right) = P\left(\Ev{X_{2} = 4}\right)
  = \dfrac{2}{9}
  \]

\item Pour tout entier $n$ supérieur ou égal à 2, donner, en
  justifiant, l'ensemble des valeurs prises par $X_{n}$.

\item
  \begin{noliste}{a)}
    \setlength{\itemsep}{2mm}
  \item Utiliser la formule des probabilités totales pour établir que,
    pour tout entier naturel $n$ supérieur ou égal à 2, on a :
    \[
    P\left(\Ev{X_{n + 1} = 1}\right) =
    \dfrac{1}{3}\left(P\left(\Ev{X_{n} = 2}\right) + P\left(\Ev{X_{n}
          = 3}\right) + P\left(\Ev{X_{n} = 4}\right)\right)
    \]
  \item Vérifier que cette relation reste valable pour $n = 0$ et $n =
    1$.
  \item Justifier que, pour tout $n$ de $\N$, on a $P\left(\Ev{X_{n} =
        1}\right) + P\left(\Ev{X_{n} = 2}\right) + P\left(\Ev{X_{n} =
        3}\right) + P\left(\Ev{X_{n} = 4}\right) = 1$ et en déduire
    l'égalité :
    \[
    \forall n \in \N, \ P\left(\Ev{X_{n + 1} = 1}\right) =
    -\dfrac{1}{3} P\left(\Ev{X_{n} = 1}\right) + \dfrac{1}{3}
    \]
  \item Établir alors que : $\forall n \in \N, \ P\left(\Ev{X_{n} =
        1}\right) = \dfrac{1}{4} +
    \dfrac{3}{4}\left(-\dfrac{1}{3}\right)^{n}$.
  \end{noliste}
  
\item
  \begin{noliste}{a)}
    \setlength{\itemsep}{2mm}
  \item En procédant de la même façon qu'à la question précédente,
    montrer que l'on a :
    \[
    \forall n \in \N, \ P\left(\Ev{X_{n + 1} = 2}\right) =
    \dfrac{1}{3}\left( P\left(\Ev{X_{n} = 1}\right) + P\left(\Ev{X_{n}
          = 3}\right) + P\left(\Ev{X_{n} = 4}\right)\right)
    \]
  \item En déduire une relation entre $P\left(\Ev{X_{n + 1} =
        2}\right)$ et $P\left(\Ev{X_{n} = 2}\right)$.
  \item Montrer enfin que : $\forall n \in \N, P\left(\Ev{X_{n} =
        2}\right) =
    \dfrac{1}{4}-\dfrac{1}{4}\left(-\dfrac{1}{3}\right)^{n}$.
  \end{noliste}

\item On admet que, pour tout entier naturel $n$, on a :
  \[
  P\left(\Ev{X_{n + 1} = 3}\right) = -\dfrac{1}{3} P\left(\Ev{X_{n} =
      3}\right) + \dfrac{1}{3} \quad \hbox{ et } \quad P\left(\Ev{X_{n
        + 1} = 4}\right) = -\dfrac{1}{3} P\left(\Ev{X_{n} = 4}\right)
  + \dfrac{1}{3}
  \]
  En déduire sans calcul que :
  \[
  \forall n \in \N, \ P\left(\Ev{X_{n} = 3}\right) = P\left(\Ev{X_{n}
      = 4}\right) =
  \dfrac{1}{4}-\dfrac{1}{4}\left(-\dfrac{1}{3}\right)^{n}
  \]
\item Déterminer, pour tout entier naturel $n$, l'espérance
  $\E(X_{n})$ de la variable aléatoire $X_{n}$.
\end{noliste}


\newpage


\subsection*{Partie 2 : calcul des puissances d'une matrice $A$}}

\noindent
Pour tout $n$ de $\N$, on considère la matrice-ligne de ${\cal
  M}_{1,4}(\R)$ :
\[
U_{n} =
\begin{smatrix}
  P\left(\Ev{X_{n} = 1}\right) & P\left(\Ev{X_{n} = 2}\right) &
  P\left(\Ev{X_{n} = 3}\right) & P\left(\Ev{X_{n} =
      4}\right)\end{smatrix}
\]
\begin{noliste}{1.}
  \setlength{\itemsep}{4mm}%
 \setcounter{enumi}{6}
\item 
  \begin{noliste}{a)}
    \setlength{\itemsep}{2mm}
  \item Montrer (grâce à certains résultats de la partie 1) que, si
    l'on pose $A = \dfrac{1}{3}\begin{smatrix}
      0 & 1 & 1 & 1\\
      1 & 0 & 1 & 1\\
      1 & 1 & 0 & 1\\
      1 & 1 & 0 & 1\\
      1 & 1 & 1 & 0
    \end{smatrix}
    $, on a :  
    \[
    \forall n \in \N, \ U_{n + 1} = U_{n} \ A 
    \]
  \item Établir par récurrence que : $\forall n \in \N, \ U_{n} =
    U_{0} \ A^{n}$.
  \item En déduire la première ligne de $A^{n}$.
  \end{noliste}
  
\item Expliquer comment choisir la position du mobile au départ pour
  trouver les trois autres lignes de la matrice $A^{n}$, puis écrire
  ces trois lignes.
\end{noliste}


\subsection*{Partie 3 : une deuxième méthode de calcul des puissances
  de $A$}

\noindent
On considère les matrices $I$ et $J$ suivantes : $I =
\begin{smatrix}
  1 & 0 & 0 & 0\\
  0 & 1 & 0 & 0\\
  0 & 0 & 1 & 0\\
  0 & 0 & 0 & 1
\end{smatrix}
$ et $J = 
\begin{smatrix}
  1 & 1 & 1 & 1\\
  1 & 1 & 1 & 1\\
  1 & 1 & 1 & 1\\
  1 & 1 & 1 & 1
\end{smatrix}
$.
\begin{noliste}{1.}
  \setlength{\itemsep}{4mm}%
 \setcounter{enumi}{8}
\item Déterminer les réels $a$ et $b$ tels que $A = aI + bJ$.
\item
  \begin{noliste}{a)}
    \setlength{\itemsep}{2mm}
  \item Calculer $J^{2}$ puis établir que, pour tout entier naturel
    $k$ non nul, on a : $J^{k} = 4^{k-1} J$.
  \item À l'aide de la formule du binôme de Newton, en déduire, pour tout
    entier $n$ non nul, l'expression de $A^{n}$ comme combinaison linéaire
    de $I$ et $J$.
  \item Vérifier que l'expression trouvée reste valable pour $n = 0$.
  \end{noliste}
  
\end{noliste}


\subsection*{Partie 4 : informatique}

\begin{noliste}{1.}
  \setlength{\itemsep}{4mm}%
  \setcounter{enumi}{10}
\item
  \begin{noliste}{a)}
    \setlength{\itemsep}{2mm}
  \item Compléter le script \Scilab{} suivant pour qu'il affiche les
    100 premières positions autres que celle d'origine, du mobile dont
    le voyage est étudié dans ce problème, ainsi que le nombre $n$ de
    fois où il est revenu sur le sommet numéroté 1 au cours de ses 100
    premiers déplacements (on pourra utiliser la commande {\tt sum}).
    \begin{scilab}
      & A = [------] /3 \nl %
      & x = grand(100,'markov',A,1) \nl %
      & n = ------ \nl %
      & disp(x) \nl %
      & disp(n) \nl %
    \end{scilab}
    
  \item Après avoir exécuté cinq fois ce script, les réponses
    concernant le nombre de fois où le mobile est revenu sur le sommet
    1 sont : $n = 23$, $n = 28$, $n = 23$, $n = 25$, $n = 26$.\\
    En quoi est-ce normal ?
  \end{noliste}
  
\end{noliste}


\end{document}

