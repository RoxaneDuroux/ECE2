\documentclass[11pt]{article}%
\usepackage{geometry}%
\geometry{a4paper,
 lmargin = 2cm,rmargin = 2cm,tmargin = 2.5cm,bmargin = 2.5cm}

\input{../../macros.tex}

\pagestyle{fancy} %
\lhead{ECE2 \hfill Mathématiques\\
} %
\chead{\hrule} %
\rhead{} %
\lfoot{} %
\cfoot{} %
\rfoot{\thepage} %

\renewcommand{\headrulewidth}{0pt}% : Trace un trait de séparation
 % de largeur 0,4 point. Mettre 0pt
 % pour supprimer le trait.

\renewcommand{\footrulewidth}{0.4pt}% : Trace un trait de séparation
 % de largeur 0,4 point. Mettre 0pt
 % pour supprimer le trait.

\setlength{\headheight}{14pt}

\title{\bf \vspace{-2cm} EDHEC 2002} %
\author{} %
\date{} %
\begin{document}

\maketitle %
\vspace{-1.4cm}\hrule %
\thispagestyle{fancy}

\vspace*{.2cm}


% DEBUT DU DOC À MODIFIER : tout virer jusqu'au début de l'exo

%Définition et changement de valeurs de
compteurs%newcounter{cpt1}{section} compteur cpt1 remis à 0 à chaque
aumentation par stepcounter du compteur section%setcounter{cpt1}{3} on
met le compteur à 3%addtocounter{cpt1}{5} on ajoute 5 au compteur%
stepcounter{cpt1} on ajoute 1% ifthenelse{test}{alors}{sinon} (page
206) pour subordonner à une condition % whiledo{test}{commande} pour
faire une boucle (page 206 aussi) % value{cpt1} pour noter dans le
document la valeur de cpt1 
%Définition définitive d'opérateurs
mathématiques\newcommand{\ch}{\operatorname{ch}} 
\newcommand{\sh}{\operatorname{sh}}
\renewcommand{\tanh}{\operatorname{th}}
\renewcommand{\sinh}{\operatorname{sh}}
\renewcommand{\cosh}{\operatorname{ch}}
\newcommand{\argsh}{\operatorname{argsh}}
\newcommand{\argch}{\operatorname{argch}}
\newcommand{\argth}{\operatorname{argth}}
\newcommand{\Id}{\operatorname{Id}}
\renewcommand{\leq}{\leq}
\renewcommand{\geq}{\geq }

\newcommand{\dlim}{\lim}
\newcommand{\dsum}{\sum}
\newcommand{\dprod}{\prod}



%Définition de nouvelles couleurs : rgb(trois paramètres red green blue
entre 0 et 1); cmyk (quatre cyan magenta yellow black) entre 0 et 1;
gray (entre 0 et 1) et black, white, red, green, blue, cyan, magenta,
yellow% definecolor{0gris}{gray}{0.8} 
% Nouvelle commande pour encadrer le titre car shabox ne veut que d'une
seule ligne; ATTENTION A LA TAILLE; petite différence avec shadowbox ou
doublebox, voire fcolorbox ou colorbox (au lieu de shabox; laisser le
parbox tranquille sauf pour la taille de la boîte
\newcommand{\Tbox}[1]{\begin{center} \shabox{\parbox{0.6
\linewidth}{#1}} \end{center}} %[1] pour 1 paramètre ; #1 pour ce que
fait le 1er paramètre; entre accolades ce que fait la commande
%Mise en page en mode fancy : en-têtes et pieds de pages puis
définition des en-têtes et pieds de pages\pagestyle{fancy}
\lhead{ECE 2 - Mathématiques \\
Quentin Dunstetter - ENC-Bessières 2011$\backslash$2012}
\chead{}
\rhead{Edhec 2002}
\rfoot[ \ \thepage]{\thepage}
\cfoot{}
\lfoot{}
\thispagestyle{fancy} %Mise en page de la 1ère page en mode fancy
%Trait en bas et en haut de la page (entre en-tête et texte et texte et
pied de page)\renewcommand{\footrulewidth}{0.4pt}
\renewcommand{\headrulewidth}{0.4pt}

\begin{center}
{\Huge EDHEC}

\textbf{School of management\vspace{3cm}}

{\large ECOLE DE HAUTES ETUDES COMMERCIALES DU NORD}

{\large Concours d'admission sur classes préparatoires}

\underline{\hspace{3cm}}

{\LARGE MATHEMATIQUES}

{\large Option économique}

\textbf{Année 2002}
\end{center}

\noindent \textsl{La présentation, la lisibilité, l'orthographe, la
qualité
de la rédaction, la clarté et la précision des raisonnements entreront
pour
une part importante dans l'appréciation des copies.}

\noindent \textsl{Les candidats sont invités à encadrer dans la mesure
du
possible les résultats de leurs calculs.}

\noindent \textsl{Ils ne doivent faire usage d'aucun document : seule
l'utilisation d'une règle graduée est autorisée.\vspace{1cm}}

\noindent \textbf{L'utilisation de toute calculatrice et de tout
matériel électronique est interdite.}\vspace{1cm}

\section*{Exercice 1}

Pour tout réel $x,$ on note $[x]$ la partie entière de $x,$
c'est-à-dire
l'unique nombre entier vérifiant $ :[x]\leq x<[x] + 1.$\\
Soit $X$ une variable aléatoire suivant la loi exponentielle de
paramètre $
\lambda $ $(\lambda >0).$\\
On pose $Y = [X],$ $Y$ est donc la partie entière de $X$ et on a $
:\forall
k\in \Z,\quad (Y = k)\Leftrightarrow (k\leq X<k + 1).$

\begin{noliste}{1.}
 \setlength{\itemsep}{4mm}
\item 

\begin{noliste}{a)}
 \setlength{\itemsep}{2mm}
\item Montrer que $Y$ prend des valeurs dans $\N.$

\item Pour tout $k$ de $\N^{\times },$ calculer $P\left(\Ev{Y =
k-1}\right).$

\item En déduire que la variable aléatoire $Y + 1$ suit une loi
géométrique
dont on déterminera le paramètre.

\item Donner l'espérance et la variance de $Y + 1.$ En déduire
l'espérance et
la variance de $Y.$
\end{noliste}

\item On pose $Z = X-Y.$

\begin{noliste}{a)}
 \setlength{\itemsep}{2mm}
\item Déterminer $Z(\Omega ).$

\item En utilisant le système complet d'événements $(Y = k)_{k\in \N},$
montrer que :
\[
\forall x\in \lbrack 0;1[,\quad P\left(\Ev{Z\leq x}\right) =
\dfrac{1-e^{- \lambda x}}{1-e^{- \lambda }}.
\]

\item En déduire une densité $f$ de $Z.$

\item Déterminer l'espérance $\E(Z)$ de $Z.$ Ce résultat était-il
prévisible ?
\vspace{1cm}
\end{noliste}
\end{noliste}

\section*{Exercice 2}

On désigne par $n$ un entier naturel non nul.\\
On lance $n$ fois une pièce de monnaie donnant "\textsl{pile}" avec la
probabilité $p$ (avec $0<p<1)$ et "\textsl{face}" avec la probabilité
$q = 1-p.
$ On appelle $k$-chaine une suite de $k$ lancers consécutifs ayant tous
donné
"\textsl{pile}", cette suite devant être suivie d'un "\textsl{face}" ou
être
la dernière suite du tirage.\\
Pour tout $k$ de $\{1,..,n\},$ on note $Y_{k}$ la variable aléatoire
égale
au nombre total de $k$-chaines de "\textsl{pile}" obtenues au cours de
ces $n
$ lancers.\\
Pour tout $k$ de $\{1,..,n\},$ on pourra noter $P_{k}$ l'évènement "on
obtient "\textsl{pile}" au $k^{\grave{e}me}$ lancer"\\
Par exemple, avec $n = 11,$ si l'on a obtenu les résultats $
P_{1}P_{2}F_{3}F_{4}P_{5}P_{6}P_{7}F_{8}P_{9}F_{10}P_{11}$ alors $Y_{1}
= 2,$ $
Y_{2} = 1$ et $Y_{3} = 1.$\\
Le but de cet exercice est de déterminer, pour tout $k$ de
$\{1,..,n\},$
l'espérance de $Y_{k},$ noté $\E(Y_{k}).$

\begin{noliste}{1.}
 \setlength{\itemsep}{4mm}
\item Déterminer la loi de $Y_{n}$ et donner $\E(Y_{n}).$

\item Montrer que $P\left(\Ev{Y_{n-1} = 1}\right) = 2qp^{n-1}$ et
donner $\E(Y_{n-1}).$

\item Dans cette question, $k$ désigne un élément de $\{1,..,n-2\}.$\\
Pour tout $i$ de $\{1,..,n\},$ on note $X_{i,k}$ la variable aléatoire
qui
vaut $1$ si une $k$-chaine de "\textsl{pile}" commence au
$i^{\grave{e}me}$
lancer et qui vaut $0$ sinon.

\begin{noliste}{a)}
 \setlength{\itemsep}{2mm}
\item Calculer $P\left(\Ev{X_{1,k} = 1}\right).$

\item Soit $i\in \{2,..,n-k\}.$ Montrer que $P\left(\Ev{X_{i,k} =
1}\right) = q^{2}p^{k}.$

\item Montrer que $P\left(\Ev{X_{n-k + 1,k} = 1}\right) = qp^{k}.$

\item Exprimer $Y_{k}$ en fonction des variables $X_{i,k}$ puis
déterminer $\E(Y_{k}).$\vspace{1cm}
\end{noliste}
\end{noliste}

\section*{Exercice 3}

On note $f$ la fonction définie sur $\R_{+}$ par $ :\left\{ 
\begin{array}{ll}
f(x) = \dfrac{-x\ln x}{1 + x^{2}} & \forall x>0, \\
f(0) = 0. & 
\end{array}
\right. $

\begin{noliste}{1.}
 \setlength{\itemsep}{4mm}
\item 

\begin{noliste}{a)}
 \setlength{\itemsep}{2mm}
\item Vérifier que $f$ est continue sur $\R_{+}.$

\item Étudier le signe de $f(x).$
\end{noliste}

\item Montrer que l'on définit bien une fonction $F$ sur $\R_{+}$ en
posant :
\[
\forall x\in \R_{+},\quad F(x) = \dint{0}{x}f(t)dt.
\]

\item Pour tout $x$ de $\R_{+},$ on pose $g(x) = F(x)-x.$

\begin{noliste}{a)}
 \setlength{\itemsep}{2mm}
\item Montrer que $g$ est dérivable sur $\R_{+}$ et que, pour $x>0,$
on peut écrire $g^{\prime }(x)$ sous la forme $g^{\prime }(x) =
\dfrac{-xh(x)}{1 + x^{2}}.$

\item Étudier les variations de $h,$ puis en déduire son signe (on
donne $\ln \dfrac{\sqrt{5}-1}{2}\simeq -0,48).$

\item En déduire le signe de $g(x).$
\end{noliste}

\item On définit la suite $(u_{n})$ par la donnée de son premier terme
$u_{0} = 1$ et la relation de récurrence, valable pour tout $n$ de $\N$
: $u_{n + 1} = F(u_{n}).$

\begin{noliste}{a)}
 \setlength{\itemsep}{2mm}
\item Établir par récurrence que $ :\forall n\in \N,\quad u_{n}\in
\lbrack 0;1].$

\item Montrer, en utilisant le résultat de la troisième question, que
$(u_{n})$ est décroissante.

\item En déduire que la suite $(u_{n})$ converge et donner
$\underset{n\rightarrow + \infty }{\lim }u_{n}.$
\end{noliste}
\end{noliste}

\section*{Problème}

\subsection*{Partie 1 : Étude d'un ensemble de matrices.}

On considère les matrices suivantes de $\mathfrak{M}_{4}(\R) :$
\[
I = 
\begin{smatrix}
1 & 0 & 0 & 0 \\
0 & 1 & 0 & 0 \\
0 & 0 & 1 & 0 \\
0 & 0 & 0 & 1
\end{smatrix}
\quad ;\quad J = 
\begin{smatrix}
0 & 0 & 0 & 1 \\
1 & 0 & 0 & 0 \\
0 & 1 & 0 & 0 \\
0 & 0 & 1 & 0
\end{smatrix}
\quad ;\quad K = 
\begin{smatrix}
0 & 1 & 0 & 0 \\
0 & 0 & 1 & 0 \\
0 & 0 & 0 & 1 \\
1 & 0 & 0 & 0
\end{smatrix}
\quad ;\quad L = 
\begin{smatrix}
0 & 0 & 1 & 0 \\
0 & 0 & 0 & 1 \\
1 & 0 & 0 & 0 \\
0 & 1 & 0 & 0
\end{smatrix}.
\]
On note $E$ l'ensemble des matrices $M$ s'écrivant $M = aI + bJ + cK +
dL,$ où $a,b,c
$ et $d$ décrivent $\R.$

\begin{noliste}{1.}
 \setlength{\itemsep}{4mm}
\item 

\begin{noliste}{a)}
 \setlength{\itemsep}{2mm}
\item Montrer que $E$ est un espace vectoriel.

\item Montrer que la famille $(I,J,K,L)$ est libre.

\item Donner la dimension de $E.$
\end{noliste}

\item 

\begin{noliste}{a)}
 \setlength{\itemsep}{2mm}
\item Montrer, en les calculant explicitement que
$J^{2},K^{2},L^{2},J^{3}$
et $K^{3}$ appartiennent à $E.$

\item En déduire, sans aucun calcul matriciel, que $JK,KJ,KL,LK,JL$ et
$LJ$
appartiennent aussi à $E.$

\item Établir enfin que le produit de deux matrices de $E$ est encore
une
matrice de $E.$
\end{noliste}

\item 

\begin{noliste}{a)}
 \setlength{\itemsep}{2mm}
\item Montrer que $L$ est diagonalisable.

\item Déterminer les valeurs propres de $L$ ainsi que les sous-espaces
propres associés à ces valeurs propres.
\end{noliste}

\item On considère les vecteurs : $u_{1} = 
\begin{smatrix}
1 \\
1 \\
1 \\
1
\end{smatrix}
$ $;$ $u_{2} = 
\begin{smatrix}
1 \\
-1 \\
1 \\
-1
\end{smatrix}
$ $;$ $u_{3} = 
\begin{smatrix}
1 \\
1 \\
-1 \\
-1
\end{smatrix}
$ $;$ $u_{4} = 
\begin{smatrix}
1 \\
-1 \\
-1 \\
1
\end{smatrix}.$

\begin{noliste}{a)}
 \setlength{\itemsep}{2mm}
\item Montrer que $(u_{1},u_{2},u_{3},u_{4})$ est une base de
$\mathfrak{M}_{4,1}(\R).$

\item Vérifier que $u_{1},u_{2},u_{3},u_{4}$ sont des vecteurs propres
de $L$ et de $J + K.${\Large }
\end{noliste}
\end{noliste}

\subsection*{Partie 2 : Étude d'un mouvement aléatoire.}

Dans cette partie, $p$ désigne un réel de $]0;1[.$\\
Les sommets d'un carré sont numérotés $1,2,3$ et $4$ de telle façon que
les
cotés du carré relient le sommet $1$ au sommet $2,$ le sommet $2$ au
sommet $
3,$ le sommet $3$ au sommet $4,$ le sommet $4$ au sommet $1,$ les
diagonales
reliant elles le sommet $1$ au sommet $3$ ainsi que le sommet $2$ au
sommet $
4.$\\
Un pion se déplace sur les sommets de ce carré selon le protocole
suivant :

\begin{noliste}{$\sbullet$}
\item Le pion est sur le sommet $1$ au départ.

\item Lorsque le pion est à un instant donné sur un sommet du carré, il
se déplace à l'instant suivant vers un sommet voisin (relié par un
coté) avec la
probabilité $p$ ou vers un sommet opposé (relié par une diagonale) avec
la
probabilité $1-2p.$
\end{noliste}

\noindent On note $X_{n}$ la variable aléatoire égale au numéro du
sommet
sur lequel se trouve le pion à l'instant $n.$ On a donc $X_{0} = 1$

\begin{noliste}{1.}
 \setlength{\itemsep}{4mm}
\item 

\begin{noliste}{a)}
 \setlength{\itemsep}{2mm}
\item Écrire la matrice $A,$ carré d'ordre $4,$ dont le terme situé à
l'intersection de la $i^{\grave{e}me}$ ligne et de la $j^{\grave{e}me}$
colonne est égal à la probabilité conditionnelle $P\left(\Ev{X_{n + 1}
= i/X_{n} = j}\right).$

\item Vérifier que $A$ s'écrit comme combinaison linéaire de $J + K$ et
$L.$
\end{noliste}

\begin{noliste}{a)}
 \setlength{\itemsep}{2mm}
\item Pour tout $i$ de $\{1,2,3,4\},$ calculer $Au_{i}.$ En déduire
qu'il
existe une matrice $D$ diagonale et une matrice $P$ inversible telles
que $
A = PDP^{-1}.\ $Expliciter $D$ et $P.$

\item Calculer $P^{2}$ puis en déduire $P^{-1}.$
\end{noliste}

\item Pour tout $n$ de $\N,$ on pose $C_{n} = 
\begin{smatrix}
P\left(\Ev{X_{n} = 1}\right) \\
P\left(\Ev{X_{n} = 2}\right) \\
P\left(\Ev{X_{n} = 3}\right) \\
P\left(\Ev{X_{n} = 4}\right)
\end{smatrix}.$

\begin{noliste}{a)}
 \setlength{\itemsep}{2mm}
\item Montrer, à l'aide de la formule des probabilités totales que
$C_{n + 1} = AC_{n}.$

\item En déduire que $C_{n} = \dfrac{1}{4}PD^{n}PC_{0},$ puis donner la
loi de 
$X_{n}$ pour tout entier naturel $n$ supérieur ou égal à $1.$
\end{noliste}
\end{noliste}

\label{fin}

\end{document}

