\documentclass[11pt]{article}%
\usepackage{geometry}%
\geometry{a4paper,
 lmargin = 2cm,rmargin = 2cm,tmargin = 2.5cm,bmargin = 2.5cm}

\input{../../macros.tex}

\pagestyle{fancy} %
\lhead{ECE2 \hfill Mathématiques\\
} %
\chead{\hrule} %
\rhead{} %
\lfoot{} %
\cfoot{} %
\rfoot{\thepage} %

\renewcommand{\headrulewidth}{0pt}% : Trace un trait de séparation
 % de largeur 0,4 point. Mettre 0pt
 % pour supprimer le trait.

\renewcommand{\footrulewidth}{0.4pt}% : Trace un trait de séparation
 % de largeur 0,4 point. Mettre 0pt
 % pour supprimer le trait.

\setlength{\headheight}{14pt}

\title{\bf \vspace{-2cm} EDHEC 1999} %
\author{} %
\date{} %
\begin{document}

\maketitle %
\vspace{-1.4cm}\hrule %
\thispagestyle{fancy}

\vspace*{.2cm}


% DEBUT DU DOC À MODIFIER : tout virer jusqu'au début de l'exo

%Définition et changement de valeurs de
compteurs%newcounter{cpt1}{section} compteur cpt1 remis à 0 à chaque
aumentation par stepcounter du compteur section%setcounter{cpt1}{3} on
met le compteur à 3%addtocounter{cpt1}{5} on ajoute 5 au compteur%
stepcounter{cpt1} on ajoute 1% ifthenelse{test}{alors}{sinon} (page
206) pour subordonner à une condition % whiledo{test}{commande} pour
faire une boucle (page 206 aussi) % value{cpt1} pour noter dans le
document la valeur de cpt1 
%Définition définitive d'opérateurs
mathématiques\newcommand{\ch}{\operatorname{ch}} 
\newcommand{\sh}{\operatorname{sh}}
\renewcommand{\tanh}{\operatorname{th}}
\renewcommand{\sinh}{\operatorname{sh}}
\renewcommand{\cosh}{\operatorname{ch}}
\newcommand{\argsh}{\operatorname{argsh}}
\newcommand{\argch}{\operatorname{argch}}
\newcommand{\argth}{\operatorname{argth}}
\newcommand{\ker}{\operatorname{Ker}}
\renewcommand{\im}{\operatorname{Im}}
\newcommand{\rg}{\operatorname{rg}}
\newcommand{\Id}{\operatorname{Id}}
\newcommand{\id}{\operatorname{id}}
\renewcommand{\leq}{\leq}
\renewcommand{\geq}{\geq }

%Définition de nouvelles couleurs : rgb(trois paramètres red green blue
entre 0 et 1); cmyk (quatre cyan magenta yellow black) entre 0 et 1;
gray (entre 0 et 1) et black, white, red, green, blue, cyan, magenta,
yellow% definecolor{0gris}{gray}{0.8} 
% Nouvelle commande pour encadrer le titre car shabox ne veut que d'une
seule ligne; ATTENTION A LA TAILLE; petite différence avec shadowbox ou
doublebox, voire fcolorbox ou colorbox (au lieu de shabox; laisser le
parbox tranquille sauf pour la taille de la boîte
\newcommand{\Tbox}[1]{\begin{center} \shabox{\parbox{0.6
\linewidth}{#1}} \end{center}} %[1] pour 1 paramètre ; #1 pour ce que
fait le 1er paramètre; entre accolades ce que fait la commande
%Mise en page en mode fancy : en-têtes et pieds de pages puis
définition des en-têtes et pieds de pages\pagestyle{fancy}
\lhead{ECE 2 - Mathématiques \\
Quentin Dunstetter - ENC-Bessières 2011$\backslash$2012}
\chead{}
\rhead{Edhec 1999}
\rfoot[ \ \thepage]{\thepage}
\cfoot{}
\lfoot{}
\thispagestyle{fancy} %Mise en page de la 1ère page en mode fancy
%Trait en bas et en haut de la page (entre en-tête et texte et texte et
pied de page)\renewcommand{\footrulewidth}{0.4pt}
\renewcommand{\headrulewidth}{0.4pt}


%DEBUT DU DOCUMENT\vspace*{3cm}

\begin{center}
{\LARG\E\textbf{BANQUE COMMUNE D'ÉPREUVES}}



{\large \textsc{CONCOURS D ADMISSION DE 1999}}



{\large \textbf{Concepteur : Edhec}}



\rule{2.39cm}{0.05cm}



{\Large \textbf{OPTION ÉCONOMIQUE}}



{\Large \textbf{MATHÉMATIQUES }}



{\Large Lundi 9 mai, de 14h à 18h}



\rule{2.39cm}{0.05cm}
\end{center}

\textit{La présentation, la lisibilité, l'orthographe, la qualité
de la rédaction, la clarté et la précision des raisonnements
entreront pour une part importante dans l'appréciation des copies.}

\textit{Les candidats sont invités à \textbf{encadrer} dans la mesure
du possible les résultats de leurs calculs.}

\textit{Ils ne doivent faire usage d'aucun document. L'utilisation de
toute
calculatrice et de tout matériel électronique est interdite. Seule
l'utilisation d'une règle graduée est autorisée.}

\textit{Si au cours de l'épreuve, un candidat repère ce qui lui semble
être une erreur d'énoncé, il la signalera sur sa copie et
poursuivra sa composition en expliquant les raisons des initiatives
qu'il sera
amené à prendre.}

\vspace*{3cm}

\section*{EXERCICE 1 :}

Soit $a$ un réel positif ou nul. On considère la matrice $A\left(
a\right) = \left( 
\begin{array}{cccc}
1 & a-2 & a & 1 \\
a & -1 & 1 & a \\
0 & 0 & -a & 1 \\
0 & 0 & -1 & 0
\end{array}
\right) $

\begin{noliste}{1.}
 \setlength{\itemsep}{4mm}
\item Montrer que $A\left( 0\right) $ admet 1 et -1 comme seules
valeurs
propres.

Donner les sous-espaces propres correspondants.

\hspace{-1cm}Dans la suite, on suppose $a>0.$

\item Montrer que les valeurs propres de $A\left( a\right) $ sont les
réels $\lambda $ solutions de l'une des équations : 
\[
\lambda ^{2} = \left( a-1\right) ^{2}\quad \text{et\quad }\lambda
^{2} + a\lambda + 1 = 0.
\]

\item 
\begin{noliste}{a)}
 \setlength{\itemsep}{2mm}
\item Déduire de la question précédente la valeur de $a$ pour
laquelle $A\left( a\right) $ n'est pas inversible.

\item Pour cette valeur, dire si $A\left( a\right) $ est
diagonalisable.
\end{noliste}

\item On suppose dans cette question que $a>2.$

\begin{noliste}{a)}
 \setlength{\itemsep}{2mm}
\item Montrer que $A\left( a\right) $ possède 4 valeurs propres
distinctes deux à deux.

\item En déduire que $A\left( a\right) $ est diagonalisable.
\end{noliste}
\end{noliste}

\section*{Exercice 2}

Pour tout réel $a$, on considère la fonction $f_{a}$ de
$\mathbb{R\times R}$ dans $\R$, définie par :

$\forall \left( x,y\right) \in \mathbb{R\times R},f_{a}\left(
x,y\right)
 = \left( 1 + y + xy + ax^{2}\right) e^{y}.$

\subsection*{Partie 1 : étude des extrema de $f_{a}$}

Dans cette partie, on suppose $a\neq 0$ et $a\neq -\dfrac{1}{2}$

\begin{noliste}{1.}
 \setlength{\itemsep}{4mm}
\item 
\begin{noliste}{a)}
 \setlength{\itemsep}{2mm}
\item Calculer les dérivées partielles premières de $f_{a}.$

\item En déduire que $f_{a}$ possède deux points critiques
(c'est-à-dire des couples de $\mathbb{R\times R}$ en lesquels $f_{a}$
est
susceptible de présenter un extremum local) et donner leurs coordonnées
\end{noliste}

\item Calculer les dérivées partielles secondes de $f_{a}.
$

\item 
\begin{noliste}{a)}
 \setlength{\itemsep}{2mm}
\item Examiner, pour chacun des deux points critiques, à quelle
condition portant sur $a$, $f_{a}$ présente en ces points un extremum
local.

\item Déterminer, en distinguant trois cas, si $f_{a}$ présente sur
$\mathbb{R\times R}$ un maximum local ou un minimum local et donner sa
valeur
en fonction de $a.$
\end{noliste}
\end{noliste}

\subsection*{Partie 2 : étude d'un fonction définie à l'aide de
$f_{a}$}

\begin{noliste}{1.}
 \setlength{\itemsep}{4mm}
\item 
\begin{noliste}{a)}
 \setlength{\itemsep}{2mm}
\item Pour tout réel $x$ et pour tout réel $t$ inférieur à $x $,
calculer $\dint \nolimits_{t}{x}e^{y}dy.$\\
En déduire que l'intégrale $I = \dint \nolimits_{-\infty }{x}e^{y}dy$
converge et donner sa valeur.

\item Pour tout réel $x$, montrer gr\^{a}ce à une intégration
par parties, que l'intégrale $J = \dint \nolimits_{-\infty
}{x}ye^{y}dy$
converge et donner sa valeur
\end{noliste}

\item 
\begin{noliste}{a)}
 \setlength{\itemsep}{2mm}
\item Déduire des deux questions précédentes que l'on définit bien une
fonction $F_{a}$, de $\R$ dans $\R$, en
posant : $F_{a}\left( x\right) = \dint \nolimits_{-\infty
}{x}f_{a}\left(
x,y\right) dy.$

\item Après avoir écrit $F_{a}\left( x\right) $ en fonction de $a$
et de $x$, donner le tableau de variations de $F_{a}.$\\
(On distinguera les trois cas : $a = -1,a<-1$ et $a>-1$)
\end{noliste}
\end{noliste}

\section*{EXERCICE 3}

Soient $X,Y$ et $Z$ trois variables aléatoires mutuellement
indépendantes et définies sur le même espace probabilisé $\left(
\Omega,\mathcal{A},P\right).$ On suppose que $X,Y$ et $Z$ suivent la
loi $\mathcal{U}_{\left[ \ \left| 1,n\right| \right] }$ 

( c'est-à-dire que : $\forall k\in \left[ \ \left| 1,n\right|
\right],P\left(\Ev{ X = k}\right) = P\left(\Ev{ Y = k}\right) =
P\left(\Ev{ Z = k}\right) = \dfrac{1}{n}$
).

\begin{noliste}{1.}
 \setlength{\itemsep}{4mm}
\item 
\begin{noliste}{a)}
 \setlength{\itemsep}{2mm}
\item Montrer que : $\forall k\in \left[ \ \left| 2,n + 1\right|
\right],P\left(\Ev{ X + Y = k}\right) = \dfrac{k-1}{n^{2}}.$

\item Montrer que : $\forall k\in \left[ \ \left| n + 2,2n\right|
\right],P\left(\Ev{ X + Y = k}\right) = \dfrac{2n-k + 1}{n^{2}}.$
\end{noliste}

\item Utiliser la formule des probabilités totales pour déduire de
la première question que :
\[
P\left(\Ev{ X + Y = Z}\right) = \dfrac{n-1}{2n^{2}}.
\]

\item 
\begin{noliste}{a)}
 \setlength{\itemsep}{2mm}
\item Montrer que la variable aléatoire $T = n + 1-Z$ suit la loi
$\mathcal{U}_{\left[ \ \left| 1,n\right| \right] }$.

\item Pourquoi $T$ est-elle indépendante de $X$ et de $Y$ ?

\item En faisant intervenir la variable $T$ et en utilisant la deuxième
question, déterminer la probabilité $P\left(\Ev{ X + Y + Z = n +
1}\right).$
\end{noliste}
\end{noliste}

\section*{PROBLEME}

Les parties 1 et 2 sont indépendantes.

\subsection*{Partie 1}

On pose, pour tout $n$ élément de $\N^{*},u_{n} = \dsum\limits_{p =
1}{n}\dfrac{1}{p}.$

\begin{noliste}{1.}
 \setlength{\itemsep}{4mm}
\item 
\begin{noliste}{a)}
 \setlength{\itemsep}{2mm}
\item Montrer que : $\forall p\in \N^{\ast },\dint_{p}{p +
1}\dfrac{dt}{t}\geq \dfrac{1}{p + 1}.$

\item En déduire que : $\forall n\in \N^{\ast },u_{n}\leq
1 + \ln \left( n\right).$
\end{noliste}

\item On considère la fonction $\varphi_{1}$ définie sur $\R_{+}$ par :
$\left\{ 
\begin{array}{l}
\varphi_{1}\left( 0\right) = 0 \\
\varphi_{1}\left( x\right) = x\left( 1 + \ln \left( x\right) \right)
\text{ si 
}x>0
\end{array}
\right. $\\
Montrer que $\varphi_{1}$ est continue sur $\R_{+}.$

\item Pour tout réel $x$ positif et pour tout entier naturel $n$ non
nul, on pose :\\
$\varphi_{n + 1}\left( x\right) = \dint_{0}{x}\varphi_{n}\left(
t\right\ dt$
( On rappelle que $\varphi_{1}$ a été définie à la question
2).

\begin{noliste}{a)}
 \setlength{\itemsep}{2mm}
\item Montrer que, pour tout $n$ élément de $\N^{\ast }$, la
fonction $\varphi_{n}$ est parfaitement définie et continue sur
$\R_{+}$. Que vaut $\varphi_{n}\left( 0\right) $ ?

\item Vérifier qu'il existe deux suites $\left( a_{n}\right)_{n\in 
\N^{\ast }}$ et $\left( b_{n}\right)_{n\in \N^{\ast }}$
telles que :\\
$\forall n\in \N^{\ast },\forall x\in \R_{+}{\ast },\varphi
_{n}\left( x\right) = x^{n}\left( a_{n} + b_{n}\ln x\right).$\\
On montrera que : $\forall n\in \N^{\ast },a_{n + 1} = \dfrac{a_{n}}{n
+ 1}-\dfrac{b_{n}}{\left( n + 1\right) ^{2}}$ et $b_{n + 1} =
\dfrac{b_{n}}{n + 1}$
\end{noliste}

\item Écrire un programme en \Scilab{} qui calcule et affiche les $n$
premiers termes de chacune des suites $\left( a_{n}\right) $ et $\left(
b_{n}\right) $ pour une valeur de $n$ entrée par l'utilisateur.

\item Calculer $b_{n}.$

\item Pour tout $n$ élément de $\N^{\ast }$, on pose : $c_{n} = n!$
$a_{n}.$

\begin{noliste}{a)}
 \setlength{\itemsep}{2mm}
\item Montrer que $c_{n} = 2-u_{n}.$

\item En déduire que, pour tout entier $n$ supérieur ou égal 
à 2 : $\left| c_{n}\right| \leq 1 + \ln \left( n\right).$

\item Conclure que $\lim \limits_{n\rightarrow + \infty }a_{n} = 0.$

\item Montrer enfin que la série de terme général $a_{n}$ est
absolument convergente.
\end{noliste}
\end{noliste}

\subsection*{Partie 2}

On considère les fonctions $e_{1},e_{2},e_{3}$ et $e_{4}$ définies
par :\\
$\forall x\in \R_{+}{\ast },e_{1}\left( x\right) = x,e_{2}\left(
x\right) = x^{2},e_{3}\left( x\right) = x\ln \left( x\right) $ et
$e_{4}\left(
x\right) = x^{2}\ln \left( x\right).$\\
On note $E$ l'espace vectoriel engendré par $e_{1},e_{2},e_{3}$ et
$e_{4}.$

\begin{noliste}{1.}
 \setlength{\itemsep}{4mm}
\item On suppose dans cette question que $a,b,c$ et $d$ sont 4 réels
tels que :\\
(*) $\forall x\in \R_{+}{\ast },ax + bx^{2} + cx\ln \left( x\right)
 + dx^{2}\ln \left( x\right) = 0.$

\begin{noliste}{a)}
 \setlength{\itemsep}{2mm}
\item Montrer que $a + b = 0.$

\item Établir que : $\forall x>1,\dfrac{a}{x\ln \left( x\right) } +
\dfrac{b}{\ln \left( x\right) } + \dfrac{c}{x} + d = 0.$ En déduire que
$d = 0.$

\item Établir ensuite que : $\forall x\in \R_{+}{\ast },\dfrac{a}{x} +
b + c\dfrac{\ln \left( x\right) }{x} = 0.$ En déduire que $b = 0.$

\item Montrer finalement que $a = b = c = d = 0.$
\end{noliste}

\item 
\begin{noliste}{a)}
 \setlength{\itemsep}{2mm}
\item Déduire de la question précédente que $\left(
e_{1},e_{2},e_{3},e_{4}\right) $ est une famille libre.

\item Montrer que $\left( e_{1},e_{2},e_{3},e_{4}\right) $ est une base
de $E.$
\end{noliste}

\item On note $u$ l'application qui à toute fonction $f$ de $E$ associe
la fonction $g = u\left( f\right) $ définie par :\\
$\forall x\in \R_{+}{\ast },g\left( x\right) = xf^{\prime }\left(
x\right).$

\begin{noliste}{a)}
 \setlength{\itemsep}{2mm}
\item Montrer que $u$ est une application linéaire.

\item Déterminer $u\left( e_{1}\right),u\left( e_{2}\right),u\left(
e_{3}\right) $ et $u\left( e_{4}\right).$

\item En déduire que $u$ est un endomorphisme de $E.$
\end{noliste}

\item 
\begin{noliste}{a)}
 \setlength{\itemsep}{2mm}
\item Donner la matrice $A$ de $u$ dans la base $\left(
e_{1},e_{2},e_{3},e_{4}\right).$

\item Montrer que $u$ est un automorphisme de $E.$
\end{noliste}
\end{noliste}

\end{document}

