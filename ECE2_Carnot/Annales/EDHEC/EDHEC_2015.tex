\documentclass[11pt]{article}%
\usepackage{geometry}%
\geometry{a4paper,
 lmargin = 2cm,rmargin = 2cm,tmargin = 2.5cm,bmargin = 2.5cm}

\input{../../macros.tex}

\pagestyle{fancy} %
\lhead{ECE2 \hfill Mathématiques\\
} %
\chead{\hrule} %
\rhead{} %
\lfoot{} %
\cfoot{} %
\rfoot{\thepage} %

\renewcommand{\headrulewidth}{0pt}% : Trace un trait de séparation
 % de largeur 0,4 point. Mettre 0pt
 % pour supprimer le trait.

\renewcommand{\footrulewidth}{0.4pt}% : Trace un trait de séparation
 % de largeur 0,4 point. Mettre 0pt
 % pour supprimer le trait.

\setlength{\headheight}{14pt}

\title{\bf \vspace{-2cm} EDHEC 2015} %
\author{} %
\date{} %
\begin{document}

\maketitle %
\vspace{-1.4cm}\hrule %
\thispagestyle{fancy}

\vspace*{.2cm}


% DEBUT DU DOC À MODIFIER : tout virer jusqu'au début de l'exo

%Définition et changement de valeurs de
compteurs%newcounter{cpt1}{section} compteur cpt1 remis à 0 à chaque
aumentation par stepcounter du compteur section%setcounter{cpt1}{3} on
met le compteur à 3%addtocounter{cpt1}{5} on ajoute 5 au compteur%
stepcounter{cpt1} on ajoute 1% ifthenelse{test}{alors}{sinon} (page
206) pour subordonner à une condition % whiledo{test}{commande} pour
faire une boucle (page 206 aussi) % value{cpt1} pour noter dans le
document la valeur de cpt1 
%Définition définitive d'opérateurs
mathématiques\newcommand{\ch}{\operatorname{ch}} 
\newcommand{\sh}{\operatorname{sh}}
\renewcommand{\tanh}{\operatorname{th}}
\renewcommand{\sinh}{\operatorname{sh}}
\renewcommand{\cosh}{\operatorname{ch}}
\newcommand{\argsh}{\operatorname{argsh}}
\newcommand{\argch}{\operatorname{argch}}
\newcommand{\argth}{\operatorname{argth}}
\newcommand{\Id}{\operatorname{Id}}
\newcommand{\id}{\operatorname{id}}
\renewcommand{\im}{\operatorname{Im}}
\renewcommand{\leq}{\leq}
\renewcommand{\geq}{\geq }

\newcommand{\dlim}{\lim}
\newcommand{\dprod}{\prod}
\newcommand{\dsum}{\sum}
\newcommand{\dint}{\int}
\newcommand{\lb}{\llbracket}
\newcommand{\rb}{\rrbracket}







%Définition de nouvelles couleurs : rgb(trois paramètres red green blue
entre 0 et 1); cmyk (quatre cyan magenta yellow black) entre 0 et 1;
gray (entre 0 et 1) et black, white, red, green, blue, cyan, magenta,
yellow% definecolor{0gris}{gray}{0.8} 
% Nouvelle commande pour encadrer le titre car shabox ne veut que d'une
seule ligne; ATTENTION A LA TAILLE; petite différence avec shadowbox ou
doublebox, voire fcolorbox ou colorbox (au lieu de shabox; laisser le
parbox tranquille sauf pour la taille de la boîte
\newcommand{\Tbox}[1]{\begin{center} \shabox{\parbox{0.6
\linewidth}{#1}} \end{center}} %[1] pour 1 paramètre ; #1 pour ce que
fait le 1er paramètre; entre accolades ce que fait la commande
%Mise en page en mode fancy : en-têtes et pieds de pages puis
définition des en-têtes et pieds de pages\pagestyle{fancy}
\lhead{ECE 2\\
Mathématiques
}
\chead{}
\rhead{Edhec 2015}
\rfoot[ \ \thepage]{\thepage}
\cfoot{}
\lfoot{}
\thispagestyle{fancy} %Mise en page de la 1ère page en mode fancy
%Trait en bas et en haut de la page (entre en-tête et texte et texte et
pied de page)\renewcommand{\footrulewidth}{0.4pt}
\renewcommand{\headrulewidth}{0.4pt}

\indent \vspace{0.3cm}

\Tbox{\begin{center} \textbf{\Huge Edhec 2015} \end{center} }

\vspace{0.5cm}

\section*{Exercice 1}

\noindent On note $\mathcal{B} = (e_{1},e_{2},e_{3},e_{4},e_{5})$ la
base canonique de $\R^{5}$. On désigne par $I$ la matrice identité de
$\mathcal{M}_{5} (\R)$ et on considère l'endomorphisme $f$ de $\R^{5}$
dont la matrice dans la base $\mathcal{B}$ est :

\begin{center}
$\begin{smatrix}
1 & 1 & 1 & 1 & 1 \\
1 & 0 & 0 & 0 & 0 \\
1 & 0 & 0 & 0 & 0 \\
1 & 0 & 0 & 0 & 0 \\
1 & 1 & 1 & 1 & 1
\end{smatrix}
$
\end{center}

\begin{noliste}{1.}
 \setlength{\itemsep}{4mm}
\item 
\begin{noliste}{a)}
 \setlength{\itemsep}{2mm}
\item Déterminer la dimension de $\im(f)$, puis montrer que la famille
$(e_{2} + e_{3} + e_{4}, e_{1} + e_{5} )$ est une base de $\im(f)$.
\item En déduire la dimension de $\ker(f)$ puis donner une base de
$\ker(f)$.
\end{noliste}

\item
On note $u = e_{2} + e_{3} + e_{4}$ et $v = e_{1} + e_{5}$.

\begin{noliste}{a)}
 \setlength{\itemsep}{2mm}
\item Écrire $f(u)$ et $f(v)$ comme combinaisons linéaires de
$e_{1},e_{2},e_{3},e_{4},e_{5}$, puis $f(u-v)$ et
$f (u + 3v)$ comme combinaisons linéaires de $u$ et $v$.

\item En déduire les valeurs propres de $f$ et préciser les
sous-espaces propres associés.

\item Établir que $C$ est diagonalisable et déterminer une matrice $D$
diagonale et une matrice $R$ inversible telles que $C = R D R^{-1}$.

\end{noliste}

\item
\begin{noliste}{a)}
 \setlength{\itemsep}{2mm}
\item Établir la relation suivante : $D(D + I)(D-3I) = 0$.

\item En déduire que le polynôme $P$ défini par $P\left(\Ev{X}\right) =
X^{3} -2X^{2} -3X$ est un polynôme annulateur
de $C$.

\end{noliste}

\item On admet que (principe de la division euclidienne), pour tout
entier naturel $n$ non nul, il existe un unique polynôme Qn et trois
réels $a_{n}$, $b_{n}$ et $c_{n}$ tels que :
\[
 X^{n} = (X^{3} -2X^{2} -3X)Q_{n}(X) + a_{n} X^{2} + b_{n} X + c_{n}
\]

\begin{noliste}{a)}
 \setlength{\itemsep}{2mm}
\item En utilisant les racines de $P$, déterminer les valeurs de
$a_{n}$, $b_{n}$ et $c_{n}$ en fonction de $n$.

\item Déduire de ce qui précède l'expression, pour tout entier naturel
$n$ non nul, de $C^{n}$	en fonction de $C$ et $C^{2}$.
\end{noliste}

 \item Compléter, à l'aide de matrices de type {\tt zeros} et {\tt
ones}, les deux espaces laissés libres dans la commande \texttt{Scilab}
suivante pour qu'elle permette de construire la matrice $C$.
 
\begin{center}
{\tt C = [ones(1,5);------,------;ones(1,5)]}
\end{center}


\end{noliste}








\section*{Exercice 2}

\noindent Trois personnes, notées $A$, $B$ et $C$ entrent simultanément
dans une agence bancaire disposant de deux guichets. Les clients $A$ et
$B$ occupent simultanément à l'instant 0 les deux guichets tandis que
$C$ attend que l'un de ces deux guichets se libère pour se faire
servir. \\
\noindent On suppose que :

\begin{noliste}{$\sbullet$}
\item Les durées de passage au guichet des trois personnes $A$, $B$ et
$C$ sont mesurées en heures et on suppose que ce sont des variables
aléatoires indépendantes, notées respectivement $X$, $Y$ et $Z$, et
suivant toutes la loi uniforme sur $[0, 1[$.

\item La durée du changement de personne à un guichet est négligeable.
\end{noliste}

\begin{noliste}{1.}
 \setlength{\itemsep}{4mm}
\item On pose $U = \min(X, Y)$ et $V = \max(X, Y)$ et on admet que $U$
et $V$ sont des variables aléatoires.
\begin{noliste}{a)}
 \setlength{\itemsep}{2mm}
\item Montrer que la fonction de répartition $F_{U} $ de $U$ est
définie par : $F_{U} (x) = 
\left\lbrace 
\begin{array}{ccc}
0 & \text{si} & x<0 \\
2x-x^{2} & \text{si} & 0\leq x\leq 1 \\
1 & \text{si} & x>1
\end{array}
\right.$.

\item En déduire que $U$ est une variable aléatoire à densité et donner
une densité $f_{U}$ de $U$.


\item Déterminer l'espérance et la variance de $U$.

\end{noliste}

\item On note $T$ le temps total passé par $C$ dans l'agence bancaire.

\begin{noliste}{a)}
 \setlength{\itemsep}{2mm}
\item Exprimer $T$ en fonction de certaines des variables précédentes.

\item En déduire $\E(T)$ et $\V(T)$.
\end{noliste}

\item 
\begin{noliste}{a)}
 \setlength{\itemsep}{2mm}
\item On rappelle que, si \texttt{a} et \texttt{b} sont deux vecteurs
lignes de taille $n$, les commandes \texttt{m = min(a,b)} et \texttt{M
= max(a,b)} renvoient les vecteurs \texttt{m} et \texttt{M}, de même
taille que \texttt{a} et \texttt{b}, et tels que, pour tout $i$ de
$\lb 1, n \rb$, on ait : \texttt{m(i) = min(a(i), b(i))} et
\texttt{M(i) = max(a(i), b(i))}.

\noindent On rappelle également que \texttt{grand(1,n,'unf',0,1)}
simule $n$ variables aléatoires indépendantes suivant la loi uniforme
sur $[0, 1[$. \\
Compléter les commandes \texttt{Scilab} suivantes pour qu'elles
permettent de simuler $n$ fois les variables aléatoires $U$, $V$ et
$T$, pour $n$ entré par l'utilisateur :\\
\texttt{n = input('entrez la valeur de n :')\\
x = grand(1,n,'unf',0,1) \\
y = grand(1,n,'unf,0,1) \\
z = grand(1,n,'unf,0,1)
u = ------ ; disp(u, 'u = ') \\
v = ------ ; disp(v, 'v = ') \\
t = ------ ; disp(t, 't = ')}

\item Que représente l'évènement $(T \geq V )$ ?

\item On souhaite déterminer une valeur approchée de la probabilité $p
= P \left(\Ev{T \geq V }\right)$ en simulant un
grand nombre de fois le passage des clients $A$, $B$ et $C$ aux
guichets. Compléter les commandes \texttt{p = ------;disp(p, 'p = ')}
pour que, placées sous les commandes écrites à la question 3a), elles
permettent d'obtenir une valeur approchée de $p$.

\item Lors de plusieurs essais des commandes ci-dessus, avec $n =
10000$, la réponse donnée par \texttt{Scilab} est comprise entre 0.66
et 0.67. Que peut-on conjecturer quant à la valeur exacte de $p$ ?
\end{noliste}





\end{noliste}










\section*{Exercice 3}

\begin{noliste}{1.}
 \setlength{\itemsep}{4mm}
\item Pour tout entier naturel $k$, on pose :
\[
 I_{k} = \dint_{0}{+ \infty} t^{k} e^{-t \ dt 
\]

\begin{noliste}{a)}
 \setlength{\itemsep}{2mm}
\item Justifier que $I_{0}$, $I_{1}$ et $I_{2}$ sont des intégrales
convergentes et donner leur valeur (on pourra s'appuyer sur le cours de
probabilité).

\item Pour tout réel $a$ positif et tout entier naturel $k$, on pose :
$ I_{k} (a) = \dint_{0}{a} t^{k} e^{-t \ dt $.\\
Établir, grâce à une intégration par parties,que : $I_{k + 1}(a) = (k +
1) I_{k} (a)-a^{k + 1 }e^{-a}$.

\item En déduire que $I_{3}$ et $I_{4}$ sont des intégrales
convergentes et vérifier que : $I_{3} = 6$ et $I_{4} = 24$.

\end{noliste}

\item Déduire des questions précédentes que, pour tout couple $(x,y)$
de réels, $ \dint_{0}{+ \infty} (y + xt + t^{2})^{2} e^{-t \ dt $ est
une intégrale convergente.\\
\noindent On considère, pour toute la suite, la fonction $f$ de
$\R^{2}$ dans $\R$ définie par :
\[
 \forall (x,y) \in \R^{2}, \; f(x,y) = \dint_{0}{+ \infty} (y + xt +
t^{2})^{2} e^{-t \ dt
\]

\item 
\begin{noliste}{a)}
 \setlength{\itemsep}{2mm}
\item Vérifier que l'on a : $\forall (x,y) \in \R^{2}, \; f(x,y) =
2x^{2} + y^{2} + 12x + 4y + 2xy + 24$.

\item Justifier que $f$ est de classe $C^{2}$ sur $\R^{2}$.


\end{noliste}

\item 
\begin{noliste}{a)}
 \setlength{\itemsep}{2mm}
\item Calculer les dérivées partielles d'ordre 1 de $f$ puis déterminer
le seul point critique $(a,b)$ de $f$.

\item Calculer les dérivées partielles d'ordre 2 de $f$ et écrire la
matrice hessienne $\nabla^{2} ( f )(a,b)$ de $f$ en son point critique.

\item Déterminer les valeurs propres de $\nabla^{2} ( f )(a,b)$ et en
déduire que $f$ admet un extremum local $m$ au point $(a,b)$ dont on
précisera la nature (minimum ou maximum) et la valeur.

\end{noliste}

\item Le but de cette question est de montrer qu'en fait cet extremum
est global.
\begin{noliste}{a)}
 \setlength{\itemsep}{2mm}
\item Compléter le membre de droite de l'égalité suivante : 
\[
 2x^{2} + 2xy + 12x = 2\left( x + \dfrac{y}{2} + 3\right)^{2} - \cdots 
\]

\item Compléter de même l'égalité : $\dfrac{y^{2}}{2} -2y + 6 =
\dfrac{1}{2}(y-2)^{2} + \cdots $

\item En déduire une autre écriture de $f (x, y)$ montrant que
l'extremum trouvé plus haut est global.

\end{noliste}

\end{noliste}










\section*{Problème}

\subsection*{\textit{Partie 1}}

Dans cette partie, $x$ désigne un réel de $[0, 1[$.
\begin{noliste}{1.}
 \setlength{\itemsep}{4mm}
\item 
\begin{noliste}{a)}
 \setlength{\itemsep}{2mm}
\item Montrer que : $\forall m\in \N$, $0\leq \dint_{0}{x}
\dfrac{t^{m}}{1-t^{2} \ dt\leq \dfrac{1}{1-x^{2}}\times \dfrac{1}{m +
1}$.

\item En déduire que : $\dlim{x \rightarrow + \infty} \dint_{0}{x}
\dfrac{t^{m}}{1-t^{2} \ dt = 0$.


\end{noliste}

\item 
\begin{noliste}{a)}
 \setlength{\itemsep}{2mm}
\item Pour tout réel $t$ de $[0,1[$ et pour tout $k$ élément de
$\N^\star$, calculer $\dsum\limits_{j = 0}{k-1} t^{2j}$.

\item En déduire que : $\dsum\limits_{j = 0}{k-1} \dfrac{x^{2j + 1}}{2j
+ 1} = \dint_{0}{x} \dfrac{1}{1-t^{2} \ dt - \dint_{0}{x}
\dfrac{t^{2k}}{1-t^{2} \ dt$.

\item Utiliser la question 1) pour montrer que la série de terme
général $ \dfrac{x^{2j + 1}}{2j + 1} $ converge et exprimer
$\dsum\limits_{j = 0}{+ \infty} \dfrac{x^{2j + 1}}{2j + 1}$ sous forme
d'une intégrale que l'on ne cherchera pas à calculer.

\item Conclure que : $\forall k\in \N$, $\dsum\limits_{j = k}{+ \infty}
\dfrac{x^{2j + 1}}{2j + 1} = \dint_{0}{x} \dfrac{t^{2k}}{1-t^{2} \
dt$.\\

On admet sans démonstration que l'on a aussi : $\forall k\in \N$,
$\dsum\limits_{j = k + 1}{+ \infty} \dfrac{x^{2j}}{2j} = \dint_{0}{x}
\dfrac{t^{2k + 1}}{1-t^{2} \ dt$.
\end{noliste}


\end{noliste}

\subsection*{\textit{Partie 2}}

\noindent Un joueur réalise une suite de lancers indépendants d'une
pièce. Cette pièce donne "pile" avec la probabilité $p$ $( 0 < p < 1 )$
et "face" avec la probabilité $q = 1 - p$.\\
\noindent On note $N$ la variable aléatoire égale au rang d'apparition
du premier "pile".\\
\noindent Si $N$ prend la valeur $n$, le joueur place $n$ boules
numérotées de 1 à $n$ dans une urne, puis il extrait une boule au
hasard de cette urne. On dit que ce joueur a gagné si le numéro porté
par la boule tirée est impair et on désigne par $A$ l'événement : « le
joueur a gagné ».\\
\noindent On appelle $X$ la variable aléatoire égale au numéro porté
par la boule extraite de l'urne.

\begin{noliste}{1.}
 \setlength{\itemsep}{4mm}
\item Reconnaître la loi de $N$.

\item 
\begin{noliste}{a)}
 \setlength{\itemsep}{2mm}
\item Montrer que, si $m$ est un entier naturel, la commande
\texttt{2*floor(m/2)} renvoie la valeur $m$ si et seulement si $m$ est
pair.

\item Compléter les commandes \texttt{Scilab} suivantes pour qu'elles
simulent $N$ et $X$ puis renvoient l'un des deux messages : « le joueur
a gagné » ou « le joueur a perdu ».\\
\texttt{p = input('donner la valeur de p') \\
N = grand(1,1,'geom',---)//'geom' désigne une loi géométrique\\
X = grand(1,1,'uin',---)//'uin' désigne une loi uniforme discrète \\
if ------ then disp('------'), else disp('------'), end}
\end{noliste}

\item 
\begin{noliste}{a)}
 \setlength{\itemsep}{2mm}
\item Donner, pour tout entier naturel $k$ supérieur ou égal à $j$, la
valeur de $P_{(N = 2j)}(X = 2k + 1)$.

\item Donner, pour tout entier naturel $k$ supérieur ou égal à $j + 1$,
la valeur de 

$P_{(N = 2j + 1)}(X = 2k + 1)$.

\item Déterminer $P_{(N = 2j)}(X = 2k + 1)$ lorsque $k$ appartient à
$\lb 0 ; j-1 \rb$.

\item Déterminer $P_{(N = 2j + 1)}(X = 2k + 1)$ lorsque $k$ appartient
à $\lb 0 ; j \rb$.
\end{noliste}

\item 
\begin{noliste}{a)}
 \setlength{\itemsep}{2mm}
\item 
Justifier que $P\left(\Ev{X = 2k + 1}\right) = \dsum\limits_{n = 1}{+
\infty} P\left(\Ev{N = n}\right)P_{\left(\Ev{N = n}\right)}(X = 2k +
1)$.\\
En admettant que l'on peut scinder la somme précédente selon la parité
de $n$, montrer que :
\[
\forall k\in \N,\; P\left(\Ev{X = 2k + 1}\right) = \dfrac{p}{q}\left(
\dsum\limits_{j = k + 1}{+ \infty} \dfrac{q^{2j}}{2j}-\dsum\limits_{j =
k}{+ \infty} \dfrac{q^{2j + 1}}{2j + 1}\right) 
\]

\item En déduire que : $\forall k\in \N,\; P\left(\Ev{X = 2k +
1}\right) = \dfrac{p}{q}\dint_{0}{q} \dfrac{t^{2k}}{1-t \ dt$.
\end{noliste}

\item 
\begin{noliste}{a)}
 \setlength{\itemsep}{2mm}
\item Montrer que $\dlim{x \rightarrow + \infty} \dint_{0}{q}
\dfrac{t^{2n + 2}}{(1-t)^{2}(1 + t) \ dt = 0$.

\item Montrer que $\dsum\limits_{k = 0}{n} P\left(\Ev{X = 2k +
1}\right) = \dfrac{p}{q}\left( \dint_{0}{q} \dfrac{1}{(1-t)^{2}(1 + t)
\ dt-\dint_{0}{q} \dfrac{t^{2n + 2}}{(1-t)^{2}(1 + t)}dt\right) $.

\item En déduire $P\left(\Ev{A}\right) = \dfrac{p}{q}\dint_{0}{q}
\dfrac{1}{(1-t)^{2}(1 + t) \ dt$.

\end{noliste}

\item 
\begin{noliste}{a)}
 \setlength{\itemsep}{2mm}
\item Trouver trois constantes réelles $a$, $b$ et $c$ telles que, pour
tout $t$ différent de 1 et de -1, on ait :
\[
 \dfrac{1}{(1-t)^{2}(1 + t)} = \dfrac{a}{1-t} + \dfrac{b}{1 + t} +
\dfrac{c}{(1 + t)^{2}}
\]

\item Écrire $P\left(\Ev{A}\right)$ explicitement en fonction de $q$.

\item En déduire que $P\left(\Ev{A}\right)>\dfrac{1}{2}$.
\end{noliste}

\end{noliste}










\end{document}


\Prob\left(\Ev{
\Prob (
\Prob \left(
\Prob\left(
\Prob \left(

P(
P (
P \left(
P\left(
P \left}\right)\left(\Ev{
\Prob \left(
\Prob\left(
\Prob \left(

P(
P (
P \left(
P\left(
P \left}\right)\left(\Ev{
\Prob\left(
\Prob \left(

P(
P (
P \left(
P\left(
P \left}\right)\left(\Ev{
\Prob \left(

P(
P (
P \left(
P\left(
P \left}\right)\left(\Ev{

P(
P (
P \left(
P\left(
P \left}\right)\left(\Ev{
P (
P \left(
P\left(
P \left}\right)\left(\Ev{
P \left(
P\left(
P \left}\right)\left(\Ev{
P\left(
P \left}\right)\left(\Ev{
P \left}\right)\left(\Ev{}\right)(