\documentclass[11pt]{article}%
\usepackage{geometry}%
\geometry{a4paper,
 lmargin = 2cm,rmargin = 2cm,tmargin = 2.5cm,bmargin = 2.5cm}

\input{../../macros.tex}

\pagestyle{fancy} %
\lhead{ECE2 \hfill Mathématiques\\
} %
\chead{\hrule} %
\rhead{} %
\lfoot{} %
\cfoot{} %
\rfoot{\thepage} %

\renewcommand{\headrulewidth}{0pt}% : Trace un trait de séparation
 % de largeur 0,4 point. Mettre 0pt
 % pour supprimer le trait.

\renewcommand{\footrulewidth}{0.4pt}% : Trace un trait de séparation
 % de largeur 0,4 point. Mettre 0pt
 % pour supprimer le trait.

\setlength{\headheight}{14pt}

\title{\bf \vspace{-2cm} EDHEC 2003} %
\author{} %
\date{} %
\begin{document}

\maketitle %
\vspace{-1.4cm}\hrule %
\thispagestyle{fancy}

\vspace*{.2cm}


% DEBUT DU DOC À MODIFIER : tout virer jusqu'au début de l'exo

%Définition et changement de valeurs de
compteurs%newcounter{cpt1}{section} compteur cpt1 remis à 0 à chaque
aumentation par stepcounter du compteur section%setcounter{cpt1}{3} on
met le compteur à 3%addtocounter{cpt1}{5} on ajoute 5 au compteur%
stepcounter{cpt1} on ajoute 1% ifthenelse{test}{alors}{sinon} (page
206) pour subordonner à une condition % whiledo{test}{commande} pour
faire une boucle (page 206 aussi) % value{cpt1} pour noter dans le
document la valeur de cpt1 
%Définition définitive d'opérateurs
mathématiques\newcommand{\ch}{\operatorname{ch}} 
\newcommand{\sh}{\operatorname{sh}}
\renewcommand{\tanh}{\operatorname{th}}
\renewcommand{\sinh}{\operatorname{sh}}
\renewcommand{\cosh}{\operatorname{ch}}
\newcommand{\argsh}{\operatorname{argsh}}
\newcommand{\argch}{\operatorname{argch}}
\newcommand{\argth}{\operatorname{argth}}
\newcommand{\Id}{\operatorname{Id}}
\renewcommand{\leq}{\leq}
\renewcommand{\geq}{\geq }

\newcommand{\dlim}{\lim}
\newcommand{\dsum}{\sum}
\newcommand{\dprod}{\prod}



%Définition de nouvelles couleurs : rgb(trois paramètres red green blue
entre 0 et 1); cmyk (quatre cyan magenta yellow black) entre 0 et 1;
gray (entre 0 et 1) et black, white, red, green, blue, cyan, magenta,
yellow% definecolor{0gris}{gray}{0.8} 
% Nouvelle commande pour encadrer le titre car shabox ne veut que d'une
seule ligne; ATTENTION A LA TAILLE; petite différence avec shadowbox ou
doublebox, voire fcolorbox ou colorbox (au lieu de shabox; laisser le
parbox tranquille sauf pour la taille de la boîte
\newcommand{\Tbox}[1]{\begin{center} \shabox{\parbox{0.6
\linewidth}{#1}} \end{center}} %[1] pour 1 paramètre ; #1 pour ce que
fait le 1er paramètre; entre accolades ce que fait la commande
%Mise en page en mode fancy : en-têtes et pieds de pages puis
définition des en-têtes et pieds de pages\pagestyle{fancy}
\lhead{ECE 2 - Mathématiques \\
Quentin Dunstetter - ENC-Bessières 2011$\backslash$2012}
\chead{}
\rhead{Edhec 2003}
\rfoot[ \ \thepage]{\thepage}
\cfoot{}
\lfoot{}
\thispagestyle{fancy} %Mise en page de la 1ère page en mode fancy
%Trait en bas et en haut de la page (entre en-tête et texte et texte et
pied de page)\renewcommand{\footrulewidth}{0.4pt}
\renewcommand{\headrulewidth}{0.4pt}

\begin{center}
{\Huge EDHEC}

\textbf{School of management\vspace{3cm}}

{\large ECOLE DE HAUTES ETUDES COMMERCIALES DU NORD}

{\large Concours d'admission sur classes préparatoires}

\underline{\hspace{3cm}}

{\LARGE MATHEMATIQUES}

{\large Option économique}

\textbf{Année 2003}{\large }
\end{center}

\noindent \textsl{La présentation, la lisibilité, l'orthographe, la
qualité
de la rédaction, la clarté et la précision des raisonnements entreront
pour
une part importante dans l'appréciation des copies.}

\noindent \textsl{Les candidats sont invités à encadrer dans la mesure
du
possible les résultats de leurs calculs.}

\noindent \textsl{Ils ne doivent faire usage d'aucun document : seule
l'utilisation d'une règle graduée est autorisée.\vspace{1cm}}

\noindent \textbf{L'utilisation de toute calculatrice et de tout
matériel électronique est interdite.}\vspace{1cm}

\section*{Exercice 1}

On note $f$ la fonction définie, pour tout réel $x$ strictement
positif, par : $f\left( x\right) = \dfrac{e^{\dfrac{1}{x}}}{x^{2}}$.

\begin{noliste}{1.}
 \setlength{\itemsep}{4mm}
\item 

\begin{noliste}{a)}
 \setlength{\itemsep}{2mm}
\item Pour tout entier naturel $n$ supérieur ou égal à $1$, montrer que
l'intégrale $I_{n} = \dint{n}{+ \infty }f\left( x\right\dx$ est
convergente et exprimer $I_{n}$ en fonction de $n$.

\item En déduire que $I_{n}\underset{n\rightarrow + \infty }{\sim
}\dfrac{1}{n}$.
\end{noliste}

\item Montrer que la série de terme général $u_{n} = f\left( {n}\right)
$ est convergente.

\item 

\begin{noliste}{a)}
 \setlength{\itemsep}{2mm}
\item Établir que : \quad $\forall k\in ${$\N$}$^{\times },\quad
f\left( k + 1\right) \leq \dint{k}{k + 1}f\left( x\right)
dx\leq f\left( k\right) $.

\item En sommant soigneusement cette dernière inégalité, montrer que : 
\[
\forall n\in {\N}{\times },\quad \Sum{k = n + 1}{+ \infty
}u_{k}\leq I_{n}\leq \Sum{k = n + 1}{+ \infty }u_{k} +
\dfrac{e^{\dfrac{
1}{n}}}{n^{2}}
\]

\item Déduire des questions précédentes un équivalent simple, lorsque
$n$ est au voisinage de $ + \infty $, de $\Sum{k = n + 1}{+ \infty
}\dfrac{e^{\dfrac{1}{k}}}{k^{2}}$.
\end{noliste}
\end{noliste}

\section*{Exercice 2}

Dans cet exercice, $n$ désigne un entier naturel non nul.

\begin{noliste}{1.}
 \setlength{\itemsep}{4mm}
\item Soit $f_{n}$ la fonction définie par : $f_{n}\left( x\right) = 
\left\{
\begin{array}{cl}
nx^{n-1} & \text{si }x\in \left[ 0;1\right] \\
0 & \text{sinon}
\end{array}
\right.
$\\
Montrer que $f_{n}$ est une densité de probabilité.

\item On considère une variable aléatoire $X_{n}$ réelle dont une
densité de probabilité est $f_{n}$. On dit alors que $X_{n}$ suit une
loi monôme d'ordre $n$.

\begin{noliste}{a)}
 \setlength{\itemsep}{2mm}
\item Reconnaître la loi de $X_{1}$.

\item Dans le cas où $n$ est supérieur ou égal à $2$, déterminer la
fonction
de répartition $F_{n}$ de $X_{n}$, ainsi que son espérance $\E\left(
x{_{n}}
\right) $ et sa variance $\V\left( x{_{n}}\right) $.
\end{noliste}

\item On considère deux variables aléatoires $U_{n}$ et $V_{n}$
définies sur
le même espace probabilisé $\left( {\Omega,\mathcal{A},P}\right) $,
suivant
la loi monôme d'ordre $n\ \left( n\geq 2\right) $ et indépendantes,
c'est-à-dire qu'elles vérifient en particulier l'égalité suivante : 
\[
\forall x\in \R,\quad P\left( U_{n}\leq x\cap V_{n}\leq
x\right) = P\left(\Ev{ U_{n}\leq x}\right) P\left(\Ev{ V_{n}\leq
x}\right)
\]

On pose $M_{n} = \sup \left( U_{n},V_{n}\right) $ et on admet que
$M_{n}$ est
une variable aléatoire définie, elle aussi, sur $\left(
{\Omega,\mathcal{A},P}\right) $.

\begin{noliste}{a)}
 \setlength{\itemsep}{2mm}
\item Pour tout réel $x$, écrire, en justifiant la réponse, l'évènement
$
\left( M_{n}\leq x\right) $ à l'aide des évènements $\left(
U_{n}\leq x\right) $ et $\left( V_{n}\leq x\right) $.

\item En déduire une densité de $M_{n}$. Vérifier que $M_{n}$ suit une
loi
monôme dont on donnera l'ordre, puis déterminer sans calcul $\E\left(
M_{n}\right) $.

\item On pose $T_{n} = \inf \left( U_{n},V_{n}\right) $. Exprimer
$M_{n} + T_{n}$
en fonction de $U_{n}$ et $V_{n}$, puis en déduire, sans calcul
d'intégrale,
la valeur de $\E\left( T_{n}\right) $.
\end{noliste}
\end{noliste}

\section*{Exercice 3}

\begin{noliste}{1.}
 \setlength{\itemsep}{4mm}
\item Montrer que : $\forall x\in ${$\R$}$^{\times },\quad \dfrac{
e^{x}-1}{x}>0$.\\
On considère la fonction $f$ définie sur $\R$ par : $
\left\{
\begin{array}{cl}
f\left( x\right) = \ln \left( \dfrac{e^{x}-1}{x}\right) & \text{si
}x\neq 0
\\
f\left( {0}\right) = 0 & 
\end{array}
\right.
$

\item Montrer que $f$ est continue sur $\R$.

\item Montrer que $f$ est de classe $C^{1}$ sur $\left] {-\infty
;0}\right[ $
et sur $\left] {0; + \infty }\right[ $, puis préciser $f^{\prime
}\left(
x\right) $ pour tout $x$ de {$\R$}$^{\times }$.

\item 

\begin{noliste}{a)}
 \setlength{\itemsep}{2mm}
\item Montrer que $\underset{x\rightarrow 0}{\lim }f^{\prime }\left(
x\right) = \dfrac{1}{2}$.

\item En déduire que $f$ est de classe $C^{1}$ sur $\R$ et donner
$f^{\prime }\left( {0}\right) $.
\end{noliste}

\item 

\begin{noliste}{a)}
 \setlength{\itemsep}{2mm}
\item Étudier les variations de la fonction $g$ définie par : $\forall
x\in 
\R,\quad g\left( x\right) = xe^{x}-e^{x} + 1$.

\item En déduire le signe de $g\left( x\right) $, puis dresser le
tableau de
variations de $f$ (limites comprises).
\end{noliste}

\vspace{0pt} On considère la suite $\left( u_{n}\right) $ définie par
la donnée de son premier terme $u_{0}>0$ et par la relation, valable
pour tout entier naturel $n$ : $u_{n + 1} = f\left( u_{n}\right) $.

\vspace{0pt}

\item Montrer que : $\forall n\in \N,\quad u_{n}>0$.

\item 

\begin{noliste}{a)}
 \setlength{\itemsep}{2mm}
\item Vérifier que : $\forall x\in \R,\quad f\left( x\right)
-x = f\left( {-}x\right) $.

\item En déduire le signe de $f\left( x\right) -x$ sur
{$\R$}$_{+}{\times }$.

\item Montrer que la suite $\left( u_{n}\right) $ est décroissante.
\end{noliste}

\item En déduire que $\left( u_{n}\right) $ converge et donner sa
limite.

\item Écrire un programme en \texttt{\Scilab{}} permettant de
déterminer et
d'afficher le plus petit entier naturel $n$ pour lequel $u_{n}\leq
10^{-3}$, dans le cas où $u_{0} = 1$.
\end{noliste}

\section*{Problème}

Un joueur participe à un jeu se jouant en plusieurs parties. Ses
observations lui permettent d'affirmer que :

\begin{noliste}{$\sbullet$}
\item s'il gagne deux parties consécutives, alors il gagne la prochaine
avec
la probabilité $\dfrac{2}{3}$.

\item s'il perd une partie et gagne la suivante, alors il gagne la
prochaine
avec la probabilité $\dfrac{1}{2}$.

\item s'il gagne une partie et perd la suivante, alors il gagne la
prochaine
avec la probabilité $\dfrac{1}{2}$.

\item s'il perd deux parties consécutives, alors il gagne la prochaine
avec
la probabilité $\dfrac{1}{3}$.
\end{noliste}

Pour tout entier naturel $n$ non nul, on note $A_{n}$ l'évènement : "le
joueur gagne la $n^{\text{ième}}$ partie".

De plus, pour tout entier naturel $n$ supérieur ou égal à 2, on pose : 
\[
E_{n} = A_{n-1}\cap A_{n}\qquad F_{n} = \overline{A_{n-1}}\cap
A_{n}\qquad
G_{n} = A_{n-1}\cap \overline{A_{n}}\qquad H_{n} =
\overline{A_{n-1}}\cap 
\overline{A_{n}}
\]

\begin{noliste}{1.}
 \setlength{\itemsep}{4mm}
\item On admet que $\left( E_{n},F_{n},G_{n},H_{n}\right) $ est un
système
complet d'évènements.

\begin{noliste}{a)}
 \setlength{\itemsep}{2mm}
\item Utiliser la formule des probabilités totales pour montrer que,
pour
tout entier naturel $n$ supérieur ou égal à $2$, on a : $P\left(\Ev{
E_{n + 1}}\right) = \dfrac{2}{3}P\left(\Ev{ E_{n}}\right) +
\dfrac{1}{2}P\left(\Ev{
F_{n}}\right) $.

\item Exprimer de la même façon (aucune explication n'est exigée) les
probabilités $P\left(\Ev{ F_{n + 1}}\right) $, $P\left(\Ev{ G_{n +
1}}\right) $ et $
P\left(\Ev{ H_{n + 1}}\right) $ en fonction de $P\left(\Ev{
E_{n}}\right) $, $P\left(\Ev{
F_{n}}\right) $, $P\left(\Ev{ G_{n}}\right) $ et $P\left(\Ev{
H_{n}}\right) $.

\item Pour tout entier naturel $n$ supérieur ou égal à $2$, on pose :
$U_{n} = 
\begin{smatrix}
P\left(\Ev{ E_{n}}\right) \\
P\left(\Ev{ F_{n}}\right) \\
P\left(\Ev{ G_{n}}\right) \\
P\left(\Ev{ H_{n}}\right)
\end{smatrix}
$.

Vérifier que $U_{n + 1} = MU_{n}$, où $M = 
\begin{smatrix}
2/3 & 1/2 & 0 & 0 \\
0 & 0 & 1/2 & 1/3 \\
1/3 & 1/2 & 0 & 0 \\
0 & 0 & 1/2 & 2/3
\end{smatrix}
$.
\end{noliste}

\item 

\begin{noliste}{a)}
 \setlength{\itemsep}{2mm}
\item Soient $P = 
\begin{smatrix}
1 & 1 & 3 & 3 \\
-2 & -1 & -1 & 2 \\
2 & -1 & 1 & 2 \\
-1 & 1 & -3 & 3
\end{smatrix}
$ et $Q = 
\begin{smatrix}
-1 & -3 & 3 & 1 \\
2 & -3 & -3 & 2 \\
2 & 1 & -1 & -2 \\
1 & 1 & 1 & 1
\end{smatrix}
$.

Calculer $PQ$. En déduire que $P$ est inversible et donner son inverse.

\item On note $C_{1}$, $C_{2}$, $C_{3}$ et $C_{4}$ les colonnes de $P$.
Calculer $MC_{1}$, $MC_{2}$, $MC_{3}$ et $MC_{4}$, puis en déduire que
$-
\dfrac{1}{3}$, $\dfrac{1}{6}$, $\dfrac{1}{2}$ et $1$ sont les valeurs
propres de $M$.

\item Justifier que $M = PDP^{-1}$, où $D$ est une matrice diagonale
que l'on déterminera.
\end{noliste}

\vspace{0pt} \textbf{\emph{Dans toute la suite, on suppose que le
joueur a
gagné les deux premières parties.}}

\vspace{\normalbaselineskip}

\item 
\begin{noliste}{a)}
 \setlength{\itemsep}{2mm}
\item Montrer par récurrence que : $\forall n\in \N,\quad
M^{n} = PD^{n}P^{-1}$.

\item Montrer, également par récurrence, que : $\forall n\geq 2,\quad
U_{n} = M^{n-2}U_{2}$.

\item Pour tout entier naturel $n$ supérieur ou égal à 2, donner la
première
colonne de $M^{n}$, puis en déduire $P\left(\Ev{ E_{n}}\right) $,
$P\left(\Ev{
F_{n}}\right) $, $P\left(\Ev{ G_{n}}\right) $ et $P\left(\Ev{
H_{n}}\right) $.

\item Montrer que l'on a : 
\[
\underset{n\rightarrow + \infty }{\lim }P\left(\Ev{ E_{n}}\right) =
\dfrac{3}{10}
\qquad \underset{n\rightarrow + \infty }{\lim }P\left(\Ev{
F_{n}}\right) = \dfrac{2
}{10}\qquad \underset{n\rightarrow + \infty }{\lim }P\left(\Ev{
G_{n}}\right) = 
\dfrac{2}{10}\qquad \underset{n\rightarrow + \infty }{\lim }P\left(\Ev{
H_{n}}\right) = \dfrac{3}{10}\qquad
\]
\end{noliste}

\item Pour tout entier naturel $k$ non nul, on note $X_{k}$ la variable
aléatoire qui vaut $1$ si le joueur gagne la $k^{\text{ième}}$ partie
et qui vaut $0$ sinon ($X_{1}$ et $X_{2}$ sont donc deux variables
certaines).

\begin{noliste}{a)}
 \setlength{\itemsep}{2mm}
\item Pour tout entier naturel $k$ supérieur ou égal à $2$, exprimer
$A_{k}$ en fonction de $E_{k}$ et $F_{k}$.

\item En déduire, pour tout entier naturel $k$ supérieur ou égal à $2$,
la loi de $X_{k}$.
\end{noliste}

\item Pour tout entier naturel $n$ supérieur ou égal à $2$, on note
$S_{n}$ la variable aléatoire égale au nombre de parties gagnées par le
joueur lors des $n$ premières parties.

\begin{noliste}{a)}
 \setlength{\itemsep}{2mm}
\item Calculer $P\left(\Ev{ S_{n} = 2}\right) $ en distinguant les cas
$n = 2$, $n = 3$
et $n\geq 4$.

\item Déterminer $P\left(\Ev{ S_{n} = n}\right) $.

\item Pour tout entier $n$ supérieur ou égal à 3, écrire $S_{n}$ en
fonction
des variables $X_{k}$, puis déterminer $\E\left( S_{n}\right) $ en
fonction
de $n$.
\end{noliste}
\end{noliste}

\label{fin}

\end{document}

