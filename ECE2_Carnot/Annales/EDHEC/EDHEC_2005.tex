\documentclass[11pt]{article}%
\usepackage{geometry}%
\geometry{a4paper,
 lmargin = 2cm,rmargin = 2cm,tmargin = 2.5cm,bmargin = 2.5cm}

\input{../../macros.tex}

\pagestyle{fancy} %
\lhead{ECE2 \hfill Mathématiques\\
} %
\chead{\hrule} %
\rhead{} %
\lfoot{} %
\cfoot{} %
\rfoot{\thepage} %

\renewcommand{\headrulewidth}{0pt}% : Trace un trait de séparation
 % de largeur 0,4 point. Mettre 0pt
 % pour supprimer le trait.

\renewcommand{\footrulewidth}{0.4pt}% : Trace un trait de séparation
 % de largeur 0,4 point. Mettre 0pt
 % pour supprimer le trait.

\setlength{\headheight}{14pt}

\title{\bf \vspace{-2cm} EDHEC 2005} %
\author{} %
\date{} %
\begin{document}

\maketitle %
\vspace{-1.4cm}\hrule %
\thispagestyle{fancy}

\vspace*{.2cm}

\section*{Exercice 1}

On note $J_{1} = \left( 
\begin{array}{cc}
1 & 0 \\
0 & 0
\end{array}
\right),\,J_{2} = \left( 
\begin{array}{cc}
0 & 1 \\
0 & 0
\end{array}
\right),\,J_{3} = \left( 
\begin{array}{cc}
0 & 0 \\
1 & 0
\end{array}
\right),\,$et $J_{4} = \left( 
\begin{array}{cc}
0 & 0 \\
0 & 1
\end{array}
\right),$ et on rappelle que la famille $\left(
J_{1},J_{2},J_{3},J_{4}\right) $ est une base de
$\mathfrak{M}_{2}\left( 
\R\right) $.

Soit $f$ l'application qui, à toute matrice $M = \left( 
\begin{array}{cc}
a & b \\
c & d
\end{array}
\right) $ de $\mathfrak{M}_{2}\left( \R\right),$ associe $f\left(
M\right) = M + \left( a + b\right) I$ où $I$ désigne la matrice $\left(

\begin{array}{cc}
1 & 0 \\
0 & 1
\end{array}
\right) $.

\begin{noliste}{1.}
 \setlength{\itemsep}{4mm}
\item Montrer que $f$ est un endomorphisme de $\mathfrak{M}_{2}\left( 
\R\right) $.

\item 

\begin{noliste}{a)}
 \setlength{\itemsep}{2mm}
\item Exprimer $f\left( J_{1}\right),$ $f\left( J_{2}\right),$ $f\left(
J_{3}\right),$ et $f\left( J_{4}\right) $ comme combinaisons linéaires
de $
J_{1},\,J_{2},\,J_{3}$ et $J_{4}.$

\item Vérifier que la matrice $A$ de $f$ dans la base $\left(
J_{1},J_{2},J_{3},J_{4}\right) $ est $A = \left( 
\begin{array}{rrrr}
2 & 0 & 0 & 1 \\
0 & 1 & 0 & 0 \\
0 & 0 & 1 & 0 \\
1 & 0 & 0 & 2
\end{array}
\right) $.

\item Justifier que $f$ est diagonalisable.
\end{noliste}

\item 

\begin{noliste}{a)}
 \setlength{\itemsep}{2mm}
\item Montrer que $\left( J_{1}-J_{4},J_{2},J_{3},I\right) $ est une
base de 
$\mathfrak{M}_{2}\left( \R\right) $.

\item Écrire la matrice $D$ de $f$ dans cette base.

\item En déduire l'existence d'une matrice $P$ inversible telle que $
A = PDP^{-1}$.
\end{noliste}

\item 

\begin{noliste}{a)}
 \setlength{\itemsep}{2mm}
\item Déterminer la matrice $P^{-1}$.

\item Montrer que, pour tout $n$ de $\N$, $A^{n} = PD^{n}P^{-1}$.

\item En déduire explicitement la matrice $A^{n}$.
\end{noliste}
\end{noliste}

\section*{Exercice 2}

Soit $f$ la fonction définie sur $\R^{2}$ par : $\forall \left(
x,y\right) \in \R^{2},\quad f\left( x,y\right) = x\,e^{x\left(
y^{2} + 1\right) }$.

\begin{noliste}{1.}
 \setlength{\itemsep}{4mm}
\item Justifier que $f$ est de classe $C^{2}$ sur $\R^{2}$.

\item 

\begin{noliste}{a)}
 \setlength{\itemsep}{2mm}
\item Déterminer les dérivées partielles premières de $f$.

\item En déduire que le seul point en lequel $f$ est susceptible de
présenter un extremum local est $A = \left( -1,0\right) $.
\end{noliste}

\item 

\begin{noliste}{a)}
 \setlength{\itemsep}{2mm}
\item Déterminer les dérivées partielles secondes de $f$.

\item Montrer qu'effectivement, $f$ présente un extremum local en $A$.
En préciser la nature et la valeur.
\end{noliste}

\item 

\begin{noliste}{a)}
 \setlength{\itemsep}{2mm}
\item Montrer que : $\forall \left( x,y\right) \in \R^{2},\;f\left(
x,y\right) \geq x\,e^{x}$.

\item En étudiant la fonction $g$ définie sur $\R$ par $g\left(
x\right) = x\,e^{x},$ conclure que l'extremum trouvé à la question 2b)
est un
extremum global de $f$ sur $\R^{2}$.
\end{noliste}
\end{noliste}

\section*{Exercice 3}

Dans cet exercice, $a$ désigne un réel strictement positif.

\begin{noliste}{1.}
 \setlength{\itemsep}{4mm}
\item On considère la fonction $f$ sur $\R$ par : $f\left( t\right)
 = \left\{ 
\begin{array}{ccc}
a\left( 1-t\right) ^{a-1} & \text{si} & t\in \left[ 0,1\right[ \\
0 & \text{si} & t\notin \left[ 0,1\right[ 
\end{array}
\right. $

\begin{noliste}{a)}
 \setlength{\itemsep}{2mm}
\item Pour tout $x$ de $\left[ 0,1\right[,$ calculer $\dint{0}{x}f
\left( t\right) \,dt$.

\item En déduire que $\dint{0}{1}f\left( t\right) \,dt$ est une
intégrale convergente et donner sa valeur.

\item Montrer que $f$ peut être considérée comme une fonction densité
de
probabilité.\\
On considère maintenant une variable aléatoire $X$ admettant $f$ comme
densité et on note $F$ sa fonction de répartition.
\end{noliste}

\item Expliciter $F(x)$ pour tout réel $x$.\\
On se propose de déterminer l'espérance $\E\left( X\right) $et la
variance $
\V\left( X\right) $ de la variable aléatoire $X$. Pour ce faire, on
pose $
Y = -\ln \left( 1-X\right) $ et on admet que $Y$ est une variable
aléatoire à
densité. On note alors $G$ sa fonction de répartition.

\item 

\begin{noliste}{a)}
 \setlength{\itemsep}{2mm}
\item Pour tout réel $x$ positif, exprimer $G\left( x\right) $ en
fonction
de $x.$

\item En déduire que $Y$ suit la loi exponentielle de paramètre $a$.
\end{noliste}

\item 

\begin{noliste}{a)}
 \setlength{\itemsep}{2mm}
\item Pour tout réel $\lambda >0$, donner la valeur de $\dint{0}{+
\infty }e^{-\lambda \,x}dx$.

\item En déduire que la variable aléatoire $e^{-Y}$ possède une
espérance et
donner sa valeur en fonction de $a$.

\item Exprimer $X$ en fonction de $Y$, puis en déduire que $X$ possède
une
espérance dont on donnera l'expression en fonction de $a$.

\item Montrer que la variable aléatoire $e^{-2Y}$ possède une espérance
et
que $\E\left( e^{-2Y}\right) = \dfrac{a}{a + 2}$. \\
En déduire la variance de $e^{-Y}$ puis la variance de $X$.
\end{noliste}
\end{noliste}

\section*{Problème}

Un mobile se déplace sur les points à coordonnées entières d'un axe
d'origine $O.$\\
Au départ, le mobile est à l'origine.\\
Le mobile se déplace selon la règle suivante : s'il est sur le point
d'abscisse $k$ à l'instant $n$, alors, à l'instant $\left( n + 1\right)
$ il
sera sur le point d'abscisse $\left( k + 1\right) $ avec la probabilité
$p$ ($
0<p<1$) ou sur le point d'abscisse $0$ avec la probabilité $1-p$.\\
Pour tout $n$ de $\N$, on note $X_{n}$ l'abscisse de ce point à
l'instant $n$ et l'on a donc $X_{0} = 0$.\\
On admet que, pour tout $n$ de $\N$ $X_{n}$ est définie sur un
espace probabilisé $\left( \Omega,\mathcal{A},P\right) $.\\
Par ailleurs, on note $T$ l'instant auquel le mobile se trouve pour la
première fois à l'origine (sans compter son positionnement au
départ).\\
Par exemple, si les abscisses successives du mobile après son départ
sont $0$, $0$, $1$, $2$, $0$, $0$, $1$, alors on a $T = 1$. Si les
abscisses successives sont : $1$, $2$, $3$, $0$, $0$, $1$, alors on a
$T = 4$.\\
On admet que $T$ est une variable aléatoire définie sur $\left(
\Omega,\mathcal{A},\Prob\right) $.

\begin{noliste}{1.}
 \setlength{\itemsep}{4mm}
\item 

\begin{noliste}{a)}
 \setlength{\itemsep}{2mm}
\item Pour tout $k$ de $\N^{\times }$, exprimer l'évènement $\left(
T = k\right) $ en fonction d'évènements mettant en jeu certaines des
variables 
$X_{i}$.

\item Donner la loi de $X_{1}$.

\item En déduire $P\left(\Ev{ T = k}\right) $ pour tout $k$ de
$\N^{\times }
$, puis reconnaître la loi de $T$.
\end{noliste}

\item 

\begin{noliste}{a)}
 \setlength{\itemsep}{2mm}
\item Montrer par récurrence que, pour tout entier naturel $n$,
$X_{n}\left(
\Omega \right) = [ \ \hspace{-0.15em}[0,n]\hspace{-0.13em}]$.

\item Pour tout $n$ de $\N^{\times }$, utiliser le système complet
d'évènements $\left( X_{n-1} = k\right)_{0\leq k\leq n-1}$ pour montrer
que : $P\left(\Ev{ X_{n} = 0}\right) = 1-p$.
\end{noliste}

\item 

\begin{noliste}{a)}
 \setlength{\itemsep}{2mm}
\item Établir que : $\forall n\in \N,\quad \forall k\in \left\{
1,2,\dots n + 1\right\},\quad P\left(\Ev{ X_{n + 1} = k}\right) =
pP\left(\Ev{
X_{n} = k-1}\right) $.

\item En déduire que : $\forall n\in \N^{\times },\quad \forall k\in
\left\{ 0,1,2\dots,n-1\right\},\quad P\left(\Ev{ X_{n} = k}\right) =
p^{k}\left(
1-p\right).$\\
En déduire également la valeur de $P\left(\Ev{ X_{n} = n}\right) $.
Donner une
explication probabiliste de ce dernier résultat.

\item Vérifier que $\Sum{k = 0}{n}P\left(\Ev{ X_{n} = k}\right) = 1$.
\end{noliste}

\item Dans cette question et dans cette question seulement, on prend $p
= 
\dfrac{1}{3}$.\\
On rappelle que \texttt{random(3)} renvoie au hasard un entier de
$\left\{
0,1,2\right\} $.\\
Compléter le programme suivant pour qu'il simule l'expérience aléatoire
étudiée et affiche la valeur prise par $X_{n}$ pour une valeur de $n$
entrée par l'utilisateur.\\
\texttt{Program edhec2005 ;}

\texttt{Var k, n, u, X : integer ;}

\texttt{begin}

\texttt{\hspace{1cm}Readln(n) ;}

\texttt{\hspace{1cm}Randomize ; }

\texttt{\hspace{1cm}X : = O;}

\texttt{\hspace{1cm}For k : = 1 to n do }

\texttt{\hspace{1cm}begin}

\texttt{\hspace{2cm}u : = random(3) ;}

\texttt{\hspace{2cm}if (u = 2) then X : =.........;}

\texttt{\hspace{2cm}\hspace{1cm}else X : =.......;}

\texttt{\hspace{2cm}end ;}

\texttt{\hspace{1cm}Writeln (X) ;}

\texttt{end.}

\item 

\begin{noliste}{a)}
 \setlength{\itemsep}{2mm}
\item Montrer que : $\forall n\geq 2,\quad
\Sum{k = 1}{n-1}kp^{k-1} = \dfrac{\left( n-1\right) p^{n}-n\,p^{n-1} +
1}{\left( 1-p\right) ^{2}}$.

\item En déduire que $\E\left( X_{n} \right) = \dfrac{p\left(
1-p^{n}\right) }{1-p}$.
\end{noliste}

\item 

\begin{noliste}{a)}
 \setlength{\itemsep}{2mm}
\item Montrer, en utilisant la question 3a), que : $\forall n\in
\N,\quad E\left( X_{n + 1}{2}\right) = p\left( E\left( X_{n}{2}\right)
+ 2\E\left(
X_{n}\right) + 1\right) $.

\item Pour tout entier naturel $n$, on pose $u_{n} = E\left(
X_{n}{2}\right)
 + \left( 2n-1\right) \dfrac{p^{n + 1}}{1-p}$. \\
Montrer que $u_{n + 1} = pu_{n} + \dfrac{p\left( 1 + p\right) }{1-p}$.

\item En déduire l'expression de $u_{n}$, puis celle de $\E\left(
X_{n}{2}\right) $ en fonction de $p$ et $n$.

\item Montrer enfin que : $\V\left( X_{n}\right) = \dfrac{p}{\left(
1-p\right)
^{2}}\left( 1-\left( 2n + 1\right) p^{n}\left( 1-p\right) -p^{2n +
1}\right) $.
\end{noliste}
\end{noliste}

\label{fin}




\end{document}

