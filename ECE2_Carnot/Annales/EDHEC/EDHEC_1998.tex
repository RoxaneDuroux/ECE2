\documentclass[11pt]{article}%
\usepackage{geometry}%
\geometry{a4paper,
 lmargin = 2cm,rmargin = 2cm,tmargin = 2.5cm,bmargin = 2.5cm}

\input{../../macros.tex}

\pagestyle{fancy} %
\lhead{ECE2 \hfill Mathématiques\\
} %
\chead{\hrule} %
\rhead{} %
\lfoot{} %
\cfoot{} %
\rfoot{\thepage} %

\renewcommand{\headrulewidth}{0pt}% : Trace un trait de séparation
 % de largeur 0,4 point. Mettre 0pt
 % pour supprimer le trait.

\renewcommand{\footrulewidth}{0.4pt}% : Trace un trait de séparation
 % de largeur 0,4 point. Mettre 0pt
 % pour supprimer le trait.

\setlength{\headheight}{14pt}

\title{\bf \vspace{-2cm} EDHEC 1998} %
\author{} %
\date{} %
\begin{document}

\maketitle %
\vspace{-1.4cm}\hrule %
\thispagestyle{fancy}

\vspace*{.2cm}


% DEBUT DU DOC À MODIFIER : tout virer jusqu'au début de l'exo

%Définition et changement de valeurs de
compteurs%newcounter{cpt1}{section} compteur cpt1 remis à 0 à chaque
aumentation par stepcounter du compteur section%setcounter{cpt1}{3} on
met le compteur à 3%addtocounter{cpt1}{5} on ajoute 5 au compteur%
stepcounter{cpt1} on ajoute 1% ifthenelse{test}{alors}{sinon} (page
206) pour subordonner à une condition % whiledo{test}{commande} pour
faire une boucle (page 206 aussi) % value{cpt1} pour noter dans le
document la valeur de cpt1 
%Définition définitive d'opérateurs
mathématiques\newcommand{\ch}{\operatorname{ch}} 
\newcommand{\sh}{\operatorname{sh}}
\renewcommand{\tanh}{\operatorname{th}}
\renewcommand{\sinh}{\operatorname{sh}}
\renewcommand{\cosh}{\operatorname{ch}}
\newcommand{\argsh}{\operatorname{argsh}}
\newcommand{\argch}{\operatorname{argch}}
\newcommand{\argth}{\operatorname{argth}}
\newcommand{\ker}{\operatorname{Ker}}
\renewcommand{\im}{\operatorname{Im}}
\newcommand{\rg}{\operatorname{rg}}
\newcommand{\Id}{\operatorname{Id}}
\newcommand{\id}{\operatorname{id}}
\renewcommand{\leq}{\leq}
\renewcommand{\geq}{\geq }

%Définition de nouvelles couleurs : rgb(trois paramètres red green blue
entre 0 et 1); cmyk (quatre cyan magenta yellow black) entre 0 et 1;
gray (entre 0 et 1) et black, white, red, green, blue, cyan, magenta,
yellow% definecolor{0gris}{gray}{0.8} 
% Nouvelle commande pour encadrer le titre car shabox ne veut que d'une
seule ligne; ATTENTION A LA TAILLE; petite différence avec shadowbox ou
doublebox, voire fcolorbox ou colorbox (au lieu de shabox; laisser le
parbox tranquille sauf pour la taille de la boîte
\newcommand{\Tbox}[1]{\begin{center} \shabox{\parbox{0.6
\linewidth}{#1}} \end{center}} %[1] pour 1 paramètre ; #1 pour ce que
fait le 1er paramètre; entre accolades ce que fait la commande
%Mise en page en mode fancy : en-têtes et pieds de pages puis
définition des en-têtes et pieds de pages\pagestyle{fancy}
\lhead{ECE 2 - Mathématiques \\
Quentin Dunstetter - ENC-Bessières 2011$\backslash$2012}
\chead{}
\rhead{Edhec 1998}
\rfoot[ \ \thepage]{\thepage}
\cfoot{}
\lfoot{}
\thispagestyle{fancy} %Mise en page de la 1ère page en mode fancy
%Trait en bas et en haut de la page (entre en-tête et texte et texte et
pied de page)\renewcommand{\footrulewidth}{0.4pt}
\renewcommand{\headrulewidth}{0.4pt}


%DEBUT DU DOCUMENT\vspace*{3cm}

\begin{center}
{\LARG\E\textbf{BANQUE COMMUNE D'ÉPREUVES}}



{\large \textsc{CONCOURS D ADMISSION DE 1998}}



{\large \textbf{Concepteur : Edhec}}



\rule{2.39cm}{0.05cm}



{\Large \textbf{OPTION ÉCONOMIQUE}}



{\Large \textbf{MATHÉMATIQUES }}



{\Large Lundi 9 mai, de 14h à 18h}



\rule{2.39cm}{0.05cm}
\end{center}

\textit{La présentation, la lisibilité, l'orthographe, la qualité
de la rédaction, la clarté et la précision des raisonnements
entreront pour une part importante dans l'appréciation des copies.}

\textit{Les candidats sont invités à \textbf{encadrer} dans la mesure
du possible les résultats de leurs calculs.}

\textit{Ils ne doivent faire usage d'aucun document. L'utilisation de
toute
calculatrice et de tout matériel électronique est interdite. Seule
l'utilisation d'une règle graduée est autorisée.}

\textit{Si au cours de l'épreuve, un candidat repère ce qui lui semble
être une erreur d'énoncé, il la signalera sur sa copie et
poursuivra sa composition en expliquant les raisons des initiatives
qu'il sera
amené à prendre.}

\vspace*{3cm}

\section*{Exercice 1}

\begin{noliste}{1.}
 \setlength{\itemsep}{4mm}
\item On considère la fonction $g$ définie pour tout $x$ élément de
$\R_{+}{\times }$ par $g(x) = \ln x + 2x + 1$

\begin{noliste}{a)}
 \setlength{\itemsep}{2mm}
\item Étudier les variations de $g$ et donner les limites de g en
$0^{+}$ et
en $ + \infty $.

\item En déduire qu'il existe un unique réel $\alpha $, élément de
$]0,\dfrac{1}{e}[$ tel que $g(\alpha ) = 0$.
\end{noliste}

\item On considère la fonction de deux variables réelles $f$ définie
par : 
\[
\forall (x,y)\in \R_{+}{\times }\times \R\qquad
f(x,y) = x(\ln x + x + y^{2})
\]

\begin{noliste}{a)}
 \setlength{\itemsep}{2mm}
\item Déterminer le seul point critique de $f$, c'est à dire le seul
couple
de $\R_{+}{\times }\times \R$ en lequel $f$ est susceptible
de présenter un extremum.

\item Vérifier que $f$ présente un minimum relatif $m$ en ce point.

\item Montrer que $m = -\alpha (\alpha + 1)$
\end{noliste}
\end{noliste}

\section*{Exercice 2}

$E$ désigne un espace vectoriel sur $\R$, rapporté à une base
$\mathcal{B} = (e_{1},\ e_{2},\ e_{3})$.\\
Pour tout réel $a$, on considère l'endomorphisme $f_{a}$ de $E$ défini
par : 
\[
f_{a}(e_{2}) = 0\qquad \text{et}\qquad
f_{a}(e_{1}) = f_{a}(e_{3}) = ae_{1} + e_{2}-ae_{3}
\]

\begin{noliste}{1.}
 \setlength{\itemsep}{4mm}
\item 

\begin{noliste}{a)}
 \setlength{\itemsep}{2mm}
\item Déterminer une base de $\operatorname{Im}$ $f_{a}$.

\item Montrer qu'une base de $\ker f_{a}$ est $(e_{2},\ e_{1}-e_{3})$.
\end{noliste}

\item Écrire la matrice de $f_{a}$ dans $\mathcal{B}$ et calculer
$A^{2}$.
En déduire sans calcul $f_{a}\circ f_{a}$

\item On pose $e_{1}{\prime } = f_{a}(e_{1})\quad e_{2}{\prime
} = e_{1}-e_{3}\quad e_{3}{\prime } = e_{3}$

\begin{noliste}{a)}
 \setlength{\itemsep}{2mm}
\item Montrer que $(e_{1}{\prime },\ e_{2}{\prime },\ e_{3}{\prime })$
est une base de $E$.

\item Donner la matrice $A^{\prime }$ de $f_{a}$ dans cette base.

\item En déduire que $0$ est la seule valeur propre de $A$. $A$
est-elle
inversible ? $A$ est-elle diagonalisable ?
\end{noliste}

\item Pour tout réel $x$ non nul, on pose $B(x) = A-xI$, $I$ désignant
la
matrice identité de $\M{3}$.

\begin{noliste}{a)}
 \setlength{\itemsep}{2mm}
\item Montrer sans calcul que $B(x)$ est inversible.

\item Calculer $(A-xI)(A + xI)$ puis écrire $(B(x))^{-1}$ en fonction
de $x$, $I$ et $A$.

\item Pour tout $n$ de $\N$, déterminer $(B(x))^{n}$ en fonction de
$x$, $n$, $I$ et $A$.
\end{noliste}
\end{noliste}

\section*{Exercice 3}

On réalise une suite de lancers d'une pièce équilibrée, chaque lancer
amenant donc pile ou face avec une probabilité 1/2.\\
On note $P_{k}$ (resp. $F_{k}$ ) l'évènement : \textquotedblleft on
obtient
pile (resp. face) au $k^{\acute{e}me}$ lancer".\\
Pour ne pas surcharger l'écriture, on écrira, par exemple, $P_{1}F_{2}$
à la
place de $P_{1}\cap F_{2}$.\\
On note $X$ la variable aléatoire qui prend la valeur $k$ si l'on
obtient
pour la première fois pile puis face dans cet ordre aux lancers $k-1$
et $k$
($k$ désignant un entier supérieur ou égal à 2), $X$ prenant la valeur
$0$
si l'on obtient jamais une telle succession.

\begin{noliste}{1.}
 \setlength{\itemsep}{4mm}
\item Calculer $P\left(\Ev{X = 2}\right).$

\item 
\begin{noliste}{a)}
 \setlength{\itemsep}{2mm}
\item En remarquant que $(X = 3) = P_{1}P_{2}F_{3}\cup
F_{1}P_{2}F_{3}$,
calculer $P\left(\Ev{X = 3}\right)$.

\item Sur le modèle de la question précédente, écrire, pour tout entier
$k$
supérieur ou égal à 3, l'évènement $(X = k)$ comme réunion de $(k-1)$
évènements incompatibles.

\item Déterminer P$(X = k)$ pour tout entier $k$ supérieur ou égal à 2.

\item Calculer P$(X = 0)$.
\end{noliste}

\item On se propose, dans cette question, de retrouver le résultat de
la
question 2) c : par une autre méthode.

\begin{noliste}{a)}
 \setlength{\itemsep}{2mm}
\item Montrer que, $k$ désignant un entier supérieur ou égal à 3, si le
premier lancer est un pile, alors il faut et il suffit que
$P_{2}P_{3}\dots
P_{k-1}F_{k}$ se réalise pour que $(X = k)$ se réalise.

\item En déduire, en utilisant la formule des probabilités totales que
: 
\[
\forall k\geq 3\qquad P\left(\Ev{X = k}\right) =
\dfrac{1}{2}P\left(\Ev{X = k-1}\right) + \dfrac{1}{2^{k}}
\]

\item On pose, pour tout entier $k$ supérieur ou égal à 2, $u_{k} =
2^{k}P\left(\Ev{X = k}\right)
$.\\
Montrer que la suite $(u_{k})_{k\geq 2}$ est arithmétique. Retrouver le
résultat annoncé.
\end{noliste}

\item Montrer que $X$ a une espérance $\E(X)$, puis la calculer.
\end{noliste}

\section*{Problème}

La partie I permet d'établir des résultats utiles pour les parties II
et III.\\
Les parties II et III sont indépendantes entre elles.\\
On considère le fonction $f$ définie pour tout réel $x$ positif ou nul
par $f(x) = 1-e^{-x}$.

\subsection*{Partie I}

\begin{noliste}{1.}
 \setlength{\itemsep}{4mm}
\item 
\begin{noliste}{a)}
 \setlength{\itemsep}{2mm}
\item Dresser le tableau de variations de $f$.

\item Montrer que : $\forall x\in \R^{+}\qquad f(x)\leq x$
\qquad, l'égalité ayant lieu seulement pour $x = 0$.
\end{noliste}

\item 
\begin{noliste}{a)}
 \setlength{\itemsep}{2mm}
\item Montrer que, pour tout entier naturel $n$ et pour tout réel $x$ :

\[
e^{-x} = \Sum{k = 0}{n}\dfrac{(-1)^{k}x^{k}}{k!} + (-1)^{n +
1}\dint{0}{x}\dfrac{(x-t)^{n}}{n!}e^{-t}dt
\]

\item En écrivant l'égalité précédente pour $n = 2$, puis pour $n = 3$,
montrer
que : 
\[
\forall x\in \R\qquad \dfrac{x^{2}}{2}-\dfrac{x^{3}}{6}\leq
x-f(x)\leq \dfrac{x^{2}}{2}
\]
\end{noliste}
\end{noliste}

\subsection*{Partie II}

On considère la suite $(u_{n})$ définie par son premier terme $u_{0} =
1$ et
par la relation : 
\[
\forall n\in \N\qquad u_{n + 1} = f(u_{n})
\]

\begin{noliste}{1.}
 \setlength{\itemsep}{4mm}
\item 
\begin{noliste}{a)}
 \setlength{\itemsep}{2mm}
\item Montrer que : $\forall n\in \N\qquad u_{n}\in \ ]0,1]$

\item Montrer, grâce à la question I 1), que : $\forall n\in \N\qquad
u_{n}-u_{n + 1}\geq 0$

\item Conclure quant à la convergence de la suite $(u_{n})$ et donner
sa
limite.
\end{noliste}

\item 
\begin{noliste}{a)}
 \setlength{\itemsep}{2mm}
\item Simplifier, pour tout $n$ élément de $\N^{\times }$, la somme
$\Sum{k = 0}{n-1}(u_{k}-u_{k + 1})$.

\item En déduire que la série de terme général $(u_{n}-u_{n + 1})$ est
convergente.

\item En utilisant la question I 2), montrer que $u_{n}-u_{n + 1}\sim
\dfrac{u_{n}{2}}{2}$ en $ + \infty $.

\item Donner enfin la nature de la série de terme général $u_{n}{2}$.
\end{noliste}
\end{noliste}

\subsection*{Partie III}

\begin{noliste}{1.}
 \setlength{\itemsep}{4mm}
\item On note $\phi $ la fonction définie sur $\R$ par : $\phi (0) = 1$
et $\forall x\in \R_{\times }{+}$, \qquad $\phi (x) =
\dfrac{f(x)}{x}$.\\
Montrer que $\phi $ est continue sur $\R^{+}$.

\hspace{-1cm}On considère la fonction réelle $g$ définie par $g(0) = 1$
et $\forall x\in \R_{\times }{+}$, \qquad $g(x) =
\dfrac{1}{x}\dint{0}{x}\phi (t)dt$.

\item 
\begin{noliste}{a)}
 \setlength{\itemsep}{2mm}
\item Vérifier que $g$ est bien définie et continue sur $\R_{\times
}{+}$.

\item Montrer que $\forall x\in \R_{\times }{+}\qquad
1-\dfrac{x}{4}\leq g(x)\leq 1-\dfrac{x}{4} + \dfrac{x^{2}}{18}$.

\item En déduire que $g$ est continue en $0$, dérivable en $0$ puis
donner $g^{\prime }(0)$.
\end{noliste}

\item 
\begin{noliste}{a)}
 \setlength{\itemsep}{2mm}
\item Montrer que : $\forall x\in \ ]1, + \infty \lbrack \qquad
\dint{1}{x}\phi (t)dt\leq \ln x$.

\item En déduire que $g$ a une limite finie en $ + \infty $ et donner
la
valeur de cette limite.
\end{noliste}

\item 
\begin{noliste}{a)}
 \setlength{\itemsep}{2mm}
\item Pour tout réel $x$ strictement positif, calculer $g^{\prime }(x)$
et l'écrire sous la forme $g^{\prime }(x) = \dfrac{h(x)}{x^{2}}$

\item Montrer alors que $xh^{\prime }(x) = (x + 1)e^{-x}-1$

\item Étudier la fonction notée $k$ définie par :$\forall x\in
\R^{+}\qquad k(x) = (x + 1)e^{-x}-1$

\item Donner le signe de $k$, puis les variations de $g$ et enfin
celles de $g$.

\item Dresser le tableau de variations de $g$ et tracer l'allure de sa
courbe représentative dans un repère orthonormé.
\end{noliste}
\end{noliste}

\label{fin}

\end{document}

