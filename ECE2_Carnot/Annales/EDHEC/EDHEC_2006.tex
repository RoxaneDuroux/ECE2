\documentclass[11pt]{article}%
\usepackage{geometry}%
\geometry{a4paper,
 lmargin = 2cm,rmargin = 2cm,tmargin = 2.5cm,bmargin = 2.5cm}

\input{../../macros.tex}

\pagestyle{fancy} %
\lhead{ECE2 \hfill Mathématiques\\
} %
\chead{\hrule} %
\rhead{} %
\lfoot{} %
\cfoot{} %
\rfoot{\thepage} %

\renewcommand{\headrulewidth}{0pt}% : Trace un trait de séparation
 % de largeur 0,4 point. Mettre 0pt
 % pour supprimer le trait.

\renewcommand{\footrulewidth}{0.4pt}% : Trace un trait de séparation
 % de largeur 0,4 point. Mettre 0pt
 % pour supprimer le trait.

\setlength{\headheight}{14pt}

\title{\bf \vspace{-2cm} EDHEC 2006} %
\author{} %
\date{} %
\begin{document}

\maketitle %
\vspace{-1.4cm}\hrule %
\thispagestyle{fancy}

\vspace*{.2cm}


% DEBUT DU DOC À MODIFIER : tout virer jusqu'au début de l'exo

%Définition et changement de valeurs de
compteurs%newcounter{cpt1}{section} compteur cpt1 remis à 0 à chaque
aumentation par stepcounter du compteur section%setcounter{cpt1}{3} on
met le compteur à 3%addtocounter{cpt1}{5} on ajoute 5 au compteur%
stepcounter{cpt1} on ajoute 1% ifthenelse{test}{alors}{sinon} (page
206) pour subordonner à une condition % whiledo{test}{commande} pour
faire une boucle (page 206 aussi) % value{cpt1} pour noter dans le
document la valeur de cpt1 
%Définition définitive d'opérateurs
mathématiques\newcommand{\ch}{\operatorname{ch}} 
\newcommand{\sh}{\operatorname{sh}}
\renewcommand{\tanh}{\operatorname{th}}
\renewcommand{\sinh}{\operatorname{sh}}
\renewcommand{\cosh}{\operatorname{ch}}
\newcommand{\argsh}{\operatorname{argsh}}
\newcommand{\argch}{\operatorname{argch}}
\newcommand{\argth}{\operatorname{argth}}
\newcommand{\Id}{\operatorname{Id}}
\renewcommand{\leq}{\leq}
\renewcommand{\geq}{\geq }

\newcommand{\dlim}{\lim}
\newcommand{\dsum}{\sum}
\newcommand{\dprod}{\prod}



%Définition de nouvelles couleurs : rgb(trois paramètres red green blue
entre 0 et 1); cmyk (quatre cyan magenta yellow black) entre 0 et 1;
gray (entre 0 et 1) et black, white, red, green, blue, cyan, magenta,
yellow% definecolor{0gris}{gray}{0.8} 
% Nouvelle commande pour encadrer le titre car shabox ne veut que d'une
seule ligne; ATTENTION A LA TAILLE; petite différence avec shadowbox ou
doublebox, voire fcolorbox ou colorbox (au lieu de shabox; laisser le
parbox tranquille sauf pour la taille de la boîte
\newcommand{\Tbox}[1]{\begin{center} \shabox{\parbox{0.6
\linewidth}{#1}} \end{center}} %[1] pour 1 paramètre ; #1 pour ce que
fait le 1er paramètre; entre accolades ce que fait la commande
%Mise en page en mode fancy : en-têtes et pieds de pages puis
définition des en-têtes et pieds de pages\pagestyle{fancy}
\lhead{ECE 2 - Mathématiques \\
Quentin Dunstetter - ENC-Bessières 2011$\backslash$2012}
\chead{}
\rhead{Edhec 2006}
\rfoot[ \ \thepage]{\thepage}
\cfoot{}
\lfoot{}
\thispagestyle{fancy} %Mise en page de la 1ère page en mode fancy
%Trait en bas et en haut de la page (entre en-tête et texte et texte et
pied de page)\renewcommand{\footrulewidth}{0.4pt}
\renewcommand{\headrulewidth}{0.4pt}

\begin{center}
{\Huge EDHEC}

\textbf{School of management\vspace{3cm}}

{\large ECOLE D\E\ HAUTES\ ETUDES\ COMMERCIALES\ DU\ NORD}

{\large Concours d'admission sur classes préparatoires}

\underline{\hspace{3cm}}

{\LARGE MATHEMATIQUES}

{\large Option économique}

\textbf{Année 2006}{\large }
\end{center}

\noindent \textsl{La présentation, la lisibilité, l'orthographe, la
qualité
de la rédaction, la clarté et la précision des raisonnements entreront
pour
une part importante dans l'appréciation des copies.}

\noindent \textsl{Les candidats sont invités à encadrer dans la mesure
du
possible les résultats de leurs calculs.}

\noindent \textsl{Ils ne doivent faire usage d'aucun document : seule
l'utilisation d'une règle graduée est autorisée.\vspace{1cm}}

\noindent \textbf{L'utilisation de toute calculatrice et de tout
matériel électronique est interdite.}\vspace{1cm}

\section*{Exercice 1}

Soit $f$ l'endomorphisme de $\R^{3}$ dont la matrice dans la base
canonique $\mathcal{B}$ de $\R^{3}$ est : $A = 
\begin{smatrix}
2 & 10 & 7 \\
1 & 4 & 3 \\
-2 & -8 & -6
\end{smatrix}
$

On note $I$ la matrice unité et $O$ la matrice nulle de
$\mathfrak{M}_{3}(
\R)$ et on pose $u = (2;1;-2)$

\begin{noliste}{1.}
 \setlength{\itemsep}{4mm}
\item 

\begin{noliste}{a)}
 \setlength{\itemsep}{2mm}
\item Montrer que $\ker (f) = \Vect(u)$.

\item La matrice $A$ est-elle inversible ?
\end{noliste}

\item 

\begin{noliste}{a)}
 \setlength{\itemsep}{2mm}
\item Déterminer le vecteur v de $\R^{3}$ dont la $2$-ième coordonnée
dans $B$ vaut 1, et tel que $f(v) = u.$

\item Démontrer que le vecteur $w$ de $\R^{3}$,dont la $2$-ième
coordonnée dans $B$ vaut 1, et qui vérifie $f(w) = v$ est $w =
(0;1;-1)$.

\item Montrer que $(u;v;w)$ est une base de $\R^{3}$ que l'on notera 
$\mathcal{B}{\prime }$. On note $P$ la matrice de passage de la base
$\mathcal{B}$ à la base $\mathcal{B}{\prime }$.
\end{noliste}

\item 

\begin{noliste}{a)}
 \setlength{\itemsep}{2mm}
\item Écrire la matrice $N$ de f relativement à la base
$\mathcal{B}{\prime
}$. En déduire la seule valeur propre de $f$. L'endomorphisme $f$
est-il
diagonalisable ?

\item Donner la relation liant les matrices $A,N,P$ et $P^{-1}$, puis
en déduire que, pour tout entier $k$ supérieur ou égal à 3, on a :
$A^{k} = O$.
\end{noliste}

\item On note $C_{N}$ (respectivement $C_{A}$) l'ensemble des matrices
de $
\mathfrak{M}_{3}(\R)$ qui commutent avec N (respectivement A),

\begin{noliste}{a)}
 \setlength{\itemsep}{2mm}
\item Montrer que $C_{N}$ est un sous-espace vectoriel de
$\mathfrak{M}_{3}(
\R)$ et que $C_{N} = \Vect(I;N;N^{2})$.\\
On admet que $C_{A}$ est aussi un sous-espace vectoriel de
$\mathfrak{M}_{3}(
\R)$.

\item Établir que : \textquotedblleft $M\in C_{A}$ \textquotedblright\
$
\Leftrightarrow $ \textquotedblleft\ $P^{-1}MP\in C_{N}$
\textquotedblright. En déduire que $C_{A} = \Vect(I;A;A^{2})$. Quelle
est la dimension de $C_{A}$ ?
\end{noliste}
\end{noliste}

\subsection*{Exercice 2}

On considère la fonction f définie par f (x) = $\left\{ 
\begin{array}{cc}
\dfrac{1}{2(1-x)^{2}} & \text{si }x\in \left[ 0,\dfrac{1}{2}\right[
\\
\dfrac{1}{2x^{2}} & \text{si }x\in \left[ \ \dfrac{1}{2},1\right[ \ 
\\
0 & \text{si }x\in \R\backslash \left[ 0,1\right[ \ 
\end{array}
\right. $ 

\begin{noliste}{1.}
 \setlength{\itemsep}{4mm}
\item Montrer que f peut être considérée comme une densité de
probabilité.
\\
Dans toute la suite, on considère une variable aléatoire $X$ définie
sur un
certain espace probabilisé \\
$(\Omega ;A;P)$ et admettant la fonction f pour densité.

\item Déterminer la fonction de répartition $F$ de $X$.

\item Montrer que $X$ a une espérance et que celle-ci vaut
$\dfrac{1}{2}$.

\item 

\begin{noliste}{a)}
 \setlength{\itemsep}{2mm}
\item Déterminer $\E((X-1)^{2})$.

\item En déduire que $X$ a une variance et que $\V(X) =
\dfrac{3}{4}-\ln (2)$.
\end{noliste}

\item On appelle variable indicatrice d'un évènement A, la variable de
Bernoulli qui vaut 1 si A est réalisé et 0 sinon. On considère
maintenant la
variable aléatoire $Y$, indicatrice de l'évènement $(X\leq 1/2)$ et la
variable aléatoire $Z$, indicatrice de l'évènement $(X>1/2)$.

\begin{noliste}{a)}
 \setlength{\itemsep}{2mm}
\item Préciser la relation liant $Y$ et $Z$ puis établir sans calcul
que le
coefficient de corrélation linéaire de $Y$ et $Z$, noté $\rho (Y;Z)$,
est égal à $-1.$

\item En déduire la valeur de la covariance de $Y$ et $Z$.
\end{noliste}
\end{noliste}

\subsubsection*{Exercice 3}

Soit $f$ la fonction définie pour tout couple $(x;y)$ de $\R^{2}$ par :
$f(x;y) = 2x^{2} + 2y^{2} + 2xy-x-y.$

\begin{noliste}{1.}
 \setlength{\itemsep}{4mm}
\item 

\begin{noliste}{a)}
 \setlength{\itemsep}{2mm}
\item Calculer les dérivées partielles premières de $f$.

\item En déduire que le seul point critique de $f$ est $A = \left(
\dfrac{1}{6},\dfrac{1}{6}\right) $.
\end{noliste}

\item 

\begin{noliste}{a)}
 \setlength{\itemsep}{2mm}
\item Calculer les dérivées partielles secondes de $f$.

\item Montrer que $f$ présente un minimum local en $A$ et donner la
valeur $m
$ de ce minimum.
\end{noliste}

\item 

\begin{noliste}{a)}
 \setlength{\itemsep}{2mm}
\item Développer $2\left( x + \dfrac{y}{2}-\dfrac{1}{4}\right) ^{2} +
\dfrac{3}{2}\left( y-\dfrac{1}{6}\right) ^{2}$.

\item En déduire que m est le minimum global de $f$ sur $\R^{2}$.
\end{noliste}

\item On considère la fonction $g$ définie pour tout couple $(x;y)$ de
$
\R^{2}$, par :$g(x;y) = 2e^{2x} + 2e^{2y} + 2e^{x + y}-e^{x}-e^{y}.$

\begin{noliste}{a)}
 \setlength{\itemsep}{2mm}
\item Utiliser la question 3. pour établir que : $\forall (x;y)\in \R
^{2},\quad g(x;y)\geq -1/6$.

\item En déduire que $g$ possède un minimum global sur $\R^{2}$ et
préciser en quel point ce minimum est atteint.
\end{noliste}
\end{noliste}

\section*{Problème}

\subsection*{Partie I : Étude d'une variable discrète sans mémoire.}

Soit $X$ une variable aléatoire discrète, à valeurs dans $\N$ telle
que : $\forall m\in \N,\quad P\left(\Ev{X\geq m}\right)>0$. \\
On suppose également que $X$ vérifie : $\forall (m;n)\in \N\times 
\N,\quad P_{(x\geq m)}(X\geq n + m) = P\left(\Ev{X\geq n}\right)$.\\
On pose $P\left(\Ev{X = 0}\right) = p$ et on suppose que $p>0$. 

\begin{noliste}{1.}
 \setlength{\itemsep}{4mm}
\item On pose $q = 1-p$. Montrer que $P\left(\Ev{X\geq 1}\right) = q$.
En déduire que $0<q<1
$.

\item Montrer que : $\forall (m;n)\in \N\times \N,\quad
P\left(\Ev{X\geq n + m}\right) = P\left(\Ev{X\geq
m}\right)P\left(\Ev{X\geq n}\right)$. 

\item Pour tout n de $N$ on pose $u_{n} = P\left(\Ev{X\geq n}\right)$.

\begin{noliste}{a)}
 \setlength{\itemsep}{2mm}
\item Utiliser la relation obtenue à la deuxième question pour montrer
que la suite $(u_{n})$ est géométrique. 

\item Pour tout $n$ de $\N$, exprimer $P\left(\Ev{X\geq n}\right)$ en
fonction de $n$ et de $q$. 

\item Établir que : $\forall n\in \N,\quad P\left(\Ev{X = n}\right) =
P\left(\Ev{X\geq 
n}\right)-P\left(\Ev{X\geq n + 1}\right)$ 

\item En déduire que, pour tout $n$ de $\N$, on a $P\left(\Ev{X =
n}\right) = q^{n}p$.
\end{noliste}

\item 

\begin{noliste}{a)}
 \setlength{\itemsep}{2mm}
\item Reconnaître la loi suivie par la variable $X + 1$.

\item En déduire $\E(X)$ et $\V(X)$.
\end{noliste}
\end{noliste}

\subsection*{Partie II : Taux de panne d'une variable discrète.}

Pour toute variable aléatoire $Y$ à valeurs dans $\N$ et telle que,
pour tout $n$ de $\N$, $P\left(\Ev{Y\geq n}\right)>0$, on définit le
taux de
panne de $Y$ à l'instant $n$, noté $\lambda_{n}$ par : $\forall n\in 
\N,\quad \lambda_{n} = P_{(y\geq n)}(Y\geq n).$

\begin{noliste}{1.}
 \setlength{\itemsep}{4mm}
\item 

\begin{noliste}{a)}
 \setlength{\itemsep}{2mm}
\item Montrer que : $\forall n\in \N,\quad \lambda_{n} =
\dfrac{P\left(\Ev{Y = n}\right)
}{P\left(\Ev{Y\geq n}\right)}.$

\item En déduire que :$\forall n\in \N,\quad 1-\lambda_{n} = \dfrac{
P\left(\Ev{Y\geq n + 1}\right)}{P\left(\Ev{Y\geq n}\right)}.$

\item Établir alors que : $\forall n\in \N$, $0\leq \lambda
_{n}<1$.

\item Montrer par récurrence, que : $\forall n\in \N^{\times },\quad
P\left(\Ev{Y\geq n}\right) = \dprod\limits_{k =
0}{n-1}(1-\lambda_{k}).$
\end{noliste}

\item 

\begin{noliste}{a)}
 \setlength{\itemsep}{2mm}
\item Montrer que :$\forall n\in \N^{\times },\quad
\dsum\limits_{k = 0}{n-1}P\left(\Ev{Y = k}\right) = 1-P\left(\Ev{Y\geq
n}\right)$

\item En déduire que $\dlim{n\rightarrow + \infty }P\left(\Ev{Y\geq
n}\right) = 0$. 

\item Montrer que $\dlim{n\rightarrow + \infty
}\dsum\limits_{k = 0}{n-1}-\ln (1-\lambda_{k}) = + \infty $. 

\item Conclure quant à la nature de la série de terme général
$\lambda_{n}$.
\end{noliste}

\item 

\begin{noliste}{a)}
 \setlength{\itemsep}{2mm}
\item On considère la déclaration de fonction, en \Scilab{}, rédigée de
manière
récursive : \\
Function f (n : integer) : integer ;\\
Begin \\
If (n = 0) then f : = ------ \\
else f : = ------- ;\\
end;\\
Compléter cette déclaration pour qu'elle renvoie n! lorsqu'on appelle f
(n).

\item On considère la déclaration de fonction récursive suivante \\
Function g (a : real ; n : integer) : real ;\\
Begin \\
lf (n = 0) then g : = 1 else g : = a *g(a,n -1); \\
end ;\\
Dire quel est le résultat retourné à l'appel de g(a, n).

\item Proposer un programme (sans écrire la partie déclarative)
utilisant
ces deux fonctions et permettant d'une part le calcul de la somme $
\dsum\limits_{k = 0}{n-1}\dfrac{a^{k}}{k!}e^{-a}$ et d'autre part, à
l'aide
du résultat de la question l.a), le calcul et l'affichage du taux de
panne à
l'instant n d'une variable aléatoire suivant la loi de Poisson de
paramètre $
a>0$, lorsque n et a sont entrés au clavier par l'utilisateur \\
(on supposera $n\geq 1$).

\item Compléter la déclaration de fonction suivante pour qu'elle
renvoie la
valeur de $\dsum\limits_{k = 0}{n-1}\dfrac{a^{k}}{k!}e^{-a}$ à l'appel
de
sigma (a, n).\\
Function sigma(a : real ; n : integer) : real ;\\
var k : integer ; p : real ;\\
Begin \\
p : = 1 ; s : = 1 ; \\
For k : = 1 to n -1 do begin p : = p*a/ k ; s : =....... end ; \\
s. =.....\dots \dots.. ; \\
sigma : = s;\\
end ;
\end{noliste}
\end{noliste}

\subsection*{Partie III : Caractérisation des variables dont la loi est
du
type de celle de X.}

\begin{noliste}{1.}
 \setlength{\itemsep}{4mm}
\item Déterminer le taux de panne de la variable $X$ dont la loi a été
trouvée à la question 3d) de la partie 1.

\item On considère une variable aléatoire $Z$, à valeurs dans $\N$,
et vérifiant : $\forall n\in \N^{\times },\quad P\left(\Ev{Z\geq
n}\right)>0$.
\\
On suppose que le taux de panne de $Z$ est constant, c'est-à-dire que
l'on a : $\forall n\in \N,\quad \lambda_{n} = \lambda $.

\begin{noliste}{a)}
 \setlength{\itemsep}{2mm}
\item Montrer que $0<\lambda <1$. 

\item Pour tout $n$ de $\N$, déterminer $P\left(\Ev{Z\geq n$}\right) en
fonction de $\lambda $ et $n$.

\item Conclure que les seules variables aléatoires $Z$ à valeurs dans $
\N$ dont le taux de panne est constant et telles que pour tout $n$
de $\N,\quad P\left(\Ev{Z\geq n}\right)>0$, sont les variables dont la
loi est
du type de celle de $X$.
\end{noliste}
\end{noliste}

\label{fin}

\end{document}

