\documentclass[11pt]{article}%
\usepackage{geometry}%
\geometry{a4paper,
 lmargin = 2cm,rmargin = 2cm,tmargin = 2.5cm,bmargin = 2.5cm}

\input{../../macros.tex}

\pagestyle{fancy} %
\lhead{ECE2 \hfill Mathématiques\\
} %
\chead{\hrule} %
\rhead{} %
\lfoot{} %
\cfoot{} %
\rfoot{\thepage} %

\renewcommand{\headrulewidth}{0pt}% : Trace un trait de séparation
 % de largeur 0,4 point. Mettre 0pt
 % pour supprimer le trait.

\renewcommand{\footrulewidth}{0.4pt}% : Trace un trait de séparation
 % de largeur 0,4 point. Mettre 0pt
 % pour supprimer le trait.

\setlength{\headheight}{14pt}

\title{\bf \vspace{-2cm} EDHEC 2013} %
\author{} %
\date{} %
\begin{document}

\maketitle %
\vspace{-1.4cm}\hrule %
\thispagestyle{fancy}

\vspace*{.2cm}


% DEBUT DU DOC À MODIFIER : tout virer jusqu'au début de l'exo

%Définition et changement de valeurs de
compteurs%newcounter{cpt1}{section} compteur cpt1 remis à 0 à chaque
aumentation par stepcounter du compteur section%setcounter{cpt1}{3} on
met le compteur à 3%addtocounter{cpt1}{5} on ajoute 5 au compteur%
stepcounter{cpt1} on ajoute 1% ifthenelse{test}{alors}{sinon} (page
206) pour subordonner à une condition % whiledo{test}{commande} pour
faire une boucle (page 206 aussi) % value{cpt1} pour noter dans le
document la valeur de cpt1 
%Définition définitive d'opérateurs
mathématiques\newcommand{\ch}{\operatorname{ch}} 
\newcommand{\sh}{\operatorname{sh}}
\renewcommand{\tanh}{\operatorname{th}}
\renewcommand{\sinh}{\operatorname{sh}}
\renewcommand{\cosh}{\operatorname{ch}}
\newcommand{\argsh}{\operatorname{argsh}}
\newcommand{\argch}{\operatorname{argch}}
\newcommand{\argth}{\operatorname{argth}}
\newcommand{\Id}{\operatorname{Id}}
\newcommand{\id}{\operatorname{id}}
\renewcommand{\im}{\operatorname{Im}}
\renewcommand{\leq}{\leq}
\renewcommand{\geq}{\geq }

\newcommand{\dlim}{\lim}
\newcommand{\dsum}{\sum\limits}
\newcommand{\dprod}{\prod}
\newcommand{\lb}{\llbracket}
\newcommand{\rb}{\rrbracket}


%Définition de nouvelles couleurs : rgb(trois paramètres red green blue
entre 0 et 1); cmyk (quatre cyan magenta yellow black) entre 0 et 1;
gray (entre 0 et 1) et black, white, red, green, blue, cyan, magenta,
yellow% definecolor{0gris}{gray}{0.8} 
% Nouvelle commande pour encadrer le titre car shabox ne veut que d'une
seule ligne; ATTENTION A LA TAILLE; petite différence avec shadowbox ou
doublebox, voire fcolorbox ou colorbox (au lieu de shabox; laisser le
parbox tranquille sauf pour la taille de la boîte
\newcommand{\Tbox}[1]{\begin{center} \shabox{\parbox{0.5
\linewidth}{#1}} \end{center}} %[1] pour 1 paramètre ; #1 pour ce que
fait le 1er paramètre; entre accolades ce que fait la commande
%Mise en page en mode fancy : en-têtes et pieds de pages puis
définition des en-têtes et pieds de pages\pagestyle{fancy}
\lhead{ECE 2 - Mathématiques \\
Quentin Dunstetter - ENC-Bessières 2011$\backslash$2012}
\chead{}
\rhead{Edhec 2013}
\rfoot[ \ \thepage]{\thepage}
\cfoot{}
\lfoot{}
\thispagestyle{fancy} %Mise en page de la 1ère page en mode fancy
%Trait en bas et en haut de la page (entre en-tête et texte et texte et
pied de page)\renewcommand{\footrulewidth}{0.4pt}
\renewcommand{\headrulewidth}{0.4pt}

\indent \vspace{0.3cm}

\Tbox{\begin{center} \textbf{\Huge Edhec 2013} \end{center} }

\vspace{0.5cm}

\section*{Exercice 1}

\noindent On se propose d'étudier la suite $(u_{n})_{n \in \N}$,
définie par la donnée de $u_{0} = 0$ et par la relation, valable pour
tout entier naturel $n$ : $u_{n + 1} = \frac{u_{n}{2} + 1}{2}$.

\begin{noliste}{1.}
 \setlength{\itemsep}{4mm}

\item \begin{noliste}{a)}
 \setlength{\itemsep}{2mm}

\item Montrer que, pour tout entier naturel $n$, on a : $0 \leq u_{n}
\leq 1$. 

\item Étudier les variations de la suite $(u_{n})$. 

\item Déduire des questions précédentes que la suite $(u_{n})$ converge
et donner sa limite. \\

\end{noliste}

\item \begin{noliste}{a)}
 \setlength{\itemsep}{2mm}

\item Écrire une fonction \Scilab{} qui renvoie la valeur de $u_{n}$.

\item En déduire un programme,rédigé en \Scilab{}, qui permet de
déterminer et d'afficher la plus petite valeur de $n$ pour laquelle on
a : $0 < 1 - u_{n} < 10^{-3}$. \\

\end{noliste}

\item Pour tout entier naturel $n$, on pose $v_{n} = 1 - u_{n}$.

\begin{noliste}{a)}
 \setlength{\itemsep}{2mm}

\item Pour tout entier naturel $k$, exprimer $v_{k} - v_{k + 1}$ en
fonction de $v_{k}$.

\item Simplifier, pour tout entier naturel $n$ non nul, la somme
$\Sum{k = 0}{n-1} (v_{k} - v_{k + 1})$.

\item Donner pour finir la nature de la série de terme général
$v_{n}{2}$ ainsi que la valeur de $\Sum{n = 0}{+ \infty} v_{n}{2}$.

\end{noliste}

\end{noliste}

\section*{Exercice 2}

\begin{noliste}{1.}
 \setlength{\itemsep}{4mm}

\item On note $\mathcal{B} = (e_{1}, e_{2}, e_{3})$ la base canonique
de $\R^{3}$ et on considère l'endomorphisme $f$ de $\R^{3}$ dont la
matrice dans la base $\mathcal{B}$ est : 
\[
 A = \begin{smatrix}
2 & 1 & 2 \\
-1 & -1 & -1 \\
-1 & 0 & -1 \\
\end{smatrix}
\]

\begin{noliste}{a)}
 \setlength{\itemsep}{2mm}

\item Vérifier que l'on a $A^{2} \neq 0$ et calculer $A^{3}$. 

\item Déterminer une base $(a)$ de $\kerf$ ainsi qu'une base $(b,c)$ de
$\imf$.

\item Montrer que $\imf^{2} = \kerf$. \\

\end{noliste}

\end{noliste}

\noindent Dans la suite, on considère un endomorphisme $g$ de $\R^{3}$
tel que : $g^{2} \neq 0$ et $g^{3} = 0$, ce qui signifie que $g \circ
g$ n'est pas l'endomorphisme nul, mais que $g \circ g \circ g $ est
l'endomorphisme nul. \\
En désignant par $M$ la matrice de $g$ dans la base canonique
$\mathcal{B}$ de $\R^{3}$, on a donc : 
\[
 M^{2} \neq 0 \ \text{ et } \ M^{3} = 0 
\]

On se propose de montrer, dans ce cas plus général, que $\img^{2} =
\kerg$. \\

\begin{noliste}{1.}
 \setlength{\itemsep}{4mm}

\item \begin{noliste}{a)}
 \setlength{\itemsep}{2mm}

\item Montrer que 0 est la seule valeur propre possible de $g$. 

\item Montrer, en raisonnant par l'absurde, que 0 est effectivement la
seule valeur propre de $g$.

\item En déduire, toujours en raisonnant par l'absurde, que $g$ n'est
pas diagonalisable. \\

\end{noliste}

\item \begin{noliste}{a)}
 \setlength{\itemsep}{2mm}

\item Justifier qu'il existe un vecteur $u$ de $\R^{3}$ tel que
$g^{2}(u) \neq 0$.

\item Montrer que $( \ u, g(u), g^{2}(u) \ )$ est une base de $\R^{3}$,
que l'on notera $\mathcal{B}'$.

\item Donner la matrice $N$ de $g$ dans la base $\mathcal{B}'$.

\item Déterminer $\img$ et donner sa dimension. En déduire une base de
$\kerg$. Pour finir, déterminer $\img^{2}$ puis conclure.

\end{noliste}

\end{noliste}

\section*{Exercice 3}

\noindent Dans cet exercice, la lettre $n$ désigne un entier naturel.
\\
On dispose d'une urne contenant au départ $n$ boules blanches et $(n +
2)$ boules noires. On dispose également d'une réserve infinie de boules
blanches et de boules noires. \\
Pour tout entier naturel $j$, on dit que l'urne est dans l'état $j$
lorsqu'elle contient $j$ boules blanches et $(j + 2)$ boules noires. Au
départ, l'urne est donc dans l'état $n$. \\

\noindent On réalise une succession d'épreuves, chaque épreuve se
déroulant selon le protocole suivant : \\
Pour tout entier naturel $j$ non nul, si l'urne est dans l'état $j$, on
extrait une boule au hasard de l'urne.
\begin{noliste}{$\sbullet$}

\item Si l'on obtient une boule blanche, alors cette boule n'est pas
remise dans l'urne et on enlève de plus une boule noire de l'urne,
l'urne est alors dans l'état $(j-1)$. 

\item Si l'on obtient une boule noire, alors cette boule est remise
dans l'urne et on remet en plus une boule blanche et une boule noire
dans l'urne, l'urne est alors dans l'état $(j + 1)$. \\

\end{noliste}

\begin{noliste}{1.}
 \setlength{\itemsep}{4mm}

\item Dans cette question, on suppose que $n = 1$ (l'urne contient donc
une boule blanche et 3 boules noires) et on note $X_{1}$ la variable
aléatoire égale au nombre de boules blanches encore présentes dans
l'urne après la première épreuve et $X_{2}$ la variable aléatoire égale
au nombre de boules blanches encore présentes dans l'urne après la
deuxième épreuve. \\
On admet que $X_{1}$ et $X_{2}$ sont définies sur un certain espace
probabilisé $(\Omega, \mathcal{A}, \Pr)$ que l'on ne cherchera pas à
détermine.

\begin{noliste}{a)}
 \setlength{\itemsep}{2mm}

\item Donner la loi de $X_{1}$.

\item Utiliser la formule des probabilités totales pour déterminer la
loi de $X_{2}$.

\item Simulation informatique de l'expérience aléatoire décrite
ci-dessus. \\

On rappelle que random($n$) renvoie au hasard un entier compris entre 0
et $n-1$. \\
Compléter le programme suivant pour qu'il simule l'expérience aléatoire
décrite dans cet exercice et pour qu'il affiche les valeurs des
variables aléatoires $X_{1}$ et $X_{2}$. \\

\begin{verbatim}
Program simul ;
Var X1, X2, tirage : integer ;
Begin
Randomize;
tirage : = random(4); If tirage = 0 then X1 : =..........else X1 :
=..........;
If (X1 = 0) then X2 : =..........
 Else begin tirage : = random(6);
 If tirage< = 1 then X2 : =..........else X2 : =..........;
 end;
Writeln(X1,X2);
end.

\end{verbatim}

\end{noliste}

\end{noliste}

\noindent On revient au cas général ($n$ est donc un entier naturel
quelconque supérieur ou égal à 1) et on décide que les tirages
s'arrêtent dès que l'urne ne contient plus de boules blanches. \\
Pour tout $j$ de $\N$, on note alors $E_{j}$ l'évènement : \og l'urne
est dans l'état $j$ initialement et les tirages s'arrêtent au bout d'un
temps fini \fg. On pose $e_{j} = \Prob\left(\Ev{E_{j}}\right)$ et l'on
a bien sûr $e_{0} = 1$. \\

\begin{noliste}{1.}
 \setlength{\itemsep}{4mm}

\item Montrer, en considérant les deux résultats possibles du premier
tirage (c'est-à-dire au début du jeu lorsque l'urne est dans l'état
$n$) que : 
\[
 \forall n \in \N^{\ast}, \ \ \ e_{n} = \frac{n}{2n + 2} e_{n-1} +
\frac{n + 2}{2n + 2} e_{n + 1}. 
\]

\item \begin{noliste}{a)}
 \setlength{\itemsep}{2mm}

\item Montrer par récurrence que : $\forall n \in \N, \ e_{n} \geq e_{n
+ 1}$.

\item En déduire que la suite $(e_{n})$ est convergente.

\end{noliste}

\begin{center} On admet pour la suite que $\dlim{n \rightarrow +
\infty} e_{n} = 0$. \end{center}

\item Pour tout entier naturel $n$, on pose $u_{n} = (n + 1) e_{n}$.

\begin{noliste}{a)}
 \setlength{\itemsep}{2mm}

\item Pour tout entier naturel $n$ de $\N^{\ast}$, écrire $u_{n + 1}$
en fonction de $u_{n}$ et $u_{n-1}$.

\item En déduire l'expression de $u_{n}$ en fonction de $n$ et $e_{1}$.

\item Montrer enfin que l'on a : $\forall n \in \N, \ e_{n} = (2e_{1} -
1) \frac{n}{n + 1} + \frac{1}{n + 1}$. 

\end{noliste}

Déterminer la valeur de $e_{1}$, puis en déduire, pour tout entier
naturel $n$, l'expression de $e_{n}$ en fonction de $n$.

\end{noliste}

\section*{Problème}

\begin{noliste}{1.}
 \setlength{\itemsep}{4mm}

\item On considère la fonction $f$ définie pour tout $x$ réel par :
$f(x) = \left\{
\begin{array}{cl}
 1 - | x | \text{ si } x \in [-1 ; 1] \\
0 \text{ sinon} \\
\end{array}
\right.$.

\begin{noliste}{a)}
 \setlength{\itemsep}{2mm}

\item Calculer $\dint{0}{1} f(x)\ dx$. En déduire sans calcul
$\dint{-1}{0} f(x)\ dx$.

\item Vérifier que $f$ peut être considérée comme une densité. \\

\end{noliste}

\end{noliste}

\noindent On considère dorénavant une variable aléatoire $X$, définie
sur un espace probabilisé $(\Omega, \mathcal{A}, \Pr)$, et admettant
$f$ comme densité

\begin{noliste}{1.}
 \setlength{\itemsep}{4mm}

\item \begin{noliste}{a)}
 \setlength{\itemsep}{2mm}

\item Établir l'existence de l'espérance de $X$, puis donner sa valeur.

\item Établir l'existence de la variance de $X$, puis donner sa valeur.
\\

\end{noliste}

\item Montrer que la fonction de répartition de $X$, notée $F_{X}$, est
définie par : 
\[
F_{X}(x) = \left\{ 
\begin{array}{l}
 0 \ \text{ si } \ x < -1 \\
\dfrac{1}{2} + x + \dfrac{x^{2}}{2} \ \text{ si } -1 \leq x \leq 0 \\
\dfrac{1}{2} + x -\dfrac{x^{2}}{2} \ \text{ si } 0 < x \leq 1
\rule{0cm}{0.6cm} \\
1 \ \text{ si } x > 1 \rule{0cm}{0.4cm} \\
\end{array}
\right. 
\]

\end{noliste}

\noindent On pose $Y = \left| X \right|$ et on admet que $Y$ est une
variable aléatoire à densité, elle aussi définie sur l'espace
probabilisé $(\Omega, \mathcal{A}, \Pr)$. On note $F_{Y}$ sa fonction
de répartition. \\

\begin{noliste}{1.}
 \setlength{\itemsep}{4mm}

\item \begin{noliste}{a)}
 \setlength{\itemsep}{2mm}

\item Donner la valeur de $F_{Y}(x)$ lorsque $x$ est strictement
négatif.

\item Pour tout réel $x$ positif ou nul, exprimer $F_{Y}(x)$ à l'aide
de la fonction $F_{X}$.

\item En déduire qu'une densité de $Y$ est la fonction $g$ définie par
: 
\[
g(x) = \left\{
\begin{array}{cl}
 2(1-x) \text{ si } x \in [ 0 ; 1] \\
\\0 \text{ sinon}
\end{array}
\right. 
\]

\item Montrer que $Y$ possède une espérance et une variance et les
déterminer. \\

\end{noliste}

\item On considère deux variables aléatoires $U$ et $V$, elles aussi
définies sur $(\Omega, \mathcal{A}, \Pr)$, indépendantes et suivant
toutes les deux la loi uniforme sur $[0 ; 1]$. \\
On pose $I = \Inf ( U,V)$, c'est-à-dire que, pour tout $\omega$ de
$\Omega$, on a $I(\omega) = \operatorname{Min}( \ U(\omega), V(\omega)
\ )$. \\
On admet que $I$ est une variable aléatoire à densité, elle aussi
définie sur $(\Omega, \mathcal{A}, \Pr)$, et on rappelle que, pour tout
réel $x$, on a $\Prob\left(\Ev{U > x}\right) = \Prob( \ [U > x] \cap [
V > x ] \ )$. \\
Pour finir, on note $F_{I}$ la fonction de répartition de $I$.

\begin{noliste}{a)}
 \setlength{\itemsep}{2mm}

\item Expliciter $F_{I}(x)$ pour tout réel $x$.

\item En déduire que $I$ suit la même loi que $Y$. \\

\end{noliste}

\item On considère plus généralement $n$ variables aléatoires $X_{1},
X_{2}, \dots, X_{n}$ ( $n \geq 2$ ), toutes définies sur $(\Omega,
\mathcal{A}, \Pr)$, indépendantes et suivant la même loi uniforme sur
$[ 0 ; 1]$. On pose $I_{n} = \Inf ( X_{1}, X_{2}, \dots, X_{n})$. \\
Déterminer la fonction de répartition de $I_{n}$ et montrer que la
suite $(I_{n})$ converge en loi vers une variable aléatoire dont on
précisera la loi. \\

\item Simulation informatique de la loi de $Y$. \\
Compléter la déclaration de fonction suivante pour qu'elle simule la
loi de $Y$. 

\begin{verbatim}
 Function y : real;
 Var u,v : real;
 Begin
 Randomize;
 u : =..........;v : =..........;
 If(u<v) then y : =..........else y : =..........;
 End;
\end{verbatim}
\end{noliste}

\end{document} 