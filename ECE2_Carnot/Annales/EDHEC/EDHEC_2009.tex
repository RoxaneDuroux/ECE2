\documentclass[11pt]{article}%
\usepackage{geometry}%
\geometry{a4paper,
 lmargin = 2cm,rmargin = 2cm,tmargin = 2.5cm,bmargin = 2.5cm}

\input{../../macros.tex}

\pagestyle{fancy} %
\lhead{ECE2 \hfill Mathématiques\\
} %
\chead{\hrule} %
\rhead{} %
\lfoot{} %
\cfoot{} %
\rfoot{\thepage} %

\renewcommand{\headrulewidth}{0pt}% : Trace un trait de séparation
 % de largeur 0,4 point. Mettre 0pt
 % pour supprimer le trait.

\renewcommand{\footrulewidth}{0.4pt}% : Trace un trait de séparation
 % de largeur 0,4 point. Mettre 0pt
 % pour supprimer le trait.

\setlength{\headheight}{14pt}

\title{\bf \vspace{-2cm} EDHEC 2009} %
\author{} %
\date{} %
\begin{document}

\maketitle %
\vspace{-1.4cm}\hrule %
\thispagestyle{fancy}

\vspace*{.2cm}


% DEBUT DU DOC À MODIFIER : tout virer jusqu'au début de l'exo

%Définition et changement de valeurs de
compteurs%newcounter{cpt1}{section} compteur cpt1 remis à 0 à chaque
aumentation par stepcounter du compteur section%setcounter{cpt1}{3} on
met le compteur à 3%addtocounter{cpt1}{5} on ajoute 5 au compteur%
stepcounter{cpt1} on ajoute 1% ifthenelse{test}{alors}{sinon} (page
206) pour subordonner à une condition % whiledo{test}{commande} pour
faire une boucle (page 206 aussi) % value{cpt1} pour noter dans le
document la valeur de cpt1 
%Définition définitive d'opérateurs
mathématiques\newcommand{\ch}{\operatorname{ch}} 
\newcommand{\sh}{\operatorname{sh}}
\renewcommand{\tanh}{\operatorname{th}}
\renewcommand{\sinh}{\operatorname{sh}}
\renewcommand{\cosh}{\operatorname{ch}}
\newcommand{\argsh}{\operatorname{argsh}}
\newcommand{\argch}{\operatorname{argch}}
\newcommand{\argth}{\operatorname{argth}}
\newcommand{\ker}{\operatorname{Ker}}
\renewcommand{\im}{\operatorname{Im}}
\newcommand{\rg}{\operatorname{rg}}
\newcommand{\Id}{\operatorname{Id}}
\newcommand{\id}{\operatorname{id}}
\renewcommand{\leq}{\leq}
\renewcommand{\geq}{\geq }

%Définition de nouvelles couleurs : rgb(trois paramètres red green blue
entre 0 et 1); cmyk (quatre cyan magenta yellow black) entre 0 et 1;
gray (entre 0 et 1) et black, white, red, green, blue, cyan, magenta,
yellow% definecolor{0gris}{gray}{0.8} 
% Nouvelle commande pour encadrer le titre car shabox ne veut que d'une
seule ligne; ATTENTION A LA TAILLE; petite différence avec shadowbox ou
doublebox, voire fcolorbox ou colorbox (au lieu de shabox; laisser le
parbox tranquille sauf pour la taille de la boîte
\newcommand{\Tbox}[1]{\begin{center} \shabox{\parbox{0.6
\linewidth}{#1}} \end{center}} %[1] pour 1 paramètre ; #1 pour ce que
fait le 1er paramètre; entre accolades ce que fait la commande
%Mise en page en mode fancy : en-têtes et pieds de pages puis
définition des en-têtes et pieds de pages\pagestyle{fancy}
\lhead{ECE 2 - Mathématiques \\
Quentin Dunstetter - ENC-Bessières 2011$\backslash$2012}
\chead{}
\rhead{Edhec 2009}
\rfoot[ \ \thepage]{\thepage}
\cfoot{}
\lfoot{}
\thispagestyle{fancy} %Mise en page de la 1ère page en mode fancy
%Trait en bas et en haut de la page (entre en-tête et texte et texte et
pied de page)\renewcommand{\footrulewidth}{0.4pt}
\renewcommand{\headrulewidth}{0.4pt}


%DEBUT DU DOCUMENT\vspace*{3cm}

\begin{center}
{\LARG\E\textbf{BANQUE COMMUNE D'ÉPREUVES}}



{\large \textsc{CONCOURS D ADMISSION DE 2009}}



{\large \textbf{Concepteur : Edhec}}



\rule{2.39cm}{0.05cm}



{\Large \textbf{OPTION ÉCONOMIQUE}}



{\Large \textbf{MATHÉMATIQUES }}



{\Large Lundi 9 mai, de 14h à 18h}



\rule{2.39cm}{0.05cm}
\end{center}

\textit{La présentation, la lisibilité, l'orthographe, la qualité
de la rédaction, la clarté et la précision des raisonnements
entreront pour une part importante dans l'appréciation des copies.}

\textit{Les candidats sont invités à \textbf{encadrer} dans la mesure
du possible les résultats de leurs calculs.}

\textit{Ils ne doivent faire usage d'aucun document. L'utilisation de
toute
calculatrice et de tout matériel électronique est interdite. Seule
l'utilisation d'une règle graduée est autorisée.}

\textit{Si au cours de l'épreuve, un candidat repère ce qui lui semble
être une erreur d'énoncé, il la signalera sur sa copie et
poursuivra sa composition en expliquant les raisons des initiatives
qu'il sera
amené à prendre.}

\vspace*{3cm}

\section*{EXERCICE 1}

Dans cet exercice, on considère la fonction $f$ définie comme suit :\\
$f(0) = 1$, et pour tout $x$ non nul de $]-\infty ;1[,\quad f(x) =
\dfrac{-x}{(1-x)\ln (1-x)}$.

\begin{noliste}{1.}
 \setlength{\itemsep}{4mm}
\item Montrer que $f$ est continue sur $]-\infty ;1[$.

\item 
\begin{noliste}{a)}
 \setlength{\itemsep}{2mm}
\item Déterminer le développement limité de $\ln(1-x)$ à
l'ordre 2 lorsque $x$ est au voisinage de 0.

\item En déduire que $f$ est dérivable en 0, puis vérifier que
$f^{\prime }(0) = \dfrac{1}{2}$.
\end{noliste}

\item 
\begin{noliste}{a)}
 \setlength{\itemsep}{2mm}
\item Montrer que $f$ est dérivable sur $]-\infty;0[$ et sur $]0;1[$,
puis calculer $f^{\prime }(x)$ pour tout réel $x$ élément de
$]-\infty;0[ \ \cup ]0;1[$.

\item Déterminer le signe de la quantité $\ln(1-x) + x$ lorsque $x$
appartient à $]-\infty;1[$, puis en déduire les variations de $f$.

\item Déterminer les limites de $f$ aux bornes de son domaine de
définition, puis dresser son tableau de variation.
\end{noliste}

\item 
\begin{noliste}{a)}
 \setlength{\itemsep}{2mm}
\item Établir que, pour tout $n\in \N^*$, il existe un seul réel
de $[0;1[$, noté $u_{n}$, tel que $f(u_{n}) = n$ et donner la valeur de
$u_{1}$.

\item Montrer que la suite $(u_{n})$ converge et que
$\underset{n\rightarrow
 + \infty }{\lim }u_{n} = 1$
\end{noliste}
\end{noliste}

\section*{EXERCICE 2}

Dans cet exercice, $p$ désigne un réel de $]0;1[$ et on note $q =
1-p$.\\
On considère deux variables aléatoires $X$ et $Y$ définies sur
le même espace probabilisé $(\Omega,\mathcal{A},P)$, indépendantes et
suivant toutes deux la même loi géométrique de paramètre $p$.

\begin{noliste}{1.}
 \setlength{\itemsep}{4mm}
\item On pose $Z = \inf (X,Y)$ et on admet que $Z$ est une variable
aléatoire, elle aussi définie sur l'espace probabilisé
$(\Omega,\mathcal{A},P)$.\\
On rappelle que, pour tout entier naturel $k$, on a l'égalité : $(Z>k)
= (X>k)\cap (Y>k)$.

\begin{noliste}{a)}
 \setlength{\itemsep}{2mm}
\item Pour tout entier naturel $k$, calculer $P\left(\Ev{Z>k}\right)$.

\item Établir que, pour tout entier naturel $k$ supérieur ou égal 
à 1, on a :
\[
P\left(\Ev{Z = k}\right) =
P\left(\Ev{Z>k-1}\right)-P\left(\Ev{Z>k}\right)
\]

\item En déduire que $Z$ suit la loi géométrique de paramètre
$(1-q^{2})$.
\end{noliste}

\item On définit la variable aléatoire $T$ de la façon suivante :\\
Pour tout $\omega $ de $\Omega $ tel que $X(\omega )$ est un entier
naturel
pair, on pose $T(\omega ) = \dfrac{X(\omega )}{2}$, et, pour tout
$\omega $ de 
$\Omega $ tel que $X(\omega )$ est un entier naturel impair, on pose
$T(\omega ) = \dfrac{1 + X(\omega )}{2}$.\\
On admet que $T$ est une variable aléatoire, elle aussi définie sur
$(\Omega,\mathcal{A},P)$.

\begin{noliste}{a)}
 \setlength{\itemsep}{2mm}
\item Montrer que $T$ prend des valeurs entières non nulles.

\item Réciproquement, justifier que tout entier naturel $k$ non nul est

élément de $T(\Omega )$ et en déduire que $T(\Omega ) = \N^{\ast }$.

\item Exprimer l'événement $(T = k)$ en fonction de certains événements
$(X = i)$ puis montrer que $T$ suit la même loi que $Z$.
\end{noliste}

\item On rappelle que la fonction random renvoie de façon uniforme un
réel aléatoire élément de $[0;1[$.\\
Compléter le programme suivant pour que, d'une part, il simule les
lancers d'une pièce donnant $"pile"$ avec la probabilité $p$ et
calculer la valeur prise par la variable aléatoire $X$ égale au rang
du premier $"pile"$ obtenu lors de ces lancers ($X$ suit bien la loi
géométrique de paramètre $p$) et pour que, d'autre part, il calcule et
affiche la valeur prise par $T$, la variable aléatoire $T$ ayant été
définie dans la deuxième question.\\
\texttt{Program edhec2009;\\
Var x,t,lancer :integer;\\
Begin\\
\hspace*{1cm}Randomize; }x\texttt{ : = 0;\\
\hspace*{1cm}Repeat lancer : = random; x : =.......; until(lancer < =
p);\\
\hspace*{1cm}If(x mod 2 = 0) then.... else....;\\
\hspace*{1cm}Writeln(t);\\
End.}
\end{noliste}

\section*{EXERCICE 3}

Dans tout l'exercice $\lambda $ désigne un réel strictement positif.

\begin{noliste}{1.}
 \setlength{\itemsep}{4mm}
\item On considère la fonction $h$ définie sur $\R$ par : 
\[
h(x) = \left\{ 
\begin{array}{ll}
\lambda^{2}xe^{-\lambda x} \text{ si } x\geq 0 & \\
0 \text{ si } x<0 & 
\end{array}
\right.
\]

\begin{noliste}{a)}
 \setlength{\itemsep}{2mm}
\item En se référant éventuellement à une loi exponentielle,
montrer la convergence de l'intégrale $\int \limits_{0}{+
\infty}h(x)dx$
puis donner sa valeur.

\item Montrer que $h$ peut être considérée comme la densité
d'une variable aléatoire $X$.

\item Montrer la convergence de l'intégrale $\int
\limits_{0}{+ \infty}xh(x)dx$ puis donner sa valeur. En déduire que $X$
possède une espérance et la déterminer.
\end{noliste}

\item Dans cette question, on considère une variable aléatoire $Y$
de densité $f$, nulle sur $]-\infty ;0[$, continue sur $[0; + \infty
\lbrack $ et strictement positive sur $[0; + \infty \lbrack $. On note
alors $F $ la fonction de répartition de $Y$.\\
Justifier que, pour tout réel $x$, on a : $1-F(x)>0$.\\
On définit alors la fonction $g$ par : 
\[
g(x) = \left\{ 
\begin{array}{ll}
-f(x)\ln (1-F(x))\text{ } & \text{si }x\geq 0 \\
0 & \text{si }x<0
\end{array}
\right.
\]

\item 
\begin{noliste}{a)}
 \setlength{\itemsep}{2mm}
\item Montrer que $g$ est positive sur $\R$.

\item Montrer que $g$ est continue sur $]-\infty ;0[$ et sur $[0; +
\infty
\lbrack $.

\item En remarquant que, si l'on pose $u^{\prime }(x) = -f(x)$, on peut
choisir $u(x) = 1-F(x)$, montrer gr\^{a}ce à une intégration par
parties que $\int \limits_{0}{+ \infty }g(x)dx$ est une intégrale
convergente et que $\int \limits_{0}{+ \infty }g(x)dx = 1$.

\item Établir que $g$ peut être considérée comme la densité
d'une variable aléatoire $Z$.

\item Étude d'un cas particulier.\\
Vérifier que la variable aléatoire $Y$ suivant la loi exponetielle
de paramètre $\lambda $ (avec $\lambda >0$) vérifie les conditions
imposées dans la deuxième question. Montrer alors que $Z$ suit la même
loi que $X$.\\
\end{noliste}
\end{noliste}

\section*{PROBLEME}

Les parties 1 et 2 sont indépendantes.

\subsection*{Partie 1}

On note $e_{0},e_{1},e_{2}$ les fonctions définies par : 
\[
\forall t\in \R,\quad e_{0}(t) = 1,\quad e_{1}(t) = t,\quad
e_{2}(t) = t^{2}
\]
On rappelle que la famille $(e_{0},e_{1},e_{2})$ est une base de
l'espace
vectoriel $E$ constitué des fonctions polynomiales de degré inférieur
ou égal à 2.\\
\\
On considère l'application $f$ qui, à tout élément $P$ de $E$
associe $f(P) = P^{\prime \prime }-5P^{\prime } + 6P$, où $P^{\prime }$
et $P^{\prime \prime }$ désignent respectivement les dérivées première
et seconde de $P$.

\begin{noliste}{1.}
 \setlength{\itemsep}{4mm}
\item Montrer que $f$ est un endomorphisme de $E$.

\item Écrire la matrice $A$ de $f$ relativement à la base $(e_{0},
e_{1},
e_{2})$.

\item 
\begin{noliste}{a)}
 \setlength{\itemsep}{2mm}
\item Établir que $f$ est un automorphisme de $E$. En déduire Ker($f$).

\item Écrire la matrice de $f^{-1}$ relativement à la base $(e_{0},
e_{1},
e_{2})$.
\end{noliste}

\item 
\begin{noliste}{a)}
 \setlength{\itemsep}{2mm}
\item Déterminer la seule valeur propre $\lambda$ de $f$.
L'endomorphisme $f$ est-il diagonalisable ?

\item Préciser le sous-espace propre associé à la valeur propre
$\lambda $.
\end{noliste}
\end{noliste}

\subsection*{Partie 2}

On note $F$ l'espace vectoriel des fonctions de classe $C^{\infty }$
sur $\R$ et $Id$ l'endomorphisme identité de $F$.\\
On considère l'application $g$ qui, à toute fonction $u$ de $F$,
associe $g(u) = u^{\prime \prime }-5u^{\prime } + 6u$, où $u^{\prime }$
et $u^{\prime \prime }$ désignent respectivement les dérivées première
et seconde de $u$.

\begin{noliste}{1.}
 \setlength{\itemsep}{4mm}
\item Montrer que $g$ est un endomorphisme de $F$.

\item Dans cette question, on se propose de déterminer $\ker \left(
g-6Id\right) $. On considère donc une fonction $u$ élément de $\ker
\left( g-6Id\right) $.

\begin{noliste}{a)}
 \setlength{\itemsep}{2mm}
\item Montrer que la fonction $j$, définie par tout réel $x$ par $j(x)
= u^{\prime }\left( x\right) e^{-5x}$ est constante.

\item En déduire que $\ker \left( g-6Id\right) = \mathrm{Vect}\left(
u_{1},u_{2}\right) $, où $u_{1}$ est la fonction constante égale 
à 1 et $u_{2}$ la fonction définie pour tout réel $x$ par $u_{2}(x) =
e^{5x}$.\\
\end{noliste}

On se propose, dans les trois questions suivantes de déterminer $\ker
\left( g\right) $. On considère donc $u$ une fonction de $\ker \left(
g\right) $.

\item On pose $v = u^{\prime }-2u$.

\begin{noliste}{a)}
 \setlength{\itemsep}{2mm}
\item Montrer que $v^{\prime } = 3v$.

\item En déduire que la fonction $h$, définie pour tout réel $x$
par $h(x) = v(x)e^{-3x}$, est constante.

\item Conclure qu'il existe un réel $\alpha $ tel que : $\forall x\in 
\R,\quad v(x) = \alpha e^{3x}$.
\end{noliste}

\item On pose $w = u^{\prime }-3u$.

\begin{noliste}{a)}
 \setlength{\itemsep}{2mm}
\item Montrer que $w^{\prime } = 2w$.

\item En déduire que la fonction $k$, définie pour tout réel $x$
par $k(x) = w(x)e^{-2x}$, est constante.

\item Conclure qu'il existe un réel $\beta $ tel que : $\forall x\in 
\R,\quad w(x) = \beta e^{2x}$.
\end{noliste}

\item 
\begin{noliste}{a)}
 \setlength{\itemsep}{2mm}
\item Montrer, en utilisant les deux questions précédentes, que $\ker
\left( g\right) = \mathrm{Vect}\left( u_{3},u_{4}\right) $, où les
fonctions $u_{3}$ et $u_{4}$ sont définies pour tout réel $x$ par
$u_{3}(x) = e^{3x}$ et $u_{4}(x) = e^{2x}$.

\item Montrer enfin que dim(Ker($g$)) = 2.
\end{noliste}
\end{noliste}

\label{fin}



\end{document}

