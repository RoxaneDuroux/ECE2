\documentclass[11pt]{article}%
\usepackage{geometry}%
\geometry{a4paper,
 lmargin = 2cm,rmargin = 2cm,tmargin = 2.5cm,bmargin = 2.5cm}

\input{../../macros.tex}

\pagestyle{fancy} %
\lhead{ECE2 \hfill Mathématiques\\
} %
\chead{\hrule} %
\rhead{} %
\lfoot{} %
\cfoot{} %
\rfoot{\thepage} %

\renewcommand{\headrulewidth}{0pt}% : Trace un trait de séparation
 % de largeur 0,4 point. Mettre 0pt
 % pour supprimer le trait.

\renewcommand{\footrulewidth}{0.4pt}% : Trace un trait de séparation
 % de largeur 0,4 point. Mettre 0pt
 % pour supprimer le trait.

\setlength{\headheight}{14pt}

\title{\bf \vspace{-2cm} EDHEC 1997} %
\author{} %
\date{} %
\begin{document}

\maketitle %
\vspace{-1.4cm}\hrule %
\thispagestyle{fancy}

\vspace*{.2cm}


% DEBUT DU DOC À MODIFIER : tout virer jusqu'au début de l'exo

%Définition et changement de valeurs de
compteurs%newcounter{cpt1}{section} compteur cpt1 remis à 0 à chaque
aumentation par stepcounter du compteur section%setcounter{cpt1}{3} on
met le compteur à 3%addtocounter{cpt1}{5} on ajoute 5 au compteur%
stepcounter{cpt1} on ajoute 1% ifthenelse{test}{alors}{sinon} (page
206) pour subordonner à une condition % whiledo{test}{commande} pour
faire une boucle (page 206 aussi) % value{cpt1} pour noter dans le
document la valeur de cpt1 
%Définition définitive d'opérateurs
mathématiques\newcommand{\ch}{\operatorname{ch}} 
\newcommand{\sh}{\operatorname{sh}}
\renewcommand{\tanh}{\operatorname{th}}
\renewcommand{\sinh}{\operatorname{sh}}
\renewcommand{\cosh}{\operatorname{ch}}
\newcommand{\argsh}{\operatorname{argsh}}
\newcommand{\argch}{\operatorname{argch}}
\newcommand{\argth}{\operatorname{argth}}
\newcommand{\ker}{\operatorname{Ker}}
\renewcommand{\im}{\operatorname{Im}}
\newcommand{\rg}{\operatorname{rg}}
\newcommand{\Id}{\operatorname{Id}}
\newcommand{\id}{\operatorname{id}}
\renewcommand{\leq}{\leq}
\renewcommand{\geq}{\geq }

%Définition de nouvelles couleurs : rgb(trois paramètres red green blue
entre 0 et 1); cmyk (quatre cyan magenta yellow black) entre 0 et 1;
gray (entre 0 et 1) et black, white, red, green, blue, cyan, magenta,
yellow% definecolor{0gris}{gray}{0.8} 
% Nouvelle commande pour encadrer le titre car shabox ne veut que d'une
seule ligne; ATTENTION A LA TAILLE; petite différence avec shadowbox ou
doublebox, voire fcolorbox ou colorbox (au lieu de shabox; laisser le
parbox tranquille sauf pour la taille de la boîte
\newcommand{\Tbox}[1]{\begin{center} \shabox{\parbox{0.6
\linewidth}{#1}} \end{center}} %[1] pour 1 paramètre ; #1 pour ce que
fait le 1er paramètre; entre accolades ce que fait la commande
%Mise en page en mode fancy : en-têtes et pieds de pages puis
définition des en-têtes et pieds de pages\pagestyle{fancy}
\lhead{ECE 2 - Mathématiques \\
Quentin Dunstetter - ENC-Bessières 2011$\backslash$2012}
\chead{}
\rhead{Edhec 1997}
\rfoot[ \ \thepage]{\thepage}
\cfoot{}
\lfoot{}
\thispagestyle{fancy} %Mise en page de la 1ère page en mode fancy
%Trait en bas et en haut de la page (entre en-tête et texte et texte et
pied de page)\renewcommand{\footrulewidth}{0.4pt}
\renewcommand{\headrulewidth}{0.4pt}


%DEBUT DU DOCUMENT\vspace*{3cm}

\begin{center}
{\LARG\E\textbf{BANQUE COMMUNE D'ÉPREUVES}}



{\large \textsc{CONCOURS D ADMISSION DE 1997}}



{\large \textbf{Concepteur : Edhec}}



\rule{2.39cm}{0.05cm}



{\Large \textbf{OPTION ÉCONOMIQUE}}



{\Large \textbf{MATHÉMATIQUES }}



{\Large Lundi 9 mai, de 14h à 18h}



\rule{2.39cm}{0.05cm}
\end{center}

\textit{La présentation, la lisibilité, l'orthographe, la qualité
de la rédaction, la clarté et la précision des raisonnements
entreront pour une part importante dans l'appréciation des copies.}

\textit{Les candidats sont invités à \textbf{encadrer} dans la mesure
du possible les résultats de leurs calculs.}

\textit{Ils ne doivent faire usage d'aucun document. L'utilisation de
toute
calculatrice et de tout matériel électronique est interdite. Seule
l'utilisation d'une règle graduée est autorisée.}

\textit{Si au cours de l'épreuve, un candidat repère ce qui lui semble
être une erreur d'énoncé, il la signalera sur sa copie et
poursuivra sa composition en expliquant les raisons des initiatives
qu'il sera
amené à prendre.}

\vspace*{3cm}

{\Large Exercice 1}

Pour tout entier naturel $n$ non nul, on note $f_{n}$ la fonction
définie par : $\forall x\in \R_{+}{*},\;f_{n}(x) = x-n.\ln (x)$.

\begin{noliste}{1.}
 \setlength{\itemsep}{4mm}
\item 
\begin{noliste}{a)}
 \setlength{\itemsep}{2mm}
\item Étudier cette fonction et dresser son tableau de variations.

\item En déduire, lorsque $n$ est supérieur ou égal à $3$,
l'existence de deux réels $u_{n}$et $v_{n}$solutions de l'équation
$f_{n}(x) = 0$ et vérifiants $0<u_{n}<n<v_{n}$.
\end{noliste}

\item Étude de la suite $(u_{n})_{n\geq 3}$.

\begin{noliste}{a)}
 \setlength{\itemsep}{2mm}
\item Montrer que $\forall n\geq 3,1<u_{n}<e$.

\item Montrer que $f_{n}(u_{n + 1}) = \ln (u_{n + 1})$, puis en
conclure que $(u_{n})$ est décroissante.

\item En déduire que $(u_{n})_{n\geq 3}$ converge et montrer, en
encadrant $\ln (u_{n}),$ que $\dlim{n\rightarrow + \infty }u_{n} = 1$.

\item Montrer que $\dlim{n\rightarrow + \infty }{\frac{\ln
(u_{n})}{u_{n}-1}} = 1$ ; en déduire que $\thicksim $ ${\frac{1}{n}}$.
\end{noliste}

\item Étude de la suite $\mathbf{(}v_{n}\mathbf{)}_{n\geq 3}$

\begin{noliste}{a)}
 \setlength{\itemsep}{2mm}
\item Calculer $\dlim{n\rightarrow + \infty }}v_{n}$.

\item Calculer $f_{n}(n\cdot \ln (n))$ puis montrer que $\forall n\geq
3$, $n.\ln (n)<v_{n}.$

\item Soit $g$ la fonction définie par : $\forall x\in \R^{*},g(x) =
x-2\ln (x)$.

Étudier $g$ et donner son signe. En déduire que $\forall n\in
\N^{*},n>2\ln (n)$.

\item En déduire le signe de $f_{n}(2n.\ln (n))$, puis établir que : 
$n\ln \left( n\right) <v_{n}<2n.\ln (n)$

\item Montrer enfin que : $\ln (v_{n})\underset{n\rightarrow + \infty
}{\thicksim }n\cdot \ln (n)$
\end{noliste}
\end{noliste}

{\Large Exercice 2}

Pour tout entier naturel $n$ non nul, on pose $u_{n} = 1/\binom{n +
p}{n},$où
$p$ désigne un entier naturel fixé.

\begin{noliste}{1.}
 \setlength{\itemsep}{4mm}
\item {Montrer que si $p = 0$ ou si $p = 1$ la série de terme général
$u_{n}$diverge. }

\hspace{-1cm}{On suppose dans toute la suite que $p$ est supérieur ou 
égal à $2$ et on pose $S_{n} = \Sum{k = 1}{n}u_{k}$ }

\item 
\begin{noliste}{a)}
 \setlength{\itemsep}{2mm}
\item {Montrer que $\forall n\in \N,\left( n + p + 2\right)
u_{n + 2} = \left( n + 2\right) u_{n + 1}$. }

\item {En déduire par récurrence sur $n$ que }$ =
{\frac{1}{p-1}}$}$\left( {1-\left( n + p + 1\right) u_{n + 1}}\right)
${\ 
}
\end{noliste}

\item 
\begin{noliste}{a)}
 \setlength{\itemsep}{2mm}
\item {On pose $v_{n} = (n + p)u_{n}$. Montrer que la suite $(v_{n})$
est décroissante. }

\item {En déduire que la suite $(v_{n})$ converge et que sa limite
$\ell 
$ est positive ou nulle. }

\item {Utiliser le résultat pprécédent pour montrer que la série de
terme général $u_{n}$converge et donner sa somme en fonction
de $p$ et de $\ell $. }
\end{noliste}

\item On suppose dans cette question seulement que $\ell \neq 0$.

\begin{noliste}{a)}
 \setlength{\itemsep}{2mm}
\item {Montrer qu'au voisinage de }$ + \infty,\;${$u_{n}\thicksim 
\frac{\ell }{n}$}

\item {En déduire une contradiction avec la troisième question. }
\end{noliste}

\item {Donner la valeur de $\ell $ et en déduire en fonction de $p,$la
somme de la série de terme général $u_{n}$ }
\end{noliste}

{\Large Exercice 3}

$I$ désigne la matrice unité de $\M{3} $ et $M$ une matrice de $\M{3} $
pour laquelle il existe un entier $p$ supérieur ou égal à 2 tel
que $M^{p} = 0$ et $M^{p-1}\ne 0$

On définit alors les matrices suivantes :

\begin{noliste}{$\sbullet$}
\item $\exp \left( M\right) = \Sum{k = 0}{p-1}\frac{1}{k!}M^{k},$ avec
la convention 
$M^{0} = I$

\item $\ln \left( I + M\right) = \Sum{k = 1}{p-1}\frac{\left( -1\right)
^{k + 1}}{k}M^{k} $
\end{noliste}

On considère l'enemble $E$ des matrices triangulaires supérieures de 
$\M{3} $ dont tous les éléments
diagonaux sont nuls.

\begin{noliste}{1.}
 \setlength{\itemsep}{4mm}
\item 
\begin{noliste}{a)}
 \setlength{\itemsep}{2mm}
\item Montrer que $E$ est un espace vatoriel de sur $\R$ et donner
sa dimension.

\item Montrer que les matrices de $E$ ne sont pas inversibles.

\item Montrer que toute matrice non nulle de $E$ n'est pas
diagonalisable.
\end{noliste}

Dans la suite, $A$ désigne une matrice quelconque de $E.$

\item 
\begin{noliste}{a)}
 \setlength{\itemsep}{2mm}
\item calculer $A^{k}$\hspace{-1cm} pour tout $k$ de $\N$.

\item Exprimer$\exp \left( A\right) $ et $\ln \left( I + A\right) $ en
fonction de $I$ et de $A.$
\end{noliste}

\item Montrer que $\ln \left[ \ \exp \left( A\right) \right] = A$

\item 
\begin{noliste}{a)}
 \setlength{\itemsep}{2mm}
\item Vérifier que $\ln \left( I + A\right) $ appartient à $E.$

\item Montrer que $\exp \left[ \ \ln \left( I + A\right) \right] = I +
A$
\end{noliste}

\item Montrer que $\forall m\in \N,\,\exp \left( m\,A\right) = \left[
\exp \left( A\right) \right] ^{m}$

\item Montrer que $\exp \left( A\right) $ est inversible et que sont
inverse
est : $\left[ \ \exp \left( A\right) \right] ^{-1} = \exp \left(
-A\right) $

\item Quelle condition doivent vérifier deux matrices $A$ et $B$ de $E$
pour que $\exp \left( A + B\right) = \exp \left( A\right) \,\exp \left(
B\right) $\ ?
\end{noliste}

{\Large Problème}

Dans ce problème $n$ désigne un entier supérieur ou égal 
à $2$.

\textbf{Partie I}

On effectue $2n$ tirages au hasard dans une urne contenant $n$ boules
numérotées de $1$ à $n$.

Un tirage consiste à extraire une boule de l'urne, la boule tirée 
étant remise dans l'urne.

On note $N$ la variable aléatoire égale au numéro du tirage au
cours duquel, pour la première fois, on a obtenu une boule déjà
obtenue auparavant.

\begin{noliste}{1.}
 \setlength{\itemsep}{4mm}
\item {Montrer que $N(\Omega ) = [[2,n + 1]]$. }

\item {Montrer que $\forall k\in [[1,n]],\;p\left( N>k\right) =
\frac{A_{n}{k}}{n^{k}}$ }

\hspace{-1cm}Rappel : $A_{n}{k}$ désigne le nombre d'arrangements de
$k$
éléments d'un ensemble à $n$ éléments.

\item 
\begin{noliste}{a)}
 \setlength{\itemsep}{2mm}
\item {\ Montrer que : }$\forall k${$\in [[2,n]],\;p(N = k) =
p(N>k-1)-p(N>k)$. }

\item {Calculer $p(N = n + 1)$ puis en déduire la loi de }$N{.}$
\end{noliste}

\item Montrer que l'espérance $\E(N)$ de la variable {aléatoire} $N$
est : $\left( N\right) = \Sum{k = 0}{n}\frac{A_{n}{k}}{n^{k}}$
\end{noliste}

\textbf{Partie II}

Soit $X$ une variable aléatoire, à valeurs dans $\R^{+},$ de
densité $f$ (nulle sur $\R_{-}{*})$ et de fonction de répartition $F.$
On suppose, de plus, $f$ continue sur $\R^{+}.$

On pose, pour tout réel $x$ positif, $\varphi \left( x\right)
 = \dint{0}{x}t\,f\left( t\right) \,dt$.

\begin{noliste}{1.}
 \setlength{\itemsep}{4mm}
\item Montrer, gràce à une intégration par parties, que : 
\[
\forall x\in \R^{+},\;\varphi \left( x\right) = \dint{0}{x}\left[
1-F\left( t\right) \right] \,dt-x\cdot P\left(\Ev{ X>x}\right)
\]

\item On suppose, dans cette question, que l'intégrale $\dint{0}{+
\infty }\left[ 1-F\left( t\right) \right] \,dt$ converge.

\begin{noliste}{a)}
 \setlength{\itemsep}{2mm}
\item Calculer $\varphi ^{\prime }\left( x\right) $ et en déduire que
la
fonction $\varphi $ est croissante sur $\R^{+}.$

\item Montrer que $\varphi $ est majorée et en déduire que $X$ a une
espérance.

\item Montrer que : $\forall x\in \R^{+},\;0\leq x\cdot P\left(\Ev{
X>x}\right) \leq \dint{x}{+ \infty }t\,f\left( t\right) \,dt$

\item En utilisant le fait que $X$ a une espérance, montrer que
$\dlim{x\rightarrow + \infty }\dint{x}{+ \infty }t\,f\left( t\right)
\,dt = 0$

En déduire $\dlim{x\rightarrow \infty }x\,P\left(\Ev{
X>x}\right),$ puis montrer que : $\E\left( X\right) = \dint{0}{+ \infty
}\left[
1-F\left( t\right) \right] \,dt$
\end{noliste}
\end{noliste}

\textbf{Partie III}

On considère la fonction$F_{n}$ définie par :$\left\{ 
\begin{array}{l}
F_{n}\left( x\right) = 0\text{ si }x<0 \\
 F_{n}\left( x\right) = 1-\left( 1 + \frac{x}{n}\right)
^{n}e^{-x}\text{ si }x\geq 0
\end{array}
\right. $

\begin{noliste}{1.}
 \setlength{\itemsep}{4mm}
\item Montrer que $F_{n}$ est la fontion de répartition d'une variable
aléatoire à densité $T_{n}.$

\item 
\begin{noliste}{a)}
 \setlength{\itemsep}{2mm}
\item Montrer que, pour tout entier naturel $k,$ l'intégrale $ =
\dint{0}{+ \infty }x^{k}e^{-x}dx$ converge.

\item Montrer que $I_{k + 1} = \left( k + 1\right) I_{k}$ puis donner
la valeur de 
$I_{k}.$
\end{noliste}

\item En déduire, en utilisant la partie II, que $T_{n}$ a une
espérance etque $\E\left( T_{n}\right) = E\left( N\right).$
\end{noliste}

\textbf{Partie VI}

On cosidère la déclaration de fonction suivante, rédigée en
\Scilab{} :

\texttt{function f(p,q :integer) :real;}

\texttt{var j :integer; z :real;}

\texttt{begin}

\texttt{\hspace{1cm}if (p< = 0) or (q<0) then write('valeurs
incorrectes')}

\texttt{\hspace{1cm}else }

\texttt{\hspace{1cm}begin}

\texttt{\hspace{2cm}z : = 1;}

\texttt{\hspace{2cm}for j : = 1 to (q-1) do z : = z*(1-j/p)}

\texttt{\hspace{2cm}f : = z;}

\texttt{\hspace{1cm}end;}

\texttt{end;}

\begin{noliste}{1.}
 \setlength{\itemsep}{4mm}
\item Montrer que si $p$ est un entier naturel non nnul et si $q$ est
un
entier naturel alors $ f\left( p,q\right) = \frac{A_{p}{q}}{p^{q}}$

\item Utiliser cette déclaration pour écrire un algorithme en
\Scilab{} donnant la valeur commune de $\E\left( N\right) $ et
$\E\left(
T_{n}\right) $ lorsque l'utilisateur entre la valeur de $n$ au clavier.
\end{noliste}

\end{document}

