\documentclass[11pt]{article}%
\usepackage{geometry}%
\geometry{a4paper,
 lmargin = 2cm,rmargin = 2cm,tmargin = 2.5cm,bmargin = 2.5cm}

\input{../../macros.tex}

\pagestyle{fancy} %
\lhead{ECE2 \hfill Mathématiques\\
} %
\chead{\hrule} %
\rhead{} %
\lfoot{} %
\cfoot{} %
\rfoot{\thepage} %

\renewcommand{\headrulewidth}{0pt}% : Trace un trait de séparation
 % de largeur 0,4 point. Mettre 0pt
 % pour supprimer le trait.

\renewcommand{\footrulewidth}{0.4pt}% : Trace un trait de séparation
 % de largeur 0,4 point. Mettre 0pt
 % pour supprimer le trait.

\setlength{\headheight}{14pt}

\title{\bf \vspace{-2cm} EDHEC 2007} %
\author{} %
\date{} %
\begin{document}

\maketitle %
\vspace{-1.4cm}\hrule %
\thispagestyle{fancy}

\vspace*{.2cm}


% DEBUT DU DOC À MODIFIER : tout virer jusqu'au début de l'exo

%Définition et changement de valeurs de
compteurs%newcounter{cpt1}{section} compteur cpt1 remis à 0 à chaque
aumentation par stepcounter du compteur section%setcounter{cpt1}{3} on
met le compteur à 3%addtocounter{cpt1}{5} on ajoute 5 au compteur%
stepcounter{cpt1} on ajoute 1% ifthenelse{test}{alors}{sinon} (page
206) pour subordonner à une condition % whiledo{test}{commande} pour
faire une boucle (page 206 aussi) % value{cpt1} pour noter dans le
document la valeur de cpt1 
%Définition définitive d'opérateurs
mathématiques\newcommand{\ch}{\operatorname{ch}} 
\newcommand{\sh}{\operatorname{sh}}
\renewcommand{\tanh}{\operatorname{th}}
\renewcommand{\sinh}{\operatorname{sh}}
\renewcommand{\cosh}{\operatorname{ch}}
\newcommand{\argsh}{\operatorname{argsh}}
\newcommand{\argch}{\operatorname{argch}}
\newcommand{\argth}{\operatorname{argth}}
\newcommand{\ker}{\operatorname{Ker}}
\renewcommand{\im}{\operatorname{Im}}
\newcommand{\rg}{\operatorname{rg}}
\newcommand{\Id}{\operatorname{Id}}
\newcommand{\id}{\operatorname{id}}
\renewcommand{\leq}{\leq}
\renewcommand{\geq}{\geq }

%Définition de nouvelles couleurs : rgb(trois paramètres red green blue
entre 0 et 1); cmyk (quatre cyan magenta yellow black) entre 0 et 1;
gray (entre 0 et 1) et black, white, red, green, blue, cyan, magenta,
yellow% definecolor{0gris}{gray}{0.8} 
% Nouvelle commande pour encadrer le titre car shabox ne veut que d'une
seule ligne; ATTENTION A LA TAILLE; petite différence avec shadowbox ou
doublebox, voire fcolorbox ou colorbox (au lieu de shabox; laisser le
parbox tranquille sauf pour la taille de la boîte
\newcommand{\Tbox}[1]{\begin{center} \shabox{\parbox{0.6
\linewidth}{#1}} \end{center}} %[1] pour 1 paramètre ; #1 pour ce que
fait le 1er paramètre; entre accolades ce que fait la commande
%Mise en page en mode fancy : en-têtes et pieds de pages puis
définition des en-têtes et pieds de pages\pagestyle{fancy}
\lhead{ECE 2 - Mathématiques \\
Quentin Dunstetter - ENC-Bessières 2011$\backslash$2012}
\chead{}
\rhead{Edhec 2007}
\rfoot[ \ \thepage]{\thepage}
\cfoot{}
\lfoot{}
\thispagestyle{fancy} %Mise en page de la 1ère page en mode fancy
%Trait en bas et en haut de la page (entre en-tête et texte et texte et
pied de page)\renewcommand{\footrulewidth}{0.4pt}
\renewcommand{\headrulewidth}{0.4pt}


%DEBUT DU DOCUMENT\vspace*{3cm}

\begin{center}
{\LARG\E\textbf{BANQUE COMMUNE D'ÉPREUVES}}



{\large \textsc{CONCOURS D ADMISSION DE 2007}}



{\large \textbf{Concepteur : Edhec}}



\rule{2.39cm}{0.05cm}



{\Large \textbf{OPTION ÉCONOMIQUE}}



{\Large \textbf{MATHÉMATIQUES }}



{\Large Lundi 9 mai, de 14h à 18h}



\rule{2.39cm}{0.05cm}
\end{center}

\textit{La présentation, la lisibilité, l'orthographe, la qualité
de la rédaction, la clarté et la précision des raisonnements
entreront pour une part importante dans l'appréciation des copies.}

\textit{Les candidats sont invités à \textbf{encadrer} dans la mesure
du possible les résultats de leurs calculs.}

\textit{Ils ne doivent faire usage d'aucun document. L'utilisation de
toute
calculatrice et de tout matériel électronique est interdite. Seule
l'utilisation d'une règle graduée est autorisée.}

\textit{Si au cours de l'épreuve, un candidat repère ce qui lui semble
être une erreur d'énoncé, il la signalera sur sa copie et
poursuivra sa composition en expliquant les raisons des initiatives
qu'il sera
amené à prendre.}

\vspace*{3cm}

\section*{Exercice 1}

Pour toute matrice $M$ élément de $\M{2},$ on note $^{t}M$ la matrice
transposée de $M,$ définie de
la façon suivante : si $M = 
\begin{smatrix}
a & b \\
c & d
\end{smatrix}
$ alors $^{t}M = 
\begin{smatrix}
a & c \\
b & d
\end{smatrix}.$

On pose $E_{1} = 
\begin{smatrix}
1 & 0 \\
0 & 0
\end{smatrix},\ E_{2} = 
\begin{smatrix}
0 & 1 \\
0 & 0
\end{smatrix},\ E_{3} = 
\begin{smatrix}
0 & 0 \\
1 & 0
\end{smatrix}
$ et $E_{4} = 
\begin{smatrix}
0 & 0 \\
0 & 1
\end{smatrix}
$

On rappelle que $\mathcal{B = }\left( E_{1},E_{2},E_{3},E_{4}\right) $
est une
base de $\M{2}.$

On note $\varphi $ l'application qui à toute matrice $M$ de $\M{2} $
associe $\varphi \left( M\right) = M + \left.
^{t}M\right..$

\begin{noliste}{1.}
 \setlength{\itemsep}{4mm}
\item 
\begin{noliste}{a)}
 \setlength{\itemsep}{2mm}
\item Montrer que $\varphi $ est un endomorphisme de $\M{2}.$

\item Écrire la matrice $A$ de $\varphi $ dans $\mathcal{B}.$

\item En déduire que $\varphi $ est diagonalisable et non bijectif.
\end{noliste}

\item calculer $A^{2}$ et en déduire que, pour tout $n$ de$\N^{\ast }
:A^{n} = 2^{n-1}A$

\item 
\begin{noliste}{a)}
 \setlength{\itemsep}{2mm}
\item Monterr que $\mathrm{Im}\varphi = \mathrm{Vect}\left( E_{1},\
E_{2} + E_{3},\ E_{4}\right),$ puis établir que $\dim \mathrm{Im}\left(
\varphi \right) = 3.$

\item En déduire la dimension de $\ker \varphi $ puis déterminer une
base de $\ker \varphi.$

\item Établir que $\mathrm{Im}\varphi $ est le sous espace propre
associé
à la valeur propre 2

\item Donner,pour résumer, les valeurs propres de $\varphi $ ainsi
qu'une base de chacun des sous-espaces propres associés.
\end{noliste}
\end{noliste}

\section*{Exercice 2}

On admet que si $Z_{1}$ et $Z_{2}$ sont deux variables aléatoires à
densité, définies sur le même espace probabilisé, alors leur
covariance, si elle existe, est définie par :
\[
\mathrm{Cov}\left( Z_{1},Z_{2}\right) = E\left( Z_{1}Z_{2}\right)
-\E\left(
Z_{1}\right) E\left( Z_{2}\right)
\]
\\
On admet également que si $Z_{1}$ et $Z_{2}$ sont indépendantes
alors leur covariance est nulle.

On considère deux variables aléatoires réelles $X$ et $U$ définies sur
le même espace probabilisé $\left( \Omega,\mathcal{A},\Prob\right),$
indépendantes, $X$ suivant la loi normale $\mathcal{N}\left( 0,1\right)
$ et $U$ suivant la loi discrète uniforme sur $\left\{ -1,1\right\}.$

On pose \ $Y = UX$ et on admet que $Y$ est une variable aléatoire à
densité, définie elle aussi sur l'espace probabilisé $\left(
\Omega,\mathcal{A},\Prob\right).$

\begin{noliste}{1.}
 \setlength{\itemsep}{4mm}
\item 
\begin{noliste}{a)}
 \setlength{\itemsep}{2mm}
\item En utilisant la formule des probabilités totales, montrer que :
\[
\Prob\left(\Ev{ Y\leq x}\right) = \mathrm{P}\left( \left[ U = 1\right]
\cap \left[ X\leq x\right] \right) + \mathrm{P}\left( \left[ U =
-1\right] \cap \left[ X\geq -x\right] \right)
\]

\item En déduire que $Y$ suit la même loi que $X.$
\end{noliste}

\item 
\begin{noliste}{a)}
 \setlength{\itemsep}{2mm}
\item Calculer l'espérance de $U$ puis montrer que $\E\left( XY\right)
= 0$

\item En déduire que $\mathrm{Cov}\left( X,Y\right) = 0$.
\end{noliste}

\item 
\begin{noliste}{a)}
 \setlength{\itemsep}{2mm}
\item Rappeler la valeur de $\E\left( X^{2}\right) $ et en déduire que
$\dint{0}{+ \infty }x^{2}e^{-\frac{x^{2}}{2}} = \frac{1}{2}\sqrt{2\pi
}$

\item Montrer, grace à une intégrationpar parties que 
\[
\forall A\in \R_{+} :\quad \dint{0}{A}x^{4}e^{-\frac{x^{2}}{2}}dx =
-A^{3}e^{-\frac{A^{2}}{2}} + 3\dint{0}{A}x^{2}e^{-\frac{x^{2}}{2}}dx
\]

\item En déduire que l'intégrale $\dint{0}{+ \infty
}x^{4}e^{-\frac{x^{2}}{2}}dx$ converge et vaut $\dfrac{3}{2}\sqrt{2\pi
}.$

\item Établir finalement que $X$ possède un moment d'ordre $4$ et que
$\E\left( X^{4}\right) = 3$
\end{noliste}

\item 
\begin{noliste}{a)}
 \setlength{\itemsep}{2mm}
\item Vérifier que $\E\left( X^{2}Y^{2}\right) = 3$

\item Déterminer $\mathrm{Cov}\left( X^{2},Y^{2}\right) $

\item En déduire que $X^{2}$ et $Y^{2}$ ne sont pas indépendantes.
Montrer alors que $X$ et $Y$ ne le sont pas non plus.

\item Cet exercice a permis de montrer qu'un résultat classique
concernant les variables discrètes est encore valable pour les
variabales à densité. Lequel ?
\end{noliste}
\end{noliste}

\section*{Exercice 3}

\begin{noliste}{1.}
 \setlength{\itemsep}{4mm}
\item 
\begin{noliste}{a)}
 \setlength{\itemsep}{2mm}
\item Montrer que pour tout $x>0 :x-\ln \left( x\right) >0$

\item On pose alors $\left\{ 
\begin{array}{cc}
f\left( x\right) = \dfrac{\ln \left( x\right) }{x-\ln \left( x\right) }
& 
\text{si }x>0 \\
f\left( 0\right) = -1 & 
\end{array}
\right. $

Déterminer l'ensemble de définition $D$ de la fonction $f.$
\end{noliste}

\item 
\begin{noliste}{a)}
 \setlength{\itemsep}{2mm}
\item Montrer que $f$ est continue sur $D.$

\item Montrer que $f$ est dérivable (à droite ) en $0$ et que
$f_{d}{\prime }\left( 0\right) = 0.$
\end{noliste}

\item 
\begin{noliste}{a)}
 \setlength{\itemsep}{2mm}
\item Justifier que $f$ est dérivable sur $D\setminus \left\{ 0\right\}
$
et calculer $f^{\prime }\left( x\right) $ pour tout $x$ de $D\setminus
\left\{ 0\right\}.$

\item Déterminer la limite de $f$ en $ + \infty.$

\item Dresser le tableau de variations de $f.$
\end{noliste}

\item Étudier le signe de $f\left( x\right).$

\item Pour tout réel $x$ élément de $D,$ on pose $F\left(
x\right) = \dint{0}{x}f\left( t\right\ dt$

\begin{noliste}{a)}
 \setlength{\itemsep}{2mm}
\item Montrer que $F$ est de classe $\mathcal{C}{1}$ sur $D$ puis
étudier ses variations.

\item Déterminer $\dlim{x\rightarrow + \infty }\dint{1}{x}\frac{\ln
t}{t}dt.$

\item En déduire $\dlim{x\rightarrow + \infty }\dint{1}{x}\frac{\ln
t}{t-\ln t}dt$ puis $\dlim{x\rightarrow + \infty }F\left( x\right)..$
\end{noliste}
\end{noliste}

\section*{Problème}

On lance une pièce équilibrée. (la probabilité d'obtenir
"pile" et celle d'obtenir "face" étant toutes deux égales à
$\frac{1}{2}$ ) et on note $Z$\ la variable aléatoire égale au rang
du lancer où l'on obtient le premier "pile".

Après cette série de lancers, si $Z$ a pris la valeur $k$ ($k\in 
\N^{\ast }$), on remplit une urne de $k$ boules numérotées
1,\ 2,\ $\cdots,\ k,$ puis on extrait au hasard une boule de cette
urne.

On note $X$ la variable aléatoire égale au numéro de la boule tirée
après la procédure décrite ci-dessus.

\begin{noliste}{1.}
 \setlength{\itemsep}{4mm}
\item On décide de coder l'événement $\ll $obtenir un "pile"$\gg 
$ par $1$ et l'événement $\ll $obtenir un "face"$\gg $\ par $0.$

On rappelle que la fonction \texttt{random }renvoie, pour un argument
$k$ de
type \texttt{integer} (où $k$ désigne un entier supérieur ou 
égal à $1$) un entier aléatoire compris entre $0$ et $k-1.$

\begin{noliste}{a)}
 \setlength{\itemsep}{2mm}
\item Compléter le programme suivant pour qu'il affiche la valeur prise
par $Z$ lors de la première partie de l'expérience décrite
ci-dessus.

\texttt{Program edhec\_{2}007;}

\texttt{Var z,hasard :integer;}

\texttt{begin}

\texttt{\hspace{1cm}randomize; z : = 0;}

\texttt{\hspace{1cm}repeat }

\texttt{\hspace{2cm}z : = \ldots \ldots \ldots ; hasard : = \ldots
\ldots \ldots ;
until (hasard = 1);}

\texttt{\hspace{1cm}writeln(z);}

\texttt{end.}

\item Quelle instruction faut-il rajouter avant la dernière ligne de ce
programme pour qu'il simule l'expérience aléatoire décrite dans
ce problème et affiche la valeur prise par la variable aléatoire $X$
 ?
\end{noliste}

\item Établir la convergence de la série de terme général
$\frac{1}{k}\left( \frac{1}{2}\right) ^{k}$ ($k\in \N^{\ast }$).

\item Rappeler la loi de $Z$ ainsi que son espérance et sa variance.

\item 
\begin{noliste}{a)}
 \setlength{\itemsep}{2mm}
\item Pour tout couple $\left( i,k\right) $ de $\N^{\ast }\times 
\N^{\ast },$ déterminer la probabilité $\mathrm{P}_{\left[
Z = k\right] }\left( X = i\right) $

\item En déduire que $\forall i\in \N^{\ast }$, $\Prob\left(\Ev{ X =
i}\right) = \Sum{k = i}{+ \infty }\frac{1}{k}\left( \frac{1}{2}\right)
^{k}.$

\item On admet dans cette question que $\Sum{i = 1}{+ \infty
}\Sum{k = i}{+ \infty } = \Sum{k = 1}{+ \infty
}\Sum{i = 1}{k}$. Vérifier que $\Sum{i = 1}{+ \infty }\Prob\left(\Ev{ X
= i}\right) = 1$
\end{noliste}

\item 
\begin{noliste}{a)}
 \setlength{\itemsep}{2mm}
\item Montrer que, pour tout entier naturel $i$ non nul, on a :
$i\Prob\left(\Ev{ X = i}\right) \leq \left( \frac{1}{2}\right) ^{i-1}$

\item En déduire que $X$ possède une espérance.

\item Montrer, en admettant qu'il est licite de permuter les symboles
$\sum $
comme dans la question 4c), que 
\[
\E\left( X\right) = \frac{3}{2}
\]
\end{noliste}

\item 
\begin{noliste}{a)}
 \setlength{\itemsep}{2mm}
\item Utiliser le résultat de la question 5a) pour montrer que $X$ a un
moment d'ordre 2.

\item Établir, alors, toujours en admettant qu'il est licite de
permuter les
symboles $\sum $ comme dans la question 4c), que
\[
\E\left( X^{2}\right) = \frac{1}{6}\Sum{k = 1}{+ \infty }\left( k +
1\right)
\left( 2k + 1\right) \left( \frac{1}{2}\right) ^{k}
\]

\item Déterminer les réels $a,b$ et $c$ tels que : $\forall k\in 
\N^{\ast }$, $\left( k + 1\right) \left( 2k + 1\right) = ak\left(
k-1\right) + bk + c$.

\item En déduire la valeur de $\E\left( X^{2}\right) $ et vérifier
que $\V\left( X\right) = \dfrac{11}{12}$.
\end{noliste}

\item 
\begin{noliste}{a)}
 \setlength{\itemsep}{2mm}
\item Écrire l'inégalité de Bienaymé-Chebychev, pour la variable 
$X$.

\item En déduire $\Prob\left(\Ev{ X\geq 3}\right) \leq \dfrac{11}{27}$
\end{noliste}

\item On se propose de calculer $\Prob\left(\Ev{ X = 1}\right),$
$\Prob\left(\Ev{ X = 2}\right) $ et $\Prob\left(\Ev{ X\geq 3}\right).$

\begin{noliste}{a)}
 \setlength{\itemsep}{2mm}
\item Écrire explicitement en fonction de $x$ et $n$ la somme $\Sum{k =
1}{n}x^{k-1}$ ($n$ désignant un entier naturel non nul
et $x$ un réel différent de 1).

\item En déduire que : $\forall n\in \N^{\ast
} :\Sum{k = 1}{n}\frac{1}{k}\left( \frac{1}{2}\right) ^{k} = \ln \left(
2\right) -\dint{0}{1/2}\frac{x^{n}}{1-x}dx$

\item Montrer que $\forall n\in \N^{\ast } :0\leq
\dint{0}{1/2}\frac{x^{n}}{1-x}dx\leq \left( \frac{1}{2}\right) ^{n}$\\
En déduire la valeur de $\dlim{n\rightarrow + \infty
}\dint{0}{1/2}\frac{x^{n}}{1-x}dx$

\item Établir alors que $\Prob\left(\Ev{ X = 1}\right) = \ln \left(
2\right) $
puis donner la valeur de $\Prob\left(\Ev{ X = 2}\right) $.

\item Utiliser les résultats précédents pour calculer $\Prob\left(\Ev{
X\geq 3}\right) $, puis donner une valeur approchée de $\Prob\left(\Ev{
X\geq 3}\right) $ en prenant $\ln 2\simeq 0,7$. Que peut-on en déduire
en ce qui concerne le majorant trouvé à la septième
question ?
\end{noliste}
\end{noliste}

\end{document}

