\documentclass[11pt]{article}%
\usepackage{geometry}%
\geometry{a4paper,
 lmargin = 2cm,rmargin = 2cm,tmargin = 2.5cm,bmargin = 2.5cm}

\input{../../macros.tex}

\pagestyle{fancy} %
\lhead{ECE2 \hfill Mathématiques\\
} %
\chead{\hrule} %
\rhead{} %
\lfoot{} %
\cfoot{} %
\rfoot{\thepage} %

\renewcommand{\headrulewidth}{0pt}% : Trace un trait de séparation
 % de largeur 0,4 point. Mettre 0pt
 % pour supprimer le trait.

\renewcommand{\footrulewidth}{0.4pt}% : Trace un trait de séparation
 % de largeur 0,4 point. Mettre 0pt
 % pour supprimer le trait.

\setlength{\headheight}{14pt}

\title{\bf \vspace{-2cm} EDHEC 2014} %
\author{} %
\date{} %
\begin{document}

\maketitle %
\vspace{-1.4cm}\hrule %
\thispagestyle{fancy}

\vspace*{.2cm}


% DEBUT DU DOC À MODIFIER : tout virer jusqu'au début de l'exo

%Définition et changement de valeurs de
compteurs%newcounter{cpt1}{section} compteur cpt1 remis à 0 à chaque
aumentation par stepcounter du compteur section%setcounter{cpt1}{3} on
met le compteur à 3%addtocounter{cpt1}{5} on ajoute 5 au compteur%
stepcounter{cpt1} on ajoute 1% ifthenelse{test}{alors}{sinon} (page
206) pour subordonner à une condition % whiledo{test}{commande} pour
faire une boucle (page 206 aussi) % value{cpt1} pour noter dans le
document la valeur de cpt1 
%Définition définitive d'opérateurs
mathématiques\newcommand{\ch}{\operatorname{ch}} 
\newcommand{\sh}{\operatorname{sh}}
\renewcommand{\tanh}{\operatorname{th}}
\renewcommand{\sinh}{\operatorname{sh}}
\renewcommand{\cosh}{\operatorname{ch}}
\newcommand{\argsh}{\operatorname{argsh}}
\newcommand{\argch}{\operatorname{argch}}
\newcommand{\argth}{\operatorname{argth}}
\newcommand{\Id}{\operatorname{Id}}
\newcommand{\id}{\operatorname{id}}
\renewcommand{\im}{\operatorname{Im}}
\renewcommand{\leq}{\leq}
\renewcommand{\geq}{\geq }

\newcommand{\dlim}{\lim}
\newcommand{\dsum}{\sum\limits}
\newcommand{\dprod}{\prod}
\newcommand{\lb}{\llbracket}
\newcommand{\rb}{\rrbracket}


%Définition de nouvelles couleurs : rgb(trois paramètres red green blue
entre 0 et 1); cmyk (quatre cyan magenta yellow black) entre 0 et 1;
gray (entre 0 et 1) et black, white, red, green, blue, cyan, magenta,
yellow% definecolor{0gris}{gray}{0.8} 
% Nouvelle commande pour encadrer le titre car shabox ne veut que d'une
seule ligne; ATTENTION A LA TAILLE; petite différence avec shadowbox ou
doublebox, voire fcolorbox ou colorbox (au lieu de shabox; laisser le
parbox tranquille sauf pour la taille de la boîte
\newcommand{\Tbox}[1]{\begin{center} \shabox{\parbox{0.8
\linewidth}{#1}} \end{center}} %[1] pour 1 paramètre ; #1 pour ce que
fait le 1er paramètre; entre accolades ce que fait la commande
%Mise en page en mode fancy : en-têtes et pieds de pages puis
définition des en-têtes et pieds de pages\pagestyle{fancy}
\lhead{ECE 2 - Mathématiques \\
Quentin Dunstetter - ENC-Bessières 2011$\backslash$2012}
\chead{}
\rhead{Edhec 2014}
\rfoot[ \ \thepage]{\thepage}
\cfoot{}
\lfoot{}
\thispagestyle{fancy} %Mise en page de la 1ère page en mode fancy
%Trait en bas et en haut de la page (entre en-tête et texte et texte et
pied de page)\renewcommand{\footrulewidth}{0.4pt}
\renewcommand{\headrulewidth}{0.4pt}

\indent \vspace{0.3cm}

\Tbox{\begin{center} \textbf{\Huge Edhec 2014} \end{center} }

\vspace{0.5cm}

\section*{Exercice 1}

\noindent On note $I$ la matrice $I = \begin{smatrix}
1 & 0 & 0 \\
0 & 1 & 0 \\
0 & 0 & 1 \\
\end{smatrix}
$ et on considère la matrice $A = \begin{smatrix}
7 & 5 & 1 \\
6 & -1 & 2 \\
6 & 1 & 3 \\
\end{smatrix}
$. 

\begin{noliste}{1.}
 \setlength{\itemsep}{4mm}

\item \begin{noliste}{a)}
 \setlength{\itemsep}{2mm}

\item Montrer, grâce à la méthode du pivot de Gauss, que les valeurs
propres $\lambda$ de $A$ sont les solutions de l'équation $\lambda^{3}
- 9 \lambda^{2} - 27 \lambda + 53 = 0$. 

\item Étudier la fonction $f$ qui, à tout réel $x$, associe $f(x) =
x^{3} - 9 x^{2} - 27 x + 53$, puis dresser son tableau de variation (on
précisera les limites de $f$ en $-\infty$ et $ + \infty$, on notera $m$
le minimum local de $f$ sur $\R$, $M$ le maximum local de $f$ sur $\R$
et on ne cherchera ni à calculer $m$, ni à calculer $M$).

\item Calculer $f(0)$ et $f(3)$ puis déterminer les signes de $m$ et
$M$.

\item Montrer que $A$ possède trois valeurs propres, que l'on ne
cherchera pas à calculer et que l'on notera $\lambda_{1}$,
$\lambda_{2}$ et $\lambda_{3}$, avec $\lambda_{1} < \lambda_{2} <
\lambda_{3}$.

\item En déduire qu'il existe une matrice $P$ inversible telle que $A =
P D P^{-1}$, avec $D = \begin{smatrix}
\lambda_{1} & 0 & 0 \\
0 & \lambda_{2} & 0 \\
0 & 0 & \lambda_{3} \\
\end{smatrix}
$.

\end{noliste}

\item L'objectif de cette question est de déterminer l'ensemble $E$ des
matrices $M$ de $\mathcal{M}_{3} (\R)$ qui commutent avec $A$
c'est-à-dire qui vérifient $A M = MA$.

\begin{noliste}{a)}
 \setlength{\itemsep}{2mm}

\item Montrer que les matrices qui commutent avec $D$ sont les matrices
diagonales. 

\item Montrer l'équivalence entre les deux propositions suivantes : 

\begin{nonoliste}{(i)}

\item $M$ est une matrice de $E$.

\item $P^{-1} M P$ commute avec $D$.

\end{nonoliste}

\item Établir que toute matrice $M$ de $E$ est combinaison linéaires
des trois matrices suivantes : 
\[
 P \begin{smatrix}
1 & 0 & 0 \\
0 & 0 & 0 \\
0 & 0 & 0 \\
\end{smatrix}
P^{-1} \, \ P \begin{smatrix}
0 & 0 & 0 \\
0 & 1 & 0 \\
0 & 0 & 0 \\
\end{smatrix}
P^{-1} \, \ P \begin{smatrix}
0 & 0 & 0 \\
0 & 0 & 0 \\
0 & 0 & 1 \\
\end{smatrix}
P^{-1} 
\]

\item En déduire que $E$ est un sous-espace vectoriel de
$\mathcal{M}_{3} (\R)$ et donner sa dimension.

\item Montrer, en raisonnant sur les valeurs propres de $A$, qu'il
n'existe aucun polynôme annulateur non nul de $A$ de degré inférieur ou
égal à 2. En déduire que $(I, \ A,\ A^{2})$ est une base de $E$.

\end{noliste}

\end{noliste}

\section*{Exercice 2}


\begin{noliste}{1.}
 \setlength{\itemsep}{4mm}

\item Montrer que l'intégrale $\dint{x}{2x} \frac{ 1 }{ \sqrt{ t^{2} +
1 } } $ est définie pour tout réel $x$.

\end{noliste}

\noindent On considère désormais la fonction $f$ définie par : 
\[
 \forall x \in \R, \ f(x) = \dint{x}{2x} \frac{ 1 }{ \sqrt{ t^{2} + 1 }
} \ dt 
\]

\begin{noliste}{1.}
 \setlength{\itemsep}{4mm}

\item Établir que $f$ est impaire.

\item \begin{noliste}{a)}
 \setlength{\itemsep}{2mm}

\item Montrer que $f$ est de classe $C^{1}$ sur $\R$.

\item Déterminer $f'(x)$, pour tout réel $x$, et en déduire que $f$ est
strictement croissante sur $\R$.

\end{noliste}

\item \begin{noliste}{a)}
 \setlength{\itemsep}{2mm}

\item En utilisant la relation $t^{2} \leq t^{2} + 1 \leq t^{2} + 2t +
1$, valable pour tout $t$ positif ou nul, montrer que l'on a
l'encadrement suivant :
\[
 \forall x \in \R_+^{\ast}, \ \ln (2x + 1) - \ln (x + 1) \leq f(x) \leq
\ln 2 
\]

\item Donner alors la limite de $f(x)$ lorsque $x$ tend vers $ +
\infty$.

\item Dresser le tableau de variation complet de $f$.

\item Résoudre l'équation $f(x) = 0$.

\end{noliste}

\item \begin{noliste}{a)}
 \setlength{\itemsep}{2mm}

\item Montrer que, pour tout réel $x$, on a : $x + \sqrt{ x^{2} + 1 } >
0$.

\item Déterminer la dérivée de la fonction $h$ qui, à tout réel $x$,
associe $\ln \left( x + \sqrt{ x^{2} + 1 } \right)$.

\item En déduire une expression explicite de $f(x)$.

\end{noliste}

\item Recherche d'un équivalent de $f(x)$ lorsque $x$ est au voisinage
de $0$.

\begin{noliste}{a)}
 \setlength{\itemsep}{2mm}

\item Établir que, pour tout réel $x$ strictement positif, on a : $x -
f(x) = \dint{x}{2x} \frac{ t^{2} }{ \sqrt{ t^{2} + 1} \left( 1 + \sqrt{
t^{2} + 1 } \right) }\ dt $.

\item En déduire :
\[
 \forall x \in \R_+^{\ast}, \ 0 \leq x - f(x) \leq \frac{7}{6} x^{3} 
\]

\item Conclure que : $f(x) \underset{0^+}{\sim} x $.

\item Montrer que l'on a aussi : $f(x) \underset{0^-}{\sim} x $.

\end{noliste}

\end{noliste}

\section*{Exercice 3}

\noindent Dans cet exercice, $\theta$ désigne un réel strictement
positif et $n$ un entier naturel supérieur ou égal à 2. \\
Pour tout $k$ de $\N$, on pose : $u_{k} = \frac{ 1 }{ 1 + \theta }
\left( \frac{ \theta }{ 1 + \theta } \right)^{k} $.

\begin{noliste}{1.}
 \setlength{\itemsep}{4mm}

\item Montrer que la suite $(u_{k})_{ k \in \N}$ définit une loi de
probabilité.

\end{noliste}

\noindent On considère maintenant une variable aléatoire $X$ prenant
ses valeurs dans $\N$ et dont la loi est donnée par : 
\[
 \forall k \in \N, \ \Prob\left(\Ev{ X = k}\right) = u_{k}. 
\]

\begin{noliste}{1.}
 \setlength{\itemsep}{4mm}

\item \begin{noliste}{a)}
 \setlength{\itemsep}{2mm}

\item On pose $Y = X + 1$. Reconnaître la loi de $Y$, puis en déduire
l'espérance et la variance de $X$.

\item Compléter la fonction \Scilab{} suivante pour qu'elle simule la
variable aléatoire $X$ : 
\begin{verbatim}
Function X(theta : real) : interger;
var Y : real;
Begin Y : = 0; Repeat Y : = Y + 1; until (-------); X : = -------; end;
\end{verbatim}

\end{noliste}

\item Dans cette question, on souhaite estimer le paramètre $\theta$
par la méthode du maximum de vraisemblance. Pour ce faire, on considère
un échantillon $(X_{1}, X_{2}, \dots, X_{n})$ composé de variables
aléatoires indépendantes ayant toutes la même loi que $X$ et on
introduit $L$, de $\R_+^{\ast}$ dans $\R$, définie par :
\[
 \forall \theta \in \R_+^{\ast}, \ L( \theta = \prod\limits_{k = 1}{n}
\Prob\left(\Ev{ X_{k} = x_{k} }\right) 
\]

où $x_{1}, x_{2}, \dots, x_{n}$ désignent des entiers naturels éléments
de $X ( \Omega ) $. \\

L'objectif est de chosir la valeur de $\theta$ qui rend $ L (\theta)$
maximale.

\begin{noliste}{a)}
 \setlength{\itemsep}{2mm}

\item Écrire $\ln \left( \rule{0cm}{0.3cm} L( \theta) \right) $ en
fonction de $\theta$ et de $S_{n} = \Sum{k = 1}{n} x_{k}$.

\item On considère la fonction $\varphi$ définie par :
\[
 \forall \theta \in \ ] 0 ; + \infty [, \ \varphi ( \theta ) = S_{n}
\ln \theta - (S_{n} + n) \ln ( 1 + \theta ) 
\]

Montrer que la fonction $\varphi$ admet un maximum, atteint en un seul
réel que l'on notera $\widehat{\theta_{n}}$ et que l'on exprimera en
fonction de $S_{n}$. Que représente $\widehat{\theta_{n}}$ pour la
fonction $L$ ?

\end{noliste}

On pose dorénavant : $T_{n} = \frac{1}{n} \Sum{i = 1}{n} X_{i}$. La
variable $T_{n}$ est appelée estimateur du maximum de vraisemblance
pour $\theta$.

\begin{noliste}{a)}
 \setlength{\itemsep}{2mm}

\item Vérifier que $T_{n}$ est un estimateur sans biais de $\theta$.

\item Calculer le risque quadratique $r_{T_{n}} (\theta)$ de $T_{n}$ et
vérifier que $\dlim{n \rightarrow + \infty} r_{T_{n}} (\theta) = 0$.

\end{noliste}

\end{noliste}

\section*{Problème}

\begin{noliste}{1.}
 \setlength{\itemsep}{4mm}

\item Soit $x$ un réel quelconque. \begin{noliste}{a)}
 \setlength{\itemsep}{2mm}

\item Justifier que la fonction $t \mapsto \max ( x, t)$ est continue
sur $\R$.

\end{noliste}

On considère maintenant l'intégrale $y = \dint{0}{1} \max (x,t)\ dt$.

\begin{noliste}{a)}
 \setlength{\itemsep}{2mm}

\item Montrer que : $y = \left\{ 
\begin{array}{l}
 \frac{1}{2} \text{ si } x \leq 0 \\
\\\
frac{ x^{2} + 1 }{ 2 } \text{ si } 0 < x \leq 1 \\
\\x \text{ si } x > 1
\end{array}
\right.$.

\end{noliste}

\end{noliste}

\noindent Dans la suite de ce problème, on considère une variable
aléatoire $X$ définie sur un certain espace probabilisé $(\Omega,
\mathcal{A}, \Pr)$, que l'on ne cherchera pas à déterminer. \\
On admet que l'on définit une variable aléatoire $Y$, elle aussi
définie sur $(\Omega, \mathcal{A}, \Pr)$, en posant $Y = \dint{0}{x}
\max ( X, t)\ dt$, ce qui signifie que, pour tout $\omega$ de $\Omega$,
on a : 
\[
 Y (\omega) = \dint{0}{1} \max \left( X (\omega, t \rule{0cm}{0.3cm}
\right)\ dt 
\]

\noindent On note $F_{Y}$ la fonction de répartition de $Y$.
\begin{center}
\textit{On se propose dans la suite de déterminer la loi de Y
connaissant celle de X}.
\end{center}

\begin{noliste}{1.}
 \setlength{\itemsep}{4mm}

\item Vérifier que si $X$ suit une loi géométrique alors on a : $Y =
X$.

\item On suppose, dans cette question, que $X (\Omega) = \{ -1, 0, 1
\}$ et que l'on a : 
\[
 \Prob\left(\Ev{ X = -1 }\right) = \Prob\left(\Ev{ X = 1 }\right) =
\frac{1}{4} 
\]

\begin{noliste}{a)}
 \setlength{\itemsep}{2mm}

\item Déterminer la valeur de $\Prob\left(\Ev{ X = 0 }\right) $.

\item Vérifier que $Y ( \Omega ) = \left\{ \frac{1}{2} ; 1 \right\}$
puis donner la loi de $Y$.

\item Compléter la déclaration de fonction suivante pour qu'elle simule
la variable aléatoire $Y$.
\begin{verbatim}
Function y : real;
Var u : integer;
Begin
u : = random(4) ;
If ------- then ------ else y : = ------;
End;
\end{verbatim}

\end{noliste}

\item On suppose, dans cette question, que $X$ suit la loi uniforme sur
$[0;1[$, avec $X(\Omega) = [ 0 ; 1[$.

\begin{noliste}{a)}
 \setlength{\itemsep}{2mm}

\item Vérifier, en utilisant la première question, que l'on a : $Y =
\frac{ X^{2} + 1 }{2} $.

\item En déduire que $Y(\Omega) = \left[ \ \frac{1}{2}, 1 \right[ $.

\item Montrer alors que, pour tout $x$ de $\left[ \ \frac{1}{2}, 1
\right[$, on a : $F_{Y} (x) = \sqrt{ 2x -1 }$.

\item Expliquer pourquoi $Y$ est une variable à densité.

\item Donner la valeur de $\E(Y)$.

\item Compléter la déclaration de fonction suivante pour qu'elle simule
la variable aléatoire $Y$.
\begin{verbatim}
Function y :real;
Var u : real;
Begin
u : = random;
y : = ------;
End;
\end{verbatim}

\end{noliste}

\item On suppose, dans cette question, que $X$ suit la loi normale
centrée réduite. On rappelle que $X (\Omega) = \R$ et on note $\Phi$ la
fonction de répartition de $X$. 

\begin{noliste}{a)}
 \setlength{\itemsep}{2mm}

\item Vérifier que $Y (\Omega) = \left[ \ \frac{1}{2} ; + \infty
\right[$.

\item Donner la valeur de $\Prob\left(\Ev{ Y = \frac{1}{2}}\right)$.

\item Utiliser la formule des probabilités totales associées au système
complet d'évènements \\
$\left( \rule{0cm}{0.3cm} [ X \leq 0 ], [ 0 < X \leq 1 ], [ X > 1]
\right)$ pour établir l'égalité suivante :
\[
 F_{Y} (x) = \left\{ 
\begin{array}{l}
 0 \text{ si } x < \frac{1}{2} \\
\\\
Phi \left( \sqrt{ 2x - 1 } \right) \text{ si } \frac{1}{2} \leq x \leq
1 \\
\\\
Phi (x) \text{ si } x > 1
\end{array}
\right. 
\]

\item La variable aléatoire $Y$ est-elle à densité ? Est-elle discrète
?

\end{noliste}

\end{noliste}

\end{document}

