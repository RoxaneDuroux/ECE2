\documentclass[11pt]{article}%
\usepackage{geometry}%
\geometry{a4paper,
 lmargin = 2cm,rmargin = 2cm,tmargin = 2.5cm,bmargin = 2.5cm}

\input{../../macros.tex}

\pagestyle{fancy} %
\lhead{ECE2 \hfill Mathématiques\\
} %
\chead{\hrule} %
\rhead{} %
\lfoot{} %
\cfoot{} %
\rfoot{\thepage} %

\renewcommand{\headrulewidth}{0pt}% : Trace un trait de séparation
 % de largeur 0,4 point. Mettre 0pt
 % pour supprimer le trait.

\renewcommand{\footrulewidth}{0.4pt}% : Trace un trait de séparation
 % de largeur 0,4 point. Mettre 0pt
 % pour supprimer le trait.

\setlength{\headheight}{14pt}

\title{\bf \vspace{-2cm} EDHEC 1996} %
\author{} %
\date{} %
\begin{document}

\maketitle %
\vspace{-1.4cm}\hrule %
\thispagestyle{fancy}

\vspace*{.2cm}


% DEBUT DU DOC À MODIFIER : tout virer jusqu'au début de l'exo

%Définition et changement de valeurs de
compteurs%newcounter{cpt1}{section} compteur cpt1 remis à 0 à chaque
aumentation par stepcounter du compteur section%setcounter{cpt1}{3} on
met le compteur à 3%addtocounter{cpt1}{5} on ajoute 5 au compteur%
stepcounter{cpt1} on ajoute 1% ifthenelse{test}{alors}{sinon} (page
206) pour subordonner à une condition % whiledo{test}{commande} pour
faire une boucle (page 206 aussi) % value{cpt1} pour noter dans le
document la valeur de cpt1 
%Définition définitive d'opérateurs
mathématiques\newcommand{\ch}{\operatorname{ch}} 
\newcommand{\sh}{\operatorname{sh}}
\renewcommand{\tanh}{\operatorname{th}}
\renewcommand{\sinh}{\operatorname{sh}}
\renewcommand{\cosh}{\operatorname{ch}}
\newcommand{\argsh}{\operatorname{argsh}}
\newcommand{\argch}{\operatorname{argch}}
\newcommand{\argth}{\operatorname{argth}}
\newcommand{\ker}{\operatorname{Ker}}
\renewcommand{\im}{\operatorname{Im}}
\newcommand{\rg}{\operatorname{rg}}
\newcommand{\Id}{\operatorname{Id}}
\newcommand{\id}{\operatorname{id}}
\renewcommand{\leq}{\leq}
\renewcommand{\geq}{\geq }

%Définition de nouvelles couleurs : rgb(trois paramètres red green blue
entre 0 et 1); cmyk (quatre cyan magenta yellow black) entre 0 et 1;
gray (entre 0 et 1) et black, white, red, green, blue, cyan, magenta,
yellow% definecolor{0gris}{gray}{0.8} 
% Nouvelle commande pour encadrer le titre car shabox ne veut que d'une
seule ligne; ATTENTION A LA TAILLE; petite différence avec shadowbox ou
doublebox, voire fcolorbox ou colorbox (au lieu de shabox; laisser le
parbox tranquille sauf pour la taille de la boîte
\newcommand{\Tbox}[1]{\begin{center} \shabox{\parbox{0.6
\linewidth}{#1}} \end{center}} %[1] pour 1 paramètre ; #1 pour ce que
fait le 1er paramètre; entre accolades ce que fait la commande
%Mise en page en mode fancy : en-têtes et pieds de pages puis
définition des en-têtes et pieds de pages\pagestyle{fancy}
\lhead{ECE 2 - Mathématiques \\
Quentin Dunstetter - ENC-Bessières 2011$\backslash$2012}
\chead{}
\rhead{Edhec 1996}
\rfoot[ \ \thepage]{\thepage}
\cfoot{}
\lfoot{}
\thispagestyle{fancy} %Mise en page de la 1ère page en mode fancy
%Trait en bas et en haut de la page (entre en-tête et texte et texte et
pied de page)\renewcommand{\footrulewidth}{0.4pt}
\renewcommand{\headrulewidth}{0.4pt}


%DEBUT DU DOCUMENT\vspace*{3cm}

\begin{center}
{\LARG\E\textbf{BANQUE COMMUNE D'ÉPREUVES}}



{\large \textsc{CONCOURS D ADMISSION DE 1996}}



{\large \textbf{Concepteur : Edhec}}



\rule{2.39cm}{0.05cm}



{\Large \textbf{OPTION ÉCONOMIQUE}}



{\Large \textbf{MATHÉMATIQUES }}



{\Large Lundi 9 mai, de 14h à 18h}



\rule{2.39cm}{0.05cm}
\end{center}

\textit{La présentation, la lisibilité, l'orthographe, la qualité
de la rédaction, la clarté et la précision des raisonnements
entreront pour une part importante dans l'appréciation des copies.}

\textit{Les candidats sont invités à \textbf{encadrer} dans la mesure
du possible les résultats de leurs calculs.}

\textit{Ils ne doivent faire usage d'aucun document. L'utilisation de
toute
calculatrice et de tout matériel électronique est interdite. Seule
l'utilisation d'une règle graduée est autorisée.}

\textit{Si au cours de l'épreuve, un candidat repère ce qui lui semble
être une erreur d'énoncé, il la signalera sur sa copie et
poursuivra sa composition en expliquant les raisons des initiatives
qu'il sera
amené à prendre.}

\vspace*{3cm}

\section*{EXERCICE 1}

On considère la suite $(d_{n})$ définie par 
\[
d_{0} = 1,\quad d_{1} = 0\quad et\quad \forall n\in \N,^{\times }\quad
d_{n + 1} = n(d_{n} + d_{n-1}).
\]

\begin{noliste}{1.}
 \setlength{\itemsep}{4mm}
\item 

\begin{noliste}{a)}
 \setlength{\itemsep}{2mm}
\item Calculer $d_{2}$, $d_{3}$, $d_{4}$ et $d_{5}$.

\item Montrer que : $\forall n\in \N,\quad d_{n}\in \N,$.
\end{noliste}

\item Montrer que : $\forall n\in \N,,\quad
d_{n + 1} = (n + 1)d_{n} + (-1)^{n + 1}$.

\item On considère, pour tout entier naturel $n$, l'intégrale $I_{n} =
\dfrac{{1}}{{n!}}{\dint{0}{1}{t^{n}e^{t}\,dt}}$.

\begin{noliste}{a)}
 \setlength{\itemsep}{2mm}
\item Calculer $I_{0}$, puis exprimer, pour tout entier naturel $n$,
$I_{n + 1} $ en fonction de $I_{n}$.

\item En déduire que : $\forall n\in \N,\quad e\cdot
d_{n} = n![1 + (-1)^{n}I_{n}]$.

\item Montrer alors que : $\forall n\in \N,\quad {\left|
{\,d_{n}-\dfrac{{n!}}{{e}}\,}\right| }\leq \dfrac{{1}}{{n + 1}}$.

\item Vérifier que cette dernière inégalité détermine parfaitement
$d_{n}$
pour $n\geq 2$, puis retrouver la valeur de $d_{5}$ obtenue à la
deuxième question et calculer $d_{10}$.

On donne $\dfrac{5!}{e}\simeq 44,1$ et $\dfrac{10!}{e}\simeq
1334960,\allowbreak 9$
\end{noliste}
\end{noliste}

\section*{EXERCICE 2}

\subsection*{Partie I}

On considère les matrices $I = {{\begin{smatrix}
{1} & {0} & {0} \\
{0} & {1} & {0} \\
{0} & {0} & {1}\end{smatrix}
}}$ et $J = {{\begin{smatrix}
{1} & {1} & {1} \\
{1} & {1} & {1} \\
{1} & {1} & {1}\end{smatrix}
}}$.\\
Soit de plus la matrice $M = {{\begin{smatrix}
{\dfrac{{2}}{{3}}} & {\dfrac{{1}}{{6}}} & {\dfrac{{1}}{{6}}} \\
{\dfrac{{1}}{{6}}} & {\dfrac{{2}}{{3}}} & {\dfrac{{1}}{{6}}} \\
{\dfrac{{1}}{{6}}} & {\dfrac{{1}}{{6}}} &
{\dfrac{{2}}{{3}}}\end{smatrix}
}}$.

\begin{noliste}{1.}
 \setlength{\itemsep}{4mm}
\item Exprimer $J^{2}$, puis pour tout entier naturel $n$ supérieur ou
égal à
2, $J^{n}$ en fonction de $J$.

\item En déduire que, pour tout entier naturel $n\ $ : $M^{n} =
\dfrac{{1}}{{2^{n}}}I + \dfrac{{1}}{{3}}\left(
{1-\dfrac{{1}}{{2^{n}}}}\right) J$.
\end{noliste}

\subsection*{Partie II}

Un mobile se déplace aléatoirement dans l'ensemble des sommets d'un
triangle 
\textit{ABC} de la façon suivante : si, à l'instant $n$, il est sur
l'un
quelconque des trois sommets, alors à l'instant $(n + 1)$, soit il y
reste,
avec une probabilité de $\dfrac{{2}}{{3}}$, soit il se place sur l'un
des
deux autres sommets, et ceci avec la même probabilité.

On note $A_{n}$ l'évènement : \textquotedblleft\ le mobile se trouve en
$A$ à
l'instant $n$ \textquotedblright.

$B_{n}$ l'évènement : \textquotedblleft\ le mobile se trouve en $B$ à
l'instant $n$ \textquotedblright.

$C_{n}$ l'évènement : \textquotedblleft\ le mobile se trouve en $C$ à
l'instant $n$ \textquotedblright.

On pose $a_{n} = P\left(\Ev{A_{n}}\right)$, $b_{n} =
P\left(\Ev{B_{n}}\right)$ et $c_{n} = P\left(\Ev{C_{n}}\right)$.

\begin{noliste}{1.}
 \setlength{\itemsep}{4mm}
\item Pour tout $n$ entier naturel, déterminer $a_{n} + b_{n} + c_{n}$.

\item 

\begin{noliste}{a)}
 \setlength{\itemsep}{2mm}
\item Exprimer, pour tout entier naturel $n$, $a_{n + 1}$, $b_{n + 1}$
et $c_{n + 1}$ en fonction de $a_{n}$, $b_{n}$ et $c_{n}$.

\item Déduire de la question précédente que : 
\[
\forall n\in \N,\quad a_{n + 1}-b_{n + 1} =
\dfrac{{1}}{{2}}(a_{n}-b_{n})\quad \text{et}\quad a_{n + 1}-c_{n + 1} =
\dfrac{{1}}{{2}}(a_{n}-c_{n}).
\]
\end{noliste}

\item On suppose, dans cette question seulement, que le mobile se
trouve en $A$ à l'instant 0.

\begin{noliste}{a)}
 \setlength{\itemsep}{2mm}
\item Calculer $a_{n}$, $b_{n}$ et $c_{n}$ en fonction de $n$.

\item Vérifier que ${{\begin{smatrix}
{a_{n}} \\
{b_{n}} \\
{c_{n}}\end{smatrix}
}}$ est la première colonne de $M^{n}$.

\item Démontrer ce résultat.
\end{noliste}

\item Expliquer comment retrouver, grâce à une méthode analogue à celle
employée dans la troisième question, les deux autres colonnes de
$M^{n}$
(aucun calcul n'est demandé).
\end{noliste}

\section*{EXERCICE 3}

\begin{noliste}{1.}
 \setlength{\itemsep}{4mm}
\item Montrer que : $\forall t>0\quad \ln (1 + t)>\dfrac{{t}}{{1 +
t}}$.

\item Soit $f$ la fonction définie pour tout réel $x$ par : $f(x) =
e^{-x}\ln
(1 + e^{x})$.

\begin{noliste}{a)}
 \setlength{\itemsep}{2mm}
\item Pour tout réel $x$, calculer $f^{\prime }(x)$ et en déduire les
variations de $f$.

\item Calculer $\underset{{x\rightarrow -\infty }}{{\lim }}f(x)$ et
$\underset{{x\rightarrow + \infty }}{{\lim }}f(x)$.
\end{noliste}

\item 

\begin{noliste}{a)}
 \setlength{\itemsep}{2mm}
\item Pour tout réel $x$, vérifier que : $f(x) = 1-f^{\prime
}(x)-\dfrac{{e^{x}}}{{1 + e^{x}}}$.\\
En déduire, en fonction de $f$, une primitive $F$ de $f$ sur $\R$.

\item Montrer que l'intégrale impropre ${\dint{0}{+ \infty
}{f(x)\,dx}}$ est convergente et donner sa valeur.
\end{noliste}

\item Soit $\alpha $ un réel et $g$ la fonction définie par : $g(x) =
0$ si $x<0$ et $g(x) = \alpha f(x)$ si $x\geq 0$. \\
Déterminer $\alpha $ pour que $g$ puisse être considérée comme densité
d'une
variable aléatoire $X$.
\end{noliste}

\section*{PROBLEME}

\subsection*{Partie I}

On considère la fonction $f$ définie sur $\R_{+}{\times }$, par $f(x) =
x-\ln x$.

\begin{noliste}{1.}
 \setlength{\itemsep}{4mm}
\item Étudier $f$ et résumer cette étude par un tableau de variations.

\item Étudier le signe de $f(x)-x$, pour tout réel $x$ strictement
positif.
\end{noliste}

\subsection*{Partie II}

On considère l'algorithme suivant :

\texttt{Program\ iter\ ;}

\texttt{Var\ n,\ k\ :\ integer\ ;}

\texttt{a,\ u,\ p\ :\ real\ ;}

\texttt{Function\ f(x\ :real)\ :\ real\ ;}

\texttt{Begin}

\texttt{If\ x\ \ > \ 0\ then\ f\ : = \ x\ -\ ln(x)\ ;}

\texttt{End\ ;}

\texttt{Begin}

\texttt{Readln(n,\ a)\ ;\ u\ : = \ a\ ;\ p\ : = \ a\ ;}

\texttt{For\ k\ : = \ 1\ to\ n\ do\ begin}

\texttt{\ \ \ \ \ \ \ \ \ \ \ \ \ \ \ \ \ \ \ \ \ u\ : = \ f(u)\ ;}

\texttt{\ \ \ \ \ \ \ \ \ \ \ \ \ \ \ \ \ \ \ \ \ p\ : = \ p*u\ ;}

\texttt{\ \ \ \ \ \ \ \ \ \ \ \ \ \ \ \ \ \ \ end\ ;}

\texttt{Writeln(u,\ p)\ ;}

\texttt{End.}

\begin{noliste}{1.}
 \setlength{\itemsep}{4mm}
\item Dans le cas particulier où $n = 3$ et $a = 2$, donner les valeurs
approchées à $10^{-4}$ près par défaut des contenus des variables $u$et
$p$ à la fin
de l'algorithme.\\
Dorénavant, dans le cas général, on note $u_{n}$ et $p_{n}$ les
contenus
respectifs des variables $u$et $p$ à la fin de l'algorithme, lorsque
leur
calcul est possible.

\item 

\begin{noliste}{a)}
 \setlength{\itemsep}{2mm}
\item Pour quelles valeurs de $a$, peut-on définir la suite
$(u_{n})_{n\in 
\N,}$ dont le premier terme est $u_{0} = a$, et dont le terme général
$u_{n}$ est calculé par l'algorithme précédent ?

\item Pour les valeurs de $a$ trouvées ci-dessus, donner en fonction de
$n$
le nombre d'appels de fonction utilisés au cours de cet algorithme,
ainsi
que le nombre de soustractions, de multiplications et d'affectations
nécessaires au calcul de $u$ et de $p$.\\
NB : le symbole \textbf{ : = } utilisé dans l'écriture
\textquotedblleft\ 
\textbf{for k\textit{\ } : = 1 to n} \textquotedblright\ ne sera pas
considéré
comme une affectation, mais chaque appel de fonction nécessite une
affectation (\textbf{f : = }) et une soustraction.
\end{noliste}

\item Lorsque la suite $(u_{n})_{n\in \N,}$ est bien définie, écrire,
pour tout entier naturel $n$, la relation liant $u_{n + 1}$ et $u_{n}$.

\item Pour quelle valeur de $a$ la suite $(u_{n})_{n\in \N,}$
est-elle constante ?

\item On suppose, dans cette question, que $a>1$.

\begin{noliste}{a)}
 \setlength{\itemsep}{2mm}
\item Montrer que, pour tout entier naturel $n\ $ : $u_{n}>1$.

\item Étudier les variations de la suite $(u_{n})_{n\in \N,}$.

\item En déduire que $(u_{n})_{n\in \N,}$ converge et donner sa
limite.
\end{noliste}

\item On suppose, dans cette question, que $0<a<1$.

\begin{noliste}{a)}
 \setlength{\itemsep}{2mm}
\item Montrer que$\ $ : $u_{1}>1$.

\item En déduire que $(u_{n})_{n\in \N,}$ converge et donner sa
limite.
\end{noliste}
\end{noliste}

\subsection*{Partie III}

\begin{noliste}{1.}
 \setlength{\itemsep}{4mm}
\item En considérant $\ln (p_{n})$, exprimer $p_{n}$ en fonction
seulement
de $a$ et $u_{n + 1}$, puis calculer $\underset{{n\rightarrow + \infty
}}{{\lim 
}}p_{n}$.

\item Écrire alors un nouvel algorithme en \Scilab{}, ne contenant
aucune
multiplication et permettant le calcul de $p_{n}$.
\end{noliste}

\subsection*{Partie IV}

Dans cette partie, on choisit $a>1$.\\
On pose, pour tout entier naturel $n\ $ : $v_{n} = {\dint{u_{n +
1}}{u_{n}}{f(t)\,dt}}$.

\begin{noliste}{1.}
 \setlength{\itemsep}{4mm}
\item Vérifier que la suite $(v_{n})_{n\in \N,}$ est bien définie.

\item 

\begin{noliste}{a)}
 \setlength{\itemsep}{2mm}
\item Montrer que, pour tout entier naturel $n$, on a : 
\[
u_{n + 2}\ln (u_{n})\leq v_{n}\leq u_{n + 1}\ln (u_{n}).
\]

\item En déduire que $v_{n}\underset{{+ \infty }}{{\sim }}\ln (u_{n})$.
\end{noliste}
\end{noliste}

\label{fin}

\end{document}

