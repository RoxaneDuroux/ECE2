\documentclass[11pt]{article}%
\usepackage{geometry}%
\geometry{a4paper,
 lmargin = 2cm,rmargin = 2cm,tmargin = 2.5cm,bmargin = 2.5cm}

\input{../../macros.tex}

\pagestyle{fancy} %
\lhead{ECE2 \hfill Mathématiques\\
} %
\chead{\hrule} %
\rhead{} %
\lfoot{} %
\cfoot{} %
\rfoot{\thepage} %

\renewcommand{\headrulewidth}{0pt}% : Trace un trait de séparation
 % de largeur 0,4 point. Mettre 0pt
 % pour supprimer le trait.

\renewcommand{\footrulewidth}{0.4pt}% : Trace un trait de séparation
 % de largeur 0,4 point. Mettre 0pt
 % pour supprimer le trait.

\setlength{\headheight}{14pt}

\title{\bf \vspace{-2cm} EDHEC 2004} %
\author{} %
\date{} %
\begin{document}

\maketitle %
\vspace{-1.4cm}\hrule %
\thispagestyle{fancy}

\vspace*{.2cm}


% DEBUT DU DOC À MODIFIER : tout virer jusqu'au début de l'exo

%Définition et changement de valeurs de
compteurs%newcounter{cpt1}{section} compteur cpt1 remis à 0 à chaque
aumentation par stepcounter du compteur section%setcounter{cpt1}{3} on
met le compteur à 3%addtocounter{cpt1}{5} on ajoute 5 au compteur%
stepcounter{cpt1} on ajoute 1% ifthenelse{test}{alors}{sinon} (page
206) pour subordonner à une condition % whiledo{test}{commande} pour
faire une boucle (page 206 aussi) % value{cpt1} pour noter dans le
document la valeur de cpt1 
%Définition définitive d'opérateurs
mathématiques\newcommand{\ch}{\operatorname{ch}} 
\newcommand{\sh}{\operatorname{sh}}
\renewcommand{\tanh}{\operatorname{th}}
\renewcommand{\sinh}{\operatorname{sh}}
\renewcommand{\cosh}{\operatorname{ch}}
\newcommand{\argsh}{\operatorname{argsh}}
\newcommand{\argch}{\operatorname{argch}}
\newcommand{\argth}{\operatorname{argth}}
\newcommand{\Id}{\operatorname{Id}}
\renewcommand{\leq}{\leq}
\renewcommand{\geq}{\geq }

\newcommand{\dlim}{\lim}
\newcommand{\dsum}{\sum}
\newcommand{\dprod}{\prod}



%Définition de nouvelles couleurs : rgb(trois paramètres red green blue
entre 0 et 1); cmyk (quatre cyan magenta yellow black) entre 0 et 1;
gray (entre 0 et 1) et black, white, red, green, blue, cyan, magenta,
yellow% definecolor{0gris}{gray}{0.8} 
% Nouvelle commande pour encadrer le titre car shabox ne veut que d'une
seule ligne; ATTENTION A LA TAILLE; petite différence avec shadowbox ou
doublebox, voire fcolorbox ou colorbox (au lieu de shabox; laisser le
parbox tranquille sauf pour la taille de la boîte
\newcommand{\Tbox}[1]{\begin{center} \shabox{\parbox{0.6
\linewidth}{#1}} \end{center}} %[1] pour 1 paramètre ; #1 pour ce que
fait le 1er paramètre; entre accolades ce que fait la commande
%Mise en page en mode fancy : en-têtes et pieds de pages puis
définition des en-têtes et pieds de pages\pagestyle{fancy}
\lhead{ECE 2 - Mathématiques \\
Quentin Dunstetter - ENC-Bessières 2011$\backslash$2012}
\chead{}
\rhead{Edhec 2004}
\rfoot[ \ \thepage]{\thepage}
\cfoot{}
\lfoot{}
\thispagestyle{fancy} %Mise en page de la 1ère page en mode fancy
%Trait en bas et en haut de la page (entre en-tête et texte et texte et
pied de page)\renewcommand{\footrulewidth}{0.4pt}
\renewcommand{\headrulewidth}{0.4pt}

\begin{center}
{\Huge EDHEC}

\textbf{School of management\vspace{3cm}}

{\large ECOLE D\E\ HAUTES\ ETUDES\ COMMERCIALES\ DU\ NORD}

{\large Concours d'admission sur classes préparatoires}

\underline{\hspace{3cm}}

{\LARGE MATHEMATIQUES}

{\large Option économique}

\textbf{Année 2004}{\large }
\end{center}

\noindent \textsl{La présentation, la lisibilité, l'orthographe, la
qualité
de la rédaction, la clarté et la précision des raisonnements entreront
pour
une part importante dans l'appréciation des copies.}

\noindent \textsl{Les candidats sont invités à encadrer dans la mesure
du
possible les résultats de leurs calculs.}

\noindent \textsl{Ils ne doivent faire usage d'aucun document : seule
l'utilisation d'une règle graduée est autorisée.\vspace{1cm}}

\noindent \textbf{L'utilisation de toute calculatrice et de tout
matériel électronique est interdite.}\vspace{1cm}

\section*{Exercice 1}

Le but de cet exercice est de calculer $\dlim{n\rightarrow + \infty
}\dint{0}{+ \infty }\dfrac{1}{1 + t + t^{n}}dt$. \\
Pour tout n de $\N$, on pose $u_{n} = \dint{0}{1}\dfrac{1}{
1 + t + t^{n}}dt$ et on a, en particulier, $u_{0} =
\dint{0}{1}\dfrac{1}{
2 + t}dt$

\begin{noliste}{1.}
 \setlength{\itemsep}{4mm}
\item Pour tout $n$ de $\N$, justifier l'existence de $u_{n}$.

\item Calculer $u_{0}$ et $u_{1}$.

\item 

\begin{noliste}{a)}
 \setlength{\itemsep}{2mm}
\item Montrer que la suite $(u_{n})$ est croissante.

\item Montrer que : $\forall n\in \N,\quad u_{n}\leq \ln 2$.

\item En déduire que la suite $(u_{n})$ est convergente.
\end{noliste}

\item 

\begin{noliste}{a)}
 \setlength{\itemsep}{2mm}
\item Pour tout $n$ de $\N$, écrire $\ln 2-u_{n}$ sous la forme
d'une intégrale.

\item En déduire que : $\forall n\in \N,\quad \ln 2-u_{n}\leq 
\dfrac{1}{n + 1}$.

\item Donner la limite de la suite $(u_{n})$.
\end{noliste}

\item Pour tout entier naturel $n$ supérieur ou égal à 2, on pose
$v_{n} = \dint{1}{+ \infty }\dfrac{1}{1 + t + t^{n}}dt$.

\begin{noliste}{a)}
 \setlength{\itemsep}{2mm}
\item Justifier la convergence de l'intégrale défnissant $v_{n}$.

\item Montrer que : $\forall n\geq 2,\quad 0\leq v_{n}\leq 
\dfrac{1}{n-1}$.

\item En déduire $\dlim{n\rightarrow + \infty }v_{n}$, puis donner la
valeur de $\dlim{n\rightarrow + \infty }\dint{0}{+ \infty }
\dfrac{1}{1 + t + t^{n}}dt$.
\end{noliste}
\end{noliste}

\section*{Exercice 2}

On note $E$ l'espace vectoriel des fonctions polynomiales réelles de
degré inférieur ou égal à 2. \\
On note $e_{0},e_{1},e_{2}$ les fonctions définies, pour tout réel $x$
par 
\[
e_{0}(x) = 1,\quad e_{1}(x) = x\quad \text{et}\quad e_{2}(x) = x^{2}
\]
et on rappelle que $B = (e_{0},e_{1},e_{2})$ est une base de $E$.\\
Soit $f$ l'application qui à toute fonction polynomiale $P$ de $E$
associe
la fonction $Q = f(P)$, où $Q$ est la dérivée seconde de l'application
qui à
tout réel $x$ associe $(x^{2}-x)P\left(\Ev{x}\right)$.

\begin{noliste}{1.}
 \setlength{\itemsep}{4mm}
\item 

\begin{noliste}{a)}
 \setlength{\itemsep}{2mm}
\item Montrer que $f$ est un endomorphisme de $E$.

\item Déterminer $f(e_{0}),$ $f(e_{1})$ et $f(e_{2})$ en fonction de
$e_{0},$
$e_{1}$ et $e_{2}$.

\item En déduire que la matrice de $f$ dans la base $B$ est $A = \left(

\begin{array}{ccc}
2 & -2 & 0 \\
0 & 6 & -6 \\
0 & 0 & 12
\end{array}
\right) $

\item Montrer sans calcul que $f$ est un automorphisme de $E$.
\end{noliste}

\item 

\begin{noliste}{a)}
 \setlength{\itemsep}{2mm}
\item Donner les valeurs propres de $f$, puis en déduire que $f$ est
diagonalisable.

\item Déterminer les sous-espaces propres de $f$.
\end{noliste}

\item 

\begin{noliste}{a)}
 \setlength{\itemsep}{2mm}
\item Justifier l'existence d'une matrice P inversible dont la première
ligne ne contient que des \textquotedblleft 1\textquotedblright\ telle
que$
A = PDP^{-1}$, où $D = \left( 
\begin{array}{ccc}
2 & 0 & 0 \\
0 & 6 & 0 \\
0 & 0 & 12
\end{array}
\right).$

\item Montrer que : $\forall n\in \N,\quad A^{n} = PD^{n}P^{-1}$.
\end{noliste}

\item 

\begin{noliste}{a)}
 \setlength{\itemsep}{2mm}
\item Déterminer la matrice $P^{-1}$.

\item En déduire explicitement, en fonction de $n$, la matrice $A^{n}$.

\item On dit qu'une suite de matrices $(M_{n})$ tend vers la matrice
$M$,
lorsque $n$ tend vers $ + \infty $, si chaque coefficient de $M_{n}$
tend
vers le coefficient situé à la même place dans $M$. \\
On pose $B = \dfrac{1}{12}A$. Montrer que la suite $(B^{n})$ tend vers
une
matrice $J$ vérifiant $J^{2} = J$.
\end{noliste}
\end{noliste}

\newpage

\section*{Exercice 3}

On désigne par $n$ un entier naturel supérieur ou égal à $2$.\\
On lance n fois une pièce équilibrée (c'est-à-dire donnant
\textquotedblleft
pile\textquotedblright\ avec la probabilité $\dfrac{1}{2}$ et
\textquotedblleft face\textquotedblright\ également avec la probabilité
$
\dfrac{1}{2}$), les lancers étant supposés indépendants. \\
On note $Z$ la variable aléatoire qui vaut 0 si l'on n'obtient aucun
\textquotedblleft pile\textquotedblright\ pendant ces $n$ lancers et
qui,
dans le cas contraire, prend pour valeur le rang du premier
\textquotedblleft pile\textquotedblright.

\begin{noliste}{1.}
 \setlength{\itemsep}{4mm}
\item 

\begin{noliste}{a)}
 \setlength{\itemsep}{2mm}
\item Déterminer, en argumentant soigneusement, l'ensemble $Z(\Omega
)$.

\item Pour tout $k$ de $Z(\Omega )$, calculer $P\left(\Ev{Z =
k}\right)$. On distinguera les
cas $k = 0$ et $k\geq 1$.

\item Vérifier que $\Sum{k\in Z(\Omega )}P\left(\Ev{Z = k}\right) = 1.$

\item On rappelle que l'instruction \textquotedblleft
random(2)\textquotedblright\ renvoie un nombre au hasard parmi les
nombres 0
et 1. Recopier et compléter le programme suivant pour qu'il simule \\
l'expérience décrite ci-dessus, l'entier n étant entré au clavier par
l'utilisateur (\textquotedblleft pile\textquotedblright\ sera codé par
le
nombre 1 et \textquotedblleft face\textquotedblright\ par 0).

\texttt{Program EDHEC2004 ;}

\texttt{var k, n, z, lancer : integer ;}

\texttt{Begin}

\texttt{Randomize ;}

\texttt{Readln(n) ; k : = 0 ; z : = 0 ;}

\texttt{Repeat }

\texttt{k : = k + 1 ; lancer : = random(2) ; }

\texttt{If (lancer = 1) then..................;}

\texttt{until (lancer = 1 or..........) ;}

\texttt{Writeln (z) ; }

\texttt{end.}

\noindent On dispose de $n + 1$ urnes $U_{0},U_{1},...,U_{n}$ telles
que pour
tout $k$ de $\{0,1,...,n\}$, l'urne $U_{k}$ contient $k$ boules
blanches et $
n-k$ boules noires.\\
On effectue des tirages d'une boule, au hasard et avec remise dans ces
urnes
de la façon suivante : \\
si après les lancers de la pièce décrits dans la première question, la
variable $Z$ prend la valeur $k$ (avec $k\geq 1$), alors on tire une
par une et avec remise, $k$ boules dans l'urne $U_{k}$ et l'on note $X$
la
variable aléatoire égale au nombre de boules blanches obtenues à
l'issue de
ces tirages. \\
Si la variable $Z$ a pris la valeur 0, aucun tirage n'est effectué et
$X$
prend la valeur 0.
\end{noliste}

\item Déterminer $X(\Omega )$.

\item 

\begin{noliste}{a)}
 \setlength{\itemsep}{2mm}
\item Déterminer, en distinguant les cas $i = 0$ et $1\leq i\leq n$,
la probabilité $P\left(\Ev{X = i/Z = 0}\right)$.

\item Déterminer, en distinguant les cas $i = n$ et $0\leq i\leq n-1$,
la probabilité $P\left(\Ev{X = i/Z = n}\right)$.

\item Pour tout $k$ de $\{1,2,...,n-1\}$ déterminer, en distinguant les
cas $
0\leq i\leq k$ et $k<i\leq n$, la probabilité conditionnelle $
P\left(\Ev{X = i/Z = k}\right)$.
\end{noliste}

\item 

\begin{noliste}{a)}
 \setlength{\itemsep}{2mm}
\item Montrer que $P\left(\Ev{X = 0}\right) = \Sum{k =
1}{n-1}(\dfrac{n-k}{2n})^{k} + 
\dfrac{1}{2^{n}}$.

\item Montrer que $P\left(\Ev{X = n}\right) = \dfrac{1}{2^{n}}$.

\item Exprimer, pour tout $i$ de $\{1,2,...,n-1\}$, $P\left(\Ev{X =
i}\right)$ sous forme
d'une somme que l'on ne cherchera pas à réduire.
\end{noliste}

\item Vérifier, avec les expressions trouvées à la question précédente,
que $\Sum{i = 0}{n}P\left(\Ev{X = i}\right) = 1$.
\end{noliste}

\newpage

\section*{Problème}

Dans ce problème, la lettre $n$ désigne un entier naturel non nul.\\
On note $f_{n}$ la fonction définie sur $\R$ par : $f_{n}(x) =
xe^{-\dfrac{n}{x}}$ si $x\neq 0$ et $f_{n}(0) = 0$.\\
On note $(C_{n})$ la courbe représentative de $f_{n}$ dans un repère
orthonormé $(O,\vec{i},\vec{j})$.

\begin{noliste}{1.}
 \setlength{\itemsep}{4mm}
\item 

\begin{noliste}{a)}
 \setlength{\itemsep}{2mm}
\item Montrer que $f_{n}$ est continue à droite en 0.

\item Montrer que $f_{n}$ est dérivable à droite en 0 et donner la
valeur du
nombre dérivé \\
à droite en 0 de $f_{n}$.
\end{noliste}

\item 

\begin{noliste}{a)}
 \setlength{\itemsep}{2mm}
\item Montrer que $f_{n}$ est dérivable sur $]-\infty,0[$ et sur $
]0, + \infty \lbrack $. Pour tout réel $x$ non nul, \\
calculer $f_{n}{\prime }(x)$ puis étudier son signe.

\item Calculer les limites de $f_{n}$ en $ + \infty $, $-\infty $ et
$0^{-}$,
puis donner le tableau de variation de $f_{n}$.
\end{noliste}

\item 

\begin{noliste}{a)}
 \setlength{\itemsep}{2mm}
\item Rappeler le développement limité à l'ordre $2$ de $e^{u}$ lorsque
$u$
est au voisinage de 0.

\item En déduire que, lorsque $x$ est au voisinage de $ + \infty $ ou
au
voisinage de $-\infty $, on a :
\[
f_{n}(x) = x-n + \dfrac{n^{2}}{2x} + o(\dfrac{1}{x}).
\]

\item En déduire qu'au voisinage de $ + \infty $, ainsi qu'au voisinage
de $
-\infty $, $(C_{n})$ admet une asymptote \textquotedblleft
oblique\textquotedblright\ $(D_{n})$ dont on donnera une équation.
Préciser
la position relative de $(D_{n})$ et $(C_{n})$ aux voisinages de $ +
\infty $
et de $-\infty $.

\item Donner l'allure de la courbe $(C_{1})$.
\end{noliste}

\item 

\begin{noliste}{a)}
 \setlength{\itemsep}{2mm}
\item Montrer qu'il existe un unique réel, que l'on notera $u_{n}$, tel
que $
f_{n}(u_{n}) = 1$.

\item Vérifier que, pour tout $n$ de $\N^{\times }$, $u_{n}$ est
strictement supérieur à $1$ et que $u_{n}$ est solution de l'équation
$x\ln
(x) = n$.

\item Étudier la fonction $g$ définie sur $[1, + \infty \lbrack $ par $
g(x) = x\ln x$. En déduire, en utilisant la fonction $g^{-1}$, que $
\dlim{n\rightarrow + \infty }u_{n} = + \infty.$

\item Justifier la relation $\ln u_{n} + \ln (\ln u_{n}) = \ln n$, puis
montrer
que $\ln u_{n}\underset{n\rightarrow + \infty }{\sim }\ln n$. \\
En déduire un équivalent de $u_{n}$ lorsque n est au voisnage de $ +
\infty $.
\end{noliste}

\item 

\begin{noliste}{a)}
 \setlength{\itemsep}{2mm}
\item Montrer que la suite $(u_{n})_{n\geq 1}$ est strictement
croissante.

\item Montrer que : $f_{n}(u_{n + 1}) = \exp (\dfrac{1}{u_{n + 1}})$.
\end{noliste}

\item On pose $I_{n} = \dint{u_{n}}{u_{n + 1}}f_{n}(t)dt$.

\begin{noliste}{a)}
 \setlength{\itemsep}{2mm}
\item Montrer que : $1\leq \dfrac{I_{n}}{u_{n + 1}-u_{n}}\leq \exp (
\dfrac{1}{u_{n + 1}})$.

\item En déduire un équivalent de $I_{n}$ lorsque $n$ est au voisinage
de $
 + \infty $.

\item Montrer alors que la série de terme général $I_{n}$ est
divergente.
\end{noliste}
\end{noliste}

\label{fin}

\end{document}

