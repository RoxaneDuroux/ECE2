\documentclass[11pt]{article}%
\usepackage{geometry}%
\geometry{a4paper,
 lmargin = 2cm,rmargin = 2cm,tmargin = 2.5cm,bmargin = 2.5cm}

\input{../../macros.tex}

\pagestyle{fancy} %
\lhead{ECE2 \hfill Mathématiques\\
} %
\chead{\hrule} %
\rhead{} %
\lfoot{} %
\cfoot{} %
\rfoot{\thepage} %

\renewcommand{\headrulewidth}{0pt}% : Trace un trait de séparation
 % de largeur 0,4 point. Mettre 0pt
 % pour supprimer le trait.

\renewcommand{\footrulewidth}{0.4pt}% : Trace un trait de séparation
 % de largeur 0,4 point. Mettre 0pt
 % pour supprimer le trait.

\setlength{\headheight}{14pt}

\title{\bf \vspace{-2cm} EDHEC 2000} %
\author{} %
\date{} %
\begin{document}

\maketitle %
\vspace{-1.4cm}\hrule %
\thispagestyle{fancy}

\vspace*{.2cm}


% DEBUT DU DOC À MODIFIER : tout virer jusqu'au début de l'exo

%Définition et changement de valeurs de
compteurs%newcounter{cpt1}{section} compteur cpt1 remis à 0 à chaque
aumentation par stepcounter du compteur section%setcounter{cpt1}{3} on
met le compteur à 3%addtocounter{cpt1}{5} on ajoute 5 au compteur%
stepcounter{cpt1} on ajoute 1% ifthenelse{test}{alors}{sinon} (page
206) pour subordonner à une condition % whiledo{test}{commande} pour
faire une boucle (page 206 aussi) % value{cpt1} pour noter dans le
document la valeur de cpt1 
%Définition définitive d'opérateurs
mathématiques\newcommand{\ch}{\operatorname{ch}} 
\newcommand{\sh}{\operatorname{sh}}
\renewcommand{\tanh}{\operatorname{th}}
\renewcommand{\sinh}{\operatorname{sh}}
\renewcommand{\cosh}{\operatorname{ch}}
\newcommand{\argsh}{\operatorname{argsh}}
\newcommand{\argch}{\operatorname{argch}}
\newcommand{\argth}{\operatorname{argth}}
\newcommand{\ker}{\operatorname{Ker}}
\renewcommand{\im}{\operatorname{Im}}
\newcommand{\rg}{\operatorname{rg}}
\newcommand{\Id}{\operatorname{Id}}
\newcommand{\id}{\operatorname{id}}
\renewcommand{\leq}{\leq}
\renewcommand{\geq}{\geq }

%Définition de nouvelles couleurs : rgb(trois paramètres red green blue
entre 0 et 1); cmyk (quatre cyan magenta yellow black) entre 0 et 1;
gray (entre 0 et 1) et black, white, red, green, blue, cyan, magenta,
yellow% definecolor{0gris}{gray}{0.8} 
% Nouvelle commande pour encadrer le titre car shabox ne veut que d'une
seule ligne; ATTENTION A LA TAILLE; petite différence avec shadowbox ou
doublebox, voire fcolorbox ou colorbox (au lieu de shabox; laisser le
parbox tranquille sauf pour la taille de la boîte
\newcommand{\Tbox}[1]{\begin{center} \shabox{\parbox{0.6
\linewidth}{#1}} \end{center}} %[1] pour 1 paramètre ; #1 pour ce que
fait le 1er paramètre; entre accolades ce que fait la commande
%Mise en page en mode fancy : en-têtes et pieds de pages puis
définition des en-têtes et pieds de pages\pagestyle{fancy}
\lhead{ECE 2 - Mathématiques \\
Quentin Dunstetter - ENC-Bessières 2011$\backslash$2012}
\chead{}
\rhead{Edhec 2000}
\rfoot[ \ \thepage]{\thepage}
\cfoot{}
\lfoot{}
\thispagestyle{fancy} %Mise en page de la 1ère page en mode fancy
%Trait en bas et en haut de la page (entre en-tête et texte et texte et
pied de page)\renewcommand{\footrulewidth}{0.4pt}
\renewcommand{\headrulewidth}{0.4pt}


%DEBUT DU DOCUMENT\vspace*{3cm}

\begin{center}
{\LARG\E\textbf{BANQUE COMMUNE D'ÉPREUVES}}



{\large \textsc{CONCOURS D ADMISSION DE 2000}}



{\large \textbf{Concepteur : Edhec}}



\rule{2.39cm}{0.05cm}



{\Large \textbf{OPTION ÉCONOMIQUE}}



{\Large \textbf{MATHÉMATIQUES }}



{\Large Lundi 9 mai, de 14h à 18h}



\rule{2.39cm}{0.05cm}
\end{center}

\textit{La présentation, la lisibilité, l'orthographe, la qualité
de la rédaction, la clarté et la précision des raisonnements
entreront pour une part importante dans l'appréciation des copies.}

\textit{Les candidats sont invités à \textbf{encadrer} dans la mesure
du possible les résultats de leurs calculs.}

\textit{Ils ne doivent faire usage d'aucun document. L'utilisation de
toute
calculatrice et de tout matériel électronique est interdite. Seule
l'utilisation d'une règle graduée est autorisée.}

\textit{Si au cours de l'épreuve, un candidat repère ce qui lui semble
être une erreur d'énoncé, il la signalera sur sa copie et
poursuivra sa composition en expliquant les raisons des initiatives
qu'il sera
amené à prendre.}

\vspace*{3cm}

\paragraph{ Exercice 1}

\begin{noliste}{1.}
 \setlength{\itemsep}{4mm}
\item Déterminer l'ensemble $D$ des réels tels que $e^{x}-e^{-x}>0$.

On définit la fonction $f$ par~ : $\forall x \in D,\;
f(x) = \ln\left(e^{x}-e^{-x}\right)$.

On note $(C)$ sa courbe représentative dans un repère orthonormé
$(O,\vec{\imath},\vec{\jmath})$.

\item 
\begin{noliste}{a)}
 \setlength{\itemsep}{2mm}
\item Étudier les variations de $f$ et donner les limites de $f$ aux
bornes de $D$.

\item En déduire l'existence d'un unique réel $\alpha$ vérifiant
$f(\alpha) = 0$, puis donner la valeur exacte de $\alpha$.

\item Montrer que le coefficient directeur de la tangente $(T)$ à la
courbe $(C)$ au point d'abscisse $\alpha$ vaut $\sqrt{5}$.
\end{noliste}

\item 
\begin{noliste}{a)}
 \setlength{\itemsep}{2mm}
\item Calculer $ \dlim{x\rightarrow + \infty} \left(f(x)-x\right)
$.

\item En déduire l'équation de l'asymptote $(\Delta)$ à la courbe $(C)$
au voisinage de $ + \infty$.

\item Donner la position relative de $(\Delta)$ et $(C)$.
\end{noliste}

\item Donner l'allure de la courbe $(C)$ en faisant figurer les droites
$(\Delta)$ et $(T)$.

On admettra que $\alpha \simeq 0,5$ et que $\sqrt{\alpha}\simeq 2,2$.

\item Soit $\lambda$ un réel, on note $g_{\lambda}$ la fonction définie
par~ : $ \left\{
\begin{array}{ll}
g_{\lambda}(x) = 0 & \text{si} \; x<\alpha \\
g_{\lambda}(x) =  \frac{\lambda}{e^{2x}-1} & \text{si} \;
x\geq\alpha
\end{array}
\right.$.

\begin{noliste}{a)}
 \setlength{\itemsep}{2mm}
\item On pose $h(x) = f(x)-x$. Après avoir calculer $h^{\prime }(x)$,
déterminer $\lambda$ en fonction de $\alpha$ pour que $g_{\lambda}$
soit
une densité de probabilité d'une certaine variable aléatoire $X$.

\item Donner la fonction de répartition $G_{\lambda}$ de $X$.
\end{noliste}
\end{noliste}

%-------------------------------------------------------------

\paragraph{ Exercice 2}

\hfill\\

Soit la matrice $K = 
\begin{smatrix}
0 & 0 & 1 \\
0 & 1 & 0 \\
1 & 0 & 0
\end{smatrix}
$.

On note $E$ l'ensemble des matrices $M$ de $\M{3}$
vérifiant~ : $MK = KM = M$.

\begin{noliste}{1.}
 \setlength{\itemsep}{4mm}
\item 
\begin{noliste}{a)}
 \setlength{\itemsep}{2mm}
\item Montrer que $E$ est un espace vectoriel.

\item Montrer par l'absurde qu'aucune matrice de $E$ n'est inversible.
\end{noliste}

\item Soit $M = \begin{matrix}
a & b & c \\
d & e & f \\
g & h & k\end{matrix}$ une matrice de $E$.

\begin{noliste}{a)}
 \setlength{\itemsep}{2mm}
\item Montrer que $k = g = c = a$, $h = b$ et $f = d$, puis en déduire
la forme des
matrices de $E$.

\item Retrouver le fait que les matrices de $E$ ne sont pas
inversibles.

\item Déterminer une base de $E$ et vérifier que $\mathrm{dim} E = 4$.
\end{noliste}

\item On considère l'ensemble $F$ des mtrices de la forme $M =
\begin{matrix}
x & y & x \\
y & z & y \\
x & y & x\end{matrix}$ où $x$, $y$ et $z$ sont des réels.

\begin{noliste}{a)}
 \setlength{\itemsep}{2mm}
\item Vérifier que $F$ est un sous-espace vectoriel de $E$ et donner
une
base de $F$.

\item Les matrices de $F$ sont-elles diagonalisables~ ?

\item Dans cette question on appelle $U$ la matrice de $F$ telle que~ :
$x = 3$, $y = 2$ et $z = 4$.

Trouver les valeurs propres de $U$ et exhiber un vecteur colonne propre
pour
chacune d'entre elles.
\end{noliste}

\item On note $\varphi$ l'application de $F$ dans $\R$ qui à toute
matrice $A$ de $F$ associe le nombre $\Sum{i = 1}{3}\Sum{j =
1}{3}(-1)^{i + j}a_{i,j}$, où $a_{i,j}$ désigne
l'élément de la matrice $A$ situé à l'intersection de la
$i^{\grave{e}me}$ ligne et de la $j^{i\grave{e}me}$ colonne.

\begin{noliste}{a)}
 \setlength{\itemsep}{2mm}
\item Montrer que $\varphi$ est une application linéaire de $F$ dans
$\R$.

\item Déterminer $\mathrm{Im}\, \varphi$. En déduire que
$\mathrm{Ker}\,
\varphi$ est de dimension $2$.

\item Soit $M = \begin{matrix}
x & y & x \\
y & z & y \\
x & y & x\end{matrix}$ une matrice de $\mathrm{Ker}\, \varphi$.

Exprimer $\varphi (M)$ en fonction de $x$, $y$ et $z$ et en déduire une
base de $\mathrm{Ker}\, \varphi$.
\end{noliste}
\end{noliste}

\vspace{0.5cm} 
%-------------------------------------------------------------

\paragraph{ Exercice 3}

\hfill\\

Pour tout entier $n$ supérieur ou égal à $1$, on définit la fonction 
$f_{n}$ par~ : 
\[
\forall x \in \R_+,\;\; f_{n}(x) = x^{n} + 9x^{2}-4.
\]

\begin{noliste}{1.}
 \setlength{\itemsep}{4mm}
\item 
\begin{noliste}{a)}
 \setlength{\itemsep}{2mm}
\item Montrer que l'équation $f_{n}(x) = 0$ n'a qu'une seule solution
strictement positive, notée $u_{n}$.

\item Calculer $u_{1}$ et $u_{2}$.

\item Vérifier que~ : $\forall n \in \N^*,\; \in]0,\frac{2}{3}[$.
\end{noliste}

\item 
\begin{noliste}{a)}
 \setlength{\itemsep}{2mm}
\item Montrer que, pour tout $x$ élément de $]0,1[$, on a~ : $f_{n +
1}(x)<f_{n}(x)$.

\item En déduire le signe de $f_{n}(u_{n + 1})$, puis les variations de
la
suite $(u_{n})$.

\item Montrer que la suite $(u_{n})$ est convergente. On note $\ell $
sa
limite.
\end{noliste}

\item 
\begin{noliste}{a)}
 \setlength{\itemsep}{2mm}
\item Déterminer la limite de $(u_{n})^{n}$ lorsque $n$ tend vers $ +
\infty$.

\item Donner enfin la valeur de $\ell $.
\end{noliste}

\item Montrer que la série de terme général $\frac{2}{3}-u_{n}$ est
convergente.
\end{noliste}

\vspace{0.5cm} 
%-------------------------------------------------------------

\paragraph{ Problème}

\hfill\\

On lance indéfiniment une pièce donnant \textit{``Pile''} avec la
probabilité $p$ et \textit{``Face''} avec la probabilité $q = 1-p$. On
suppose que $p\in]0,1[$ et on admet que les lancers sont mutuellement
indépendants.

Pour tout entier naturel $k$, supérieur ou égal à $2$, on dit que le
$k^{i\grave{e}me}$ lancer est un changement s'il amène un résultat
différent de celui du $(k-1)^{i\grave{e}me}$ lancer.

On note $P_{k}$ (resp. $F_{k}$) l'événement~ : \textit{``on obtient
Pile
(resp. Face) au $k^{i\grave{e}me}$ lancer''}.

Pour ne pas surcharger l'écriture on écrira, par exemple, $P_{1}F_{2}$
à
la place de $P_{1}\cap F_{2}$.

Pour tout entier naturel $n$ supérieur ou égal à $2$, on note $X_{n}$
la
variable aléatoire égale au nombre de changements survenus durant les
$n$
premiers lancers.

\paragraph{Partie 1~ : étude de quelques exemples.}

\begin{noliste}{1.}
 \setlength{\itemsep}{4mm}
\item Donner la loi de $X_{2}$.

\item 
\begin{noliste}{a)}
 \setlength{\itemsep}{2mm}
\item Donner la loi de $X_{3}$.

\item Vérifier que $\E(X_{3}) = 4pq$ et que $\V(X_{3}) = 2pq(3-8pq)$.
\end{noliste}

\item 
\begin{noliste}{a)}
 \setlength{\itemsep}{2mm}
\item Trouver la loi de $X_{4}$.

\item Calculer $\E(X_{4})$.
\end{noliste}
\end{noliste}

\paragraph{Partie 2~ : étude du cas $p\neq q$.}

Dans cette partie, $n$ désigne un entier naturel supérieur ou égal à 
$2$.

\begin{noliste}{1.}
 \setlength{\itemsep}{4mm}
\item Exprimer $P\left(\Ev{X_{n} = 0}\right)$ en fonction de $p$, $q$
et $n$.

\item En décomposant l'événement $(X_{n} = 1)$ en une réunion
d'événements incompatibles, montrer que \\
$ P\left(\Ev{X_{n} = 1}\right) = \frac{2pq}{q-p}\left(
q^{n-1}-p^{n-1}\right) $.

\item En distinguant les cas $n$ pair et $n$ impair, exprimer
$P\left(\Ev{X_{n} = n-1}\right)$
en fonction de $p$ et $q$.

\item Retrouver, gr\^{a}ce aux trois questions précédentes, les lois de
$X_{3}$ et $X_{4}$.

\item Pour tout entier naturel $k$, supérieur ou égal à $2$, on note
$Z_{k}$ la variable aléatoire qui vaut $1$ si le $k^{i\grave{e}me}$
lancer
est un changement et $0$ sinon ($Z_{k}$ est donc une variable de
Bernouilli).

Écrire $X_{n}$ à l'aide de certaines des variables $Z_{k}$ et en
déduire $\E(X_{n})$.
\end{noliste}

\paragraph{Partie 3~ : étude du cas $p = q$.}

\begin{noliste}{1.}
 \setlength{\itemsep}{4mm}
\item Vérifier, en utilisant les résultats de la partie 1, que $X_{3}$
et $X_{4}$ suivent chacune une loi bin\^{o}miale.

\item Montrer que, pour tout entier naturel $n$ supérieur ou égal à
$2$, $X_{n}$ suit une loi bin\^{o}miale dont on donnera les paramètres.
\end{noliste}

\end{document}

