\documentclass[11pt]{article}%
\usepackage{geometry}%
\geometry{a4paper,
 lmargin = 2cm,rmargin = 2cm,tmargin = 2.5cm,bmargin = 2.5cm}

\input{../../macros.tex}

\pagestyle{fancy} %
\lhead{ECE2 \hfill Mathématiques\\
} %
\chead{\hrule} %
\rhead{} %
\lfoot{} %
\cfoot{} %
\rfoot{\thepage} %

\renewcommand{\headrulewidth}{0pt}% : Trace un trait de séparation
 % de largeur 0,4 point. Mettre 0pt
 % pour supprimer le trait.

\renewcommand{\footrulewidth}{0.4pt}% : Trace un trait de séparation
 % de largeur 0,4 point. Mettre 0pt
 % pour supprimer le trait.

\setlength{\headheight}{14pt}

\title{\bf \vspace{-2cm} EDHEC 2008} %
\author{} %
\date{} %
\begin{document}

\maketitle %
\vspace{-1.4cm}\hrule %
\thispagestyle{fancy}

\vspace*{.2cm}


% DEBUT DU DOC À MODIFIER : tout virer jusqu'au début de l'exo

%Définition et changement de valeurs de
compteurs%newcounter{cpt1}{section} compteur cpt1 remis à 0 à chaque
aumentation par stepcounter du compteur section%setcounter{cpt1}{3} on
met le compteur à 3%addtocounter{cpt1}{5} on ajoute 5 au compteur%
stepcounter{cpt1} on ajoute 1% ifthenelse{test}{alors}{sinon} (page
206) pour subordonner à une condition % whiledo{test}{commande} pour
faire une boucle (page 206 aussi) % value{cpt1} pour noter dans le
document la valeur de cpt1 
%Définition définitive d'opérateurs
mathématiques\newcommand{\ch}{\operatorname{ch}} 
\newcommand{\sh}{\operatorname{sh}}
\renewcommand{\tanh}{\operatorname{th}}
\renewcommand{\sinh}{\operatorname{sh}}
\renewcommand{\cosh}{\operatorname{ch}}
\newcommand{\argsh}{\operatorname{argsh}}
\newcommand{\argch}{\operatorname{argch}}
\newcommand{\argth}{\operatorname{argth}}
\newcommand{\ker}{\operatorname{Ker}}
\renewcommand{\im}{\operatorname{Im}}
\newcommand{\rg}{\operatorname{rg}}
\newcommand{\Id}{\operatorname{Id}}
\newcommand{\id}{\operatorname{id}}
\renewcommand{\leq}{\leq}
\renewcommand{\geq}{\geq }

%Définition de nouvelles couleurs : rgb(trois paramètres red green blue
entre 0 et 1); cmyk (quatre cyan magenta yellow black) entre 0 et 1;
gray (entre 0 et 1) et black, white, red, green, blue, cyan, magenta,
yellow% definecolor{0gris}{gray}{0.8} 
% Nouvelle commande pour encadrer le titre car shabox ne veut que d'une
seule ligne; ATTENTION A LA TAILLE; petite différence avec shadowbox ou
doublebox, voire fcolorbox ou colorbox (au lieu de shabox; laisser le
parbox tranquille sauf pour la taille de la boîte
\newcommand{\Tbox}[1]{\begin{center} \shabox{\parbox{0.6
\linewidth}{#1}} \end{center}} %[1] pour 1 paramètre ; #1 pour ce que
fait le 1er paramètre; entre accolades ce que fait la commande
%Mise en page en mode fancy : en-têtes et pieds de pages puis
définition des en-têtes et pieds de pages\pagestyle{fancy}
\lhead{ECE 2 - Mathématiques \\
Quentin Dunstetter - ENC-Bessières 2011$\backslash$2012}
\chead{}
\rhead{Edhec 2008}
\rfoot[ \ \thepage]{\thepage}
\cfoot{}
\lfoot{}
\thispagestyle{fancy} %Mise en page de la 1ère page en mode fancy
%Trait en bas et en haut de la page (entre en-tête et texte et texte et
pied de page)\renewcommand{\footrulewidth}{0.4pt}
\renewcommand{\headrulewidth}{0.4pt}


%DEBUT DU DOCUMENT\vspace*{3cm}

\begin{center}
{\LARG\E\textbf{BANQUE COMMUNE D'ÉPREUVES}}



{\large \textsc{CONCOURS D ADMISSION DE 2008}}



{\large \textbf{Concepteur : Edhec}}



\rule{2.39cm}{0.05cm}



{\Large \textbf{OPTION ÉCONOMIQUE}}



{\Large \textbf{MATHÉMATIQUES }}



{\Large Lundi 9 mai, de 14h à 18h}



\rule{2.39cm}{0.05cm}
\end{center}

\textit{La présentation, la lisibilité, l'orthographe, la qualité
de la rédaction, la clarté et la précision des raisonnements
entreront pour une part importante dans l'appréciation des copies.}

\textit{Les candidats sont invités à \textbf{encadrer} dans la mesure
du possible les résultats de leurs calculs.}

\textit{Ils ne doivent faire usage d'aucun document. L'utilisation de
toute
calculatrice et de tout matériel électronique est interdite. Seule
l'utilisation d'une règle graduée est autorisée.}

\textit{Si au cours de l'épreuve, un candidat repère ce qui lui semble
être une erreur d'énoncé, il la signalera sur sa copie et
poursuivra sa composition en expliquant les raisons des initiatives
qu'il sera
amené à prendre.}

\vspace*{3cm}

\section*{Exercice 1}

Pour tout entier naturel $n$ non nul, on définit la fonction $f$,, par
: $\forall x\in \R,\ f_{n}\left( x\right) = \dfrac{1}{1 + e^{x}} +
n~x$.

On appelle $\left( C_{n}\right) $ sa courbe représentative dans un
repère orthonormé $\left( O,\vec{i},\vec{j}\right) $ d'unité 5 cm.

\begin{noliste}{1.}
 \setlength{\itemsep}{4mm}
\item
\begin{noliste}{a)}
 \setlength{\itemsep}{2mm}
\item Déterminer, pour tout réel $x$, $f_{n}{\prime }\left(
x\right) $ et $f^{\prime \prime }\left( x\right) $.

\item En déduire que la fonction $f_{n}$ est strictement croissante sur
$\R$
\end{noliste}

\item
\begin{noliste}{a)}
 \setlength{\itemsep}{2mm}
\item Calculer $\dlim{x\rightarrow -\infty }f_{n}\left( x\right) $
ainsi que
$\dlim{x\rightarrow + \infty }f_{n}\left( x\right) $.

\item Montrer que les droites $\left( D_{n}\right) $ et $\left(
D_{n}{\prime }\right) $ d'équations $y = nx$ et $y = nx + 1$ sont
asymptotes
de $\left( C_{n}\right) $

\item Déterminer les coordonnées du seul point d'inflexion, noté
$A_{n}$ de $\left( C_{n}\right) $.

\item Donner l'équation de la tangente $\left( T_{1}\right) $ à la
courbe $\left( C_{1}\right) $ en $A_{1}$, puis tracer sur un même
dessin
les droites $\left( D_{1}\right) $, $\left( D_{1}{\prime }\right) $ et
$\left( T_{1}\right) $ ainsi que l'allure de la courbe $\left(
C_{1}\right) $.
\end{noliste}

\item
\begin{noliste}{a)}
 \setlength{\itemsep}{2mm}
\item Montrer que l'équation $f_{n}\left( x\right) = 0$ possède une
seule solution sur $\R$, notée $u_{n}$.

\item Montrer que l'on a : $\forall n\in \N^{\ast },\quad
\dfrac{-1}{n}<u_{n}<0$.

\item En déduire la limite de la suite $\left( u_{n}\right) $

\item En revenant à la définition de $u_{n}$, montrer que
$u_{n}\underset{n\rightarrow + \infty }{\thicksim }\dfrac{-1}{2n}.$
\end{noliste}
\end{noliste}

\section*{Exercice 2}

On considère un endomorphisme $f$ de $\R^{3}$ dont la matrice
dans la base canonique $\mathcal{B}$ de $\R^{3}$ est la matrice $A = 
\begin{smatrix}
6 & 10 & 11 \\
2 & 6 & 5 \\
-4 & -8 & -8
\end{smatrix}
$.

\begin{noliste}{1.}
 \setlength{\itemsep}{4mm}
\item
\begin{noliste}{a)}
 \setlength{\itemsep}{2mm}
\item Déterminer la matrice $A\left( A-2I\right) ^{2}$ et en déduire
les seules valeurs propres possibles de $f$ :

\item On considère les vecteurs $u = \left( 2,1,-2\right) $ et $v =
\left(
3,1,-2\right) $.\\
Déterminer $f\left( u\right) $ et $f\left( v\right) $ puis en déduire
les valeurs propres de $f$ :

\item L'endomorphisme $f$ est-il un automorphisme de $\R^{3}$ ?
\end{noliste}

\item On considère le vecteur $w = \left( -2,0,1\right) $

\begin{noliste}{a)}
 \setlength{\itemsep}{2mm}
\item Montrer que $\left( u,v,w\right) $ est une base de $\R^{3}$.

\item Exprimer $f\left( w\right) $ comme combinaison linéaire de $v$ et
$w$ puis vérifier que la matrice de $f$ dans la base $\left(
u,v,w\right)
$ est $T = 
\begin{smatrix}
0 & 0 & 0 \\
0 & 2 & 1 \\
0 & 0 & 2
\end{smatrix}
$

\item Montrer que $f$ n'est pas diagonalisable.
\end{noliste}

\item
\begin{noliste}{a)}
 \setlength{\itemsep}{2mm}
\item On pose $T = D + N$, où $D = 
\begin{smatrix}
0 & 0 & 0 \\
0 & 2 & 0 \\
0 & 0 & 2
\end{smatrix}
$ et $N = 
\begin{smatrix}
0 & 0 & 0 \\
0 & 0 & 1 \\
0 & 0 & 0
\end{smatrix}
$

Déterminer $N^{2}$ puis utiliser la formule du bin\^{o}me pour montrer
que, pour tout entier naturel $n$ non nul, on a $T^{n} = D^{n} +
nD^{n-1}N.$

\item Donner explicitement, pour tout entier naturel $n$ non nul, la
matrice
$T^{n}$ en fonction de $n$

\item Proposer une matrice $P$ telle que $A = PTP^{-1}$ puis déterminer
$P^{-1}$.

\item Montrer que, pour tout entier naturel $n$ non nul, on a : $A^{n}
= PT^{n}P^{-1}$

\item Déterminer explicitement $A^{n}$ pour tout entier $n$ supérieur
ou égal à 1.
\end{noliste}
\end{noliste}

\section*{Exercice 3}

\begin{noliste}{1.}
 \setlength{\itemsep}{4mm}
\item Montrer que l'intégrale $\dint{0}{+ \infty }\dfrac{1}{\left(
1 + x\right) ^{2}}dx$ est convergente et donner sa valeur

\item On considère la fonction $f$ définie par : $\forall x\in
\R,\ f\left( x\right) = \dfrac{1}{2\left( 1 + \left| x\right|
\right) ^{2}}$.

\begin{noliste}{a)}
 \setlength{\itemsep}{2mm}
\item Montrer que $f$ est paire.

\item Montrer que $f$ peut être considérée comme une fonction
densité de probabilité.
\end{noliste}

\hspace{-1cm}Dans la suite, on considère une variable aléatoire $X$,
définie sur un espace probabilisé $\left(
\Omega,\mathcal{A},\Prob\right) $ admettant $f$ comme densité.\\
\hspace{-1cm}On note $F$ la fonction de répartition de $X$.

\item On pose $Y = \ln \left( 1 + \left| X\right| \right) $ et on
admet que $Y$ est une variable aléatoire à densité, elle aussi définie
sur l'espace probabilisé $\left( \Omega,\mathcal{A},\Prob\right) $

\begin{noliste}{a)}
 \setlength{\itemsep}{2mm}
\item Déterminer $Y\left( \Omega \right) $

\item Exprimer la fonction de répartition $G$ de $Y$ à l'aide de $F.$

\item En déduire que $Y$ admet pour densité la fonction $g$ définie par
:
\[
g\left( x\right) = \left\{
\begin{array}{cc}
2e^{x}f\left( e^{x}-1\right) & \text{si }x\geq 0 \\
0 & \text{si }x<0
\end{array}
\right.
\]

\item Montrer enfin que $Y$ suit une loi exponentielle dont on
déterminera le paramètre.
\end{noliste}
\end{noliste}

\section*{Problème}

\subsection*{Partie 1 : préliminaires}

\begin{noliste}{1.}
 \setlength{\itemsep}{4mm}
\item Soit $f$ une fonction de classe $C^{1}$ sur $\left[ 0,1\right] $.
On
se propose, dans cette question, de démontrer un résultat classique
sur les sommes de Riemann associées à cette fonction.

\begin{noliste}{a)}
 \setlength{\itemsep}{2mm}
\item Montrer qu'il existe un réel $M$ strictement positif tel que,
pour
tout couple $\left( x,y\right) $ d'éléments de $\left[ 0,1\right] $
on a : $\left| f\left( x\right) -f\left( y\right) \right| \leq
M\left| x-y\right| $

\item En déduire que : $ \forall n\in \N^{\ast },\
\forall k\in \left[ \ \left[ 0,n-1\right] \right],\ \forall t\in \left[
\ \frac{k}{n},\frac{k + 1}{n}\right],\ \left| f\left( t\right) -f\left(
\frac{k}{n}\right) \right| \leq M\left( t-\frac{k}{n}\right) $

\item Montrer alors que : $ \forall n\in \N^{\ast },\
\forall k\in \left[ \ \left[ 0,n-1\right] \right],\ $
\[
\left| \dint{k/n}{\left( k + 1\right) /n}f\left( t\right\
dt-\frac{1}{n}f\left( \frac{k}{n}\right) \right| \leq \frac{M}{2n^{2}}
\]

\item En sommant la relation précédente, établir que : $ \forall n\in
\N^{\ast }$, $\left|
\dint{0}{1}f\left( t\right\ dt-\frac{1}{n}\Sum{k = 0}{n-1}f\left(
\frac{k}{n}\right) \right| \leq \frac{M}{2n}$

\item Conclure finalement que $ \frac{1}{n}\Sum{k = 0}{n-1}f\left(
\frac{k}{n}\right) \rightarrow \dint{0}{1}f\left( t\right\ dt.$
\end{noliste}

\item Pour tout couple $\left( p,q\right) $ d'entiers naturels, on pose
$I\left( p,q\right) = \dint{0}{1}x^{p}\left( 1-x\right) ^{q}dx$

\begin{noliste}{a)}
 \setlength{\itemsep}{2mm}
\item Montrer que : $ \forall \left( p,q\right) \in \N\times \N,\
I\left( p,q\right) = \frac{q}{p + 1}I\left( p + 1,q-1\right). $

\item En déduire que : $ \forall \left( p,q\right) \in
\N\times \N,\ I\left( p,q\right) = \frac{p!q!}{\left(
p + q\right) !}I\left( p + q,0\right).$

\item Déterminer$I\left( p + q,0\right) $ et montrer finalement que : $
\forall \left( p,q\right) \in \N\times \N^{\ast },\ I\left( p,q\right)
= \frac{p!q!}{\left( p + q + 1\right) !}.$
\end{noliste}

\item Informatique.

Compléter la déclaration récursive suivante afin qu'elle
permette le calcul de $I\left( p,q\right) :$

\texttt{Function i(p, q : integer ) : real ;\\
Begin\\
If q = 0 then i : =...... else...... ; \\
End ;\\
}
\end{noliste}

\subsection*{Partie 2 : étude d'une suite de variables aléatoires}

Dans cette partie, $m$ est un entier naturel fixé, supérieur ou égal à
2.

On considère une suite de variables aléatoires $\left( U_{n}\right)
_{n\geq 1}$, toutes définies sur le même espace probabilisé $\left(
\Omega,\mathcal{A},\Prob\right) $, telles que, pour tout entier
naturel $n$ supérieur ou égal à $1$, $U_{n}$ suit la loi
uniforme sur $ \left\{
0,\frac{1}{n},\frac{2}{n},\cdots,\frac{n-1}{n}\right\} $

On considère également une suite de variables aléatoires $\left(
X_{n}\right)_{n\geq 1}$ définies elles aussi sur $\left(
\Omega,\mathcal{A},\Prob\right) $, et telles que, pour tout entier
naturel $n$
supérieur ou égal à $1$, et pour tout $k$ de $\left[ \ \left[
0,n-1\right] \right],$ la loi de $X_{n}$ conditionnellement à
l'événement $ \left( U_{n} = \frac{k}{n}\right) $ est la loi binomiale
$ \mathcal{B}\left( m,\frac{k}{n}\right) $.

\begin{noliste}{1.}
 \setlength{\itemsep}{4mm}
\item On considère une variable aléatoire $Y$ suivant la loi
binomiale $\mathcal{B}\left( m,p\right) $. Rappeler la valeur de
l'espérance de $Y$ puis montrer que $\E\left( Y\left( Y-1\right)
\right) = m\left(
m-1\right) p^{2}$

\item Donner la loi de $X_{1}$

\hspace{-1cm}Dans toute la suite, on suppose $n$ supérieur ou égal
à 2.

\item
\begin{noliste}{a)}
 \setlength{\itemsep}{2mm}
\item Déterminer $X_{n}\left( \Omega \right) $, puis montrer que, pour
tout $i$ de $X_{n}\left( \Omega \right) $, on a :
\[
\Prob\left(\Ev{ X_{n} = i}\right) = \frac{1}{n}\binom{m}{i}\Sum{k =
0}{n-1}\left( \frac{k}{n}\right) ^{i}\left( 1-\frac{k}{n}\right) ^{m-i}
\]

\item Utiliser la première question de cette partie pour donner sans
calcul la valeur de la somme $ \Sum{i = 1}{m}i\binom{m}{i}\left(
\frac{k}{n}\right) ^{i}\left( 1-\frac{k}{n}\right) ^{m-i}.\\
$Montrer alors que l'espérance de $X_{n}$ est égale à $ \dfrac{m\left(
n-1\right) }{2n}$

\item En utilisant toujours la première question de cette partie,
donner
sans calcul la valeur de la somme $ \Sum{i = 1}{m}i\left(
i-1\right) \binom{m}{i}\left( \frac{k}{n}\right) ^{i}\left(
1-\frac{k}{n}\right) ^{m-i}$\\
Montrer alors que l'espérance de $X_{n}\left( X_{n}-1\right) $ est
égale à $\dfrac{m\left( m-1\right) \left( n-1\right) \left( 2n-1\right)
}{6n^{2}}.$

\item En déduire finalement que la variance de $X_{n}$ est égale
à $ \dfrac{m\left( m + 2\right) \left( n^{2}-1\right) }{12n^{2}}$
\end{noliste}

\item
\begin{noliste}{a)}
 \setlength{\itemsep}{2mm}
\item En utilisant les résultats obtenus aux deux premières
questions de la première partie, calculer, pour tout i de $X_{n}\left(
\Omega \right) $,\hspace{5mm} $\lim \limits_{n\rightarrow + \infty
}\Prob\left(\Ev{ X_{n} = i}\right) $.

\item En déduire que la suite $\left( X_{n}\right) $ converge en loi
vers une variable aléatoire $X$ dont on précisera la loi.

\item Vérifier que\hspace{5mm} $\lim \limits_{n\rightarrow + \infty
}\E\left( X_{n}\right) = E\left( X\right) $ et $\lim
\limits_{n\rightarrow
 + \infty }\V(X_{n}) = V(X).$
\end{noliste}
\end{noliste}

\end{document}

