\documentclass[11pt]{article}%
\usepackage{geometry}%
\geometry{a4paper,
 lmargin = 2cm,rmargin = 2cm,tmargin = 2.5cm,bmargin = 2.5cm}

\input{../../macros.tex}

\pagestyle{fancy} %
\lhead{ECE2 \hfill Mathématiques\\
} %
\chead{\hrule} %
\rhead{} %
\lfoot{} %
\cfoot{} %
\rfoot{\thepage} %

\renewcommand{\headrulewidth}{0pt}% : Trace un trait de séparation
 % de largeur 0,4 point. Mettre 0pt
 % pour supprimer le trait.

\renewcommand{\footrulewidth}{0.4pt}% : Trace un trait de séparation
 % de largeur 0,4 point. Mettre 0pt
 % pour supprimer le trait.

\setlength{\headheight}{14pt}

\title{\bf \vspace{-2cm} EDHEC 2011} %
\author{} %
\date{} %
\begin{document}

\maketitle %
\vspace{-1.4cm}\hrule %
\thispagestyle{fancy}

\vspace*{.2cm}


% DEBUT DU DOC À MODIFIER : tout virer jusqu'au début de l'exo

%Définition et changement de valeurs de
compteurs%newcounter{cpt1}{section} compteur cpt1 remis à 0 à chaque
aumentation par stepcounter du compteur section%setcounter{cpt1}{3} on
met le compteur à 3%addtocounter{cpt1}{5} on ajoute 5 au compteur%
stepcounter{cpt1} on ajoute 1% ifthenelse{test}{alors}{sinon} (page
206) pour subordonner à une condition % whiledo{test}{commande} pour
faire une boucle (page 206 aussi) % value{cpt1} pour noter dans le
document la valeur de cpt1 
%Définition définitive d'opérateurs
mathématiques\newcommand{\ch}{\operatorname{ch}} 
\newcommand{\sh}{\operatorname{sh}}
\renewcommand{\tanh}{\operatorname{th}}
\renewcommand{\sinh}{\operatorname{sh}}
\renewcommand{\cosh}{\operatorname{ch}}
\newcommand{\argsh}{\operatorname{argsh}}
\newcommand{\argch}{\operatorname{argch}}
\newcommand{\argth}{\operatorname{argth}}
\newcommand{\ker}{\operatorname{Ker}}
\renewcommand{\im}{\operatorname{Im}}
\newcommand{\rg}{\operatorname{rg}}
\newcommand{\Id}{\operatorname{Id}}
\newcommand{\id}{\operatorname{id}}
\renewcommand{\leq}{\leq}
\renewcommand{\geq}{\geq }

%Définition de nouvelles couleurs : rgb(trois paramètres red green blue
entre 0 et 1); cmyk (quatre cyan magenta yellow black) entre 0 et 1;
gray (entre 0 et 1) et black, white, red, green, blue, cyan, magenta,
yellow% definecolor{0gris}{gray}{0.8} 
% Nouvelle commande pour encadrer le titre car shabox ne veut que d'une
seule ligne; ATTENTION A LA TAILLE; petite différence avec shadowbox ou
doublebox, voire fcolorbox ou colorbox (au lieu de shabox; laisser le
parbox tranquille sauf pour la taille de la boîte
\newcommand{\Tbox}[1]{\begin{center} \shabox{\parbox{0.6
\linewidth}{#1}} \end{center}} %[1] pour 1 paramètre ; #1 pour ce que
fait le 1er paramètre; entre accolades ce que fait la commande
%Mise en page en mode fancy : en-têtes et pieds de pages puis
définition des en-têtes et pieds de pages\pagestyle{fancy}
\lhead{ECE 2 - Mathématiques \\
Quentin Dunstetter - ENC-Bessières 2011$\backslash$2012}
\chead{}
\rhead{Edhec 2011}
\rfoot[ \ \thepage]{\thepage}
\cfoot{}
\lfoot{}
\thispagestyle{fancy} %Mise en page de la 1ère page en mode fancy
%Trait en bas et en haut de la page (entre en-tête et texte et texte et
pied de page)\renewcommand{\footrulewidth}{0.4pt}
\renewcommand{\headrulewidth}{0.4pt}


%DEBUT DU DOCUMENT\vspace*{3cm}

\begin{center}
{\LARG\E\textbf{BANQUE COMMUNE D'ÉPREUVES}}



{\large \textsc{CONCOURS D ADMISSION DE 2011}}



{\large \textbf{Concepteur : Edhec}}



\rule{2.39cm}{0.05cm}



{\Large \textbf{OPTION ÉCONOMIQUE}}



{\Large \textbf{MATHÉMATIQUES }}



{\Large Lundi 9 mai, de 14h à 18h}



\rule{2.39cm}{0.05cm}
\end{center}

\textit{La présentation, la lisibilité, l'orthographe, la qualité
de la rédaction, la clarté et la précision des raisonnements
entreront pour une part importante dans l'appréciation des copies.}

\textit{Les candidats sont invités à \textbf{encadrer} dans la mesure
du possible les résultats de leurs calculs.}

\textit{Ils ne doivent faire usage d'aucun document. L'utilisation de
toute
calculatrice et de tout matériel électronique est interdite. Seule
l'utilisation d'une règle graduée est autorisée.}

\textit{Si au cours de l'épreuve, un candidat repère ce qui lui semble
être une erreur d'énoncé, il la signalera sur sa copie et
poursuivra sa composition en expliquant les raisons des initiatives
qu'il sera
amené à prendre.}

\vspace*{3cm}

\section*{Exercice 1}

On considère la fonction $f$ définie sur $\R^{+}$ par :
$ = \dfrac{2}{x^{2}}\dint{0}{x} \dfrac{t}{e^{t} + 1}dt}$ si
$x>0$ et $f(0) = \dfrac{1}{2}.$

\begin{noliste}{1.}
 \setlength{\itemsep}{4mm}
\item
\begin{noliste}{a)}
 \setlength{\itemsep}{2mm}
\item Montrer que : $\forall x\in]0, + \infty\lbrack,$ $\forall
t\in\left[
0,x\right],$ $\dfrac{1}{e^{x} + 1}\leq\dfrac{1}{e^{t} +
1}\leq\dfrac{1}{2}.$

\item Établir alors que, pour tout réel $x$ strictement positif, on a :
$\dfrac{1}{e^{x} + 1}\leq f(x)\leq\dfrac{1}{2}.$

\item En déduire que la fonction $f$ est continue (à droite) en 0.
\end{noliste}

\item
\begin{noliste}{a)}
 \setlength{\itemsep}{2mm}
\item Montrer que $f$ est de classe $C^{1}$ sur $]0, + \infty\lbrack,$
puis
vérifier que, pour tout réel $x$ strictement positif, on peut écrire :
$f^{\prime}(x) = -\dfrac{4}{x^{3}}g(x),$ où $g$ est une fonction que
l'on déterminera.

\item Étudier les variations, puis le signe de la fonction $g.$ En
déduire que
$f$ est décroissante sur $\R^{+}.$
\end{noliste}

\item
\begin{noliste}{a)}
 \setlength{\itemsep}{2mm}
\item Montrer que, pour tout réel $t$ positif, on a : $\dfrac{t}{e^{t}
+ 1}\leq1.$

\item En déduire la limite de $f(x)$ lorsque $x$ tend vers $ + \infty.$
\end{noliste}
\end{noliste}

\section*{Exercice 2}

On désigne par $E$ l'espace vectoriel des fonctions polynômiales de
degré
inférieur ou égal à 2 et on note $\mathcal{B}$ la base $\left(
e_{0},e_{1},e_{2}\right) $ de $E,$ où pour tout réel $x,$ on a :
$e_{0}\left(
x\right) = 1,$ $e_{1}\left( x\right) = x$ et $e_{2}\left( x\right) =
x^{2}.$\\
 On considère l'application, notée $f,$ qui à toute fonction
polynômiale $P$ appartenant à $E,$ associe la fonction polynômiale
$f(P)$
définie par :
\[
\forall x\in\R,\ \left( f\left( P\right) \right) \left( x\right)
 = 2xP\left(\Ev{ x}\right) -\left( x^{2}-1\right) P^{\prime}\left(
x\right).
\]


\begin{noliste}{1.}
 \setlength{\itemsep}{4mm}
\item
\begin{noliste}{a)}
 \setlength{\itemsep}{2mm}
\item Montrer que $f$ est une application linéaire.

\item En écrivant, pour tout réel $x,$ $P\left(\Ev{x}\right) = a + bx +
cx^{2},$ définir
explicitement $(f(P))(x)$ puis en déduire que $f$ est un endomorphisme
de $E.
$

\item Écrire $f(e_{0}),$ $f(e_{1})$ et $f(e_{2})$ comme des
combinaisons
linéaires de $e_{0},$ $e_{1}$ et $e_{2},$ puis en déduire la matrice
$A$ de
$f$ dans la base $\mathcal{B}.$
\end{noliste}

\item
\begin{noliste}{a)}
 \setlength{\itemsep}{2mm}
\item Vérifier que $\operatorname*{Im}f = \operatorname{vect}\left(
e_{1},e_{0} + e_{2}\right) $ et donner la dimension de
$\operatorname{Im}f.$

\item Déterminer $\operatorname{Ker}f.$
\end{noliste}

\item
\begin{noliste}{a)}
 \setlength{\itemsep}{2mm}
\item À l'aide de la méthode du pivot de Gauss, déterminer les valeurs
propres
de $A.$

\item En déduire que $f$ est diagonalisable et donner les sous-espaces
propres
de $f.$

\item Vérifier que les sous-espaces propres de $f,$ autres que
$\operatorname*{Ker}f,$ sont inclus dans $\operatorname{Im}f.$
\end{noliste}
\end{noliste}

\section*{Exercice 3}

On désigne par $n$ un entier naturel supérieur ou égal à 2. On dispose
de $n$
urnes, numérotées de $1$ à $n,$ contenant chacune $n$ boules. On répète
$n$
épreuves, chacune consistant à choisir une urne au hasard et à en
extraire une
urne au hasard. On suppose que les choix des urnes sont indépendants
les uns
des autres.\\
 Pour tout $i$ de $\left\{ 1,2,...,n\right\},$ on note
$X_{i}$ la variable aléatoire prenant la valeur 1 si l'urne numérotée
$i$
contient toujours $n$ boules au bout de ces $n$ épreuves, et qui prend
la
valeur 0 sinon.

\begin{noliste}{1.}
 \setlength{\itemsep}{4mm}
\item
\begin{noliste}{a)}
 \setlength{\itemsep}{2mm}
\item Pour tout $i$ et tout $k,$ éléments de $\{1,2,...,n\},$ on note
$U_{i,k}$ l'évènement "l'urne numéro $i$ est choisie à la
$k^{\text{ème}}$
épreuve".\\
 Écrire l'évènement $(X_{i} = 1)$ à l'aide de certains des
évènements $U_{i,k},$ puis montrer que :
\[
\forall i\in\left\{ 1,2,...,n\right\},\ P\left(\Ev{X_{i} = 1}\right) =
\left( 1-\dfrac{1}{n}\right) ^{n}.
\]


\item Justifier également que, si $i$ et $j$ sont deux entiers
distincts,
éléments de $\{1,2,...,n\},$ on a :
\[
P\left( \left[ X_{i} = 1\right] \cap\left[ X_{j} = 1\right] \right) =
\left(
1-\dfrac{2}{n}\right) ^{n}.
\]


\item Comparer $\left( 1-\dfrac{2}{n}\right) $ et $\left(
1-\dfrac{1}{n}\right) ^{2}$ et en déduire que, si $i$ et $j$ sont deux
entiers
distincts, éléments de $\left\{ 1,2,...,n\right\},$ alors les variables
$X_{i}$ et $X_{j}$ en sont pas indépendantes.
\end{noliste}

\item On pose $ = \Sum{k = 1}{n}X_{i}.}$

\begin{noliste}{a)}
 \setlength{\itemsep}{2mm}
\item Déterminer l'espérance de $Y_{n},$ notée $\E(Y_{n}).$

\item En déduire $\dlim{n\rightarrow + \infty}\dfrac{\E(Y_{n})}{n}}$ et
donner un équivalent de $\E(Y_{n})$ lorsque $n$ est au voisinage de
$ + \infty.$
\end{noliste}

\item Pour tout $i$ de $\left\{ 1,2,...,n\right\},$ on note $N_{i}$ la
variable aléatoire égale au nombre de boules manquantes dans l'urne
numérotée
$i$ à la fin de ces $n$ épreuves.

\begin{noliste}{a)}
 \setlength{\itemsep}{2mm}
\item Donner sans calcul la loi de $N_{i}$ ainsi que la valeur de
$\E(N_{i}).$

\item Que vaut le produit $N_{i}X_{i}$ ?

\item Les variables $N_{i}$ et $X_{i}$ sont-elles indépendantes ?
\end{noliste}

\item Compléter le programme informatique suivant pour qu'il simule
l'expérience décrite au début de cet exercice et affiche les valeurs
prises
par $X_{1}$ et $N_{1}$ pour une valeur de $n$ entrée par l'utilisateur.

\texttt{Program\ edhec\_{2}011;}

\texttt{Var\ x1,\ n1,\ n,\ k,\ tirage,\ hasard\ :\ integer;}

\texttt{Begin}

\texttt{\hspace*{1cm}Randomize;}

\texttt{\hspace*{1cm}Writeln('donnez\ un\ entier\ naturel\ n\
supérieur\ ou\ égal\ à\ 2');}

\texttt{\hspace*{1cm}Readln(n);}

\texttt{\hspace*{1cm}n1 : = 0;\ x1 : = 1;}

\texttt{\hspace*{1cm}For\ k : = 1\ to\ n\ do}

\texttt{\hspace*{1cm}begin}

\texttt{\hspace*{2cm}hasard\ : = \ random(n) + 1;}

\texttt{\hspace*{2cm}if\ hasard\ = \ 1\ then\ begin\ x1\ : = --------;\
n1 : = \ --------;\ end;}

\texttt{\hspace*{1cm}end;}

\texttt{\hspace*{1cm}Writeln(x1,n1);}

\texttt{End.}
\end{noliste}

\section*{Problème}

\textbf{Notations et objectifs}

On considère deux variables aléatoires $X$ et $Y,$ définies sur un
espace
probabilisé $\left( \Omega,\mathcal{A},P\right),$ et indépendantes.

On suppose que $X$ est une variable à densité et on note $F_{X}$ sa
fonction
de répartition.

On suppose par ailleurs que la loi de $Y$ est donnée par
$P\left(\Ev{Y = 1}\right) = P\left(\Ev{Y = -1}\right) = \dfrac{1}{2}.$

L'indépendance de $X$ et $Y$ se traduit par les égalités suivantes,
valables
pour tout réel $x$ :
\[
P\left( \left[ X\leq x\right] \cap\left[ Y = 1\right] \right) =
P\left(\Ev{
X\leq x}\right) P\left(\Ev{ Y = 1}\right) \quad\text{et }\quad P\left(
\left[
X\leq x\right] \cap\left[ Y = -1\right] \right) = P\left(\Ev{ X\leq
x}\right)
P\left(\Ev{ Y = -1}\right).
\]
On pose $Z = XY$ et on admet que $Z$ est, elle-aussi, une variable
aléatoire
définie sur $(\Omega,\mathcal{A},P).$

On se propose d'établir deux résultats utiles pour la suite dans la
partie 1,
puis d'en déduire la loi de la variable aléatoire $Z$ en fonction de la
loi de
$X$ dans les parties 2 et 3.

\subsection*{Partie 1 : expression de la fonction de répartition de $Z$
en
fonction de celle de $X.$}

\begin{noliste}{1.}
 \setlength{\itemsep}{4mm}
\item Rappeler l'expression des fonctions de répartition d'une variable
aléatoire suivant une loi uniforme sur $[a,b]$ (avec $a<b$) et d'une
variable
aléatoire suivant une loi exponentielle de paramètre $\lambda$ (avec
$\lambda>
0$).

\item En utilisant le système complet d'évènements $\{(Y = 1),(Y =
-1)\},$ montrer
que la fonction de répartition $F_{Z}$ de la variable aléatoire $Z$ est
donnée
par :
\[
\forall x\in\R,\ F_{Z}(x) = \dfrac{1}{2}\left( F_{X}(x)-F_{X}(-x) +
1\right).
\]

\end{noliste}

\subsection*{Partie 2 : Étude de deux premiers exemples.}

\begin{noliste}{1.}
 \setlength{\itemsep}{4mm}
\item On suppose que la loi de $X$ est la loi normale centrée
réduite.\\
Reconnaître la loi de $Z.$

\item On suppose que la loi de $X$ est la loi uniforme sur $[0,1].$

\begin{noliste}{a)}
 \setlength{\itemsep}{2mm}
\item Déterminer l'expression de $F_{X}(-x)$ selon les valeurs prises
par $x.$

\item Déterminer $F_{Z}(x)$ pour tout réel $x,$ puis reconnaître la loi
de
$Z.$
\end{noliste}
\end{noliste}

\subsection*{Partie 3 : Étude du cas où la loi de $X$ est la loi
exponentielle
de paramètre 1.}

\begin{noliste}{1.}
 \setlength{\itemsep}{4mm}
\item
\begin{noliste}{a)}
 \setlength{\itemsep}{2mm}
\item Montrer que la fonction de répartition $F_{Z}$ de la variable
aléatoire
$Z$ est définie par :
\[
F_{Z}(x) = \left\{
\begin{array}
[c]{cl}1-\frac{1}{2}e^{-x} & \text{ si } x \geq0\\
\frac{1}{2}e^{x} & \text{ si } x < 0
\end{array}
\right.
\]


\item En déduire que $Z$ est une variable aléatoire à densité.

\item Établir alors qu'une densité de $Z$ est la fonction $f_{Z}$
définie pour
tout réel $x$ par :
\[
f_{Z}(x) = \dfrac{1}{2}e^{-|x|}.
\]


\item Donner la valeur de l'intégrale $\dint{0}{+ \infty}xe^{-x}dx.}$

\item Montrer que $f_{Z}$ est une fonction paire et en déduire
l'existence et
la valeur de $\E(Z).$
\end{noliste}

\item
\begin{noliste}{a)}
 \setlength{\itemsep}{2mm}
\item Donner la valeur de l'intégrale $\dint{0}{+
\infty}x^{2}e^{-x}dx.}$

\item En déduire l'existence et la valeur de $\E(Z^{2}),$ puis donner
la valeur
de la variance de $Z.$
\end{noliste}

\item
\begin{noliste}{a)}
 \setlength{\itemsep}{2mm}
\item Déterminer $\E(X)\E(Y)$ et comparer avec $\E(Z).$ Quel résultat
retrouve-t-on ainsi ?

\item Exprimer $Z^{2}$ en fonction de $X,$ puis en déduire de nouveau
la
variance de $Z.$
\end{noliste}

\item Soit $U$ et $V$ des variables aléatoires suivant respectivement
la loi
de Bernoulli de paramètre $\dfrac{1}{2}$ et la loi uniforme sur
$[0,1[.$

\begin{noliste}{a)}
 \setlength{\itemsep}{2mm}
\item On pose $Q = -\ln(1-V)$ et on admet que $Q$ est une variable
aléatoire.
Déterminer la fonction de répartition de $Q$ et en déduire la loi
suivie par
la variable aléatoire $Q.$

\item On pose $R = 2U-1$ et on admet que $R$ est une variable
aléatoire.
Déterminer $R(\Omega)$ et donner la loi suivie par la variable $R.$

\item Informatique.

En tenant compte des résultats des questions 5a) et 5b), écrire en
\texttt{ \Scilab{}} une déclaration de fonction dont l'en-tête est
\texttt{function z :real;} pour qu'elle simule la loi de $Z.$
\end{noliste}
\end{noliste}


\end{document}