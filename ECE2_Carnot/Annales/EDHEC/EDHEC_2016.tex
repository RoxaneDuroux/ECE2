\documentclass[11pt]{article}%
\usepackage{geometry}%
\geometry{a4paper,
 lmargin = 2cm,rmargin = 2cm,tmargin = 2.5cm,bmargin = 2.5cm}

\input{../../macros.tex}

\pagestyle{fancy} %
\lhead{ECE2 \hfill Mathématiques\\
} %
\chead{\hrule} %
\rhead{} %
\lfoot{} %
\cfoot{} %
\rfoot{\thepage} %

\renewcommand{\headrulewidth}{0pt}% : Trace un trait de séparation
 % de largeur 0,4 point. Mettre 0pt
 % pour supprimer le trait.

\renewcommand{\footrulewidth}{0.4pt}% : Trace un trait de séparation
 % de largeur 0,4 point. Mettre 0pt
 % pour supprimer le trait.

\setlength{\headheight}{14pt}

\title{\bf \vspace{-2cm} EDHEC 2016} %
\author{} %
\date{} %
\begin{document}

\maketitle %
\vspace{-1.4cm}\hrule %
\thispagestyle{fancy}

\vspace*{.2cm}


% DEBUT DU DOC À MODIFIER : tout virer jusqu'au début de l'exo
\begin{center}
\shadowbox{\begin{minipage}[t]{1\columnwidth}\begin{center}
\textbf{Sujet EDHEC 2016 ECO } \foreignlanguage{english}{ }
\par\end{center}\end{minipage}} 
\par\end{center}

{}



\subsection*{{\normalsize{}Exercice 1}}

On désigne par $Id$ l'endomorphisme identité de $\R{}{3}$
et par $I$ la matrice identité de $\M{3}$. \\
 On note $\mathcal{B} = (e_{1},e_{2},e_{3})$ la base canonique de
$\R{}{3}$
et on considère l'endomorphisme $f$ de $\R{}{3}$ dont la
matrice dans la base $\mathcal{B}$ est : $A = 
\begin{smatrix}
3 & -1 & 1\\
2 & 0 & 2\\
1 & -1 & 3
\end{smatrix}
$. 
\begin{noliste}{1.}
 \setlength{\itemsep}{4mm}
\item Calculer $A^{2}-4A$ puis déterminer un polynôme annulateur de $A$
de degré $2$. 
\item 

\begin{noliste}{a)}
 \setlength{\itemsep}{2mm}
\item En déduire la seule valeur propre de $A$ (donc aussi de $f$ ).\\

\item La matrice $A$ est-elle diagonalisable ? Est-elle inversible ? 
\end{noliste}
\item Déterminer une base $(u_{1},u_{2})$ du sous-espace propre de $f$
associé à la valeur propre de $f$. 
\item 

\begin{noliste}{a)}
 \setlength{\itemsep}{2mm}
\item On pose $u_{3} = e_{1} + e_{2} + e_{3}$. Montrer que la famille
$(u_{1},u_{2},u_{3})$
est une base de $\R^{3}.$ 
\item Vérifier que la matrice $T$ de $f$ dans la base
$(u_{1},u_{2},u_{3})$
est triangulaire et que ses éléments diagonaux sont tous égaux à 2. 
\item En écrivant $T = 2I + N$, déterminer, pour tout entier naturel
$n$, la matrice $T^{n}$ comme combinaison linéaire de $I$ et $N$,
puis de $I$ et $T$. 
\end{noliste}
\item 

\begin{noliste}{a)}
 \setlength{\itemsep}{2mm}
\item Expliquer pourquoi l'on a : 
\[
\begin{array}{lll}
\forall n\in\N, & & A^{n} = n2^{n-1}A-(n-1)2^{n}I
\end{array}
\]

\item Utiliser le polynôme annulateur obtenu à la première question
pour
déterminer $A^{-1}$ en fonction de $I$ et de $A$. 
\item Vérifier que la formule trouvée à la question $5a)$ reste valable
pour $n = -1$. 
\end{noliste}
\end{noliste}

\subsection*{{\normalsize{}Exercice 2}}

Pour chaque entier naturel $n$, on définit la fonction $f_{n}$
par : $\forall x\in[n, + \infty[ \ \;,\ :f_{n}(x) =
\dint{n}{x}e^{\sqrt{t}}dt$. 
\begin{noliste}{1.}
 \setlength{\itemsep}{4mm}
\item Étude de $f_{n}$.

\begin{noliste}{a)}
 \setlength{\itemsep}{2mm}
\item Montrer que $f_{n}$ est de classe $C^{1}$sur $[n, + \infty[$ puis
déterminer $f_{n}{'}(x)$ pour tout $x$ de $[n, + \infty[$. Donner
le sens de variation de $f_{n}$. 
\item En minorant $f_{n}(x)$, établir que $\underset{x\rightarrow +
\infty}{lim}f_{n}(x) = + \infty$. 
\item En déduire que pour chaque entier naturel $n$, il existe un
unique
réel, noté $u_{n}$, élément de $[n, + \infty[$, tel que $f_{n}(u_{n}) =
1$. 
\end{noliste}
\item Étude de la suite $(u_{n})$.

\begin{noliste}{a)}
 \setlength{\itemsep}{2mm}
\item Montrer que $\underset{n\rightarrow + \infty}{lim}u_{n} = +
\infty$. 
\item Montrer que : $\forall n\in\N\ :,\ :e^{-\sqrt{u_{n}}}\leq
u_{n}-n\leq e^{-\sqrt{n}}$. 
\end{noliste}
\item 

\begin{noliste}{a)}
 \setlength{\itemsep}{2mm}
\item Utiliser la question 2b) pour compléter les commandes Scilab
suivantes
afin qu'elles permettent d'afficher un entier naturel $n$ pour lequel
$u_{n}-n$ est inférieur ou égal à $10^{-4}$.~\\
 \texttt{n = 0}


\texttt{while -{}-{}-{}-{}-{}-{}-}


\texttt{n = -{}-{}-{}-{}-{}-}


\texttt{end}


\texttt{disp(n)}

\item Le script affiche l'une des trois valeurs $n = 55$, $n = 70$ et
$n = 85$. Préciser laquelle en prenant $2,3$ comme valeur approchée de
$ln10$. 
\end{noliste}

%\newpage

\item On pose $v_{n} = u_{n}-n$.

\begin{noliste}{a)}
 \setlength{\itemsep}{2mm}
\item Montrer que $\underset{n\to + \infty}{\lim}v_{n} = 0$. 
\item Établir que, pour tout réel $x$ supérieur ou égal à $-1$,on a :
$\sqrt{1 + x}\leq1 + \frac{x}{2}$. 
\item Vérifier ensuite que : $\forall n\in\N^{*}\;
:e^{-\sqrt{u_{n}}}\geq e^{-\sqrt{n}}exp(-\frac{v_{n}}{2\sqrt{n}})$. 
\item Déduire de l'encadrement obtenu en 2b) que : $u_{n}-n\underset{+
\infty}{\sim}e^{-\sqrt{n}}$. 
\end{noliste}
\end{noliste}

\subsection*{{\normalsize{}{}Exercice 3}}

\noindent Dans cet exercice, toutes les variables aléatoires sont
supposées définies sur un même espace probabilisé
$(\varOmega,\mathcal{A},P)$. On désigne par $p$ un réel de $]0,1[$. \\
 On considère deux variables aléatoires indépendantes $U$ et $V$,
telles que $U$ suit la loi uniforme sur $[-3,1]$, et $V$ suit la
loi uniforme sur $[-1,3]$.\\
 On considère également une variable aléatoire $Z$,indépendante de
$U$ et $V$, dont la loi est donnée par : 
\begin{eqnarray*}
P\left(\Ev{Z = 1}\right) = p & \mbox{et } & P\left(\Ev{Z = -1}\right) =
1-p
\end{eqnarray*}
Enfin,on note $X$ la variable aléatoire, définie par : 
\[
\forall\omega\in\varOmega,\ :X(\omega) = \left\{ 
\begin{array}{lll}
U(\omega) & \text{si} & \mbox{\ensuremath{Z(\omega) = 1}}\\
\V(\omega) & \text{si} & \mbox{\ensuremath{Z(\omega) = -1}}
\end{array}
\right.
\]
On note $F_{X}$, $F_{U}$ et $F_{V}$ les fonctions de répartition
respectives des variables $X$,$U$ et $V$. 
\begin{noliste}{1.}
 \setlength{\itemsep}{4mm}
\item Donner les expressions de $F_{U}(x)$ et $F_{V}(x)$ selon les
valeurs
de $x$. 
\item 

\begin{noliste}{a)}
 \setlength{\itemsep}{2mm}
\item Établir, grâce au système complet d'évènements $\left((Z = 1),(Z
= -1)\right)$, que : 
\begin{align*}
\forall x\in\R,\ :F_{X}(x) & = pF_{U}(x) + (1-p)F_{V}(x)
\end{align*}

\item Vérifier que $X(\varOmega) = [-3,3]$ puis expliciter $F_{X}(x)$
dans
les cas : 
\begin{eqnarray*}
x<-3, & -3\leq x\leq-1, & -1\leq x\leq1,\ :1\leq x\leq3\ :\mbox{et }x>3
\end{eqnarray*}

\item On admet que X est une variable à densité.Donner une densité
$f_{X}$
de la variable aléatoire $X$. 
\item \noindent Établir que X admet une espérance $\E(X)$ et une
variance
$\V(X)$. 
\end{noliste}
\item On se propose de montrer d'une autre façon que $X$ possède une
espérance
et un moment d'ordre $2$ puis de les déterminer.

\begin{noliste}{a)}
 \setlength{\itemsep}{2mm}
\item Vérifier que l'on a : 
\[
X = U\frac{1 + Z}{2} + V\frac{1-Z}{2}.
\]

\item Déduire de l'égalité précédente que $X$ possède une espérance et
retrouver la valeur de $\E(X)$. 
\item En déduire également que $X$ possède un moment d'ordre $2$ et
retrouver
la valeur de $\E(X^{2})$. 
\end{noliste}
\item \noindent %\noindent 

\begin{noliste}{a)}
 \setlength{\itemsep}{2mm}
\item Soit $T$ une variable aléatoire suivant la loi de Bernoulli de
paramètre
$p$. Déterminer la loi de $2T-1$. 
\item On rappelle que \texttt{grand(1,1,'unf',a,b) }et\texttt{
grand(1,1,'bin',p)}sont
des commandes\texttt{ Scilab }permettant de simuler respectivement
une variable aléatoire à densité suivant la loi uniforme sur $[a,b]$
et une variable aléatoire suivant la loi de Bernoulli de paramètre
$p$.\\
 Écrire des commandes \texttt{Scilab }permettant de simuler $U,V,Z,$
puis $X$. 
\end{noliste}
\end{noliste}

\subsection*{{\normalsize{}{}PROBLÈME }}


\subsection*{{\normalsize{}{}Partie I : Questions préliminaires.}}

Dans cette partie,$x$ désigne un réel élément de $[0,1[$. 
\begin{noliste}{1.}
 \setlength{\itemsep}{4mm}
\item a) Pour tout $n$ de $\N{}{*}$ et pour tout $t$ de $[0,x]$,
simplifier la somme $\Sum{p = 1}{n}t^{p-1}$.\\
 b) En déduire que : $\Sum{p = 1}{n}\frac{x^{p}}{p} =
-ln(1-x)-\dint{0}{x}\frac{t^{n}}{1-t}dt$. \\
 c) Établir par encadrement que l'on a :
$\ensuremath{\dlim{n\rightarrow +
\infty}\dint{0}{x}\dfrac{t^{n}}{1-t}dt = 0}$. \\
 d) En déduire que : $\Sum{k = 1}{+ \infty}\frac{x^{k}}{k} = -ln(1-x)$.

\item Soit $m$ un entier naturel fixé. À l'aide de la formule du
triangle
de \Scilab{}, établir l'égalité : 
\[
\forall q\geq m,\;\Sum{k = m}{q}\binom{k}{m} = \binom{q + 1}{m + 1}
\]

\item Soit $n$ un entier naturel non nul. On considère une suite
$(X_{n})_{n\in\N^{*}}$
de variables aléatoires,\\
 mutuellement indépendantes, suivant toutes la loi géométrique de
paramètre $x$, et on pose $S_{n} = \Sum{k = 1}{n}X_{k}$.

\begin{noliste}{a)}
 \setlength{\itemsep}{2mm}
\item Déterminer $S_{n}(\varOmega)$ puis établir que, pour tout entier
$k$ supérieur ou égal à $n + 1$, on a :


\begin{eqnarray*}
\Prob\left(\Ev{S_{n + 1} = k}\right) & = & \Sum{j =
n}{k-1}\mathrm{P\left(\Ev{(}\right)\left(\Ev{}\right)S_{n} =
j)\cap(X_{n + 1} = k-j))
\end{eqnarray*}


\item En déduire, par récurrence sur $n$, que la loi de $S_{n}$ est
donnée
par : 
\begin{eqnarray*}
\forall k\in[[n, + \infty[[, & P\left(\Ev{S_{n} = k}\right) =
\binom{k-1}{n-1}x^{n}(1-x)^{k-n}
\end{eqnarray*}

\item En déduire, pour tout $x$ de $]0,1[$ et pour tout entier naturel
$n$ non nul : 
\begin{eqnarray*}
 & \Sum{k = n}{+ \infty}\binom{k-1}{n-1}(1-x)^{k-n} = \frac{1}{x^{n}}
\end{eqnarray*}

\item On rappelle que la commande\texttt{ grand(1,n,'geom',p)}permet
à\texttt{
Scilab }de simuler n variables aléatoires indépendantes suivant toutes
la loi géométrique de paramètre $p$.~\\
 Compléter les commandes\texttt{ Scilab} suivantes pour qu'elles
simulent
la variable aléatoire $S_{n}.$ 
\end{noliste}

\texttt{n = input('entrez une valeur de n supérieure à 1 :')}


\texttt{S = -{}-{}-{}-{}-{}-}


\texttt{disp(S)}

\end{noliste}

\subsection*{{\normalsize{}{}{}Partie 2 : étude d'une variable
aléatoire.}}

Dans cette partie, on désigne par $p$ un réel de $]0,1[$ et on pose
$q = 1-p$.\\
 On considère la suite $(u_{k})_{k\in\N^{*}}$, définie par : 
\[
\mathrm{\forall k\in\N^{*} :\ :u_{k} = -\dfrac{q^{k}}{k\,ln\,p}\ :}
\]

\begin{noliste}{1.}
 \setlength{\itemsep}{4mm}
\item 

\begin{noliste}{a)}
 \setlength{\itemsep}{2mm}
\item Vérifier que la suite $(u_{k})_{k\in\N^{*}}$ est à termes
positifs. 
\item Montrer, en utilisant un résultat de la partie 1, que $\Sum{k =
1}{+ \infty}u_{k} = 1$. 
\end{noliste}

On considère dorénavant une variable aléatoire $X$ dont la loi de
probabilité est donnée par :


\[
\mathrm{\forall k\in\N^{*},\ :P\left(\Ev{X = k}\right) = u_{k}\ :}
\]


\item 

\begin{noliste}{a)}
 \setlength{\itemsep}{2mm}
\item Montrer que $X$ possède une espérance et la déterminer. 
\item Montrer également que $X$ possède une variance et vérifier que :
$\V(X) = -\frac{q(q + \textrm{ln}\,p)}{(p\,\textrm{ln}\,p)^{2}}\ :.$ 
\end{noliste}
\item Soit $k$ un entier naturel non nul. On considère une variable
aléatoire
$Y$ dont la loi, conditionnellement à l'évènement $(X = k)$, est la
loi binomiale de paramètres $k$ et $p$.

\begin{noliste}{a)}
 \setlength{\itemsep}{2mm}
\item Montrer que $\mathrm{Y(\varOmega) = \N}$ puis utiliser la formule
des probabilités totales, ainsi que la question 1) de la partie 1,
pour montrer que :


\begin{eqnarray*}
\Prob\left(\Ev{Y = 0}\right) = & 1 + \frac{\mathrm{ln}(1 +
q)}{\mathrm{ln}\,p}
\end{eqnarray*}


\item Après avoir montré que, pour tout couple $(k,n)$ de
$\N{}{*}\times\N{}{*}$,
on a : $\frac{\binom{k}{n}}{k} = \frac{\binom{k-1}{n-1}}{n}$, établir
que, pour tout entier naturel $n$ non nul, on a : 
\[
\Prob\left(\Ev{Y = n}\right) =
-\frac{p^{n}q^{n}}{n\,\mathrm{ln}\,p}\Sum{k = 1}{+
\infty}\binom{k-1}{n-1}(q^{2})^{k-n}.
\]
En déduire, grâce à la question 3) de la première partie, l'égalité : 
\begin{eqnarray*}
\Prob\left(\Ev{Y = n}\right) = & -\frac{q^{n}}{n\,(1 +
q)^{n}\,\mathrm{ln}\,p}
\end{eqnarray*}

\item Vérifier que l'on a $\Sum{k = 0}{+ \infty}P\left(\Ev{Y =
k}\right) = 1.$ 
\item Montrer que $Y$ possède une espérance et donner son expression en
fonction de $\mathrm{ln}\,p$ et $q$. 
\item Montrer aussi que $Y$ possède une variance et que l'on a : $\V(Y)
= -\frac{q(q + (1 +
q\mathrm{)ln}\,p)}{(\mathrm{ln}\,p)^{2}}$.\end{noliste}
\end{noliste}

\end{document}

