\documentclass[11pt]{article}%
\usepackage{geometry}%
\geometry{a4paper,
  lmargin=2cm,rmargin=2cm,tmargin=2.5cm,bmargin=2.5cm}

\input{../../../macros.tex}
%\input{../../../../../../macros.tex}

\pagestyle{fancy} %
\lhead{ECE2 \\
  Mathématiques\\[.2cm]
  \hrule} %
\chead{} %
\rhead{} %
\lfoot{} %
\cfoot{} %
\rfoot{\thepage} %

\renewcommand{\headrulewidth}{0pt}% : Trace un trait de séparation
                                    % de largeur 0,4 point. Mettre 0pt
                                    % pour supprimer le trait.

\renewcommand{\footrulewidth}{0.4pt}% : Trace un trait de séparation
                                    % de largeur 0,4 point. Mettre 0pt
                                    % pour supprimer le trait.

\setlength{\headheight}{14pt}

\title{\bf \vspace{-1cm} ESSEC II 2012} %
\author{} %
\date{} %
\begin{document}

\maketitle %
\vspace{-1.2cm}\hrule %
\thispagestyle{fancy}

\vspace*{.4cm}










\noindent
En psychologie, on s'intéresse à la fa\c con dont un individu est 
amené à sélectionner une action quand un choix se présente entre 
différentes actions possibles. Ce choix peut être influencé par un 
grand nombre de facteurs impondérables, ce qui fait qu'il est 
légitime de le modéliser à l'aide de variables aléatoires. 
L'objet du problème est de présenter quelques éléments simples 
de la théorie des modèles de choix discret. Dans le modèle binaire 
le plus simple, le choix se fait en fonction de la réaction à un 
stimulus. Dans une première partie, on étudie la modélisation 
élémentaire de la réponse à un stimulus. Dans une deuxième 
partie, on considère une importante modélisation de choix 
dépendant du hasard, dit {\it modèle de Luce}, et on étudie ses 
propriétés. Enfin, dans une troisième partie, on regarde le cas 
o\`u les différents choix possibles engendrent des réactions 
aléatoires et on étudie des propriétés de la réaction 
optimale.  {\bf Les trois parties sont indépendantes}.\\

\noindent
Toutes les variables aléatoires sont définies sur un même espace 
probabilisé $(\Omega , \A , \Prob)$. Si elles existent, on note $\E(T)$ 
et $\V(T)$ l'espérance et la variance d'une variable aléatoire $T$.

\section{Modèles avec réponse discrète}
\noindent
Soit $\alpha $ un réel (positif ou négatif) représentant un niveau 
de stimulus. On considère une variable aléatoire réelle $X$ à 
valeurs dans $\R$ représentant la tolérance de l'individu au 
stimulus en question. On considère donc que l'individu réagit si $X 
\leq \alpha$ et ne réagit pas si $X >\alpha $. On considère la 
variable aléatoire $Y$ indicatrice de la réaction définie par 
\[
 Y = 
 \left\{ \begin{array}{lll}
  1 & {\rm si} & X\leq \alpha \\
  0 & {\rm si} & X>\alpha 
  \end{array} \right.
\]
Soit $F$ la fonction de répartition de la variable aléatoire $X$.
\begin{noliste}{1.} 
 \item Déterminer la loi de $Y$, son espérance $\theta$ et 
 sa variance.
 
 \item  On considère $n$ individus dont on observe la réaction 
 au stimulus. La tolérance de l'individu $i$ est une variable 
 aléatoire $X_i$ dont on suppose qu'elle suit la  même loi que $X$. 
 En outre, les tolérances pour les différents individus  sont 
 supposées indépendantes.
 \begin{noliste}{a)}
  \item Soit $N$ la variable aléatoire égale au nombre 
  d'individus réagissant au stimulus. Déterminer la loi de $N$, son 
  espérance et sa variance.
  
  \item Construire à l'aide de $N$ un estimateur sans biais de 
  $\theta$.
 \end{noliste}
 
 \item Soient $m$ un réel et $\sigma$ un réel strictement 
 positif.\\
 \noindent
 On suppose que la tolérance $X$ est obtenue comme résultante d'un 
 \og grand nombre \fg{} $n$ de facteurs indépendants de petite taille 
 c'est-à-dire  $X = \Sum{i=1}{n} X_i$, où les $X_i$ sont 
 supposées être des variables aléatoires de même loi 
 d'espérance $\dfrac{m}{n}$ et de variance $\dfrac{\sigma^2}{n}$.
 \begin{noliste}{a)}
  \item Déterminer l'espérance et l'écart-type de $X$.
  
  \item Montrer que pour tout réel $a$, 
  \[
   \dlim{n\to +\infty}\Prob\Big(\Ev{X- \dfrac{m}{\sigma}\leq a}\Big) 
   = \dfrac{1}{\sqrt{2\pi}} \dint{-\infty}{a} \ee^{-\frac{u^2}{2}} \ du
  \]
  
  \item Le résultat précédent justifie que pour $n$ grand on 
  peut considérer que la variable aléatoire  $X- \dfrac{m}{\sigma}$ 
  suit la loi normale centrée réduite.  Trouver dans ce cas 
  l'expression de $\theta$ en fonction de $\alpha $, $m$ et $\sigma$.
  
  \item Déterminer $\dlim{\sigma \to +\infty} \theta$ et interpréter le 
  résultat.
 \end{noliste}
 
 \item Plutôt que d'utiliser la loi normale, on préfère 
 souvent une loi plus simple dont on étudie dans cette question 
 quelques propriétés.
 \begin{noliste}{a)}
  \item Soit $F$ la fonction définie sur $\R$ par : 
  \[
   \forall y\in \R, \ F(y)= \dfrac{1}{1+\ee^{-y}}
  \]
  Montrer que $F$ est la fonction de répartition d'une variable 
  aléatoire réelle. On dit alors que cette variable aléatoire suit la 
  {\it loi logistique}.
  
  \item \label{niv_stimulus} On suppose que $X$ suit une loi 
  logistique. Déterminer $\theta$.\\[.2cm]
  On considère $Z$ une variable aléatoire suivant la loi logistique.
  
  \item Déterminer une densité de probabilité de $Z$.
  
  \item Soit $y$ un réel positif. Etablir une relation entre 
  $F(y)$ et $F(-y)$.
  
  \item Montrer que $Z$ admet une espérance et la déterminer.
  
  \item Soit $U$ une variable aléatoire de loi uniforme sur 
  $]0,1[$. Déterminer la loi de la variable aléatoire $\ln \Big( 
  \dfrac{U}{1-U} \Big)$.
 \end{noliste}
\end{noliste}








\section{Règles de décisions stochastiques : le modèle de Luce}

\noindent
On suppose maintenant que l'individu doit choisir une action dans un 
ensemble fini d'actions possibles $A$. On note ${\cal F} =\{ S\subset 
A \ / \ \vert S \vert \geq 2 \}$ où $\vert S \vert$ désigne le cardinal 
de l'ensemble $S$. Quand le nombre d'actions possibles est très grand, 
la procédure de choix se passe en deux temps : l'individu commence par 
sélectionner une partie $S$ de ${\cal F}$ à laquelle il va restreindre 
son choix, puis choisit une action précise à l'intérieur de $S$.\\[.1cm]
Pour chaque élément $S$ de ${\cal F}$, on définit une probabilité 
$\Prob_S$ sur $S$ : pour $a$ un élément de  $S$, $\Prob_S(\{ a\})$ 
représente la probabilité pour que l'individu ayant sélectionné 
$S$ choisisse l'action $a$. Pour simplifier la notation, on notera 
$\Prob_S(a)$ pour $\Prob_S(\{ a\} )$.  En particulier, 
$\Prob_A(S)= \Sum{a\in S}{} \Prob_A(a)$ est la probabilité pour 
que l'individu prenne dans $S$ l'action qu'il choisit.\\[.1cm]
Pour $a$ et $b$ distincts dans $A$ on note $\Prob(a,b) = \Prob_{\{ 
a,b\} }(\{ a\} )$; il s'agit donc de la probabilité de préférer l'action 
$a$ à l'action $b$ dans le cas d'un choix à faire entre $a$ et 
$b$.\\[.1cm]
On suppose que pour tout $S$ appartenant à ${\cal F}$ et tout $a$ dans 
$S$, $\Prob_S(a)\neq 0$.\\[.1cm]
On fait l'hypothèse suivante sur le modèle : \\[.1cm]
$(*)$ \qquad Pour tout couple $(S,T)$ d'éléments de ${\cal F}$ tel 
que $S$ est inclus dans $T$, pour tout $a$ élément de $S$, 
\[
 \Prob_T(a)=\Prob_T(S) \, \Prob_S(a)
\]

\begin{noliste}{1.}
 \item Interpréter le sens de la condition $(*)$ en termes de 
  probabilités conditionnelles.
  
  \item 
  \begin{noliste}{a)}
    \item Soit $k$ un réel strictement positif. On pose pour tout 
    $a\in A$,  $v(a)=k \, \Prob_A(a)$. Montrer que pour tout $S$ 
    appartenant à ${\cal F}$ et pour tout $a$ dans $S$, 
    \begin{equation}\label{**}
     \Prob_S(a)= \dfrac{v(a)}{\Sum{b\in S}v(b)}. 
    \end{equation}
    
    \item Montrer que si $v$ et $w$ sont deux fonctions réelles 
    définies sur $A$ satisfaisant (\ref{**}), il existe un réel $\mu$ 
    strictement positif tel que $v=\mu \cdot w$. 
    Une telle fonction $v$ s'appelle une {\it utilité associée au 
    système de probabilités} $(\Prob_S)_{S\in {\cal F}}$.
  \end{noliste}
  
  \item Réciproquement, soit $v$ une fonction réelle 
  strictement positive sur $A$. On pose,  pour tout $S$ dans ${\cal F}$ 
  et tout $a$ appartenant à $S$, 
  \[
   Q_S(a)= \dfrac{v(a)}{\Sum{b\in S}{} v(b)}
  \]
  Montrer qu'on définit ainsi un système de probabilités vérifiant 
  $(*)$.
  
  \item 
  \begin{noliste}{a)}
    \item Soit $v$ une utilité associée au système 
    de probabilités $(\Prob_S)_{S\in {\cal F}}$. Montrer que pour tout 
    $S\in {\cal F}$, et pour tous $a$ et $b$ dans $S$, 
    \[
     v(a)\leq v(b) \ \Rightarrow \ \Prob_S(a) \leq \Prob_S(b)
    \]
    La probabilité que $a$ soit choisi augmente donc avec son utilité.
    
    \item Montrer qu'il existe une fonction $\rho $ sur $A$ telle 
    que pour tous $a$ et $b$ distincts dans $A$,
    \[
     \Prob(a,b) = \dfrac{1}{1+\exp(\rho (b)-\rho (a))}
    \]
    
    \item Soit $X$ une variable aléatoire suivant la loi 
    logistique de fonction de répartition $F$ définie par 
    \[
      F(y)=\dfrac{1}{ 1+\ee^{-y}}
    \]
    Soient $a$ et $b$ distincts dans $A$. Trouver  en fonction de $\rho$ 
    un réel $\alpha_{a,b}$ tel que $\Prob(a,b)=\Prob(\Ev{X\leq 
    \alpha_{a,b}})$.
  \end{noliste}
  
  \item 
  \begin{noliste}{a)}
    \item Montrer que pour tout couple $(S,T)$ 
    d'éléments de ${\cal F}$ tel que $S$ est inclus dans $T$,  et pour 
    tous $a$ et $b$ dans $S$, on a 
    \[
     \dfrac{\Prob_S(a)}{\Prob_S(b)} = \dfrac{\Prob_T(a)}{\Prob_T(b)}
    \]
    Le rapport des probabilités de choix respectives de $a$ et $b$ est 
    donc indépendant de la sélection de l'ensemble d'actions contenant 
    $a$ et $b$.
    
    \item On examine ici un cas concret. On suppose que l'individu 
    devant se rendre de son domicile à son travail ait le choix entre 
    utiliser sa voiture (symbolisée par V) ou le bus, dont deux lignes 
    sont possibles : le bus rouge (symbolisé par R) ou le bus bleu (B). 
    On a donc l'ensemble d'actions $A=\{ V,R,B \}$ . On suppose que 
    l'individu est indifférent au fait de choisir sa voiture ou un bus, 
    et est également indifférent à la couleur du bus. On définit ainsi 
    un système de probabilités comme précédemment avec  
    $\Prob(V,R)=\Prob(V,B)=\dfrac{1}{2}$ et de plus 
    $\Prob_A(R)=\Prob_A(B)$. 
    Montrer que $\Prob_A(V)=\dfrac{1}{3}$. Ce résultat est-il 
    satisfaisant? 
    Interpréter.
  \end{noliste}
\end{noliste}








\section{Utilités aléatoires}

\noindent
Dans cette partie, on aborde la question du choix sous un autre aspect. 
A chaque action $i$ de l'ensemble d'actions $A =\{ 1,2, \dots , n\} $ 
est associée une variable aléatoire $U_i$ représentant l'utilité 
de l'action $i$. L'individu est alors amené à choisir l'action qui 
maximise ces utilités. On suppose que les variables $U_i$ sont 
indépendantes et que la loi de $U_i$ est donnée par la fonction de 
répartition $F_i$. On s'intéresse dans cette partie à la valeur 
$U$ de l'utilité maximale, c'est à dire à $U=\max (U_1,\dots , 
U_n)$.

\begin{noliste}{1.}
 \item 
 \begin{noliste}{a)}
  \item Déterminer la fonction de répartition 
  $G_n$ de $U$. Que vaut $G_n$ dans le cas particulier où les $U_i$ 
  suivent la même loi de fonction de répartition $F$ ?\\[.2cm]
  
  \noindent
  On suppose désormais que les $U_i$ ont même loi.
  
  \item Pour $x$ réel donné, étudier $\dlim{n\to +\infty}G_n(x)$.
  
  \item Montrer que les seules lois pour lesquelles on a $G_n=F$ 
  pour tout $n\geq 1$ sont les lois de variables aléatoires 
  constantes.
 \end{noliste}
 
 \item Pour obtenir un type de loi plus intéressant pour $U$, on va 
 chercher des lois admettant une densité strictement positive sur $\R$ 
 et dont la fonction de répartition $F$ vérifie que pour tout $n\geq 1$ 
 il existe $b_n\leq 0$ tel que pour tout $x$ réel, 
 $(F(x))^n=F(x+b_n)$.\\[.1cm]
 \noindent 
 On suppose qu'une telle loi existe et on cherche des conditions 
 qu'elle vérifie.
 \begin{noliste}{a)}
  \item Montrer que $F$ est une fonction continue et strictement 
  croissante telle que $\dlim{x\to -\infty}F(x)=0$ et 
  $\dlim{x\to +\infty}F(x)=1$. $F$ définit donc une bijection de $\R$ 
  sur $]0,1[$.
  
  \item Montrer que la suite $(b_n)_{n\geq 1}$ est décroissante.
  
  \item Soit $(n,N)$ un couple d'entiers strictement positifs. On 
  considère $U_1, \dots , U_{nN}$,   $nN$ variables aléatoires 
  indépendantes de même loi $F$, et on pose pour $j$ tel que  $1\leq j 
  \leq n$ : 
  \[
   Y_j=\max (U_{(j-1)N+1}, \dots , U_{jN})
  \]
  Montrer que les variables $Y_j$ sont indépendantes.
  
  \item Quelle est  la fonction de répartition de $Y_j$ ?
  
  \item En remarquant que $\max (Y_1,\dots , Y_n) = \max (U_1, 
  \dots , U_{nN})$, montrer que pour tout $x$ réel, 
  \[
   F(x)^{nN}=F(x+b_n+b_N)=F(x+b_{nN})
  \]
  Déduire que pour tout couple $(n,N)$ d'entiers strictement positifs , 
  $b_{nN}=b_n+b_N$.
  
  \item Montrer que pour tout entier $n$ strictement positif et 
  tout $k\in \N$ , $b_{n^k}=kb_n$.
  
  \item Soient $p$ et $m$ deux entiers strictement positifs. 
  Montrer qu'il existe un unique $k_m\in \N$ tel que 
  $2^{k_m} \leq p^m< 2^{{k_m}+1}$ et que 
  $\dlim{m\to +\infty } \dfrac{k_m}{m}=\ln \left(\dfrac{p}{\ln 
  2}\right)$.
  
  \item En déduire qu'il existe un réel $\gamma$ tel que pour 
  tout entier $p$ strictement positif,  $b_p=\gamma \ln (p)$.
  
  \item Montrer que la fonction $F(x)=\exp(-\ee^{-x})$ 
  satisfait aux conditions cherchées. La loi ainsi définie est dite 
  loi de Gumbel.
 \end{noliste}
 
 \item Dans cette section, on étudie un certain nombre de 
 propriétés de la loi de Gumbel. Soit $X$ une variable aléatoire de 
 loi de Gumbel c'est-à-dire de fonction de répartition $F$ telle que 
 $F(x)=\exp(-\ee^{-x})$.
 \begin{noliste}{a)}
  \item Déterminer une densité de probabilité de $X$.
  
  \item On pose $Z=\ee^{-X}$. Déterminer la loi de la variable 
  aléatoire $Z$.
  
  \item Soient $x$ et $y$ deux réels strictement positifs.\\
  Établir une relation entre $\Prob_{\Ev{X\leq -\ln (x)}}(\Ev{X\leq 
  -\ln(x+y)})$ et $\Prob(\Ev{X\leq -\ln(y)})$.
  
  \item On considère  $(Y_i)_{i\ge 1}$ une suite de variables 
  aléatoires indépendantes de même loi exponentielle de paramètre 
  1. Soit $L$ une variable aléatoire suivant une loi de Poisson de 
  paramètre 1 indépendante de $(Y_i)_{i\geq 1}$. 
  On considère la variable aléatoire $W=\max (Y_1,\dots , Y_L)$ telle 
  que pour tout $k\geq 1$,  et tout $\omega \in \Ev{L=k}$, $W(\omega 
  )=\max (Y_1(\omega ), \dots , Y_k(\omega ))$ et $W(\omega )=0$ si 
  $L(\omega)=0$. \\
  Montrer que pour tous réels $a$ et $b$ tels que $0< a<b$, on a 
  $\Prob(\Ev{a\leq W\leq b})=\Prob(\Ev{a\leq X\leq b})$. Que vaut 
  $\Prob(\Ev{W=0})$ ?
 \end{noliste}
\end{noliste}

\end{document}





