\documentclass[11pt]{article}%
\usepackage{geometry}%
\geometry{a4paper,
  lmargin=2cm,rmargin=2cm,tmargin=2.5cm,bmargin=2.5cm}

\input{../../macros.tex}
%\input{../../../../../../macros.tex}

\pagestyle{fancy} %
\lhead{ECE2 \\
  Mathématiques\\[.2cm]
  \hrule} %
\chead{} %
\rhead{} %
\lfoot{} %
\cfoot{} %
\rfoot{\thepage} %

\renewcommand{\headrulewidth}{0pt}% : Trace un trait de séparation
                                    % de largeur 0,4 point. Mettre 0pt
                                    % pour supprimer le trait.

\renewcommand{\footrulewidth}{0.4pt}% : Trace un trait de séparation
                                    % de largeur 0,4 point. Mettre 0pt
                                    % pour supprimer le trait.

\setlength{\headheight}{14pt}

\title{\bf \vspace{-1cm} ESSEC II 2014} %
\author{} %
\date{} %
\begin{document}

\maketitle %
\vspace{-1.2cm}\hrule %
\thispagestyle{fancy}

\vspace*{.4cm}


\section{Autour de la loi de Benford}
\noindent
Soit $x\in\R$. On note $\lfloor x\rfloor$ sa partie entière, c'est à 
dire le plus grand entier relatif inférieur ou égal à $x$, et $\{ 
x\}$ sa partie fractionnaire : $\{ x\}=x-\lfloor x\rfloor$. On note 
$\log (z)$ le logarithme \underline{en base $10$} \ du réel $z>0$. On a 
donc $\log (z)=\frac{\ln (z)}{\ln (10)}$. On rappelle en particulier 
les propriétés suivantes, qu'on poura utiliser sans démonstration 
\[
\forall z>0, \; 10^{\log (z)}=z
\]
\[
\forall z>0, \ \forall z'>0, \ \log(z \times z')=\log (z)+\log (z')
\]
\[
\forall a\in\R, \; \log\left(10^a\right)=a
\]
On a par exemple $\log (100)=2$ et $\log(\sqrt{10})=\frac 12$.
\begin{noliste}{1.}
\setlength{\itemsep}{2mm}

\item
\begin{noliste}{a)}
\item Montrer que pour tout réel $x$ positif et non nul, on a  
\[
x=10^{\{\log (x)\}} \times 10^{\lfloor\log (x)\rfloor}
\]

Cette décomposition est dite {\em notation scientifique} de $x$.
\item Montrer que pour tout $x>0$, le couple  $( 10^{\{\log (x)\}}, 
\lfloor\log (x)\rfloor )$ est l'unique couple $(\alpha , n)$ dans  
$[1,10[\times \Z$ tel que $x=\alpha \times 10^n$.

\item Soit $x>0$. On pose $\gamma=\lfloor 10^{\{ \log (x)\} }\rfloor$. 
Montrer que $\gamma\in\{ 1,2,\ldots , 9\}$.  
\end{noliste}

\item Pour tout entier naturel $k$ tel que $1\leq k\leq 9$, on pose 
$p_k=\log (1+\frac 1k)$. Montrer que 
$\Sum{k=1}{9} p_k=1$.\\
$(p_k)_{1\leq k\leq 9}$ définit donc une loi de probabilité sur $\{ 
1,2,\ldots , 9\}$ dite \emph{loi de Benford}.

\item Soit $X$ une variable aléatoire réelle strictement positive. 
On suppose que la variable aléatoire réelle $Y=\{\log (X)\}$ suit une 
loi uniforme sur $[0,1[$. 

\begin{noliste}{a)}
\item Soit $k\in\{ 1,2,\ldots , 9\}$. Montrer que $\lfloor 
10^Y\rfloor=k\Leftrightarrow k\leq 10^Y<k+1$.

\item On considère la variable aléatoire $\Gamma =\lfloor 
10^{\{\log (X)\}}\rfloor$ égale au premier chiffre significatif de $X$. 
Déterminer la loi de la variable aléatoire $\Gamma$.  
\end{noliste}

\item 
Soit $Y$ une variable aléatoire réelle admettant une densité $g$ 
continue sur $\R$. On suppose que 
\begin{nonoliste}{}
\item $(h_1)$ \quad $g$ atteint son maximum $M$ en un unique point 
$a_0\in\R$.

\item $(h_2)$ \quad $g$ est croissante sur $]-\infty , a_0]$ et 
décroissante sur $[a_0,+\infty [$.
\end{nonoliste}

\begin{noliste}{a)}
\item
\begin{nonoliste}{i.}
\item Montrer que pour tout $y\in\R$ et tout $n\in\Z$, 
$\{ y\}=\{ y-n\}$.

\item Déduire que pour tout $n\in\Z$, la loi de  $\{ Y\}$ est 
identique à celle de  $\{Y-n\}$.

\item Déterminer une fonction de densité  $\tilde {g}$ continue de 
la variable aléatoire $Y-\lfloor a_0\rfloor$. 

\item Montrer que $\tilde g$ admet un unique maximum en un point 
$\tilde a_0\in [0,1[$.

\item Montrer que $\tilde g$ vérifie les conditions $(h_1)$ et 
$(h_2)$ ci-dessus avec $\tilde a_0$ rempla\c{c}ant $a_0$.
\end{nonoliste}
On supposera donc désormais que $a_0\in [0,1[$. On fixe $x\in ]0,1[$ 
et on note $I_{n,x}=[n, n+x[$ pour tout $n\in\Z$.

\newpage

\item 
\begin{nonoliste}{i.}
\item Soit $\varphi$ une fonction positive continue et croissante sur 
$[0,1]$. Montrer, en utilisant un changement de variable, que    
$\dint{0}{x}\varphi (t)\dt\leq x \times \dint{0}{1}\varphi (u)\ du$.

\item Déduire que pour tout $n\in\Z$ tel que $n\leq -1$, on a 
$\dfrac{1}{x}\dint{I_{n,x}}{}g(t)\dt\leq 
\dint{n}{n+1}g(t)\dt$.\\
On {\bf admettra} qu'on montrerait de même que pour $n\geq 2$, 
$\dfrac{1}{x} \dint{I_{n,x}}{} g(t)\dt \leq \dint{n-1+x}{n+x} 
g(t)\dt$

\item Montrer que $\dfrac{1}{x} 
\Sum{n\geq 1}{} \dint{I_{-n,x}}{} g(t)\dt\leq \dint{-\infty}{0} 
g(t)\dt$ et que $\dfrac{1}{x} \Sum{n\geq 2}{} 
\dint{I_{n,x}}{} g(t)\dt \leq \dint{1+x}{+\infty} g(t)\dt$

\item Montrer que $\dint{I_{0,x}}{} g(u)\ du\leq xM$ et que 
$\dint{I_{1,x}}{} g(u)\ du\leq xM$

\item Conclure que 
$\dfrac{1}{x} \Sum{n\in\Z}{} \dint{I_{n,x}}{} g(u)\ du= 
\dfrac{1}{x} \Sum{n\geq 1}{} \dint{I_{-n,x}}{} g(u)\ du+
\dfrac{1}{x} \Sum{n\geq 0}{} \dint{I_{n,x}}{} g(u)\ du\leq 1+2M$.\\
On montrerait de même que $\dfrac{1}{x} 
\Sum{n\in\Z}{} \dint{I_{n,x}}{} g(u)\ du\geq 1-2M$, inégalité qu'on 
{\bf admettra}.

\item Montrer que l'événement 
$\Ev{\{Y\}<x}$ est égal à $\dcup{n\in\Z}{}\Ev{Y\in I_{n,x}}$.

\item Déduire que  $\vert \Prob(\Ev{\{Y\}<x})-x \vert \leq 2M$.
\end{nonoliste}
\end{noliste}

\item
Soit $(Z_n)_{n \geq 1}$ une suite de variables aléatoires, où 
$Z_n$ suit une loi exponentielle de paramètre $\frac{1}{n}$.\\ 
On pose $X_n=10^{\sqrt{Z_n}}$ et $Y_n=\log (X_n)=\sqrt{Z_n}$. 
 \begin{noliste}{a)}
\item Déterminer une densité $g_n$ de la loi $Y_n$, continue sur 
$\R$. 
\item \'Etudier les variations de $g_n$ sur $\R_+$ et déterminer son 
maximum.
\item Montrer que pour tout $x\in \ ]0,1[$, 
$\vert \Prob(\Ev{\{Y_n\}<x})-x \vert \leq 2\sqrt{\dfrac{2}{n}} 
\ee^{-\frac{1}{2}}$. 
\item {\bf Seulement pour les cubes} : \\
Montrer que la suite $(\{ 
Y_n\})_{n\geq 1}$ converge en loi vers la loi uniforme sur $[0,1[$.
\end{noliste}
\end{noliste}


\section{Répartition des valeurs dans une table numérique}

\noindent
Henri Poincaré (1854-1912) a proposé au début du $\eme{20}$ siècle 
une façon originale d'étudier la répartition des valeurs d'une 
table numérique en montrant que pour un bon choix d'une fonction $F$ 
de période assez grande par rapport à l'incrémentation des 
valeurs de la table, la moyenne des valeurs prises par $F$ sur la table 
sera petite, ce qui indique une certaine forme d'équilibre dans la 
répartition de ces valeurs.\\
Poincaré considère l'exemple des valeurs d'une table de logarithmes :
\[
z_n=\ln\left( 1+\frac n{100 \, 000}\right)
\]
pour $n=1,2,3,\ldots , 10 \, 000$ et pose $F(y)=\sin (1000 
\times \pi \times y)$, fonction de période $\frac{1}{500}$, grande par 
rapport à l'incrémentation $\frac{1}{100 \, 000}$ dans la table. Il 
s'intéresse à la moyenne des valeurs de $F$ sur la table, c'est à dire 
à :
\[
S=\dfrac{1}{10 \, 000}\Sum{k=1}{10 \, 000}F(z_k)
\]
et désire montrer que cette valeur est petite.\\
Posons 
\[
J=\dfrac{1}{10 \, 000}\dint{\frac{1}{2}}{10 \, 000+\frac{1}{2}} 
F\left(\ln \left(1+\frac x{100 \, 000} \right)\right) \dx
\]

\newpage

\begin{noliste}{1.}
\setlength{\itemsep}{2mm}
\setcounter{enumi}{5}
\item
\begin{noliste}{a)}
\item Soit $\varphi$ une fonction de classe ${\cal C}^2$ sur $[\frac 12 
, +\infty [$. On suppose qu'il existe un réel $M>0$ tel que, pour 
tout $u\in [\frac 12, +\infty [$, $\vert \varphi ''(u) \vert \leq M$. \\
Montrer que pour $n\in\N^*$ et $\vert h \vert \leq \frac{1}{2}$, on a 
\[
-\dfrac{M}{8}\leq \varphi (n+h)-\varphi (n)-h \, \varphi'(n)\leq
\dfrac{M}{8}
\]

\item Déduire que pour tout $n\in\N^*$, 
\[
-\dfrac{M}{8}\leq \dint{n-\frac{1}{2}}{n+\frac{1}{2}}\varphi (u)\ 
du-\varphi (n)\leq \dfrac{M}{8}
\]

\item À partir de cette question, dans la partie II, on utilise des 
fonctions trigonométriques qui ne sont pas connues en voie 
E. Je les ai donc supprimées mais vous pouvez aller les consulter sur 
internet.
\end{noliste}
\end{noliste}

\section{Sur les nombres normaux}

\noindent
Dans cette partie, on se donne un espace probabilisé $(\Omega, \A, 
\Prob)$ et on notera comme d'habitude, sous réserve d'existence, 
$\E(X)$ et $\V(X)$ l'espérance et la variance d'une variable aléatoire 
réelle $X$. \\
On commence par rappeler les deux points de théorie suivants.
\begin{nonoliste}{(i)}
\item Pour toute suite d'événements $(C_k)_{k\in\N}$ dans $\Omega$, 
on a $\Prob(\dcup{k=0}{+\infty }C_k) \leq \Sum{k=0}{\infty}\Prob(C_k)$ 
avec la convention que cette série vaut $+\infty$ si elle diverge.

\item Si $(C_k)_{k\in\N}$ est une suite décroissante d'événements 
dans $\Omega$, au sens où $\forall k\geq 0$, $C_k\supset C_{k+1}$, on 
a $\Prob(\dcap{k\geq 0}{}C_k)=\dlim{k\to+\infty}\Prob(C_k)$
\end{nonoliste}

\noindent
On rappelle aussi l'\emph {inégalité de Markov} : si $Z$  est une 
variable aléatoire positive admettant une espérance $\E(Z)$, pour 
tout $\alpha >0$, on a $\Prob(\Ev{Z>\alpha}) 
\leq \dfrac{\E(Z)}{\alpha}$.\\
On considère ici le tirage au sort d'un nombre réel entre $0$ et 
$1$ qu'on modélise de la fa\c{c}on suivante : $(X_n)_{n\geq 1}$ est 
une suite de variables aléatoires indépendantes, de m\^eme loi 
uniforme à valeurs dans $\{ 0,1,2,\ldots , 9\}$. Les $(X_n)_{n\geq 
1}$ représentent les décimales du nombre tiré au hasard c'est à 
dire que ce nombre est $\Sum{k=1}{+\infty}\dfrac{X_k}{10^k}$.\\
On définit enfin pour tout $k\geq 1$ une variable aléatoire $Y_k$ 
à valeurs $0$ ou $1$ par $Y_k=1$ si $X_k=1$ et $Y_k=0$ si $X_k\neq 
1$. 
\begin{noliste}{1.}
\setlength{\itemsep}{2mm}
\setcounter{enumi}{6}
\item 
\begin{noliste}{a)}
\item Montrer que les variables $Y_k$ sont indépendantes et de même 
loi que l'on précisera.

\item Déterminer $\E(Y_k)$ et $\V(Y_k)$. \\
On pose $S_n=\Sum{k=1}{n} Y_k$. par conséquent, $\dfrac{S_n}{n}$ 
représente la fréquence des $1$ dans la suite des décimales du 
nombre tiré. 

\item Calculer  $\V(\frac{S_n}{n})$ en fonction de $n$.

\item Soit $\varepsilon >0$ fixé. Montrer 
$\Prob(\Ev{\vert \frac{S_n}{n}-\frac{1}{10}\vert > \varepsilon})
\leq \dfrac{\V(\frac{S_n}{n})}{\varepsilon^2 } $.

\item En déduire que 
\[
\dlim{n\to+\infty} \Prob\left(\Ev{\left\vert \frac{S_n}{n}-\frac{1}{10} 
\right\vert >\varepsilon}\right)=0
\]
\end{noliste}

\noindent
On va dans la suite améliorer ce résultat en montrant qu'en fait 
pour la plupart des nombres réels, la fréquence des $1$ dans leurs 
décimales vaut $\frac{1}{10}$. 

\newpage

\item 
\begin{noliste}{a)}
\item On pose $A=\dcap{N\geq 1}{} \dcup{k=N}{\infty}A_k$. Montrer 
que $A$ est l'ensemble des $\omega$ qui appartiennent à une 
infinité d'événements $A_k$.

\item On pose, pour tout $N\geq 1$, $B_N=\dcup{k=N}{\infty} A_k$. 
Montrer que $\forall N\geq 0$, $B_N\supset B_{N+1}$.

\item Déduire que $\dlim{N\to +\infty} \Prob(B_N)=\Prob(A)$.

\item On suppose que $\Sum{k=1}{\infty} \Prob(A_k)<+\infty$. 
\begin{noliste}{i.}
\item Que vaut 
$\dlim{N\to+\infty} \Sum{k=N}{+\infty} \Prob(A_k)$ ?

\item Conclure que $\Prob(A)=0$.
\end{noliste}
\end{noliste}

\item On pose, pour tout $k\geq 1$, $Y'_k=Y_k-\frac{1}{10}$.

\begin{noliste}{a)}
\item Montrer que 
$\dfrac{S_n}{n}-\dfrac{1}{10}=\dfrac{1}{n} \Sum{k=1}{n} Y'_k$.

\item Montrer que les variables $Y'_k$  sont indépendantes, 
d'espérance nulle et telles que  $\vert Y'_k \vert \leq 1$. 

\item Montrer que 
$\E\left(\left(\Sum{k=1}{n} Y'_k\right)^4\right)\leq n+3n(n-1)$.

\item Déduire que 
\[
\E\left(\left(\dfrac{S_n}{n}-\dfrac{1}{10}\right)^4\right)\leq 
\dfrac{3}{n^2}
\]

\item On pose, pour $k\geq 1$, $A_k=\Ev{\left((\dfrac{S_k}{k}-
\dfrac{1}{10}\right)^4 > \dfrac{1}{\sqrt{k}}}$. Montrer que 
$\Prob(A_k)\leq \dfrac{3}{k^{\frac{3}{2}}}$.

\item Déduire que  $\Sum{k\geq 1}{} \Prob(A_k)$ est une série 
convergente. 

\item On considère l'événement $A=\{\omega\in\Omega , \; 
\left(\frac{S_k(\omega)}{k}- \frac{1}{10}\right)^4 > 
\frac{1}{\sqrt{k}}\mbox{ pour une infinité de } k\}$. Montrer que 
$\Prob(A)=0$.

\item Déduire qu'avec probabilité $1$, on peut trouver $N$ tel que 
pour tout $k\geq N$,  
$\left\vert \dfrac{S_k}{k}-\dfrac{1}{10}\right\vert \leq 
\dfrac{1}{\sqrt[8]{k}}$.

\item Conclure qu'avec probabilité $1$, on a , 
$\dlim{k\to+\infty}\dfrac{S_k}{k}=\dfrac{1}{10}$. 
\end{noliste}
\end{noliste}






\end{document}
