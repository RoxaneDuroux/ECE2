\documentclass[11pt]{book}%
\usepackage{geometry}%
\geometry{a4paper,
  lmargin=2cm,rmargin=2cm,tmargin=2.5cm,bmargin=2.5cm}
  
\input{../../macros.tex}



\begin{document}

\chapter*{Contenus sujets ESSEC II}

\begin{noliste}{$\sbullet$}
\item  \underline{2016} :
 \begin{noliste}{-}
  \item  classique :
  \[
   \E(X)=\Sum{j=0}{+\infty} \Prob(\Ev{X>j})
  \]
  
  \item \Scilab{} infaisable
  
  \item réduction dans $\M{2}$
  
  \item Bienaymé-Tchebychev et autres inégalités de concentration
  
  \item intervalles de confiance
 \end{noliste}

\item \underline{2015} :
 \begin{noliste}{-}
  \item $\liminf$
  
  \item lois sous-exponentielles : bien
  
  \item loi de sommes, produit de convolution
  
  \item lois à queues lourdes (\var à densité)
 \end{noliste}
 
\item \underline{2013} :
 \begin{noliste}{-}
  \item convergence loi binomiale vers loi de Poisson
  
  \item loi du $\max$
  
  \item critère de convergence des SATP
  
  \item SATP double
  
  \item utilisation approximation de la Poisson par une binomiale pour 
  un cas particulier (concentration en bactéries)
  
  \item démo inégalité de Markov
  
  \item inégalités de concentration et intervalle de confiance (pour 
  créer un test)
 \end{noliste}
 
\item \underline{2008} :
 \begin{noliste}{-}
  \item PageRank Google : très sale
  
  \item manipulations valeur absolue
  
  \item montrer qu'une matrice est une matrice de transition
  
  \item trouver un invariant pour cette matrice
  
  \item formule des proba totales
  
  \item existence d'un invariant pour une matrice stochastique.
  
  \item cas particulier des matrices de $\M{2}$
  
  \item unicité de l'invariant dans le cas général
  
  \item manipulation de normes de matrices
  
  \item convergence en loi
 \end{noliste}
 
 \newpage
 
\item \underline{2000} :
 \begin{noliste}{-}
  \item proba d'événements, intersections, unions : très classique
  
  \item matrices par blocs
 \end{noliste}
 
\item \underline{1996} :
 \begin{noliste}{-}
  \item fonction génératrice (beaucoup)
  
  \item ses propriétés (définie sur $[0,1]$, croissante, continue à 
  gauche en $1$)
  
  \item produit de fonctions génératrices
  
  \item ruine du joueur avec fonction géneratrice de la Bernoulli
  
  \item temps d'attente de la ruine du joueur
 \end{noliste}
 
\item \underline{1994} : (court)
 \begin{noliste}{-}
  \item suite récurrente linéaire d'ordre $3$
  
  \item puissance $\eme{n}$ d'une matrice connaissant un polynôme 
  annulateur
  
  \item marche aléatoire sur un cube (chaîne de Markov)
 \end{noliste}
 
\item \underline{1991} : (court)
 \begin{noliste}{-}
  \item diagonalisation d'une matrice de transition de $\M{3}$
  
  \item puissances $\eme{n}$
  
  \item chaîne de Markov associée à cette matrice de transition
  
  \item temps d'attente $T$ jusqu'au premier succès de la chaîne
  
  \item programmation de la fonction de répartition de $T$
 \end{noliste}
\end{noliste}

\end{document}