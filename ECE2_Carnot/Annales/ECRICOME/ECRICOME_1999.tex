\documentclass[11pt]{article}%
\usepackage{geometry}%
\geometry{a4paper,
 lmargin = 2cm,rmargin = 2cm,tmargin = 2.5cm,bmargin = 2.5cm}

\input{../../macros.tex}

\pagestyle{fancy} %
\lhead{ECE2 \hfill Mathématiques\\
} %
\chead{\hrule} %
\rhead{} %
\lfoot{} %
\cfoot{} %
\rfoot{\thepage} %

\renewcommand{\headrulewidth}{0pt}% : Trace un trait de séparation
 % de largeur 0,4 point. Mettre 0pt
 % pour supprimer le trait.

\renewcommand{\footrulewidth}{0.4pt}% : Trace un trait de séparation
 % de largeur 0,4 point. Mettre 0pt
 % pour supprimer le trait.

\setlength{\headheight}{14pt}

\title{\bf \vspace{-2cm} ECRICOME 1999} %
\author{} %
\date{} %
\begin{document}

\maketitle %
\vspace{-1.4cm}\hrule %
\thispagestyle{fancy}

\vspace*{.2cm}


% DEBUT DU DOC À MODIFIER : tout virer jusqu'au début de l'exo

%Définition et changement de valeurs de
compteurs%newcounter{cpt1}{section} compteur cpt1 remis à 0 à chaque
aumentation par stepcounter du compteur section%setcounter{cpt1}{3} on
met le compteur à 3%addtocounter{cpt1}{5} on ajoute 5 au compteur%
stepcounter{cpt1} on ajoute 1% ifthenelse{test}{alors}{sinon} (page
206) pour subordonner à une condition % whiledo{test}{commande} pour
faire une boucle (page 206 aussi) % value{cpt1} pour noter dans le
document la valeur de cpt1 
%Définition définitive d'opérateurs
mathématiques\newcommand{\ch}{\operatorname{ch}} 
\newcommand{\sh}{\operatorname{sh}}
\renewcommand{\tanh}{\operatorname{th}}
\renewcommand{\sinh}{\operatorname{sh}}
\renewcommand{\cosh}{\operatorname{ch}}
\newcommand{\argsh}{\operatorname{argsh}}
\newcommand{\argch}{\operatorname{argch}}
\newcommand{\argth}{\operatorname{argth}}
\newcommand{\Id}{\operatorname{Id}}
\renewcommand{\leq}{\leq}
\renewcommand{\geq}{\geq }

\newcommand{\dlim}{\lim}
\newcommand{\dsum}{\sum}
\newcommand{\dint}{\int}
\newcommand{\dprod}{\prod}



%Définition de nouvelles couleurs : rgb(trois paramètres red green blue
entre 0 et 1); cmyk (quatre cyan magenta yellow black) entre 0 et 1;
gray (entre 0 et 1) et black, white, red, green, blue, cyan, magenta,
yellow% definecolor{0gris}{gray}{0.8} 
% Nouvelle commande pour encadrer le titre car shabox ne veut que d'une
seule ligne; ATTENTION A LA TAILLE; petite différence avec shadowbox ou
doublebox, voire fcolorbox ou colorbox (au lieu de shabox; laisser le
parbox tranquille sauf pour la taille de la boîte
\newcommand{\Tbox}[1]{\begin{center} \shabox{\parbox{0.6
\linewidth}{#1}} \end{center}} %[1] pour 1 paramètre ; #1 pour ce que
fait le 1er paramètre; entre accolades ce que fait la commande
%Mise en page en mode fancy : en-têtes et pieds de pages puis
définition des en-têtes et pieds de pages\pagestyle{fancy}
\lhead{ECE 2 - Mathématiques \\
Quentin Dunstetter - ENC-Bessières 2011$\backslash$2012}
\chead{}
\rhead{Ecricome 1999}
\rfoot[ \ \thepage]{\thepage}
\cfoot{}
\lfoot{}
\thispagestyle{fancy} %Mise en page de la 1ère page en mode fancy
%Trait en bas et en haut de la page (entre en-tête et texte et texte et
pied de page)\renewcommand{\footrulewidth}{0.4pt}
\renewcommand{\headrulewidth}{0.4pt}


\begin{center}
{\Huge ECRICOME Eco 1999}
\end{center}

\begin{center}
\textbf{\large Exercice 1 }
\end{center}



\noindent\textbf{\ Préliminaire}

Soit $(x_{n})$ une suite numérique qui vérifie, pour tout entier
naturel $n
$, la relation~ : 
\[
x_{n + 2} = \frac{1}{3} x_{n + 1} + \frac{1}{3} x_{n}
\]
Montrer que~ : $ \dlim{n \rightarrow + \infty} x_{n} = 0$\\
(on donne : $\frac{1 + \sqrt{13}}{6} = 0,77$ à $10^{-2}$ près par excès
et $\frac{1-\sqrt{13}}{ 6} = -0,44$ à $10^{-2}$ près par défaut).

$a$ et $b$ sont deux réels supérieurs ou égaux à 1.\\
On étudie la suite numérique $(u_{n})$ définie par~ : $u_{0} = a
\;\;u_{1} = b$
et pour tout entier naturel $n$~ : 
\[
u_{n + 2} = \sqrt{u_{n}} + \sqrt{u_{n + 1}}
\]

\textbf{\ Question 1}

\begin{noliste}{1.}
 \setlength{\itemsep}{4mm}
\item[ \ \textbf{\ 1.a)}] Montrer que, pour tout entier naturel $n$,
$u_{n}$ est
bien défini et vérifie $u_{n} \geq 1$.

\item[ \ \textbf{\ 1.b)}] Montrer que la seule limite possible de la
suite $(u_{n})$ est 4.

\item[ \ \textbf{\ 1.c)}] Écrire un programme en \Scilab{} qui calcule
et
affiche la valeur de $u_{n}$ pour des valeurs de $a$ et $b$ réelles
supérieures ou égales à 1 et de $n$ entier supérieur ou égal à
2, entrées par l'utilisateur.
\end{noliste}

\textbf{\ Question 2}

On se propose d'établir la convergence de la suite $(u_{n})$ par
l'étude
d'une suite auxiliaire $(v_{n})$ définie, pour tout entier naturel $n$,
par~ : 
\[
v_{n} = \frac{1}{2} \sqrt{u_{n}} -1
\]

\begin{noliste}{1.}
 \setlength{\itemsep}{4mm}
\item[ \ \textbf{\ 2.a)}] Montrer que si $ \dlim{n \rightarrow
 + \infty}v_{n} = 0$ alors $ \dlim{n \rightarrow + \infty}u_{n} = 4.$

\item[ \ \textbf{\ 2.b)}] Vérifier, pour tout entier naturel $n$~ : $
v_{n + 2} = \frac{v_{n + 1} + v_{n}}{ 2(2 + v_{n + 2})}.$

En déduire que~ : $ |v_{n + 2}|\leq \frac{1}{ 3}\left(|v_{n + 1}| + 
|v_{n}| \right).$

\item[ \ \textbf{\ 2.c)}] On note $(x_{n})$ la suite définie par~ :
$x_{0} = |v_{0}|$, 
$x_{1} = |v_{1}|$ et, pour tout entier naturel $n$, 
\[
x_{n + 2} = \frac{1}{3} x_{n + 1} + \frac{1}{3}x_{n}
\]
Montrer que, pour tout entier naturel $n$, $|v_{n}| \leq x_{n}$ et
conclure
quant à la convergence de la suite $(u_{n})$.
\end{noliste}

\newpage

\begin{center}
\textbf{\large Exercice 2 }
\end{center}

\noindent Toutes les matrices de cet exercice sont des éléments
de l'ensemble $E$ des matrices carrées d'ordre 2 à coefficients réels.

On pose~ : $O = 
\begin{smatrix}
0 & 0 \\
0 & 0
\end{smatrix}
$, $I = 
\begin{smatrix}
1 & 0 \\
0 & 1
\end{smatrix}
$ et $H = \left\{M \in E \; / \; \text{il existe}\; \alpha \in \R
\;\text{ tel que }\; M = \alpha I \right\}$ et pour toute matrice
réelle $M = 
\begin{smatrix}
a & b \\
c & d
\end{smatrix}
$, $\tau(M) = a + d$ et $\delta(M) = ad-bc$.

\textbf{\ Question 1}

On dit que la suite de matrices $(A_{n})_{n \in \N}$ où $A_{n} = 
\begin{smatrix}
a_{n} & b_{n} \\
c_{n} & d_{n}
\end{smatrix}
$ converge vers la matrice $O$ si $(a_{n})$, $(b_{n})$, $(c_{n})$ et
$(d_{n})$ sont
des suites réelles de limite nulle.

Justifier les résultats suivants :

\begin{noliste}{1.}
 \setlength{\itemsep}{4mm}
\item[ \ \textbf{\ 1 a)}] Soient $(A_{n})_{n \in \N}$ et $(B_{n})_{n
\in 
\N}$ deux suites de matrices, $\lambda$ un réel et $M$ une matrice :

si $(A_{n})_{n \in \N}$ et $(B_{n})_{n \in \N}$ convergent vers
la matrice $O$, alors $(A_{n} + B_{n})_{n \in \N}$, $(\lambda \
A_{n})_{n
\in \N}$, $(M \ A_{n})_{n \in \N}$ et $(A_{n}\ M)_{n \in \N}$
convergent aussi vers $O$.

\item[ \ \textbf{\ 1 b)}] Si $D = 
\begin{smatrix}
\lambda & 0 \\
0 & \mu
\end{smatrix}
$ avec $|\lambda|<1$ et $|\mu|<1$,la suite de matrices $(D^{n})_{n \in
\N}$ converge vers $O$.

\item[ \ \textbf{\ 1 c)}] Si une matrice $A$ est diagonalisable, de
valeurs
propres $\lambda$ et $\mu$ tels que $|\lambda|<1$ et $|\mu|<1$, alors
la
suite $(A^{n})_{n \in \N}$ converge vers la matrice $O$.
\end{noliste}

\textbf{\ Question 2}

Dans toute cette question, $A$ désigne un élément de $E$ tel que
$\delta(A)<0$.\\
On se propose de montrer qu'une telle matrice est diagonalisable.

\begin{noliste}{1.}
 \setlength{\itemsep}{4mm}
\item[ \ \textbf{\ 2 a)}] Montrer que $A$ n'est pas élément de $H$.

\item[ \ \textbf{\ 2 b)}] Vérifier par le calcul que, pour tout élément
$M
$ de $E$, on a : 
\[
M^{2} = \tau(M) \ M - \delta(M) \ I \qquad(*)
\]

\item[ \ \textbf{\ 2 c)}] Montrer qu'il existe deux réels distincts
$\lambda$
et $\mu$ tels que : $\lambda + \mu = \tau(A)$ et $\lambda \mu =
\delta(A).$

\item[ \ \textbf{\ 2 d)}] On pose $M = A - \lambda I$ et $N = A - \mu
I.$\\
Montrer que $M N = O$ et en déduire que l'hypothèse ``$M$ est
inversible"
conduit à une contradiction.\\
Montrer de même que $N$ n'est pas inversible.

\item[ \ \textbf{\ 2 e)}] En déduire que $A$ est diagonalisable et
qu'il
existe une matrice $P$ de $E$ inversible telle que $A = P \ D \ P^{-1}$
avec $D = 
\begin{smatrix}
\lambda & 0 \\
0 & \mu
\end{smatrix}
$.
\end{noliste}

\textbf{\ Question 3}

On note $U$ l'ouvert de $\R^{2}$ défini par $ U = \ ]\frac{1}{ 3},
\frac{2}{ 3 }[ \ \times ]0,1[$ et $f$ l'application définie sur $U$
par~ : 
\[
(x,y) \rightarrow f(x,y) = x^{2}-x + x y^{2}-xy
\]

\begin{noliste}{1.}
 \setlength{\itemsep}{4mm}
\item[ \ \textbf{\ 3 a)}] Montrer que $f$ est strictement négative sur
$U$.

\item[ \ \textbf{\ 3 b)}] Montrer (en rédigeant soigneusement) que $f$
admet
un unique extremum sur $U$ et que celui-ci est un minimum dont on
donnera la
valeur.

En déduire que, pour tout élément $(x,y)$ de $U$ : $ -\frac{25}{64}
\leq f(x,y) <0$.
\end{noliste}

\textbf{\ Question 4}

Soient $a$ et $b$ deux réels tels que $(a,b)$ soit un élément de
l'ouvert $U$ défini précédemment.

On pose $Q = 
\begin{smatrix}
a & b \\
a(1-b) & a-1
\end{smatrix}.$

On se propose de montrer que la suite de matrices $(Q^{n})_{n \in \N}$
converge vers $O$.

\begin{noliste}{1.}
 \setlength{\itemsep}{4mm}
\item[ \ \textbf{\ 4 a)}] Calculer $\tau(Q)$ et $\delta(Q).$\\
Vérifier que les résultats de la question 2 s'appliquent pour $A = Q$
et
en déduire que $Q$ admet deux valeurs propres distinctes $\lambda$ et
$\mu$
telles que : 
\[
-\frac{1}{3} < \lambda + \mu < \frac{1}{3} \;\;\;\text{ et } \;\;\;
-\frac{25}{64} \leq \lambda \mu \leq 0
\]

\item[ \ \textbf{\ 4 b)}] Exprimer $\lambda^{2} + \mu^{2}$ en fonction
de $\lambda + \mu
$ et $\lambda \mu$ et en déduire que $\lambda^{2} + \mu^{2} <1$.\\
Pourquoi peut on affirmer que la suite $(Q^{n})_{n \in \N}$ converge
vers $O$~ ?
\end{noliste}



\begin{center}
\textbf{\large Exercice 3 }
\end{center}

On modélise la durée de fonctionnement d'un appareil par une
variable aléatoire réelle $T$ définie sur un certain espace
probabilisé $(\Omega,\mathcal{A},P)$, admettant une densité $f$.

On note $F$ sa fonction de répartition et on suppose que $F$ vérifie
les
propriétés~ :

\begin{noliste}{$\sbullet$}
\item $F(t) = 0$ pour tout réel $t \leq 0$.

\item $F$ est de classe $C^{1}$ et strictement croissante sur $\R^+ $.
\end{noliste}

Sous une hypothèse introduite dans la question 2, on se propose
d'expliciter $F$ et $f$, puis de calculer l'espérance $\E(T)$ de $T$,
\emph{``temps moyen de fonctionnement''}.



\textbf{\ Question 1}

On rappelle que l'intégrale généralisée $\dint{0}{+
\infty}e^{-u^{2}}du$ converge et vaut $ \frac{1}{2} 
\sqrt{\pi}$.\\
Soit $\alpha$ un réel strictement positif; si $x$ est un élément de
$\R^+ $, on pose~ : 
\[
I(x) = \dint{0}{x}2u^{2}e^{-u^{2}}du \;\; \text{ et } \;\; J(x) =
\dint{0}{x}
t^{2}e^{-(t/\alpha)^{2}}dt
\]

\begin{noliste}{1.}
 \setlength{\itemsep}{4mm}
\item[ \ \textbf{\ 1 a)}] À l'aide d'un changement de variable,
exprimer,
pour tout élément $x$ de $\R^+ $, $J(x)$ en fonction de $
I\left(\frac{x}{ \alpha}\right)$.

\item[ \ \textbf{\ 1 b)}] À l'aide d'une intégration par parties,
montrer
que, pour tout élément $x$ de $\R^+ $ :
\[
I(x) = \dint{0}{x} e^{-u^{2}} du - x e^{-x^{2}}
\]

\item[ \ \textbf{\ 1 c)}] En déduire que l'intégrale généralisée $
\dint{0}{+ \infty} t^{2}e^{-(t/\alpha)^{2}}dt$ converge et vaut $
\frac{\alpha^{3} \sqrt{\pi}}{4}$.
\end{noliste}

\textbf{\ Question 2}

\begin{noliste}{1.}
 \setlength{\itemsep}{4mm}
\item[ \ \textbf{\ 2 a)}] Montrer que, pour tout élément $u$ de $\R^+
$, $F(u)<1$, puis en déduire que $P\left(\Ev{T \geq u}\right) \neq 0$.

\item[ \ \textbf{\ 2 b)}] Soient $t_{0}$ et $t$ des réels tels que
$0\leq
t_{0}\leq t$.\\
On pose~ : $q(t_{0},t) = \frac{1}{t-t_{0}}P_{T\geq
t_{0}}(t_{0}\leq T\leq t\,)$ (probabilité conditionnelle).\\
$q(t_{0},t)$ est \emph{le taux d'arrêt de fonctionnement entre les
instants $t_{0}$ et $t$}.\\
On définit ensuite, sous réserve d'existence, le \emph{taux d'arrêt de
fonctionnement instantané en $t_{0}$} par~ : $\tau
(t_{0}) = \dlim{t\rightarrow t_{0}{+}}q(t_{0},t).$

Exprimer $q(t_{0},t)$ en fonction de $t$, $t_{0}$, $F(t_{0})$ et
$F(t)$.\\
En déduire que $\tau (t_{0})$ existe et que $\tau (t_{0}) =
\frac{F^{\prime }(t_{0})}{1-F(t_{0})}.$

\hspace{-1cm}Dans la suite de l'énoncé, on fait l'hypothèse
suivante :

il existe un réel $c>0$ tel que, pour tout élément $t$ de $\R^{+}$,
$\tau (t) = ct$.

\item[ \ \textbf{\ 2 c)}] Montrer que, pour tout élément $t$ de
$\R^{+}$, on a~ : $\ln \left[ 1-F(t)\right] = c\frac{t^{2}}{2}$

\item[ \ \textbf{\ 2 d)}] Soit $t$ un élément de $\R^{+}$.\\
Expliciter $F(t)$, puis montrer, en posant $\alpha =
\sqrt{\frac{2}{c}}$, que $f(t) = \frac{2}{\alpha ^{2}}te^{-(t/\alpha
)^{2}}$.

\item[ \ \textbf{\ 2 e)}] Montrer enfin que $\E(T)$ existe et donner sa
valeur
en fonction de $c$.
\end{noliste}

\end{document}

