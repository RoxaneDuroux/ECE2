\documentclass[11pt]{article}%
\usepackage{geometry}%
\geometry{a4paper,
 lmargin = 2cm,rmargin = 2cm,tmargin = 2.5cm,bmargin = 2.5cm}

\input{../../macros.tex}

\pagestyle{fancy} %
\lhead{ECE2 \hfill Mathématiques\\
} %
\chead{\hrule} %
\rhead{} %
\lfoot{} %
\cfoot{} %
\rfoot{\thepage} %

\renewcommand{\headrulewidth}{0pt}% : Trace un trait de séparation
 % de largeur 0,4 point. Mettre 0pt
 % pour supprimer le trait.

\renewcommand{\footrulewidth}{0.4pt}% : Trace un trait de séparation
 % de largeur 0,4 point. Mettre 0pt
 % pour supprimer le trait.

\setlength{\headheight}{14pt}

\title{\bf \vspace{-2cm} ECRICOME 2005} %
\author{} %
\date{} %
\begin{document}

\maketitle %
\vspace{-1.4cm}\hrule %
\thispagestyle{fancy}

\vspace*{.2cm}


% DEBUT DU DOC À MODIFIER : tout virer jusqu'au début de l'exo

%Définition et changement de valeurs de
compteurs%newcounter{cpt1}{section} compteur cpt1 remis à 0 à chaque
aumentation par stepcounter du compteur section%setcounter{cpt1}{3} on
met le compteur à 3%addtocounter{cpt1}{5} on ajoute 5 au compteur%
stepcounter{cpt1} on ajoute 1% ifthenelse{test}{alors}{sinon} (page
206) pour subordonner à une condition % whiledo{test}{commande} pour
faire une boucle (page 206 aussi) % value{cpt1} pour noter dans le
document la valeur de cpt1 
%Définition définitive d'opérateurs
mathématiques\newcommand{\ch}{\operatorname{ch}} 
\newcommand{\sh}{\operatorname{sh}}
\renewcommand{\tanh}{\operatorname{th}}
\renewcommand{\sinh}{\operatorname{sh}}
\renewcommand{\cosh}{\operatorname{ch}}
\newcommand{\argsh}{\operatorname{argsh}}
\newcommand{\argch}{\operatorname{argch}}
\newcommand{\argth}{\operatorname{argth}}
\newcommand{\ker}{\operatorname{Ker}}
\renewcommand{\im}{\operatorname{Im}}
\newcommand{\rg}{\operatorname{rg}}
\newcommand{\Id}{\operatorname{Id}}
\renewcommand{\leq}{\leq}
\renewcommand{\geq}{\geq }

\newcommand{\NN}{\mbox{${\mathbb N}$}}
\newcommand{\ZZ}{\mbox{${\mathbb Z}$}}
\newcommand{\QQ}{\mbox{${\mathbb Q}$}}
\newcommand{\RR}{\mbox{${\mathbb R}$}}
\newcommand{\MM}{\mbox{${\mathcal{M}}$}}
\newcommand{\CC}{\mbox{${\mathcal{C}}$}}
\newcommand{\DD}{\mbox{${\mathcal{D}}$}}

\newcommand{\vpp}{\vspace{0.1cm}}
\newcommand{\vp}{\vspace{0.2cm}}
\newcommand{\vg}{\vspace{0.4cm}}
\newcommand{\dsp}{\def \tend #1{\xrightarrow[ \
\phantom{a}#1\phantom{a}]{}}
\def \equi #1{\mathop{\sim}_{\substack{#1}}}
\def 
\[
 {[ \ \![}
\def 
\]
 {]\!]}
\newcommand{\pg}{\geq }
\newcommand{\pp}{\leq}
\newcommand{\dt}{\,\textrm{d}t}
\newcommand{\dx}{\,\textrm{d}x}
\newcommand{\du}{\,\textrm{d}u}
\newcommand{\Ent}{\textrm{Ent}}
\newcommand{\bul}{\item[$\bullet$]}

%Définition de nouvelles couleurs : rgb(trois paramètres red green blue
entre 0 et 1); cmyk (quatre cyan magenta yellow black) entre 0 et 1;
gray (entre 0 et 1) et black, white, red, green, blue, cyan, magenta,
yellow% definecolor{0gris}{gray}{0.8} 
% Nouvelle commande pour encadrer le titre car shabox ne veut que d'une
seule ligne; ATTENTION A LA TAILLE; petite différence avec shadowbox ou
doublebox, voire fcolorbox ou colorbox (au lieu de shabox; laisser le
parbox tranquille sauf pour la taille de la boîte
\newcommand{\Tbox}[1]{\begin{center} \shabox{\parbox{0.6
\linewidth}{#1}} \end{center}} %[1] pour 1 paramètre ; #1 pour ce que
fait le 1er paramètre; entre accolades ce que fait la commande
%Mise en page en mode fancy : en-têtes et pieds de pages puis
définition des en-têtes et pieds de pages\pagestyle{fancy}
\lhead{ECE 2 - Mathématiques \\
Quentin Dunstetter - ENC-Bessières 2011$\backslash$2012}
\chead{}
\rhead{Ecricome 2005}
\rfoot[ \ \thepage]{\thepage}
\cfoot{}
\lfoot{}
\thispagestyle{fancy} %Mise en page de la 1ère page en mode fancy
%Trait en bas et en haut de la page (entre en-tête et texte et texte et
pied de page)\renewcommand{\footrulewidth}{0.4pt}
\renewcommand{\headrulewidth}{0.4pt}


\noindent {\Large \textbf{ECRI%TCIMACRO{\TeXButton{TeX
field}{\colorbox[gray]{0.95}{COME}}}%BeginExpansion
\colorbox[gray]{0.95}{COME}%EndExpansion
}}\vspace{0.3cm}

\noindent \textbf{Banque d'épreuves communes}

\noindent aux concours des Ecoles

\noindent esc%TCIMACRO{\TeXButton{TeX
field}{\colorbox[gray]{0.95}{bordeaux}} }%BeginExpansion
\colorbox[gray]{0.95}{bordeaux}
%EndExpansion
/ esc%TCIMACRO{\TeXButton{TeX field}{\colorbox[gray]{0.95}{marseille}}
}%BeginExpansion
\colorbox[gray]{0.95}{marseille}
%EndExpansion
/ icn%TCIMACRO{\TeXButton{TeX field}{\colorbox[gray]{0.95}{nancy}}
}%BeginExpansion
\colorbox[gray]{0.95}{nancy}
%EndExpansion
/ esc%TCIMACRO{\TeXButton{TeX field}{\colorbox[gray]{0.95}{reims}}
}%BeginExpansion
\colorbox[gray]{0.95}{reims}
%EndExpansion
/ esc%TCIMACRO{\TeXButton{TeX field}{\colorbox[gray]{0.95}{rouen}}
}%BeginExpansion
\colorbox[gray]{0.95}{rouen}
%EndExpansion
/ esc%TCIMACRO{\TeXButton{TeX
field}{\colorbox[gray]{0.95}{toulouse}}}%BeginExpansion
\colorbox[gray]{0.95}{toulouse}%EndExpansion
\vspace{1cm}

\begin{center}
{\large CONCOURS D'ADMISSION}

\textbf{option économique}

{\Large \textbf{MATHÉMATIQUES}}

\textbf{Année 2005}
\end{center}

\noindent \textbf{Aucun instrument de calcul n'est autorisé.}

\noindent \textbf{Aucun document n'est autorisé.}

\noindent L'énoncé comporte \pageref{fin} pages

\begin{quotation}
\noindent Les candidats sont invités à soigner la présentation de leur
copie, à mettre en évidence les principaux résultats, à respecter les
notations de l'énoncé, et à donner des démonstrations complètes (mais
brèves) de leurs affirmations.
\end{quotation}

\vspace{13cm}

\hfill \textbf{Tournez la page}

\hfill \textbf{S.V.P\qquad }

\newpage

\section*{EXERCICE1}

On considère, pour tout entier naturel $n$, l'application
$\varphi_{n}$, définie sur $\R$ par :
\[
\forall x\in \R,\quad \varphi_{n}(x) = (1-x)^{n}e^{-2x}
\]
ainsi que l'intégrale : $I_{n} = \dint{0}{1}\varphi_{n}(x)dx$.\\
On se propose de démontrer l'existence de trois réels, $a$, $b$, $c$
tels que
\[
I_{n} = a + \dfrac{b}{n} + \dfrac{c}{n^{2}} + \dfrac{\varepsilon
(n)}{n^{2}}\quad 
\text{avec}\quad :\dlim{n\rightarrow + \infty }\varepsilon (n) = 0.
\]

\begin{noliste}{1.}
 \setlength{\itemsep}{4mm}
\item Calculer $I_{0}$, $I_{1}$.

\item Étudier la monotonie de la suite $(I_{n})_{n\in \N}$.

\item Déterminer le signe de $I_{n}$ pour tout entier naturel $n$.

\item Qu'en déduit-on pour la suite $(I_{n})_{n\in \N}$ ?

\item Majorer la fonction $g :x\mapsto e^{-2x}$ sur $[0;1]$.

\item En déduire que : $\forall n\in \N^{\times },\quad 0\leq
I_{n}\leq \dfrac{1}{n + 1}.$

\item Déterminer la limite de la suite $(I_{n})_{n\in \N}$ lorsque $
n $ tend vers l'infini.

\item À l'aide d'une intégration par parties, montrer que $\forall n\in

\N,\quad 2I_{n + 1} = 1-(n + 1)I_{n}.$

\item En déduire la limite de la suite $(nI_{n})_{n\in \N}$ lorsque $
n$ tend vers l'infini.

\item Déterminer la limite de la suite $(n(nI_{n}-1))_{n\in \N}$
lorsque $n$ tend vers l'infini.

\item Donner alors les valeurs de $a$, $b$, $c$.
\end{noliste}

\section*{EXERCICE 2.}

On considère la fonction $f$ définie par 
\[
\forall x\in \R_{+}{\times },\quad f(x) = x^{2}-x\ln (x)-1\quad \text{
et}\quad f(0) = -1.
\]
ainsi que les fonctions $\varphi $ et $g$ définies par :

\begin{noliste}{$\sbullet$}
\item $\forall x\in \R_{+}{\times },\quad \varphi (x) = \dfrac{2}{x}
 + \ln (x)$

\item $\forall (x;y)\in \R^{2},g(x\ ;y) = xe^{y}-ye^{x}.$
\end{noliste}

\noindent On donne le tableau de valeurs de $f$ :
\[
\begin{tabular}{|c|c|c|c|c|c|c|c|c|}
\hline
$x = $ & $0,5$ & $1$ & $1,5$ & $2$ & $2,5$ & $3$ & $3,5$ & $4$ \\
\hline
$f(x)\simeq $ & $-0,4$ & $0$ & $0,6$ & $1,6$ & $3$ & $4,7$ & $6,9$ &
$9,5$
\\
\hline
\end{tabular}
\]

\subsection*{2.1 Étude de deux suites associées à $f$.}

\begin{noliste}{1.}
 \setlength{\itemsep}{4mm}
\item Montrer que $f$ est continue sur $\R_{+}$.

\item Étudier la dérivabilité de la fonction $f$ en 0. En donner une
interprétation graphique.

\item Étudier la convexité de $f$ sur $\R_{+}{\times }$, puis
dresser son tableau de variations en précisant la limite de $f(x)$
lorsque $x $ tend vers l'infini.

\item Étudier la nature de la branche infinie.

\item Montrer que $f$ réalise une bijection de $\R_{+}{\times }$
sur un intervalle $J$ que l'on précisera.

\item Quel est le sens de variation de $f^{-1}$ ? Déterminer la limite
de $f^{-1}(x)$ lorsque $x$ tend vers l'infini.

\item Justifier que pour tout entier naturel $k$, il existe un unique
réel $x_{k}$ positif tel que $f(x_{k}) = k$.

\begin{noliste}{a)}
 \setlength{\itemsep}{2mm}
\item Donner la valeur de $x_{0}$.

\item Utiliser le tableau de valeurs de $f$ pour déterminer un
encadrement
de $x_{1}$ et $x_{2}$.

\item Exprimer $x_{k}$ à l'aide de $f^{-1}$ puis justifier que la suite
$
(x_{k})$ est croissante et déterminer sa limite lorsque $k$ tend vers
l'infini.
\end{noliste}

\item On définit la suite $(u_{n})$ par : $u_{0} = \dfrac{3}{2}$ et
$\forall
n\in \N,\quad u_{n + 1} = \varphi (u_{n})$

\begin{noliste}{a)}
 \setlength{\itemsep}{2mm}
\item Étudier les variations de $\varphi $ sur $\R_{+}{\times }$.

\item On donne $\varphi (\dfrac{3}{2})\simeq 1,73$ et $\varphi
(2)\simeq
1,69 $. Montrer que $\varphi \left( \left[ \ \dfrac{3}{2};2\right]
\right)
\subset \left[ \ \dfrac{3}{2};2\right] $.

\item En étudiant les variations de $\varphi ^{\prime }$, montrer que :
$
\forall x\in \left[ \ \dfrac{3}{2};2\right],\quad \left| \varphi
^{\prime
}(x)\right| \leq \dfrac{2}{9}$.

\item Montrer que les équations $x = \varphi (x)$ et $f(x) = 1$ sont
équivalentes. En déduire que le réel $x_{1}$ est l'unique solution de
l'équation $x = \varphi (x)$.

\item Montrer successivement que pour tout entier $n$ :
\[
\dfrac{3}{2}\leq u_{n}\leq 2\quad ;\quad \left|
u_{n + 1}-x_{1}\right| \leq \dfrac{2}{9}\left|
u_{n}-x_{1}\right| \quad ;\quad \left| u_{n}-x_{1}\right|
\leq \left( \dfrac{2}{9}\right) ^{n}.
\]

\item En déduire la limite de la suite $(u_{n})$.
\end{noliste}
\end{noliste}

\subsection*{2.2 Recherche d'extremum éventuel de $g$.}

\begin{noliste}{1.}
 \setlength{\itemsep}{4mm}
\item Calculer les dérivées partielles premières de la fonction $g$.

\item Montrer que si $g$ admet un extremum local en $(a;b)$ de
$\R^{2}$, alors : $ab = 1$ et $a = e^{a-1/a}$.\\
En déduire que nécessairement $a>0$, $ab = 1$ et $f(a) = 0$ et donc que
le seul
point où $g$ peut admettre un extremum est le couple $(1;1)$.

\item Calculer les réels : $r = \dfrac{\partial ^{2}g}{\partial
x^{2}}\left(
1;1\right) $ ; $s = \dfrac{\partial ^{2}g}{\partial x\partial y}\left(
1;1\right) $ ; $t = \dfrac{\partial ^{2}g}{\partial y^{2}}\left(
1;1\right) $.

\item La fonction $g$ admet-elle un extremum local sur $\R^{2}$ ?
\end{noliste}

\section*{EXERCICE 3}

On effectue une suite de lancers d'une pièce de monnaie. On suppose que
les résultats des lancers sont indépendants et qu'à chaque lancer, la
pièce donne pile avec la probabilité $p$ ($0<p<1$) et face avec la
probabilité $q = 1-p$.
\\
On s'intéresse dans cet exercice à l'apparition de deux piles
consécutifs.
\\
Pour tout entier naturel $n$ non nul, on note $A_{n}$ l'évènement :\\
\textquotedblleft\ deux piles consécutifs sont réalisés pour la
première
fois aux lancers numéro $n$ et $n + 1$ \textquotedblright.\\
On définit alors la suite $(a_{n})_{n\in \N}$ des probabilités des
évènements $A_{n}$ par :\\
Pour tout entier naturel $n$ non nul : $a_{n} =
P\left(\Ev{A_{n}}\right)$ avec la convention $a_{0} = 0$.

\subsection*{3.1. Encadrement des racines de l'équation
caractéristique.}

On considère la fonction polynomiale $f$ de la variable réelle $x$
définie par $f(x) = x^{2}-qx-pq.$

\begin{noliste}{1.}
 \setlength{\itemsep}{4mm}
\item Montrer que l'équation $f(x) = 0$ possède deux racines réelles
distinctes $r_{1}$ et $r_{2}$ avec $r_{1}<r_{2}$.\\
Exprimer $r_{1} + r_{2}$ et $r_{1}\times r_{2}$ en fonction de $p$ et
$q$.

\item Calculer $f(1)$, $f(-1)$, $f(0)$.

\item En déduire l'encadrement suivant : $\left| r_{1}\right|
<\left| r_{2}\right| <1.$
\end{noliste}

\subsection*{3.2. Équivalent de $a_{n}$ quand $n$ tend vers l'infini.}

\begin{noliste}{1.}
 \setlength{\itemsep}{4mm}
\item Déterminer $a_{1}$, $a_{2}$ et $a_{3}$ en fonction de $p$ et $q$.

\item En remarquant que l'évènement $A_{n + 2}$ est réalisé si et
seulement si :

\begin{noliste}{$\sbullet$}
\item on a obtenu pile au premier tirage, face au deuxième tirage, et à
partir de ce moment, $A_{n}$ est réalisé.
\end{noliste}

\noindent ou

\begin{noliste}{$\sbullet$}
\item on a obtenu face au premier tirage, et à partir de ce moment,
$A_{n +}{}_{1}$ est réalisé.
\end{noliste}

\noindent Montrer que l'on a, pour tout entier naturel $n$ : $a_{n +
2}-qa_{n + 1}-pqa_{n} = 0.$

\item Écrire un programme, en langage \Scilab{}, permettant de calculer
$a_{n}$
; l'entier $n$, le réel $p$ étant donné par l'utilisateur.

\item Montrer que pour tout entier naturel $n\ $; $a_{n} =
\dfrac{p^{2}}{r_{2}-r_{1}}((r_{2})^{n}-(r_{1})^{n}).$

\item Donner un équivalent de $a_{n}$ lorsque $n$ tend vers plus
l'infini.
\end{noliste}

\subsection*{3.3. Expression de $a_{n}$ en fonction de $n$ par une
méthode matricielle.}

On définit les matrices $A$ et $P$ par : 
\[
A = 
\begin{smatrix}
r_{1} + r_{2} & -r_{1}r_{2} \\
1 & 0
\end{smatrix}
\quad ;\quad P = {{
\begin{smatrix}
r_{1} & r_{2} \\
1 & 1
\end{smatrix}
}}\quad ;\quad I = {{
\begin{smatrix}
1 & 0 \\
0 & 1
\end{smatrix}
}};
\]
ainsi que les matrices unicolonnes $X_{n}$ tout entier naturel $n$, par
: $
X_{n} = {{
\begin{smatrix}
a_{n + 1} \\
a_{n}
\end{smatrix}
}}$.

\begin{noliste}{1.}
 \setlength{\itemsep}{4mm}
\item Vérifier que pour tout entier naturel $n$ : $X_{n + 1} = AX_{n}$.

\item Montrer que les matrices $A-r_{1}I$ et $A-r_{2}I$ ne sont pas
inversibles.

\item En déduire que $A$ est diagonalisable.

\item Montrer que $P$ est inversible et déterminer $P^{-1}$.

\item Calculer la matrice $D = P^{-1}AP$.\\
(Les coefficients de la matrice $D$ seront exprimés en fonction de
$r_{1}$et $r_{2}$ seulement).

\item Démontrer par récurrence, que pour tout entier naturel $n$ ;
$X_{n} = PD^{n}P^{-1}X_{0}.$

\item Retrouver ainsi l'expression de $a_{n}$ en fonction de $r_{1}$,
$r_{2}$, $p$ et $n$.
\end{noliste}

\subsection*{3.4. Étude du temps d'attente du premier double pile.}

On désigne par $T$ l'application associant à toute suite de lancers
successifs le numéro du lancer où pour la première fois on obtient un
double
pile. Ainsi, pour tout entier naturel $n$ ; $P\left(\Ev{T = n +
1}\right) = a_{n}$.

\begin{noliste}{1.}
 \setlength{\itemsep}{4mm}
\item Montrer que $T$ est une variable aléatoire, c'est-à-dire que :
$\Sum{n = 1}{+ \infty }P\left(\Ev{T = n + 1}\right) = 1.$

\item Prouver que $T$ admet une espérance $\E(T)$, et que : $\E(T) =
\dfrac{{1 + p}}{p^{2}}$.
\end{noliste}

\label{fin}

\end{document}

