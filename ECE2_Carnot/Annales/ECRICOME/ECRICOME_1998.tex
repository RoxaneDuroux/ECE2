\documentclass[11pt]{article}%
\usepackage{geometry}%
\geometry{a4paper,
 lmargin = 2cm,rmargin = 2cm,tmargin = 2.5cm,bmargin = 2.5cm}

\input{../../macros.tex}

\pagestyle{fancy} %
\lhead{ECE2 \hfill Mathématiques\\
} %
\chead{\hrule} %
\rhead{} %
\lfoot{} %
\cfoot{} %
\rfoot{\thepage} %

\renewcommand{\headrulewidth}{0pt}% : Trace un trait de séparation
 % de largeur 0,4 point. Mettre 0pt
 % pour supprimer le trait.

\renewcommand{\footrulewidth}{0.4pt}% : Trace un trait de séparation
 % de largeur 0,4 point. Mettre 0pt
 % pour supprimer le trait.

\setlength{\headheight}{14pt}

\title{\bf \vspace{-2cm} ECRICOME 1998} %
\author{} %
\date{} %
\begin{document}

\maketitle %
\vspace{-1.4cm}\hrule %
\thispagestyle{fancy}

\vspace*{.2cm}


% DEBUT DU DOC À MODIFIER : tout virer jusqu'au début de l'exo

%Définition et changement de valeurs de
compteurs%newcounter{cpt1}{section} compteur cpt1 remis à 0 à chaque
aumentation par stepcounter du compteur section%setcounter{cpt1}{3} on
met le compteur à 3%addtocounter{cpt1}{5} on ajoute 5 au compteur%
stepcounter{cpt1} on ajoute 1% ifthenelse{test}{alors}{sinon} (page
206) pour subordonner à une condition % whiledo{test}{commande} pour
faire une boucle (page 206 aussi) % value{cpt1} pour noter dans le
document la valeur de cpt1 
%Définition définitive d'opérateurs
mathématiques\newcommand{\ch}{\operatorname{ch}} 
\newcommand{\sh}{\operatorname{sh}}
\renewcommand{\tanh}{\operatorname{th}}
\renewcommand{\sinh}{\operatorname{sh}}
\renewcommand{\cosh}{\operatorname{ch}}
\newcommand{\argsh}{\operatorname{argsh}}
\newcommand{\argch}{\operatorname{argch}}
\newcommand{\argth}{\operatorname{argth}}
\newcommand{\Id}{\operatorname{Id}}
\renewcommand{\leq}{\leq}
\renewcommand{\geq}{\geq }

\newcommand{\dlim}{\lim}
\newcommand{\dsum}{\sum}
\newcommand{\dint}{\int}
\newcommand{\dprod}{\prod}



%Définition de nouvelles couleurs : rgb(trois paramètres red green blue
entre 0 et 1); cmyk (quatre cyan magenta yellow black) entre 0 et 1;
gray (entre 0 et 1) et black, white, red, green, blue, cyan, magenta,
yellow% definecolor{0gris}{gray}{0.8} 
% Nouvelle commande pour encadrer le titre car shabox ne veut que d'une
seule ligne; ATTENTION A LA TAILLE; petite différence avec shadowbox ou
doublebox, voire fcolorbox ou colorbox (au lieu de shabox; laisser le
parbox tranquille sauf pour la taille de la boîte
\newcommand{\Tbox}[1]{\begin{center} \shabox{\parbox{0.6
\linewidth}{#1}} \end{center}} %[1] pour 1 paramètre ; #1 pour ce que
fait le 1er paramètre; entre accolades ce que fait la commande
%Mise en page en mode fancy : en-têtes et pieds de pages puis
définition des en-têtes et pieds de pages\pagestyle{fancy}
\lhead{ECE 2 - Mathématiques \\
Quentin Dunstetter - ENC-Bessières 2011$\backslash$2012}
\chead{}
\rhead{Ecricome 1998}
\rfoot[ \ \thepage]{\thepage}
\cfoot{}
\lfoot{}
\thispagestyle{fancy} %Mise en page de la 1ère page en mode fancy
%Trait en bas et en haut de la page (entre en-tête et texte et texte et
pied de page)\renewcommand{\footrulewidth}{0.4pt}
\renewcommand{\headrulewidth}{0.4pt}


\begin{center}
{\Huge ECRICOME Eco 1998}
\end{center} 

\noindent \textbf{Exercice 1 }

\noindent

\begin{noliste}{1.}
 \setlength{\itemsep}{4mm}
\item[ \ \textbf{1)}] Soit $g$ l'application de $]0, + \infty [$ dans
$\R$
définie par 
\[
g(x) = \ln ({\frac{x}{1 + x}})-{\frac{\ln (1 + x)}{x}} 
\]

\begin{noliste}{$\sbullet$}
\item[ \ \textbf{a :}] Montrer que $g$ est dérivable sur $]0, + \infty
[$ et
expliciter sa dérivée.

\item[ \ \textbf{b :}] Dresser la tableau de variations de $g$, avec
ses éventuelles limites aux bornes.
\end{noliste}

\item[ \ \textbf{2)}] Soit $f$ l'application de $\R$ dans $\R$ définie
par 
\[
f(x) = e^{-x}\ln (1 + e^{x}) 
\]

\begin{noliste}{$\sbullet$}
\item[ \ \textbf{a :}] À l'aide d'un changement de variable, montrer
que pour
tout réel $x$ positif, on a : 
\[
\dint{0}{x}f(t)dt = g(e^{x}) + 2\ln 2 
\]

\item[ \ \textbf{b :}] Montrer qu'il existe un réel $c$, que l'on
explicitera, tel que l'application $h_{c}$ définie sur $\R$ par : 
\[
x\longrightarrow h_{c}(x) = \left\{ 
\begin{array}{cc}
cf(x) & \hbox{ si }x\geq 0 \\
0 & \hbox{ si }x<0
\end{array}
\right. 
\]
soit une densité d'une variable aléatoire $X$.

\item[ \ \textbf{c :}] On considère la variable aléatoire $Y =
e^{X}$.\\
Montrer que $Y$ a une densité que l'on explicitera.
\end{noliste}
\end{noliste}

\vspace{1cm} %\newpage
%------------------- Exercice n� 2 
\noindent \textbf{Exercice 2 }

\noindent Dans cet exercice, on étudie la diagonalisation des
matrices carrées d'ordre 3 antisymétriques ( c'est à dire vérifiant
$^{t}\!A = -A$ ).\\
On étudie d'abord un cas particulier avant de passer au cas général.

\textbf{Partie A}

\begin{noliste}{1.}
 \setlength{\itemsep}{4mm}
\item[ \ \textbf{1)}] On désigne par $E$ l'espace vectoriel $\R^{3}$
muni de sa base canonique $\mathcal{B} = (e_{1},e_{2},e_{3})$. On note
$0_{E}$
l'élément nul de $\R^{3}$.\\
On rappelle que toute famille libre de trois vecteurs de $E$ est une
base de 
$E$.\\
Soit $A = \left( 
\begin{array}{ccc}
0 & 2 & -1 \\
-2 & 0 & -2 \\
1 & 2 & 0
\end{array}
\right) $ et $f$ l'endomorphisme de $E$ représenté par $A$ dans la
base $\mathcal{B}$.\\
Soit $\mathcal{U} = (u_{1},u_{2},u_{3})$ où $u_{1} = -2e_{1} + e_{2} +
2e_{3}$,\ 
$u_{2} = e_{1} + 2e_{2}$ \ et \ $u_{3} = f(u_{2})$.

\begin{noliste}{$\sbullet$}
\item[ \ \textbf{a :}] Déterminer le noyau de $f$ et en donner une
base.

\item[ \ \textbf{b :}] Montrer que $\mathcal{U}$ est une base de $E$ et
déterminer la matrice $B$ représentant $f$ dans cette base.
\end{noliste}

\item[ \ \textbf{2)}] Soit $\lambda $ un réel non nul.\\
Montrer que pour tout vecteur $x$ de $E$, $[ \ \ f(x) = \lambda x\ ]$
équivaut à $[ \ \ x = 0_{E}\ ]$.\\
On pourra utiliser la décomposition de $x$ dans $\mathcal{U}$.

\item[ \ \textbf{3)}] 

\begin{noliste}{$\sbullet$}
\item[ \ \textbf{a :}] Quel est finalement l'ensemble des valeurs
propres de $A$
 ?

\item[ \ \textbf{b :}] La matrice $A$ est-elle diagonalisable ?
\end{noliste}
\end{noliste}

\textbf{Partie B}

\noindent Soient $a$, $b$ et $c$ trois réels donnés. On pose $a^{2} +
b^{2} + c^{2} = s$ et on suppose $s \neq0$.\\
On considère la matrice $M = \left( 
\begin{array}{lll}
0 & -a & -b \\
a & 0 & -c \\
b & c & 0
\end{array}
\right)$ et $g$ l'endomorphisme de $E$ représenté par $M$ dans la base
$\mathcal{B}$.

\begin{noliste}{1.}
 \setlength{\itemsep}{4mm}
\item[ \ \textbf{1)}] 

\begin{noliste}{$\sbullet$}
\item[ \ \textbf{a :}] Calculer $M^{2}$ et $M^{3}$.

\item[ \ \textbf{b :}] Vérifier que $M^{3}$ s'exprime simplement en
fonction
de $M$ et $s$.
\end{noliste}

\item[ \ \textbf{2)}] Montrer que si le réel $\lambda $ est valeur
propre
de $g$ alors $\lambda $ est nécessairement nul. On utilisera la
relation
trouvée ci dessus.

\item[ \ \textbf{3)}] Montrer que l'hypothèse ``$M$ est inversible''
conduit à une contradiction.

\item[ \ \textbf{4)}] 

\begin{noliste}{$\sbullet$}
\item[ \ \textbf{a :}] Quel est finalement l'ensemble des valeurs
propres de $M$
 ?

\item[ \ \textbf{b :}] La matrice $M$ est-elle diagonalisable ?
\end{noliste}
\end{noliste}

\vspace{2cm} 
%----------------------------- Problème
---------------------------------
\centerline{\bf Problème}

\noindent Dans tout le problème, $X$ désigne une variable aléatoire
définie sur un espace probabilisé $(\Omega,\mathcal{A},P)$ et à valeurs
dans $\N$ et $\E(X)$ l'espérance de $X$ si
elle existe.\\
On note $A$ l'événement \textquotedblleft $X$ prend une valeur
paire" (on écrira dorénavant pour abréger \textquotedblleft $X$
est pair"). On rappelle que $0$ est pair. On pose : $a =
P\left(\Ev{A}\right)$.\\
$\bullet $ On dit que $X$ a la propriété $\mathcal{P}$ si et
seulement si $a>1/2$.\\
On définit deux variables $X_{0}$ et $X_{1}$ :
\[
X_{0} = \left\{ 
\begin{array}{lc}
X & \text{si }X\text{ est pair} \\
0 & \text{sinon}
\end{array}
\right. \hspace{3cm}X_{1} = \left\{ 
\begin{array}{lc}
X & \text{si }X\text{ est impair} \\
0 & \text{sinon}
\end{array}
\right. 
\]
$\bullet $ On dit que $X$ a la propriété $\mathcal{Q}$ si et
seulement si $\E(X_{1})>\E(X_{0})$

\vspace{0.5cm} \textbf{Préliminaires}

\begin{noliste}{1.}
 \setlength{\itemsep}{4mm}
\item[ \ \textbf{1)}] Déterminer $X_{0} + X_{1}$.

\item[ \ \textbf{2)}] On note $Y$ la variable aléatoire qui vaut
$\left\{ 
\begin{array}{rc}
1 & \text{si }X\text{ est pair} \\
-1 & \text{si }X\text{ est impair}
\end{array}
\right. $\\
Montrer les relations : 
\[
X_{0} = {\frac{1}{2}}(1 + Y)X\qquad \hbox{ et }\qquad X_{1} =
{\frac{1}{2}}(1-Y)X
\]
\end{noliste}

\textbf{Partie A}

\noindent Dans cette partie, on suppose que $X$ suit une loi
géométrique de paramètre $p$ $(0<p<1)$ et on pose $q = 1-p$.

\begin{noliste}{1.}
 \setlength{\itemsep}{4mm}
\item[ \ \textbf{1)}] Montrer que $a =  {\frac{q}{q + 1}}$, puis que $X
$ ne vérifie pas la propriété $\mathcal{P}$.

\item[ \ \textbf{2)}] Montrer que $X_{1}$ admet une espérance donnée
$\E(X_{1}) = \frac{q^{2} + 1}{(1 + q)(1-q^{2})}}$

\item[ \ \textbf{3)}] Montrer que $X_{0}$ admet aussi une espérance que
l'on précisera, puis que $X$ vérifie la propriété $\mathcal{Q}$.
\end{noliste}

\textbf{Partie B}

\begin{noliste}{1.}
 \setlength{\itemsep}{4mm}
\item[ \ \textbf{1)}] Pour tout entier naturel $n$, on pose $p_{n} =
P\left(\Ev{X = n}\right)$ et
on suppose que la suite $(p_{n})_{n\in \N}$ est strictement
décroissante.\\
En écrivant $P\left(\Ev{A}\right)$ et $P\left(\Ev{\overline{A}}\right)$
à l'aide des nombres $p_{n}$, montrer que $X$ vérifie $\mathcal{P}$.

\item[ \ \textbf{2)}] On suppose maintenant que $X$ a une espérance.

\begin{noliste}{$\sbullet$}
\item[ \ \textbf{a :}] Montrer que $X_{0}$ et $X_{1}$ admettent aussi
des espérances.

\item[ \ \textbf{b :}] Montrer à l'aide des préliminaires, que $X$
vérifie $\mathcal{Q}$ si et seulement si $\E(YX)<0$.
\end{noliste}
\end{noliste}

\textbf{Partie C}

\noindent On suppose ici que la variable aléatoire $X$ est en
fait à valeurs dans l'intervalle entier $ \llbracket 0 ; 20 :\rrbracket
$ et donc que, pour tout entier $n \geq 21$, $p_{n} = 0$.\\
On définit un type :
\begin{verbatim}
 type Table = array[0..20] of real;
\end{verbatim}

et on demande d'écrire un programme en \Scilab{} :

- contenant une procédure \texttt{ENTR\E\_{L}OI} (à écrire) qui permet
à l'utilisateur d'entrer dans une variable T de type TABLE les nombres
$P\left(\Ev{X = k}\right)$ pour $k = 0,1,\ldots,20$.

- calculant $\E(Y X)$ et indiquant par un message si $X$ vérifie
$\mathcal{Q}$ ou non.

\vspace{0.5cm} \textbf{Partie D}

\noindent Le but de cette partie est d'étudier le cas o ? $X$ suit
une loi binomiale $\mathcal{B}(n,p)$ avec $n \geq 2$ et $0<p<1/2$.

\begin{noliste}{1.}
 \setlength{\itemsep}{4mm}
\item[ \ \textbf{1)}] Soit $f$ la fonction de deux variables définie
sur
l'ouvert $U = \ ]1, + \infty [ \ \times ]0,{\frac{1}{2}}$ de $\R^{2}$
par : 
\[
(x,y)\longrightarrow f(x,y) = xy(1-2y)^{x-1} = xy\exp [(x-1)\ln (1-2y)]

\]

\begin{noliste}{$\sbullet$}
\item[ \ \textbf{a :}] Montrer que $f$ admet en tout point $(x,y)$ de
$U$ des dérivées partielles premières $\frac{\partial f}{\partial
x}}(x,y)$ et $\frac{\partial f}{\partial y}}(x,y)$.
Les calculer et les mettre sous forme de produits.\\
Montrer que $f$ est de classe $C^{1}$ sur $U$.

\item[ \ \textbf{b :}] Montrer que, pour tout élément $u$ de $]0,1[
:\quad \ln (1-u)<-u$\\
En déduire que $f$ n'a pas d'extremum sur $U$.
\end{noliste}

\item[ \ \textbf{2)}] Dans cette question, $p$ est un réel vérifiant :
$0<p<1$.\\
On réalise une suite d'épreuves de Bernoulli indépendantes, de
probabilité de ``succès'' $p$ et de probabilité d'``échec'' $q =
1-p$.\\
Pour tout élément $n$ de $\N^{*}$, on définit l'événement : \\
$F_{n} = $``au cours des n premières épreuves, on obtient un nombre
pair de succès'' et on pose $u_{n} = P\left(\Ev{F_{n}}\right)$.

\begin{noliste}{$\sbullet$}
\item[ \ \textbf{a :}] Montrer que pour tout élément $n$ de $\N^{*} $,
$u_{n + 1} = (1-p)u_{n} + p(1-u_{n})$\\
On pose par convention $u_{0} = 1$. Vérifier que la relation précédent
est encore vraie pour $n = 0$.

\item[ \ \textbf{b :}] Donner l'expression générale de $u_{n}$ en
fonction de $n$ pour tout entier naturel $n$.
\end{noliste}

\item[ \ \textbf{3)}] On se place maintenant dans le cas annoncé au
début de la partie D : $X$ suit une loi binomiale $\mathcal{B}(n,p)$
avec $n\ge
2$ et $0<p<1/2$.

\begin{noliste}{$\sbullet$}
\item[ \ \textbf{a :}] En reprenant la définition du réel $a$ associé
à la variable $X$, montrer que $a = u_{n}$.\\
Montrer que $X$ vérifie la propriété $\mathcal{P}$.

\item[ \ \textbf{b :}] Calculer $\E(YX)$\\
En déduire que $\E(X_{1})-\E(X_{0}) = f(n,p)$ (o ? $f$ est la fonction
introduite en D-1) puis que $X$ a la propriété $\mathcal{Q}$.

\item[ \ \textbf{c :}] On considère l'application partielle : 
\[
g :\quad y\longrightarrow g(y) = f(n,y)\quad \hbox{ définie sur
}]0,{\frac{1}{2}}[ \ \quad (n\geq 2) 
\]
Montrer que $g$ admet un maximum $M_{n} =  {\frac{1}{2}}\left(
{\frac{n-1}{n}}\right) ^{n-1}$

\item[ \ \textbf{d :}] Montrer que la fonction $\phi :\
x\longrightarrow  (x-1)\ln (1-{\frac{1}{x}})$ est de classe $C^{2}$ sur
$[2, + \infty [$, que sa dérivée seconde est strictement positive sur
$[2, + \infty [$ et que $ \dlim{x\rightarrow + \infty }\phi
^{\prime }(x) = 0$.\\
En déduire que la suite $(M_{k})_{k\geq 2}$ est strictement
décroissante.

\item[ \ \textbf{e :}] Montrer que la suite $(M_{k})_{k\geq 2}$
converge, préciser sa limite et montrer que pour tout entier $k$
supérieur ou égal à 2, $ {\frac{1}{2e}}\leq M_{k}\leq {\frac{1}{4}}$
\end{noliste}

\item[ \ \textbf{4)}] On suppose que deux joueurs Alain et Béatrice
jouent 
à pile ou face avec une pièce désiquilibrée, la probabilité d'obtenir
``face'' étant égale à $p$ $(0<p< 
{\frac{1}{2}})$.\\
Une pièce est lancée $n$ fois $(n\geq 2)$, les lancers successifs
sont supposés indépendants.\\
Alain empoche un gain égal au nombre de faces apparues si ce nombre de
faces est pair et Béatrice, un gain égal au nombre de faces apparues
si ce nombre est impair.

\begin{noliste}{$\sbullet$}
\item[ \ \textbf{a :}] Quel est celui des joueurs qui a le plus de
chances de
gagner ?

\item[ \ \textbf{b :}] Quel est celui des joueurs qui a la plus forte
espérance de gain ?

\item[ \ \textbf{c :}] Comment interpréter l'encadrement obtenu à la
question 3-e) de cette partie ?
\end{noliste}
\end{noliste}

\end{document}

