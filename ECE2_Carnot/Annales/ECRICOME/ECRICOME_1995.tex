\documentclass[11pt]{article}%
\usepackage{geometry}%
\geometry{a4paper,
 lmargin = 2cm,rmargin = 2cm,tmargin = 2.5cm,bmargin = 2.5cm}

\input{../../macros.tex}

\pagestyle{fancy} %
\lhead{ECE2 \hfill Mathématiques\\
} %
\chead{\hrule} %
\rhead{} %
\lfoot{} %
\cfoot{} %
\rfoot{\thepage} %

\renewcommand{\headrulewidth}{0pt}% : Trace un trait de séparation
 % de largeur 0,4 point. Mettre 0pt
 % pour supprimer le trait.

\renewcommand{\footrulewidth}{0.4pt}% : Trace un trait de séparation
 % de largeur 0,4 point. Mettre 0pt
 % pour supprimer le trait.

\setlength{\headheight}{14pt}

\title{\bf \vspace{-2cm} ECRICOME 1995} %
\author{} %
\date{} %
\begin{document}

\maketitle %
\vspace{-1.4cm}\hrule %
\thispagestyle{fancy}

\vspace*{.2cm}


% DEBUT DU DOC À MODIFIER : tout virer jusqu'au début de l'exo

%Définition et changement de valeurs de
compteurs%newcounter{cpt1}{section} compteur cpt1 remis à 0 à chaque
aumentation par stepcounter du compteur section%setcounter{cpt1}{3} on
met le compteur à 3%addtocounter{cpt1}{5} on ajoute 5 au compteur%
stepcounter{cpt1} on ajoute 1% ifthenelse{test}{alors}{sinon} (page
206) pour subordonner à une condition % whiledo{test}{commande} pour
faire une boucle (page 206 aussi) % value{cpt1} pour noter dans le
document la valeur de cpt1 
%Définition définitive d'opérateurs
mathématiques\newcommand{\ch}{\operatorname{ch}} 
\newcommand{\sh}{\operatorname{sh}}
\renewcommand{\tanh}{\operatorname{th}}
\renewcommand{\sinh}{\operatorname{sh}}
\renewcommand{\cosh}{\operatorname{ch}}
\newcommand{\argsh}{\operatorname{argsh}}
\newcommand{\argch}{\operatorname{argch}}
\newcommand{\argth}{\operatorname{argth}}
\newcommand{\Id}{\operatorname{Id}}
\renewcommand{\leq}{\leq}
\renewcommand{\geq}{\geq }

\newcommand{\dlim}{\lim}
\newcommand{\dsum}{\sum}
\newcommand{\dint}{\int}
\newcommand{\dprod}{\prod}



%Définition de nouvelles couleurs : rgb(trois paramètres red green blue
entre 0 et 1); cmyk (quatre cyan magenta yellow black) entre 0 et 1;
gray (entre 0 et 1) et black, white, red, green, blue, cyan, magenta,
yellow% definecolor{0gris}{gray}{0.8} 
% Nouvelle commande pour encadrer le titre car shabox ne veut que d'une
seule ligne; ATTENTION A LA TAILLE; petite différence avec shadowbox ou
doublebox, voire fcolorbox ou colorbox (au lieu de shabox; laisser le
parbox tranquille sauf pour la taille de la boîte
\newcommand{\Tbox}[1]{\begin{center} \shabox{\parbox{0.6
\linewidth}{#1}} \end{center}} %[1] pour 1 paramètre ; #1 pour ce que
fait le 1er paramètre; entre accolades ce que fait la commande
%Mise en page en mode fancy : en-têtes et pieds de pages puis
définition des en-têtes et pieds de pages\pagestyle{fancy}
\lhead{ECE 2 - Mathématiques \\
Quentin Dunstetter - ENC-Bessières 2011$\backslash$2012}
\chead{}
\rhead{Ecricome 1995}
\rfoot[ \ \thepage]{\thepage}
\cfoot{}
\lfoot{}
\thispagestyle{fancy} %Mise en page de la 1ère page en mode fancy
%Trait en bas et en haut de la page (entre en-tête et texte et texte et
pied de page)\renewcommand{\footrulewidth}{0.4pt}
\renewcommand{\headrulewidth}{0.4pt}


\begin{center}
{\Huge ECRICOME Eco 1995}
\end{center}

\section*{Exercice I}

\begin{noliste}{1.}
 \setlength{\itemsep}{4mm}
\item Pour $n\in \N,$ on pose 
\[
I_{n} = \dint{0}{2\pi }x^{n}\sin xdx\qquad \text{et}\qquad
J_{n} = \dint{0}{2\pi }x^{n}\cos x
\]

\begin{noliste}{a)}
 \setlength{\itemsep}{2mm}
\item Justifier l'existence de $I_{n}$ et $J_{n}.$

\item Pour $n\in \N,$ établir les relations 
\[
I_{n + 1} = (n + 1)J_{n}-(2\pi )^{n + 1}\qquad \text{et}\qquad J_{n +
1} = -(n + 1)J_{n}
\]

\item Pour $n\in \{0,1,2,3\}$, calculer $I_{n}$ et $J_{n}.$
\end{noliste}

\item On considère $f :\mathbb{R\rightarrow R}$ définie par : 
\[
f(x) = \left\{ 
\begin{array}{ccc}
\dfrac{x}{2\pi ^{2}}(1-\cos x) & \text{si} & x\in [0,2\pi ] \\
0 & \text{si} & x\notin [0,2\pi ]
\end{array}
\right.
\]
Montrer que $f$ est la densité d'une variable aléatoire réelle $X.$

\item 

\begin{noliste}{a)}
 \setlength{\itemsep}{2mm}
\item Déterminer la fonction de répartition $F$ de $X.$

\item Montrer que $F$ est dérivable sur $\R.$ Préciser les valeurs
de $F^{\prime }(0)$ et $F^{\prime }(2\pi ).$

\item Donner le tableau de variation de $F$ sur $\R.$
\end{noliste}

\item Calculer l'espérance mathématique $\E(X)$ et l'écart-type $\sigma
(X)$
de la variable aléatoire $X.$\\
Donner une valeur approchée à $10^{-2}$ près de $\E(X)$ et de $\sigma
(X).$

\item Calculer les probabilités suivantes

\begin{noliste}{a)}
 \setlength{\itemsep}{2mm}
\item $P\left(\Ev{X>\dfrac{\pi }{2}}\right)$

\item $P\left(\Ev{X<\dfrac{\pi }{2}$ ou $X>\dfrac{3\pi }{2}}\right)$

\item $P\left(\Ev{\left| X-\pi \right| \leq \dfrac{\pi }{2}}\right)$

\item $P[(X\geq \dfrac{\pi }{2})/(X\leq \dfrac{3\pi }{2}))$\\
(Probabilité conditionnelle de l'évènement $(X\geq \dfrac{\pi }{2})$
sachant $(X\leq \dfrac{3\pi }{2})$
\end{noliste}
\end{noliste}

\section*{EXERCICE II}

Pour $x\in \R,$ on pose $ :\quad f(x) = x^{3} + 5x-1$

\begin{noliste}{1.}
 \setlength{\itemsep}{4mm}
\item 

\begin{noliste}{a)}
 \setlength{\itemsep}{2mm}
\item Étudier les variations de $f$ sur $\R.$

\item \label{alpha}Montrer que l'équation $x^{3} + 5x-1 = 0$ admet une
unique
solution $\alpha $ dans $\R.$

\item Établir que $0<\alpha <\dfrac{1}{2}.$
\end{noliste}

\item Le plan étant rapporté à un repère orthonormal
$(O,\overrightarrow{i},\overrightarrow{j}),$ on note $(C)$ la courbe
représentative de $f$ dans ce
repère.\\
$M_{0}$ est le point de $(C)$ d'abscisse $1.$ La tangente à $(C)$ au
point $M_{0}$ coupe l'axe $(O,\overrightarrow{i})$ en un point
d'abscisse $x_{1}.$
Soit $M_{1}$ le point de $(C)$ d'abscisse $x_{1}.$ En traçant la
tangente à $(C)$ au point $M_{1},$ on détermine de façon analogue le
point $M_{2}.$ On
construit ainsi par récurrence une suite $(M_{n})$ de points de $(C).$
On désigne enfin par $x_{n}$ l'abscisse du point $M_{n}.$\\
Établir que $ :\forall n\in \N,\quad x_{n + 1} = \dfrac{2x_{n}{3} +
1}{3x_{n}{2} + 5}.$

\item 

\begin{noliste}{a)}
 \setlength{\itemsep}{2mm}
\item Pour $x\in \R,$ on pose 
\[
g(x) = 2x^{3}-3\alpha x^{2} + 1-5\alpha
\]
où $\alpha $ est le nombre défini à la question \ref{alpha}.\\
Étudier les variation de $g$ sur $\R.$\\
Exprimer, pour $n\in \N,$ $x_{n + 1}-\alpha $ à l'aide de $g$ et
$x_{n}.$\\
Établir que 
\[
\forall n\in \N,\quad x_{n}>\alpha.
\]

\item Montrer que la suite $(x_{n})$ est strictement décroissante.\\
En déduire qu'elle est convergente. Quelle est la limite ?
\end{noliste}

\item 

\begin{noliste}{a)}
 \setlength{\itemsep}{2mm}
\item Pour $n\in \N,$ on pose $u_{n} = (x_{n + 1}-\alpha
)-(x_{n}-x_{n + 1}).$\\
Exprimer, pour $n\in \N,$ $u_{n}$ à l'aide de $\alpha $ et de $x_{n}. $

\item Pour $x\in \R,$ on pose $h(x) = x^{3}-3\alpha x^{2}-5x +
2-5\alpha.$\\
Étudier les variations de $h$ sur $[0,1].$

\item Établir que $ :\quad \forall n\in \N,\quad x_{n + 1}-\alpha
<x_{n}-x_{n + 1}.$
\end{noliste}

\item 

\begin{noliste}{a)}
 \setlength{\itemsep}{2mm}
\item Écrire en \Scilab{} un programme qui calcule $x_{N}$ et $x_{N +
1},$ où $N$
est le plus petit entier $n$ pour lequel la condition $\left|
x_{n + 1}-x_{n}\right| \leq 10^{-5}$ est réalisée.

\item Expliquer pourquoi un tel programme permet d'obtenir une valeur
approchée de $\alpha $ à $10^{-5}$ près.
\end{noliste}
\end{noliste}

\section*{PROBLEME}

\subsection*{Notations}

On note $\mathfrak{M}_{3}(\R)$ l'espace vectoriel des matrices carrées
d'ordre $3$ à coefficients dans $\R.$\\
On désigne par $\mathcal{E} =
(\varepsilon_{1},\varepsilon_{2},\varepsilon
_{3})$ la base canonique de $\R^{3}.$\\
On rappelle que, par définition $ :\varepsilon_{1} = (1,0,0),\quad
\varepsilon
_{2} = (0,1,0),\quad \varepsilon_{3} = (0,0,1).$\\
On pose : 
\[
A = 
\begin{smatrix}
16 & 4 & -4 \\
-18 & -4 & 5 \\
30 & 8 & -7
\end{smatrix}
\]
Enfin, on désigne par $u$ l'endomorphisme de $\R^{3}$ ayant $A$ pour
matrice dans la base $\mathcal{E}.$

\subsection*{PREMIERE PARTIE : étude de la matrice $A$}

\begin{noliste}{1.}
 \setlength{\itemsep}{4mm}
\item 

\begin{noliste}{a)}
 \setlength{\itemsep}{2mm}
\item Déterminer les valeurs propres de $A.$

\item $A$ est-elle inversible ?

\item $A$ est-elle diagonalisable ?
\end{noliste}

\item On pose 
\[
P = 
\begin{smatrix}
1 & 0 & -1 \\
-2 & 1 & 1 \\
2 & 1 & -2
\end{smatrix}
\qquad \text{et}\qquad D = 
\begin{smatrix}
0 & 0 & 0 \\
0 & 1 & 0 \\
0 & 0 & 4
\end{smatrix}
\]

\begin{noliste}{a)}
 \setlength{\itemsep}{2mm}
\item Montrer qu'il existe une base $E = (e_{1},e_{2},e_{3})$ de
$\R^{3}$ telle que $P$ soit la matrice de passage de la base
$\mathcal{E}$ dans
la base $E$ et telle que $D$ soit la matrice de $u$ dans la base $E.$

\item En utilisant la méthode du pivot de Gauss, montrer que $P$ est
inversible et calculer $P^{-1}.$

\item Justifier rapidement et sans calcul l'égalité $ :P^{-1}AP = D;$

\item Montrer qu'une matrice $\Delta $ de $\mathfrak{M}_{3}(\R)$
vérifie $\Delta D = D\Delta $ si et seulement $\Delta $ est diagonale.
\end{noliste}
\end{noliste}

\subsection*{DEUXIEME PARTIE : résolution dans $\mathfrak{M}_{3}(\R)$
de l'équation du second degré $ :X^{2} = A$}

On se propose dans cette partie de déterminer toutes les matrices $X$
de $\mathfrak{M}_{3}(\R)$ vérifiant 
\[
X^{2} = A
\]

\begin{noliste}{1.}
 \setlength{\itemsep}{4mm}
\item On considère $X\in \mathfrak{M}_{3}(\R)$ telle que $ :X^{2} = A;$
on pose $Y = P^{-1}XP.$\\
Vérifier que $Y^{2} = D;$ montrer que $YD = DY,$ puis établir que $Y$
est de la
forme : 
\[
Y = 
\begin{smatrix}
0 & 0 & 0 \\
0 & \gamma & 0 \\
0 & 0 & 2\gamma ^{\prime }\end{smatrix}
\qquad \text{avec }\gamma \in \{-1,1\}\text{ et }\gamma ^{\prime }\in
\{-1,1\}.
\]
En déduire la forme de la matrice $X$ puis montrer, sans calculer
explicitement les coefficients de $X^{2},$ qu'une telle matrice $X$
vérifie
bien $ :X^{2} = A.$

\item Quel est le nombre $m$ de solutions dans $\mathfrak{M}_{3}(\R) 
$ de l'équation du second degré $X^{2} = A$ ?\\
Sans calculer explicitement ces $m$ solutions $X_{1},X_{2},...,X_{m},$
déterminer leur somme $S = X_{1} + X_{2} + \cdots + X_{m}$ et exprimer
leur produit $T = X_{1}X_{2}\cdots X_{m}$ en fonction de $A.$
\end{noliste}

\subsection*{TROISIEME PARTIE : calcul de $A^{n}$ et application à une
étude
de suites}

\begin{noliste}{1.}
 \setlength{\itemsep}{4mm}
\item Soit $n\in \N^{\times };$ calculer $D^{n}$ puis en déduire
l'expression de $A^{n}$ en fonction de $n.$

\item Soient $a,b$ et $c$ trois réels.\\
On considère les suites $(p_{n}),(q_{n})$ et $(r_{n})$ définies par 
\[
p_{0} = a,\quad q_{0} = b,\quad r_{0} = c\quad \text{et, pour tout
}n\in \N,\quad \left\{ 
\begin{array}{c}
p_{n + 1} = 16p_{n} + 4q_{n}-4r_{n} \\
q_{n + 1} = -18p_{n}-4q_{n} + 5r_{n} \\
r_{n + 1} = 30p_{n} + 8q_{n}-7r_{n}
\end{array}
\right.
\]

\begin{noliste}{a)}
 \setlength{\itemsep}{2mm}
\item Pour $n\in \N,$ on pose $U_{n} = 
\begin{smatrix}
p_{n} \\
q_{n} \\
r_{n}\end{smatrix}.$\\
Exprimer $U_{n}$ à l'aide de $A$ et de $U_{0};$ en déduire, que pour
$n\geq 1,$ les expressions de $p_{n},q_{n}$ et $r_{n}$ en fonction de
$a,b,c$ et de $n.$

\item Déterminer une condition nécessaire et suffisante portant sur
$a,b$ et 
$c$ pour que les suites $(p_{n}),(q_{n})$ et $(r_{n})$ tendent vers une
limite finie lorsque $n$ tend vers plus l'infini.\\
Cette condition étant supposée remplie, que peut-on dire des suites
$(p_{n}), $ $(q_{n})$ et $(r_{n})$ ?
\end{noliste}
\end{noliste}

\subsection*{QUATRIEME PARTIE : $C(A) = \{M\in \mathfrak{M}_{3}(\R)$
telle que $AM = MA\}$}

\begin{noliste}{1.}
 \setlength{\itemsep}{4mm}
\item Montrer que $M\in C(A)$ si et seulement $P^{-1}MP$ est diagonale.

\item En déduire que $C(A)$ est égal à l'ensemble des matrices de
$\mathfrak{M}_{3}(\R)$ de la forme : 
\begin{equation}
aM_{1} + bM_{2} + cM_{3}\text{ avec }(a,b,c)\in \R^{3} \label{CL}
\end{equation}
où $M_{1},M_{2}$ et $M_{3}$ sont trois matrices que l'on déterminera.

\item Montrer que $(M_{1},M_{2},M_{3})$ est une famille libre
d'éléments de $C(A).$ En déduire l'unicité de l'écriture d'une matrice
$M$ de $C(A)$ sous
la forme (\ref{CL}).
\end{noliste}

\label{fin}

\end{document}

