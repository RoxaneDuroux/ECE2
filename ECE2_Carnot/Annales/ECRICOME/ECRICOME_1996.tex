\documentclass[11pt]{article}%
\usepackage{geometry}%
\geometry{a4paper,
 lmargin = 2cm,rmargin = 2cm,tmargin = 2.5cm,bmargin = 2.5cm}

\input{../../macros.tex}

\pagestyle{fancy} %
\lhead{ECE2 \hfill Mathématiques\\
} %
\chead{\hrule} %
\rhead{} %
\lfoot{} %
\cfoot{} %
\rfoot{\thepage} %

\renewcommand{\headrulewidth}{0pt}% : Trace un trait de séparation
 % de largeur 0,4 point. Mettre 0pt
 % pour supprimer le trait.

\renewcommand{\footrulewidth}{0.4pt}% : Trace un trait de séparation
 % de largeur 0,4 point. Mettre 0pt
 % pour supprimer le trait.

\setlength{\headheight}{14pt}

\title{\bf \vspace{-2cm} ECRICOME 1996} %
\author{} %
\date{} %
\begin{document}

\maketitle %
\vspace{-1.4cm}\hrule %
\thispagestyle{fancy}

\vspace*{.2cm}


% DEBUT DU DOC À MODIFIER : tout virer jusqu'au début de l'exo

%Définition et changement de valeurs de
compteurs%newcounter{cpt1}{section} compteur cpt1 remis à 0 à chaque
aumentation par stepcounter du compteur section%setcounter{cpt1}{3} on
met le compteur à 3%addtocounter{cpt1}{5} on ajoute 5 au compteur%
stepcounter{cpt1} on ajoute 1% ifthenelse{test}{alors}{sinon} (page
206) pour subordonner à une condition % whiledo{test}{commande} pour
faire une boucle (page 206 aussi) % value{cpt1} pour noter dans le
document la valeur de cpt1 
%Définition définitive d'opérateurs
mathématiques\newcommand{\ch}{\operatorname{ch}} 
\newcommand{\sh}{\operatorname{sh}}
\renewcommand{\tanh}{\operatorname{th}}
\renewcommand{\sinh}{\operatorname{sh}}
\renewcommand{\cosh}{\operatorname{ch}}
\newcommand{\argsh}{\operatorname{argsh}}
\newcommand{\argch}{\operatorname{argch}}
\newcommand{\argth}{\operatorname{argth}}
\newcommand{\Id}{\operatorname{Id}}
\renewcommand{\leq}{\leq}
\renewcommand{\geq}{\geq }

\newcommand{\dlim}{\lim}
\newcommand{\dsum}{\sum}
\newcommand{\dint}{\int}
\newcommand{\dprod}{\prod}



%Définition de nouvelles couleurs : rgb(trois paramètres red green blue
entre 0 et 1); cmyk (quatre cyan magenta yellow black) entre 0 et 1;
gray (entre 0 et 1) et black, white, red, green, blue, cyan, magenta,
yellow% definecolor{0gris}{gray}{0.8} 
% Nouvelle commande pour encadrer le titre car shabox ne veut que d'une
seule ligne; ATTENTION A LA TAILLE; petite différence avec shadowbox ou
doublebox, voire fcolorbox ou colorbox (au lieu de shabox; laisser le
parbox tranquille sauf pour la taille de la boîte
\newcommand{\Tbox}[1]{\begin{center} \shabox{\parbox{0.6
\linewidth}{#1}} \end{center}} %[1] pour 1 paramètre ; #1 pour ce que
fait le 1er paramètre; entre accolades ce que fait la commande
%Mise en page en mode fancy : en-têtes et pieds de pages puis
définition des en-têtes et pieds de pages\pagestyle{fancy}
\lhead{ECE 2 - Mathématiques \\
Quentin Dunstetter - ENC-Bessières 2011$\backslash$2012}
\chead{}
\rhead{Ecricome 1996}
\rfoot[ \ \thepage]{\thepage}
\cfoot{}
\lfoot{}
\thispagestyle{fancy} %Mise en page de la 1ère page en mode fancy
%Trait en bas et en haut de la page (entre en-tête et texte et texte et
pied de page)\renewcommand{\footrulewidth}{0.4pt}
\renewcommand{\headrulewidth}{0.4pt}


\begin{center}
{\Huge ECRICOME Eco 1996}
\end{center}

\section*{EXERCICE 1}

On désigne par $n$ un entier naturel non nul, et l'on se propose
d'étudier
les racines de l'équation 
\[
(E_{n}) :\ln x + x = n
\]
A cet effet, on introduit la fonction $f$ de la variable réelle $x$
définie
sur $\R_{+}{\times }$ par :
\[
f(x) = \ln x + x
\]

\subsection*{1.1. Existence des racines de $(E_{n})$}

\begin{noliste}{1.}
 \setlength{\itemsep}{4mm}
\item Étudier les variations de la fonction $f.$ Montrer que $f$ est
une
bijection de $\R_{+}{\times }$ sur $\R.$ En déduire que,
pour tout entier naturel non nul $n,$ $(E_{n})$ admet une racine et une
seule $x_{n}$ et que la suite $(x_{n})_{n\in \N}$ est strictement
croissante.

\item Donner la valeur de $x_{1}.$ Trouver à l'aide de la calculatrice,
une
valeur approchée de $x_{2}$ à $10^{-2}$ près, et déterminer l'entier
naturel 
$p$ tel que 
\[
\dfrac{p}{100}\leq x_{2}<\dfrac{p + 1}{100}.
\]

\item Représenter la fonction $f$ relativement à un repère orthonormal
du
plan (unité graphique : $2$ cm).
\end{noliste}

\subsection*{1.2. Étude de la convergence de $(x_{n})_{n\in \N^{\times
}}$}

\begin{noliste}{1.}
 \setlength{\itemsep}{4mm}
\item Montrer que :
\[
\forall x\in \R_{+}{\times },\qquad \ln x<x.
\]

\item Prouver que l'on a :
\[
\forall n\in \N^{\times },\qquad \dfrac{n}{2}\leq
x_{n}\leq n
\]

\item Quelle est la limite de $x_{n}$ quand $n$ tend vers $ + \infty $
?
\end{noliste}

\subsection*{1.3. Comportement asymptotique de $(x_{n})_{n\in
\N^{\times }}$}

\begin{noliste}{1.}
 \setlength{\itemsep}{4mm}
\item Montrer que $\dfrac{\ln (x_{n})}{n}$ tend vers $0$ quand $n$ tend
vers 
$ + \infty.$ En déduire que :
\[
x_{n}\underset{n\rightarrow + \infty }{\sim }n.
\]

\item Calculer la limite de $x_{n + 1}-x_{n}$ quand $n$ tend vers $ +
\infty.$

\item On pose :
\[
\forall n\in \N^{\times },\qquad u_{n} = \dfrac{n-x_{n}}{\ln n}.
\]

\begin{noliste}{a)}
 \setlength{\itemsep}{2mm}
\item Montrer que :
\[
\forall n\in \N^{\times },\qquad u_{n}-1 = \dfrac{\ln \left(
\dfrac{x_{n}}{n}\right) }{\ln n}.
\]

\item Quelle est la limite de $u_{n}$ quand $n$ tend vers $ + \infty $
?

\item Prouver alors que :
\[
1-u_{n}\underset{n\rightarrow + \infty }{\sim }\dfrac{1}{n}
\]
\end{noliste}

\item En déduire qu'il existe une fonction $\varepsilon $ ayant une
limite
nulle en $0$ telle que, pour tout entier supérieur ou égal à $2,$ on
ait :
\[
x_{n} = n-\ln n + \dfrac{\ln n}{n} + \dfrac{\ln n}{n}\varepsilon
(\dfrac{1}{n})
\]
\end{noliste}

\section*{EXERCICE 2}

Dans cet exercice, on se propose d'étudier la nature et la somme de la
série
de terme général :
\[
\forall n\in \N^{\times },\quad u_{n} = \dint{0}{\pi /2}x\sin
(nx)\cos ^{n}xdx.
\]
A cet effet, on pose :
\[
\forall n\in \N^{\times },\quad S_{n} = \Sum{k = 1}{n}\dfrac{1}{k}
\]
et \textit{on admettra} dans la suite que, pour tout réel $x :$
\[
\Sum{p = 1}{n}C_{n}{p}\sin (2px) = 2^{n}\sin (nx)\cos ^{n}x
\]

\subsection*{2.1. Calcul de $u_{n}$ en fonction de $S_{n}$}

\begin{noliste}{1.}
 \setlength{\itemsep}{4mm}
\item On définit la fonction $f_{n}$ par :
\[
\left\{ 
\begin{array}{cc}
\forall t\in \ ]0,1] & f_{n}(t) = \dfrac{1-(1-t)^{n}}{t} \\
 & f_{n}(0) = n
\end{array}
\right.
\]

\begin{noliste}{a)}
 \setlength{\itemsep}{2mm}
\item Montrer que $f_{n}$ est une fonction polynômiale sur $[0,1].$

\item On pose :
\[
I_{n} = \dint{0}{1}f_{n}(t)dt.
\]
Prouver que :
\[
I_{n} = \Sum{p = 1}{n}C_{n}{p}\dfrac{(-1)^{p-1}}{p}
\]
puis, après avoir calculé $\dint{0}{1}(1-t)^{k-1}dt$ montrer que 
\[
I_{n} = S_{n}
\]
\end{noliste}

\item Montrer que, pour tout entier naturel $p$ non nul :
\[
\dint{0}{\pi /2}x\sin (2px)dx = \dfrac{\pi (-1)^{p + 1}}{4p}
\]
et, à l'aide de la formule admise, en déduire que :
\[
u_{n} = \dfrac{\pi }{2^{n + 2}}S_{n}
\]
\end{noliste}

\subsection*{2.2. Convergence et somme de la série}

Soit $f$ la fonction définie sur $[0,1[$ par :
\[
f(x) = -\ln (1-x).
\]

\begin{noliste}{1.}
 \setlength{\itemsep}{4mm}
\item Étudier les variations de la fonction $\varphi $ définie par :
\[
\forall t\in \lbrack 0,x],\quad \varphi (t) = \dfrac{x-t}{1-t}
\]
et montrer que :
\[
\forall t\in \lbrack 0,x],\quad 0\leq \varphi (t)\leq x.
\]

\item Vérifier que :
\[
\forall t\in \lbrack 0,x],\quad \dfrac{\varphi (t)}{1-t} =
\dfrac{x-1}{(1-t)^{2}} + \dfrac{1}{1-t}.
\]

\item Montrer, par récurrence que :
\[
\forall n\in \N^{\times },\quad \forall x\in \lbrack 0,1[,\quad
f(x) = \Sum{k = 1}{n}\dfrac{x^{k}}{k} + R_{n}(x)
\]
où l'on a posé :
\[
R_{n}(x) = \dint{0}{x}\dfrac{(\varphi (t))^{n}}{1-t}dt.
\]

\item En déduire de $2.2.1$ que :
\[
\forall x\in \lbrack 0,1[,\quad 0\leq R_{n}(x)\leq -x^{n}\ln (1-x)
\]

\item Exprimer la quantité $(1-x)\Sum{k = 1}{n}S_{k}x^{k}$ en
fonction de $f(x),$ $R_{n}(x),$ $S_{n}$ et $x.$\\
Prouver que la série de terme général $S_{k}x^{k}$ est convergente.
Montrer
que sa somme est $\dfrac{f(x)}{1-x}.$

\item En déduire la valeur de $\Sum{p = 1}{+ \infty }u_{p}.$
\end{noliste}

\section*{PROBLEME}

\subsection*{3.1. Première partie}

On désigne par $E$ l'espace vectoriel $\R^{3},$ par $B =
(e_{1},e_{2},e_{3})$ une base de $E,$ et par $f$ l'endomorphisme de $E$
qui, à tout vecteur $u$ de coordonnées $(x,y,z)$ dans la base $B,$
associe
le vecteur $u^{\prime }$ de coordonnées $(x^{\prime },y^{\prime
},z^{\prime
})$ dans la base $B$ tel que :
\[
\left\{ 
\begin{array}{c}
4x^{\prime } = y \\
4y^{\prime } = 4x + 2y + 4z \\
4z^{\prime } = y
\end{array}
\right.
\]

\begin{noliste}{1.}
 \setlength{\itemsep}{4mm}
\item Écrire la matrice $M$ de l'endomorphisme $f$ dans la base $B.$

\item Calculer les valeurs propres de $f$\\
$f$ est-il diagonalisable ?\\
$M$ est-elle inversible ?

\item Déterminer les sous-espaces propres de $f.$
\end{noliste}

\subsection*{3.2. Deuxième partie}

On dispose de deux urnes A et B : initialement l'urne A contient $N$
boules
noires tandis que l'urne B contient $N$ boules blanches, avec $N\geq
2.$
On y effectue une suite d'épreuves, chaque épreuve étant réalisée de la
façon suivante :\\
On tire au hasard une boule dans chacune des deux urnes, la boule tirée
de
l'urne A est mise dans B, celle tirée de B est mise dans A.\\
On appelle $Y_{k}$ la variable aléatoire égale au nombre de boules
noires présentes dans l'urne A à l'issue de la $k^{i\grave{e}me}$
épreuve et l'on pose 
$Z_{k} = Y_{k-1}-Y_{k},$ pour $k$ entier naturel non nul, avec la
convention $Y_{0} = N.$\\
Pour $k$ et $j$ entiers naturels, on pose :
\[
P\left(\Ev{k,j}\right) = P\left(\Ev{Y_{k} = j}\right)
\]
où $P$ désigne la probabilité. Ainsi :\begin{eqnarray*}
P\left(\Ev{Y_{k} & = & j}\right) = 0\text{ si }j>N \\
P\left(\Ev{Y_{0} & = & N}\right) = 1 \\
P\left(\Ev{Y_{0} & = & k}\right) = 0\text{ si }k\neq N \\
P\left(\Ev{Y_{k} & = & -1}\right) = 0
\end{eqnarray*}

\subsubsection*{3.2.1 Étude du cas particulier $N = 2$}

On note $U_{k} = 
\begin{smatrix}
p(k,0) \\
p(k,1) \\
p(k,2)\end{smatrix},\quad V = \dfrac{1}{6}\begin{smatrix}
1 \\
4 \\
1
\end{smatrix}
$ et $W = \dfrac{1}{6}\begin{smatrix}
-1 \\
2 \\
-1
\end{smatrix}.$

\begin{noliste}{1.}
 \setlength{\itemsep}{4mm}
\item Déterminer $U_{1}.$\\
Calculer les probabilités conditionnelles :
\[
P_{(Y_{k} = j)}(Y_{k + 1} = i)
\]
pour $i\in \{0,1,2\}$ et $j\in \{0,1,2\}$ puis montrer que, pour tout
entier
naturel $k :$
\[
U_{k + 1} = M.U_{k}
\]

\item Prouver que, pour tout entier naturel $k$ non nul :
\[
U_{k} = \left( -\dfrac{1}{2}\right) ^{k-1}W + V
\]

\item En déduire l'expression de $p(k,0),$ $p(k,1)$ et $p(k,2)$ en
fonction
de $k$ pour $k$ entier naturel non nul.

\item Montrer que l'espérance $\E(Y_{k})$ de la variable $Y_{k}$ est
constante.

\item Calculer la variance $\V(Y_{k})$ de la variable $Y_{k}$ en
fonction de $k$ et sa limite quand $k$ tend vers $ + \infty.$
\end{noliste}

\subsubsection*{3.2.2. Retour au cas général}

Dans cette deuxième partie, on revient au cas général avec $N\geq 3$ et
on se propose d'étudier la convergence des suites $(\E(Y_{k}))_{k\in
\N^{\times }}$ et $(\V(Y_{k}))_{k\in \N^{\times }}$.

\begin{noliste}{1.}
 \setlength{\itemsep}{4mm}
\item Calcul de l'espérance $\E(Y_{k})$ de la variable $Y_{k}.$

\begin{noliste}{a)}
 \setlength{\itemsep}{2mm}
\item Quelles sont les valeurs prises par la variable $Z_{k}$ ?
Calculer :
\[
P\left(\Ev{Z_{k} = 1/Y_{k-1} = j}\right)\text{ et }P\left(\Ev{Z_{k} =
-1/Y_{k-1} = j}\right)
\]
pour $j\in \N,$ $j\leq N$ et $k\in \N^{\times }.$

\item En appliquant la formule des probabilités totales, prouver que,
pour
tout entier naturel $k$ non nul :
\[
\E(Z_{k}) = \dfrac{2}{N}\E(Y_{k-1})-1.
\]

\item Montrer que la suite $(\E(Z_{k}))_{k\in \N^{\times }}$ est
géométrique.

\item En déduire l'expression de $\E(Z_{k})$ et $\E(Y_{k})$ en fonction
de $k$
et de $N.$

\item Montrer que les suites $(\E(Z_{k}))_{k\in \N^{\times }}$ et
$(\E(Y_{k}))_{k\in \N^{\times }}$ sont convergentes et donner leur
limite quand $k$ tend vers $ + \infty.$
\end{noliste}

\item Calcul de la variance $\V(Y_{k})$ de la variable $Y_{k} :$

\begin{noliste}{a)}
 \setlength{\itemsep}{2mm}
\item Montrer que :
\[
\E(Z_{k}{2}) = E(Y_{k-1}{2})-2\E(Y_{k}Y_{k-1}) + E(Y_{k}{2})
\]
puis que 
\[
\E(Z_{k}Y_{k-1}) = E(Y_{k-1}{2})-\E(Y_{k}Y_{k-1}).
\]

\item En utilisant la méthode du $3.2.2.1.(b)$ montrer que :
\[
\E(Z_{k}Y_{k-1}) = \dfrac{2}{N}\E(Y_{k-1}{2})-\E(Y_{k-1})
\]
puis que :
\[
\E(Z_{k}{2}) = \dfrac{2}{N^{2}}\E(Y_{k-1}{2})-\dfrac{2}{N}\E(Y_{k-1}) +
1
\]

\item Déduire des résultats précédents que :
\[
\E(Y_{k}{2}) = 2\dfrac{N-1}{N}\E(Y_{k-1}) + \dfrac{N^{2}-4N +
2}{N^{2}}\E(Y_{k-1}{2}) + 1
\]
puis que :
\[
\V(Y_{k}) = \dfrac{N^{2}-4N +
2}{N^{2}}\V(Y_{k-1})-\dfrac{2}{N^{2}}\left[
\E(Y_{k-1})-\dfrac{N}{2}\right] ^{2} + \dfrac{1}{2}.
\]
(On utilisera la relation $\E(Y_{k}) = \dfrac{N-2}{N}\E(Y_{k-1}) + 1).$

\item En remarquant que :
\[
\E(Y_{k})-\dfrac{N}{2} = \dfrac{N-2}{N}\left[
E(Y_{k-1})-\dfrac{N}{2}\right]
\]
en déduire que $\V(Y_{k + 1})$ est égal à :
\[
\dfrac{2(N-1)(N-3)}{N^{2}}\V(Y_{k})-\dfrac{(N-2)^{2}(N^{2}-4N +
2)}{N^{4}}\V(Y_{k-1}) + \dfrac{2(N-1)}{N^{2}}
\]

\item On pose :
\[
u_{k} = V(Y_{k})-\dfrac{N^{2}}{4(2N-1)}
\]
Montrer que :
\[
u_{k + 1} = \dfrac{2(N-1)(N-3)}{N^{2}}u_{k}-\dfrac{(N-2)^{2}(N^{2}-4N +
2)}{N^{4}}u_{k-1}
\]

\item En déduire qu'il existe deux réels $\lambda $ et $\mu,$ que l'on
ne
calculera pas, tels que :
\[
\V(Y_{k}) = \dfrac{N^{2}}{4(2N-1)} + \lambda \left(
\dfrac{N-2}{N}\right)
^{2k} + \mu \left( \dfrac{N^{2}-4N + 2}{N^{2}}\right) ^{k}
\]

\item Montrer enfin que $\V(Y_{k})$ tend vers $\dfrac{N^{2}}{4(2N-1)}$
lorsque $k$ tend vers $ + \infty $
\end{noliste}
\end{noliste}

\label{fin}

\end{document}

