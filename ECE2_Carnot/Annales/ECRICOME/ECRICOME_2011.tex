\documentclass[11pt]{article}%
\usepackage{geometry}%
\geometry{a4paper,
 lmargin = 2cm,rmargin = 2cm,tmargin = 2.5cm,bmargin = 2.5cm}

\input{../../macros.tex}

\pagestyle{fancy} %
\lhead{ECE2 \hfill Mathématiques\\
} %
\chead{\hrule} %
\rhead{} %
\lfoot{} %
\cfoot{} %
\rfoot{\thepage} %

\renewcommand{\headrulewidth}{0pt}% : Trace un trait de séparation
 % de largeur 0,4 point. Mettre 0pt
 % pour supprimer le trait.

\renewcommand{\footrulewidth}{0.4pt}% : Trace un trait de séparation
 % de largeur 0,4 point. Mettre 0pt
 % pour supprimer le trait.

\setlength{\headheight}{14pt}

\title{\bf \vspace{-2cm} ECRICOME 2011} %
\author{} %
\date{} %
\begin{document}

\maketitle %
\vspace{-1.4cm}\hrule %
\thispagestyle{fancy}

\vspace*{.2cm}


% DEBUT DU DOC À MODIFIER : tout virer jusqu'au début de l'exo

%Définition et changement de valeurs de
compteurs%newcounter{cpt1}{section} compteur cpt1 remis à 0 à chaque
aumentation par stepcounter du compteur section%setcounter{cpt1}{3} on
met le compteur à 3%addtocounter{cpt1}{5} on ajoute 5 au compteur%
stepcounter{cpt1} on ajoute 1% ifthenelse{test}{alors}{sinon} (page
206) pour subordonner à une condition % whiledo{test}{commande} pour
faire une boucle (page 206 aussi) % value{cpt1} pour noter dans le
document la valeur de cpt1 
%Définition définitive d'opérateurs
mathématiques\newcommand{\ch}{\operatorname{ch}} 
\newcommand{\sh}{\operatorname{sh}}
\renewcommand{\tanh}{\operatorname{th}}
\renewcommand{\sinh}{\operatorname{sh}}
\renewcommand{\cosh}{\operatorname{ch}}
\newcommand{\argsh}{\operatorname{argsh}}
\newcommand{\argch}{\operatorname{argch}}
\newcommand{\argth}{\operatorname{argth}}
\newcommand{\ker}{\operatorname{Ker}}
\renewcommand{\im}{\operatorname{Im}}
\newcommand{\rg}{\operatorname{rg}}
\newcommand{\Id}{\operatorname{Id}}
\renewcommand{\leq}{\leq}
\renewcommand{\geq}{\geq }

\newcommand{\NN}{\mbox{${\mathbb N}$}}
\newcommand{\ZZ}{\mbox{${\mathbb Z}$}}
\newcommand{\QQ}{\mbox{${\mathbb Q}$}}
\newcommand{\RR}{\mbox{${\mathbb R}$}}
\newcommand{\MM}{\mbox{${\mathcal{M}}$}}
\newcommand{\CC}{\mbox{${\mathcal{C}}$}}
\newcommand{\DD}{\mbox{${\mathcal{D}}$}}

\newcommand{\vpp}{\vspace{0.1cm}}
\newcommand{\vp}{\vspace{0.2cm}}
\newcommand{\vg}{\vspace{0.4cm}}
\newcommand{\dsp}{\def \tend #1{\xrightarrow[ \
\phantom{a}#1\phantom{a}]{}}
\def \equi #1{\mathop{\sim}_{\substack{#1}}}
\def 
\[
 {[ \ \![}
\def 
\]
 {]\!]}
\newcommand{\pg}{\geq }
\newcommand{\pp}{\leq}
\newcommand{\dt}{\,\textrm{d}t}
\newcommand{\dx}{\,\textrm{d}x}
\newcommand{\du}{\,\textrm{d}u}
\newcommand{\Ent}{\textrm{Ent}}
\newcommand{\bul}{\item[$\bullet$]}



%Définition de nouvelles couleurs : rgb(trois paramètres red green blue
entre 0 et 1); cmyk (quatre cyan magenta yellow black) entre 0 et 1;
gray (entre 0 et 1) et black, white, red, green, blue, cyan, magenta,
yellow% definecolor{0gris}{gray}{0.8} 
% Nouvelle commande pour encadrer le titre car shabox ne veut que d'une
seule ligne; ATTENTION A LA TAILLE; petite différence avec shadowbox ou
doublebox, voire fcolorbox ou colorbox (au lieu de shabox; laisser le
parbox tranquille sauf pour la taille de la boîte
\newcommand{\Tbox}[1]{\begin{center} \shabox{\parbox{0.6
\linewidth}{#1}} \end{center}} %[1] pour 1 paramètre ; #1 pour ce que
fait le 1er paramètre; entre accolades ce que fait la commande
%Mise en page en mode fancy : en-têtes et pieds de pages puis
définition des en-têtes et pieds de pages\pagestyle{fancy}
\lhead{ECE 2 - Mathématiques \\
Quentin Dunstetter - ENC-Bessières 2011$\backslash$2012}
\chead{}
\rhead{Ecricome 2011}
\rfoot[ \ \thepage]{\thepage}
\cfoot{}
\lfoot{}
\thispagestyle{fancy} %Mise en page de la 1ère page en mode fancy
%Trait en bas et en haut de la page (entre en-tête et texte et texte et
pied de page)\renewcommand{\footrulewidth}{0.4pt}
\renewcommand{\headrulewidth}{0.4pt}

\begin{center}
\Large Ecricome 2011
\end{center}

\noindent {\bf\Large EXERCICE 1.}

\vp\vp
\begin{minipage}{.965\textwidth}
On dit qu'une matrice $A$ carrée d'ordre $n$ est une matrice nilpotente
s'il existe un entier naturel $k$ non nul tel que :
\[
A^{k-1}\neq 0_{n}\quad\textrm{et}\quad A^{k} = 0_{n}
\]

où $0_{n}$ représente la matrice carrée nulle d'ordre $n$.

\vpp
Soit $A$ une matrice carrée d'ordre $n$, on dit que le couple
$(\Delta,N)$ est une décomposition de Dunford de $A$ lorsque :
\[
\left\{
\begin{array}{l}
 \textrm{$\Delta$ est un matrice diagonalisable} \\
\textrm{$N$ est une matrice nilpotente} \\
\Delta N = N\Delta\;\;\textrm{et}\;\; 
A = N + \Delta
\end{array}
\right.
\]
\end{minipage}

\begin{noliste}{1.}
 \setlength{\itemsep}{4mm}
\item On pose :
\[
A = 
\begin{smatrix}
1 & 2 \\
0 & 1
\end{smatrix},\quad \Delta = 
\begin{smatrix}
1 & 0\\
0 & 1
\end{smatrix}
\quad\textrm{et}\quad
N = 
\begin{smatrix}
0 & 2 \\
0 & 0
\end{smatrix}
\]

Vérifier que $(\Delta,N)$ est une décomposition de Dunford de $A$.

\vpp
Dans toute la suite de l'exercice, on pose :
\[
A = 
\begin{smatrix}
3 & 1 & -1\\
-2 & 0 & 2\\
0 & 0 & 1
\end{smatrix},\quad N = 
\begin{smatrix}
0 & 0 & -1\\
0 & 0 & 2\\
0 & 0 & 0
\end{smatrix},\quad
\Delta = 
\begin{smatrix}
3 & 1 & 0\\
-2 & 0 & 0\\
0 & 0 & 1
\end{smatrix},\quad D = 
\begin{smatrix}
2 & 0 & 0\\
0 & 1 & 0\\
0 & 0 & 1
\end{smatrix}
\]

\item\begin{noliste}{a)}
 \setlength{\itemsep}{2mm}
\item Déterminer les valeurs propres de $A$.

\item La matrice $A$ est-elle diagonalisable ?
\end{noliste} 

\item On considère les matrices colonnes 
\[
X_{1} = 
\begin{smatrix}
1\\
-1 \\
0
\end{smatrix},\quad X_{2} = 
\begin{smatrix}
0\\
0 \\
1
\end{smatrix}
\quad\textrm{et}\quad 
X_{3} = 
\begin{smatrix}
1\\
-2 \\
0
\end{smatrix}
\]
\begin{noliste}{a)}
 \setlength{\itemsep}{2mm}
\item Calculer les produits $\Delta X_{1}$, $\Delta X_{2}$ et $\Delta
X_{3}$.

\item Justifier que la matrice $\Delta$ est diagonalisable et
déterminer une matrice $P$ inversible telle que : $P^{-1}\Delta P = D$.

\item Calculer $P^{-1}$. 
\end{noliste} 

\item\begin{noliste}{a)}
 \setlength{\itemsep}{2mm}
\item Établir que $N$ est une matrice nilpotente.

\item Vérifier que $(\Delta,N)$ est une décomposition de Dunford de la
matrice $A$.

\item En utilisant la formule du binôme de Newton que l'on justifiera,
donner l'expression de $A^{n}$ en fonction des puissances de $\Delta$,
de $N$
et de $n$.

\item Établir que : \quad Pour tout entier naturel $k\pg1$, \;\;
$\Delta^{k}N = N$ 

\item Proposer une décomposition de Dunford de $A^{n}$.
\end{noliste} 
\end{noliste}




\newpage
\vg\vg
\noindent {\bf\Large EXERCICE 2.}

\vp
On considère l'application $\varphi$ définie sur $\R_+ $ par :
\[
\left\{
\begin{array}{ll}
 \varphi(x) = 1-x^{2}\ln (x) & \textrm{si $x>0$} \\
\varphi(0) = 1
\end{array}
\right.
\]

ainsi que la fonction numérique $f$ des variables réelles $x$ et $y$
définie par :
\[
\forall\,(x,y)\in\,]0, + \infty[ \ \,\times\,]0, + \infty[,\quad f(x,y)
= xy + \ln(x)\ln(y)
\]

\vp
{\bf I. Étude des zéros de $\varphi$.}
\begin{noliste}{1.}
 \setlength{\itemsep}{4mm}
\item Déterminer la limite de $\varphi(x)$ lorsque $x$ tend vers $ +
\infty$, ainsi que la limite de $\frac{\varphi(x)}{x}$\, lorsque $x$
tend vers 
$ + \infty$. Interpréter graphiquement cette limite.

\item Prouver que $\varphi$ est continue sur $\RR_+ $.

\item Justifier la dérivabilité de $\varphi$ sur $\RR_+^*$ et calculer
sa fonction dérivée.

\item Montrer que $\varphi$ est dérivable en 0. Donner l'allure de la
représentation graphique de $\varphi$ au voisinage du point d'abscisse
0.

\item Dresser le tableau de variations de $\varphi$.

\item On rappelle que $\ln(2)\approx 0,7$. 

Montrer l'existence d'un unique réel $\alpha$ tel que :
$\varphi(\alpha) = 0$ et justifier que :
$\sqrt{2}<\alpha<2$.

\item Établir la convergence de l'intégrale $I = \dint{0}{\alpha}
\varphi(x)\dx$ et vérifier que :
\[
I = \frac{\alpha(6 + \alpha^{2})}{9}
\]

\item On considère les deux suites $(a_{n})_{n\in\NN}$ et
$(b_{n})_{n\in\NN}$ définies par :

\begin{noliste}{$\sbullet$}
\bul $a_{0} = \sqrt{2}$ et $b_{0} = 2$ ;

\bul Pour tout $n\pg0$, si $\varphi(a_{n})\varphi\!\left(\frac{a_{n} +
b_{n}}{2}\right)<0$ alors $a_{n + 1} = a_{n}$ et $b_{n + 1} =
\frac{a_{n} + b_{n}}{2}$ ;

\vpp
\bul Pour tout $n\pg0$, si $\varphi(a_{n})\varphi\!\left(\frac{a_{n} +
b_{n}}{2}\right)\pg0$ alors $a_{n + 1} = \frac{a_{n} + b_{n}}{2}$ et
$b_{n + 1} = b_{n}$.
\end{noliste}

Écrire une programme en \Scilab{} calculant $a_{7}$ et $b_{7}$.
\end{noliste} 


\vp\vpp
{\bf II. Extrema de $f$ sur $]0, + \infty[ \ \,\times\,]0, + \infty[$.}

\vp
Rappelons que $\alpha$ est l'unique réel vérifiant $\varphi(\alpha) =
0$.

\begin{noliste}{1.}
 \setlength{\itemsep}{4mm}
\item Justifier que $f$ est de classe $\mathcal{C}{2}$ sur $]0, +
\infty[ \ \,\times\,]0, + \infty[$.

\item Calculer les dérivées partielles premières et prouver que le
point de coordonnées $\left(\frac{1}{\alpha}\,,\frac{1}{\alpha}\right)$
est
l'unique point critique de $f$ sur $]0, + \infty[ \ \,\times\,]0, +
\infty[$.

\item Calculer les dérivées partielles secondes sur $]0, + \infty[ \
\,\times\,]0, + \infty[$ et établir que pour tous réels $x$ et $y$
strictement positifs :
\[
\left\{
\begin{array}{l}
\frac{\partial^{2}f}{\partial x^{2}}\,(x,y) =
\left(\frac{y}{x}\right)^{2}\left(1-\varphi\!\left(\frac{1}{y}\right)\r
ght) \\
\\
\frac{\partial^{2}f}{\partial y\partial x}\,(x,y) = 1 + \frac{1}{xy} \\
\\
\frac{\partial^{2}f}{\partial y^{2}}\,(x,y) =
\left(\frac{x}{y}\right)^{2}\left(1-\varphi\!\left(\frac{1}{x}\right)\r
ght)
\end{array}
\right.
\]

\item La fonction $f$ présente-t-elle un extremum local sur $]0, +
\infty[ \ \,\times\,]0, + \infty[$ ? Si oui, en donner sa nature
(maximum ou minimum).
\end{noliste} 




\newpage
\vg
\noindent {\bf\Large EXERCICE 3.}

\vp\vpp
{\bf I. Un jeu en ligne.}

\vp\vpp
\begin{minipage}{.965\textwidth}
\noindent Le société Lehazard met à la disposition de ses clients un
nouveau jeu en ligne dont la page d'écran affiche une grille à trois
lignes et trois colonnes.

\vpp
\noindent Après une mise initiale de 2 euros du joueur, une fonction
aléatoire place au hasard successivement trois jetons ($\bigstar$) dans
trois cases différentes.
La partie est gagnée si les trois jetons sont alignés. Le gagnant
empoche 10 fois sa mise, ce qui lui rapporte 18 euros à l'issue du jeu.
Dans le cas
contraire la mise initiale est perdue par le joueur.
\[
\begin{array}{|c|c|c|c|}
 \hline & \textrm{A} & \textrm{B} & \textrm{C} \\
\hline 1 & \bigstar & & \\
\hline 2 & \bigstar & &\\
\hline 3 & &\bigstar & \\
\hline
\end{array}
\]

\vpp
On définit les événements $H$, $V$, $D$, $N$ par :

\hskip 0,5cm - $H$ : \og les trois jetons sont alignés horizontalement
\fg.

\hskip 0,5cm - $V$ : \og les trois jetons sont alignés verticalement
\fg.

\hskip 0,5cm - $D$ : \og les trois jetons sont alignés en diagonale
\fg.

\hskip 0,5cm - $N$ : \og les trois jetons ne sont pas alignés \fg.
\end{minipage}

\vpp
\begin{noliste}{1.}
 \setlength{\itemsep}{4mm}
\item Justifier qu'il y a 84 positionnements possibles des trois jetons
dans les trois cases.

\item Déterminer les probabilités $p(H)$, $p(V)$, $p(D)$ des évènements
$H,V,D$.

\item En déduire que la probabilité de l'événement $N$ est égale à :
\[
p(N) = \frac{19}{21}\approx 0,9048
\]

\item La société peut s'attendre à 10\,000 relances par jour de ce jeu.
\begin{noliste}{a)}
 \setlength{\itemsep}{2mm}
\item Pour chaque entier naturel $i$ non nul, on note $Z_{i}$ le gain
de la société à la $i$-ième relance.

\noindent Calculer l'espérance mathématique $\E(Z_{i})$ de $Z_{i}$.

\item Quel gain journalier $Z$ la société peut-elle espérer ? 
\end{noliste} 
\end{noliste} 


\vp\vpp
\noindent {\bf II. Cas de joueurs invétérés.}

\begin{noliste}{1.}
 \setlength{\itemsep}{4mm}
\item Un joueur décide de jouer 100 parties consécutives que l'on
suppose indépendantes.
\begin{noliste}{a)}
 \setlength{\itemsep}{2mm}
\item Donner la loi de la variable aléatoire $X$ égale au nombre de
parties gagnées.

\item Indiquer l'espérance et la variance de $X$.

\item Exprimer la perte $T$ du joueur en fonction de $X$.
\end{noliste} 

\item Quel est le nombre minimum $n$ de parties qu'il devrait jouer
pour que la probabilité de gagner au moins une partie soit supérieure
ou égale à 50\% ? ({\it On admettra que
$\ln\!\left(\frac{19}{21}\right)\approx-0,1$ et $\ln(2)\approx0,7$}).

\item Un autre joueur décide de jouer et de miser tant qu'une partie
n'est pas gagnée. On note $Y$ la variable aléatoire égale au nombre de
parties jouées pour gagner la première fois.
\begin{noliste}{a)}
 \setlength{\itemsep}{2mm}
\item Donner la loi de $Y$.

\item Indiquer l'espérance et la variance de $Y$.

\item Pour tout entier naturel $k$, montrer que la probabilité $p_{k}$
que le joueur joue au plus $k$ parties avant de gagner pour la première
fois
est donnée par la formule :
\[
p_{k} = 1-\left(\frac{19}{21}\right)^{k}
\]
\end{noliste} 
\end{noliste} 


\vp\vpp
\noindent {\bf III. Contrôle de la qualité du jeu.}

\vp\vpp
\begin{minipage}{.965\textwidth}
\noindent On constate que, parfois, la fonction aléatoire est déréglée.
Dans ce cas, elle place le premier jeton dans la case $(\textrm{A},1)$,
les deux autres étant placés au hasard dans les cases restantes. On
note $\Delta$ l'évènement \og la fonction aléatoire est déréglée \fg
\;et on pose $p(\Delta) = x$ avec $x\in\,]0,1[$.
\end{minipage}

\begin{noliste}{1.}
 \setlength{\itemsep}{4mm}
\item Calculer les probabilités conditionnelles $p_{\Delta}(H)$,
$p_{\Delta}(V)$ et $p_{\Delta}(D)$ des évènements $H,V,D$ sachant
l'évènement $\Delta$.

\item Utiliser la formule des probabilités totales avec le système
complet d'évènements $(\Delta,\overline{\Delta})$ pour en déduire que
la probabilité que les jetons ne soient pas alignés est égale à :
\[
P\left(\Ev{N}\right) = -\frac{x}{84} + \frac{19}{21}
\]

\item Soit $G$ la variable aléatoire égale au gain réalisé par la
société de jeu lors d'une partie jouée. Déterminer la valeur maximale
de $x$ pour que l'espérance du gain soit positive.

\item On joue une partie. On constate que les jetons sont alignés.
Quelle est la probabilité, en fonction de $x$, que la fonction
aléatoire
ait été déréglée ?
\end{noliste} 





\end{document}