\documentclass[11pt]{article}%
\usepackage{geometry}%
\geometry{a4paper,
 lmargin = 2cm,rmargin = 2cm,tmargin = 2.5cm,bmargin = 2.5cm}

\input{../../macros.tex}

\pagestyle{fancy} %
\lhead{ECE2 \hfill Mathématiques\\
} %
\chead{\hrule} %
\rhead{} %
\lfoot{} %
\cfoot{} %
\rfoot{\thepage} %

\renewcommand{\headrulewidth}{0pt}% : Trace un trait de séparation
 % de largeur 0,4 point. Mettre 0pt
 % pour supprimer le trait.

\renewcommand{\footrulewidth}{0.4pt}% : Trace un trait de séparation
 % de largeur 0,4 point. Mettre 0pt
 % pour supprimer le trait.

\setlength{\headheight}{14pt}

\title{\bf \vspace{-2cm} ECRICOME 2007} %
\author{} %
\date{} %
\begin{document}

\maketitle %
\vspace{-1.4cm}\hrule %
\thispagestyle{fancy}

\vspace*{.2cm}


% DEBUT DU DOC À MODIFIER : tout virer jusqu'au début de l'exo

%Définition et changement de valeurs de
compteurs%newcounter{cpt1}{section} compteur cpt1 remis à 0 à chaque
aumentation par stepcounter du compteur section%setcounter{cpt1}{3} on
met le compteur à 3%addtocounter{cpt1}{5} on ajoute 5 au compteur%
stepcounter{cpt1} on ajoute 1% ifthenelse{test}{alors}{sinon} (page
206) pour subordonner à une condition % whiledo{test}{commande} pour
faire une boucle (page 206 aussi) % value{cpt1} pour noter dans le
document la valeur de cpt1 
%Définition définitive d'opérateurs
mathématiques\newcommand{\ch}{\operatorname{ch}} 
\newcommand{\sh}{\operatorname{sh}}
\renewcommand{\tanh}{\operatorname{th}}
\renewcommand{\sinh}{\operatorname{sh}}
\renewcommand{\cosh}{\operatorname{ch}}
\newcommand{\argsh}{\operatorname{argsh}}
\newcommand{\argch}{\operatorname{argch}}
\newcommand{\argth}{\operatorname{argth}}
\newcommand{\Id}{\operatorname{Id}}
\renewcommand{\leq}{\leq}
\renewcommand{\geq}{\geq }

\newcommand{\dlim}{\lim}
\newcommand{\dsum}{\sum}
\newcommand{\dprod}{\prod}



%Définition de nouvelles couleurs : rgb(trois paramètres red green blue
entre 0 et 1); cmyk (quatre cyan magenta yellow black) entre 0 et 1;
gray (entre 0 et 1) et black, white, red, green, blue, cyan, magenta,
yellow% definecolor{0gris}{gray}{0.8} 
% Nouvelle commande pour encadrer le titre car shabox ne veut que d'une
seule ligne; ATTENTION A LA TAILLE; petite différence avec shadowbox ou
doublebox, voire fcolorbox ou colorbox (au lieu de shabox; laisser le
parbox tranquille sauf pour la taille de la boîte
\newcommand{\Tbox}[1]{\begin{center} \shabox{\parbox{0.6
\linewidth}{#1}} \end{center}} %[1] pour 1 paramètre ; #1 pour ce que
fait le 1er paramètre; entre accolades ce que fait la commande
%Mise en page en mode fancy : en-têtes et pieds de pages puis
définition des en-têtes et pieds de pages\pagestyle{fancy}
\lhead{ECE 2 - Mathématiques \\
Quentin Dunstetter - ENC-Bessières 2011$\backslash$2012}
\chead{}
\rhead{Edhec 2004}
\rfoot[ \ \thepage]{\thepage}
\cfoot{}
\lfoot{}
\thispagestyle{fancy} %Mise en page de la 1ère page en mode fancy
%Trait en bas et en haut de la page (entre en-tête et texte et texte et
pied de page)\renewcommand{\footrulewidth}{0.4pt}
\renewcommand{\headrulewidth}{0.4pt}

\begin{center}
\textbf{ECRICOME 2007}
\end{center}

\section*{Exercice 1}

Soit $a$ un réel strictement positif. On considère la fonction $f_{a}
$ définie pour tout réel $t$ strictement positif par :

\[
f_{a}(t) = \dfrac 12(t + \dfrac{a^{2}}t)
\]
ainsi que la suite $(u_{n})_{n\epsilon \N}$ de nombre réels déterminée
par son premier terme $u_{0}>0$ et par la relation de récurrence : 
\[
\forall n\in \mathbb{N\qquad }u_{n + 1} = f_{a}(u_{n})
\]

\subsection*{1) Étude des variations de la fonction $f_{a}$.}

\begin{noliste}{1.}
 \setlength{\itemsep}{4mm}
\item Déterminer la limite de $f_{a}(t)$ lorsque $t$ tend vers $ +
\infty $.
Justifier l'existence d'une asymptote oblique au voisinage de $ +
\infty $ et
donner la position de la courbe représentative de $f_{a}$ par rapport à
cette asymptote.

\item Déterminer la limite de $f_{a}(t)$ lorsque $t$ tend vers $0$ par
valeurs positives. Interpréter graphiquement cette limite.

\item Donner l'expression de la fonction dérivée de $f_{a}$ sur $\R^{*
+}$ et dresser le tableau de variation de $f_{a}$.

\item En déduire que : 
\[
\forall t>0\qquad f_{a}(t)\geq a
\]
\end{noliste}

\subsection*{2) Étude de la convergence de la suite $(u_{n})_{n\protect
\epsilon 
\N}$..}

\begin{noliste}{1.}
 \setlength{\itemsep}{4mm}
\item Que dire de la suite $(u_{n})_{n\epsilon \N}$ dans le cas
particulier où $u_{0} = a$ ?

\item Dans la suite on revient au cas général $u_{0}>0$.\\
Démontrer que : 
\[
\forall t>a\qquad 0<f_{a}{\prime }(t)<\frac 12
\]

\item Montrer que pour tout entier $n$, non nul : 
\[
\left. u_{n}\geq a\right.
\]

\item Prouver alors que pour tout entier $n$ non nul : 
\[
0\leq u_{n + 1}-a\leq \dfrac{1}{2}\left( u_{n}-a\right)
\]
Puis que : 
\[
\left| u_{n}-a\right| \leq \left( \dfrac{1}{2}\right)
^{n-1}\left| u_{1}-a\right|
\]

\item En déduire la convergence de la suite $(u_{n})$ et indiquer sa
limite.

\item En utilisant ce qui précède, écrire un programme en
langage \Scilab{} permettant d'afficher les $100$ premiers termes d'une
suite $(u_{n})$, de premier terme $1$, convergeant vers $\sqrt{2}$.
\end{noliste}

\subsection*{3) Recherche d'extremum d'une fonction à deux variables.}

On considère, sur $\R_{+}{*}\times \R_{+}{*}$, la
fonction $g$ définie par~ : 
\[
g(x,y) = \frac 12(\frac 1x + \frac 1y)(1 + x)(1 + y)
\]

\begin{noliste}{1.}
 \setlength{\itemsep}{4mm}
\item Calculer les dérivées partielles d'ordre $1$ et $2$ de $g$ sur 
$\R_{+}{*}\times \R_{+}{*}.$

\item Montrer que $g$ admet un extremum local sur $\R_{+}{*}\times 
\R_{+}{*}$ dont on précisera la nature.

\item Vérifier que~ : 
\[
g(x,y) = 1 + f_{1}(x) + f_{1}(y) + f_{1}(\frac xy)
\]

\item En déduire que l'extremum local est un extremum global de $g$ sur
$\R_{+}{\ast }\times \R_{+}{\ast }$.
\end{noliste}

\section*{Exercice 2}

$M_{2}(\R)$ désigne l'espace vectoriel des matrices carrées
d'ordre $2$ à coefficients réels.\\
La matrice $A$ suivante étant donnée 
\[
A = \left( 
\begin{array}{cc}
3 & -1 \\
6 & -2
\end{array}
\right) 
\]
on définit l'application $\phi_{A}$ par : 
\[
\left. 
\begin{array}{c}
\phi_{A} :M_{2}(\R)\rightarrow M_{2}(\R) \\
\qquad \quad \qquad \qquad M\mapsto \phi_{A}(M) = AM-MA
\end{array}
\right. 
\]

\subsection*{1) Diagonalisation de A.}

\begin{noliste}{1.}
 \setlength{\itemsep}{4mm}
\item Vérifier que $A^{2} = A.$ En déduire les valeurs propres
possibles
de $A$.

\item Prouver que la matrice $A$ est diagonalisable et déterminer une
matrice $P$ inversible de $M_{2}(\R)$ et une matrice diagonale $D$ de
$M_{2}(\R)$ dont la première colonne est nulle vérifiant la
relation : 
\[
A = PDP^{-1}
\]
Donner l'écriture matricielle de $P^{-1}.$
\end{noliste}

\subsection*{2) Diagonalisation de $\protect \phi_{A}.$}

\begin{noliste}{1.}
 \setlength{\itemsep}{4mm}
\item Montrer que $\phi_{A}$ est un endomorphisme de $M_{2}(\R)$.

\item Établir que $X^{3}-X$ est un polyn\^{o}me annulateur de
$\phi_{A}$. En déduire les valeurs propres possibles de $\phi_{A}$.

\item Montrer que la matrice $M$ est un vecteur propre de $\phi_{A}$
associée à la valeur propre $\lambda $ si et seulement si la matrice $N
= P^{-1}MP$ est non nulle et vérifie l'équation matricielle : 
\[
DN-ND = \lambda N
\]

\item On pose $N = \left( 
\begin{array}{cc}
a & b \\
c & d
\end{array}
\right).$

\begin{noliste}{a)}
 \setlength{\itemsep}{2mm}
\item Trouver l'ensemble des matrices $N$ telles que $DN-ND = 0$.

\item En déduire que la famille $(A$,$M_{1})$ avec $M_{1} = \left( 
\begin{array}{cc}
-2 & 1 \\
-6 & 3
\end{array}
\right) $ est une base du sous-espace propre $Ker\phi_{A}$ associé à
la valeur propre $0$.

\item Déterminer les deux autres valeurs propres non nulles
$\lambda_{1}$
et $\lambda_{2}$ de $\phi_{A}$ et \\
caractériser les matrices $N$ associées.

\item En déduire une base de chaque sous-espace propre
$E_{\lambda_{1}}$
et $E_{\lambda_{1}}$associé aux valeurs propres $\lambda_{1}$ et
$\lambda
_{2}. $
\end{noliste}

\item L'endomorphisme $\phi_{A}$ est-il diagonalisable ?
\end{noliste}

\section*{Exercice 3}

Soucieux d'améliorer le flux de sa clientèle lors du passage en
caisse, un gérant de magasin a réalisé les observations
suivantes :

\subsection*{1) Mode de paiement de la clientèle.}

\begin{noliste}{1.}
 \setlength{\itemsep}{4mm}
\item L'étude du mode de paiement en fonction du montant des achats a
permis d'établir les probabilités suivantes : 
\[
\left. 
\begin{array}{c}
P\left[ S = 0\cap U = 0\right] = 0.4 \\
P\left[ S = 0\cap U = 1\right] = 0.3 \\
P\left[ S = 1\cap U = 0\right] = 0.2 \\
P\left[ S = 1\cap U = 1\right] = 0.1
\end{array}
\right.
\]
où $S$ représente la variable aléatoire prenant la valeur $0$ si
le montant des achats est inférieur ou égal à $50$ euros,
prenant la valeur $1$ sinon, et $U$ la variable aléatoire prenant la
valeur $0$ si la somme est réglée par carte bancaire, prenant la
valeur $1$ sinon.

\begin{noliste}{a)}
 \setlength{\itemsep}{2mm}
\item Déterminer les lois de $S$ et $U$ et vérifier que la probabilité
que le client règle par carte bancaire est égale à $p = \dfrac 35$.

\item Calculer la covariance du couple $(S,U)$. Les variables $S$ et
$U$
sont-elles indé-pendantes ?

\item Quelle est la probabilité que la somme réglée soit supérieure
strictement à $50$ euros sachant que le client utilise un autre
moyen de paiement que la carte bancaire ?
\end{noliste}

\item On suppose que les modes de règlement sont indépendants entre
les individus.\\
Une caissière reçoit $n$ clients dans sa journée $(n\geq 2)$.\\
On définit trois variables aléatoires $C_{n},L_{1},L_{2}$ par :\\
-$C_{n}$ comptabilise le nombre de clients qui paient par carte
bancaire.\\
-$L_{1}$(resp.$L_{2})$ est égale au rang du $1^{er}$ (resp.du
$2^{\grave{e}me}$) client utilisant la carte bancaire comme moyen de
paiement, s'il y
en a au moins un (resp.au moins deux) et à zéro sinon.

\begin{noliste}{a)}
 \setlength{\itemsep}{2mm}
\item Reconna\^{\i} tre la loi de $C_{n}$, rappeler la valeur de
l'espérance et de la variance de cette variable aléatoire.

\item Déterminer la loi de $L_{1}$ et vérifier que : 
\[
\Sum{k = 0}{n} P\left[ L_{1} = k\right] = 1
\]

\item Déterminer la loi de $L_{2}$.
\end{noliste}
\end{noliste}

\subsection*{2) Étude du temps moyen de passage en caisse.}

Après enquête, on estime que le temps de passage à une caisse,
exprimé en unités de temps, est une variable aléatoire $T$ dont
une densité de probabilité est donnée par la fonction $f$ définie par :

\[
\left\{ \left. 
\begin{array}{l}
f(x) = xe^{-x}\quad \text{si }x\geq 0 \\
f(x) = 0\quad \quad \quad \text{si }x<0
\end{array}
\right. \right.
\]

\begin{noliste}{1.}
 \setlength{\itemsep}{4mm}
\item Rappeler la définition d'une densité de probabilité d'une
variable aléatoire $X$ suivant une loi exponentielle de paramètre
$\lambda = 1$. Donner la valeur de l'espérance et de la variance de
$X$.

\item Utiliser la question précédente pour vérifier que $f$ est
bien une densité de probabilité, puis montrer que $T$ admet une
espérance que l'on déterminera.\\
Quel est le temps moyen de passage en caisse ?

\item

\begin{noliste}{a)}
 \setlength{\itemsep}{2mm}
\item Démontrer que la fonction de répartition de $T$, notée $F_{T}$
est définie par : 
\[
\left. 
\begin{array}{l}
\forall x<0\qquad F_{T}(x) = 0 \\
\forall x\geq 0\qquad F_{T}(x) = 1-(x + 1)e^{-x}
\end{array}
\right.
\]

\item Montrer que la probabilité que le temps de passage en caisse soit
inférieur à deux unités(de temps) sachant qu'il est supérieur à une
unité est égale à $\dfrac{2e-3}{2e}.$
\end{noliste}

\item Un jour donné, trois clients $A,B,C$ se présentent simultanément
devant deux caisses libres. Par courtoisie, $C$ décide de
laisser passer $A $ et $B$ et de prendre la place du premier d'entre
eux qui
aura terminé. On suppose que les variables $T_{A}$ et $T_{B}$
correspondant
au temps de passage en caisse de $A$ et $B$ sont indépendantes.

\begin{noliste}{a)}
 \setlength{\itemsep}{2mm}
\item $M$ désignant le temps d'attente du client $C$ exprimer $M$ en
fonction de $T_{A}$ et $T_{B}$.

\item Montrer que la fonction de répartition de la variable aléatoire
$M$ est donnée par : 
\[
\left\{ 
\begin{array}{l}
\forall t\in \R^{+}\mathbb{\qquad }P\left[ M\leq t\right]
 = 1-(1 + t)^{2}e^{-2t} \\
\forall t\in \R^{-\ast }\mathbb{\qquad }P\left[ M\leq t\right] = 0
\end{array}
\right.
\]

\item Prouver que $M$ est une variable à densité et expliciter une
densité de $M$.
\end{noliste}
\end{noliste}

\end{document}

