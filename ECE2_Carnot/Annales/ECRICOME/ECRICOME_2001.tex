\documentclass[11pt]{article}%
\usepackage{geometry}%
\geometry{a4paper,
 lmargin = 2cm,rmargin = 2cm,tmargin = 2.5cm,bmargin = 2.5cm}

\input{../../macros.tex}

\pagestyle{fancy} %
\lhead{ECE2 \hfill Mathématiques\\
} %
\chead{\hrule} %
\rhead{} %
\lfoot{} %
\cfoot{} %
\rfoot{\thepage} %

\renewcommand{\headrulewidth}{0pt}% : Trace un trait de séparation
 % de largeur 0,4 point. Mettre 0pt
 % pour supprimer le trait.

\renewcommand{\footrulewidth}{0.4pt}% : Trace un trait de séparation
 % de largeur 0,4 point. Mettre 0pt
 % pour supprimer le trait.

\setlength{\headheight}{14pt}

\title{\bf \vspace{-2cm} ECRICOME 2001} %
\author{} %
\date{} %
\begin{document}

\maketitle %
\vspace{-1.4cm}\hrule %
\thispagestyle{fancy}

\vspace*{.2cm}


% DEBUT DU DOC À MODIFIER : tout virer jusqu'au début de l'exo

%Définition et changement de valeurs de
compteurs%newcounter{cpt1}{section} compteur cpt1 remis à 0 à chaque
aumentation par stepcounter du compteur section%setcounter{cpt1}{3} on
met le compteur à 3%addtocounter{cpt1}{5} on ajoute 5 au compteur%
stepcounter{cpt1} on ajoute 1% ifthenelse{test}{alors}{sinon} (page
206) pour subordonner à une condition % whiledo{test}{commande} pour
faire une boucle (page 206 aussi) % value{cpt1} pour noter dans le
document la valeur de cpt1 
%Définition définitive d'opérateurs
mathématiques\newcommand{\ch}{\operatorname{ch}} 
\newcommand{\sh}{\operatorname{sh}}
\renewcommand{\tanh}{\operatorname{th}}
\renewcommand{\sinh}{\operatorname{sh}}
\renewcommand{\cosh}{\operatorname{ch}}
\newcommand{\argsh}{\operatorname{argsh}}
\newcommand{\argch}{\operatorname{argch}}
\newcommand{\argth}{\operatorname{argth}}
\newcommand{\Id}{\operatorname{Id}}

%Définition de nouvelles couleurs : rgb(trois paramètres red green blue
entre 0 et 1); cmyk (quatre cyan magenta yellow black) entre 0 et 1;
gray (entre 0 et 1) et black, white, red, green, blue, cyan, magenta,
yellow% definecolor{0gris}{gray}{0.8} 
% Nouvelle commande pour encadrer le titre car shabox ne veut que d'une
seule ligne; ATTENTION A LA TAILLE; petite différence avec shadowbox ou
doublebox, voire fcolorbox ou colorbox (au lieu de shabox; laisser le
parbox tranquille sauf pour la taille de la boîte
\newcommand{\Tbox}[1]{\begin{center} \shabox{\parbox{0.6
\linewidth}{#1}} \end{center}} %[1] pour 1 paramètre ; #1 pour ce que
fait le 1er paramètre; entre accolades ce que fait la commande
%Mise en page en mode fancy : en-têtes et pieds de pages puis
définition des en-têtes et pieds de pages\pagestyle{fancy}
\lhead{ECE 2 - Mathématiques \\
Quentin Dunstetter - ENC-Bessières 2011$\backslash$2012}
\chead{}
\rhead{Epreuve Ecricome 2001}
\rfoot[ \ \thepage]{\thepage}
\cfoot{}
\lfoot{}
\thispagestyle{fancy} %Mise en page de la 1ère page en mode fancy
%Trait en bas et en haut de la page (entre en-tête et texte et texte et
pied de page)\renewcommand{\footrulewidth}{0.4pt}
\renewcommand{\headrulewidth}{0.4pt}


%DEBUT DU DOCUMENT
\noindent {\Large \textbf{ECRI%TCIMACRO{\TeXButton{TeX
field}{\colorbox[gray]{0.95}{COME}}}%BeginExpansion
\colorbox[gray]{0.95}{COME}%EndExpansion
}}\vspace{0.3cm}

\noindent \textbf{Banque d'épreuves communes}

\noindent aux concours des Ecoles

\noindent esc%TCIMACRO{\TeXButton{TeX
field}{\colorbox[gray]{0.95}{bordeaux}} }%BeginExpansion
\colorbox[gray]{0.95}{bordeaux}
%EndExpansion
/ esc%TCIMACRO{\TeXButton{TeX field}{\colorbox[gray]{0.95}{marseille}}
}%BeginExpansion
\colorbox[gray]{0.95}{marseille}
%EndExpansion
/ icn%TCIMACRO{\TeXButton{TeX field}{\colorbox[gray]{0.95}{nancy}}
}%BeginExpansion
\colorbox[gray]{0.95}{nancy}
%EndExpansion
/ esc%TCIMACRO{\TeXButton{TeX field}{\colorbox[gray]{0.95}{reims}}
}%BeginExpansion
\colorbox[gray]{0.95}{reims}
%EndExpansion
/ esc%TCIMACRO{\TeXButton{TeX field}{\colorbox[gray]{0.95}{rouen}}
}%BeginExpansion
\colorbox[gray]{0.95}{rouen}
%EndExpansion
/ esc%TCIMACRO{\TeXButton{TeX
field}{\colorbox[gray]{0.95}{toulouse}}}%BeginExpansion
\colorbox[gray]{0.95}{toulouse}%EndExpansion
\vspace{1cm}

\begin{center}
{\large CONCOURS D'ADMISSION }\vspace{0.5cm}

\textbf{option économique} \vspace{0.5cm}

{\Large \textbf{MATHÉMATIQUES}} \vspace{0.5cm}

\textbf{Année 2001}
\end{center}

\noindent \textbf{Aucun instrument de calcul n'est autorisé.}

\noindent \textbf{Aucun document n'est autorisé.}

\noindent L'énoncé comporte \pageref{fin} pages

\begin{quotation}
\noindent Les candidats sont invités à soigner la présentation de leur
copie, à mettre en évidence les principaux résultats, à respecter les
notations de l'énoncé, et à donner des démonstrations complètes (mais
brèves) de leurs affirmations.
\end{quotation}

\newpage

\section*{EXERCICE 1}

Dans cet exercice on étudie l'évolution au cours du temps d'un titre
dans
une bourse de valeurs.

\subsection*{I. Le but de la première partie est de calculer les
puissances
successives de la matrice :}

\[
M(a) = 
\begin{smatrix}
1-2a & a & a \\
a & 1-2a & a \\
a & a & 1-2a
\end{smatrix}
\]
où $a$ représente un nombre réel.

\begin{noliste}{1.}
 \setlength{\itemsep}{4mm}
\item Montrer que, pour tous réels $a$, $b$, on a : $M(a).M(b) = M(a +
b-3ab)$.

\item En déduire les valeurs de $a$ pour lesquelles la matrice $M(a)$
est
inversible et exprimer son inverse.

\item Justifier le fait que $M(a)$ est diagonalisable.

\item Déterminer le réel $a_{0}$ non nul, tel que : 
\[
\left[ M(a_{0})\right] ^{2} = M(a_{0})
\]

\item On considère les matrices : 
\[
P = M(a_{0})\quad \text{et}\quad Q = I-P
\]
où $I$ désigne la matrice carrée unité d'ordre $3$.

\begin{noliste}{a)}
 \setlength{\itemsep}{2mm}
\item Montrer qu'il existe un réel $\alpha $, que l'on exprimera en
fonction
de $a$, tel que : 
\[
M(a) = P + \alpha Q
\]

\item Calculer $P^{2}$, $QP$, $PQ$, $Q^{2}$.

\item Pour tout entier naturel $n$, non nul, montrer que $\left[
M(a)\right]
^{n}$ s'écrit comme combinaison linéaire de $P$ et $Q$.

\item Expliciter alors la matrice $\left[ M(a)\right] ^{n}$.
\end{noliste}
\end{noliste}

\subsection*{II. \textbf{Évolution d'un titre boursier au cours du
temps.}}

Dans la suite de l'exercice, on suppose que $a\in \left]\ 0\ ;\
\dfrac{2}{3}\
\right[ $.

\begin{noliste}{1.}
 \setlength{\itemsep}{4mm}
\item On définit des suites $(p_{n})_{n\in \N^{\ast }}$, $
(q_{n})_{n\in \N^{\ast }}$, $(r_{n})_{n\in \N^{\ast }}$ par
leur premier terme $p_{1}$, $q_{1}$, $r_{1}$, et les relations de
récurrence : 
\[
\left\{
\begin{array}{cl}
p_{n + 1} = (1-2a)p_{n} + aq_{n} + ar_{n} \\
q_{n + 1} = ap_{n} + (1-2a)q_{n} + ar_{n} \\
r_{n + 1} = ap_{n} + aq_{n} + (1-2a)r_{n}
\end{array}
\right.
\]

\begin{noliste}{a)}
 \setlength{\itemsep}{2mm}
\item Exprimer $p_{n}$, $q_{n}$, $r_{n}$ en fonction de $
n,p_{1},q_{1},r_{1}. $

\item Étudier la convergence de ces suites.
\end{noliste}

\item Dans une bourse de valeurs, un titre donné peut monter, rester
stable,
ou baisser. Dans un modèle mathématique, on considère que :

\begin{noliste}{$\sbullet$}
\item le premier jour le titre est stable ;

\item si un jour $n$, le titre monte, le jour $n + 1$, il montera avec
la
probabilité $\dfrac{2}{3}$, restera stable avec la probabilité
$\dfrac{1}{6}$, et baissera avec la probabilité $\dfrac{1}{6}$;

\item si un jour $n$, le titre est stable, le jour $n + 1$, il montera
avec la
probabilité $\dfrac{1}{6}$, restera stable avec la probabilité
$\dfrac{2}{3}$, et baissera avec la probabilité $\dfrac{1}{6}$;

\item si un jour $n$, le titre baisse, le jour $n + 1$, il montera avec
la
probabilité $\dfrac{1}{6}$, restera stable avec la probabilité
$\dfrac{1}{6}$, et baissera avec la probabilité $\dfrac{2}{3}$.
\end{noliste}

\noindent On note $M_{n}$ (respectivement $S_{n}$, respectivement
$B_{n}$) l'
évènement \textquotedblleft le titre donné monte (respectivement reste
stable, respectivement baisse) le jour $n$.

\begin{noliste}{a)}
 \setlength{\itemsep}{2mm}
\item Exprimer les probabilités de hausse, de stabilité, et de baisse
au
jour $n + 1$ en fonction de ces mêmes probabilités au jour $n$.

\item En déduire les probabilités de hausse, de stabilité, et de baisse
au
jour $n$.

\item Quelles sont les limites de ces probabilités lorsque $n$ tend
vers
l'infini ?
\end{noliste}
\end{noliste}

\section*{EXERCICE 2}

Un système est constitué de $n$ composants. On suppose que les
variables aléatoires $T_{1},T_{2},\ldots,T_{n}$ mesurant le temps de
bon fonctionnement de chacun des $n$ composants sont indépendantes, de
même loi, la loi exponentielle de paramètre $\lambda >0$.

\subsection*{I. Calcul du nombre moyen de composants défaillants entre
les
instants $0$ et $t$.}

On note $N_{t}$ la variable aléatoire égale au nombre de composants
défaillants entre les instants $0$ et $t$ avec $t\geq 0$.

\begin{noliste}{1.}
 \setlength{\itemsep}{4mm}
\item Pour tous les entiers $i$ de $\{1,2,\ldots,n\}$, calculer la
probabilité de l'évènement $\{T_{i}<t\}$.

\item Montrer que $N_{t}$ suit une loi binômiale. Préciser ses
paramètres et
son espérance $\E(N_{t})$.

\item \'A partir de quel instant $t_{0}$ le nombre moyen de composants
défaillants dépasse-t-il la moitié du nombre de composants ?
\end{noliste}

\subsection*{II. Montage en série.}

On suppose que le système fonctionne correctement si tous les
composants
eux-mêmes fonctionnent correctement et note $S_{n}$ la variable
aléatoire
mesurant le temps de bon fonctionnement du système.

\begin{noliste}{1.}
 \setlength{\itemsep}{4mm}
\item Pour $t\in \R$, exprimer l'évènement $\{S_{n}>t\}$ en fonction
des évènements : 
\[
\{T_{1}>t\},\;\{T_{2}>t\},\;\ldots \;\{T_{n}>t\}.
\]

\item Déterminer alors la fonction de répartition $F_{n}$, de $S_{n}$,
puis définir sa densité $f_{n}$.

\item Reconnaître la loi de $S_{n}$ et donner sans calcul l'espérance $
\E(S_{n})$ et la variance $\V(S_{n})$ de $S_{n}$.
\end{noliste}

\subsection*{III. Montage en parallèle.}

On suppose maintenant que le système fonctionne correctement si l'un au
moins des composants fonctionne correctement et note $U_{n}$ la
variable aléatoire mesurant le temps de bon fonctionnement du système.

\begin{noliste}{1.}
 \setlength{\itemsep}{4mm}
\item Exprimer $\{U_{n}<t\}$ en fonction des évènements $\{T_{1}<t\},\;
\{T_{2}<t\},\;\ldots \;\{T_{n}<t\}$.

\item Déterminer alors la fonction de répartition $G_{n}$ de $U_{n}$
puis
montrer que sa densité $g_{n}$ est définie par : 
\[
\left\{
\begin{array}{cl}
g_{n}(t) = n\lambda \left( 1-e^{-\lambda t}\right) ^{n-1}e^{-\lambda
t}, & 
t\geq 0 \\
g_{n}(t) = 0, & t<0
\end{array}
\right.
\]

\item Montrer l'existence de l'espérance $\E(U_{n})$ de $U_{n}$ et
prouver
que : 
\[
\E(U_{n}) = \dfrac{1}{\lambda }\Sum{k = 0}{n-1}\dfrac{(-1)^{k}}{k +
1}C_{n}{k + 1}
\]
puis, que pour tous entiers naturels $n$, 
\[
\E(U_{n + 1})-\E(U_{n}) = \dfrac{1}{\lambda (n + 1)}
\]

\item Par sommation de la relation précédente, et en utilisant
l'équivalent
simple : 
\[
\Sum{k = 1}{n}\dfrac{1}{k}\underset{n\rightarrow + \infty }{\sim }\ln n
\]
donner un équivalent simple de $\E(U_{n})$ lorsque $n$ tend vers $ +
\infty $.
\end{noliste}

\section*{EXERCICE 3}

On désigne par $n$ un entier naturel non nul et $a$ un réel strictement
positif.\\
On se propose d'étudier les racines de l'équation : 
\[
(E_{n})\; :\;\dfrac{1}{x} + \dfrac{1}{x + 1} + \dfrac{1}{x + 2} +
\ldots + \dfrac{1}{x + 2n} = a
\]
\'A cet effet, on introduit la fonction $f_{n}$, de la variable réelle
$x$ dé
finie par : 
\[
f_{n}(x) = \dfrac{1}{x} + \dfrac{1}{x + 1} + \dfrac{1}{x + 2} + \cdots
+ \dfrac{1}{x + 2n}-a
\]

\subsection*{I. Étude d'un cas particulier.}

Pour cette question seulement, on prend $a = \dfrac{11}{6}$ et $n = 1$.

\begin{noliste}{1.}
 \setlength{\itemsep}{4mm}
\item Représenter la fonction $f_{1}$ relativement à un repère
orthonormal
du plan. (unité graphique 2 cm)

\item Calculer $f_{1}(1)$, puis déterminer les racines de $(E_{1})$.\\
(On donne $\sqrt{37} = 6{,}08$ à $10^{-2}$ près par défaut)
\end{noliste}

\subsection*{II. Dénombrement des racines de $(E_{n})$.}

\begin{noliste}{1.}
 \setlength{\itemsep}{4mm}
\item Dresser le tableau de variations de $f_{n}$.

\item Justifier l'existence de racines de l'équation $(E_{n})$ et en
déterminer le nombre.
\end{noliste}

\subsection*{III. Équivalent de la plus grande des racines quand $n$
tend
vers $ + \infty $.}

On note $x_{n}$ la plus grande des racines de $(E_{n})$.

\begin{noliste}{1.}
 \setlength{\itemsep}{4mm}
\item Justifier que $x_{n}>0$.

\item Démontrer que pour tout réel $x>1$ : 
\[
\dfrac{1}{x}<\ln \dfrac{x}{x-1}<\dfrac{1}{x-1}
\]
En déduire que pour $x$ réel strictement positif : 
\[
f_{n}(x)-\dfrac{1}{x} + a<\ln (1 + \dfrac{2n}{x})<f_{n}(x)-\dfrac{1}{x
+ 2n} + a
\]
puis, que : 
\[
a-\dfrac{1}{x_{n}}<\ln \left( 1 + \dfrac{2n}{x_{n}}\right)
<a-\dfrac{1}{x_{n} + 2n}
\]

\item Montrer que pour tout $n$ entier naturel, non nul : 
\[
x_{n}>\dfrac{2n}{\exp a-1}
\]

\item Quelle est la limite de $x_{n}$, puis la limite de $\ln (1 +
\dfrac{2n}{x_{n}})$, lorsque $n$ tend vers $ + \infty $ ?

\item Prouver enfin l'existence d'un réel $\delta $, que l'on exprimera
en
fonction de $a$, tel que l'on ait, au voisinage de l'infini,
l'équivalent
suivant
\[
x_{n}\underset{n\rightarrow + \infty }{\sim }\delta.n
\]

\end{noliste}

\begin{center}
Fin de l'épreuve \label{fin}
\end{center}

\end{document}

