\documentclass[11pt]{article}%
\usepackage{geometry}%
\geometry{a4paper,
 lmargin = 2cm,rmargin = 2cm,tmargin = 2.5cm,bmargin = 2.5cm}

\input{../../macros.tex}

\pagestyle{fancy} %
\lhead{ECE2 \hfill Mathématiques\\
} %
\chead{\hrule} %
\rhead{} %
\lfoot{} %
\cfoot{} %
\rfoot{\thepage} %

\renewcommand{\headrulewidth}{0pt}% : Trace un trait de séparation
 % de largeur 0,4 point. Mettre 0pt
 % pour supprimer le trait.

\renewcommand{\footrulewidth}{0.4pt}% : Trace un trait de séparation
 % de largeur 0,4 point. Mettre 0pt
 % pour supprimer le trait.

\setlength{\headheight}{14pt}

\title{\bf \vspace{-2cm} ECRICOME 1991} %
\author{} %
\date{} %
\begin{document}

\maketitle %
\vspace{-1.4cm}\hrule %
\thispagestyle{fancy}

\vspace*{.2cm}


% DEBUT DU DOC À MODIFIER : tout virer jusqu'au début de l'exo

%Définition et changement de valeurs de
compteurs%newcounter{cpt1}{section} compteur cpt1 remis à 0 à chaque
aumentation par stepcounter du compteur section%setcounter{cpt1}{3} on
met le compteur à 3%addtocounter{cpt1}{5} on ajoute 5 au compteur%
stepcounter{cpt1} on ajoute 1% ifthenelse{test}{alors}{sinon} (page
206) pour subordonner à une condition % whiledo{test}{commande} pour
faire une boucle (page 206 aussi) % value{cpt1} pour noter dans le
document la valeur de cpt1 
%Définition définitive d'opérateurs
mathématiques\newcommand{\ch}{\operatorname{ch}} 
\newcommand{\sh}{\operatorname{sh}}
\renewcommand{\tanh}{\operatorname{th}}
\renewcommand{\sinh}{\operatorname{sh}}
\renewcommand{\cosh}{\operatorname{ch}}
\newcommand{\argsh}{\operatorname{argsh}}
\newcommand{\argch}{\operatorname{argch}}
\newcommand{\argth}{\operatorname{argth}}
\newcommand{\Id}{\operatorname{Id}}
\renewcommand{\leq}{\leq}
\renewcommand{\geq}{\geq }

\newcommand{\dlim}{\lim}
\newcommand{\dsum}{\sum}
\newcommand{\dint}{\int}
\newcommand{\dprod}{\prod}



%Définition de nouvelles couleurs : rgb(trois paramètres red green blue
entre 0 et 1); cmyk (quatre cyan magenta yellow black) entre 0 et 1;
gray (entre 0 et 1) et black, white, red, green, blue, cyan, magenta,
yellow% definecolor{0gris}{gray}{0.8} 
% Nouvelle commande pour encadrer le titre car shabox ne veut que d'une
seule ligne; ATTENTION A LA TAILLE; petite différence avec shadowbox ou
doublebox, voire fcolorbox ou colorbox (au lieu de shabox; laisser le
parbox tranquille sauf pour la taille de la boîte
\newcommand{\Tbox}[1]{\begin{center} \shabox{\parbox{0.6
\linewidth}{#1}} \end{center}} %[1] pour 1 paramètre ; #1 pour ce que
fait le 1er paramètre; entre accolades ce que fait la commande
%Mise en page en mode fancy : en-têtes et pieds de pages puis
définition des en-têtes et pieds de pages\pagestyle{fancy}
\lhead{ECE 2 - Mathématiques \\
Quentin Dunstetter - ENC-Bessières 2011$\backslash$2012}
\chead{}
\rhead{Ecricome 1991}
\rfoot[ \ \thepage]{\thepage}
\cfoot{}
\lfoot{}
\thispagestyle{fancy} %Mise en page de la 1ère page en mode fancy
%Trait en bas et en haut de la page (entre en-tête et texte et texte et
pied de page)\renewcommand{\footrulewidth}{0.4pt}
\renewcommand{\headrulewidth}{0.4pt}


\begin{center}
{\Huge ECRICOME Eco 1991}
\end{center}

\section*{\textbf{Exercice 1}}

L'espace vectoriel $E = \R^{3}$ est rapporté à sa base canoniqur
$\mathcal{B} = (i,j,k)$ $i = (1,0,0)$, $j = (0,1,0)$ et $k = (0,0,1)$.
\\
On appelle $f$ l'endomorphisme de $E$ dont la matrice relativement à
$\mathcal{B}$ est la matrice $A$ suivante : 
\[
A = \left( 
\begin{array}{rrr}
-2 & -1 & 2 \\
-15 & -6 & 11 \\
-14 & -6 & 11
\end{array}
\right) 
\]

\begin{noliste}{1.}
 \setlength{\itemsep}{4mm}
\item Déterminer l'image du vecteur $u = i + j + 2k$ par l'application
$f$.\\
Que peut-on dire du vecteur $u$ pour l'application $f$ ?

\item On pose $v = 3j + 2k$ et $\mathcal{B}{\prime } = (u,v,k)$

\begin{noliste}{a)}
 \setlength{\itemsep}{2mm}
\item Démontrer que $\mathcal{B^{\prime }}$ est une base de $E$

\item Soit $P$ la matrice de passage de la base $\mathcal{B}$ à la base
$\mathcal{B}{\prime }$.\\
Calculer la matrice inverse de $P$.

\item Déterminer la matrice $T$ de $f$ relativement à la base
$\mathcal{B}{\prime }$.
\end{noliste}

\item On considère la matrice $N$ suivante :
\[
N = \left( 
\begin{array}{rrr}
0 & 1 & 2 \\
0 & 0 & 3 \\
0 & 0 & 0
\end{array}
\right)
\]

\begin{noliste}{a)}
 \setlength{\itemsep}{2mm}
\item Calculer $N^{2}$. En déduire $N^{k}$ si $k$ est un entier
supérieur ou 
égal à $2$.

\item Calculer $I + N$. Déterminer $T^{n}$ en fonction de $n$ et de
$N$, puis
de $n$ uniquement.

\item Montrer que $P.N.P^{-1} = A-I$ et que $P.N^{2}.P^{-1} = A^{2}-2A
+ I$

\item Donner l'expression de $A^{n}$ en fonction de $n$, $I$, $A$ et
$A^{2}$.
\end{noliste}
\end{noliste}

\section*{\textbf{Exercice 2}}

On considère la suite $(u_{n})$ dont le terme général est défini par la
relation : 
\[
\forall n\in \N^{\ast }\qquad u_{n} = \dint{0}{1}\sqrt[3]{1-x^{n}}dx =
\dint{0}{1}(1-x^{n})^{1/3}dx
\]

\begin{noliste}{1.}
 \setlength{\itemsep}{4mm}
\item Calculer $u_{1}$.

\item Démontrer que la suite $(u_{n})$ est croissante et convergente.

\item Démontrer que, pour tout réel $x$ tel que $0\leq x\leq 1$,
on a : 
\[
1-x\leq \sqrt[3]{1-x}\leq 1-\dfrac{x}{3}
\]
Interpréter graphiquement la double inégalité précédente.

\item En déduire un encadrement de $u_{n}$ et la limite de $u_{n}$
quand $n$
tend vers $ + \infty $.

\item On considère la fonction $f$ définie par : 
\[
\left\{ 
\begin{array}{ll}
f(x) = \dfrac{4}{3}\sqrt[3]{1-x} & \text{si }0\leq x\leq 1 \\
f(x) = 0 & \text{si }x<0\text{ ou }x>1
\end{array}
\right.
\]

\begin{noliste}{a)}
 \setlength{\itemsep}{2mm}
\item Montrer que $f$ est la densité d'une variable aléatoire $Y$.

\item Déterminer la fonction de répartition $F$ de la variable $Y$.\\
Construire sa représentation graphique dans un plan rapporté à un
repère
orthonrmé, d'unité 3cm sur chaque axe.

\item Calculer l'espérance de la variable $Y$

\item Calculer la probabilité de l'évènement : $(0,488<Y\leq 1,2)$
\end{noliste}

On rapelle que : $2^{9} = 512$.
\end{noliste}

\section*{\textbf{Problème}}

Le problème est constitué de deux parties largement indépendantes.\\
La première partie propose une estimation du paramètre de la loi de
Poisson
suivie par une variable aléatoire; la seconde partie étudie
l'intervalle de
temps qui sépare deux réalisations sucessives d'un phénomène aléatoire.

\subsection*{\textbf{Partie I}}

La société SA POIN est spécialisée dans la location de camions. La
direction
pense que le nombre de véhicules loués un jour $j$ donné est une
variable aléatoire $X_{j}$, que les variables $X_{j}$ sont mutuellement
épendantes et
suivent toutes la même loi de Poisson de paramètre $a$ inconnu.

\begin{noliste}{1.}
 \setlength{\itemsep}{4mm}
\item On considère, pour tout entier $k\in \N^{\ast }$ la fonction
$f_{k}$ définie pour tout $x$ réel positif ou nul par : \qquad
$f_{k}(x) = x^{k}\ e^{-x}$\\
On appelle $(C_{k})$ la courbe représentative de $f_{k}$ dans un plan
rapporté à un repère orthogonal (O;i,j), l'unité étant 2cm sur l'axe
des abscisses,
10cm sur l'axe des ordonnées.

\begin{noliste}{a)}
 \setlength{\itemsep}{2mm}
\item Construire le tableau des variations de $f_{k}$.

\item Préciser l'équation de la demi-tangente à la courbe $(C_{k})$ au
point 
$O(0,0)$.

\item Tracer les courbes $(C_{1})$ et $(C_{2})$ sur le même graphique.
\end{noliste}

\item On considère, pour tous les entiers $k$ et $q$ non nuls, la
fonction $g $ définie pour tout $x$ réel positif ou nul par : \quad
$g(x) = \left(
f_{k}(x)\right) ^{q}$.

\begin{noliste}{a)}
 \setlength{\itemsep}{2mm}
\item Démontrer que $g$ et $f_{k}$ ont même sens de variation.

\item Donner le tableau de variation de $g$ lorsque $k = 8$ et $q = 4$.
\end{noliste}

\item La société SA POIN relève le nombre de véhicules loués au cours
de 4
journées choisies au hasard, soit : 9, 11, 4, 8 pour les jours $j = 1$,
$j = 2$, $j = 3$, $j = 4$ respectivement.

\begin{noliste}{a)}
 \setlength{\itemsep}{2mm}
\item Déterminer la probabilité P$_{a}$ de l'évènement : 
\[
(X_{1} = 9)\cap (X_{2} = 11)\cap (X_{3} = 4)\cap (X_{4} = 8)
\]
Exprimer P$_{a}$ en fonction de $g_{a}$ la fonction définie en 2) b :.

\item La direction de la société SA POIN décide de retenir la valeur
$a$ qui
donne à la probabilité P$_{a}$ une valeur maximum.

\begin{noliste}{$\sbullet$}
\item Trouver le réel $a$

\item Calculer alors la probabilité qu'un jour donné, la société SA
POIN
loue 10 camions.
\end{noliste}
\end{noliste}

\item On se propose de généraliser le procédé. On a noté le nombre de
locations au cours de $n$ journées cchoisies au hasard, soit :\\
$x_{1}$, $x_{2}$, $\ldots $, $x_{n}$ pour les jours $j = 1$, $j = 2$,
$\ldots $, 
$j = n$ respectivement.\\
On note $m$ la moyenne arithmétique de ces observations.\\
L'entreprise décide toujours de choisir $a$ de telle sorte que la
probabilité
P$_{a}$ de l'évènement : 
\[
E = (X_{1} = x_{1})\cap (X_{2} = x_{2})\cap \ldots \cap (X_{n} = x_{n})
\]
soit maximum.

\begin{noliste}{a)}
 \setlength{\itemsep}{2mm}
\item Calculer $P_{a}$ en fonction de $a$.

\item Démontrer que la valeur de $a$ qui maximise $P_{a}$ est $m$.
\end{noliste}
\end{noliste}

\subsection*{\textbf{Partie II}}

La socité SA POIN étudie, pour un jour donné, le temps $T$ exprimé en
heures, entre deux locations successives : $T$ est une variable
aléatoire qui
suit une loi exponentielle de paramètre 1/3.

\begin{noliste}{1.}
 \setlength{\itemsep}{4mm}
\item 
\begin{noliste}{a)}
 \setlength{\itemsep}{2mm}
\item Rappeler l'espérance et l'écart type de la variable aléatoire
$T$.

\item Redémontrer que la fonction de répartition $F$ de la variable
aléatoire $T$ est définie par : 
\[
F(x) = P\left(\Ev{T\leq x}\right) = \left\{ 
\begin{array}{ll}
0 & \text{si }x\leq 0 \\
1-e^{-x/3} & \text{si }x>0
\end{array}
\right.
\]

\item En déduire la probabilité (exprimée en fonction du nombre $e$)
que le
temps séparant deux locations successives soit, pour ce jour observé,
compris entre 1 heure et 2 heures.
\end{noliste}

\item Soit $t$ un réel positif fixé pour toute la suite du problème.
Soit la
variable aléatoire $Y_{t}$ ayant pour fonction de répartition la
fonction $G_{t}$ définie par : 
\[
G_{t}(y) = P\left(\Ev{Y_{t}\leq y}\right) = \left\{ 
\begin{array}{ll}
1 & \text{si }y\geq t \\
\dfrac{F(y)}{F(t)} & \text{si }y<t
\end{array}
\right.
\]
En déduire une densité de probabilité $g_{t}$ de la variable $Y_{t}$.

\item Soit $Z_{1},\ Z_{2}\,\ldots Z_{n}$ $n$ variables aléatoires
indépendantes de même loi que $Y_{t}$.\\
On pose $S_{n} = \sup (Z_{1},Z_{2},\ldots,Z_{n})$.

\begin{noliste}{a)}
 \setlength{\itemsep}{2mm}
\item Démontrer que la fonction de répartition $H_{t}$ de $S_{n}$ est
définie par : 
\[
H_{t}(s) = \left( G_{t}(s)\right) ^{n}
\]

\item En déduire la densité de probabilité de la variable $S_{n}$ notée
$h_{t}$.
\end{noliste}

\item Soit $b$ un réel positif.

\begin{noliste}{a)}
 \setlength{\itemsep}{2mm}
\item On pose $p_{n}(b) = P\left(\Ev{|S_{n}-t|>b}\right)$. Montrer que
$p_{n}(b) = P\left(\Ev{S_{n}<t-b}\right).$

\item Calculer $p_{n}(b)$.

\item Déterminer la limite de $p_{n}(b)$ lorsque $n$ tend vers $ +
\infty $.
\end{noliste}

\item On définit la variable aléatoire $U_{n} = n(t-S_{n})$. Pour tout
réel $u$, on pose 
\[
K_{n}(u) = P\left(\Ev{U_{n}\leq u}\right)
\]

\begin{noliste}{a)}
 \setlength{\itemsep}{2mm}
\item Exprimer $K_{n}(u)$ en fonction de $u$.

\item Écrire le développement limité à l'ordre $1$, par rapport à
$\dfrac{u}{n}$ quand $n$ tend vers $ + \infty $ de $\exp \left(
-\dfrac{1}{3}(t-\dfrac{u}{n})\right) $

\item On pose $c = \dfrac{\exp (-t/3)}{1-\exp (-t/3)}$ et on considère
un réel 
$u$ fixé.\\
En déduire la limite de $K_{n}(u)$ lorsque $n$ tend vers $ + \infty $
en
fonction de $u$ et $c$.

\item On note $K(u)$ cette limite. Reconnaitre la loi de la variable
aléatoire $U$ dont la fonction de répartition est la fonction $K$ ainsi
définie.
\end{noliste}
\end{noliste}

\label{fin}

\end{document}

