\documentclass[11pt]{article}%
\usepackage{geometry}%
\geometry{a4paper,
 lmargin = 2cm,rmargin = 2cm,tmargin = 2.5cm,bmargin = 2.5cm}

\input{../../macros.tex}

\pagestyle{fancy} %
\lhead{ECE2 \hfill Mathématiques\\
} %
\chead{\hrule} %
\rhead{} %
\lfoot{} %
\cfoot{} %
\rfoot{\thepage} %

\renewcommand{\headrulewidth}{0pt}% : Trace un trait de séparation
 % de largeur 0,4 point. Mettre 0pt
 % pour supprimer le trait.

\renewcommand{\footrulewidth}{0.4pt}% : Trace un trait de séparation
 % de largeur 0,4 point. Mettre 0pt
 % pour supprimer le trait.

\setlength{\headheight}{14pt}

\title{\bf \vspace{-2cm} ECRICOME 1990} %
\author{} %
\date{} %
\begin{document}

\maketitle %
\vspace{-1.4cm}\hrule %
\thispagestyle{fancy}

\vspace*{.2cm}


% DEBUT DU DOC À MODIFIER : tout virer jusqu'au début de l'exo

%Définition et changement de valeurs de
compteurs%newcounter{cpt1}{section} compteur cpt1 remis à 0 à chaque
aumentation par stepcounter du compteur section%setcounter{cpt1}{3} on
met le compteur à 3%addtocounter{cpt1}{5} on ajoute 5 au compteur%
stepcounter{cpt1} on ajoute 1% ifthenelse{test}{alors}{sinon} (page
206) pour subordonner à une condition % whiledo{test}{commande} pour
faire une boucle (page 206 aussi) % value{cpt1} pour noter dans le
document la valeur de cpt1 
%Définition définitive d'opérateurs
mathématiques\newcommand{\ch}{\operatorname{ch}} 
\newcommand{\sh}{\operatorname{sh}}
\renewcommand{\tanh}{\operatorname{th}}
\renewcommand{\sinh}{\operatorname{sh}}
\renewcommand{\cosh}{\operatorname{ch}}
\newcommand{\argsh}{\operatorname{argsh}}
\newcommand{\argch}{\operatorname{argch}}
\newcommand{\argth}{\operatorname{argth}}
\newcommand{\Id}{\operatorname{Id}}
\renewcommand{\leq}{\leq}
\renewcommand{\geq}{\geq }

\newcommand{\dlim}{\lim}
\newcommand{\dsum}{\sum}
\newcommand{\dint}{\int}
\newcommand{\dprod}{\prod}



%Définition de nouvelles couleurs : rgb(trois paramètres red green blue
entre 0 et 1); cmyk (quatre cyan magenta yellow black) entre 0 et 1;
gray (entre 0 et 1) et black, white, red, green, blue, cyan, magenta,
yellow% definecolor{0gris}{gray}{0.8} 
% Nouvelle commande pour encadrer le titre car shabox ne veut que d'une
seule ligne; ATTENTION A LA TAILLE; petite différence avec shadowbox ou
doublebox, voire fcolorbox ou colorbox (au lieu de shabox; laisser le
parbox tranquille sauf pour la taille de la boîte
\newcommand{\Tbox}[1]{\begin{center} \shabox{\parbox{0.6
\linewidth}{#1}} \end{center}} %[1] pour 1 paramètre ; #1 pour ce que
fait le 1er paramètre; entre accolades ce que fait la commande
%Mise en page en mode fancy : en-têtes et pieds de pages puis
définition des en-têtes et pieds de pages\pagestyle{fancy}
\lhead{ECE 2 - Mathématiques \\
Quentin Dunstetter - ENC-Bessières 2011$\backslash$2012}
\chead{}
\rhead{Ecricome 1990}
\rfoot[ \ \thepage]{\thepage}
\cfoot{}
\lfoot{}
\thispagestyle{fancy} %Mise en page de la 1ère page en mode fancy
%Trait en bas et en haut de la page (entre en-tête et texte et texte et
pied de page)\renewcommand{\footrulewidth}{0.4pt}
\renewcommand{\headrulewidth}{0.4pt}


\begin{center}
{\Huge ECRICOME Eco 1990}
\end{center}

\begin{center}
\textbf{EXERCICE 1}
\end{center}

On considère une base $B = (e_{1},e_{2},e_{3})$ de l'espace vectoriel
$\R^{3}$ sur le corps des r"els et l'endomorphisme $f$ de $\R^{3}$
d"fini par :
\[
f(e_{1}) = ae_{1}\quad ;\quad f(e_{2}) = ae_{2} + be_{1}\quad ;\quad
f(e_{3}) = ae_{3} + be_{2} + ce_{3},
\]
où $a,b,c$ sont trois réels fixés.

\begin{center}
On pose $I = 
\begin{smatrix}
1 & 0 & 0 \\
0 & 1 & 0 \\
0 & 0 & 1
\end{smatrix}
\qquad J = 
\begin{smatrix}
0 & 1 & 0 \\
0 & 0 & 1 \\
0 & 0 & 0
\end{smatrix}
\qquad K = 
\begin{smatrix}
0 & 0 & 1 \\
0 & 0 & 0 \\
0 & 0 & 0
\end{smatrix}
$
\end{center}

\begin{noliste}{1.}
 \setlength{\itemsep}{4mm}
\item 

\begin{noliste}{a)}
 \setlength{\itemsep}{2mm}
\item Écrire la matrice $M(a,b,c)$ de $f$ relativement à la base $B.$

\item On appelle $F$ l'ensemble de toutes les matrices $M(a,b,c)$ où
$a,b,c$
sont des réels quelconques.\\
Démontrer que $F$ est un espace vectoriel sur $\R$ et que tout élement
de $F$ se décompose de manière unique comme combinaison linéaire de
$I,J,K.$
\end{noliste}

\item 

\begin{noliste}{a)}
 \setlength{\itemsep}{2mm}
\item Calculer les matrices $J^{2},K^{2},JK,KJ.$\\
En déduire que $F$ est stable pour la multiplication des matrices.

\item Déterminer les éléments de $F$ qui sont inversibles et donner
l'inverse d'une telle matrice.
\end{noliste}

\item On note $M$ pour $M(a,b,c).$

\begin{noliste}{a)}
 \setlength{\itemsep}{2mm}
\item Démontrer qu'il existe trois suites réelles
$(x_{n}),(y_{n}),(z_{n})$
telles que 
\[
M^{n} = x_{n}I + y_{n}J + z_{n}K\quad \text{et}\quad \left\{ 
\begin{array}{c}
x_{n + 1} = ax_{n} \\
y_{n + 1} = bx_{n} + ay_{n} \\
z_{n + 1} = cx_{n} + by_{n} + az_{n}
\end{array}
\right. 
\]

\item Déterminer $x_{n}$ en fonction de $n.$ On suppose désormais
$a\neq 0.$

\item On pose $u_{n} = \dfrac{y_{n}}{x_{n}}.$ Déterminer $u_{n},$ puis
$y_{n}$
en fonction de $n.$

\item En déduire la valeur de $z_{n}.$
\end{noliste}
\end{noliste}

\begin{center}
\textbf{EXERCICE 2}
\end{center}

\textbf{A- }On considère la fonction $g$ définie pour tout réel $x$ par
:
\[
g(x) = x^{2}e^{-\dfrac{(x-1)^{2}}{2}}
\]

\begin{noliste}{1.}
 \setlength{\itemsep}{4mm}
\item Étudier la fonction $g.$

\item Construire la courbe représentative $(C)$ dans un repère
$(O,\overrightarrow{i},\overrightarrow{j})$ orthogonal.\\
(On prendra comme unité graphique $2$ cm en abscisses et $4$ cm en
ordonnées.

\item On pose $I = \dint{1}{+ \infty
}x^{2}e^{-\dfrac{(x-1)^{2}}{2}}dx.$

\begin{noliste}{a)}
 \setlength{\itemsep}{2mm}
\item Démontrer que $I$ est une intégrale convergente.

\item Démontrer que $I = \dint{0}{+ \infty }(t +
1)^{2}e^{-\dfrac{t^{2}}{2}}dt.$

\item Démontrer que $\dint{0}{+ \infty }t^{2}e^{-\dfrac{t^{2}}{2}}dt =
\dfrac{\sqrt{2\pi }}{2}$.

\item En déduire que $I = 2 + \sqrt{2\pi }.$
\end{noliste}
\end{noliste}

\textbf{B-} On considère la fonction $f$ définie par :
\[
\left\{ 
\begin{array}{ccc}
f(x) = & A^{2}e^{-\dfrac{(x-1)^{2}}{2}} & \text{si }x>1 \\
f(x) = & 0 & \text{si }x\leq 1
\end{array}
\right. 
\]

\begin{noliste}{1.}
 \setlength{\itemsep}{4mm}
\item Déterminer la valeur de $A$ pour que la fonction $f$ soit la
densité
de probabilité d'une variable aléatoire $X.$

\item Déterminer alors la probabilité $P\left(\Ev{X\leq 2}\right).$ En
donner une
valeur approchée à $10^{-3}$ près.\\
N.B. Des tables relatives à la loi normale $\mathcal{N}(0,1)$ sont
fournies
en annexe au sujet.
\end{noliste}

\begin{center}
\textbf{PROBLEME}
\end{center}

\noindent La société MARSCHA réalise auprès de sa clientèle une étude
statistique relative au lancement d'un nouveau produit référencé $A.$\\
L'observation des ventes permet d'admettre que chaque client achètera
le
produit $A,$ avec la probabilité $0,5$ et que les achats des clients
sont
mutuellement indépendants.\\
On considère un échantillon de $n$ clients susceptibles d'acheter le
produit 
$A,$ et on appelle $X_{n}$ le nombre de clients, parmi les $n$
interrogés,
qui achèterons effectivement le produit $A.$\\
On pose $F_{n} = \dfrac{X_{n}}{n}.$

\begin{center}
\textbf{Première partie - A}
\end{center}

\begin{noliste}{1.}
 \setlength{\itemsep}{4mm}
\item 

\begin{noliste}{a)}
 \setlength{\itemsep}{2mm}
\item Préciser la loi de $X_{n}.$ Quel est l'ensemble des valeurs
prises par 
$F_{n}$ ?

\item Calculer $P\left(\Ev{F_{n} = 1-\dfrac{1}{n}}\right).$

\item Calculer $\E(X_{n})$ et $\V(X_{n}).$
\end{noliste}

\item Trouver à l'aide de l'inégalité de Bienaymé-Tchebichev, le plus
petit
entier $n$ tel que :
\[
P\left[ \ \left| F_{n}-0,5\right| \geq 0,20\right] <0,05.
\]
$$
\end{noliste}

\begin{center}
\textbf{Deuxième partie - B}
\end{center}

L'objet de cette partie du problème est la recherche d'une autre
majoration
de 
\[
P\left[ \ \left| F_{n}-0,5\right| \geq a\right],\text{ où }0,5>a>0.
\]

\begin{noliste}{1.}
 \setlength{\itemsep}{4mm}
\item On considère un entier naturel non nul $m,$ un réel $t$ tel que
$0,5\leq t\leq 1$ et la fonction $f$ définie, pour $x$ positif ou
nul, par :
\[
f(x) = \left( e^{-tx} + e^{(1-t)x}.\right) ^{m}.
\]

\begin{noliste}{a)}
 \setlength{\itemsep}{2mm}
\item Placer $\dfrac{t}{1-t}$ par rapport à $1.$

\item Étudier les variations de $f$ sur $\R_{+}$ et construire son
tableau de variation. On appelera $b$ le minimum de $f$ sur $\R_{+}.$

\item Montrer que l'on a $f(x) = e^{-mtx}(1 + e^{x})^{m}.$

\item En déduire que $b = \left( \dfrac{1}{t^{t}(1-t)^{1-t}}\right)
^{m}.$
\end{noliste}

\item Soient $m\in \N^{\times },$ $t\in \lbrack 0,5;1[.$ On note $r$
le plus petit entier strictement supérieur à $mt.$\\
On pose $S = \Sum{k = r}{m}C_{m}{k}.$

\begin{noliste}{a)}
 \setlength{\itemsep}{2mm}
\item Montrer que si $k$ est un entier supérieur ou égal à $r$ et si
$x$ est
un réel positif ou nul, on a :
\[
e^{x(k-mt)}\geq 1.
\]

\item En déduire que, si $x$ est positif ou nul, on a $ :S<\Sum{k =
0}{m}C_{m}{k}e^{x(k-mt)}.$

\item Démontrer alors que, pour tout réel $x$ positif ou nul, on a
$S<e^{-mtx}(1 + e^{x})^{m}.$

\item On pose, pour tout $t$ tel que $0,5\leq t<1,\quad g(t) = t\ln
t + (1-t)\ln (1-t).$\\
Montrer que $S<e^{-mg(t)}.$\\
On pourra utiliser le \textbf{B}-1.(d)
\end{noliste}

\item Soit $a$ un réel positif, inférieur à $0,5$ et $n$ un entier non
nul.\\
On note $u$ le plus petit entier supérieur à $(a + 0,5)n$ et $v$ le
plus grand
entier inférieur à $(0,5-a)n.$

\begin{noliste}{a)}
 \setlength{\itemsep}{2mm}
\item Montrer que $\Sum{i = u}{n}C_{n}{i} = \Sum{j = 0}{v}C_{n}{j}.$

\item Montrer que $\left[ \ \left| F_{n}-0,5\right| >a\right] = \left[
X_{n}>n(a + 0,5)\right] \cup \left[ X_{n}<(0,5-a)n\right].$

\item En déduire que $P\left[ \ \left| F_{n}-0,5\right| >a\right]
 = (0,5)^{n-1}\Sum{i = u}{n}C_{n}{i}.$

\item Conclure que $P\left[ \ \left| F_{n}-0,5\right| >a\right] <e(n)$
avec $e(n) = (0,5)^{n-1}e^{-ng(a + 0,5)},$ $g$ étant définie au
\textbf{B}-2.(d)
\end{noliste}

\item On pose $\left\{ 
\begin{array}{cc}
w(t) = g(t) & \text{si }t\in \lbrack 0,5;1[ \ \\
w(t) = 0 & \text{si }t = 1
\end{array}
\right. $\\
Étudier la fonction $w$ sur $[0,5;1].$ Préciser la tangente au point
$A(1,0)$
à la courbe $C(w)$ représentative de $w.$ Tracer la courbe $C(w).$

\item 

\begin{noliste}{a)}
 \setlength{\itemsep}{2mm}
\item Déduire du 3.(d) et du 4. que la limite de $P\left[ \ \left|
F_{n}-0,5\right| >a\right] $ est nulle lorsque $n$ tend vers $ +
\infty.$

\item Exprimer en fonction de $a,$ la valeur du plus petit entier $n$
pour
que $e(n)\leq 0,05.$ On appellera $n_{1}$ cet entier naturel.\\
Calculer $n_{1}$ si $a = 0,20.$
\end{noliste}

\item Il semble donc que les conditions d'approximation de la loi de
$X_{n}$
par une loi normale soient remplies.

\begin{noliste}{a)}
 \setlength{\itemsep}{2mm}
\item Préciser les paramètres de la loi normale approchant celle suivie
par $X_{n}.$

\item En utilisant l'approximation précédente, déterminer le plus petit
entier $n_{2}$ tel que $P\left[ \ \left| F_{n}-0,5\right| >0,2\right]
\leq 0,05.$
\end{noliste}
\end{noliste}

\label{fin}

\end{document}

