\documentclass[11pt]{article}%
\usepackage{geometry}%
\geometry{a4paper,
 lmargin = 2cm,rmargin = 2cm,tmargin = 2.5cm,bmargin = 2.5cm}

\input{../../macros.tex}

\pagestyle{fancy} %
\lhead{ECE2 \hfill Mathématiques\\
} %
\chead{\hrule} %
\rhead{} %
\lfoot{} %
\cfoot{} %
\rfoot{\thepage} %

\renewcommand{\headrulewidth}{0pt}% : Trace un trait de séparation
 % de largeur 0,4 point. Mettre 0pt
 % pour supprimer le trait.

\renewcommand{\footrulewidth}{0.4pt}% : Trace un trait de séparation
 % de largeur 0,4 point. Mettre 0pt
 % pour supprimer le trait.

\setlength{\headheight}{14pt}

\title{\bf \vspace{-2cm} ECRICOME 2004} %
\author{} %
\date{} %
\begin{document}

\maketitle %
\vspace{-1.4cm}\hrule %
\thispagestyle{fancy}

\vspace*{.2cm}


% DEBUT DU DOC À MODIFIER : tout virer jusqu'au début de l'exo

%Définition et changement de valeurs de
compteurs%newcounter{cpt1}{section} compteur cpt1 remis à 0 à chaque
aumentation par stepcounter du compteur section%setcounter{cpt1}{3} on
met le compteur à 3%addtocounter{cpt1}{5} on ajoute 5 au compteur%
stepcounter{cpt1} on ajoute 1% ifthenelse{test}{alors}{sinon} (page
206) pour subordonner à une condition % whiledo{test}{commande} pour
faire une boucle (page 206 aussi) % value{cpt1} pour noter dans le
document la valeur de cpt1 
%Définition définitive d'opérateurs
mathématiques\newcommand{\ch}{\operatorname{ch}} 
\newcommand{\sh}{\operatorname{sh}}
\renewcommand{\tanh}{\operatorname{th}}
\renewcommand{\sinh}{\operatorname{sh}}
\renewcommand{\cosh}{\operatorname{ch}}
\newcommand{\argsh}{\operatorname{argsh}}
\newcommand{\argch}{\operatorname{argch}}
\newcommand{\argth}{\operatorname{argth}}
\newcommand{\Id}{\operatorname{Id}}
\renewcommand{\leq}{\leq}
\renewcommand{\geq}{\geq }

%Définition de nouvelles couleurs : rgb(trois paramètres red green blue
entre 0 et 1); cmyk (quatre cyan magenta yellow black) entre 0 et 1;
gray (entre 0 et 1) et black, white, red, green, blue, cyan, magenta,
yellow% definecolor{0gris}{gray}{0.8} 
% Nouvelle commande pour encadrer le titre car shabox ne veut que d'une
seule ligne; ATTENTION A LA TAILLE; petite différence avec shadowbox ou
doublebox, voire fcolorbox ou colorbox (au lieu de shabox; laisser le
parbox tranquille sauf pour la taille de la boîte
\newcommand{\Tbox}[1]{\begin{center} \shabox{\parbox{0.6
\linewidth}{#1}} \end{center}} %[1] pour 1 paramètre ; #1 pour ce que
fait le 1er paramètre; entre accolades ce que fait la commande
%Mise en page en mode fancy : en-têtes et pieds de pages puis
définition des en-têtes et pieds de pages\pagestyle{fancy}
\lhead{ECE 2 - Mathématiques \\
Quentin Dunstetter - ENC-Bessières 2011$\backslash$2012}
\chead{}
\rhead{Epreuve Ecricome 2004}
\rfoot[ \ \thepage]{\thepage}
\cfoot{}
\lfoot{}
\thispagestyle{fancy} %Mise en page de la 1ère page en mode fancy
%Trait en bas et en haut de la page (entre en-tête et texte et texte et
pied de page)\renewcommand{\footrulewidth}{0.4pt}
\renewcommand{\headrulewidth}{0.4pt}


%DEBUT DU DOCUMENT

\noindent {\Large \textbf{ECRI%TCIMACRO{\TeXButton{TeX
field}{\colorbox[gray]{0.95}{COME}}}%BeginExpansion
\colorbox[gray]{0.95}{COME}%EndExpansion
}}\vspace{0.3cm}

\noindent \textbf{Banque d'épreuves communes}

\noindent aux concours des Ecoles

\noindent esc%TCIMACRO{\TeXButton{TeX
field}{\colorbox[gray]{0.95}{bordeaux}} }%BeginExpansion
\colorbox[gray]{0.95}{bordeaux}
%EndExpansion
/ esc%TCIMACRO{\TeXButton{TeX field}{\colorbox[gray]{0.95}{marseille}}
}%BeginExpansion
\colorbox[gray]{0.95}{marseille}
%EndExpansion
/ icn%TCIMACRO{\TeXButton{TeX field}{\colorbox[gray]{0.95}{nancy}}
}%BeginExpansion
\colorbox[gray]{0.95}{nancy}
%EndExpansion
/ esc%TCIMACRO{\TeXButton{TeX field}{\colorbox[gray]{0.95}{reims}}
}%BeginExpansion
\colorbox[gray]{0.95}{reims}
%EndExpansion
/ esc%TCIMACRO{\TeXButton{TeX field}{\colorbox[gray]{0.95}{rouen}}
}%BeginExpansion
\colorbox[gray]{0.95}{rouen}
%EndExpansion
/ esc%TCIMACRO{\TeXButton{TeX
field}{\colorbox[gray]{0.95}{toulouse}}}%BeginExpansion
\colorbox[gray]{0.95}{toulouse}%EndExpansion
\vspace{1cm}

\begin{center}
{\large CONCOURS D'ADMISSION }\vspace{0.5cm}

\textbf{option économique} \vspace{0.5cm}

{\Large \textbf{MATHÉMATIQUES}} \vspace{0.5cm}

\textbf{Année 2004}
\end{center}

\noindent \textbf{Aucun instrument de calcul n'est autorisé.}

\noindent \textbf{Aucun document n'est autorisé.}

\noindent L'énoncé comporte \pageref{fin} pages

\begin{quotation}
\noindent Les candidats sont invités à soigner la présentation de leur
copie, à mettre en évidence les principaux résultats, à respecter les
notations de l'énoncé, et à donner des démonstrations complètes (mais
brèves) de leurs affirmations.
\end{quotation}

\vspace{13cm}

\hfill \textbf{Tournez la page}

\hfill \textbf{S.V.P\qquad }

\newpage

\section*{Exercice 1}

Soient $f$ la fonction numérique de la variable réelle définie par : 
\[
\forall x\in \R,\mathbb{\quad }f\left( x\right) = \frac{1}{\sqrt{1 +
x^{2}}}
\]
et $\left( u_{n}\right) $ la suite de nombres réels déterminée par :
\[
\left\{ 
\begin{array}{cc}
 & u_{0} = \dint{0}{1}f\left( x\right\dx \\
\forall n\in \N^{\times }, & u_{n} = \dint{0}{1}x^{n}f\left(
x\right\dx
\end{array}
\right. 
\]
On note $\mathcal{C}_{f}$ la représentation graphique de $f,$
relativement à
un repère orthonormal $\left( O,\vec{i},\vec{j}\right).$

\subsection*{I. Étude de $f.$}

\begin{noliste}{1.}
 \setlength{\itemsep}{4mm}
\item Montrer que la fonction $f$ est paire sur $\R$.

\item Étudier les variations de $f$ sur l'intervalle $\left[ 0, +
\infty \right[ $.

\item Déterminer la lmite de $f$ lorsque $x$ tend vers $ + \infty.$

\item Montrer que $f$ est bornée sur $\R$.

\item Donner l'allure de $\mathcal{C}_{f}$.

\item Montrer que $f$ réaise une bijection de l'intervalle $\left[
0, + \infty \right[ $ sur un intervalle $J$ à préciser.

\item Pour tout $y$ de l'intervalle $\left] 0,1\right],$ déterminer
l'unqiue réel $x$ appartenant à l'intervalle $\left[ 0, + \infty
\right[ $ tel que : 
\[
f\left( x\right) = y
\]

\item Déterminer alors la bijection réciproque $f^{-1}$.
\end{noliste}

\subsection*{II. Calcul d'aire}

On considère la fonction numérique $F$ de la variable réelle $x$
définie par : 
\[
F\left( x\right) = \ln \left( x + \sqrt{x^{2} + 1}\right) 
\]
Pour tout réel $\lambda $ strictement positif, on note
$\mathcal{A}\left(
\lambda \right) $ l'aire (exprimée en unité d'aire) du domaine
constitué par
l'ensemble des points $M\left( x,y\right) $ tels que : 
\[
\lambda \leq x\leq 2\lambda \quad \text{et\quad }0\leq
y\leq f\left( x\right) 
\]
ainsi 
\[
\mathcal{A}\left( \lambda \right) = \dint{\lambda }{2\lambda }f\left(
x\right\dx
\]

\begin{noliste}{1.}
 \setlength{\itemsep}{4mm}
\item Montrer que : 
\[
\forall x\in \R,\quad x + \sqrt{x^{2} + 1}>0
\]

En déduire l'ensemble de définition de $F.$

\item Montrer que $F$ est une primitive de $f$ sur $\R$.

\item Montrer que $F$ est impaire sur son ensemble de définition.

\item Déterminer la limite de $F$ lorsque $x$ tend vers $ + \infty.$ En
déduire la limite de $F$ quand $x$ tend vers $-\infty $.

\item Exprimer $\mathcal{A}\left( \lambda \right) $ en fonction de
$\lambda $
et calculer la limite de $\mathcal{A}\left( \lambda \right) $ lorsque
$\lambda $ tend vers $ + \infty.$
\end{noliste}

\subsection*{III. Étude de la suite $\left( u_{n}\right).$}

\begin{noliste}{1.}
 \setlength{\itemsep}{4mm}
\item Calculer $u_{0}$ et $u_{1}.$

\item Effectuer une intégration par parties et calculer $u_{3}.$

(On pourra remarquer que $\dfrac{x^{3}}{\sqrt{1 + x^{2}}} =
x^{2}\dfrac{x}{\sqrt{1 + x^{2}}}$ )

\item Déterminer le sens de variations de la suite $\left(
u_{n}\right).$

\item Montrer que la suite $\left( u_{n}\right) $ est convergente. (On
ne
cherchera pas sa limite dans cette question)

\item Justifier l'encadrement suivant : 
\[
\forall x\in \left[ 0,1\right],\quad \forall n\in \N,\mathbb{\quad }
0\mathbb{\leq }\frac{x^{n}}{\sqrt{1 + x^{2}}}\leq x^{n}
\]
en déduire que : 
\[
\forall n\in \N^{\times },\quad 0\leq u_{n}\leq \frac{1}{n + 1}
\]

\item Déterminer alors la limite de la suite $\left( u_{n}\right) $
\end{noliste}

\section*{EXERCICE 2}

Dans cet exercice, on étudie l'exponentielle d'une matrice pour une
matrice
carrée d'ordre 3, puis d'ordre 2.

\subsection*{I. Exponentielle d'une matrice carrée d'ordre 3.}

Soient $A$ et $P$ les matrice définies par : 
\[
A = \left( 
\begin{array}{ccc}
1 & 1 & 1 \\
-1 & 1 & -1 \\
-2 & 0 & -2
\end{array}
\right),\quad P = \left( 
\begin{array}{rrr}
2 & 1 & 1 \\
-1 & 2 & -1 \\
1 & -1 & 1
\end{array}
\right) 
\]

\begin{noliste}{1.}
 \setlength{\itemsep}{4mm}
\item Montrer que la matrice $P$ est inversible et déterminer $P^{-1}$

\item On pose $T = P\,A\,P^{-1}.$

\begin{noliste}{a)}
 \setlength{\itemsep}{2mm}
\item Calculer la matrice $T$

\item Calculer $T^{2},\;T^{3},$ puis $T^{n}$ pour tout entier naturel
$n\geq 3.$
\end{noliste}

\item En déduire que : 
\[
\forall n\geq 3,\quad A^{n} = 0
\]
où $0$ désigne la matrice nulle d'ordre 3.

\item Pour tout réel $t,$ on définit la matrice $\E\left( t\right) $
par : 
\[
\E\left( t\right) = I + tA + \frac{t^{2}}{2}A^{2}
\]
où $I$ désigne la matrice unité d'ordre 3.

\begin{noliste}{a)}
 \setlength{\itemsep}{2mm}
\item Montrer que : 
\[
\forall \left( t,t^{\prime }\right) \in \R^{2},\quad E\left(
t\right) E\left( t^{\prime }\right) = E\left( t + t^{\prime }\right) 
\]

\item Pour tout $t$ réel, calculer $\E\left( t\right) E\left(
-t\right).$ En
déduire que la matrice $\E\left( t\right) $ est inversible et
déterminer son
inverse en fonction de $I,\;A,\;A^{2},\;t$.

\item Pour tout $t$ réel et pour tout entier naturel $n,$ déterminer
$\left[
\E\left( t\right) \right] ^{n}$ en fonction de $I,\;A,\;A^{2},\;t$ et
$n.$
\end{noliste}
\end{noliste}

\subsection*{II. Exponentielle d'une matrice carrée d'ordre 2.}

Soient $B$ et $D$ les matrices définies par : 
\[
B = \left( 
\begin{array}{rr}
0 & -1 \\
2 & 3
\end{array}
\right),\quad D = \left( 
\begin{array}{rr}
1 & 0 \\
0 & 2
\end{array}
\right) 
\]
Pour tout entier naturel $n$ non nul, et pour tout réel $t,$ on définit
la
matrice $E_{n}\left( t\right) $ par : 
\[
E_{n}\left( t\right) = \Sum{k = 0}{n}\frac{t^{k}}{k!}B^{k}\quad
\text{que
l'on note }E_{n}\left( t\right) = \left( 
\begin{array}{rr}
a_{n}\left( t\right) & c_{n}\left( t\right) \\
b_{n}\left( t\right) & d_{n}\left( t\right) 
\end{array}
\right) 
\]

\begin{noliste}{1.}
 \setlength{\itemsep}{4mm}
\item Montrer que $B$ est diagonalisable.

\item Déterminer une matrice $Q$ d'ordre 2, inversible telle que 
\[
Q^{-1}BQ = D
\]

\item Pour tout entier naturel $n,$ montrer que : 
\[
B^{n} = \left( 
\begin{array}{cc}
2-2^{n} & 1-2^{n} \\
2^{n + 1}-2 & 2^{n + 1}-1
\end{array}
\right) 
\]

\item Montrer que : 
\[
\forall n\in \N,\quad a_{n}\left( t\right) = \Sum{k =
0}{n}\frac{2t^{k}-\left( 2t\right) ^{k}}{k!}
\]
exprimer de même $b_{n}\left( t\right),$ $c_{n}\left( t\right),$
$d_{n}\left( t\right) $ sous le forme d'une somme.

\item Déterminer les limites de $a_{n}\left( t\right),$ $b_{n}\left(
t\right),$ $c_{n}\left( t\right),$ $d_{n}\left( t\right) $ lorsque $n$
tend vers $ + \infty.$

\item Pour tout $t$ réel, on pose alors : 
\[
\E\left( t\right) = \left( 
\begin{array}{cc}
\dlim{n\rightarrow + \infty }a_{n}\left( t\right) & 
\dlim{n\rightarrow + \infty }c_{n}\left( t\right) \\
\dlim{n\rightarrow + \infty }b_{n}\left( t\right) & 
\dlim{n\rightarrow + \infty }d_{n}\left( t\right) 
\end{array}
\right) 
\]

\begin{noliste}{a)}
 \setlength{\itemsep}{2mm}
\item Montrer que 
\[
\E\left( t\right) = \left( 
\begin{array}{cc}
2e^{t}-e^{2t} & e^{t}-e^{2t} \\
2e^{2t}-2e^{t} & 2e^{2t}-e^{t}
\end{array}
\right) 
\]

\item Déterminer les matrice $E_{1}$ et $\;E_{2},$ telles que pour tout
$t$ réel on ait : 
\[
\E\left( t\right) = e^{t}E_{1} + e^{2t}E_{2}
\]

\item Calculer $E_{1}{2},$ $E_{2}{2},$ $E_{1}E_{2},$ $E_{2}E_{1}.$

\item En déduire que pour tout $t$ réel, $\E\left( t\right) $ est
inversible
et déterminer son inverse.
\end{noliste}
\end{noliste}

\section*{Exercice 3}

Une personne envoie chaque jour un courrier électronique par
l'intermédiaire
de deux serveurs : le serveur $A$ ou le serveur $B$.\\
On constate que le serveur $A$ est choisi dans 70\% des cas et donc que
le
serveur $B$ est choisi dans 30\% des cas. (Ce qui revient à dire que la
probabilité pour que le serveur $A$ soit choisi est de $0.7$). Les
choix des
serveurs sont supposés indépendants les uns des autres.

\begin{noliste}{1.}
 \setlength{\itemsep}{4mm}
\item Dans cette question, on suppose que la probabilité d'une erreur
de
transmission avec le serveur $A$ est de $0.1$, alors que la probabilité
d'erreur de transmission avec le serveur $B$ est de $0.05$.

\begin{noliste}{a)}
 \setlength{\itemsep}{2mm}
\item Calculer la probabilité pour qu'il y ait une erreur de
transmission
lors de l'envoi d'un courrier.

\item Si le courrier a subi une erreur de transmission, quelle est la
probabilité pour que le serveur utilisé soit le serveur $A$ ?
\end{noliste}

\item Un jour donné, appelé le jour 1, on note les différents serveurs
utilisé par l'ordinateur par une suite de lettres. Par exemple, la
suite $AABBBA\dots $ signifie que les deux premiers jours l'ordinateur
a choisi le
serveur $A,$ les jours 3. 4 et 5 il a choisi le le serveur $B$, et le
jour 6
le serveur $A$. Dans cet exemple, on dit que l'on a une première série
de
longueur 2 et une deuxième série de longueur 3 (Ce qui est également le
cas
de la série $BBAAAB\dots $)

On note $L_{1}$ la variable aléatoire représentant la longueur de la
premièer série et $L_{2}$ la variable aléatoire représentant la
longueur de la deuxième série.

Ainsi, pour $k\geq 1,$ dire que $L_{1} = k$ signifie que pendant les
$k$
premiers jours, c'est le même serveur qui a été choisi et le jour
suivant
l'autre serveur.

\begin{noliste}{a)}
 \setlength{\itemsep}{2mm}
\item Jusitifier soigneusement la formule : 
\[
\forall k\geq 1\quad P\left(\Ev{ L_{1} = k}\right) = \left( 0.3\right)
^{k}\left( 0.7\right) + \left( 0.7\right) ^{k}\left( 0.3\right) 
\]

\item Vérifier par le calcul que 
\[
\Sum{k = 1}{+ \infty }p\left( L_{1} = k\right) = 1
\]

\item Déterminer l'espérance mathématique de $L_{1}.$

\item Déterminer la loi du couple aléatoire $\left(
L_{1},L_{2}\right).$

\item En déduire la loi de $L_{2}$
\end{noliste}

\item Soit $n\in \N^{\times }.$ A partir d'un jour donné, que l'on
appelera le jour $1,$ on note : $N_{n}$ la variable aléatoire
représentant
le nombrede fois où l'ordinateur choisit le serveur $A$ pendant les $n$
premiers jours, $T_{1}$ le numéro du jour où pour la première fois le
serveur $A$ est choisi et $T_{2}$ le numéro du jour où pour la deuxième
fois
le serveur $A$ est choisi.

\begin{noliste}{a)}
 \setlength{\itemsep}{2mm}
\item Déterminer la loi de $N_{n}$, son espérance mathématique et sa
variance.

\item Déterminer la loi de $T_{1}$, son espérance mathématique et sa
variance.

\item Montrer que 
\[
\forall k\geq 2,\quad P\left(\Ev{ T_{2} = k}\right) = \left( k-1\right)
\left(
0.7\right) ^{2}\left( 0.3\right) ^{k-2}
\]
\end{noliste}

\item Le temps de transmission en seconde d'un message par le serveur
$A$
est une variable aléatoire $Z$ qui suit une loi exponentielle de
paramètre 1.

Le prix en euros $W$ de cette trammission, est calculé de la façon
suivante : on multiplie la durée de transmission en seconde par $0.1$
euro, auquel on
ajoute une somme forfaitaire de $1$ euro.

\begin{noliste}{a)}
 \setlength{\itemsep}{2mm}
\item Rappeler une densité $f_{Z}$ de $Z$ ainsi que sa fonction de
répartition $F_{Z}$.

\item Quel est le temps moyen (en seconde) de la transmission d'un
message
par le serveur $A$ ?

\item Exprimer $W$ en fonction de $Z$.

\item Montrer que $W$ est une variable aléatoire à densité. En
déterminer
une densité $f_{W}$.

\item Déterminer l'espérance de la variable $W.$
\end{noliste}

\item On suppose que le temps de transmission d'un message en seconde
par le
serveur $B$ est représenté par la variable aléatoire $X$ dont une
densité de
probabilité $f$ est donnée par : 
\[
\left\{ 
\begin{array}{c}
f\left( t\right) = te^{-t^{2}/2}\;\text{si }t\geq 0 \\
f\left( t\right) = 0\quad \text{si }t<0
\end{array}
\right. 
\]
(On rappelle que $\dint{0}{+ \infty }e^{-\frac{t^{2}}{2}}dt =
\sqrt{\dfrac{\pi }{2}}$ )

\begin{noliste}{a)}
 \setlength{\itemsep}{2mm}
\item Vérifier que $f$ est bien une densité de probabiltié.

\item Déterminer la fonction de répartition $F_{X}$ de $X.$

\item Calculer l'espérance de la variable $X.$
\end{noliste}
\end{noliste}

\label{fin}

\end{document}

