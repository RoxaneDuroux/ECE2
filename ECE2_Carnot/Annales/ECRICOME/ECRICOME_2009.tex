\documentclass[11pt]{article}%
\usepackage{geometry}%
\geometry{a4paper,
 lmargin = 2cm,rmargin = 2cm,tmargin = 2.5cm,bmargin = 2.5cm}

\input{../../macros.tex}

\pagestyle{fancy} %
\lhead{ECE2 \hfill Mathématiques\\
} %
\chead{\hrule} %
\rhead{} %
\lfoot{} %
\cfoot{} %
\rfoot{\thepage} %

\renewcommand{\headrulewidth}{0pt}% : Trace un trait de séparation
 % de largeur 0,4 point. Mettre 0pt
 % pour supprimer le trait.

\renewcommand{\footrulewidth}{0.4pt}% : Trace un trait de séparation
 % de largeur 0,4 point. Mettre 0pt
 % pour supprimer le trait.

\setlength{\headheight}{14pt}

\title{\bf \vspace{-2cm} ECRICOME 2009} %
\author{} %
\date{} %
\begin{document}

\maketitle %
\vspace{-1.4cm}\hrule %
\thispagestyle{fancy}

\vspace*{.2cm}


% DEBUT DU DOC À MODIFIER : tout virer jusqu'au début de l'exo

%Définition et changement de valeurs de
compteurs%newcounter{cpt1}{section} compteur cpt1 remis à 0 à chaque
aumentation par stepcounter du compteur section%setcounter{cpt1}{3} on
met le compteur à 3%addtocounter{cpt1}{5} on ajoute 5 au compteur%
stepcounter{cpt1} on ajoute 1% ifthenelse{test}{alors}{sinon} (page
206) pour subordonner à une condition % whiledo{test}{commande} pour
faire une boucle (page 206 aussi) % value{cpt1} pour noter dans le
document la valeur de cpt1 
%Définition définitive d'opérateurs
mathématiques\newcommand{\ch}{\operatorname{ch}} 
\newcommand{\sh}{\operatorname{sh}}
\renewcommand{\tanh}{\operatorname{th}}
\renewcommand{\sinh}{\operatorname{sh}}
\renewcommand{\cosh}{\operatorname{ch}}
\newcommand{\argsh}{\operatorname{argsh}}
\newcommand{\argch}{\operatorname{argch}}
\newcommand{\argth}{\operatorname{argth}}
\newcommand{\Id}{\operatorname{Id}}
\renewcommand{\leq}{\leq}
\renewcommand{\geq}{\geq }

\newcommand{\dlim}{\lim}
\newcommand{\dsum}{\sum}
\newcommand{\dprod}{\prod}



%Définition de nouvelles couleurs : rgb(trois paramètres red green blue
entre 0 et 1); cmyk (quatre cyan magenta yellow black) entre 0 et 1;
gray (entre 0 et 1) et black, white, red, green, blue, cyan, magenta,
yellow% definecolor{0gris}{gray}{0.8} 
% Nouvelle commande pour encadrer le titre car shabox ne veut que d'une
seule ligne; ATTENTION A LA TAILLE; petite différence avec shadowbox ou
doublebox, voire fcolorbox ou colorbox (au lieu de shabox; laisser le
parbox tranquille sauf pour la taille de la boîte
\newcommand{\Tbox}[1]{\begin{center} \shabox{\parbox{0.6
\linewidth}{#1}} \end{center}} %[1] pour 1 paramètre ; #1 pour ce que
fait le 1er paramètre; entre accolades ce que fait la commande
%Mise en page en mode fancy : en-têtes et pieds de pages puis
définition des en-têtes et pieds de pages\pagestyle{fancy}
\lhead{ECE 2 - Mathématiques \\
Quentin Dunstetter - ENC-Bessières 2011$\backslash$2012}
\chead{}
\rhead{Ecricome 2009}
\rfoot[ \ \thepage]{\thepage}
\cfoot{}
\lfoot{}
\thispagestyle{fancy} %Mise en page de la 1ère page en mode fancy
%Trait en bas et en haut de la page (entre en-tête et texte et texte et
pied de page)\renewcommand{\footrulewidth}{0.4pt}
\renewcommand{\headrulewidth}{0.4pt}


\begin{center}
{\Huge ECRICOME Eco 2009}
\end{center}

\section{EXERCICE.}

A tout triplet $\left( a,b,c\right) $ de réels, on associe la matrice
$M\left( a,b,c\right) $ définie par :
\[
M\left( a,b,c\right) = 
\begin{smatrix}
a & a & a \\
0 & b & b \\
0 & 0 & c
\end{smatrix}
\]

On désigne par $E$ l'ensemble des matrices $M\left( a,b,c\right) $ où
$a$, $b$, $c$ sont des réels. Ainsi : 
\[
E = \left\{ M\left( a,b,c\right) \text{ avec }a,b,c\text{
réels}\right\} 
\]

\subsection{Recherche d'une base de $E$.}

\begin{noliste}{1.}
 \setlength{\itemsep}{4mm}
\item Montrer que $E$ est un sous-espace vectoriel de l'espace
vectoriel.$\M{3} $ des matrices carrées réelles d'ordre 3.

\item Donner une base de $E$ ainsi que sa dimension.
\end{noliste}

\subsection{Cas particulier de la matrice $M\left( 1,2,3\right) $.}

\begin{noliste}{1.}
 \setlength{\itemsep}{4mm}
\item Donner les valeurs propres de $M\left( 1,2,3\right) $.

\item Déterminer une matrice $P$ inversible et une matrice $D$
diagonale
de $\M{3} $ telle que :
\[
D = P^{-1}M\left( 1,2,3\right) P
\]

\item Donner l'expression de $P^{-1}$ et en déduire la matrice $M\left(
1,2,3\right) $ en fonction de l'entier naturel $n$.
\end{noliste}

\subsection{Cas particulier de la matrice $M\left( 1,1,1\right) $}

On pose $J = M\left( 1,1,1\right) -I_{3}$, la matrice $I_{3}$
représentant
la matrice unité de $\M{3} $.

\begin{noliste}{1.}
 \setlength{\itemsep}{4mm}
\item Calculer les matrices $J^{2},$ $J^{3}.$. En déduire, sans
démonstration, l'expression de $J^{n}$, pour tout entier naturel $n\geq
3.$

\item Montrer que pour tout entier naturel $n\geq 2 :$
\[
\left[ M\left( 1,1,1\right) \right] ^{n} = I_{3} + nJ + \frac{n\left(
n-1\right) }{2}J^{2}
\]
L'écriture obtenue est-elle encore valable pour les entiers $n = 0$ et
$n = 1
$ ?

\item En déduire l'écriture matricielle de $\left[ M\left(
1,1,1\right) \right] ^{n}$.
\end{noliste}

\subsection{Cas particulier de la matrice $M\left( 1,1,2\right) $.}

On note $f$ l'endomorphisme de $\R^{3}$ dont la matrice dans la base
canonique de $\R^{3}$ est la matrice $M\left( 1,1,2\right) $. On
définit la famille de vecteurs $\mathcal{C} = \left(
\vec{u},\vec{v},\vec{w}\right) $ par :

$\vec{u} = \left( 1,0,0\right),\ \vec{v} = \left( 0,1,0\right),\
\vec{w} = \left( 2,1,1\right) $

\begin{noliste}{1.}
 \setlength{\itemsep}{4mm}
\item Démontrer que $\mathcal{C}$ est une base de $\R^{3}.$

\item Prouver que les vecteurs $\vec{u}$ et $\vec{w}$ sont deux
vecteurs
propres de $f$ associés à deux valeurs propres que l'on précisera.

\item Exprimer $f\left( \vec{v}\right) $ comme combinaison linéaire des
vecteurs $\vec{u}$ et $\vec{v}$. En déduire la matrice $T$ de $f$ dans
la base $\mathcal{C}$.

\item Montrer que pour tout entier naturel $n :$
\[
T^{n} = 
\begin{smatrix}
1 & n & 0 \\
0 & 1 & 0 \\
0 & 0 & 2^{n}\end{smatrix}
\]

\item Montrer que la matrice de passage $R$ de la base canonique à la
base $\mathcal{C}$ a pour matrice inverse la matrice $Q = 
\begin{smatrix}
1 & 0 & -2 \\
0 & 1 & -1 \\
0 & 0 & 1
\end{smatrix}
$

\item Donner une relation reliant les matrices $M\left( 1,1,2\right) $,
$Q$, 
$R$ et $T$.

\item Sans l'expliciter, écrire $\left[ M\left( 1,1,2\right) \right]
^{n}
$ en fonction de $n,~Q,~R,~T.$
\end{noliste}

\section{EXERCICE.}

On considère l'application $\varphi $ définie sur $\R^{+ \ast
}$ par : 
\[
\varphi \left( x\right) = 2\ln \left( \dfrac{x}{2}\right) +
\dfrac{1}{x}
\]
ainsi que la fonction numérique $f$ des variables réelles $x$ et $y$
définie par :
\[
\forall \left( x,y\right) \in \left] 0, + \infty \right[ \ \times
\left]
0, + \infty \right[,\quad f\left( x,y\right) = e^{x + 4y}\ln \left(
xy\right) 
\]

\subsection{Étude des zéros de $\protect \varphi $.}

\begin{noliste}{1.}
 \setlength{\itemsep}{4mm}
\item Déterminer la limite de $\varphi \left( x\right) $ lorsque $x$
tend vers $0$ par valeurs positives. Interpréter graphiquement cette
limite.

\item Déterminer la limite de $\varphi \left( x\right) $ lorsque $x$
tend vers $ + \infty $, ainsi que la limite de $\dfrac{\varphi \left(
x\right) 
}{x}$ lorsque $x$ tend vers $ + \infty $. Interpréter graphiquement
cette
limite.

\item Justifier la dérivabilité de $\varphi $ sur $\R^{+ \ast
}$, déterminer sa dérivée.

\item Dresser le tableau de variation de $\varphi $, faire
appara\^{\i}tre
les limites de $\varphi $ en $0^{+}$ et $ + \infty $.

\item On rappelle que $\ln \left( 2\right) \simeq 0,7$. Montrer
l'existence
de deux réels positifs $\alpha $ et $\beta $ tels que :
\[
\varphi \left( \alpha \right) = \left( \beta \right) = 0
\]

\item Proposer un programme en \Scilab{} permettant d'encadrer $\alpha
$ dans
un intervalle d'amplitude $10^{-2}$.
\end{noliste}

\subsection{Extrema de $f$ sur $\left] 0, + \infty \right[ \ \times
\left]
0, + \infty \right[ $}

\begin{noliste}{1.}
 \setlength{\itemsep}{4mm}
\item Justifier que $f$ est de classe $C^{2}$ sur $\left] 0, + \infty
\right[
\times \left] 0, + \infty \right[.$

\item Calculer les dérivées partielles premières et prouver que
pour $x$, et $y$ strictement positifs
\[
\left\{ 
\begin{array}{c}
\dfrac{\partial f}{\partial x}\left( x,y\right) = f\left( x,y\right) +
\dfrac{1}{x}e^{x + 4y} \\
\dfrac{\partial f}{\partial x}\left( x,y\right) = 4f\left( x,y\right) +
\dfrac{1}{y}e^{x + 4y}
\end{array}
\right. 
\]

\item Montrer que les points de coordonnées respectives $\left(
\alpha,\dfrac{\alpha }{4}\right) $ et $\left( \beta,\dfrac{\beta
}{4}\right) $sont
des points critiques de $f$ sur $\left] 0, + \infty \right[ \ \times
\left]
0, + \infty \right[.$

\item Calculer les dérivées partielles secondes sur $\left]
0, + \infty \right[ \ \times \left] 0, + \infty \right[ $ et établir
que :
\[
\left\{ 
\begin{array}{c}
\dfrac{\partial ^{2}f}{\partial x^{2}}\left( \alpha,\dfrac{\alpha
}{4}\right) = \dfrac{\alpha -1}{\alpha ^{2}}e^{2\alpha } \\
\dfrac{\partial ^{2}f}{\partial y^{2}}\left( \alpha,\dfrac{\alpha
}{4}\right) = 16\dfrac{\alpha -1}{\alpha ^{2}}e^{2\alpha } \\
\dfrac{\partial ^{2}f}{\partial y\partial x}\left( \alpha,\dfrac{\alpha
}{4}\right) = \dfrac{4}{\alpha }e^{2\alpha }
\end{array}
\right. 
\]

\item La fonction $f$ présente-t-elle un extremum local sur $\left]
0, + \infty \right[ \ \times \left] 0, + \infty \right[ $ au point de
coordonnées $\left( \alpha,\dfrac{\alpha }{4}\right) $ ? Si oui, en
donner sa nature
(maximum on minimum)

\item De même, $f$ présente-t-elle un extremum local sur $\left]
0, + \infty \right[ \ \times \left] 0, + \infty \right[ $ au point de
coordonnées $\left( \beta,\dfrac{\beta }{4}\right) $ ?
\end{noliste}

\section{EXERCICE.}

\subsection{Liminaire.}

Soient $x$ un réel dans l'intervalle $\left[ 0,1\right[ $, $n$ un
entier
naturel non nul et $S_{n}$ la fonction définie par :
\[
S_{n}\left( x\right) = \Sum{k = 0}{n}x^{k}
\]

\begin{noliste}{1.}
 \setlength{\itemsep}{4mm}
\item Calculer la somme $S_{n}\left( x\right) $.

\item Dériver l'égalité obtenue et montrer que :
\[
\Sum{k = 1}{n}kx^{k-1} = \frac{nx^{n + 1}-\left( n + 1\right) x^{n} +
1}{\left(
1-x\right) ^{2}}
\]
\end{noliste}

Une municipalité a lancé une étude concernant les problèmes
liés au transport.

\subsection{Partie 1.}

Sur une ligne de bus, une enquête a permis de révéler que le
retard (ou l'avance) sur l'horaire officiel du bus à une station
donnée, peut être représenté(e) par une variable aléatoire réelle,
notée $X$, exprimée en minutes, qui suit une loi normale
$\mathcal{N}\left( m,\sigma ^{2}\right) $.

On admet de plus que la probabilité que le retard soit inférieur 
à $7$ minutes est égale à $p = 0.8413$ et que l'espérance de $X
$ est de $5$ minutes.

\begin{noliste}{1.}
 \setlength{\itemsep}{4mm}
\item Déterminer la valeur de $\sigma $ en utilisant la table jointe en
annexe.

\item Quelle est la probabilité que le retard soit supérieur à $9
$ minutes ?

\item Sachant que le retard est supérieur à $3$ minutes, quelle est
la probabilité que le retard soit inférieur à $7$ minutes ? (On
exprimera cette probabilité à l'aide de la fonction de répartition de
la loi normale centrée réduite, puis on utilisera la
table jointe en annexe).

\item Monsieur Thierex fréquente cette ligne de bus tous les jours
pendant 10 jours. On suppose que les retards journaliers sont
indépendants.

\begin{noliste}{a)}
 \setlength{\itemsep}{2mm}
\item On désigne par $Y$ la variable aléatoire réelle égale
au nombre de jours où Monsieur Thierex a attendu moins de 7 minutes.\\
Déterminer la loi de $Y$, donner sans calcul, son espérance et sa
variance.

\item On définit par $Z$ la variable aléatoire discrète réelle
indiquant le rang $k$ du jour où pour la première fois Monsieur
Thierex attend plus de 7 minutes si cet événement se produit. Dans
le cas contraire si le temps d'attente est inférieur à 7 minutes
pendant les dix jours, $Z$ prend la valeur $0$.\\
Déterminer en fonction de $p$ la probabilité des événements $\left[ Z =
0\right] $, puis $\left[ Z = k\right] $ pour $1\leq k\leq 10$.\\
Utiliser le liminaire pour calculer l'espérance de $Z$ en fonction de
$p$.
\end{noliste}

\item Lassé des retards de son bus, Monsieur Thurman décide de
prendre le bus ou le métro selon le protocole suivant :

\begin{noliste}{$\sbullet$}
\item Le premier jour, il prend le bus.

\item Si le jour $n$ $\left( n\in \N^{\ast }\right) $ il attend plus
de 7 minutes pour prendre le bus, le jour $n + 1$ il prend le métro,
sinon
il prend de nouveau le bus.

\item Si le jour $n$ il prend le métro, le jour $n + 1$ il prend le
métro ou le bus de façon équiprobable.
\end{noliste}

On note $p_{n}$ la probabilité de l'événement $A_{n} = $" Monsieur
Thurman prend le bus le jour $n$"

\begin{noliste}{a)}
 \setlength{\itemsep}{2mm}
\item Justifier que pour tout entier naturel $n$ non nul :
\[
p_{n + 1} = \left( p-\frac{1}{2}\right) p_{n} + \frac{1}{2}
\]

\item Soit $\alpha $ le réel vérifiant :
\[
\alpha = \left( p-\frac{1}{2}\right) \alpha + \frac{1}{2}
\]
Déterminer $\alpha $ en fonction de $p$, puis montrer que, pour tout
entier naturel $n$ non nul :
\[
p_{n} = \left( p-\frac{1}{2}\right) ^{n-1}\left( 1-\alpha \right) +
\alpha 
\]

\item La suite $\left( p_{n}\right) $ est-elle convergente ? Si oui
quelle
est sa limite ?
\end{noliste}
\end{noliste}

\subsection{3.3. Partie 2.}

\begin{noliste}{1.}
 \setlength{\itemsep}{4mm}
\item Le nombre d'appels reçus par le standard d'une société de
taxis pendant une période de durée $t$ suit une loi de Poisson $Y_{t}
$ de paramètre $\lambda t$, $\lambda $ étant une constante
strictement positive. Une origine de temps étant choisie, on note $T$
la
variable aléatoire réelle représentant le temps d'attente du
premier appel vers ce standard. Par convention $\Prob\left(\Ev{ T\leq
t}\right) = 0$ pour $t<0$.

\begin{noliste}{a)}
 \setlength{\itemsep}{2mm}
\item Pour tout entier naturel $k$, rappeler la valeur de la
probabilité
de l'événement $\left[ Yt = k\right] $, ainsi que l'espérance et
la variance de $Y_{t}$.

\item Que peut-on dire des événements $\left[ Y_{t} = 0\right] $] et
$\left[ T>t\right] $ pour $t>0$. En déduire la probabilité des
événements $\left[ T>t\right] $ et $\left[ T\leq t\right] $] pour
$t>0$.

\item Expliciter la fonction de répartition $F_{T}$ de $T$.
Reconna\^{\i}tre la loi de $T$ et donner son espérance et sa variance.
\end{noliste}

\item La durée, exprimée en heures, du transport d'un client par la
société est une variable aléatoire $U$ à densité dont
une densité est donnée par :
\[
\left\{ 
\begin{array}{cc}
g\left( t\right) = t~e^{-t} & \text{ si }t\geq 0 \\
g\left( t\right) = 0 & \text{si }t<0
\end{array}
\right. 
\]

\begin{noliste}{a)}
 \setlength{\itemsep}{2mm}
\item Vérifier que $g$ est bien une densité de probabilité.

\item Montrer que $U$ admet une espérance que l'on déterminera. Que
représente cette espérance ?
\end{noliste}
\end{noliste}

\textbf{Table}

\rule{17cm}{0.05cm}

La table ci-dessous comporte les valeurs de la fonction de répartition
de la loi normale centrée réduite, à savoir les valeurs de :
\[
\Phi \left( x\right) = \frac{1}{\sqrt{2\pi }}\dint{-\infty }{x}\exp
\left( -\frac{t^{2}}{2}\right\ dt
\]

par exemple $\Phi \left( 0,67\right) = 0,7486$

\begin{tabular}{|l|l|l|l|l|l|l|l|l|l|l|}
\hline
$x$ & \textbf{0,00} & \textbf{0,01} & \textbf{0,02} & \textbf{0,03} & 
\textbf{0,04} & \textbf{0,05} & \textbf{0,06} & \textbf{0,07} &
\textbf{0,08}
 & \textbf{0,09} \\
\hline
\textbf{0,0} & 0,5000 & 0,5040 & 0,5080 & 0,5120 & 0,5160 & 0,5199 &
0,5239
 & 0,5279 & 0,5319 & 0,5359 \\
\hline
\textbf{0,1} & 0,5398 & 0,5438 & 0,5478 & 0,5517 & 0,5557 & 0,5596 &
0,5636
 & 0,5675 & 0,5714 & 0,5753 \\
\hline
\textbf{0,2} & 0,5793 & 0,5832 & 0,5871 & 0,5910 & 0,5948 & 0,5987 &
0,6026
 & 0,6064 & 0,6103 & 0,6141 \\
\hline
\textbf{0,3} & 0,6179 & 0,6217 & 0,6255 & 0,6293 & 0,6331 & 0,6368 &
0,6406
 & 0,6443 & 0,6480 & 0,6517 \\
\hline
\textbf{0,4} & 0,6554 & 0,6591 & 0,6628 & 0,6664 & 0,6700 & 0,6736 &
0,6772
 & 0,6808 & 0,6844 & 0,6879 \\
\hline
\textbf{0,5} & 0,6915 & 0,6950 & 0,6985 & 0,7019 & 0,7054 & 0,7088 &
0,7123
 & 0,7157 & 0,7190 & 0,7224 \\
\hline
\textbf{0,6} & 0,7257 & 0,7291 & 0,7324 & 0,7357 & 0,7389 & 0,7422 &
0,7454
 & 0,7486 & 0,7517 & 0,7549 \\
\hline
\textbf{0,7} & 0,758 & 0,7611 & 0,7642 & 0,7673 & 0,7704 & 0,7734 &
0,7764 & 
0,7794 & 0,7823 & 0,7852 \\
\hline
\textbf{0,8} & 0,7881 & 0,7910 & 0,7939 & 0,7967 & 0,7995 & 0,8023 &
0,8051
 & 0,8078 & 0,8106 & 0,8133 \\
\hline
\textbf{0,9} & 0,8159 & 0,8186 & 0,8212 & 0,8238 & 0,8264 & 0,8289 &
0,8315
 & 0,8340 & 0,8365 & 0,8389 \\
\hline
\textbf{1,0} & 0,8413 & 0,8438 & 0,8461 & 0,8485 & 0,8508 & 0,8531 &
0,8554
 & 0,8577 & 0,8599 & 0,8621 \\
\hline
\textbf{1,1} & 0,8643 & 0,8665 & 0,8686 & 0,8708 & 0,8729 & 0,8749 &
0,8770
 & 0,8790 & 0,8810 & 0,8830 \\
\hline
\textbf{1,2} & 0,8849 & 0,8869 & 0,8888 & 0,8907 & 0,8925 & 0,8944 &
0,8962
 & 0,8980 & 0,8997 & 0,9015 \\
\hline
\textbf{1,3} & 0,9032 & 0,9049 & 0,9066 & 0,9082 & 0,9099 & 0,9115 &
0,9131
 & 0,9147 & 0,9162 & 0,9177 \\
\hline
\textbf{1,4} & 0,9192 & 0,9207 & 0,9222 & 0,9236 & 0,9251 & 0,9265 &
0,9279
 & 0,9292 & 0,9306 & 0,9319 \\
\hline
\textbf{1,5} & 0,9332 & 0,9345 & 0,9357 & 0,9370 & 0,9382 & 0,9394 &
0,9406
 & 0,9418 & 0,9429 & 0,9441 \\
\hline
\textbf{1,6} & 0,9452 & 0,9463 & 0,9474 & 0,9484 & 0,9495 & 0,9505 &
0,9515
 & 0,9525 & 0,9535 & 0,9545 \\
\hline
\textbf{1,7} & 0,9554 & 0,9564 & 0,9573 & 0,9582 & 0,9591 & 0,9599 &
0,9608
 & 0,9616 & 0,9625 & 0,9633 \\
\hline
\textbf{1,8} & 0,9641 & 0,9649 & 0,9656 & 0,9664 & 0,9671 & 0,9678 &
0,9686
 & 0,9693 & 0,9699 & 0,9706 \\
\hline
\textbf{1,9} & 0,9713 & 0,9719 & 0,9726 & 0,9732 & 0,9738 & 0,9744 &
0,975 & 
0,9756 & 0,9761 & 0,9767 \\
\hline
\textbf{2,0} & 0,9772 & 0,9778 & 0,9783 & 0,9788 & 0,9793 & 0,9798 &
0,9803
 & 0,9808 & 0,9812 & 0,9817 \\
\hline
\end{tabular}

\end{document}

