\documentclass[11pt]{article}%
\usepackage{geometry}%
\geometry{a4paper,
 lmargin = 2cm,rmargin = 2cm,tmargin = 2.5cm,bmargin = 2.5cm}

\input{../../macros.tex}

\pagestyle{fancy} %
\lhead{ECE2 \hfill Mathématiques\\
} %
\chead{\hrule} %
\rhead{} %
\lfoot{} %
\cfoot{} %
\rfoot{\thepage} %

\renewcommand{\headrulewidth}{0pt}% : Trace un trait de séparation
 % de largeur 0,4 point. Mettre 0pt
 % pour supprimer le trait.

\renewcommand{\footrulewidth}{0.4pt}% : Trace un trait de séparation
 % de largeur 0,4 point. Mettre 0pt
 % pour supprimer le trait.

\setlength{\headheight}{14pt}

\title{\bf \vspace{-2cm} ECRICOME 1993} %
\author{} %
\date{} %
\begin{document}

\maketitle %
\vspace{-1.4cm}\hrule %
\thispagestyle{fancy}

\vspace*{.2cm}


% DEBUT DU DOC À MODIFIER : tout virer jusqu'au début de l'exo

%Définition et changement de valeurs de
compteurs%newcounter{cpt1}{section} compteur cpt1 remis à 0 à chaque
aumentation par stepcounter du compteur section%setcounter{cpt1}{3} on
met le compteur à 3%addtocounter{cpt1}{5} on ajoute 5 au compteur%
stepcounter{cpt1} on ajoute 1% ifthenelse{test}{alors}{sinon} (page
206) pour subordonner à une condition % whiledo{test}{commande} pour
faire une boucle (page 206 aussi) % value{cpt1} pour noter dans le
document la valeur de cpt1 
%Définition définitive d'opérateurs
mathématiques\newcommand{\ch}{\operatorname{ch}} 
\newcommand{\sh}{\operatorname{sh}}
\renewcommand{\tanh}{\operatorname{th}}
\renewcommand{\sinh}{\operatorname{sh}}
\renewcommand{\cosh}{\operatorname{ch}}
\newcommand{\argsh}{\operatorname{argsh}}
\newcommand{\argch}{\operatorname{argch}}
\newcommand{\argth}{\operatorname{argth}}
\newcommand{\Id}{\operatorname{Id}}
\renewcommand{\leq}{\leq}
\renewcommand{\geq}{\geq }

\newcommand{\dlim}{\lim}
\newcommand{\dsum}{\sum}
\newcommand{\dint}{\int}
\newcommand{\dprod}{\prod}



%Définition de nouvelles couleurs : rgb(trois paramètres red green blue
entre 0 et 1); cmyk (quatre cyan magenta yellow black) entre 0 et 1;
gray (entre 0 et 1) et black, white, red, green, blue, cyan, magenta,
yellow% definecolor{0gris}{gray}{0.8} 
% Nouvelle commande pour encadrer le titre car shabox ne veut que d'une
seule ligne; ATTENTION A LA TAILLE; petite différence avec shadowbox ou
doublebox, voire fcolorbox ou colorbox (au lieu de shabox; laisser le
parbox tranquille sauf pour la taille de la boîte
\newcommand{\Tbox}[1]{\begin{center} \shabox{\parbox{0.6
\linewidth}{#1}} \end{center}} %[1] pour 1 paramètre ; #1 pour ce que
fait le 1er paramètre; entre accolades ce que fait la commande
%Mise en page en mode fancy : en-têtes et pieds de pages puis
définition des en-têtes et pieds de pages\pagestyle{fancy}
\lhead{ECE 2 - Mathématiques \\
Quentin Dunstetter - ENC-Bessières 2011$\backslash$2012}
\chead{}
\rhead{Ecricome 1993}
\rfoot[ \ \thepage]{\thepage}
\cfoot{}
\lfoot{}
\thispagestyle{fancy} %Mise en page de la 1ère page en mode fancy
%Trait en bas et en haut de la page (entre en-tête et texte et texte et
pied de page)\renewcommand{\footrulewidth}{0.4pt}
\renewcommand{\headrulewidth}{0.4pt}


\begin{center}
{\Huge ECRICOME Eco 1993}
\end{center}

\section*{EXERCICE1}

Soit $x$ un réel strictement positif.\\
On pose pour tout entier naturel $n :$
\[
S_{n}(x) = \Sum{k = 0}{n}(-1)^{k}\dfrac{1}{k + x + 1} = \dfrac{1}{x +
1}-\dfrac{1}{x + 2} + \cdots + (-1)^{n}\dfrac{1}{n + x + 1}.
\]
On se propose d'étudier la limite $S(x)$ de la somme $S_{n}(x)$ lorsque
$n$
tend vers $ + \infty.$

\begin{noliste}{1.}
 \setlength{\itemsep}{4mm}
\item Pour tout entier naturel $p,$ on pose $f_{p}(x) = \left\{ 
\begin{array}{ccc}
\dfrac{t^{x + p}}{1 + t} & \text{si} & 0\leq t\leq 1 \\
0 & \text{si} & t = 0
\end{array}
\right. $ et $I_{p}(x) = \dint{0}{1}f_{p}(t)dt.$\\
Montrer que, pour tout entier naturel $p,$ l'intégrale $I_{p}(x)$
existe.

\item Montrer que, pour tout réel $t$ élément de $[0,1],$ on a :
\[
\dfrac{1}{1 + t} = \Sum{k = 0}{n}(-1)^{k}t^{k} + (-1)^{n +
1}\dfrac{t^{n + 1}}{1 + t}.
\]

\item Déduire de ce qui précède que l'on a :
\[
\dint{0}{1}\dfrac{t^{x}}{1 + t}dt = S_{n}(x) + R_{n}(x)\quad
\text{où}\quad R_{n}(x) = (-1)^{n + 1}\dint{0}{1}\dfrac{t^{n + x +
1}}{1 + t}dt.
\]

\item Démonter que l'on a pour tout entier naturel $n :0\leq
\dint{0}{1}\dfrac{t^{n + x + 1}}{1 + t}dt\leq \dfrac{1}{n + 2}.$

\item Conclure que l'on a $ :\quad S(x) = \dint{0}{1}\dfrac{t^{x}}{1 +
t}dt.$

\item Étude du cas particulier où $x = \dfrac{1}{2}.$

\begin{noliste}{a)}
 \setlength{\itemsep}{2mm}
\item En utilisant le changement de variable $u = t^{1/2},$ calculer
$S(\dfrac{1}{2})$.\\
(On rappelle que $\dint{0}{1}\dfrac{dx}{1 + x^{2}}).$

\item En déduire que l'on a $ :\quad \dlim{n\rightarrow + \infty
}\Sum{k = 0}{n}(-1)^{k}\dfrac{1}{2k + 1} = \dfrac{\pi }{4}.$
\end{noliste}
\end{noliste}

\section*{EXERCICE 2}

Les produits référencés $X,Y,Z$ se partagent un marché. On note
$x_{n},y_{n},z_{n}$ les proportions de consommateurs utilisant
respectivement
les produits $X,Y,Z$ au $n^{i\grave{e}me}$ mois, où $n$ est un entier
naturel.\\
On observe les données suivantes $ :x_{0} = 0,1,\quad y_{0} = 0,2,\quad
z_{0} = 0,7.
$\\
Par ailleurs les sondages mensuels ont permis de déterminer les
intentions
des consommateurs, supposées constantes :

\begin{noliste}{$\sbullet$}
\item Utilisant le produit $X$ un mois donné, respectivement $40$ \%,
$30$
\%, $30$ \% des consommateurs ont l'intention d'adopter les produits
$X,Y,Z$
le mois suivant.

\item Utilisant le produit $Y$ un mois donné, respectivement $30$ \%,
$40$
\%, $30$ \% des consommateurs ont l'intention d'adopter les produits
$X,Y,Z$
le mois suivant.

\item Utilisant le produit $Z$ un mois donné, respectivement $20$ \%,
$10$
\%, $70$ \% des consommateurs ont l'intention d'adopter les produits
$X,Y,Z$
le mois suivant.
\end{noliste}

\begin{noliste}{1.}
 \setlength{\itemsep}{4mm}
\item Exprimer $x_{n + 1},$ $y_{n + 1},$ $z_{n + 1}$ en fonction de
$x_{n},$ $y_{n},$ $z_{n}.$

\item On considère la matrice $A = 
\begin{smatrix}
0,2 & 0,1 \\
0,2 & 0,3
\end{smatrix}
;\quad U_{n} = 
\begin{smatrix}
x_{n} \\
y_{n}\end{smatrix}
;\quad B = 
\begin{smatrix}
0,2 \\
0,1
\end{smatrix}.$\\
Montrer que l'on a, pour tout entier $n,$ l'égalité matricielle $ :U_{n
+ 1} = AU_{n} + B.$

\item Déterminer la matrice $C$ telle que $C = AC + B.$

\item On considère la matrice $V_{n} = U_{n}-C;$ démonter, pour tout
entier $n,
$ que $V_{n} = A^{n}V_{0}.$

\item 

\begin{noliste}{a)}
 \setlength{\itemsep}{2mm}
\item Calculer les valeurs propres de la matrice $A.$

\item Trouver une matrice $P$ inversible telle que $P^{-1}AP$ soit
diagonale.

\item En déduire, pour tout entier naturel non nul, l'expression de la
matrice $A^{n}.$
\end{noliste}

\item En déduire les valeurs $x_{n},y_{n}$ en fonction de $n.$

\item Calculer $z_{n}$ en fonction de $n.$

\item Quelles sont à long terme les proportions des consommateurs
utilisant
respectivement les produits $X,Y,Z$ ?
\end{noliste}

\section*{PROBLEME}

\begin{center}
\textbf{Les deux parties sont indépendantes}
\end{center}

\subsection*{Première partie}

La société ALFDIS étudie à la fin de chaque mois le coût mensuel de
gestion
de l'article A, lié au nombre $n$ de centaines d'articles A en stock au
début du mois (le stock est dit de niveau $100n)$ et au nombre $k$ de
centaines d'articles A demandés pendant ce même mois.\\
La demande mensuelle de cet article A est une variable aléatoire $X$
qui
suit une loi de Poisson de paramètre $5$ (en centaines d'articles).\\
La société estime qu'un article A restant en stock à la fin du mois
coûte à
l'entreprise $300\ $francs alors qu'un article A manquant lui coûte
$500$
francs.

\begin{noliste}{1.}
 \setlength{\itemsep}{4mm}
\item Pour tout entier naturel $n,$ on pose $p_{n} = P\left(\Ev{X\leq
n}\right).$

\begin{noliste}{a)}
 \setlength{\itemsep}{2mm}
\item Exprimer, pour $n$ non nul, $p_{n}$ et $\Sum{k =
0}{n}k.P\left(\Ev{X = k}\right)$
en fonction de $p_{n-1}$ et de $n.$

\item En utilisant les probabilités $p_{n},$ calculer les sommes
$u_{n}$ et $v_{n}$ suivantes :
\[
u_{n} = \Sum{k = n + 1}{+ \infty }P\left(\Ev{X = k}\right)\quad
\text{et}\quad
v_{n} = \Sum{k = n + 1}{+ \infty }k.P\left(\Ev{X = k}\right).
\]

\item Calculer $u_{4}$ et $v_{4}$ à $10^{-6}$ près au mieux (on
utilisera la
table de la page \ref{fin}).
\end{noliste}

\item Montrer que, pour un stock de niveau $100n$ et pour une demande
mensuelle de $X = k$ centaines d'articles, le coût $C_{n}(k)$ de
gestion de
l'article A s'écrit :
\[
C_{n}(k) = \left\{ 
\begin{array}{ccc}
3.10^{4}.(n-k) & \text{si} & 0\leq k\leq n \\
5.10^{4}(k-n) & \text{si} & n<k
\end{array}
\right. 
\]

\item On note $C_{n}$ la variable aléatoire prenant les valeurs
$C_{n}(k).$\\
Calculer, en fonction de $n,$ de $p_{n-1}$ et de $p_{n},$ l'espérance
mathématique :
\[
\E(C_{n}) = \Sum{k = 0}{+ \infty }C_{n}(k).P\left(\Ev{X = k}\right).
\]

\item Démontrer la relation $ :\E(C_{n + 1})-\E(C_{n}) =
(8p_{n}-5).10^{4}.$

\item Trouver la valeur de l'entier naturel $n$ solution arrondie par
excès
de l'équation :
\[
\E(C_{n + 1}) = E(C_{n}).
\]

\item En déduire le sens de variation de la suite de terme général
$\E(C_{n}).
$\\
Montrer que $\E(C_{5})>\E(C_{6}).$\\
Conclure quant à l'existence d'un niveau $100n$ du stock d'articles A
qui
minimise l'espérance $\E(C_{n})$ du coût de gestion de cet article.
\end{noliste}

\subsection*{Deuxième partie}

La société ALFDIS distribue aussi de l'essence dont la demande est
aléatoire. Elle a procédé à une étude des coûts mensuels de gestion de
ce
produit.

\begin{noliste}{1.}
 \setlength{\itemsep}{4mm}
\item On considère la fonction $f$ définie pour tout réel $t$ par $
:f(t) = \left\{ 
\begin{array}{ccc}
0 & \text{si} & t\leq 0 \\
4.t.e^{-2t} & \text{si} & t>0
\end{array}
\right..$\\
On appelle $\mathcal{C}$ la courbe représentative de $f$ dans un repère
$(O,\overrightarrow{i},\overrightarrow{j})$, d'unités $4$ cm sur l'axe
des
abscisses et $10$ cm sur l'axe des ordonnées.

\begin{noliste}{a)}
 \setlength{\itemsep}{2mm}
\item Étudier la dérivabilité de $f$ en $0.$ Conclure pour la courbe
$\mathcal{C}.$

\item Construire le tableau de variation de $f.$

\item Déterminer les coordonnées du point d'inflexion $1$ de
$\mathcal{C}.$

\item Tracer la courbe $\mathcal{C}.$
\end{noliste}

\item Pour tout entier naturel $p,$ on pose $I_{p} = \dint{0}{+ \infty
}t^{p}e^{-2t}dt$ et pour $a\geq 0,$ $I_{p}(a) =
\dint{0}{a}t^{p}e^{-2t}dt.$

\begin{noliste}{a)}
 \setlength{\itemsep}{2mm}
\item Calculer $I_{0}(a).$

\item Déterminer une relation de récurrence entre $I_{p + 1}(a)$ et
$I_{p}(a).$

\item En déduire la valeur de $I_{1}(a)$ et de $I_{2}(a).$

\item Prouver que $I_{p}$ est une intégrale impropre convergente.
Calculer $I_{p}$ en fonction de $p.$

\item Démontrer que la fonction $f$ est la densité de probabilité d'une
variable aléatoire $Y$, dont on déterminera la fonction de répartition
$F.$

\item Les statistiques des ventes de la société permettent de
considérer
dans toute la suite du problème que la variable aléatoire $Y$
représente la
demande mensuelle, en millions de litres d'essence.\\
Déterminer la valeur du moment d'ordre $p$ de la variable aléatoire
$Y.$ En déduire la demande mensuelle en litre que la société ALFDIS
peut espérer, et
avec quel écart-type (les valeurs seront arrondies au $100$ litres près
au
mieux).
\end{noliste}

\item Les services de gestion de la société ALFDIS indiquent que pour
un
niveau $s$ de stock d'essence fixé et réalisé en début de mois ($s$ en
millions de litres), le coût de gestion mensuel est une variable
$g_{s}$ dépendant du nombre aléatoire $t$ de millions de litres
d'essence demandés, et
dont la valeur en millions de francs est :
\[
g_{s}(t) = \left\{ 
\begin{array}{ccc}
0 & \text{si} & t<0 \\
3.(s-t) & \text{si} & 0\leq t\leq s \\
2.(t-s) & \text{si} & t>s
\end{array}
\right. 
\]
\\
On admet que l'espérance du coût mensuel est définie par $ :\E(g_{s}) =
\dint{-\infty }{+ \infty }g_{s}(t)f(t)dt.$

\begin{noliste}{a)}
 \setlength{\itemsep}{2mm}
\item Exprimer $\dint{s}{+ \infty }t.f(t)dt$ en fonction de $\E(Y)$ et
de $I_{2}(s)$ pour $s\geq 0.$

\item En déduire que l'on a $ :\E(g_{s}) = 5s.F(s) +
2\E(Y)-2s-20I_{2}(s).$

\item Déterminer alors l'expression de $\E(g_{s})$ en fonction de $s.$
\end{noliste}

\item On considère la fonction $\varphi $ définie, pour tout $s$
positif ou
nul, par :
\[
\varphi (s) = 5.(s + 1)e^{-2s} + 3.(s-1).
\]

\begin{noliste}{a)}
 \setlength{\itemsep}{2mm}
\item Déterminer $\varphi ^{\prime }(s)$ et $\varphi "(s).$

\item En déduire que l'équation $\varphi ^{\prime }(s) = 0$ admet une
unique
solution $s_{0}.$ Donner l'entier naturel $q$ tel que l'on ait :
\[
\dfrac{q}{10^{3}}<s_{0}<\dfrac{q + 1}{10^{3}}.
\]

\item Construire le tableau de variation de la fonction $\varphi.$

\item Conclure que l'espérance du coût mensuel $\E(g_{s})$ admet pour
valeur
minimale le nombre réel :\begin{gather*}
\dfrac{6s_{0}{2}}{2s_{0} + 1}. \\
\underline{\hspace{6cm}}
\end{gather*}
\end{noliste}
\end{noliste}

Table donnant certaines valeurs des probabilités $P\left(\Ev{X =
n}\right)$ et $p_{n} = P\left(\Ev{X\leq n}\right),$ si $X$ suit une loi
de Poisson de paramètre $5$

\[
\begin{tabular}{|c|c|c|c|c|c|c|c|c|}
\hline
$n$ & $0$ & $1$ & $2$ & $3$ & $4$ & $5$ & $6$ & $7$ \\
\hline
$P\left(\Ev{X = n}\right)$ & $0,0067379$ & $0,0336897$ & $0,0842243$ &
$0,1403739$ & $0,1754674 $ & $0,1754674$ & $0,1462228$ & $0,1044449$ \\
\hline
$p_{n}$ & $0,0067379$ & $0,0404277$ & $0,1246520$ & $0,2650259$ &
$0,4404933$
 & $0,6159607$ & $0,7621835$ & $0,8666283$ \\
\hline
\end{tabular}
\]

\label{fin}

\end{document}

