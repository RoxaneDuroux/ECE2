\documentclass[11pt]{article}%
\usepackage{geometry}%
\geometry{a4paper,
 lmargin = 2cm,rmargin = 2cm,tmargin = 2.5cm,bmargin = 2.5cm}

\input{../../macros.tex}

\pagestyle{fancy} %
\lhead{ECE2 \hfill Mathématiques\\
} %
\chead{\hrule} %
\rhead{} %
\lfoot{} %
\cfoot{} %
\rfoot{\thepage} %

\renewcommand{\headrulewidth}{0pt}% : Trace un trait de séparation
 % de largeur 0,4 point. Mettre 0pt
 % pour supprimer le trait.

\renewcommand{\footrulewidth}{0.4pt}% : Trace un trait de séparation
 % de largeur 0,4 point. Mettre 0pt
 % pour supprimer le trait.

\setlength{\headheight}{14pt}

\title{\bf \vspace{-2cm} ECRICOME 1988} %
\author{} %
\date{} %
\begin{document}

\maketitle %
\vspace{-1.4cm}\hrule %
\thispagestyle{fancy}

\vspace*{.2cm}


% DEBUT DU DOC À MODIFIER : tout virer jusqu'au début de l'exo

%Définition et changement de valeurs de
compteurs%newcounter{cpt1}{section} compteur cpt1 remis à 0 à chaque
aumentation par stepcounter du compteur section%setcounter{cpt1}{3} on
met le compteur à 3%addtocounter{cpt1}{5} on ajoute 5 au compteur%
stepcounter{cpt1} on ajoute 1% ifthenelse{test}{alors}{sinon} (page
206) pour subordonner à une condition % whiledo{test}{commande} pour
faire une boucle (page 206 aussi) % value{cpt1} pour noter dans le
document la valeur de cpt1 
%Définition définitive d'opérateurs
mathématiques\newcommand{\ch}{\operatorname{ch}} 
\newcommand{\sh}{\operatorname{sh}}
\renewcommand{\tanh}{\operatorname{th}}
\renewcommand{\sinh}{\operatorname{sh}}
\renewcommand{\cosh}{\operatorname{ch}}
\newcommand{\argsh}{\operatorname{argsh}}
\newcommand{\argch}{\operatorname{argch}}
\newcommand{\argth}{\operatorname{argth}}
\newcommand{\Id}{\operatorname{Id}}
\renewcommand{\leq}{\leq}
\renewcommand{\geq}{\geq }

\newcommand{\dlim}{\lim}
\newcommand{\dsum}{\sum}
\newcommand{\dint}{\int}
\newcommand{\dprod}{\prod}



%Définition de nouvelles couleurs : rgb(trois paramètres red green blue
entre 0 et 1); cmyk (quatre cyan magenta yellow black) entre 0 et 1;
gray (entre 0 et 1) et black, white, red, green, blue, cyan, magenta,
yellow% definecolor{0gris}{gray}{0.8} 
% Nouvelle commande pour encadrer le titre car shabox ne veut que d'une
seule ligne; ATTENTION A LA TAILLE; petite différence avec shadowbox ou
doublebox, voire fcolorbox ou colorbox (au lieu de shabox; laisser le
parbox tranquille sauf pour la taille de la boîte
\newcommand{\Tbox}[1]{\begin{center} \shabox{\parbox{0.6
\linewidth}{#1}} \end{center}} %[1] pour 1 paramètre ; #1 pour ce que
fait le 1er paramètre; entre accolades ce que fait la commande
%Mise en page en mode fancy : en-têtes et pieds de pages puis
définition des en-têtes et pieds de pages\pagestyle{fancy}
\lhead{ECE 2 - Mathématiques \\
Quentin Dunstetter - ENC-Bessières 2011$\backslash$2012}
\chead{}
\rhead{Ecricome 1988}
\rfoot[ \ \thepage]{\thepage}
\cfoot{}
\lfoot{}
\thispagestyle{fancy} %Mise en page de la 1ère page en mode fancy
%Trait en bas et en haut de la page (entre en-tête et texte et texte et
pied de page)\renewcommand{\footrulewidth}{0.4pt}
\renewcommand{\headrulewidth}{0.4pt}


\begin{center}
{\Huge ECRICOME Eco 1988}
\end{center}

\begin{center}
\textbf{EXERCICE DE MATHEMATIQUES}
\end{center}

Soit $E$ un espace vectoriel rapporté à une base $B =
(e_{1},e_{2},e_{3});$ on
considère l'endomorphisme $f$ de $E$ dont la matrice relativement à la
base $B$ est :
\[
A = 
\begin{smatrix}
2 & 0 & 0 \\
1 & 3 & -1 \\
1 & 1 & 1
\end{smatrix}.
\]

\begin{noliste}{1.}
 \setlength{\itemsep}{4mm}
\item Vérifier que $A$ n'admet comme valeur propre que le réel $\lambda
= 2.$

\item Déterminer l'ensemble des vecteurs propres associées à $\lambda =
2.$

\item On pose $\left\{ 
\begin{array}{cccccc}
e_{1}{\prime } = & & & e_{2} & + & e_{3} \\
e_{2}{\prime } = & e_{1} & & & + & e_{3} \\
e_{1}{\prime } = & e_{1} & + & e_{2} & & 
\end{array}
\right..$\\
Montrer que la matrice $A^{\prime }$ de $f$ relativement à la nouvelle
base $B^{\prime } = (e_{1}{\prime },e_{2}{\prime },e_{3}{\prime })$ est
:
\[
A^{\prime } = 
\begin{smatrix}
2 & 0 & 2 \\
0 & 2 & 0 \\
0 & 0 & 2
\end{smatrix}.
\]

\item On écrit $A^{\prime } = 2I + J$ avec $I = 
\begin{smatrix}
1 & 0 & 0 \\
0 & 1 & 0 \\
0 & 0 & 1
\end{smatrix}
$ et $J = 
\begin{smatrix}
0 & 0 & 2 \\
0 & 0 & 0 \\
0 & 0 & 0
\end{smatrix}.$\\
Pour $n\in \N,$ calculer $A^{\prime n}$ puis $A^{n}.$

\item \underline{Applications} : déterminer en fonction de $n$ les
valeurs
de $x_{n},y_{n},z_{n}$ définis par 
\[
x_{0} = y_{0} = z_{0} = 1\quad \left\{ 
\begin{array}{l}
x_{n + 1} = 2x_{n} \\
y_{n + 1} = x_{n} + 3y_{n}-z_{n} \\
z_{n + 1} = x_{n} + y_{n} + z_{n}
\end{array}
\right. 
\]
\end{noliste}

\begin{center}
\textbf{EXERCICE DE STATISTIQUES-PROBABILITES}
\end{center}

Dans un hôpital de la région parisienne, le nombre d'admis dans le
service
des urgences, au cours du samedi est une variable aléatoire $X$ de
distribution
\[
\begin{tabular}{c|c|c|c|c|c|c}
$x_{i}$ & $0$ & $1$ & $2$ & $3$ & $4$ & $5$ \\
\hline
$P\left(\Ev{X = x_{i}}\right)$ & $0,22$ & $0,33$ & $0,25$ & $0,13$ &
$0,05$ & $0,02$\end{tabular}
\]
La probabilité que la personne admise soit un homme est $0,3.$ Soit $Y$
la
variable aléatoire "nombre d'hommes admis dans le service des urgences
au
cours de la nuit du samedi".

\begin{noliste}{1.}
 \setlength{\itemsep}{4mm}
\item Quelle est la loi conditionnelle de $Y$ pour $X = x_{i}$ ?

\item En déduire $P\left(\Ev{Y = 4}\right)$ à $10^{-4}$ près.

\item Si, un samedi donné, il n'y a que trois lits disponibles pour les
hommes et deux pour les femmes, quelle est la probabilité de refuser un
ou
plusieurs patients hommes ou femmes ?

\item Déterminer la loi de $Y$ en supposant que $X$ suit une loi de
Poisson
de paramètre $\lambda = 1,5.$
\end{noliste}

\begin{center}
\textbf{Problème}
\end{center}

Les parties $A$ et $B$ sont largement indépendantes.

Ce problème étudie, à partir de différentes hypothèses, la probabilité
pour
qu'une personne donnée soit retenue à la suite d'une offre d'emploi.

La notation $\ln $ désigne le logarithme népérien.

\section*{Première Partie -A}

Une entreprise souhaite recruter un responsable pour son service
commercial.
Un certain nombre de personnes se présentent à la suite de cette offre.
L'ordre des convocations pour l'entretien se fait au hasard. Pour
diverses
raisons, il a décidé que le premier candidat qui, au cours des ces
entretiens, présenterait les qualités requises, serait immédiatement
embauché.

Le nombre de personnes candidates à ce poste, convoquées et ayant les
qualités exigées est une variable aléatoire $X.$

On considère que les ordres de passage de ces candidats "embauchables"
devant la commission de recrutement sont équiprobables.

Enfin, dans tout le problème, on nomme $A$ une personne ayant toutes
les
qualités pour être embauchée, et faisant partie des personnes
convoquées.

\begin{noliste}{1.}
 \setlength{\itemsep}{4mm}
\item On suppose dans cette question que $X$ peut prendre comme valeur
tout
entier $k$ vérifiant $1\leq k\leq N,$ $N$ étant un entier fixé,
toutes ces valeurs étant équiprobables.

\begin{noliste}{a)}
 \setlength{\itemsep}{2mm}
\item Montrer que la probabilité $\alpha_{N}$ pour que $A$ soit
embauchée
est donnée par 
\[
\alpha_{N} = \left( 1 + \dfrac{1}{2} + \dfrac{1}{3} + \cdots +
\dfrac{1}{N}\right)
\times \dfrac{1}{N}.
\]

\item Utiliser, après l'avoir justifiée, l'inégalité
\[
\forall k\in \N^{\times },\quad \dfrac{1}{k + 1}\leq
\dint{k}{k + 1}\dfrac{dt}{t}\leq \dfrac{1}{k}
\]
pour obtenir un encadrement de $\alpha_{N}.$\\
En déduire la limite de $\alpha_{N}$ lorsque $N\rightarrow + \infty.$
\end{noliste}

\item On suppose dans cette question que $X$ peut encore prendre comme
valeur tout entier $k$ vérifiant $1\leq k\leq N,$ sa loi étant définie
par 
\[
\forall k\in \lbrack \hspace{-0.15em}[1,N]\hspace{-0.13em}],\quad
P\left(\Ev{X = k}\right) = \dfrac{C}{k^{r}}
\]
où $r$ est un réel supérieur ou égale à $1,$ et $C$ une constante
réelle.

\begin{noliste}{a)}
 \setlength{\itemsep}{2mm}
\item A quelle condition doit satisfaire la constante $C$ ?

\item Montrer que la probabilité $\beta_{N}$ pour que $A$ soit
embauchée
est donnée par :
\[
\beta_{N} = \dfrac{1 + \dfrac{1}{2^{r + 1}} + \cdots + \dfrac{1}{N^{r +
1}}}{1 + \dfrac{1}{2^{r}} + \cdots + \dfrac{1}{N^{r}}}.
\]

\item On considère la suite $(u_{N}),$ $N\in \N^{\times }$ définie
par $u_{N} = \Sum{k = 1}{N}\dfrac{1}{k^{r}}.$\\
En utilisant, comme en 1.b), l'inégalité pour $k\geq 1,$
\[
\dfrac{1}{(k + 1)^{r}}\leq \dint{k}{k + 1}\dfrac{dx}{x^{r}}\leq
\dfrac{1}{k^{r}}
\]
discuter, selon la valeur de $r,$ l'existence d'une limite pour $u_{N}$
quant $N$ tend vers $ + \infty.$\\
En déduire le comportement de $\beta_{N}$ quand $N$ tend vers $ +
\infty.$
\end{noliste}

\item On suppose dans cette question que le nombre $X-1$ de personnes
autres
que $A$ suit une loi de Poisson de paramètre $\lambda.$

\begin{noliste}{a)}
 \setlength{\itemsep}{2mm}
\item Exprimer la probabilité pour que $A$ soit embauchée.

\item On considère la fonction définie sur $\R^{\times }$ par 
\[
\varphi (\lambda ) = \dfrac{1-e^{-\lambda }}{\lambda }.
\]

\begin{nonoliste}{(i)}
\item Montrer que $\varphi $ se prolonge en une fonction continue
$\widetilde{\varphi }$ sur $\R.$

\item Montrer que $\varphi $ est dérivable sur $\R^{\times }$ et
calculer $\varphi ^{\prime }.$ \\
Étudier la dérivabilité de $\widetilde{\varphi }{\prime }$ en $0.$

\item Soit $h$ la fonction définie sur $\R$ par $h(t) = e^{t}-1-t.$\\
Montrer que $ :\forall t\in \R,\quad h(t)\geq 0.$ Résoudre l'équation
$h(t) = 0.$

\item Étudier le signe de $\varphi ^{\prime },$ puis dresser le tableau
de
variations de $\widetilde{\varphi }.$\\
Construire la courbe représentative de $\widetilde{\varphi }.$
\end{nonoliste}
\end{noliste}
\end{noliste}

\section*{Deuxième partie -B}

Cette seconde partie précise les résultats obtenus en $A$ en étudiant
notamment le comportement asymptotique de $\alpha_{N}$ quand $N$ tend
vers $ + \infty,$ et celui de $\beta_{N}$ dans le cas $r = 2.$

\begin{noliste}{1.}
 \setlength{\itemsep}{4mm}
\item 

\begin{noliste}{a)}
 \setlength{\itemsep}{2mm}
\item Soit $(x_{N})$ et $(y_{N})$ les suites définies par 
\begin{eqnarray*}
x_{N} & = & 1 + \dfrac{1}{2} + \cdots + \dfrac{1}{N}-\ln N\text{ où
}N\in \N^{\times } \\
y_{N} & = & 1 + \dfrac{1}{2} + \cdots + \dfrac{1}{N}-\ln (N + 1)\text{
où }N\in 
\N^{\times }
\end{eqnarray*}Montrer que $(x_{N})$ et $(y_{N})$ sont monotones. \\
En déduire leur convergence vers une même limite $L.$

\item Montrer que $ :\forall N\in \N^{\times },\quad
0<L-y_{N}<\dfrac{1}{N}.$\\
En déduire une valeur approchée à $10^{-1}$ près.

\item Montrer alors que la probabilité $\alpha_{N}$ peut s'écrire sous
la
forme 
\[
\alpha_{N} = \dfrac{\ln N}{N} + \dfrac{L}{N} + \dfrac{\varepsilon
(N)}{N}
\]
où $\varepsilon (N)$ vérifie $\dlim{N\rightarrow + \infty }\varepsilon
(N) = 0.$
\end{noliste}

\item Cette question étudie le comportement de $\beta_{N},$ dans le cas
$r = 2,$ lorsque $N$ tend vers $ + \infty.$

\begin{noliste}{a)}
 \setlength{\itemsep}{2mm}
\item Étude de $l_{2} = \dlim{N\rightarrow + \infty }\left( 1 +
\dfrac{1}{2^{2}} + \cdots + \dfrac{1}{N^{2}}\right).$

\begin{nonoliste}{(i)}
\item On note $s_{N} = \Sum{k = 1}{N}\dfrac{1}{k^{2}}$ et $r_{N + 1} =
l_{2}-s_{N} = \Sum{k = n + 1}{+ \infty }\dfrac{1}{k^{2}}.$\\
Établir que $0\leq r_{N + 1}\leq \dint{N}{+ \infty
}\dfrac{dt}{t^{2}}.$\\
Déterminer une valeur $N_{1}$ de l'entier $N$ à partir de laquelle
$s_{N}$
est une valeur approchée de $l_{2}$ à $10^{-3}$ près.

\item Établir que $ :\dint{N + 1}{+ \infty }\dfrac{dt}{t^{2}}\leq
r_{N + 1}\leq \dint{N}{+ \infty }\dfrac{dt}{t^{2}}.$\\
En déduire que le réel $s_{N}{\prime }$ défini par 
\[
s_{N}{\prime } = s_{N} + \dfrac{1}{2}\left( \dint{N + 1}{+ \infty
}\dfrac{dt}{t^{2}} + \dint{N}{+ \infty }\dfrac{dt}{t^{2}}\right) 
\]
est une valeur approchée de $l_{2}$ telle que 
\[
\left| s_{n}{\prime }-l_{2}\right| \leq \dfrac{1}{2}\dint{N}{N +
1}\dfrac{dt}{t^{2}}.
\]
Déterminer une valeur $N_{2}$ de l'entier $N$ à partir de laquelle
$s_{N}{\prime }$ est une valeur approchée de $l_{2}$ à $10^{-3}$ près.

\item Calculer $s_{N_{2}}{\prime }.$\\
On précisera un algorithme simple permettant d'obtenir $s_{N}{\prime }$
pour une valeur de $N$ que cet algorithme laisserait au choix de
l'utilisateur.
\end{nonoliste}

\item \underline{Accélération de convergence} :

\begin{nonoliste}{(i)}
\item On pose $u_{k} = \dfrac{1}{k(k + 1)}.$ \\
En décomposant $u_{k}$ sous la forme $\dfrac{a}{k} + \dfrac{b}{k + 1},$
calculer 
$U_{N} = \Sum{k = 1}{N}u_{k},$ puis $U = \dlim{N\rightarrow
 + \infty }U_{N}.$

\item On pose, pour tout $k\in \N^{\times },$ $\dfrac{1}{k^{2}} = u_{k}
+ u_{k}{\prime }$ puis $s_{N} = U_{N} + U_{N}{\prime }$ où
$U_{N}{\prime
} = \Sum{k = 1}{N}u_{k}{\prime }.$\\
Quelle est la limite $U^{\prime }$ de $U_{N}{\prime },$ quand $N$ tend
vers 
$ + \infty $ ? \\
Exprimer $u_{k}{\prime }$ en fonction de $k.$\\
Après avoir étudié la monotonie sur $\R^{\times }$ de la fonction
$t\mapsto \dfrac{1}{t^{2}(t + 1)},$ justifier :
\[
\dint{k}{k + 1}\dfrac{dt}{t^{2}(t + 1)}\leq u_{k}{\prime
}\leq \dint{k-1}{k}\dfrac{dt}{t^{2}(t + 1)}
\]
puis
\[
\dint{N + 1}{N + p + 1}\dfrac{dt}{t^{2}(t + 1)}\leq
\Sum{k = N + 1}{N + p}u_{k}{\prime }\leq \dint{N}{N +
p}\dfrac{dt}{t^{2}(t + 1)}
\]
pour tout entier $p>0.$ En déduire comme en 2(a)ii. que 
\[
\widetilde{U}_{N} = U_{N}{\prime } + \dfrac{1}{2}\left(
\dint{N + 1}{+ \infty }\dfrac{dt}{t^{2}(t + 1)} + \dint{N}{+ \infty
}\dfrac{dt}{t^{2}(t + 1)}\right) 
\]
est une valeur approchée de $U^{\prime }$ telle que $\left| U^{\prime
}-\widetilde{U}_{N}\right| \leq \dfrac{1}{2}\dint{N}{N +
1}\dfrac{dt}{t^{2}(t + 1)}.$

\item Montrer que $\left| U^{\prime }-\widetilde{U}_{N}\right|
\leq \dfrac{1}{2}\dfrac{1}{N^{2}(N + 1)}.$\\
Déterminer une valeur de $N_{3}$ de l'entier $N$ à partir de laquelle
$\widetilde{U}_{N}$ est une valeur approchée de $U^{\prime }$ à
$10^{-3}$ près.
\end{nonoliste}

\item 

\begin{nonoliste}{(i)}
\item Déterminer $a,b,c$ trois constantes réelles telles que 
\[
\forall t\in \R^{\times },\quad \dfrac{1}{t^{2}(t + 1)} =
\dfrac{a}{t^{2}} + \dfrac{b}{t} + \dfrac{c}{t + 1}.
\]
En déduire l'expression de $\widetilde{U}_{N}$ en fonction de $N.$

\item Calculer $\widetilde{U}_{N_{3}},$ puis la valeur approchée de
$l_{2}$
qui s'en déduit.

\item En admettant qu'une valeur approchée à $10^{-3}$ près de 
\[
l_{3} = \dlim{N\rightarrow + \infty }\left( 1 + \dfrac{1}{2^{3}} +
\cdots + \dfrac{1}{N^{3}}\right) 
\]
soit $l_{3}\simeq 1,202,$ préciser la limite $\beta_{N}$ dans le cas $r
= 2.$
\end{nonoliste}
\end{noliste}
\end{noliste}

\label{fin}

\end{document}

