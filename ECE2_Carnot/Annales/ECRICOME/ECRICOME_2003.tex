\documentclass[11pt]{article}%
\usepackage{geometry}%
\geometry{a4paper,
 lmargin = 2cm,rmargin = 2cm,tmargin = 2.5cm,bmargin = 2.5cm}

\input{../../macros.tex}

\pagestyle{fancy} %
\lhead{ECE2 \hfill Mathématiques\\
} %
\chead{\hrule} %
\rhead{} %
\lfoot{} %
\cfoot{} %
\rfoot{\thepage} %

\renewcommand{\headrulewidth}{0pt}% : Trace un trait de séparation
 % de largeur 0,4 point. Mettre 0pt
 % pour supprimer le trait.

\renewcommand{\footrulewidth}{0.4pt}% : Trace un trait de séparation
 % de largeur 0,4 point. Mettre 0pt
 % pour supprimer le trait.

\setlength{\headheight}{14pt}

\title{\bf \vspace{-2cm} ECRICOME 2003} %
\author{} %
\date{} %
\begin{document}

\maketitle %
\vspace{-1.4cm}\hrule %
\thispagestyle{fancy}

\vspace*{.2cm}


% DEBUT DU DOC À MODIFIER : tout virer jusqu'au début de l'exo

%Définition et changement de valeurs de
compteurs%newcounter{cpt1}{section} compteur cpt1 remis à 0 à chaque
aumentation par stepcounter du compteur section%setcounter{cpt1}{3} on
met le compteur à 3%addtocounter{cpt1}{5} on ajoute 5 au compteur%
stepcounter{cpt1} on ajoute 1% ifthenelse{test}{alors}{sinon} (page
206) pour subordonner à une condition % whiledo{test}{commande} pour
faire une boucle (page 206 aussi) % value{cpt1} pour noter dans le
document la valeur de cpt1 
%Définition définitive d'opérateurs
mathématiques\newcommand{\ch}{\operatorname{ch}} 
\newcommand{\sh}{\operatorname{sh}}
\renewcommand{\tanh}{\operatorname{th}}
\renewcommand{\sinh}{\operatorname{sh}}
\renewcommand{\cosh}{\operatorname{ch}}
\newcommand{\argsh}{\operatorname{argsh}}
\newcommand{\argch}{\operatorname{argch}}
\newcommand{\argth}{\operatorname{argth}}
\newcommand{\Id}{\operatorname{Id}}
\renewcommand{\leq}{\leq}
\renewcommand{\geq}{\geq }

%Définition de nouvelles couleurs : rgb(trois paramètres red green blue
entre 0 et 1); cmyk (quatre cyan magenta yellow black) entre 0 et 1;
gray (entre 0 et 1) et black, white, red, green, blue, cyan, magenta,
yellow% definecolor{0gris}{gray}{0.8} 
% Nouvelle commande pour encadrer le titre car shabox ne veut que d'une
seule ligne; ATTENTION A LA TAILLE; petite différence avec shadowbox ou
doublebox, voire fcolorbox ou colorbox (au lieu de shabox; laisser le
parbox tranquille sauf pour la taille de la boîte
\newcommand{\Tbox}[1]{\begin{center} \shabox{\parbox{0.6
\linewidth}{#1}} \end{center}} %[1] pour 1 paramètre ; #1 pour ce que
fait le 1er paramètre; entre accolades ce que fait la commande
%Mise en page en mode fancy : en-têtes et pieds de pages puis
définition des en-têtes et pieds de pages\pagestyle{fancy}
\lhead{ECE 2 - Mathématiques \\
Quentin Dunstetter - ENC-Bessières 2011$\backslash$2012}
\chead{}
\rhead{Epreuve Ecricome 2003}
\rfoot[ \ \thepage]{\thepage}
\cfoot{}
\lfoot{}
\thispagestyle{fancy} %Mise en page de la 1ère page en mode fancy
%Trait en bas et en haut de la page (entre en-tête et texte et texte et
pied de page)\renewcommand{\footrulewidth}{0.4pt}
\renewcommand{\headrulewidth}{0.4pt}


%DEBUT DU DOCUMENT

\noindent {\Large \textbf{ECRI%TCIMACRO{\TeXButton{TeX
field}{\colorbox[gray]{0.95}{COME}}}%BeginExpansion
\colorbox[gray]{0.95}{COME}%EndExpansion
}}\vspace{0.3cm}

\noindent \textbf{Banque d'épreuves communes}

\noindent aux concours des Ecoles

\noindent esc%TCIMACRO{\TeXButton{TeX
field}{\colorbox[gray]{0.95}{bordeaux}} }%BeginExpansion
\colorbox[gray]{0.95}{bordeaux}
%EndExpansion
/ esc%TCIMACRO{\TeXButton{TeX field}{\colorbox[gray]{0.95}{marseille}}
}%BeginExpansion
\colorbox[gray]{0.95}{marseille}
%EndExpansion
/ icn%TCIMACRO{\TeXButton{TeX field}{\colorbox[gray]{0.95}{nancy}}
}%BeginExpansion
\colorbox[gray]{0.95}{nancy}
%EndExpansion
/ esc%TCIMACRO{\TeXButton{TeX field}{\colorbox[gray]{0.95}{reims}}
}%BeginExpansion
\colorbox[gray]{0.95}{reims}
%EndExpansion
/ esc%TCIMACRO{\TeXButton{TeX field}{\colorbox[gray]{0.95}{rouen}}
}%BeginExpansion
\colorbox[gray]{0.95}{rouen}
%EndExpansion
/ esc%TCIMACRO{\TeXButton{TeX
field}{\colorbox[gray]{0.95}{toulouse}}}%BeginExpansion
\colorbox[gray]{0.95}{toulouse}%EndExpansion
\vspace{1cm}

\begin{center}
{\large CONCOURS D'ADMISSION }\vspace{0.5cm}

\textbf{option economique} \vspace{0.5cm}

{\Large \textbf{MATHÉMATIQUES}} \vspace{0.5cm}

\textbf{Année 2003}
\end{center}

\noindent \textbf{Aucun instrument de calcul n'est autorisé.}

\noindent \textbf{Aucun document n'est autorisé.}

\noindent L'énoncé comporte \pageref{fin} pages

\begin{quotation}
\noindent Les candidats sont invités à soigner la présentation de leur
copie, à mettre en évidence les principaux résultats, à respecter les
notations de l'énoncé, et à donner des démonstrations complètes (mais
brèves) de leurs affirmations.
\end{quotation}

\vspace{13cm}

\hfill \textbf{Tournez la page}

\hfill \textbf{S.V.P\qquad }

\newpage

\section*{Exercice 1}

On considère l'espace vectoriel $E = \R^{3}$ et $f$ l'endomorphisme de 
$E$ dont la matrice dans la base canonique $\mathcal{B} = \left( 
\overrightarrow{e_{1}},\overrightarrow{e_{2}},\overrightarrow{e_{3}}\ri
ht) $
est la matrice $A$ :

\[
A = \left( 
\begin{array}{ccc}
3 & -2 & 3 \\
1 & 0 & 2 \\
0 & 0 & 2
\end{array}
\right)
\]

\noindent \textbf{1. Calcul des puissances de }$A$

\begin{noliste}{1.}
 \setlength{\itemsep}{4mm}
\item Déterminer les valeurs propres $\lambda_{1}$ et $\lambda_{2}$ de
l'endomorphisme $f$, avec $\lambda_{1}<\lambda_{2}$.

\item La matrice $A$ est-elle inversible ? (On ne demande pas la
matrice $A^{-1}$).

\item Déterminer une base et la dimension de chacun des sous-espaces
propres
de $f$.

\item Justifier que $f$ n'est pas diagonalisable.

\item Déterminer le vecteur $\overrightarrow{u_{1}}$ de $E$ vérifiant :

\begin{noliste}{$\sbullet$}
\item $\overrightarrow{u_{1}}$ est un vecteur propre de $f$ associé à
la
valeur propre $\lambda_{1}$.

\item la première composante de $\overrightarrow{u_{1}}$ est 1.
\end{noliste}

\item Déterminer le vecteur $\overrightarrow{u_{2}}$ de $E$ vérifiant :

\begin{noliste}{$\sbullet$}
\item $\overrightarrow{u_{2}}$ est un vecteur propre de $f$ associe à
la
valeur propre $\lambda_{2}$.

\item la deuxième composante de $\overrightarrow{u_{2}}$ est 1.
\end{noliste}

\item Soit $\overrightarrow{u_{3}} = (1,1,1)$. Montrer que $\mathcal{C}
= \left( 
\overrightarrow{u_{1}},\overrightarrow{u_{2}},\overrightarrow{u_{3}}\ri
ht) $
est une base de $E$.

\item Déterminer la matrice de passage $P$ de la la base $\mathcal{B}$
à
la base $\mathcal{C}$ puis la matrice de passage de la base
$\mathcal{C}$ à
la base $\mathcal{B}$.

\item Montrer que : $f\left( \overrightarrow{u_{3}}\right) =
\overrightarrow{u_{2}} + 2\overrightarrow{u_{3}}$.

\item En déduire que la matrice de $f$ dans la base $\mathcal{C}$ est
la
matrice : 
\[
T = \left( 
\begin{array}{lll}
1 & 0 & 0 \\
0 & 2 & 1 \\
0 & 0 & 2
\end{array}
\right)
\]

\item Rappeler la relation matricielle entre $A$ et $T$.

\item Prouver que pour tout élément $n$ de $\N^{\times }$ il existe
un réel $\alpha_{n}$ tel que : 
\[
T^{n} = \left( 
\begin{array}{lll}
1 & 0 & 0 \\
0 & 2^{n} & \alpha_{n} \\
0 & 0 & 2^{n}
\end{array}
\right)
\]
On donnera le réel $\alpha_{1}$ ainsi qu'une relation entre $\alpha_{n
+ 1}$
et $\alpha_{n}$.

\item Montrer que : 
\[
\forall n\in \N^{\times },\;\alpha_{n} = n\ 2^{n-1}
\]
En déduire l'écriture matricielle de $A^{n}$ en fonction de $n$.
\end{noliste}

\noindent \textbf{2. Matrices commutant avec }$A$\textbf{.}

\noindent $\mathfrak{M}_{3}\left( \R\right) $ désignant l'ensemble
des matrices carrées d'ordre 3, on considère le sous-ensemble $C\left(
A\right) $ de $\mathfrak{M}_{3}\left( \R\right) $ des matrices $M$
telles que : 
\[
AM = MA
\]

\begin{noliste}{1.}
 \setlength{\itemsep}{4mm}
\item Montrer que $C(A)$ est un sous-espace vectoriel de
$\mathfrak{M}_{3}\left( \R\right) $.

\item Pour $M$ appartenant à $\mathfrak{M}_{3}\left( \R\right) $ on
pose $M^{\prime } = P^{-1}MP.$

Montrer que : 
\[
AM = MA\Longleftrightarrow TM^{\prime } = M^{\prime }T
\]
($T$ est définie dans la question \textbf{1}.10)

\item Montrer qu'une matrice $M^{\prime }$ de $\mathfrak{M}_{3}\left( 
\R\right) $) vérifie $TM^{\prime } = M^{\prime }T$ si et seulement si
$M^{\prime }$ est de la forme $\left( 
\begin{array}{lll}
a & 0 & 0 \\
0 & b & c \\
0 & 0 & b
\end{array}
\right) $ où $a,$ $b$, $c$ sont trois réels.

\item En déduire que $M$ appartient à $C(A)$ si et seulement si il
existe
des réels $a,b,c$ tels que : 
\[
M = \left( 
\begin{array}{ccc}
-a + 2b & 2a-2b & -a + b + 2c \\
-a + b & 2a-b & -a + b + c \\
0 & 0 & b
\end{array}
\right)
\]

\item Déterminer alors une base de $C(A)$ ainsi que la dimension de
$C(A)$.
\end{noliste}

\section*{\protect\LARGE Exercice 2}

On considère les fonctions $ch$ et $sh$ définies sur $\R$ par :

\[
ch\left( x\right) = \dfrac{e^{x} + e^{-x}}{2} \text{ \quad et \quad
}sh\left( x\right) = \dfrac{e^{x}-e^{-x}}{2}
\]
ainsi que la fonction $f$ définie sur $\R$ par : 
\[
\left\{ 
\begin{array}{l}
f\left( x\right) = \dfrac{x}{sh\left( x\right) }\text{ si }x\neq 0 \\
f\left( 0\right) = 1
\end{array}
\right.
\]
On s'intéresse dans cet exercice à la convergence de la suite $\left(
u_{n}\right)_{n\in \N}$ définie par la relation de récurrence : 
\[
\left\{ 
\begin{array}{l}
u_{0} = 1 \\
\forall n\in \mathbb{N\;\;}u_{n + 1} = f\left( u_{n}\right)
\end{array}
\right.
\]

\noindent \textbf{1. Étude des fonctions }$ch$\textbf{, }$sh$\textbf{,
et }$f $\textbf{.}

\begin{noliste}{1.}
 \setlength{\itemsep}{4mm}
\item Étudier la parité des fonctions $ch$ et $sh$.

\item Dresser le tableau de variations de la fonction $sh$, puis en
déduire
le signe de $sh\left( x\right) $ pour $x$ appartenant à $\R$.

\item Déterminer un équivalent en $ + \infty $ de $sh(x)$. En déduire
l'allure
de la courbe représentative de la fonction $sh$ en $ + \infty $.

\item Montrer que la fonction $sh$ réalise une bijection de $\R$
dans $\R$.

\item Étudier les variations de la fonction $ch$.

\item Montrer que : 
\[
\forall x\in \R,\;\;ch\left( x\right) >sh\left( x\right)
\]

\item Donner sur un même graphique l'allure des courbes représentatives
des
fonctions $ch$ et $sh$.

\item Étudier la parité de la fonction $f$.

\item Déterminer le développement limité d'ordre 3 en 0 de la fontion
$sh.$

\item En déduire que la fonction $f$ est continue en $0,$ dérivable en
0 et déterminer $f^{\prime }\left( 0\right) $.

\item Justifier que $f$ est dérivable sur $\R_{+}{\times }$ et sur
$\R_{-}{* }$ et calculer $f^{\prime }\left( x\right) $ pour $x\in \R^{*
}$.

\item On pose : 
\[
\forall x\in \R^{+},\;\;h\left( x\right) = shx-xch\left( x\right)
\]
Étudier les variations de $h$, puis en déduire le signe de $h\left(
x\right) $.

\item Déterminer les variations de $f$ sur $\R^{+}$ et donner
l'allure de la courbe représentative de la fonction $f$. (On ne
cherchera
pas les points d'inflexion).
\end{noliste}

\noindent \textbf{2. Étude de la suite }$\left( u_{n}\right)_{n\in
\N}$\textbf{.}

\noindent On donne :

\begin{eqnarray*}
f\left( 0.8\right) & \simeq & 0.9,\;\;f(1)\simeq 0.85, \\
sh(0.6) & \simeq & 0.64,\;\;sh(0.8)\simeq 0.89,\;\;sh(1)\simeq
1.18,\;\;sh(1.2)\simeq 1.51
\end{eqnarray*}

\begin{noliste}{1.}
 \setlength{\itemsep}{4mm}
\item Justifier que $f\left( \left[ 0.8,1\right] \right) \subset \left[
0.8,1\right] $, puis que : 
\[
\forall n\in \N,\;\;u_{n}\in \left[ 0.8,1\right]
\]

\item Montrer que l'équation $f\left( x\right) = x$ admet une unique
solution 
$\alpha $ sur $\R$ (on pourra utiliser la question \textbf{1.4, }
sans cherche à déterminer $\alpha $).

\item Donner um encadrement de $\alpha $ et justifier que : 
\[
\forall x\in \left[ 0.8,1\right],\;\;\dfrac{h\left( 1\right)
}{sh^{2}\left(
0.8\right) }\leq f^{\prime }\left( x\right) \leq \dfrac{h\left(
0.8\right) }{sh^{2}\left( 1\right) }
\]

\item On donne : 
\[
\dfrac{h\left( 1\right) }{sh^{2}\left( 0.8\right) }\simeq -0.47\text{
et }\dfrac{h\left( 0.8\right) }{sh^{2}\left( 1\right) }\simeq -0.13
\]
Montrer que : 
\[
\forall n\in \N,\mathbb{\;\;}\left| u_{n + 1}-\alpha \right|
\leq 0.5\left| u_{n}-\alpha \right|
\]
Puis que : 
\[
\forall n\in \N,\mathbb{\;\;}\left| u_{n}-\alpha \right|
\leq 0.2\left( 0.5\right) ^{n}
\]

\item En déduire la limite de la suite $\left( u_{n}\right) $ quand $n$
tend
vers $ + \infty $.

\item Écrire un programme en -\Scilab{} permettant de calculer et
d'afficher $u_{10}$.
\end{noliste}

\section*{\protect\LARGE Exercice 3}

Sous diverses hypothèses, l'exercice étudie différentes situations
probabilistes concernant une entreprise de construction produisant des
objets sur deux chaînes de montage $A$ et $B$ qui fonctionnent
indépendemment l'une de l'autre.\\
Pour une chaîne donnée, les fabrications des pièces sont indépendantes.

\noindent \textbf{Partie 1.}

On suppose que $A$ produit $60\%$ des objets et $B$ produit $40\%$ des
objets. La probabilité qu'un objet construit par la chaine $A$ soit
défectueux est $0.1$ alors que la probabilité pour qu'un objet
construit par
la chaine $B$ soit défectueux est $0.2$.

\begin{noliste}{1.}
 \setlength{\itemsep}{4mm}
\item On choisit au hasard un objet à la sortie de l'entreprise. On
constate
que cet objet est défectueux. Calculer la probabilité de l'évènement
\textquotedblleft l'objet provient de la chaîne A\textquotedblleft\.

\item On suppose de plus que le nombre d'objets produits en une heure
par $A$
est une variable aléatoire $Y$ qui suit une loi de Poisson de paramètre
$\lambda = 20.$

On considère la variable aléatoire $X$ représentant le nombre d'objets
défectueux produits par la chaîne $A$ en une heure.

\begin{noliste}{a)}
 \setlength{\itemsep}{2mm}
\item Rappeler la loi de $Y$ ainsi que la valeur de l'espérance et de
la
variance de $Y$.

\item Soient $k$ et $n$ deux entiers naturels, déterminer la
probabilité
conditionnelle $P\left[ X = k/Y = n\right] $. (On distinguera les cas
$k\leq n$ et $k>n$ ).

\item En déduire, en utilisant le système complet d'évènements $\left(
Y = i\right)_{i\in \N},$ que $X$ suit une loi de Poisson de paramètre
2.
\end{noliste}
\end{noliste}

\noindent \textbf{Partie 2.}

\noindent Soit $f$ la fonction définie sur $\R$ par : 
\[
\left\{ 
\begin{array}{ccc}
f\left( t\right) = \dfrac{2}{\left( 1 + t\right) ^{3}}\text{ } &
\text{si} & 
t\geq 0 \\
f\left( t\right) = 0 & \text{si} & t<0
\end{array}
\right.
\]

\begin{noliste}{1.}
 \setlength{\itemsep}{4mm}
\item Montrer que $f$ est une densité d'une variable aléatoire $Z$.

\item Déterminer la fonction de répartition $F_{Z}$ de $Z$.

\item Justifier la convergence de l'intégrale : 
\[
\dint{0}{+ \infty }\dfrac{2t}{\left( 1 + t\right) ^{3}}dt
\]
La calculer en effectuant le changement de variable $u = t + 1$.

\item Prouver que $Z$ admet une espérance et la déterminer.

\item $Z$ admet-elle une variance ?

\item Dans cette partie, on suppose que le temps de fabrication,
exprimé en
minutes d'un pièce par la chaîne $A$ (respectivement $B$) est une
variable aléatoire $Z_{1}$ ( respectivement $Z_{2}$) où $Z_{1}$
et$Z_{2}$ sont deux variables aléatoires indépendantes suivant la même
loi que que $Z$.

\begin{noliste}{a)}
 \setlength{\itemsep}{2mm}
\item On considère les évènements :

$C = $ \textquotedblleft le temps de fabrication d'une pièce sur la
chaîne $B$
est supérieur à 2 minutes\textquotedblleft.

$D = $ \textquotedblleft le temps de fabrication d'une pièce sur la
chaîne $B$
est inférieur à 3 minutes\textquotedblleft.

Calculer les probabilités suivante : $P\left(\Ev{ C}\right),P\left(\Ev{
D}\right),P\left(\Ev{ D/C}\right).$

\item On note $T = \max (Z_{1},Z_{2})$ et $G_{T}$ la fonction de
répartition
de $T$.

\begin{nonoliste}{(i)}
\item Exprimer l'évènement $\left( T\leq x\right) $ en fonction des
évènements $\left( Z_{1}\leq x\right) $ et $\left( Z_{2}\leq x\right)
$.

\item Montrer que : 
\[
\forall x\in \R,\;\;G_{T}\left( x\right) = \left[ F_{Z}\left(
x\right) \right] ^{2}
\]
\end{nonoliste}

\item En déduire que $T$ est une variable aléatoire à densité dont on
donnera une densité.
\end{noliste}
\end{noliste}

\noindent \textbf{Partie 3.}

\noindent On suppose maintenant que pour qu'une pièce soit terminée, il
faut
qu'elle passe par la chaîne $A$ puis par la chaîne $B$.\\
Le temps de passage exprimé en minutes pour un objet sur la chaîne $A$
est
une variable aléatoire $M$ suivant une loi exponentielle de paramètre
2.
\\
Le temps de passage exprimé en minutes pour un objet sur la chaîne $B$
est
une variable aléatoire $N$ suivant une loi uniforme sur $\left[
0,1\right] $.
\\
Les variables $M$ et $N$ sont indépendantes.

\begin{noliste}{1.}
 \setlength{\itemsep}{4mm}
\item Rappeler l'expression d'une densité de probabilité $v$ de $M$ et
d'une
densité $w$ de $N$.

\item On note $S$ la variable aléatoire représentant le temps total de
fabrication d'une pièce.

Exprimer $S$ en fonction de $M$ et de $N$ et déterminer le temps moyen
de
fabrication d'une pièce.
\end{noliste}

\label{fin}

\end{document}

