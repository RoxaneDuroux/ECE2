\documentclass[11pt]{article}%
\usepackage{geometry}%
\geometry{a4paper,
 lmargin = 2cm,rmargin = 2cm,tmargin = 2.5cm,bmargin = 2.5cm}

\input{../../macros.tex}

\pagestyle{fancy} %
\lhead{ECE2 \hfill Mathématiques\\
} %
\chead{\hrule} %
\rhead{} %
\lfoot{} %
\cfoot{} %
\rfoot{\thepage} %

\renewcommand{\headrulewidth}{0pt}% : Trace un trait de séparation
 % de largeur 0,4 point. Mettre 0pt
 % pour supprimer le trait.

\renewcommand{\footrulewidth}{0.4pt}% : Trace un trait de séparation
 % de largeur 0,4 point. Mettre 0pt
 % pour supprimer le trait.

\setlength{\headheight}{14pt}

\title{\bf \vspace{-2cm} ECRICOME 2012} %
\author{} %
\date{} %
\begin{document}

\maketitle %
\vspace{-1.4cm}\hrule %
\thispagestyle{fancy}

\vspace*{.2cm}


% DEBUT DU DOC À MODIFIER : tout virer jusqu'au début de l'exo

%Définition et changement de valeurs de
compteurs%newcounter{cpt1}{section} compteur cpt1 remis à 0 à chaque
aumentation par stepcounter du compteur section%setcounter{cpt1}{3} on
met le compteur à 3%addtocounter{cpt1}{5} on ajoute 5 au compteur%
stepcounter{cpt1} on ajoute 1% ifthenelse{test}{alors}{sinon} (page
206) pour subordonner à une condition % whiledo{test}{commande} pour
faire une boucle (page 206 aussi) % value{cpt1} pour noter dans le
document la valeur de cpt1 
%Définition définitive d'opérateurs
mathématiques\newcommand{\ch}{\operatorname{ch}} 
\newcommand{\sh}{\operatorname{sh}}
\renewcommand{\tanh}{\operatorname{th}}
\renewcommand{\sinh}{\operatorname{sh}}
\renewcommand{\cosh}{\operatorname{ch}}
\newcommand{\argsh}{\operatorname{argsh}}
\newcommand{\argch}{\operatorname{argch}}
\newcommand{\argth}{\operatorname{argth}}
\newcommand{\Id}{\operatorname{Id}}
\renewcommand{\leq}{\leq}
\renewcommand{\geq}{\geq }

\newcommand{\dlim}{\lim}
\newcommand{\dsum}{\sum}
\newcommand{\dint}{\int}
\newcommand{\dprod}{\prod}



%Définition de nouvelles couleurs : rgb(trois paramètres red green blue
entre 0 et 1); cmyk (quatre cyan magenta yellow black) entre 0 et 1;
gray (entre 0 et 1) et black, white, red, green, blue, cyan, magenta,
yellow% definecolor{0gris}{gray}{0.8} 
% Nouvelle commande pour encadrer le titre car shabox ne veut que d'une
seule ligne; ATTENTION A LA TAILLE; petite différence avec shadowbox ou
doublebox, voire fcolorbox ou colorbox (au lieu de shabox; laisser le
parbox tranquille sauf pour la taille de la boîte
\newcommand{\Tbox}[1]{\begin{center} \shabox{\parbox{0.6
\linewidth}{#1}} \end{center}} %[1] pour 1 paramètre ; #1 pour ce que
fait le 1er paramètre; entre accolades ce que fait la commande
%Mise en page en mode fancy : en-têtes et pieds de pages puis
définition des en-têtes et pieds de pages\pagestyle{fancy}
\lhead{ECE 2 - Mathématiques \\
Quentin Dunstetter - ENC-Bessières 2011$\backslash$2012}
\chead{}
\rhead{Ecricome 2012}
\rfoot[ \ \thepage]{\thepage}
\cfoot{}
\lfoot{}
\thispagestyle{fancy} %Mise en page de la 1ère page en mode fancy
%Trait en bas et en haut de la page (entre en-tête et texte et texte et
pied de page)\renewcommand{\footrulewidth}{0.4pt}
\renewcommand{\headrulewidth}{0.4pt}


\begin{center}
{\Huge ECRICOME Eco 2012}
\end{center}

\section*{EXERCICE 1}

$\left( \M{3}, +,\cdot.\right) $ désigne l'espace vectoriel des
matrices carrées d'ordre 3 à
coefficients réels.

Deux matrices $A$ et $B$ de $\M{3} $ étant données, on suppose qu'il
existe une matrice $L$ appartenant à $\mathcal{M}_{3}\R$ telle que :
\[
L = AL + B.
\]

On définit la suite de matrices $\left( U_{n}\right)_{nIN}$ de $\M{3} $
de la manière suivante :
\[
\left\{ 
\begin{array}{l}
U_{0}\in \M{3} \\
\forall n\in \N,\ U_{n + 1} = AU_{n} + B
\end{array}
\right.
\]

\begin{noliste}{1.}
 \setlength{\itemsep}{4mm}
\item Démontrer par récurrence que, pour tout entier naturel $n$ :
\[
U_{n} = L + A^{n}\left( U_{0}-L\right).
\]

Dans la suite du problème les matrices $A$ et $B$ sont choisies de
telle
sorte que :
\[
A = \frac{1}{6}\begin{smatrix}
0 & 3 & 3 \\
-4 & 6 & 4 \\
-2 & 3 & 5
\end{smatrix},\quad B = 
\begin{smatrix}
3 & -1 & -2 \\
1 & 0 & -1 \\
2 & -1 & -1
\end{smatrix}
\]
On note :

\begin{noliste}{$\sbullet$}
\item $\mathrm{Id}$ l'endomorphisme identité de $\R^{3}$ ;

\item $a$ l'endomorphisme de $\R^{3}$ dont la matrice dans la base
canonique est la matrice $A$ ;

\item $b$ l'endomorphisme de $\R^{3}$ dont la matrice dans la base
canonique est la matrice $B$ ;

\item $\operatorname{Im}\left( b\right) $ l'image de l'endomorphisme
$b$ ;

\item $\operatorname{Im}\left( Id-a\right) $ l'image de l'endomorphisme
$\mathrm{Id}-a$.
\end{noliste}

\item Prouver que le vecteur $u = \left( x,y,z\right) $ appartient à
l'image de $b$ si et seulement si 
\[
-x + y + z = 0
\]
puis montrer que :
\[
\operatorname{Im}\left( b\right) = \operatorname{Im}\left(
\mathrm{Id}-a\right)
\]

\item Montrer que la matrice $P = 
\begin{smatrix}
1 & 1 & 0 \\
1 & 0 & -1 \\
1 & 1 & 1
\end{smatrix}
$ peut être considérée comme la matrice de passage de la base
canonique de $\R^{3}$ à une base de vecteurs propres de $a.$

\item Écrire la matrice $D$ de l'endomorphisme $a$ ainsi que la matrice
$B^{\prime }$ de l'endomorphisme $b$ dans cette base de vecteurs
propres.

\item Démontrer que, pour tout entier naturel $n$,
\[
A^{n} = PD^{n}P^{-1}
\]

\item En écrivant convenablement $D^{n}$ comme la somme de trois
matrices diagonales judicieusement choisies, prouver l'existence de
trois
matrices $E$, $F$, $G$ indépendantes de $n$ telles que pour tout entier
naturel $n$ :
\[
A^{n} = E + \left( \frac{1}{2}\right) ^{n}F + \left( \frac{1}{3}\right)
^{n}G.
\]
Expliciter uniquement la matrice $E$ sous la forme d'un tableau de
nombres.

\item Déterminer par le calcul, une matrice $L^{\prime }$ de la forme
$\begin{smatrix}
0 & 0 & 0 \\
0 & p & q \\
0 & 0 & r
\end{smatrix}
$ telle que :
\[
L^{\prime } = DL^{\prime } + B^{\prime }
\]

\item Montrer que la matrice $L = PL^{\prime }P^{-1}$ vérifie : 
\[
L = AL + B.
\]

\item Établir que $EL = 0.$

\item Montrer que chacun des coefficients de la matrice $U_{n}$ a pour
limite, lorsque $n$ tend vers $ + \infty $, les coefficients de la
matrice $EU_{0} + L$.
\end{noliste}

\section*{EXERCICE 2.}

\subsection*{Partie I. Étude d'une fonction $f$.}

On considère la fonction définie sur l'ensemble des réels
positifs par :
\[
\left\{ 
\begin{array}{lc}
f\left( x\right) = \dfrac{1-e^{-x}}{x} & \text{si }x>0 \\
f\left( 0\right) = 1 & 
\end{array}
\right.
\]

\begin{noliste}{1.}
 \setlength{\itemsep}{4mm}
\item Écrire le développement limité de $f\left( x\right) $ l'ordre
2, au voisinage de $0$ En déduire que $f$ est continue sur $\left[
0, + \infty \right[.$

\item Montrer que $f$ est dérivable en $0$ et donner la valeur de
$f^{\prime }\left( 0\right) $.

\item Justifier la dérivabilité de $f$ sur l'intervalle $\left]
0, + \infty \right[ $ puis déterminer la fonction $\varphi $ telle que
:
\[
\forall x>0\quad f^{\prime }\left( x\right) = \frac{\varphi \left(
x\right) }{x^{2}}
\]

\item Étudier les variations de $\varphi $ ;. En déduire le tableau
de variation $f$ qui sera complété par la limite de $f$ en $ + \infty
$.
\end{noliste}

\subsection*{Partie II. Étude d'une suite.}

On introduit la suite $\left( u_{n}\right)_{n\in \N^{\ast }}$ définie
par :
\[
\forall n\in \N^{\ast }\quad u_{n} =
\dint_{0}{n}\frac{e^{-\frac{u}{n}}}{1 + u}du
\]

\begin{noliste}{1.}
 \setlength{\itemsep}{4mm}
\item Démontrer que pour tout entier naturel $n$ non nul :
\[
u_{n}\geq \frac{1}{e}\ln \left( n + 1\right)
\]
Donner la limite de la suite $\left( u_{n}\right)_{n\in \N^{\ast }}$.

\item Prouver l'existence de l'intégrale $\dint{0}{1}f\left( x\right)
dx $'.

\item Utiliser un changement de variable affine pour montrer que, pour
tout
entier naturel $n$ non nul :
\[
0\leq \dint{0}{n}\frac{1}{1 + u}du-u_{n}\leq \dint{0}{1}f\left(
x\right\dx
\]

\item Donner alors un équivalent simple de $u_{n}$ lorsque $n$ tend
vers 
$ + \infty.$
\end{noliste}

\subsection*{EXERCICE 3.}

Soit $n$ un entier naturel non nul. Une entreprise dispose d'un lot du
$n$
feuilles originales qu'elle a numérotées $l,\ 2,\ \cdots,\ n.$ Elle
photocopie ces $n$ feuilles originales et souhaite que chaque original
soit
agrafé avec sa copie. L'entreprise programme le photocopieur afin que
chaque original soit agrafé avec sa copie. Cependant. suite à un défaut
informatique, la photocopieuse a mélangé les originaux et
les copies. L'entreprise décide donc de placer les $n$ originaux et les
$n$ copies dans une boite. Une personne est alors chargée du travail
suivant : elle pioche simultanément et au hasard 2 feuilles dans la
boite. S'il s'agit d'un original et de sa copie, elle les agrafe et les
sort
de la boite. Sinon, elle repose les deux feuilles dans la, boite et
elle
recommence.

On modélise l'expérience par un espace probabilité $\left(
\Omega,\mathcal{B},\operatorname{P}\right) $. Soit $T_{n}$ la variable
aléatoire égale au nombre de pioches qui sont nécessaires pour vider la
boite lorsque celle-ci contient $n$ originaux et $n$ copies (soit $2n$
feuilles).

On considère l'événement $A_{n}$ : \guillemotleft\ à l'issue
de la première pioche, les deux feuilles piochées ne sont pas agrafées
\guillemotright\ et $a_{n}$ sa probabilité c'est-à-dire que $a_{n} =
\operatorname{P}\left( A_{n}\right) $.

\begin{noliste}{1.}
 \setlength{\itemsep}{4mm}
\item Calculer $a_{n}$.

\item \underline{Étude de $T_{2}$}. On suppose dans cette question que
$n = 2$, c'est-à-dire que la boite contient deux originaux et deux
copies.

\begin{noliste}{a)}
 \setlength{\itemsep}{2mm}
\item Montrer que pour tout entier $k\geq 2 :\operatorname{P}\left(
T_{2} = k\right)
 = \left( 1-a_{2}\right) \left( a_{2}\right) ^{k-2}.$

\item Justifier que la variable $S_{2} = T_{2}-1$ suit une loi
géométrique dont on précisera le paramètre.\\
En déduire l'espérance et la variance de $T_{2}$ en fonction de $a_{2}$
\end{noliste}

\item \underline{Étude de $T_{3}$}. On suppose dans cette question que
$n = 3$, c'est-à-dire que la boite, contient trois originaux et trois
copies.

\begin{noliste}{a)}
 \setlength{\itemsep}{2mm}
\item Calculer $\operatorname{P}\left( T_{3} = 2\right) $ puis
$\operatorname{P}\left(
T_{3} = 3\right) $ en fonction de $a_{2}$ et $a_{3}$

\item À l'aide du système complet d'événements $\left(
A_{3},\overline{A_{3}}\right) $ démontrer pour tout $k\geq 2$ que :
\[
\operatorname{P}\left( T_{3} = k + 1\right) = \left( 1-a_{3}\right)
\operatorname{P}\left(
T_{2} = k\right) + a_{3}\operatorname{P}\left( T_{3} = k\right)
\]

\item Montrer que :
\[
k\geq 2,\quad \operatorname{P}\left( T_{3} = k\right) = \frac{\left(
1-a_{2}\right)
\left( 1-a_{3}\right) }{a_{3}-a_{2}}\left[ \ \left( a_{3}\right)
^{k-2}-\left(
a_{2}\right) ^{k-2}\right].
\]

\item Calculer $\Sum{k = 2}{+ \infty }\operatorname{P}\left(
T_{3} = k\right).$

\item Prouver que la variable aléatoire $T_{3}-1$ admet une espérance
et calculer $\E\left( T_{3}-1\right) $.

Donner la valeur de $\E\left( T_{3}\right) $ en fonction de $a_{2}$ et
$a_{3}$.

\item Établir que la variable aléatoire $T_{3}\left( T_{3}-1\right) $
admet une espérance et donner sa valeur en fonction de $a_{2}$ et
$a_{3}$.

En déduire que $T_{3}$ admet une variance.
\end{noliste}
\end{noliste}

\end{document}

