\documentclass[11pt]{article}%
\usepackage{geometry}%
\geometry{a4paper,
 lmargin = 2cm,rmargin = 2cm,tmargin = 2.5cm,bmargin = 2.5cm}

\input{../../macros.tex}

\pagestyle{fancy} %
\lhead{ECE2 \hfill Mathématiques\\
} %
\chead{\hrule} %
\rhead{} %
\lfoot{} %
\cfoot{} %
\rfoot{\thepage} %

\renewcommand{\headrulewidth}{0pt}% : Trace un trait de séparation
 % de largeur 0,4 point. Mettre 0pt
 % pour supprimer le trait.

\renewcommand{\footrulewidth}{0.4pt}% : Trace un trait de séparation
 % de largeur 0,4 point. Mettre 0pt
 % pour supprimer le trait.

\setlength{\headheight}{14pt}

\title{\bf \vspace{-2cm} ECRICOME 2014} %
\author{} %
\date{} %
\begin{document}

\maketitle %
\vspace{-1.4cm}\hrule %
\thispagestyle{fancy}

\vspace*{.2cm}


% DEBUT DU DOC À MODIFIER : tout virer jusqu'au début de l'exo

%Définition et changement de valeurs de
compteurs%newcounter{cpt1}{section} compteur cpt1 remis à 0 à chaque
aumentation par stepcounter du compteur section%setcounter{cpt1}{3} on
met le compteur à 3%addtocounter{cpt1}{5} on ajoute 5 au compteur%
stepcounter{cpt1} on ajoute 1% ifthenelse{test}{alors}{sinon} (page
206) pour subordonner à une condition % whiledo{test}{commande} pour
faire une boucle (page 206 aussi) % value{cpt1} pour noter dans le
document la valeur de cpt1 
%Définition définitive d'opérateurs
mathématiques\newcommand{\ch}{\operatorname{ch}} 
\newcommand{\sh}{\operatorname{sh}}
\renewcommand{\tanh}{\operatorname{th}}
\renewcommand{\sinh}{\operatorname{sh}}
\renewcommand{\cosh}{\operatorname{ch}}
\newcommand{\argsh}{\operatorname{argsh}}
\newcommand{\argch}{\operatorname{argch}}
\newcommand{\argth}{\operatorname{argth}}
\newcommand{\Id}{\operatorname{Id}}
\newcommand{\id}{\operatorname{id}}
\renewcommand{\im}{\operatorname{Im}}
\renewcommand{\leq}{\leq}
\renewcommand{\geq}{\geq }

\newcommand{\dlim}{\lim}
\newcommand{\dsum}{\sum\limits}
\newcommand{\dprod}{\prod}
\newcommand{\lb}{\llbracket}
\newcommand{\rb}{\rrbracket}


%Définition de nouvelles couleurs : rgb(trois paramètres red green blue
entre 0 et 1); cmyk (quatre cyan magenta yellow black) entre 0 et 1;
gray (entre 0 et 1) et black, white, red, green, blue, cyan, magenta,
yellow% definecolor{0gris}{gray}{0.8} 
% Nouvelle commande pour encadrer le titre car shabox ne veut que d'une
seule ligne; ATTENTION A LA TAILLE; petite différence avec shadowbox ou
doublebox, voire fcolorbox ou colorbox (au lieu de shabox; laisser le
parbox tranquille sauf pour la taille de la boîte
\newcommand{\Tbox}[1]{\begin{center} \shabox{\parbox{0.8
\linewidth}{#1}} \end{center}} %[1] pour 1 paramètre ; #1 pour ce que
fait le 1er paramètre; entre accolades ce que fait la commande
%Mise en page en mode fancy : en-têtes et pieds de pages puis
définition des en-têtes et pieds de pages\pagestyle{fancy}
\lhead{ECE 2 - Mathématiques \\
Quentin Dunstetter - ENC-Bessières 2011$\backslash$2012}
\chead{}
\rhead{Ecricome 2014}
\rfoot[ \ \thepage]{\thepage}
\cfoot{}
\lfoot{}
\thispagestyle{fancy} %Mise en page de la 1ère page en mode fancy
%Trait en bas et en haut de la page (entre en-tête et texte et texte et
pied de page)\renewcommand{\footrulewidth}{0.4pt}
\renewcommand{\headrulewidth}{0.4pt}

\indent \vspace{0.3cm}

\Tbox{\begin{center} \textbf{\Huge Ecricome 2014} \end{center} }

\vspace{0.5cm}


\section*{Exercice 1}

\noindent Soit $f$ l'endomorphisme de $\R^{3}$ dont la matrice dans la
base canonique de $\R^{3}$ est :
\[
 A = \begin{smatrix}
1 & 0 & 0 \\
1 & 2 & 1 \\
2 & -2 & -1
\end{smatrix}
\]
On considère les vecteurs $u$ et $v$ de $\R^{3}$ définis par :
\[
 u = (0,1,-2) \quad\text{et}\quad v = (0,1,-1) 
\]
On note $\mathrm{Ker}(f)$ le noyau de $f$ et $\mathrm{Im}(f)$ son
image.
Si $\lambda$ est une valeur propre de $f$, on désigne par
$E_{\lambda}(f)$ l'espace propre de $f$ associé à la valeur propre
$\lambda$.

\subsection*{Partie I : Réduction de l'endomorphisme $f$}

\begin{noliste}{1.}
 \setlength{\itemsep}{4mm}
\item Déterminer une base de $\mathrm{Ker}(f)$ et une base de
$\mathrm{Im}(f)$.

\item Justifier que $f$ n'est pas bijectif. En déduire, \textit{sans le
moindre calcul}, une valeur propre de $f$.

\item Prouver que $u$ et $v$ sont deux vecteurs propres de $f$.

Préciser la valeur propre $\lambda$ (respectivement $\mu$) associée à
$u$ (respectivement à $v$).

Donner la dimension de l'espace propre $E_{\lambda}(f)$ (respectivement
$E_{\mu}(f)$).

\item L'endomorphisme $f$ est-il diagonalisable ?

\item Rechercher tous les vecteurs $t = (x,y,z)$ de $\R^{3}$ vérifiant
l'équation :
\[
 f(t) = t + v 
\]

\item Déterminer un vecteur $w$ de $\R^{3}$, dont la troisième
coordonnée (dans la base canonique de $\R^{3}$) est nulle, telle que la
famille $C = (u,v,w)$ soit une base de $\R^{3}$ et que la matrice de
$f$ dans la base $C$ soit la matrice 
\[
 T = \begin{smatrix}
0 & 0 & 0 \\
0 & 1 & 1 \\
0 & 0 & 1
\end{smatrix}
\]
\end{noliste}

\subsection*{Partie II : Résolution d'une équation}

\noindent Dans les questions 1,2 et 3 de cette partie, on suppose qu'il
existe un endomorphisme $g$ de $\R^{3}$ vérifiant : 
\[
 g\circ g = f 
\]

\begin{noliste}{1.}
 \setlength{\itemsep}{4mm}
\item Montrer que : 
\[
 f\circ g = g \circ f 
\]
En déduire que :
\[
 f(g(u)) = 0 \quad\text{et}\quad f(g(v)) = g(v) 
\]

\item Justifier qu'il existe deux réels $a$ et $b$ tels que $g(u) = au$
et $g(v) = bv$.

\item On note $N$ la matrice de $g$ dans la base $C = (u,v,w)$ définie
à la question I.6. Justifier que :
\[
 N = \begin{smatrix}
a & 0 & c \\
0 & b & d \\
0 & 0 & e \\
\end{smatrix}
\]
où $a$ et $b$ sont les deux réels définis à la question précédente
(II.2) et $c,d,e$ des réels.

\item Existe-t-il des endomorphismes $g$ de $\R^{3}$ tels que $g\circ g
= f$ ?

\textit{Indication : Utiliser les matrices de $f$ et $g$ dans la base
$C = (u,v,w)$ définie à la question I.6.}
\end{noliste}

\section*{Exercice 2}

\noindent On considère la fonction $f$ définie sur $[0; + \infty[$ par
:
\[
 f(x) = \left\{ 
\begin{array}{cl}
 1 & \text{si }x = 0 \\
 & \\
\dfrac{x}{\ln(1 + x)} & \text{si }x\in \left]0; + \infty\right[ \
\end{array}
\right. 
\]
ainsi que la suite $(u_{n})_{n\in \N}$ définie par :
\[
 u_{0} = e \quad\text{et}\quad \forall n\in \N,\; u_{n + 1} = f(u_{n}) 
\]

\begin{noliste}{1.}
 \setlength{\itemsep}{4mm}
\item Déterminer le signe de $f$ sur l'intervalle $[0; + \infty[$. En
déduire que, pour tout entier naturel $n$, $u_{n}$ existe.

\item Écrire un programme en -\Scilab{} qui, pour une valeur $N$
fournie par l'utilisateur, calcule et affiche $u_{N}$.

\item Montrer que $f$ est continue sur $[0; + \infty[$.

\item Établir que $f$ est de classe $\mathcal{C}{1}$ sur $\left]0; +
\infty\right[$.

\item Donner le développement limité à l'ordre $2$ au voisinage de $0$
de 
\[
 \ \ln(1 + x) - \dfrac{x}{1 + x} 
\]
puis déterminer un équivalent de $f'(x)$ lorsque $x$ tend vers $0$.

\item Prouver que $f$ est de classe $\mathcal{C}{1}$ sur $[0; +
\infty[$.

\item Établir que :
\[
 \ \forall x\geq e-1,\quad f(x)\leq x \quad\text{et}\quad (x + 1)\ln(x
+ 1)\geq (x + 1) 
\]
En déduire que :
\[
 \ \forall x\geq e-1,\quad f'(x)\geq 0 
\]

\item Démontrer que :
\[
 \ \forall n\in \N,\quad e-1\leq u_{n} 
\]

\item Établir que la suite $(u_{n})_{n\in \N}$ converge et préciser la
valeur de sa limite $L$.
\end{noliste}




\section*{Exercice 3}

\noindent Soit $p$ un réel appartenant à l'intervalle ouvert $]0;1[$.
On note $q = 1-p$.\\

\noindent On dispose dans tout l'exercice d'une même pièce dont la
probabilité d'obtenir PILE vaut $p$.

\subsection*{Partie I : Étude d'une première expérience}

\noindent On procède à l'expérience suivante $\mathcal{E}$ : \og
\textit{On effectue une succession illimitée de lancers de la pièce}
\fg.\\

\noindent On note :
\begin{noliste}{$\sbullet$}
\item[$\bullet$] pour tout entier naturel non nul $n$, $X_{n}$ la
variable aléatoire égale au nombre de PILE obtenus lors des $n$
premiers lancers de la pièce ;
\item[$\bullet$] pour tout entier naturel non nul $j$, $F_{j}$
l'évènement : \og La pièce donne FACE lors du $j$-ième lancer \fg ;
\item[$\bullet$] $Y$ la variable aléatoire égale au nombre de FACE
obtenus avant l'apparition du second PILE.\\
\end{noliste}

\noindent Par exemple, si les lancers ont donné dans cet ordre : 
\begin{center} \og FACE, PILE, FACE, FACE, FACE, PIL\E\fg \end{center}
alors $Y = 4$.\\

\noindent On admet que les variables aléatoires $X_{n}$ ($n\in \N^*$)
et $Y$ sont définies sur un même espace probabilisé modélisant
l'expérience $\mathcal{E}$.\\

\begin{noliste}{1.}
 \setlength{\itemsep}{4mm}
\item Simulation informatique.
\begin{noliste}{a)}
 \setlength{\itemsep}{2mm}
\item Écrire une fonction en -\Scilab{} d'en-tête : 
\begin{center} \texttt{function LANCER(p : real) : integer ;}
\end{center}
qui crée un nombre aléatoire dans l'intervalle $[0;1]$ et renvoie $1$
si ce nombre aléatoire est strictement inférieur à $p$ et $0$ sinon.

\item Écrire une fonction en -\Scilab{} d'en-tête : 
\begin{center} \texttt{function PREMIER\_{P}IL\E(p : real) : integer ;}
\end{center}
qui simule autant de lancers de la pièce que nécessaire jusqu'à
l'obtention du premier PILE et renvoie le nombre de lancers effectués.

\textit{Indication : si on le souhaite, on pourra utiliser la fonction
}\texttt{LANCER}\textit{ en la répétant convenablement.}

\item Écrire un programme en -\Scilab{} qui demande un réel $p$ à
l'utilisateur, puis qui simule autant de lancers de la pièce que
nécessaire jusqu'à l'obtention du second PILE, et affiche le nombre de
FACE obtenus en tout.

\textit{Indication : on pourra utiliser la fonction
}\texttt{PREMIER\_{P}ILE}\textit{ en la répétant convenablement.}
\end{noliste}

\item Soit $n$ un entier naturel non nul. Donner la loi de $X_{n}$.
Préciser la valeur de son espérance $\E(X_{n})$ et de sa variance
$\V(X_{n})$.

\item Déterminer les valeurs prises par la variable aléatoire $Y$.

\item Donner les valeurs des probabilités : 
\[
 P\left(\Ev{Y = 0}\right),\quad P\left(\Ev{Y = 1}\right) \quad
\text{et}\quad P\left(\Ev{Y = 2}\right) 
\]

\item Soit $n$ un entier naturel. Justifier que les évènements :
\[
 (Y = n) \quad\text{et}\quad (X_{n + 1} = 1)\cap \overline{F_{n + 2}} 
\]
sont égaux.

\item Prouver que :
\[
 \ \forall n\in \N,\quad P\left(\Ev{Y = n}\right) = \left(\Ev{n +
1}\right)p^{2}q^{n} 
\]

\item Vérifier par le calcul que :
\[
 \ \Sum{n = 0}{+ \infty} P\left(\Ev{Y = n}\right) = 1 
\]

\item Démontrer que la variable aléatoire $Y$ possède une espérance
$\E(Y)$ et donner sa valeur.

\item Soit $k\in\N^*$. On note $Y_{k}$ la variable aléatoire égale au
nombre de FACE obtenus avant l'apparition du $k$-ième PILE. En
particulier, on a $Y_{2} = Y$.

En généralisant la méthode utilisée dans les questions précédentes,
déterminer la loi de $Y_{k}$.
\end{noliste}

\subsection*{Partie II : Étude d'une seconde expérience}

\noindent On procède à l'expérience suivante : \\
\\
\noindent $\mathcal{F}$ : \og \textit{Deux joueurs se relaient pour
effectuer des lancers successifs de la pièce pendant la pause déjeuner.
\\
Le joueur 1 arrive à 12h (considéré comme l'instant 0) et joue jusqu'à
l'arrivée du joueur 2. \\
 Le joueur 2 arrive au hasard entre 12h et 13h puis joue jusqu'à 13h
(considéré comme l'instant 1).} \fg\\

\noindent On note :
\begin{noliste}{$\sbullet$}
\item[$\bullet$] $R$ la variable aléatoire égale à la durée (en heure)
du jeu pour le joueur 1 ;
\item[$\bullet$] $S$ la variable aléatoire égale à la durée (en heure)
du jeu pour le joueur 2 ;
\item[$\bullet$] $T$ la variable aléatoire égale à la durée (en heure)
de jeu effectuée par le joueur ayant joué le plus longtemps
c'est-à-dire que : 
\[
 T = \max(R,S) 
\]
\end{noliste}

\noindent Pour tout variable aléatoire $X$, on note $F_{X}$ la fonction
de répartition de $X$.\\

\noindent On admet que $R$ et $S$ sont deux variables aléatoires
définies sur un même espace probabilisé muni d'une probabilité $P$
modélisant l'expérience $\mathcal{F}$. En outre, on suppose que :
\begin{center}\fbox{$R$ suit la loi uniforme sur $[0;1]$ et que $S =
1-R$} \end{center}
(cette dernière relation traduisant que le temps total consacré au jeu
par le joueur 1 et le joueur 2 est exactement d'une heure).\\

\begin{noliste}{1.}
 \setlength{\itemsep}{4mm}
\item Expliciter la fonction $F_{R}$ puis la fonction $F_{S}$.
Reconnaître alors la loi suivie par la variable aléatoire $S$.

\item Pour tout réel $t$, prouver que :
\[
 P\left(\Ev{T\leq t}\right) = P\Big( (R\leq t)\cap (R\geq 1-t) \Big) 
\]

\item Déterminer, pour tout $t\in \left[ \ \dfrac{1}{2} ; 1\right]$,
l'expression de $F_{T}$ en fonction de $T$.

\item Justifier que $T$ suit la loi uniforme sur $\left[ \ \dfrac{1}{2}
; 1\right]$.

\item En déduire que $T$ admet une espérance $\E(T)$ et une variance
$\V(T)$ que l'on précisera.
\end{noliste}

\end{document}

