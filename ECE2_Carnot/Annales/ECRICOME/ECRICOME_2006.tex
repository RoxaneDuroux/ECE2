\documentclass[11pt]{article}%
\usepackage{geometry}%
\geometry{a4paper,
 lmargin = 2cm,rmargin = 2cm,tmargin = 2.5cm,bmargin = 2.5cm}

\input{../../macros.tex}

\pagestyle{fancy} %
\lhead{ECE2 \hfill Mathématiques\\
} %
\chead{\hrule} %
\rhead{} %
\lfoot{} %
\cfoot{} %
\rfoot{\thepage} %

\renewcommand{\headrulewidth}{0pt}% : Trace un trait de séparation
 % de largeur 0,4 point. Mettre 0pt
 % pour supprimer le trait.

\renewcommand{\footrulewidth}{0.4pt}% : Trace un trait de séparation
 % de largeur 0,4 point. Mettre 0pt
 % pour supprimer le trait.

\setlength{\headheight}{14pt}

\title{\bf \vspace{-2cm} ECRICOME 2006} %
\author{} %
\date{} %
\begin{document}

\maketitle %
\vspace{-1.4cm}\hrule %
\thispagestyle{fancy}

\vspace*{.2cm}


% DEBUT DU DOC À MODIFIER : tout virer jusqu'au début de l'exo

%Définition et changement de valeurs de
compteurs%newcounter{cpt1}{section} compteur cpt1 remis à 0 à chaque
aumentation par stepcounter du compteur section%setcounter{cpt1}{3} on
met le compteur à 3%addtocounter{cpt1}{5} on ajoute 5 au compteur%
stepcounter{cpt1} on ajoute 1% ifthenelse{test}{alors}{sinon} (page
206) pour subordonner à une condition % whiledo{test}{commande} pour
faire une boucle (page 206 aussi) % value{cpt1} pour noter dans le
document la valeur de cpt1 
%Définition définitive d'opérateurs
mathématiques\newcommand{\ch}{\operatorname{ch}} 
\newcommand{\sh}{\operatorname{sh}}
\renewcommand{\tanh}{\operatorname{th}}
\renewcommand{\sinh}{\operatorname{sh}}
\renewcommand{\cosh}{\operatorname{ch}}
\newcommand{\argsh}{\operatorname{argsh}}
\newcommand{\argch}{\operatorname{argch}}
\newcommand{\argth}{\operatorname{argth}}
\newcommand{\Id}{\operatorname{Id}}
\renewcommand{\leq}{\leq}
\renewcommand{\geq}{\geq }

\newcommand{\dlim}{\lim}
\newcommand{\dsum}{\sum}
\newcommand{\dint}{\int}
\newcommand{\dprod}{\prod}



%Définition de nouvelles couleurs : rgb(trois paramètres red green blue
entre 0 et 1); cmyk (quatre cyan magenta yellow black) entre 0 et 1;
gray (entre 0 et 1) et black, white, red, green, blue, cyan, magenta,
yellow% definecolor{0gris}{gray}{0.8} 
% Nouvelle commande pour encadrer le titre car shabox ne veut que d'une
seule ligne; ATTENTION A LA TAILLE; petite différence avec shadowbox ou
doublebox, voire fcolorbox ou colorbox (au lieu de shabox; laisser le
parbox tranquille sauf pour la taille de la boîte
\newcommand{\Tbox}[1]{\begin{center} \shabox{\parbox{0.6
\linewidth}{#1}} \end{center}} %[1] pour 1 paramètre ; #1 pour ce que
fait le 1er paramètre; entre accolades ce que fait la commande
%Mise en page en mode fancy : en-têtes et pieds de pages puis
définition des en-têtes et pieds de pages\pagestyle{fancy}
\lhead{ECE 2 - Mathématiques \\
Quentin Dunstetter - ENC-Bessières 2011$\backslash$2012}
\chead{}
\rhead{Ecricome 2006}
\rfoot[ \ \thepage]{\thepage}
\cfoot{}
\lfoot{}
\thispagestyle{fancy} %Mise en page de la 1ère page en mode fancy
%Trait en bas et en haut de la page (entre en-tête et texte et texte et
pied de page)\renewcommand{\footrulewidth}{0.4pt}
\renewcommand{\headrulewidth}{0.4pt}


\begin{center}
{\Huge ECRICOME Eco 2006}
\end{center}

\section*{EXERCICE 1}. On considère la fonction $f$ définie pour tout
réel $x$ par : 
\[.f\left( x\right) = x + 1 + 2e^{x}
\]

ainsi que la fonction $g\ $des deux variables réelles $x$ et $y$
définie par : 
\[
g\left( x,y\right) = e^{x}\left( x + y^{2} + e^{x}\right)
\]

\subsection*{1. Recherche d'extremum local de $g.$}

\begin{noliste}{1.}
 \setlength{\itemsep}{4mm}
\item Étudier les variations de $f$ et donner les limites de $f\left(
x\right) $ lorsque $x$ tend vers $ + \infty $ et $-\infty $.

\item Justifier l'existence d'une asymptote oblique au voisinage de
$-\infty 
$ et donner la position de la courbe représentative de $f$ par rapport
à
cette asymptote.

\item Déduire des variations de $f$ l'existence d'un unique réel
$\alpha $, élément de l'intervalle $\left[ -2,-1\right] $ tel que
$f\left( \alpha \right) = 0$. ( on rappelle que $e\simeq 2,7$ )

\item Déterminer le seul point critique de $g$, c'est-à-dire le seul
couple
de $\R^{2}$, en lequel $g$ est susceptible de présenter un extremum.

\item Vérifier que $g$ présente un extremum relatif $\beta $ en ce
point.
Est-ce un maximum ou un minimum ?

\item Montrer que l'on a : 
\[
4\beta + \alpha ^{2}-1 = 0
\]
\end{noliste}

\subsection*{2. Étude d'une suite réelle.}

On s'intéresse à la suite $\left( u_{n}\right)_{n\in \N}$ définie par
son premier terme $u_{0} = -1$ et par la relation 
\[
\forall n\in \mathbb{N\quad }u_{n + 1} = u_{n}-\frac{f\left(
u_{n}\right) }{f^{\prime }\left( u_{n}\right) }
\]

\begin{noliste}{1.}
 \setlength{\itemsep}{4mm}
\item Prouver que $f$ est convexe sur $\R$. En déduire que que pour
tous réels $x$ et $t$ :

\[
f\left( x\right) + \left( t-x\right) f^{\prime }\left( x\right) \leq
f\left( t\right)
\]

\item En déduire l'inégalité suivante : 
\[
\forall n\in \mathbb{N\quad \alpha }\leq u_{n + 1}
\]
Puis que pour tout entier naturel $n$. : 
\[
\alpha \leq u_{n + 1}\leq u_{n}\leq -1
\]
En déduire que la suite $\left( u_{n}\right)_{n\in \N}$ est convergente
vers un réel à préciser.

\item On admet que pour tout $x$ de l'intervalle$\left[ -2,-1\right]$ :
\[
0\leq \left( x-\alpha \right) f^{\prime }\left( x\right) -f\left(
x\right) \leq \frac{\left( x-\alpha \right) ^{2}}{e}
\]

\begin{noliste}{a)}
 \setlength{\itemsep}{2mm}
\item Prouver alors que pour tout entier naturel $n$ : 
\[
0\leq u_{n + 1}-\alpha \leq \frac{\left( u_{n}-\alpha \right) ^{2}}{e}
\]

\item Puis démontrer par récurrence que pour tout entier naturel $n$ :

\[
0\leq u_{n}-\alpha \leq \frac{1}{e^{2^{n}-1}}
\]
\end{noliste}

\item Écrire un programme en langage \Scilab{} permettant, lorsque
l'entier
naturel $p$ est donné par l'utilisateur, de calculer une valeur
approchée de 
$\alpha $, de telle sorte que l'on ait : 
\[
0\leq u_{n}-\alpha \leq 10^{-p}
\]
\end{noliste}

\section*{EXERCICE 2}

Pour tout entier naturel $n$, on définit la fonction $f_{n}$ de la
variable réelle $x$ par : 
\[
f_{n}\left( x\right) = x^{n}\exp \left( -\frac{x^{2}}{2}\right)
\]

\begin{noliste}{1.}
 \setlength{\itemsep}{4mm}
\item Justifier que $f_{n}\left( x\right) $ est négligeable devant
$\dfrac{1}{x^{2}}$ au voisinage de $ + \infty $.

\item Prouver la convergence de l'intégrale $\dint{0}{+
\infty}f_{n}\left( x\right\dx$.

\item O pose $I_{n} = \dint{0}{+ \infty }f_{n}\left( x\right\dx$

\begin{noliste}{a)}
 \setlength{\itemsep}{2mm}
\item À l'aide d'une intégration par parties portant sur des intégrales
définies sur le segment $\left[ 0,A\right] $ avec $A\geq 0$, prouver
que pour tout entier naturel $n :$
\[
I_{n + 2} = \left( n + 1\right) I_{n}
\]

\item En utilisant la loi normale centrée réduite, justifier que : 
\[
I_{0} = \sqrt{\frac{\pi }{2}}
\]

\item Donner la valeur de $I_{1}$.

\item Démontrer par récurrence que pour tout entier naturel $n$ :

\begin{eqnarray*}
I_{2n} & = & \sqrt{\frac{\pi }{2}}\frac{\left( 2n\right) !}{2^{n}n!} \\
I_{2n + 1} & = & 2^{n}n!
\end{eqnarray*}
\end{noliste}

\item Soit $f$ la fonction définie pour tout réel $x$ par : 
\[
\left\{ 
\begin{array}{cc}
f\left( x\right) = f_{1}\left( x\right) & \text{si }x\geq 0 \\
f\left( x\right) = 0 & \text{si }x<0
\end{array}
\right.
\]

\begin{noliste}{a)}
 \setlength{\itemsep}{2mm}
\item Démontrer que $f$ est une densité de probabilité.

\item Soit $X$ une variable aléatoire réelle qui admet $f$ pour densité
de probabilité.

\begin{nonoliste}{(i)}
\item Justifier que $X$ admet une espérance $\E\left( X\right) $, et
préciser sa valeur.

\item Justifier que $X$ admet une variance $\V\left( X\right) $, et
préciser sa valeur.
\end{nonoliste}
\end{noliste}

\item On désigne par $F$ et $G$ les fonctions de répartitions
respectives de $X$ et de $Y = X^{2}$.

\begin{noliste}{a)}
 \setlength{\itemsep}{2mm}
\item Exprimer $G\left( x\right) $ en fonction de $F\left( x\right) $
en distinguant les deux cas : $x<0$ et $x\geq 0$

\item En déduire que $Y$ est une variable à densité. Reconnaître la loi
de $Y $ et donner la valeur de $\E\left( Y\right) $ et $\V\left(
Y\right) $.
\end{noliste}
\end{noliste}

\newpage

\section*{EXERCICE 3}

$E$ désigne l'espace des fonctions polynômes à coefficients réels, dont
le degré est inférieur ou égal à l'entier naturel 2.

\subsection*{1. Étude d'un endomorphisme de $E$.}

On considère l'application $f$ qui, à tout élément $P$ de $E, $ associe
la fonction polynôme $Q$ telle que : 
\[
\text{pour tout }x\text{ réel :\qquad }Q\left( x\right) = \left(
x-1\right)
P^{\prime }\left( x\right) + P\left(\Ev{ x}\right)
\]
et $\mathcal{B = }\left( P_{0},P_{1},P_{2}\right) $ la base canonique
de $E$ définie par : 
\[
\text{pour tout réel }x :\qquad P_{0}\left( x\right) = 1,\;P_{1}\left(
x\right) = x\text{ et }P_{2}\left( x\right) = x^{2}
\]

\begin{noliste}{1.}
 \setlength{\itemsep}{4mm}
\item Montrer que $f$ est un endomorphisme de $E$.

\item Vérifier que la matrice $A$ de $f$ dans $\mathcal{B}$, s'écrit
sous la
forme : 
\[
A = \left( 
\begin{array}{rrr}
1 & -1 & 0 \\
0 & 2 & -2 \\
0 & 0 & 3
\end{array}
\right)
\]

\item Quelles sont les valeurs propres de $f$ ? $f$ est-il
diagonalisable ? $f$ est-il un automorphisme de $E$ ?

\item Déterminer l'image par $f$ des fonctions polynômes
$R_{0},\,R_{1},\,R_{2}$ définies par : 
\[
\text{pour tout réel }x :\qquad R_{0}\left( x\right) = 1,\quad
R_{1}\left(
x\right) = x-1\text{ et }R_{2}\left( x\right) = \left( x-1\right) ^{2}
\]

\item Montrer que $\mathcal{B}{\prime } = \left(
R_{0},R_{1},R_{2}\right) $
est une base de vecteurs propres de $f$. \ 'Écrire la matrice de
passage $P$ de
la base $\mathcal{B}$ à la base $\mathcal{B}{\prime }$ ainsi que la
matrice 
$D$ de $f$ dans la base $\mathcal{B}{\prime }.$

\item Vérifier que pour tout réel $x$ : 
\[
\left\{ 
\begin{array}{c}
R_{2}x + 2R_{1}\left( x\right) + R_{0}\left( x\right) = P_{2}\left(
x\right) \\
R_{1}\left( x\right) + R_{0}\left( x\right) = P_{1}\left( x\right)
\end{array}
\right.
\]

En déduire la matrice de passage de la base $\mathcal{B}{\prime }$ à la
base $\mathcal{B}$.

\item Écrire $A^{-1}$ en fonction de $D^{-1}.$ Démontrer par récurrence
que pour tout entier naturel $n$ : 
\[
\left[ A^{-1}\right] ^{n} = P\left[ D^{-1}\right] ^{n}P^{-1}
\]
et expliciter la troisième colonne de la matrice $\left[ A^{-1}\right]
^{n}$.
\end{noliste}

\subsection*{2. Suite d'épreuves aléatoires.}

On dispose d'une urne qui contient trois boules numérotées de 0 à 2.

On s'intéresse à une suite d'épreuves définies de la manière suivante :

\begin{noliste}{$\sbullet$}
\item La première épreuve consiste à choisir au hasard une boule dans
cette urne.

\item Si $j$ est le numéro de la boule tirée, on enlève de l'urne
toutes les
boules dont le numéro est strictement supérieur à $j$, le tirage
suivant se
faisant alors dans l'urne ne contenant plus que les boules numérotées
de $0$ à $j$.
\end{noliste}

On considère alors la variable aléatoire réelle $X_{k}$ égale au numéro
de
la boule obtenue à la $k^{\grave{e}me}$ épreuve ($k\geq 0$).

On note alors $U_{k}$ la matrice unicolonne définie par : 
\[
U_{k} = \left( 
\begin{array}{c}
P\left[ X_{k} = 0\right] \\
P\left[ X_{k} = 1\right] \\
P\left[ X_{k} = 2\right]
\end{array}
\right)
\]
où $P\left[ X_{k} = j\right] $ est la probabilité de tirer la boule
numéro $j$ à la $k^{\grave{e}me}$ épreuve.

On convient de définir la matrice $U_{0}$ par : 
\[
U_{0} = \left( 
\begin{array}{c}
0 \\
0 \\
1
\end{array}
\right)
\]

\begin{noliste}{1.}
 \setlength{\itemsep}{4mm}
\item Déterminer la loi de $X_{2}$ (On pourra s'aider d'un arbre).
Calculer l'espérance et la variance de $X_{2}$.

\item Par utilisation de la formule des probabilités totales, prouver
que pour tout entier naturel $k$ :

\[
U_{k + 1} = A^{-1}U_{k}
\]

\item Écrire $U_{k}$ en fonction de $A^{-1}$ et $U_{0}$.

\item Pour tout $k$ de $\N$, donner la loi de $X_{k}$ et vérifier que
l'on a : 
\[
\dlim{k\rightarrow + \infty }P\left[ X_{k} = 0\right] = 1,\quad
\dlim{k\rightarrow + \infty }P\left[ X_{k} = 1\right] = 0,\quad
\dlim{k\rightarrow + \infty }P\left[ X_{k} = 2\right] = 0
\]
\end{noliste}

\label{fin}

\end{document}

