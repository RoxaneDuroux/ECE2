\documentclass[11pt]{article}%
\usepackage{geometry}%
\geometry{a4paper,
 lmargin = 2cm,rmargin = 2cm,tmargin = 2.5cm,bmargin = 2.5cm}

\input{../../macros.tex}

\pagestyle{fancy} %
\lhead{ECE2 \hfill Mathématiques\\
} %
\chead{\hrule} %
\rhead{} %
\lfoot{} %
\cfoot{} %
\rfoot{\thepage} %

\renewcommand{\headrulewidth}{0pt}% : Trace un trait de séparation
 % de largeur 0,4 point. Mettre 0pt
 % pour supprimer le trait.

\renewcommand{\footrulewidth}{0.4pt}% : Trace un trait de séparation
 % de largeur 0,4 point. Mettre 0pt
 % pour supprimer le trait.

\setlength{\headheight}{14pt}

\title{\bf \vspace{-2cm} ECRICOME 2008} %
\author{} %
\date{} %
\begin{document}

\maketitle %
\vspace{-1.4cm}\hrule %
\thispagestyle{fancy}

\vspace*{.2cm}


% DEBUT DU DOC À MODIFIER : tout virer jusqu'au début de l'exo

%Définition et changement de valeurs de
compteurs%newcounter{cpt1}{section} compteur cpt1 remis à 0 à chaque
aumentation par stepcounter du compteur section%setcounter{cpt1}{3} on
met le compteur à 3%addtocounter{cpt1}{5} on ajoute 5 au compteur%
stepcounter{cpt1} on ajoute 1% ifthenelse{test}{alors}{sinon} (page
206) pour subordonner à une condition % whiledo{test}{commande} pour
faire une boucle (page 206 aussi) % value{cpt1} pour noter dans le
document la valeur de cpt1 
%Définition définitive d'opérateurs
mathématiques\newcommand{\ch}{\operatorname{ch}} 
\newcommand{\sh}{\operatorname{sh}}
\renewcommand{\tanh}{\operatorname{th}}
\renewcommand{\sinh}{\operatorname{sh}}
\renewcommand{\cosh}{\operatorname{ch}}
\newcommand{\argsh}{\operatorname{argsh}}
\newcommand{\argch}{\operatorname{argch}}
\newcommand{\argth}{\operatorname{argth}}
\newcommand{\Id}{\operatorname{Id}}
\renewcommand{\leq}{\leq}
\renewcommand{\geq}{\geq }

\newcommand{\dlim}{\lim}
\newcommand{\dsum}{\sum}
\newcommand{\dprod}{\prod}



%Définition de nouvelles couleurs : rgb(trois paramètres red green blue
entre 0 et 1); cmyk (quatre cyan magenta yellow black) entre 0 et 1;
gray (entre 0 et 1) et black, white, red, green, blue, cyan, magenta,
yellow% definecolor{0gris}{gray}{0.8} 
% Nouvelle commande pour encadrer le titre car shabox ne veut que d'une
seule ligne; ATTENTION A LA TAILLE; petite différence avec shadowbox ou
doublebox, voire fcolorbox ou colorbox (au lieu de shabox; laisser le
parbox tranquille sauf pour la taille de la boîte
\newcommand{\Tbox}[1]{\begin{center} \shabox{\parbox{0.6
\linewidth}{#1}} \end{center}} %[1] pour 1 paramètre ; #1 pour ce que
fait le 1er paramètre; entre accolades ce que fait la commande
%Mise en page en mode fancy : en-têtes et pieds de pages puis
définition des en-têtes et pieds de pages\pagestyle{fancy}
\lhead{ECE 2 - Mathématiques \\
Quentin Dunstetter - ENC-Bessières 2011$\backslash$2012}
\chead{}
\rhead{Ecricome 2008}
\rfoot[ \ \thepage]{\thepage}
\cfoot{}
\lfoot{}
\thispagestyle{fancy} %Mise en page de la 1ère page en mode fancy
%Trait en bas et en haut de la page (entre en-tête et texte et texte et
pied de page)\renewcommand{\footrulewidth}{0.4pt}
\renewcommand{\headrulewidth}{0.4pt}

\begin{center}
\fbox{\bf {\large ECRICOME 2008}}
\end{center}
\vspace{1,5 cm}
\underline{\bf Exercice 1}\\
A tout couple $(a,b)$ de deux réels, on associe la matrice $M(a,b)$
définie par :
\[
M(a,b) = 
\begin{smatrix}
a + 2b & -b & -2b\\
2b & a-b & -4b\\
-b & b & a + 3b
\end{smatrix}
\]
On désigne par $E$ l'ensemble des matrices $M(a,b)$ où $a$ et $b$
décrivent $\R$. Ainsi : 
\[
E = \{M(a,b)/ (a,b)\in \R^{2} \}
\]
On note $I$ la matrice identité $M(0,1)$ et $A$ la matrice suivante :
\[
A = 
\begin{smatrix}
2 & -1 & -2\\
2 & -1 & -4\\
-1 & 1 & 3
\end{smatrix}
\]
\begin{noliste}{1.}
 \setlength{\itemsep}{4mm}
\item Montrer que $E$ est un sous-espace vectoriel de l'espace
vectoriel $\mathfrak{M}_{3}(\R)$ des matrices carrées d'ordre 3.
\item Donner une base de $E$ ainsi que sa dimension.
\item Vérifier que les réels 1 et 2 sont deux valeurs propres de $A$.\\
Donner les dimensions des sous-espaces propres associés à ces valeurs
propres. \\
En déduire que $A$ est diagonalisable.
\item Déterminer deux matrices $P$ et $D$ de $\mathfrak{M}_{3}(\R)$
vérifiant les conditions suivantes :
 \begin{noliste}{$\sbullet$}
 \item $P$ est inversible et ses trois éléments diagonaux sont égaux à
1.
 \item $D = (d_{i,j})$ est diagonale avec $d_{1,1} = 2$
 \item $D = P^{-1}AP$
 \end{noliste}
 Donner l'expression de la matrice $P^{-1}$.
\item Prouver que la matrice $D(a,b) = P^{-1}M(a,b)P$ est une matrice
diagonale.
\item Montrer que $M(a,b)$ est inversible si et seulement si $D(a,b)$
est inversible.\\
En déduire une condition nécessaire et suffisante portant sur $a$ et
$b$ pour que $M(a,b)$ soit inversible.
\item Prouver que $[M(a,b)]^{2} = I$ si et seulement si $[D(a,b)]^{2} =
I$\\
En déduire l'existence de quatre matrices $M(a,b)$ que l'on
déterminera, vérifiant \\
$[M(a,b)]^{2} = I$.
\end{noliste}

\newpage

\underline{\bf Exercice 2}\\

On considère les fonctions suivantes :
$g(x,y) = 1 + \ln(x + y) $ (fonction des variables $x$ et $y$)\\
et pour $p\in \N^*, \left\{ 
\begin{array}{ll}
 f_{p}(x) = g(x,p) \\
h_{p}(x) = x-f_{p}(x)
\end{array}
\right.$ (famille de fonctions de la variable réelle $x$).\\
On note $(C_{p})$ la courbe représentative de la fonction $f_{p}$.

\begin{noliste}{1.}
 \setlength{\itemsep}{4mm}
\item {\bf Recherche d'extremum éventuel de la fonction $g$}
\begin{noliste}{a)}
 \setlength{\itemsep}{2mm}
\item Représenter, relativement à un repère orthonormé du plan, le
domaine de définition $D$ de la fonction $g$. On hachurera $D$. On
admet que cet ensemble de définition est un ouvert de $\R^{2}$.
\item Déterminer sur $D$ les dérivées partielles premières de $g$. La
fonction $g$ admet-elle un extremum sur $D$ ?
\end{noliste}
\item {\bf Étude de la fonction $f_{1}$}
\begin{noliste}{a)}
 \setlength{\itemsep}{2mm}
\item Donner le domaine de définition de $f_{1}$.
\item Déterminer le développement limité en 0, à l'ordre 2, de la
fonction $f_{1}$.
\item En déduire une équation de la tangente à $C_{1}$ au point
d'abscisse 0, et la position locale de la courbe $C_{1}$ par rapport à
cette tangente.
\item Déterminer $\underset{x\to + \infty}{\lim}f_{1}(x)$,
$\underset{x\to + \infty}{\lim}\dfrac{f_{1}(x)}{x}$. Donner une
interprétation graphique de ces limites.
\end{noliste}
\item {\bf Étude d'une suite $(\alpha_{p})_{p\in \N^*}$}
\begin{noliste}{a)}
 \setlength{\itemsep}{2mm}
\item Montrer que l'équation $f_{p}(x) = x$ admet une unique solution
$\alpha_{p}$ sur l'intevalle $]0; + \infty[$. (On ne cherchera pas à
calculer $\alpha_{p}$).
\item Déterminer le signe de $h_{p}(\alpha_{p + 1})$ et en déduire que
la suite $(\alpha_{p})_{p\geq 1}$ est monotone.
\item Prouver que l'on a :
\[
\forall p\in \N^*, \quad \alpha_{p}\geq 1 + \ln(p)
\]
Quel est le comportement de la suite $(\alpha_{p})_{p\geq 1}$ lorsque
$p$ tend vers $ + \infty$ ?
\end{noliste}
\item {\bf Valeur approchée de $\alpha_{1}$}\\
On admet que le réel $\alpha_{1}$ appartient à l'intervalle $[1;3]$.
On définit la suite $(u_{n})_{n\in \N}$ par : $\left\{ 
\begin{array}{ll}
 u_{0} = 1\\
\forall n\in \N, u_{n + 1} = f_{1}(u_{n})
\end{array}
\right.$

\begin{noliste}{a)}
 \setlength{\itemsep}{2mm}
\item Démontrer par récurrence que pour tout entier naturel $n$ :
$u_{n}\geq 1$.
\item Appliquer à $f_{1}$ l'inégalité des accroissements finis entre
$\alpha_{1}$ et $u_{n}$ et en déduire que pour tout entier naturel $n$
: 
\[
|u_{n}-\alpha_{1}|\leq (\dfrac{1}{2})^{n-1}
\]
\item Déterminer un entier naturel $n_{0}$ de telle sorte que si $n$
est supérieur ou égal à $n_{0}$ alors $|u_{n}-\alpha_{1}|$ est
inférieur ou égal à $10^{-4}$.
\item Écrire un programme en langage \Scilab{} permettant d'obtenir les
valeurs de $n_{0}$ et de $u_{n_{0}}$
\end{noliste}
\end{noliste}

\underline{\bf Exercice 3}\\
On s'intéresse dans cet exercice à l'étude de trois jeux présents dans
une fête foraine.
\begin{noliste}{1.}
 \setlength{\itemsep}{4mm}
\item {\bf Premier jeu}\\
Pour ce premier jeu de hasard, la mise pour chaque partie est de 1
euro. L'observation montre qu'une partie est gagnée avec la probabilité
$\dfrac{1}{10}$, perdue avec la probabilité $\dfrac{9}{10}$.\\
Toute partie gagnée rapporte 3 euros. Les différentes parties sont
indépendantes.
Une personne décide de jouer $N$ parties $(N\geq 2)$. On note $X_{N}$
la variable aléatoire représentant le nombre de parties gagnées et
$Y_{N}$ la variable aléatoire représentant le gain algébrique du
joueur.
\begin{noliste}{a)}
 \setlength{\itemsep}{2mm}
\item Donner la loi de $X_{N}$ ainsi que la valeur de l'espérance et de
la variance de cette variable.
\item Exprimer $Y_{N}$ en fonction de $X_{N}$. En déduire la valeur de
l'espérance et de la variance de $Y_{N}$.
\item La personne décide de jouer 60 parties. On admet que l'on peut
approcher $X_{60}$ par une loi de Poisson.
\begin{nonoliste}{(i)}
\item Donner le paramètre de cette loi de Poisson.
\item A l'issue des 60 parties, quelle est la probabilité que le joueur
perde moins de 50 euros ? (cette probabilité sera impérativement
calculée en utilisant l'annexe située à la fin de l'exercice)\\
\end{nonoliste}
\end{noliste}
\newpage
\item{\bf Deuxième jeu}\\
Pour ce deuxième jeu, le participant lance trois fléchettes dans une
cible circulaire de centre O et de rayon 1. Pour $1\leq i\leq 3$, on
note $X_{i}$ la variable aléatoire égale à la distance du point
d'impact de centre O de la $i^{ème}$ fléchette. Ces trois variables
aléatoires $X_{1}, X_{2}, X_{3}$ de même loi, indépendantes, sont des
variables à densité dont une densité $f$ est définie par :
\[
f(x) = \left\{ 
\begin{array}{ll}
 2x \quad si \quad x\in [0;1] \\
0\quad sinon
\end{array}
\right.
\]
Le joueur gagne si la distance la plus proche du centre O se trouve à
une distance inférieure à $\dfrac{1}{5}$ de ce centre. Enfin, on note
$M$ la variable aléatoire représentant la plus petite des trois
distances $X_{1}, X_{2}, X_{3}$.
\begin{noliste}{a)}
 \setlength{\itemsep}{2mm}
\item Vérifier que $f$ est une densité de probabilité et déterminer la
fonction de répartition $F$ de $X_{i}$.
\item Déterminer l'espérance de $X_{i}$.
\item Exprimer l'évènement $[M>t]$ à l'aide des évènements $[X_{1}>t],
[X_{2}>t], [X_{3}>t]$ pour tout réel $t$.
\item Déterminer la fonction de répartition $F_{M}$ de $M$ et montrer
que $M$ est une variable à densité et en donner une densité notée
$f_{M}$.
\item Quelle est la probabilité de l'évènement $G = $"le joueur gagne
la partie" ?\\
\end{noliste}
\underline{\bf Troisième jeu}\\
Pour ce dernier jeu, le participant lance successivement $n$ boules au
hasard dans $N$ cases numérotées de 1 à $N$ avec $N\geq 2$. On suppose
que les différents lancers de boules sont indépendants et que la
probabilité pour qu'une boule quelconque tombe dans une case donnée est
$\dfrac{1}{N}$. Une case peut contenir plusieurs boules. \\
Le gain étant fonction du nombre de cases atteintes, on étudie la
variable aléatoire $T_{n}$ égale au nombre de cases non vides à l'issue
des $n$ lancers.
\begin{noliste}{a)}
 \setlength{\itemsep}{2mm}
\item Déterminer en fonction de $n$ et de $N$ les valeurs prises par
$T_{n}$. 
\item Donner les lois de $T_{1}$ et $T_{2}$.
\item Déterminer, lorsque $n\geq 2$, la probabilité des évènements
$[T_{n} = 1], [T_{n} = 2]$,\\
$[T_{n} = n]$. (pour la dernière probabilité on distinguera deux cas
$n>N$ et $n\leq N$).
\item À l'aide de la formule des probabilités totales, justifier
l'égalité $(I)$ suivante, pour tout entier $k$ tel que $1\leq k \leq
n$,
 
\[
(I)\qquad P\left(\Ev{T_{n + 1} = k]}\right) =
\dfrac{k}{N}P\left(\Ev{T_{N} = k]}\right) + \dfrac{N-k +
1}{N}P\left(\Ev{T_{n} = k-1]}\right)
\]
\item Afin de calculer l'espérance $\E(T_{n})$ de la variable $T_{n}$,
on considère la fonction polynômiale $G_{n}$ définie par :
\[
\forall x\in \R, \quad G_{n}(x) = \Sum{k = 1}{n} P\left(\Ev{T_{n} =
k]}\right)x^{k}
\]
 \begin{nonoliste}{(i)}
 \item Quelle est la valeur de $G_{n}(1)$ ?
 \item Exprimer $\E(T_{n})$ en fonction de $G_{n}'(1)$.
 \item En utilisant la relation $(I)$, montrer que :
 
\[
\forall x\in \R, \quad G_{n + 1}(x) = \dfrac{1}{N}(x-x^{2})G_{N}'(x) +
xG_{n}(x)
\]
 \item En dérivant l'expression précédente, en déduire que :
 
\[
\E(T_{n + 1}) = (1-\dfrac{1}{N})\E(T_{n}) + 1
\]
 \item Prouver enfin que l'espérance de la variable $T_{n}$ est donnée
par :
 
\[
\E(T_{n}) = N\left[1-\left(1-\dfrac{1}{N}\right)^{n} \right]
\]
\end{nonoliste}
\end{noliste}



\end{noliste}



\end{document} 