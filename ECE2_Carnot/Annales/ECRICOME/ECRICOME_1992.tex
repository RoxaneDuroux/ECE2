\documentclass[11pt]{article}%
\usepackage{geometry}%
\geometry{a4paper,
 lmargin = 2cm,rmargin = 2cm,tmargin = 2.5cm,bmargin = 2.5cm}

\input{../../macros.tex}

\pagestyle{fancy} %
\lhead{ECE2 \hfill Mathématiques\\
} %
\chead{\hrule} %
\rhead{} %
\lfoot{} %
\cfoot{} %
\rfoot{\thepage} %

\renewcommand{\headrulewidth}{0pt}% : Trace un trait de séparation
 % de largeur 0,4 point. Mettre 0pt
 % pour supprimer le trait.

\renewcommand{\footrulewidth}{0.4pt}% : Trace un trait de séparation
 % de largeur 0,4 point. Mettre 0pt
 % pour supprimer le trait.

\setlength{\headheight}{14pt}

\title{\bf \vspace{-2cm} ECRICOME 1992} %
\author{} %
\date{} %
\begin{document}

\maketitle %
\vspace{-1.4cm}\hrule %
\thispagestyle{fancy}

\vspace*{.2cm}


% DEBUT DU DOC À MODIFIER : tout virer jusqu'au début de l'exo

%Définition et changement de valeurs de
compteurs%newcounter{cpt1}{section} compteur cpt1 remis à 0 à chaque
aumentation par stepcounter du compteur section%setcounter{cpt1}{3} on
met le compteur à 3%addtocounter{cpt1}{5} on ajoute 5 au compteur%
stepcounter{cpt1} on ajoute 1% ifthenelse{test}{alors}{sinon} (page
206) pour subordonner à une condition % whiledo{test}{commande} pour
faire une boucle (page 206 aussi) % value{cpt1} pour noter dans le
document la valeur de cpt1 
%Définition définitive d'opérateurs
mathématiques\newcommand{\ch}{\operatorname{ch}} 
\newcommand{\sh}{\operatorname{sh}}
\renewcommand{\tanh}{\operatorname{th}}
\renewcommand{\sinh}{\operatorname{sh}}
\renewcommand{\cosh}{\operatorname{ch}}
\newcommand{\argsh}{\operatorname{argsh}}
\newcommand{\argch}{\operatorname{argch}}
\newcommand{\argth}{\operatorname{argth}}
\newcommand{\Id}{\operatorname{Id}}
\renewcommand{\leq}{\leq}
\renewcommand{\geq}{\geq }

\newcommand{\dlim}{\lim}
\newcommand{\dsum}{\sum}
\newcommand{\dint}{\int}
\newcommand{\dprod}{\prod}



%Définition de nouvelles couleurs : rgb(trois paramètres red green blue
entre 0 et 1); cmyk (quatre cyan magenta yellow black) entre 0 et 1;
gray (entre 0 et 1) et black, white, red, green, blue, cyan, magenta,
yellow% definecolor{0gris}{gray}{0.8} 
% Nouvelle commande pour encadrer le titre car shabox ne veut que d'une
seule ligne; ATTENTION A LA TAILLE; petite différence avec shadowbox ou
doublebox, voire fcolorbox ou colorbox (au lieu de shabox; laisser le
parbox tranquille sauf pour la taille de la boîte
\newcommand{\Tbox}[1]{\begin{center} \shabox{\parbox{0.6
\linewidth}{#1}} \end{center}} %[1] pour 1 paramètre ; #1 pour ce que
fait le 1er paramètre; entre accolades ce que fait la commande
%Mise en page en mode fancy : en-têtes et pieds de pages puis
définition des en-têtes et pieds de pages\pagestyle{fancy}
\lhead{ECE 2 - Mathématiques \\
Quentin Dunstetter - ENC-Bessières 2011$\backslash$2012}
\chead{}
\rhead{Ecricome 1992}
\rfoot[ \ \thepage]{\thepage}
\cfoot{}
\lfoot{}
\thispagestyle{fancy} %Mise en page de la 1ère page en mode fancy
%Trait en bas et en haut de la page (entre en-tête et texte et texte et
pied de page)\renewcommand{\footrulewidth}{0.4pt}
\renewcommand{\headrulewidth}{0.4pt}


\begin{center}
{\Huge ECRICOME Eco 1992}
\end{center}

\section*{Exercice1}

On définit la suite $(u_{n})_{n}
\[
_{\geq }
\]
_{0}$ par $\forall n\in 
\N,\quad u_{n} = \dint{0}{\dfrac{\pi }{4}}\tan ^{2n + 2}(t)dt$

\begin{noliste}{1.}
 \setlength{\itemsep}{4mm}
\item 
\begin{noliste}{a)}
 \setlength{\itemsep}{2mm}
\item Rappeler la valeur de la dérivée de la fonction tangente sur
$\left] -\dfrac{\pi }{2},\dfrac{\pi }{2}\right[ $.

\item Calculer alors $u_{0}$.
\end{noliste}

\item Montrer que la suite $(u_{n})_{n}
\[
_{\in }
\]
_{\N}$ est décroissante.

\item Montrer que : $\forall n\in \N,\quad u_{n + 1} + u_{n} =
\dfrac{1}{2n + 3}$

\item En déduire que : $\forall n\in \N$,$\quad \dfrac{1}{2n + 3}\leq
u_{n}\leq \dfrac{1}{2(2n + 1)}$ Puis donner un équivalent
simple de $u_{n}$ lorsque $n$ tend vers + $\infty $.

\item On pose $S_{n} = \Sum{k = 0}{n}\dfrac{(-1)^{k}}{2k + 1}$

\begin{noliste}{a)}
 \setlength{\itemsep}{2mm}
\item Montrer que : $\forall n\in \N$,$\quad S_{n} = \dfrac{\pi }{4} +
(-1)^{n}u_{n}$

\item En déduire la limite de $(S_{n})_{n}
\[
_{\in }
\]
_{\N}$ et un équivalent de $\left( S_{n}-\dfrac{\pi }{4}\right) $
lorsque $n$ tend vers + $\infty $.
\end{noliste}
\end{noliste}

\section*{Exercice 2}

Un distributeur de jouets distingue trois catégories de jouets :

\begin{quote}
\noindent T : les jouets traditionnels tels que poupées, peluches ;\\
M : les jouets liés à la mode inspirés directement d'un livre, un film,
une émission ;\\
S : les jouets scientifiques vulgarisant une technique récente.
\end{quote}

\noindent Il estime que

\begin{noliste}{$\sbullet$}
\item[i)] Le client qui a acheté un jouet traditionnel une année pour
Noël
choisira, l'année suivante, un jouet de l'une des trois catégories avec
une équiprobabilité ;

\item[ii)] Le client qui a acheté un jouet inspiré par la mode optera
l'année suivante \\
pour un jouet T avec la probabilité $\dfrac{1}{4}$,\\
pour un jouet M avec la probabilité $\dfrac{1}{4}$,\\
pour un jouet S avec la probabilité $\dfrac{1}{2}$ ;

\item[iii)] pour un jouet T avec la probabilité $\dfrac{1}{4}$,\\
pour un jouet M avec la probabilité $\dfrac{1}{2}$,\\
pour un jouet S avec la probabilité $\dfrac{1}{4}$.
\end{noliste}

\noindent Le volume des ventes de ce commerçant vient de se composer

\begin{quote}
\noindent d'une part $p_{0} = \dfrac{45}{100}$ de jouets de la
catégorie T\\
d'une part $q_{0} = \dfrac{25}{100}$ de jouets de la catégorie M\\
et d'une part $r_{0} = \dfrac{30}{100}$ de jouets de la catégorie S.
\end{quote}

\noindent On désigne par $p_{n},q_{n},r_{n}$, les parts respectives des
jouets T, M, S dans les ventes du distributeur le $n^{\grave{e}me}$
Noël
suivant.

\begin{noliste}{1.}
 \setlength{\itemsep}{4mm}
\item Montrer que le triplet $(p_{n + 1},q_{n + 1},r_{n + 1})$
s'exprime en
fonction du triplet $(p_{n},q_{n},r_{n})$ au moyen d'une matrice $A$
qu'on
formera.

\item Soit $P$ la matrice définie par : $P = \left( 
\begin{array}{ccc}
3 & 2 & 0 \\
4 & -1 & 1 \\
4 & -1 & -1
\end{array}
\right) $

\begin{noliste}{a)}
 \setlength{\itemsep}{2mm}
\item Montrer à l'aide de la méthode de Gauss que $P$ est inversible et
déterminer $P^{-1}$.

\item Déterminer la matrice $D = P^{-1}AP$.

\item Montrer par récurrence sur $n\in \N^{\times }$ que : $A^{n} =
PD^{n}P^{-1}$.

\item Calculer $A^{n}$ pour $n\in \N^{\times }$.
\end{noliste}

\item Exprimer $(p_{n},q_{n},r_{n})$ directement en fonction de $n$.

\item Quelles parts à long terme les trois catégories de jouets
représenteront-elles dans la vente si l'attitude des consommateurs
reste
constante ?
\end{noliste}

\section*{Problème}

Dans tout le problème, si $a$ et $b$ sont des entiers naturels, on note
$[\hspace{-0.15em}[a,b]\hspace{-0.13em}] = \{i\in \mathbb{N\quad
}/\quad
a\leq i\leq b\}.$

\subsection*{Première partie}

La société ACTUEL SA est une société de vente à domicile dont la
politique
de vente est la suivante : un collaborateur de cette société contacte
par téléphone 100 clients potentiels auxquels il propose deux produits
que l'on
nommera A et B.\\
Pour un entier i compris entre 1 et 100, on notera :

\begin{quote}
\noindent R$_{i}$ l'évènement : {"} la ième personne contactée reçoit
le
collaborateur {"}.\\
A$_{i}$ l'évènement : {"} la i-ème personne contactée achète le produit
A {"}.\\
B$_{i}$ l'évènement : {"} la i-ème personne contactée achète le produit
B {"}.
\end{quote}

\noindent On fait les hypothèses suivantes :

\begin{quote}
\noindent (i) les évènements $R_{1},R_{2},...,R_{100}$ sont
mutuellement indépendants et $\forall i\in \lbrack
\hspace{-0.15em}[1,100]\hspace{-0.13em}],\quad P\left(\Ev{R_{i}}\right)
= 0,2$.\\
(ii) les évènements $A_{1},B_{1},A_{2},B_{2},...,A_{100},B_{100}$ sont
mutuellement indépendants.\\
(iii) $\forall i\in \lbrack
\hspace{-0.15em}[1,100]\hspace{-0.13em}],\quad
P\left(\Ev{A_{i}/R_{i}}\right) = 0,5$ et
$P\left(\Ev{B_{i}/R_{i}}\right) = 0,1$.
\end{quote}

\noindent Enfin, si $i\in \lbrack
\hspace{-0.15em}[1,100]\hspace{-0.13em}]$,
on notera $X_{i}$ et $Y_{i}$ les variables aléatoires définies par :

\begin{quote}
\noindent $(X_{i} = 1)$ si et seulement si ($A_{i}$ est réalisé)\\
$(X_{i} = 0)$ si et seulement si ($A_{i}$ n'est pas réalisé),\\
$(Y_{i} = 2$) si et seulement si ($A_{i}$ et $B_{i}$ sont réalisés),\\
$(Y_{i} = 1)$ si et seulement si (un des deux évènements (et un
seulement) $A_{i}$ ou $B_{i}$ est réalisé),\\
$(Y_{i} = 0)$ si et seulement si ($A_{i}$ et $B_{i}$ ne sont pas
réalisés).
\end{quote}

\noindent Par hypothèse $X_{1},X_{2},...,X_{100}$ sont mutuellement
indépendantes ainsi que $Y_{1},Y_{2},...,Y_{100}$.

\begin{noliste}{1.}
 \setlength{\itemsep}{4mm}
\item Soit $i\in \lbrack \hspace{-0.15em}[1,100]\hspace{-0.13em}]$,
calculer 
$P\left(\Ev{A_{i}}\right)$. Quelle est la loi de $X_{i}$ ?\\
On note S la variable aléatoire $S = X_{1} + X_{2} +... + X_{100}$.

\item Quelle est la loi de $S$ ? Rappeler les valeurs de $\E(S)$ et de
$\V(S)$.\\
Compte tenu des hypothèses, on estime que l'on peut approcher $S$ par
une
loi de Poisson de paramètre $\lambda = 10$.\\
\textbf{N.B. Une table relative à la loi de poisson est fournie en
annexe du
sujet.}\\
Dans les questions 3 et 4, on utilisera une précision de $10^{-4}$.

\item Calculer alors $P\left(\Ev{S\geq 7}\right)$.

\item On suppose que la vente d'un produit A rapporte au collaborateur
un
gain de 1000 F.

\begin{noliste}{a)}
 \setlength{\itemsep}{2mm}
\item Quelle est la probabilité qu'à la fin du mois le collaborateur
gagne
au moins 7000 F ?

\item On suppose de plus, que s'il vend au moins 8 produits A dans le
mois,
le collaborateur perçoit une prime de 500 F.\\
Sachant qu'il a gagné au moins 7000 F dans le mois, quelle est la
probabilité
que le collaborateur ait touché la prime ?
\end{noliste}

\item Soit $i\in \lbrack \hspace{-0.15em}[1,100]\hspace{-0.13em}]$

\begin{noliste}{a)}
 \setlength{\itemsep}{2mm}
\item Pour $k\in \{0,1,2\}$, calculer $P\left(\Ev{Y_{i} = k}\right)$.

\item Calculer l'espérance et la variance de $Y_{i}$.
\end{noliste}

\item On note $Z = Y_{1} + Y_{2}$

\begin{noliste}{a)}
 \setlength{\itemsep}{2mm}
\item Quelles sont les valeurs que peut prendre Z ?

\item Déterminer la loi de Z, puis son espérance et sa variance.
\end{noliste}
\end{noliste}

\subsection*{Deuxième partie}

Dans cette partie $n$ désigne un entier naturel supérieur ou égal à
1.\\
Si $p\in \ ]0,1[$, on note $F_{n,p}$ la fonction de répartition d'une
loi
binomiale de paramètres $n$ et $p$.\\
On notera $q = 1-p$.\\
Si $\lambda \in \ ]0, + \infty \lbrack,$ on note $\pi 
\[
_{\lambda }$ la
fonction de répartition d'une loi de Poisson de paramètre $\lambda $.\\
Ainsi : $\forall k\in \lbrack
\hspace{-0.15em}[0,n]\hspace{-0.13em}],\quad
F_{n,p}(k) = \Sum{i = 0}{k}C_{n}{i}p^{i}q^{n-i}$ et $\forall
k\geq n,\quad F_{n,p}(k) = 1$\\
et $\forall k\in N,\quad \pi_{\lambda }(k) = \left[ \ \Sum{i =
0}{k}\dfrac{\lambda ^{i}}{i!}\right] e^{-\lambda }$

\begin{noliste}{1.}
 \setlength{\itemsep}{4mm}
\item \textbf{Expression intégrale de }$\pi_{\mathbf{\lambda
}}(k)$\textbf{\ :} dans cette question $k\in \N$. \\
On note $I_{k} = \dfrac{1}{k!}\dint{\lambda }{+ \infty }e^{-x}x^{k}dx$

\begin{noliste}{a)}
 \setlength{\itemsep}{2mm}
\item Montrer l'existence de $I_{0}$ et donner sa valeur.

\item Démontrer que pour tout $k\in \N$ l'existence de $I_{k}$ et la
relation :
\[
I_{k + 1} = \dfrac{\lambda ^{k + 1}}{(k + 1)!}e^{-\lambda } + I_{k}
\]

\item En déduire que : $\forall k\in N,\quad \pi_{\lambda }(k) =
I_{k}$.
\end{noliste}

\item \textbf{Expression intégrale de }$F_{n,p}(k)$ \textbf{ :}

\begin{noliste}{a)}
 \setlength{\itemsep}{2mm}
\item Vérifier que $\forall k\in \lbrack
\hspace{-0.15em}[1,n]\hspace{-0.13em}],\quad kC_{n}{k} = (n-k +
1)C_{n}{k-1}$

\item Montrer que pour $n\geq 2,$ $\forall k\in \lbrack
\hspace{-0.15em}[1,n-1]\hspace{-0.13em}]$, on a :
\[
\dint{0}{q}t^{n-k-1}(1-t)^{k}dt = \dfrac{p^{k}q^{n-k}}{n-k} +
\dfrac{k}{n-k}\dint{0}{q}t^{n-k}(1-t)^{k-1}dt
\]

\item En déduire 
\[
\forall n\in \N^{\times },\forall k\in \lbrack
\hspace{-0.15em}[0,n-1]\hspace{-0.13em}],\quad
F_{n,p}(k) = (n-k)C_{n}{k}\dint{0}{q}t^{n-k-1}(1-t)^{k-1}dt
\]
(On n'oubliera pas les cas où $k = 0$ et le cas particulier de $n = 1$)

\item Vérifier alors que $\forall n\in \N^{\times },\forall k\in
\lbrack \hspace{-0.15em}[0,n-1]\hspace{-0.13em}]$, on a : 
\[
F_{n,p}(k) = (n-k)C_{n}{k}\int\nolimits_{p}{1}(1-t)^{n-k-1}t^{k}dt
\]
\end{noliste}

Dans toute la suite du problème $\lambda $ désigne un réel strictement
positif.

\item \textbf{Convergence en loi d'une suite de variables aléatoires
:}\\
Pour $n\in \N^{\times }$, on définit la variable aléatoire discrète
$S_{n}$ par : 
\[
\left\{ 
\begin{tabular}{ll}
$\forall k\in \lbrack \lbrack 0,n]]$ & $P\left(\Ev{S_{n} = k}\right) =
C_{n}{k}\left[ \ \dfrac{\lambda }{n}\right] ^{k}\left[ 1-\dfrac{\lambda
}{n}\right] ^{n-k}$ \\
$\forall k>n$ & $P\left(\Ev{S_{n} = k}\right) = 0$\end{tabular}\right.
\]

\begin{noliste}{a)}
 \setlength{\itemsep}{2mm}
\item Vérifier que $\forall k\in \lbrack
\hspace{-0.15em}[1,n]\hspace{-0.13em}],\quad P\left(\Ev{S_{n} =
k}\right) = \dfrac{\lambda ^{k}}{k!}\left[ \ \dprod\limits_{j =
0}{k-1}(1-\dfrac{j}{n})\right] \dfrac{\left[ 1-\dfrac{\lambda
}{n}\right] ^{n}}{\left[ 1-\dfrac{\lambda }{n}\right] ^{k}}$

\item Montrer que $\underset{n\rightarrow + \infty }{\lim }\left[
1-\dfrac{\lambda }{n}\right] ^{n} = e^{-\lambda }$

\item Soit $k\in \N$, déduire de ce qui précède que
$\underset{n\rightarrow + \infty }{\lim }P\left(\Ev{S_{n} = k}\right) =
\dfrac{\lambda ^{k}}{k!}e^{-\lambda
} $ \\
Ainsi, si l'on considère une variable aléatoire $X$ qui suit une loi de
Poisson de paramètre $\lambda $; on a : 
\[
\forall k\in \N,\quad \underset{n\rightarrow + \infty }{\lim
}P\left(\Ev{S_{n} = k}\right) = P\left(\Ev{X = k}\right)
\]
\end{noliste}

\item \textbf{Dans cette question, on suppose que} $n>\lambda $

\begin{noliste}{a)}
 \setlength{\itemsep}{2mm}
\item Déduire de la question 3 que $\forall k\in \N,\quad
\underset{n\rightarrow + \infty }{\lim }F_{n,\dfrac{\lambda }{n}}(k) =
\pi_{\lambda }(x)$

\item À l'aide de la question 2, vérifier que $\forall k\in \lbrack
\hspace{-0.15em}[0,n-1]\hspace{-0.13em}],$ on a :
\[
F_{n,\dfrac{\lambda }{n}}(k) =
\dfrac{1}{k!}\dfrac{(n-1)...(n-k)}{n^{k}}\dint{\lambda }{n}\left[
1-\dfrac{x}{n}\right] ^{n-k-1}x^{k}dx
\]

\item Déduire alors de ce qui précède que pour tout k fixé tel que
$0\leq k\leq n-1$, on a :
\[
\underset{n\rightarrow + \infty }{\lim }\dint{\lambda }{n}\left[
1-\dfrac{x}{n}\right] ^{n-k-1}x^{k}dx = \int\nolimits_{\lambda }{+
\infty
}e^{-x}x^{k}dx
\]
\end{noliste}
\end{noliste}

\label{fin}

\end{document}

