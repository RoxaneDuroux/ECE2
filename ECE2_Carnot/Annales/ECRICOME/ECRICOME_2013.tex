\documentclass[11pt]{article}%
\usepackage{geometry}%
\geometry{a4paper,
 lmargin = 2cm,rmargin = 2cm,tmargin = 2.5cm,bmargin = 2.5cm}

\input{../../macros.tex}

\pagestyle{fancy} %
\lhead{ECE2 \hfill Mathématiques\\
} %
\chead{\hrule} %
\rhead{} %
\lfoot{} %
\cfoot{} %
\rfoot{\thepage} %

\renewcommand{\headrulewidth}{0pt}% : Trace un trait de séparation
 % de largeur 0,4 point. Mettre 0pt
 % pour supprimer le trait.

\renewcommand{\footrulewidth}{0.4pt}% : Trace un trait de séparation
 % de largeur 0,4 point. Mettre 0pt
 % pour supprimer le trait.

\setlength{\headheight}{14pt}

\title{\bf \vspace{-2cm} ECRICOME 2013} %
\author{} %
\date{} %
\begin{document}

\maketitle %
\vspace{-1.4cm}\hrule %
\thispagestyle{fancy}

\vspace*{.2cm}


% DEBUT DU DOC À MODIFIER : tout virer jusqu'au début de l'exo

%Définition et changement de valeurs de
compteurs%newcounter{cpt1}{section} compteur cpt1 remis à 0 à chaque
aumentation par stepcounter du compteur section%setcounter{cpt1}{3} on
met le compteur à 3%addtocounter{cpt1}{5} on ajoute 5 au compteur%
stepcounter{cpt1} on ajoute 1% ifthenelse{test}{alors}{sinon} (page
206) pour subordonner à une condition % whiledo{test}{commande} pour
faire une boucle (page 206 aussi) % value{cpt1} pour noter dans le
document la valeur de cpt1 
%Définition définitive d'opérateurs
mathématiques\newcommand{\ch}{\operatorname{ch}} 
\newcommand{\sh}{\operatorname{sh}}
\renewcommand{\tanh}{\operatorname{th}}
\renewcommand{\sinh}{\operatorname{sh}}
\renewcommand{\cosh}{\operatorname{ch}}
\newcommand{\argsh}{\operatorname{argsh}}
\newcommand{\argch}{\operatorname{argch}}
\newcommand{\argth}{\operatorname{argth}}
\newcommand{\Id}{\operatorname{Id}}
\renewcommand{\leq}{\leq}
\renewcommand{\geq}{\geq }

\newcommand{\dlim}{\lim}
\newcommand{\dsum}{\sum}
\newcommand{\dint}{\int}
\newcommand{\dprod}{\prod}



%Définition de nouvelles couleurs : rgb(trois paramètres red green blue
entre 0 et 1); cmyk (quatre cyan magenta yellow black) entre 0 et 1;
gray (entre 0 et 1) et black, white, red, green, blue, cyan, magenta,
yellow% definecolor{0gris}{gray}{0.8} 
% Nouvelle commande pour encadrer le titre car shabox ne veut que d'une
seule ligne; ATTENTION A LA TAILLE; petite différence avec shadowbox ou
doublebox, voire fcolorbox ou colorbox (au lieu de shabox; laisser le
parbox tranquille sauf pour la taille de la boîte
\newcommand{\Tbox}[1]{\begin{center} \shabox{\parbox{0.6
\linewidth}{#1}} \end{center}} %[1] pour 1 paramètre ; #1 pour ce que
fait le 1er paramètre; entre accolades ce que fait la commande
%Mise en page en mode fancy : en-têtes et pieds de pages puis
définition des en-têtes et pieds de pages\pagestyle{fancy}
\lhead{ECE 2 - Mathématiques \\
Quentin Dunstetter - ENC-Bessières 2012$\backslash$2013}
\chead{}
\rhead{Ecricome 2013}
\rfoot[ \ \thepage]{\thepage}
\cfoot{}
\lfoot{}
\thispagestyle{fancy} %Mise en page de la 1ère page en mode fancy
%Trait en bas et en haut de la page (entre en-tête et texte et texte et
pied de page)\renewcommand{\footrulewidth}{0.4pt}
\renewcommand{\headrulewidth}{0.4pt}


\begin{center}
{\Huge ECRICOME Eco 2013}
\end{center}

\section*{EXERCICE 1}
\noindent On désigne par $\mathfrak{M}_{3} (\R)$ l'ensemble des
matrices carrées de taille 3 à coefficients réels et par $0_{3}$ la
matrice nulle de $\mathfrak{M}_{3}(\R)$. \\
On pose $A = \begin{smatrix}
0 & 1 & 0 \\
0 & 0 & 1 \\
3 & -9 & 6 \\
\end{smatrix}
\in \mathfrak{M}_{3}(\R)$ ainsi que le polynôme $R$ défini par : 
\[
 \forall x \in \R, \ \ \ R(x) = x^{3} - 6 x^{2} + 9 x - 3. 
\]
Pour tout réel $\lambda$, on pose $X_\lambda = \begin{smatrix}
1 \\
\lambda \\
\lambda^{2} \\
\end{smatrix}
$. Pour finir, on introduit l'application $f$ définie par : 
\[
 \forall M \in \mathfrak{M}_{3}(\R), \ \ \ f(M) = AM + MA.
\]
\begin{noliste}{1.}
 \setlength{\itemsep}{4mm}

\item Montrer que $R'$ (la dérivée de $R$) admet deux racines réelles
distinctes $r_{1},\ r_{2}$ avec $r_{1}<r_{2}$ que l'on précisera. \\

\item Dresser le tableau de variations de $R$ en y ajoutant les valeurs
de $R$ en $r_{1}$ et $r_{2}$. \\

\item Justifier que $R$ admet trois raicnes $a,\ b,\ c$ avec $0 < a < b
< c$. \textit{On ne cherchera pas à calculer ces racines}. \\

\item Soit $\lambda$ un réel, calculer $A X_{\lambda}$ puis démontrer
que $X_{\lambda}$ est un vecteur propre de $A$ associé à la valeur
propre $\lambda$ si et seulement si $R(\lambda) = 0$. \\

\item Établir l'existence d'une matrice inversible $P$ et d'une matrice
diagonale $D$ telle que $A = P D P^{-1}$. Expliciter les matrices $P$
et $D$ en fonction des réels $a,\ b,\ c$. \\

\item Prouver que $f$ est une application linéaire et que : 
\[
 \forall M \in \mathfrak{M}_{3}(\R), \ \ \ f(M) = 0_{3} \Leftrightarrow
D M' + M' D = 0_{3}.
\]
où l'on a posé $M' = P^{-1} M P$. \\
\item Soit $N = \begin{smatrix}
p & q & r \\
s & t & u \\
v & w & x \\
\end{smatrix}
$. Déterminer les neuf coefficients de la matrice $DN + N D$. Que dire
de $N$ si $DN + ND = 0_{3}$ ? \\

\item Démontrer que $f$ est un isomorphisme.

\end{noliste}

\section*{EXERCICE 2}
\noindent On considère l'application $\varphi$ définie sur
$\R_+^{\ast}$ par : 
\[
 \forall x \in \R_+^{\ast}, \ \ \ \varphi(x) = \frac{ x \ln (x) - 1
}{x}.
\]
ainsi que la fonction numérique $f$ des variables réelles $x$ et $y$
définie par : 
\[
 \forall (x,y) \in \ ]0 ; + \infty[ \ \times ]0 ; + \infty[, \ \ \
f(x,y) = \frac{1}{x^{2}} - \frac{y}{x} + \frac{y^{2}}{2} + \exp \left(
- \frac{1}{x} \right) 
\]
où $ \exp$ désigne la fonction exponentielle.

\section*{I. Étude des zéros de $\varphi$.}

\begin{noliste}{1.}
 \setlength{\itemsep}{4mm}

\item Déterminer la limite de $\varphi (x)$ lorsque $x$ tend vers 0 par
valeurs positives. \\
Interpréter graphiquement cette limite. \\

\item Déterminer la limite de $\varphi(x)$ lorsque $x$ tend vers $ +
\infty$, ainsi que la limite de $\frac{\varphi(x)}{x}$ lorsque $x$ tend
vers $ + \infty$. Interpréter graphiquement cette limite. \\

\item Justifier la dérivabilité de $\varphi$ sur $\R_+^{\ast}$ et
déterminer sa dérivée. \\

\item Dresser le tableau de variation de $\varphi$ en faisant
apparaître les limites de $\varphi$ en $0^+ $ et $ + \infty$. \\

\item Prouver l'existence d'un unique réel $\alpha \in \R_+^{\ast}$ tel
que : 
\[
 \varphi(\alpha) = 0.
\]
Justifier que $\alpha \in [1 ; e ]$. 

\end{noliste}

\section*{II. Étude d'une suite réelle.}
\noindent On considère la suite $u$ définie par la relation de
récurrence suivante : 
\[
 \left\{
\begin{array}{cl}
 u_{0} = e ; \\
\forall n \in \N, \ \ \ u_{n + 1} = \varphi(u_{n}) + u_{n}. \\
\end{array}
\right.
\]

\begin{noliste}{1.}
 \setlength{\itemsep}{4mm}

\item Démontrer que pour tout entier $n$, $u_{n}$ existe et $u_{n} >
\alpha$. \\

\item Si cette suite est convergente de limite $L$, que peut valoir $L$
? \\

\item Prouver que la suite $u$ est strictement croissante. \\

\item La suite $u$ est-elle convergente ? \\

\item Soit $A$ un réel. Recopier et compléter le programme suivant afin
qu'il affiche le plus petit entier $n$ tel que $u_{n} \geq A$ : 

\begin{center} \fbox{ \parbox{0.4 \linewidth}{
\noindent program ecricome2013 ; \\
var n : integer ; \\
\rule{0.5cm}{0cm} u : real ; \\
\rule{0.5cm}{0cm} A : real ; \\
\rule{0.2cm}{0cm} function g(x : real) : real ; \\
\rule{0.2cm}{0cm} begin \\
\rule{0.2cm}{0cm} g : =.................; \\
\rule{0.2cm}{0cm} end ; \\
begin \\
\rule{0.5cm}{0cm} writeln('entrer un réel $A>0$') ; \\
\rule{0.5cm}{0cm} readln(A) ; \\
\rule{0.5cm}{0cm} u : = exp(1) ; n : = 0 ; \\
\rule{0.5cm}{0cm} while.............. do \\
\rule{0.7cm}{0cm} begin' \\
\rule{0.9cm}{0cm}................ ; \\
\rule{0.9cm}{0cm}................ ; \\
\rule{0.7cm}{0cm} end ; \\
writeln(.............. ) ; \\
end.
}
 } \end{center}

\end{noliste}

\section*{III. Extrema de $f$ sur $]0; + \infty[ \ \times ]0 ; + \infty
[$. }

\begin{noliste}{1.}
 \setlength{\itemsep}{4mm}

\item Justifier que $f$ est de classe $C^{2}$ sur l'ouvert $]0 ; +
\infty[ \ \times ]0 ; + \infty[ $. \\

\item Calculer les dérivées partielles premières et prouver que $f$
possède un unique point critique $A$ d'abscisse $\alpha$ et d'ordonnée
$y_{\alpha}$ à déterminer en fonction de $\alpha$. \\

\item Calculer les dérivées partielles secondes sur $]0; + \infty[ \
\times ] 0 ; + \infty[$ et établir que
\[
 \partialssc{f}{x} (\alpha, y_{\alpha} ) = \frac{ 2 \alpha +
1}{\alpha^{5}} 
\]

\item La fonction $f$ présente-t-elle un extremum local en $A$ sur
l'ouvert $]0; + \infty[ \ \times ]0 ; + \infty[$ ? Si oui, en donner sa
nature (maximum ou minimum).

\end{noliste}


\section*{EXERCICE 3}
\noindent Soient $n$ et $b$ deux entiers avec $n \geq 1$ et $b \geq 2$.
On considère une urne contenant $n$ boules noires et $b$ boules
blanches, toutes indiscernables. \vspace{0.5cm} \\
Un joueur $A$ effectue des tirages successifs d'une boule
\textbf{\underline{sans remise}} dans l'urne jusqu'à obtenir une boule
blanche. \\
Il laisse alors la place au joueur $B$ qui effectue des tirages
successifs d'une boule \textbf{\underline{avec remise}} dans l'urne
jusqu'à obtenir une boule blanche. \vspace{0.5cm} \\
On note $X$ la variable aléatoire réelle égale au nombre de boules
noires tirées par $A$ avant de tirer une boule blanche et on appelle
$Y$ la variable aléatoire réelle égale au nombre de boules noires
tirées par $B$ avant de tirer une boule blanche (s'il ne reste plus de
boule noire, on a donc $Y = 0$). \vspace{0.5cm} \\
Par exemple, si $n = 3$ et $b = 7$ et que les tirages successifs ont
donné une boule : \\
\og\ noire, blanche, noire, noire, noire, noire, blanche \fg\ alors :
\begin{noliste}{$\sbullet$}

\item $A$ a effectué deux tirages, il a retiré une boule noire puis une
boule blanche de l'urne ; 

\item l'urne contient maintenant 8 boules dont deux noires et six
blanches ; 

\item $B$ a effectué ensuite cinq tirages dans cette urne, il a pioché
4 boules noires qu'il a reposé dans l'urne après chaque tirage puis il
a pioché une boule blanche ;

\item $X$ vaut 1 et $Y$ vaut 4.

\end{noliste}

\section*{I. Étude d'un cas particulier $b = n = 2$.}
\noindent Pour ce cas particulier on pourra s'aider d'un arbre pondéré.
\\
On suppose donc ici que l'urne contient initialement 2 boules blanches
et 2 boules noires. \begin{noliste}{1.}
 \setlength{\itemsep}{4mm}

\item Donner les probabilités des évènements : $[ X = 0],\ [X = 1],\ [X
= 2]$. \\

\item En déduire l'espérance et la variance de $X$. \\

\item Montrer que la probabilité de l'évènement $[Y = 0]$ est donnée
par : 
\[
 P\left(\Ev{Y = 0]}\right) = \frac{1}{2} 
\]

\item Pour tout entier $i$ naturel non nul, déterminer les probabilités
suivantes : 
\[
 P( [X = 0] \cap [Y = i] ), \ \ \ P([X = 1] \cap [Y = i]), \ \ \ P([X =
2] \cap [Y = i]).
\]

\item En déduire la loi de $Y$. Uniquement à l'aide de l'expression de
$P\left(\Ev{Y = i]}\right)$ en fonction de $i$, vérifier que :
\[
\Sum{i = 0}{+ \infty} P\left(\Ev{Y = i]}\right) = 1. 
\]

\item Montrer que $Y$ admet une espérance et la calculer.

\end{noliste}

\section*{II. Retour au cas général.}

\begin{noliste}{1.}
 \setlength{\itemsep}{4mm}

\item Pour tout $k \in \{0, 1..., n\}$, calculer la probabilité
$P\left(\Ev{X = k]}\right)$ puis vérifier que :
\Large 
\[
  P\left(\Ev{X = k]}\right) = \frac{ \binom{n-k + b-1}{b-1} }{ \binom{n
+ b}{b} }.
\]
 \normalsize

\item Utiliser la question qui précède pour justifier que : 
\[
 \Sum{k = 0}{n} \binom{k + b-1}{b-1} = \binom{n + b}{b}.
\]

\textit{Par conséquent, on vient de démontrer la formule suivante : }
\[
 (\mathcal{S}) : \forall N \in \N, \ \ \ \forall a \in \N, \ \ \ \Sum{k
= 0}{N} \binom{k + a}{a} = \binom{N + a + 1}{a + 1}.
\]

\item Soient $k \geq 1$, $N \geq 1$ et $a \in \N$. Comparer $  k
\binom{k + a}{a}$ et $  (a + 1) \binom{k + a}{a + 1}$ puis justifier
que : 
\[
 \Sum{k = 0}{N} k \binom{k + a}{a} = (a + 1) \Sum{k = 0}{N-1} \binom{k
+ a + 1}{a + 1}.
\]

\item À l'aide des questions précédentes, montrer que l'espérance de la
variable $n - X$ est donnée par : 
\[
 E (n - X) = \frac{bn}{b + 1}.
\]
En déduire l'espérance $\E(X)$ de $X$. \\

\item Pour tout $k$ de $X(\Omega)$, et pour tout entier $i$ non nul,
déterminer la probabilité suivante : 
\[
P( [X = k] \cap [Y = i] ).
\]

\item Pour tout $k$ de $X(\Omega)$, et pour tout entier $i$, non nul,
justifier que la série $ \Sum{i \geq 1} i \left( \frac{n-k}{n + b-k-1}
\right)^{i-1}$ est convergente et déterminer sa somme. \\

\item Montrer que $Y$ admet une espérance et vérifier que : 
\[
 E(Y) = \frac{b n}{b^{2}-1}.
\]

\end{noliste}

\end{document}

