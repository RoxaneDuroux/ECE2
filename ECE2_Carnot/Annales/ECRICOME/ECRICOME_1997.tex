\documentclass[11pt]{article}%
\usepackage{geometry}%
\geometry{a4paper,
 lmargin = 2cm,rmargin = 2cm,tmargin = 2.5cm,bmargin = 2.5cm}

\input{../../macros.tex}

\pagestyle{fancy} %
\lhead{ECE2 \hfill Mathématiques\\
} %
\chead{\hrule} %
\rhead{} %
\lfoot{} %
\cfoot{} %
\rfoot{\thepage} %

\renewcommand{\headrulewidth}{0pt}% : Trace un trait de séparation
 % de largeur 0,4 point. Mettre 0pt
 % pour supprimer le trait.

\renewcommand{\footrulewidth}{0.4pt}% : Trace un trait de séparation
 % de largeur 0,4 point. Mettre 0pt
 % pour supprimer le trait.

\setlength{\headheight}{14pt}

\title{\bf \vspace{-2cm} ECRICOME 1997} %
\author{} %
\date{} %
\begin{document}

\maketitle %
\vspace{-1.4cm}\hrule %
\thispagestyle{fancy}

\vspace*{.2cm}


% DEBUT DU DOC À MODIFIER : tout virer jusqu'au début de l'exo

%Définition et changement de valeurs de
compteurs%newcounter{cpt1}{section} compteur cpt1 remis à 0 à chaque
aumentation par stepcounter du compteur section%setcounter{cpt1}{3} on
met le compteur à 3%addtocounter{cpt1}{5} on ajoute 5 au compteur%
stepcounter{cpt1} on ajoute 1% ifthenelse{test}{alors}{sinon} (page
206) pour subordonner à une condition % whiledo{test}{commande} pour
faire une boucle (page 206 aussi) % value{cpt1} pour noter dans le
document la valeur de cpt1 
%Définition définitive d'opérateurs
mathématiques\newcommand{\ch}{\operatorname{ch}} 
\newcommand{\sh}{\operatorname{sh}}
\renewcommand{\tanh}{\operatorname{th}}
\renewcommand{\sinh}{\operatorname{sh}}
\renewcommand{\cosh}{\operatorname{ch}}
\newcommand{\argsh}{\operatorname{argsh}}
\newcommand{\argch}{\operatorname{argch}}
\newcommand{\argth}{\operatorname{argth}}
\newcommand{\Id}{\operatorname{Id}}
\renewcommand{\leq}{\leq}
\renewcommand{\geq}{\geq }

\newcommand{\dlim}{\lim}
\newcommand{\dsum}{\sum}
\newcommand{\dint}{\int}
\newcommand{\dprod}{\prod}



%Définition de nouvelles couleurs : rgb(trois paramètres red green blue
entre 0 et 1); cmyk (quatre cyan magenta yellow black) entre 0 et 1;
gray (entre 0 et 1) et black, white, red, green, blue, cyan, magenta,
yellow% definecolor{0gris}{gray}{0.8} 
% Nouvelle commande pour encadrer le titre car shabox ne veut que d'une
seule ligne; ATTENTION A LA TAILLE; petite différence avec shadowbox ou
doublebox, voire fcolorbox ou colorbox (au lieu de shabox; laisser le
parbox tranquille sauf pour la taille de la boîte
\newcommand{\Tbox}[1]{\begin{center} \shabox{\parbox{0.6
\linewidth}{#1}} \end{center}} %[1] pour 1 paramètre ; #1 pour ce que
fait le 1er paramètre; entre accolades ce que fait la commande
%Mise en page en mode fancy : en-têtes et pieds de pages puis
définition des en-têtes et pieds de pages\pagestyle{fancy}
\lhead{ECE 2 - Mathématiques \\
Quentin Dunstetter - ENC-Bessières 2011$\backslash$2012}
\chead{}
\rhead{Ecricome 1997}
\rfoot[ \ \thepage]{\thepage}
\cfoot{}
\lfoot{}
\thispagestyle{fancy} %Mise en page de la 1ère page en mode fancy
%Trait en bas et en haut de la page (entre en-tête et texte et texte et
pied de page)\renewcommand{\footrulewidth}{0.4pt}
\renewcommand{\headrulewidth}{0.4pt}


\begin{center}
{\Huge ECRICOME Eco 1997}
\end{center}


\begin{center}
{\LARGE Exercice 1}
\end{center}

$\alpha $ est un réel strictement positif. Pour tout $n\in \N$
on pose : 
\[
u_{n}\left( \alpha \right) = \frac{n!}{\prod\limits_{k = 0}{n}\left(
\alpha
 + k\right) } 
\]

\begin{noliste}{1.}
 \setlength{\itemsep}{4mm}
\item Étude de la convergence de la suite $\left( u_{n}\left( \alpha
\right)
\right)_{n\in \N}$

\begin{noliste}{a)}
 \setlength{\itemsep}{2mm}
\item Montrer que la suite $\left( u_{n}\left( \alpha \right)
\right)_{n\in 
\N}$ est monotone et convergente. Que peut-on déduire pour la série de
terme général $\left( u_{n}\left( \alpha \right)
-u_{n + 1}\left( \alpha \right) \right) $ ?

On note $\ell \left( \alpha \right) $ la limite de la suite $\left(
u_{n}\left( \alpha \right) \right)_{n\in \N}$

\item On suppose que $\ell \left( \alpha \right) $ est non nulle.
Démontrer que : 
\[
u_{n}\left( \alpha \right) -u_{n + 1}\left( \alpha \right)
\underset{n\rightarrow + \infty }{\thicksim }\frac{\alpha \ell \left(
\alpha \right) }{n} 
\]

\item Déduire de ce qui précède que $\ell \left( \alpha \right)
 = 0$
\end{noliste}

\item Dans cette question : $\alpha \in \ ]0,1]$

\begin{noliste}{a)}
 \setlength{\itemsep}{2mm}
\item Montrer que : 
\[
\forall n\in \mathbb{N\;\;}u_{n}\left( \alpha \right) \geq \frac{1}{n +
\alpha }
\]

\item Quelle est la nature de la série de terme général $u_{n}\left(
\alpha \right) $ ?
\end{noliste}

\item On pose pour tout entier naturel $n$ : 
\[
I_{n}\left( \alpha \right) = \dint{0}{+ \infty }e^{-\alpha t}\left(
1-e^{-t}\right) ^{n}dt 
\]

\begin{noliste}{a)}
 \setlength{\itemsep}{2mm}
\item Étudier la convergence de l'intégrale généralisée $I_{n}\left(
\alpha \right) $ et calculer $I_{0}\left( \alpha \right) $

\item Soit un réel $x$ strictement positif. Intégrer par parties : 
\[
\dint{0}{x}e^{-\alpha t}\left( 1-e^{-t}\right) ^{n}dt 
\]
et en déduire une relation simple entre $I_{n}\left( \alpha \right) $
et 
$I_{n-1}\left( \alpha + 1\right) $, pour tout $n$ entier naturel non
nul.

\item En déduire : $\forall n\in \mathbb{N\;\;}I_{n}\left( \alpha
\right) = u_{n}$
\end{noliste}

\item On suppose désormais que $\alpha >1$

\begin{noliste}{a)}
 \setlength{\itemsep}{2mm}
\item Montrer que, pour tout $N$ entier naturel : 
\[
\Sum{n = 0}{N}I_{n}\left( \alpha \right) = \frac{1}{\alpha -1}-I_{N +
1}\left(
\alpha -1\right) 
\]

\item En déduire que la série de terme général $u_{n}\left(
\alpha \right) $ est convergente, et donner en fonction de $\alpha $ la
valeur de $\Sum{n = 0}{+ \infty }u_{n}\left( \alpha \right) $.
\end{noliste}
\end{noliste}

\begin{center}
{\LARGE EXERCICE 2}
\end{center}

$\M{3} $ désigne l'ensemble des
matrices carrées d'ordre 3 à coefficients réels.

$\M{3,1} $ est l'ensemble des matrices
colonnes à trois lignes dont les coefficients sont réels.

On pose : 
\[
A = \left( 
\begin{array}{lll}
1 & 0 & 2 \\
\frac{3}{2} & -2 & 6 \\
\frac{1}{2} & -1 & \frac{5}{2}
\end{array}
\right) \;\text{et }B = \left( 
\begin{array}{c}
x \\
y \\
z
\end{array}
\right) 
\]
où $x,\;y$ et $z$ sont des nombres réels.

On définit alors une suite de matrices colonnes $\left( X_{n}\right)
_{n\in \N}$ de la manière suivante :

\[
\left\{ 
\begin{array}{l}
X_{0}\in \M{3,1} \\
\forall n\in \mathbb{N\;}X_{n + 1} = AX_{n} + B
\end{array}
\right. 
\]

\begin{noliste}{1.}
 \setlength{\itemsep}{4mm}
\item Montrer que 0, $\frac{1}{2}$ et $1$ sont les valeurs propres de
A, et préciser des
vecteurs propres $u,\;v$ et $w$ qui leur sont respectivement associés.

\item Justifier les affirmations suivantes :

\begin{noliste}{$\sbullet$}
\item il existe un unique triplet $\left( \alpha,\beta,\gamma \right) $
de 
$\R^{3}$ tel que : 
\[
B = \alpha u + \beta v + \gamma w 
\]

\item Pour tout entier naturel $n,$ il existe un unique triplet $\left(
\alpha_{n},\beta_{n},\gamma_{n}\right) $ de $\R^{3}$ tel que : 
\[
X_{n} = \alpha_{n}u + \beta_{n}v + \gamma_{n}w 
\]
\end{noliste}

\item Établir par récurrence que 
\[
n\in \N^{*}\;\;\left\{ 
\begin{array}{l}
\alpha_{n} = \alpha \\
\beta_{n} = \left( \frac{1}{2}\right) ^{n}\left( \beta_{0}-2\beta
\right)
 + 2\beta \\
\gamma_{n} = \gamma_{0} + n\gamma
\end{array}
\right. 
\]

\item Soit $\left( a_{n}\right)_{n\in \N},\;\left( b_{n}\right)
_{n\in \N}$ et $\left( c_{n}\right)_{n\in \N}$ les suites réelles
telles que : 
\[
\forall n\in \mathbb{N\;}X_{n} = \left( 
\begin{array}{c}
a_{n} \\
b_{n} \\
c_{n}
\end{array}
\right) 
\]
On dit que la suite de matrices colonnes $\left( X_{n}\right)_{n\in \N}
$ converge si les suites réelles $\left( a_{n}\right)_{n\in
\N},\;\left( b_{n}\right)_{n\in \N}$ et $\left( c_{n}\right)_{n\in 
\N}$ convergent. Dans ce cas on écrit : 
\[
\lim X_{n} = \left( 
\begin{array}{c}
\lim a_{n} \\
\lim b_{n} \\
\lim c_{n}
\end{array}
\right) 
\]

\begin{noliste}{a)}
 \setlength{\itemsep}{2mm}
\item Prouver que $\left( X_{n}\right)_{n\in \N}$ converge si et
seulement si le réel $\gamma $ (introduit en 2.) est nul.

\item En déduire que $\left( X_{n}\right)_{n\in \N}$ converge
si et seulement si : 
\[
3x-4y + 12z = 0 
\]
\end{noliste}

\item On dit que le couple $(A,B)$ admet une position d'équilibre
stable
si la suite $\left( X_{n}\right)_{n\in \N}$ converge vers la même
limite quelle que soit la valeur de $X_{0}$.
\end{noliste}

Expliquer pourquoi, quelle que soit la valeur de $B$, le couple $\left(
A,B\right) $ n'admet pas de position d'équilibre stable.

\begin{center}
{\Large Exercice 3}
\end{center}

Dans tout le problème (qui comporte deux parties indépendantes), on
suppose que la durée, exprimée en minutes, d'une communication
téléphonique est une variable aléatoire réelle $D$ qui suit la loi
exponentielle de paramètre $\alpha $

\textbf{I Comparaison de deux tarifications}

\begin{noliste}{1.}
 \setlength{\itemsep}{4mm}
\item Pour ses communications, on propose à l'utilisateur d'une ligne
téléphonique deux tarifications $T_{1}$ et $T_{2}$, exprimées en
francs, définies de la façon suivante :
\end{noliste}

\begin{noliste}{$\sbullet$}
\item $T_{1} = aD$, où $a$ est un nombre réel strictement supérieur à 1
qui représente le prix d'une minute de communication

\item $T_{2}$ est à valeurs dans $\N^{*}$ et, pour tout $n$
entier naturel non nul : $\{T_{2} = n\} = \{n-1<D\leq n\}$
\end{noliste}

\begin{noliste}{1.}
 \setlength{\itemsep}{4mm}
\item Calculer $\E(T_{1})$ en fonction de $a$ et de $\alpha $.

\item Déterminer la loi de $T_{2}$. De quelle loi s'agit-il ? Exprimer
$\E(T_{2})$ en fonction de $\alpha $

\item On pose : 
\[
\left\{ 
\begin{array}{c}
\forall t\in \R_{+}{*} : \varphi \left( t\right) = \frac{t}{1-e^{-t}}
\\
\varphi \left( 0\right) = 1
\end{array}
\right. 
\]

\begin{noliste}{a)}
 \setlength{\itemsep}{2mm}
\item Montrer que $\varphi $ est une fonction de classe $C^{1}$ sur
$[0, + \infty [$

\item On définit de plus la fonction $\psi $ sur $[0, + \infty [$ par :

\[
\forall t\in \R :\psi \left( t\right) = 1-\left( 1 + t\right) e^{-t} 
\]
Utiliser cette fonction pour en déduire que $\varphi $ réalise une
bijection de$[0, + \infty [$ vers $[1, + \infty [$
\end{noliste}

\item Comparaison des tarifications

\begin{noliste}{a)}
 \setlength{\itemsep}{2mm}
\item Montrer qu'il existe un unique réel $\alpha_{0}$ strictement
positif tel que $\varphi \left( \alpha_{0}\right) = a$

\item Préciser quelle est, en moyenne, la tarification la plus
avantageuse suivant la valeur de la durée moyenne d'une communication.
\end{noliste}

\item Pour $a = 1,25$ donner, en utilisant votre calculatrice. une
valeur
approchée de $\frac{1}{\alpha_{0}}$

(on ne donnera que les deux premières décimales fournies par la
calculatrice).
\end{noliste}

\textbf{II Étude d'un standard téléphonique }

Dans toute cette partie, $\theta $ est un nombre réel strictement
positif représentant un temps exprimé en minutes. Un standard
téléphonique de capacité illimitée reçoit des communications
téléphoniques entre l'instant 0 et l'instant $\theta $ inclus.

\textbf{II A Cas d'une seule communication}

On désigne par n un entier naturel non nul. L'instant où débute
la communication est une variable aléatoire réelle $I_{n}$ telle que : 
\[
\left\{ 
\begin{array}{c}
 I_{n}\left( \Omega \right) = \left\{ \frac{\theta }{n},\frac{2\theta
}{n},\dots,\frac{\left( n-1\right) \theta }{n},\frac{n\theta
}{n}\right\} \\
 \forall k\in \left[ \ \left[ 1,n\right] \right] :p\left( I_{n} =
\frac{k\theta }{n}\right) = \frac{1}{n}
\end{array}
\right. 
\]
où $p$ désigne la probabilité. De plus $I_{n}$ et $D$ (la durée
aléatoire de la communication) sont indépendantes.

\begin{noliste}{1.}
 \setlength{\itemsep}{4mm}
\item Pour tout réel positif $t,$ rappeler quelle est l'expression de
$p\left( D>t\right) $ en fonction de $t$ et de $\alpha $.

\item En déduire, pour $k$ élément de $\left[ \ \left[ 1,n\right]
\right] $ la probabilité conditionnelle de $\left\{ D + I_{n}>\theta
\right\} $ sachant $\left\{ I_{n} = \frac{k\theta }{n}\right\}.$

\item Démontrer l'égalité suivante : 
\[
p\left( D + I_{n}>\theta \right) = \frac{1}{n}\left( \frac{\alpha
\theta }}{\frac{\alpha \theta }{n}}}\right) 
\]

\item Déterminer : 
\[
\dlim{n\rightarrow + \infty }p\left( D + I_{n}>\theta \right) 
\]
\end{noliste}

\textbf{II B Étude de l'encombrement du standard à l'instant }$\theta $

Dans cette partie on définit les nombres réels $p$ et $q$ par : 
\[
p = \frac{1-e^{-\alpha \theta }}{\alpha \theta }\text{ \ et \ \ }q =
1-p 
\]
On suppose désormais que la probabilité qu'une communication reçue dans
l'intervalle de temps $\left[ 0,\theta \right] $ se poursuive au-delà
de l'instant $\theta $ est égale à $p$.

On note $N_{\theta }$ la variable aléatoire réelle égale au
nombre de communications reçues dans l'intervalle de temps $\left[
0,\theta \right] $ et l'on suppose que $N_{\theta }$ suit une loi de
Poisson
de paramètre $\theta $.

On note$C_{\theta }$ la variable aléatoire réelle égale au
nombre de communications reçues dans l'intervalle de temps $\left[
0,\theta \right] $ qui se poursuivent au-delà de l'instant $\theta $

Les instants aléatoires où les communications se terminent sont
mutuellement indépendant.

\begin{noliste}{1.}
 \setlength{\itemsep}{4mm}
\item Loi de probabilité de $C_{\theta }$

\begin{noliste}{a)}
 \setlength{\itemsep}{2mm}
\item Soit $r$ un entier naturel. Quelle est la loi conditionnelle de
$C_{\theta }$ sachant que $\left\{ N_{\theta } = r\right\} $ ?

\item Démontrer que l'on a : 
\[
\forall r\in \mathbb{N\;\;\forall }k\in \left[ \ \left[ 0,r\right]
\right]
\;\;p\left( \left\{ C_{\theta } = k\right\} \cap \left\{ N_{\theta } =
r\right\}
\right) = \frac{e^{-\theta }\left( p\theta \right) ^{k}\left( q\theta
\right) ^{r-k}}{k!\left( r-k\right) !}\;\;\; 
\]

\item En déduire, pour tout entier naturel $k$, une expression simple
de 
$p\left( C_{\theta } = k\right) $ en fonction de $k$, $p,$ et $\theta
$.
Quelle est la loi de probabilité de $C_{\theta } ?$
\end{noliste}

\item Étude de l'espérance de $C_{\theta }$

\begin{noliste}{a)}
 \setlength{\itemsep}{2mm}
\item Déterminer l'expression de $\E\left( C_{\theta }\right) $ en
fonction de $\theta $ et de $\alpha $.

\item Quelle est la limite de $\E\left( C_{\theta }\right) $ lorsque
$\theta $
tend vers$ + \infty $ ? Vérifier qu'elle majore $\E\left( C_{\theta
}\right) $.
\end{noliste}
\end{noliste}

\end{document}

