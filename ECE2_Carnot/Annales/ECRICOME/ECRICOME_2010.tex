\documentclass[11pt]{article}%
\usepackage{geometry}%
\geometry{a4paper,
 lmargin = 2cm,rmargin = 2cm,tmargin = 2.5cm,bmargin = 2.5cm}

\input{../../macros.tex}

\pagestyle{fancy} %
\lhead{ECE2 \hfill Mathématiques\\
} %
\chead{\hrule} %
\rhead{} %
\lfoot{} %
\cfoot{} %
\rfoot{\thepage} %

\renewcommand{\headrulewidth}{0pt}% : Trace un trait de séparation
 % de largeur 0,4 point. Mettre 0pt
 % pour supprimer le trait.

\renewcommand{\footrulewidth}{0.4pt}% : Trace un trait de séparation
 % de largeur 0,4 point. Mettre 0pt
 % pour supprimer le trait.

\setlength{\headheight}{14pt}

\title{\bf \vspace{-2cm} ECRICOME 2010} %
\author{} %
\date{} %
\begin{document}

\maketitle %
\vspace{-1.4cm}\hrule %
\thispagestyle{fancy}

\vspace*{.2cm}


% DEBUT DU DOC À MODIFIER : tout virer jusqu'au début de l'exo

%Définition et changement de valeurs de
compteurs%newcounter{cpt1}{section} compteur cpt1 remis à 0 à chaque
aumentation par stepcounter du compteur section%setcounter{cpt1}{3} on
met le compteur à 3%addtocounter{cpt1}{5} on ajoute 5 au compteur%
stepcounter{cpt1} on ajoute 1% ifthenelse{test}{alors}{sinon} (page
206) pour subordonner à une condition % whiledo{test}{commande} pour
faire une boucle (page 206 aussi) % value{cpt1} pour noter dans le
document la valeur de cpt1 
%Définition définitive d'opérateurs
mathématiques\newcommand{\ch}{\operatorname{ch}} 
\newcommand{\sh}{\operatorname{sh}}
\renewcommand{\tanh}{\operatorname{th}}
\renewcommand{\sinh}{\operatorname{sh}}
\renewcommand{\cosh}{\operatorname{ch}}
\newcommand{\argsh}{\operatorname{argsh}}
\newcommand{\argch}{\operatorname{argch}}
\newcommand{\argth}{\operatorname{argth}}
\newcommand{\Id}{\operatorname{Id}}
\renewcommand{\leq}{\leq}
\renewcommand{\geq}{\geq }

\newcommand{\dlim}{\lim}
\newcommand{\dsum}{\sum}
\newcommand{\dprod}{\prod}



%Définition de nouvelles couleurs : rgb(trois paramètres red green blue
entre 0 et 1); cmyk (quatre cyan magenta yellow black) entre 0 et 1;
gray (entre 0 et 1) et black, white, red, green, blue, cyan, magenta,
yellow% definecolor{0gris}{gray}{0.8} 
% Nouvelle commande pour encadrer le titre car shabox ne veut que d'une
seule ligne; ATTENTION A LA TAILLE; petite différence avec shadowbox ou
doublebox, voire fcolorbox ou colorbox (au lieu de shabox; laisser le
parbox tranquille sauf pour la taille de la boîte
\newcommand{\Tbox}[1]{\begin{center} \shabox{\parbox{0.6
\linewidth}{#1}} \end{center}} %[1] pour 1 paramètre ; #1 pour ce que
fait le 1er paramètre; entre accolades ce que fait la commande
%Mise en page en mode fancy : en-têtes et pieds de pages puis
définition des en-têtes et pieds de pages\pagestyle{fancy}
\lhead{ECE 2 - Mathématiques \\
Quentin Dunstetter - ENC-Bessières 2011$\backslash$2012}
\chead{}
\rhead{Ecricome 2010}
\rfoot[ \ \thepage]{\thepage}
\cfoot{}
\lfoot{}
\thispagestyle{fancy} %Mise en page de la 1ère page en mode fancy
%Trait en bas et en haut de la page (entre en-tête et texte et texte et
pied de page)\renewcommand{\footrulewidth}{0.4pt}
\renewcommand{\headrulewidth}{0.4pt}


\begin{center}
{\Huge ECRICOME Eco 2010}
\end{center}

\section*{EXERCICE 1.}

Soit $E$ un espace vectoriel et $\mathcal{B = }\left(
e_{1},e_{2},e_{3}\right)
$ une base de $E$. Pour tout réel $a$, on considère l'endomorphisme
$f_{a}$ de l'espace vectoriel $E$ dont la matrice dans la

base,$\mathcal{B = }$ $\left( e_{1},e_{2},e_{3}\right) $est donnée par
:
\[
M_{a} = 
\begin{smatrix}
a + 2 & -\left( 2a + 1\right) & a \\
1 & 0 & 0 \\
0 & 1 & 0
\end{smatrix}
\]
ainsi que la fonction polyn\^{o}miale $Q$ qui à tout réel $x$
associe le réel :
\[
Q\left( x\right) = x^{3}-\left( a + 2\right) x^{2} + \left( 2a +
1\right) x-a
\]

\subsection*{I. Recherche des valeurs propres de $f_{a}$.}

\begin{noliste}{1.}
 \setlength{\itemsep}{4mm}
\item Montrer que le réel $\lambda $ est une valeur propre de $f_{a}$
si
et seulement si $\lambda $ est racine du polyn\^{o}me $Q$.

\item Vérifier que le réel $\lambda = 1$ est racine de $Q$.

\item En déduire les racines de $Q$ ainsi que leur nombre en fonction
de
$a$.

\item Lorsque $a = 1$, l'endomorphisme $f_{1}$ est-il diagonalisable ?
\end{noliste}

\subsection*{II. Réduction de la matrice $M_{a}$.}

\textbf{Dans toute la suite de l'exercice on suppose }$a$\textbf{\
différent de 1.}

Soit $\mathcal{B}{\prime } = \left( e_{1}{\prime },e_{2}{\prime
},e_{3}{\prime }\right) $ la famille de vecteurs de $E$ définie par
\[
\left\{
\begin{array}{l}
e_{1}{\prime } = a^{2}e_{1} + ae_{2} + e_{3} \\
e_{2}{\prime } = e_{1} + e_{2} + e_{3} \\
e_{3}{\prime } = 2e_{1} + e_{2}
\end{array}
\right.
\]

\begin{noliste}{1.}
 \setlength{\itemsep}{4mm}
\item Prouver que $\mathcal{B}{\prime }$ est une base de $E$.

\hspace{-1cm}On note $P_{a}$ la matrice de passage de la base
$\mathcal{B}$
à la base $\mathcal{B}{\prime }$.

\item Montrer que $e_{1}{\prime }$ est un vecteur propre de $f_{a}$.

\item Vérifier que le sous-espace vectoriel $F$ engendré par les
vecteurs $e_{2}{\prime }$ et $e_{3}{\prime }$ est stable par $f_{a}$
c'est-à-dire :
\[
f_{a}\left( F\right) \subset F
\]

\item Donner l'expression de la matrice $T_{a}$ de l'endomorphisme
$f_{a}$
dans la nouvelle base $\mathcal{B}{\prime }$.

\item Démontrer par récurrence que pour tout entier naturel $n$ :
\[
T^{n} = 
\begin{smatrix}
a^{n} & 0 & 0 \\
0 & 1 & n \\
0 & 0 & 1
\end{smatrix}
\]
où, par convention, on pose $T_{a}{0} = 
\begin{smatrix}
1 & 0 & 0 \\
0 & 1 & 0 \\
0 & 0 & 1
\end{smatrix}
$
\end{noliste}

\section*{III. Étude d'une suite récurrente linéaire.}

Soit $\left( u_{n}\right)_{n\in \N}$ la suite de nombres réels définie
par la relation de récurrence suivante :
\[
\left\{
\begin{array}{l}
u_{0} = 1,\ u_{1} = -1,\ u_{2} = 0 \\
\text{pour tout entier naturel }n :u_{n + 3} = 4u_{n + 2}-5u_{n + 1} +
2u_{n}
\end{array}
\right.
\]

\begin{noliste}{1.}
 \setlength{\itemsep}{4mm}
\item Vérifier que pour tout entier naturel $n$ :
\[
\begin{smatrix}
u_{n + 3} \\
u_{n + 2} \\
u_{n + 1}\end{smatrix}
 = M_{2}\begin{smatrix}
u_{n + 2} \\
u_{n + 1} \\
u_{n}\end{smatrix}
\]

\item Établir par récurrence que pour tout entier naturel $n$ :
\[
\begin{smatrix}
u_{n + 2} \\
u_{n + 1} \\
u_{n}\end{smatrix}
 = P_{2}T_{2}{n}P_{2}{-1}\begin{smatrix}
u_{2} \\
u_{1} \\
u_{0}\end{smatrix}
\]

\item Donner l'expression matricielle de la matrice inverse de $P_{2}$
puis
exprimer $u_{n}$ en fonction de $n$.

\item La suite $\left( u_{n}\right)_{n\in \N}$ est-elle convergente
 ?
\end{noliste}

\section*{EXERCICE 2}

On considère l'application $\varphi $ définie sur $\R_{+}{\ast }$ par :
\[
\forall x\in \R_{+}{\ast },\quad \varphi \left( x\right) = \ln
\left( x\right) -\ln \left( x + 1\right) + \frac{1}{x}
\]

\subsection*{I. Résolution de l'équation $\protect \varphi \left(
x\right) = 1$.}

\begin{noliste}{1.}
 \setlength{\itemsep}{4mm}
\item Déterminer la limite de $\varphi \left( x\right) $ lorsque $x$
tend vers $0$ par valeurs positives.\\
Interpréter graphiquement cette limite.

\item Déterminer la limite de $\varphi \left( x\right) $ lorsque $x$
tend vers $ + \infty $.\\
Interpréter graphiquement cette limite.

\item Prouver que $\varphi $ est strictement monotone sur $\R_{+}{\ast
}$.

\item Dresser le tableau de variation de $\varphi $ et y faire
appara\^{\i}tre les limites de $\varphi $ en $0^{+}$ et $ + \infty.$

\item On rappelle que $\ln \left( 2\right) \simeq 0,7$ et $\ln \left(
3\right) \simeq 1,1$.

Montrer que l'équation $\varphi \left( x\right) = 1$ possède une
unique solution notée $\alpha $ et que :
\[
\frac{1}{3}<\alpha <\frac{1}{2}
\]

\item Proposer un programme en \Scilab{} permettant d'encadrer $\alpha
$ dans
un intervalle d'amplitude $10^{-2}$.
\end{noliste}

\section*{II. Une variable à densité.}

Soit $\alpha $ le réel défini à la question \textbf{I.5.} On
considère la variable aléatoire réelle $X$ dont une densité
de probabilité est donnée par :
\[
\left\{
\begin{array}{ll}
f\left( x\right) = \dfrac{1}{x^{2}\left( x + 1\right) } & \text{si
}x>\alpha
\\
f\left( x\right) = 0 & \text{si }x\leq \alpha
\end{array}
\right.
\]

\begin{noliste}{1.}
 \setlength{\itemsep}{4mm}
\item Vérifier que $f$ est bien une densité de probabilité.

\item Montrer que $X$ admet une espérance $\E\left( X\right) $.

\item Démontrer que pour $x>\alpha $ :
\[
xf\left( x\right) = \varphi ^{\prime }\left( x\right) + \frac{1}{x^{2}}
\]
En déduire que l'espérance de $X$ est donnée par :
\[
\E\left( X\right) = \frac{1-\alpha }{\alpha }
\]
Donner un encadrement de $\E\left( X\right) $ par deux entiers
consécutifs.

\item La variable aléatoire réelle $X$ admet-elle une variance ?
\end{noliste}

\section*{EXERCICE 3}

Dans cet exercice, on étudie des situations probabilistes liées à
un jeu de dés à six faces.

Pour ce jeu, effectuer une partie consiste à lancer successivement deux
dés équilibrés.

On note :

\begin{noliste}{$\sbullet$}
\item $D_{1}$ le résultat du premier dé et $D_{2}$ le résultat
du deuxième dé

\item $E_{1}$ l'événement : $\left( D_{1}<D_{2}\right) $, $E_{2}$
l'événement : $\left( D_{1} = D_{2}\right) $ et $E_{3}$ l'événement :
$\left( D_{1}>D_{2}\right) $
\end{noliste}

Lors d'une partie,

\begin{noliste}{$\sbullet$}
\item si l'événement $E_{1}$ se produit alors le joueur ne marque
pas de point,

\item si l'événement $E_{2}$ se produit alors le joueur marque 2
points,

\item si l'événement $E_{3}$ se produit alors le joueur marque 1
point.
\end{noliste}

\subsection*{I. Étude de parties successives.}

Soit $n$ un entier naturel non nul. Le joueur joue successivement $n$
parties.

Pour tout entier naturel $i\geq 1$, on note :

\begin{noliste}{$\sbullet$}
\item $X_{i}$ la variable aléatoire représentant le nombre de points
marqués lors de la $i^{\grave{e}me}$ partie ;

\item $Y_{i}$ le nombre de points marqués après $i$ parties.
\end{noliste}

\begin{noliste}{1.}
 \setlength{\itemsep}{4mm}
\item Calculer la probabilité de chacun des événements $E_{1}$, $E_{2}$
et $E_{3}$.

\item Soit $i\in \left\{ 1,2,\dots,n\right\} $, déterminer la loi de la
variable aléatoire $X_{i}$ puis calculer son espérance et sa
variance.

\item Trouver la loi de la variable aléatoire $Y_{1}$.

\item Quelle est la loi de la variable aléatoire $Y_{2}$ ?

\item
\begin{noliste}{a)}
 \setlength{\itemsep}{2mm}
\item Préciser l'ensemble $Y_{3}\left( \Omega \right) $ des valeurs
prises par la variable aléatoire $Y_{3}$.

\item Construire et remplir le tableau de la loi conjointe du couple
$\left(
Y_{2},Y_{3}\right).$

\textit{On justifiera précisément une valeur non nulle de ce
tableau, les autres pouvant être données directement.}

\item En déduire la loi de la variable aléatoire $Y_{3}$.
\end{noliste}

\item
\begin{noliste}{a)}
 \setlength{\itemsep}{2mm}
\item Écrire $Y_{n}$ en fonction des variables aléatoires $X_{1}$,
$X_{2}$,..., $X_{n}$.\\
En déduire l'espérance mathématique et la variance de $Y_{n}$.

\item En moyenne, combien de parties au minimum doit faire le joueur
pour
obtenir plus de $10$ points ?
\end{noliste}
\end{noliste}

\subsection*{II. Étude du temps d'attente.}

Le joueur joue maintenant jusqu'à ce qu'il dépasse un nombre de
points donné.

Plus précisément on note : \\
$T_{1}$ (respectivement $T_{2}$) la variable aléatoire représentant
le nombre de parties effectuées par le joueur lorsque le total de ses
points est supérieur ou égal à $1$ (respectivement $2$) pour la
première fois (si cet événement se produit).

Par exemple si les points marqués par le joueur sont dans l'ordre :

\textbf{Exemple 1 :} 0 0 l 0 1 2..... alors $T_{1} = 3$ et $T_{2} = 5$.

\textbf{Exemple 2 :} 0 0 0 2 1 2.... alors $T_{1} = 4$ et $T_{2} = 4$.

\begin{noliste}{1.}
 \setlength{\itemsep}{4mm}
\item
\begin{noliste}{a)}
 \setlength{\itemsep}{2mm}
\item Préciser l'ensemble $T_{1}\left( \Omega \right) $ des valeurs
prises par la variable aléatoire $T_{1}$ puis, pour tout $k$
appartenant
à $T_{1}\left( \Omega \right) $, donner la valeur de la probabilité
$\Prob\left(\Ev{ T_{1} = k}\right) $.

\item Donner la valeur de l'espérance et de la variance de la variable
aléatoire $T_{1}$.
\end{noliste}

\item
\begin{noliste}{a)}
 \setlength{\itemsep}{2mm}
\item Déterminer l'ensemble $T_{2}\left( \Omega \right) $ des valeurs
prises par la variable aléatoire $T_{2}$.

\item Calculer les probabilités $\Prob\left(\Ev{ T_{2} = 1}\right) $ et
$\Prob\left(\Ev{ T_{2} = 2}\right) $.

\item Prouver que, pour $k\geq 3$, on a :
\[
\Prob\left(\Ev{ T_{2} = k}\right) = \left( \frac{5}{12}\right)
^{k-1}\times
\frac{1}{6} + \left( k-1\right) \left( \frac{5}{12}\right) ^{k-1}\times
\frac{7}{12}
\]

\item Ce résultat est-il valable pour $k = 1$ et $k = 2$ ?

\item Établir que : $\dsum \limits_{k = 1}{+ \infty }\Prob\left(\Ev{
T_{2} = k}\right) = 1$.

\item Que peut-on en déduire pour l'événement \guillemotleft \ le
joueur n'obtient jamais un score cumulé supérieur ou égal à
2 \guillemotright \ ?

\item Calculer $\E\left( T_{2}\right) $.
\end{noliste}
\end{noliste}

\end{document}

