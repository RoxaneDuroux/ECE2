\documentclass[11pt]{article}%
\usepackage{geometry}%
\geometry{a4paper,
 lmargin = 2cm,rmargin = 2cm,tmargin = 2.5cm,bmargin = 2.5cm}

\input{../../macros.tex}

\pagestyle{fancy} %
\lhead{ECE2 \hfill Mathématiques\\
} %
\chead{\hrule} %
\rhead{} %
\lfoot{} %
\cfoot{} %
\rfoot{\thepage} %

\renewcommand{\headrulewidth}{0pt}% : Trace un trait de séparation
 % de largeur 0,4 point. Mettre 0pt
 % pour supprimer le trait.

\renewcommand{\footrulewidth}{0.4pt}% : Trace un trait de séparation
 % de largeur 0,4 point. Mettre 0pt
 % pour supprimer le trait.

\setlength{\headheight}{14pt}

\title{\bf \vspace{-2cm} ECRICOME 2000} %
\author{} %
\date{} %
\begin{document}

\maketitle %
\vspace{-1.4cm}\hrule %
\thispagestyle{fancy}

\vspace*{.2cm}


% DEBUT DU DOC À MODIFIER : tout virer jusqu'au début de l'exo

%Définition et changement de valeurs de
compteurs%newcounter{cpt1}{section} compteur cpt1 remis à 0 à chaque
aumentation par stepcounter du compteur section%setcounter{cpt1}{3} on
met le compteur à 3%addtocounter{cpt1}{5} on ajoute 5 au compteur%
stepcounter{cpt1} on ajoute 1% ifthenelse{test}{alors}{sinon} (page
206) pour subordonner à une condition % whiledo{test}{commande} pour
faire une boucle (page 206 aussi) % value{cpt1} pour noter dans le
document la valeur de cpt1 
%Définition définitive d'opérateurs
mathématiques\newcommand{\ch}{\operatorname{ch}} 
\newcommand{\sh}{\operatorname{sh}}
\renewcommand{\tanh}{\operatorname{th}}
\renewcommand{\sinh}{\operatorname{sh}}
\renewcommand{\cosh}{\operatorname{ch}}
\newcommand{\argsh}{\operatorname{argsh}}
\newcommand{\argch}{\operatorname{argch}}
\newcommand{\argth}{\operatorname{argth}}
\newcommand{\Id}{\operatorname{Id}}
\renewcommand{\leq}{\leq}
\renewcommand{\geq}{\geq }

\newcommand{\dlim}{\lim}
\newcommand{\dsum}{\sum}
\newcommand{\dint}{\int}
\newcommand{\dprod}{\prod}



%Définition de nouvelles couleurs : rgb(trois paramètres red green blue
entre 0 et 1); cmyk (quatre cyan magenta yellow black) entre 0 et 1;
gray (entre 0 et 1) et black, white, red, green, blue, cyan, magenta,
yellow% definecolor{0gris}{gray}{0.8} 
% Nouvelle commande pour encadrer le titre car shabox ne veut que d'une
seule ligne; ATTENTION A LA TAILLE; petite différence avec shadowbox ou
doublebox, voire fcolorbox ou colorbox (au lieu de shabox; laisser le
parbox tranquille sauf pour la taille de la boîte
\newcommand{\Tbox}[1]{\begin{center} \shabox{\parbox{0.6
\linewidth}{#1}} \end{center}} %[1] pour 1 paramètre ; #1 pour ce que
fait le 1er paramètre; entre accolades ce que fait la commande
%Mise en page en mode fancy : en-têtes et pieds de pages puis
définition des en-têtes et pieds de pages\pagestyle{fancy}
\lhead{ECE 2 - Mathématiques \\
Quentin Dunstetter - ENC-Bessières 2011$\backslash$2012}
\chead{}
\rhead{Ecricome 2000}
\rfoot[ \ \thepage]{\thepage}
\cfoot{}
\lfoot{}
\thispagestyle{fancy} %Mise en page de la 1ère page en mode fancy
%Trait en bas et en haut de la page (entre en-tête et texte et texte et
pied de page)\renewcommand{\footrulewidth}{0.4pt}
\renewcommand{\headrulewidth}{0.4pt}


\begin{center}
{\Huge ECRICOME Eco 2000}
\end{center}

\begin{center}
{\LARGE Exercice 1}
\end{center}

\noindent Soit $X$ une variable aléatoire à densité définie sur un
espace probabilisé. On note $f$ une densité de $X$, $F$ sa fonction de
répartition. On fait les trois hypothèses
suivantes :

\begin{noliste}{1.}
 \setlength{\itemsep}{4mm}
\item 
\begin{noliste}{a)}
 \setlength{\itemsep}{2mm}
\item 
\begin{nonoliste}{(i)}
\item Si $t$ appartient à $]-\infty,0[$, $f(t) = 0$.

\item Si $t$ appartient à $[0, + \infty [$, $f(t)$ est strictement
positif.

\item $f$ est continue sur $]0, + \infty [$.
\end{nonoliste}
\end{noliste}

\item Montrer que l'équation $F(x) =  {\frac{1}{2}}$ admet
une solution unique sur $]0, + \infty [$.\\
Cet unique réel, que l'on notera $m$, sera appelé \textbf{médiane} de
$X$.

\item Dans cette question, on suppose que $X$ suit une loi
exponentielle de
paramètre 1.\\
Montrer que $X$ satisfait aux hypothèses du début de l'exercice et
déterminer la médiane de $X$.

\item On suppose dans cette question que la densité de $X$ est donnée
sur $[0, + \infty [$ par $f(t) = t\ e^{-t}$ et sur $]-\infty,0[$ par
$f(t) = 0$.

\begin{noliste}{a)}
 \setlength{\itemsep}{2mm}
\item Vérifier que $f$ satisfait aux hypothèses du début de
l'exercice.

\item Déterminer la fonction de répartition $F$ de $X$.

\item Montrer, sans chercher à la calculer, que la médiane $m$ de $X 
$ vérifie $1\leq m\leq 2$.\\
(On donne $6<e^{2}<9)$.\\
On se propose, dans la suite de cette question, de calculer une valeur
approchée de $m$. On introduit pour cela la fonction $g$ définie sur 
$[1,2]$ par $g(x) = \ln (2x + 2)$, fonction qui va permettre de
construire une
suite convergeant vers $m$.

\item Montrer que $g(m) = m$.

\item Montrer que si $x$ appartient à $[1,2]$ alors $g(x)$ appartient 
à $[1,2]$ et 
\[
|g(x)-m|\leq {\frac{1}{2}}|x-m| 
\]

\item On considère la suite $(u_{n})$ définie par $u_{0} = 1$ et pour
$n>0$ par $u_{n} = g(u_{n-1})$.\\
Montrer que\ $|u_{n}-m|\leq \left(  {\frac{1}{2}}\right) ^{n}$

\item Déterminer un entier $n$ tel que $u_{n}$ soit une valeur
approchée de $m$ à $10^{-2}$ près.
\end{noliste}

\item On revient maintenant au cas général et on suppose que la
variable $X$ admet une espérance $\E(X)$ et une variance $\V(X)$. On
note
toujours $m$ la médiane de $X$.

\begin{noliste}{a)}
 \setlength{\itemsep}{2mm}
\item Montrer qu'on a les inégalités : 
\[
\V(X)\geq \dint{0}{m}(t-\E(X))^{2}\ f(t)\ dt\qquad \text{et }\qquad
V(X)\geq
\dint{m}{+ \infty }(t-\E(X))^{2}\ f(t)\ dt
\]

\item En distinguant les cas $m\leq E(X)$ et $m>\E(X)$, montrer que : 
\[
|m-\E(X)|\leq \sqrt{2\V(X)}
\]
\end{noliste}
\end{noliste}

\begin{center}
{\LARGE Exercice 2}
\end{center}

\noindent $E$ est l'espace vectoriel des polyn\^{o}mes à
coefficients réels et de degré inférieur ou égal à 3. On
désigne par $f$ l'application qui à un polyn\^{o}me $P$ de $E$
associe le polyn\^{o}me $f(P)$ défini par : 
\[
\forall x\in \R :f(P)(x) = P\left(\Ev{x + 1}\right) +
P\left(\Ev{x}\right) 
\]

\begin{noliste}{1.}
 \setlength{\itemsep}{4mm}
\item[ \ \textbf{1)}] Montrer que $f$ est un endomorphisme de $E$.

\item[ \ \textbf{2)}] On nore $\mathcal{B}$ la base usuelle de $E$
constituée, dans cet ordre des quatre polyn\^{o}mes $1$, $X$, $X^{2}$,
$X^{3}$.\\
Montrer que la matrice de $f$ dans la base $\mathcal{B}$ est $\left( 
\begin{array}{cccc}
2 & 1 & 1 & 1 \\
0 & 2 & 2 & 3 \\
0 & 0 & 2 & 3 \\
0 & 0 & 0 & 2
\end{array}
\right) $

\item[ \ \textbf{3)}] Montrer que $f$ est bijectif.

\item[ \ \textbf{4)}] Calculer la matrice de $f^{-1}$ dans la base
$\mathcal{B} 
$.

\item[ \ \textbf{5)}] Soit $P$ un élément de $E$ défini par :
$P\left(\Ev{X}\right) = a_{0} + a_{1}X + a_{2}X^{2} + a_{3}X^{3}$.

\begin{noliste}{$\sbullet$}
\item[ \ \textbf{a :}] Expliciter en fonction des réels $a_{0}$,
$a_{1}$, $a_{2}$, $a_{3}$ le polyn\^{o}me $Q = f^{-1}(P)$.

\item[ \ \textbf{b :}] On considère pour tout entier strictement
positif $n$
la somme 
\[
S(n) = \Sum{k = 1}{k = n}(-1)^{k}P\left(\Ev{k}\right) 
\]
Exprimer simplement $S(n)$ en fonction de $(-1)^{n}$, $Q(n + 1)$ et
$Q(1)$.

\item[ \ \textbf{c :}] Expliciter alors la valeur de $S(n)$ en fonction
de $n$, $a_{0}$, $a_{1}$, $a_{2}$, $a_{3}$
\end{noliste}
\end{noliste}

\begin{center}
{\LARGE Exercice 3}
\end{center}

\noindent $T$ est l'ensemble des couples $(x,y)$ de réels
solutions du système d'inéquations 
\[
x\geq \frac{1}{4}}\quad y\geq \frac{1}{4}}\quad
x + y\leq \frac{3}{4}} 
\]
On note $T^{\prime }$ l'\ ``intérieur'' de $T$ à savoir l'ensemble
des couples $(x,y)$ solutions du système d'inéquations 
\[
x>\frac{1}{4}}\quad y>\frac{1}{4}}\quad x + y<\frac{3}{4}} 
\]
Soit $f$ la fonction définie sur $T$ par : \quad $f(x,y) = \frac{1}{x}}
+ {\frac{1}{y}}-{\frac{2}{x + y}}$

\begin{noliste}{1.}
 \setlength{\itemsep}{4mm}
\item[ \ \textbf{1)}] Représenter sur un même graphique $T$ et
$T^{\prime }$.

\item[ \ \textbf{2)}] On admet que $T^{\prime }$ est un ouvert de
$\R^{2}$.

\begin{noliste}{$\sbullet$}
\item[ \ \textbf{a :}] Déterminer les dérivées partielles d'ordre 1
sur $T^{\prime }$ de la fonction $f$.

\item[ \ \textbf{b :}] Montrer que $f$ n'admet pas d'extremum local (et
donc a
fortiori absolu) sur $T^{\prime }$.
\end{noliste}

\item[ \ \textbf{3)}] Démontrer par de simples considérations sur des
inégalités que l'on a pour tout couple $(x,y)$ de $T$ : 
\[
2\leq f(x,y)\leq {\frac{16}{3}} 
\]
On considère une urne contenant des boules blanches (en proportion
$p$),
des boules rouges (en proportion $r$) et des boules vertes (en
proportion $u$).\\
On suppose que \quad $p\geq  {\frac{1}{4}}$ \quad $r\geq 
{\frac{1}{4}}$ \quad $u\geq  {\frac{1}{4}}$ \quad
et que \quad $p + r + u = 1$.\\
\\
On effectue indéfiniment des tirages successifs d'une boule dans cette
urne avec remise entre deux tirages.\\
Pour tout entier naturel $n$ supérieur ou égal à 1, on note $B_{n}$
(respectivement $R_{n}$, $V_{n}$) l'événement : ``Tirer une
boule blanche (respectivement rouge, verte) au $n^{i\grave{e}me}$
tirage''. 
\\
On appelle $X$ (resp $Y$) la variable aléatoire égale au rang
d'apparition de la première blanche (resp rouge).\\
On définit alors la variable $D = |X-Y|$ égale au nombre de tirages
séparant la sortie de la première blanche et de la première rouge.

\item[ \ \textbf{4)}] Déterminer la loi de $X$. Faire de même pour $Y$.

\item[ \ \textbf{5)}] Soit $i$ et $j$ des entiers naturels non nuls. En
distinguant les cas $i = j$, $i<j$ et $i>j$, exprimer l'événement $(X =
i)\cap (Y = j)$ à l'aide des événements décrits dans l'énoncé.\\
En déduire la loi du couple $(X,Y)$.

\item[ \ \textbf{6)}] Les variables $X$ et $Y$ sont-elles indépendantes
?

\item[ \ \textbf{7)}] Soit $k$ un entier naturel non nul, montrer
l'égalité : 
\[
P\left(\Ev{D = k}\right) = {\frac{pr}{p + r}}\left[ (1-p)^{k-1} +
(1-r)^{k-1}\right] 
\]

\item[ \ \textbf{8)}] Montrer que $D$ admet une espérance et que $\E(D)
= f(p,r)$. Encadrer alors $\E(D)$.
\end{noliste}

\vspace{1cm}

\end{document}

