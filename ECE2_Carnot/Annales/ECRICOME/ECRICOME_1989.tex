\documentclass[11pt]{article}%
\usepackage{geometry}%
\geometry{a4paper,
 lmargin = 2cm,rmargin = 2cm,tmargin = 2.5cm,bmargin = 2.5cm}

\input{../../macros.tex}

\pagestyle{fancy} %
\lhead{ECE2 \hfill Mathématiques\\
} %
\chead{\hrule} %
\rhead{} %
\lfoot{} %
\cfoot{} %
\rfoot{\thepage} %

\renewcommand{\headrulewidth}{0pt}% : Trace un trait de séparation
 % de largeur 0,4 point. Mettre 0pt
 % pour supprimer le trait.

\renewcommand{\footrulewidth}{0.4pt}% : Trace un trait de séparation
 % de largeur 0,4 point. Mettre 0pt
 % pour supprimer le trait.

\setlength{\headheight}{14pt}

\title{\bf \vspace{-2cm} ECRICOME 1989} %
\author{} %
\date{} %
\begin{document}

\maketitle %
\vspace{-1.4cm}\hrule %
\thispagestyle{fancy}

\vspace*{.2cm}


% DEBUT DU DOC À MODIFIER : tout virer jusqu'au début de l'exo

%Définition et changement de valeurs de
compteurs%newcounter{cpt1}{section} compteur cpt1 remis à 0 à chaque
aumentation par stepcounter du compteur section%setcounter{cpt1}{3} on
met le compteur à 3%addtocounter{cpt1}{5} on ajoute 5 au compteur%
stepcounter{cpt1} on ajoute 1% ifthenelse{test}{alors}{sinon} (page
206) pour subordonner à une condition % whiledo{test}{commande} pour
faire une boucle (page 206 aussi) % value{cpt1} pour noter dans le
document la valeur de cpt1 
%Définition définitive d'opérateurs
mathématiques\newcommand{\ch}{\operatorname{ch}} 
\newcommand{\sh}{\operatorname{sh}}
\renewcommand{\tanh}{\operatorname{th}}
\renewcommand{\sinh}{\operatorname{sh}}
\renewcommand{\cosh}{\operatorname{ch}}
\newcommand{\argsh}{\operatorname{argsh}}
\newcommand{\argch}{\operatorname{argch}}
\newcommand{\argth}{\operatorname{argth}}
\newcommand{\Id}{\operatorname{Id}}
\renewcommand{\leq}{\leq}
\renewcommand{\geq}{\geq }

\newcommand{\dlim}{\lim}
\newcommand{\dsum}{\sum}
\newcommand{\dint}{\int}
\newcommand{\dprod}{\prod}



%Définition de nouvelles couleurs : rgb(trois paramètres red green blue
entre 0 et 1); cmyk (quatre cyan magenta yellow black) entre 0 et 1;
gray (entre 0 et 1) et black, white, red, green, blue, cyan, magenta,
yellow% definecolor{0gris}{gray}{0.8} 
% Nouvelle commande pour encadrer le titre car shabox ne veut que d'une
seule ligne; ATTENTION A LA TAILLE; petite différence avec shadowbox ou
doublebox, voire fcolorbox ou colorbox (au lieu de shabox; laisser le
parbox tranquille sauf pour la taille de la boîte
\newcommand{\Tbox}[1]{\begin{center} \shabox{\parbox{0.6
\linewidth}{#1}} \end{center}} %[1] pour 1 paramètre ; #1 pour ce que
fait le 1er paramètre; entre accolades ce que fait la commande
%Mise en page en mode fancy : en-têtes et pieds de pages puis
définition des en-têtes et pieds de pages\pagestyle{fancy}
\lhead{ECE 2 - Mathématiques \\
Quentin Dunstetter - ENC-Bessières 2011$\backslash$2012}
\chead{}
\rhead{Ecricome 1989}
\rfoot[ \ \thepage]{\thepage}
\cfoot{}
\lfoot{}
\thispagestyle{fancy} %Mise en page de la 1ère page en mode fancy
%Trait en bas et en haut de la page (entre en-tête et texte et texte et
pied de page)\renewcommand{\footrulewidth}{0.4pt}
\renewcommand{\headrulewidth}{0.4pt}


\begin{center}
{\Huge ECRICOME Eco 1989}
\end{center}

\begin{center}
\textbf{EXERCICE DE MATHEMATIQUES}
\end{center}

On considère l'ensemble $E$ des suites de nombres réels $u_{n}$ (où
$n\in 
\N)$ vérifiant 
\[
\forall n\in \N,\quad u_{n + 3} = 4u_{n + 2}-4u_{n + 1} + u_{n}.
\]

\begin{noliste}{1.}
 \setlength{\itemsep}{4mm}
\item Montrer que pour tous $\lambda $ et $\mu $ réels et toutes suites
$u$
et $v$ de $E,$ la suite $\lambda u + \mu v$ appartient à $E.$

\item Vérifier l'existence, et préciser la valeur, de trois réels
distincts $r$ non nuls tels que la suite de terme général $u_{n} =
r^{n}$ soit élément de 
$E.$ Ces trois réels seront notés $r_{1},r_{2},r_{3}.$

\item Soit $u$ un élement de $E.$ Montrer qu'il existe
$\alpha,\beta,\gamma $ réels tels que la relation 
\[
u_{n} = \alpha r_{1}{n} + \beta r_{2}{n} + \gamma r_{3}{n}
\]
soit vérifiée pour $n = 0,$ $n = 1$ et $n = 2.$

\item Montrer alors que $\forall n\in \N,\quad u_{n} = \alpha
r_{1}{n} + \beta r_{2}{n} + \gamma r_{3}{n}.$
\end{noliste}

\begin{center}
\textbf{EXERCICE DE STATISTIQUES-PROBABILITES}
\end{center}

\noindent Deux personnes $A$ et $B$ partent en vacances de façons
indépendantes dans un pays $E.$ \\
Leur séjour dans ce pays peut s'étaler sur $n$ journées $(n>3)$
numérotées $1,2,...,n.$\\
Pour éventuellement s'y rencontrer, elles ont projeté de séjourner
trois
jours consécutifs (et trois jours seulement) dans un hotel $H,$ choisi
par
elles.\\
On suppose que les jours d'arrivée possibles $1,2,..,n-2$ de ces deux
personnes dans cet hôtel sont deux variables aléatoires uniformes et
indépendantes.\\
Les arrivées ont lieu le matin et les départs le soir, deux jours plus
tard.

\begin{noliste}{1.}
 \setlength{\itemsep}{4mm}
\item 

\begin{noliste}{a)}
 \setlength{\itemsep}{2mm}
\item Quelle est la probabilité que $A$ et $B$ arrivent le même jour ?

\item Quelle est la probabilité qu'elles arrivent avec un jour d'écart
?

\item Quelle est la probabilité qu'elles puissent se rencontrer dans
l'hôtel
 ?
\end{noliste}

\item Sachant que $A$ et $B$ se sont rencontrées, quelle est la
probabilité
qu'elles ne puissent passer qu'une journée ensemble ?
\end{noliste}

\begin{center}
\textbf{PROBLEME}
\end{center}

\noindent Les bons de commande que renvoient les clients d'une
entreprise de
vente par correspondance portent des numéros tous différents qui
permettent
de les archiver.\\
Le service chargé de les classer, les reçoit par paquets de $n$ ($n$
fixe $\in \N^{\times }$), auxquels s'ajoute éventuellement un paquet
incomplet contenant donc un nombre de bons inférieur ou égal à $n-1.$\\
Sur une période donnée, le service d'archivage reçoit toujours $p$
paquets
de $n$ fiches, $p$ étant un entier naturel non nul fixé, supérieur ou
égal à 
$2$ et un complément éventuel. Le nombre de fiches reçues est donc $pn
+ X$ où 
$X$ est une variable aléatoire prenant ses valeurs dans l'ensemble
$\{0,..,n-1\}.$

\begin{center}
\textbf{Première partie -A }
\end{center}

\noindent On suppose ici que $X$ suit la loi uniforme sur
$\{1,..,n-1\}.$

\begin{noliste}{1.}
 \setlength{\itemsep}{4mm}
\item 

\begin{noliste}{a)}
 \setlength{\itemsep}{2mm}
\item On note $F_{1}$ l'évènement :\\
"la première fiche extraite au hasard de l'ensemble de toutes les
fiches a
comme numéro le plus petit des numéros".\\
Décomposer l'évènement $F_{1}$ à l'aide du système complet d'évènements
$(X = 0),(X = 1),...,(X = n-1).$\\
En déduire que la probabilité $\alpha_{n,1}$ de l'évènement $F_{1}$ est
: $\alpha_{n,1} = \dfrac{1}{n}\Sum{k = 1}{n-1}\dfrac{1}{pk + n}.$

\item Soit $i\in \N^{\times },$ $i\leq np.$ \\
On note $F_{i}$ l'évènement :\\
"les $i$ premières fiches successivement extraites de l'ensemble sans
remise
ont, et dans le bon ordre, les $i$ numéros les plus faibles".\\
Montrer que la probabilité $\alpha_{n,i}$ de l'évènement $F_{i}$ est :
$\alpha_{n,i} = \dfrac{1}{n}\Sum{k = 1}{n-1}\dfrac{1}{A_{pn}{i} +
k}.$\\
Rappel : on note $A_{n}{k} = n(n-1)\cdots (n-k + 1).$
\end{noliste}

\item \textbf{Étude de} $\alpha_{n,2}.$

\begin{noliste}{a)}
 \setlength{\itemsep}{2mm}
\item Vérifier l'existence de deux réels $a$ et $b$ tels que, pour tout
$x\in \R^{\times }\backslash \{1\}$
\[
\dfrac{1}{x(x-1)} = \dfrac{a}{x} + \dfrac{b}{x-1}.
\]

\item En déduire $\alpha_{n,2} = \dfrac{1}{(pn-1)(pn + n-1)}.$
\end{noliste}

\item \textbf{Étude de} $\alpha_{n,1}.$

\begin{noliste}{a)}
 \setlength{\itemsep}{2mm}
\item Montrer, en utilisant l'approximation d'une intégrale par la
méthode
des rectangles, que 
\[
\dlim{n\rightarrow + \infty }\dfrac{1}{n}\Sum{k = 0}{n-1}\dfrac{1}{pn +
k} = \ln \left( \dfrac{p + 1}{p}\right) 
\]
et en déduire que :
\[
\alpha_{n,1} = \dfrac{1}{n}\ln \left( \dfrac{p + 1}{p}\right) +
\dfrac{1}{n}\varepsilon (n)\text{ où }\dlim{n\rightarrow + \infty
}\varepsilon
(n) = 0.
\]

\item En utilisant la monotonie de la fonction $x\mapsto \dfrac{1}{p +
x},$ établir plus précisément que :
\[
\dint{k/n}{(k + 1)/n}\dfrac{dx}{p + x}\leq \dfrac{1}{n}\times 
\dfrac{1}{p + (k/n)}\leq \dint{(k-1)/n}{k/n}\dfrac{dx}{p + x}
\]
puis que 
\[
\dfrac{1}{n}\ln \left( \dfrac{p + 1}{p}\right) \leq \alpha
_{n,1}\leq \dfrac{1}{n}\ln \left( \dfrac{p + 1-(1/n)}{p-(1/n)}\right).
\]

\item Étudier et représenter graphiquement la fonction $f$ définie par
:
\[
f(x) = \ln \left( \dfrac{p + 1-x}{p-x}\right) 
\]
où $p$ désigne un entier naturel non nul.

\item On suppose $p\geq 2.$\\
Vérifier que, pour tout $x\in \lbrack 0,1],\quad \left| f^{\prime
}(x)\right| \leq \dfrac{1}{p(p-1)}.$\\
En déduire que pour $n\in \N$ et $n\geq 2,\quad \left|
f(\dfrac{1}{n})-f(0)\right| \leq \dfrac{1}{np(p-1)}$\\
puis que 
\[
0\leq \alpha_{n,1}-\dfrac{1}{n}\ln \left( \dfrac{p + 1}{p}\right)
\leq \dfrac{1}{n^{2}p(p-1)}.
\]
Pour quelles valeurs de $n$ peut-on alors considérer que
$\dfrac{1}{n}\ln
\left( \dfrac{p + 1}{p}\right) $ est une valeur approchée de
$\alpha_{n,1}$
avec une valeur inférieure ou égale à $10^{-2}$ ?
\end{noliste}
\end{noliste}

\begin{center}
\textbf{Deuxième partie -B}
\end{center}

I) Dans cette partie, $\alpha_{n,1}$ représente toujours la probabilité
que
la première fiche extraite porte le premier numéro, mais en faisant
l'hypothèse suivante :\\
le nombre de fiches reçues est $pn + X$ où $X$ est une variable
aléatoire
prenant ses valeurs dans l'ensemble $\{0,1,2,...,n-1\}$ selon la loi
précédente
\[
P\left(\Ev{X = k}\right) = C_{n}e^{-k/n}\qquad C_{n}\text{ désignant
une constante réelle ne dépendant que de }n.
\]

\begin{noliste}{1.}
 \setlength{\itemsep}{4mm}
\item Montrer que $C_{n} = \dfrac{e}{e-1}(1-e^{-1//n}).$

\item Déterminer la valeur de $\alpha_{n,1}$ et montrer que :
$\dlim{n\rightarrow + \infty }n\alpha_{n,1} =
\dfrac{e}{e-1}\dint{0}{1}\dfrac{e^{-x}}{p + x}dx.$
\end{noliste}

II) On se propose de déterminer selon différentes méthodes des valeurs
approchées de :
\[
I = \dint{0}{1}\dfrac{e^{-x}}{1 + x}dx.
\]

\begin{noliste}{1.}
 \setlength{\itemsep}{4mm}
\item 

\begin{noliste}{a)}
 \setlength{\itemsep}{2mm}
\item Montrer que pour tout $x\in \lbrack 0,1],$
\[
1-x\leq e^{-x}\leq 1-x + \dfrac{x^{2}}{2}.
\]

\item En déduire que pour tout $x\in \lbrack 0,1],$
\[
-1 + \dfrac{2}{1 + x}\leq \dfrac{e^{-x}}{1 + x}\leq
\dfrac{x}{2}-\dfrac{3}{2} + \dfrac{5/2}{1 + x}.
\]

\item Préciser alors un encadrement de $I$ et en conclure qu'une valeur
approchée de $I$ est 
\[
I_{1} = \dfrac{9}{4}\ln 2-\dfrac{9}{8}\qquad \text{avec}\qquad \left|
I-I_{1}\right| \leq 0,05.
\]
\end{noliste}

\item 

\begin{noliste}{a)}
 \setlength{\itemsep}{2mm}
\item Étudier et représenter la fonction $g$ définie par : $g(x) =
\dfrac{e^{-x}}{1 + x}.$

\item Jusitifier les inégalités
\[
\dfrac{1}{n}\Sum{k = 1}{n}g(\dfrac{k}{n})\leq I\leq \dfrac{1}{n}\Sum{k
= 0}{n-1}g(\dfrac{k}{n}).
\]
En déduire qu'une valeur approchée de $I$ est :
\[
I_{2}(n) = \dfrac{1}{2}\left[ \ \dfrac{1}{n}\Sum{k =
1}{n}g(\dfrac{k}{n}) + \dfrac{1}{n}\Sum{k =
0}{n-1}g(\dfrac{k}{n})\right] 
\]
avec $\left| I-I_{2}(n)\right| \leq \dfrac{2e-1}{4e}\dfrac{1}{n}.$\\
Quelle valeur conviendrait-il de donner à $n$ pour obtenir $I_{2}(n)$
comme
valeur approchée à $10^{-2}$ près ?
\end{noliste}

\item 

\begin{noliste}{a)}
 \setlength{\itemsep}{2mm}
\item Constater que :
\[
I_{2}(n) = \Sum{k = 0}{n-1}\dfrac{1}{n}\left[ \ \dfrac{g(\dfrac{k}{n})
+ g(\dfrac{k + 1}{n})}{2}\right] 
\]
puis que 
\[
I_{2}(n) = \Sum{k = 0}{n}\dint{k/n}{(k + 1)/n}\left[ g(\dfrac{k}{n}) +
\left( \dfrac{g(\dfrac{k + 1}{n})-g(\dfrac{k}{n})}{\dfrac{1}{n}}\right)
(x-\dfrac{k}{n})\right\dx
\]
et 
\[
I-I_{2}(n) = \Sum{k = 0}{n}\dint{k/n}{(k + 1)/n}\left[
g(x)-g(\dfrac{k}{n})-\left( \dfrac{g(\dfrac{k +
1}{n})-g(\dfrac{k}{n})}{\dfrac{1}{n}}\right) (x-\dfrac{k}{n})\right\dx.
\]

\item Soit $[a,b]$ un segment inclus dans $[0,1],$ avec $a<b.$\\
On pose, pour $x\in \lbrack a,b],$
\[
\Phi (x) = g(x)-g(a)-\left[ \ \dfrac{g(b)-g(a)}{b-a}\right] (x-a)\qquad
\text{et}\qquad \alpha = \dint{a}{b}\Phi (x)dx.
\]
Montrer, en utilisant par exemple une intégration par parties, que pour
$t\in \lbrack a,b] :$
\[
g(t)-g(a) = (t-a)g^{\prime }(a) + \dint{a}{t}(t-u)g"(u)du.
\]
En déduire $\forall x\in \lbrack a,b],\quad \Phi
(x) = \dint{a}{x}(x-u)g"(u)du-\dfrac{x-a}{b-a}\dint{a}{b}(b-u)g"(u)du.$

\item Étudier le signe de $g"$ sur $[0,1]$ et établir que 
\[
\left| \Phi (x)\right| \leq 2(x-a)\left[ g^{\prime
}(b)-g^{\prime }(a)\right] 
\]
puis que 
\[
\left| \alpha \right| \leq (b-a)^{2}\left[ g^{\prime
}(b)-g^{\prime }(a)\right].
\]

\item Déduire du résultat précédent que :
\[
\left| I-I_{2}(n)\right| \leq \dfrac{1}{n^{2}}\left[ g^{\prime
}(1)-g^{\prime }(0)\right] \leq \dfrac{1,75}{n^{2}}.
\]
Quelle valeur $n_{0}$ convient-il de donner à $n$ pour obtenir
$I_{2}(n)$
comme valeur approchée de $I$ à $10^{-2}$ près ?\\
Calculer $I_{2}(n_{0}).$
\end{noliste}
\end{noliste}

\label{fin}

\end{document}

