\documentclass[11pt]{article}%
\usepackage{geometry}%
\geometry{a4paper,
 lmargin = 2cm,rmargin = 2cm,tmargin = 2.5cm,bmargin = 2.5cm}

\input{../../macros.tex}

\pagestyle{fancy} %
\lhead{ECE2 \hfill Mathématiques\\
} %
\chead{\hrule} %
\rhead{} %
\lfoot{} %
\cfoot{} %
\rfoot{\thepage} %

\renewcommand{\headrulewidth}{0pt}% : Trace un trait de séparation
 % de largeur 0,4 point. Mettre 0pt
 % pour supprimer le trait.

\renewcommand{\footrulewidth}{0.4pt}% : Trace un trait de séparation
 % de largeur 0,4 point. Mettre 0pt
 % pour supprimer le trait.

\setlength{\headheight}{14pt}

\title{\bf \vspace{-2cm} ECRICOME 2002} %
\author{} %
\date{} %
\begin{document}

\maketitle %
\vspace{-1.4cm}\hrule %
\thispagestyle{fancy}

\vspace*{.2cm}


% DEBUT DU DOC À MODIFIER : tout virer jusqu'au début de l'exo

%Définition et changement de valeurs de
compteurs%newcounter{cpt1}{section} compteur cpt1 remis à 0 à chaque
aumentation par stepcounter du compteur section%setcounter{cpt1}{3} on
met le compteur à 3%addtocounter{cpt1}{5} on ajoute 5 au compteur%
stepcounter{cpt1} on ajoute 1% ifthenelse{test}{alors}{sinon} (page
206) pour subordonner à une condition % whiledo{test}{commande} pour
faire une boucle (page 206 aussi) % value{cpt1} pour noter dans le
document la valeur de cpt1 
%Définition définitive d'opérateurs
mathématiques\newcommand{\ch}{\operatorname{ch}} 
\newcommand{\sh}{\operatorname{sh}}
\renewcommand{\tanh}{\operatorname{th}}
\renewcommand{\sinh}{\operatorname{sh}}
\renewcommand{\cosh}{\operatorname{ch}}
\newcommand{\argsh}{\operatorname{argsh}}
\newcommand{\argch}{\operatorname{argch}}
\newcommand{\argth}{\operatorname{argth}}
\newcommand{\Id}{\operatorname{Id}}
\renewcommand{\leq}{\leq}
\renewcommand{\geq}{\geq }

%Définition de nouvelles couleurs : rgb(trois paramètres red green blue
entre 0 et 1); cmyk (quatre cyan magenta yellow black) entre 0 et 1;
gray (entre 0 et 1) et black, white, red, green, blue, cyan, magenta,
yellow% definecolor{0gris}{gray}{0.8} 
% Nouvelle commande pour encadrer le titre car shabox ne veut que d'une
seule ligne; ATTENTION A LA TAILLE; petite différence avec shadowbox ou
doublebox, voire fcolorbox ou colorbox (au lieu de shabox; laisser le
parbox tranquille sauf pour la taille de la boîte
\newcommand{\Tbox}[1]{\begin{center} \shabox{\parbox{0.6
\linewidth}{#1}} \end{center}} %[1] pour 1 paramètre ; #1 pour ce que
fait le 1er paramètre; entre accolades ce que fait la commande
%Mise en page en mode fancy : en-têtes et pieds de pages puis
définition des en-têtes et pieds de pages\pagestyle{fancy}
\lhead{ECE 2 - Mathématiques \\
Quentin Dunstetter - ENC-Bessières 2011$\backslash$2012}
\chead{}
\rhead{Epreuve Ecricome 2002}
\rfoot[ \ \thepage]{\thepage}
\cfoot{}
\lfoot{}
\thispagestyle{fancy} %Mise en page de la 1ère page en mode fancy
%Trait en bas et en haut de la page (entre en-tête et texte et texte et
pied de page)\renewcommand{\footrulewidth}{0.4pt}
\renewcommand{\headrulewidth}{0.4pt}


%DEBUT DU DOCUMENT

\noindent {\Large \textbf{ECRI%TCIMACRO{\TeXButton{TeX
field}{\colorbox[gray]{0.95}{COME}}}%BeginExpansion
\colorbox[gray]{0.95}{COME}%EndExpansion
}}\vspace{0.3cm}

\noindent \textbf{Banque d'épreuves communes}

\noindent aux concours des Ecoles

\noindent esc%TCIMACRO{\TeXButton{TeX
field}{\colorbox[gray]{0.95}{bordeaux}} }%BeginExpansion
\colorbox[gray]{0.95}{bordeaux}
%EndExpansion
/ esc%TCIMACRO{\TeXButton{TeX field}{\colorbox[gray]{0.95}{marseille}}
}%BeginExpansion
\colorbox[gray]{0.95}{marseille}
%EndExpansion
/ icn%TCIMACRO{\TeXButton{TeX field}{\colorbox[gray]{0.95}{nancy}}
}%BeginExpansion
\colorbox[gray]{0.95}{nancy}
%EndExpansion
/ esc%TCIMACRO{\TeXButton{TeX field}{\colorbox[gray]{0.95}{reims}}
}%BeginExpansion
\colorbox[gray]{0.95}{reims}
%EndExpansion
/ esc%TCIMACRO{\TeXButton{TeX field}{\colorbox[gray]{0.95}{rouen}}
}%BeginExpansion
\colorbox[gray]{0.95}{rouen}
%EndExpansion
/ esc%TCIMACRO{\TeXButton{TeX
field}{\colorbox[gray]{0.95}{toulouse}}}%BeginExpansion
\colorbox[gray]{0.95}{toulouse}%EndExpansion
\vspace{1cm}

\begin{center}
{\large CONCOURS D'ADMISSION }\vspace{0.5cm}

\textbf{option économique} \vspace{0.5cm}

{\Large \textbf{MATHÉMATIQUES}} \vspace{0.5cm}

\textbf{Année 2002}
\end{center}

\noindent \textbf{Aucun instrument de calcul n'est autorisé.}

\noindent \textbf{Aucun document n'est autorisé.}

\noindent L'énoncé comporte \pageref{fin} pages

\begin{quotation}
\noindent Les candidats sont invités à soigner la présentation de leur
copie, à mettre en évidence les principaux résultats, à respecter les
notations de l'énoncé, et à donner des démonstrations complètes (mais
brèves) de leurs affirmations.
\end{quotation}

\newpage

\section*{EXERCICE 1}

Dans l'ensemble $\mathfrak{M}_{3}(\R)$ des matrices carrées d'ordre $
3$ à coefficients réels, on considère le sous-ensemble $E$ des matrices
$
M(a,b)$ définies par : 
\[
M(a,b) = 
\begin{smatrix}
b & a & b \\
a & b & b \\
b & b & a
\end{smatrix}.
\]
Ainsi : 
\[
E = \{M(a,b)\quad a,b\in \R\}.
\]
On note $f_{a,b}$ l'endomorphisme de $\R^{3}$ représenté par la
matrice $M(a,b)$ dans la base canonique $\mathcal{B} =
(e_{1},e_{2},e_{3})$ de 
$\R^{3}$.

\subsection*{I. Structure de $E$.}

\begin{noliste}{1.}
 \setlength{\itemsep}{4mm}
\item Montrer que $E$ est un sous-espace vectoriel de $\M{3}$.

\item Donner une base de $E$, ainsi que sa dimension.
\end{noliste}

\subsection*{II. Étude d'un cas particulier.}

On pose $A = M(1,0)$.

\begin{noliste}{1.}
 \setlength{\itemsep}{4mm}
\item Calculer $A^{2}$. En déduire que $A$ est une matrice inversible
et
exprimer $A^{-1}$ en fonction de $A$.

\item Déterminer les valeurs propres de $A$.

\item Trouver une base de $\R^{3}$ dans laquelle la matrice de $
f_{1,0}$ est : 
\[
\begin{smatrix}
1 & 0 & 0 \\
0 & 1 & 0 \\
0 & 0 & -1
\end{smatrix}.
\]
\end{noliste}

\subsection*{III. Diagonalisation des éléments de $E$ et application.}

On considère les vecteurs de $\R^{3}$ suivants : 
\[
\vec{u} = (1,1,1),\quad \vec{v} = (1,-1,0),\quad \vec{w} = (1,1,-2).
\]

\begin{noliste}{1.}
 \setlength{\itemsep}{4mm}
\item Justifier que les matrices de l'ensemble $E$ sont
diagonalisables.

\item Montrer que $\mathcal{C} = \left( \vec{u},\vec{v},\vec{w}\right)
$ est
une base de $\R^{3}$.

\item On note $P$ la matrice de passage de la base $\mathcal{B}$ à la
base $\mathcal{C}$. Écrire $P$.

\item Déterminer $P^{-1}$.

\item Exprimer les vecteurs $f_{a,b}\left( \vec{u}\right) $,
$f_{a,b}\left( 
\vec{v}\right) $, $f_{a,b}\left( \vec{w}\right) $ en fonction de
$\vec{u}$, $\vec{v}$, $\vec{w}$.

\item En déduire l'expression de la matrice $D_{a,b}$ de $f_{a,b}$ dans
la
base $\mathcal{C}$.

\item Justifier l'égalité : 
\[
P^{-1}M_{a,b}P = D_{a,b}.
\]

\item Donner une condition nécessaire et suffisante portant sur $a$ et
$b$
pour que $D_{a,b}$ soit inversible.

\item Cette condition étant réalisée, déterminer la matrice inverse de
$
D_{a,b}$.

\item Donner une condition nécessaire et suffisante portant sur $a$ et
$b$
pour que $M_{a,b}$ soit inversible.
\end{noliste}

\section*{EXERCICE 2}

On considère la famille de fonctions $(f_{n})_{n\in \N^{\ast }}$
définies sur $]-1, + \infty \lbrack $ par : 
\[
f_{n}(x) = x^{n}\ln (1 + x).
\]

\subsection*{I. Étude des fonctions $f_{n}$.}

Soit $n\in \N^{\ast }$. On note $h_{n}$ la fonction définie sur $]-1, +
\infty \lbrack $ par : 
\[
h_{n}(x) = n\ln (1 + x) + \dfrac{x}{1 + x}.
\]

\begin{noliste}{1.}
 \setlength{\itemsep}{4mm}
\item Étudier le sens de variation des fonctions $h_{n}$.

\item Calculer $h_{n}(0)$, puis en déduire le signe de $h_{n}$.

\item Étude du cas particulier $n = 1$.

\begin{noliste}{a)}
 \setlength{\itemsep}{2mm}
\item Après avoir justifié la dérivabilité de $f_{1}$ sur $]-1, +
\infty
\lbrack $, exprimer $f_{1}{\prime }(x)$ en fonction de $h_{1}(x)$.

\item En déduire les variations de la fonction $f_{1}$ sur $]-1, +
\infty
\lbrack $.
\end{noliste}

\item Soit $n\in \N^{\ast }\setminus \{1\}$.

\begin{noliste}{a)}
 \setlength{\itemsep}{2mm}
\item Justifier la dérivabilité de $f_{n}$ sur $]-1, + \infty \lbrack $
et
exprimer $f_{n}{\prime }(x)$ en fonction de $h_{n}(x)$.

\item En déduire les variations de $f_{n}$ sur $]-1, + \infty \lbrack
$. (On
distinguera les cas $n$ pair et $n$ impair). On précisera les limites
aux
bornes sans étudier les branches infinies.
\end{noliste}
\end{noliste}

\subsection*{II. Étude d'une suite.}

On considère la suite $\left( U_{n}\right)_{n\in \N^{\ast }}$ définie
par : 
\[
U_{n} = \dint{0}{1}f_{n}(x)\,dx.
\]

\subsubsection*{Calcul de $U_{1}$.}

\begin{noliste}{1.}
 \setlength{\itemsep}{4mm}
\item Prouver l'existence de trois réels $a$, $b$, $c$ tels que : 
\[
\forall x\in \lbrack 0,1],\quad \dfrac{x^{2}}{x + 1} = ax + b +
\dfrac{c}{x + 1}.
\]

\item En déduire la valeur de l'intégrale : 
\[
\dint{0}{1}\dfrac{x^{2}}{x + 1}\,dx.
\]

\item Montrer que $U_{1} = \dfrac{1}{4}$.
\end{noliste}

\subsubsection*{Convergence de la suite $\left( U_{n}\right)_{n\in
\N^{\ast }}$.}

\begin{noliste}{1.}
 \setlength{\itemsep}{4mm}
\item Montrer que la suite $\left( U_{n}\right)_{n\in \N^{\ast }}$
est monotone.

\item Justifier la convergence de la suite $\left( U_{n}\right)_{n\in 
\N^{\ast }}$. (On ne demande pas sa limite.)

\item Démontrer que : 
\[
\forall n\in \N^{\ast },\quad 0\leq U_{n}\leq \dfrac{\ln 2}{n + 1}.
\]

\item En déduire la limite de la suite $\left( U_{n}\right)_{n\in
\N^{\ast }}$.
\end{noliste}

\subsubsection*{Calcul de $U_{n}$ pour $n\geq 2$.}

Pour $x\in \lbrack 0,1]$ et $n\in \N^{\ast }\setminus \{1\}$, on
pose : 
\[
S_{n}(x) = 1-x + x^{2} + \cdots + (-1)^{n}x^{n} = \Sum{k =
0}{n}(-1)^{k}x^{k}.
\]

\begin{noliste}{1.}
 \setlength{\itemsep}{4mm}
\item Montrer que : 
\[
S_{n}(x) = \dfrac{1}{1 + x} + \dfrac{(-1)^{n}x^{n + 1}}{1 + x}.
\]

\item En déduire que : 
\[
\Sum{k = 0}{n}\dfrac{(-1)^{k}}{k + 1} = \ln 2 +
(-1)^{n}\dint{0}{1}\dfrac{x^{n + 1}}{1 + x}\,dx.
\]

\item En utilisant une intégration par parties dans le calcul de
$U_{n}$,
montrer que : 
\[
U_{n} = \dfrac{\ln 2}{n + 1} + \dfrac{(-1)^{n}}{n + 1}\left[ \ \ln
2-\left( 1-\dfrac{1}{2} + \cdots + \dfrac{(-1)^{k}}{k + 1} + \cdots +
\dfrac{(-1)^{n}}{n + 1}\right) \right].
\]
\end{noliste}

\section*{EXERCICE 3}

Une urne contient une boule blanche et une boule noire, les boules
étant
indiscernables au toucher.

On y prélève une boule, chaque boule ayant la même probabilité d'être
tirée,
on note sa couleur, et on la remet dans l'urne avec $c$ boules de la
couleur
de la boule tirée. On répète cette épreuve, on réalise ainsi une
succession
de $n$ tirages ($n\geq 2$).

\subsection*{I. Étude du cas $c = 0$.}

On effectue donc ici $n$ tirages avec remise de la boule dans l'urne.

On note $X$ la variable aléatoire réelle égale au nombre de boules
blanches
obtenues au cours des $n$ tirages et $Y$ la variable aléatoire réelle
définie par : 
\[
\left\{
\begin{array}{cl}
Y = k & \text{si l'on obtient une boule blanche pour la première fois
au }k^{\grave{e}me}\text{ tirage.} \\
Y = 0 & \text{si les $n$ boules tirées sont noires.}
\end{array}
\right.
\]

\begin{noliste}{1.}
 \setlength{\itemsep}{4mm}
\item Déterminer la loi de $X$. Donner la valeur de $\E(X)$ et de
$\V(X)$.

\item Pour $k\in \{1,\ldots,n\}$, déterminer la probabilité
$P\left(\Ev{Y = k}\right)$ de l'évènement $(Y = k)$, puis déterminer
$P\left(\Ev{Y = 0}\right)$.

\item Vérifier que : 
\[
\Sum{k = 0}{n}P\left(\Ev{Y = k}\right) = 1.
\]

\item Pour $x\neq 1$ et $n$ entier naturel non nul, montrer que : 
\[
\Sum{k = 1}{n}kx^{k} = \dfrac{nx^{n + 2}-(n + 1)x^{n + 1} +
x}{(1-x)^{2}}.
\]

\item En déduire $\E(Y)$.
\end{noliste}

\subsection*{II. Étude du cas $c\neq 0$.}

On considère les variables aléatoires $\left( X_{i}\right)_{1\leq
i\leq n}$ définies par : 
\[
\left\{
\begin{array}{cl}
X_{i} = 1 & \text{si on obtient une boule blanche au
}i^{\grave{e}me}\text{
tirage.} \\
X_{i} = 0 & \text{sinon.}
\end{array}
\right.
\]
On définit alors, pour $2\leq p\leq n$, la variable aléatoire $Z_{p}$,
par : 
\[
Z_{p} = \Sum{i = 1}{p}X_{i}.
\]

\begin{noliste}{1.}
 \setlength{\itemsep}{4mm}
\item Que représente la variable $Z_{p}$ ?

\item Donner la loi de $X_{1}$ et l'espérance $\E(X_{1})$ de $X_{1}$.

\item Déterminer la loi du couple $(X_{1},X_{2})$. En déduire la loi de
$X_{2}$ puis l'espérance $\E(X_{2})$.

\item Déterminer la loi de probabilité de $Z_{2}$.

\item Déterminer l'univers image $Z_{p}\left( \Omega \right) $ de
$Z_{p}$.

\item Soit $p\leq n-1$.

\begin{noliste}{a)}
 \setlength{\itemsep}{2mm}
\item Déterminer $P\left(\Ev{X_{p + 1} = 1\,/Z_{p} = k}\right)$ pour
$k\in Z_{p}\left( \Omega
\right) $.

\item En utilisant la formule des probabilités totales, montrer que : 
\[
P\left(\Ev{X_{p + 1} = 1}\right) = \dfrac{1 + c\E(Z_{p})}{2 + pc}.
\]

\item En déduire que $X_{p}$ est une variable aléatoire de Bernoulli de
paramètre $\dfrac{1}{2}$.

(On raisonnera par récurrence sur $p$ : les variables $X_{1}$,
$X_{2}$,...., $X_{p}$ étant supposées suivre une loi de de Bernoulli de
paramètre $\dfrac{1}{2}$, et on calculera $\E(Z_{p})$).
\end{noliste}
\end{noliste}

\label{fin}

\end{document}

