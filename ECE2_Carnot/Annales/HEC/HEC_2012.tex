\documentclass[11pt]{article}%
\usepackage{geometry}%
\geometry{a4paper,
 lmargin = 2cm,rmargin = 2cm,tmargin = 2.5cm,bmargin = 2.5cm}

\input{../../../../../../macros.tex}

\pagestyle{fancy} %
\lhead{ECE2 \hfill septembre 2017 \\
 Mathématiques\\[.2cm]} %
\chead{\hrule} %
\rhead{} %
\lfoot{} %
\cfoot{} %
\rfoot{\thepage} %

\renewcommand{\headrulewidth}{0pt}% : Trace un trait de séparation
 % de largeur 0,4 point. Mettre 0pt
 % pour supprimer le trait.

\renewcommand{\footrulewidth}{0.4pt}% : Trace un trait de séparation
 % de largeur 0,4 point. Mettre 0pt
 % pour supprimer le trait.

\setlength{\headheight}{14pt}

\title{\bf \vspace{-1cm} HEC 2012} %
\author{} %
\date{} %

\begin{document}

\maketitle %
\vspace{-1.2cm}\hrule %
\thispagestyle{fancy}

\vspace*{.4cm}

% DEBUT DU DOC À MODIFIER : tout virer jusqu'au début de l'exo

%Définition et changement de valeurs de
compteurs%newcounter{cpt1}{section} compteur cpt1 remis à 0 à chaque
aumentation par stepcounter du compteur section%setcounter{cpt1}{3} on
met le compteur à 3%addtocounter{cpt1}{5} on ajoute 5 au compteur%
stepcounter{cpt1} on ajoute 1% ifthenelse{test}{alors}{sinon} (page
206) pour subordonner à une condition % whiledo{test}{commande} pour
faire une boucle (page 206 aussi) % value{cpt1} pour noter dans le
document la valeur de cpt1 
%Définition définitive d'opérateurs
mathématiques\newcommand{\ch}{\operatorname{ch}} 
\newcommand{\sh}{\operatorname{sh}}
\renewcommand{\tanh}{\operatorname{th}}
\renewcommand{\sinh}{\operatorname{sh}}
\renewcommand{\cosh}{\operatorname{ch}}
\newcommand{\argsh}{\operatorname{argsh}}
\newcommand{\argch}{\operatorname{argch}}
\newcommand{\argth}{\operatorname{argth}}
\newcommand{\Id}{\operatorname{Id}}
\renewcommand{\leq}{\leq}
\renewcommand{\geq}{\geq }

\newcommand{\dlim}{\lim}
\newcommand{\dsum}{\sum}
\newcommand{\dprod}{\prod}



%Définition de nouvelles couleurs : rgb(trois paramètres red green blue
entre 0 et 1); cmyk (quatre cyan magenta yellow black) entre 0 et 1;
gray (entre 0 et 1) et black, white, red, green, blue, cyan, magenta,
yellow% definecolor{0gris}{gray}{0.8} 
% Nouvelle commande pour encadrer le titre car shabox ne veut que d'une
seule ligne; ATTENTION A LA TAILLE; petite différence avec shadowbox ou
doublebox, voire fcolorbox ou colorbox (au lieu de shabox; laisser le
parbox tranquille sauf pour la taille de la boîte
\newcommand{\Tbox}[1]{\begin{center} \shabox{\parbox{0.6
\linewidth}{#1}} \end{center}} %[1] pour 1 paramètre ; #1 pour ce que
fait le 1er paramètre; entre accolades ce que fait la commande
%Mise en page en mode fancy : en-têtes et pieds de pages puis
définition des en-têtes et pieds de pages\pagestyle{fancy}
\lhead{ECE 2 - Mathématiques \\
Quentin Dunstetter - ENC-Bessières 2011$\backslash$2012}
\chead{}
\rhead{HEC 2012}
\rfoot[ \ \thepage]{\thepage}
\cfoot{}
\lfoot{}
\thispagestyle{fancy} %Mise en page de la 1ère page en mode fancy
%Trait en bas et en haut de la page (entre en-tête et texte et texte et
pied de page)\renewcommand{\footrulewidth}{0.4pt}
\renewcommand{\headrulewidth}{0.4pt}

\begin{center}
{\huge HEC Eco 2012}
\end{center}

\section*{EXERCICE}

\begin{flushleft}
Soit $m$ un réel donné strictement positif et $f$ l'endomorphisme de
$\R^{3}$ dont la matrice $M$ dans la base canonique de $\R^{3}$ est
donnée par :\\


\end{flushleft}

\noindent

\begin{flushleft}
$M = \left(
\begin{array}
[c]{ccc}0 & 1/m & 1/m^{2}\\
m & 0 & 1/m\\
m^{2} & m & 0
\end{array}
\right) $.


\end{flushleft}

\noindent

\begin{flushleft}
On note $I$ la matrice identité de $\M{3} $ et $\text{Id}$
l'endomorphisme identité de $\R^{3}$.


\end{flushleft}

\noindent

\begin{flushleft}
Pour tout endomorphisme $g$ de $\R^{3}$ on pose $g^{0} = \text{Id}$ et
pour tout $k$ de $\N^{*}$, $g^{k} = g\circ g^{k-1}$.


\end{flushleft}

\begin{noliste}{1.}
 \setlength{\itemsep}{4mm}
\item Déterminer le noyau $\ker\left( f\right) $ et l'image
$\text{im}\left( f\right) $ de l'endomorphisme $f.$ La matrice $M$
est-elle inversible ?

\item {}

\begin{noliste}{a)}
 \setlength{\itemsep}{2mm}
\item Montrer que la matrice $M^{2}$ est une combinaison linéaire de
$I$
et de $M$.

\item Déterminer un polyn\^{o}me annulateur non nul de la matrice $M$.

\item Déterminer les valeurs propres et les sous-espaces propres de
$M$.
La matrice $M$ est-elle diagonalisable ?
\end{noliste}

\item \`{A} l'aide des résultats de la question 2.(c), indiquer une
méthode, sans faire les calculs, qui permettrait d'obtenir pour tout
$n$
de $\N$, l'expression de $M^{n}$ en fonction de $n$.

\item On pose : $p = \dfrac{1}{3}\left( f + \text{Id}\right) $ et $q =
-\dfrac
{1}{3}\left( f-2\text{Id}\right) $.

\begin{noliste}{a)}
 \setlength{\itemsep}{2mm}
\item Calculer $p\circ q$ et $q\circ p$, puis pour tout $n$ de $\N$,
$p^{n}$ et $q^{n}$.

\item En déduire pour tout $n$ de $\N$, l'expression de $f^{n}$ en
fonction de $p$ et $q$.

\item Déterminer les deux suites réelles telles que pour tout $n$ de
$\N$, on ait : $M^{n} = a_{n}I + b_{n}M$.

\item La formule précédente reste-t-elle valable si $n$ appartient
à $\Z$ ?
\end{noliste}
\end{noliste}

\subsection*{PROBL\`{E}ME}

\begin{flushleft}
Sous réserve d'existence, on note $\E(U)$ et $\V(U)$ respectivement,
l'espérance mathématique et la variance d'une variable aléatoire
$U$ définie sur l'espace probabilisé $(\Omega;\mathcal{A};P)$.
\\
 Pour $p$ entier supérieur ou égal à 2, on dit que les
variables aléatoires à densité $U_{1};\ldots;U_{p}$ sont
indépendantes si pour tout $p$- uplet $\left( u_{1};\ldots;u_{p}\right)
$ de réels, les événements $\left[ U_{1}\leq u_{1}\right]
;\ldots;\left[ U_{p}\leq u_{p}\right] $ sont indépendants.


\end{flushleft}

\noindent

\begin{flushleft}
L'objet du problème est l'étude de quelques propriétés d'une
loi de probabilité utilisée notamment en fiabilité.


\end{flushleft}

\noindent

\begin{flushleft}
Les parties I et II sont largement indépendantes. La partie III est
indépendante des parties I et II.


\end{flushleft}

\subsection*{Partie I. Loi à 1 paramètre.}

On note $\lambda$ un paramètre réel strictement positif. On
considère la fonction $f_{\lambda}$ de $\R$ dans $\R$
définie par :
\[
f_{\lambda}\left( x\right) = \left\{
\begin{array}
[c]{ccc}\dfrac{\lambda}{2\sqrt{x}}e^{-\lambda\sqrt{x}} & \text{si} &
x>0\\
 & & \\
0 & \text{si} & x\leq0
\end{array}
\right..
\]


\begin{noliste}{1.}
 \setlength{\itemsep}{4mm}
\item {}

\begin{noliste}{a)}
 \setlength{\itemsep}{2mm}
\item Montrer que la fonction $f_{\lambda}$ est de classe $C^{2}$ sur
$\R_{+}{*}$.

\item Dresser le tableau de variation de $f_{\lambda}$ sur
$\R_{+}{\ast}$ et préciser les limites suivantes :\\
${\dlim{x\rightarrow0^{+}}f_{\lambda}\left( x\right) }$,
${\dlim{x\rightarrow + \infty}f_{\lambda}\left( x\right) }$.

\item Établir la convexité de la fonction $f_{\lambda}$ sur
$\R_{+}{*}$.

\item Tracer l'allure de la courbe représentative de $f_{\lambda}$ dans
le
plan rapporté à un repère orthogonal.
\end{noliste}

\item {}

\begin{noliste}{a)}
 \setlength{\itemsep}{2mm}
\item Vérifier que la fonction $x\mapsto-e^{-\lambda\sqrt{x}}$ est une
primitive de $f_{\lambda}$ sur $\R_{+}{*}$.

\item ) Établir la convergence de l'intégrale
${ \dint{0}{+ \infty}f_{\lambda}\left( x\right\dx}$ et
calculer sa valeur.

\item En déduire que la fonction $f_{\lambda}$ est une densité de
probabilité sur $\R_{+}{*}$.
\end{noliste}

\item Soit $X$ une variable aléatoire définie sur un espace
probabilisé $(\Omega;\mathcal{A};P)$, à valeurs strictement
positives, ayant $f_{\lambda}$ pour densité. On note $F_{\lambda}$ la
fonction de répartition de $X$ et on pose : $Y = \lambda\sqrt{X}$.

\begin{noliste}{a)}
 \setlength{\itemsep}{2mm}
\item Calculer pour tout $x$ réel, $F_{\lambda}\left( x\right) $.

\item Montrer que $Y$ suit la loi exponentielle de paramètre 1.

\item Établir pour tout $r$ de $\N^{*}$, l'existence de $\E\left(
Y^{r}\right) $.

\item Montrer que pour tout $r$ de $\N^{*}$, on a : $\E\left(
Y^{r + 1}\right) = \left( r + 1\right) E\left( Y^{r}\right) $.

\item En déduire pour tout $r$ de $\N^{*}$, $\E\left(
Y^{r}\right) $ et $\E\left( X^{r}\right) $. En particulier, calculer
$\E(X)$
et $\V(X)$.
\end{noliste}

\item Soit $\left( X_{n}\right) _{n\in\N^{\ast}}$ une suite de
variables aléatoires définies sur $(\Omega;\mathcal{A};P)$,
indépendantes et de même loi que $X$. \\
 Soit $\left(
a_{n}\right) _{n\in\N^{\ast}}$ et $\left( b{}_{n}\right)
_{n\in\N^{\ast}}$ deux suites de réels strictement positifs
vérifiant ${\dlim{n\rightarrow + \infty}n^{2}a_{n} = 1}$ et
${\dlim{n\rightarrow + \infty}n^{2}b_{n} = 0}$. \\
 On pose
pour tout $n$ de $\N^{\ast}$, $M_{n} = {\min_{1\leq
k\leq n}\left( X_{k}\right) }$ et $J_{n} = \dfrac{M_{n}-b_{n}}{a_{n}}$.
\\
 On admet que $M_{n}$ et $J_{n}$ sont des variables aléatoires
à densité définies sur $(\Omega;\mathcal{A};P)$. \\
 Montrer
que la suite $\left( J_{n}\right) _{n\in\N^{\ast}}$ converge en loi
vers une variable aléatoire dont on précisera la loi.
\end{noliste}

\subsection*{Partie II. Estimation ponctuelle de $\lambda$.}

\begin{flushleft}
Pour $n$ entier de $\N^{\ast}$, on note $\left( X_{1};\ldots
;X_{n}\right) $ un $n$-échantillon de variables aléatoires à
valeurs strictement positives, indépendantes et de même loi que la
variable aléatoire $X$ définie dans la question 3. On rappelle que
$Y = \lambda\sqrt{X}$, et on pose pour tout $k$ de $\left[ \!\left[
1;n\right] \!\right] $, $Y_{k} = \lambda\sqrt{X_{k}}$, $S_{k} = {\Sum{j
= 1}{k}Y_{j}}$ et $g_{k}$ une densité de $S_{k}$.\\
\textbf{On admet} que pour tout entier $n$ supérieur ou égal
à $2$, les variables aléatoires $Y_{1};\ldots;Y_{n}$ sont
indépendantes et que pour tout $k$ de $\left[ \!\left[ 1;n\right]
\!\right] $, les variables aléatoires $S_{k}$ et $Y_{k + 1}$ sont
indépendantes. On admet que si $T$ et $Z$ sont deux variables
aléatoires à densité indépendantes définies sur le
même espace probabilisé, de densités respectives $f_{T}$ et
$f_{Z}$ telles que $f_{T}$ et $f_{Z}$ soit bornée, alors la variable
aléatoire $T + Z$ admet une densité $f_{T + Z}$ définie pour tout $x$
réel par
\[
f_{T + Z}\left( x\right) = \dint{-\infty}{+ \infty} f_{T}\left(
y\right)
f_{Z}\left( x-y\right) dy.
\]

\end{flushleft}

\begin{noliste}{1.}
 \setlength{\itemsep}{4mm}
\item[5.]
\begin{noliste}{a)}
 \setlength{\itemsep}{2mm}
\item En utilisant les propriétés admises, montrer que $g_{2}\left(
x\right) = \left\{
\begin{array}
[c]{ccc}xe^{-x} & \text{si} & x>0\\
0 & \text{si} & x\leq0
\end{array}
\right. $.

\item \`{A} l'aide d'un raisonnement par récurrence, montrer que pour
tout
$n$ de $\N^{\ast}$, on a $g_{n}\left( x\right) = \left\{
\begin{array}
[c]{ccc}\dfrac{1}{\left( n-1\right) !}x^{n-1}e^{-x} & \text{si} & x>0\\
0 & \text{si} & x\leq0
\end{array}
\right. $.

\item On admet que pour tout $n$ de $\N^{\ast}$, $\dfrac{1}{S_{n}}$
est une variable aléatoire à densité. Pour quelles valeurs de $n$,
l'espérance $\E\left( \dfrac{1}{S_{n}}\right) $ et la variance
$\V\left( \dfrac{1}{S_{n}}\right) $ existent-elles ? Calculer alors
leurs
valeurs respectives.
\end{noliste}

\item[6.] On note $\left( x_{1};\ldots;x_{n}\right) $ un $n$-uplet de
$\left( \R_{+}{\ast}\right) ^{n}$ constituant une réalisation
du $n$-échantillon $\left( X_{1};\ldots;X_{n}\right) $.\\
 On
suppose que le paramètre $\lambda$ est inconnu. Soit $H$ la fonction de
$\R_{+}{\ast}$dans $\R$ définie par :
\[
H\left( \lambda\right) = \ln\left( {\prod_{k = 1}{n}f_{\lambda}\left(
x_{k}\right) }\right).
\]
\\
 Montrer que la fonction $H$ admet un maximum atteint en un unique
point $\lambda_{0}$ dont on donnera la valeur.

\item[7.] On pose pour tout entier $n$ supérieur ou égal à $3$ ou
$\lambda_{n}{\ast} = \dfrac{n}{{\Sum{k = 1}{n}\sqrt{X_{k}}}}$.

\begin{noliste}{a)}
 \setlength{\itemsep}{2mm}
\item Que représente $\lambda_{0}$ pour $\lambda_{n}{\ast}$ ?

\item Construire à partir de $\lambda_{n}{\ast}$ un estimateur sans
biais
$\hat{\lambda}_{n}$ de $\lambda$ et calculer le risque quadratique
$\rho\left( \hat{\lambda}_{n}\right) $ de $\hat{\lambda}_{n}$.

\item Calculer ${\dlim{n\rightarrow + \infty}\rho\left(
\hat{\lambda}_{n}\right) }$. Commenter.
\end{noliste}

\subsection*{Partie III. Loi à 2 paramètres.}

\item[8.] Soit $\lambda$ et $\alpha$ deux paramètres réels strictement
positifs et $f_{\left( \lambda;\alpha\right) }$ la fonction définie sur
$\R$ par
\[
f_{\left( \lambda;\alpha\right) }\left( x\right) = \left\{
\begin{array}
[c]{ccc}\lambda\alpha x^{\alpha-1}e^{-\lambda x^{\alpha}} & \text{si} &
x>0\\
0 & \text{si} & x\leq0
\end{array}
\right.
\]


\begin{noliste}{a)}
 \setlength{\itemsep}{2mm}
\item Montrer que $f_{\left( \lambda;\alpha\right) }$ est une densité
de
probabilité sur $\R$. \\
 Soit $W$ une variable aléatoire
définie sur un espace probabilisé $(\Omega;\mathcal{A};P)$, à
valeurs strictement positives, de densité $f_{\left( \lambda
;\alpha\right) }$. On dit que $W$ suit la loi $\mathcal{WB}\left(
\lambda;\alpha\right) $.

\item On note $F_{\left( \lambda;\alpha\right) }$ la fonction de
répartition de $W$. Calculer pour tout $x$ réel, $F_{\left(
\lambda;\alpha\right) }\left( x\right) $.

\item Montrer que la variable aléatoire $F_{\left(
\lambda;\alpha\right)
}\left( W\right) $ suit la loi uniforme sur $[0;1]$.

\item Écrire nne fonction \Scilab{} d'en-tête function
\texttt{W(lambda,alpha :real) :real;} permettant de simuler $W$.
\end{noliste}

\item[9.] Soit $K$ une variable aléatoire à densité définie
sur un espace probabilisé $(\Omega;\mathcal{A};P)$, à valeurs
strictement positives, de densité $f_{K}$ nulle sur $\R_{-}$,
continue sur $\R$, de classe $C^{1}$ sur $\R_{+}{*}$ et
strictement positive sur $\R_{+}{*}$. On note $F_{K}$ la fonction de
répartition de $K$. On pose pour tout $X$ réel $R\left( x\right)
 = -\ln\left( 1-F_{K}\left( x\right) \right) $ et $r\left( x\right)
 = R^{\prime}\left( x\right) $, où $R^{\prime}$ est la fonction
dérivée de $R$.

\begin{noliste}{a)}
 \setlength{\itemsep}{2mm}
\item On suppose dans cette question que $K$ suit la loi
$\mathcal{WB}\left(
\lambda;2\right) $ avec $\lambda>0$. \\
Établir les
propriétés (i) et (ii) suivantes : \\
(i) la fonction $r$ est
continue et strictement croissante sur $\R_{+}$, et $r\left(
0\right) = 0$. \\
(ii) la variable aléatoire $r(K)$ suit la loi
$\mathcal{WB}\left( \dfrac{1}{4\lambda};2\right) $.

\item Réciproquement, on suppose dans cette question que les
propriétés (i) et (ii) sont vérifiées. \\
 Montrer que
$K$ suit la loi $\mathcal{WB}\left( \lambda;2\right) $. Conclusion ?
\end{noliste}

Dans les questions 10 et 11, l'entier $n$ est supérieur ou égal à
2. On note $w_{1};\ldots;w_{n}$ des réels strictement positifs et non
tous égaux.

\item[10.] Soit $\varphi$ la fonction de $\R_{+}{\ast}$ dans
$\R$ définie par
\[
\varphi\left( x\right) = \dfrac{{\Sum{k = 1}{n}\left(
w_{k}\right) ^{x}\ln\left( w_{k}\right) }}{{\Sum{k = 1}{n}}\left(
w_{k}\right) ^{x}}-\dfrac{1}{x}.
\]


\begin{noliste}{a)}
 \setlength{\itemsep}{2mm}
\item Soit $y_{1};\ldots;y_{n}$des réels non tous nuls et
$z_{1};\ldots;z_{n}$ des réels quelconques. \\
 En étudiant la fonction
polynomiale du second degré $Q$ définie sur $\R$ par
$Q(t) = {\Sum{k = 1}{n}}\left( z_{k}-ty_{k}\right) ^{2}$,
établir l'inégalité \\
\[
\left( {\Sum{k = 1}{n}}y_{k}z_{k}\right) ^{2}\leq{\left( \sum
_{k = 1}{n}y_{k}{2}\right) {\left( \Sum{k = 1}{n}z_{k}{2}\right) }}.
\]


\item Montrer que la fonction $\varphi$ est strictement croissante sur
$\R_{+}{*}$.

\item On note $n_{0}$ le nombre d'entiers $k_{0}$ de $\left[ \!\left[
1;n\right] \!\right] $ vérifiant $w_{k_{0}} = { \max
_{1\leq k\leq n}\left( w_{k}\right) }$. Montrer que $1\leq
n_{0}\leq n-1$.

\item Donner un équivalent de ${ \Sum{k = 1}{n}}\left(
w_{k}\right) ^{x}$ en fonction de $n_{0}$ et $w_{k_{0}}$, lorsque $x$
tend
vers $ + \infty$.

\item Calculer en fonction de $w_{k_{0}}$, la limite de $\varphi\left(
x\right) $ lorsque $x$ tend vers $ + \infty$. \\
(on distinguera les deux
cas $w_{k_{0}} = 1$ et $w_{k_{0}}\neq1$).

\item En déduire que sur $\R_{+}{*}$ l'équation
$\varphi\left( x\right) = \dfrac{1}{n}{ \Sum{k = 1}{n}}\ln\left(
w_{k}\right) $ admet une unique solution.
\end{noliste}

\item[11.] On note $\left( W_{1};\ldots;W_{n}\right) $ un
$n$-échantillon de variables aléatoires à valeurs strictement
positives, indépendantes et de même loi $\mathcal{WB}\left(
\lambda;\alpha\right) $ définie dans la question 8. dont une
réalisation est le $n$-uplet $\left( w_{1};\ldots;w_{n}\right) $.\\
 On suppose que les paramètres $\lambda$ et $\alpha$ sont
inconnus.\\
 Soit $G$ la fonction de $\left( \R_{+}{*}\right)
^{2}$ dans $\R$ définie par $G\left( \lambda;\alpha\right)
 = \ln\left( { \prod_{k = 1}{n}f_{\left( \lambda;\alpha\right)
}\left( w_{k}\right) }\right) $.

\begin{noliste}{a)}
 \setlength{\itemsep}{2mm}
\item Montrer que la fonction $G$ admet un unique point critique
$\mbox{\ensuremath{\left(\hat{\lambda};\hat{\alpha}\right)}}$ sur
$\left(
\R_{+}{*}\right) ^{2}$.

\item Montrer que la fonction $G$ admet un maximum local au point
$\mbox{\ensuremath{\left(\hat{\lambda};\hat{\alpha}\right)}}$.
\end{noliste}
\end{noliste}


\end{document}