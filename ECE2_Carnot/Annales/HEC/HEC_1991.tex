\documentclass[11pt]{article}%
\usepackage{geometry}%
\geometry{a4paper,
 lmargin = 2cm,rmargin = 2cm,tmargin = 2.5cm,bmargin = 2.5cm}

\input{../../../../../../macros.tex}

\pagestyle{fancy} %
\lhead{ECE2 \hfill septembre 2017 \\
 Mathématiques\\[.2cm]} %
\chead{\hrule} %
\rhead{} %
\lfoot{} %
\cfoot{} %
\rfoot{\thepage} %

\renewcommand{\headrulewidth}{0pt}% : Trace un trait de séparation
 % de largeur 0,4 point. Mettre 0pt
 % pour supprimer le trait.

\renewcommand{\footrulewidth}{0.4pt}% : Trace un trait de séparation
 % de largeur 0,4 point. Mettre 0pt
 % pour supprimer le trait.

\setlength{\headheight}{14pt}

\title{\bf \vspace{-1cm} HEC 1991} %
\author{} %
\date{} %

\begin{document}

\maketitle %
\vspace{-1.2cm}\hrule %
\thispagestyle{fancy}

\vspace*{.4cm}

% DEBUT DU DOC À MODIFIER : tout virer jusqu'au début de l'exo

%Définition et changement de valeurs de
compteurs%newcounter{cpt1}{section} compteur cpt1 remis à 0 à chaque
aumentation par stepcounter du compteur section%setcounter{cpt1}{3} on
met le compteur à 3%addtocounter{cpt1}{5} on ajoute 5 au compteur%
stepcounter{cpt1} on ajoute 1% ifthenelse{test}{alors}{sinon} (page
206) pour subordonner à une condition % whiledo{test}{commande} pour
faire une boucle (page 206 aussi) % value{cpt1} pour noter dans le
document la valeur de cpt1 
%Définition définitive d'opérateurs
mathématiques\newcommand{\ch}{\operatorname{ch}} 
\newcommand{\sh}{\operatorname{sh}}
\renewcommand{\tanh}{\operatorname{th}}
\renewcommand{\sinh}{\operatorname{sh}}
\renewcommand{\cosh}{\operatorname{ch}}
\newcommand{\argsh}{\operatorname{argsh}}
\newcommand{\argch}{\operatorname{argch}}
\newcommand{\argth}{\operatorname{argth}}
\newcommand{\Id}{\operatorname{Id}}
\renewcommand{\leq}{\leq}
\renewcommand{\geq}{\geq }

\newcommand{\dlim}{\lim}
\newcommand{\dsum}{\sum}
\newcommand{\dprod}{\prod}



%Définition de nouvelles couleurs : rgb(trois paramètres red green blue
entre 0 et 1); cmyk (quatre cyan magenta yellow black) entre 0 et 1;
gray (entre 0 et 1) et black, white, red, green, blue, cyan, magenta,
yellow% definecolor{0gris}{gray}{0.8} 
% Nouvelle commande pour encadrer le titre car shabox ne veut que d'une
seule ligne; ATTENTION A LA TAILLE; petite différence avec shadowbox ou
doublebox, voire fcolorbox ou colorbox (au lieu de shabox; laisser le
parbox tranquille sauf pour la taille de la boîte
\newcommand{\Tbox}[1]{\begin{center} \shabox{\parbox{0.6
\linewidth}{#1}} \end{center}} %[1] pour 1 paramètre ; #1 pour ce que
fait le 1er paramètre; entre accolades ce que fait la commande
%Mise en page en mode fancy : en-têtes et pieds de pages puis
définition des en-têtes et pieds de pages\pagestyle{fancy}
\lhead{ECE 2 - Mathématiques \\
Quentin Dunstetter - ENC-Bessières 2011$\backslash$2012}
\chead{}
\rhead{HEC 1991}
\rfoot[ \ \thepage]{\thepage}
\cfoot{}
\lfoot{}
\thispagestyle{fancy} %Mise en page de la 1ère page en mode fancy
%Trait en bas et en haut de la page (entre en-tête et texte et texte et
pied de page)\renewcommand{\footrulewidth}{0.4pt}
\renewcommand{\headrulewidth}{0.4pt}

\begin{center}
{\huge HEC Eco 1991}
\end{center}

\section*{EXERCICE 1}

On considère la matrice : $M = {{\begin{smatrix}
{1} & {-3} & {3} \\
{-2} & {0} & {2} \\
{1} & {-1} & {3}\end{smatrix}
}}$ et l'endomorphisme $f$ de $\R^{3}$ représenté par $M$ dans la
base canonique de $\R^{3}$.

\begin{noliste}{1.}
 \setlength{\itemsep}{4mm}
\item 

\begin{noliste}{a)}
 \setlength{\itemsep}{2mm}
\item Résoudre par la méthode du pivot de Gauss, en discutant suivant
la
valeur du paramètre $\lambda $, le système d'équations : 
\[
\left\{ 
\begin{array}{c}
\left( {1-\lambda }\right) x_{1}-3x_{2} + 3x_{3} = 0 \\
{-2x_{1}-\lambda x_{2} + 2x_{3} = 0} \\
{x_{1}-\,x_{2} + \left( {3-\lambda }\right) x_{3} = 0}
\end{array}
\right. 
\]
En déduire les valeurs propres de la matrice $M.$

\item Montrer que l'endomorphisme $f$ est diagonalisable. Déterminer
une
base $(V_{1};V_{2};V_{3}\ )$ de vecteurs propres de $f$ que l'on
choisira de
manière que chacun ait, dans la base canonique de $\R^{3}$ des
coordonnées égales à $0$ ou à $1$.
\end{noliste}

\item A tout vecteur $y = (y_{1};y_{2};y_{3})$ de $\R^{3}$ fixé, on
associe la fonction $\varphi_{y}$ de $\R^{3}$ dans $\R$
telle que, pour tout élément $x = (x_{1};x_{2};x_{3})$ de $\R^{3}$ :
\[
\varphi_{y}(x) = (y_{1} + y_{2})x_{1} + (y_{1} + 2y_{2})x_{2} +
y_{3}x_{3}.
\]

\begin{noliste}{a)}
 \setlength{\itemsep}{2mm}
\item Le vecteur $y = (y_{1};y_{2};y_{3})$ étant fixé dans $\R^{3}$,
exprimer $\varphi_{y}(f(x))$ en fonction des coordonnées
$x_{1}$;$x_{2}$;$x_{3}$ de $x$.

\item Montrer qu'à chaque vecteur $y = (y_{1};y_{2};y_{3})$ de
$\R^{3}$, on peut faire correspondre un vecteur $Y =
(Y_{1};Y_{2};Y_{3})$ et un seul
de $\R^{3}$ tel que pour tout vecteur $x$ de $\R^{3}$ :
$\varphi_{y}(f(x)) = \varphi_{Y}(x)$ à cet effet, on exprimera
$Y_{1}$,$Y_{2}$
et $Y_{3}$ en fonction de $y_{1}$,$y_{2}$ et $y_{3}$.\\
On pose $Y = g(y)$. Vérifier que $g$ est une application linéaire de
$\R^{3}$ dans $\R^{3}$ ; en donner la matrice $N$ dans la base
canonique de $\R^{3}$.
\end{noliste}

\item On considère la matrice : $S = {{\begin{smatrix}
{1} & {1} & {0} \\
{1} & {2} & {0} \\
{0} & {0} & {1}\end{smatrix}
}}$, et l'endomorphisme $s$ de $\R^{3}$ représenté par $S$ dans la
base canonique de $\R^{3}$.

\begin{noliste}{a)}
 \setlength{\itemsep}{2mm}
\item Calculer l'inverse $S^{-1}$de $S$.

\item Calculer le produit matriciel $SMS^{-1}$.

\item En déduire que les vecteurs $s(V_{1})$, $s(V_{2})$ et $s(V_{3})$
sont
des vecteurs propres de la matrice $^{t}N$, transposée de la matrice
$N.$Préciser les valeurs propres associées.
\end{noliste}
\end{noliste}

\section*{EXERCICE 2}

Soit $a$ un nombre réel strictement positif. Pour tout nombre entier
nature1
non nul $n$, on considère la fonction polynomiale $P_{n}$ définie par
la
relation : $P_{n}(x) = \Sum{k = 1}{n}x^{k}-a$.

\begin{noliste}{1.}
 \setlength{\itemsep}{4mm}
\item Montrer que l'équation $P_{n}(x) = 0$ admet une solution positive
et une
seule, que l'on notera $x_{n}$. Montrer que $x_{n}\leq a$.

\item Étudier le signe de $P_{n + 1}(x_{n})$. En déduire que la suite
$(x_{n})_{n\geq 1}$est monotone.

\item Montrer que la suite $(x_{n})_{n\geq 1}$ est convergente. On note

$l$ sa limite. Prouver que $0\leq l<1$.

\item Montrer que, pour tout nombre entier naturel non nul $n$, le
nombre $x_{n}$ est solution de l'équation : 
\[
x^{n + 1}-(a + 1)x + a = 0.
\]
En déduire que : $l = \dfrac{{a}}{{a + 1}}$.
\end{noliste}

\section*{EXERCICE 3}

On désigne par $(\Omega ;A;P)$ un espace probabilisé et par $p$ un
nombre réel tel que $0<p<1$.\\
On rappelle que le symbole $P\left(\Ev{A/B}\right)$ désigne la
probabilité conditionnelle de 
$A$ sachant que $B$ est réalisé.

\subsection*{I. }

Un jour donné, un modèle de voiture est successivement examiné par $N$
clients éventuels. On suppose que $N$ est une variable aléatoire
définie sur 
$(\Omega ;A;P)$, à valeurs dans l'ensemble $N$ des nombres entiers
naturels
telle que, pour tout couple $(r;s)$ de nombre entiers naturels :
\[
P\left(\Ev{N\geq r + s/N\geq r}\right) = P\left(\Ev{N\geq s}\right).
\]

\begin{noliste}{1.}
 \setlength{\itemsep}{4mm}
\item Déterminer la probabilité $P\left(\Ev{N\geq 1}\right)$. En
choisissant $r = s = 1$,
calculer $P\left(\Ev{N\geq 2}\right)$. En déduire la valeur de
$P\left(\Ev{N = 1}\right)$.

\item En raisonnant par récurrence, trouver, pour tout nombre entier
naturel 
$n$, les probabilités $P\left(\Ev{N\geq n}\right)$ et $P\left(\Ev{N =
n}\right)$.

\item Trouver l'espérance $\E(N)$ et la variance $\V(N)$ de $N$.

\item Application numérique. On suppose que $p = \dfrac{{1}}{{10}}$.
Calculer$P\left(\Ev{N = n}\right)$, $\E\left(\Ev{N}\right)$ et $\V(N)$.
\end{noliste}

\subsection*{II.}

Parmi les clients qui examinent le modèle, certains passent commande,
les
autres non. Pour tout nombre entier naturel non nul $i$, on note
$X_{i}$ la
variable aléatoire qui prend la valeur 1 si le $i^{i\grave{e}me}$
client
passe commande d'une voiture et 0 dans le cas contraire. On convient
que $X_{0} = 0$.\\
On désigne par $\alpha $ un nombre réel tel que $0<\alpha <1$. On
suppose
que la suite $(X_{i})_{i\in \N}$ est définie sur l'espace probabilisé
$(\Omega ;A;P)$ et que, pour tout nombre entier naturel non nul $i$,
$P\left(\Ev{X_{i} = 1}\right) = \alpha.$\\
On suppose enfin que les variables aléatoires $X_{i}$ sont mutuellement
indépendantes et indépendantes de $N.$

\begin{noliste}{1.}
 \setlength{\itemsep}{4mm}
\item Pour tout nombre entier naturel $n$, on pose $S_{n} = X_{0} +
X_{1} +... + X_{n}$.\\
Déterminer la loi de probabilité de $S_{n}$.

\item On note $S$ la variable aléatoire définie sur $(\Omega ;A;P)$ de
la façon suivante : pour tout élément $\omega $ de $\Omega $, si
$N(\omega ) = n$,
on pose $S(\omega ) = S_{n}(\omega )$. Ainsi, $S$ représente le nombre
de
voitures commandées. On se propose de trouver la loi de probabilité de
$S.$

\begin{noliste}{a)}
 \setlength{\itemsep}{2mm}
\item Soit $k$un nombre entier naturel. On admet (la justification
n'est pas
demandée) que, pour tout nombre réel fixé $x$ appartenant à
l'intervalle $[0;1[$ la série $\Sum{n = k}{+ \infty }C_{n}{k}x^{n}$ est
convergente.\\
On note $\sigma_{k}$ la somme de cette série. Ainsi $\sigma
_{k} = \Sum{n = k}{+ \infty }C_{n}{k}x^{n}$.\\
Montrer que pour tout entier naturel $k$ : $\sigma_{k + 1} = x\sigma
_{k} + x\sigma_{k + 1}$.

\item En déduire par récurrence sur $k$, que : $\sigma_{k} =
\dfrac{{x^{k}}}{{\left( {1-x}\right) ^{k + 1}}}$

\item Trouver la loi de probabilité de $S$.

\item Application numérique.\\
On suppose que $p = \dfrac{1}{10}$ et $\alpha = \dfrac{1}{2}$.
Déterminer la
loi de probabilité de $S$.\\
Vérifier qu'elle est du même type que la loi de $N$ obtenue dans la
partie I.\\
Calculer l'espérance et la variance de $S$.
\end{noliste}
\end{noliste}

\label{fin}

\end{document}

