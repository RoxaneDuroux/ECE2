\documentclass[11pt]{article}%
\usepackage{geometry}%
\geometry{a4paper,
 lmargin = 2cm,rmargin = 2cm,tmargin = 2.5cm,bmargin = 2.5cm}

\input{../../../../../../macros.tex}

\pagestyle{fancy} %
\lhead{ECE2 \hfill septembre 2017 \\
 Mathématiques\\[.2cm]} %
\chead{\hrule} %
\rhead{} %
\lfoot{} %
\cfoot{} %
\rfoot{\thepage} %

\renewcommand{\headrulewidth}{0pt}% : Trace un trait de séparation
 % de largeur 0,4 point. Mettre 0pt
 % pour supprimer le trait.

\renewcommand{\footrulewidth}{0.4pt}% : Trace un trait de séparation
 % de largeur 0,4 point. Mettre 0pt
 % pour supprimer le trait.

\setlength{\headheight}{14pt}

\title{\bf \vspace{-1cm} HEC 2006} %
\author{} %
\date{} %

\begin{document}

\maketitle %
\vspace{-1.2cm}\hrule %
\thispagestyle{fancy}

\vspace*{.4cm}

% DEBUT DU DOC À MODIFIER : tout virer jusqu'au début de l'exo

%Définition et changement de valeurs de
compteurs%newcounter{cpt1}{section} compteur cpt1 remis à 0 à chaque
aumentation par stepcounter du compteur section%setcounter{cpt1}{3} on
met le compteur à 3%addtocounter{cpt1}{5} on ajoute 5 au compteur%
stepcounter{cpt1} on ajoute 1% ifthenelse{test}{alors}{sinon} (page
206) pour subordonner à une condition % whiledo{test}{commande} pour
faire une boucle (page 206 aussi) % value{cpt1} pour noter dans le
document la valeur de cpt1 
%Définition définitive d'opérateurs
mathématiques\newcommand{\ch}{\operatorname{ch}} 
\newcommand{\sh}{\operatorname{sh}}
\renewcommand{\tanh}{\operatorname{th}}
\renewcommand{\sinh}{\operatorname{sh}}
\renewcommand{\cosh}{\operatorname{ch}}
\newcommand{\argsh}{\operatorname{argsh}}
\newcommand{\argch}{\operatorname{argch}}
\newcommand{\argth}{\operatorname{argth}}
\newcommand{\Id}{\operatorname{Id}}
\renewcommand{\leq}{\leq}
\renewcommand{\geq}{\geq }

\newcommand{\dlim}{\lim}
\newcommand{\dsum}{\sum}
\newcommand{\dprod}{\prod}



%Définition de nouvelles couleurs : rgb(trois paramètres red green blue
entre 0 et 1); cmyk (quatre cyan magenta yellow black) entre 0 et 1;
gray (entre 0 et 1) et black, white, red, green, blue, cyan, magenta,
yellow% definecolor{0gris}{gray}{0.8} 
% Nouvelle commande pour encadrer le titre car shabox ne veut que d'une
seule ligne; ATTENTION A LA TAILLE; petite différence avec shadowbox ou
doublebox, voire fcolorbox ou colorbox (au lieu de shabox; laisser le
parbox tranquille sauf pour la taille de la boîte
\newcommand{\Tbox}[1]{\begin{center} \shabox{\parbox{0.6
\linewidth}{#1}} \end{center}} %[1] pour 1 paramètre ; #1 pour ce que
fait le 1er paramètre; entre accolades ce que fait la commande
%Mise en page en mode fancy : en-têtes et pieds de pages puis
définition des en-têtes et pieds de pages\pagestyle{fancy}
\lhead{ECE 2 - Mathématiques \\
Quentin Dunstetter - ENC-Bessières 2011$\backslash$2012}
\chead{}
\rhead{HEC III 2006}
\rfoot[ \ \thepage]{\thepage}
\cfoot{}
\lfoot{}
\thispagestyle{fancy} %Mise en page de la 1ère page en mode fancy
%Trait en bas et en haut de la page (entre en-tête et texte et texte et
pied de page)\renewcommand{\footrulewidth}{0.4pt}
\renewcommand{\headrulewidth}{0.4pt}

\begin{center}
{\small CHAMBRE D\E\ COMMERCE ET D'INDUSTRIE DE PARIS}

\textbf{DIRECTION DE L'ENSEIGNEMENT}

Direction des Admissions et concours

\underline{\hspace*{3cm}}

{\Large ECOLE DES\ HAUTES\ ETUDES\ COMMERCIALES}

{\Large E.S.C.P.-E.A.P.}

{\Large ECOL\E\ SUPERIEUR\E\ D\E\ COMMERC\E\ D\E\ LYON}{\large }

CONCOURS D'ADMISSION\ SUR\ CLASSES\ PREPARATOIRES

\underline{\hspace*{3cm}}

\textbf{OPTION ECONOMIQUE}

{\Large MATHEMATIQUES III}

\textbf{Année 2006}

\underline{\hspace*{3cm}}
\end{center}

\begin{quotation}
\noindent \textsl{La présentation, la lisibilité, l'orthographe, la
qualité
de la rédaction, la clarté et la précision des raisonnements entreront
pour
une part importante dans l'appréciation des copies.}

\noindent \textsl{Les candidats sont invités à encadrer dans la mesure
du
possible les résultats de leurs calculs.}

\noindent \textsl{Ils ne doivent faire usage d'aucun document :
l'utilisation de toute calculatrice et de tout matériel électronique
est
interdite.}

\noindent \textsl{Seule l'utilisation d'une règle graduée est
autorisée.}

\noindent \textsl{\hrulefill }
\end{quotation}

\section*{EXERCICE}

Dans cet exercice, $n$ désigne un entier supérieur ou égal à 2,
$\lambda $
et $\mu $ deux nombres réels strictement positifs et $B$ la matrice de
$\mathfrak{M}_{n}(\R)$ suivante : 
\[
B = 
\begin{smatrix}
0 & \lambda & 0 & \ldots & \ldots & 0 \\
\mu & 0 & \lambda & \ddots & & \vdots \\
0 & \mu & \ddots & \ddots & \ddots & \vdots \\
\vdots & \ddots & \ddots & \ddots & \ddots & 0 \\
\vdots & & \ddots & \mu & 0 & \lambda \\
0 & \dots & \dots & 0 & \mu & 0
\end{smatrix},\quad \text{c'est-à-dire}\quad B = (b_{i,j})\quad a =
\text{avec}\quad \left\{ 
\begin{array}{ll}
b_{i,j} = \lambda & \text{si }j = i + 1 \\
b_{i,j} = \mu & \text{si }j = i-1 \\
b_{i,j} = 0 & \text{sinon}
\end{array}
\right.
\]
On s'intéresse aux valeurs propres de $B$ et pour cela, pour $a$ réel,
on
note $A_{a} = B-aI_{n}$, où $I_{n}$ désigne la matrice unité d'ordre
$n$.

\begin{noliste}{1.}
 \setlength{\itemsep}{4mm}
\item \textbf{Exemple.} Dans cette question, on considère la matrice $B
= 
\begin{smatrix}
0 & 1 & 0 & 0 & 0 \\
1 & 0 & 1 & 0 & 0 \\
0 & 1 & 0 & 1 & 0 \\
0 & 0 & 1 & 0 & 1 \\
0 & 0 & 0 & 1 & 0
\end{smatrix}
$

\begin{noliste}{a)}
 \setlength{\itemsep}{2mm}
\item La matrice $B$ est-elle diagonalisable ?

\item Déterminer les valeurs propres et les vecteurs propres de
l'endomorphisme de $\R^{5}$ canoniquement associé à la matrice $B.$\\
\textsl{On revient maintenant au cas général. On dira qu'une suite
$(u_{k})_{k\in \N}$ vérifie la propriété (R) lorsque l'on a, pour
tout $k$ de $\N$ : $\mu u_{k}-au_{k + 1} + \lambda u_{k + 2} = 0$}
\end{noliste}

\item Montrer qu'un vecteur $X = 
\begin{smatrix}
x_{1} \\
x_{2} \\
\vdots \\
x_{n}\end{smatrix}
$ de $\mathfrak{M}_{n,1}(\R)$ vérifie $A_{a}X = 0$ si et seulement
si,\\
en posant $x_{0} = x_{n + 1} = 0$, les nombres
$x_{0},x_{1},...,x_{n},x_{n + 1}$
sont les $n + 2$ premiers termes d'une suite vérifiant $(R)$.

\item On suppose dans cette question que $a^{2}> 4 \lambda \mu$.

\begin{noliste}{a)}
 \setlength{\itemsep}{2mm}
\item Déterminer l'ensemble des suites vérifiant $(R)$

\item Montrer que si un vecteur $X$ de $\mathfrak{M}_{n,1}(\R)$ vérifie
$A_{a}X = 0$, alors $X$ est le vecteur nul.
\end{noliste}

\item On suppose dans cette question que $a^{2} = 4 \lambda \mu$.

\begin{noliste}{a)}
 \setlength{\itemsep}{2mm}
\item Déterminer l'ensemble des suites vérifiant $(R)$.

\item Montrer que si un vecteur $X$ de $\mathfrak{M}_{n,1}(\R)$ vérifie
$A_{a}X = 0$, alors $X$ est le vecteur nul.
\end{noliste}

\item 

\begin{noliste}{a)}
 \setlength{\itemsep}{2mm}
\item En déduire que si $B$ admet des valeurs propres, elles
appartiennent à
l'intervalle $]-2\sqrt{\lambda \mu },2\sqrt{\lambda \mu }[$

\item Un théorème classique du à Jacques Hadamard, affirme que si le
réel $a$
est valeur propre de $B$, alors $|a|\leq \lambda + \mu $ (ce théorème
n'est pas à démontrer).\\
Le résultat que l'on a obtenu en [5)a :] est-il meilleur que le
résultat du théorème d'Hadamard ?
\end{noliste}
\end{noliste}

\section*{Problème}

\textsl{Ce problème a pour objet principal la modélisation d'un
processus aléatoire ponctuel (discret) représenté par une suite de
variables aléatoires de Bernoulli. Ce modèle est ensuite approché par
un modèle continu, et dans la
dernière partie, on s'intéresse, dans un cas particulier, à
l'adéquation de ce modèle continu au modèle discret initial.} \\
\textsl{Dans tout le problème, $\lambda $ désigne un nombre réel de
l'intervalle ouvert $]0,1[$.} \vspace{0.5cm}

\subsection*{Partie I : Modèle discret.}

On suppose donnée une suite $(X_{n})_{n\in \N}$ de variables aléatoires
de Bernoulli, définies sur un espace probabilisé
$(\Omega,\mathcal{A},P)$. Pour tout $n$ de $\N$, on note $p_{n}$ le
paramètre de la
variable aléatoire $X_{n}$.\\
On suppose que $p_{0}$ appartient à l'intervalle ouvert $]0,1[$ et que
pour
tout $n$ de $\N$, on a les probabilités conditionnelles suivantes : 
\[
P_{(X_{n} = 1)}(X_{n + 1} = 1) = P\left(\Ev{X_{n} = 1}\right) =
p_{n}\quad \text{et}\quad
P_{(X_{n} = 0)}(X_{n + 1} = 1) = \lambda P\left(\Ev{X_{n} = 1}\right) =
\lambda p_{n}
\]
\textsl{[On rappelle que la probabilité conditionnelle $P_{A}(B)$ peut
aussi
se noter $P\left(\Ev{B/A}\right)$]}

\begin{noliste}{1.}
 \setlength{\itemsep}{4mm}
\item 

\begin{noliste}{a)}
 \setlength{\itemsep}{2mm}
\item Montrer que pour tout entier $n$ de $\N$, on a : \qquad $p_{n +
1} = (1-\lambda )p_{n}{2} + \lambda p_{n}$.

\item En déduire que pour tout entier $n$ de $\N$, on a : \qquad
$0<p_{n}<1$
\end{noliste}

\item 

\begin{noliste}{a)}
 \setlength{\itemsep}{2mm}
\item Montrer que la suite $(p_{n})_{n\in \N}$ est convergente et
déterminer sa limite.

\item On pose $a = (1-\lambda )p_{0} + \lambda $. Établir, pour tout
$n$ de $\N$, l'inégalité : \qquad $p_{n}\leq a^{n}$.\\
En déduire que la série de terme général $p_{n}$ est convergente.
\end{noliste}

\item Pour tout $n$ de $N$, on définit la variable aléatoire $Y_{n}$
par :
\quad $Y_{n} = \dsum\limits_{k = 0}{n}X_{k}$ \quad et on note
$\E(Y_{n})$ son espérance.

\begin{noliste}{a)}
 \setlength{\itemsep}{2mm}
\item Justifier l'existence de la limite $L$ de la suite $\left(
\E(Y_{n})\right)_{n\in \N}$.

\item Écrire une fonction \Scilab{} permettant de calculer une valeur
approchée
de $\E(Y_{n})$. L'en-tête de cette fonction sera : \\
\texttt{function approx( n :integer ; p0, lambda : real) : real}
\end{noliste}

\item 

\begin{noliste}{a)}
 \setlength{\itemsep}{2mm}
\item Exprimer, pour tout $n$ de $\N$, la covariance $\Cov(X_{n},X_{n +
1})$ de $X_{n}$ et $X_{n + 1}$ en fonction de $p_{n}$ et $p_{n + 1}$.
\\
Les variables $X_{n}$ et $X_{n + 1}$ sont-elles indépendantes ?

\item Montrer que $\dlim{n\rightarrow \infty }\left( \dfrac{p_{n +
1}}{p_{n}}\right) = \lambda $.

\item Pour tout $n$ de $\N$, on note $r_{n}$ le coefficient de
corrélation linéaire entre $X_{n}$ et $X_{n + 1}$ : 
\[
r_{n} = \dfrac{\Cov(X_{n},X_{n + 1})}{\sqrt{\V(X_{n})\V(X_{n +
1})}}\qquad 
\text{où }\V\text{ désigne la variance}
\]
Exprimer $r_{n}$ en fonction de $p_{n}$ et $p_{n + 1}$.\\
Montrer que lorsque $n$ tend vers $ + \infty $, $r_{n}$ est équivalent
à $\dfrac{1-\lambda }{\sqrt{\lambda }}p_{n}$.
\end{noliste}
\end{noliste}

\subsection*{Partie II : Simulation.}

\textsl{On rappelle que la fonction \Scilab{} \texttt{random } simule
une
variable aléatoire suivant une loi uniforme sur l'intervalle $[0,1]$.}
\\
Soit $N$ un entier naturel non nul et inférieur ou égal à $200$.\\
On considère la suite finie des $N + 1$ variables aléatoires
$X_{0},X_{1},...,X_{N}$ vérifiant les conditions de la partie I,
modélisée
par l'arbre pondéré suivant, et on note encore $Y_{N} = X_{0} + \cdots
+ X_{N}$.

\begin{center}
\includegraphics[height = 7.5cm]{hec3_{2}006.png}
\end{center}

\noindent On cherche à étudier cette situation à l'aide du programme
suivant :
\begin{verbatim}
Program evaluation;
 var lambda,p0 : real;
 
function bernoulli(p :real) :integer;
 begin
 if random < = p then bernoulli : = 1 else bernoulli : = 0;
 end;
 
function simulation(N :integer) :integer;
 var c,i,x : integer; a,p,q :real;
 begin
 p : = p0; x : = bernoulli(p); c : = x;
 for i : = 1 to N do
 begin
 q : = p;
 if x = 0 then q : = p*lambda;
 x : = bernoulli(q); c : = c + x; p : = (1-lambda)*p*p + lambda * p;
 end;
 simulation : = c;
 end;
 
var y,k, N :integer ; T : array[0..200] of integer; begin
 readln(lambda);readln(p0);readln(N);randomize;
 for k : = 0 to N do T[k] : = 0;
 for k : = 1 to 10000 do
 begin
 y : = simulation(N); T[y] : = T[y] + 1;
 end;
 for k : = 0 to N do
 begin
 write(T[k]); write(' ');
 end;
 readln;
end.
\end{verbatim}

\begin{noliste}{1.}
 \setlength{\itemsep}{4mm}
\item Expliquer le résultat rendu par la fonction \texttt{bernoulli}.

\item Expliquer le fonctionnement de la fonction \texttt{simulation} et
donner en particulier la signification du résultat rendu.

\item Le programme \texttt{evaluation} permet de simuler une variable
aléatoire. En se référant à la loi faible des grands nombres, quelle
loi de probabilité peut-on simuler grâce à ce programme ?
\end{noliste}

\subsection*{Partie III : Modèle continu.}

soit $\ell $ tel que $0<\ell <1$ et soit $T$ un réel strictement
positif.
Pour tout $t$ de $[0,T]$, on définit une variable aléatoire $X(t)$ sur
un
espace probabilisé $(\Omega,\mathcal{A},P)$ qui suit une loi de
Bernoulli
de paramètre $p(t)$, c'est-à-dire que : $p(t) = P\left(\Ev{X(t) =
1}\right)\left(\Ev{t}\right) = 1)$. On suppose que la
fonction $p$ est définie et dérivable sur $[0,T]$, de dérivée
$p^{\prime }$,
et vérifie la relation :
\[
\forall t\in \lbrack 0,T]\qquad p^{\prime }(t) = (1-\ell )p(t)(p(t)-1)
\]
On note $p(0) = p_{0}$ et on suppose que $p_{0}$ appartient à
l'intervalle
ouvert $]0,1[$.

\begin{noliste}{1.}
 \setlength{\itemsep}{4mm}
\item Soit $f$ la fonction définie sur $[0,T]$ par $f(t) = p(t) \times
e^{(1-\ell)t}$. Montrer que $f$ est croissante sur $[0,T]$ et en
déduire que
la fonction $p$ ne s'annule pas sur $[0,T]$.

\item 

\begin{noliste}{a)}
 \setlength{\itemsep}{2mm}
\item Soit $g$ la fonction définie sur $[0,T]$ par : $g(t) =
\dfrac{e^{-(1-\ell )t}}{p(t)}$. \quad Exprimer $g^{\prime }(t)$ en
fonction de $\ell $ et $t$ et en déduire qu'il existe une constante $k$
telle que, pour tout $t$ de $[0,T]$, \quad $g(t) = k + e^{(\ell -1)t}$.

\item Montrer que, pour tout $t$ de $[0,T]$, on a : $p(t) =
\dfrac{p_{0}}{p_{0} + (1-p_{0})e^{(1-\ell )t}}$.

\item Dresser le tableau de variations de $p$ sur $[0,T]$. Soit $(C)$
la
courbe représentative de $p$ dans le plan rapporté à un repère
orthogonal. A
quelle condition, portant sur $p_{0}$, la courbe $(C)$ présente-t-elle
un
point d'inflexion ? Quelles sont alors les coordonnées de ce point ?
\end{noliste}

\item Pour tout $n\in \N^{\ast }$, on note $\delta = \dfrac{T}{n}$ et
pour tout $k\in \lbrack \hspace{-0.15em}[0,n]\hspace{-0.13em}]$, $t_{k}
= k\delta $.\\
Pour tout $n\in \N^{\ast }$, on définit la variable aléatoire $Z_{n}$
par : $Z_{n} = \dsum\limits_{k = 0}{n}X(t_{k})$, d'espérance
$\E(Z_{n})$.

\begin{noliste}{a)}
 \setlength{\itemsep}{2mm}
\item Montrer que la suite $\left( \dfrac{\E(Z_{n})}{n}\right)_{n\in
\N^{\times }}$ est convergente et de limite
$\dfrac{1}{T}\dint_{0}{T}p(t)\ dt$. Cette limite sera notée $m(T)$ dans
la suite de cette partie.

\item Justifier la validité du changement de variable $u = e^{(1-\ell
)t}$
dans l'intégrale $\dint\limits_{0}{T}p(t)\ dt$ et en déduire que l'on a
: 
\[
m(T) = \dfrac{1}{(1-\ell )T}\dint\limits_{1}{e^{(1-\ell )T}}\left(
\dfrac{1}{u}-\dfrac{1-p_{0}}{p_{0} + (1-p_{0})u}\right) \ du
\]

\item En déduire une expression de $m(T)$ en fonction de $p_{0}$, $\ell
$ et 
$T$ et montrer que, lorsque $T$ tend vers $ + \infty $, $p_{0}$ et
$\ell $ étant fixés, $m(T)$ est équivalent à $-\dfrac{\ln
(1-p_{0})}{(1-\ell )T}$
\end{noliste}
\end{noliste}

\subsection*{Partie IV : Retour au modèle discret.}

Soit $n$ un entier naturel non fixé. Avec les notations des parties I
et
III, on suppose que $p_{0} = \dfrac{1}{3}$, $\ell = \dfrac{1}{2}$ et $T
= 2n(1-\lambda )$.

\begin{noliste}{1.}
 \setlength{\itemsep}{4mm}
\item Montrer que la fonction $p$ définie dans la partie III est deux
fois dérivable sur $[0,T]$, et montrer que pour tout $t$ de $[0,T]$ :
\qquad $p^{\prime \prime }(t) = \dfrac{1}{4}(2p(t)-1)p(t)(p(t)-1)$
\quad où $p^{\prime
\prime }$ désigne la dérivée seconde de $p$.

\item On rappelle que pour tout $k$ de $[ \
\hspace{-0.15em}[0,n]\hspace{-0.13em}]$, \quad $t_{k} = k\delta =
k\dfrac{T}{n}$ et que $p_{k}$ a été défini dans
la partie I.\\
Pour tout $k$ de $[ \ \hspace{-0.15em}[0,n]\hspace{-0.13em}]$, on pose
$\varepsilon_{k} = p(t_{k})-p_{k}$.

\begin{noliste}{a)}
 \setlength{\itemsep}{2mm}
\item Établir, pour tout $k$ de $[ \
\hspace{-0.15em}[0,n-1]\hspace{-0.13em}]$,
l'inégalité suivante : \quad $|p(t_{k + 1} ) -p(t_{k})-\delta p^{\prime
}(t_{k})|\leq \dfrac{\delta ^{2}}{8}$.

\item Établir, pour tout $k$ de $[ \
\hspace{-0.15em}[0,n-1]\hspace{-0.13em}]$,
l'égalité : $p(t_{k}) + \delta p^{\prime }(t_{k})-p_{k + 1} =
\varepsilon
_{k}[1-(1-\lambda )(1-p(t_{k})-p_{k})]$.

\item En déduire, pour tout $k$ de $[ \
\hspace{-0.15em}[0,n-1]\hspace{-0.13em}]
$, l'inégalité suivante : $|\varepsilon_{k + 1}|\leq \dfrac{\delta
^{2}}{8} + \dfrac{1}{3}(\lambda + 2)|\varepsilon_{k}|$.

\item Établir, pour tout $k$ de $[ \
\hspace{-0.15em}[0,n]\hspace{-0.13em}]$,
l'inégalité : $|\varepsilon_{k}|\leq 6(1-\lambda )$.
\end{noliste}

\item Pour tout réel $\alpha $ tel que $\alpha >18(1-\lambda )$, on
pose : $N(\alpha ) = \dfrac{1}{1-\lambda }\ln \left( \dfrac{\alpha
}{12(1-\lambda )}-\dfrac{1}{2}\right) $.

\begin{noliste}{a)}
 \setlength{\itemsep}{2mm}
\item Vérifier que pour tout réel $\alpha> 18(1-\lambda)$, on a
$N(\alpha)>0$.

\item Montrer que si $n\leq N(\alpha )$, alors pour tout $k$ de
$[\hspace{-0.15em}[0,n]\hspace{-0.13em}]$, on a : $\left|
\dfrac{p(t_{k})-p_{k}}{p(t_{k})}\right| \leq \alpha $.

\item Montrer que, pour $\alpha $ fixé, $\dlim{\lambda \rightarrow
1}N(\alpha ) = + \infty $

\item Conclure sur la qualité de l'approximation du modèle discret par
le modèle continu, lorsque $\lambda$ se "rapproche" de $1$.
\end{noliste}
\end{noliste}

\label{fin}

\end{document}

