\documentclass[11pt]{article}%
\usepackage{geometry}%
\geometry{a4paper,
 lmargin = 2cm,rmargin = 2cm,tmargin = 2.5cm,bmargin = 2.5cm}

\input{../../../../../../macros.tex}

\pagestyle{fancy} %
\lhead{ECE2 \hfill septembre 2017 \\
 Mathématiques\\[.2cm]} %
\chead{\hrule} %
\rhead{} %
\lfoot{} %
\cfoot{} %
\rfoot{\thepage} %

\renewcommand{\headrulewidth}{0pt}% : Trace un trait de séparation
 % de largeur 0,4 point. Mettre 0pt
 % pour supprimer le trait.

\renewcommand{\footrulewidth}{0.4pt}% : Trace un trait de séparation
 % de largeur 0,4 point. Mettre 0pt
 % pour supprimer le trait.

\setlength{\headheight}{14pt}

\title{\bf \vspace{-1cm} HEC 1998} %
\author{} %
\date{} %

\begin{document}

\maketitle %
\vspace{-1.2cm}\hrule %
\thispagestyle{fancy}

\vspace*{.4cm}

% DEBUT DU DOC À MODIFIER : tout virer jusqu'au début de l'exo

%Définition et changement de valeurs de
compteurs%newcounter{cpt1}{section} compteur cpt1 remis à 0 à chaque
aumentation par stepcounter du compteur section%setcounter{cpt1}{3} on
met le compteur à 3%addtocounter{cpt1}{5} on ajoute 5 au compteur%
stepcounter{cpt1} on ajoute 1% ifthenelse{test}{alors}{sinon} (page
206) pour subordonner à une condition % whiledo{test}{commande} pour
faire une boucle (page 206 aussi) % value{cpt1} pour noter dans le
document la valeur de cpt1 
%Définition définitive d'opérateurs
mathématiques\newcommand{\ch}{\operatorname{ch}} 
\newcommand{\sh}{\operatorname{sh}}
\renewcommand{\tanh}{\operatorname{th}}
\renewcommand{\sinh}{\operatorname{sh}}
\renewcommand{\cosh}{\operatorname{ch}}
\newcommand{\argsh}{\operatorname{argsh}}
\newcommand{\argch}{\operatorname{argch}}
\newcommand{\argth}{\operatorname{argth}}
\newcommand{\Id}{\operatorname{Id}}
\renewcommand{\leq}{\leq}
\renewcommand{\geq}{\geq }

\newcommand{\dlim}{\lim}
\newcommand{\dsum}{\sum}
\newcommand{\dprod}{\prod}



%Définition de nouvelles couleurs : rgb(trois paramètres red green blue
entre 0 et 1); cmyk (quatre cyan magenta yellow black) entre 0 et 1;
gray (entre 0 et 1) et black, white, red, green, blue, cyan, magenta,
yellow% definecolor{0gris}{gray}{0.8} 
% Nouvelle commande pour encadrer le titre car shabox ne veut que d'une
seule ligne; ATTENTION A LA TAILLE; petite différence avec shadowbox ou
doublebox, voire fcolorbox ou colorbox (au lieu de shabox; laisser le
parbox tranquille sauf pour la taille de la boîte
\newcommand{\Tbox}[1]{\begin{center} \shabox{\parbox{0.6
\linewidth}{#1}} \end{center}} %[1] pour 1 paramètre ; #1 pour ce que
fait le 1er paramètre; entre accolades ce que fait la commande
%Mise en page en mode fancy : en-têtes et pieds de pages puis
définition des en-têtes et pieds de pages\pagestyle{fancy}
\lhead{ECE 2 - Mathématiques \\
Quentin Dunstetter - ENC-Bessières 2011$\backslash$2012}
\chead{}
\rhead{HEC 1998}
\rfoot[ \ \thepage]{\thepage}
\cfoot{}
\lfoot{}
\thispagestyle{fancy} %Mise en page de la 1ère page en mode fancy
%Trait en bas et en haut de la page (entre en-tête et texte et texte et
pied de page)\renewcommand{\footrulewidth}{0.4pt}
\renewcommand{\headrulewidth}{0.4pt}

\begin{center}
{\huge HEC Eco 1998}
\end{center}

\section*{EXERCICE I}

\begin{noliste}{1.}
 \setlength{\itemsep}{4mm}
\item 

\begin{noliste}{a)}
 \setlength{\itemsep}{2mm}
\item Soit $\mu $ un paramètre réel. On considère le système
d'équations

$(1)\left\{ 
\begin{tabular}{l}
$\mu x_{1} + x_{2}\hfill = 0$ \\
$3x_{1} + \mu x_{2} + 2x_{3}\hfill = 0$ \\
$2x_{2} + \mu x_{3} + 3x_{4}\hfill = 0$ \\
$x_{3} + \mu x_{4}\hfill = 0$\end{tabular}\ \right. $ d'inconnues
$x_{1}$, $x_{2}$, $x_{3}$, $x_{4}$.

\begin{nonoliste}{(i)}
\item Montrer que ce système admet les mêmes solutions que le système
$(2)\left\{ 
\begin{tabular}{lcr}
$\mu x_{1} + x_{2}\hfill $ & $ = $ & $0$ \\
$x_{3} + \mu x_{4}\hfill $ & $ = $ & $0$ \\
$(3-\mu ^{2})x_{1}-2\mu x_{4}$ & $ = $ & $0$ \\
$-2\mu x_{1} + (3-\mu ^{2})x_{4}$ & $ = $ & $0$\end{tabular}\ \right. $

\item Résoudre, en discutant suivant les valeurs de $\mu $, le sysème
$\quad \left\{ 
\begin{tabular}{lcr}
$(3-\mu ^{2})x_{1}-2\mu x_{4}$ & $ = $ & $0$ \\
$-2\mu x_{1} + (3-\mu ^{2})x_{4}$ & $ = $ & $0$\end{tabular}\ \right. $

\item Déterminer enfin, suivant les valeurs de $\mu $, les solutions du
système (1).
\end{nonoliste}

\item Déterminer les valeurs propres et les vecteurs propres de la
matrice

$A = \left( 
\begin{tabular}{cccc}
$1$ & $1$ & $0$ & $0$ \\
$3$ & $1$ & $2$ & $0$ \\
$0$ & $2$ & $1$ & $3$ \\
$0$ & $0$ & $1$ & $1$\end{tabular}\right).$
\end{noliste}

\item Pour tout entier $n\geq 1$ on note $\R_{n}[x]$ l'espace
vectoriel des fonctions polynômes de degré inférieur ou égal à $n$ et,
à
toute fonction polynôme $P$ de $\R_{n}[x]$, on associe la fonction
polynôme $T_{n}P$ définie sur $\R$ par $T_{n}P\left(\Ev{x}\right) =
\left(\Ev{nx + 1}\right)\,P\left(\Ev{x}\right) +
\left(\Ev{1-x^{2}}\right)\,P^{\prime }(x).$

\begin{noliste}{a)}
 \setlength{\itemsep}{2mm}
\item Montrer que, pour tout $n\geq 1$, l'application $P\mapsto T_{n}P$
est un endomorphisme de $\R_{n}[x]$.

\item Donner la matrice $M_{n}$ de cet endomorphisme $T_{n}$ dans la
base de 
$\R_{n}[x]$ formée des fonctions polynômes $1,\;X,\dots \;X^{n}$ où
$X^{k}$ désigne, pour tout $k\in \{0,1,...,n\}$, la fonction $x\mapsto
x^{k}$.

\item Dans le cas $n = 3$, donner les valeurs propres de $T_{3}$ et
écrire
les fonctions polyn\^{o}mes formant une base de vecteurs propres.

\item En faisant la somme des lignes de la matrice $M_{n}$, déterminer
simplement une valeur propre de $T_{n}.$
\end{noliste}

\item On se propose de déterminer plus généralement toutes les valeurs
propres de $T_{n}$.

\begin{noliste}{a)}
 \setlength{\itemsep}{2mm}
\item Etant donné un réel $\lambda $ calculer, pour $\,-1<x<1$,
l'intégrale $g_{\lambda }(x) = \dint{0}{x}\dfrac{nt + 1-\lambda
}{1-t^{2}}\,dt.$ On
cherchera d'abord deux réels $a$ et $b$ tels que $\,\,\dfrac{nt +
1-\lambda }{1-t^{2}} = \dfrac{a}{1-t} + \dfrac{b}{1 + t}\,$

\item Montrer que si les nombres $\,h = n + 1 - \lambda\, $ et $\, k =
n - 1
 + \lambda\, $ sont des nombres entiers, positifs ou nuls, pairs, alors
la
fonction $\exp(-g_{\lambda}(x))$ est une fonction polyn\^{o}me.
Vérifier que
ces conditions impliquent que $\; -(n-1) \leq \lambda \leq n + 1$.

\item Pour $n = 3$ quels sont les réels $\lambda$ qui vérifient les
conditions précédentes ? Pour un entier $n \geq 1$ quelconque,
combien de réels $\lambda$ vérifient ces conditions ?

\item Montrer que si $\lambda$ est une valeur propre de $T_{n}$ et si
$P_\lambda$ est un vecteur propre associé, alors la fonction $h$,
définie
sur $]-1,1[$ par $h(x) = P_\lambda(x) \exp(g_\lambda(x))$, a une
dérivée
nulle. Que vaut alors $P_\lambda$ ?

\item Déterminer les valeurs propres de $T_{n}$ et une base de vecteurs
propres (on pourra distinguer les cas $n$ pair et $n$ impair).
\end{noliste}
\end{noliste}

\section*{EXERCICE II}

On effectue une suite de lancers d'une pièce de monnaie. On suppose que
les résultats des lancers sont indépendants et que, à chaque lancer, la
pièce
donne face avec la probabilité $p$ $\,(0<p<1\,)$ et pile avec la
probabilité 
$\,q = 1-p$.\\
L'objet de l'exercice est l'étude du nombre de lancers nécessaires pour
obtenir deux faces de suite, c'est à dire lors de deux lancers
consécutifs.\\
On suppose donné un espace probabilisé, muni d'une probabilité $P$,
modélisant cette expérience.\\
Pour tout entier $n\geq 1$, on note

\begin{noliste}{$\sbullet$}
\item $U_{n}$ l'événement~ : \textsl{on obtient 2 faces de suite, pour
la
première fois, aux lancers numéro $n$ et $n + 1$,}
\end{noliste}

et on pose $u_{n} = P\left(\Ev{U_{n}}\right)$.

Pour tout entier $n \geq 2$, on note

\begin{noliste}{$\sbullet$}
\item $A_{n}$ l'événement : \textsl{les $n$ premiers lancers ne donnent
pas deux faces de suite et le $n$-ième lancer donne face},

\item $B_{n}$ l'événement : \textsl{les $n$ premiers lancers ne donnent
pas deux faces de suite et le $n$-ième lancer donne pile,}
\end{noliste}

et on pose $\; x_{n} = P\left(\Ev{A_{n}}\right)$,\quad $y_{n} =
P\left(\Ev{B_{n}}\right)$.

\begin{noliste}{1.}
 \setlength{\itemsep}{4mm}
\item 

\begin{noliste}{a)}
 \setlength{\itemsep}{2mm}
\item Déterminer $u_{1}\,$ ; $\,x_{2}$, $\,y_{2}$, $\,u_{2}\,$ ;
$\;x_{3}$, $\,y_{3}$, $\,u_{3}$.

\item Trouver, pour $n\geq 2$, une relation simple entre $x_{n}$ et
$u_{n}$.

\item Pour tout $n\geq 2$ déterminer les probabilités conditionnelles

$\,P\left(\Ev{A_{n + 1}\,/\,A_{n}}\right),\quad P\left(\Ev{A_{n +
1}\,/\,B_{n}}\right),\quad
P\left(\Ev{B_{n + 1}\,/\,A_{n}}\right),\quad P\left(\Ev{B_{n +
1}\,/\,B_{n}}\right).$

\item En déduire, pour tout $n\geq 2$, les relations de récurrence
suivantes~ : $\left\{ 
\begin{tabular}{rcl}
$x_{n + 1}$ & $ = $ & $py_{n}$ \\
$y_{n + 1}$ & $ = $ & $q(x_{n} + y_{n})$\end{tabular}\ \right. $
\end{noliste}

\item On suppose, dans cette question, que $p = q = \dfrac{1}{2}\cdot $

\begin{noliste}{a)}
 \setlength{\itemsep}{2mm}
\item Soit $(f_{n})_{n\geq 0}$ la suite de nombres entiers définie par
les conditions :

$\,f_{0} = 1,\,f_{1} = 1\,\,$et pour tout entier $n\geq 
0,\,\,f_{n + 2}\, = \,f_{n + 1}\, + \,f_{n}\,.$

Montrer que, pour tout $n\geq 2$, on a $2^{n}y_{n} = f_{n}$.

\item On pose $\;\alpha = \dfrac{1 + \sqrt{5}}{2}\;$ et $\;\beta =
\dfrac{1-\sqrt{5}}{2}\cdot $ Montrer que l'on a, pour tout entier
$n\geq 0,\quad
f_{n} = \dfrac{\alpha ^{n + 1}-\beta ^{n + 1}}{\alpha -\beta }\,\cdot $

\item En déduire, pour tout entier $n\geq 2$, une expression de
$\,x_{n}
$, puis de $\,u_{n}$, en fonction de $n,\alpha \,$et $\beta $.

\item Vérifier que $\Sum{n = 1}{\infty }u_{n} = 1$, c'est à dire que la
probabilité d'obtenir deux faces de suite au bout d'un nombre fini de
lancers est égale à 1.
\end{noliste}

\item On considère maintenant le cas où $p = \dfrac{2}{3}.$\\
Donner, pour tout entier $n\geq 1$, une expression de $\,u_{n}$ en
fonction de $n$.
\end{noliste}

\label{fin}

\end{document}

