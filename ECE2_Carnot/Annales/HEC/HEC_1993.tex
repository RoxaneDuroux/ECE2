\documentclass[11pt]{article}%
\usepackage{geometry}%
\geometry{a4paper,
 lmargin = 2cm,rmargin = 2cm,tmargin = 2.5cm,bmargin = 2.5cm}

\input{../../../../../../macros.tex}

\pagestyle{fancy} %
\lhead{ECE2 \hfill septembre 2017 \\
 Mathématiques\\[.2cm]} %
\chead{\hrule} %
\rhead{} %
\lfoot{} %
\cfoot{} %
\rfoot{\thepage} %

\renewcommand{\headrulewidth}{0pt}% : Trace un trait de séparation
 % de largeur 0,4 point. Mettre 0pt
 % pour supprimer le trait.

\renewcommand{\footrulewidth}{0.4pt}% : Trace un trait de séparation
 % de largeur 0,4 point. Mettre 0pt
 % pour supprimer le trait.

\setlength{\headheight}{14pt}

\title{\bf \vspace{-1cm} HEC 1993} %
\author{} %
\date{} %

\begin{document}

\maketitle %
\vspace{-1.2cm}\hrule %
\thispagestyle{fancy}

\vspace*{.4cm}

% DEBUT DU DOC À MODIFIER : tout virer jusqu'au début de l'exo

%Définition et changement de valeurs de
compteurs%newcounter{cpt1}{section} compteur cpt1 remis à 0 à chaque
aumentation par stepcounter du compteur section%setcounter{cpt1}{3} on
met le compteur à 3%addtocounter{cpt1}{5} on ajoute 5 au compteur%
stepcounter{cpt1} on ajoute 1% ifthenelse{test}{alors}{sinon} (page
206) pour subordonner à une condition % whiledo{test}{commande} pour
faire une boucle (page 206 aussi) % value{cpt1} pour noter dans le
document la valeur de cpt1 
%Définition définitive d'opérateurs
mathématiques\newcommand{\ch}{\operatorname{ch}} 
\newcommand{\sh}{\operatorname{sh}}
\renewcommand{\tanh}{\operatorname{th}}
\renewcommand{\sinh}{\operatorname{sh}}
\renewcommand{\cosh}{\operatorname{ch}}
\newcommand{\argsh}{\operatorname{argsh}}
\newcommand{\argch}{\operatorname{argch}}
\newcommand{\argth}{\operatorname{argth}}
\newcommand{\Id}{\operatorname{Id}}
\renewcommand{\leq}{\leq}
\renewcommand{\geq}{\geq }

\newcommand{\dlim}{\lim}
\newcommand{\dsum}{\sum}
\newcommand{\dprod}{\prod}



%Définition de nouvelles couleurs : rgb(trois paramètres red green blue
entre 0 et 1); cmyk (quatre cyan magenta yellow black) entre 0 et 1;
gray (entre 0 et 1) et black, white, red, green, blue, cyan, magenta,
yellow% definecolor{0gris}{gray}{0.8} 
% Nouvelle commande pour encadrer le titre car shabox ne veut que d'une
seule ligne; ATTENTION A LA TAILLE; petite différence avec shadowbox ou
doublebox, voire fcolorbox ou colorbox (au lieu de shabox; laisser le
parbox tranquille sauf pour la taille de la boîte
\newcommand{\Tbox}[1]{\begin{center} \shabox{\parbox{0.6
\linewidth}{#1}} \end{center}} %[1] pour 1 paramètre ; #1 pour ce que
fait le 1er paramètre; entre accolades ce que fait la commande
%Mise en page en mode fancy : en-têtes et pieds de pages puis
définition des en-têtes et pieds de pages\pagestyle{fancy}
\lhead{ECE 2 - Mathématiques \\
Quentin Dunstetter - ENC-Bessières 2011$\backslash$2012}
\chead{}
\rhead{HEC 1993}
\rfoot[ \ \thepage]{\thepage}
\cfoot{}
\lfoot{}
\thispagestyle{fancy} %Mise en page de la 1ère page en mode fancy
%Trait en bas et en haut de la page (entre en-tête et texte et texte et
pied de page)\renewcommand{\footrulewidth}{0.4pt}
\renewcommand{\headrulewidth}{0.4pt}

\begin{center}
{\huge HEC Eco 1993}
\end{center}

\section*{EXERCIC\E\ I}

On note $E$ l'espace vectoriel des fonctions numériques définies et de
classe $C^{2}$ sur $\R$. Soit $F$ l'ensemble des éléments $f$ de $E$
tels que :
\[
f^{\prime \prime }-3f^{\prime } + 2f = 0
\]
Soit $F_{0}$ l'ensemble des éléments de $F$ vérifiant en outre la
relation $f(0) = f^{\prime }(0) = 0$.

\begin{noliste}{1.}
 \setlength{\itemsep}{4mm}
\item Montrer que $F$ et $F_{0}$ sont des sous-espaces vectoriels de
$E$.

\item Soit $f_{1}$ et $f_{2}$ les fonctions définies sur $\R$ par
les relations :
\[
f_{1}(x) = e^{x}\qquad \text{et}\qquad f_{2}(x) = e^{2x}
\]
Montrer que $f_{1}$ et $f_{2}$ sont des éléments de $F$ linéairement
indépendants.

\item Soit $f$ un élément de $F$.

\begin{noliste}{a)}
 \setlength{\itemsep}{2mm}
\item Montrer qu'il existe un couple $(a_{1},a_{2})$ et un seul de
nombres réels tel que $f-a_{1}f_{1}-a_{2}f_{2}$ appartienne à $F_{0}$.

\item Soit $g_{1}$ et $g_{1}$ les fonctions définies sur $\R$ par
les relations :
\[
g_{1}(x) = e^{-x}(f^{\prime }(x)-2f(x))\qquad \text{et}\qquad
g_{2}(x) = e^{-2x}(f^{\prime }(x)-f(x)).
\]
Montrer que ces fonctions sont constantes.

\item En appliquant le résultat précédent, montrer que si f appartient
à $F_{0}$, alors $f = 0$.
\end{noliste}

\item Soit $\Phi $ l'application de $F$ dans $\R^{2}$ définie par la
relation :
\[
\Phi (f) = (f(0),f^{\prime }(0)).
\]
Montrer que $\Phi $ est un isomorphisme de l'espace vectoriel $F$ sur
$\R^{2}$.
\end{noliste}

\section*{EXERCICE 2}

On considère la fonction f définie sur l'intervalle $[0, + \infty
\lbrack $
par $f(x) = \sqrt{1 + x}$ et la fonction $g$ définie sur $]1, + \infty
\lbrack $
par $g(x) = \dfrac{1}{x-1}$.\\
L'objet de l'exercice est l'étude de la suite $(u_{n})_{n\geq 0}$ de
nombres réels déterminée par la condition initiale $u_{0} = 1$ et par
la
relation de récurrence, valables pour tout nombre entier naturel n :

\[
\left\{ 
\begin{array}{lll}
u_{2n + 1} & = & f(u_{2n}) \\
u_{2n + 2} & = & g(u_{2n + 1})
\end{array}
\right. 
\]

\begin{noliste}{1.}
 \setlength{\itemsep}{4mm}
\item Justifier l'existence de $(u_{n})_{n\geq 0}$ et montrer que pour
tout nombre entier naturel $n$, $u_{2n + 1}>1$ et $u_{2n + 2}>0$.

\item Montrer que les équations $f(x) = x$ et $g(x) = x$ ont une
solution
commune et une seule. On note $\alpha $ cette solution.

\item Montrer que la fonction $\varphi = f\circ g$ est définie et
décroissante sur $]1, + \infty \lbrack $.\\
Montrer que la fonction $\psi = \varphi \circ \varphi $ est définie et
croissante sur $]1, + \infty \lbrack $.Calculer $\psi (\alpha )$.

\item Montrer que, pour tout élément $x$ de $]1,\alpha \lbrack $ :
\[
\sqrt{\dfrac{\sqrt{\dfrac{x}{x-1}}}{\sqrt{\dfrac{x}{x-1}-1}}}>x
\]

\item Pour tout nombre entier naturel $n$, on pose $v_{n} = u_{4n +
1}$.

\begin{noliste}{a)}
 \setlength{\itemsep}{2mm}
\item Montrer que $v_{0}<\alpha $.

\item Montrer que, pour tout nombre entier naturel $n$, $v_{n + 1} =
\psi (v_{n})
$.

\item À l'aide du résultat de la question 4, montrer que la suite
$(v_{n})_{n\geq 0}$ est croissante et que, pour tout nombre entier
naturel $n$, 
\[
1<v_{n}\leq \alpha.
\]

\item En déduire que la suite $(v_{n})_{n\geq 0}$ est convergente et
trouver sa limite.
\end{noliste}

\item On considère les trois suites $(u_{4n + 2})_{n\geq 0}$, $(u_{4n +
3})_{n\geq 0}$ et $(u_{4n + 4})_{n\geq 0}$. Montrer que ces
suites s'expriment simplement à l'aide de la suite $(v_{n})_{n\geq 0}$.
En déduire qu'elles sont convergentes et trouver leurs limites.

\item Montrer que la suite $(u_{n})_{n\geq 0}$ est convergente et
trouver sa limite.
\end{noliste}

\section*{EXERCICE 3}

L'objet de cet exercice est l'étude d'un exemple de discrétisation
d'une
variable aléatoire.\\
Pour tout nombre réel $x$, on note $[x]$ sa partie entière,
c'est-à-dire le
plus grand des nombres entiers inférieurs ou égaux à $x.$ On note
$\{x\}$ la
partie décimale de $x$, définie par :
\[
\{x\} = x-[x]
\]

\subsection*{A.}

\begin{noliste}{1.}
 \setlength{\itemsep}{4mm}
\item Représenter graphiquement les fonctions définies sur $\R$ par 
\[
x\mapsto \lbrack x],\quad x\mapsto \{x\},\quad x\mapsto \lbrack
2x]\quad 
\text{et}\quad x\mapsto \{2x\}
\]

\item Montrer que, pour tout nombre réel $x$ :$[2x] = [x] + [x +
\dfrac{1}{2}]$
\end{noliste}

\subsection*{B. }

Soit $X$ une variable aléatoire positive admettant pour densité une
fonction 
$f$ continue sur $[0, + \infty \lbrack $. On note $F$ la fonction de
répartition de $X$

\begin{noliste}{1.}
 \setlength{\itemsep}{4mm}
\item Pour tout nombre entier naturel non nul $n$, exprimer à l'aide de
$f$
la densité de la variable aléatoire $nX$.

\item Pour tout nombre entier naturel $k$, exprimer à l'aide de $f$ ou
de $F$
la probabilité que $[X] = k$ (c'est-à-dire que la partie entière de $X$
soit égale à $k$).

\item Pour tout nombre entier naturel non nul $n$, exprimer à l'aide de
$f$
ou de $F$ la loi de la variable aléatoire à valeurs entières $[nX]$.
\end{noliste}

\subsection*{C. }

Soit $\lambda $ un nombre réel strictement positif. On suppose
maintenant
que la fonction f est définie par les relations :
\[
f(x) = \left\{ 
\begin{array}{lll}
\lambda e^{-\lambda x} & \text{si} & x\geq 0 \\
0 & \text{si} & x<0
\end{array}
\right. 
\]

\begin{noliste}{1.}
 \setlength{\itemsep}{4mm}
\item Déterminer la fonction de répartition et l'espérance de $X$.

\item Déterminer la loi de la variable aléatoire $[X]$. De quelle loi
s'agit-il ? Calculer l'espérance de $[X]$.

\item Pour tout nombre entier naturel non nul $n$, on considère la
variable
aléatoire réelle discrète $Y_{n} = \dfrac{1}{n}[nX]$\\
Trouver l'espérance $m_{n}$ de $Y_{n}$. Montrer que la suite
$(m_{n})_{n\geq 1}$ est convergente et trouver sa limite.

\item Déterminer la fonction de répartition de la variable aléatoire
réelle $\{X\}$. En déduire la densité de $[X]$ et calculer son
espérance.
\end{noliste}

\label{fin}

\end{document}

