\documentclass[11pt]{article}%
\usepackage{geometry}%
\geometry{a4paper,
 lmargin = 2cm,rmargin = 2cm,tmargin = 2.5cm,bmargin = 2.5cm}

\input{../../../../../../macros.tex}

\pagestyle{fancy} %
\lhead{ECE2 \hfill septembre 2017 \\
 Mathématiques\\[.2cm]} %
\chead{\hrule} %
\rhead{} %
\lfoot{} %
\cfoot{} %
\rfoot{\thepage} %

\renewcommand{\headrulewidth}{0pt}% : Trace un trait de séparation
 % de largeur 0,4 point. Mettre 0pt
 % pour supprimer le trait.

\renewcommand{\footrulewidth}{0.4pt}% : Trace un trait de séparation
 % de largeur 0,4 point. Mettre 0pt
 % pour supprimer le trait.

\setlength{\headheight}{14pt}

\title{\bf \vspace{-1cm} HEC 1996} %
\author{} %
\date{} %

\begin{document}

\maketitle %
\vspace{-1.2cm}\hrule %
\thispagestyle{fancy}

\vspace*{.4cm}

% DEBUT DU DOC À MODIFIER : tout virer jusqu'au début de l'exo

%Définition et changement de valeurs de
compteurs%newcounter{cpt1}{section} compteur cpt1 remis à 0 à chaque
aumentation par stepcounter du compteur section%setcounter{cpt1}{3} on
met le compteur à 3%addtocounter{cpt1}{5} on ajoute 5 au compteur%
stepcounter{cpt1} on ajoute 1% ifthenelse{test}{alors}{sinon} (page
206) pour subordonner à une condition % whiledo{test}{commande} pour
faire une boucle (page 206 aussi) % value{cpt1} pour noter dans le
document la valeur de cpt1 
%Définition définitive d'opérateurs
mathématiques\newcommand{\ch}{\operatorname{ch}} 
\newcommand{\sh}{\operatorname{sh}}
\renewcommand{\tanh}{\operatorname{th}}
\renewcommand{\sinh}{\operatorname{sh}}
\renewcommand{\cosh}{\operatorname{ch}}
\newcommand{\argsh}{\operatorname{argsh}}
\newcommand{\argch}{\operatorname{argch}}
\newcommand{\argth}{\operatorname{argth}}
\newcommand{\Id}{\operatorname{Id}}
\renewcommand{\leq}{\leq}
\renewcommand{\geq}{\geq }

\newcommand{\dlim}{\lim}
\newcommand{\dsum}{\sum}
\newcommand{\dprod}{\prod}



%Définition de nouvelles couleurs : rgb(trois paramètres red green blue
entre 0 et 1); cmyk (quatre cyan magenta yellow black) entre 0 et 1;
gray (entre 0 et 1) et black, white, red, green, blue, cyan, magenta,
yellow% definecolor{0gris}{gray}{0.8} 
% Nouvelle commande pour encadrer le titre car shabox ne veut que d'une
seule ligne; ATTENTION A LA TAILLE; petite différence avec shadowbox ou
doublebox, voire fcolorbox ou colorbox (au lieu de shabox; laisser le
parbox tranquille sauf pour la taille de la boîte
\newcommand{\Tbox}[1]{\begin{center} \shabox{\parbox{0.6
\linewidth}{#1}} \end{center}} %[1] pour 1 paramètre ; #1 pour ce que
fait le 1er paramètre; entre accolades ce que fait la commande
%Mise en page en mode fancy : en-têtes et pieds de pages puis
définition des en-têtes et pieds de pages\pagestyle{fancy}
\lhead{ECE 2 - Mathématiques \\
Quentin Dunstetter - ENC-Bessières 2011$\backslash$2012}
\chead{}
\rhead{HEC 1996}
\rfoot[ \ \thepage]{\thepage}
\cfoot{}
\lfoot{}
\thispagestyle{fancy} %Mise en page de la 1ère page en mode fancy
%Trait en bas et en haut de la page (entre en-tête et texte et texte et
pied de page)\renewcommand{\footrulewidth}{0.4pt}
\renewcommand{\headrulewidth}{0.4pt}

\begin{center}
{\huge HEC Eco 1996}
\end{center}
\section{Exercice}

 Cet exercice a pour objet l'étude d'un espace vectoriel de
matrices.\par
 On note $\M{3}$ l'ensemble des matrices carrées d'ordre 3 à
coefficients réels et on considère les matrices :\par
 \begin{displaymath}
 \mathbf{I} = 
 \left(
\begin{array}{ccc}
	1 & 0 & 0\\
	0 & 1 & 0\\
	0 & 0 & 1\\
\end{array}
\right)
\hspace{1.5cm}
 \mathbf{J} = 
 \left(
\begin{array}{ccc}
	0 & 1 & 0\\
	1 & 0 & 1\\
	0 & 1 & 0\\
\end{array}
\right)
\hspace{1.5cm}
 \mathbf{K} = 
 \left(
\begin{array}{ccc}
	0 & 0 & 1\\
	0 & 1 & 0\\
	1 & 0 & 0\\
\end{array}
\right)
\end{displaymath}
On note $Id$, $j$, $k$ les endomorphismes de $\R^{3}$ représentés
respectivement par ces matrices dans la base canonique de $\R^{3}$.
%%%%%%%%%%%%%%%%%%%%%%%%\begin{noliste}{1.}
 \setlength{\itemsep}{4mm}
\item
\begin{noliste}{a)}
 \setlength{\itemsep}{2mm}
\item
Déterminer les valeurs propres des matrices $J$ et $K$.
\item
Montrer qu'on peut trouver une base de $\R^{3}$ dans laquelle les
endomorphismes $j$ et $k$ sont tous deux diagonalisés.
\\
N.B. L'utilisation de cette base pourra permettre de simplifier la
résolution de certaines des questions de la suite.
\end{noliste}
\item
\begin{noliste}{a)}
 \setlength{\itemsep}{2mm}
\item
Montrer que $(I,J,K)$ est une famille libre de l'espace $\M{3}$.\\
Dans toute la suite, on note $E$ le sous-espace de $\M{3}$ engendré par
les éléments $I$, $J$ et $K$.
\item
Montrer que si $A$ et $B$ sont deux éléments de $E$, leur produit est
aussi dans $E$.
\item
Etant donnés trois réels $x$, $y$ et $z$, déterminer les valeurs
propres de la matrice 
\[
T = xI + yJ + zK.
\]
 \end{noliste}
 \item
 Montrer qu'il existe trois suites $(a_{n})_{n\geq1}$,
$(b_{n})_{n\geq1}$ et $(c_{n})_{n\geq1}$ de réels telles que
 
\[
 J^{n} = a_{n}I + b_{n}J + c_{n}K
 
\]
 pour tout $n\geq1$. Donner l'expression du terme général de chacune de
ces suites en fonction de $n$.
 \item
 \begin{noliste}{a)}
 \setlength{\itemsep}{2mm}
 \item
 On note $(e_{1}, e_{2}, e_{3})$ la base canonique de l'espace
$\R^{3}$. on définit une application $\Phi$ de $\R^{3}$ dans $E$ en
posant, pour $V = xe_{1} + ye_{2} + ze_{3}$,
 
\[
 \Phi(V) = xI + yJ + zK.
 
\]
 Montrer que $\Phi$ est un isomorphisme de $\R^{3}$ sur $E$.
 \item
 Déterminer l'ensemble $H$ des vecteurs $V$ de $\R^{3}$ tels que
$\Phi(V)$ soit une matrice non inversible. L'ensemble $H$ est-il un
sous-espace vectoriel de $\R^{3}$ ?
 \end{noliste}
\end{noliste}
%%%%%%%%%%%%%%%%%%%%%%%%%%%%%%%%%%%%%%%%%%%%%%%%%%%%%%%%%%%%%%%%%%%%%%%
%%%%%%%%%%%%%%%%%%%%%%%%%%%%%%%%%%%%%%%%%%%%%%%%%%%%%%%%%%%%%%%%%%%%%%%
%%%%%%%%%%%%%%%%%%%%%%%%%%%%%%%%%%%%%%%%%%%%%%%%%%%%%%%%%%%%%%%%%%%%%%%
%%%%%%%%%%%%%%%%%%%%%%%%%%\section{Exercice}
\begin{noliste}{1.}
 \setlength{\itemsep}{4mm}
\item
On désigne par $a$ un paramètre réel.\\
On note $\mathcal{D}_{a}$ l'ensemble des nombres réels $x$ vérifiant
les conditions $x>0$ et $a\sqrt{x}\neq3$ et on pose, pour tout
$x\in\mathcal{D}_{a}$,
\[
f_{a}(x) = \frac{3-a}{3-a\sqrt{x}}.
\]
%%%%%%%%%%%%%%%%%%%%%%%%%%\begin{noliste}{a)}
 \setlength{\itemsep}{2mm}
\item
Montrer que la fonction $x\mapsto f_{a}(x)$ est dérivable sur
$\mathcal{D}_{a}$ et calculer sa dérivée.
\item
Étudier, en discutant suivant les valeurs du paramètre $a$, les
variations de la fonction $f_{a}$. On donnera, dans chacun des cas
$a<0$, $0<a<3$ et $a>3$ le tableau de variations et l'allure de la
courbe représentative de $f_{a}$.
\item
A l'aide des résultats précédents, montrer que, quand $a<0$ et quand
$a>3$, l'équation $f_{a}(x) = x$ admet une racine unique dans
$\mathcal{D}_{a}$.
\end{noliste}
\item
On fixe, dans cette partie, $a = 1$.
\begin{noliste}{a)}
 \setlength{\itemsep}{2mm}
\item
Montrer que l'équation $f_{1}(x) = x$ admet deux racines : la racine 1
et une racine $\lambda_{1} >1$.
\item
Préciser la position relative, dans le plan rapporté à un repère d'axes
$(Ox, Oy)$, de la courbe représentative de la fonction $f_{1}$ et de la
droite d'équation $y = x$.
\item
Montrer que, pour tout réel $\alpha$ vérifiant $0 < \alpha\leq
\lambda_{1}$, la suite $(u_{n})_{n\geq 0}$ définie par 
\[
u_{0} = \alpha \hspace{0.5cm} \textrm{et, pour tout
entier}\hspace{0.2cm} n,\hspace{0.5cm} u_{n + 1} = f_{1}(u_{n})
\]
est bien définie.
\item
Pour chaque valeur de $\alpha$ ($0 < \alpha\leq \lambda_{1}$) montrer
que la suite obtenue est monotone et déterminer sa limite.
\end{noliste}
%%%%%%%%%%%%%%%%%%%%%%%%%%%%%%%%\item
On revient maintenant au cas plus général où $0<a<3$.
\begin{noliste}{a)}
 \setlength{\itemsep}{2mm}
\item
Résoudre l'équation d'inconnue $z$ :
\[
(*) \hspace{0.5cm} az^{3}-3z^{2} + 3-a = 0
\]
\item
Montrer que si $x$ est racine de l'équation $f_{a}(x) = x$, alors
$\sqrt{x}$ est racine de l'équation (*).
\item
En déduire les racines de l'équation $f_{a}(x) = x$. Discuter, suivant
les valeurs de $a$, la position de ces racines l'une par rapport à
l'autre.
\end{noliste}
%%%%%%%%%%%%%%%%%%%%%%%%%%%%%%%%%%%%%%%%%%\end{noliste}
%%%%%%%%%%%%%%%%%%%%%%%%%%%%%%%%%%%%%%%%%%%%%%%%%%%%%%%%%%%%%%%%%%%%%%%
%%%%%%%%%%%%%%%%%%%%%%%%%%%%%%%%%%%%%%%%%%%%%%%%%%%%%%%%%%%%%%%%%%%%%%%
%%%%%%%%%%%%%%%%%%%%%%%%%%%%%%%%%%%%%%%%%%%%%%%%%%%%%%%%%%%%%%%%%%%%%%%
%%%%%%%%%%%%%%%%%%\section{Exercice}
On désigne par $m$ un entier fixé supérieur ou égal à 2.\\
Une urne contient $m$ boules numérotées de 1 à $m$. On note $E$
l'ensemble de ces boules et $\mathcal{P}(E)$ l'ensemble des parties de
$E$.\\
Un dispositif permet d'effectuer le tirage au hasard d'une partie de
ces boules, de telle manière que chacune des parties de $E$
(c'est-à-dire chacun des éléments de $\mathcal{P}(E)$, y compris la
partie vide ou l'ensemble de toutes les boules) ait la même probabilité
d'être tirée.
%%%%%%%%%%%%%%\begin{noliste}{1.}
 \setlength{\itemsep}{4mm}
\item
On effectue un tirage.
\begin{noliste}{a)}
 \setlength{\itemsep}{2mm}
\item
Quelle est la probabilité que la boule portant le numéro 1 appartienne
à l'ensemble de boules tirées ?
\item
Pour tout entier $i$ vérifiant $1\leq i \leq m$ on note $A_{i}$
l'évènement : `` la boule portant le numéro $i$ appartient à l'ensemble
de boules tirées ''. Les évènements $A_{i}$ sont-ils indépendants ?
\item
Quelle est l'espérance de la variable aléatoire égale au nombre de
boules qui ont été tirées ? Quelle est sa variance ?
\item
La probabilité de tirer un nombre pair de boules est-elle supérieure à
la probabilité d'en tirer un nombre impair ?
\end{noliste}
%%%%%%%%%%%%%%%%%%%%%%%%%%%%%%%%%%%%%%%%%%%%%%%%%%%%%%%%%%%%%%%%%%%%%%%
%\item
On effectue maintenant une suite de tirages de la forme précédente, en
remettant dans l'urne l'ensemble des boules tirées, après chaque
tirage.
\begin{noliste}{a)}
 \setlength{\itemsep}{2mm}
\item
Déterminer, pour tout entier $k\geq 1$, la probabilité que la boule
numéro $i$ soit tirée pour la première fois au $k^\textrm{ième}$
tirage.
\item
On note $T_{i}$ la variable aléatoire qui prend la valeur $k$ ($k$
entier $\geq 1$) si la boule numéro $i$ est tirée pour la première fois
au $k^\textrm{ième}$ tirage. Déterminer l'espérance de $T_{i}$.
\item
On admet, sans que la justification en soit demandée, que les variables
aléatoires $T_{1}$, $T_{2}$,..., $T_{m}$ sont indépendantes. On note
$T$ le nombre minimum de tirages qu'il faut effectuer pour que chacune
des $m$ boules ait été tirée au moins une fois. Déterminer, pour tout
entier $k\geq 1$, la probabilité que $T$ soit inférieure ou égale à
$k$. En déduire la loi de probabilité de la variable aléatoire $T$.
\end{noliste}
%%%%%%%%%%%%%%%%%%%%%%%%\item
On effectue maintenant une suite de tirages, sans remettre dans l'urne,
après chaque tirage, les boules tirées. Chaque tirage consiste encore à
prendre au hasard une partie des boules qui restent dans l'urne,
chacune des parties de l'ensemble des boules restantes ayant la même
probabilité d'être tirée.
\begin{noliste}{a)}
 \setlength{\itemsep}{2mm}
\item
Calculer la probabilité pour que les $m$ boules soient toutes tirées en
au plus deux tirages. Calculer la probabilité pour que les $m$ boules
soient toutes tirées en exactement deux tirages.
\item
Pour tout $k\geq 1$ déterminer plus généralement la probabilité pour
que les $m$ boules soient toutes tirées en au plus $k$ tirages (On
pourra raisonner par récurrence).
\end{noliste}
%%%%%%%%%%%%%%%%%%%%%%%%%%%%%%%%%%%%%%%%%%%%%\end{noliste}
%%%%%%%%%%%%%%%%%%%%%%%%%%%%%%%%%%%%%%%%%%%%%%%%%%%%%%%%%%%%%%%%\end{document}