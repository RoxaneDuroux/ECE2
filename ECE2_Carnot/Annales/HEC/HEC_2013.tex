\documentclass[11pt]{article}%
\usepackage{geometry}%
\geometry{a4paper,
 lmargin = 2cm,rmargin = 2cm,tmargin = 2.5cm,bmargin = 2.5cm}

\input{../../../../../../macros.tex}

\pagestyle{fancy} %
\lhead{ECE2 \hfill septembre 2017 \\
 Mathématiques\\[.2cm]} %
\chead{\hrule} %
\rhead{} %
\lfoot{} %
\cfoot{} %
\rfoot{\thepage} %

\renewcommand{\headrulewidth}{0pt}% : Trace un trait de séparation
 % de largeur 0,4 point. Mettre 0pt
 % pour supprimer le trait.

\renewcommand{\footrulewidth}{0.4pt}% : Trace un trait de séparation
 % de largeur 0,4 point. Mettre 0pt
 % pour supprimer le trait.

\setlength{\headheight}{14pt}

\title{\bf \vspace{-1cm} HEC 2013} %
\author{} %
\date{} %

\begin{document}

\maketitle %
\vspace{-1.2cm}\hrule %
\thispagestyle{fancy}

\vspace*{.4cm}

% DEBUT DU DOC À MODIFIER : tout virer jusqu'au début de l'exo

%Définition et changement de valeurs de
compteurs%newcounter{cpt1}{section} compteur cpt1 remis à 0 à chaque
aumentation par stepcounter du compteur section%setcounter{cpt1}{3} on
met le compteur à 3%addtocounter{cpt1}{5} on ajoute 5 au compteur%
stepcounter{cpt1} on ajoute 1% ifthenelse{test}{alors}{sinon} (page
206) pour subordonner à une condition % whiledo{test}{commande} pour
faire une boucle (page 206 aussi) % value{cpt1} pour noter dans le
document la valeur de cpt1 
%Définition définitive d'opérateurs
mathématiques\newcommand{\ch}{\operatorname{ch}} 
\newcommand{\sh}{\operatorname{sh}}
\renewcommand{\tanh}{\operatorname{th}}
\renewcommand{\sinh}{\operatorname{sh}}
\renewcommand{\cosh}{\operatorname{ch}}
\newcommand{\argsh}{\operatorname{argsh}}
\newcommand{\argch}{\operatorname{argch}}
\newcommand{\argth}{\operatorname{argth}}
\newcommand{\Id}{\operatorname{Id}}
\newcommand{\id}{\operatorname{id}}
\renewcommand{\im}{\operatorname{Im}}
\renewcommand{\leq}{\leq}
\renewcommand{\geq}{\geq }
\newcommand{\lb}{\llbracket}
\newcommand{\rb}{\rrbracket}

\newcommand{\dlim}{\lim}
\newcommand{\dsum}{\sum}
\newcommand{\dprod}{\prod}



%Définition de nouvelles couleurs : rgb(trois paramètres red green blue
entre 0 et 1); cmyk (quatre cyan magenta yellow black) entre 0 et 1;
gray (entre 0 et 1) et black, white, red, green, blue, cyan, magenta,
yellow% definecolor{0gris}{gray}{0.8} 
% Nouvelle commande pour encadrer le titre car shabox ne veut que d'une
seule ligne; ATTENTION A LA TAILLE; petite différence avec shadowbox ou
doublebox, voire fcolorbox ou colorbox (au lieu de shabox; laisser le
parbox tranquille sauf pour la taille de la boîte
\newcommand{\Tbox}[1]{\begin{center} \shabox{\parbox{0.6
\linewidth}{#1}} \end{center}} %[1] pour 1 paramètre ; #1 pour ce que
fait le 1er paramètre; entre accolades ce que fait la commande
%Mise en page en mode fancy : en-têtes et pieds de pages puis
définition des en-têtes et pieds de pages\pagestyle{fancy}
\lhead{ECE 2 - Mathématiques \\
Quentin Dunstetter - ENC-Bessières 2011$\backslash$2012}
\chead{}
\rhead{HEC 2013}
\rfoot[ \ \thepage]{\thepage}
\cfoot{}
\lfoot{}
\thispagestyle{fancy} %Mise en page de la 1ère page en mode fancy
%Trait en bas et en haut de la page (entre en-tête et texte et texte et
pied de page)\renewcommand{\footrulewidth}{0.4pt}
\renewcommand{\headrulewidth}{0.4pt}

\indent \vspace{0.2cm}

\Tbox{ \begin{center} \textbf{\Huge HEC 2013} \end{center} }

\vspace{0.5cm}

\section*{EXERCICE}

\noindent On note $E = \R_{3} [X] = $ l'espace vectoriel des polynômes
à coefficients réels de degré inférieur ou égal à 3. \\
Soit $f$ l'application définie sur $E$ qui associe à tout polynôme $P
\in E$, le polynôme $f(P)$ défini par : 
\[
 f(P) (X) = - 3 X P\left(\Ev{X}\right) + X^{2} P'\left(\Ev{X}\right),
\text{ où } P' \text{ est la dérivée du polynôme } P. 
\]

\begin{noliste}{1.}
 \setlength{\itemsep}{4mm}

\item \begin{noliste}{a)}
 \setlength{\itemsep}{2mm}

\item Rappeler la dimension de $E$. \\

\item Montrer que $f$ est un endomorphisme de $E$. \\

\item Déterminer la matrice $M$ de $f$ dans la base canonique de $E$.
\\

\item La matrice $M$ est-elle inversible ? Est-elle diagonalisable ?
Calculer pour tout $n \in \N^{\ast}$, $M^{n}$. \\

\item Préciser le noyau $\kerf$ de $f$ ainsi qu'une base de $\kerf$. \\

\item Déterminer l'image $\imf$ de $f$. \\

\end{noliste}

\item On note $\id_{E}$ et $0_{E}$ respectivement, l'endomorphisme
identité et l'endomorphisme nul de $E$, et pour tout endomorphisme $v$
de $E$, on pose $v^{0} = \id_{E}$ et pour tout $k$ de $\N^{\ast}$,
$v^{k} = v \circ v^{k-1}$. \\
Soit $u$ et $g$ deux endomorphismes de $E$ tels que : $u^{4} = 0_{E}, \
u^{3} \neq 0_{E}$ et $g = \id_{E} + u + u^{2} + u^{3}$.

\begin{noliste}{a)}
 \setlength{\itemsep}{2mm}

\item Soit $P$ un polynôme de $E$ tel que $P \notin \ker(u^{3})$.
Montrer que la famille $( P, u(P), u^{2}(P), u^{3}(P) )$ est un base de
$E$. \\

\item Montrer que $g$ est un automorphisme de $E$. Déterminer
l'automorphisme réciproque $g^{-1}$ en fonction de $u$. \\

\item Établir l'égalité : $\keru = \ker(g-\id_{E})$. \\

\item Montrer que 1 est la seule valeur propre de $g$. \\

\end{noliste}

\end{noliste}

\section*{PROBL\`{E}ME}

\begin{noliste}{$\sbullet$}

\item \textit{Le problème aborde d'une part, l'analyse mathématique de
l'évolution du prix de vente d'un bien sous différents modes
d'anticipation d'agents économiques et d'autre part, la mise en
évidence de certaines propriétés de la fonction de profit d'une
entreprise. }

\item \textit{On note $\E(X)$ et $\V(X)$ respectivement, l'espérance et
la variance d'une variable aléatoire $X$ définie sur un espace
probabilisé $\Omega, \mathcal{A}, \Pr)$.}

\item \textit{Les quatre parties du problème sont très largement
indépendantes. Les questions 10 et 11 dont appel aux résultats de la
partie III.} \\

\end{noliste}

\subsection*{Partie I. Prix d'équilibre}

\noindent Sur le marché d'un certain bien, on note $D$ la fonction de
demande globale (des consommateurs), $O$ la fonction d'offre globale
(des entreprises) et $p$ le prix de vente du bien. \\
On suppose habituellement que la fonction $D : p \mapsto D(p)$ définie
sur $\R_+ $ à valeurs réelles est décroissante et que la fonction $O :
p \mapsto O(p)$ définie sur $\R_+ $ à valeurs réelles est croissante.
\\
Si l'équation $O(p) = D(p)$ admet une solution $p^{\ast}$, on dit que
$p^{\ast}$ \textit{est un prix d'équilibre du marché}. \\
Avant d'atteindre un niveau d'équilibre, le prix $p$ peut être soumis à
des fluctuations provoquées par des excès d'offre $\big(O(p) >
D(p)\big)$ ou des excès de demande $\big(D(p) > O(p)\big)$ au cours du
temps. \\
Afin de rendre compte de cette évolution, on note pour tout $n \in \N$,
$p_{n}$ la valeur du prix à l'instant $n$. \\
On suppose que la demande dépend de la valeur du prix selon la relation
$D_{n} = D(p_{n})$ valable pour tout $n \in \N$. \\
Quant aux entreprises, elles adaptent à chaque instant $n \in \N$, la
quantité offerte $O_{n}$ à l'instant $n$ à un \textit{prix anticipé à
l'instant} $(n-1)$, noté $\widehat{p_{n}}$, selon la relation $O_{n} =
O(\widehat{p_{n}}$, où $\widehat{p_{0}}$ peut être interprété comme un
prix d'étude du marché. \\
On suppose qu'à chaque instant, l'offre est égale à la demande,
c'est-à-dire : pour tout $n \in \N$, $O_{n} = D_{n}$. \\

\textit{Dans toute cette partie, on considère quatre paramètres réels
strictement positifs} $a,\ b, \ c$ \textit{et} $d$, \textit{avec} $a >
d$, \textit{et on suppose que les fonctions} $D$ \textit{et} $O$
\textit{sont définies sur} $\R_+ $ \textit{par} : $D(p) = a - b p$
\textit{et} $O(p) = c p + d$. \\
\textit{Par suite, on a pour tout} $n \in \N$, $D(p_{n}) = a - b p_{n}$
\textit{et} $O(\widehat{p_{n}}) = c \widehat{p_{n}} + d$.

\begin{noliste}{1.}
 \setlength{\itemsep}{4mm}

\item Dans cette question \textit{uniquement}, les réels $a,\ b,\ c$ et
$d$ ont les valeurs suivantes : $a = 40,\ b = 8,\ c = 2$ et $d = 20$.
\\
On suppose que $p_{0}$ et $p_{1}$ sont donnés et que pour tout entier
$n \geq 2$, on a $\widehat{p_{n}} = 2 p_{n-1} - p_{n-2}$.

\begin{noliste}{a)}
 \setlength{\itemsep}{2mm}

\item Établir l'existence et l'unicité d'un prix d'équilibre
$p^{\ast}$. Calculer $p^{\ast}$. \\

\item Montrer que pour tout $n \geq 2$, on a : $p_{n} = - \frac{1}{2}
p_{n-1} + \frac{1}{4} p_{n-2} + \frac{5}{2}$. \\

\item Écrire une fonction \Scilab{} récursive, d'en-tête
\texttt{function p (p0,p1 :real;n :integer) :real;} qui renvoie, pour
$p_{0}$ et $p_{1}$ fixés, le terme $p_{n}$. \\

\item On pose pour tout $n \in \N$ : $v_{n} = p_{n} - p^{\ast}$.
Montrer que $(v_{n})_{n \in \N}$ est une suite récurrente linéaire
d'ordre 2. \\

\item Calculer les solutions $r_{1}$ et $r_{2}$ de l'équation
caractéristique de la suite $(v_{n})_{n \in \N}$. \\

\item Exprimer pour tout $n \in \N$, $p_{n}$ en fonction de $n,\
r_{1},\ r_{2},\ p_{0},\ p_{1}$ et $p^{\ast}$. \\

\item Montrer que la suite $(p_{n})_{n \in \N}$ est convergente. Quelle
est sa limite ? Interpréter. \\

\end{noliste}

\item Soit $\beta$ un paramètre réel vérifiant $0 < \beta \leq 1$. On
suppose que le prix $p_{0}$ est donné et que les anticipations du prix
sont \textit{adaptatives}, c'est-à-dire que pour tout entier $n \geq
1$, on a : \\
\[
\widehat{p_{n}} = \widehat{p_{n-1} } + \beta ( p_{n-1} -
\widehat{p_{n-1} } ).
\]

\begin{noliste}{a)}
 \setlength{\itemsep}{2mm}

\item Exprimer pour tout $n \in \N$, le prix courant $p_{n}$ en
fonction du prix anticipé $\widehat{p_{n}}$. \\

\item En déduire que pour tout $n \in \N^{\ast}$, le prix $p_{n}$
vérifie l'équation de récurrence suivante : 
\[
p_{n} = \left( 1 - \beta \frac{b + c}{b} \right) p_{n-1} + \beta
\frac{a-d}{b}. 
\]

\item Quel est le prix d'équilibre $p^{\ast}$ ? Déterminer l'expression
de $p_{n}$ en fonction de $n,\ p_{0},\ p^{\ast},\ b,\ c$ et $\beta$. \\

\item En supposant que $p_{0} \neq p^{\ast}$, montrer que la suite
$(p_{n})_{n \in \N}$ converge si et seulement si : $\frac{c}{b} <
\frac{2}{\beta} - 1$. \\
Quelle est alors sa limite ? \\

\item Étudier la convergence de la suite $(p_{n})_{n \in \N}$ lorsque
$c < b$. \\

\end{noliste}

\end{noliste}

\subsection*{Partie II. Convexité du profit et prix aléatoire}

\begin{noliste}{1.}
 \setlength{\itemsep}{4mm}

\item Soit $p$ un paramètre réel positif ou nul et $h_{p}$ la fonction
définie sur $\R_+ $ à valeurs dans $\R$ donnée par : 
\[
h_{p}(x) = p x - \frac{x^{3}}{3} 
\]

\begin{noliste}{a)}
 \setlength{\itemsep}{2mm}

\item Dresser le tableau de variation de $h_{p}$ sur $\R_+ $. Préciser
les limites aux bornes de l'intervalle de définition, les racines de
l'équation $h_{p}(x) = 0$ et la valeur maximale de $h_{p}$ sur $\R_+ $.
\\

\item Tracer la courbe représentative de $h_{1}$ dans le plan rapporté
à un repère orthonormé. 

\end{noliste}

\end{noliste}

\noindent On considère une entreprise présente sur le marché d'un bien
qui adapte son volume de production $x \in \R_+ $ à un niveau de prix
$p \in \R_+ $ donné (par l'équilibre du marché) ou administré (par
l'État). \\
On modélise \textit{le coût total de l'entreprise} par une fonction $F$
définie et de classe $C^{2}$ sur $\R_+ $, strictement croissante sur
$\R_+ $ ainsi que sa dérivée $F'$, telle que $F(0) = F'(0) = 0$ et
$F(x)$ équivalent à $s x^{r}$ avec $s > 0$ et $r > 1$, lorsque $x$ tend
vers $ + \infty$. On note $F''$ la dérivée seconde de $F$ et on suppose
que pour tout $x \in \R_+ $, $F''(x) > 0$. \\
Soit $\Pi_{p}$ la fonction défini sur $\R_+ $ à valeurs réelles telle
que : $\Pi_{p}(x) = p x - F (x)$.

\begin{noliste}{1.}
 \setlength{\itemsep}{4mm}

\item \begin{noliste}{a)}
 \setlength{\itemsep}{2mm}

\item Montrer que $\dlim{x \rightarrow + \infty} F'(x) = + \infty$ et
que $F'$ admet sur $\R_+ $ une fonction réciproque, que l'on note $S$,
dont on précisera l'ensemble de définition (\textit{la fonction} $S$
\textit{est la fonction d'offre de l'entreprise}). \\

\item Montrer que $\Pi_{p}$ est concave sur $\R_+ $ et admet sur $\R_+
$ un maximum atteint en un seul point. \\

\end{noliste}

\item Soit $M$ la fonction définie sur $\R_+ $ à valeurs réelles telle
que : $M(p) = \max\limits_{x \in \R_+} \Pi_{p}(x)$ (\textit{la
fonction} $M$ \textit{est la fonction de profit de l'entreprise}).

\begin{noliste}{a)}
 \setlength{\itemsep}{2mm}

\item Pour tout $p \in \R_+ $, exprimer $M(p)$ à l'aide de $p$, $F$ et
$S$. \\

\item Montrer que la fonction $M$ est dérivable sur $\R_+ $ et calculer
sa dérivée $M'$. \\

\item Montrer que la fonction $M$ est convexe et croissante sur $\R_+
$. \\

\end{noliste}

\item On suppose qie le prix $p$ est une variable aléatoire discrète
définie sur un espace probabilisé $(\Omega, \mathcal{A}, \Pr)$, à
valeurs dans l'ensemble $\{p^{(1)}, p^{(2)}, \dots, p^{(k)} \} \subset
\R_+ $, où $k$ est un entier fixé supérieur ou égal à 2.

\begin{noliste}{a)}
 \setlength{\itemsep}{2mm}

\item Montrer que pour tout $i \in \lb 1 ; k \rb$ et pour tout $y \in
\R_+ $, on a : $M \big( p^{(i)} \big) - M(y) \geq M'(y) \big( p^{(i)} -
y \big)$. \\

\item En déduire pour tout $y \in \R_+ $, l'inégalité : $\E\big( M(p )
\big) \geq M(y) + M'(y) \big( E(p) - y \big)$. \\

\item Établir l'inégalité : $\E\big( M(p) \big) \geq M \big( E(p)
\big)$. Quelle conclusion peut-on en tirer ? \\

\end{noliste}

\item On suppose que le prix $p$ est une variable aléatoire à densité
définie sur un espace probabilisé $(\Omega, \mathcal{A}, \Pr)$, à
valeurs dans $\R_+ $, dont une densité $f$ est nulle sur $\R_-$ et
continue sur $\R_+ $. On suppose l'existence de l'intégrale $
\dint{0}{+ \infty} M(x) f(x)\ dx$. Justifier que $p$ admet une
espérance et montrer que : 
\[
\E\big( M(p) \big) \geq M \big( E(p) \big)
\]

\end{noliste}

\newpage

\subsection*{Partie III. Espérance conditionnelle}

\noindent Soit $X$ et $Y$ deux variables aléatoires discrètes définies
sur un espace probabilisé $(\Omega, \mathcal{A}, \Pr)$, à valeurs dans
$\{ x_{1}, x_{2}, \dots, x_{q} \} \subset \R$ et $\{ y_{1}, y_{2},
\dots y_{r} \} \subset \R$, respectivement ($q \geq 2$ et $r \geq 2$).
\\
On suppose que pour tout $i \in \lb 1 ; q \rb$, on a : $P\left(\Ev{X =
x_{i}]}\right) > 0$. \\
Soit $\varphi$ la fonction définie sur $\{ x_{1} ; x_{2} ; \dots x_{q}
\}$ à valeurs réelles, telle que : 
\[
 \forall i \in \lb 1 ; q \rb, \ \ \varphi (x_{i}) = \Sum{j = 1}{r}
y_{j} P_{[X = x_{i}]} ( [Y = y_{j}] ). 
\]
Ainsi, pour tout $i \in \lb 1 ; q \rb$, $\varphi(x_{i})$ est
\textit{l'espérance conditionnelle de} $Y$ \textit{sachant l'évènement}
$[X = x_{i}]$, notée également $\E(Y | [X = x_{i}])$. On définit alors
une variable aléatoire $Z$ sur $\Omega$ en posant pour tout $\omega \in
\Omega$, $Z(\omega) = E( Y | [X = X(\omega)] )$ et on note $Z = E (Y |
X) = \varphi (X)$.

\begin{noliste}{1.}
 \setlength{\itemsep}{4mm}

\item \begin{noliste}{a)}
 \setlength{\itemsep}{2mm}

\item On suppose que $X$ et $Y$ sont indépendantes. Déterminer la
variable aléatoire $\E(Y | X)$. \\

\item Quelle est la variable aléatoire $\E(X | X)$ ? \\

\item On suppose que les réels $\varphi (x_{1}), \varphi(x_{2}), \dots
\varphi(x_{q})$ sont deux à deux distincts. \\
Déterminer pour tout $i \in \lb 1 ; q \rb,\ \Prob\left(\Ev{E(Y | X) =
\varphi(x_{i}) ] }\right)\left(\Ev{Y | X}\right) = \varphi(x_{i}) ] )$.
\\

\item Montrer que $\E\big( E(Y | X) \big) = E(Y)$ (on pourra appliquer
le théorème du transfert). \\

\item Soit les réels $\lambda,\ \rho$ et $\mu$. Exprimer $\E( \lambda Y
+ \rho X + \mu | X)$ en fonction de $\lambda,\ \rho,\ \mu,\ X$ et $\E(Y
| X)$.

\end{noliste}

\end{noliste}

\subsection*{Partie IV. Anticipation naïve et anticipation rationnelle}

\noindent Dans cette partie, on suppose qu'à chaque instant $n$ ($n \in
\N$), le prix $p_{n}$ d'un certain bien est une variable aléatoire
discrète définie sur un espace probabilisé $(\Omega, \mathcal{A},
\Pr)$, à valeurs dans $\{ p^{(1)}, p^{(2)}, \dots, p^{(k)} \} \subset
\R_+ $, où $k$ est un entier fixé supérieur ou égal à 2. On suppose que
la suite $(p_{n})_{n \in \N}$ est constituée de variables aléatoires de
même loi, et que pour tout $n \in \N$ et pour tout $i \in \lb 1 ; k
\rb$, on a : $\Prob\left(\Ev{p_{n} = p^{(i)}]}\right) > 0$. \\
Soit $(u_{n})_{n \in \N}$ une suite de variables aléatoires discrètes
finies indépendantes et de même loi, telles que pour tout $n \in \N$,
on a $\E(u_{n}) = 0$ et $\V(u_{n}) = \sigma^{2} > 0$. \\
On suppose que pour tout $n \in \N^{\ast}$, les variables aléatoires
$u_{n}$ et $p_{n-1}$ sont indépendantes. \\
Soit $\theta$ et $p^{\ast}$ deux paramètres réels vérifiant $-1 <
\theta < 1$ et $p^{\ast} \geq 0$. \\
On suppose que $p_{0}$ est de la forme $p_{0} = l u_{0} + m$, où $l$ et
$m$ sont des constantes réelles, et que pour tout $n \in \N^{\ast}$, on
a : $p_{n} = \theta p_{n-1} + ( 1 - \theta) p^{\ast} + u_{n}. \ \ \ \
(1)$

\begin{noliste}{1.}
 \setlength{\itemsep}{4mm}

\item \begin{noliste}{a)}
 \setlength{\itemsep}{2mm}

\item Calculer pour tout $n \in \N^{\ast}$, $\E(p_{n})$ et $\V(p_{n})$.
Déterminer les constantes $l$ et $m$ en fonction de $\theta$ et
$p^{\ast}$. \\

\item Calculer pour tout $n \in \N^{\ast}$, la covariance $\Cov (p_{n},
p_{n-1} )$. \\
Que représente le paramètre $\theta$ pour le couple de variables
aléatoires $(p_{n}, p_{n-1})$ ? \\

\end{noliste}

\item Déterminer la variable aléatoire $\E(p_{n} | p_{n-1})$. \\

\item On rappelle que l'on note $\widehat{p_{n}}$ l'anticipation de
$p_{n}$ faite à l'instant $(n-1)$. \\
Pour tout $n \in \N^{\ast}$, on pose : $e_{n} = p_{n} -
\widehat{p_{n}}$ (\textit{erreur d'anticipation à l'instant} $n$).

\begin{noliste}{a)}
 \setlength{\itemsep}{2mm}

\item On suppose dans cette question que les anticipations sont naïves,
c'est-à-dire que pour tout $n \in \N^{\ast}$, on a : $\widehat{p_{n}} =
p_{n-1}$. \\
Déterminer $\E(\widehat{p_{n}} | p_{n-1})$. Calculer
$\E(\widehat{p_{n}})$, $\V(\widehat{p_{n}})$, $\E(e_{n})$ et
$\V(e_{n})$. \\

\item On suppose dans cette question que les anticipations de prix sont
\textit{rationnelles}, ce qui se traduit dans le cadre du modèle (1)
par : $\widehat{p_{n}} = E(p_{n} | p_{n-1})$. \\
Déterminer $\E(\widehat{p_{n}} | p_{n-1})$. Calculer
$\E(\widehat{p_{n}})$, $\V(\widehat{p_{n}})$, $\E(e_{n})$ et
$\V(e_{n})$. \\

\item Comparer les deux types d'anticipation \textit{naïve} et
\textit{rationnelle}.

\end{noliste}

\end{noliste}

\end{document}