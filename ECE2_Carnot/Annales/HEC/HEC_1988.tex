\documentclass[11pt]{article}%
\usepackage{geometry}%
\geometry{a4paper,
 lmargin = 2cm,rmargin = 2cm,tmargin = 2.5cm,bmargin = 2.5cm}

\input{../../../../../../macros.tex}

\pagestyle{fancy} %
\lhead{ECE2 \hfill septembre 2017 \\
 Mathématiques\\[.2cm]} %
\chead{\hrule} %
\rhead{} %
\lfoot{} %
\cfoot{} %
\rfoot{\thepage} %

\renewcommand{\headrulewidth}{0pt}% : Trace un trait de séparation
 % de largeur 0,4 point. Mettre 0pt
 % pour supprimer le trait.

\renewcommand{\footrulewidth}{0.4pt}% : Trace un trait de séparation
 % de largeur 0,4 point. Mettre 0pt
 % pour supprimer le trait.

\setlength{\headheight}{14pt}

\title{\bf \vspace{-1cm} HEC 1988} %
\author{} %
\date{} %

\begin{document}

\maketitle %
\vspace{-1.2cm}\hrule %
\thispagestyle{fancy}

\vspace*{.4cm}

% DEBUT DU DOC À MODIFIER : tout virer jusqu'au début de l'exo

%Définition et changement de valeurs de
compteurs%newcounter{cpt1}{section} compteur cpt1 remis à 0 à chaque
aumentation par stepcounter du compteur section%setcounter{cpt1}{3} on
met le compteur à 3%addtocounter{cpt1}{5} on ajoute 5 au compteur%
stepcounter{cpt1} on ajoute 1% ifthenelse{test}{alors}{sinon} (page
206) pour subordonner à une condition % whiledo{test}{commande} pour
faire une boucle (page 206 aussi) % value{cpt1} pour noter dans le
document la valeur de cpt1 
%Définition définitive d'opérateurs
mathématiques\newcommand{\ch}{\operatorname{ch}} 
\newcommand{\sh}{\operatorname{sh}}
\renewcommand{\tanh}{\operatorname{th}}
\renewcommand{\sinh}{\operatorname{sh}}
\renewcommand{\cosh}{\operatorname{ch}}
\newcommand{\argsh}{\operatorname{argsh}}
\newcommand{\argch}{\operatorname{argch}}
\newcommand{\argth}{\operatorname{argth}}
\newcommand{\Id}{\operatorname{Id}}
\renewcommand{\leq}{\leq}
\renewcommand{\geq}{\geq }

\newcommand{\dlim}{\lim}
\newcommand{\dsum}{\sum}
\newcommand{\dprod}{\prod}



%Définition de nouvelles couleurs : rgb(trois paramètres red green blue
entre 0 et 1); cmyk (quatre cyan magenta yellow black) entre 0 et 1;
gray (entre 0 et 1) et black, white, red, green, blue, cyan, magenta,
yellow% definecolor{0gris}{gray}{0.8} 
% Nouvelle commande pour encadrer le titre car shabox ne veut que d'une
seule ligne; ATTENTION A LA TAILLE; petite différence avec shadowbox ou
doublebox, voire fcolorbox ou colorbox (au lieu de shabox; laisser le
parbox tranquille sauf pour la taille de la boîte
\newcommand{\Tbox}[1]{\begin{center} \shabox{\parbox{0.6
\linewidth}{#1}} \end{center}} %[1] pour 1 paramètre ; #1 pour ce que
fait le 1er paramètre; entre accolades ce que fait la commande
%Mise en page en mode fancy : en-têtes et pieds de pages puis
définition des en-têtes et pieds de pages\pagestyle{fancy}
\lhead{ECE 2 - Mathématiques \\
Quentin Dunstetter - ENC-Bessières 2011$\backslash$2012}
\chead{}
\rhead{HEC 1988}
\rfoot[ \ \thepage]{\thepage}
\cfoot{}
\lfoot{}
\thispagestyle{fancy} %Mise en page de la 1ère page en mode fancy
%Trait en bas et en haut de la page (entre en-tête et texte et texte et
pied de page)\renewcommand{\footrulewidth}{0.4pt}
\renewcommand{\headrulewidth}{0.4pt}

\begin{center}
{\huge HEC Eco 1988}
\end{center}

\section*{EXERCIC\E\ 1}

Pour tout nombre entier non nul $n,$ on pose : $I_{n} =
\dint{0}{1}\dfrac{x^{n}}{1 + x^{n}}dx$

\begin{noliste}{1.}
 \setlength{\itemsep}{4mm}
\item À l'aide d'un encadrement convenable de $I_{n},$ déterminer la
limite
de la suite $(I_{n}).$

\item Montrer que $\dlim{n\rightarrow + \infty }\dint{0}{1}\dfrac{dx}{1
+ x^{n}} = 1.$

\item Pour tout nombre entier naturel non nul $n,$ on pose : $J_{n} =
nI_{n}.$

\begin{noliste}{a)}
 \setlength{\itemsep}{2mm}
\item Montrer que : $J_{n} = \ln 2-\dint{0}{1}\ln (1 + x^{n})dx.$

\item Montrer que, pour tout nombre réel positif $t$ : $0\leq \ln
(1 + t)\leq t.$\\
En déduire la limite de la suite $(J_{n}).$
\end{noliste}
\end{noliste}

\section*{EXERCIC\E\ 2}

\begin{noliste}{1.}
 \setlength{\itemsep}{4mm}
\item Montrer que, pour tout nombre réel $t$ : $-1\leq \dfrac{2t}{1 +
t^{2}}\leq 1.$

\item On désigne désormais par $f$ une fonction dérivable d'une
variable réelle telle que 
\[
f(x + y) = \dfrac{f(x) + f(y)}{1 + f(x)f(y)}\qquad (1)
\]

\begin{noliste}{a)}
 \setlength{\itemsep}{2mm}
\item Montrer que, pour tout nombre réel $x$ : $-1\leq f(x)\leq 1.$

\item Trouver toutes les fonctions constantes $f$ vérifiant la relation
(1).\\
Dans la suite de l'exercice, on suppose que $f$ n'est pas constante.

\item Montrer que, pour tout nombre réel $x$ : $-1\leq f(x)\leq 1.$

\item Calculer $f(0).$ On pose $a = f^{\prime }(0).$

\item En utilisant la définition de la dérivée, exprimer $f^{\prime
}(x)$ en
fonction de $f(x)$ et de $a.$

\item Montrer que $f$ définit une bijection de $\R$ sur l'intervalle 
$]-1,1[$ et que la fonction $f^{-1}$ est dérivable sur cet
intervalle.\\
Calculer la dérivée de $f^{-1}.$

\item Expliciter $f^{-1}(y)$ en fonction de $y$ et de $a.$ Trouver
enfin
toutes les fonctions $f$ non constantes, dérivables et vérifiant la
relation 
$(1).$
\end{noliste}
\end{noliste}

\section*{EXERCICE 3}

Soit $X$ une variable aléatoire discrète définie sur un espace
probabilisé $(\Omega,\mathcal{A},P)$ dont la variance $\sigma ^{2}$
existe et est non
nulle. On suppose que $X$ est symétrique, c'est-à-dire que $X$ et $-X$
ont
la même loi de probabilité.

\begin{noliste}{1.}
 \setlength{\itemsep}{4mm}
\item Calculer l'espérance de $X.$

\item Soit $Z$ une variable aléatoire définie sur
$(\Omega,\mathcal{A},P),$
indépendante de $X,$ de loi de probabilité :
\[
P\left(\Ev{Z = 1}\right) = p\qquad P\left(\Ev{Z = -1}\right) = 1-p
\]
où $p$ est un nombre réel tel que $0<p<1.$ On pourra poser $q = 1-p.$

\begin{noliste}{a)}
 \setlength{\itemsep}{2mm}
\item Montrer que la variable aléatoire $ZX$ a la même loi que $X.$

\item Calculer la covariance des variables aléatoires $X$ et $ZX;$
montrer
que le coefficient de corrélation de $X$ et $ZX$ ne dépend pas de
$\sigma
^{2}.$
\end{noliste}

\item On définit les variables aléatoires $U,V$ et $Y$ par les
conditions 
\[
U = \left\{ 
\begin{array}{cc}
1 & \text{si }X\geq 0 \\
0 & \text{si }X<0
\end{array}
\right. \qquad V = \left\{ 
\begin{array}{cc}
1 & \text{si }X\leq 0 \\
0 & \text{si }X>0
\end{array}
\right. \qquad Y = U-V
\]

\begin{noliste}{a)}
 \setlength{\itemsep}{2mm}
\item Montrer que la variable aléatoire $Y$ est symétrique.

\item On note $\left| X\right| $ la valeur absolue de $X.$ Comparer $X$
et $Y\left| X\right|.$ En déduire la valeur du coefficient de
corrélation de $Y$ et $\left| X\right|.$

\item Montrer que les variables aléatoires $Y$ et $\left| X\right| $
sont indépendantes si et seulement si $P\left(\Ev{X = 0}\right) = 0$
\end{noliste}
\end{noliste}

\label{fin}

\end{document}

