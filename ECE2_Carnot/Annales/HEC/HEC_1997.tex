\documentclass[11pt]{article}%
\usepackage{geometry}%
\geometry{a4paper,
 lmargin = 2cm,rmargin = 2cm,tmargin = 2.5cm,bmargin = 2.5cm}

\input{../../../../../../macros.tex}

\pagestyle{fancy} %
\lhead{ECE2 \hfill septembre 2017 \\
 Mathématiques\\[.2cm]} %
\chead{\hrule} %
\rhead{} %
\lfoot{} %
\cfoot{} %
\rfoot{\thepage} %

\renewcommand{\headrulewidth}{0pt}% : Trace un trait de séparation
 % de largeur 0,4 point. Mettre 0pt
 % pour supprimer le trait.

\renewcommand{\footrulewidth}{0.4pt}% : Trace un trait de séparation
 % de largeur 0,4 point. Mettre 0pt
 % pour supprimer le trait.

\setlength{\headheight}{14pt}

\title{\bf \vspace{-1cm} HEC 1997} %
\author{} %
\date{} %

\begin{document}

\maketitle %
\vspace{-1.2cm}\hrule %
\thispagestyle{fancy}

\vspace*{.4cm}

% DEBUT DU DOC À MODIFIER : tout virer jusqu'au début de l'exo

%Définition et changement de valeurs de
compteurs%newcounter{cpt1}{section} compteur cpt1 remis à 0 à chaque
aumentation par stepcounter du compteur section%setcounter{cpt1}{3} on
met le compteur à 3%addtocounter{cpt1}{5} on ajoute 5 au compteur%
stepcounter{cpt1} on ajoute 1% ifthenelse{test}{alors}{sinon} (page
206) pour subordonner à une condition % whiledo{test}{commande} pour
faire une boucle (page 206 aussi) % value{cpt1} pour noter dans le
document la valeur de cpt1 
%Définition définitive d'opérateurs
mathématiques\newcommand{\ch}{\operatorname{ch}} 
\newcommand{\sh}{\operatorname{sh}}
\renewcommand{\tanh}{\operatorname{th}}
\renewcommand{\sinh}{\operatorname{sh}}
\renewcommand{\cosh}{\operatorname{ch}}
\newcommand{\argsh}{\operatorname{argsh}}
\newcommand{\argch}{\operatorname{argch}}
\newcommand{\argth}{\operatorname{argth}}
\newcommand{\Id}{\operatorname{Id}}
\renewcommand{\leq}{\leq}
\renewcommand{\geq}{\geq }

\newcommand{\dlim}{\lim}
\newcommand{\dsum}{\sum}
\newcommand{\dprod}{\prod}



%Définition de nouvelles couleurs : rgb(trois paramètres red green blue
entre 0 et 1); cmyk (quatre cyan magenta yellow black) entre 0 et 1;
gray (entre 0 et 1) et black, white, red, green, blue, cyan, magenta,
yellow% definecolor{0gris}{gray}{0.8} 
% Nouvelle commande pour encadrer le titre car shabox ne veut que d'une
seule ligne; ATTENTION A LA TAILLE; petite différence avec shadowbox ou
doublebox, voire fcolorbox ou colorbox (au lieu de shabox; laisser le
parbox tranquille sauf pour la taille de la boîte
\newcommand{\Tbox}[1]{\begin{center} \shabox{\parbox{0.6
\linewidth}{#1}} \end{center}} %[1] pour 1 paramètre ; #1 pour ce que
fait le 1er paramètre; entre accolades ce que fait la commande
%Mise en page en mode fancy : en-têtes et pieds de pages puis
définition des en-têtes et pieds de pages\pagestyle{fancy}
\lhead{ECE 2 - Mathématiques \\
Quentin Dunstetter - ENC-Bessières 2011$\backslash$2012}
\chead{}
\rhead{HEC 1997}
\rfoot[ \ \thepage]{\thepage}
\cfoot{}
\lfoot{}
\thispagestyle{fancy} %Mise en page de la 1ère page en mode fancy
%Trait en bas et en haut de la page (entre en-tête et texte et texte et
pied de page)\renewcommand{\footrulewidth}{0.4pt}
\renewcommand{\headrulewidth}{0.4pt}

\begin{center}
{\huge HEC Eco 1997}
\end{center}

\section*{EXERCIC\E\ I}

\begin{noliste}{1.}
 \setlength{\itemsep}{4mm}
\item Soit $f$ l'endomorphisme de $\R^{3}$ dont la matrice dans la
base canonique $(e_{1},e_{2},e_{3})$
\[
A = \left( 
\begin{array}{ccc}
4 & -5 & 5 \\
3 & -4 & 5 \\
0 & 0 & 1
\end{array}
\right) 
\]

\begin{noliste}{a)}
 \setlength{\itemsep}{2mm}
\item Montrer que $-1$ est une valeur propre de $A$ et trouver un
vecteur
propre, noté $V$, associé à cette valeur propre.

\item Montrer que $(V,e_{2},e_{3})$ est une base de $\R^{3}$. Donner
la matrice de $f$ dans cette base.

\item Montrer que si $X$ est un vecteur quelconque de $\R^{3}$,
alors $f(X)$ est une combinaison linéaire de $X$ et de $V$. En déduire
que,
dans toute base de $\R^{3}$ de la forme $(V,V_{2},V_{3})$ où $V_{2}$
et $V_{3}$ sont des vecteurs de $\R^{3}$, la matrice de $f$ est la
forme :
\[
\left( 
\begin{array}{ccc}
-1 & a & b \\
0 & 1 & 0 \\
0 & 0 & 1
\end{array}
\right) 
\]
où $a$ et $b$ sont des réels.
\end{noliste}

\item L'endomorphisme $f$ est-il diagonalisable ?

\item Plus généralement, soit $U_{1}$ un vecteur non nul de $\R^{3}$
et h un endomorphisme de $\R^{3}$ ayant la propriété suivante :\\
$(P) :$ Pour tout vecteur X de $\R^{3}$, $h(X)$ est une combinaison
linéaire de $X$ et de $U_{1}$.\\
Soit $(U_{1},U_{2},U_{3})$ une base de $\R^{3}$ obtenue en
adjoignant à $U_{1}$ deux autres vecteurs $U_{2}$ et $U_{3}$.

\begin{noliste}{a)}
 \setlength{\itemsep}{2mm}
\item En appliquant la propriété $(P)$ à trois vecteurs particuliers,
montrer qu'il existe des réels $\alpha $, $\beta $, $\gamma $ tels que
:
\[
h(U_{2}) = \alpha U_{1} + \gamma U_{2}\quad \text{et}\quad h(U_{3}) =
\beta
U_{1} + \gamma U_{3}.
\]

\item En déduire que la matrice de h dans la base $(U_{1},U_{2},U_{3})$
est
de la forme
\[
\left( 
\begin{array}{ccc}
\lambda & \alpha & \beta \\
0 & \gamma & 0 \\
0 & 0 & \gamma 
\end{array}
\right) 
\]
où $\alpha $, $\beta $, $\gamma $ et $\lambda $ sont des réels.

\item L'endomorphisme $h$ est-il diagonalisable ?
\end{noliste}
\end{noliste}

\section*{EXERCICE II}

\begin{noliste}{1.}
 \setlength{\itemsep}{4mm}
\item Montrer que, si $g$ est une fonction réelle définie sur$[0,1]$,
continue et positive sur cet intervalle, la fonction $h$, définie par
$h(x) = 2\dint{0}{x}\sqrt{g(t)}dt$, est aussi une fonction continue et
positive sur $[0,1].$

\item Compte tenu du résultat précédent, on définit sur $[0,1]$ une
suite de
fonctions réelles $(f_{n})_{n\in \N}$ en posant, pour tout $x\in
\lbrack 0,1]$, 
\[
f_{0}(x) = 1,\quad \text{et pour tout }n\geq 1,\quad
f_{n}(x) = 2\dint{0}{x}\sqrt{f_{n-1}(t)}dt
\]
Calculer $f_{1}$, $f_{2}$ et $f_{3}$.

\item Montrer qu'il existe deux suites de réels $(a_{n})_{n\in \N}$
et $(b_{n})_{n\in \N}$ telles que, pour tout $n\geq 0$ et tout $x\in
\lbrack 0,1]$, on ait : $f_{n}(x) = a_{n}x^{b_{n}}.$\\
On calculera $b_{n}$ en fonction de $n$ et on écrira une relation de
récurrence donnant $a_{n}$ en fonction de $a_{n-1}$.

\item Démontrer que, pour tout $n\geq 1$, $2^{n}\ln
(a_{n}) = -\Sum{k = 1}{n}2^{k}\ln (1-2^{-k}).$

\item 

\begin{noliste}{a)}
 \setlength{\itemsep}{2mm}
\item Montrer que, pour tout$x\in \lbrack 0,1[$, on a : $0\leq -\ln
(1-x)-x\leq \dfrac{1}{2}\times \dfrac{x^{2}}{1-x}$

\item En déduire que, pour tout $k\geq 1$, $\left| -2^{k}\ln
(1-2^{-k})-1\right| \leq 2^{-k}$

\item Montrer que $\ln (a_{n})$ est équivalent à
$\dfrac{n}{2^{n}}$quand $n$
tend vers l'infini.
\end{noliste}

\item Montrer que, pour tout réel $x\in \lbrack 0,1]$, la suite
$(f_{n}(x))_{n\in \N}$ converge et préciser sa limite en fonction de
$x$.
\end{noliste}

\section*{EXERCICE III}

\noindent Un pion est déplacé de manière aléatoire sur un damier de
quatre
cases numérotées de 1 à 4. On considère une suite de variables
aléatoires, $(X_{n})_{n\geq 0}$, définies sur un espace probabilisé
$(\Omega,A,P)$, 
à valeurs dans l'ensemble $\{1,2,3,4\}$, représentant la position du
pion
aux instants successifs : pour $i\in \{1,2,3,4\}$, on a $X_{n} = i$ si
le pion
est sur la case $i$ à l'instant $n$.\\
On note $\pi 
\[
_{ij}$ l'élément situé à la i-ème ligne et à la j-ième
colonne de la matrice

\[
\Pi = \left( 
\begin{array}{cccc}
1/4 & 1/4 & 1/4 & 1/4 \\
1/3 & 1/3 & 1/3 & 0 \\
1/3 & 1/3 & 1/3 & 0 \\
0 & 0 & 0 & 1
\end{array}
\right) 
\]
Pour $i$ et $j$ dans $\{1,2,3,4\}$, on suppose que, pour tout $n\geq 0$
tel que $P\left(\Ev{X_{n} = i}\right)\neq 0$, la probabilité
conditionnelle $P\left(\Ev{X_{n + 1} = j/X_{n} = i}\right)$
(probabilité que le pion soit sur la case $j$ à
l'instant $n + 1$ sachant qu'il est sur la case $i$ à l'instant $n$)
est égale 
à $\pi 
\]
_{ij}$.\\
On note, pour tout $n\geq 0$, 
\[
p_{n} = P\left(\Ev{X_{n} = 1}\right),\quad q_{n} = P\left(\Ev{X_{n} =
2}\right),\quad r_{n} = P\left(\Ev{X_{n} = 3}\right),\quad
s_{n} = P\left(\Ev{X_{n} = 4}\right).
\]
Enfin on suppose que le pion est sur la case $1$ à l'instant $n = 0$ et
donc
que $p_{0} = 1$. Si le pion vient sur la case 4 à l'instant $n>0$, il y
reste à
l'instant $n + 1$, d'où la quatrième ligne de la matrice $\Pi $.

\begin{noliste}{1.}
 \setlength{\itemsep}{4mm}
\item Calculer $p_{1},q_{1},r_{l},s_{1}$ et $p_{2},q_{2},r_{2},s_{2}.$

\item 

\begin{noliste}{a)}
 \setlength{\itemsep}{2mm}
\item Donner, pour tout entier $n\geq 0$, l'expression de $p_{n +
1},q_{n + 1},r_{n + 1},s_{n + 1}$ en fonction de
$p_{n},q_{n},r_{n},s_{n}.$

\item En déduire, pour $n\geq 1$, les expressions de
$p_{n},q_{n},r_{n}$
et $s_{n}$ en fonction de $n$.
\end{noliste}

\item 

\begin{noliste}{a)}
 \setlength{\itemsep}{2mm}
\item Calculer, pour $n\geq 0$, la probabilité conditionnelle
$P\left(\Ev{X_{n + l} = 4/X_{n}\neq 4}\right).$

\item Déterminer, pour tout entier $n\geq 1$, la probabilité que le
pion passe par la case 4, pour la première fois, à l'instant n. On
notera $t_{n}$ cette probabilité.

\item Déterminer la probabilité qu'il ne passe jamais par la case 4.

\item Soit T une variable aléatoire définie sur $(\Omega,A,P)$, à
valeurs
dans $\N^{\times }$ et vérifiant, pour tout entier $n\geq 1$,
$P\left(\Ev{T = n}\right) = t_{n}$. \\
Calculer l'espérance et la variance de $T$.
\end{noliste}

\item Calculer la probabilité que le pion ne passe jamais ni par la
case 2
ni par la case 3.

\item 

\begin{noliste}{a)}
 \setlength{\itemsep}{2mm}
\item Déterminer la probabilité que le pion se soit trouvé sur la case
1 à
l'instant $n = 1$, sachant qu'il est sur la case 4 à l'instant $n = 2$.

\item Plus généralement, soit m et n deux entiers vérifiant $0<m<n$.
Déterminer la probabilité que le pion se soit trouvé sur la case 1 à
l'instant 
$m$, sachant qu'il est sur la case 4 à l'instant $n$.
\end{noliste}
\end{noliste}

\label{fin}

\end{document}

