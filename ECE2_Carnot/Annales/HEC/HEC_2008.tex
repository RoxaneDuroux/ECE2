\documentclass[11pt]{article}%
\usepackage{geometry}%
\geometry{a4paper,
 lmargin = 2cm,rmargin = 2cm,tmargin = 2.5cm,bmargin = 2.5cm}

\input{../../../../../../macros.tex}

\pagestyle{fancy} %
\lhead{ECE2 \hfill septembre 2017 \\
 Mathématiques\\[.2cm]} %
\chead{\hrule} %
\rhead{} %
\lfoot{} %
\cfoot{} %
\rfoot{\thepage} %

\renewcommand{\headrulewidth}{0pt}% : Trace un trait de séparation
 % de largeur 0,4 point. Mettre 0pt
 % pour supprimer le trait.

\renewcommand{\footrulewidth}{0.4pt}% : Trace un trait de séparation
 % de largeur 0,4 point. Mettre 0pt
 % pour supprimer le trait.

\setlength{\headheight}{14pt}

\title{\bf \vspace{-1cm} HEC 2008} %
\author{} %
\date{} %

\begin{document}

\maketitle %
\vspace{-1.2cm}\hrule %
\thispagestyle{fancy}

\vspace*{.4cm}

% DEBUT DU DOC À MODIFIER : tout virer jusqu'au début de l'exo

%Définition et changement de valeurs de
compteurs%newcounter{cpt1}{section} compteur cpt1 remis à 0 à chaque
aumentation par stepcounter du compteur section%setcounter{cpt1}{3} on
met le compteur à 3%addtocounter{cpt1}{5} on ajoute 5 au compteur%
stepcounter{cpt1} on ajoute 1% ifthenelse{test}{alors}{sinon} (page
206) pour subordonner à une condition % whiledo{test}{commande} pour
faire une boucle (page 206 aussi) % value{cpt1} pour noter dans le
document la valeur de cpt1 
%Définition définitive d'opérateurs mathématiques

%Définition de nouvelles couleurs : rgb(trois paramètres red green blue
entre 0 et 1); cmyk (quatre cyan magenta yellow black) entre 0 et 1;
gray (entre 0 et 1) et black, white, red, green, blue, cyan, magenta,
yellow% definecolor{0gris}{gray}{0.8} 
% Nouvelle commande pour encadrer le titre car shabox ne veut que d'une
seule ligne; ATTENTION A LA TAILLE; petite différence avec shadowbox ou
doublebox, voire fcolorbox ou colorbox (au lieu de shabox; laisser le
parbox tranquille sauf pour la taille de la boîte
\newcommand{\Tbox}[1]{\begin{center} \shabox{\parbox{0.6
\linewidth}{#1}} \end{center}} %[1] pour 1 paramètre ; #1 pour ce que
fait le 1er paramètre; entre accolades ce que fait la commande
%Mise en page en mode fancy : en-têtes et pieds de pages puis
définition des en-têtes et pieds de pages\pagestyle{fancy}
\lhead{ECE 2 - Mathématiques \\
Quentin Dunstetter - ENC-Bessières 2011$\backslash$2012}
\chead{}
\rhead{HEC 2008}
\rfoot[ \ \thepage]{\thepage}
\cfoot{}
\lfoot{}
\thispagestyle{fancy} %Mise en page de la 1ère page en mode fancy
%Trait en bas et en haut de la page (entre en-tête et texte et texte et
pied de page)\renewcommand{\footrulewidth}{0.4pt}
\renewcommand{\headrulewidth}{0.4pt}

\bi\begin{center}\textbf{ \Large HEC 2008 voie E - Mathématiques III}
\end{center}\bi\bi

\n La présentation, la lisibilité, l'orthographe, la qualité de la
rédaction, la clarté et la précision des raisonnements entreront pour
une part importante dans l'appréciation des copies.\\
\n Les candidats sont invités à encadrer dans la mesure du possible les
résultats de leurs calculs.\\
\n Ils ne doivent faire usage d'aucun document : \textbf{l'utilisation
de toute calculatrice et de tout matériel électronique est
interdite.}\\
\n Seule l'utilisation d'une règle graduée est autorisée. \bi\bi

\hrule\bi\bi\bi\bi

%\begin{center}
%\large{\textit{Quelques consignes de présentation :}}\\
%\textit{- rédiger sur des copies doubles en laissant une marge
suffisante}\\
%\textit{- changez de page ou de feuille pour chaque exercice}\\
%\textit{ ou chaque partie d'un problème}\\
%\textit{- inscrivez clairement en titre le numéro de l'exercice,}\\
%\textit{ numérotez les questions traitées}\\
%\textit{- chaque question aura droit à une réponse finale}\\
%\textit{ soulignée ou encadrée en rouge.}
%\end{center}\bi


\n\textbf{\large{\underline{EXERCICE}}}\quad\\

\n Étant donné un entier $n$ supérieur ou égal à $2$, on considère un
nuage de $n$ points du plan, c'est-à-dire un $n$-uplet
$\big((x_{1},y_{1}),(x_{2},y_{2}),\dots,(x_{n},y_{n})\big)$ d'éléments
de $\R^{2}$. On suppose que les réels $x_{1}$, $x_{2}$, $\dots$,
$x_{n}$ (resp. $y_{1}$, $y_{2}$, $\dots$, $y_{n}$) ne sont pas tous
égaux.\bi

\n On appelle moyenne arithmétique $\overline{x}$ et écart-type
$\sigma_{x}$ du $n$-uplet $x = (x_{1},\dots,x_{n})$, les réels suivants
:
\[
 \ \overline{x} = \frac{1}{n}\Sum{k = 1}{n}
x_{k}\quad\mathrm{et}\quad\sigma_{x} = \sqrt{\frac{1}{n}\Sum{k =
1}{n}(x_{k}-\overline{x})^{2}}
\]

\n On définit de même la moyenne arithmétique $\overline{y}$ et
l'écart-type $\sigma_{y}$ du $n$-uplet $y = (y_{1},\dots,y_{n})$.\bi

\n La covariance $\cov(x,y)$ et le coefficient de corrélation linéaire
$r(x,y)$ du couple $(x,y)$ sont donnés par :
\[
 \ \cov(x,y) = \frac{1}{n}\Sum{k =
1}{n}(x_{k}-\overline{x})(y_{k}-\overline{y})\quad\mathrm{et}\quad
r(x,y) = \frac{\cov(x,y)}{\sigma_{x}\times\sigma_{y}}
\]

\n Soit $f$ la fonction définie sur $\R^{2}$ à valeurs réelles qui, à
tout couple $(a,b)$ de $\R^{2}$, associe le réel $f(a,b)$ tel que :
\[
f(a,b) = \Sum{k = 1}{n} (ax_{k} + b-y_{k})^{2}
\]

\en\item Justifier que $f$ est de classe $\mathscr C^{2}$ sur $\R^{2}$.
\item \en\item Écrire le système d'équations $(S)$ permettant de
déterminer les points critiques de $f$.
\item Résoudre le système $(S)$. En déduire que $f$ admet un unique
point critique $(\hat{a},\hat{b})$ que l'on exprimera en fonction de
$\overline{x}$, $\overline{y}$, $\sigma_{x}{2}$ et $\cov(x,y)$.
\item Montrer que ce point critique correspond à un minimum local de
$f$.
\item Établir la formule suivante : $f(\hat{a},\hat{b}) =
n\sigma_{y}{2} \big(1-r^{2}(x,y)\big)$.
\een\item
\en\item Montrer que l'on a : $|r(x,y)|\leq 1$.
\item Que peut-on dire du nuage de points lorsque $|r(x,y)| = 1$
?\een\een
\newpage


\n\textbf{\large{\underline{PROBL\`{E}ME } - Propagation d'un
virus}}\quad\\

\n Dans tout le problème, $N$ désigne un entier naturel fixé 
supérieur ou égal à $2$, et $p$ un réel fixé de
l'intervalle $\left] 0,1\right[ $.\\
\n On pose : $q = 1-p$. Soit $n$ un entier naturel quelconque.

\n Dans une population de $N$ individus, on s'intéresse à la
propagation d'un certain virus. \\
\n Chaque jour, on distingue dans cette population trois catégories
d'individus : en premier lieu, les individus sains, c'est-à-dire ceux
qui ne sont pas porteurs du virus, ensuite les individus qui viennent
d'être contaminés et qui sont inoffensifs pour les autres, et enfin,
les
individus contaminés par le virus et qui sont contagieux.\\

\n Ces trois catégories évoluent jour après jour selon le modèle
suivant :

\begin{noliste}{$\sbullet$}
\item chaque jour $n$, chaque individu sain peut être contaminé par
n'importe lequel des individus contagieux ce jour avec la même
probabilité $p$, ces contaminations éventuelles étant indépendantes
les unes des autres ;

\item un individu contaminé le jour $n$ devient contagieux le jour $n +
1$
;

\item chaque individu contagieux le jour $n$ redevient sain le jour $n
+ 1$.
\end{noliste}

\n On note alors $X_{n}$ le nombre aléatoire d'individus contagieux le
jour 
$n$.\\

\n On remarquera que si, pour un certain entier naturel $i$, on a
$X_{i} = 0$,
alors on a aussi $X_{i + 1} = 0$.\\

\n Les variables aléatoires $X_{0},X_{1},\dots,X_{n}$ sont supposées
définies sur un même espace probabilisé $\left(
\Omega,\mathcal{A},\Prob\right) $, et$\E\left( X_{n}\right) $ désigne,
pour tout $n$
de $\N$, l'espérance de $X_{n}$.\bi

\begin{center}\textbf{Partie I. Un cas particulier}\end{center}


\n \textit{Dans cette partie uniquement, on suppose que l'on a : $N =
3$ et $p = 1/3$.}\\

\n On
considère les matrices $S$ et $R$ suivantes :
\[
S = 
\begin{smatrix}
9 & 4 & 4 & 9 \\
0 & 4 & 5 & 0 \\
0 & 1 & 0 & 0 \\
0 & 0 & 0 & 0
\end{smatrix},\quad R = 
\begin{smatrix}
0 & 1 & -6 & 1 \\
1 & 0 & 5 & 0 \\
-1 & 0 & 1 & 0 \\
0 & -1 & 0 & 0
\end{smatrix}
\]

L'ensemble $\M{4,1} $ des matrices
colonnes à quatre lignes est confondu avec l'espace vectoriel $\R^{4}$.

\begin{noliste}{1.}
 \setlength{\itemsep}{4mm}
\item Montrer que la matrice $R$ est inversible et calculer son inverse
$R^{-1}.$

\item 
\begin{noliste}{a)}
 \setlength{\itemsep}{2mm}
\item Montrer que les réels $-1$, $0$, $5$ et $9$ sont des valeurs
propres de $S.$

\item Calculer le produit matriciel $R^{-1}S\ R$.

\item En déduire, pour tout $n$ de $\N$ l'expression de la
matrice $S^{n}$; (On pose $S^{0} = I$. où $I$ désigne la matrice
identité de $\M{4} $ )
\end{noliste}

\item Soit $n$ un entier fixé de $\N$.

\begin{noliste}{a)}
 \setlength{\itemsep}{2mm}
\item Déterminer la loi de probabilité conditionnelle de $X_{n + 1}$
sachant l'événement $\left[ X_{n} = 0\right].$

\item Déterminer la loi de probabilité conditionnelle de $X_{n + 1}$
sachant l'événement $\left[ X_{n} = 3\right].$

\item Vérifier que la loi de probabilité conditionnelle de $X_{n + 1}$
sachant l'événement $\left[ X_{n} = 1\right].$ (resp $\left[ X_{n} =
2\right] $ ) est la loi binomiale de paramètres $\left(
2,\frac{1}{3}\right) $ (resp \ $\left( 1,\frac{5}{9}\right) $ )

\item On note $\E\left( X_{n + 1}|X_{n} = i\right) $ l'espérance de la
loi de
probabilité conditionnelle de $X_{n + 1}$ sachant l'événement $\left[
X_{n} = i\right] $

Déterminer les valeurs respectives de $\E\left( X_{n + 1}|X_{n} =
1\right) $
et $\E\left( X_{n + 1}|X_{n} = 2\right) $.
\end{noliste}

\item On suppose, \textit{uniquement} dans cette question, que $X_{0}$
suit la loi
binomiale de paramètres $\left( 3,\frac{1}{3}\right) $

\begin{noliste}{a)}
 \setlength{\itemsep}{2mm}
\item Déterminer la loi de $X_{1}$ et calculer $\E\left( X_{1}\right).$

\item Vérifier la formule suivante : $ \E\left( X_{1}\right)
 = \Sum{i = 0}{3}\E\left( X_{1}|X_{0} = i\right)\times \mathrm{P}\left(
X_{0} = i
 \right) $
\end{noliste}

\item Pour tout entier naturel $n$, on considère le vecteur $U_{n}$ de
$\R^{4}$ défini par :
\[
U_{n} = 
\begin{smatrix}
u_{n} \\
v_{n} \\
w_{n} \\
t_{n}\end{smatrix}
 = 
\begin{smatrix}
\Prob\left(\Ev{ X_{n} = 0}\right) \\
\Prob\left(\Ev{ X_{n} = 1}\right) \\
\Prob\left(\Ev{ X_{n} = 2}\right) \\
\Prob\left(\Ev{ X_{n} = 3}\right)\end{smatrix}
\]

\begin{noliste}{a)}
 \setlength{\itemsep}{2mm}
\item Déterminer une relation entre $u_{n},v_{n},w_{n}$ et $t_{n}$.

\item À l'aide de la formule des probabilités totales appliquée au
système complet d'événements $\left( \left[ X_{n} = i\right]
\right)_{0\leq i\leq 3}$ déterminer une matrice $M$ de $\M{4} $
indépendante de $n$, telle que : $U_{n + 1} = M~U_{n}$

\item Exprimer $M$ en fonction de $S$. En déduire les valeurs propres
de 
$M$.

\item Donner l'expression des réels $u_{n}$, et $v_{n}$\ en fonction de
$n$, $v_{0}$ et $w_{0}$.
\end{noliste}

\item On pose : $F = \di\dcup{n = 0}{+ \infty }\left[ X_{n} = 0\right]
$

\begin{noliste}{a)}
 \setlength{\itemsep}{2mm}
\item Que signifie l'événement $F$ ?

\item Montrer que le virus finit par dispara\^{\i}tre presque
s\^{u}rement,
quelle que soit la loi de la variable aléatoire initiale $X_{0}$.
\end{noliste}
\end{noliste}\bi

\begin{center}\textbf{Partie II. Le cas général}\end{center}

\n On suppose que pour tout entier naturel $n$ et pour tout entier $i$
de $\lc 0,N\rc $, on a : $\Prob\left(\Ev{ X_{n} = i}\right)
>0$. On suppose également que pour tout couple $\left( i,j\right) $ de
$\lc 0,N\rc ^{2}$, le réel $q_{i,j}$ défini par : $ q_{i,j} =
\mathrm{P}_{\left[ X_{n} = i\right] }\left( 
X_{n + 1} = j\right) $ est indépendant de $n$.\\

\n Soit $Q$ la matrice de $\M{N + 1} $ définie par : $Q = \left(
q_{i,j}\right)_{0\leq i,j\leq N}$

\begin{noliste}{1.}
 \setlength{\itemsep}{4mm}
\item 
\begin{noliste}{a)}
 \setlength{\itemsep}{2mm}
\item Déterminer, pour tout $j$ de $\lc0,N\rc $
les probabilités $q_{0,j}$ et $q_{N,j}$. De même, déterminer
pour tout $i$ de $\lc 0,N\rc $ la probabilité $q_{i,N}$.

\item Justifier que si l'on a $j>$ $N-i$, alors $q_{i,j} = 0$.

\item Montrer que pour tout $i$ de $\lc 1,N-1\rc,$ la
loi de probabilité conditionnelle de $X_{n + 1}$ sachant $\left[ X_{n}
= i\right] $, est \ une loi binomiale dont on déterminera les
paramètres.
\end{noliste}

\item 
\begin{noliste}{a)}
 \setlength{\itemsep}{2mm}
\item Montrer que $1$ est valeur propre de la matrice $Q$.

\item Soit $\lambda $ une valeur propre de $Q$, et $V = 
\begin{smatrix}
\V\left( 0\right) \\
\V\left( 1\right) \\
\vdots \\
\V\left( N\right)\end{smatrix}
$ un vecteur propre associé à $\lambda.$\\
On pose : $\left| V\left( i\right) \right| = \max \limits_{0\leq
j\leq N}\left| V\left( j\right) \right| $. \\
Justifier que la composante $\V\left( i\right) $ n'est pas nulle, puis,
en
examinant la ligne $i$ du système $QV = \lambda V$, montrer que l'on a
: $\left| \lambda \right| \leq 1$.
\end{noliste}

\item On pose pour tout entier naturel $n$ : $U_{n} = 
\begin{smatrix}
\Prob\left(\Ev{ X_{n} = 0}\right) \\
\Prob\left(\Ev{ X_{n} = 1}\right) \\
\vdots \\
\Prob\left(\Ev{ X_{n} = N}\right)\end{smatrix}
$

\n À l'aide de la formule des probabilités totales, déterminer en
fonction de $Q$ une matrice $M$ de $\M{n + 1} $ indépendante de $n$ et
vérifiant, pour tout $n$ de $\N$, la relation : $U_{n + 1} =
M~U_{n}$.\\

\n\textit{On suppose jusqu'à la fin de la partie II que la matrice $M$
est
diagonalisable, et que $\mathcal{B = }\left( V_{0},\dots,V_{N}\right) $
est
une base de vecteurs propres de $M$ telle que, pour tout $k$ de $\lc
0,N\rc $ le vecteur propre $V_{k}$ est associé à
une valeur propre $\lambda_{k}$.}\\

\n \textit{De plus, on suppose que : $\lambda_{0} = 1$, $V_{0} = 
\begin{smatrix}
1 \\
0 \\
\vdots \\
0
\end{smatrix}
$, et que pour tout $k$ de $\lc 1,N\rc $, on a : $\left|
\lambda_{k}\right| <1$.}

\item On décompose alors le vecteur $U_{0}$ sur la base $\mathcal{B}$ :
$ U_{0} = \Sum{k = 0}{N}\alpha_{k}V_{k}$

\begin{noliste}{a)}
 \setlength{\itemsep}{2mm}
\item Déterminer, pour tout $n$ de $\N$, la décomposition du
vecteur $U_{n}$\ sur la base $\mathcal{B}.$

\item On note, pour tout couple $\left( k,i\right) $ de $\lc 0,N\rc
^{2}$, $V_{k}\left( i\right) $ la $\left( i + 1\right) $-ième
composante du vecteur $V_{k}$.\\
Exprimer, pour tout $n$ de $\N$ et pour tout $i$ de $\lc 0,N\rc $, la
probabilité de l'événement $\left[ X_{n} = i\right] $, en fonction des
réels $\alpha_{k}$, $\lambda_{k}$ et $V_{k}\left( i\right) $ (pour
$k\in \lc 0,N\rc $ ).

\item Montrer que, pour tout $i$ de $\lc 1,N\rc $, on
a : $\di\dlim{n\rightarrow + \infty }\Prob\left(\Ev{ X_{n} = i
}\right) = 0$.

\item En déduire que le virus finit par dispara\^{\i}tre presque
s\^{u}rement, quelle que soit la loi de la variable aléatoire initiale
$X_{0}$.
\end{noliste}
\end{noliste}\bi

\begin{center}\textbf{Partie III. Estimations ponctuelle et par
intervalle de
confiance de $p$}\end{center}

\n On suppose que le paramètre $p$, qui exprime la probabilité qu'un
individu contagieux transmette le virus à un individu sain, est
inconnu,
et on cherche à l'estimer. \\

\n On rappelle que : $q = 1-p$.\\

\n Pour $m$ entier supérieur ou égal à $m$, on considère un
$m$-échantillon $\left( Y_{1},Y_{2},\dots,Y_{m}\right) $ de variables
aléatoires indépendantes, de même loi de Bernoulli de paramètre $p$,
définies sur un espace probabilisé $\left(
\Omega,\mathcal{A},\Prob\right) $.\\

\n On pose : $\di\overline{Y_{m}} = \frac{1}{m}\Sum{i = 1}{m}Y_{i}$.\\

\n \textit{Dans toute la suite de cette partie, on note $\varepsilon $
un réel
strictement positif quelconque.}

\begin{noliste}{1.}
 \setlength{\itemsep}{4mm}
\item 
\begin{noliste}{a)}
 \setlength{\itemsep}{2mm}
\item Montrer que $\overline{Y_{m}}$ est un estimateur sans biais de
$p$; déterminer son risque quadratique.

\item \`{A} l'aide de l'inégalité de Bienaymé-Tchebycheff,
montrer que l'intervalle $ \left[ \
\overline{Y_{m}}-\sqrt{\frac{5}{m}},\overline{Y_{m}} +
\sqrt{\frac{5}{m}}\right] $ est un intervalle de
confiance de $p$ au niveau de confiance $0,95$.
\end{noliste}

\item Soit $\theta $ un réel positif.

\begin{noliste}{a)}
 \setlength{\itemsep}{2mm}
\item Établir l'égalité suivante : $ \mathrm{P}\left(
\overline{Y_{m}}-p\geq \varepsilon \right) = \mathrm{P}\left(
e^{m\theta \overline{Y_{m}}}\geq e^{m\theta \left(
p + \varepsilon \right) } \right) $

\item Montrer que si $T$ est une variable aléatoire discrète finie 
à valeurs positives d'espérance $\E\left( T\right) $, et $a$ un réel
strictement positif, on a l'inégalité : $\Prob\left(\Ev{
T\geq a}\right) \leq \dfrac{\E\left( T\right) }{a}$

\item Soit $g$ la fonction définie sur $\R^{+}$ par : $g\left(
x\right) = \ln \left( p~e^{x} + q\right) $.\\
Déduire des questions précédentes, l'inégalité suivante : $
\mathrm{P}\left( \overline{Y_{m}}-p\geq \varepsilon \right) \leq
e^{m\left[ g\left( \theta \right) -\theta \left(
p + \varepsilon \right) \right] }$

\item Montrer que la fonction $g$ est de classe $C^{2}$ sur $\R^{+}$
et vérifie, pour tout $x$ de $\R^{+}$, l'inégalité : $\left| g^{\prime
\prime }\left( x\right) \right| \leq \dfrac{1}{4}$

\item En déduire l'inégalité suivante : $g\left( \theta \right)
\leq \theta p + \dfrac{\theta ^{2}}{8}$

\item Étudier les variations de la fonction $h$ définie sur $\R^{+}$
par : $ h\left( x\right) = \frac{x^{2}}{8}-\varepsilon x$.\\
En déduire l'inégalité : $\Prob\left(\Ev{ \overline{Y_{m}}-p\geq
\varepsilon}\right) \leq e^{-2m\varepsilon ^{2}}$
\end{noliste}

\item On pose : $\di\overline{W_{m}} = \frac{1}{m}\Sum{i = 1}{m}\left(
1-Y_{i}\right) $\\
Établir l'inégalité : $\Prob\left(\Ev{\overline{W_{m}}-q\geq
\varepsilon}\right) \leq e^{-2m\varepsilon ^{2}}$

\item 
\begin{noliste}{a)}
 \setlength{\itemsep}{2mm}
\item Déduire des questions \textbf{2f)} et 3, l'inégalité
suivante : $ \mathrm{P}\left( \left| \overline{Y_{m}}-p\right| \geq
\varepsilon \right) \leq 2~e^{-2m\varepsilon ^{2}}.$

\item Sachant que $\ln \left( 0.025\right) \approx -3.688$, calculer
$2e^{-2m\varepsilon ^{2}}$ pour $\varepsilon =
\sqrt{\frac{1.844}{m}}.$\\
En déduire un nouvel intervalle de confiance de $p$ au niveau de
confiance $0.95$. Comparer cet intervalle de confiance à celui obtenu 
à la question \textbf{l.b)}. \\
Conclure.
\end{noliste}
\end{noliste}

\end{document}

