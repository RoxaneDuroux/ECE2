\documentclass[11pt]{article}%
\usepackage{geometry}%
\geometry{a4paper,
 lmargin = 2cm,rmargin = 2cm,tmargin = 2.5cm,bmargin = 2.5cm}

\input{../../../../../../macros.tex}

\pagestyle{fancy} %
\lhead{ECE2 \hfill septembre 2017 \\
 Mathématiques\\[.2cm]} %
\chead{\hrule} %
\rhead{} %
\lfoot{} %
\cfoot{} %
\rfoot{\thepage} %

\renewcommand{\headrulewidth}{0pt}% : Trace un trait de séparation
 % de largeur 0,4 point. Mettre 0pt
 % pour supprimer le trait.

\renewcommand{\footrulewidth}{0.4pt}% : Trace un trait de séparation
 % de largeur 0,4 point. Mettre 0pt
 % pour supprimer le trait.

\setlength{\headheight}{14pt}

\title{\bf \vspace{-1cm} HEC 1987} %
\author{} %
\date{} %

\begin{document}

\maketitle %
\vspace{-1.2cm}\hrule %
\thispagestyle{fancy}

\vspace*{.4cm}

% DEBUT DU DOC À MODIFIER : tout virer jusqu'au début de l'exo

%Définition et changement de valeurs de
compteurs%newcounter{cpt1}{section} compteur cpt1 remis à 0 à chaque
aumentation par stepcounter du compteur section%setcounter{cpt1}{3} on
met le compteur à 3%addtocounter{cpt1}{5} on ajoute 5 au compteur%
stepcounter{cpt1} on ajoute 1% ifthenelse{test}{alors}{sinon} (page
206) pour subordonner à une condition % whiledo{test}{commande} pour
faire une boucle (page 206 aussi) % value{cpt1} pour noter dans le
document la valeur de cpt1 
%Définition définitive d'opérateurs
mathématiques\newcommand{\ch}{\operatorname{ch}} 
\newcommand{\sh}{\operatorname{sh}}
\renewcommand{\tanh}{\operatorname{th}}
\renewcommand{\sinh}{\operatorname{sh}}
\renewcommand{\cosh}{\operatorname{ch}}
\newcommand{\argsh}{\operatorname{argsh}}
\newcommand{\argch}{\operatorname{argch}}
\newcommand{\argth}{\operatorname{argth}}
\newcommand{\Id}{\operatorname{Id}}
\renewcommand{\leq}{\leq}
\renewcommand{\geq}{\geq }

\newcommand{\dlim}{\lim}
\newcommand{\dsum}{\sum}
\newcommand{\dprod}{\prod}



%Définition de nouvelles couleurs : rgb(trois paramètres red green blue
entre 0 et 1); cmyk (quatre cyan magenta yellow black) entre 0 et 1;
gray (entre 0 et 1) et black, white, red, green, blue, cyan, magenta,
yellow% definecolor{0gris}{gray}{0.8} 
% Nouvelle commande pour encadrer le titre car shabox ne veut que d'une
seule ligne; ATTENTION A LA TAILLE; petite différence avec shadowbox ou
doublebox, voire fcolorbox ou colorbox (au lieu de shabox; laisser le
parbox tranquille sauf pour la taille de la boîte
\newcommand{\Tbox}[1]{\begin{center} \shabox{\parbox{0.6
\linewidth}{#1}} \end{center}} %[1] pour 1 paramètre ; #1 pour ce que
fait le 1er paramètre; entre accolades ce que fait la commande
%Mise en page en mode fancy : en-têtes et pieds de pages puis
définition des en-têtes et pieds de pages\pagestyle{fancy}
\lhead{ECE 2 - Mathématiques \\
Quentin Dunstetter - ENC-Bessières 2011$\backslash$2012}
\chead{}
\rhead{HEC 1987}
\rfoot[ \ \thepage]{\thepage}
\cfoot{}
\lfoot{}
\thispagestyle{fancy} %Mise en page de la 1ère page en mode fancy
%Trait en bas et en haut de la page (entre en-tête et texte et texte et
pied de page)\renewcommand{\footrulewidth}{0.4pt}
\renewcommand{\headrulewidth}{0.4pt}

\begin{center}
{\huge HEC Eco 1987}
\end{center}

\section*{EXERCIC\E\ 1}

Pour tout nombre entier naturel non nul $n,$ on pose :
\[
I_{n} = \dint{0}{n}e^{-x}x^{n}dx
\]

\begin{noliste}{1.}
 \setlength{\itemsep}{4mm}
\item Calculer $I_{1}$ et $I_{2}.$

\item Montrer que :
\[
I_{n} = e^{-n}n^{n}\dint{0}{n}e^{t}(1-\dfrac{t}{n})^{n}dt
\]

\item 

\begin{noliste}{a)}
 \setlength{\itemsep}{2mm}
\item Montrer que, pour tout élément $x$ de l'intervalle $[0,1[$ :
\[
\ln (1-x)\leq -x-\dfrac{x^{2}}{2}
\]

\item En déduire que, pour tout élément de l'intervalle $[0,n]$ :
\[
e^{t}\left( 1-\dfrac{t}{n}\right) ^{n}\leq e^{-t^{2}/2n}
\]
\end{noliste}

\item 

\begin{noliste}{a)}
 \setlength{\itemsep}{2mm}
\item Prouver que :
\[
\dfrac{e^{n}}{n^{n}}I_{n}\leq \sqrt{2n}\dint{0}{\sqrt{n/2}}e^{-u^{2}}du
\]

\item Montrer que, pour tout élément $u$ de l'intervalle $[1, + \infty
\lbrack 
$
\[
e^{-u^{2}}\leq e^{-u}
\]
En déduire la limite lorsque $n$ tend vers $ + \infty $ de
$\dfrac{e^{n}}{n^{n + 1}}I_{n}.$
\end{noliste}
\end{noliste}

\section*{EXERCIC\E\ 2}

On désigne par $E$ l'espace vectoriel des polynômes réels de degré
inférieur
ou égal à 4.\\
On considère l'application $f$ qui à tout élément $P$ de $E$ associe le
polynôme $f(P)$ défini par :
\[
f(P)(X) = (X-1)P^{\prime }(X)-P\left(\Ev{X}\right)
\]
où $P^{\prime }$ est le polynôme dérivé de $P.$

\begin{noliste}{1.}
 \setlength{\itemsep}{4mm}
\item Montrer que $f$ définit un endomorphisme de $E.$

\item On considère la base canonique $B = (1,X,X^{2},X^{3},X^{4})$ de
$E.$
Écrire la matrice associée à $f$ dans cette base.

\item Déterminer les éléments du noyau de $f.$ En déduire le rang de
$f.$

\item Déterminer les éléments de l'image de $f.$

\item On considère l'application $f^{2} = f\circ f.$

\begin{noliste}{a)}
 \setlength{\itemsep}{2mm}
\item Pour tout élément $P$ de $E,$ exprimer $f^{2}(P)$ à l'aide de
$P$, $P^{\prime }$ et $P^{\prime \prime }.$

\item Écrire la matrice associée à $f^{2}$ dans la base $B.$

\item Déterminer le noyau de $f^{2}.$ En déduire le rang de $f^{2}.$
\end{noliste}
\end{noliste}

\section*{EXERCIC\E\ 3}

Trois boules de couleurs différentes sont réparties dans deux urnes
$U_{1}$
et $U_{2}.$ On effectue une succession de tirages au hasard dans
l'ensemble
des couleurs; chaque couleur a la même probabilité d'être tirée et les
tirages sont indépendants. A l'issue de chaque tirage, la boule dont la
couleur a été tirée est changée d'urne.\\
On suppose que le nombre de boules situées dans l'urne $U_{1}$ avant le
premier tirage est aléatoire et représenté par la variable aléatoire
$X_{0}.$
Pour tout nombre entier naturel non nul $n,$ on note $X_{n}$ le nombre
de
boules situées dans l'urne $U_{1}$ après le $n^{\text{ème}}$ tirage.
Les
variables aléatoires $X_{n}$, où $n\geq 0,$ prennent leurs valeurs dans
l'ensemble $I = \{0,1,2,3\}.$

\begin{noliste}{1.}
 \setlength{\itemsep}{4mm}
\item 

\begin{noliste}{a)}
 \setlength{\itemsep}{2mm}
\item Pour tout couple $(i,j)$ d'éléments de $I,$ calculer la
probabilité
conditionnelle $p_{ij}$ pour que $X_{1} = j$ sachant que $X_{0} = i.$
On écrira
les résultats sous la forme d'une matrice $P = (p_{ij})$ et on
vérifiera que,
pour tout élément $i$ de $I,$
\[
\Sum{j = 0}{3}p_{ij} = 1
\]

\item On suppose que la loi de probabilité de $X_{0}$ est une loi
binomiale
de paramètres $(3,\dfrac{1}{2}),$ notée $\mathcal{B}(3,\dfrac{1}{2}).$
Quelle est la loi de probabilité de $X_{1}$ ?
\end{noliste}

\item 

\begin{noliste}{a)}
 \setlength{\itemsep}{2mm}
\item Expliquez pourquoi la probabilité conditionnelle que $X_{2} = j$
sachant
qeu $X_{1} = i$ vaut encore $p_{ij}.$

\item Pour tout couple $(i,j)$ d'éléments de $I,$ on note $p_{ij}{(2)}$
la
probabilité conditionnelle que $X_{2} = j$ sachant que $X_{0} = i.$
Montrer que :
\[
p_{ij}{(2)} = \Sum{k = 0}{3}p_{ik}.p_{kj}
\]
et en déduire les valeurs des nombres $p_{ij}{(2)},$ où $(i,j)\in
I^{2},$
qu'on écrira sous forme d'une matrice $P_{2} = (p_{ij}{(2)}).$ Pour
tout élément $i$ de $I,$ calculer $\Sum{j = 0}{3}p_{ij}{(2)}$

\item La loi de $X_{0}$ étant encore la loi
$\mathcal{B}(3,\dfrac{1}{2}),$
trouver celle de $X_{2}.$
\end{noliste}

\item Soit, pour tout nombre entier naturel non nul $n,$ $p_{ij}{(n)}$
la
probabilité conditionnelle que $X_{n} = j$ sachant que $X_{0} = i$.
Exprimer la
matrice $P_{n} = (p_{ij}{(n)})$ à l'aide de $P.$ Trouver la loi de
$X_{n},$
quand la loi de $X_{0}$ est $\mathcal{B}(3,\dfrac{1}{2}).$

\item On note $A_{n}$ l'évènement $(X_{1}\neq 0,X_{2}\neq
0,...,X_{n-1}\neq
0,X_{n} = 0)$ et $q_{in}$ la probabilité conditionnelle de $A_{n}$
sachant que 
$X_{0} = i.$ Calculer, pour tout entier $i$ de $I,$ les valeurs de
$q_{in}$
pour $n = 1,$ $n = 2$ et $n = 3.$
\end{noliste}

\label{fin}

\end{document}

