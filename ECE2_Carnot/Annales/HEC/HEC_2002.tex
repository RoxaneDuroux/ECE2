\documentclass[11pt]{article}%
\usepackage{geometry}%
\geometry{a4paper,
 lmargin = 2cm,rmargin = 2cm,tmargin = 2.5cm,bmargin = 2.5cm}

\input{../../../../../../macros.tex}

\pagestyle{fancy} %
\lhead{ECE2 \hfill septembre 2017 \\
 Mathématiques\\[.2cm]} %
\chead{\hrule} %
\rhead{} %
\lfoot{} %
\cfoot{} %
\rfoot{\thepage} %

\renewcommand{\headrulewidth}{0pt}% : Trace un trait de séparation
 % de largeur 0,4 point. Mettre 0pt
 % pour supprimer le trait.

\renewcommand{\footrulewidth}{0.4pt}% : Trace un trait de séparation
 % de largeur 0,4 point. Mettre 0pt
 % pour supprimer le trait.

\setlength{\headheight}{14pt}

\title{\bf \vspace{-1cm} HEC 2002} %
\author{} %
\date{} %

\begin{document}

\maketitle %
\vspace{-1.2cm}\hrule %
\thispagestyle{fancy}

\vspace*{.4cm}

% DEBUT DU DOC À MODIFIER : tout virer jusqu'au début de l'exo

%Définition et changement de valeurs de
compteurs%newcounter{cpt1}{section} compteur cpt1 remis à 0 à chaque
aumentation par stepcounter du compteur section%setcounter{cpt1}{3} on
met le compteur à 3%addtocounter{cpt1}{5} on ajoute 5 au compteur%
stepcounter{cpt1} on ajoute 1% ifthenelse{test}{alors}{sinon} (page
206) pour subordonner à une condition % whiledo{test}{commande} pour
faire une boucle (page 206 aussi) % value{cpt1} pour noter dans le
document la valeur de cpt1 
%Définition définitive d'opérateurs
mathématiques\newcommand{\ch}{\operatorname{ch}} 
\newcommand{\sh}{\operatorname{sh}}
\renewcommand{\tanh}{\operatorname{th}}
\renewcommand{\sinh}{\operatorname{sh}}
\renewcommand{\cosh}{\operatorname{ch}}
\newcommand{\argsh}{\operatorname{argsh}}
\newcommand{\argch}{\operatorname{argch}}
\newcommand{\argth}{\operatorname{argth}}
\newcommand{\Id}{\operatorname{Id}}
\renewcommand{\leq}{\leq}
\renewcommand{\geq}{\geq }

\newcommand{\dlim}{\lim}
\newcommand{\dsum}{\sum}
\newcommand{\dprod}{\prod}



%Définition de nouvelles couleurs : rgb(trois paramètres red green blue
entre 0 et 1); cmyk (quatre cyan magenta yellow black) entre 0 et 1;
gray (entre 0 et 1) et black, white, red, green, blue, cyan, magenta,
yellow% definecolor{0gris}{gray}{0.8} 
% Nouvelle commande pour encadrer le titre car shabox ne veut que d'une
seule ligne; ATTENTION A LA TAILLE; petite différence avec shadowbox ou
doublebox, voire fcolorbox ou colorbox (au lieu de shabox; laisser le
parbox tranquille sauf pour la taille de la boîte
\newcommand{\Tbox}[1]{\begin{center} \shabox{\parbox{0.6
\linewidth}{#1}} \end{center}} %[1] pour 1 paramètre ; #1 pour ce que
fait le 1er paramètre; entre accolades ce que fait la commande
%Mise en page en mode fancy : en-têtes et pieds de pages puis
définition des en-têtes et pieds de pages\pagestyle{fancy}
\lhead{ECE 2 - Mathématiques \\
Quentin Dunstetter - ENC-Bessières 2011$\backslash$2012}
\chead{}
\rhead{HEC 2002}
\rfoot[ \ \thepage]{\thepage}
\cfoot{}
\lfoot{}
\thispagestyle{fancy} %Mise en page de la 1ère page en mode fancy
%Trait en bas et en haut de la page (entre en-tête et texte et texte et
pied de page)\renewcommand{\footrulewidth}{0.4pt}
\renewcommand{\headrulewidth}{0.4pt}

\begin{center}
{\Large {\textbf{ECOLE DES HAUTES ETUDES COMMERCIALES}} \vspace{0.3cm}
}

{\normalsize CONCOURS D'ADMISSION SUR CLASSES PREPARATOIRES
\vspace{0.5cm}}

{\normalsize \hskip 0.0cm\hbox to 3cm{\hrulefill}}

{\normalsize \vspace{0.3cm} }

{\normalsize \textbf{OPTION ECONOMIQUE} \vspace{0.5cm} }

{\Large {\textbf{MATHEMATIQUES III}} \vspace{0.5cm} }

{\normalsize Jeudi 16 mai 2002, de 14h à 18h.}

{\normalsize \hskip 0.0cm\hbox to 3cm{\hrulefill}}

{\normalsize \vspace{0.3cm} }
\end{center}

\textit{La présentation, la lisibilité, l'orthographe, la qualité de
la rédaction, la clarté et la précision des raisonnements entreront
pour une part importante dans l'appréciation des copies. \hfill\break
Les
candidats sont invités à encadrer dans la mesure du possible les
résultats de leurs calculs. \hfill\break Ils ne doivent faire usage
d'aucun document ; l'utilisation de toute calcultrice et de tout
matériel\break électronique est interdite. \hfill\break Seule
l'utilisation d'une règle graduée est autorisée.} \vspace{0.8cm}

{\LARG\E\ EXERCICE I \vspace{0.5cm}}

{Le but de cet exercice est la résolution de l'équation matricielle $AM
= MB$, d'inconnue $M$, dans l'espace vectoriel $E$ des matrices carrées
d'ordre $2$ à coefficients réels. }\vspace{2mm} {On rappelle que si
$U_{1},\,U_{2},\,U_{3},\,U_{4}$ sont les matrices définies par : 
\[
\;U_{1} = 
\begin{smatrix}
1 & 0 \\
0 & 0
\end{smatrix}
\quad U_{2} = 
\begin{smatrix}
0 & 1 \\
0 & 0
\end{smatrix}
\quad U_{3} = 
\begin{smatrix}
0 & 0 \\
1 & 0
\end{smatrix}
\quad U_{4} = 
\begin{smatrix}
0 & 0 \\
0 & 1
\end{smatrix}
\]
la famille $(U_{1},\,U_{2},\,U_{3},\,U_{4})$ est une base de $E$, qui
est
donc de dimension $4$.}\vspace{2mm} {Si $A$ et $B$ sont deux matrices
de $E$, l'ensemble des matrices $M$ de $E$ vérifiant $AM = MB$ est noté
$V_{A,B}$.}

\begin{noliste}{1.}
 \setlength{\itemsep}{4mm}
\item Soit $A$ et $B$ deux matrices de $E$ et $\varphi_{A,B}$
l'application
qui, à toute matrice $M$ de $E$, associe \hfill\break la matrice
$AM-MB$.

\begin{noliste}{a)}
 \setlength{\itemsep}{2mm}
\item Montrer que $\varphi_{A,B}$ est un endomorphisme de $E$ et en
déduire que $V_{A,B}$ est un sous-espace vectoriel de $E$.

\item Dans le cas particulier où $ A = 
\begin{smatrix}
1 & -1 \\
-1 & \;1
\end{smatrix}
\;\text{et}\;B = 
\begin{smatrix}
-1 & 0 \\
2 & \;1
\end{smatrix}
$, construire la matrice carrée d'ordre $4$ qui représente $\varphi
_{A,B}$ dans la base $\,U_{2},\,U_{3},\,U_{4})$.

Montrer que cette matrice est inversible et en déduire l'ensemble
$V_{A,B}$.
\end{noliste}

\item \label{d} Dans cette question, $r$ et $s$ désignent deux réels
distincts et différents de $1$, et on pose : 
\[
D = 
\begin{smatrix}
1 & 0 \\
0 & r
\end{smatrix}
\quad \text{et}\quad \Delta = 
\begin{smatrix}
1 & 0 \\
0 & s
\end{smatrix}
\]

\begin{noliste}{a)}
 \setlength{\itemsep}{2mm}
\item Soit $ M = 
\begin{smatrix}
x & y \\
z & t
\end{smatrix}
$ une matrice quelconque de $E$. Donner des conditions nécessaires et
suffisantes sur $x,\,y,\,z,\,t$ pour que $M$ appartienne à $V_{D,\Delta
} $.

\item En déduire une base de $V_{D,\Delta}$.
\end{noliste}

\item \label{d2} Soit $a,b,c,d$ des réels non nuls vérifiant $a-b\neq
c-d,\;a-b\neq 1,\;c-d\neq 1$, $A$ et $B$ les matrices définies
par : \quad 
\[
A = 
\begin{smatrix}
a & 1-a \\
b & 1-b
\end{smatrix}
\quad,\quad B = 
\begin{smatrix}
c & 1-c \\
d & 1-d
\end{smatrix}
\]

\begin{noliste}{a)}
 \setlength{\itemsep}{2mm}
\item Montrer que les valeurs propres de $A$ sont $1$ et $a-b$. En
déduire
qu'il existe une matrice inversible $P$ de $E$, et une matrice $D$
égale
à celle de la question \ref{d} pour une valeur convenable de $r$,
telles
que l'on ait : $D = P^{-1} A P$.

\item Justifier de même l'existence d'une matrice inversible $Q$ de
$E$,
et d'une matrice $\Delta$ égale à celle de la question \ref{d} pour une
valeur convenable de $s$, telles que l'on ait : $\Delta = Q^{-1} B Q$.

\item Pour toute matrice $M$ de $E$, montrer qu'elle appartient à
$V_{A,B}$
si et seulement si la matrice $P^{-1}M Q$ appartient à $V_{D,\Delta}$.
En
déduire une base de $V_{A,B}$.
\end{noliste}

\item Dans cette question $r,\,s$ et $u,\,v$ désignent quatre réels
vérifiant $r\neq s,\,r\neq v,\;u\neq s,\;u\neq v$, et on pose : 
\[
D = 
\begin{smatrix}
u & 0 \\
0 & r
\end{smatrix}
\quad \text{et}\quad \Delta = 
\begin{smatrix}
v & 0 \\
0 & s
\end{smatrix}
\]

\begin{noliste}{a)}
 \setlength{\itemsep}{2mm}
\item \vspace{-2mm} Par une méthode analogue à celle de la question
\ref{d}, déterminer $V_{D,\Delta}$.

\item En déduire, par une méthode analogue à celle de la question
\ref{d2}, le sous-espace vectoriel $V_{A,B}$ dans le cas où $A$ et $B$
sont
deux matrices diagonalisables n'ayant aucune valeur propre commune.
\end{noliste}
\end{noliste}

{\LARGE EXERCICE II \vspace{0.5cm} }

{Cet exercice met en évidence le fait que l'existence d'une espérance
finie, pour une variable aléatoire, n'est pas toujours
intuitive.}\vspace{1mm} {Dans tout l'exercice, $I$ désigne l'intervalle
réel $[1, + \infty \lbrack $ et on suppose que toutes les variables
aléatoires
envisagées sont définies sur le même espace probabilisé
$(\Omega,\mathcal{A},\mathbf{P})$. } 

{\large Première approche\label{partieA}}

\begin{noliste}{1.}
 \setlength{\itemsep}{4mm}
\item Montrer que l'application $g$ définie par : \ $\left\{ 
\begin{matrix}
 = \frac{1}{t^{2}}} & \text{si}\;t\in I \\
g(t) = 0 & \text{sinon}\end{matrix}\right. $ \ est une densité de
probabilité.

\item Soit $X$ une variable aléatoire à valeurs dans $I$ admettant $g$
pour densité. Déterminer, pour tout réel $t$, la probabilité
$\mathbf{P}([X\leq t])$ et montrer que $X$ n'admet pas d'espérance.

\item Soit $X$ et $Y$ deux variables aléatoires à valeurs dans $I$
admettant $g$ pour densité et telles que, pour tout réel $t$, les
événements $[X\leq t]$ et $[Y\leq t]$ sont indépendants. On
définit alors deux variables aléatoires $U$ et $V$ par :\; $U =
\min(X,Y)$
et $V = \max(X,Y)$, c'est-à-dire que, pour tout $\omega$ de $\Omega$,
$U(\omega)$ est le plus petit des nombres $X(\omega)$ et $Y(\omega)$,
tandis
que $\V(\omega)$ est le plus grand de ces nombres.

\begin{noliste}{a)}
 \setlength{\itemsep}{2mm}
\item Pour tout réel $t$, exprimer l'événement $[\V\leq t]$ à
l'aide des variables aléatoires $X$ et $Y$; en déduire la probabilité
$\mathbf{P}([\V\leq t])$.

\item Montrer que la variable aléatoire $V$ admet pour densité
l'application $h$ définie par : \vspace{-2mm}
\[
\left\{ 
\begin{matrix}
h(t) = & \frac{2(t-1)}{t^{3}}} & \text{si}\;t\in I \\
h(t) = & 0 & \text{sinon}\end{matrix}\right.
\]

\item De façon analogue, calculer pour tout réel $t$ la probabilité
$\mathbf{P}([U>t])$ et en déduire que la variable aléatoire $U$
admet pour densité l'application $m$ définie par :\vspace{-2mm}
\[
\left\{ 
\begin{matrix}
m(t) = & \frac{2}{t^{3}}} & \text{si}\;t\in I \\
m(t) = & 0 & \text{sinon}\end{matrix}\right.
\]

\item Montrer que $V$ n'admet pas d'espérance et que $U$ admet une
espérance que l'on calculera.
\end{noliste}
\end{noliste}

{\large Situation plus générale \vspace{3mm} }

{Dans cette partie, $n$ désigne un entier supérieur ou égal à
$2$ et on suppose que $n$ visiteurs, numérotés de $1$ à $n$, se
rendent aléatoirement dans un musée et que, pour tout entier de
l'intervalle $[1,n]$, l'heure d'arrivée du visiteur numéro $k$ est
une variable aléatoire $X_{k}$ admettant pour densité l'application $g$
définie dans la partie \ref{partieA}.}

On suppose de plus que, pour tout réel $t$, les événements $[X_{1}\leq
t],\; [X_{2}\leq t],\; \ldots,\;[X_{n}\leq t]$ sont
mutuellement indépendants.

Si $r$ est un entier de l'intervalle $[1,n]$, on note $T_{r}$ la
variable aléatoire désignant l'heure d'arrivée du $r$-ième arrivant.

La partie \ref{partieA} traite donc du cas $n = 2$, les variables
aléatoires 
$U$ et $V$ étant respectivement égales à $T_{1}$ et $T_{2}$.

\begin{noliste}{1.}
 \setlength{\itemsep}{4mm}
\item Soit $t$ un élément de $I$ fixé. Pour tout entier $k$ de $[1,n]$,
on note $B_{k}$ la variable aléatoire prenant la valeur $1$
lorsque l'événement $[X_{k}\leq t]$ est réalisé et la
valeur $0$ sinon.

\begin{noliste}{a)}
 \setlength{\itemsep}{2mm}
\item Préciser, en la justifiant soigneusement, la loi de la variable
aléatoire $Z$ définie par : \vspace{-3mm}
\[
Z = B_{1} + \, \ldots \, + \, B_{n}
\]

\item Pour tout entier $r$ de l'intervalle $[1,n]$, exprimer
l'événement $[T_{r}\leq t]$ à l'aide de la variable aléatoire $Z$ et
en déduire l'égalité : \ $\text{P}([T_{r}\leq
t]) = \Sum{k = r}{n}\Sum{k = r}{n}\binom{n}{k}\left(
1-\frac{1}{t}\right)
^{k}\left( \frac{1}{t}\right) ^{n-k}\cdotp$

\item Vérifier, pour tout entier $k$ de l'intervalle $[1,n]$, l'égalité
: \ $ k\binom{n}{k}-\left( n + 1-k\right) \binom{n}{k-1} = 0$.

\item En déduire que, pour tout entier $r$ de l'intervalle $[1,n]$, la
variable aléatoire $T_{r}$ admet pour densité l'application $f_{r}$
définie par :\vspace{-3mm} 
\[
\left\{ 
\begin{matrix}
f_{r}(t) = \binom{n}{r}\left( \frac{1}{t}\right) ^{n + 2-r}\left(
1-\frac{1}{t}\right) ^{r-1}} & \text{si}\;t\in I \\
f_{r}(t) = 0 & \text{sinon}\end{matrix}\right. 
\]

\item Donner un équivalent à \ $tf_{r}(t)$ quand $t$ tend vers $ +
\infty $ et en déduire que les variables aléatoires
$T_{1},\,T_{2},\,\ldots \,,\,T_{n-1}$ admettent une espérance alors que
$T_{n}$ n'en admet pas.
\end{noliste}

\item Pour tout couple $(p,q)$ d'entiers naturels, on pose :\ $ J(p,q)
= \dint{0}{1}x^{p}\,(1-x)^{q}\,\text{d}x.$

\begin{noliste}{a)}
 \setlength{\itemsep}{2mm}
\item \`{A} l'aide d'une intégration par parties, établir pour tout
couple $(p,q)$ d'entiers naturels, la relation : \vspace{-3mm} 
\[
(p + 1)\,J(p,q + 1) = (q + 1)\,J(p + 1,q)
\]

\item Calculer, pour tout entier naturel $q$, l'intégrale $J(0,q)$.

\item Montrer par récurrence sur $p$ que, pour tout couple d'entiers
naturels $(p,q)$, on a :\vspace{-3mm} 
\[
J(p,q) = \frac{p!\,q!}{(1 + p + q)!}
\]
\end{noliste}

\item Soit $r$ un entier de l'intervalle $[1,n-1]$.

\begin{noliste}{a)}
 \setlength{\itemsep}{2mm}
\item Si $a$ est un réel strictement supérieur à $1$,
transformer en effectuant le changement de variable $ = \frac{1}{t}$ \
l'intégrale\ $\dint{1}{a}t\,f_{r}(t)\,\text{d}t$.

\item En déduire la valeur de l'espérance de la variable aléatoire
$T_{r}$ en fonction de $n$ et de $r$.
\end{noliste}
\end{noliste}

\end{document}

