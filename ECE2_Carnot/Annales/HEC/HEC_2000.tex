\documentclass[11pt]{article}%
\usepackage{geometry}%
\geometry{a4paper,
 lmargin = 2cm,rmargin = 2cm,tmargin = 2.5cm,bmargin = 2.5cm}

\input{../../../../../../macros.tex}

\pagestyle{fancy} %
\lhead{ECE2 \hfill septembre 2017 \\
 Mathématiques\\[.2cm]} %
\chead{\hrule} %
\rhead{} %
\lfoot{} %
\cfoot{} %
\rfoot{\thepage} %

\renewcommand{\headrulewidth}{0pt}% : Trace un trait de séparation
 % de largeur 0,4 point. Mettre 0pt
 % pour supprimer le trait.

\renewcommand{\footrulewidth}{0.4pt}% : Trace un trait de séparation
 % de largeur 0,4 point. Mettre 0pt
 % pour supprimer le trait.

\setlength{\headheight}{14pt}

\title{\bf \vspace{-1cm} HEC 2000} %
\author{} %
\date{} %

\begin{document}

\maketitle %
\vspace{-1.2cm}\hrule %
\thispagestyle{fancy}

\vspace*{.4cm}

% DEBUT DU DOC À MODIFIER : tout virer jusqu'au début de l'exo

%Définition et changement de valeurs de
compteurs%newcounter{cpt1}{section} compteur cpt1 remis à 0 à chaque
aumentation par stepcounter du compteur section%setcounter{cpt1}{3} on
met le compteur à 3%addtocounter{cpt1}{5} on ajoute 5 au compteur%
stepcounter{cpt1} on ajoute 1% ifthenelse{test}{alors}{sinon} (page
206) pour subordonner à une condition % whiledo{test}{commande} pour
faire une boucle (page 206 aussi) % value{cpt1} pour noter dans le
document la valeur de cpt1 
%Définition définitive d'opérateurs
mathématiques\newcommand{\ch}{\operatorname{ch}} 
\newcommand{\sh}{\operatorname{sh}}
\renewcommand{\tanh}{\operatorname{th}}
\renewcommand{\sinh}{\operatorname{sh}}
\renewcommand{\cosh}{\operatorname{ch}}
\newcommand{\argsh}{\operatorname{argsh}}
\newcommand{\argch}{\operatorname{argch}}
\newcommand{\argth}{\operatorname{argth}}
\newcommand{\Id}{\operatorname{Id}}
\renewcommand{\leq}{\leq}
\renewcommand{\geq}{\geq }

\newcommand{\dlim}{\lim}
\newcommand{\dsum}{\sum}
\newcommand{\dprod}{\prod}



%Définition de nouvelles couleurs : rgb(trois paramètres red green blue
entre 0 et 1); cmyk (quatre cyan magenta yellow black) entre 0 et 1;
gray (entre 0 et 1) et black, white, red, green, blue, cyan, magenta,
yellow% definecolor{0gris}{gray}{0.8} 
% Nouvelle commande pour encadrer le titre car shabox ne veut que d'une
seule ligne; ATTENTION A LA TAILLE; petite différence avec shadowbox ou
doublebox, voire fcolorbox ou colorbox (au lieu de shabox; laisser le
parbox tranquille sauf pour la taille de la boîte
\newcommand{\Tbox}[1]{\begin{center} \shabox{\parbox{0.6
\linewidth}{#1}} \end{center}} %[1] pour 1 paramètre ; #1 pour ce que
fait le 1er paramètre; entre accolades ce que fait la commande
%Mise en page en mode fancy : en-têtes et pieds de pages puis
définition des en-têtes et pieds de pages\pagestyle{fancy}
\lhead{ECE 2 - Mathématiques \\
Quentin Dunstetter - ENC-Bessières 2011$\backslash$2012}
\chead{}
\rhead{HEC 2000}
\rfoot[ \ \thepage]{\thepage}
\cfoot{}
\lfoot{}
\thispagestyle{fancy} %Mise en page de la 1ère page en mode fancy
%Trait en bas et en haut de la page (entre en-tête et texte et texte et
pied de page)\renewcommand{\footrulewidth}{0.4pt}
\renewcommand{\headrulewidth}{0.4pt}


\begin{center}
{\LARGE Ecole des Hautes Études Commerciales }

{\Large Option économique 2000}

{\Large Mathématiques III}
\end{center}

{\LARGE Exercice 1}

\begin{noliste}{1.}
 \setlength{\itemsep}{4mm}
\item Montrer que, pour tout nombre réel $x>0$ et tout entier naturel
$k,$ l'intégrale 
\[
\dint{1}{\infty }\frac{t^{k}e^{-xt}}{1 + t^{5}}dt
\]
est convergente.

 
Pour quelles valeurs de l'entier $k$ cette intégralle est-elle aussi
convergente pour $x = 0$ ?


\item On se propose d'étudier la fonction $F$ définie, pour $x\geq 0,
$ par $\left( x\right) = \dint{1}{\infty }\frac{e^{-xt}}{1 + t^{5}}dt.$


Montrer que $F$ est une fonction strictement positive, décroissante et
que 
\[
\dlim{x\longrightarrow + \infty }F\left( x\right) = 0
\]


\item 
\begin{noliste}{a)}
 \setlength{\itemsep}{2mm}
\item Montrer que, pour tout réel $t\geq 0,$ tout réel $x\geq 0$ et
tout réel $h\geq 0,$ on a : 
\[
\left| e^{-t(x + h)}-e^{-tx} + \left. t\right. \left. h\right.
e^{-tx}\right|
\leq \frac{t^{2}h^{2}}{2}e^{-tx}
\]


\item Montrer de même que, pour tout réel $t\geq 0,$ tout réel $x\geq
0$ et tout réel $h\leq 0,$ on a : 
\[
\left| e^{-t(x + h)}-e^{-tx} + \left. t\right. \left. h\right.
e^{-tx}\right|
\leq \frac{t^{2}h^{2}}{2}e^{-t(x + h)}
\]


\item En déduire que pour tout réel $x\geq 0$ et tout réel $h$
tel que $x + h\geq 0,$ on a : 
\[
\left| F(x + h)-F(x) + h\dint{1}{\infty }\frac{te^{-xt}}{1 +
t^{5}}dt\right| \leq 
\frac{h^{2}}{2}\dint{1}{\infty }\frac{t^{2}}{1 + t^{5}}dt
\]


\item Montrer enfin que la fonction $F$ est dérivable sur$\left[
0, + \infty \right[ $ et donner une expression de sa fonction dérivée
$F^{\prime }.$
\end{noliste}


\item Montrer de même que $F^{\prime }$ est dérivable sur $\left[
0, + \infty \right[ $ et que $ = \dint{1}{\infty }\frac{t^{2}e^{-xt}}{1
+ t^{5}}dt$


\item On se propose de montrer que la fonction $\ln (F)$ est convexe.


\begin{noliste}{a)}
 \setlength{\itemsep}{2mm}
\item Montrer que si $a$, $b$ et $c$ sont trois nombres réels tels que,
pour tout réel $\lambda,$ on ait l'inégalité : $a\lambda
^{2} + 2b\lambda + c\geq 0,$ alors, nécessairement, $ac-b^{2}\geq 0.$


\item En déduire que la fonction $\ln (F)$ est une fonction convexe.



\end{noliste}
\end{noliste}


{\LARGE Exercice II}


On dispose de deux jetons $A$ et $B$ que l'on peut placer dans deux
cases $C_{0}$ et $C_{1},$ et d'un dispositif permettant de tirer au
hasard et de
manière équiprobable, l'une des lettre $a$, $b$ ou $c$. Au début
de l'expérience, les deux jetons sont placés dans $C_{0}.$ On
procède alors à une série de tirages indépendants de l'une
des trois lettres $a$, $b$ ou $c$.


A la suite de chaque tirage, on effectue l'opération suivante :


\begin{noliste}{$\sbullet$}
\item si la lettre $a$ est tirée, on change le jeton $A$ de case,


\item si la lettre $b$ est tirée, on change le jeton $B$ de case,


\item si la lettre $c$ est tirée, on ne change pas le placement des
jetons.
\end{noliste}


On suppose qu'il existe un espace probabilisé dont la probabilité est
notée $p$, qui modélise cette expérience et que l'on définit
deux suites de variables aléatoires sur cet espace, $\left(
X_{n}\right)
_{n\geq 0}$ et $\left( Y_{n}\right)_{n\geq 0}$, décrivant les positions
respectives de $A$ et $B$, en posant :


$X_{0} = Y_{0} = 0$, et pour tout entier naturel n non nul, $X_{n} = 0$
si à
l'issue de la $n^{i\grave{e}me}$ opération, le jeton $A$ se trouve dans
$C_{0}$ et $X_{n} = 1$ s'il se trouve dans $C_{1}$; de même, $Y_{n} =
0$ si
à l'issue de la $n^{i\grave{e}me}$ opération, le jeton $B$ se trouve
dans $C_{0}$ et $Y_{n} = 1$ s'il se trouve dans $C_{1}.$


{\large I Simulation}


Écrire un programme en -\Scilab{} permettant de simuler l'expérience,
qui lira un entier $N$ entré au clavier, représentant le nombre de
tirages à effectuer, et qui affichera à l'écran la liste des
couples observés $\left( X_{n},Y_{n}\right) $ pour $1\leq n\leq N.$


Ce programme utilisera la fonction \texttt{RANDOM }qui renvoie, pour un
argument $m$ de type \texttt{INTEGER}, un nombre entier de l'intervalle
$\left[ 0,m-1\right] $, tiré au hasard et de manière équiprobable.


(Cette fonction doit être initialisé par la commande
\texttt{RANDOMIZE})





{\Large II Étude du mouvement du jeton A.}


\begin{noliste}{1.}
 \setlength{\itemsep}{4mm}
\item 
\begin{noliste}{a)}
 \setlength{\itemsep}{2mm}
\item Soit n un entier strictement positif. Déterminer la
probabilité que, à l'issue de la $n^{i\grave{e}me}$ opération,
le jeton $A$ n'ait jamais quitté $C_{0}.$


\item Quelle est la proabilité que le jeton $A$ reste indéfiniment
dans $C_{0}$ ?
\end{noliste}


\item Pour tout entier naturel $k$ supérieur ou égal à 2, on
s'interresse à l'événement $D_{k} :$ à l'issue de la $k^{i\grave{e}me}$
opération, le jeton $A$ revient pour la première fois
dans $C_{0}.$ Déterminer la probabilité $p\left( D_{k}\right) $.

\item Soit M la matrice 
\[
M = \left( 
\begin{array}{ll}
2 & 1 \\
1 & 2
\end{array}
\right) 
\]


\begin{noliste}{a)}
 \setlength{\itemsep}{2mm}
\item Déterminer les valeurs propres de $M$ et donner une base de
vecteurs propres.


\item En déduire l'expression de $M^{n,}$ pour tout entier $n$
strictement positif.
\end{noliste}


\item 
\begin{noliste}{a)}
 \setlength{\itemsep}{2mm}
\item Calculer les probabilités $p\left( X_{1} = 0\right) $ et $p\left(
X_{1} = 1\right).$


\item Déterminer une matrice $Q$ telle que, pour tout entier naturel
$n,
$ on ait l'égalité matricielle : 
\[
\left( 
\begin{array}{l}
p(X_{n + 1} = 0) \\
p(X_{n + 1} = 1)
\end{array}
\right) = Q\left( 
\begin{array}{l}
p(X_{n} = 0) \\
p(X_{n} = 1)
\end{array}
\right) 
\]


\item Pour tout entier naturel $n$ non nul, calculer la matrice $Q^{n}$
et
en déduire la loi de la variable $X_{n}.$
\end{noliste}
\end{noliste}


{\Large III Étude d'un mouvement du couple de jetons }$(A,B)$


On suppose que l'on définit sur le même espace probabilisé une
suite de variables aléatoires $\left( W_{n}\right)_{n\geq 0}$, à
valeurs dans $\left\{ 0,1,2,3\right\} $, décrivant les positions des
dexu jetons $A$ et $B,$ en posant :


$W_{0} = 0,$ et pour tout entier naturel $n$ non nul,


$W_{n} = 0,$ si à l'issue de la $n^{i\grave{e}me}$ opération, $A$ et $B
$ se trouvent tous les deux dans $C_{0},$


$W_{n} = 1,$ si à l'issue de la $n^{i\grave{e}me}$ opération, $A$ se
trouve dans $C_{0},$et $B$ dans $C_{1},$


$W_{n} = 2,$ si à l'issue de la $n^{i\grave{e}me}$ opération, $A$ se
trouve dans $C_{1},$et $B$ dans $C_{0},$


$W_{n} = 3,$ si à l'issue de la $n^{i\grave{e}me}$ opération, les deux
jetons $A$ et $B$ se trouvent dans $C_{1}.$


\begin{noliste}{1.}
 \setlength{\itemsep}{4mm}
\item Calculer la probabilité$p(W_{1} = i)$ pour $i$ égal à 0, 1,
2 et 3.


\item Déterminer la matrice $R$ telle que, pour tout entier naturel
$n,$
on ait l'égalité matricielle : 
\[
\left( 
\begin{array}{l}
p\left( W_{n + 1} = 0\right) \\
p\left( W_{n + 1} = 1\right) \\
p\left( W_{n + 1} = 2\right) \\
p\left( W_{n + 1} = 3\right) 
\end{array}
\right) = R\left( 
\begin{array}{l}
p\left( W_{n} = 0\right) \\
p\left( W_{n} = 1\right) \\
p\left( W_{n} = 2\right) \\
p\left( W_{n} = 3\right) 
\end{array}
\right) 
\]


\item On considère les matrices :


\[
I = \left( 
\begin{array}{llll}
1 & 0 & 0 & 0 \\
0 & 1 & 0 & 0 \\
0 & 0 & 1 & 0 \\
0 & 0 & 0 & 0
\end{array}
\right),U = \left( 
\begin{array}{llll}
1 & 1 & 1 & 1 \\
1 & 1 & 1 & 1 \\
1 & 1 & 1 & 1 \\
1 & 1 & 1 & 1
\end{array}
\right),V = \left( 
\begin{array}{llll}
0 & 0 & 0 & 1 \\
0 & 0 & 1 & 0 \\
0 & 1 & 0 & 0 \\
1 & 0 & 0 & 0
\end{array}
\right) 
\]


\begin{noliste}{a)}
 \setlength{\itemsep}{2mm}
\item Pour tout entier naturel $n$ non nul, calculer $U^{n}$ et
$V^{n}.$


\item Établir, pour tout entier naturel non nul $n,$ l'égalité 
\[
\left( U-\V\right) ^{n} = \Sum{k = 0}{n}\left( -1\right)
^{k}C_{n}{k}U^{n-k}V^{k}
\]


où par convention on pose : $U^{0} = V^{0} = I.$


\item En déduire, pour tout entier naturel non nul $n,$
l'égalité 
\[
\left( U-\V\right) ^{n} = \frac{1}{4}\left[ 3^{n}-\left( -1\right)
^{n}\right]
U + (-1)^{n}V^{n}
\]
\end{noliste}


\item Pour tout entier naturel $n$ non nul, calculer $R^{n}et$ donner
la
loi de la variable $W_{n}.$ (on distinguera les cas $n$ pair et $n$
impair)


\item Déterminer, pour tout entier naturel $n$ non nul, la covariance
de $X_{n}$ et $Y_{n}$ et calculer la limite de cette covariance quand
$n$
tend vers + $\infty.$
\end{noliste}


{\Large VI Étude d'un long séjour.}


On suppose que chaque tirage, avec l'opération qui le suit, dure une
minute. Ainsi, à lissue de la $n^{i\grave{e}me}$ opération, $n$
minutes se sont écoulées depuis le début de l'expérience.


Soit $n$ un entier naturel non nul.


On suppose que le nombre de minutes écoulées pendant lesquelles le
jeton $A$ a séjourné dans $C_{1}$, entre le début de
l'expérience et l'issue de la $n^{i\grave{e}me}$ opération, est une
variable aléatoire que l'on note $T_{n}.$


\begin{noliste}{1.}
 \setlength{\itemsep}{4mm}
\item Exprimer $T_{n}$ à l'aide des variables $X_{k}$, pour $k$ compris
entre 1 et $n.$


\item En dédure l'espérance $\E\left( T_{n}\right) $.


Calculer la limite de $\frac{1}{n}\E\left( T_{n}\right) $ quand $n$
tend vers l'infini.\newpage 
\end{noliste}

\end{document}

