\documentclass[11pt]{article}%
\usepackage{geometry}%
\geometry{a4paper,
 lmargin = 2cm,rmargin = 2cm,tmargin = 2.5cm,bmargin = 2.5cm}

\input{../../../../../../macros.tex}

\pagestyle{fancy} %
\lhead{ECE2 \hfill septembre 2017 \\
 Mathématiques\\[.2cm]} %
\chead{\hrule} %
\rhead{} %
\lfoot{} %
\cfoot{} %
\rfoot{\thepage} %

\renewcommand{\headrulewidth}{0pt}% : Trace un trait de séparation
 % de largeur 0,4 point. Mettre 0pt
 % pour supprimer le trait.

\renewcommand{\footrulewidth}{0.4pt}% : Trace un trait de séparation
 % de largeur 0,4 point. Mettre 0pt
 % pour supprimer le trait.

\setlength{\headheight}{14pt}

\title{\bf \vspace{-1cm} HEC 2001} %
\author{} %
\date{} %

\begin{document}

\maketitle %
\vspace{-1.2cm}\hrule %
\thispagestyle{fancy}

\vspace*{.4cm}

% DEBUT DU DOC À MODIFIER : tout virer jusqu'au début de l'exo

%Définition et changement de valeurs de
compteurs%newcounter{cpt1}{section} compteur cpt1 remis à 0 à chaque
aumentation par stepcounter du compteur section%setcounter{cpt1}{3} on
met le compteur à 3%addtocounter{cpt1}{5} on ajoute 5 au compteur%
stepcounter{cpt1} on ajoute 1% ifthenelse{test}{alors}{sinon} (page
206) pour subordonner à une condition % whiledo{test}{commande} pour
faire une boucle (page 206 aussi) % value{cpt1} pour noter dans le
document la valeur de cpt1 
%Définition définitive d'opérateurs
mathématiques\newcommand{\ch}{\operatorname{ch}} 
\newcommand{\sh}{\operatorname{sh}}
\renewcommand{\tanh}{\operatorname{th}}
\renewcommand{\sinh}{\operatorname{sh}}
\renewcommand{\cosh}{\operatorname{ch}}
\newcommand{\argsh}{\operatorname{argsh}}
\newcommand{\argch}{\operatorname{argch}}
\newcommand{\argth}{\operatorname{argth}}
\newcommand{\Id}{\operatorname{Id}}
\renewcommand{\leq}{\leq}
\renewcommand{\geq}{\geq }

\newcommand{\dlim}{\lim}
\newcommand{\dsum}{\sum}
\newcommand{\dprod}{\prod}



%Définition de nouvelles couleurs : rgb(trois paramètres red green blue
entre 0 et 1); cmyk (quatre cyan magenta yellow black) entre 0 et 1;
gray (entre 0 et 1) et black, white, red, green, blue, cyan, magenta,
yellow% definecolor{0gris}{gray}{0.8} 
% Nouvelle commande pour encadrer le titre car shabox ne veut que d'une
seule ligne; ATTENTION A LA TAILLE; petite différence avec shadowbox ou
doublebox, voire fcolorbox ou colorbox (au lieu de shabox; laisser le
parbox tranquille sauf pour la taille de la boîte
\newcommand{\Tbox}[1]{\begin{center} \shabox{\parbox{0.6
\linewidth}{#1}} \end{center}} %[1] pour 1 paramètre ; #1 pour ce que
fait le 1er paramètre; entre accolades ce que fait la commande
%Mise en page en mode fancy : en-têtes et pieds de pages puis
définition des en-têtes et pieds de pages\pagestyle{fancy}
\lhead{ECE 2 - Mathématiques \\
Quentin Dunstetter - ENC-Bessières 2011$\backslash$2012}
\chead{}
\rhead{HEC 2001}
\rfoot[ \ \thepage]{\thepage}
\cfoot{}
\lfoot{}
\thispagestyle{fancy} %Mise en page de la 1ère page en mode fancy
%Trait en bas et en haut de la page (entre en-tête et texte et texte et
pied de page)\renewcommand{\footrulewidth}{0.4pt}
\renewcommand{\headrulewidth}{0.4pt}

\begin{center}
{\huge HEC Eco 2001}
\end{center}

{\LARGE Exercice 1}

On note $m$ un paramètre réel et on considère les matrices $H_{m}$
définies
par 
\[
H_{m} = \left( 
\begin{array}{ccc}
-1-m & m & 2 \\
-m & 1 & m \\
-2 & m & 3-m
\end{array}
\right) 
\]
On note $h_{m}$ l'endomorphisme de $\mathbf{R}{3}$ ayant pour matrice
$H_{m} $ dans la base canonique de $\mathbf{R}{3}$.

\begin{noliste}{1.}
 \setlength{\itemsep}{4mm}
\item {On suppose dans cette question que $m = 2$. }

\begin{noliste}{a)}
 \setlength{\itemsep}{2mm}
\item {Déterminer les valeurs propres de la matrice $H_{2}$ et les
sous-espaces propres associés. }

\item {La matrice $H_{2}$ est-elle diagonalisable ? Si oui, donner une
base
de vecteurs propres. }
\end{noliste}

\item {Étudier de même les valeurs propres et les sous-espaces
propres de $H_{0}$. Cette matrice est-elle diagonalisable ? }

\item 
\begin{noliste}{a)}
 \setlength{\itemsep}{2mm}
\item {Montrer qu'il existe un réel $a$, qu'on déterminera, qui est
valeur
propre de la matrice $H_{m}$ pour toutes les valeurs du paramètre $m$.
}

\item {Déterminer, pour chaque valeur de $m$, le sous-espace propre
associé
à la valeur propre $a$. Montrer qu'on peut trouver un vecteur non nul
$v_{1}$
appartenant à tous ces sous-espaces. }
\end{noliste}

{%%%}

\item {Soit $F$ le sous-espace de $\mathbf{R}{3}$ engendré par les
vecteurs $v_{2} = (1,0,1)$ et $v_{3} = (1,1,0)$.\\
Déterminer les vecteurs $h_{m}(v_{2})$ et $h_{m}(v_{3})$ et montrer que
ces
vecteurs appartiennent à $F$ pour tout $m$ réel. }

\item {En se placant dans la base de $\mathbf{R}{3}$ formée des
vecteurs $v_{1}$, $v_{2}$ et $v_{3}$, déterminer les valeurs de $m$
pour lesquelles la
matrice $H_{m}$ est diagonalisable. }
\end{noliste}

{\LARGE Exercice 2.}

On réalise une suite de lancers indépendants d'une pièce de monnaie
équilibrée. On associe à cette expérience une suite $(X_{n})_{n\geq 1}$
de variables aléatoires indépendantes, définies sur un espace
probabilisé $(\Omega,\mathcal{A},\mathbf{P})$ et suivant toutes la loi
de Bernoulli de paramètre $\frac{1}{2}$.\\
Pour tout entier $n$ supérieur ou égal à 1, on pose $S_{n} = X_{1} +
\cdots
 + X_{n}$.\\
\emph{Notation} : Si $Z$ est une variable aléatoire définie sur
$(\Omega,\mathcal{A},\mathbf{P})$, on note $\E(Z)$ son espérance.\\
{N.B.} La partie {II} peut être traitée indépendamment de la partie
{I}.

\textbf{Partie I. Préliminaire}

\begin{noliste}{1.}
 \setlength{\itemsep}{4mm}
\item 
\begin{noliste}{a)}
 \setlength{\itemsep}{2mm}
\item {Déterminer la loi de probabilité de la variable $S_{n}$. }

\item {Quelles sont l'espérance et la variance de $S_{n}$ ? }
\end{noliste}

\item 
\begin{noliste}{a)}
 \setlength{\itemsep}{2mm}
\item {Montrer que pour tout réel }$\varepsilon ${\ strictement
positif, on
peut trouver une constante $K_{\varepsilon }$ telle que, pour tout
entier $n$
supérieur ou égal à 1, on ait l'inégalité : 
\[
\mathbf{P}\left( \left| \frac{S_{n}}{n}-\frac{1}{2}\right| \geq
\varepsilon
\right) \leq \frac{K_{\varepsilon }}{n}
\]
}

\item {Déduire de la majoration obtenue que, pour tout réel $r$
vérifiant $0<r<\frac{1}{2}$, on a : 
\[
\dlim{n\to + \infty }\mathbf{P}\left( \left|
\frac{S_{n}}{n}-\frac{1}{2}\right| \geq \frac{1}{n^{r}}\right) = 0
\]
}
\end{noliste}

\item {Montrer d'autre part, à l'aide du théorème de la limite centrée,
que
la suite définie pour $n$ supérieur ou égal à 1 par $ n\to
P\left(\Ev{ \left| \frac{S_{n}}{n}-\frac{1}{2}\right|
>\frac{1}{\sqrt{n}}}\right) 
$ admet une limite non nulle }
\end{noliste}

\emph{L'objet de la suite de l'exercice est l'étude d'une
majoration de la probabilité $\mathbf{P}\left( \left|
\frac{S_{n}}{n}-\frac{1}{2}\right| \geq \varepsilon \right) $ meilleure
que la
majoration obtenue à la question 2.a.}

\textbf{Partie II. Étude de fonctions}\\
On considère la fonction $f$ définie sur $\mathbf{R}$ par $f(x) = \ln
\left( \frac{e^{x} + 1}{2}\right) $.

\begin{noliste}{1.}
 \setlength{\itemsep}{4mm}
\item 
\begin{noliste}{a)}
 \setlength{\itemsep}{2mm}
\item {Étudier les variations de $f$. }

\item {Montrer qu'il existe des réels $\alpha $ et $\beta $, que l'on
déterminera, tels que : 
\[
\dlim{x\to + \infty }(f(x)-(\alpha x + \beta )) = 0
\]
}

\item {Montrer de même qu'il existe des réels $\alpha ^{\prime }$ et
$\beta
^{\prime }$ tels que : 
\[
\dlim{x\to -\infty }(f(x)-(\alpha ^{\prime }x + \beta ^{\prime })) = 0
\]
}
\end{noliste}

\item {Soit $a$ un réel vérifiant $0<a<1$ et soit $\varphi_{a}$ la
fonction définie sur $\mathbf{R}$ par 
\[
\forall x\in \mathbf{R},\ \varphi_{a}(x) = f(x)-ax
\]
}

\begin{noliste}{a)}
 \setlength{\itemsep}{2mm}
\item {Étudier les variations de la fonction $\varphi_{a}$ et préciser
les valeurs de $\varphi_{a}(0)$ et de $\varphi_{a}{\prime }(0)$.
Montrer
que la fonction $\varphi_{a}$ atteint un minimum en un unique point
$x_{a}$
de $\mathbf{R}$ dont on donnera l'expression en fonction de $a$. }

\item {Étudier le signe de $x_{a}$ suivant les valeurs de $a$. Montrer
que, pour tout réel $a$ vérifiant $a\ne \frac{1}{2}$, on a : 
\[
e^{\varphi_{a}(x_{a})}<1
\]
}
\end{noliste}
\end{noliste}

\emph{Dans toute la suite, pour tout réel a vérifiant $0<a<1$, on
pose $h_{a} = e^{\varphi_{a}(x_{a})}$}

\textbf{Partie III. Étude de l'écart de \boldmath
}$\frac{S_{n}}{n}$\textbf{\unboldmath\ à sa moyenne}

\begin{noliste}{1.}
 \setlength{\itemsep}{4mm}
\item {Montrer que si $Z$ est une variable aléatoire prenant un nombre
fini
de valeurs réelles positives $z_{1},\ldots,z_{r}$, on a la majoration
}$\mathbf{P}${$(Z\geq 1)\leq E(Z)$. }

\item {Calculer, pour tout réel $u$ et tout entier $n$ supérieur ou
égal à
1, l'espérance $\E(e^{uS_{n}})$. En déduire que, pour tout entier $n$
supérieur ou égal à 1, si $a$ est un réel vérifiant $0<a<1$ et $t$ un
réel
quelconque, on a : 
\[
\E\left( e^{t\,(\frac{S_{n}}{n}-a)}\right) = e^{\varphi_{a}\left(
\frac{t}{n}\right) }
\]
}

\item {\emph{On suppose dans cette question que $a$ est un réel
vérifiant $\frac{1}{2}<a<1$.} }

\begin{noliste}{a)}
 \setlength{\itemsep}{2mm}
\item {Montrer que, pour tout réel $t$ strictement positif et tout
entier $n
$ supérieur ou égal à 1, on a l'inégalité : 
\[
\mathbf{P}\left( \frac{S_{n}}{n}-a\geq 0\right) \leq
e^{\varphi_{a}(\frac{t}{n})}
\]
}

\item {En donnant à $t$, dans l'inégalité précédente, une valeur
convenablement choisie, établir, pour tout entier $n$ supérieur ou égal
à
1, l'inégalité : 
\[
\mathbf{P}\left( \frac{S_{n}}{n}\geq a\right) \leq (h_{a})^{n}
\]
}
\end{noliste}

\item {Soit }$\varepsilon ${\ un réel vérifiant $0<\varepsilon
<\frac{1}{2}$; on pose $a = \frac{1}{2} + \varepsilon $. }

\begin{noliste}{a)}
 \setlength{\itemsep}{2mm}
\item {Comparer, pour tout entier $n$ supérieur ou égal à 1, les lois
de
probabilité des variables aléatoires $S_{n}$ et $n-S_{n}$. En déduire
l'égalité : 
\[
\mathbf{P}\left( \frac{S_{n}}{n}\geq a\right) = \mathbf{P}\left(
\frac{S_{n}}{n}\leq 1-a\right) 
\]
puis la majoration : 
\[
\mathbf{P}\left( \left| \frac{S_{n}}{n}-\frac{1}{2}\right| \geq
\varepsilon
\right) \leq 2(h_{a})^{n}
\]
En quoi cette majoration peut-elle être considérée, pour }$\varepsilon
${\
strictement positif fixé, comme meilleure que celle de la question
I.2.a ? }

\item {\`{A}l'aide de l'expression de $x_{a}$ trouvée à la question
II.2.a, 
établir l'égalité 
\[
\varphi_{a}(x_{a}) = -\left( (\frac{1}{2} + \varepsilon )\ln (1 +
2\varepsilon ) + (\frac{1}{2}-\varepsilon )\ln (1-2\varepsilon )\right)

\]
}
\end{noliste}

\item 
\begin{noliste}{a)}
 \setlength{\itemsep}{2mm}
\item {En déduire qu'on a : $\varphi_{\frac{1}{2} + \varepsilon
}(x_{\frac{1}{2} + \varepsilon }) = -2\varepsilon ^{2} + o(\varepsilon
^{2})$, quand }$\varepsilon ${\ tend vers 0. }

\item {Montrer qu'on peut retrouver ainsi la limite obtenue en I.2.b. }
\end{noliste}
\end{noliste}

\textbf{Partie IV. Étude d'un algorithme}\\
On se propose d'illustrer cet exercice par une simulation. On considère
pour
cela le programme -\Scilab{} nommé \texttt{Simulation} et reproduit
ci-dessous, dans lequel \texttt{RANDOM(100)} désigne un nombre entier
tiré
au hasard par l'ordinateur, avec la loi uniforme, dans l'intervalle
$[0,99]$
(la procédure \texttt{RANDOMIZE} sert à initialiser la fonction
\texttt{RANDOM}).

\begin{noliste}{1.}
 \setlength{\itemsep}{4mm}
\item {Que fait la procédure \texttt{EP} de ce programme ? }

\item {Quels nombres entiers sont comptabilisés dans les variables
\texttt{U}, \texttt{V} et \texttt{W} à la fin de ce programme ? }

\item {De quels nombres les valeurs \texttt{P1}, \texttt{P2} et
\texttt{P3}
fournies par le programme sont-elles des estimations ? }

\item {\`{A} quoi peut-on s'attendre pour la valeur de \texttt{P1} ? }

\texttt{PROGRAM Simulation;}

\texttt{\hspace{1cm}VAR}

\texttt{\hspace{1cm}n,K,U,V,W,i,j,X :integer;}

\texttt{\hspace{1cm}S :real;}

\texttt{CONST}

\texttt{\hspace{1cm}nombredelancers = 20000;}

\texttt{\hspace{1cm}nombredessais = 2000;}

\texttt{PROCEDURE EP\left(\Ev{n :integer}\right);}

\texttt{begin}

\texttt{\hspace{1cm}S : = 0;}

\texttt{\hspace{1cm}for i : = 1 to n do}

\texttt{\hspace{1cm}begin}

\texttt{\hspace{2cm}X : = RANDOM(100);}

\texttt{\hspace{2cm}if X>49 then S : = S + 1;}

\texttt{\hspace{1cm}end;}

\texttt{end;}

\texttt{BEGIN}

\texttt{\hspace{1cm}RANDOMIZE;}

\texttt{\hspace{1cm}n : = nombredelancers;}

\texttt{\hspace{1cm}K : = nombredessais;}

\texttt{\hspace{1cm}U : = 0; V : = 0; W : = 0;}

\texttt{\hspace{1cm}for j : = 1 to K do}

\texttt{\hspace{1cm}begin}

\texttt{\hspace{2cm}EP\left(\Ev{n}\right);}

\texttt{\hspace{2cm}if abs(S/n-0.5) > exp((-0.4)*ln(n)) then
U : = U + 1;}
 
\texttt{\hspace{2cm}if abs(S/n-0.5) > exp((-0.5)*ln(n)) then
V : = V + 1;}

\texttt{\hspace{2cm}if abs(S/n-0.5) > exp((-0.9)*ln(n)) then
W : = W + 1;}

\texttt{\hspace{1cm}end;}

\texttt{\hspace{1cm}writeln('P1 = ',U/K);}

\texttt{\hspace{1cm}writeln('P2 = ',V/K);}

\texttt{\hspace{1cm}writeln('P3 = ',W/K);}

\texttt{END.}
\end{noliste}

\end{document}

