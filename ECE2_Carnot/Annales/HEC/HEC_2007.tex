\documentclass[11pt]{article}%
\usepackage{geometry}%
\geometry{a4paper,
 lmargin = 2cm,rmargin = 2cm,tmargin = 2.5cm,bmargin = 2.5cm}

\input{../../../../../../macros.tex}

\pagestyle{fancy} %
\lhead{ECE2 \hfill septembre 2017 \\
 Mathématiques\\[.2cm]} %
\chead{\hrule} %
\rhead{} %
\lfoot{} %
\cfoot{} %
\rfoot{\thepage} %

\renewcommand{\headrulewidth}{0pt}% : Trace un trait de séparation
 % de largeur 0,4 point. Mettre 0pt
 % pour supprimer le trait.

\renewcommand{\footrulewidth}{0.4pt}% : Trace un trait de séparation
 % de largeur 0,4 point. Mettre 0pt
 % pour supprimer le trait.

\setlength{\headheight}{14pt}

\title{\bf \vspace{-1cm} HEC 2007} %
\author{} %
\date{} %

\begin{document}

\maketitle %
\vspace{-1.2cm}\hrule %
\thispagestyle{fancy}

\vspace*{.4cm}

% DEBUT DU DOC À MODIFIER : tout virer jusqu'au début de l'exo

%Définition et changement de valeurs de
compteurs%newcounter{cpt1}{section} compteur cpt1 remis à 0 à chaque
aumentation par stepcounter du compteur section%setcounter{cpt1}{3} on
met le compteur à 3%addtocounter{cpt1}{5} on ajoute 5 au compteur%
stepcounter{cpt1} on ajoute 1% ifthenelse{test}{alors}{sinon} (page
206) pour subordonner à une condition % whiledo{test}{commande} pour
faire une boucle (page 206 aussi) % value{cpt1} pour noter dans le
document la valeur de cpt1 
%Définition définitive d'opérateurs
mathématiques\newcommand{\ch}{\operatorname{ch}} 
\newcommand{\sh}{\operatorname{sh}}
\renewcommand{\tanh}{\operatorname{th}}
\renewcommand{\sinh}{\operatorname{sh}}
\renewcommand{\cosh}{\operatorname{ch}}
\newcommand{\argsh}{\operatorname{argsh}}
\newcommand{\argch}{\operatorname{argch}}
\newcommand{\argth}{\operatorname{argth}}
\newcommand{\Id}{\operatorname{Id}}
\renewcommand{\leq}{\leq}
\renewcommand{\geq}{\geq }

\newcommand{\dlim}{\lim}
\newcommand{\dsum}{\sum}
\newcommand{\dprod}{\prod}



%Définition de nouvelles couleurs : rgb(trois paramètres red green blue
entre 0 et 1); cmyk (quatre cyan magenta yellow black) entre 0 et 1;
gray (entre 0 et 1) et black, white, red, green, blue, cyan, magenta,
yellow% definecolor{0gris}{gray}{0.8} 
% Nouvelle commande pour encadrer le titre car shabox ne veut que d'une
seule ligne; ATTENTION A LA TAILLE; petite différence avec shadowbox ou
doublebox, voire fcolorbox ou colorbox (au lieu de shabox; laisser le
parbox tranquille sauf pour la taille de la boîte
\newcommand{\Tbox}[1]{\begin{center} \shabox{\parbox{0.6
\linewidth}{#1}} \end{center}} %[1] pour 1 paramètre ; #1 pour ce que
fait le 1er paramètre; entre accolades ce que fait la commande
%Mise en page en mode fancy : en-têtes et pieds de pages puis
définition des en-têtes et pieds de pages\pagestyle{fancy}
\lhead{ECE 2 - Mathématiques \\
Quentin Dunstetter - ENC-Bessières 2011$\backslash$2012}
\chead{}
\rhead{HEC III 2007}
\rfoot[ \ \thepage]{\thepage}
\cfoot{}
\lfoot{}
\thispagestyle{fancy} %Mise en page de la 1ère page en mode fancy
%Trait en bas et en haut de la page (entre en-tête et texte et texte et
pied de page)\renewcommand{\footrulewidth}{0.4pt}
\renewcommand{\headrulewidth}{0.4pt}


\begin{center}
{\Huge \ H.E.C. 2007}

{\LARGE OPTION : ÉCONOMIQUE}

{\Huge MATHÉMATIQUES III }
\end{center}

La présentation, la lisibilité, l'orthographe, la qualité de la
rédaction, la clarté et la précision des raisonnements entreront
pour une part importante dans l'appréciation des copies.

Les candidats sont invités à encadrer dans \{a mesure du possible
les résultats de leurs calculs.

Ils ne doivent faire usage d'aucun document : l'utilisation de toute
calculatrice et de tout matériel électronique est interdite.

Seule l'utilisation d'une règle graduée est autorisée.

\section*{EXERCICE}

\begin{noliste}{1.}
 \setlength{\itemsep}{4mm}
\item On considère $\R^{3}$ muni de sa base canonique $\left(
e_{1},e_{2},e_{3}\right) $ ; soit $t$ l'endomorphisme de $\R^{3}$,
dont la matrice associée $T$ relativement à cette base s'écrit : 
\[
T = 
\begin{smatrix}
1 & 1 & 1 \\
0 & 1 & 0 \\
0 & 1 & 0
\end{smatrix}
\]
\\
Calculer les valeurs propres de $t$. Déterminer les sous-espaces
propres
de $t$ associés, et donner une base de chacun d'entre eux.\\
L'endomorphisme $t$ est-il diagonalisable ? Est-il bijectif ?

\hspace{-1cm}L'objet des questions suivantes est une généralisation
des résultats précédents.

\item Soit $n$ un entier de $\N^{\ast }$. On considère l'espace
vectoriel $\R^{2n + 1}$ 1 muni de sa base canonique $\left(
e_{1},e_{2},\cdots,e_{2n + 1}\right) $. Soit $t$ l'endomorphisme de
$\R^{2n + 1}$ défini par :

\begin{noliste}{$\sbullet$}
\item pour tout entier $i$ de $\left[ 1,2n + 1\right] $, avec $i\neq
n + 1 :t\left( e_{i}\right) = e_{1}$ ;

\item $t\left( e_{n + 1}\right) = e_{1} + e_{2} + \cdots + e_{2n + 1}$.
\end{noliste}

\begin{noliste}{a)}
 \setlength{\itemsep}{2mm}
\item Déterminer la matrice $T$ associée à l'endomorphisme $t$
relativement à la base $\left( e_{1},e_{2},\cdots,e_{2n + 1}\right) $

\item Déterminer le rang de $t$, ainsi que la dimension du noyau de
$t$.

\item Justifier que $0$ est valeur propre de $t$. Déterminer la
dimension du sous-espace propre associé à la valeur propre $0$,
ainsi qu'une base de ce sous-espace.
\end{noliste}

\item Montrer que $\mathrm{Im}\left( t\circ t\right) \subset
\mathrm{Im}\left( t\right) $, où $\mathrm{Im}\left( u\right) $ désigne
l'image
d'un endomorphisme $u$ de $\R^{2n + 1}$

\item Soit $\tilde{t}$ l'endomorphisme défini sur $\mathrm{Im}\left(
t\right) $ par : pour tout $x$ de $\mathrm{Im}\left( t\right) $,
$\tilde{t}\left( x\right) = t\left( x\right) $\\
Établir que $\mathcal{B = }\left( e_{1},\ \Sum{i = 1}{2n +
1}e_{i}\right) $
constitue une base de $\mathrm{Im}\left( t\right) $. Écrire la matrice
associée à $\tilde{t}$ relativement à la base $\mathcal{B}$

\item 
\begin{noliste}{a)}
 \setlength{\itemsep}{2mm}
\item Soit $\lambda $ une valeur propre non nulle de $t$, et $x$ un
vecteur
propre associé à $\lambda $. Montrer que $x$ appartient à
$\mathrm{Im}\left( t\right) $.

\item En déduire toutes les valeurs propres de $t$. L'endomorphisme $t$
est-il diagonalisable ?
\end{noliste}
\end{noliste}

\section*{PROBL\`{E}ME}

Toutes les variables aléatoires qui interviennent dans ce problème
sont considérées comme définies sur des espaces probabilisés
non nécessairement identiques, mais qui, par souci de simplification,
seront tous notés $\left( \Omega,\ \mathcal{A},\ \mathrm{P}\right) $

\subsection*{Partie I}

\begin{noliste}{1.}
 \setlength{\itemsep}{4mm}
\item On considère la fonction $g$ définie sur $\R$ par : $g\left(
x\right) = \dfrac{1}{2}\times e^{-\left| x\right| }$

\begin{noliste}{a)}
 \setlength{\itemsep}{2mm}
\item Montrer que les intégrales $\dint{-\infty }{0}g\left( x\right\dx$
et $\dint{0}{+ \infty }g\left( x\right\dx$ sont convergentes et de même
valeur.

\item Établir que $g$ est une densité de probabilité sur $\R$.
\end{noliste}

\hspace{-1cm}Soit $Y$ une variable aléatoire à valeurs réelles
admettant $g$ pour densité. On dit que $Y$ suit la loi
$\mathcal{L}\left( 0\right) $.

\item Étudier les variations de $g$ et tracer l'allure de sa
représentation graphique dans le plan rapporté à un repère orthonormé.

\item 
\begin{noliste}{a)}
 \setlength{\itemsep}{2mm}
\item Montrer, pour tout $r$ de $\N$, l'existence du moment
$m_{r}\left( Y\right) $ d'ordre $r$ de la variable aléatoire $Y$.

\item Calculer, pour tout $r$ de $\N$, $m_{r}\left( Y\right) $ en
fonction de $r$. Quelles sont les valeurs de l'espérance $\E\left(
Y\right) $ et de la variance $\V\left( Y\right) $ de la variable
aléatoire $Y$ ?
\end{noliste}

\item 
\begin{noliste}{a)}
 \setlength{\itemsep}{2mm}
\item Déterminer la fonction de répartition $G$ de $Y$.

\item Établir que $G$ est une bijection de $\R$ sur $\left] 0;1\right[
$.

\item Montrer que l'équation $G\left( x\right) = \dfrac{1}{2}$ admet
une
unique solution que l'on déterminera.

\item Établir que la fonction qui, à tout réel $x$ associe $G\left(
x\right) \left( 1-G\left( x\right) \right) $, est paire.
\end{noliste}

\item 
\begin{noliste}{a)}
 \setlength{\itemsep}{2mm}
\item Montrer que l'application réciproque $G^{-1}$ de $G$ est définie
par :
\[
G^{-1}\left( x\right) = \left\{ 
\begin{array}{cc}
\ln \left( 2x\right) & \text{si }0<x\leq 1/2 \\
-\ln \left( 2\left( 1-x\right) \right) & \text{si }1/2\leq x<1
\end{array}
\right.
\]

\item Écrire une fonction \Scilab{} dont l'en-tête est \texttt{Laplace}
qui permet de simuler la loi $\mathcal{L}\left( 0\right) $. On rappelle
que
la fonction \texttt{random} permet de simuler en \Scilab{} une loi
uniforme sur 
$\left] 0;1\right[ $.
\end{noliste}

\item Pour tout entier $n$ de $\N^{\ast }$, on considère la
fonction $g_{n}$\ définie par :
\[
g_{n}\left( x\right) = g\left( x\right) \left( 1 + xe^{-n\left|
x\right| }\right)
\]
\\
Montrer que $g_{n}$ définit une densité de probabilité sur $\R$.

\hspace{-1cm}Pour tout $n$ de $\N^{\ast }$, on désigne par $Y_{n} $ une
variable aléatoire de densité $g_{n}$, et on note $G_{n}$
la fonction de répartition de $Y_{n}$.

\item 
\begin{noliste}{a)}
 \setlength{\itemsep}{2mm}
\item Établir pour tout réel $x$, la majoration suivante : $\left|
G_{n}\left( x\right) -G\left( x\right) \right| \leq \dfrac{1}{ne}\times
G\left( x\right) $

\item En déduire que la suite de variables aléatoires $\left(
Y_{n}\right)_{n\geq 1}$ converge en loi vers la loi $\mathcal{L}\left(
0\right) $
\end{noliste}
\end{noliste}

\subsection*{Partie II}

Soit $\theta $ un paramètre réel inconnu et $X$ une variable aléatoire
à densité. On dit que $X$ suit la loi $\mathcal{L}\left(
\theta \right) $, si une densité $f$ de $X$ est donnée par : pour
tout $x$ réel, $f\left( x\right) = \dfrac{e^{-\left| x-\theta
\right| }}{2}$

Soit $n$ un entier naturel. On considère un $\left( 2n + 1\right)
$-échantillon $\left( X_{1},X_{2},\cdots,X_{2n + 1}\right) $ de
variables aléatoires réelles indépendantes et de même loi
$\mathcal{L}\left(
\theta \right) $

\begin{noliste}{1.}
 \setlength{\itemsep}{4mm}
\item 
\begin{noliste}{a)}
 \setlength{\itemsep}{2mm}
\item Déterminer la fonction de répartition $F$ de la variable
aléatoire $X$.

\item En déduire que la variable aléatoire $\left( X-\theta \right) $
suit la loi $\mathcal{L}\left( 0\right) $ définie dans la partie I.

\item Calculer l'espérance et la variance de la variable aléatoire $X
$.

\item Résoudre l'équation $F\left( x\right) = 1/2$
\end{noliste}

\item Soit $x$ un réel fixé. Pour tout $i$ de $\left[ \ \left[ 1,2n +
1\right] \right] $, on note $Z_{i}$ la variable aléatoire de Bernoulli
telle que $\Prob\left(\Ev{ Z_{i} = 1}\right) = \mathrm{P}\left(
X_{i}\leq
x\right) $

\begin{noliste}{a)}
 \setlength{\itemsep}{2mm}
\item Établir l'indépendance des variables aléatoires $Z_{1},\
Z_{2},\dots,Z_{2n + 1}$

\item Soit $S_{2n + 1}$ la variable aléatoire définie par : $S_{2n + 1}
= \Sum{i = 1}{2n + 1}Z_{i}.$ Quelle est la loi de probabilité de $S_{2n
+ 1}$ ?

Préciser l'espérance et la variance de $S_{2n + 1}$.
\end{noliste}

\item On pose $\overline{X}_{2n + 1} = \dfrac{1}{2n + 1}\Sum{i = 1}{2n
+ 1}X_{i}$

\begin{noliste}{a)}
 \setlength{\itemsep}{2mm}
\item Montrer que $\overline{X}_{2n + 1}$ est un estimateur sans biais
du paramètre $\theta $

\item Calculer le risque quadratique de $\overline{X}_{2n + 1}$ en
$\theta.$
\end{noliste}
\end{noliste}

\subsection*{Partie III}

Le contexte de cette partie est identique à celui de la partie
précédente.

Pour tout $\omega $ de $\Omega $, on réordonne par ordre croissant les
réels $X_{1}\left( \omega \right),X_{2}\left( \omega
\right),\cdots,X_{2n + 1}\left( \omega \right) $, et on note
$\widehat{X}_{1}\left( \omega
\right),\ \widehat{X}_{2}\left( \omega \right),\cdots,\ \widehat{X}_{2n
+ 1}\left( \omega \right) $ les nombres ainsi rangés, c'est-à-dire que
$\widehat{X}_{1}\left( \omega \right) \leq \widehat{X}_{2}\left(
\omega \right) \leq \cdots \leq \widehat{X}_{2n + 1}\left( \omega
\right) $.
On définit ainsi $\left( 2n + 1\right) $ variables aléatoires
$\widehat{X}_{1},\ \widehat{X}_{2}\left( \omega \right),\cdots,\
\widehat{X}_{2n + 1}$, telles que $\widehat{X}_{1}\leq
\widehat{X}_{2}\leq \cdots \leq 
\widehat{X}_{2n + 1}$ qui constituent un réarrangement par ordre
croissant
des variables aléatoires $X_{1},\ X_{2},\cdots,\ X_{2n + 1}$. On admet
que $\Prob\left(\Ev{ \ \widehat{X}_{1}<\widehat{X}_{2}<\cdots
<\widehat{X}_{2n + 1}}\right) = 1$.

On s'intéresse dans cette partie à la variable aléatoire
$\widehat{X}_{n + 1}.$

\begin{noliste}{1.}
 \setlength{\itemsep}{4mm}
\item 
\begin{noliste}{a)}
 \setlength{\itemsep}{2mm}
\item Pour tout réel $x$, justifier l'égalité entre événements suivante
: $\left[ \ \widehat{X}_{n + 1}\leq x\right] = \left[
S_{2n + 1}\geq n + 1\right] $.

\item En déduire la fonction de répartition $\widehat{F}_{n + 1}$ de
$\widehat{X}_{n + 1}$ en fonction de $F$ (on exprimera cette fonction
sous
forme d'une somme que l'on ne cherchera pas à calculer).
\end{noliste}

\item On note $\widehat{f}_{n + 1}$ une densité de $\widehat{X}_{n +
1}$, et 
$\widehat{g}_{n + 1}$ une densité de $\left( \widehat{X}_{n + 1}-\theta
\right) $.

\begin{noliste}{a)}
 \setlength{\itemsep}{2mm}
\item Établir pour tout $j$ de $\left[ \ \left[ 0;2n\right] \right] $,
l'égalité suivante : $\left( j + 1\right) \binom{2n + 1}{j + 1} =
\left(
2n-j + 1\right) \binom{2n + 1}{j}$.

\item En déduire, pour tout $x$ réel, l'égalité :
\[
\widehat{f}_{n + 1}\left( x\right) = \frac{\left( 2n + 1\right)
!}{\left(
n!\right) ^{2}}\left( F\left( x\right) \right) ^{n}\left( 1-F\left(
x\right)
\right) ^{n}f\left( x\right)
\]

\item Établir, pour tout $x$ réel, l'égalité suivante :
\[
\widehat{g}_{n + 1}\left( x\right) = \frac{\left( 2n + 1\right)
!}{\left(
n!\right) ^{2}}\left( G\left( x\right) \right) ^{n}\left( 1-G\left(
x\right)
\right) ^{n}g\left( x\right)
\]
où $g$ et $G$ ont été définies dans la partie I.

\item En utilisant la question I.4.d), montrer que $\widehat{X}_{n +
1}$ est
un estimateur sans biais du paramètre $\theta $.
\end{noliste}

\item Dans cette question, on étudie le comportement de la suite
$\left( 
\widehat{X}_{n + 1}\right)_{n\in \N}$, lorsque $n$ tend vers $ + \infty

$.\\
On désigne par $\widehat{h}_{n + 1}$ une densité de la variable
aléatoire $\sqrt{2n + 1}\left( \widehat{X}_{n + 1}-\theta \right) $.

\begin{noliste}{a)}
 \setlength{\itemsep}{2mm}
\item Montrer que pour tout réel $x$, on a :
\[
\widehat{h}_{n + 1}\left( x\right) = \frac{1}{\sqrt{2n + 1}}\times
\frac{\left(
2n + 1\right) !}{\left( n!\right) ^{2}}\left( G\left( \frac{x}{\sqrt{2n
+ 1}}\right) \right) ^{n}\left( 1-G\left( \frac{x}{\sqrt{2n +
1}}\right) \right)
^{n}g\left( \frac{x}{\sqrt{2n + 1}}\right)
\]

\item Écrire, lorsque $u$ tend vers 0, le développement limité 
à l'ordre 2 de $e^{-u}$, et le développement limité à
l'ordre $1$ de $\ln \left( 1-u\right) $.

\item Soit $x$ un réel fixé. Montrer que lorsque $n$ tend vers $ +
\infty $, on a :
\[
G\left( \frac{x}{\sqrt{2n + 1}}\right) \left( 1-G\left(
\frac{x}{\sqrt{2n + 1}}\right) \right) = \frac{1}{4}\left(
1-\frac{x^{2}}{2n + 1} + o\left( \frac{1}{\left( 2n + 1\right) }\right)
\right)
\]

\item En déduire que : 
\[
\dlim{n\rightarrow + \infty }4^{n}\left[ G\left( \frac{x}{\sqrt{2n +
1}}\right)
\left( 1-G\left( \frac{x}{\sqrt{2n + 1}}\right) \right) \right]
^{n} = e^{-x^{2}/2}
\]

\item On admet que lorsque $n$ tend vers $ + \infty $, 
\[
\binom{2n}{n}\times \frac{1}{4^{n}}\sim \frac{1}{\sqrt{\pi n}}
\]
Montrer alors que pour tout réel $x$, on a :
\[
\dlim{n + \infty }\widehat{h}_{n + 1}\left( x\right) =
\frac{1}{\sqrt{2\pi }}e^{-x^{2}/2}
\]
\end{noliste}

\item On admet que le résultat de la question précédente entra\^{\i}ne
la convergence en loi de la suite de variables aléatoires $\sqrt{2n +
1}\left( \widehat{X}_{n + 1}-\theta \right) $ vers une variable
aléatoire qui suit la loi normale centrée réduite.\\
Montrer qu'un intervalle de confiance $\left[ I_{n};J_{n}\right] $ pour
le
paramètre $\theta $ au risque $\alpha = 0,05$, est donné par :
\[
I_{n} = \widehat{X}_{n + 1}-\frac{1,96}{\sqrt{2n + 1}}\text{\quad
et\quad }J_{n} = \widehat{X}_{n + 1} + \frac{1,96}{\sqrt{2n + 1}}
\]
\end{noliste}

(on rappelle que si $\Phi $ désigne la fonction de répartition de la
loi normale centrée réduite, on a $\Phi \left( 1,96\right) \simeq
0,975$ )

\end{document}

