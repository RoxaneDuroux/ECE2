\documentclass[11pt]{article}%
\usepackage{geometry}%
\geometry{a4paper,
 lmargin = 2cm,rmargin = 2cm,tmargin = 2.5cm,bmargin = 2.5cm}

\input{../../../../../../macros.tex}

\pagestyle{fancy} %
\lhead{ECE2 \hfill septembre 2017 \\
 Mathématiques\\[.2cm]} %
\chead{\hrule} %
\rhead{} %
\lfoot{} %
\cfoot{} %
\rfoot{\thepage} %

\renewcommand{\headrulewidth}{0pt}% : Trace un trait de séparation
 % de largeur 0,4 point. Mettre 0pt
 % pour supprimer le trait.

\renewcommand{\footrulewidth}{0.4pt}% : Trace un trait de séparation
 % de largeur 0,4 point. Mettre 0pt
 % pour supprimer le trait.

\setlength{\headheight}{14pt}

\title{\bf \vspace{-1cm} HEC 1994} %
\author{} %
\date{} %

\begin{document}

\maketitle %
\vspace{-1.2cm}\hrule %
\thispagestyle{fancy}

\vspace*{.4cm}

% DEBUT DU DOC À MODIFIER : tout virer jusqu'au début de l'exo

%Définition et changement de valeurs de
compteurs%newcounter{cpt1}{section} compteur cpt1 remis à 0 à chaque
aumentation par stepcounter du compteur section%setcounter{cpt1}{3} on
met le compteur à 3%addtocounter{cpt1}{5} on ajoute 5 au compteur%
stepcounter{cpt1} on ajoute 1% ifthenelse{test}{alors}{sinon} (page
206) pour subordonner à une condition % whiledo{test}{commande} pour
faire une boucle (page 206 aussi) % value{cpt1} pour noter dans le
document la valeur de cpt1 
%Définition définitive d'opérateurs
mathématiques\newcommand{\ch}{\operatorname{ch}} 
\newcommand{\sh}{\operatorname{sh}}
\renewcommand{\tanh}{\operatorname{th}}
\renewcommand{\sinh}{\operatorname{sh}}
\renewcommand{\cosh}{\operatorname{ch}}
\newcommand{\argsh}{\operatorname{argsh}}
\newcommand{\argch}{\operatorname{argch}}
\newcommand{\argth}{\operatorname{argth}}
\newcommand{\Id}{\operatorname{Id}}
\renewcommand{\leq}{\leq}
\renewcommand{\geq}{\geq }

\newcommand{\dlim}{\lim}
\newcommand{\dsum}{\sum}
\newcommand{\dprod}{\prod}



%Définition de nouvelles couleurs : rgb(trois paramètres red green blue
entre 0 et 1); cmyk (quatre cyan magenta yellow black) entre 0 et 1;
gray (entre 0 et 1) et black, white, red, green, blue, cyan, magenta,
yellow% definecolor{0gris}{gray}{0.8} 
% Nouvelle commande pour encadrer le titre car shabox ne veut que d'une
seule ligne; ATTENTION A LA TAILLE; petite différence avec shadowbox ou
doublebox, voire fcolorbox ou colorbox (au lieu de shabox; laisser le
parbox tranquille sauf pour la taille de la boîte
\newcommand{\Tbox}[1]{\begin{center} \shabox{\parbox{0.6
\linewidth}{#1}} \end{center}} %[1] pour 1 paramètre ; #1 pour ce que
fait le 1er paramètre; entre accolades ce que fait la commande
%Mise en page en mode fancy : en-têtes et pieds de pages puis
définition des en-têtes et pieds de pages\pagestyle{fancy}
\lhead{ECE 2 - Mathématiques \\
Quentin Dunstetter - ENC-Bessières 2011$\backslash$2012}
\chead{}
\rhead{HEC 1994}
\rfoot[ \ \thepage]{\thepage}
\cfoot{}
\lfoot{}
\thispagestyle{fancy} %Mise en page de la 1ère page en mode fancy
%Trait en bas et en haut de la page (entre en-tête et texte et texte et
pied de page)\renewcommand{\footrulewidth}{0.4pt}
\renewcommand{\headrulewidth}{0.4pt}

\begin{center}
{\huge HEC Eco 1994}
\end{center}

\section*{Exercice 1}

\begin{noliste}{1.}
 \setlength{\itemsep}{4mm}
\item Pour tout entier naturel $n$, on note $f_{n}$ la fonction définie
sur $[0,1]$ par 
\[
f_{n}(x) = \dint{0}{x}e^{nt^{2}}dt-\dint{x}{1}e^{-nt^{2}}dt
\]

\begin{noliste}{a)}
 \setlength{\itemsep}{2mm}
\item Montrer que $f_{n}$ est dérivable sur $[0,1]$.

\item Étudier le sens de variation de $f_{n}$.
\end{noliste}

\item Montrer qu'il existe un unique réel $c_{n}$ de $[0,1]$ tel que 
\[
\dint{0}{c_{n}}e^{nt^{2}}dt = \dint{c_{n}}{1}e^{-nt^{2}}dt
\]
Donner la valeur de $c_{0}$.

\item On considère la suite $(c_{n})_{n\geq 0}$ des nombres définis à
la question précédente; montrer qu'elle est croissante et qu'elle
converge
vers une limite $\ell $ appartenant à $[0,1]$.

\item 

\begin{noliste}{a)}
 \setlength{\itemsep}{2mm}
\item Montrer que, pour tout nombre réel fixé de $[0,1]$,
$\dlim{n\rightarrow + \infty }\dint{0}{r}e^{nt^{2}}dt = + \infty $.

\item Montrer que pour tout entier naturel $n$ on a
$\dint{c_{n}}{1}e^{-nt^{2}}dt\leq 1$.

\item En déduire la valeur de $\ell $.
\end{noliste}
\end{noliste}

\section*{Exercice 2}

On note $(e_{1},e_{2},e_{3})$ la base canonique de $\R^{3}$ et on
désigne par $\Gamma $ la partie de $\M_{3}(\R)$ constituée des
matrices dont les trois colonnes sont égales.

\begin{noliste}{1.}
 \setlength{\itemsep}{4mm}
\item Soit $T = \left( 
\begin{array}{ccc}
a & a & a \\
b & b & b \\
c & c & c
\end{array}
\right) $ un élément de $\Gamma $. On pose $s = a + b + c$ et on note
$u$
l'endomorphisme de $\R^{3}$ de matrice $T$ dans la base
$(e_{1},e_{2},e_{3})$.

\begin{noliste}{a)}
 \setlength{\itemsep}{2mm}
\item On suppose dans cette question que $s\neq 0$.

\begin{nonoliste}{(i)}
\item Montrer que les vecteurs 
\[
f_{1} = e_{1}-e_{2}\qquad f_{2} = e_{2}-e_{3}\qquad f_{3} = ae_{1} +
be_{2} + ce_{3}
\]
forment une base de $\R^{3}$.

\item Déterminer la matrice de $u$ dans cette base.

\item En déduire les valeurs propres de $T$.
\end{nonoliste}

\item On suppose dans cette question que $s = 0$.

\begin{nonoliste}{(i)}
\item Calculer $T^{2}$.

\item En déduire que $T$ admet 0 pour unique valeur propre.
\end{nonoliste}
\end{noliste}

\item Soit de plus $T^{\prime } = \left( 
\begin{array}{ccc}
a^{\prime } & a^{\prime } & a^{\prime } \\
b^{\prime } & b^{\prime } & b^{\prime } \\
c^{\prime } & c^{\prime } & c^{\prime }
\end{array}
\right) $ un autre élément de $\Gamma $ et $s^{\prime } = a^{\prime
} + b^{\prime } + c^{\prime }$.

\begin{noliste}{a)}
 \setlength{\itemsep}{2mm}
\item Exprimer le produit $TT^{\prime }$ en fonction de $T$.

\item A quelles conditions les matrices $TT^{\prime }$ et $T$ ont-elles
les mêmes valeurs propres ? (On pourra distinguer le cas $s = 0$ et
$s\neq 0$).
\end{noliste}

\item Soit $M$ un élément de $\M_{3}(\R)$ de la forme $M = T + I$
où $T$ appartient à $\Gamma $ et $I$ est la matrice unité de
$\M_{3}(\R)$. On suppose que $T$ n'est pas la matrice nulle.

\begin{noliste}{a)}
 \setlength{\itemsep}{2mm}
\item Pour tout couple $(\alpha,\beta )$ de nombres réels, exprimer le
produit $M(\alpha I + \beta M)$ comme combinaison linéaire de $I$ et
$M$.

\item Trouver les valeurs de $s$ pour lesquelles $M$ possède une
matrice
inverse la forme $(\alpha I + \beta M)$ et calculer cette inverse.\\
Pour les autres valeurs de $s$, la matrice $M$ est-elle inversible ?

\item Montrer que la matrice $N = \left( 
\begin{array}{rrr}
2 & 1 & 1 \\
0 & 1 & 0 \\
-1 & -1 & 0
\end{array}
\right) $ est inversible et calculer sont inverse.

\item Trouver les valeurs propres et les vecteurs propres de $N$.
\end{noliste}
\end{noliste}

\section*{Exercice 3}

\begin{noliste}{1.}
 \setlength{\itemsep}{4mm}
\item 

\begin{noliste}{a)}
 \setlength{\itemsep}{2mm}
\item Rappeler la valeur de la quantité ${\dfrac{1}{\sqrt{2\pi
}}}\dint{-\infty }{+ \infty }e^{-{\frac{x^{2}}{2}}}dx$ et en déduire la
valeur de $\dint{0}{+ \infty }e^{-{\frac{x^{2}}{2}}}dx$

\item Déterminer la constante $k$ telle que la fonction $f$ définie par

\[
f(x) = \left\{ 
\begin{array}{rrr}
ke^{-{\frac{t^{2}}{2}}} & \text{si} & x\geq 0 \\
0 & \text{si} & x<0
\end{array}
\right. 
\]
soit une densité de probabilité.
\end{noliste}

\item Un barrage alimente l'irrigation d'une région donnée. La quantité
d'eau de pluie tombée en un mois en amont du barrage est une variable
aléatoire $X$ qui suit une loi de densité $f$. On suppose que, dans un
système
d'unités convenablement choisi, la quantité d'eau fournie par le
barrage est 
égale a la variable aléatoire $Q = (1-ae^{-bX^{2}})$ où $b$ est un
nombre
strictement positif et $a$ un nombre tel que $0<a<1$.

\begin{noliste}{a)}
 \setlength{\itemsep}{2mm}
\item Calculer l'espérance $\E(X)$ et la variance $\V(X)$ de la
variable aléatoire $X$.

\item Étudier la fonction $g$ définie par $g(x) = (1-ae^{-bx^{2}})$.\\
Montrer qu'elle admet une fonction réciproque $g^{-1}$ dont on
précisera
l'ensemble de définition.\\
Expliciter la valeur $g^{-1}(y)$ en fonction de $y$.

\item Calculer l'espérance $\E(Q)$ et la variance $\V(Q)$ de la
variable $Q$
en fonction de $a$ et $b$.
\end{noliste}

\item Exprimer la probabilité $p$ que $Q$ soit supérieure à une valeur
fixée 
$\alpha $ à l'aide de la fonction de répartition $\Phi $ d'une loi
normale
centrée réduite, de la fonction $g^{-1}$ et de $\alpha $.
\end{noliste}

\label{fin}

\end{document}

