\documentclass[11pt]{article}%
\usepackage{geometry}%
\geometry{a4paper,
 lmargin = 2cm,rmargin = 2cm,tmargin = 2.5cm,bmargin = 2.5cm}

\input{../../../../../../macros.tex}

\pagestyle{fancy} %
\lhead{ECE2 \hfill septembre 2017 \\
 Mathématiques\\[.2cm]} %
\chead{\hrule} %
\rhead{} %
\lfoot{} %
\cfoot{} %
\rfoot{\thepage} %

\renewcommand{\headrulewidth}{0pt}% : Trace un trait de séparation
 % de largeur 0,4 point. Mettre 0pt
 % pour supprimer le trait.

\renewcommand{\footrulewidth}{0.4pt}% : Trace un trait de séparation
 % de largeur 0,4 point. Mettre 0pt
 % pour supprimer le trait.

\setlength{\headheight}{14pt}

\title{\bf \vspace{-1cm} HEC 2004} %
\author{} %
\date{} %

\begin{document}

\maketitle %
\vspace{-1.2cm}\hrule %
\thispagestyle{fancy}

\vspace*{.4cm}

% DEBUT DU DOC À MODIFIER : tout virer jusqu'au début de l'exo

%Définition et changement de valeurs de
compteurs%newcounter{cpt1}{section} compteur cpt1 remis à 0 à chaque
aumentation par stepcounter du compteur section%setcounter{cpt1}{3} on
met le compteur à 3%addtocounter{cpt1}{5} on ajoute 5 au compteur%
stepcounter{cpt1} on ajoute 1% ifthenelse{test}{alors}{sinon} (page
206) pour subordonner à une condition % whiledo{test}{commande} pour
faire une boucle (page 206 aussi) % value{cpt1} pour noter dans le
document la valeur de cpt1 
%Définition définitive d'opérateurs
mathématiques\newcommand{\ch}{\operatorname{ch}} 
\newcommand{\sh}{\operatorname{sh}}
\renewcommand{\tanh}{\operatorname{th}}
\renewcommand{\sinh}{\operatorname{sh}}
\renewcommand{\cosh}{\operatorname{ch}}
\newcommand{\argsh}{\operatorname{argsh}}
\newcommand{\argch}{\operatorname{argch}}
\newcommand{\argth}{\operatorname{argth}}
\newcommand{\Id}{\operatorname{Id}}
\renewcommand{\leq}{\leq}
\renewcommand{\geq}{\geq }

\newcommand{\dlim}{\lim}
\newcommand{\dsum}{\sum}
\newcommand{\dprod}{\prod}



%Définition de nouvelles couleurs : rgb(trois paramètres red green blue
entre 0 et 1); cmyk (quatre cyan magenta yellow black) entre 0 et 1;
gray (entre 0 et 1) et black, white, red, green, blue, cyan, magenta,
yellow% definecolor{0gris}{gray}{0.8} 
% Nouvelle commande pour encadrer le titre car shabox ne veut que d'une
seule ligne; ATTENTION A LA TAILLE; petite différence avec shadowbox ou
doublebox, voire fcolorbox ou colorbox (au lieu de shabox; laisser le
parbox tranquille sauf pour la taille de la boîte
\newcommand{\Tbox}[1]{\begin{center} \shabox{\parbox{0.6
\linewidth}{#1}} \end{center}} %[1] pour 1 paramètre ; #1 pour ce que
fait le 1er paramètre; entre accolades ce que fait la commande
%Mise en page en mode fancy : en-têtes et pieds de pages puis
définition des en-têtes et pieds de pages\pagestyle{fancy}
\lhead{ECE 2 - Mathématiques \\
Quentin Dunstetter - ENC-Bessières 2011$\backslash$2012}
\chead{}
\rhead{HEC 2004}
\rfoot[ \ \thepage]{\thepage}
\cfoot{}
\lfoot{}
\thispagestyle{fancy} %Mise en page de la 1ère page en mode fancy
%Trait en bas et en haut de la page (entre en-tête et texte et texte et
pied de page)\renewcommand{\footrulewidth}{0.4pt}
\renewcommand{\headrulewidth}{0.4pt}

\begin{center}
{\huge HEC Eco 2004}
\end{center}


\section*{EXERCICE}

\begin{noliste}{1.}
 \setlength{\itemsep}{4mm}
\item \textbf{Étude d'une suite et programmation}\\
On note $(c_{n})_{\,n\in \N^{\times }}$ la suite réelle définie pour
tout entier $n$ strictement positif par : 
\[
c_{n} = \dint{0}{1}\frac{x^{n-1}}{1 + x}dx
\]

\begin{noliste}{a)}
 \setlength{\itemsep}{2mm}
\item Montrer que $(c_{n})_{\,n\in \N^{\times }}$ est une suite
décroissante de réels positifs.

\item Montrer que, pour tout entier $n$ strictement positif, l'on a
:\quad $c_{n + 1} + c_{n} = \dfrac{1}{n}$

\item Établir, pour tout entier $n$ supérieur ou égal à $2$, la double
inégalité :
\[
\dfrac{1}{n}\leq 2c_{n}\leq \dfrac{1}{n-1}
\]
En déduire un équivalent simple de $c_{n}$ quand $n$ tend vers
l'infini.

\item Calculer $c_{1}$ et prouver, pour tout entier $n$ supérieur ou
égal à $2$, l'égalité :
\[
c_{n} = (-1)^{n}\left( \Sum{k = 1}{n-1}\dfrac{(-1)^{k + 1}}{k}-\ln
2\right)
\]

\item Écrire un programme en -\Scilab{} qui, pour une valeur d'un
entier $n $ strictement positif entrée par l'utilisateur, calcule et
affiche la
valeur de $c_{n}$.
\end{noliste}

\item \textbf{Étude d'une suite de variables aléatoires à densité}\\
Pour tout entier $n$ strictement positif, on note $f_{n}$ l'application
de $\R$ dans $\R$ définie par : 
\[
f_{n}(t) = \left\{ 
\begin{matrix}
0 & \text{si} & t<1 \\
\dfrac{1}{c_{n}\,t^{n}(1 + t)} & \text{si} & t\geq 1\end{matrix}\right.
\]

\begin{noliste}{a)}
 \setlength{\itemsep}{2mm}
\item À l'aide d'un changement de variable, établir pour tout entier
$n$
strictement positif et pour tout réel $x$ supérieur ou égal à $1$,
l'égalité :
\[
\dint{1}{x}\dfrac{1}{t^{n}(1 + t)}\,dt = \dint{1/x}{1}\dfrac{u^{n-1}}{1
+ u}\,du
\]

\item En déduire que, pour tout entier $n$ strictement positif, $f_{n}$
est
une densité de probabilité. Dans la suite de l'exercice, on suppose que
$(X_{n})_{\,n\in \N^{\times }}$ est une suite de variables aléatoires
définies sur le même espace probabilisé $\left(
\Omega,\mathcal{A},\mathbf{P}\right) $, telle que, pour tout entier $n$
strictement positif, $X_{n}$
prend ses valeurs dans $[1, + \infty \lbrack $ et admet $f_{n}$ comme
densité.
On note $F_{n}$ la fonction de répartition de $X_{n}$.

\item Pour quelles valeurs de $n$ la variable aléatoire $X_{n}$
admet-elle
une espérance ? Dans le cas où l'espérance de $X_{n}$ existe, calculer
cette
espérance en fonction de $c_{n}$ et de $c_{n-1}$.

\item Dans cette question, exclusivement, on suppose que $n$ est égal à
$1$.
Préciser la fonction $F_{1}$.\\
En déduire l'ensemble des réels $y$ vérifiant $\mathbf{P}([X_{1}\leq
y])\geq \dfrac{1}{2}$ Déterminer une densité de la variable aléatoire
$Z = \ln (X_{1})$.

\item Soit $x$ un réel strictement supérieur à $1$.\\
Justifier l'encadrement :
\[
0\leq \dint{1/x}{1}\dfrac{u^{n}}{(1 + u)^{2}}du\leq \dfrac{1}{n + 1}
\]
En déduire la limite suivante :
\[
\dlim{n\rightarrow + \infty }\left( \dint{1/x}{1}\dfrac{u^{n}}{(1 +
u)^{2}}du\right).
\]

Transformer, pour tout entier naturel $n$ non nul, $F_{n}(x)$ à l'aide
d'une
intégration par parties et en déduire l'égalité suivante :
\[
\dlim{n\rightarrow + \infty }F_{n}(x) = 1.
\]

\item Que vaut $\dlim{n\rightarrow + \infty }F_{n}(x)$ si $x$ est un
réel inférieur ou égal à $1$ ? \\
Montrer que la suite de variables aléatoires $(X_{n})_{\,n\in
\N^{\times }}$ converge en loi vers une variable que l'on précisera.
\end{noliste}
\end{noliste}

\section*{PROBLEME}

Dans ce problème, $n$ désigne un entier naturel non nul et $E$ désigne
l'espace vectoriel des polynômes à coefficients réels, de degré
inférieur ou 
égal à $2n$. \\
Pour tout entier naturel non nul $k$, on note $X^{k}$ le polynôme
$x\mapsto
x^{k}$ et on rappelle que la famille $(1,X,\ldots,X^{2n})$ est une base
de $E$. \\
Si $a_{0},a_{1},\ldots,a_{2n}$ sont $2n + 1$ réels et $Q$ est le
polynôme défini sur $\R$ par : 
\[
\ Q(x) = \Sum{k = 0}{2n}a_{k}x^{k},
\]
on définit le polynôme $s(Q)$ par :
\[
\ s(Q)(x) = \Sum{k = 0}{2n}a_{2n-k}x^{k}.
\]
\textit{Autrement dit, }$s(Q)$\textit{\ est le polynôme obtenu à partir
de }$Q$\textit{\ en "inversant l'ordre des coefficients".\\
Par exemple, si }$n$\textit{\ est égal à }$2$\textit{\ et si }$Q(x) =
4x^{4} + 7x^{3} + 2x^{2} + 1$\textit{, on obtient \ }$s(Q)(x) = x^{4} +
2x^{2} + 7x + 4$.\\
Les trois parties de ce problème sont largement indépendantes.

\subsection*{PARTIE A}

\begin{noliste}{1.}
 \setlength{\itemsep}{4mm}
\item \textbf{Linéarité de $s$}\\
Montrer que l'application $s$ : $Q\mapsto s(Q)$ est une application
linéaire
de $E$ dans lui-même.

\item \textbf{Diagonalisation dans un cas particulier}

\begin{noliste}{a)}
 \setlength{\itemsep}{2mm}
\item On considère la matrice carrée d'ordre 3 : \quad $M = $ $\left( 
\begin{array}{ccc}
0 & 0 & 1 \\
0 & 1 & 0 \\
1 & 0 & 0
\end{array}
\right) $.\\
Justifier sans calcul que la matrice $M$ est diagonalisable. Déterminer
les
valeurs propres de $M$ et, pour chacune d'entre elles, donner une base
du
sous-espace propre associé.

\item Vérifier que, dans le cas particulier $n = 1$, $M$ est la matrice
de
l'application linéaire $s$ dans la base $(1,X,X^{2})$. Donner alors une
base
de vecteurs propres pour $s$.
\end{noliste}

\item \textbf{Étude du cas général}\\
On définit la famille de polynômes $(A_{0},....,A_{2n})$ par : 
\[
\text{pour tout réel }x,\qquad \left\{ 
\begin{matrix}
A_{k}(x) = & x^{2n-k} + x^{k} & \text{si} & 0\leq k\leq {n-1} \\
A_{n}(x) = & x^{n} & & \\
A_{k}(x) = & x^{k}-x^{2n-k} & \text{si} & {n + 1}\leq k\leq
{2n}\end{matrix}\right.
\]

\begin{noliste}{a)}
 \setlength{\itemsep}{2mm}
\item Déterminer l'endomorphisme $s\circ s$.

\item Soit $P$ un polynôme non nul et $\lambda $ un réel vérifiant
$s(P) = \lambda P$.\\
Calculer $s\circ s(P)$ et en déduire que les valeurs propres de $s$
appartiennent à $\{1,-1\}$.

\item Déterminer $s(A_{k})$ pour tout entier $k$ vérifiant $0\leq
k\leq {2n}$.

\item Montrer que la famille $(A_{0},....,A_{2n})$ est libre.

\item En déduire que l'endomorphisme $s$ est diagonalisable, préciser
ses
valeurs propres et la dimension de chacun de ses sous-espaces propres.
\end{noliste}
\end{noliste}

\subsection*{PARTIE B}

\begin{noliste}{1.}
 \setlength{\itemsep}{4mm}
\item \textbf{Préliminaires}\\
On définit une suite $(R_{k})_{\,k\in \N^{\times }}$ de polynômes
par :\\
pour tout réel $x,$
\[
R_{1}(x) = x,\quad R_{2}(x) = x^{2}-2
\]
et pour tout entier $k$ supérieur ou égal à $2,$
\[
R_{k + 1}(x) = xR_{k}(x)-R_{k-1}(x)
\]

\begin{noliste}{a)}
 \setlength{\itemsep}{2mm}
\item Déterminer les polynômes $R_{3}$ et $R_{4}$.

\item Montrer que, pour tout entier $k$ strictement positif, $R_{k}$
est un
polynôme de degré $k$ vérifiant pour tout réel $x$ non nul, l'égalité :
\[
R_{k}\left( x + \dfrac{1}{x}\right) = x^{k} + \dfrac{1}{x^{k}}
\]

\item Pour tout réel $a$, déterminer, s'ils existent, les réels $x$ non
nuls
qui vérifient la relation suivante : $x + \dfrac{1}{x} = a$.
\end{noliste}

\item \textbf{Étude des racines des polynômes vecteurs propres de $s$
associés à la valeur propre $1$}\\
Dans cette question, $Q$ désigne un polynôme de degré $2n$ défini par :
$Q(x) = \Sum{k = 0}{2n}a_{k}x^{k}$, tel que $a_{2n}$ soit non nul et
tel
que, pour tout entier $k$ de l'intervalle $[ \
\hspace{-0.15em}[0,n]\hspace{-0.13em}]$, l'on ait : $a_{k} = a_{2n-k}$.
\\
On définit alors le polynôme $\widetilde{Q}$ par : 
\[
\ \widetilde{Q}(x) = a_{n} + \Sum{k = 1}{n}a_{n-k}R_{k}(x).
\]

\begin{noliste}{a)}
 \setlength{\itemsep}{2mm}
\item Vérifier que $0$ n'est pas racine de $Q$.

\item Soit $x$ un réel non nul, on pose : $y = x + \dfrac{1}{x}$. \\
Montrer que $\dfrac{Q(x)}{x^{n}}$ est nul si et seulement si
$\widetilde{Q}(y)$ est nul. \\
Quel est l'intérêt de ce résultat dans la recherche des racines de $Q$
?

\item On suppose que $n$ est égal à $3$ et que $Q$ est défini par :
\[
Q(x) = x^{6} + x^{5}-9x^{4} + 2x^{3}-9x^{2} + x + 1.
\]
Déterminer les racines de $Q$.
\end{noliste}
\end{noliste}

\subsection*{PARTIE C}

Dans cette partie, $p$ désigne un entier supérieur ou égal à $2$.\\
On désigne par $\Omega $ l'ensemble des éléments de $E$ dont les
coefficients sont des entiers de l'intervalle $[ \
\hspace{-0.15em}[1,p]\hspace{-0.13em}]$, par $\mathcal{A}$ l'ensemble
des parties de $\Omega $ et par $\mathbf{P}$ la probabilité uniforme
sur $\mathcal{A}$, c'est à dire que,
pour tout polynôme $Q$ de $\Omega $, l'on a :
\[
\mathbf{P}\left( \{Q\}\right) = \dfrac{1}{\operatorname{card}(\Omega )}
\]
Si $Q$ est un élément de $\Omega $ et $i$ un entier naturel non nul, on
dit
que $Q$ et $s(Q)$ présentent $i$ coïncidences lorsqu'il existe
exactement $i$
entiers $k$ qui vérifient $a_{k} = a_{2n-k}$. \\
On définit alors la variable aléatoire $Z$ qui, à tout polynôme $Q$ de
$\Omega $, associe le nombre de coïncidences entre $Q$ et $s(Q)$. \\
\textit{Par exemple pour }$n = 2$\textit{, si }$Q(x) = x^{4} + 7x^{3} +
2x^{2} + 5x + 1$\textit{, on a }$Z(Q) = 3$\textit{.}

\begin{noliste}{1.}
 \setlength{\itemsep}{4mm}
\item \textbf{Description d'un cas simple}\\
Dans cette question, on suppose que $n$ est égal à $1$ et que $p$ est
égal à 
$2$.\\
Écrire tous les éléments de $\Omega $ puis déterminer la loi,
l'espérance et
la variance de $Z$.

\item \textbf{Étude générale de la variable aléatoire $Z$}\\
On revient au cas général : $n$ est strictement positif et $p$ est
supérieur
ou égal à $2$.

\begin{noliste}{a)}
 \setlength{\itemsep}{2mm}
\item Calculer le cardinal de $\Omega $.

\item Montrer que la plus petite valeur que peut prendre $Z$ est 1 et
justifier l'égalité suivante :
\[
\mathbf{P}([Z = 1]) = \left( {\dfrac{p-1}{p}}\right) ^{n}
\]

\item Montrer que la plus grande valeur que peut prendre $Z$ est $2n +
1$ et
justifier l'égalité suivante :
\[
\mathbf{P}([Z = 2n + 1]) = \dfrac{1}{p^{n}}
\]

\item Montrer que $Z$ ne peut prendre que des valeurs impaires et, pour
un
entier $j$ vérifiant $0\leq j\leq n$, calculer $\mathbf{P}([Z = 2j +
1])$.

\item On pose $Y = \dfrac{Z-1}{2}.$ Montrer que $Y$ suit une loi
binomiale
dont on précisera les paramètres. \\
En déduire l'espérance et la variance de $Z$ en fonction de $n$ et de
$p$.
\end{noliste}
\end{noliste}

\label{fin}

\end{document}

