\documentclass[11pt]{article}%
\usepackage{geometry}%
\geometry{a4paper,
 lmargin = 2cm,rmargin = 2cm,tmargin = 2.5cm,bmargin = 2.5cm}

\input{../../../../../../macros.tex}

\pagestyle{fancy} %
\lhead{ECE2 \hfill septembre 2017 \\
 Mathématiques\\[.2cm]} %
\chead{\hrule} %
\rhead{} %
\lfoot{} %
\cfoot{} %
\rfoot{\thepage} %

\renewcommand{\headrulewidth}{0pt}% : Trace un trait de séparation
 % de largeur 0,4 point. Mettre 0pt
 % pour supprimer le trait.

\renewcommand{\footrulewidth}{0.4pt}% : Trace un trait de séparation
 % de largeur 0,4 point. Mettre 0pt
 % pour supprimer le trait.

\setlength{\headheight}{14pt}

\title{\bf \vspace{-1cm} HEC 2014} %
\author{} %
\date{} %

\begin{document}

\maketitle %
\vspace{-1.2cm}\hrule %
\thispagestyle{fancy}

\vspace*{.4cm}

% DEBUT DU DOC À MODIFIER : tout virer jusqu'au début de l'exo

%Définition et changement de valeurs de
compteurs%newcounter{cpt1}{section} compteur cpt1 remis à 0 à chaque
aumentation par stepcounter du compteur section%setcounter{cpt1}{3} on
met le compteur à 3%addtocounter{cpt1}{5} on ajoute 5 au compteur%
stepcounter{cpt1} on ajoute 1% ifthenelse{test}{alors}{sinon} (page
206) pour subordonner à une condition % whiledo{test}{commande} pour
faire une boucle (page 206 aussi) % value{cpt1} pour noter dans le
document la valeur de cpt1 
%Définition définitive d'opérateurs
mathématiques\newcommand{\ch}{\operatorname{ch}} 
\newcommand{\sh}{\operatorname{sh}}
\renewcommand{\tanh}{\operatorname{th}}
\renewcommand{\sinh}{\operatorname{sh}}
\renewcommand{\cosh}{\operatorname{ch}}
\newcommand{\argsh}{\operatorname{argsh}}
\newcommand{\argch}{\operatorname{argch}}
\newcommand{\argth}{\operatorname{argth}}
\newcommand{\Id}{\operatorname{Id}}
\newcommand{\id}{\operatorname{id}}
\renewcommand{\im}{\operatorname{Im}}
\renewcommand{\leq}{\leq}
\renewcommand{\geq}{\geq }

\newcommand{\dlim}{\lim}
\newcommand{\dsum}{\sum\limits}
\newcommand{\dprod}{\prod}
\newcommand{\lb}{\llbracket}
\newcommand{\rb}{\rrbracket}


%Définition de nouvelles couleurs : rgb(trois paramètres red green blue
entre 0 et 1); cmyk (quatre cyan magenta yellow black) entre 0 et 1;
gray (entre 0 et 1) et black, white, red, green, blue, cyan, magenta,
yellow% definecolor{0gris}{gray}{0.8} 
% Nouvelle commande pour encadrer le titre car shabox ne veut que d'une
seule ligne; ATTENTION A LA TAILLE; petite différence avec shadowbox ou
doublebox, voire fcolorbox ou colorbox (au lieu de shabox; laisser le
parbox tranquille sauf pour la taille de la boîte
\newcommand{\Tbox}[1]{\begin{center} \shabox{\parbox{0.8
\linewidth}{#1}} \end{center}} %[1] pour 1 paramètre ; #1 pour ce que
fait le 1er paramètre; entre accolades ce que fait la commande
%Mise en page en mode fancy : en-têtes et pieds de pages puis
définition des en-têtes et pieds de pages\pagestyle{fancy}
\lhead{ECE 2 - Mathématiques \\
Quentin Dunstetter - ENC-Bessières 2011$\backslash$2012}
\chead{}
\rhead{HEC 2014}
\rfoot[ \ \thepage]{\thepage}
\cfoot{}
\lfoot{}
\thispagestyle{fancy} %Mise en page de la 1ère page en mode fancy
%Trait en bas et en haut de la page (entre en-tête et texte et texte et
pied de page)\renewcommand{\footrulewidth}{0.4pt}
\renewcommand{\headrulewidth}{0.4pt}

\indent \vspace{0.3cm}

\Tbox{\begin{center} \textbf{\Huge HEC 2014} \end{center} }

\vspace{0.5cm}

\section*{Exercice}

\noindent On note $\M{3} $ l'espace vectoriel des matrices carrées
d'ordre 3 à coefficients réels. \\

\noindent On pose pour toute matrice $A = \left( a_{i,j}\right)_{1\leq
i,j\leq3}\in \M{3} $ : \\

\noindent $s_{1}\left( A\right) = \Sum{j = 1}{3}a_{1,j}~$,\quad
$s_{2}\left( A\right) = \Sum{j = 1}{3}a_{2,j}~$,\quad $s_{3}\left(
A\right) = \Sum{j = 1}{3}a_{3,j}$\quad\ (somme des coefficients des
lignes) \\

\noindent $s_{4}\left( A\right) = \Sum{i = 1}{3}a_{i,1}~$,\quad
$s_{5}\left( A\right) = \Sum{i = 1}{3}a_{i,2}~$,\quad $s_{6}\left(
A\right) = \Sum{i = 1}{3}a_{i,3}$\quad (somme des coefficients des
colonnes) \\

\noindent $s_{7}\left( A\right) = \Sum{i = 1}{3}a_{i,i}~$,\quad
$s_{8}\left( A\right) = \Sum{i = 1}{3}a_{i,4-i}$,\quad\ (somme des
coefficients des diagonales) \\

\noindent Pour tout couple $\left( k,l\right) \in \lb1,3\rb^{2} $, on
note $E_{k,l}$ la matrice de $\M{3} $ dont tous les coefficients sont
nuls excepté celui situé à l'intersection de la $k$-ième ligne et de la
$l$-ième colonne qui vaut 1. \\

\noindent On rappelle que la famille $\left(
E_{1,1},E_{1,2},E_{1,3},E_{2,1},E_{2,2},E_{2,3},E_{3,1},E_{3,2},E_{3,3}
\right) $ est une base de $\M{3} $ ; on note $\mathcal{B}$ cette base.

\begin{noliste}{1.}
 \setlength{\itemsep}{4mm}
\item Soit $\mathcal{E}$ l'ensemble des matrices $A\in\M{3} $ telles
que $s_{7}\left( A\right) = 0$.

\begin{noliste}{a)}
 \setlength{\itemsep}{2mm}
\item Montrer que $\mathcal{E}$\ est un sous-espace vectoriel de $\M{3}
$.

\item Quelle est la dimension de $\mathcal{E}$ ?
\end{noliste}

Soit $f$ l'application de $\M{3} $ dans $\R^{8}$ qui à toute matrice
$A$ fait correspondre le vecteur 
\\
$f\left( A\right) = \left( s_{1}\left( A\right),s_{2}\left(
A\right),s_{3}\left( A\right),s_{4}\left( A\right),s_{5}\left(
A\right),s_{6}\left( A\right),s_{7}\left( A\right),s_{8}\left( A\right)
\right) $
de $\R^{8}$

\item 

\begin{noliste}{a)}
 \setlength{\itemsep}{2mm}
\item Montrer que $f$ est une application linéaire.

\item On note $\mathcal{C}$ la base canonique de $\R^{8}$. Écrire la
matrice $F$ de $f$ dans les bases $\mathcal{B}$ et $\mathcal{C}$.
\end{noliste}

\item On note $\mathcal{G}$ l'ensemble des matrices $A\in\M{3} $ telles
que :
\[
s_{1}\left( A\right) = s_{2}\left( A\right) = s_{3}\left( A\right)
 = s_{4}\left( A\right) = s_{5}\left( A\right) = s_{6}\left( A\right)
 = s_{7}\left( A\right) = s_{8}\left( A\right)
\]

\begin{noliste}{a)}
 \setlength{\itemsep}{2mm}
\item Montrer que $\mathcal{G}$ est un sous-espace vectoriel de $\M{3}
$.

\item On note $\kerf$ le noyau de l'application linéaire $f$.
Montrer que $ \mathcal{G} \cap \mathcal{E} = \kerf $.

\item On note $J$ la matrice de $\M{3} $
dont tous les coefficients sont égaux à 1. Montrer que toute matrice
de $\mathcal{G}$ s'écrit de manière unique comme la somme d'une
matrice de $\kerf$ et d'une matrice de $ \Vect \left( J \right) $.

\item Quel est le rang de l'application $f$ ?

\item Déterminer la dimension de $\kerf$ ainsi qu'une base de $\kerf$.
\end{noliste}
\end{noliste}

\section*{Problème}

\begin{noliste}{$\sbullet$}
\item La fonction de répartition de la loi normale centrée réduite est
notée $\Phi$.

\item La notation $\exp$ désigne la fonction exponentielle.

\item Les trois parties du problème sont très largement indépendantes.
\end{noliste}



\textbf{Partie I. Un équivalent d'une intégrale}

\begin{noliste}{1.}
 \setlength{\itemsep}{4mm}
\item Soit $N$ la fonction définie sur l'intervalle $\left[ 0,1\right[
$, à valeurs réelles, telle que :
\[
 N\left( x\right) = x^{2}-2x-2\left( 1-x\right) \ln \left( 1-x\right).
\]

\begin{noliste}{a)}
 \setlength{\itemsep}{2mm}
\item Montrer que la fonction $N$ est de classe $\mathcal{C}{1}$ sur
$\left[
0,1\right[ $.

\item Montrer que pour tout $x\in\left[ 0,1\right[ $, on a : $\ln\left(
1-x\right) \leq-x$.

\item On note $N^{\prime}$ la fonction dérivée de la fonction $N$.
Montrer que pour tout $x\in\left[ 0,1\right[ $, on a $N^{\prime}\left(
x\right) \leq0$.

\item En déduire pour tout $x\in\left[ 0,1\right[ $, un encadrement de
$N(x)$.
\end{noliste}

\item Soit $f$ la fonction définie sur l'intervalle $\left] 0,1\right[
$, à valeurs réelles, telle que : $f\left( x\right) = -2\dfrac {x +
\ln\left( 1-x\right) }{x^{2}}$

\begin{noliste}{a)}
 \setlength{\itemsep}{2mm}
\item Rappeler le développement limité en 0 à l'ordre 2 de $\ln(1-x)$

\item Calculer $\dlim{x\rightarrow0}f\left( x\right) $. En déduire que
la fonction $f$ est prolongeable par continuité en $0$.

On note encore $f$ la fonction ainsi prolongée.

\item Sous réserve d'existence, on note $f^{\prime}$ la fonction
dérivée de $f$\\
Montrer que pour tout\ $x\in\left] 0,1\right[ $, on a :
$f^{\prime}\left(
x\right) = -2\dfrac{N\left( x\right) }{x^{3}\left( 1-x\right) }$.

\item Dresser le tableau de variation de la fonction $f$ sur $\left[
0,1\right[ $.\\
En déduire que $f$ réalise une bijection strictement croissante de
$\left[ 0,1\right[ $ dans $\left[ 1, + \infty \right[ $.
\end{noliste}

\item On pose pour tout $n\in\N^{\ast}$ et pour tout $x\in\left[
0,1\right[ $ : $g_{n}\left( x\right) = \exp\left(
-\frac{nx^{2}}{2}f\left(
x\right) \right) $

\begin{noliste}{a)}
 \setlength{\itemsep}{2mm}
\item Établir la convergence de l'intégrale $\dint{0}{1}g_{n}\left(
x\right\dx$. On pose alors pour tout n$\in\N^{\ast} :I_{n} = \int
_{0}{1}g_{n}\left( x\right\dx$.

\item Montrer que pour tout $x\in\left[ 0,1\right[ $, on a : $0\leq
g_{n}\left( x\right) \leq\exp\left( -\frac{nx^{2}}{2}\right) $.

\item En déduire l'encadrement : $0\leq
I_{n}\leq\sqrt{\frac{2\pi}{n}}\left( \Phi\left( \sqrt{n}\right)
-\frac{1}{2}\right) $

\item Montrer que pour tout $n\in\N^{\ast}$, on a : $0\leq
I_{n}\leq\sqrt{\frac{\pi}{2n}}$
\end{noliste}

\item \label{a}Soit $\left( v_{n}\right)_{n\in\N^{\ast}}$ la suite
réelle définie par : pour tout $n\in\N^{\ast}$, $v_{n} =
\frac{1}{\ln\left( n + 2\right) }$

\begin{noliste}{a)}
 \setlength{\itemsep}{2mm}
\item Montrer que pour tout $n\in\N^{\ast}$, on a : $0<v_{n}<1$.

\item On pose pour tout $n\in\N^{\ast}$ : $w_{n} = f\left(
v_{n}\right) $. Établir la convergence de la suite $\left( w_{n}\right)
_{n\in\N^{\ast}}$ ; déterminer sa limite.

\item Établir pour tout $n\in \N^{\ast }$ les inégalités
suivantes : 
\[
I_{n}\geq \dint{0}{v_{n}}g_{n}\left( x\right\dx\geq \dint{0}{v_{n}}\exp
\left( -\frac{nx^{2}}{2}w_{n}\right\dx\geq
\frac{1}{\sqrt{nw_{n}}}\dint{0}{v_{n}\sqrt{nw_{n}}}\exp \left(
-\frac{u^{2}}{2}\right) du
\]

\item Établir pour tout $n\in\N^{\ast}$, l'encadrement, :
$\frac{2}{\sqrt{w_{n}}}\left( \Phi\left( v_{n}\sqrt{nw_{n}}\right)
-\frac{1}{2}\right) \leq I_{n}\sqrt{\frac{2n}{\pi}}\leq1$

\item En déduire un équivalent de $I_{n}$ lorsque $n$ tend vers $ +
\infty $
\end{noliste}
\end{noliste}

\textbf{Partie II. Quelques propriétés asymptotiques de la loi de
Poisson}

\textit{Les notations sont identiques à celles de la Partie I.}

\begin{noliste}{1.}
 \setlength{\itemsep}{4mm}

\item \label{b}On pose pour tout réel $x>0$ et pour tout $n\in \N^{\ast
}$ : $J_{0}\left( x\right) = 1-e^{-x}$ et $J_{n}\left( x\right) =
\frac{1}{n!}\dint{0}{x}t^{n}e^{-t}dt$

\begin{noliste}{a)}
 \setlength{\itemsep}{2mm}
\item Calculer pour tout réel $x>0$, $J_{1}\left( x\right) $.

\item Établir pour tout réel $x>0$ et pour tout $n\in\N^{\ast }$, la
relation : $J_{n}\left( x\right) = J_{n-1}\left( x\right) -\frac
{1}{n!}x^{n}e^{-x}$.

\item En déduire pour tout réel $x>0$ et pour tout $n\in \N^{\ast}$,
une expression de $J_{n}\left( x\right) $ sous forme de somme.

\item Montrer que pour tout $n\in\N^{\ast}$, l'intégrale $\int
_{0}{+ \infty}t^{n}e^{-t}dt$ est convergente et calculer sa valeur.

\item À l'aide du changement de variable $t = n(1-x)$, montrer que pour
tout $n\in \N^{\ast }$, on a : 
\[
I_{n} = n!\frac{e^{n}}{n^{n + 1}}J_{n}\left( n\right) 
\]
\end{noliste}

Soit $\left( X_{n}\right)_{n\in\N^{\ast}}$ une suite de variables
aléatoires indépendantes, définies sur un espace probabilisé $\left(
\Omega,\mathcal{A},P\right) $, de même loi de Poisson de paramètre 1.
On pose pour tout $n\in\N^{\ast}$ : $S_{n} = \Sum{i = 1}{n}X_{i}$.

\item 

\begin{noliste}{a)}
 \setlength{\itemsep}{2mm}
\item Rappeler, sans démonstration mais en citant le résultat de
cours utilisé, la loi de la variable aléatoire $S_{n}$.

\item Exprimer pour tout $n\in \N^{\ast }$, $P\left(\Ev{S_{n}\leq
n]}\right)$ et $P\left(\Ev{S_{n}\geq n]}\right)$ en fonction de
$J_{n}\left( n\right) $ et $J_{n-1}\left(
n\right) $ respectivement.
\end{noliste}

\item \label{7}Pour tout $n\in \N^{\ast }$, on note $h_{n}$ la
fonction définie sur $\R_{+}$ à valeurs réelles, telle
que : $h_{n}\left( x\right) = x^{n}e^{-x}$.

\begin{noliste}{a)}
 \setlength{\itemsep}{2mm}
\item Étudier les variations de $h_{n}$ sur $\R_{+}$.

\item Établir pour tout $n\in\N^{\ast}$, la relation :
\[
P\left(\Ev{ S_{n + 1}\leq n + 1\right] }\right)-P\left(\Ev{S_{n}\leq
n]}\right) = -\frac{1}{\left(
n + 1\right) !}\dint{n}{n + 1}\left( h_{n + 1}\left( t\right) -h_{n +
1}\left(
n\right) \right\ dt
\]

\item En déduire que la suite $\left( P\left(\Ev{S_{n}\leq
n]}\right)\right)_{n\in\N^{\ast}}$ est décroissante.

\item Étudier la monotonie de la suite $\left( P\left(\Ev{S_{n}\geq
n]}\right)\right)
_{n\in\N^{\ast}}$.

\item Montrer que les deux suites $\left( P\left(\Ev{S_{n}\leq
n]}\right)\right)_{n\in\N^{\ast}}$ et $\left( P\left(\Ev{S_{n}\geq
n]}\right)\right)_{n\in \N^{\ast}}$ sont convergentes.
\end{noliste}

\item \label{8}

\begin{noliste}{a)}
 \setlength{\itemsep}{2mm}
\item Énoncer le théorème de la limite centrée et déterminer
$\dlim{n\rightarrow + \infty}P\left(\Ev{S_{n}\leq n]}\right)$

\item En déduire, à l'aide des questions \ref{a} et \ref{b}, un
équivalent de $n!$ lorsque $n$ tend vers $ + \infty$

\item \label{8c}Donner un équivalent et la limite de $P\left(\Ev{S_{n}
= n]}\right)$
lorsque $n$ tend vers $ + \infty$.

\item Déterminer $\dlim{n\rightarrow + \infty }P\left(\Ev{S_{n}\geq
n]}\right)$.
\end{noliste}
\end{noliste}

\textbf{Partie III, Médianes : cas des variables aléatoires discrètes
et des variables aléatoires à densité}

\noindent Soit $X$ une variable aléatoire réelle définie sur un espace
probabilisé $\left( \Omega,\mathcal{A},P\right) $, de fonction de
répartition $F$. On appelle médiane de $X$, tout réel $m$ vérifiant les
deux conditions : $P\left(\Ev{ X\leq m}\right) \geq \frac{1}{2}$ et
$P\left(\Ev{ X\geq m}\right) \geq\frac{1}{2}.$ \\

\noindent On admet qu'un tel réel $m$ existe toujours.

\begin{noliste}{1.}
 \setlength{\itemsep}{4mm}

\item On suppose que l'on a défini un entier $N$ supérieur ou égal à
$1$ et un type \texttt{type proba = array[l..N] of real}. Soit $X$
une variable aléatoire discrète à valeurs dans $\left[ \ \left[
1,N\right] \right] $. On suppose que la loi de $X$ est stockée dans une
variable \texttt{loi} de type \texttt{proba}.\\
Écrire une fonction \Scilab{} d'en-tête \texttt{function
mediane(loi :proba) :real;} qui renvoie une médiane de $X$.

\item Dans cette question, $X$ est une variable aléatoire discrète 
à valeurs dans $\N$ admettant une espérance $\E(X)$.

\begin{noliste}{a)}
 \setlength{\itemsep}{2mm}
\item \label{10a}Montrer que pour tout $r\in\N^{\ast}$, on a :
$\E\left( \left| X-r\right| \right) = E\left( X\right)
-r + 2\Sum{k = 0}{r-1}\left( r-k\right) P\left(\Ev{ X = k
}\right) $.

\item Montrer que : $\Sum{k = 0}{r-1}F\left( k\right) = \sum
\limits_{k = 0}{r-1}\left( r-k\right) P\left(\Ev{ X = k}\right) $.\\
En déduire que pour tout $r\in\N^{\ast}$, on a : $\E\left(
\left| X-r\right| \right) = E\left( X\right)
 + 2\Sum{k = 0}{r-1}\left( F\left( k\right) -\frac{1}{2}\right) $

\item Soit $m$ une médiane de $X$. On suppose que $m\in \N^{\ast
}$.\\
Déterminer, pour tout $r\in \N^{\ast }$, le signe de $\E\left(
\left| X-r\right| \right) -\E\left( \left| X-m\right| \right) 
$. Conclure.

\item On suppose que $X$ suit la loi de Poisson de paramètre $n$\
($n\in 
\N^{\ast }$).\\
En utilisant les questions \ref{7} et \ref{8}, justifier que $n$ est
une médiane de $X$.\\
En utilisant les questions \ref{10a} et \ref{8c}, montrer que $\E\left(
\left| X-n\right| \right) $ est équivalent à $\sqrt{\frac{2n}{\pi }}$
quand $n$ tend vers $ + \infty $.
\end{noliste}

\item Dans cette question, $X$ est une variable aléatoire à densité
dont une densité $f$ est continue sur $\R$.\\
On suppose que $X$ admet une espérance $\E(X)$. Soit $M$ la fonction de
$\R$ dans $\R$ définie par \\
$M(x) = E\left( \left| X-x\right| \right) $.

\begin{noliste}{a)}
 \setlength{\itemsep}{2mm}
\item Établir pour tout $x\geq0$ l'encadrement $0\leq x\left( 1-F\left(
x\right) \right) \leq\dint{x}{+ \infty}tf\left( t\right\ dt$\\
En déduire que $\dlim{x\rightarrow + \infty}x\left( 1-F\left(
x\right) \right) = 0$.\\
En considérant la variable aléatoire $-X$. montrer que
$\dlim{x\rightarrow-\infty}xF\left( x\right) = 0$.

\item Établir pour tout $x$ réel, la relation : $M\left( x\right)
 = \dint{-\infty}{x}F\left( t\right\ dt + \dint{x}{+ \infty}\left(
1-F\left(
t\right) \right\ dt$

\item Montrer que pour tout couple $\left( a,b\right) \in\R^{2}$, on
a : $M\left( b\right) -M\left( a\right) = \dint{a}{b}\left( 2F\left(
t\right) -1\right\ dt$.

\item On note $m$ une médiane de $X$. Montrer que $m$ est un point en
lequel la fonction $M$ atteint son minimum.
\end{noliste}
\end{noliste}

\end{document}

