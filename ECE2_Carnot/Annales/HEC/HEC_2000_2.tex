\documentclass[11pt]{article}%
\usepackage{geometry}%
\geometry{a4paper,
  lmargin=2cm,rmargin=2cm,tmargin=2.5cm,bmargin=2.5cm}

\input{../../../../macros.tex}
%\input{../../../../../../macros.tex}

\pagestyle{fancy} %
\lhead{ECE2 \\
  Mathématiques\\[.2cm]
  \hrule} %
\chead{} %
\rhead{} %
\lfoot{} %
\cfoot{} %
\rfoot{\thepage} %

\renewcommand{\headrulewidth}{0pt}% : Trace un trait de séparation
                                    % de largeur 0,4 point. Mettre 0pt
                                    % pour supprimer le trait.

\renewcommand{\footrulewidth}{0.4pt}% : Trace un trait de séparation
                                    % de largeur 0,4 point. Mettre 0pt
                                    % pour supprimer le trait.

\setlength{\headheight}{14pt}

\title{\bf \vspace{-1cm} HEC 2000 II} %
\author{} %
\date{} %
\begin{document}

\maketitle %
\vspace{-1.2cm}\hrule %
\thispagestyle{fancy}

\vspace*{.4cm}

\noindent
Ce problème se compose de cinq parties : il étudie deux suites de 
variables aléatoires discrètes et une simulation informatique. Si le 
candidat ne parvient pas à établir un résultat demandé, il l'indiquera 
clairement, et il pourra pour la suite, admettre ce résultat.\\
Dans tout le problème, $n$ désigne un entier naturel non nul.\\
On considère une urne $U_n$ contenant $n$ boules numérotées de $1$ à 
$n$. On tire une boule au hasard dans $U_n$. On note $k$ le numéro de 
cette boule. Si $k$ est égal à $1$, on arrête les tirages. Si $k$ est 
supérieur ou égal à $2$, on enlève de l'urne $U_n$ les boules 
numérotées de $k$ à $n$ (il reste donc les boules numérotées de $1$ à 
$k-1$), et on effectue à nouveau un tirage dans l'urne. On répète ces 
tirages nécessaires pour l'obtention de la boule numéro $1$. On note 
$Y_n$ la variable aléatoire égale au nombre de tirages nécessaires pour 
l'obtention de la boule numéro $1$. On note $Y_n$ la variable aléatoire 
égale à la somme des numéros des boules tirées. On note $\E(X_n)$ et 
$\V(X_n)$ (respectivement $\E(Y_n)$ et $\V(Y_n)$) l'espérance et la 
variance de $X_n$ (respectivement $Y_n$).

\section*{Partie I}

\begin{noliste}{1.}
\item On pose : $h_{n}=\DSum{k=1}{n}\dfrac{1}{k}=1+\dfrac{1}{2}%
+.....+\dfrac{1}{n}$

\begin{noliste}{a)}
\item Montrer, pour tout entier naturel $k$ non nul, les inégalités : 
\begin{equation*}
\dfrac{1}{k+1}\leqslant \ln (k+1)-\ln k\leqslant \dfrac{1}{k}
\end{equation*}%
où $\ln $ désigne le logarithme népérien.

\item En déduire les inégalités : $\ln (n+1)\leqslant h_{n}\leqslant 
1+\ln n$

\item Déterminer un équivalent simple de $h_{n}$ quand $n$ tend vers
l'infini.
\end{noliste}

\item On pose : 
$k_{n}=\DSum{k=1}{n}\dfrac{1}{k^{2}}=1+\dfrac{1}{%
2^{2}}+.....+\dfrac{1}{n^{2}}$

\begin{noliste}{a)}
\item Montrer, pour tout entier $k$ supérieur ou égal à $2$, 
l'inégalité 
\[
\dfrac{1}{k^{2}}\leq \dfrac{1}{k-1}-\dfrac{1}{k}
\]

\item En déduire la majoration $k_{n}\leq 2$

\item Déterminer un équivalent simple de $h_{n}-k_{n}$ quand n tend vers
l'infini.
\end{noliste}
\end{noliste}

\section*{Partie II : Étude de la variable aléatoire $X_{n}$}

\noindent On note $I_{n}$ la variable aléatoire égale au numéro de la 
premiè%
re boule tirée dans l'urne $U_{n}$.

\begin{noliste}{1.}
\item

\begin{noliste}{a)}
\item Quelle est la loi de $I_{n}$?

\item Quelle est la loi conditionnelle de $X_{n}$ sachant 
$\Ev{I_{n}=1}$ ?

\item Si $n$ est supérieur ou égal à $2$, montrer :%
\[
\forall j\in \mathbb{N}^*, \ \forall k\in \left\{ 1,2\ 
,...,n\right\}
,\quad \Prob_{\Ev{I_n=k}}\left( \Ev{X_{n}=j}\right) =\Prob\left( 
\Ev{X_{k-1}=j-1}\right)
\]
\end{noliste}

\item

\begin{noliste}{a)}
\item Quelle est la loi de $X_{1}$ ?

\item Quel est l'événement $\Ev{X_{2}=1}$ ? Donner la loi de $X_{2}$ , 
son espérance et sa variance.

\item Calculer $\Prob_{\Ev{I_3=1}}(\Ev{X_{3}=2})$, 
$\Prob_{\Ev{I_3=2}}(\Ev{X_{3}=2})$, $%
\Prob_{\Ev{I_3=3}}(\Ev{X_{3}=2})$ . Déterminer la loi de 
$X_{3}$, son espérance et sa
variance.
\end{noliste}

\item 
\begin{noliste}{a)}
\item Montrer que $X_{n}$ prend ses valeurs dans ${\{}1,2,...,n{\}}$.

\item Déterminer $\Prob(\Ev{X_{n}=1})$ et $\Prob(\Ev{X_{n}=2})$

\item Si $n$ est supérieur ou égal à $2$, montrer la relation :%
\[
\forall j\geq 2,\quad \Prob\left(\Ev{X_{n}=j}\right) =\dfrac{1}{n}%
\DSum{k=1}{n-1} \Prob\left( \Ev{X_{k}=j-1}\right)
\]

\item Si $n$ est supérieur ou égal à $3$ et 
$j$
supérieur ou égal à $2$, calculer : 
\[
n \ \Prob\left( \Ev{X_{n}=j}\right) 
-(n-1) \Prob\left(
\Ev{X_{n-1}=j}\right)
\]
En déduire, si $n$ est un entier supérieur ou égal à $2$ :%
\[
\forall j\geq 1, \quad \Prob\left( \Ev{X_{n}=j}\right) 
=\dfrac{n-1}{n} \Prob\left(
\Ev{X_{n-1}=j}\right) +\dfrac{1}{n} \Prob\left(\Ev{X_{n-1}=j-1}\right) 
\]
\end{noliste}

\item
\begin{noliste}{a)}
\item Si $n$ est supérieur ou égal à $2$, montrer, en utilisant la 
question \itbf{3.d)}. :%
\[
\E\left( X_{n}\right) =\E\left( X_{n-1}\right) +\dfrac{1}{n}
\]

\item En déduire $\E(X_{n})$ et donner un équivalent simple de 
$\E(X_{n})$
quand $n$ tend vers l'infini.
\end{noliste}

\item 
\begin{noliste}{a)}
\item Si $n$ est supérieur ou égal à $2$, calculer $\E(X_{n}^{2})$ en
fonction de $\E(X_{n-1}^{2})$ et de $\E(X_{n-1})$.

\item En déduire: $\V(X_{n})=h_{n}{-}k_{n}$ (en reprenant les notations
introduites en \textbf{Partie I}).

\item Donner un équivalent de $\V(X_{n})$ quand $n$ tend vers l'infini.
\end{noliste}

\item Soit $\left( T_{i}\right) _{i\geqslant 1}$une suite de variables 
alé%
atoires indépendantes telle que, pour tout i entier naturel non nul, 
$T_{i}$
suit la loi de Bernoulli de paramètre $\dfrac{1}{i}$. On pose : 
\[
S_{n}=\sum\limits_{i=1}^{n}T_{i}=T_{1}+.....+T_{n}
\]

\begin{noliste}{a)}
\item Vérifier que $X_{1}$ et $T_{1}$ ont même loi.

\item Si $n$ est supérieur ou égal à $2$, montrer, pour tout entier $j$ 
non nul :%
\[
\Prob\left(\Ev{S_{n}=j}\right) =\dfrac{1}{n}\Prob\left( 
\Ev{S_{n-1}=j-1}\right) +\dfrac{n-1}{n%
} \Prob\left(\Ev{S_{n-1}=j}\right)
\]
En déduire que $X_{n}$ et $S_{n}$ ont même loi.

\item Retrouver ainsi $\E(X_{n})$ et $\V(X_{n})$.
\end{noliste}
\end{noliste}

\section*{Partie III : Étude de la variable aléatoire $Y_{n}$.}

\begin{noliste}{1.}
\item Donner la loi de $Y_{n}$.

\begin{noliste}{a)}
\item Quelles sont les valeurs prises par $Y_{2}$ ?

\item Déterminer la loi de $Y_{2}$.
\end{noliste}

\item 
\begin{noliste}{a)}
\item Si $n$ est supérieur ou égal à $2$, montrer, pour tout entier $j$ 
non nul et tout entier $k$ supérieur ou égal à $2$%
\[
\Prob_{\Ev{I_n=k}}\left(\Ev{Y_{n}=j}\right) = \Prob\left( 
\Ev{Y_{k-1}=j-k}\right)
\]

\item Si $n$ est supérieur ou égal à $2$, en déduire, pour tout entier 
$j$
supérieur ou égal à $1$%
\[
\Prob\left(\Ev{Y_{n}=j}\right) =\dfrac{n-1}{n}\Prob\left( 
\Ev{Y_{n-1}=j}\right) +\dfrac{1}{n}%
\Prob\left(\Ev{Y_{n-1}=j-n}\right)
\]

\item Si $n$ est supérieur ou égal à $2$, montrer 
$\E(Y_{n})=\E(Y_{n-1})+1$\\
Que vaut $\E(Y_{n})$ pour tout entier $n$ supérieur ou égal à $1$ ?
\end{noliste}
\end{noliste}

\section*{Partie IV : Simulation informatique.}

\noindent Dans le langage informatique \Scilab{}, la fonction 
\texttt{grand(1,1,\ttq{}uin\ttq{},1,n)}
renvoie un entier aléatoire compris entre $1$ et $n$. On donne la 
procé%
dure suivante

\begin{scilab}
  & n = input(\ttq{}Entrer un entier naturel : \ttq{}) \nl %
  & a = 1 \nl %
  & b = 1 \nl %
  & alea = grand(1,1,\ttq{}uin\ttq{},1,n) \nl %
  & \tcFor{while} alea > 1 \nl %
  & \qquad a = a + 1 \nl %
  & \qquad b = b + alea \nl %
  & \qquad alea = grand(1,1,\ttq{}uin\ttq{},1,(alea - 1)) \nl %
  & \tcFor{end} \nl %
  & disp(a) \nl %
  & disp(b)
\end{scilab}

\noindent Que fait ce programme ? Que représentent a et b ?

\section*{Partie V}
\noindent
On considère l'urne $U_{n}$ contenant $n$ boules numérotées entre $1$ et 
$n$%
. A partir de l'urne $U_{n}$ on effectue la suite de tirages décrite 
dans
l'en{\-}tête du problème. Pour $i$ entier de ${\{}1,...,n{\}}$ , on 
définit $%
Z_{i}^{(n)}$ la variable aléatoire égal à $1$ si, lors d'un quelconque 
de
ces tirages, on a obtenu la boule numéro $i$, égale à $0$ sinon.

\begin{noliste}{1.}
\item Quelle est la loi de $Z_{n}^{(n)}$ ? Que dire de la variable $%
Z_{1}^{(n)}$ ?

\item 
\begin{noliste}{a)}
\item Si $n$ est supérieur ou égal à $2$, et $i$ un entier de 
${\{}1,...,n{-}1{%
\}}$, montrer la relation%
\[
\Prob\left(\Ev{Z_{i}^{(n)}=1}\right) 
=\dfrac{1}{n}+\DSum{k=i+1}{n}\dfrac{1}{n%
}\Prob\left(\Ev{Z_{i}^{(k-1)}=1}\right)
\]

\item Montrer par récurrence que, pour tout $n$ de $\mathbb{N}^*$ 
et
pour tout $i$ de ${\{}1,...,n{\}}$, $Z_{i}^{(n)}$ suit la loi de 
Bernoulli
de paramètre $\dfrac{1}{i}$.
\end{noliste}

\item Que vaut $\DSum{i=1}{n}Z_{i}^{(n)}$ ? Retrouver ainsi $%
\E(X_{n}) $.

\item Retrouver $\E(Y_{n})$.
\end{noliste}




\end{document}
