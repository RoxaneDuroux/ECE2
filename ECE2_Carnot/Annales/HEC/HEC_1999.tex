\documentclass[11pt]{article}%
\usepackage{geometry}%
\geometry{a4paper,
 lmargin = 2cm,rmargin = 2cm,tmargin = 2.5cm,bmargin = 2.5cm}

\input{../../../../../../macros.tex}

\pagestyle{fancy} %
\lhead{ECE2 \hfill septembre 2017 \\
 Mathématiques\\[.2cm]} %
\chead{\hrule} %
\rhead{} %
\lfoot{} %
\cfoot{} %
\rfoot{\thepage} %

\renewcommand{\headrulewidth}{0pt}% : Trace un trait de séparation
 % de largeur 0,4 point. Mettre 0pt
 % pour supprimer le trait.

\renewcommand{\footrulewidth}{0.4pt}% : Trace un trait de séparation
 % de largeur 0,4 point. Mettre 0pt
 % pour supprimer le trait.

\setlength{\headheight}{14pt}

\title{\bf \vspace{-1cm} HEC 1999} %
\author{} %
\date{} %

\begin{document}

\maketitle %
\vspace{-1.2cm}\hrule %
\thispagestyle{fancy}

\vspace*{.4cm}

% DEBUT DU DOC À MODIFIER : tout virer jusqu'au début de l'exo

%Définition et changement de valeurs de
compteurs%newcounter{cpt1}{section} compteur cpt1 remis à 0 à chaque
aumentation par stepcounter du compteur section%setcounter{cpt1}{3} on
met le compteur à 3%addtocounter{cpt1}{5} on ajoute 5 au compteur%
stepcounter{cpt1} on ajoute 1% ifthenelse{test}{alors}{sinon} (page
206) pour subordonner à une condition % whiledo{test}{commande} pour
faire une boucle (page 206 aussi) % value{cpt1} pour noter dans le
document la valeur de cpt1 
%Définition définitive d'opérateurs
mathématiques\newcommand{\ch}{\operatorname{ch}} 
\newcommand{\sh}{\operatorname{sh}}
\renewcommand{\tanh}{\operatorname{th}}
\renewcommand{\sinh}{\operatorname{sh}}
\renewcommand{\cosh}{\operatorname{ch}}
\newcommand{\argsh}{\operatorname{argsh}}
\newcommand{\argch}{\operatorname{argch}}
\newcommand{\argth}{\operatorname{argth}}
\newcommand{\Id}{\operatorname{Id}}
\renewcommand{\leq}{\leq}
\renewcommand{\geq}{\geq }

\newcommand{\dlim}{\lim}
\newcommand{\dsum}{\sum}
\newcommand{\dprod}{\prod}



%Définition de nouvelles couleurs : rgb(trois paramètres red green blue
entre 0 et 1); cmyk (quatre cyan magenta yellow black) entre 0 et 1;
gray (entre 0 et 1) et black, white, red, green, blue, cyan, magenta,
yellow% definecolor{0gris}{gray}{0.8} 
% Nouvelle commande pour encadrer le titre car shabox ne veut que d'une
seule ligne; ATTENTION A LA TAILLE; petite différence avec shadowbox ou
doublebox, voire fcolorbox ou colorbox (au lieu de shabox; laisser le
parbox tranquille sauf pour la taille de la boîte
\newcommand{\Tbox}[1]{\begin{center} \shabox{\parbox{0.6
\linewidth}{#1}} \end{center}} %[1] pour 1 paramètre ; #1 pour ce que
fait le 1er paramètre; entre accolades ce que fait la commande
%Mise en page en mode fancy : en-têtes et pieds de pages puis
définition des en-têtes et pieds de pages\pagestyle{fancy}
\lhead{ECE 2 - Mathématiques \\
Quentin Dunstetter - ENC-Bessières 2011$\backslash$2012}
\chead{}
\rhead{HEC 1999}
\rfoot[ \ \thepage]{\thepage}
\cfoot{}
\lfoot{}
\thispagestyle{fancy} %Mise en page de la 1ère page en mode fancy
%Trait en bas et en haut de la page (entre en-tête et texte et texte et
pied de page)\renewcommand{\footrulewidth}{0.4pt}
\renewcommand{\headrulewidth}{0.4pt}

\begin{center}
{\huge HEC Eco 1999}
\end{center}

\section*{Exercice 1}

On note $\mathcal{L}(\R^{4})$ l'ensemble des endomorphismes de
l'espace vectoriel $\R^{4}$, $\Id$ l'endomorphisme identique
de $\R^{4}$ et $I$ la matrice identité de $\mathfrak{M}_{4}(\R)$. On
considère les matrices :
\[
L = \left(
\begin{array}{rrrr}
0 & 1 & 0 & 0 \\
1 & 0 & 0 & 0 \\
0 & 0 & 0 & 1 \\
0 & 0 & 1 & 0
\end{array}
\right),\qquad \qquad M = \left(
\begin{array}{rrrr}
0 & 0 & 1 & 0 \\
0 & 0 & 0 & 1 \\
1 & 0 & 0 & 0 \\
0 & 1 & 0 & 0
\end{array}
\right)
\]
On désigne respectivement par $\varphi $ et $\psi $ les endomorphismes
de $\R^{4}$ représentés par $L$ et $M$ dans la base canonique
$\mathcal{E} = (e_{1},e_{2},e_{3},e_{4})$ de $\R^{4}$.

\begin{noliste}{1.}
 \setlength{\itemsep}{4mm}
\item 

\begin{noliste}{a)}
 \setlength{\itemsep}{2mm}
\item Montrer que $\varphi $ et $\psi $ sont des automorphismes de
$\R^{4}$ et en déterminer les automorphismes réciproques.

\item Déterminer les valeurs propres et les sous espaces propres
associés de
l'endomorphisme $\varphi $.\\
Déterminer de même les valeurs propres et les sous espaces propres
associés
de $\psi $.

\item Montrer que l'on peut trouver un vecteur $f_{1}$ non nul de
$\R^{4}$ vérifiant $\varphi (f_{1}) = f_{1}$ et $\psi (f_{1}) = f_{1}$.

\item Déterminer, plus généralement, une base $\mathcal{F} =
(f_{1},f_{2},f_{3},f_{4})$ de $\R^{4}$ dont chaque vecteur est à la
fois un vecteur propre de $\varphi $ et un vecteur propre de $\psi $.
Donner
la matrice $L^{\prime }$ de $\varphi $ et la matrice $M^{\prime }$ de
$\psi $
dans cette base $\mathcal{F}$.
\end{noliste}

\item On se propose d'étudier l'ensemble $C$ des endomorphismes $\gamma
$ de
$\R^{4}$ vérifiant $\gamma \circ \varphi = \varphi \circ \gamma $ et
$\gamma \circ \psi = \psi \circ \gamma $.

\begin{noliste}{a)}
 \setlength{\itemsep}{2mm}
\item Montrer que $C$ est un sous espace vectoriel de
$\mathcal{L}(\R^{4})$ qui contient $\varphi $ et $\psi $.

\item Montrer que si $\gamma \in C$ et $\gamma ^{\prime }\in C$, alors
$\gamma \circ \gamma ^{\prime }\in C$.

\item Soit $\gamma \in \mathcal{L}(\R^{4})$ un endomorphisme de
$\R^{4}$ et soit $G$ la matrice de $\gamma $ dans la base $\mathcal{F}
$, constituée de vecteurs propres de $\varphi $ et $\psi $, déterminée
à la
question 1.d). Montrer que $\gamma \in C$ si et seulement si $G$ est
une
matrice diagonale.

\item En déduire que $C$ est un sous espace vectoriel de dimension 4 de
$\mathcal{L}(\R^{4})$ et que les en domorphismes $Id$, $\varphi $,
$\psi $ et $\varphi \circ \psi $ forment une base de $C$.
\end{noliste}
\end{noliste}

\section*{Exercice 2}

On considère un entier naturel $N$ supérieur ou égal à 3, et on note
$\{1,2,\ldots,N\}$ l'ensemble des entiers strictement positifs,
inférieurs
ou égaux à $N$.\\
Une urne contient $N$ boules numérotées de $1$ à $N$. On y effectue des
tirages successifs d'une boule avec remise de la boule tirée après
chaque
tirage, jusqu'à obtenir pour la première fois un numéro déjà tiré. On
note
alors $T_{N}$ le rang aléatoire de ce dernier tirage.\\
C'est ainsi que, si on a obtenu successivement les numéros
-1-5-4-7-3-5-, la
variable $T_{N}$ prend la valeur 6, alors que si l'on a obtenu
-5-4-2-2- la
variable $T_{N}$ prend la valeur 4.\\
On admet qu'on définit ainsi une variable aléatoire sur un espace
probabilisé, dont la probabilité est notée $\mathbf{P}$. Toutes les
variables aléatoires introduites dans le problème seront supposées
définies sur cet
espace. Si $Z$ est une telle variable, son espérance sera notée E($Z$)
et sa
variance V($Z$).\\
N.B. \textit{Les parties II et III sont indépendantes.}

\subsection*{I. Étude de la variable aléatoire $T_{N}$}

\begin{noliste}{1.}
 \setlength{\itemsep}{4mm}
\item Dans cette question, on se place dans le cas particulier où
l'entier $N $ est égal à 3.\\
Déterminer la loi de $T_{3}$ et calculer son espérance et sa variance.

\item On revient désormais au cas général où $N$ est supérieur ou égal
à 3.

\begin{noliste}{a)}
 \setlength{\itemsep}{2mm}
\item Déterminer l'ensemble des valeurs que peut prendre $T_{N}$.

\item Calculer $P\left(\Ev{T_{N} = 2}\right)$, $P\left(\Ev{T_{N} =
3}\right)$, et $P\left(\Ev{T_{N} = N + 1}\right)$.

\item Prouver, pour tout entier $k$ de $\{1,2,\ldots,N\}$, les égalités
\[
P\left(\Ev{T_{N}>k}\right) = \dfrac{N!}{(N-k)!N^{k}} = \prod_{i =
0}{k-1}(1-\dfrac{i}{N})
\]
En déduire la loi de la variable aléatoire $T_{N}$.

\item Déterminer, pour tout entier $k$ fixé, la limite
$\dlim{N\rightarrow + \infty }P\left(\Ev{T_{n}>k}\right)$.\\
Pouvait-on prévoir ce résultat ?
\end{noliste}
\end{noliste}

\subsection*{II. Étude d'un algorithme}

Dans le programme -\Scilab{} suivant, la fonction \texttt{RANDOM}
renvoie,
pour un argument $M$ de type \texttt{INTEGER}, un nombre entier
aléatoire de
l'intervalle $[0,M-1]$.

\texttt{PROGRAM\ simulation;\ VAR\ T :ARRAY[1..20001]\ OF\ INTEGER;}

\texttt{\ \ \ \ U,S,i,n :INTEGER;}

\texttt{\ \ \ \ coincide :BOOLEAN;}

\texttt{\ }

\texttt{PROCEDUR\E\ X;}

\texttt{\ \ \ \ BEGIN}

\texttt{\ \ \ \ \ \ \ RANDOMIZE;\ \ \{initialisation\ de\ la\ fonction\
RANDOM\}}

\texttt{\ \ \ \ \ \ \ FOR\ i : = 1\ to\ 20001\ DO\ T[i] : = 1 +
RANDOM(20000);}

\texttt{\ \ \ \ END;}

\texttt{\ }

\texttt{BEGIN}

\texttt{\ \ \ X;}

\texttt{\ \ \ i : = 1;\ coincide : = FALSE;}

\texttt{\ \ \ REPEAT}

\texttt{\ \ \ \ \ i : = i + 1;}

\texttt{\ \ \ \ \ S : = 0;}

\texttt{\ \ \ \ \ WHIL\E\ (S < i-1)\ and\ NOT\ coincide\ DO}

\texttt{\ \ \ \ \ BEGIN}

\texttt{\ \ \ \ \ \ \ \ \ S : = S + 1;}

\texttt{\ \ \ \ \ \ \ \ \ IF\ T[S] = T[i]\ THEN\ conicide\ : = TRUE;}

\texttt{\ \ \ \ \ END;}

\texttt{\ \ \ UNTIL\ coincide\ = \ TRUE;}

\texttt{\ \ \ U : = i;}

\texttt{\ \ \ FOR\ n : = 1\ to\ i\ DO\ WRITELN(T[n],',\ ');}

\texttt{\ \ \ WRITELN;}

\texttt{\ \ \ WRITELN('U = \ ',U);}

\texttt{\ \ \ WRITELN('S = \ ',S);}

\texttt{\ \ \ READLN;}

\texttt{END.}

\begin{noliste}{1.}
 \setlength{\itemsep}{4mm}
\item[a)] Que fait la procédure X ?

\item[b)] Que représentent les variables U et S à la fin du programme ?

\item[c] Pourquoi est-il certain que le nombre de passages dans la
boucle
\texttt{REPEAT... UNTIL} est fini ?
\end{noliste}

\section*{III. Étude du comportement asymptotique de la suite
$(T_{N})_{N\geq 3}$}

\begin{noliste}{1.}
 \setlength{\itemsep}{4mm}
\item Une formule pour l'espérance de $T_{N}$.

\begin{noliste}{a)}
 \setlength{\itemsep}{2mm}
\item Justifier l'égalité suivante : \quad $\E(T_{N}) = \Sum{k =
0}{N}P\left(\Ev{T_{n}>k}\right)$.

\item En déduire l'égalité : \quad $\E(T_{N}) = \dfrac{N!}{N^{N}}\Sum{h
= 0}{N}\dfrac{N^{h}}{h!}$.
\end{noliste}

\item Un résultat utile sur les lois de Poisson.\\
Soit $(X_{n})_{n\geq 1}$ une suite de variables aléatoires de Poisson
indépendantes de paramètre $\lambda = 1$ et soit, pour tout entier
$N\geq 1$, $Y_{N} = X_{1} + X_{2} + \ldots + X_{N}$.

\begin{noliste}{a)}
 \setlength{\itemsep}{2mm}
\item Montrer par récurrence sur $N$, que la loi de $Y_{N}$ est une loi
de
Poisson de paramètre $N$. Donner l'espérance et la variance de $Y_{N}$.

\item Justifier l'égalité :
\[
\dlim{N\rightarrow + \infty }P\left(\Ev{ \dfrac{Y_{n}-N}{\sqrt{N}}\leq
0}\right) = \dfrac{1}{\sqrt{2\pi }}\dint{-\infty }{0}e^{-t^{2}/2}\ dt
\]

\item En déduire l'égalite : \quad $\dlim{N\rightarrow
 + \infty }P\left(\Ev{Y_{n}\leq N}\right) = \dfrac{1}{2}$
\end{noliste}

\item En appliquant ce résultat, montrer que $\E(T_{N})$ est équivalent
à $\dfrac{1}{2}\dfrac{N!\ e^{N}}{N^{N}}$ quand $N$ tend vers l'infini.

\item Une expression de la variance de $T_{N}$.

\begin{noliste}{a)}
 \setlength{\itemsep}{2mm}
\item Montrer l'égalité \quad $\E(T_{N}{2}) = \Sum{k = 0}{N}(2k +
1)P\left(\Ev{T_{N}>k}\right).$

\item Établir la relation $\Sum{k = 0}{N}kP\left(\Ev{T_{N}>k}\right) =
\dfrac{N!}{N^{N}}\Sum{h = 0}{N}(N-h)\dfrac{N^{h}}{h!}$.

\item Montrer l'égalité : \quad $\Sum{h = 0}{N}(N-h)\dfrac{N^{h}}{h!} =
\dfrac{N^{N + 1}}{N!}$.

\item En déduire que la variance $\V(T_{N})$ de $T_{N}$ et son
espérance vérifient la relation :
\[
\V(T_{N}) = 2N + E(T_{N})-(\E(T_{N}))^{2}
\]
\end{noliste}

\item En admettant le résultat classique : $\quad N!\sim
N^{N}e^{-N}\sqrt{2\pi N}$ \quad quand $N$ tend vers l'infini, donner,
en conclusion, des équivalents simples de $\E(T_{N})$ et $\V(T_{N}).$
\end{noliste}

\begin{center}
{* FIN *}
\end{center}

\label{fin}

\end{document}

