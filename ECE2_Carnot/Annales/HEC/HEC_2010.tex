\documentclass[11pt]{article}%
\usepackage{geometry}%
\geometry{a4paper,
 lmargin = 2cm,rmargin = 2cm,tmargin = 2.5cm,bmargin = 2.5cm}

\input{../../../../../../macros.tex}

\pagestyle{fancy} %
\lhead{ECE2 \hfill septembre 2017 \\
 Mathématiques\\[.2cm]} %
\chead{\hrule} %
\rhead{} %
\lfoot{} %
\cfoot{} %
\rfoot{\thepage} %

\renewcommand{\headrulewidth}{0pt}% : Trace un trait de séparation
 % de largeur 0,4 point. Mettre 0pt
 % pour supprimer le trait.

\renewcommand{\footrulewidth}{0.4pt}% : Trace un trait de séparation
 % de largeur 0,4 point. Mettre 0pt
 % pour supprimer le trait.

\setlength{\headheight}{14pt}

\title{\bf \vspace{-1cm} HEC 2010} %
\author{} %
\date{} %

\begin{document}

\maketitle %
\vspace{-1.2cm}\hrule %
\thispagestyle{fancy}

\vspace*{.4cm}

% DEBUT DU DOC À MODIFIER : tout virer jusqu'au début de l'exo

%Définition et changement de valeurs de
compteurs%newcounter{cpt1}{section} compteur cpt1 remis à 0 à chaque
aumentation par stepcounter du compteur section%setcounter{cpt1}{3} on
met le compteur à 3%addtocounter{cpt1}{5} on ajoute 5 au compteur%
stepcounter{cpt1} on ajoute 1% ifthenelse{test}{alors}{sinon} (page
206) pour subordonner à une condition % whiledo{test}{commande} pour
faire une boucle (page 206 aussi) % value{cpt1} pour noter dans le
document la valeur de cpt1 
%Définition définitive d'opérateurs
mathématiques\newcommand{\ch}{\operatorname{ch}} 
\newcommand{\sh}{\operatorname{sh}}
\renewcommand{\tanh}{\operatorname{th}}
\renewcommand{\sinh}{\operatorname{sh}}
\renewcommand{\cosh}{\operatorname{ch}}
\newcommand{\argsh}{\operatorname{argsh}}
\newcommand{\argch}{\operatorname{argch}}
\newcommand{\argth}{\operatorname{argth}}
\newcommand{\Id}{\operatorname{Id}}
\renewcommand{\leq}{\leq}
\renewcommand{\geq}{\geq }

\newcommand{\dlim}{\lim}
\newcommand{\dsum}{\sum}
\newcommand{\dprod}{\prod}



%Définition de nouvelles couleurs : rgb(trois paramètres red green blue
entre 0 et 1); cmyk (quatre cyan magenta yellow black) entre 0 et 1;
gray (entre 0 et 1) et black, white, red, green, blue, cyan, magenta,
yellow% definecolor{0gris}{gray}{0.8} 
% Nouvelle commande pour encadrer le titre car shabox ne veut que d'une
seule ligne; ATTENTION A LA TAILLE; petite différence avec shadowbox ou
doublebox, voire fcolorbox ou colorbox (au lieu de shabox; laisser le
parbox tranquille sauf pour la taille de la boîte
\newcommand{\Tbox}[1]{\begin{center} \shabox{\parbox{0.6
\linewidth}{#1}} \end{center}} %[1] pour 1 paramètre ; #1 pour ce que
fait le 1er paramètre; entre accolades ce que fait la commande
%Mise en page en mode fancy : en-têtes et pieds de pages puis
définition des en-têtes et pieds de pages\pagestyle{fancy}
\lhead{ECE 2 - Mathématiques \\
Quentin Dunstetter - ENC-Bessières 2011$\backslash$2012}
\chead{}
\rhead{HEC 2010}
\rfoot[ \ \thepage]{\thepage}
\cfoot{}
\lfoot{}
\thispagestyle{fancy} %Mise en page de la 1ère page en mode fancy
%Trait en bas et en haut de la page (entre en-tête et texte et texte et
pied de page)\renewcommand{\footrulewidth}{0.4pt}
\renewcommand{\headrulewidth}{0.4pt}


\begin{center}
{\LARGE BANQUE COMMUNE D'EPREUVES}

{\large CONCOURS D'ADMISSION DE 2010}

{\large Conceptions : H.E.C.- E.S.C.P. / EUROPE}

\rule{2.39cm}{0.05cm}

{\large OPTION : ÉCONOMIQUE}

{\Huge MATHÉMATIQUES }

Mardi 4 mai 2010, de 8 h. à 12 h.

\rule{2.39cm}{0.05cm}
\end{center}

La présentation, la lisibilité, l'orthographe, la qualité de la
rédaction, la clarté et la précision des raisonnements entreront
pour une part importante dans l'appréciation des copies.

Les candidats sont invités à \textbf{encadrer} dans la mesure du
possible les résultats de leurs calculs.

Ils ne doivent faire usage d'aucun document : \textbf{l'utilisation de
toute
calculatrice et de tout matériel électronique est interdite}. Seule
l'utilisation d'une règle graduée est autorisée.

Si au cours de l'épreuve, un candidat repère ce qui lui semble être une
erreur d'énoncé, il la signalera sur sa copie et poursuivra
sa composition en expliquant les raisons des initiatives qu'il sera
amené
à prendre

\begin{center}
\rule{16.5cm}{0.05cm}
\end{center}

\section*{EXERCICE}

Soit $E = \R_{3}\left[ X\right] $ l'espace vectoriel des polyn\^{o}mes
de degré inférieur ou égal à 3 à coefficients réels.
On confond polyn\^{o}me de E et fonction polynomiale associée définie
sur $\R$.

Soit $d$ l'application définie sur $E$ qui à tout polyn\^{o}me $P$,
associe le polyn\^{o}me $d\left( P\right) = P^{\prime}$, où
$P^{\prime}$ désigne la dérivée de $P$.

\begin{noliste}{1.}
 \setlength{\itemsep}{4mm}
\item Rappeler sans démonstration la dimension de $E$ et la base
canonique $\mathcal{B}$ de $E$.

\item Montrer que $d$ est un endomorphisme de $E$ et donner la matrice
associée à $d$ dans la base $\mathcal{B}$.

\item Déterminer le noyau de $d$, $\kerd,$ l'image de $d$,
$\operatorname{Im}d$,
ainsi que leurs dimensions respectives.

\item Déterminer les valeurs propres de $d$ ainsi que les polyn\^{o}mes
propres associés. L'endomorphisme $d$ est-il diagonalisable ?

On désigne par $\left( d^{k}\right)_{k\geq0}$, la suite
d'endomorphismes de $E$ définie par : $d^{0} = I$, où $I$ représente
l'endomorphisme identité et, pour tout $k$ de $\N$, $d^{k + 1} =
d^{k}\circ d$. Pour tout $k$ de $\N$, $Ker~d^{k}$ désigne le noyau de
$d^{k}$.

\item 

\begin{noliste}{a)}
 \setlength{\itemsep}{2mm}
\item Déterminer pour tout $k$ de $\left[ \ \left[ 1,4\right] \right]
$,
le sous espace $\kerd^{k}$ ainsi que sa dimension.\\
Vérifier que pour tout $k$ de $\left[ \ \left[ 1,4\right] \right] $,
$d\left( Ker~d^{k}\right) \subset Ker~d^{k}$.

\item Soit $P$ un polyn\^{o}me de degré $r$, avec $r\in\left[ \ \left[
0,3\right] \right] $. Montrer que la famille $\left( d^{k}\left(
P\right)
\right)_{0\leq k\leq r}$ est libre.
\end{noliste}

\item Dans cette question, on cherche à déterminer les sous espaces
vectoriels $F$ de $E$ tels que $d\left( F\right) \subset F$.

\begin{noliste}{a)}
 \setlength{\itemsep}{2mm}
\item On suppose que $\dim F = 1$. Montrer que $F$ est un sous-espace
propre
de $d$. En déduire $F$.

\item On suppose que $\dim F = 2$. Montrer qu'il existe dans $F$ un
polyn\^{o}me de degré supérieur ou égal à 1. En déduire $F$.

\item On suppose que $\dim F = 3$. On note $\tilde{d}$ l'endomorphisme
de $F$ défini par : pour tout $P$ de $F$, $\tilde{d}\left( P\right) =
d\left(
P\right) $. Montrer que $\left( \tilde{d}\right) ^{3} = 0$. En déduire
$F$.
\end{noliste}
\end{noliste}

\newpage

\section*{PROBLEME}

Toutes les variables aléatoires qui interviennent dans ce problème
sont supposées définies sur un espace probabilisé $\left(
\Omega,\mathcal{A,P}\right) $. Sous réserve d'existence, on note
$\E\left(
X\right) $ et $\V\left( X\right) $ respectivement l'espérance et la
variance d'une variable aléatoire $X$, et $Cov\left( X,Y\right) $ la
covariance de deux variables aléatoires $X$ et $Y$.

Dans les parties I et III, la fonction de répartition et une densité
d'une variable aléatoire $X$ à densité sont notées
respectivement $F_{X}$ et $f_{X}$.

\textbf{On admet }que les formules donnant l'espérance et la variance
d'une somme de variables aléatoires discrètes, ainsi que la définition
et les propriétés de la covariance et du coefficient de corrélation
linéaire de deux variables aléatoires discrètes,
s'appliquent au cas de variables aléatoires à densité.

Pour $n$ entier supérieur ou égal à $2$, on dit que les
variables aléatoires à densité $X_{1},X_{2},...,X_{n}$ sont
indépendantes si pour tout $n-uplet$ $\left(
x_{1},x_{2},...,x_{n}\right) $
de réels, les évènements $\left[ X_{1}\leq x_{1}\right],\left[
X_{2}\leq x_{2}\right],...,\left[ X_{n}\leq x_{n}\right] $ sont
indépendants.

L'objet du problème est double. D'une part, montrer certaines analogies
entre les lois géométriques et exponentielles, d'autre part mettre
en évidence quelques propriétés asymptotiques de variables aléatoires
issues de la loi exponentielle.

La partie II est indépendante de la partie I. La partie III est
indépendante de la partie II et largement indépendante de la partie I.

\subsection*{Partie I. Loi exponentielle}

\begin{noliste}{1.}
 \setlength{\itemsep}{4mm}
\item 

\begin{noliste}{a)}
 \setlength{\itemsep}{2mm}
\item Rappeler la valeur de $\dint{0}{+ \infty}e^{-t}dt$. Établir pour
tout $n$ de $\N^{\ast}$ la convergence de l'intégrale $\int
_{0}{+ \infty}t^{n}e^{-t}dt$. \\
On pose alors $I_{0} = \dint{0}{+ \infty }e^{-t}dt$ et,.pour tout $n$
de $\N^{\ast}$ $I_{n} = \dint{0}{+ \infty }t^{n}e^{-t}dt$.

\item Soit $n$ un entier de $\N^{\ast}$. À l'aide d'une intégration par
parties, établir une relation de récurrence entre $I_{n}$
et $I_{n-1}$. En déduire la valeur de $I_{n}$ en fonction de $n$.
\end{noliste}

Soit $\lambda$ un réel strictement positif. Soit $X_{1}$ et $X_{2}$
deux
variables indépendantes de même loi exponentielle de paramètre
$\lambda$ (d'espérance $1/\lambda$).\\
on pose : $Y = X_{1}-X_{2}$, $T = \max\left( X_{1},X_{2}\right) $ et $Z
= \min\left( X_{1},X_{2}\right) $.

\item Justifier les relations $T + Z = X_{1} + X_{2}$ et $T-Z = \left|
X_{1}-X_{2}\right| = \left| Y\right| $.

\item 

\begin{noliste}{a)}
 \setlength{\itemsep}{2mm}
\item Rappeler sans démonstration les valeurs respectives de $\V\left(
X_{1}\right) $ et de $P\left(\Ev{ X_{1}\leq x}\right) $, pour tout
réel $x$.

\item Calculer $\E\left( X_{1} + X_{2}\right) $, $\V\left( X_{1} +
X_{2}\right) $, 
$\E\left( Y\right) $, $\V\left( Y\right) $.
\end{noliste}

\item Déterminer pour tout réel $z$, $F_{Z}\left( z\right) $ et
$f_{Z}\left( z\right) $. Reconna\^{\i}tre la loi de $Z$ et en déduire
$\E\left( Z\right) $ et $\V\left( Z\right) $.

\item 

\begin{noliste}{a)}
 \setlength{\itemsep}{2mm}
\item Montrer que pour tout réel $t$, on a : $F_{T}\left( t\right)
 = \left\{ 
\begin{array}{c}
\left( 1-e^{-\lambda t}\right) ^{2} \\
0
\end{array}
\right. 
\begin{tabular}{l}
si $t\geq0$ \\
si $t<0$\end{tabular}
$. Exprimer pour tout réel $t$, $f_{T}\left( t\right) $.

\item Justifier l'existence de $\E\left( T\right) $ et $\V\left(
T\right) $.
Montrer que $\E\left( T\right) = \frac{3}{2\lambda}$ et $\V\left(
T\right) = \frac{5}{4\lambda^{2}}$. \\
(on pourra utiliser des changements de variables affine).
\end{noliste}

\item On note $r$ le coefficient de corrélation linéaire de $Z$ et $T
$. Montrer que $r = 1/\sqrt{5}$.

\item 

\begin{noliste}{a)}
 \setlength{\itemsep}{2mm}
\item Préciser $Y\left( \Omega\right) $ et $\left| Y\right|
\left( \Omega\right) $.

\item Déterminer une densité de la variable aléatoire $-X_{2}$.

\item Montrer que pour tout réel $y$, l'intégrale $\dint{-\infty
}{+ \infty}f_{X_{1}}\left( t\right) f_{-X_{2}}\left( y-t\right\ dt$ est
convergente et qu'elle vaut $\frac{\lambda}{2}e^{-\lambda\left|
y\right| }$.\\
(on distinguera les deux cas : $y\geq0$ et $y<0$)

\item Établir que la fonction
$y\mapsto\frac{\lambda}{2}e^{-\lambda\left| y\right| }$ est une densité
de probabilité sur $\R$
; on admet que c'est une densité de la variable aléatoire $Y$.

\item Déterminer pour tout $y$ réel, $f_{\left| Y\right|
}\left( y\right) $. Reconna\^{\i}tre la loi de $\left| Y\right| = T-Z$.
\end{noliste}
\end{noliste}

\newpage

\subsection*{Partie II. Loi géométrique}

Soit $p$ un réel de $\left] 0,1\right[ $ et $q = 1-p$. Soit $X_{1}$ et
$X_{2}$ deux variables indépendantes de même loi géométrique
de paramètre $p$ (d'espérance $1/p$).\\
on pose : $Y = X_{1}-X_{2}$, $T = \max\left( X_{1},X_{2}\right) $ et $Z
= \min\left( X_{1},X_{2}\right) $. On rappelle que $T + Z = X_{1} +
X_{2}$ et $T-Z = \left| X_{1}-X_{2}\right| = \left| Y\right| $.

\begin{noliste}{1.}
 \setlength{\itemsep}{4mm}
\item 

\begin{noliste}{a)}
 \setlength{\itemsep}{2mm}
\item Rappeler sans démonstration les valeurs respectives de $\V\left(
X_{1}\right) $ et de $P\left(\Ev{ X_{1}\leq k}\right) $, pour tout 
$k$ de $X_{1}\left( \Omega\right) $.

\item Calculer $\E\left( X_{1} + X_{2}\right) $, $\V\left( X_{1} +
X_{2}\right) $, 
$\E\left( X_{1}-X_{2}\right) $, $\V\left( X_{1}-X_{2}\right) $.

\item Établir la relation : $P\left(\Ev{ X_{1} = X_{2}}\right) =
\frac{p}{1 + q}$
\end{noliste}

\item 

\begin{noliste}{a)}
 \setlength{\itemsep}{2mm}
\item Montrer que $Z$ suit la loi géométrique de paramètre $1-q^{2}$.
En déduire $\E\left( Z\right) $, $\V\left( Z\right) $ et $\E\left(
T\right) $.

\item Soit $k$ un entier de $\N^{\ast}$.Justifier l'égalité : $\left[ Z
= k\right] \cup\left[ T = k\right] = \left[ X_{1} = k\right] \cup\left[
X_{2} = k\right] $.\\
En déduire la relation suivante : $P\left(\Ev{ T = k}\right) =
2P\left(\Ev{
X_{1} = k}\right) -P\left(\Ev{ Z = k}\right) $.

\item Établir la formule : $\V\left( T\right) = \frac{q\left(
2q^{2} + q + 2\right) }{\left( 1-q^{2}\right) ^{2}}.$
\end{noliste}

\item 

\begin{noliste}{a)}
 \setlength{\itemsep}{2mm}
\item Préciser $\left( T-Z\right) \left( \Omega\right) $. Exprimer pour
tout $j$ de $\N^{\ast}$, l'évènement $\left[ Z = j\right] \cap\left[ Z
= T\right] $ en fonction des évènements $\left[ X_{1} = j\right]
$ et $\left[ X_{2} = j\right] $. En déduire pour tout $j$ de
$\N^{\ast}$, l'expression de $P\left( \left[ Z = j\right] \cap\left[ Z
= T\right]
\right) $

\item Montrer que pour tout couple $\left( j,l\right) $ de $\left(
\N^{\ast}\right) ^{2}$, on a : $P\left( \left[ Z = j\right] \cap\left[
T-Z = l\right] \right) = 2p^{2}q^{2j + l-2}$

\item Montrer que pour tout $k$ de $\Z$, $P\left(\Ev{
X_{1}-X_{2} = k}\right) = \frac{pq^{\left| k\right| }}{1 + q}$ (on
distinguera trois cas : $k = 0$, $k>0$ et $k<0$).

\item En déduire la loi de la variable aléatoire $\left|
X_{1}-X_{2}\right| $.

\item Établir à l'aide des questions précédentes que les
variables $Z$ et $T-Z$ sont indépendantes.
\end{noliste}

\item 

\begin{noliste}{a)}
 \setlength{\itemsep}{2mm}
\item À l'aide du résultat de la question 3.e, calculer $Cov\left(
Z,T\right) $. Les variables $Z$ et $T$ sont-elles indépendantes ?

\item Calculer en fonction de $q$, le coefficient de corrélation
linéaire $\rho $ de $Z$ et $T$.

\item Déterminer la loi de probabilité du couple $\left( Z,T\right) $.

\item Déterminer pour tout $j$ de $\N^{\ast }$, la loi de
probabilité conditionnelle de $T$ sachant l'évènement $\left[ Z =
j\right] $.

\item Soit $j$ un élément de $\N^{\ast }$. On suppose qu'il
existe une variable aléatoire $D_{j}$ à valeur dans $\N^{\ast }$, dont
la loi de probabilité est la loi conditionnelle de $T$
sachant l'évènement $\left[ Z = j\right] $. Calculer $\E\left(
D_{j}\right) $.
\end{noliste}
\end{noliste}

\subsection*{Partie III. Convergences}

Dans les questions \ref{a} à \ref{b}, $\lambda $ désigne un paramètre
réel strictement positif, inconnu.

pour $n$ élément de $\N^{\ast }$, on considère un $n$-échantillon
$\left( X_{1},X_{2},...,X_{n}\right) $ de variables aléatoires à
valeurs strictement positives, indépendantes, de même
loi exponentielle de paramètre $\lambda $.

On pose pour tout $n$ de $\N^{\ast }$ : $S_{n} = \Sum{k = 1}{n}X_{k}$
et $J_{n} = \lambda S_{n}$.

\begin{noliste}{1.}
 \setlength{\itemsep}{4mm}
\item \label{a}Calculer pour tout $n$ de $\N^{\ast }$, $\E\left(
S_{n}\right) $, $\V\left( S_{n}\right),\E\left( J_{n}\right) $ et
$\V\left(
J_{n}\right) $.

\item On admet qu'une densité $f_{J_{n}}$ de $J_{n}$ est donnée par
$f_{J_{n}}\left( x\right) = \left\{ 
\begin{array}{c}
\frac{e^{-x}x^{n-1}}{\left( n-1\right) !} \\
0
\end{array}
\right. 
\begin{tabular}{l}
si $x>0$ \\
si $x\leq 0$\end{tabular}
$.

\begin{noliste}{a)}
 \setlength{\itemsep}{2mm}
\item À l'aide du théorème de transfert, établir pour tout $n$
supérieur ou égal à 3, l'existernce de $\E\left( \frac{1}{J_{n}}\right)
$ et de $\E\left( \frac{1}{J_{n}{2}}\right) $, et donner leur
valeurs respectives.

\item On pose pour tout $n$ supérieur ou égal à 3 :
$\widehat{\lambda_{n}} = \frac{n}{S_{n}}$. Justifier que
$\widehat{\lambda_{n}}$ est
un estimateur de $\lambda $. Est-il sans biais ? Calculer la limite,
lorsque $n$ tend vers $ + \infty $, du risque quadratique associé à
$\widehat{\lambda_{n}}$ en $\lambda $.
\end{noliste}

\newpage

\item Dans cette question, on veut déterminer un intervalle de
confiance
du paramètre $\lambda $ au risque $\alpha $. On note $\Phi $ la
fonction
de répartition de la loi normale centrée réduite, et $u_{\alpha }
$ le réel strictement positif tel que $\Phi \left( u_{\alpha }\right) =
1-\frac{\alpha }{2}$.

\begin{noliste}{a)}
 \setlength{\itemsep}{2mm}
\item Enoncer le théorème de la limite centrée. En déduire
que la variable aléatoire $N_{n}$ définie par $N_{n} = \lambda
\frac{S_{n}}{\sqrt{n}}-\sqrt{n}$ converge en loi vers la loi normale
centrée réduite.

\item En déduire que pour $n$ assez grand, on a approximativement :
$P\left(\Ev{ -u_{\alpha }\leq N_{n}\leq u_{\alpha }}\right)
 = 1-\alpha $.

\item Montrer que pour $n$ assez grand, l'intervalle $\left[ \ \left(
1-\frac{u_{\alpha }}{\sqrt{n}}\right) \widehat{\lambda_{n}},\left( 1 +
\frac{u_{\alpha }}{\sqrt{n}}\right) \widehat{\lambda_{n}}\right] $ est
un
intervalle de confiance de $\lambda $ au risque $\alpha $. On note
$\lambda
_{0}$ la réalisation de $\widehat{\lambda_{n}}$ sur le $n$-échantillon.
\end{noliste}

\item \label{b}Avec le $n$-échantillon $\left(
X_{1},X_{2},...,X_{n}\right) $, on construit un nouvel intervalle de
confiance de $\lambda $ au risque $\beta $ ($\beta \neq \alpha $), tel
que
la longueur de cet intervalle soit $k$ ($k>1$) fois plus petite que
celle
obtenue avec le risque $\alpha $.

\begin{noliste}{a)}
 \setlength{\itemsep}{2mm}
\item Justifier l'existence de la fonction réciproque $\Phi ^{-1}$ de
$\Phi $. Quel est le domaine de définition de $\Phi ^{-1}$ ?

\item Établir l'égalité $\beta = 2\Phi \left( \frac{1}{k}\Phi
^{-1}\left( \alpha /2\right) \right) $. En déduire que $\beta >\alpha
$.
Ce dernier résultat était-il prévisible ?
\end{noliste}

Dans les questions \ref{c} à \ref{d}, on suppose que $\lambda = 1$.

\item \label{c}On pose pour tou $n$ de $\N^{\ast }$ : $T_{n} = \max
\left( X_{1},X_{2},...,X_{n}\right) $.\\
Pour tout $n$ de $\N^{\ast }$, pour tout réel $x$ positif ou
nul, on pose : $g_{n}\left( x\right) = \dint{0}{x}F_{T_{n}}\left(
t\right)
dt $ et $h_{n}\left( x\right) = \dint{0}{x}tf_{T_{n}}\left( t\right\
dt$

\begin{noliste}{a)}
 \setlength{\itemsep}{2mm}
\item Exprimer $h_{n}\left( x\right) $ en fonction de $F_{T_{n}}\left(
x\right) $
et $g_{n}\left( x\right) $.

\item Déterminer pour tout réel $t$, l'expression de $F_{T_{n}}\left(
t\right) $ en fonction de $t$.\\
Établir pour tout $n$ supérieur ou égal à 2, la relation :
$g_{n-1}\left( x\right) -g_{n}\left( x\right) =
\frac{1}{n}F_{T_{n}}\left(
x\right) $

\item En déduire que pour tout $n$ de $\N^{\ast }$, pour tout réel $x$
positif ou nul, l'expression de $g_{n}\left( x\right) $ en
fonction de $x,F_{T_{1}}\left( X\right),F_{T_{2}}\left(
x\right),...,F_{T_{n}}\left( x\right) $.

\item Montrer que $F_{T_{n}}\left( x\right) -1$ est équivalent à
$-ne^{-x}$, lorsque $x$ tend vers $ + \infty $.

\item Déduire des questions c) et d) l'existence de $\E\left(
T_{n}\right) $ et montrer que $\E\left( T_{n}\right) = \Sum{k =
1}{n}\frac{1}{k}$.
\end{noliste}

\item On veut étudier dans cette question la convergence en loi de la
suite de variables aléatoires $\left( G_{n}\right)_{n\geq 1}$ définie
par : pour tout $n$ de $\N^{\ast }$, $G_{n} = T_{n}-\E\left(
T_{n}\right) $. \\
On pose pour tout $n$ de $\N^{\ast }$ : $\gamma_{n} = -\ln n + E\left(
T_{n}\right) $ et on admet sans démonstration que la suite $\left(
\gamma_{n}\right)_{n\geq 1}$ est convergente ; on note $\gamma $ sa
limite.

\begin{noliste}{a)}
 \setlength{\itemsep}{2mm}
\item Montrer que pour tout $x$ réel et $n$ assez grand, on a :
$F_{G_{n}}\left( x\right) = \left( 1-\frac{1}{n}e^{-\left( x + \gamma
_{n}\right) }\right) ^{n}$.

\item En déduire que pour tout $x$ réel, on a : $\dlim{n\rightarrow +
\infty }F_{G_{n}}\left( x\right) = e^{-e^{-\left( x + \gamma
\right) }}$

\item Montrer que la fonction $F_{G} :\mathbb{R\rightarrow R}$ définie
par $F_{G}\left( x\right) = e^{-e^{-\left( x + \gamma \right) }}$ est
la
fonction de répartition d'une variable aléatoire $G$ à densité.
Conclure.
\end{noliste}

\item \label{d}

\begin{noliste}{a)}
 \setlength{\itemsep}{2mm}
\item Soit $X$ une variable aléatoire à densité de fonction de
répartition $F_{X}$ strictement croissante. Déterminer la loi de la
variable alétoire $Y$ définie par $Y = F_{X}\left( X\right) $.

\item Écrire une fonction \Scilab{} d'en-tête \texttt{Gumbel} qui
permet de
simuler la variable aléatoire $G$. On supposera que la constante
$\gamma 
$ est définie en langage \Scilab{} par une constante
\texttt{gamma}\textbf{. }On rappelle que la fonction \Scilab{}
\texttt{random }permet de simuler la
loi uniforme sur $\left] 0,1\right[ $.
\end{noliste}
\end{noliste}

\end{document}

