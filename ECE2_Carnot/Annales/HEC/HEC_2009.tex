\documentclass[11pt]{article}%
\usepackage{geometry}%
\geometry{a4paper,
 lmargin = 2cm,rmargin = 2cm,tmargin = 2.5cm,bmargin = 2.5cm}

\input{../../../../../../macros.tex}

\pagestyle{fancy} %
\lhead{ECE2 \hfill septembre 2017 \\
 Mathématiques\\[.2cm]} %
\chead{\hrule} %
\rhead{} %
\lfoot{} %
\cfoot{} %
\rfoot{\thepage} %

\renewcommand{\headrulewidth}{0pt}% : Trace un trait de séparation
 % de largeur 0,4 point. Mettre 0pt
 % pour supprimer le trait.

\renewcommand{\footrulewidth}{0.4pt}% : Trace un trait de séparation
 % de largeur 0,4 point. Mettre 0pt
 % pour supprimer le trait.

\setlength{\headheight}{14pt}

\title{\bf \vspace{-1cm} HEC 2009} %
\author{} %
\date{} %

\begin{document}

\maketitle %
\vspace{-1.2cm}\hrule %
\thispagestyle{fancy}

\vspace*{.4cm}

% DEBUT DU DOC À MODIFIER : tout virer jusqu'au début de l'exo

%Définition et changement de valeurs de
compteurs%newcounter{cpt1}{section} compteur cpt1 remis à 0 à chaque
aumentation par stepcounter du compteur section%setcounter{cpt1}{3} on
met le compteur à 3%addtocounter{cpt1}{5} on ajoute 5 au compteur%
stepcounter{cpt1} on ajoute 1% ifthenelse{test}{alors}{sinon} (page
206) pour subordonner à une condition % whiledo{test}{commande} pour
faire une boucle (page 206 aussi) % value{cpt1} pour noter dans le
document la valeur de cpt1 
%Définition définitive d'opérateurs
mathématiques\newcommand{\ch}{\operatorname{ch}} 
\newcommand{\sh}{\operatorname{sh}}
\renewcommand{\tanh}{\operatorname{th}}
\renewcommand{\sinh}{\operatorname{sh}}
\renewcommand{\cosh}{\operatorname{ch}}
\newcommand{\argsh}{\operatorname{argsh}}
\newcommand{\argch}{\operatorname{argch}}
\newcommand{\argth}{\operatorname{argth}}
\newcommand{\Id}{\operatorname{Id}}
\renewcommand{\leq}{\leq}
\renewcommand{\geq}{\geq }

\newcommand{\dlim}{\lim}
\newcommand{\dsum}{\sum}
\newcommand{\dprod}{\prod}



%Définition de nouvelles couleurs : rgb(trois paramètres red green blue
entre 0 et 1); cmyk (quatre cyan magenta yellow black) entre 0 et 1;
gray (entre 0 et 1) et black, white, red, green, blue, cyan, magenta,
yellow% definecolor{0gris}{gray}{0.8} 
% Nouvelle commande pour encadrer le titre car shabox ne veut que d'une
seule ligne; ATTENTION A LA TAILLE; petite différence avec shadowbox ou
doublebox, voire fcolorbox ou colorbox (au lieu de shabox; laisser le
parbox tranquille sauf pour la taille de la boîte
\newcommand{\Tbox}[1]{\begin{center} \shabox{\parbox{0.6
\linewidth}{#1}} \end{center}} %[1] pour 1 paramètre ; #1 pour ce que
fait le 1er paramètre; entre accolades ce que fait la commande
%Mise en page en mode fancy : en-têtes et pieds de pages puis
définition des en-têtes et pieds de pages\pagestyle{fancy}
\lhead{ECE 2 - Mathématiques \\
Quentin Dunstetter - ENC-Bessières 2011$\backslash$2012}
\chead{}
\rhead{HEC 2009}
\rfoot[ \ \thepage]{\thepage}
\cfoot{}
\lfoot{}
\thispagestyle{fancy} %Mise en page de la 1ère page en mode fancy
%Trait en bas et en haut de la page (entre en-tête et texte et texte et
pied de page)\renewcommand{\footrulewidth}{0.4pt}
\renewcommand{\headrulewidth}{0.4pt}

\begin{center}
 HEC 2009 Eco 
\end{center}

\section*{Exercice}

Toutes les matrices de cet exercice sont des éléments de l'ensemble
$\M{2} $ des matrices carrées d'ordre 2 
à. coefficients réels. On note $I$ la matrice identité de $\M{2} $. On
rappelle qu'un élément 
$A$ de $\M{2} $ est colinéaire $I$
s'il existe un réel $\lambda $ tel que $A = \lambda I$.

On définit les deux applications suivantes de $\M{2} $ dans $\R$,
notées $d$ et $t$, par : pour
tout élément $A = \left( a_{i,j}\right)_{1\leq i,j\leq2}$ de.$\M{2} $
$d(A) = a_{1,1}a_{2,2}-a_{1,2}a_{2,1} $ et $t(A) = a_{1,1} + a_{2,2}.$

\begin{noliste}{1.}
 \setlength{\itemsep}{4mm}
\item Soit $A$ et $B$ deux éléments de $\M{2} $.

\begin{noliste}{a)}
 \setlength{\itemsep}{2mm}
\item Calculer $d(2I)$. En déduire que l'application $d$ n'est pas
linéaire.

\item Établir la formule : $d(AB) = d(A)\times d(B)$.

\item En déduire que si $A$ et $B$ sont semblables, on a : $d(A) =
d(B)$.
\end{noliste}

\item 

\begin{noliste}{a)}
 \setlength{\itemsep}{2mm}
\item Montrer que $t$ est une application linéaire de $\M{2} $ dans
$\R$. Déterminer la
dimension de son image et celle de son noyau.

\item Établir que si $A$ et $B$ sont deux éléments de $\M{2} $ on a :
$t(AB) = t(BA)$.

\item En déduire que si $A$ et $B$ sont semblables, on a : $t(A) =
t(B)$.
\end{noliste}

\item Soit $A$ un élément donné de $\M{2} $ non colinéaire à $I$.

\begin{noliste}{a)}
 \setlength{\itemsep}{2mm}
\item Établir l'existence d'un unique couple $\left( \alpha,\beta
\right) 
$ de réels vérifiant : $A^{2} = \alpha A + \beta I.$

\item Exprimer $\alpha$ et $\beta$ en fonction de $d(A)$ et $t(A)$.
\end{noliste}

\item Soit $A$ un élément donné de $\M{2} $ non colinéaire à $I$. On
note $u$
l'endomorphisme de $\R^{2}$ dont $A$ est la matrice associée
dans la base canonique $\left( e_{1},e_{2}\right) $ de $\R^{2}$ On
pose : $w = e_{1} + e_{2}$.

\begin{noliste}{a)}
 \setlength{\itemsep}{2mm}
\item Montrer que les trois vecteurs $e_{1},e_{2}$ et $w$ ne peuvent
être simultanément vecteurs propres de $u$.

\item En déduire qu'il existe au moins un élément non nul $x$ de 
$\R^{2}$ tel que la famille $\left( x,u\left( x\right) \right) $
soit une base de $\R^{2}$

\item Montrer que la matrice $M$ associée à $u$ dans la base $\left(
x,u\left( x\right) \right) $ est de la forme $\left( 
\begin{array}{cc}
0 & a \\
1 & b
\end{array}
\right) $ où $a$ et $b$ sont deux réels, indépendants de la base 
$(x,u(x))$, que l'on exprimera en fonction de $d(A)$ et $t(A)$.

\item En déduire que la matrice $A$ est semblable à sa transposée
$^{t}A$
\end{noliste}

\item Soit $A$ un élément donné de $\M{2} $ et $\mathcal{C}(A)$
l'ensemble défini par $\mathcal{C}(A) = \{B\in \M{2} $ / $AB = BA\}$.

\begin{noliste}{a)}
 \setlength{\itemsep}{2mm}
\item Montrer que $\mathcal{C}(A)$ est un sous-espace vectoriel de
$\M{2} $

\item Déterminer une base et la dimension de $\mathcal{C}(A)$ (on
discutera selon que $A$ est ou n'est pas colinéaire à $I$).
\end{noliste}
\end{noliste}

\section*{Problème}

Dans tout le problème, on considère la suite $\left( u_{n}\right)
_{n\in \N}$\ définie par $u_{0} = 0$, $u_{1} = 1$ et la relation pour
tout $n$ de $\N$, $u_{n + 2} = u_{n + 1} + u_{n}$.

La partie II est indépendante de la partie I et la partie III est
indépendante de la partie II.

\subsection*{Partie I. Analyse}

\begin{noliste}{1.}
 \setlength{\itemsep}{4mm}
\item 

\begin{noliste}{a)}
 \setlength{\itemsep}{2mm}
\item Montrer que la suite $\left( u_{n}\right)_{n\in \N}$ est une
suite croissante d'entiers naturels.

\item La suite est-elle convergente ?
\end{noliste}

\textbf{Dans toute la suite du problème, }$a$\textbf{\ et }$b$\textbf{\
(}$a>b$\textbf{) désignent les deux solutions de l'équation du second
degré suivante : }$x^{2}-x-1 = 0$

\item 

\begin{noliste}{a)}
 \setlength{\itemsep}{2mm}
\item Montrer que : $b = 1-a = -\frac{1}{a}$. Établir l'encadrement
suivant $1<a<2$.

\item Montrer que, pour tout $n$ de $\N$, on a : $u_{n} =
\frac{1}{\sqrt{5}}\left( a^{n}-b^{n}\right) $

\item En déduire un équivalent de $u$ lorsque $n$ tend vers $ + \infty$
\end{noliste}

\item On pose, pour tout $n$ de $\N$ : $\beta_{n} = u_{n + 1}-au_{n}$.
Exprimer, pour tout $n$ de $\N$, $\beta_{n}$ en fonction de $n$ et $b
$.

\item On rappelle que pour tout réel $x$, la partie entière de $x$
est l'entier noté $\left\lfloor x\right\rfloor $ qui vérifie :
$\left\lfloor x\right\rfloor \leq x<\left\lfloor x\right\rfloor + 1$.

\begin{noliste}{a)}
 \setlength{\itemsep}{2mm}
\item Établir, pour tout $n$ de $\N$, l'égalité suivante :
$\left\lfloor au_{2n}\right\rfloor = u_{2n + 1}-1$

\item Exprimer, pour tout $n$ de $\N^{\ast}$ $\left\lfloor
au_{2n-1}\right\rfloor $, en fonction de $u_{2n}$
\end{noliste}

\item Soit $y$ un réel fixé vérifiant $\left| y\right|
<1 $ et $k$ un entier fixé de $\N$.

\begin{noliste}{a)}
 \setlength{\itemsep}{2mm}
\item Montrer que la série $\sum \limits_{n\geq1}n^{k}y^{n}$\ est
absolument convergente.

\item En deduire la convergence de la serie $\sum
\limits_{n\geq1}n^{k}\frac{u_{n}}{2^{n + 1}}$

\item En utilisant la définition de la suite $\left(
u_{n}\right)_{n\in\N}$, calculer$\sum \limits_{n =
1}{\infty}\frac{u_{n}}{2^{n + 1}}$
\end{noliste}
\end{noliste}

\subsection*{Partie II. Algèbre et algorithmique}

\begin{noliste}{1.}
 \setlength{\itemsep}{4mm}
\item Soit A la matrice carrée d'ordre 4 definie par : $A =
\frac{1}{2}\left( 
\begin{array}{cccc}
1 & 1 & 1 & 1 \\
1 & 0 & 0 & 1 \\
1 & 0 & 0 & 1 \\
1 & 1 & 1 & 1
\end{array}
\right) $

\begin{noliste}{a)}
 \setlength{\itemsep}{2mm}
\item La matrice $A$ est-elle inversible ? $A$ est-elle diagonalisable
?

\item Calculer $A^{2}$ et $A^{3}$ Vérifier que $A^{3}$ est une
combinaison linéaire de $A$ et $A^{2}$

\item Déterminer les valeurs propres de $A$.

\item Établir l'existence de deux suites $\left(
a_{n}\right)_{n\in\N^{\ast}}$et $\left( b_{n}\right)_{n\in \N^{\ast}}$
telles
que, pour tout $n$ de $\N^{\ast}$ On ait : $A^{n} = a_{n}A +
b_{n}A^{2}$

\item Exprimer, pour tout $n$ de $\N^{\ast}$, $a_{n + 1}$ et $b_{n +
1}$
en fonction de $a_{n}$ et $b_{n}$. Montrer que les suites $\left(
a_{n}\right)_{n\in \N^{\ast}}$ et $\left( b_{n}\right)_{n\in
\N^{\ast}}$ vérifient une relation de récurrence linéaire
d'ordre 2.
\end{noliste}

\item On propose la fonction \Scilab{} suivante : \\
Function f(n : integer) integer ;\\
var temp,u,v,k : integer ;\\
Begin\\
u : = 0 ; v : = 1 ;\\
for k : = 1 to n-1 do \\
\qquad Begin\\
\qquad temp : = \_ \_ \_ \_ \_ ; v : = \_ \_ \_ \_ \_ ; u : = \_ \_ \_
\_ \_ ;\\
\qquad end;\\
f : = \_ \_ \_ \_ \_ ;\\
end;\\
Compléter cette fonction aux quatre places signalées par des tirets
de façon que la valeur rendue soit $u_{n}$.

\item Soit $n$ un entier de $\N^{\ast}$. On dit que $n$ admet une $Z 
$-décomposition s'il existe un entier $r$ de $\N^{\ast}$ tel que
l'on puisse écrire $n = u_{k_{1}} + u_{k_{2}} + \cdots + u_{k_{r}}$,
où,
pour tout $i$ de $\left[ \ \left[ 1,r\right] \right] $ $k_{i}$\ est un
entier
supérieur ou égal à 2 et où, pour tout $i$ de $\left[ \ \left[
1,r-1\right] \right] $ (avec $r\geq2$), on a : $k_{i + 1}-k_{i}\geq2$.

\begin{noliste}{a)}
 \setlength{\itemsep}{2mm}
\item Montrer que les entiers 37 et 272 admettent une
$Z$-décomposition.

\item Soit $n$ un entier admettant une $Z$-décomposition de la forme $n
= u_{k_{1}} + u_{k_{2}} + \cdots + u_{k_{r}}.$ Montrer, par récurrence
sur $r$, que l'on a : $n<u_{k_{r} + 1}$ En déduire l'unicité de $r$.

\item Montrer que, pour tout entier $p$ supérieur ou égal à $2$,
tout entier $n$ qui vérifie $1\leq n\leq u_{p}$ admet une unique
$Z$-décomposition (on pourra faire un raisonnement par récurrence sur
$p$).
\end{noliste}

\item On suppose que l'on a défini en \Scilab{} une constante p et un
type
tab par les instructions suivantes :

const p = 20 ; type tab = array[2..p] of integer
\end{noliste}

On suppose également que l'on a défini une variable u de type tab
telle que, pour tout k de [[2,p]] la variable u[k] contient la valeur
$u_{k}$. On se donne un entier $n$ vérifiant : $1\leq n\leq u_{p}$

Rédiger la procédure d'en-tête : procedure Z (n integer ; var
Res : tab) de façon que :

Res[k] = $\left\{ 
\begin{array}{c}
u_{k_{1}}\text{ si }k = k_{1} \\
u_{k_{2}}\text{ si }k = k_{2} \\
\vdots \\
u_{k_{r}}\text{ si }k = k_{r} \\
0\text{ sinon}
\end{array}
\right. $

Expliquer et justifier l'algorithme utilisé.

\subsection*{Partie III. Probabilités}

On effectue dans une urne qui contient des boules numérotées 0 ou 1
une suite illimitée de tirages avec remise d'une boule. \`{A} chaque
tirage, la probabilité de tirer une boule numérotée 1 est $p$ ($0<p<1$)
et la probabilité de tirer une boule numérotée $0$ est $q $, avec $q =
1-p$, et on suppose que les résultats des différents
tirages sont indépendants.

On suppose que cette expérience est modélisée par un espace
probabilisé $\left( \Omega,\mathcal{A},\mathcal{P}\right) $. On
s'intéresse au nombre de tirages nécessaires pour obtenir deux boules
numérotées 1 de suite, c'est-à-dire lors de deux tirages consécutifs.
On définit, pour tout $i$ de $\N^{\ast}$, les événements $S_{i}$ : "le
i-ième tirage donne une boule numérotée 1
", et $B_{i} = S_{i}\cap S_{i + 1}.$

Si au moins un des événements $B_{i}$ se réalise au cours de
l'expérience, on note $Y$ la valeur de l'entier $j$ correspondant au
premier événement $B_{j}$ réalisé. Sinon, c'est-à-dire
si aucun des événements $B_{i}$ ne se réalise, on attribue à 
$Y$ la valeur 0. On admet que $Y$ est une variable aléatoire définie
sur $\left( \Omega,\mathcal{A},\mathcal{P}\right) $.

Par exemple, si le résultat de l'expérience est : 0,1,0,0,0,
1,1,0,1,1,1,0,., alors Y prend la valeur 6.

\begin{noliste}{1.}
 \setlength{\itemsep}{4mm}
\item 

\begin{noliste}{a)}
 \setlength{\itemsep}{2mm}
\item Calculer, pour tout $i$ de $\N^{\ast}$ la probabilité
$P\left(\Ev{ B_{i}}\right) $.

\item Déterminer $Y\left( \Omega \right) $. Calculer $P\left(\Ev{ Y =
1}\right),P\left(\Ev{ Y = 2}\right) $ et $P\left(\Ev{ Y = 3}\right) $.
\end{noliste}

\item Pour tout $n$ de $\N^{\ast}$, on note $C_{n}$, l'événement "lors
des n premiers tirages. il n'apparait jamais deux fois de suite
une boule numérotée 1 ". On pose : $C_{0} = \Omega$.

\begin{noliste}{a)}
 \setlength{\itemsep}{2mm}
\item Calculer $P\left(\Ev{ C_{0}}\right) $, $P\left(\Ev{ C_{1}}\right)
$ et $P\left(\Ev{
C_{2}}\right) $

\item Établir, pour tout $n$ de $\N$, la relation : $P\left(\Ev{ Y = n
+ 2}\right) = p^{2}qP\left(\Ev{ C_{n}}\right) $
\end{noliste}

\item 

\begin{noliste}{a)}
 \setlength{\itemsep}{2mm}
\item En considérant les résultats possibles des deux premiers
tirages, montrer, pour tout entier $n$ supérieur ou égal à 2, l'égalité
: $P\left(\Ev{ C_{n}}\right) = qP\left(\Ev{ C_{n-1}}\right) +
pqP\left(\Ev{
C_{n-2}}\right) $

\item Déterminer, pour tout $n$ de $\N^{\ast}$ une relation
entre $P\left(\Ev{ Y = n + 2}\right) $, $P\left(\Ev{ Y = n + 1
}\right) $ et $P\left(\Ev{ Y = n}\right) $
\end{noliste}

\item On suppose dans cette question que $p = q = 1/2$.

\begin{noliste}{a)}
 \setlength{\itemsep}{2mm}
\item Montrer que, pour tout $n$ de $\N^{\ast}$, on a : $P\left(\Ev{ Y
= n}\right) = \frac{u_{n}}{2^{n + 1}}$ où la suite $\left(
u_{n}\right)_{n\in \N}$ a été définie dans le préambule du problème.

\item Que vaut $P\left(\Ev{ Y = 0}\right) $ ?

\item On note $\E(Y)$ l'espérance de $Y$. Montrer que $\E(Y) = 5$.

\item Calculer la variance $\V(Y)$ de $Y$.
\end{noliste}

\item On revient au cas général : $0<p<1$ et $q = 1-p$.

\begin{noliste}{a)}
 \setlength{\itemsep}{2mm}
\item Montrer que l'équation du second degré $x^{2}-qx-pq$ = 0 admet
deux racines distinctes. On les note $r$ et $s$. avec $r>s$.

\item Établir les inégalités suivantes. $-1<s<0<r<1$ et $r>\left|
s\right| $.

\item On pose $\Delta = q^{2} + 4pq$. Montrer que, pour tout $n$ de
$\N^{\ast }$ on a : $P\left(\Ev{ Y = n}\right) =
\frac{p^{2}}{\sqrt{\Delta}}\left( r^{n}-s^{n}\right) $

\item Calculer $P\left(\Ev{ Y = 0}\right) $.

\item Montrer que $Y$ admet des moments de tous ordres et calculer
l'espérance de $Y$.
\end{noliste}

\item ~

\begin{noliste}{a)}
 \setlength{\itemsep}{2mm}
\item Montrer, pour tout réel $x$ vérifiant $\left| x\right|
<\frac{1}{r}$, la convergence de la série $\sum
\limits_{n\geq1}P\left(\Ev{ Y = n}\right) x^{n}$. On pose alors : $g(x)
= \sum \limits_{n = 1}{+ \infty}P\left(\Ev{ Y = n}\right) x^{n}$.

\item Établir, pour tout réel $x$ vérifiant $\left|
x\right| <\frac{1}{r}$, la formule suivante : $g(x) =
\frac{p^{2}x}{1-qx-pqx^{2}}$.
\end{noliste}

\item On suppose dans cette question que $p = 2/3$.

\begin{noliste}{a)}
 \setlength{\itemsep}{2mm}
\item Étudier les variations de la fonction $g$ sur l'intervalle $I =
\left] -\frac{3}{2},\frac{3}{2}\right[ $

\item Montrer l'existence d'un unique réel $\alpha$ de $\left] -\frac
{1}{2},0\right[ $ tel que $g$ soit concave sur l'intervalle $\left]
-\frac{3}{2},\alpha \right[ $ et convexe sur l'intervalle $\left]
\alpha,\frac{3}{2}\right[ $

\item Tracer l'allure de la courbe représentative de $g$ sur $I$ dans
le
plan rapporté à un repère orthonormé.
\end{noliste}
\end{noliste}

\end{document}

