\documentclass[11pt]{article}%
\usepackage{geometry}%
\geometry{a4paper,
 lmargin = 2cm,rmargin = 2cm,tmargin = 2.5cm,bmargin = 2.5cm}

\input{../../../../../../macros.tex}

\pagestyle{fancy} %
\lhead{ECE2 \hfill septembre 2017 \\
 Mathématiques\\[.2cm]} %
\chead{\hrule} %
\rhead{} %
\lfoot{} %
\cfoot{} %
\rfoot{\thepage} %

\renewcommand{\headrulewidth}{0pt}% : Trace un trait de séparation
 % de largeur 0,4 point. Mettre 0pt
 % pour supprimer le trait.

\renewcommand{\footrulewidth}{0.4pt}% : Trace un trait de séparation
 % de largeur 0,4 point. Mettre 0pt
 % pour supprimer le trait.

\setlength{\headheight}{14pt}

\title{\bf \vspace{-1cm} HEC 1982} %
\author{} %
\date{} %

\begin{document}

\maketitle %
\vspace{-1.2cm}\hrule %
\thispagestyle{fancy}

\vspace*{.4cm}

% DEBUT DU DOC À MODIFIER : tout virer jusqu'au début de l'exo

%Définition et changement de valeurs de
compteurs%newcounter{cpt1}{section} compteur cpt1 remis à 0 à chaque
aumentation par stepcounter du compteur section%setcounter{cpt1}{3} on
met le compteur à 3%addtocounter{cpt1}{5} on ajoute 5 au compteur%
stepcounter{cpt1} on ajoute 1% ifthenelse{test}{alors}{sinon} (page
206) pour subordonner à une condition % whiledo{test}{commande} pour
faire une boucle (page 206 aussi) % value{cpt1} pour noter dans le
document la valeur de cpt1 
%Définition définitive d'opérateurs
mathématiques\newcommand{\ch}{\operatorname{ch}} 
\newcommand{\sh}{\operatorname{sh}}
\renewcommand{\tanh}{\operatorname{th}}
\renewcommand{\sinh}{\operatorname{sh}}
\renewcommand{\cosh}{\operatorname{ch}}
\newcommand{\argsh}{\operatorname{argsh}}
\newcommand{\argch}{\operatorname{argch}}
\newcommand{\argth}{\operatorname{argth}}
\newcommand{\Id}{\operatorname{Id}}
\renewcommand{\leq}{\leq}
\renewcommand{\geq}{\geq }

\newcommand{\dlim}{\lim}
\newcommand{\dsum}{\sum}
\newcommand{\dprod}{\prod}



%Définition de nouvelles couleurs : rgb(trois paramètres red green blue
entre 0 et 1); cmyk (quatre cyan magenta yellow black) entre 0 et 1;
gray (entre 0 et 1) et black, white, red, green, blue, cyan, magenta,
yellow% definecolor{0gris}{gray}{0.8} 
% Nouvelle commande pour encadrer le titre car shabox ne veut que d'une
seule ligne; ATTENTION A LA TAILLE; petite différence avec shadowbox ou
doublebox, voire fcolorbox ou colorbox (au lieu de shabox; laisser le
parbox tranquille sauf pour la taille de la boîte
\newcommand{\Tbox}[1]{\begin{center} \shabox{\parbox{0.6
\linewidth}{#1}} \end{center}} %[1] pour 1 paramètre ; #1 pour ce que
fait le 1er paramètre; entre accolades ce que fait la commande
%Mise en page en mode fancy : en-têtes et pieds de pages puis
définition des en-têtes et pieds de pages\pagestyle{fancy}
\lhead{ECE 2 - Mathématiques \\
Quentin Dunstetter - ENC-Bessières 2011$\backslash$2012}
\chead{}
\rhead{HEC 1982}
\rfoot[ \ \thepage]{\thepage}
\cfoot{}
\lfoot{}
\thispagestyle{fancy} %Mise en page de la 1ère page en mode fancy
%Trait en bas et en haut de la page (entre en-tête et texte et texte et
pied de page)\renewcommand{\footrulewidth}{0.4pt}
\renewcommand{\headrulewidth}{0.4pt}

\begin{center}
{\huge HEC Eco 1982}
\end{center}

\section*{PREMIER\ EXERCICE}

On désigne par $\mathfrak{M}_{3}(\R)$ l'algèbre des matrices carrées
d'ordre 3 à éléments réels.\\
On considère les éléments suivants de $\mathfrak{M}_{3}(\R)$
\[
I = 
\begin{smatrix}
1 & 0 & 0 \\
0 & 1 & 0 \\
0 & 0 & 1
\end{smatrix}
\qquad J = 
\begin{smatrix}
1 & 1 & 1 \\
1 & 1 & 1 \\
1 & 1 & 1
\end{smatrix}
\qquad \text{et}\qquad K = 
\begin{smatrix}
4 & 3 & 3 \\
3 & 4 & 3 \\
3 & 3 & 4
\end{smatrix}
\]

\begin{noliste}{1.}
 \setlength{\itemsep}{4mm}
\item Exprimer $K$ comme combinaison linéaire de $I$ et $J.$ Pour tout
entier naturel non nul $n,$ calculer $J^{n}$ en fonction de $J$ et de
$n.$
En déduire $K^{n}.$ Expliciter les éléments de $K^{n}$ en numération
décimale.

\item Soit $E$ l'ensemble des matrices de la forme $M(a,b) = 
\begin{smatrix}
a & b & b \\
b & a & b \\
b & b & a
\end{smatrix}
$ où $a$ et $b$ parcourent $\R.$ Montrer que $E$ est une sous-algèbre
de $\mathfrak{M}_{3}(\R).$\\
(On rappelle qu'une sous-algèbre d'une algèbre en est exactement une
partie
non vide, stable par les lois induites, internes et externes).\\
Quelle est la dimension de l'espace vectoriel réel sous-jacent à $E$ ?

\item Soit $B =
(\overrightarrow{i},\overrightarrow{j},\overrightarrow{k})$ la
base canonique de $\R^{3},$ soit $f_{a,b}$ l'endomorphisme de $\R^{3}$
dont la matrice dans la base $B$ est $M(a,b).$

\begin{noliste}{a)}
 \setlength{\itemsep}{2mm}
\item Vérifier que $Q = 
\begin{smatrix}
1 & 1 & -1 \\
1 & 1 & 1 \\
1 & -2 & 0
\end{smatrix}
$ est la matrice de passage de $B$ à une base
$(\overrightarrow{I},\overrightarrow{J},\overrightarrow{K})$ constituée
par des vecteurs propres
de $f_{a,b}.$ Préciser la valeur propre associée à chacun de ces
vecteurs
propres. Déterminer tous les sous-espaces propres et donner leurs
dimensions.

\item On désigne par $^{t}Q$ la matrice transposée de $Q.$ Calculer le
produit $^{t}Q.Q.$ En déduire la matrice $Q^{-1}.$

\item Quelle est la matrice de $f_{a,b}$ dans la base
$(\overrightarrow{I},\overrightarrow{J},\overrightarrow{K})$ ? Et à
l'aide des questions a) et b)
calculer $M^{n}(a,b)$ pour tout entier naturel non nul.

\item Retrouver ainsi l'expression de $K^{n}$ obtenu à la question 1.
\end{noliste}
\end{noliste}

\section*{DEUXIEM\E\ EXERCICE}

Pour tout entier naturel $n,$ on pose $I_{n} = \dint{0}{\ln \sqrt{3}}(
\th x)^{n}dx$, où $\ln $ désigne le logarithme népérien.\\
On rappelle que, pour tout nombre réel $x,$ $\th x =
\dfrac{e^{x}-e^{-x}}{e^{x} + e^{-x}} = \dfrac{e^{2x}-1}{e^{2x} + 1}.$
On rappelle aussi que la fonction 
$\th$ est strictement croissante, dérivable sur $\R$ et que $(\th
x)^{\prime } = 1-(\th x)^{2}.$

\begin{noliste}{1.}
 \setlength{\itemsep}{4mm}
\item Montrer que, pour tout entier naturel $n,$ on a : $0\leq
I_{n}\leq \dfrac{\ln \sqrt{3}}{2^{n}}.$ En déduire la limite de $I_{n}$
quand $n$ tend vers $ + \infty.$

\item Calculer $I_{0}$ et $I_{1}.$

\item Montrer que $I_{n}-I_{n + 2} = \dfrac{1}{(n + 1)2^{n + 1}}.$

\item On considère les séries de termes généraux $u_{n} = \dfrac{1}{(2n
+ 1)2^{2n + 1}},$ où $n\geq 0$, et $v_{n} = \dfrac{1}{n2^{n}}$ où
$n\geq 1.$

\begin{noliste}{a)}
 \setlength{\itemsep}{2mm}
\item Montrer que ces deux séries sont convergentes.

\item En utilisant les deux questions 2) et 3), calculer les sommes de
ces
deux séries.
\end{noliste}
\end{noliste}

\newpage

\section*{TROISIEM\E\ EXERCICE}

On considère deux pièces de monnaie notées $A_{1}$ et $A_{2}$.
Lorsqu'on
lance la pièce $A_{1},$ la probabilité d'obtenir "face" est $p_{1},$
$0\leq p_{1}\leq 1,$ celle d'obtenir "pile" est $q_{1} = 1-p_{1}.$ De
même, lorsqu'on lance la pièce $A_{2},$ la probabilité d'obtenir "face"
est $p_{2},$ $0\leq p_{2}\leq 1,$ celle d'obtenir "pile" est $q_{2} =
1-p_{2}.$\\
On effectue une suite de parties de la façon suivante : à la première
partie, on choisit une pièce au hasard (avec la probabilité
$\dfrac{1}{2})$
et on joue avec cette pièce; si le résultat est "face", on joue la
deuxième
partie avec $A_{1},$ si le résultat est "pile", on joue la deuxième
partie
avec $A_{2};$ ensuite, pour tout entier $n\geq 1,$ on joue la $(n +
1)^{\text{ème}}$ partie avec $A_{1}$ si l'on a obtenu "face" à la
$n^{\text{ème}} $ partie, on joue la $(n + 1)^{\text{ème}}$ partie avec
$A_{2}$ si on a
obtenu "pile" à la $n^{\text{ème}}$ partie.

\begin{noliste}{1.}
 \setlength{\itemsep}{4mm}
\item Pour tout entier $n\geq 1,$ on note $u_{n}$ la probabilité
d'avoir "face" à la $n^{\text{ème}}$ partie.

\begin{noliste}{a)}
 \setlength{\itemsep}{2mm}
\item Exprimer $u_{1},$ puis $u_{2}$ en fonction de $p_{1}$ et $p_{2}.$

\item Montrer que, pour tout $n\geq 1,$
\[
u_{n + 1} = (p_{1}-p_{2})u_{n} + p_{2}.
\]

\item Montrer que la suite $(u_{n})_{n\geq 1}$ tend, quand $n$ tend
vers l'infini, vers une limite $u$ que l'on calculera.\\
Dans quels cas a-t-on $u = \dfrac{1}{2}$ ?
\end{noliste}

\item Pour tout entier $n\geq 1,$ on note $X_{n}$ la variable
aléatoire, associée à la $n^{\text{ème}}$ partie, qui prend la valeur
$1$ si le résultat de la $n^{\text{ème}}$ partie est "face", la valeur
$0$ si le résultat de la $n^{\text{ème}}$ partie est "pile".

\begin{noliste}{a)}
 \setlength{\itemsep}{2mm}
\item Déterminer les lois de probabilité des variables aléatoires
$X_{1}$ et 
$X_{2}$ et calculer leurs espérances mathématiques.

\item Les variables aléatoires $X_{1}$ et $X_{2}$ sont-elles
indépendantes ?
\end{noliste}
\end{noliste}

\section*{QUATRIEM\E\ EXERCICE}

Soit $X$ une variable aléatoire réelle, définie sur l'espace de
probabilité $(\Omega,\mathcal{A},P),$ dont la loi de probabilité admet
une densité $f$ définie, pour tout $x$ réel, par 
\[
f(x) = \left\{ 
\begin{array}{cc}
\dfrac{3}{x^{4}} & \text{si }x\geq 1\\
0 & \text{si }x<1
\end{array}
\right.
\]

\begin{noliste}{1.}
 \setlength{\itemsep}{4mm}
\item Vérifier que $f$ définit bien une densité de probabilité et la
représenter graphiquement.

\item Déterminer la fonction de répartition $F$ de $X$ et représenter
son
graphe.

\item La variable aléatoire $X$ possède-t-elle une espérance
mathématique ?
La calculer si elle existe.

\item Calculer, pour $x$ et $a$ réels, l'expression $F_{a}(x) =
P\left(\Ev{X<x/X\geq a}\right)$, probabilité conditionnelle de
l'évènement $\{X<x\}$ sachant
que l'évènement $\{X\geq a\}$ est réalisé.

\item Soit $a\geq 1$ et soit $Y$ la variable aléatoire définie sur
$(\Omega,\mathcal{A},P)$ par $Y = aX.$ Trouver la fonction de
répartition de $Y;$ en déduire sa densité de probabilité. Calculer
l'espérance mathématique
de $Y.$
\end{noliste}

\begin{center}
- FIN -
\end{center}

\label{fin}

\end{document}

