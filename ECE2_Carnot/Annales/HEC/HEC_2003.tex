\documentclass[11pt]{article}%
\usepackage{geometry}%
\geometry{a4paper,
 lmargin = 2cm,rmargin = 2cm,tmargin = 2.5cm,bmargin = 2.5cm}

\input{../../../../../../macros.tex}

\pagestyle{fancy} %
\lhead{ECE2 \hfill septembre 2017 \\
 Mathématiques\\[.2cm]} %
\chead{\hrule} %
\rhead{} %
\lfoot{} %
\cfoot{} %
\rfoot{\thepage} %

\renewcommand{\headrulewidth}{0pt}% : Trace un trait de séparation
 % de largeur 0,4 point. Mettre 0pt
 % pour supprimer le trait.

\renewcommand{\footrulewidth}{0.4pt}% : Trace un trait de séparation
 % de largeur 0,4 point. Mettre 0pt
 % pour supprimer le trait.

\setlength{\headheight}{14pt}

\title{\bf \vspace{-1cm} HEC 2003} %
\author{} %
\date{} %

\begin{document}

\maketitle %
\vspace{-1.2cm}\hrule %
\thispagestyle{fancy}

\vspace*{.4cm}

% DEBUT DU DOC À MODIFIER : tout virer jusqu'au début de l'exo

%Définition et changement de valeurs de
compteurs%newcounter{cpt1}{section} compteur cpt1 remis à 0 à chaque
aumentation par stepcounter du compteur section%setcounter{cpt1}{3} on
met le compteur à 3%addtocounter{cpt1}{5} on ajoute 5 au compteur%
stepcounter{cpt1} on ajoute 1% ifthenelse{test}{alors}{sinon} (page
206) pour subordonner à une condition % whiledo{test}{commande} pour
faire une boucle (page 206 aussi) % value{cpt1} pour noter dans le
document la valeur de cpt1 
%Définition définitive d'opérateurs
mathématiques\newcommand{\ch}{\operatorname{ch}} 
\newcommand{\sh}{\operatorname{sh}}
\renewcommand{\tanh}{\operatorname{th}}
\renewcommand{\sinh}{\operatorname{sh}}
\renewcommand{\cosh}{\operatorname{ch}}
\newcommand{\argsh}{\operatorname{argsh}}
\newcommand{\argch}{\operatorname{argch}}
\newcommand{\argth}{\operatorname{argth}}
\newcommand{\Id}{\operatorname{Id}}
\renewcommand{\leq}{\leq}
\renewcommand{\geq}{\geq }

\newcommand{\dlim}{\lim}
\newcommand{\dsum}{\sum}
\newcommand{\dprod}{\prod}



%Définition de nouvelles couleurs : rgb(trois paramètres red green blue
entre 0 et 1); cmyk (quatre cyan magenta yellow black) entre 0 et 1;
gray (entre 0 et 1) et black, white, red, green, blue, cyan, magenta,
yellow% definecolor{0gris}{gray}{0.8} 
% Nouvelle commande pour encadrer le titre car shabox ne veut que d'une
seule ligne; ATTENTION A LA TAILLE; petite différence avec shadowbox ou
doublebox, voire fcolorbox ou colorbox (au lieu de shabox; laisser le
parbox tranquille sauf pour la taille de la boîte
\newcommand{\Tbox}[1]{\begin{center} \shabox{\parbox{0.6
\linewidth}{#1}} \end{center}} %[1] pour 1 paramètre ; #1 pour ce que
fait le 1er paramètre; entre accolades ce que fait la commande
%Mise en page en mode fancy : en-têtes et pieds de pages puis
définition des en-têtes et pieds de pages\pagestyle{fancy}
\lhead{ECE 2 - Mathématiques \\
Quentin Dunstetter - ENC-Bessières 2011$\backslash$2012}
\chead{}
\rhead{HEC 2003}
\rfoot[ \ \thepage]{\thepage}
\cfoot{}
\lfoot{}
\thispagestyle{fancy} %Mise en page de la 1ère page en mode fancy
%Trait en bas et en haut de la page (entre en-tête et texte et texte et
pied de page)\renewcommand{\footrulewidth}{0.4pt}
\renewcommand{\headrulewidth}{0.4pt}

\vbox{} \vskip 3cm

\begin{center}
{\Large {\textbf{ÉCOLE DES HAUTES ÉTUDES COMMERCIALES}} \vspace{0.3cm}
}

{\normalsize CONCOURS D'ADMISSION SUR CLASSES PREPARATOIRES
\vspace{0.5cm}}

{\normalsize \hskip 0.0cm\hbox to 3cm{\hrulefill}}

{\normalsize \vspace{0.3cm} }

{\normalsize \textbf{OPTION ÉCONOMIQUE} \vspace{0.5cm} }

{\Large {\textbf{MATHEMATIQUES III}} \vspace{0.5cm} }

{\normalsize Mercredi 7 mai 2003, de 8h à 12h.}

{\normalsize \hskip 0.0cm\hbox to 3cm{\hrulefill}}

{\normalsize \vspace{0.3cm} }
\end{center}

\textit{La présentation, la lisibilité, l'orthographe, la qualité de
la rédaction, la clarté et la précision des\hfill\break raisonnements
entreront pour une part importante dans l'appréciation des copies.
\hfill\break Les candidats sont invités à encadrer dans la mesure du
possible les résultats de leurs calculs. \hfill\break Ils ne doivent
faire
usage d'aucun document ; l'utilisation de toute calculatrice et de tout
matériel\break électronique est interdite. \hfill\break Seule
l'utilisation d'une règle graduée est autorisée.} \vspace{0.5cm}

\hrule
\vspace{0.5cm}

\begin{center}
\textbf{EXERCICE}
\end{center}

\begin{noliste}{1.}
 \setlength{\itemsep}{4mm}
\item Soit $a$ et $b$ deux réels strictement positifs et $A$ la matrice
carré d'ordre $2$ définie par :\ $ A = 
\begin{smatrix}
a & b \\
b & a
\end{smatrix}
$.

\begin{noliste}{a)}
 \setlength{\itemsep}{2mm}
\item Montrer que si $a$ et $b$ sont égaux, la matrice $A$ n'est pas
inversible.

\item Calculer la matrice \ $A^{2}-2aA$. En déduire que, si $a$ et $b$
sont distincts, la matrice $A$ est inversible et donner la matrice
$A^{-1}$.

\item Montrer que les valeurs propres de $A$ sont $a + b$ et $a-b$.

\item On pose $ \Delta = 
\begin{smatrix}
a + b & 0 \\
0 & a-b
\end{smatrix}
$\. Déterminer une matrice $Q$, carrée d'ordre $2$ à
coefficients réels, inversible et dont les éléments de la première
ligne sont égaux à $1$, vérifiant \ $A = Q\Delta Q^{-1}$.

\item Calculer la matrice $Q^{-1}$ et, à l'aide de la question
précédente, calculer la matrice $A^{n}$ pour tout entier naturel non
nul $n$.
\end{noliste}

\item Soit $p$ un réel vérifiant $0<p<1$ et $q$ le réel $1-p$.
On suppose que $X$ et $Y$ sont deux variables aléatoires définies
sur le même espace probabilisé $(\Omega,\mathcal{A},\mathbf{P})$,
indépendantes et suivant la même loi géométrique de paramètre $p$.\\
Pour tout $\omega $ de $\Omega $, on désigne par $M(\omega )$ la
matrice
carrée d'ordre $2$ :\ $ 
\begin{smatrix}
X(\omega) & Y(\omega) \\
Y(\omega) & X(\omega)\end{smatrix}
$ et on note $S(\omega )$ (respectivement $D(\omega )$) la plus grande
(respectivement la plus petite) valeur propre de $M(\omega )$ et on
définit ainsi deux variables aléatoires sur
$(\Omega,\mathcal{A},\mathbf{P})$.

\begin{noliste}{a)}
 \setlength{\itemsep}{2mm}
\item Montrer que la probabilité de l'événement $[X = Y]$ est donnée
par :\ $\mathbf{P}([X = Y]) = \dfrac{p}{2-p}$ \ et en déduire la
probabilité de l'événement $\{\omega \in \Omega \;;\;M(\omega
)\;\text{est inversible}\}$.

\item Calculer la covariance des variables aléatoires $S$ et $D$.

\item Calculer les probabilités \ $\mathbf{P}([S = 2]\cap [D =
0]),\;\mathbf{P}([S = 2])$ \ et\ $\mathbf{P}([D = 0])$.\\
Les variables aléatoires $S$ et $D$ sont-elles indépendantes ?

\item Établir, pour tout entier naturel $n$ supérieur ou égal 
à $2$ : \ $\mathbf{P}([S = n]) = (n-1)p^{2}q^{n-2}$.

\item En déduire, lorsque $p$ est égal à $\dfrac{2}{21}$\,\ que
la valeur la plus probable de la plus grande valeur propre des matrices
$M(\omega )$ possibles est $11$.
\end{noliste}
\end{noliste}

\section*{PROBL\`{E}ME}

\subsection*{Partie A : Étude d'une fonction\protect\vspace{0.4cm}}

{\normalsize \label{P1} }

\begin{noliste}{1.}
 \setlength{\itemsep}{4mm}
\item 
\begin{noliste}{a)}
 \setlength{\itemsep}{2mm}
\item On suppose, dans cette question, qu'il existe une fonction $f$ de
classe $\mathcal{C}{1}$ sur les intervalles $]-\infty,0[$ et $]0,1[$,
vérifiant pour tout réel $x$ appartenant à $]-\infty,0[ \ \cup ]0,1[
$, l'égalité :
\[
x(1-x)f^{\prime }(x) + (1-x)f(x) = 1
\]
Soit $h$ la fonction définie sur $]-\infty,0[ \ \cup ]0,1[$, par : \
$h(x) = xf(x)$.\\
Montrer que $h$ est de classe $\mathcal{C}{1}$ sur les intervalles
$]-\infty,0[$, $]0,1[$ et calculer sa dérivée.\\
En déduire qu'il existe deux constantes réelles $c_{1}$ et $c_{2}$
vérifiant 
\[
\left\{ 
\begin{array}{cccc}
\forall x\in \ ]-\infty,0[, & h(x) & = & -\ln (1-x) + c_{1} \\
\forall x\in \ ]0,1[, & h(x) & = & -\ln (1-x) + c_{2}
\end{array}
\right. 
\]

\item On définit une fonction $f$ sur les intervalles $]-\infty,0[$ et
$]0,1[$ par :
\[
\left\{ 
\begin{array}{cccc}
\forall x\in \ ]-\infty,0[, & f(x) & = & \dfrac{-\ln (1-x) + c_{1}}{x}
\\
\forall x\in \ ]0,1[, & f(x) & = & \dfrac{-\ln (1-x) + c_{2}}{x}
\end{array}
\right. 
\]
où $c_{1}$ et $c_{2}$ sont deux constantes réelles.\\
Déterminer les constantes $c_{1}$ et $c_{2}$ pour que la fonction $f$
soit prolongeable par continuité en $0$.
\end{noliste}

\item \label{deff} Dans toute la suite de cette partie, $f$ désigne la
fonction définie sur $]-\infty,1[$ par :
\[
\left\{ 
\begin{array}{cc}
f(x) = -\dfrac{\ln (1-x)}{x} & \text{si }x\neq 0 \\
f(0) = 1 & 
\end{array}
\right. 
\]

\begin{noliste}{a)}
 \setlength{\itemsep}{2mm}
\item Donner le développement limité en $0$ à l'ordre $3$ de la
fonction \ $x\mapsto \ln (1-x)$ puis le développement limité en $0$ 
à l'ordre $2$ de la fonction $ x\mapsto -\frac{\ln (1-x)}{x}\;\cdotp$

\item En déduire que la fonction $f$ est continue en $0$, dérivable
en $0$ et préciser la valeur de $f^{\prime }(0)$.

\item Montrer que, pour tout $x$ de $]-\infty,0[ \ \cup ]0,1[$, on a : 
\[
f^{\prime }(x) = \left( \frac{1}{1-x}-f(x)\right) \,\frac{1}{x}
\]
En utilisant le développement limité de la question précédente, montrer
que $f$ est de classe $\mathcal{C}{1}$ sur $]-\infty,1[$.
\end{noliste}

\item 
\begin{noliste}{a)}
 \setlength{\itemsep}{2mm}
\item Étudier le signe de la fonction $\varphi $ définie sur
$]-\infty,1[$ par : \ $\varphi (x) = \dfrac{x}{1-x} + \ln
(1-x)\;\cdotp$ \\
En déduire les variations de la fonction $f$

\item Donner le tableau de variation de la fonction $f$ et l'allure de
la
représentation graphique de $f$ en précisant les asymptotes, la
tangente à l'origine et la position de la courbe par rapport à cette
tangente au voisinage de l'origine.
\end{noliste}

\newpage

\item Soit $x$ un réel de l'intervalle $]0,1[$.

\begin{noliste}{a)}
 \setlength{\itemsep}{2mm}
\item Soit $h$ la fonction définie sur $]-\infty,1[$ par : \ $h(t) =
-\ln
(1-t)$\.\\
Calculer, pour tout réel $t$ de $]-\infty,1[$, $h^{\prime
}(t),\;h^{\prime \prime }(t)$, puis pour tout entier naturel $n$ non
nul, $h^{(n)}(t)$.

\item Justifier, pour tout entier naturel $n$, l'égalité : 
\[
h(x) = \Sum{k = 1}{n + 1}\frac{x^{k}}{k} + \dint{0}{x}\frac{(x-t)^{n +
1}}{(1-t)^{n + 2}}\;\text{d}t
\]

\item Établir, pour tout réel $t$ de l'intervalle $[0,x]$, la double
inégalité :\ $0\leq \dfrac{x-t}{1-t}\leq x\;\cdotp$ \\
En déduire, pour tout entier naturel $n$ non nul, la double inégalité :
\[
0\leq f(x)-\Sum{k = 0}{n}\frac{x^{k}}{k + 1}\leq x^{n + 1}f(x)
\]

\item Justifier l'égalité :\qquad $ f(x) = \Sum{n = 0}{\infty
}\frac{x^{n}}{n + 1}\;\cdotp$\ 
\end{noliste}
\end{noliste}

\subsection*{Partie B : Étude d'une variable aléatoire à densité}

\begin{noliste}{1.}
 \setlength{\itemsep}{4mm}
\item Dans cette question $f$ est la fonction définie à la question
\ref{deff} de la partie \ref{P1}.

\begin{noliste}{a)}
 \setlength{\itemsep}{2mm}
\item Soit $f_{1}$ la fonction définie sur $]0,1]$ par :\ $\left\{ 
\begin{matrix}
f_{1}(t) = \dfrac{\ln t}{t-1} & \text{si} & t\neq 1 \\
f_{1}(1) = 1 & & 
\end{matrix}\right. $\\
Justifier la continuité de $f_{1}$ sur $]0,1]$ et établir, pour tout
réel $x$ de $]0,1[$, l'égalité :\ 
\[
\dint{0}{x}f(t)\,\text{d}t = \dint{1-x}{1}f_{1}(t)\,\text{d}t
\]

\item \label{1b} Soit $a$ un réel de l'ntervalle $]0,1[$. Établir
pour tout entier naturel $n$, l'égalité :
\[
\dint{a}{1}t^{n}\ln t\,\text{d}t = -\frac{a^{n + 1}\ln a}{n +
1}-\frac{1}{(n + 1)^{2}}(1-a^{n + 1})
\]
En déduire la convergence de l'intégrale\ $\dint{0}{1}t^{n}\ln
t\,\text{d}t$ et l'égalité :\quad $\dint{0}{1}t^{n}\ln t\,\text{d}t =
-\frac{1}{(n + 1)^{2}}\;\cdotp$

\item Soit $a$ un réel de l'intervalle $]0,1[$ et $n$ un entier
naturel,
démontrer pour tout $t$ de $[a,1]$, l'égalité
\[
\dint{a}{1}f_{1}(t)\,\text{d}t + \Sum{k = 0}{n}\dint{a}{1}t^{k}\ln
t\,\text{d}t = \dint{a}{1}t^{n + 1}f_{1}(t)\,\text{d}t
\]

\item Montrer que la fonction $t\mapsto t\,f_{1}(t)$ est prolongeable
en une
fonction $h_{1}$ continue sur $[0,1]$.\\
En déduire que l'intégrale \ $\dint{0}{1}f_{1}(t)\,\text{d}t$ converge
et qu'elle vérifie :
\[
\dint{0}{1}f_{1}(t)\,\text{d}t = \Sum{k = 0}{n}\frac{1}{(k + 1)^{2}} +
\dint{0}{1}t^{n}\,h_{1}(t)\,\text{d}t
\]

\item On désigne alors par $M$ le maximum sur $[0,1]$ de la fonction
$h_{1}$.\\
Établir, pour tout entier naturel $n$, l'inégalité : 
\[
0\leq \dint{0}{1}f_{1}(t)\,\text{d}t-\Sum{k = 0}{n}\frac{1}{(k +
1)^{2}}\leq \frac{M}{n + 1}
\]

\item Justifier la convergence de la série de terme général
$\dfrac{1}{n^{2}}$\ \ puis l'égalité : \ $\dint{0}{1}f(t)\,\text{d}t =
\Sum{n = 1}{\infty }\frac{1}{n^{2}}\;\cdotp$
\end{noliste}

\item On donne : \ $\Sum{n = 1}{\infty }\frac{1}{n^{2}} = \frac{\pi
^{2}}{6}$\ et on désigne par $g$ la fonction définie sur $\R$ par : 
\[
\left\{ 
\begin{array}{cc}
g(t) = \dfrac{6}{\pi ^{2}}\,f(t) & \text{si}\quad t\in \lbrack 0,1[ \
\\
g(t) = 0 & \text{sinon}
\end{array}
\right. 
\]

\begin{noliste}{a)}
 \setlength{\itemsep}{2mm}
\item Vérifier que $g$ est une densité de probabilité.

\item Soit $X$ une variable aléatoire ayant pour densité $g$. 
\\
Vérifier, pour tout réel $x$ de $]0,1[$, l'égalité \ $ \dint{0}{x}\ln
(1-t)\,\text{d}t = \dint{1-x}{1}\ln t\,\text{d}t $.\ \\
Utiliser alors le résultat de la question \ref{1b} pour prouver que $X$
possède une espérance et la calculer.

\item Par une méthode analogue, montrer que l'intégrale \ $
\dint{0}{1}(t-1)\ln (1-t)\,\text{d}t$ est égale à
$\dfrac{1}{4}\;\cdotp$ \\
En déduire que laa variable aléatoire $X^{2}$ admet une espérance,
préciser sa valeur et calculer la variance de la variable aléatoire
$X$.
\end{noliste}
\end{noliste}

\subsection*{Partie C : Encadrement d'une fonction de deux variables}

Dans cette partie, on désigne par $V$ l'ensemble ouvert défini par :
\[
V = \{(x,y)\in
\R^{2};\;-\dfrac{1}{2}<x<\dfrac{1}{2},\;-\dfrac{1}{2}<y<\dfrac{1}{2}\}
\]

\begin{noliste}{1.}
 \setlength{\itemsep}{4mm}
\item Soit $u$ la fonction de $V$ dans $\R$ :\quad $(x,y)\mapsto
u(x,y) = xy^{2} + x^{2} + y^{2} + \dfrac{1}{4}$.

\begin{noliste}{a)}
 \setlength{\itemsep}{2mm}
\item Montrer que la fonction $u$ admet un minimum sur $V$ dont on
précisera la valeur, mais n'admet pas de maximum.

\item Montrer que la fonction $u$ est majorée par $\dfrac{7}{8}$ sur
l'ouvert $V$.
\end{noliste}

\item Soit $F$ la fonction :\qquad $ (x,y)\longmapsto F(x,y) =
\dfrac{\ln \left( \dfrac{3}{4}-xy^{2}-x^{2}-y^{2}\right) }{\dfrac{1}{4}
+ xy^{2} + x^{2} + y^{2}}\;\cdotp$

\begin{noliste}{a)}
 \setlength{\itemsep}{2mm}
\item \`{A} l'aide des résultats de la partie \ref{P1}, montrer que $F$
est définie sur l'ouvert $V$ et qu'elle y admet un maximum. Préciser
la valeur de ce maximum.

\item Donner un encadrement de $F(x,y)$ pour tout $(x,y)$ de $V$.
\end{noliste}
\end{noliste}

\label{derniere_{p}age}

\end{document}

