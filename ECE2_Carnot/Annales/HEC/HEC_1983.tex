\documentclass[11pt]{article}%
\usepackage{geometry}%
\geometry{a4paper,
 lmargin = 2cm,rmargin = 2cm,tmargin = 2.5cm,bmargin = 2.5cm}

\input{../../../../../../macros.tex}

\pagestyle{fancy} %
\lhead{ECE2 \hfill septembre 2017 \\
 Mathématiques\\[.2cm]} %
\chead{\hrule} %
\rhead{} %
\lfoot{} %
\cfoot{} %
\rfoot{\thepage} %

\renewcommand{\headrulewidth}{0pt}% : Trace un trait de séparation
 % de largeur 0,4 point. Mettre 0pt
 % pour supprimer le trait.

\renewcommand{\footrulewidth}{0.4pt}% : Trace un trait de séparation
 % de largeur 0,4 point. Mettre 0pt
 % pour supprimer le trait.

\setlength{\headheight}{14pt}

\title{\bf \vspace{-1cm} HEC 1983} %
\author{} %
\date{} %

\begin{document}

\maketitle %
\vspace{-1.2cm}\hrule %
\thispagestyle{fancy}

\vspace*{.4cm}

% DEBUT DU DOC À MODIFIER : tout virer jusqu'au début de l'exo

%Définition et changement de valeurs de
compteurs%newcounter{cpt1}{section} compteur cpt1 remis à 0 à chaque
aumentation par stepcounter du compteur section%setcounter{cpt1}{3} on
met le compteur à 3%addtocounter{cpt1}{5} on ajoute 5 au compteur%
stepcounter{cpt1} on ajoute 1% ifthenelse{test}{alors}{sinon} (page
206) pour subordonner à une condition % whiledo{test}{commande} pour
faire une boucle (page 206 aussi) % value{cpt1} pour noter dans le
document la valeur de cpt1 
%Définition définitive d'opérateurs
mathématiques\newcommand{\ch}{\operatorname{ch}} 
\newcommand{\sh}{\operatorname{sh}}
\renewcommand{\tanh}{\operatorname{th}}
\renewcommand{\sinh}{\operatorname{sh}}
\renewcommand{\cosh}{\operatorname{ch}}
\newcommand{\argsh}{\operatorname{argsh}}
\newcommand{\argch}{\operatorname{argch}}
\newcommand{\argth}{\operatorname{argth}}
\newcommand{\Id}{\operatorname{Id}}
\renewcommand{\leq}{\leq}
\renewcommand{\geq}{\geq }

\newcommand{\dlim}{\lim}
\newcommand{\dsum}{\sum}
\newcommand{\dprod}{\prod}



%Définition de nouvelles couleurs : rgb(trois paramètres red green blue
entre 0 et 1); cmyk (quatre cyan magenta yellow black) entre 0 et 1;
gray (entre 0 et 1) et black, white, red, green, blue, cyan, magenta,
yellow% definecolor{0gris}{gray}{0.8} 
% Nouvelle commande pour encadrer le titre car shabox ne veut que d'une
seule ligne; ATTENTION A LA TAILLE; petite différence avec shadowbox ou
doublebox, voire fcolorbox ou colorbox (au lieu de shabox; laisser le
parbox tranquille sauf pour la taille de la boîte
\newcommand{\Tbox}[1]{\begin{center} \shabox{\parbox{0.6
\linewidth}{#1}} \end{center}} %[1] pour 1 paramètre ; #1 pour ce que
fait le 1er paramètre; entre accolades ce que fait la commande
%Mise en page en mode fancy : en-têtes et pieds de pages puis
définition des en-têtes et pieds de pages\pagestyle{fancy}
\lhead{ECE 2 - Mathématiques \\
Quentin Dunstetter - ENC-Bessières 2011$\backslash$2012}
\chead{}
\rhead{HEC 1983}
\rfoot[ \ \thepage]{\thepage}
\cfoot{}
\lfoot{}
\thispagestyle{fancy} %Mise en page de la 1ère page en mode fancy
%Trait en bas et en haut de la page (entre en-tête et texte et texte et
pied de page)\renewcommand{\footrulewidth}{0.4pt}
\renewcommand{\headrulewidth}{0.4pt}

\begin{center}
{\huge HEC Eco 1983}
\end{center}

\section*{PREMIER\ EXERCICE}

On considère deux verres A et B de volume unité, pleins à ras bord, le
premier d'eau, le second d'alcool pur. On dispose en outre d'un flacon
F
pour faire les mélanges.

\begin{noliste}{1.}
 \setlength{\itemsep}{4mm}
\item 

\begin{noliste}{a)}
 \setlength{\itemsep}{2mm}
\item On verse dans F, d'une part le volume $\dfrac{1}{n}$ du contenu
de A (où $n$ est un entier naturel non nul), d'autre part tout le
contenu du verre
B. Après mélange, on remplit B avec le contenu de F. L'excédent de
liquide
resté dans F, de volume $\dfrac{1}{n},$ est jeté.\\
Quel est le volume $q_{n}(1)$ d'alcool se trouvant dans le verre B à
l'issue
de cette opération (a) ?

\item On répète $k$ fois l'opération (a) ($k\leq n$), en tirant à A à
chaque fois un volume $\dfrac{1}{n}.$\\
Quel est le volume $q_{n}(k)$ d'alcool se trouvant dans le verre B à
l'issue
de la $k^{\text{ème}}$ opération (a) ?

\item On pose $q_{n} = q_{n}(n).$ Montrer que la suite $(q_{n})$ admet
une
limite $q.$

\item Calculer la précision de $10^{-3}$ de combien $q$ est inférieur à
$q_{1}(1).$\\
\emph{Par commodité de langage, on parlera dans la suite de "versement
goutte à goutte d'un verre sur un autre" (goutte }$\dfrac{1}{n}$\emph{\
aussi petite qu'on le veut) pour désigner à la limite le processus
décrit en
1. Ainsi le versement goutte à goutte du verre d'eau A sur le verre
d'alcool
B produit dans B le mélange des volumes }$1-q$\emph{\ d'eau et
}$q$\emph{\
d'alcool, comme il a été calculé ci-dessus. (On se tiendra à l'écart de
toute discussion relative à la physique des liquides)}
\end{noliste}

\item 

\begin{noliste}{a)}
 \setlength{\itemsep}{2mm}
\item Le verre A est plein d'eau, le verre B plein du mélange des
volumes $P_{0}$ d'eau et $q_{0}$ d'alcool ($p_{0} + q_{0} = 1)$\\
Quel est l'effet du versement goutte à goutte de A sur B ?

\item Le verre A est plein d'alcool, le verre B plein du mélange des
volumes 
$P_{0}$ d'eau et $q_{0}$ d'alcool.\\
Quel est l'effet du versement goutte à goutte de A sur B ?

\item Sur le verre B dans l'état initial $(p_{0},q_{0})$ on verse
goutte à
goutte un verre d'eau, puis un verre d'alcool.\\
Exprimer le volume $p_{1}$ d'eau obtenu dans B, en fonction de
$p_{0}.$\\
On note $f$ la fonction ainsi définie $f :p_{0}\mapsto p_{1}.$ Tracer
dans un
repère orthonormé les représentations graphiques des fonctions $f$ et
$I :p_{0}\mapsto p_{0}.$

\item Que devient le mélange du verre B si on itère la double opération
du
versement goutte à goutte sur B d'un verre d'eau et d'un verre d'alcool
?
\end{noliste}
\end{noliste}

\section*{DEUXIEM\E\ EXERCICE}

Soit $k$ un nombre réel.\\
On considère la suite réelle $(u_{n})$ définie par la donnée de ses
premiers
termes :
\[
u_{0} = 2k,\quad u_{1} = 1 + k
\]
et par la relation de récurrence :
\[
(\forall n\in \N)\qquad u_{n + 2} = u_{n + 1} +
(k^{2}-\dfrac{1}{4})u_{n}.
\]

\begin{noliste}{1.}
 \setlength{\itemsep}{4mm}
\item Calculer $u_{n}$ en fonction de $n,$ suivant les valeurs de $k.$

\item Déterminer la limite, lorsqu'elle existe de $u_{n}$ quand $n$
tend
vers $ + \infty.$ Ici encore on discutera suivant les valeurs de $k.$
\end{noliste}

\section*{TROISIEM\E\ EXERCICE}

Soient $n$ un entier naturel non nul et $f$ une fonction numérique $n$
fois dérivable sur $\R,$ dont la dérivée $n^{\text{ème}}$ est continue.
Soit $(a_{0},a_{1},...,a_{n})$ une suite strictement croissante de
réels. On
considère le polynôme
\[
P\left(\Ev{X}\right) = \left(\Ev{X-a_{0}}\right)(X-a_{1})\cdots
(X-a_{n})
\]
Pour tout entier naturel $k$ tel que $k\leq n,$ on note $A_{k}$ le
polynôme défini par 
\[
A_{k}(X) = \dfrac{P\left(\Ev{X}\right)}{X-a_{k}}
\]

\begin{noliste}{1.}
 \setlength{\itemsep}{4mm}
\item Exprimer $A_{k}(a_{k})$ à l'aide de la dérivée de $P.$

\item Soit $g$ la fonction numérique définie sur $\R$ par la
relation :
\[
g(x) = f(x)-\Sum{k = 0}{n}f(a_{k})\dfrac{A_{k}(x)}{A_{k}(a_{k})}.
\]

\begin{noliste}{a)}
 \setlength{\itemsep}{2mm}
\item Calculer les nombres $g(a_{0}),$ $g(a_{1}),...,g(a_{n}).$

\item Montrer qu'il existe un élément $\lambda $ de $]a_{0},a_{n}[$ tel
que :
\[
f^{(n)}(\lambda ) = n!\Sum{k = 0}{n}\dfrac{f(a_{k})}{A_{k}(a_{k})}.
\]

\item Montrer qu'il existe un élément $x_{1}$ de $]a_{0},\lambda
\lbrack $
et un élément $x_{2}$ de $]\lambda,a_{n}[$ tels que :
\[
f^{(n-1)}(x_{2})-f^{(n-1)}(x_{1}) = (x_{2}-x_{1})f^{(n)}(\lambda )
\]
\end{noliste}

\item On considère le cas particulier suivant :\begin{eqnarray*}
f(x) & = & \sqrt{2}\cos (\dfrac{\pi }{4}x) \\
n & = & 2,\quad a_{0} = -1,\quad a_{1} = 0,\quad a_{2} = 1
\end{eqnarray*}

\begin{noliste}{a)}
 \setlength{\itemsep}{2mm}
\item Expliciter $g.$

\item Calculer $\dfrac{\pi ^{2}}{16}\cos (\dfrac{\pi }{4}\lambda ).$
\end{noliste}
\end{noliste}

\section*{QUATRIEM\E\ EXERCICE}

Soit $(X_{n},\;n\geq 1)$ une suite de variables aléatoires réelles,
définies sur l'espace de probabilité $(\Omega,\mathcal{A},P).$ On
suppose que
ces variables suivant toutes la même loi de Bernouilli de paramètre $p$
:
\[
P\left(\Ev{X_{n} = 1}\right) = p\qquad P\left(\Ev{X_{n} = 0}\right) =
1-p,\qquad 0<p<1.
\]
On construit la suite $(Y_{n},\;n\geq 1)$ de variables aléatoires de la
façon suivante :
\[
Y_{1} = X_{1},\quad Y_{2} = Y_{1}X_{2},...,\quad Y_{n} =
Y_{n-1}X_{n},...
\]
$Y_{n}$ est la variable aléatoire produit des variables $Y_{n-1}$ et
$X_{n}.$\\
On suppose que pour tout entier $n\geq 2$ les variables aléatoires
$Y_{n-1}$ et $X_{n}$ sont indépendantes.

\begin{noliste}{1.}
 \setlength{\itemsep}{4mm}
\item Déterminer, pour tout $n\geq 1,$ la loi de probabilité de la
variable aléatoire $Y_{n}.$\\
Calculer l'espérance mathématique et la variance de $Y_{n}.$

\item Trouver, pour tout $n\geq 1,$ la loi de probabilité conjointe de
la variable aléatoire à deux dimensions $(X_{n},Y_{n}).$ Calculer la
covariance de $X_{n}$ et $Y_{n}.$

\item Déterminer, pour tout $n\geq 1,$ les lois de probabilité
conditionnelles de $X_{n}$ sachant $Y_{n}$ prend la valeur $0,$ et de
$X_{n}$
sachant que $Y_{n}$ prend la valeur $1.$

\item La suite des variables aléatoires $(Y_{n},\;n\geq 1)$
converge-t-elle en probabilité quand $n$ tend vers l'infini ? Si oui,
vers
quelle limite ?
\end{noliste}

\begin{center}
- FIN -
\end{center}

\label{fin}

\end{document}

