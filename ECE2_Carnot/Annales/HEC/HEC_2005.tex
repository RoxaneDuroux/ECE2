\documentclass[11pt]{article}%
\usepackage{geometry}%
\geometry{a4paper,
 lmargin = 2cm,rmargin = 2cm,tmargin = 2.5cm,bmargin = 2.5cm}

\input{../../../../../../macros.tex}

\pagestyle{fancy} %
\lhead{ECE2 \hfill septembre 2017 \\
 Mathématiques\\[.2cm]} %
\chead{\hrule} %
\rhead{} %
\lfoot{} %
\cfoot{} %
\rfoot{\thepage} %

\renewcommand{\headrulewidth}{0pt}% : Trace un trait de séparation
 % de largeur 0,4 point. Mettre 0pt
 % pour supprimer le trait.

\renewcommand{\footrulewidth}{0.4pt}% : Trace un trait de séparation
 % de largeur 0,4 point. Mettre 0pt
 % pour supprimer le trait.

\setlength{\headheight}{14pt}

\title{\bf \vspace{-1cm} HEC 2005} %
\author{} %
\date{} %

\begin{document}

\maketitle %
\vspace{-1.2cm}\hrule %
\thispagestyle{fancy}

\vspace*{.4cm}

% DEBUT DU DOC À MODIFIER : tout virer jusqu'au début de l'exo

%Définition et changement de valeurs de
compteurs%newcounter{cpt1}{section} compteur cpt1 remis à 0 à chaque
aumentation par stepcounter du compteur section%setcounter{cpt1}{3} on
met le compteur à 3%addtocounter{cpt1}{5} on ajoute 5 au compteur%
stepcounter{cpt1} on ajoute 1% ifthenelse{test}{alors}{sinon} (page
206) pour subordonner à une condition % whiledo{test}{commande} pour
faire une boucle (page 206 aussi) % value{cpt1} pour noter dans le
document la valeur de cpt1 
%Définition définitive d'opérateurs
mathématiques\newcommand{\ch}{\operatorname{ch}} 
\newcommand{\sh}{\operatorname{sh}}
\renewcommand{\tanh}{\operatorname{th}}
\renewcommand{\sinh}{\operatorname{sh}}
\renewcommand{\cosh}{\operatorname{ch}}
\newcommand{\argsh}{\operatorname{argsh}}
\newcommand{\argch}{\operatorname{argch}}
\newcommand{\argth}{\operatorname{argth}}
\newcommand{\Id}{\operatorname{Id}}
\renewcommand{\leq}{\leq}
\renewcommand{\geq}{\geq }

\newcommand{\dlim}{\lim}
\newcommand{\dsum}{\sum}
\newcommand{\dprod}{\prod}



%Définition de nouvelles couleurs : rgb(trois paramètres red green blue
entre 0 et 1); cmyk (quatre cyan magenta yellow black) entre 0 et 1;
gray (entre 0 et 1) et black, white, red, green, blue, cyan, magenta,
yellow% definecolor{0gris}{gray}{0.8} 
% Nouvelle commande pour encadrer le titre car shabox ne veut que d'une
seule ligne; ATTENTION A LA TAILLE; petite différence avec shadowbox ou
doublebox, voire fcolorbox ou colorbox (au lieu de shabox; laisser le
parbox tranquille sauf pour la taille de la boîte
\newcommand{\Tbox}[1]{\begin{center} \shabox{\parbox{0.6
\linewidth}{#1}} \end{center}} %[1] pour 1 paramètre ; #1 pour ce que
fait le 1er paramètre; entre accolades ce que fait la commande
%Mise en page en mode fancy : en-têtes et pieds de pages puis
définition des en-têtes et pieds de pages\pagestyle{fancy}
\lhead{ECE 2 - Mathématiques \\
Quentin Dunstetter - ENC-Bessières 2011$\backslash$2012}
\chead{}
\rhead{HEC 2005}
\rfoot[ \ \thepage]{\thepage}
\cfoot{}
\lfoot{}
\thispagestyle{fancy} %Mise en page de la 1ère page en mode fancy
%Trait en bas et en haut de la page (entre en-tête et texte et texte et
pied de page)\renewcommand{\footrulewidth}{0.4pt}
\renewcommand{\headrulewidth}{0.4pt}


\begin{center}
{\Huge HEC\ III 2005}
\end{center}

{\Large EXERCICE.}

Dans cet exercice, $n$ est un entier supérieur ou égal à 2. On
note $E$ l'espace vectoriel $\R^{n}$ et $\mathrm{Id}$ l'application
identité de $E$.

L'objet de l'exercice est l'étude des endomorphismes $f$ de $E$
vérifiant l'équation $\left( *\right) :f\circ f = 4\mathrm{Id}$

\textbf{A. Étude du cas }$\mathbf{n = 2}$\textbf{.}

Soit $f$ l'endomorphisme de $\R^{2}$ dont la matrice dans la base
canonique est : $A = \sqrt{2}\left( 
\begin{array}{cc}
1 & 1 \\
1 & -1
\end{array}
\right) $

Soit $u$ le vecteur de $\R^{2}$ défini par $u = \left( 
\begin{array}{c}
\sqrt{2}-2 \\
\sqrt{2}
\end{array}
\right) $.

\begin{noliste}{1.}
 \setlength{\itemsep}{4mm}
\item Montrer que $f$ vérifie l'équation $\left( *\right) $, puis
préciser le noyau et l'image de $f$.

\item On note $F = \ker \left( f-2\mathrm{Id}\right) $ et $G =
\mathrm{{Im}\left( f-2\mathrm{Id}\right).}$

\begin{noliste}{a)}
 \setlength{\itemsep}{2mm}
\item Montrer que $G$ est engendré par le vecteur $u$. En déduire la
dimension de $F$ et donner une base de $F$.

\item Vérifier que $G$ est le sous-espace propre de $f$ associé à
la valeur propre $-2$.
\end{noliste}

\item Montrer que $f$ est diagonalisable; préciser les valeurs propres
de $f$ et donner la matrice de passage de la base canonique à une base
de vecteurs propres.
\end{noliste}

\textbf{B. Étude du cas général.}

On se place désormais dans le cas où $n$ est supérieur ou égal à $2$,
et on considère un endomorphisme $f$ de $E$ vérifiant
l'équation $\left( *\right) $.

\begin{noliste}{1.}
 \setlength{\itemsep}{4mm}
\item 
\begin{noliste}{a)}
 \setlength{\itemsep}{2mm}
\item Justifier que $f$ est un automorphisme de $E$ et exprimer
l'automorphisme réciproque $f^{-1}$ en fonction de $f$

\item Déterminer les valeurs propres possibles de $f$.

\item Vérifier que $2\mathrm{Id}$ et $-2\mathrm{Id}$ satisfont
l'équation $\left( *\right) $.
\end{noliste}

\hspace{-1cm}On suppose dans la suite de l'exercice que $f\ne
2\mathrm{Id}$f
et $f\ne -2\mathrm{Id}$ et on note $F = \ker \left(
f-2\mathrm{Id}\right) $ et 
$G = \mathrm{{Im}\left( f-2\mathrm{Id}\right).}$

\item Soit $x$ un élément de $E$. Montrer que $\left( f\left(
x\right) -2x\right) $ appartient à $\ker \left( f + 2\mathrm{Id}\right)
$
et que $\left( f\left( x\right) + 2x\right) $ appartient à $F$.

En déduire que $G\subset \ker \left( f + 2\mathrm{Id}\right) $ et que
$\mathrm{{Im}\left( f + 2\mathrm{Id}\right) \subset F}$.

Montrer que $2$ et $-2$ sont les valeurs propres de $f$

\item Soit $x$ un vecteur de $\ker \left( f + 2\mathrm{Id}\right) $.

\begin{noliste}{a)}
 \setlength{\itemsep}{2mm}
\item Exprimer $\left( f-2\mathrm{Id}\right) \left( x\right) $ en
fonction
de $x$ uniquement.

En déduire que $x$ appartient à $G$, puis que $G = \ker \left( f +
2\mathrm{Id}\right) $

\item Montrer que $f$ est diagonalisable.
\end{noliste}
\end{noliste}



{\Large PROBL\`{E}ME.}

Dans tout le problème, $n$ désigne un entier naturel non nul.

On considère une urne blanche contenant $n$ boules blanches numérotées
de 1 à $n$ et une urne noire contenant $n$ boules noires numérotées de
1 à $n$, dans lesquelles on effectue des suites de
tirages. \`{A} chaque tirage, on tire simultanément et au hasard une
boule de chaque urne. On obtient ainsi à chaque tirage, deux boules,
une
blanche et une noire.

On dira qu'on a obtenu une paire lors d'un tirage, si la boule blanche
et la
boule noire tirées portent le même numéro.

\textbf{Partie I. Tirages avec remise}

\begin{noliste}{1.}
 \setlength{\itemsep}{4mm}
\item Dans cette question, on effectue les tirages avec remise jusqu'à
ce que l'on obtienne pour la première fois une paire.

\begin{noliste}{a)}
 \setlength{\itemsep}{2mm}
\item Préciser l'espace probabilisé $\left(
\Omega,\mathcal{A},\Prob\right) $ qui modélise cette expérience.

\item On note $Y$ la variable aléatoire égale au nombre de tirages
(de deux boules) effectués.

Déterminer la loi de $Y$ ; donner son espérance et sa variance.
\end{noliste}

\item Écrire en \Scilab{} une fonction dont l'en-tête est \texttt{pgrml
(n : integer) : integer} qui modélise l'expérience précédente.

\item Dans cette question, on suppose que $n = 2$. On effectue des
tirages
avec remise jusqu'à ce que l'on obtienne pour la première fois la
boule blanche numérotée 1. On note $U$ la variable aléatoire 
égale au nombre de tirages effectués, et $Z$ la variable aléatoire
égale au nombre de paires obtenues à l'issue de ces tirages.

\begin{noliste}{a)}
 \setlength{\itemsep}{2mm}
\item Calculer, pour tout $k$ de $\N^{*}$,$\Prob\left(\Ev{ U =
k}\right) $.

En déduire la probabilité que l'on n'obtienne jamais la boule
blanche numéro 1.

Reconna\^{\i}tre la loi de $U$.

\item Déterminer la loi conjointe du couple $\left( U,Z\right) $.

\item Montrer que, pour tout $k$ de $\N^{*}$,\ $\Prob\left(\Ev{ Z =
k}\right) = \Sum{\ell = k}{+ \infty }\binom{\ell }{k}\left( 
\frac{1}{4}\right) ^{\ell }$

\item Calculer $\Prob\left(\Ev{ Z = 1}\right) $.

Montrer que $\Prob\left(\Ev{ Z = 0}\right) = \frac{1}{3}$

\item En utilisant la formule dite du triangle de \Scilab{} et le
résultat
de la question c) pour ${k = i + 1}$, justifier, pour tout $i$ de
$\N^{*} $, l'égalité : \\
$\Prob\left(\Ev{ Z = i + 1}\right) = \frac{1}{4}\Prob\left(\Ev{ Z = i +
1}\right) + \frac{1}{4}\Prob\left(\Ev{ Z = i}\right) $

\item En déduire la loi de $Z$.
\end{noliste}
\end{noliste}

\textbf{Partie II. Tirages sans remise.}

Dans cette partie, les tirages se font sans remise dans les deux urnes,
jusqu'à ce que les urnes soient vides. On note $X_{n}$ le nombre de
paires obtenues à l'issue des $n$ tirages.

\textbf{A. Étude de cas particuliers}.

\begin{noliste}{1.}
 \setlength{\itemsep}{4mm}
\item Déterminer la loi de $X_{1}.$

\item On suppose dans cette question que $n = 2$.

Combien y a-t-il de résultats possibles ? Quelles sont les valeurs
prises par $X_{2}$ ?

On précisera pour chaque valeur prise par $X_{2}$, l'ensemble des
événements élémentaires permettant de l'obtenir.

En déduire la loi de $X_{2}$.
\end{noliste}

\textbf{B. Étude du cas général}.

On se place dans le cas où $n$ est un entier naturel non nul.

\begin{noliste}{1.}
 \setlength{\itemsep}{4mm}
\item 
\begin{noliste}{a)}
 \setlength{\itemsep}{2mm}
\item Décrire l'univers $\Omega $ des événements observables.

\item Déterminer le nombre total de suites de tirages possibles.

\item Déterminer l'ensemble des valeurs prises par $X_{n}$.

\hspace{-1cm}Pour tout entier naturel $k$, on note $a\left( n,k\right)
$ le
cardinal de $\left\{ \omega \in \Omega \,\left| \,X_{n}\left( \omega
\right)
 = k\right. \right\} $. Par convention, $a\left( 0,0\right) = 1$.
\end{noliste}

\item 
\begin{noliste}{a)}
 \setlength{\itemsep}{2mm}
\item Préciser la valeur de $\Sum{j = 0}{n}a\left( n,j\right) $

\item Déterminer $a\left( n,n\right) $ et $a\left( n,n-1\right) $.
\end{noliste}

\item 
\begin{noliste}{a)}
 \setlength{\itemsep}{2mm}
\item Justifier, pour tout entier $j$ tel que $0\leq j\leq n$,
l'égalité suivante : 
\[
\frac{a\left( n,j\right) }{n!} = \binom{n}{j}\frac{a\left( n-j,0\right)
}{\left( n-j\right) !}
\]
En déduire la relation : 
\[
\Sum{j = 0}{n}\binom{n}{j}\frac{a\left( j,0\right) }{j!} = n!
\]
Donner l'expression de $a\left( n,0\right) $ en fonction des nombres
$\left(
a\left( j,0\right) \right)_{0\leq j\leq n-1}$

\item Soit $k$ un entier compris entre $1$ et $n$ et $i$ un entier
compris
entre $0$ et $k-1$.

Justifier l'égalité : $\binom{j}{i}\binom{k}{j} =
\binom{k}{i}\binom{k-i}{j-i}$, puis montrer que 
\[
\Sum{j = i}{k}\left( -1\right) ^{j}\binom{j}{i}\binom{k}{j} = 0
\]
En déduire la valeur de la somme : 
\[
\Sum{j = i}{k-1}\left( -1\right) ^{j}\binom{j}{i}\binom{k}{j}
\]
\end{noliste}

\item 
\begin{noliste}{a)}
 \setlength{\itemsep}{2mm}
\item Soit $k$ un entier tel que $1\leq k\leq n$.

On suppose que, pour tout entier $j$ compris entre $0$ et $k-1$, on a
les $k$
égalités : 
\[
a\left( j,0\right) = j!\Sum{i = 0}{j}\binom{j}{i}\left( -1\right)
^{j-i}i!
\]
Montrer l'égalité : 
\[
a\left( k,0\right) = k!\Sum{i = 0}{k}\binom{k}{i}\left( -1\right)
^{k-i}i!
\]
(On pourra utiliser l'expression, pour $n = k$, de $a\left( n,0\right)
$ trouvée dans la question \textbf{3.a)}

\item En déduire, pour tout entier naturel non nul $k$, la valeur de
$a\left( k,0\right) $.

\item Déterminer l'ensemble des valeurs prises par $X_{n}$ et exprimer
la loi de $X_{n}$ à l'aide d'une somme.
\end{noliste}
\end{noliste}

\textbf{Partie III. Tirages mixtes}

Dans cette partie, les tirages se font sans remise dans l'urne blanche
et
avec remise dans l'urne noire, jusqu'à ce que l'urne blanche soit vide.
On note $X_{n}$, le nombre de paires obtenues à l'issue des $n$
tirages.

\begin{noliste}{1.}
 \setlength{\itemsep}{4mm}
\item 
\begin{noliste}{a)}
 \setlength{\itemsep}{2mm}
\item Montrer que $X_{n}$ suit une loi binomiale dont on précisera les
paramètres.

\item Donner, sans démonstration, l'espérance et la variance de
$X_{n}$.
\end{noliste}

\hspace{-1cm}On désire modéliser cette expérience. On suppose
que $n$ est une constante fixée.

\item Définir un type tableau de $n$ entiers noté \texttt{tab}, puis
deux variables de type \texttt{tab}, dont les identificateurs sont
blanc et
noir.

\item 
\begin{noliste}{a)}
 \setlength{\itemsep}{2mm}
\item Soit s un tableau de type \texttt{tab}. Écrire une procédure
dont l'en-tête est \texttt{ECHANG\E(Var s : tab ; i, j : integer)} qui 
échange les éléments \texttt{s[i]} et \texttt{s[j]} du tableau 
\texttt{s}.

\item On considère les lignes de programme suivantes utilisant la
procédure \texttt{ECHANGE}.

\texttt{Begin}

\texttt{For i : = 1 to n do blanc[i] : = i; }

\texttt{For i : = 1 to n-1 do}

\texttt{\hspace{1cm}Begin}

\texttt{\hspace{1cm}j : = RANDOM(n + l-i) + i ; }

\texttt{\hspace{1cm}ECHANG\E(blanc,i,j) ; }

\texttt{\hspace{1cm}end ;}

\texttt{writeln;}

\texttt{For i : = 1 to n do write(blanc[i],' ') }

\texttt{end.}

Expliquer le fonctionnement de ce programme et son résultat.

On précisera ce qui se passe au premier passage puis au $i$-ème
passage dans la deuxième boucle \texttt{For}, et en particulier, la
raison pour laquelle on écrit l'instruction \texttt{j : = RANDOM(n +
i-i) + i}.

\item Construire une procédure qui s'appellera \texttt{INITIALISE}
permettant de simuler le tirage sans remise et au hasard des $n$ boules
numérotées, en mettant dans la variable \texttt{s[i]} le numéro de
la $i$-ème boule tirée (On pourra s'inspirer de la question
précédente).
\end{noliste}

\item Écrire un programme complet permettant de simuler l'expérience
de cette partie III lorsque $n = 20$, puis de donner la valeur de
$X_{n}$ (Il
n'est pas nécessaire ici de recopier les procédures \texttt{ECHANGE}
et \texttt{INITIALISE}).
\end{noliste}

\end{document}

