\documentclass[11pt]{article}%
\usepackage{geometry}%
\geometry{a4paper,
 lmargin = 2cm,rmargin = 2cm,tmargin = 2.5cm,bmargin = 2.5cm}

\input{../../../../../../macros.tex}

\pagestyle{fancy} %
\lhead{ECE2 \hfill septembre 2017 \\
 Mathématiques\\[.2cm]} %
\chead{\hrule} %
\rhead{} %
\lfoot{} %
\cfoot{} %
\rfoot{\thepage} %

\renewcommand{\headrulewidth}{0pt}% : Trace un trait de séparation
 % de largeur 0,4 point. Mettre 0pt
 % pour supprimer le trait.

\renewcommand{\footrulewidth}{0.4pt}% : Trace un trait de séparation
 % de largeur 0,4 point. Mettre 0pt
 % pour supprimer le trait.

\setlength{\headheight}{14pt}

\title{\bf \vspace{-1cm} HEC 2011} %
\author{} %
\date{} %

\begin{document}

\maketitle %
\vspace{-1.2cm}\hrule %
\thispagestyle{fancy}

\vspace*{.4cm}

% DEBUT DU DOC À MODIFIER : tout virer jusqu'au début de l'exo

%Définition et changement de valeurs de
compteurs%newcounter{cpt1}{section} compteur cpt1 remis à 0 à chaque
aumentation par stepcounter du compteur section%setcounter{cpt1}{3} on
met le compteur à 3%addtocounter{cpt1}{5} on ajoute 5 au compteur%
stepcounter{cpt1} on ajoute 1% ifthenelse{test}{alors}{sinon} (page
206) pour subordonner à une condition % whiledo{test}{commande} pour
faire une boucle (page 206 aussi) % value{cpt1} pour noter dans le
document la valeur de cpt1 
%Définition définitive d'opérateurs
mathématiques\newcommand{\ch}{\operatorname{ch}} 
\newcommand{\sh}{\operatorname{sh}}
\renewcommand{\tanh}{\operatorname{th}}
\renewcommand{\sinh}{\operatorname{sh}}
\renewcommand{\cosh}{\operatorname{ch}}
\newcommand{\argsh}{\operatorname{argsh}}
\newcommand{\argch}{\operatorname{argch}}
\newcommand{\argth}{\operatorname{argth}}
\newcommand{\Id}{\operatorname{Id}}
\renewcommand{\leq}{\leq}
\renewcommand{\geq}{\geq }

\newcommand{\dlim}{\lim}
\newcommand{\dsum}{\sum}
\newcommand{\dprod}{\prod}



%Définition de nouvelles couleurs : rgb(trois paramètres red green blue
entre 0 et 1); cmyk (quatre cyan magenta yellow black) entre 0 et 1;
gray (entre 0 et 1) et black, white, red, green, blue, cyan, magenta,
yellow% definecolor{0gris}{gray}{0.8} 
% Nouvelle commande pour encadrer le titre car shabox ne veut que d'une
seule ligne; ATTENTION A LA TAILLE; petite différence avec shadowbox ou
doublebox, voire fcolorbox ou colorbox (au lieu de shabox; laisser le
parbox tranquille sauf pour la taille de la boîte
\newcommand{\Tbox}[1]{\begin{center} \shabox{\parbox{0.6
\linewidth}{#1}} \end{center}} %[1] pour 1 paramètre ; #1 pour ce que
fait le 1er paramètre; entre accolades ce que fait la commande
%Mise en page en mode fancy : en-têtes et pieds de pages puis
définition des en-têtes et pieds de pages\pagestyle{fancy}
\lhead{ECE 2 - Mathématiques \\
Quentin Dunstetter - ENC-Bessières 2011$\backslash$2012}
\chead{}
\rhead{HEC 2011}
\rfoot[ \ \thepage]{\thepage}
\cfoot{}
\lfoot{}
\thispagestyle{fancy} %Mise en page de la 1ère page en mode fancy
%Trait en bas et en haut de la page (entre en-tête et texte et texte et
pied de page)\renewcommand{\footrulewidth}{0.4pt}
\renewcommand{\headrulewidth}{0.4pt}


\begin{center}
{\LARGE BANQUE COMMUNE D'EPREUVES}

{\large CONCOURS D'ADMISSION DE 2010}

{\large Conceptions : H.E.C.- E.S.C.P. / EUROPE}

\rule{2.39cm}{0.05cm}

{\large OPTION : ÉCONOMIQUE}

{\Huge MATHÉMATIQUES }

Année 2011

\rule{2.39cm}{0.05cm}
\end{center}

La présentation, la lisibilité, l'orthographe, la qualité de la
rédaction, la clarté et la précision des raisonnements entreront
pour une part importante dans l'appréciation des copies.

Les candidats sont invités à \textbf{encadrer} dans la mesure du
possible les résultats de leurs calculs.

Ils ne doivent faire usage d'aucun document : \textbf{l'utilisation de
toute
calculatrice et de tout matériel électronique est interdite}. Seule
l'utilisation d'une règle graduée est autorisée.

Si au cours de l'épreuve, un candidat repère ce qui lui semble être une
erreur d'énoncé, il la signalera sur sa copie et poursuivra
sa composition en expliquant les raisons des initiatives qu'il sera
amené
à prendre

\begin{center}
\rule{16.5cm}{0.05cm}
\end{center}
 \centerline{\bf EXERCICE}
 

 Soit $A$ la matrice de ${\cal M}_{3}(\R)$ définie par : $A = 
\begin{smatrix}
0 & 1/2 & 1/2 \\
1/2 & 0 & 1/2 \\
1/2 & 1/2 & 0
\end{smatrix}
$.
 
 1. a) Justifier que la matrice $A$ est diagonalisable.
 
 b) Vérifier que $1$ est une valeur propre de $A$ et déterminer un
vecteur-colonne propre associé.
 
 c) Calculer les valeurs propres de $A$ et déterminer une base de
$\R^{3}$ formée de vecteurs propres de $A$.

 

 {\sl Dans la suite de l'exercice, $n$ est un entier supérieur ou égal
à $2$}. On considère l'ensemble ${\cal S}_{n}$ des matrices $A =
(a_{i,j})_{1\leq i,j\leq n}$ de ${\cal M}_{n}(\R)$ qui vérifient les
propriétés suivantes :
 

 \par
 $\bullet\ $ pour tout $(i,j)$ de $ 
\[
1,n 
\]
^{2}$, $a_{i,j}\geq 0$ ;

 $\bullet\ $ $A$ admet la valeur propre $1$ et $X_{0} = 
\begin{smatrix}
1 \\
\vdots \\
1
\end{smatrix}
$ est un vecteur-colonne propre associé à cette valeur propre.

 
 2. L'ensemble ${\cal S}_{n}$, muni des lois usuelles sur les matrices,
est-il un espace vectoriel ?
 
 3. Montrer que le produit de deux matrices de ${\cal S}_{n}$ est une
matrice de ${\cal S}_{n}$.

 

 4. Soit $A$ un élément de ${\cal S}_{n}$ et $\lambda$ une valeur
propre de $A$.

 a) Montrer qu'il existe un vecteur-colonne propre $V = \begin{smatrix}
v_{1} \\
\vdots \\
v_{n}
\end{smatrix}
$ associé à la valeur propre
 $\lambda$, pour lequel il existe un entier $k$ de $
\[
1,n
\]
$, vérifiant $v_{k} = 1$ et
 pour tout $i$ de $
\[
1,n 
\]
$, $|v_{i}|\leq 1$.
 
 b) En déduire que l'on a : $|\lambda| \leq 1$ et $|\lambda - a_{k,k}|
\leq 1-a_{k,k}$.
 
 5. Montrer que si les éléments diagonaux d'une matrice $A$ de ${\cal
S}_{n}$ sont tous strictement supérieurs à $1/2$, la matrice $A$ est
inversible.

 \vfill\eject

 \centerline{\bf PROBL\`{E}ME}
 

 {\bf Dans tout le problème :}

 $\bullet\ $ le réel $x$ est fixé, $\alpha$ est un réel strictement
positif et $f$ est une fonction définie et continue sur $[x-\alpha, x +
\alpha]$ à valeurs réelles ;

\par

 $\bullet\ $ pour tout réel $h$ vérifiant $0<h\leq \alpha$, on pose :
$S_{x}(h) = \dis {1\over 2h} \dint{-h}{h} f(x + t)dt$ et $G_{x}(h) =
\dis {3\over 2h^{3}}\dint{-h}{h} t f(x + t \ dt$ ;

 \par
 $\bullet\ $ sous réserve d'existence, on pose : $\dis s(x) =
\dlim{h\to 0^+} S_{x}(h)$ et $\dis g(x) = \dlim{h\to 0^+} G_{x}(h)$.

 
 {\sl L'objet du problème est l'étude d'une généralisation de la notion
de dérivée d'une fonction à partir de fonctions définies par des
intégrales}.
 
 {\sl La partie III est indépendante de la partie II, et la partie II
est largement indépendante de la partie I}.
 

 {\bf Partie I. Quelques exemples. Premières propriétés}
 

 1. a) Calculer $\dis \dint{-h}^ \ dt, \dint{-h}{h} (x + t \ dt,
\dint{-h}{h} t(x + t \ dt$ et $\dis \dint{-h}{h} (x + t)^{2}dt$.

 
 b) Montrer que $\dis \dint{-h}{h} t(x + t)^ \ dt = {4h^{3}x\over 3}$.
 
 c) Établir les deux formules : $\dis S_{x}(h) = {1\over 2}
\dint{-1}{1} f(x + ht \ dt$ et $\dis G_{x}(h) = {3\over 2h}
\dint{-1}{1}t f(x + ht \ dt$.
 
 2. {\sl Dans cette question uniquement}, soit $n$ un entier naturel
donné et $f$ la fonction définie par $f(t) = t^{n}$.

 a) Soit $k$ un entier naturel. Calculer suivant la parité de $k$, la
valeur de $h^{k}(1-(-1)^{k})$.
 
 b) Calculer $S_{x}(h)$ et $G_{x}(h)$ lorsque $x = 0$.
 
 c) \`{A} l'aide de la formule du bin\^{o}me, montrer que :
 $S_{x}(h) = x^{n} + h^{2}A_{x}(h)$ et $G_{x}(h) = nx^{n-1} +
h^{2}B_{x}(h)$,
 où $A_{x}$ et $B_{x}$ sont deux fonctions polynomiales en $h$ (dont
les coefficients dépendent de $x$).

 En déduire l'existence et l'expression de $s(x)$ et $g(x)$.


 
 3. {\sl Dans cette question uniquement}, la fonction $f$ est définie
par $f(t) = |t|$ et on choisit $x = 0$.

 a) La fonction $f$ est-elle dérivable en $0$ ?
 
 b) Calculer $S_{x}(h)$ et $G_{x}(h)$ pour $x = 0$. En déduire
l'existence et la valeur de $s(0)$ et $g(0)$.

 

 {\sl Dans les questions 4 à 6, on revient au cas général}.
 

 4. a) Exprimer $S_{x}(h)$ à l'aide d'une primitive $F$ de $f$.
 
 b) Établir l'existence de $s(x)$ et montrer que $s(x) = f(x)$.
 


 5. {\sl On suppose dans cette question que $f$ est dérivable en $x$ de
dérivée $f'(x)$}.

 a) Montrer l'existence d'une fonction $v_{x}$ définie et continue sur
$\R$, vérifiant $v_{x}(0) = 0$ et telle que

 pour tout réel $t$ on ait : $f(x + t) = f(x) + tf'(x) + tv_{x}(t)$.

 
 b) En déduire l'égalité : $G_{x}(h) = \dis f'(x) + {3\over 2h^{3}}
\dint{-h}{h} t^{2}v_{x}(t \ dt$.
 

 c) Soit $\varepsilon$ un réel strictement positif. Montrer qu'il
existe $\delta>0$ tel que si $h\in]0,\delta]$, on a :
 $\dis \left| {3\over 2h^{3}} \dint{-h}{h} t^{2}v_{x}(t \
dt\right|\leq\varepsilon$.

 
 d) En déduire l'expression de $g(x)$ en fonction de $f'(x)$.
 
 6. Soit $\varphi$ la fonction définie sur $\R^{2}$ par : pour tout
couple $(a,b)$ de $\R^{2}$,
 $\dis \varphi(a,b) = \dint{-h}{h} \left( f(x + t)-(b + at)\right)^ \
dt$.

 a) Établir la relation :
 $\dis \varphi(a,b) = \dint{-h}{h} (f(x + t))^ \ dt -2a\dint{-h}{h}
tf(x + t \ dt -2b\dint{-h}{h}f(x + t \ dt + {2h^{3}\over 3} a^{2} +
2hb^{2}$.

 
 b) Montrer que $\varphi$ est de classe $C^{2}$ sur $\R^{2}$ et qu'elle
admet un unique point critique $(a^\star, b^\star)$ que l'on exprimera
en fonction de $S_{x}(h)$ et $G_{x}(h)$.

 \newpage

 {\bf Partie II. Probabilités}
 

 {\sl Les notations sont celles de la partie I}.

 Les variables aléatoires qui interviennent dans cette partie sont
définies sur un espace probabilisé $(\Omega, {\cal A}, P)$.

 \par
 On note $E,V$ et $\hbox{Cov}$ respectivement, l'espérance, la variance
et la covariance.

 \par
 On suppose la fonction $f$ dérivable en $x$, de dérivée $f'(x)$.

 
 Si $X$ est une variable aléatoire définie sur $(\Omega, {\cal A}, P)$,
on s'intéresse au coefficient de corrélation linéaire $r_{x}(h)$ entre
les variables aléatoires $hX$ et $f(hX + x)$ dans quelques cas
particuliers.
 

 7. {\sl Dans cette question uniquement, la variable aléatoire $X$ suit
la loi de Bernoulli de paramètre $1/2$}.
 
 a) Rappeler la valeur de $\E(X)$ et de $\V(X)$. En déduire la valeur
de $\E(hX)$ et $\V(hX)$.
 
 b) Montrer que $\E(f(hX + x)) = \dis {1\over 2}\left( f(x + h) +
f(x)\right)$. Calculer $\V(f(hX + x))$ et $\hbox{Cov}(hX,f(hX + x))$.
 
 c) On suppose que $f(x + h)\neq f(x)$. Calculer $r_{x}(h)$. En déduire
l'existence d'une relation affine entre $f(hX + x)$ et $hX$. Expliciter
cette relation en fonction de $f(x)$ et $f(x + h)$.
 
 d) Vérifier la validité de la relation précédente lorsque $f(x + h) =
f(x)$.

 
 {\sl Dans les questions 8 à 11, la variable aléatoire $X$ suit la loi
uniforme sur l'intervalle $[-1,1]$}.

 

 {\bf On admet} que la définition et les propriétés de la covariance et
du coefficient de corrélation linéaire de deux variables aléatoires
discrètes s'appliquent au cas de deux variables aléatoires à densité.
 
 8. a) Rappeler les expressions d'une densité, de la fonction de
répartition, de l'espérance et de la variance de $X$.
 
 b) Établir les égalités suivantes : $\E(f(hX + x)) = S_{x}(h)$,
 
\[
 V(f(hX + x)) = {1\over 2h}\dint{-h}{h} (f(x + t))^ \ dt-(S_{x}(h))^{2}
\hbox{ et } \hbox{Cov}(hX,f(hX + x)) = {h^{2}\over 3}G_{x}(h)
\]

 

 9. {\sl On suppose dans cette question que $x = 0$}.

 a) Si $f$ est une fonction paire, calculer $f'(0)$ et $r_{0}(h)$.

 
 b) Soit $n$ un entier naturel impair et $f$ la fonction définie par
$f(t) = t^{n}$. Calculer $f'(0)$, $r_{0}(h)$ et $\dis \dlim{h\to 0^+}
r_{0}(h)$.

 
 10. {\sl On suppose dans cette question que $f'(x)\neq 0$ et $f(x) =
0$}.

 a) Montrer que $\hbox{Cov}(hX,f(hX + x))$ est équivalent à $\dis
{h^{2}\over 3}f'(x)$, lorsque $h$ tend vers $0^+ $.

 
 b) On pose : $w_{x}(t) = v_{x}{2}(t) + 2f'(x)v_{x}(t)$, où la fonction
$v_{x}$ a été définie dans la question 5.a).

 Établir les deux relations suivantes :
 
\[
\E(f(hX + x)) = {1\over 2h} \dint{-h}{h} tv_{x}(t \ dt\ \hbox{ et }\
E((f(hX + x))^{2}) = {h^{2}\over 3}(f'(x))^{2} + {1\over
2h}\dint{-h}{h} t^{2}w_{x}(t \ dt
\]
 
 c) Montrer que $\dis \dlim{h\to 0^+} {1\over h^{2}}\dint{-h}{h}
tv_{x}(t \ dt = 0$\hskip 2mm et $\dis \dlim{h\to 0^+} {1\over
h^{3}}\dint{-h}{h} t^{2}w_{x}(t \ dt = 0$.

 
 d) En déduire un équivalent de $\V(f(hX + x))$ lorsque $h$ tend vers
$0^+ $.

 
 e) Calculer $\dis \dlim{h\to 0^+} r_{x}(h)$.

 
 11. {\sl On suppose dans cette question que $f'(x)\neq 0$ et $f(x)\neq
0$}. Que peut-on dire de $\dis \dlim{h\to 0^+} r_{x}(h)$ ?


 \newpage

 {\bf Partie III. Généralisation aux dérivées d'ordre supérieur}
 

 On suppose dans cette partie que la fonction $f$ est de classe
$C^\infty$ sur $\R$, et on note pour tout entier naturel $k$, $f^{(k)}$
la dérivée $k$-ième de $f$, avec la convention $f^{(0)} = f$.

 On confond tout polyn\^{o}me $P$ de $\R[X]$ avec la fonction
polynomiale associée.
 
 Pour tout entier naturel $n$, on définit le polyn\^{o}me $P_{n}$ par :
pour tout $t$ réel, $P_{n}(t) = \dis {1\over 2^{n}n !}(t^{2}-1)^{n}$.
 
 12. a) Calculer $P_{0}(t),P_{1}(t), P_{2}(t)$, ainsi que leurs
dérivées première et seconde.

 
 b) Déterminer le degré de $P_{n}$ et, pour tout entier naturel $k$,
celui de sa dérivée $k$-ième $P_{n}{(k)}$.
 
 c) Déterminer le terme de plus haut degré de $P_{n}{(n)}(t)$ ainsi que
la valeur de $P_{n}{(2n)}(t)$.
 
 d) Soit $n$ un entier supérieur ou égal à $1$. Montrer que pour tout
$k$ de $
\[
0,n-1
\]
$, on peut écrire :

 $P_{n}{(k)}(t) = (t^{2}-1)^{n-k}T_{k}(t)$, où $T_{k}$ est un
polyn\^{o}me de degré $k$.

 En déduire les valeurs de $P_{n}{(k)}(1)$ et de $P_{n}{(k)}(-1)$, pour
tout $k$ de $
\[
0,n-1
\]
$.
 
 e) Établir pour tout entier naturel $n$, la formule : $\dis
\dint{-1}{1} (1-t^{2})^ \ dt = {2^{2n + 1} (n!)^{2}\over (2n + 1)!}$.

 (on remarquera que $1-t^{2} = (1-t)(1 + t)$)

 
 13. Soit $R$ un polyn\^{o}me de $\R[X]$ et $n$ un entier de $\N^*$.

 a) Montrer que $\dis \dint{-1}{1} R(t) P_{n}{(n)}(t \ dt =
-\dint{-1}{1} R'(t) P_{n}{(n-1)}(t \ dt$.
 
 b) En déduire l'égalité : $\dis \dint{-1}{1} R(t) P_{n}{(n)}(t \ dt =
(-1)^{n} \dint{-1}{1} R^{(n)}(t) P_{n}(t \ dt$.

 
 c) On suppose que $R$ est de degré strictement inférieur à $n$.
Calculer $\dis \dint{-1}{1} R(t) P_{n}{(n)}(t \ dt$.
 
 d) En choisissant un polyn\^{o}me $R$ particulier, calculer $\dis
\dint{-1}{1} (P_{n}{(n)}(t))^ \ dt$.


 
 14. On pose, pour tout entier naturel $n$ et sous réserve d'existence
:

 
\[
\dis L_{n}(x) = \dlim{h\to 0^+} {(2n + 1)!\over 2^{n + 1} h^{n} n!}
\dint{-1}{1} f(x + ht) P_{n}{(n)}(t \ dt
\]



 a) \`{A} l'aide des questions 4 et 5, vérifier que $L_{0}(x) = f(x)$
et $L_{1}(x) = f'(x)$.
 
 b) Montrer que pour tout entier naturel $n$, on a : $\dis \dint{-1}{1}
f(x + ht) P_{n}{(n)}(t \ dt = {h^{n}\over 2^{n} n!} \dint{-1}{1}
f^{(n)}(x + ht) (1-t^{2})^ \ dt$.
 

 c) En appliquant le résultat de la question 5.a) à la fonction
$f^{(n)}$, montrer que pour tout entier naturel $n$, on a :
 
\[
\dlim{h\to 0^+} \dint{-1}{1} \left(f^{(n)}(x + ht)-f^{(n)}(x)\right)
(1-t^{2})^ \ dt = 0
\]

 En déduire l'existence et l'expression de $L_{n}(x)$.

 
 15. On rappelle que $\alpha$ est le réel défini dans le préambule du
problème. Soit $n$ un entier supérieur ou égal à $1$ et $Q_{n}$ un
polyn\^{o}me de degré inférieur ou égal à $n$. On pose :
$\Phi_{n}(Q_{n}) = \dis\dint{-1}{1} \left( f(x + \alpha t)-Q_{n}(t)
\right)^ \ dt$.

 
 a) Montrer que la famille $(P_{0}, P'_{1}, P''_{2}, \ldots,
P_{n}{(n)})$ est une base de $\R_{n}[X]$, espace vectoriel des
polyn\^{o}mes de degré inférieur ou égal à $n$.

 La fonction polynomiale $Q_{n}(t)$ s'écrit alors dans cette base :
$Q_{n}(t) = \dis \Sum{k = 0}{n} a_{k} P_{k}{(k)}(t)$.
 
 b) En utilisant la question 13, établir l'égalité : $\dis \dint{-1}{1}
(Q_{n}(t))^ \ dt = \dis \Sum{k = 0}{n} {2\over 2k + 1} a_{k}{2}$.

 

 c) On pose, pour tout $k$ de $
\[
0,n
\]
$ : $y_{k} = \dis \dint{-1}{1} f(x + \alpha t) P_{k}{(k)}(t \ dt$.

 Montrer que
 $\dis \Phi_{n}(Q_{n}) = \dint{-1}{1} (f(x + \alpha t))^ \ dt + \Sum{k
= 0}{n} {2\over 2k + 1} \left(a_{k} -{2k + 1\over 2}
y_{k}\right)^{2}-\Sum{k = 0}{n} {2k + 1\over 2} y_{k}{2}$.

 

 d) Pour quel polyn\^{o}me $Q_{n}^\star$, la quantité $
\Phi_{n}(Q_{n})$ est-elle minimale ? On exprimera pour tout $k$ de $
\[
0,n
\]
$, le coefficient $a_{k}^\star$ de $Q_{n}^\star$ en fonction de $k$ et
de $y_{k}$.


 \end{document}