\documentclass[11pt]{article}%
\usepackage{geometry}%
\geometry{a4paper,
 lmargin = 2cm,rmargin = 2cm,tmargin = 2.5cm,bmargin = 2.5cm}

\input{../../../../../../macros.tex}

\pagestyle{fancy} %
\lhead{ECE2 \hfill septembre 2017 \\
 Mathématiques\\[.2cm]} %
\chead{\hrule} %
\rhead{} %
\lfoot{} %
\cfoot{} %
\rfoot{\thepage} %

\renewcommand{\headrulewidth}{0pt}% : Trace un trait de séparation
 % de largeur 0,4 point. Mettre 0pt
 % pour supprimer le trait.

\renewcommand{\footrulewidth}{0.4pt}% : Trace un trait de séparation
 % de largeur 0,4 point. Mettre 0pt
 % pour supprimer le trait.

\setlength{\headheight}{14pt}

\title{\bf \vspace{-1cm} HEC 1995} %
\author{} %
\date{} %

\begin{document}

\maketitle %
\vspace{-1.2cm}\hrule %
\thispagestyle{fancy}

\vspace*{.4cm}

% DEBUT DU DOC À MODIFIER : tout virer jusqu'au début de l'exo

%Définition et changement de valeurs de
compteurs%newcounter{cpt1}{section} compteur cpt1 remis à 0 à chaque
aumentation par stepcounter du compteur section%setcounter{cpt1}{3} on
met le compteur à 3%addtocounter{cpt1}{5} on ajoute 5 au compteur%
stepcounter{cpt1} on ajoute 1% ifthenelse{test}{alors}{sinon} (page
206) pour subordonner à une condition % whiledo{test}{commande} pour
faire une boucle (page 206 aussi) % value{cpt1} pour noter dans le
document la valeur de cpt1 
%Définition définitive d'opérateurs
mathématiques\newcommand{\ch}{\operatorname{ch}} 
\newcommand{\sh}{\operatorname{sh}}
\renewcommand{\tanh}{\operatorname{th}}
\renewcommand{\sinh}{\operatorname{sh}}
\renewcommand{\cosh}{\operatorname{ch}}
\newcommand{\argsh}{\operatorname{argsh}}
\newcommand{\argch}{\operatorname{argch}}
\newcommand{\argth}{\operatorname{argth}}
\newcommand{\Id}{\operatorname{Id}}
\renewcommand{\leq}{\leq}
\renewcommand{\geq}{\geq }

\newcommand{\dlim}{\lim}
\newcommand{\dsum}{\sum}
\newcommand{\dprod}{\prod}



%Définition de nouvelles couleurs : rgb(trois paramètres red green blue
entre 0 et 1); cmyk (quatre cyan magenta yellow black) entre 0 et 1;
gray (entre 0 et 1) et black, white, red, green, blue, cyan, magenta,
yellow% definecolor{0gris}{gray}{0.8} 
% Nouvelle commande pour encadrer le titre car shabox ne veut que d'une
seule ligne; ATTENTION A LA TAILLE; petite différence avec shadowbox ou
doublebox, voire fcolorbox ou colorbox (au lieu de shabox; laisser le
parbox tranquille sauf pour la taille de la boîte
\newcommand{\Tbox}[1]{\begin{center} \shabox{\parbox{0.6
\linewidth}{#1}} \end{center}} %[1] pour 1 paramètre ; #1 pour ce que
fait le 1er paramètre; entre accolades ce que fait la commande
%Mise en page en mode fancy : en-têtes et pieds de pages puis
définition des en-têtes et pieds de pages\pagestyle{fancy}
\lhead{ECE 2 - Mathématiques \\
Quentin Dunstetter - ENC-Bessières 2011$\backslash$2012}
\chead{}
\rhead{HEC 1995}
\rfoot[ \ \thepage]{\thepage}
\cfoot{}
\lfoot{}
\thispagestyle{fancy} %Mise en page de la 1ère page en mode fancy
%Trait en bas et en haut de la page (entre en-tête et texte et texte et
pied de page)\renewcommand{\footrulewidth}{0.4pt}
\renewcommand{\headrulewidth}{0.4pt}

\begin{center}
{\huge HEC Eco 1995}
\end{center}

\section*{EXERCICE I}

Soit $A$ la matrice de $\mathfrak{M}_{3}(\R)$ définie par 
\[
A = \left( 
\begin{array}{ccc}
-4 & -6 & 0 \\
6 & 7 & 2 \\
0 & 2 & -2
\end{array}
\right) 
\]
et soit $n$ un entier, strictement positif, fixé. \\

Cet exercice a pour but de déterminer toutes les matrices $X$ de
$\mathfrak{M}_{3}(\R)$ vérifiant l'équation 
\[
(\ast )\qquad X^{n} = A^{n}.
\]
\\
On désigne par $(e_{1},e_{2},e_{3})$ la base canonique de $\R^{3}$
et par $\phi $ l'endomorphisme de $\R^{3}$ représenté par la matrice 
$A$ dans cette base.

\begin{noliste}{1.}
 \setlength{\itemsep}{4mm}
\item 

\begin{noliste}{a)}
 \setlength{\itemsep}{2mm}
\item Déterminer l'image par $\phi $ du vecteur $(-2e_{1} + e_{2} +
2e_{3})$.

\item Déterminer le noyau de $\phi $.

\item Montrer que 2 est une valeur propre de $\phi $ et déterminer le
sous-espace propre associé.

\item Montrer que la matrice $A$ est diagonalisable.

\item Déterminer les valeurs propres de la matrice $A^{n}$.
\end{noliste}

\item 

\begin{noliste}{a)}
 \setlength{\itemsep}{2mm}
\item Montrer que si l'équation (*) admet une solution $X$, alors
$XA^{n} = A^{n}X$.

\item En déduire que si $(f_{1},f_{2},f_{3})$ est une base formée de
vecteurs propres de $A$, c'est aussi une base de vecteurs propres de
$X$.

\item Déterminer une matrice $Q$ telles que les matrices $Q^{-1}AQ$ et
$Q^{-1}XQ$ soient toutes les deux diagonales.
\end{noliste}

\item 

\begin{noliste}{a)}
 \setlength{\itemsep}{2mm}
\item Si $n$ est un nombre impair, déterminer toutes les solutions de
(*).

\item Si $n$ est un nombre pair, déterminer toutes les solutions de
(*).
\end{noliste}
\end{noliste}

\section*{EXERCICE II}

Soit $f$ la fonction numérique définie sur $\left[ 1, + \infty \right[
$ par
la relation 
\[
f(t) = \dfrac{e^{t}}{t}
\]

\begin{noliste}{1.}
 \setlength{\itemsep}{4mm}
\item 

\begin{noliste}{a)}
 \setlength{\itemsep}{2mm}
\item Montrer que la fonction $f$ est strictement croissante.

\item En déduire que, pour tout nombre entier naturel non nul $n$, 
\[
\dfrac{e^{n}}{n}\leq \dint{n}{n + 1}f(t)\ dt\leq \dfrac{e^{n + 1}}{n +
1}
\]
Montrer que l'équation 
\[
\dfrac{e^{x}}{x} = \dint{n}{n + 1}f(t)\ dt
\]
admet une solution et une seule dans l'intervalle $\left[ n;n +
1\right] $. On
notera $u_{n}$ cette solution, ce qui définit une suite
$(u_{n})_{n>0}$.
\end{noliste}

\item 

\begin{noliste}{a)}
 \setlength{\itemsep}{2mm}
\item Montrer que, pour tout nombre entier naturel non nul $p$, 
\[
0\leq \dint{1}{2}f(t)\ dt-\dfrac{1}{p}\Sum{k = 0}{p-1}f(1 +
\dfrac{k}{p})\leq \dfrac{1}{p}\left[ f(2)-f(1)\right]
\]

\item En utilisant la méthode des rectangles, donner une valeur
approchée à
0,1 près de l'intégrale $\dint{1}{2}f(t)\ dt$. En déduire une valeur
approchée de $u_{1}$ à 0,1 près.
\end{noliste}

\item 

\begin{noliste}{a)}
 \setlength{\itemsep}{2mm}
\item Montrer que 
\[
\dlim{x\rightarrow + \infty }\dfrac{u_{n}}{n} = 1
\]

\item Montrer que 
\[
\dlim{x\rightarrow + \infty }\dfrac{\dint{n}{n +
1}\dfrac{e^{t}}{t^{2}}\ dt}{\dint{n}{n + 1}\dfrac{e^{t}}{t}\ dt} = 0
\]
A l'aide d'une intégration par parties, montrer que 
\[
\dlim{x\rightarrow + \infty }(u_{n}-n) = \ln {(e-1)}
\]
\end{noliste}
\end{noliste}

\section*{EXERCICE III}

La roue d'une loterie est représentée par un disque de rayon 1 dont le
centre $O$ est pris pour origine d'un repère orthonormé $\left(
O,\overset{\rightarrow }{i},\overset{\rightarrow }{j}\right) $. La roue
est lancée dans
le sens trigonométrique. L'angle (exprimé en radians) dont elle tourne
avant
de s'arrêter est une variable aléatoire qu'on note $U$ et qui est
supposée
suivre une loi exponentielle de paramètre $\lambda $, de densité la
fonction
valant $\lambda e^{-\lambda x}$ si $x\geq 0$ et 0 si $x<0$.\\
La roue porte une marque $M$ qui, au départ, est située au point de
coordonnées $(1,0)$ et qui, après l'arrêt de la roue, est située au
point de coordonnées 
\[
X = \cos {U}\qquad Y = \sin {U}
\]

\begin{noliste}{1.}
 \setlength{\itemsep}{4mm}
\item 

\begin{noliste}{a)}
 \setlength{\itemsep}{2mm}
\item On admet (la justification n'est pas demandée) que les intégrales

\[
\dint{0}{+ \infty }\cos {u}\ e^{-\lambda u}du\qquad \text{et}\qquad
\dint{0}{+ \infty }\sin {u}\ e^{-\lambda u}du
\]
sont absolument convergentes et on se propose de les calculer. \\
Pour cela on pose, pour $\lambda >0$ et $T>0$ : 
\[
I(T) = \dint{0}{T}\cos {u}\ e^{-\lambda u}du\qquad \text{et}\qquad
J(T) = \dint{0}{T}\sin {u}\ e^{-\lambda u}du.
\]
A l'aide d'intégrations par parties, établir les relations 
\[
I(T) = \sin {T}\ e^{-\lambda T}-\lambda \cos {T}\ e^{-\lambda T} +
\lambda
-\lambda ^{2}I(T)
\]
\[
J(T) = -\cos {T}\ e^{-\lambda T} + 1-\lambda \sin {T}\ e^{-\lambda
T}-\lambda
^{2}J(T)
\]

\item En déduire les limites de $I(T)$ et de $J(T)$ quand $T$ tend vers
$ + \infty $.

\item Calculer l'espérance des variables aléatoires $X$ et $Y$ en
fonction
du paramètre $\lambda $.
\end{noliste}

\item Un joueur gagne à cette loterie si, à l'arrêt de la roue,
l'ordonnée
de $M$ vérifie la condition : $Y\geq \dfrac{1}{2}$.

\begin{noliste}{a)}
 \setlength{\itemsep}{2mm}
\item Pour quelles valeurs de $U$ le joueur a-t-il gagné ?

\item Calculer la probabilité $q(\lambda )$ que le joueur gagne.

\item Montrer que, quand $\lambda $ tend vers 0, $q(\lambda )$ a une
limite
que l'on déterminera.
\end{noliste}

\item On suppose dans cette question, pour simplifier, que $\lambda =
1$.

\begin{noliste}{a)}
 \setlength{\itemsep}{2mm}
\item Calculer les espérances de $X^{2}$, $Y^{2}$ et $XY$.

\item Calculer, pour toute valeur du nombre réel $a$, la variance de la
variable aléatoire 
\[
Z = X-aY
\]

\item Montrer qu'il existe une valeur de $a$, que l'on déterminera,
pour
laquelle la variance de $Z$ est minimum.
\end{noliste}
\end{noliste}

\label{fin}

\end{document}

