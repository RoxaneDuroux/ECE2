\documentclass[11pt]{article}%
\usepackage{geometry}%
\geometry{a4paper,
 lmargin = 2cm,rmargin = 2cm,tmargin = 2.5cm,bmargin = 2.5cm}

\input{../../../../../../macros.tex}

\pagestyle{fancy} %
\lhead{ECE2 \hfill septembre 2017 \\
 Mathématiques\\[.2cm]} %
\chead{\hrule} %
\rhead{} %
\lfoot{} %
\cfoot{} %
\rfoot{\thepage} %

\renewcommand{\headrulewidth}{0pt}% : Trace un trait de séparation
 % de largeur 0,4 point. Mettre 0pt
 % pour supprimer le trait.

\renewcommand{\footrulewidth}{0.4pt}% : Trace un trait de séparation
 % de largeur 0,4 point. Mettre 0pt
 % pour supprimer le trait.

\setlength{\headheight}{14pt}

\title{\bf \vspace{-1cm} HEC 1986} %
\author{} %
\date{} %

\begin{document}

\maketitle %
\vspace{-1.2cm}\hrule %
\thispagestyle{fancy}

\vspace*{.4cm}

% DEBUT DU DOC À MODIFIER : tout virer jusqu'au début de l'exo

%Définition et changement de valeurs de
compteurs%newcounter{cpt1}{section} compteur cpt1 remis à 0 à chaque
aumentation par stepcounter du compteur section%setcounter{cpt1}{3} on
met le compteur à 3%addtocounter{cpt1}{5} on ajoute 5 au compteur%
stepcounter{cpt1} on ajoute 1% ifthenelse{test}{alors}{sinon} (page
206) pour subordonner à une condition % whiledo{test}{commande} pour
faire une boucle (page 206 aussi) % value{cpt1} pour noter dans le
document la valeur de cpt1 
%Définition définitive d'opérateurs
mathématiques\newcommand{\ch}{\operatorname{ch}} 
\newcommand{\sh}{\operatorname{sh}}
\renewcommand{\tanh}{\operatorname{th}}
\renewcommand{\sinh}{\operatorname{sh}}
\renewcommand{\cosh}{\operatorname{ch}}
\newcommand{\argsh}{\operatorname{argsh}}
\newcommand{\argch}{\operatorname{argch}}
\newcommand{\argth}{\operatorname{argth}}
\newcommand{\Id}{\operatorname{Id}}
\renewcommand{\leq}{\leq}
\renewcommand{\geq}{\geq }

\newcommand{\dlim}{\lim}
\newcommand{\dsum}{\sum}
\newcommand{\dprod}{\prod}



%Définition de nouvelles couleurs : rgb(trois paramètres red green blue
entre 0 et 1); cmyk (quatre cyan magenta yellow black) entre 0 et 1;
gray (entre 0 et 1) et black, white, red, green, blue, cyan, magenta,
yellow% definecolor{0gris}{gray}{0.8} 
% Nouvelle commande pour encadrer le titre car shabox ne veut que d'une
seule ligne; ATTENTION A LA TAILLE; petite différence avec shadowbox ou
doublebox, voire fcolorbox ou colorbox (au lieu de shabox; laisser le
parbox tranquille sauf pour la taille de la boîte
\newcommand{\Tbox}[1]{\begin{center} \shabox{\parbox{0.6
\linewidth}{#1}} \end{center}} %[1] pour 1 paramètre ; #1 pour ce que
fait le 1er paramètre; entre accolades ce que fait la commande
%Mise en page en mode fancy : en-têtes et pieds de pages puis
définition des en-têtes et pieds de pages\pagestyle{fancy}
\lhead{ECE 2 - Mathématiques \\
Quentin Dunstetter - ENC-Bessières 2011$\backslash$2012}
\chead{}
\rhead{HEC 1986}
\rfoot[ \ \thepage]{\thepage}
\cfoot{}
\lfoot{}
\thispagestyle{fancy} %Mise en page de la 1ère page en mode fancy
%Trait en bas et en haut de la page (entre en-tête et texte et texte et
pied de page)\renewcommand{\footrulewidth}{0.4pt}
\renewcommand{\headrulewidth}{0.4pt}

\begin{center}
{\huge HEC Eco 1986}
\end{center}

\section*{EXERCIC\E\ 1}

On considère la matrice 
\[
M = 
\begin{smatrix}
10 & 1 \\
1 & 0
\end{smatrix}
\]

\begin{noliste}{1.}
 \setlength{\itemsep}{4mm}
\item Calculer les valeurs propres $\lambda $ et $\mu $ de $M;$ on
notera $\lambda $ celle de ces valeurs propres qui a la plus grande
valeur absolue.
Montrer que $\lambda \mu = -1.$\\
Pour tout nombre entier naturel non nul $n,$ calculer la matrice
$M^{n}.$

\item Soit $(u_{n})$ une suite de nombres réels. Pour tout nombre
entier
naturel $n,$ on pose :
\[
X_{n} = 
\begin{smatrix}
u_{n + 1} \\
u_{n}\end{smatrix}.
\]
Montrer que les deux conditions suivantes sont équivalentes.

\begin{noliste}{a)}
 \setlength{\itemsep}{2mm}
\item Pour tout nombre entier naturel $n$
\[
(1)\qquad u_{n + 2} = 10u_{n + 1} + u_{n}
\]

\item Pour tout entier naturel $n,$ 
\[
(2)\qquad X_{n + 1} = MX_{n}
\]
\end{noliste}

\item Soit $\alpha $ un nombre réel. On note $(u_{n}(\alpha ))$
l'unique
suite satisfaisant aux conditions équivalentes précédentes et telle que
$u_{0}(\alpha ) = 1$ et $u_{1}(\alpha ) = \alpha.$\\
Exprimer $u_{n}(\alpha )$ en fonction de $\alpha,$ $\lambda,$ $\mu $ et
$n.
$

\item Montrer que la suite $(u_{n}(\alpha ))$ est convergente si et
seulement si $\alpha = \mu ;$ déterminer alors la limite de cette
suite.

\item Soit $\mu ^{\prime }$ une valeur décimale approchée de $\mu $ à
la précision $10^{-p},$ où $p\in \N^{\times }.$ Autrement dit, $\mu
^{\prime } = \mu + \delta,$ où $\left| \delta \right| \leq
10^{-p}.$ On se propose d'examiner si $u_{n}(\mu ^{\prime }),$ que l'on
calcule par récurrence (à l'aide d'une calculatrice) grâce à la
relation
(1), fournit une bonne approximation de $u_{n}(\mu ).$

\begin{noliste}{a)}
 \setlength{\itemsep}{2mm}
\item Dans cette question, on prend $\mu ^{\prime } = -$ 0,099\ 02
(valeur
approchée de $\mu $ à la précision $10^{-4}).$ À l'aide de la
calculatrice,
calculer $u_{n}(\mu ^{\prime })$ pour $2\leq n\leq 10.$ Exprimer
d'autre part $u_{n}(\mu )$ en fonction de $\lambda $ et de $n;$ A
partir de
cette relation, calculer les valeurs décimales approchées de $u_{n}(\mu
)$
pour $2\leq n\leq 10.$

\item Exprimer $u_{n}(\mu ^{\prime })-u_{n}(\mu )$ en fonction de
$\delta,$ 
$\lambda,$ $\mu $ et $n.$ En déduire l'ordre de grandeur de $u_{10}(\mu
^{\prime })-u_{10}(\mu )$ lorsqu'on prend $\mu ^{\prime }$ la valeur
approchée définie au a).\\
Expliquer ainsi le phénomène observé à la question a).
\end{noliste}
\end{noliste}

\section*{EXERCIC\E\ 2}

Pour tout entier naturel non nul $n,$ on note $f_{n}$ la fonction
numérique définie sur $\R_{+}$ par la relation :
\[
f_{n}(x) = \dfrac{x-n}{x + n}-e^{-x}
\]

\begin{noliste}{1.}
 \setlength{\itemsep}{4mm}
\item Étudier la variation de la fonction $f_{n}.$

\item Montrer que l'équation $f_{n}(x) = 0$ admet une solution $u_{n}$
et une
seule.

\item On se propose d'étudier le comportement asymptotique de la suite
$(u_{n}).$

\begin{noliste}{a)}
 \setlength{\itemsep}{2mm}
\item En étudiant le signe de $f_{n}(n),$ montrer que $u_{n}>n.$ En
déduire
la limite de la suite $(u_{n}).$

\item Montrer que $f_{n}(n + 1)$ est positif à partir d'un certain
rang. En déduire la limite de $\dfrac{u_{n}}{n}.$

\item Montrer que :
\[
\dlim{n\rightarrow + \infty }(u_{n}-n) = 0
\]
(On pourra étudier le signe de $f_{n}(n + \varepsilon )$ où
$\varepsilon $ est
un nombre réel strictement positif.)
\end{noliste}

\item On se propose d'étude la suite de terme général $a_{n} =
u_{n}-n.$
Expliciter la relation $f_{n}(n + a_{n}) = 0.$ En déduire la limite de
$\dfrac{e^{n}}{n}a_{n}.$
\end{noliste}

\section*{EXERCIC\E\ 3}

Soit $n$ un nombre entier naturel supérieur ou égal à $2.$ On note $I$
l'ensemble $\{1,2,...,n\}.$\\
Soit $X$ une variable aléatoire à valeurs dans $I.$ Pour tout entier
$i$ de $I,$ on note $p_{i}$ la probabilité que $X$ prenne la valeur
$i;$ on suppose
que $p_{i}>0.$\\
On pose enfin :
\[
H(X) = \Sum{i = 1}{n}p_{i}\ln \dfrac{1}{p_{i}}.
\]

\begin{noliste}{1.}
 \setlength{\itemsep}{4mm}
\item Montrer que la quantité $H(X)$ est strictement positive. Quelle
est sa
valeur lorsque $X$ suit une loi uniforme sur $I$ ?

\item Soit $(q_{1},q_{2},...,q_{n})$ une suite de $n$ nombres réels
strictement positifs telle que $\Sum{i = 1}{n}q_{i} = 1.$

\begin{noliste}{a)}
 \setlength{\itemsep}{2mm}
\item Vérifier que, pour tout nombre réel strictement positif $x$ :
\[
\ln x\leq x-1
\]

\item En déduire que :
\[
\Sum{i = 1}{n}p_{i}\ln \dfrac{q_{i}}{p_{i}}\leq 0
\]
Dans quel cas le premier membre est-il nul ?

\item Montrer que $H(X)\leq \ln n.$ Caractériser le cas où il y a
égalité.
\end{noliste}

\item Soit $(X,Y)$ un couple de variables aléatoires à valeurs dans
$I\times
I.$ Pour tout élément $(i,j)$ de $I\times I.$ On note $r_{ij}$ la
probabilité
que $X$ prenne $i$ et $Y$ la valeur $j;$ on suppose que $r_{ij}>0.$ On
note
respectivement $(p_{i})_{i\in I}$ et $(q_{j})_{j\in I}$ les lois
marginales
de $X$ et de $Y.$ On pose :
\[
H(X,Y) = \Sum{i = 1}{n}\Sum{j = 1}{n}r_{ij}\ln \dfrac{1}{r_{ij}}.
\]
\end{noliste}

\label{fin}

\end{document}

