\documentclass[11pt]{article}%
\usepackage{geometry}%
\geometry{a4paper,
 lmargin = 2cm,rmargin = 2cm,tmargin = 2.5cm,bmargin = 2.5cm}

\input{../../macros.tex}

\pagestyle{fancy} %
\lhead{ECE2 \hfill Mathématiques\\
} %
\chead{\hrule} %
\rhead{} %
\lfoot{} %
\cfoot{} %
\rfoot{\thepage} %

\renewcommand{\headrulewidth}{0pt}% : Trace un trait de séparation
 % de largeur 0,4 point. Mettre 0pt
 % pour supprimer le trait.

\renewcommand{\footrulewidth}{0.4pt}% : Trace un trait de séparation
 % de largeur 0,4 point. Mettre 0pt
 % pour supprimer le trait.

\setlength{\headheight}{14pt}

\title{\bf \vspace{-2cm} ESCP 1991 - voie Générale} %
\author{} %
\date{} %
\begin{document}

\maketitle %
\vspace{-1.4cm}\hrule %
\thispagestyle{fancy}

\vspace*{.2cm}


% DEBUT DU DOC À MODIFIER : tout virer jusqu'au début de l'exo


\begin{center}
{\small CHAMBRE D\E\ COMMERCE ET D'INDUSTRIE DE PARIS}

\textbf{DIRECTION DE L'ENSEIGNEMENT}

Direction des Admissions et concours

\underline{\hspace*{3cm}}

{\Large ECOLE DES\ HAUTES\ ETUDES\ COMMERCIALES}

{\Large E.S.C.P.-E.A.P.}

{\Large ECOL\E\ SUPERIEUR\E\ D\E\ COMMERC\E\ D\E\ LYON}{\large }

CONCOURS D'ADMISSION\ SUR\ CLASSES\ PREPARATOIRES

\underline{\hspace*{3cm}}

\textbf{OPTION GENERALE}

{\Large MATHEMATIQUES I}

\textbf{Année 1991}

\underline{\hspace*{3cm}}
\end{center}

\begin{quotation}
\noindent \textsl{La présentation, la lisibilité, l'orthographe, la
qualité
de la rédaction, la clarté et la précision des raisonnements entreront
pour
une part importante dans l'appréciation des copies.}

\noindent \textsl{Les candidats sont invités à encadrer dans la mesure
du
possible les résultats de leurs calculs.}

\noindent \textsl{Ils ne doivent faire usage d'aucun document :
l'utilisation de toute calculatrice et de tout matériel électronique
est
interdite.}

\noindent \textsl{Seule l'utilisation d'une règle graduée est
autorisée.}

\noindent \textsl{\hrulefill }
\end{quotation}

\noindent L'objet de ce problème est la recherche du comportement
asymptotique du maximum sur $[0,1]$ d'une suite de fonction.

\section*{Partie I :\hspace{0.2cm} Étude du maximum d'une fonction}

Soit $f$ la fonction numérique définie sur $\R_{+}$ par la relation : 
\[
f(x) = \dfrac{x}{e^{x} + 1}
\]

\begin{noliste}{1.}
 \setlength{\itemsep}{4mm}
\item \textit{Variation de }$f$

\begin{noliste}{a)}
 \setlength{\itemsep}{2mm}
\item Calculer la dérivée $f^{\prime }$ de $f$.

\item Montrer que l'équation $f^{\prime }(x) = 0$ admet une solution
$\alpha $
et une seule sur $\R_{+}$.\\
Indication : On étudiera la variation de la fonction $g$ définie sur
$\R_{+}$ par : 
\[
g(x) = (1-x)e^{x} + 1
\]

\item Prouver que : $f(\alpha ) = \alpha -1$

\item Dresser le tableau de variation de $f$ et donner l'allure du
graphe de
cette fonction.
\end{noliste}

\item \textit{Approximation de }$\alpha $\\
Soit $\varphi $ la fonction définie sur $\R_{+}$ par la relation : 
\[
\varphi (x) = 1 + e^{-x}
\]

\begin{noliste}{a)}
 \setlength{\itemsep}{2mm}
\item Prouver que $\alpha $ est l'unique solution de l'équation
$\varphi
(x) = x$.

\item Montrer que $\alpha >1$. En déduire que : 
\[
\alpha -1<e^{-1}
\]

\item Établir que, pour tout nombre réel $x$ supérieur ou égal à 1 : 
\[
\varphi (x)\geq 1
\]
et que 
\[
\left| \varphi (x)-\alpha \right| \leq e^{-1}\left|
x-\alpha \right|
\]

\item Soit $(\alpha_{n})_{n\in \N}$ la suite d'éléments de $[1; +
\infty \lbrack $ définie par la condition initiale $\alpha_{0} = 1$ et
par la relation de récurrence 
\[
\alpha_{n + 1} = \varphi (\alpha_{n})
\]
Montrer que pour tout nombre entier naturel $n$ : 
\[
\left| \alpha_{n}-\alpha \right| \leq e^{-(n + 1)}
\]

\item En déterminant au préalable le nombre d'itérations nécessaires,
expliciter, à l'aide d'une calculatrice, une valeur décimale approchée
$\tilde{\alpha}$ de $\alpha $ telle que : 
\[
\left| \tilde{\alpha}-\alpha \right| \leq 10^{-6}
\]
\end{noliste}
\end{noliste}

\section*{Partie II :\hspace{0.2cm}\ Étude d'une suite de
fonction\protect\vspace{0.1cm}}

\emph{Excepté la question \ref{question}, la partie II est indépendante
de
la partie I.}\\
Soit $(u_{n})_{n\in \N}$ la suite de fonctions numériques définies
sur $[0;1]$ par la condition initiale $u_{0}(x) = 1$ et la relation de
récurrence : 
\[
u_{n + 1}(x) = \left[ 1-x + \dfrac{x}{2}u_{n}(x)\right] u_{n}(x)
\]

\begin{noliste}{1.}
 \setlength{\itemsep}{4mm}
\item Montrer que, pour tout nombre entier naturel $n$, la fonction
$u_{n}$
est polynômiale; déterminer son degré et son coefficient dominant.

\item Montrer par récurrence sur $n$ que, pour tout nombre entier
naturel $n$
et pour tout élément $x$ de $[0;1]$ : 
\[
0<u_{n}(x)\leq 1
\]
En déduire, toujours par récurrence sur $n$, que, pour tout nombre
entier
naturel $n$ et pour tout élément $x$ de $[0;1]$ : 
\[
u_{n}{^{\prime }}(x)\leq 0
\]

\item 

\begin{noliste}{a)}
 \setlength{\itemsep}{2mm}
\item Soit $x$ un élément de $[0;1]$ et $n$ un entier naturel non nul.
Établir les inégalités : 
\[
(1-x)^{n}\leq u_{n}(x)\leq \left( 1-\dfrac{x}{2}\right) ^{n}
\]

\item En déduire que, lorsque $x$ est fixé dans $[0;1]$, la suite de
terme général $u_{n}(x)$ converge. Exprimer sa limite en fonction de
$x$.

\item Montrer que cette suite est décroissante.
\end{noliste}

\emph{Dans toute la suite du problème, on note }$h$\emph{\ la fonction
définie sur }$[0;1]$\emph{\ par la relation :} 
\[
h(t) = t\left( 1-\dfrac{t}{2}\right)
\]

\item 

\begin{noliste}{a)}
 \setlength{\itemsep}{2mm}
\item Montrer que, pour tout couple $(a,b)$ d'éléments de $[0;1]$ : 
\[
\left| h(b)-h(a)\right| \leq \left| b-a\right|
\]

\item En déduire que, pour tout nombre entier naturel $k$ et pour tout
élément $x$ de $[0;1]$ : 
\[
\left| \left[ u_{k}(x)-u_{k + 1}(x)\right] -\left[ u_{k + 1}(x)-u_{k +
2}(x)\right] \right| \leq x\left| u_{k}(x)-u_{k + 1}(x)\right|
\]

\item Montrer enfin que, pour tout nombre entier naturel $k$ et pour
tout élément $x$ de $[0;1]$ : 
\[
0\leq u_{k}(x)-u_{k + 1}(x)\leq \dfrac{u_{k + 1}(x)-u_{k + 2}(x)}{1-x}
\]
\end{noliste}

\item Dans cette question, on fixe $k$ dans $\N$ et $x$ dans $[0;1[$.

\begin{noliste}{a)}
 \setlength{\itemsep}{2mm}
\item En utilisant le dernier encadrement et le sens de variation de
$h$,
montrer que : 
\[
x\leq \dint{u_{k + 1}(x)}{u_{k}(x)}\dfrac{dt}{h(t)}\leq 
\dfrac{x}{1-x}
\]

\item En déduire que, pour tout nombre entier naturel $n$ : 
\[
nx\leq \dint{u_{n}(x)}{1}\dfrac{dt}{h(t)}\leq \dfrac{nx}{1-x}
\]
\end{noliste}

\item 

\begin{noliste}{a)}
 \setlength{\itemsep}{2mm}
\item Trouver un couple $(A,B)$ de nombre réels tels que, pour tout
élément $t$ de l'intervalle $]0;1]$ : 
\[
\dfrac{1}{h(t)} = \dfrac{A}{t} + \dfrac{B}{1-\dfrac{t}{2}}
\]

\item En déduire la valeur de l'intégrale : 
\[
\dint{u_{n}(x)}{1}\dfrac{dt}{h(t)}
\]
en fonction de $u_{n}(x)$. En conclure que : 
\[
\dfrac{2}{1 + \exp \left( \dfrac{nx}{1-x}\right) }\leq u_{n}(x)\leq 
\dfrac{2}{1 + e^{nx}}
\]
\end{noliste}

\item Soit $y$ un élément de $\R_{+}$. Déterminer la limite de la
suite de terme général $u_{n}\left( \dfrac{y}{n}\right) $.

\item \label{question} Soit $(v_{n})_{n\in \N}$ la suite de
fonctions définie sur $[0;1]$ par : 
\[
v_{n}(x) = x\,u_{n}(x)
\]
On note $M_{n}$ le maximum de la fonction $v_{n}$ sur l'intervalle
$[0;1]$.

\begin{noliste}{a)}
 \setlength{\itemsep}{2mm}
\item Montrer que : 
\[
v_{n}\left( \dfrac{\alpha }{n + \alpha }\right) \geq \dfrac{2(\alpha
-1)}{n + \alpha }
\]

\item Établir l'encadrement : 
\[
\dfrac{2(\alpha -1)}{n + \alpha }\leq M_{n}\leq \dfrac{2(\alpha -1)}{n}
\]
En déduire un équivalent simple de $M_{n}$ lorsque $n$ tend vers $ +
\infty $
\end{noliste}
\end{noliste}

\label{fin}

\end{document}

