\documentclass[11pt]{article}%
\usepackage{geometry}%
\geometry{a4paper,
 lmargin = 2cm,rmargin = 2cm,tmargin = 2.5cm,bmargin = 2.5cm}

\input{../../macros.tex}

\pagestyle{fancy} %
\lhead{ECE2 \hfill Mathématiques\\
} %
\chead{\hrule} %
\rhead{} %
\lfoot{} %
\cfoot{} %
\rfoot{\thepage} %

\renewcommand{\headrulewidth}{0pt}% : Trace un trait de séparation
 % de largeur 0,4 point. Mettre 0pt
 % pour supprimer le trait.

\renewcommand{\footrulewidth}{0.4pt}% : Trace un trait de séparation
 % de largeur 0,4 point. Mettre 0pt
 % pour supprimer le trait.

\setlength{\headheight}{14pt}

\title{\bf \vspace{-2cm} ESCP 2003 - voie Générale} %
\author{} %
\date{} %
\begin{document}

\maketitle %
\vspace{-1.4cm}\hrule %
\thispagestyle{fancy}

\vspace*{.2cm}


% DEBUT DU DOC À MODIFIER : tout virer jusqu'au début de l'exo


\begin{center}
{\small CHAMBRE D\E\ COMMERCE ET D'INDUSTRIE DE PARIS}

\textbf{DIRECTION DE L'ENSEIGNEMENT}

Direction des Admissions et concours

\underline{\hspace*{3cm}}

{\Large ECOLE DES\ HAUTES\ ETUDES\ COMMERCIALES}

{\Large E.S.C.P.-E.A.P.}

{\Large ECOL\E\ SUPERIEUR\E\ D\E\ COMMERC\E\ D\E\ LYON}{\large }

CONCOURS D'ADMISSION\ SUR\ CLASSES\ PREPARATOIRES

\underline{\hspace*{3cm}}

\textbf{OPTION SCIENTIFIQUE}

{\Large MATHEMATIQUES I}

\textbf{Année 2003}

\underline{\hspace*{3cm}}
\end{center}

\begin{quotation}
\noindent \textsl{La présentation, la lisibilité, l'orthographe, la
qualité
de la rédaction, la clarté et la précision des raisonnements entreront
pour
une part importante dans l'appréciation des copies.}

\noindent \textsl{Les candidats sont invités à encadrer dans la mesure
du
possible les résultats de leurs calculs.}

\noindent \textsl{Ils ne doivent faire usage d'aucun document :
l'utilisation de toute calculatrice et de tout matériel électronique
est
interdite.}

\noindent \textsl{Seule l'utilisation d'une règle graduée est
autorisée.}

\noindent \textsl{\hrulefill }
\end{quotation}

\noindent Dans tout le problème, on considère un entier naturel $p$
supérieur ou égal à $2$. Pour tout entier naturel $q$, on note
$\R_{q}[X]
$ (resp. $\C_{q}[X]$ l'espace vectoriel réel (resp. complexe) des
polynômes à coefficients réels (resp. complexes) de degré au plus égal
à $q$. On pourra confondre polynôme et fonction polynomiale associée.\\
On note $\mathcal{S}_{r}$ (resp. $\mathcal{S}_{c}$) l'espace vectoriel
réel
(resp. complexe) des suites réelles (resp. complexes)..\\
\textbf{Préliminaire}\\
On considère la fonction réelle $f$ qui à tout réel $x$ positif ou nul
associe $f(x) = x^{p}-x^{p-1}-1$.

\begin{noliste}{1.}
 \setlength{\itemsep}{4mm}
\item 

\begin{noliste}{a)}
 \setlength{\itemsep}{2mm}
\item Donner le tableau de variation de la fonction $f$.

\item En déduire les résultats suivants :

\begin{noliste}{$\sbullet$}
\item la fonction $f$ s'annule une seule fois en un réel noté $C$ qui
est
strictement supérieur à $1$.

\item pour tout réel $x$ positif ou nul, le réel $f(x)$ est strictement
positif si et seulement si $x$ est strictement supérieur à $C$.
\end{noliste}
\end{noliste}

\item Dans le cas particulier où l'entier $p$ est égal à $4$, comparer
$C$
et ${\dfrac{3}{2}}$.
\end{noliste}

\section*{Partie I}

On rappelle que si $a$ est un nombre complexe et $Q(X)$ un polynôme à
coefficients complexes alors le polynôme $Q(X)$ est divisible par $X-a$
si
et seulement si le complexe $Q(a)$ est nul.

\begin{noliste}{1.}
 \setlength{\itemsep}{4mm}
\item Soit $a$ un nombre complexe, $n$ un entier naturel au moins égal
à $2$
et $P\left(\Ev{X}\right)$ un polynôme à coefficients complexes de degré
$n$ s'écrivant $P\left(\Ev{X}\right) = \Sum{k = 0}{n}\alpha_{k}X^{k}$.

\begin{noliste}{a)}
 \setlength{\itemsep}{2mm}
\item Établir l'égalité : $P\left(\Ev{X}\right)-P\left(\Ev{a}\right) =
\left(\Ev{X-a}\right)Q\left(\Ev{X}\right)$ où $Q(X) = \Sum{k =
1}{n}\alpha_{k}(\Sum{i = 0}{k-1}a^{k-i-1}X^{i})$.

\item En déduire que le polynôme
$P\left(\Ev{X}\right)-P\left(\Ev{a}\right)$ est divisible par
$(X-a)^{2}$
si et seulement si le nombre complexe $\Sum{k = 1}{n}k\alpha
_{k}a^{k-1}$ est nul.

\item À quelle condition nécessaire et suffisante le nombre complexe
$a$
est-il racine au moins double du polynôme $P\left(\Ev{X}\right)$ ?
\end{noliste}

\item Montrer que le polynôme $X^{p}-X^{p-1}-1$ a $p$ racines simples
dans $\C$ et qu'elles sont toutes non nulles. Ces racines seront notées
$Z_{1},Z_{2},\dots,Z_{p}$ avec la convention que $Z_{p}$ est égal à
$C$.

\item 

\begin{noliste}{a)}
 \setlength{\itemsep}{2mm}
\item Établir, pour tout couple $(x,y)$ de nombres complexes,
l'inégalité : $\left| x\right| -\left| y\right| \leq \left|
x-y\right| $. Quand a-t-on l'égalité ?

\item Établir, pour tout entier $k$ tel que $1\leq k\leq p$,
l'inégalité : $\left| Z_{k}\right| \leq C$.

\item Montrer que si $k$ est un entier tel que $1\leq k\leq p$,
l'égalité $\left| Z_{k}\right| = C$ n'a lieu que si $k$ est égal à $p$.
\end{noliste}

\item Soit $\theta $ l'application de $\C_{p-1}[X]$ dans $\C^{p}$ qui à
tout polynôme complexe $P\left(\Ev{X}\right)$ de degré au plus égal à
$p-1$
associe l'élément
$(Z_{1}P\left(\Ev{Z_{1}}\right),Z_{2}P\left(\Ev{Z_{2}}\right),\dots,Z_{
}P\left(\Ev{Z_{p}}\right))$ de $\C^{p}$.

\begin{noliste}{a)}
 \setlength{\itemsep}{2mm}
\item Montrer que l'application $\theta $ est un isomorphisme.

\item En déduire que, pour tout élément $(u_{1},u_{2},\dots,u_{p})$ de
$\C^{p}$, il existe un unique élément $(\lambda_{1},\lambda
_{2},\dots,\lambda_{p})$ de $\C^{p}$ vérifiant 
\[
\left\{ 
\begin{tabular}{lllllllll}
$\lambda_{1}Z_{1}$ & $ + $ & $\lambda_{2}Z_{2}$ & $ + $ & $\;\cdots $ &
$ + $ & $\lambda_{p}Z_{p}$ & $\; = $ & $u_{1}$ \\
$\lambda_{1}Z_{1}{2}$ & $ + $ & $\lambda_{2}Z_{2}{2}$ & $ + $ &
$\;\cdots $
 & $ + $ & $\lambda_{p}Z_{p}{2}$ & $\; = $ & $u_{2}$ \\
$\vdots $ & & $\vdots $ & & & & $\vdots $ & & $\vdots $ \\
$\lambda_{1}Z_{1}{p}$ & $ + $ & $\lambda_{2}Z_{2}{p}$ & $ + $ &
$\;\cdots $
 & $ + $ & $\lambda_{p}Z_{p}{p}$ & $\; = $ &
$u_{p}$\end{tabular}\right.
\]
\end{noliste}

\item On note $F$ l'espace vectoriel complexe des suites complexes $u =
(u_{n})_{n\in \N^{\times }}$ vérifiant, pour tout entier $n$
strictement supérieur à $p$, l'égalité $u_{n} = u_{n-1} + u_{n-p}$.
Autrement
dit, on a :
\[
F = \{(u_{n})_{n\in \N^{\times }}\in \mathcal{S}_{c};\;\forall
n>p\quad u_{n} = u_{n-1} + u_{n-p}\}
\]

\begin{noliste}{a)}
 \setlength{\itemsep}{2mm}
\item Vérifier que $F$ est un sous-espace vectoriel de l'espace
vectoriel
complexe des suites complexes.

\item Montrer que, pour tout entier $k$ vérifiant les inégalités $1\leq
k\leq p$, la suite géométrique $(Z_{k}{n})_{n\in \N^{\times }}$
est élément de $F$.

\item Soit $u = (u_{n})_{n\in \N^{\times }}$ une suite élément de $F$
et soit $(\lambda_{1},\lambda_{2},\dots,\lambda_{p})$ l'unique solution
du système 
\[
\left\{ 
\begin{tabular}{lllllllll}
$\lambda_{1}Z_{1}$ & $ + $ & $\lambda_{2}Z_{2}$ & $ + $ & $\;\cdots $ &
$ + $ & $\lambda_{p}Z_{p}$ & $\; = $ & $u_{1}$ \\
$\lambda_{1}Z_{1}{2}$ & $ + $ & $\lambda_{2}Z_{2}{2}$ & $ + $ &
$\;\cdots $
 & $ + $ & $\lambda_{p}Z_{p}{2}$ & $\; = $ & $u_{2}$ \\
$\vdots $ & & $\vdots $ & & & & $\vdots $ & & $\vdots $ \\
$\lambda_{1}Z_{1}{p}$ & $ + $ & $\lambda_{2}Z_{2}{p}$ & $ + $ &
$\;\cdots $
 & $ + $ & $\lambda_{p}Z_{p}{p}$ & $\; = $ &
$u_{p}$\end{tabular}\right.
\]
On note $v = (v_{n})_{n\in \N^{\times }}$ la suite complexe de terme
général $v_{n} = \Sum{k = 1}{p}\lambda_{k}Z_{k}{n}$.\\
Montrer que les suites $u$ et $v$ sont égales.

\item Montrer que $((Z_{1}{n})_{n\in \N^{\times
}},(Z_{2}{n})_{n\in \N^{\times }},\dots,(Z_{p}{n})_{n\in \N^{\times
}})$ est une base de $F$.
\end{noliste}
\end{noliste}

\section*{Partie II}

\begin{noliste}{1.}
 \setlength{\itemsep}{4mm}
\item Pour tout entier naturel $q$, on considère l'application
$\Phi_{q}$
de $\R_{q}[X]$ dans lui-même qui à tout polynôme $A(X)$ de $\R_{q}[X]$
associe le polynôme : $\Phi_{q}(A(X)) = A(X)-A(X-1)-A(X-p)$.\\
Montrer que l'application $\Phi_{q}$ est un automorphisme de $\R_{q}X$.

\item Soit $Q(X)$ un polynôme à coefficients réels. On note $E_{Q}$
l'ensemble des suites complexes $u = (u_{n})_{n\in \N^{\times }}$
vérifiant, pour tout entier $n$ strictement supérieur à $p$, l'égalité
:
\[
u_{n} = u_{n-1} + u_{n-p} + Q(n).
\]
Autrement dit, on a : 
\[
E_{Q} = \{(u_{n})_{n\in \N^{\times }}\in \mathcal{S}_{c};\quad \forall
n>p\quad u_{n} = u_{n-1} + u_{n-p} + Q(n)\}
\]

\begin{noliste}{a)}
 \setlength{\itemsep}{2mm}
\item Montrer qu'il existe un unique polynôme à coefficients réels,
noté $A_{0}(X)$, tel que la suite $(A_{0}(n))_{n\in \N^{\times }}$ est
élément de $E_{Q}$.

\item Prouver qu'une suite complexe $u = (u_{n})_{n\in \N^{\times }}$
est élément de $E_{Q}$ si et seulement si la suite
$(u_{n}-A_{0}(n))_{n\in 
\N^{\times }}$ appartient à l'espace vectoriel $F$ défini dans la
question \textbf{I-5}).

\item En déduire que pour toute suite $(u_{n})_{n\in \N^{\times }}$
élément de $E_{Q}$, il existe un élément
$(\alpha_{1},\alpha_{2},\dots,\alpha_{p})$ de $\C^{p}$ tel que, pour
tout entier naturel $n$ non
nul, on a l'égalité :
\[
u_{n} = \alpha_{1}Z_{1}{n} + \alpha_{2}Z_{2}{n} + \cdots + \alpha
_{p-1}Z_{p-1}{n} + \alpha_{p}C^{n} + A_{0}(n).
\]

\item Soit $(u_{n})_{n\in \N^{\times }}$ une suite \textbf{réelle}
élément de $E_{Q}$. Déduire des questions précédentes que, soit il
existe un réel $\alpha $ non nul tel que $u_{n}\sim \alpha C^{n}$, soit
la suite $(u_{n})_{n\in \N^{\times }}$ est négligeable devant la suite
$(C^{n})_{n\in \N^{\times }}$ c'est-à-dire $u_{n} = o(C^{n})$.
\end{noliste}

\item Soit $Q(X)$ un polynôme à coefficients réels. On note $I_{Q}$
l'ensemble des suites réelles $u = (u_{n})_{n\in \N^{\times }}$
vérifiant, pour tout entier $n$ strictement supérieur à $p$,
l'inégalité : 
\[
u_{n}\leq u_{n-1} + u_{n-p} + Q(n).
\]
Autrement dit, on a : 
\[
I_{Q} = \{(u_{n})_{n\in \N^{\times }}\in \mathcal{S}_{r};\quad \forall
n>p\quad u_{n}\leq u_{n-1} + u_{n-p} + Q(n)\}
\]

\begin{noliste}{a)}
 \setlength{\itemsep}{2mm}
\item Soit $(w_{n})_{n\in \N^{\times }}$ une suite réelle élément de 
$F$ et à termes \textbf{strictement positifs}. Pour tout réel $a$ et
tout
entier naturel $n$ non nul, on pose $v_{a,n} = aw_{n} + A_{0}(n)$ et on
note $v_{a}$ la suite $(v_{a,n})_{n\in \N^{\times }}$.\\
Montrer que pour toute suite $(u_{n})_{n\in \N^{\times }}$ élément
de $I_{Q}$, il existe un réel $a$ tel que, pour tout entier naturel $n$
non
nul, on a l'inégalité : $u_{n}\leq v_{a,n}$.

\item Justifier l'existence d'une suite réelle élément de $F$ et à
termes 
\textbf{strictement positifs}. En déduire que si $(u_{n})_{n\in
\N^{\times }}$ est une suite réelle élément de $I_{Q}$ et à termes
\textbf{positifs ou nuls} alors la suite $(u_{n})_{n\in \N^{\times }}$
est
dominée par la suite $(C^{n})_{n\in \N^{\times }}$ c'est-à-dire $u_{n}
= O(C^{n})$.
\end{noliste}
\end{noliste}

\section*{Partie III}

Pour tout entier naturel $n$ au moins égal à $2$, on note $T_{n}$, ou
plus
simplement $T$, l'ensemble $\{1,2,\dots,n\}$ des entiers compris entre
$1$
et $n$. Pour toute partie $A$ de $T$ on note $\func{card}A$ le nombre
d'éléments de $A$. On considère une matrice $M = (\alpha
_{ij})_{l\leq i,j\leq n}$ carrée d'ordre $n$, \textbf{symétrique},
dont les coefficients valent $0$ ou $1$, les coefficients diagonaux
étant
nuls (on dit que $M$ est une matrice \textbf{d'incidence} d'ordre $n$).
On a
donc : 
\[
(\forall (i,j)\in T^{2}\quad (\alpha_{ij} = 0\;\text{ou}\;\alpha_{ij} =
1))\text{ et }(\forall i\in T\quad \alpha_{ii} = 0)
\]
Pour tout couple $(i,j)$ d'éléments de $T$ on dit que $i$ et $j$ sont
voisins si $\alpha_{ij} = 1$. Pour toute partie non vide $A$ de $T$ et
tout élément $i$ de $A$ on note $A(i)$ l'ensemble des éléments de $A$
voisins de $i$
et on dit que $A(i)$ est l'ensemble des voisins de $i$ dans $A$;
autrement
dit, on a : $A(i) = \{j\in A;\alpha_{ij} = 1\}$. \\
Une partie non vide $S$ de $T$ est dite stable si, pour tout élément
$i$ de $S$, $S(i)$ est vide. \textit{On remarquera que les singletons
de }$T$\textit{\ sont stables.} \\
Pour toute partie non vide $A$ de $T$, on appelle nombre de stabilité
de $A$
relativement à $M$ et on note $\omega (A,M)$, le maximum des cardinaux
des
parties stables incluses dans $A$, et on pose $\omega (\emptyset,M) =
0$.

\begin{noliste}{1.}
 \setlength{\itemsep}{4mm}
\item Dans cette question, on suppose que $n = 4$ et que $M = 
\begin{smatrix}
0 & 1 & 0 & 1 \\
1 & 0 & 1 & 1 \\
0 & 1 & 0 & 0 \\
1 & 1 & 0 & 0
\end{smatrix}
$.

\begin{noliste}{a)}
 \setlength{\itemsep}{2mm}
\item Déterminer $A(1)$ et $\omega (A,M)$ pour $A = \{1,3,4\}$.

\item Déterminer le nombre $\omega (T,M)$.
\end{noliste}

\item Dans le cas particulier où $M = (\alpha_{ij})_{1\leq i,j\leq
n}$ est la matrice dont les coefficients vérifient les conditions : 
\[
\alpha_{ij} = 1\quad \text{si}\quad \left| i-j\right| = 1\text{ et
}\alpha_{ij} = 0\quad \text{sinon},
\]
déterminer le nombre $\omega (T,M)$. \\
L'objet des questions suivantes est l'étude de la complexité de deux
algorithmes de calcul du nombre de stabilité de $T$ relativement à $M$,
la
complexité d'un tel algorithme étant définie comme étant le nombre
maximum
de "\ lectures "\ de coefficients de la matrice $M$ que nécessite, dans
le
pire des cas (suivant les valeurs de $M$), l'exécution de cet
algorithme.

\item Un algorithme "\ naïf "\ consiste à examiner, une à une, les
parties à
au moins deux éléments de $T$, supposées rangées selon un ordre
décroissant
de leur cardinal (ce rangement étant indépendant de $M$), jusqu'à
rencontrer
une partie stable (et c'est ce test qui nécessite des lectures dans
$M$) ;
bien entendu, si aucune partie stable n'a été rencontrée, $\omega
(T,M)$
vaut $1$.

\begin{noliste}{a)}
 \setlength{\itemsep}{2mm}
\item Calculer la somme $\Sum{k = 2}{n}C_{n}{k}C_{k}{2}$.

\item Montrer que, pour tout entier $n$ au moins égal à $2$, la
complexité
de l'algorithme "\ naïf "\ est supérieure ou égale à $2^{n}-(n + 1)$ et
inférieure ou égale à $C_{n}{2}2^{n-2}$.
\end{noliste}

\item Soit $A$ une partie non vide de $T$.\\
Montrer que, pour tout élément $i$ de $A$, on a l'égalité : 
\[
\omega (A,M) = \max \left( \omega (A\backslash \{i\},M),1 + \omega
(A\backslash
(\{i\}\cup A(i)),M)\right)
\]

\item On suppose données, en langage \Scilab{},

\begin{noliste}{$\sbullet$}
\item une déclaration de constante permettant de stocker la valeur de
l'entier $n$, la déclaration du type \texttt{tab} permettant de stocker
les
parties de $T$, et la déclaration du type \texttt{matrice} permettant
de
stocker les matrices d'incidence d'ordre $n$ ;

\item une fonction d'en-tête
\[
\begin{tabular}{l}
\texttt{function Appartient (i : integer ; A : tab) :
boolean;}\end{tabular}
\]
qui renvoie la valeur \texttt{true} si l'élément $i$ est dans la partie
$A$
et la valeur \texttt{false} sinon.
\end{noliste}

\begin{noliste}{a)}
 \setlength{\itemsep}{2mm}
\item Écrire, en langage \Scilab{}, une fonction d'en-tête :
\[
\begin{tabular}{l}
\texttt{function Recherche (A : tab; M : matrice) :
integer;}\end{tabular}
\]
qui renvoie le plus petit des éléments $i$ de $A$ pour lequel
$\func{card}A(i)$ est supérieur ou égal à $3$ si un tel plus petit
élément existe et qui
renvoie $0$ sinon.

\item Évaluer le nombre maximum de "\ lectures "\ de coefficients de la
matrice $M$ que nécessite cette fonction quand elle est appliquée à la
partie $A$.
\end{noliste}

\item \textbf{On admet }qu'il est possible de concevoir une fonction,
notée $Deux(A,M)$ renvoyant, lorsque, pour tout élément $i$ de $A$,
$\func{card}A(i) $ est inférieur ou égal à $2$, le nombre $\omega
(A,M)$ avec une
complexité inférieure ou égale à $(\func{card}A)^{2}$. \\
On considère maintenant la suite d'instructions $Omega$ dont on admet
qu'elle permet récursivement, quand elle est appliquée à la partie $A$
de $T$, d'obtenir la valeur de $\omega (A,M)$ :\\
DÉBUT 

\begin{noliste}{$\sbullet$}
\item Exécuter $Recherche(A,M)$ ; 

\item Si on a obtenu un élément $i$ de $A$ tel que
$\func{card}(A(i))\geq 3$ alors 

$\begin{tabular}{l}
Exécuter $Omega$ pour la partie $A\backslash \{i\}$ afin d'obtenir $a =
\omega
(A\backslash \{i\},M)$ ; \\
Exécuter $Omega$ pour la partie $A\backslash (\{i\}\cup A(i))$ afin
d'obtenir $b = \omega (A\backslash (\{i\}\cup A(i)),M)$ \\
Calculer $\max (a,1 + b)$ (qui est la valeur de $\omega (A,M)$
cherchée)
\end{tabular}$
\end{noliste}

Sinon exécuter $Deux(A,M)$ pour obtenir $\omega (A,M)$ ; \\
FIN \\
On note $u_{n}$ la complexité de cet algorithme lorsqu'il est appliqué
à $A = T $.\\
Justifier, pour tout entier $n$ au moins égal à $6$, l'inégalité :
$u_{n}\leq u_{n-1} + u_{n-4} + 2n^{2}$.

\item Comparer, pour de grandes valeurs de l'entier $n$, les
complexités de
l'algorithme "\ naïf "\ et de l'algorithme récursif.
\end{noliste}

\label{fin}

\end{document}

