\documentclass[11pt]{article}%
\usepackage{geometry}%
\geometry{a4paper,
 lmargin = 2cm,rmargin = 2cm,tmargin = 2.5cm,bmargin = 2.5cm}

\input{../../macros.tex}

\pagestyle{fancy} %
\lhead{ECE2 \hfill Mathématiques\\
} %
\chead{\hrule} %
\rhead{} %
\lfoot{} %
\cfoot{} %
\rfoot{\thepage} %

\renewcommand{\headrulewidth}{0pt}% : Trace un trait de séparation
 % de largeur 0,4 point. Mettre 0pt
 % pour supprimer le trait.

\renewcommand{\footrulewidth}{0.4pt}% : Trace un trait de séparation
 % de largeur 0,4 point. Mettre 0pt
 % pour supprimer le trait.

\setlength{\headheight}{14pt}

\title{\bf \vspace{-2cm} ESCP 1980} %
\author{} %
\date{} %
\begin{document}

\maketitle %
\vspace{-1.4cm}\hrule %
\thispagestyle{fancy}

\vspace*{.2cm}


% DEBUT DU DOC À MODIFIER : tout virer jusqu'au début de l'exo


\begin{center}
{\small CHAMBRE D\E\ COMMERCE ET D'INDUSTRIE DE PARIS}

\textbf{DIRECTION DE L'ENSEIGNEMENT}

Direction des Admissions et concours

\underline{\hspace*{3cm}}

{\Large ECOLE DES\ HAUTES\ ETUDES\ COMMERCIALES}

{\Large E.S.C.P.-E.A.P.}

{\Large ECOL\E\ SUPERIEUR\E\ D\E\ COMMERC\E\ D\E\ LYON}{\large }

CONCOURS D'ADMISSION\ SUR\ CLASSES\ PREPARATOIRES

\underline{\hspace*{3cm}}

\textbf{OPTION GENERALE}

{\Large MATHEMATIQUES I}

\textbf{Année 1980}

\underline{\hspace*{3cm}}
\end{center}

\begin{quotation}
\noindent \textsl{La présentation, la lisibilité, l'orthographe, la
qualité
de la rédaction, la clarté et la précision des raisonnements entreront
pour
une part importante dans l'appréciation des copies.}

\noindent \textsl{Les candidats sont invités à encadrer dans la mesure
du
possible les résultats de leurs calculs.}

\noindent \textsl{Ils ne doivent faire usage d'aucun document :
l'utilisation de toute calculatrice et de tout matériel électronique
est
interdite.}

\noindent \textsl{Seule l'utilisation d'une règle graduée est
autorisée.}

\noindent \textsl{\hrulefill }
\end{quotation}

\noindent Dans tout le problème, on désigne par :

\begin{noliste}{$\sbullet$}
\item $n$ un entier naturel non nul,

\item $p$ un nombre de l'intervalle $]0,1[$

\item $p_{1},$ $p_{2},$ $p_{3}$ des nombres réels strictement positifs

\item $\N_{n}$ le sous-ensemble de $\N$ formé par les
entiers naturels au plus égaux à $n$
\end{noliste}

\noindent Si $X$ est une variable aléatoire réelle discrète dont la
distribution de probabilité est définie par :
\[
P\left(\Ev{X = x_{i}}\right) = a_{i},\quad (i\in \N_{n}),
\]
et si $k$ est un entier naturel non nul, on appelle :

\begin{noliste}{$\sbullet$}
\item moment d'ordre $k$ de $X$ le nombre $\Sum{i\in
\N_{n}}a_{i}x_{i}{k}$

\item moment centré d'ordre $k$ le nombre $\Sum{i\in
\N_{n}}a_{i}(x_{i}-\E(X))^{k}$ où $\E(X)$ désigne l'espérance
mathématique de $X.
$
\end{noliste}

\section*{PARTI\E\ PRELIMINAIRE}

\begin{noliste}{1.}
 \setlength{\itemsep}{4mm}
\item Montrer que si $x$ et $y$ sont des entiers naturels tels que $x +
y\leq n$ on a les égalités :
\[
C_{n}{x}.C_{n-x}{y} = C_{n}{y}.C_{n-y}{x} = \dfrac{n!}{x!y!(n-x-x)!}
\]

\item Soit $Z$ une variable aléatoire dont la distribution de
probabilité
est binomiale de paramètres $n,p.$ Pour tout entier naturel $k$ non
nul, on désigne respectivement par $\mu_{k}$ et $m_{k}$ les moment et
moment centré
d'ordre $k$ de la variable aléatoire $Z.$\\
Démontrer que :
\[
\mu_{k + 1} = np\mu_{k} + p(1-p)\dfrac{d\mu_{k}}{dp}
\]
et pour $k\geq 2$ :
\[
m_{k + 1} = p(1-p)\left[ nkm_{k-1} + \dfrac{dm_{k}}{dp}\right] 
\]
où $\dfrac{d\mu_{k}}{dp}$ et $\dfrac{dm_{k}}{dp}$ sont les dérivées de
$\mu
_{k}$ et $m_{k}$ par rapport à $p.$
\end{noliste}

\section*{PARTI\E\ I}

Soit $T_{n}$ le sous-ensemble de $\N^{2}$ défini par :
\[
T_{n} = \{(x,y)\quad /\quad x\in \N,\quad y\in \N,\quad
x + y\leq n\}.
\]
On considère la fonction $f$ définie sur $T_{n}$ par :
\[
f(x,y) = \dfrac{n!}{x!y!(n-x-y)!}p_{1}{x}.p_{2}{y}.p_{3}{n-x-y}
\]

\begin{noliste}{1.}
 \setlength{\itemsep}{4mm}
\item A quelle condition doivent satisfaire $p_{1},$ $p_{2},$ $p_{3}$
pour
que l'on ait 
\[
\Sum{(x,y)\in T_{n}}f(x,y) = 1\text{ ?}
\]
\underline{On supposera cette condition réalisée dans toute la suite du
problème}

\item On considère la variable aléatoire discrète à deux dimensions,
notées $(X,Y)$, dont la distribution de probabilité conjointe est
définie pour tout $(x,y)$ élément de $T_{n}$ par :
\[
P\left(\Ev{X = x,Y = y}\right) = f(x,y)
\]

\begin{noliste}{a)}
 \setlength{\itemsep}{2mm}
\item Déterminer les distributions de probabilité marginales des
variables aléatoires $X$ et $Y.$

\item En déduire les espérances mathématiques $\E(X)$, $\E(Y)$ ainsi
que les
variances $\V(X)$ et $\V(Y)$ des variables aléatoires $X$ et $Y.$

\item Les variables aléatoires $X$ et $Y$ sont-elles indépendantes ?
Justifier votre réponse.
\end{noliste}

\item 

\begin{noliste}{a)}
 \setlength{\itemsep}{2mm}
\item Déterminer la distribution de probabilité de la variable
aléatoire $S = X + Y.$

\item Déduire du (I,3,a) la valeur de la covariance des variables
aléatoires 
$X$ et $Y,$ puis celle de leur coefficient de corrélation linéaire.
\end{noliste}
\end{noliste}

\section*{PARTI\E\ II}

\begin{noliste}{1.}
 \setlength{\itemsep}{4mm}
\item 

\begin{noliste}{a)}
 \setlength{\itemsep}{2mm}
\item Etant donné $y$ éléments de $\N_{n},$ déterminer la
distribution de probabilité conditionnelle de $X$ sachant $(Y = y).$\\
\textit{Dans la suite du problème on notera }$X_{y}$\textit{\ la
variable aléatoire ayant cette distribution de probabilité}.

\item La distribution de probabilité de $X_{y}$ est " classique ", la
reconnaître et en donner les paramètres.
\end{noliste}

\item Déterminer en fonction de $n,$ $y,$ $p_{1},$ $p_{2}$ l'espérance
mathématique $\E(X_{y})$ et la variance $\V(X_{y})$ de la variable
aléatoire $X_{y}.
$

\item Pour tout entier naturel non nul $k,$ on désigne respectivement
par $\mu_{y,k}$ les moment et moment centré d'ordre $k$ de la variable
aléatoire 
$X_{y}.$ Établir les relations (1) et (2) suivantes :
\[
(1)\qquad \mu_{y,k + 1} = (n-y)\dfrac{p_{1}}{1-p_{2}}\mu_{y,k} +
\dfrac{p_{1}p_{3}}{(1-p_{2})^{2}}\dfrac{d\mu_{y,k}}{d\left(
\dfrac{p_{1}}{1-p_{2}}\right) }
\]
et pour $k\geq 2$ :
\[
(2)\qquad m_{y,k + 1} = \dfrac{p_{1}p_{3}}{(1-p_{2})^{2}}\left[
(n-y)km_{y,k-1} + \dfrac{dm_{y,k}}{d\left(
\dfrac{p_{1}}{1-p_{2}}\right) }\right],
\]
où $\dfrac{d\mu_{y,k}}{d\left( \dfrac{p_{1}}{1-p_{2}}\right) }$ et
$\dfrac{dm_{y,k}}{d\left( \dfrac{p_{1}}{1-p_{2}}\right) }$ représentent
les dérivées
par rapport à $\dfrac{p_{1}}{1-p_{2}}$ de $\mu_{y,k}$ et $m_{y,k}.$

\item Que deviennent les résultats des questions II. 1), 2), 3),
lorsque
l'on considère la distribution de probabilité conditionnelle de $Y$
sachant $(X = x)$ ?\\
\textit{On notera dans la suite du problème }$Y_{X}$\textit{\ la
variable aléatoire ayant cette distribution de probabilité}
\end{noliste}

\section*{PARTI\E\ III}

Soient $\varphi $ et $\psi $ les applications de $\N_{n}$ dans $\R$
définies par :
\[
\varphi (x) = E(Y_{x})\qquad \text{et}\qquad \psi (y) = E(X_{y})
\]
Le plan affine étant rapporté au repère
$(O,\overrightarrow{x},\overrightarrow{y})$ on considère les
représentations graphiques $C$ et $\Gamma $ des fonctions :
\[
y = \varphi (x)\qquad \text{et}\qquad x = \psi (y).
\]

\begin{noliste}{1.}
 \setlength{\itemsep}{4mm}
\item Construire $C$ et $\Gamma $ dans le cas particulier :
\[
n = 12,\quad p_{1} = \dfrac{1}{2},\quad p_{2} = \dfrac{1}{3},\quad
p_{3} = \dfrac{1}{6}
\]

\item On appelle support $\varphi ^{s}$ et $\psi ^{s}$ des fonctions
$\varphi $ et $\psi $, les fonctions affines réelles dont les
restrictions à $\N_{n}$ sont respectivement $\varphi $ et $\psi.$ On
note $C^{s}$
et $\Gamma ^{s}$ leur représentation graphique.

\begin{noliste}{a)}
 \setlength{\itemsep}{2mm}
\item Déterminer les coordonnées du point d'intersection de $C^{s}$ et
$\Gamma ^{s}.$

\item A quelle(s) condition(s) $C$ et $\Gamma $ ont-elles un point
commun ?
\end{noliste}
\end{noliste}

\label{fin}

\end{document}

