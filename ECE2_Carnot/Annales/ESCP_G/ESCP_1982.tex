\documentclass[11pt]{article}%
\usepackage{geometry}%
\geometry{a4paper,
 lmargin = 2cm,rmargin = 2cm,tmargin = 2.5cm,bmargin = 2.5cm}

\input{../../macros.tex}

\pagestyle{fancy} %
\lhead{ECE2 \hfill Mathématiques\\
} %
\chead{\hrule} %
\rhead{} %
\lfoot{} %
\cfoot{} %
\rfoot{\thepage} %

\renewcommand{\headrulewidth}{0pt}% : Trace un trait de séparation
 % de largeur 0,4 point. Mettre 0pt
 % pour supprimer le trait.

\renewcommand{\footrulewidth}{0.4pt}% : Trace un trait de séparation
 % de largeur 0,4 point. Mettre 0pt
 % pour supprimer le trait.

\setlength{\headheight}{14pt}

\title{\bf \vspace{-2cm} ESCP 1982} %
\author{} %
\date{} %
\begin{document}

\maketitle %
\vspace{-1.4cm}\hrule %
\thispagestyle{fancy}

\vspace*{.2cm}


% DEBUT DU DOC À MODIFIER : tout virer jusqu'au début de l'exo


\begin{center}
{\small CHAMBRE D\E\ COMMERCE ET D'INDUSTRIE DE PARIS}

\textbf{DIRECTION DE L'ENSEIGNEMENT}

Direction des Admissions et concours

\underline{\hspace*{3cm}}

{\Large ECOLE DES\ HAUTES\ ETUDES\ COMMERCIALES}

{\Large E.S.C.P.-E.A.P.}

{\Large ECOL\E\ SUPERIEUR\E\ D\E\ COMMERC\E\ D\E\ LYON}{\large }

CONCOURS D'ADMISSION\ SUR\ CLASSES\ PREPARATOIRES

\underline{\hspace*{3cm}}

\textbf{OPTION GENERALE}

{\Large MATHEMATIQUES I}

\textbf{Année 1982}

\underline{\hspace*{3cm}}
\end{center}

\begin{quotation}
\noindent \textsl{La présentation, la lisibilité, l'orthographe, la
qualité
de la rédaction, la clarté et la précision des raisonnements entreront
pour
une part importante dans l'appréciation des copies.}

\noindent \textsl{Les candidats sont invités à encadrer dans la mesure
du
possible les résultats de leurs calculs.}

\noindent \textsl{Ils ne doivent faire usage d'aucun document :
l'utilisation de toute calculatrice et de tout matériel électronique
est
interdite.}

\noindent \textsl{Seule l'utilisation d'une règle graduée est
autorisée.}

\noindent \textsl{\hrulefill }
\end{quotation}

\noindent Toutes les matrices considérées dans ce problème sont à
coefficients réels.\\
La matrice $A$ a $n$ lignes et $p$ colonnes $(n,p)\in \N^{\times
}\times \N^{\times }$ de rang égal à $\inf (n,p)$ est donnée.\\
Quel que soit $k\in \N^{\times },$ on désigne par $I_{k}$ la matrice
unité d'ordre $k.$

\section*{PARTI\E\ I}

\begin{noliste}{1.}
 \setlength{\itemsep}{4mm}
\item On suppose $p<n$ et on désigne par $\Gamma (A)$ l'ensemble des
matrices $G$ carrés d'ordre $n,$ dites invariantes à gauche de $A,$
satisfaisant à la condition 
\[
(1)\qquad GA = A
\]

\begin{noliste}{a)}
 \setlength{\itemsep}{2mm}
\item Montrer que $\Gamma (A)$ est non vide et est stable sous la
multiplication des matrices.

\item Montrer que le sous-ensemble $\Gamma ^{\prime }(A)$ de $\Gamma
(A)$
formé par les matrices inversibles de $\Gamma (A)$ a une structure de
groupe
pour le produit matriciel.

\item Démontrer que toute matrice $G$ élément de $\Gamma (A)$ admet la
valeur propre $ + 1.$ Déterminer un minorant de la dimension du
sous-espace
propre associé à cette valeur propre.
\end{noliste}

\item On suppose $p>n;$ déterminer l'ensemble $\Gamma (A)$ défini comme
dans
la question 1.
\end{noliste}

\section*{PARTI\E\ II}

Dans cette partie, la matrice $G$ élément de $\Gamma (A)$, ainsi que la
matrice $A,$ sont décomposées en deux blocs de la manière suivante :
\[
A = 
\begin{smatrix}
A_{1} \\... \\
A_{2}\end{smatrix}
\qquad G = 
\begin{smatrix}
G_{1} & \vdots & G_{2}\end{smatrix}
\]
où \\
$A_{1}$ est une matrice carrée d'ordre $p$, SUPPOSE\E\ INVERSIBL\E\\
$A_{2}$ est une matrice à $n-p$ lignes et $p$ colonnes\\
$G_{1}$ est une matrice à $n$ lignes et $p$ colonnes\\
$G_{2}$ est une matrice à $n$ lignes et $n-p$ colonnes.

\begin{noliste}{1.}
 \setlength{\itemsep}{4mm}
\item Exprimer le produit $GA$ en fonction des matrices
$G_{1},G_{2},A_{1}$
et $A_{2}.$

\item Montrer que $G$ est parfaitement déterminée par le choix de
$G_{2}$ et
donner une décomposition en deux blocs de $G$ exprimée en fonction de
$A_{1},A_{2}$ et $G_{2}.$

\item On désigne par $\Delta (A)$ l'ensemble des matrices $D$ carrées
dites
invariantes à droite de $A$ satisfaisant à la condition $AD = A.$ On
note $^{t}A$ la transposée de $A.$

\begin{noliste}{a)}
 \setlength{\itemsep}{2mm}
\item Montrer que les ensembles $\Gamma (A)$ et $\Delta (^{t}A)$ sont
en
bijection.

\item Montrer que les ensembles $\Delta (A)$ et $\Gamma (^{t}A)$ sont
égaux.
\end{noliste}
\end{noliste}

\section*{PARTI\E\ III}

Dans cette partie, la matrice $A$ conserve sa décomposition en bloc
définie
en II, et la matrice $A_{1}$ est toujours supposée inversible. Quant à
la
matrice $G,$ élément de $\Gamma (A),$ on considère sa décomposition en
quatre blocs.
\[
G = 
\begin{smatrix}
G_{1}{\prime } & \vdots & G_{2}{\prime } \\
\cdots & \vdots & \cdots \\
G_{3}{\prime } & \vdots & G_{4}{\prime }\end{smatrix}
\]
où \\
$G_{1}{\prime }$ est une matrice carrée d'ordre $p.$\\
$G_{2}{\prime }$ est une matrice à $p$ lignes et $n-p$ colonnes\\
$G_{3}{\prime }$ est une matrice à $n-p$ lignes et $p$ colonnes\\
$G_{4}{\prime }$ est une matrice carrée d'ordre $n-p.$

\begin{noliste}{1.}
 \setlength{\itemsep}{4mm}
\item Établir la relation $G_{1}{\prime } = I_{p}-G_{2}{\prime
}A_{2}A_{1}{-1}$ et donner l'expression de $G_{3}{\prime }$ en fonction
de 
$G_{4}{\prime },$ $A_{1}$ et $A_{2}.$

\item Soit $A^{d}$ la matrice à $p$ lignes et $n$ colonnes décomposée
en
deux blocs de la manière suivante :
\[
A^{q} = 
\begin{smatrix}
A_{1}{d} & \vdots & A_{2}{d}\end{smatrix}
\]
où \\
$A_{1}{d}$ est une matrice carrée d'ordre $p$\\
$A_{2}{d}$ est une matrice à $p$ lignes et $n-p$ colonnes.\\
Exprimer le produit $AA^{d}$ en fonction de $A_{1},$ $A_{2},$
$A_{1}{d}$ et 
$A_{2}{d}.$

\item On dit que la matrice $A^{d}$ est une inverse à droite de $A$
relative 
à l'élément invariant $G$ de $\Gamma (A)$ si $A,$ $A^{d}$ et $G$
satisfont à
la relation (2)
\[
(2)\qquad AA^{d} = G
\]

\begin{noliste}{a)}
 \setlength{\itemsep}{2mm}
\item Exprimer $A_{1}{d}$ et $A_{2}{d}$ en fonction des matrices
$A_{1},$ $A_{2}$ et $G_{2}{\prime }.$

\item Déduire des résultats précédents que $G_{2}{\prime }$ et
$G_{4}{\prime }$ sont liées par la relation de compatibilité (3)
suivante :
\[
(3)\qquad A_{2}A_{1}{-1}G_{2}{\prime } = G_{4}{\prime }
\]
\end{noliste}

\item Soit $\Gamma ^{\prime \prime }(A)$ le sous-ensemble de $\Gamma
(A)$
formé des matrices $G$ relativement auxquelles $A$ a un inverse à
droite $A^{d}.$

\begin{noliste}{a)}
 \setlength{\itemsep}{2mm}
\item Démontrer que $\Gamma ^{\prime \prime }(A)$ est inclus dans
$\Delta
(A^{d})$\\
En déduire que $G$ élément de $\Gamma ^{\prime \prime }(A)$ est
idempotent
(c'est-à-dire : $G^{2} = G)$

\item Montrer que $G$ élément de $\Gamma ^{\prime \prime }(A)$ est
diagonalisable.
\end{noliste}

\item Soit $G$ appartenant à $\Gamma ^{\prime \prime }(A);$ on suppose
qu'il
existe une matrice $A^{g}$, inverse à gauche de $A$ relativement à un
élément $D$ de $\Delta (A),$ c'est-à-dire vérifiant $A^{g}A = D.$\\
Montrer que $A^{g} = A^{d}.$
\end{noliste}

\label{fin}

\end{document}

