\documentclass[11pt]{article}%
\usepackage{geometry}%
\geometry{a4paper,
 lmargin = 2cm,rmargin = 2cm,tmargin = 2.5cm,bmargin = 2.5cm}

\input{../../macros.tex}

\pagestyle{fancy} %
\lhead{ECE2 \hfill Mathématiques\\
} %
\chead{\hrule} %
\rhead{} %
\lfoot{} %
\cfoot{} %
\rfoot{\thepage} %

\renewcommand{\headrulewidth}{0pt}% : Trace un trait de séparation
 % de largeur 0,4 point. Mettre 0pt
 % pour supprimer le trait.

\renewcommand{\footrulewidth}{0.4pt}% : Trace un trait de séparation
 % de largeur 0,4 point. Mettre 0pt
 % pour supprimer le trait.

\setlength{\headheight}{14pt}

\title{\bf \vspace{-2cm} ESCP 1997 - voie Générale} %
\author{} %
\date{} %
\begin{document}

\maketitle %
\vspace{-1.4cm}\hrule %
\thispagestyle{fancy}

\vspace*{.2cm}


% DEBUT DU DOC À MODIFIER : tout virer jusqu'au début de l'exo


\begin{center}
{\small CHAMBRE D\E\ COMMERCE ET D'INDUSTRIE DE PARIS}

\textbf{DIRECTION DE L'ENSEIGNEMENT}

Direction des Admissions et concours

\underline{\hspace*{3cm}}

{\Large ECOLE DES\ HAUTES\ ETUDES\ COMMERCIALES}

{\Large E.S.C.P.-E.A.P.}

{\Large ECOL\E\ SUPERIEUR\E\ D\E\ COMMERC\E\ D\E\ LYON}{\large }

CONCOURS D'ADMISSION\ SUR\ CLASSES\ PREPARATOIRES

\underline{\hspace*{3cm}}

\textbf{OPTION SCIENTIFIQUE}

{\Large MATHEMATIQUES I}

\textbf{Année 1997}

\underline{\hspace*{3cm}}
\end{center}

\begin{quotation}
\noindent \textsl{La présentation, la lisibilité, l'orthographe, la
qualité
de la rédaction, la clarté et la précision des raisonnements entreront
pour
une part importante dans l'appréciation des copies.}

\noindent \textsl{Les candidats sont invités à encadrer dans la mesure
du
possible les résultats de leurs calculs.}

\noindent \textsl{Ils ne doivent faire usage d'aucun document :
l'utilisation de toute calculatrice et de tout matériel électronique
est
interdite.}

\noindent \textsl{Seule l'utilisation d'une règle graduée est
autorisée.}

\noindent \textsl{\hrulefill }
\end{quotation}

Le problème traite de quelques propriétés des polynômes de HERMITE qui
constituent une famille orthogonale pour un certain produit scalaire
qui
sera étudié dans ce problème.\\
On notera $\R[X]$ (resp. $\R_{n}[X]$) l'espace vectoriel des
polynômes à coefficients réels (resp. l'espace vectoriel des polynômes
de
degré inférieur ou égal à $n$), y compris le polynôme nul. Pour tout
entier
naturel $k$ le polynôme $X^{k}$ se confond avec la fonction polynomiale
réelle $x\mapsto x^{k},$ en particulier $X^{0}$ est la fonction
constante égale à 1.\\
On notera $[x]$ la partie entière d'un réel $x$.\\
Enfin on rappelle que $\dint{-\infty }{+ \infty
}e^{-\frac{x^{2}}{2}}\,dx = \sqrt{2\pi }$.

\section*{Partie I : Trois résultats utiles par la suite}

\begin{noliste}{1.}
 \setlength{\itemsep}{4mm}
\item 

\begin{noliste}{a)}
 \setlength{\itemsep}{2mm}
\item Pour tout entier naturel $n$, justifier la convergence de
l'intégrale 
\[
I_{n} = {\dfrac{1}{\sqrt{2\pi }}}\dint{-\infty }{+ \infty
}x^{n}e^{-\frac{x^{2}}{2}}dx
\]

\item Établir, pour tout entier $n\geq 2$, l'égalité : $I_{n} =
(n-1)I_{n-2}$.

\item Soit $n$ un entier naturel. Donner la valeur de $I_{2n + 1}$ et
montrer
que $I_{2n} = \dfrac{(2n)!}{2^{n}n!}$.

\item Pour toute fonction polynomiale $P$, justifier la convergence de
l'intégrale 
\[
{\dfrac{1}{\sqrt{2\pi }}}\dint{-\infty }{+ \infty
}P\left(\Ev{x}\right)e^{-\frac{x^{2}}{2}}dx
\]
\end{noliste}

\item On rappelle que si une suite de terme général $v_{n}$ est telle
que
les deux sous-suites de termes généraux $v_{2n}$ et $v_{2n + 1}$
convergent
vers le même réel $\ell $ alors la suite $(v_{n})_{n\in \N}$ est
elle-même convergente de limite $\ell $.\\
Soit $C$ un réel positif. Pour tout entier naturel $n$ on pose $u_{n} =
\dfrac{C^{n}}{\left[ \ \dfrac{n}{2}\right] \text{ }!}$.

\begin{noliste}{a)}
 \setlength{\itemsep}{2mm}
\item Déterminer $\dlim{n\rightarrow + \infty }{\dfrac{C^{2n}}{n!}}$.
En déduire la limite de la suite $(u_{n})_{n\in \N}$.

\item Montrer que la série de terme général $u_{2k} + u_{2k + 1}$ (o{ù}
$k\in 
\N$) converge et donner sa somme.

\item En déduire la convergence de la série de terme général $u_{n}$ et
la
valeur de la somme $\Sum{n = 0}{+ \infty }u_{n}$.
\end{noliste}

\item Soit $a$ un réel strictement positif et soit $g$ une fonction
réelle
indéfiniment dérivable sur $[-a,a]$ pour laquelle existe un réel
positif $K$
tel que, pour tout entier $n$ : 
\[
\max_{t\in \lbrack -a,a]}|g^{(n)}(t)|\leq {\dfrac{K^{n}n!}{\left[ 
\dfrac{n}{2}\right] !}}
\]

\begin{noliste}{a)}
 \setlength{\itemsep}{2mm}
\item Montrer que pour tout $\lambda {}\in \lbrack -a,a]$ : 
\[
\dlim{n\rightarrow + \infty }\dint{0}{\lambda {}}{\dfrac{(\lambda
{}-t)^{n}}{n!}}g^{(n + 1)}(t)dt = 0
\]

\item En déduire l'égalité suivante, valable pour tout $\lambda {}\in
\lbrack -a,a] :$ 
\[
g(\lambda {}) = \Sum{n = 0}{+ \infty }{\dfrac{g^{(n)}(0)}{n!}}\lambda
{}{n}
\]
Quelle simplification obtient-on si $g$ coïncide sur $[-a,a]$ avec une
fonction polynomiale de degré $d$ ?
\end{noliste}
\end{noliste}

\section*{Partie II : Les polynômes de Hermite}

\begin{noliste}{1.}
 \setlength{\itemsep}{4mm}
\item Montrer que l'application $(P,Q)\mapsto <P,Q> =
${$\dfrac{1}{\sqrt{2\pi }}$}$\dint{-\infty }{+ \infty
}e^{-\frac{x^{2}}{2}}P\left(\Ev{x}\right)Q\left(\Ev{x}\right)\,dx$ est
un produit scalaire sur $\R[X]$. On notera $\Vert.\Vert $ la norme
associée.\\
Ainsi, si $n$ est un entier naturel, la restriction de ce produit
scalaire
aux polynômes de degré au plus $n$ fait de $(\R_{n}[X],<.,.>.)$ un
espace euclidien.

\item À l'aide de la base $(1,X,X^{2},X^{3})$, construire une base
orthogonale de $(\R_{3}[X],<.,.>)$ formée de polynômes dont le
coefficient de plus haut degré est 1.\\
Pour tout entier naturel $n$, on considère l'application $H_{n}$ de
$\R$ dans $\R$ définie pour tout réel $x$ par : 
\[
H_{n}(x) = (-1)^{n}e^{\frac{x^{2}}{2}}\left(
e^{-\frac{x^{2}}{2}}\right) ^{(n)}
\]
o{ù}, selon l'usage, $f^{(n)}(x)$ désigne la valeur en $x$ de la
dérivée $n^{i\grave{e}me}$ de $f$ (en particulier $f^{(0)}(x) = f(x)$).

\item 

\begin{noliste}{a)}
 \setlength{\itemsep}{2mm}
\item Pour tout réel $x$ calculer
$H_{0}(x),H_{1}(x),H_{2}(x),H_{3}(x)$.

\item Soit $n\in \N^{\ast }$. Établir les relations 
\[
(1)\qquad {}H_{n + 1} = XH_{n}-nH_{n-1},\hspace{2cm}(2)\qquad
{}H_{n}{\prime
} = nH_{n-1}
\]
\textit{Pour établir (1), on pourra remarquer que }$\left(
e^{-\frac{x^{2}}{2}}\right) ^{(n + 1)} = \left(
-xe^{-\frac{x^{2}}{2}}\right) ^{(n)}$\textit{.}

\item Montrer que, pour tout $n\in \N$, $H_{n}$ est une fonction
polynomiale dont on précisera, en fonction de $n$, le degré, la parité
et le
coefficient de plus haut degré.
\end{noliste}

\item On dispose du 
\[
\begin{tabular}{c}
\texttt{Type \ \ Poly = array[0..20] of integer;}\end{tabular}
\]
On nommera de la même façon un polynôme de degré au plus 20 et la
variable
de type \texttt{Poly} obtenue en stockant dans la case numéro $k$,
$k\in {[\hspace{-0.15em}[0,20]\hspace{-0.13em}]}$, le coefficient de
$X^{k}$ dudit
polynôme.

\begin{noliste}{a)}
 \setlength{\itemsep}{2mm}
\item Écrire la partie instruction (i.e. sans les déclarations) d'une
procédure \Scilab{} dont l'en-tête est 
\[
\begin{tabular}{c}
\texttt{Procedure MULTIX (P, Var Q : Poly);}\end{tabular}
\]
qui stocke dans $Q$ les coefficients du polynôme $XP$, $P$ étant un
polynôme
de degré au plus 19.

\item À l'aide de (1), écrire la partie instruction d'une procédure
\Scilab{}
dont l'en-tête est 
\[
\begin{tabular}{c}
\texttt{Procedure HERMIT\E(n :integer; Var H :Poly);}\end{tabular}
\]
qui, étant donné un entier $n$, $n\in {[ \
\hspace{-0.15em}[2,20]\hspace{-0.13em}]}$, stocke les coefficients de
$H_{n}$ dans la variable de \texttt{H} de
type \texttt{Poly}.
\end{noliste}
\end{noliste}

\section*{Partie III : $(H_{n})_{n\in \N}$ comme famille de polynômes
orthogonaux}

\begin{noliste}{1.}
 \setlength{\itemsep}{4mm}
\item 

\begin{noliste}{a)}
 \setlength{\itemsep}{2mm}
\item Montrer que si $P$ est un polynôme et $n$ un entier naturel non
nul
alors :
\[
\dlim{x\rightarrow + \infty }P\left(\Ev{x}\right)\left(\Ev{
e^{-\frac{x^{2}}{2}}}\right)
^{(n-1)} = 0.
\]
De même, on montrerait et on \textbf{admet} que $\dlim{x\rightarrow
-\infty }P\left(\Ev{x}\right)\left(\Ev{ e^{-\frac{x^{2}}{2}}}\right)
^{(n-1)} = 0$.

\item Pour tout $n\in \N$, calculer 
\[
<H_{n},H_{0}> = {\dfrac{1}{\sqrt{2\pi }}}\dint{-\infty }{+ \infty
}H_{n}(x)e^{-\frac{x^{2}}{2}}dx
\]
\textit{Pour }$n$\textit{\ non nul, on utilisera la définition de
}$H_{n}$\textit{.}

\item Soit $(n,m)\in \left( \N^{\ast }\right) ^{2}$. En remarquant
que 
\[
<H_{n},H_{m}> = {\dfrac{(-1)^{n}}{\sqrt{2\pi }}}\dint{-\infty
}{+ \infty }H_{m}(x)\left( e^{-\frac{x^{2}}{2}}\right) ^{(n)}\,dx
\]
et à l'aide d'une intégration par parties que l'on effectuera avec
soin,
montrer que 
\[
<H_{n},H_{m}> = m<H_{n-1},H_{m-1}>
\]
En déduire que $(H_{n})_{n\in \N}$ est une famille orthogonale de
$\R[X]$.\\
Pour $n\in \N$, que vaut $<H_{n},H_{n}>$ ?
\end{noliste}

\item 

\begin{noliste}{a)}
 \setlength{\itemsep}{2mm}
\item Soit $k\in \N$ et $R$ une fonction polynomiale de degré au
plus $k$. Que vaut $<H_{k + 1},R>$ ?

\item Soit $n$ un entier naturel, $k$ un entier de ${[ \
\hspace{-0.15em}[0,n]\hspace{-0.13em}]}$ et $P$ un polynôme de degré au
plus $k$. Établir l'égalité 
\[
\Vert X^{k + 1}-P\Vert ^{2} = \Vert H_{k + 1}\Vert ^{2} + \Vert
Q-P\Vert ^{2}
\]
o{ù} $Q = X^{k + 1}-H_{k + 1}$. \textit{On pourra calculer }$<H_{k +
1},Q-P>$\textit{.}\\
Quelle est, dans l'espace euclidien $(\R_{n + 1}[X],<.,.>),$ la
projection orthogonale de $X^{k + 1}$ sur le sous-espace vectoriel
$\R_{k}[X]$ ?

\item On note $(G_{k})_{k\in {[ \ \hspace{-0.15em}[0,n +
1]\hspace{-0.13em}]}}$
la famille orthonormale de $(\R_{n + 1}[X],<.,.>)$ obtenue par le
procédé de SCHMIDT à partir de la base $(X^{k})_{k\in {[ \
\hspace{-0.15em}[0,n + 1]\hspace{-0.13em}]}}$. Pour tout $k\in {[ \
\hspace{-0.15em}[0,n + 1]\hspace{-0.13em}]}$, déterminer $G_{k}$ en
fonction de $H_{0},H_{1},\ldots
{},H_{n + 1} $.
\end{noliste}
\end{noliste}

\section*{Partie IV : Un développement en série de Hermite}

Soit $n$ un entier naturel non nul.

\begin{noliste}{1.}
 \setlength{\itemsep}{4mm}
\item Soit $P$ un polynôme de degré au plus $n$. Justifier l'égalité
suivante : 
\[
P = \Sum{k = 0}{n}<P,H_{k}>{\dfrac{H_{k}}{k!}}
\]

\item Pour tout couple $(b,c)$ de réels vérifiant $b\leq c$, on
\textbf{admet} qu'il existe un réel $K$ (dépendant de $b$ et $c$) tel
que pour tout
entier $n$ et tout $x\in \lbrack b,c]$ : 
\[
\left| \dfrac{H_{n}(x)}{n!}\right| \leq {\dfrac{K^{n}}{\left[ 
\dfrac{n}{2}\right] !}}
\]

\begin{noliste}{a)}
 \setlength{\itemsep}{2mm}
\item Soit $x$ un réel donné. À l'aide du 2) de la partie I établir,
pour
tout réel $\lambda {}$, la convergence de la série de terme général
$\dfrac{H_{n}(x)}{n!}\lambda {}{n}$.

\item Soit $g_{x}$ ($x$ est toujours un réel fixé) la fonction définie
pour
tout réel $\lambda {}$ par $g_{x}(\lambda {}) = e^{-\frac{(\lambda
{}-x)^{2}}{2}}$.\\
Pour tout réel $\lambda {}$ et tout entier naturel $n$, calculer
$g_{x}{(n)}(\lambda {})$ (c'est-à-dire ${\dfrac{d^{n}g_{x}}{d\lambda
{}{n}}}(\lambda {})$) en fonction de $H_{n}$.\\
Montrer que $g_{x}$ vérifie les hypothèses du 3) de la partie I et en
déduire que pour tout $(x,\lambda {})\in \R^{2}$ : 
\[
e^{\lambda x-\frac{\lambda {}{2}}{2}} = \Sum{n = 0}{+ \infty
}\dfrac{H_{n}(x)}{n!}\lambda {}{n}
\]

\item On note $\exp $ la fonction exponentielle $x\mapsto e^{x}$.\\
Pour tout entier naturel $n$, justifier rapidement la convergence de
l'intégrale 
\[
{\dfrac{1}{\sqrt{2\pi }}}\dint{-\infty }{+ \infty
}e^{x}H_{n}(x)e^{-\dfrac{x^{2}}{2}}\,dx
\]
dont, par analogie, on note $<\exp,H_{n}>$ la valeur.\\
Calculer $<\exp,H_{n}>$ puis, pour tout réel $x$, conclure à l'égalité 
\[
\exp \,x = \Sum{n = 0}{+ \infty }<\exp,H_{n}>\dfrac{H_{n}(x)}{n!}
\]
\textit{Pour calculer }$<\exp,H_{n}>$\textit{, on pourra utiliser la
définition de }$H_{n}$\textit{\ et intégrer par parties (avec soin)
afin
d'obtenir }$<\exp,H_{n}> = <\exp,H_{n-1}>$\textit{. }
\end{noliste}
\end{noliste}

\label{fin}

\end{document}

