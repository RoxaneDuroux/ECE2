\documentclass[11pt]{article}%
\usepackage{geometry}%
\geometry{a4paper,
 lmargin = 2cm,rmargin = 2cm,tmargin = 2.5cm,bmargin = 2.5cm}

\input{../../macros.tex}

\pagestyle{fancy} %
\lhead{ECE2 \hfill Mathématiques\\
} %
\chead{\hrule} %
\rhead{} %
\lfoot{} %
\cfoot{} %
\rfoot{\thepage} %

\renewcommand{\headrulewidth}{0pt}% : Trace un trait de séparation
 % de largeur 0,4 point. Mettre 0pt
 % pour supprimer le trait.

\renewcommand{\footrulewidth}{0.4pt}% : Trace un trait de séparation
 % de largeur 0,4 point. Mettre 0pt
 % pour supprimer le trait.

\setlength{\headheight}{14pt}

\title{\bf \vspace{-2cm} ESCP 1984 - voie Générale} %
\author{} %
\date{} %
\begin{document}

\maketitle %
\vspace{-1.4cm}\hrule %
\thispagestyle{fancy}

\vspace*{.2cm}


% DEBUT DU DOC À MODIFIER : tout virer jusqu'au début de l'exo


\begin{center}
{\small CHAMBRE D\E\ COMMERCE ET D'INDUSTRIE DE PARIS}

\textbf{DIRECTION DE L'ENSEIGNEMENT}

Direction des Admissions et concours

\underline{\hspace*{3cm}}

{\Large ECOLE DES\ HAUTES\ ETUDES\ COMMERCIALES}

{\Large E.S.C.P.-E.A.P.}

{\Large ECOL\E\ SUPERIEUR\E\ D\E\ COMMERC\E\ D\E\ LYON}{\large }

CONCOURS D'ADMISSION\ SUR\ CLASSES\ PREPARATOIRES

\underline{\hspace*{3cm}}

\textbf{OPTION GENERALE}

{\Large MATHEMATIQUES I}

\textbf{Année 1984}

\underline{\hspace*{3cm}}
\end{center}

\begin{quotation}
\noindent \textsl{La présentation, la lisibilité, l'orthographe, la
qualité
de la rédaction, la clarté et la précision des raisonnements entreront
pour
une part importante dans l'appréciation des copies.}

\noindent \textsl{Les candidats sont invités à encadrer dans la mesure
du
possible les résultats de leurs calculs.}

\noindent \textsl{Ils ne doivent faire usage d'aucun document :
l'utilisation de toute calculatrice et de tout matériel électronique
est
interdite.}

\noindent \textsl{Seule l'utilisation d'une règle graduée est
autorisée.}

\noindent \textsl{\hrulefill }
\end{quotation}

\noindent Dans ce problème, on dira qu'une fonction numérique $f$
possède un
développement asymptotique à l'ordre $n$ au voisinage de $ + \infty $
si :

\begin{noliste}{1.}
 \setlength{\itemsep}{4mm}
\item $f$ est définie sur un intervalle du type $]\alpha, + \infty
\lbrack
\subset \R_{+}$

\item il existe $n + 1$ nombres réels :
$a_{0},a_{1},...,a_{i},...,a_{n}$ et
une fonction $\varepsilon_{n},$ avec $\dlim{x\rightarrow + \infty
}\varepsilon_{n}(x) = 0$, tels que :
\[
(1)\qquad f(x) = a_{0} + \dfrac{a_{1}}{x} + \cdots +
\dfrac{a_{i}}{x^{i}} + \cdots + \dfrac{a_{n}}{x^{n}} +
(\dfrac{1}{x^{n}})\varepsilon_{n}(x)
\]
La formule (1) s'appelle développement asymptotique de $f,$ à l'ordre
$n,$
au voisinage de $ + \infty $\\
On pose $S_{n}(x) = a_{0} + \dfrac{a_{1}}{x} + \cdots +
\dfrac{a_{i}}{x^{i}} + \cdots
 + \dfrac{a_{n}}{x^{n}}$; on l'appelle partie régulière à l'ordre $n$
du développement asymptotique de $f.$
\end{noliste}

\noindent \emph{On définit de même un développement asymptotique à
l'ordre }$n$\emph{\ au voisinage de }$-\infty $

\section*{PARTI\E\ I}

\begin{noliste}{1.}
 \setlength{\itemsep}{4mm}
\item Démontrer que si $f$ admet un développement asymptotique à
l'ordre $n,$
au voisinage de $ + \infty,$ celui-ci est unique.

\item Démontrer que si $f$ admet un développement asymptotique à
l'ordre $n,$
au voisinage de $ + \infty,$ il en est de même pour la fonction $g$
définie
par :
\[
g(x) = f(x) + e^{-bx}\qquad \text{(}b>0\text{ donné)}
\]

\item Soit $\alpha $ un réel fixé, et soit $E_{\alpha,n}$ l'ensemble
des
fonctions $f$ définies sur $]\alpha, + \infty \lbrack $ à valeurs dans
$\R$ et admettant un développement asymptotique à l'ordre $n$ au
voisinage de $ + \infty.$

\begin{noliste}{a)}
 \setlength{\itemsep}{2mm}
\item Montrer que pour les opérations usuelles $E_{\alpha,n}$ possède
d'une
part une structure d'espace vectoriel sur $\R,$ d'autre part une
structure d'anneau commutatif unitaire.

\item Comment obtient-on la partie régulière du développement
asymptotique
du produit de deux fonctions de $E_{\alpha,n}$ ?
\end{noliste}

\item 

\begin{noliste}{a)}
 \setlength{\itemsep}{2mm}
\item Montrer que pour tout $x$ strictement positif, on a :
\[
\arctan x = \dfrac{\pi }{2}-\arctan \dfrac{1}{x}
\]

\item Déterminer le développement à l'ordre $2n + 1,$ au voisinage de $
+ \infty 
$ de la fonction $x\mapsto \arctan x.$

\item Quel est le développement asymptotique à l'ordre $2n + 1$ au
voisinage
de $-\infty $ de la fonction $x\mapsto \arctan x$
\end{noliste}
\end{noliste}

\section*{PARTI\E\ II}

Soient $\theta $ et $K$ les fonctions de la variable réelle $x$
définies par :
\[
\theta (x) = \dfrac{2}{\sqrt{\pi }}\dint{0}{x}e^{-t^{2}}dt\qquad
\text{et}\qquad K(x) = 1-\theta (x)
\]

\begin{noliste}{1.}
 \setlength{\itemsep}{4mm}
\item 

\begin{noliste}{a)}
 \setlength{\itemsep}{2mm}
\item Montrer que, pour tout $x\in \R :$
\[
K(x) = \dfrac{2}{\sqrt{\pi }}\dint{x}{+ \infty }e^{-t^{2}}dt
\]

\item En déduire que pour $x>0$ :
\[
K(x) = \dfrac{1}{\sqrt{\pi }}\dint{x^{2}}{+ \infty }u^{-1/2}e^{-u}du
\]
\end{noliste}

\item On considère l'intégrale :
\[
I_{n}(x) = \dint{x^{2}}{+ \infty }\dfrac{e^{-u}}{e^{(2n +
1)/2}}du\qquad 
\text{où }n\in \N\quad \text{et}\quad x\in \R^{\times }
\]

\begin{noliste}{a)}
 \setlength{\itemsep}{2mm}
\item Prouver que quel soit $n,$ $I_{n}(x)$ est convergente.

\item Établir la relation de récurrence :
\[
I_{n}(x) = \dfrac{e^{-x^{2}}}{x^{2n + 1}}-\dfrac{2n + 1}{2}.I_{n +
1}(x)
\]

\item Démontrer que pour $n\geq 0$ :
\[
\dlim{x\rightarrow + \infty }e^{x^{2}}I_{n}(x) = 0
\]
\end{noliste}

\item Déduire de ce qui précède le développement asymptotique à l'ordre
$2n + 1 $ : $e^{x^{2}}K(x),$ au voisinage de $ + \infty.$

\item Soit $X$ une variable aléatoire suivant la loi de Laplace-Gauss
centrée réduite, et soit $f$ sa fonction de répartition.

\begin{noliste}{a)}
 \setlength{\itemsep}{2mm}
\item Établir la relation :
\[
2F(x) = 1 + \theta (\dfrac{x}{\sqrt{2}})
\]

\item En déduire que $f(x) = 1-e^{-x^{2/2}}H(x)$ est une fonction dont
on
donnera le développement asymptotique à l'ordre $2n + 1$ au voisinage
de $ + \infty $
\end{noliste}
\end{noliste}

\label{fin}

\end{document}

