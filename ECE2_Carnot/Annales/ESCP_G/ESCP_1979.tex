\documentclass[11pt]{article}%
\usepackage{geometry}%
\geometry{a4paper,
 lmargin = 2cm,rmargin = 2cm,tmargin = 2.5cm,bmargin = 2.5cm}

\input{../../macros.tex}

\pagestyle{fancy} %
\lhead{ECE2 \hfill Mathématiques\\
} %
\chead{\hrule} %
\rhead{} %
\lfoot{} %
\cfoot{} %
\rfoot{\thepage} %

\renewcommand{\headrulewidth}{0pt}% : Trace un trait de séparation
 % de largeur 0,4 point. Mettre 0pt
 % pour supprimer le trait.

\renewcommand{\footrulewidth}{0.4pt}% : Trace un trait de séparation
 % de largeur 0,4 point. Mettre 0pt
 % pour supprimer le trait.

\setlength{\headheight}{14pt}

\title{\bf \vspace{-2cm} ESCP 1979} %
\author{} %
\date{} %
\begin{document}

\maketitle %
\vspace{-1.4cm}\hrule %
\thispagestyle{fancy}

\vspace*{.2cm}


% DEBUT DU DOC À MODIFIER : tout virer jusqu'au début de l'exo


\begin{center}
{\small CHAMBRE D\E\ COMMERCE ET D'INDUSTRIE DE PARIS}

\textbf{DIRECTION DE L'ENSEIGNEMENT}

Direction des Admissions et concours

\underline{\hspace*{3cm}}

{\Large ECOLE DES\ HAUTES\ ETUDES\ COMMERCIALES}

{\Large E.S.C.P.-E.A.P.}

{\Large ECOL\E\ SUPERIEUR\E\ D\E\ COMMERC\E\ D\E\ LYON}{\large }

CONCOURS D'ADMISSION\ SUR\ CLASSES\ PREPARATOIRES

\underline{\hspace*{3cm}}

\textbf{OPTION GENERALE}

{\Large MATHEMATIQUES I}

\textbf{Année 1979}

\underline{\hspace*{3cm}}
\end{center}

\begin{quotation}
\noindent \textsl{La présentation, la lisibilité, l'orthographe, la
qualité
de la rédaction, la clarté et la précision des raisonnements entreront
pour
une part importante dans l'appréciation des copies.}

\noindent \textsl{Les candidats sont invités à encadrer dans la mesure
du
possible les résultats de leurs calculs.}

\noindent \textsl{Ils ne doivent faire usage d'aucun document :
l'utilisation de toute calculatrice et de tout matériel électronique
est
interdite.}

\noindent \textsl{Seule l'utilisation d'une règle graduée est
autorisée.}

\noindent \textsl{\hrulefill }
\end{quotation}

\noindent Dans ce problème, on note 

\begin{noliste}{$\sbullet$}
\item $J$ l'ensemble $\{1,2,3\}$

\item $M$ une matrice carrée d'ordre $3$ à coefficients réels,
d'élément générique $a_{ij},$ $i$ étant l'indice de ligne, $j$ l'indice
de colonne.

\item $\lambda_{1},$ $\lambda_{2},$ $\lambda_{3}$ les valeurs propres
réelles ou complexes, distinctes ou confondues de $M$ et $\left|
\lambda
_{1}\right|,$ $\left| \lambda_{2}\right|,$ $\left|
\lambda_{3}\right| $ leurs modules.

\item $I$ la matrice unité de dimension 3
\[
I = 
\begin{smatrix}
1 & 0 & 0 \\
0 & 1 & 0 \\
0 & 0 & 1
\end{smatrix}
\]

\item $\func{tr}(M)$ le nombre : $\Sum{i\in J}a_{ii}$ appelé trace
de la matrice $M$
\end{noliste}

\section*{PARTI\E\ I}

\begin{noliste}{1.}
 \setlength{\itemsep}{4mm}
\item Démontrer que $\func{tr}(M) = \Sum{i\in J}\lambda_{i}$

\item Démontrer que l'une, au moins, des valeurs propres de $M$ est
réelle.

\item Soit $M^{2}$ le carré de la matrice $M.$ Démontrer que pour tout
$i\in
J,$ $\lambda_{i}{2}$ est valeur propre de $M^{2}.$\\
Réciproquement, si $\mu $ est une valeur propre de la matrice $M^{2},$
démontrer que l'une au moins des racines carrées (dans $\C)$ de $\mu $
est valeur propre de $M$ (on pourra poser $\mu = \omega ^{2})$\\
Quelle conclusion dégage-t-on de cette étude ?
\end{noliste}

\section*{PARTI\E\ II}

Dans cette partie, on suppose que $M$ satisfait à la condition (1)
suivante :
\[
(1)\qquad \exists k\in \R_{+}{\times },\quad \text{tel que}\quad 
\func{tr}M = k\quad \text{et}\quad \func{tr}(M^{2}) = k^{2}
\]

\begin{noliste}{1.}
 \setlength{\itemsep}{4mm}
\item Démontrer que $M$ admet au moins une valeur propre réelle non
nulle.

\item On suppose $\lambda_{1}>k.$

\begin{noliste}{a)}
 \setlength{\itemsep}{2mm}
\item Démontrer que les deux autres valeurs propres $\lambda_{2}$ et
$\lambda_{3}$ sont non réelles ou nulles.

\item Démontrer que $\left| \lambda_{2}\right| <\lambda_{1}$ et $\left|
\lambda_{3}\right| <\lambda_{1}.$
\end{noliste}
\end{noliste}

\section*{PARTI\E\ III}

Dans cette partie, on suppose que $M$ satisfait à la condition (2)
suivante :
\[
(2)\qquad \forall (i,j)\in J^{2},\quad a_{ij}\in \R_{+}{\times
}\quad \text{et}\quad a_{ij} = \dfrac{1}{a_{ij}}
\]

\begin{noliste}{1.}
 \setlength{\itemsep}{4mm}
\item Pour cette question, on suppose, de plus, $M$ singulière
(c'est-à-dire
non inversible).

\begin{noliste}{a)}
 \setlength{\itemsep}{2mm}
\item Déterminer les valeurs propres de $M.$

\item Démontrer que les vecteurs colonnes de $M$ sont vecteurs propres
de $M.
$ A quelle(s) valeur(s) propre(s) sont associés ces vecteurs propres ?

\item Démontrer que $M$ satisfait à la condition (3) suivante :
\[
(3)\qquad \forall n\in \N^{\times },\qquad M^{n} = 3^{n-1}M
\]

\item Soit $\Delta (\lambda ) = a_{3}\lambda ^{3} + a_{2}\lambda
^{2} + a_{1}\lambda + a_{0},$ le déterminant de la matrice $M-\lambda
I,$ où $\lambda $ est un complexe. Déduire de $(3)$ que, si on note
$\Delta (M)$ la
matrice : 
\[
a_{3}M^{3} + a_{2}M^{2} + a_{1}M + a_{0}I,
\]
alors $\Delta (M)$ est la matrice nulle.

\item Peut-on trouver un polynôme $\varphi $ à coefficients réels de
degré
strictement inférieur à trois tel que $\varphi (M)$ soit la matrice
nulle ?
\end{noliste}

\item Démontrer qu'une condition nécessaire et suffisante pour que $M$
soit
singulière est :
\[
\forall (i,j,k)\in J^{3},\qquad a_{ik}.a_{kj} = a_{ij}.
\]

\item 

\begin{noliste}{a)}
 \setlength{\itemsep}{2mm}
\item Démontrer que si $M$ est régulière (c'est-à-dire inversible) elle
admet alors une seule valeur propre réelle $\lambda_{1}$ et que
$\lambda
_{1}>3.$

\item En déduire que $\left| \lambda_{2}\right| <\lambda_{1}$ et
$\left| \lambda_{3}\right| <\lambda_{1}.$
\end{noliste}

\item Démontrer que $M$ est singulière si et seulement elle admet la
valeur
propre $3.$
\end{noliste}

\section*{PARTI\E\ IV}

L'espace vectoriel $\R^{3}$ étant rapporté à la base canonique $B_{0},$
soit $f_{1}$ l'endomorphisme de $\R^{3}$ dont la matrice
dans la base $B_{0}$ est :
\[
M_{0} = 
\begin{smatrix}
1 & \dfrac{1}{x} & \dfrac{1}{xz}\\
x & 1 & \dfrac{1}{z}\\
xz & z & 1
\end{smatrix},\qquad x>0\quad \text{et}\quad z>0.
\]

\begin{noliste}{1.}
 \setlength{\itemsep}{4mm}
\item Déterminer dans la base $B_{1}$ définie par les vecteurs :
\[
V_{1} = 
\begin{smatrix}
1 \\
x \\
xz
\end{smatrix}
\qquad V_{2} = 
\begin{smatrix}
1 \\
x \\
0
\end{smatrix}
\qquad V_{3} = 
\begin{smatrix}
0 \\
1 \\
-z
\end{smatrix}
\]
la matrice $T_{0}$ de l'endomorphisme $f_{1}.$

\item Soit $f_{\alpha }$ l'endomorphisme de $\R^{3}$ dont la matrice 
$T_{\alpha }$ dans la base $B_{1}$ est définie par :
\[
T_{\alpha } = T_{0} + \alpha 
\begin{smatrix}
1 & 0 & 0 \\
0 & 1 & 0 \\
0 & 0 & 0
\end{smatrix}
\]
Montrer que quel que soit $\alpha $ réel, $\alpha \neq 0$ et $\alpha
\neq 3,$
$T_{\alpha }$ est singulière et possède trois valeurs propres
distinctes de
que l'on déterminera.

\item 

\begin{noliste}{a)}
 \setlength{\itemsep}{2mm}
\item Déterminer la matrice $M_{\alpha }$ de l'endomorphisme $f_{\alpha
}$
dans la base $B_{0}.$

\item Déterminer une base $B_{2}$ de $\R^{3}$ dont les vecteurs
sont, pour tout $\alpha $ réel, $\alpha \neq 0$ et $\alpha \neq 3,$
vecteurs
propres de $f_{\alpha }$.

\item La matrice $M_{\alpha }$ est-elle diagonalisable ?
\end{noliste}
\end{noliste}

\label{fin}

\end{document}

