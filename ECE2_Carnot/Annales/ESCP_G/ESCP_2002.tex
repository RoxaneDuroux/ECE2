\documentclass[11pt]{article}%
\usepackage{geometry}%
\geometry{a4paper,
 lmargin = 2cm,rmargin = 2cm,tmargin = 2.5cm,bmargin = 2.5cm}

\input{../../macros.tex}

\pagestyle{fancy} %
\lhead{ECE2 \hfill Mathématiques\\
} %
\chead{\hrule} %
\rhead{} %
\lfoot{} %
\cfoot{} %
\rfoot{\thepage} %

\renewcommand{\headrulewidth}{0pt}% : Trace un trait de séparation
 % de largeur 0,4 point. Mettre 0pt
 % pour supprimer le trait.

\renewcommand{\footrulewidth}{0.4pt}% : Trace un trait de séparation
 % de largeur 0,4 point. Mettre 0pt
 % pour supprimer le trait.

\setlength{\headheight}{14pt}

\title{\bf \vspace{-2cm} ESCP 2002 - voie Générale} %
\author{} %
\date{} %
\begin{document}

\maketitle %
\vspace{-1.4cm}\hrule %
\thispagestyle{fancy}

\vspace*{.2cm}


% DEBUT DU DOC À MODIFIER : tout virer jusqu'au début de l'exo


\begin{center}
{\small CHAMBRE D\E\ COMMERCE ET D'INDUSTRIE DE PARIS}

\textbf{DIRECTION DE L'ENSEIGNEMENT}

Direction des Admissions et concours

\underline{\hspace*{3cm}}

{\Large ECOLE DES\ HAUTES\ ETUDES\ COMMERCIALES}

{\Large E.S.C.P.-E.A.P.}

{\Large ECOL\E\ SUPERIEUR\E\ D\E\ COMMERC\E\ D\E\ LYON}{\large }

CONCOURS D'ADMISSION\ SUR\ CLASSES\ PREPARATOIRES

\underline{\hspace*{3cm}}

\textbf{OPTION SCIENTIFIQUE}

{\Large MATHEMATIQUES I}

\textbf{Année 2002}

\underline{\hspace*{3cm}}
\end{center}

\begin{quotation}
\noindent \textsl{La présentation, la lisibilité, l'orthographe, la
qualité
de la rédaction, la clarté et la précision des raisonnements entreront
pour
une part importante dans l'appréciation des copies.}

\noindent \textsl{Les candidats sont invités à encadrer dans la mesure
du
possible les résultats de leurs calculs.}

\noindent \textsl{Ils ne doivent faire usage d'aucun document :
l'utilisation de toute calculatrice et de tout matériel électronique
est
interdite.}

\noindent \textsl{Seule l'utilisation d'une règle graduée est
autorisée.}

\noindent \textsl{\hrulefill }
\end{quotation}

\noindent Dans tout le problème, $n$ désigne un entier supérieur ou
égal à $2 $.\\
On considère une fonction réelle $f$ de classe $C^{\infty }$ sur
$[-1,1]$,
et on note $I(f)$ l'intégrale : $\dint{-1}{1}f(x)dx$.\\
Pour tout entier naturel $k$ non nul, on pose :
\[
M_{k}(f) = \sup_{x\in \lbrack -1,1]}f^{(k)}(x),\text{ où }f^{(k)\text{
}}\text{désigne la dérivée d'ordre }k\text{ de }f.
\]
Les polynômes considérés sont à coefficients réels, et on confond
polynôme
et fonction polynomiale associée.\\
Pour tout entier naturel $m$, on note $\R_{m}[X]$ le $\R$-espace
vectoriel des polynômes de degré inférieur ou égal à $m$.\\
On rappelle que si $r_{1},r_{2},\dots,r_{p}$ sont des racines réelles
distinctes d'un polynôme $P$, avec des multiplicités respectives
$k_{1},k_{2},\dots,k_{p}$, alors il existe un polynôme $Q$ tel que $P =
Q\prod\limits_{i = 1}{p}(X-r_{i})^{k_{i}}$.\\
Enfin, $a_{1},a_{2},\dots,a_{n}$ désignent $n$ réels deux à deux
distincts
de $[-1,1]$, et on note $A_{n}$ le polynôme : 
\[
A_{n} = \prod_{i = 1}{n}(X-a_{i})
\]
L'objet de ce problème est l'approximation de $I(f)$ par des intégrales
de
fonctions polynomiales.

\section*{Préliminaire}

\begin{noliste}{1.}
 \setlength{\itemsep}{4mm}
\item Énoncer le théorème de Rolle.

\item Soit $g$ une fonction de classe $C^{n}$ sur $[-1,1]$, s'annulant
en $n + 1$ points distincts de $[-1,1]$.

\begin{noliste}{a)}
 \setlength{\itemsep}{2mm}
\item Montrer que la dérivée de $g$ s'annule en au moins $n$ points
distincts de $]-1,1[$.

\item Montrer qu'il existe un réel $c$ de $]-1,1[$ tel que $g^{(n)}(c)
= 0$.
\end{noliste}
\end{noliste}

\section*{Partie I}

\textit{Dans cette partie, on va proposer comme valeur approchée de
}$I(f)$\textit{\ la valeur de l'intégrale obtenue en remplaçant la
fonction }$f$\textit{\ par la fonction polynomiale de degré inférieur
ou égal à }$n-1$\textit{, introduite ci-dessous, qui coïncide avec
}$f$\textit{\ sur chacun
des points }$a_{i}$\textit{.}\\
Pour tout entier $i$ de $\{1,2,\dots,n\}$, on note $L_{i}$ le polynôme
: $L_{i} = \prod\limits_{\QATOP{k\in \{1,\dots,n\}}{k\not =
i}}(X-a_{k})$.\\
Par exemple, si $n = 3$, $a_{1} = -1$, $a_{2} = 0$, et $a_{3} = 1$,
alors : $L_{1} = X(X-1)$, $L_{2} = (X-1)(X + 1)$, $L_{3} = X(X + 1)$.

\begin{noliste}{1.}
 \setlength{\itemsep}{4mm}
\item 

\begin{noliste}{a)}
 \setlength{\itemsep}{2mm}
\item Vérifier que, pour tous entiers $i$ et $j$ de $\{1,2,\dots,n\}$,
le réel $L_{i}(a_{j})$ est nul lorsque $i$ est différent de $j$, et est
non nul
lorsque $i$ est égal à $j$.

\item Montrer qu'il existe un \textbf{unique} polynôme, que l'on note
$P_{f}$, de degré inférieur ou égal à $n-1$, tel que, pour tout entier
$j$ de $\{1,2,\dots,n\}$, on a l'égalité $P_{f}(a_{j}) = f(a_{j})$, et
que ce polynôme est donné par la formule : 
\[
P = \Sum{i = 1}{n}\dfrac{f(a_{i})}{L_{i}(a_{i})}L_{i}
\]
\end{noliste}

\item Pour tout entier $i$ de $\{1,2,\dots,n\}$, on pose : $\delta_{i}
= \dfrac{1}{L_{i}(a_{i})}\dint{-1}{1}L_{i}(x)dx$.\\
Montrer que : $\dint{-1}{1}P_{f}(x)dx = \Sum{i = 1}{n}\delta
_{i}f(a_{i})$.\\
\textit{Dans toute la suite, on note :} $J_{n}(f) =
\dint{-1}{1}P_{f}(x)dx = \Sum{i = 1}{n}\delta_{i}f(a_{i})$.

\item Que peut-on dire de $I(f)$ et $J_{n}(f)$ lorsque $f$ est une
fonction
polynomiale de degré inférieur ou égal à $n-1$ ?

\item Soit $x$ un élément fixé de $[-1,1]$, distinct de chacun des
réels $a_{i}$.

\begin{noliste}{a)}
 \setlength{\itemsep}{2mm}
\item Justifier l'existence d'un réel $\lambda $ vérifiant l'égalité :
$f(x)-P_{f}(x)-\lambda A_{n}(x) = 0$.\\
On note maintenant $g_{\lambda }$ l'application qui à tout réel $t$ de
$[-1,1]$ associe : 
\[
g_{\lambda }(t) = f(t)-P_{f}(t)-\lambda A_{n}(t)
\]

\item Calculer $g_{\lambda }(a_{i})$ pour chaque entier $i$ de
$\{1,2,\dots,n\}$.

\item Montrer qu'il existe un réel $c$ de $]-1,1[$ tel que $g_{\lambda
}{(n)}(c) = 0$, puis établir l'égalité : 
\[
\lambda = \dfrac{f^{(n)}(c)}{n!}
\]
\end{noliste}

\item En déduire que, pour tout réel $x$ de $[-1,1]$, \quad $\left|
f(x)-P_{f}(x)\right| \leq \dfrac{M_{n}(f)}{n!}\left|
A_{n}(x)\right| $.\\
puis établir l'inégalité : 
\[
\left| I(f)-J_{n}(f)\right| \leq \dfrac{M_{n}(f)}{n!}\dint{-1}{1}\left|
A_{n}(x)\right\dx
\]

\item Dans cette question, on suppose que $a_{1} = -1$, $a_{n} = 1$ et
que $a_{1},a_{2},\dots,a_{n}$ sont répartis régulièrement, c'est-à-dire
que,
pour tout entier $i$ de $\{1,2,\dots,n\}$, on a : $a_{i} = -1 +
\dfrac{2(i-1)}{n-1}$.

\begin{noliste}{a)}
 \setlength{\itemsep}{2mm}
\item Soit $k$ un entier tel que $1\leq k\leq n-1$ et soit $x$ un réel
de $[a_{k},a_{k + 1}]$. Justifier l'inégalité : 
\[
\left| A_{n}(x)\right| \leq (\dfrac{2}{n-1})^{n}k!(n-k)!
\]

\item En déduire que, pour tout réel $x$ de $[-1,1]$, on a : $\left|
A_{n}(x)\right| \leq \left( \dfrac{2}{n-1}\right) ^{n}(n-1)!$.

\item \textbf{On admet} que, quand l'entier naturel $p$ tend vers
l'infini,
on a l'équivalence suivante : $p!\underset{p\rightarrow + \infty }{\sim
}\left( \dfrac{p}{e}\right) ^{p}\sqrt{2\pi p}$.\\
Montrer que, si l'entier $n$ est assez grand, on a, pour tout réel $x$
de $[-1,1]$, la majoration : 
\[
\left| A_{n}(x)\right| \leq \left( \dfrac{2}{e}\right) ^{n}
\]
\end{noliste}
\end{noliste}

\section*{Partie II}

\textit{Dans cette partie, on va proposer comme valeur approchée de
}$I(f)$\textit{\ la valeur de l'intégrale obtenue en remplaçant la
fonction }$f$\textit{\ par une certaine fonction polynomiale de degré
inférieur ou égal à 
}$2n-1$\textit{, qui réalise une approximation de }$f$\textit{\ plus
fine
que la fonction polynomiale de la partie précédente.}\\
Pour tout polynôme $Q$, on note $Q^{\prime }$ le polynôme dérivé de
$Q$.


\begin{noliste}{1.}
 \setlength{\itemsep}{4mm}
\item On considère l'application $T$ de $\R_{2n-1}[X]$ dans $\R^{2n}$
définie par : 
\[
\forall Q\in \R_{2n-1}[X],\;T(Q) =
(Q(a_{1}),Q(a_{2}),\dots,Q(a_{n}),Q^{\prime }(a_{1}),Q^{\prime
}(a_{2}),\dots,Q^{\prime }(a_{n}))
\]

\begin{noliste}{a)}
 \setlength{\itemsep}{2mm}
\item Montrer que $T$ est une application linéaire de $\R_{2n-1}[X]$
dans $\R^{2n}$.

\item Montrer que $T$ est injective (\textit{on rappelle qu'un réel
}$a$\textit{\ est racine au moins double d'un polynôme }$Q$\textit{\ si
et
seulement si }$Q(a) = Q^{\prime }(a) = 0$\textit{)}. En déduire que $T$
est
bijective.

\item Utiliser la question précédente pour montrer qu'il existe un
unique
polynôme, noté $Q_{f}$, de degré inférieur ou égal à $2n-1$, tel que,
pour
tout entier $j$ de $\{1,2,\dots,n\}$ : 
\[
Q_{f}(a_{j}) = f(a_{j})\text{ et }Q_{f}{\prime }(a_{j}) = f^{\prime
}(a_{j})
\]
(on ne demande pas d'expliciter $Q_{f}$)\\
\textit{Dans toute la suite, on note } : $K_{n}(f) =
\dint{-1}{1}Q_{f}(x)dx$.
\end{noliste}

\item Que peut-on dire de $I(f)$ et $K_{n}(f)$ lorsque $f$ est une
fonction
polynomiale de degré inférieur ou égal à $2n-1$ ?

\item Par une méthode analogue à celle de la partie précédente, on
pourrait démontrer, et \textbf{on admettra}, la majoration : 
\[
\left| I(f)-K_{n}(f)\right| \leq
\dfrac{M_{2n}(f)}{(2n)!}\dint{-1}{1}A_{n}{2}(x)dx
\]
Que vaut le polynôme $Q_{f}$ lorsque $f$ est la fonction polynomiale
$x\mapsto A_{n}{2}(x)$ ?\\
Montrer que, dans ce cas, l'inégalité précédente est une égalité.

\item Soit $\Phi $ l'application définie sur $\R_{n}[X]\times 
\R_{n}[X]$ par : 
\[
\forall (P,Q)\in \R_{n}[X]\times \R_{n}[X],\quad \Phi
(P,Q) = \dint{-1}{1}P\left(\Ev{x}\right)Q\left(\Ev{x}\right)dx
\]
Montrer que $\Phi $ est un produit scalaire.

\item L'espace vectoriel $\R_{n}[X]$ est maintenant muni de ce
produit scalaire.

\begin{noliste}{a)}
 \setlength{\itemsep}{2mm}
\item Justifier l'existence d'un polynôme $V$ de degré au plus $n-1$
vérifiant : $Q_{f}-P_{f} = A_{n}V$.\\
En déduire que si le polynôme $A_{n}$ est orthogonal à tout polynôme de
$\R_{n-1}[X]$, alors $K_{n}(f) = J_{n}(f)$.

\item Inversement, si le polynôme $A_{n}$ n'est pas orthogonal à tout
polynôme de $\R_{n-1}[X]$, montrer qu'il existe une fonction $f$ telle
que 
$K_{n}(f)\not = J_{n}(f)$.
\end{noliste}
\end{noliste}

\section*{Partie III}

\textit{Dans cette partie, l'espace vectoriel }$\R_{n}[X]$\textit{\
est toujours muni du produit scalaire }$\Phi $\textit{\ introduit dans
}\textbf{II.}4)$.$\\
On note $R_{n}$ l'image du polynôme $X^{n}$ par la projection
orthogonale
sur le sous espace-vectoriel $\R_{n-1}[X]$, et on pose : $S_{n} =
X^{n}-R_{n}$. Ainsi, $S_{n}$ est orthogonal à tout polynôme de
$\R_{n-1}[X]$, et $X^{n} = R_{n} + S_{n}$.

\begin{noliste}{1.}
 \setlength{\itemsep}{4mm}
\item En se plaçant dans le cas particulier où $n = 3$, déterminer
$S_{3}$.

\item On revient désormais au cas général.

\begin{noliste}{a)}
 \setlength{\itemsep}{2mm}
\item Déterminer le degré et le coefficient dominant de $S_{n}$.

\item Justifier l'égalité : $\dint{-1}{1}S_{n}(x)dx = 0$.\\
En déduire que le polynôme $S_{n}$ admet au moins une racine dans
$]-1,1[$.
\end{noliste}

\item On se propose de montrer que $S_{n}$ admet exactement $n$ racines
réelles distinctes dans l'intervalle $]-1,1[$.

\begin{noliste}{a)}
 \setlength{\itemsep}{2mm}
\item On suppose qu'il existe un réel $\alpha $ et un polynôme $Q$ tels
que $S_{n} = (X-\alpha )^{2}Q$.\\
Aboutir à une contradiction en considérant le signe de $S_{n}Q$ et la
valeur
de $\Phi (S_{n},Q)$.\\
En déduire que toutes les racines réelles de $S_{n}$ sont simples.

\item Soit $p$ le nombre de racines distinctes de $S_{n}$ qui
appartiennent à
$]-1,1[$, et soient $\alpha_{1},\alpha_{2},\dots,\alpha_{p}$ ces
racines.\\
On définit le polynôme : 
\[
U = \prod_{i = 1}{p}(X-\alpha_{i})
\]
Montrer que le polynôme $S_{n}U$ est de signe constant sur $[-1,1]$, et
en déduire, en considérant $\Phi (S_{n},U)$, que $p$ n'est pas
inférieur ou égal à
$n-1$.\\
Conclure que $S_{n}$ admet exactement $n$ racines réelles distinctes
$\alpha
_{1},\alpha_{2},\dots,\alpha_{n}$ dans $]-1,1[$, et que : 
\[
S_{n} = \prod_{j = 1}{n}(X-\alpha_{j})
\]
\textit{Dans toute la suite du problème, on suppose que
}$(a_{1},a_{2},\dots,a_{n}) =
(\alpha_{1},\alpha_{2},\dots,\alpha_{n})$\textit{, et on
conserve toutes les notations précédentes. En particulier, on a
maintenant }$A_{n} = S_{n}$\textit{, et, avec les réels
}$\delta_{1},\delta_{2},\dots,\delta_{n}$\textit{\ introduits dans la
partie I, (et qui sont indépendants de }$f$\textit{), on note toujours
}$J_{n}(f) = \dint{-1}{1}P_{f}(x)dx = \Sum{i =
1}{n}\delta_{i}f(\alpha_{i})$\textit{. }
\end{noliste}

\item En utilisant les résultats de la \textbf{partie II}, montrer que
: 
\[
\left| I(f)-J_{n}(f)\right| \leq
\dfrac{M_{2n}(f)}{(2n)!}\dint{-1}{1}S_{n}{2}(x)dx
\]

\item En se plaçant à nouveau dans le cas particulier où $n = 3$,
montrer que : 
\[
J_{3}(f) = \dfrac{1}{9}\left( 5f(-\sqrt{\dfrac{3}{5}}) + 8f(0) +
5f(\sqrt{\dfrac{3}{5}})\right)
\]

\item \textit{Étude des réels
}$\alpha_{1},\alpha_{2},\dots,\alpha_{n}$\textit{\ et
}$\delta_{1},\delta_{2},\dots,\delta_{n}$\textit{\ }

\begin{noliste}{a)}
 \setlength{\itemsep}{2mm}
\item En considérant $J_{n}(f)$ lorsque $f$ est constante égale à $1$,
donner la valeur de $\delta_{1} + \delta_{2} + \cdots + \delta_{n}$.

\item Pour chaque entier $j$ de $\{1,2,\dots,n\}$, montrer, en
considérant
la valeur de $J_{n}(f)$ lorsque $f$ est la fonction polynomiale
$x\mapsto
L_{j}{2}(x)$, que $\delta_{j}$ est positif.

\item On suppose dans cette question que les racines de $S_{n}$ sont
numérotées par ordre croissant, c'est-à-dire : 
\[
\alpha_{1}<\alpha_{2}<\dots <\alpha_{n}
\]
Justifier que $S_{n}(-X) = (-1)^{n}S_{n}(X)$.\\
En déduire que les réels $\alpha_{1},\alpha_{2},\dots,\alpha_{n}$ sont
répartis symétriquement par rapport à $0$, autrement dit que pour tout
entier $i$ de $\{1,2,\dots,n\}$, on a l'égalité : $\alpha_{n + 1-i} =
-\alpha_{i}$.\\
En conclure que, pour tout entier $j$ de $\{1,2,\dots,n\}$, on a
l'égalité : $\delta_{n + 1-j} = \delta_{j}$.
\end{noliste}

\item \textit{Majoration de }$\dint{-1}{1}S_{n}{2}(x)dx$\textit{\ }

\begin{noliste}{a)}
 \setlength{\itemsep}{2mm}
\item Montrer que pour tout polynôme $P$ de degré $n$ et de coefficient
dominant $1$, on a l'inégalité : 
\[
\dint{-1}{1}S_{n}{2}(x)dx\leq \dint{-1}{1}P^{2}(x)dx
\]

\item Montrer par récurrence que, pour tout entier naturel non nul $k$,
il
existe un polynôme $T_{k}$ de degré $k$ et de coefficient dominant $1$
tel
que, pour tout réel $\theta $ :\quad $\cos (k\theta ) =
2^{k-1}T_{k}(\cos
\theta )$.\\
En déduire la majoration : 
\[
\dint{-1}{1}S_{n}{2}(x)dx\leq \dfrac{\pi }{2^{2n-2}}
\]
\end{noliste}
\end{noliste}

\label{fin}

\end{document}

