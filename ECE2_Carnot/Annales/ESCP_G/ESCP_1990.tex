\documentclass[11pt]{article}%
\usepackage{geometry}%
\geometry{a4paper,
 lmargin = 2cm,rmargin = 2cm,tmargin = 2.5cm,bmargin = 2.5cm}

\input{../../macros.tex}

\pagestyle{fancy} %
\lhead{ECE2 \hfill Mathématiques\\
} %
\chead{\hrule} %
\rhead{} %
\lfoot{} %
\cfoot{} %
\rfoot{\thepage} %

\renewcommand{\headrulewidth}{0pt}% : Trace un trait de séparation
 % de largeur 0,4 point. Mettre 0pt
 % pour supprimer le trait.

\renewcommand{\footrulewidth}{0.4pt}% : Trace un trait de séparation
 % de largeur 0,4 point. Mettre 0pt
 % pour supprimer le trait.

\setlength{\headheight}{14pt}

\title{\bf \vspace{-2cm} ESCP 1990 - voie Générale} %
\author{} %
\date{} %
\begin{document}

\maketitle %
\vspace{-1.4cm}\hrule %
\thispagestyle{fancy}

\vspace*{.2cm}


% DEBUT DU DOC À MODIFIER : tout virer jusqu'au début de l'exo


\begin{center}
{\small CHAMBRE D\E\ COMMERCE ET D'INDUSTRIE DE PARIS}

\textbf{DIRECTION DE L'ENSEIGNEMENT}

Direction des Admissions et concours

\underline{\hspace*{3cm}}

{\Large ECOLE DES\ HAUTES\ ETUDES\ COMMERCIALES}

{\Large E.S.C.P.-E.A.P.}

{\Large ECOL\E\ SUPERIEUR\E\ D\E\ COMMERC\E\ D\E\ LYON}{\large }

CONCOURS D'ADMISSION\ SUR\ CLASSES\ PREPARATOIRES

\underline{\hspace*{3cm}}

\textbf{OPTION GENERALE}

{\Large MATHEMATIQUES I}

\textbf{Année 1990}

\underline{\hspace*{3cm}}
\end{center}

\begin{quotation}
\noindent \textsl{La présentation, la lisibilité, l'orthographe, la
qualité
de la rédaction, la clarté et la précision des raisonnements entreront
pour
une part importante dans l'appréciation des copies.}

\noindent \textsl{Les candidats sont invités à encadrer dans la mesure
du
possible les résultats de leurs calculs.}

\noindent \textsl{Ils ne doivent faire usage d'aucun document :
l'utilisation de toute calculatrice et de tout matériel électronique
est
interdite.}

\noindent \textsl{Seule l'utilisation d'une règle graduée est
autorisée.}

\noindent \textsl{\hrulefill }
\end{quotation}

\noindent Soit $f$ une fonction continue de $[a;b]$ dans $\R$, avec
$a{\large <}b$. L'objet de ce problème est l'étude d'approximations de
l'intégrale $J = \dint\limits_{a}{b}f(x)dx$ par la méthode des
rectangles, c'est à
dire par la suite $(u_{n})_{n\geq 1}$définie par :
\[
u_{n} = \dfrac{b-a}{n}\Sum{k = 0}{n-1}f(x_{k})\text{ avec} :x_{k} = a +
k\dfrac{b-a}{n}\text{ pour }0\leq k\leq n
\]
puis, de façon plus performante, par les suites $(v_{n})_{n\geq 1}$et
$(w_{n})_{n\geq 1}$définies par :

\[
\forall n\in \N^{\times },\quad v_{n} = 2u_{2n}-u_{n}\quad w_{n} =
\dfrac{4v_{2n}-v_{n}}{3}
\]

\section*{Partie I : Étude d'un premier exemple}

Dans cette partie on suppose que $[a,b] = [0,1]$ et que : $f(x) =
\dfrac{4}{1 + x^{2}}$.

\begin{noliste}{1.}
 \setlength{\itemsep}{4mm}
\item Écrire un algorithme de calcul de $u_{n}$ lorsque l'entier
naturel non
nul $n$ est donné. À l'aide de cet algorithme, remplir le tableau
suivant : 
\[
\begin{tabular}{|l|l|l|}
\hline
$u_{1}$ & $v_{1}$ & $w_{1}$ \\
\hline
$u_{2}$ & $v_{2}$ & $w_{2}$ \\
\hline
$u_{4}$ & $v_{4}$ & $w_{4}$ \\
\hline
$u_{8}$ & $v_{8}$ & \\
\hline
$u_{16}$ & & \\
\hline
\end{tabular}
\]
(on donnera les résultats numériques de ce tableau avec six décimales)

\item Calculer l'intégrale $J$. Évaluer la précision des résultats
numériques obtenus ci-dessus.
\end{noliste}

\section*{Partie II : Étude d'un second exemple}

Dans cette partie, on considère un réel $\lambda $ strictement positif
et
différent de 1. On suppose que $[a,b] = [0,\pi ]$ et que :
\[
f(x) = \ln (\lambda ^{2}-2\lambda \cos x + 1).
\]

\begin{noliste}{1.}
 \setlength{\itemsep}{4mm}
\item Soit $n$ un entier naturel non nul.

\begin{noliste}{a)}
 \setlength{\itemsep}{2mm}
\item Déterminer sous forme trigonométrique les racines dans $\C$ de
l'équation :
\[
y^{2n}-1 = 0.
\]

\item En déduire, en comparant leur coefficients dominants et leurs
racines,
l'égalité suivante entre polynômes :
\[
y^{2n}-1 = (y^{2}-1)\prod\limits_{k = 1}{n-1}\left( y^{2}-2ycos\left(
\dfrac{k\pi }{n}\right) + 1\right)
\]
\end{noliste}

\item 

\begin{noliste}{a)}
 \setlength{\itemsep}{2mm}
\item Déduire de la question précédente une expression simplifiée de
$u_{n}$
et de $v_{n}$.

\item En distinguant les cas $\lambda {\large <}1$ et $\lambda {\large
>}$1,
calculer l'intégrale $J$ en déterminant la limite de la suite
$(u_{n})_{n\geq 1}$, puis donner des équivalents de $u_{n}-J$ et de
$v_{n}-J$.
\end{noliste}
\end{noliste}

\section*{Partie III : Étude du cas général}

On suppose désormais que la fonction $f$ est de classe $C^{4}$sur
$[a,b]$.
Pour tout nombre entier naturel $k$ tel que $1\leq k\leq 4$, on
pose :

\begin{center}
\[
M_{k} = \sup\limits_{a\leq x\leq b}f^{(k)}(x)
\]
\end{center}

\begin{noliste}{1.}
 \setlength{\itemsep}{4mm}
\item Soit $[ \ \alpha,\beta ]$ un segment inclus dans $[a,b]$. On
considère
la fonction auxiliaire $p$ définie sur $[ \ \alpha,\beta ]$ par la
relation : 
\[
p(x) = \dint\limits_{\alpha }{x}f(t)dt-(x-\alpha )f(\alpha ).
\]

\begin{noliste}{a)}
 \setlength{\itemsep}{2mm}
\item Calculer les deux premières dérivées de $p$.

\item Montrer que, pour tout élément $x$ de $[ \ \alpha,\beta ]$ on a :
$\left| p(x)\right| \leq M_{1}$.\\
En déduire par intégration un encadrement de $p(x)$ pour $\alpha \leq
x\leq \beta $, puis établir l'inégalité suivante :
\[
\left| \dint\limits_{\alpha }{\beta }f(t)dt-(\beta -\alpha )f(\alpha
)\right| \leq \dfrac{(\beta -\alpha )^{2}}{2}M_{1}
\]

\item En appliquant cette inégalité aux segments $[x_{k},x_{k + 1}]$
pour $0\leq k\leq n-1$, Prouver enfin que :
\[
\left| J-u_{n}\right| \leq \dfrac{(b-a)^{2}}{2n}M_{1}
\]
\end{noliste}

\item Soit $[ \ \alpha,\beta ]$ un segment inclus dans $[a,b]$. On
considère
la fonction auxiliaire q définie sur $[ \ \alpha,\beta ]$ par la
relation :
\[
q(x) = p(x)-\dfrac{(x-\alpha )(f(x)-f(\alpha ))}{2}.
\]

\begin{noliste}{a)}
 \setlength{\itemsep}{2mm}
\item Calculer les deux premières dérivées de $q$.

\item Établir l'inégalité suivante :
\[
\dint\limits_{\alpha }{\beta }f(t)dt-(\beta -\alpha )f(\alpha
)-\dfrac{(\beta -\alpha )(f(\beta )-f(\alpha ))}{2}\leq \dfrac{(\beta
-\alpha
)^{3}}{12}M_{2}.
\]
(On pourra encadrer $q\textquotedblright (x)$ sur $[ \ \alpha,\beta ]$,
puis
par intégration, en déduire un encadrement de $q(x)$.)

\item Prouver enfin que :
\[
\left| J-u_{n}-\dfrac{(b-a)(f(b)-f(a))}{2n}\right| \leq
\dfrac{(b-a)^{3}}{12n^{2}}M_{2}.
\]
\end{noliste}

\item On considère cette fois la fonction auxiliaire $r$ définie sur $[
\ \alpha,\beta ]$ par la relation :
\[
r(x) = q(x) + \dfrac{(x-\alpha )^{2}(f(x)-f(\alpha ))}{12}.
\]
En procédant encore de la même manière, établir que :
\[
\left| J-u_{n}-\dfrac{(b-a)(f(b)-f(a))}{2n} +
\dfrac{(b-a)^{2}(f(b)-f(a))}{12n^{2}}\right| \leq
\dfrac{(b-a)^{5}}{720n^{4}}M_{4}.
\]

\item Déterminer à l'aide des résultats précédents le développement
limité à
l'ordre $3$ de $u_{n}$, c'est à dire des nombres réels $A,B,C$ et $D$
tels
que : 
\[
u_{n} = A + \dfrac{B}{n} + \dfrac{C}{n^{2}} + \dfrac{D}{n^{3}} +
\dfrac{\varepsilon
_{n}}{n^{3}}\text{ avec }\underset{n\rightarrow + \infty }{\lim
}\varepsilon
_{n} = 0.
\]
En déduire les développements limités à l'ordre $3$ de $v_{n}$et de
$w_{n}$.
Conclure.
\end{noliste}

\label{fin}

\end{document}

