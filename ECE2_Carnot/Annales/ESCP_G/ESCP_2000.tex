\documentclass[11pt]{article}%
\usepackage{geometry}%
\geometry{a4paper,
 lmargin = 2cm,rmargin = 2cm,tmargin = 2.5cm,bmargin = 2.5cm}

\input{../../macros.tex}

\pagestyle{fancy} %
\lhead{ECE2 \hfill Mathématiques\\
} %
\chead{\hrule} %
\rhead{} %
\lfoot{} %
\cfoot{} %
\rfoot{\thepage} %

\renewcommand{\headrulewidth}{0pt}% : Trace un trait de séparation
 % de largeur 0,4 point. Mettre 0pt
 % pour supprimer le trait.

\renewcommand{\footrulewidth}{0.4pt}% : Trace un trait de séparation
 % de largeur 0,4 point. Mettre 0pt
 % pour supprimer le trait.

\setlength{\headheight}{14pt}

\title{\bf \vspace{-2cm} ESCP 2000 - voie Générale} %
\author{} %
\date{} %
\begin{document}

\maketitle %
\vspace{-1.4cm}\hrule %
\thispagestyle{fancy}

\vspace*{.2cm}


% DEBUT DU DOC À MODIFIER : tout virer jusqu'au début de l'exo


\begin{center}
{\small CHAMBRE D\E\ COMMERCE ET D'INDUSTRIE DE PARIS}

\textbf{DIRECTION DE L'ENSEIGNEMENT}

Direction des Admissions et concours

\underline{\hspace*{3cm}}

{\Large ECOLE DES\ HAUTES\ ETUDES\ COMMERCIALES}

{\Large E.S.C.P.-E.A.P.}

{\Large ECOL\E\ SUPERIEUR\E\ D\E\ COMMERC\E\ D\E\ LYON}{\large }

CONCOURS D'ADMISSION\ SUR\ CLASSES\ PREPARATOIRES

\underline{\hspace*{3cm}}

\textbf{OPTION SCIENTIFIQUE}

{\Large MATHEMATIQUES I}

\textbf{Année 2000}

\underline{\hspace*{3cm}}
\end{center}

\begin{quotation}
\noindent \textsl{La présentation, la lisibilité, l'orthographe, la
qualité
de la rédaction, la clarté et la précision des raisonnements entreront
pour
une part importante dans l'appréciation des copies.}

\noindent \textsl{Les candidats sont invités à encadrer dans la mesure
du
possible les résultats de leurs calculs.}

\noindent \textsl{Ils ne doivent faire usage d'aucun document :
l'utilisation de toute calculatrice et de tout matériel électronique
est
interdite.}

\noindent \textsl{Seule l'utilisation d'une règle graduée est
autorisée.}

\noindent \textsl{\hrulefill }
\end{quotation}

\noindent Les \textbf{parties III }et \textbf{IV }sont indépendantes
des 
\textbf{parties I }et \textbf{II}.

\section*{Partie I}

On considère la fonction indéfiniment dérivable $\varphi $ définie,
pour
tout réel $x$ de $[0,1[$, par : $\varphi (x) = \dfrac{1}{\sqrt{1-x}}$.

\begin{noliste}{1.}
 \setlength{\itemsep}{4mm}
\item Pour tout réel $x$ de $[0,1[$ et tout entier naturel $n$, établir
l'égalité 
\[
\varphi ^{(n)}(x) = \dfrac{(2n)!}{4^{n}n!}(1-x)^{-\dfrac{2n + 1}{2}}
\]
où $\varphi ^{(n)}$ désigne la dérivée $n$-ième de $\varphi $ (avec, en
particulier, $\varphi ^{(0)} = \varphi $).

\item Pour tout entier naturel $n$ et tout réel $x$ de $[0,1[$,
justifier l'égalité suivante 
\[
\varphi (x) = \Sum{k = 0}{n}\dfrac{C_{2k}{k}}{4^{k}}x^{k} +
\dint{0}{x}\dfrac{(x-t)^{n}}{n!}\varphi ^{(n + 1)}(t)dt
\]

\item 

\begin{noliste}{a)}
 \setlength{\itemsep}{2mm}
\item Pour tout entier naturel $n$, prouver l'inégalité : $C_{2n + 2}{n
+ 1}\leq 4^{n + 1}$.

\item Pour tout couple $(t,x)$ de réels tel que $0\leq t\leq x<1$,
vérifier les inégalités : $0\leq \dfrac{x-t}{1-t}\leq x$.

\item En déduire que, pour tout réel $x$ de $[0,1[$, on a 
\[
\dlim{n\rightarrow + \infty }\dint{0}{x}\dfrac{(x-t)^{n}}{n!}\varphi
^{(n + 1)}(t)dt = 0
\]
\end{noliste}

\item Pour tout réel $x$ de $[0,1[$, démontrer l'égalité 
\[
\dfrac{1}{\sqrt{1-x}} = \Sum{k = 0}{+ \infty
}\dfrac{C_{2k}{k}}{4^{k}}x^{k}
\]
\end{noliste}

\section*{Partie II}

On se donne un espace probabilisé $(\Omega,\mathcal{B},P)$. Sur cet
espace,
on considère une suite $(X_{n})_{n\in \N^{\times }}$ de variables
aléatoires \underline{indépendantes} et de même loi que $X_{1}$, cette
loi étant définie par 
\[
P\left(\Ev{X_{1} = 1]}\right) = P\left(\Ev{X_{1} = -1]}\right) =
\dfrac{1}{2}
\]
On pose \underline{$S_{0} = 0$} et, pour tout entier naturel $n$ non
nul, $S_{n} = \Sum{k = 1}{n}X_{k} = X_{1} + X_{2} + \cdots + X_{n}$.\\
\textit{Par exemple, }$S_{n}$\textit{\ pourrait représenter l'abscisse
(aléatoire) au temps }$n$\textit{\ d'une particule se déplaçant sur un
axe et
partie de l'origine au temps }$0$\textit{, qui saute à chaque instant
d'une
unité à gauche ou d'une unité à droite avec une égale probabilité. \\
On note }$\min R$\textit{\ le plus petit élément d'une partie non vide
}$R$\textit{\ de }$\N$\textit{.}\\
On pose aussi, pour tout élément $\omega $ de $\Omega $, 
\[
R_{\omega } = \{n\in \N^{\times }/S_{n}(\omega ) = 0\}\text{ et
}T(\omega ) = \begin{tabular}{lll}
$\min R_{\omega }$ & si & $R_{\omega }\not = \emptyset $ \\
$0$ & si & $R_{\omega } = \emptyset $\end{tabular}
\]
On \textbf{admet} que $T$ est une variable aléatoire.\\
\textit{Ainsi }$T$\textit{\ pourrait être le temps d'attente
(aléatoire) du
premier retour à l'origine de la particule évoquée plus haut.}\\
Pour tout entier naturel $n$, on note $E_{n}$ l'événement $E_{n} =
[T>n]\cup
\lbrack T = 0]$.

\begin{noliste}{1.}
 \setlength{\itemsep}{4mm}
\item Soit $n$ un entier naturel non nul. On pose $A_{n} = [S_{n} = 0]$
et, pour
tout entier naturel $k$ tel que $0\leq k\leq n-1$, 
\[
A_{k} = (([S_{k} = 0]\cap \lbrack S_{k + 1}\not = 0]\cap \lbrack S_{k +
2}\not = 0]\cap
\dots \cap \lbrack S_{n}\not = 0]]) = ([S_{k} = 0]\cap
(\bigcap_{i = k + 1}{n}[S_{i}\not = 0]))
\]
\textit{Ainsi, pour tout entier }$k$\textit{\ tel que }$0\leq
k\leq n$\textit{, }$A_{k}$\textit{\ serait l'événement : }\\
\textit{"Pour la }\underline{dernière}\textit{\ fois avant l'instant
}$n$\textit{\ la particule est à l'origine à l'instant }$k$\textit{".}

\begin{noliste}{a)}
 \setlength{\itemsep}{2mm}
\item Pour tout entier $k$ tel que $0\leq k\leq n$, justifier l'égalité
suivante 
\[
P\left(\Ev{A_{k}}\right) = P\left(\Ev{S_{k} =
0]}\right)P\left(\Ev{E_{n-k}}\right)
\]

\item En déduire l'égalité : $1 = \Sum{k = 0}{n}P\left(\Ev{S_{k} =
0]}\right)P\left(\Ev{E_{n-k}}\right)$\\
On \textbf{admet} que, si deux suites $(a_{n})_{n\in \N}$ et
$(b_{n})_{n\in \N}$, à termes \textbf{positifs ou nuls}, sont telles
que les séries de termes généraux $a_{n}$ et $b_{n}$ convergent, alors
en
posant, pour tout entier naturel $n$, $\;c_{n} = \Sum{k =
0}{n}a_{k}b_{n-k}$, la série de terme général $c_{n}$ converge
et sa somme vérifie 
\[
\Sum{n = 0}{+ \infty }c_{n} = \left( \Sum{n = 0}{+ \infty }a_{n}\right)
\left(
\Sum{n = 0}{+ \infty }b_{n}\right)
\]
\end{noliste}

\item Pour tout réel $x$ de $[0,1[$, établir l'égalité 
\[
\dfrac{1}{1-x} = \left( \Sum{n = 0}{\infty }P\left(\Ev{S_{n} =
0]}\right)x^{n}\right) \left(
\Sum{n = 0}{\infty }P\left(\Ev{E_{n}}\right)x^{n}\right)
\]

\item 

\begin{noliste}{a)}
 \setlength{\itemsep}{2mm}
\item Pour tout entier naturel $n$, calculer $P\left(\Ev{S_{n} =
0]}\right)$.

\item À l'aide de la \textbf{partie I}, en déduire que, pour tout réel
$x$
de $[0,1[$, on a 
\[
\Sum{n = 0}{+ \infty }P\left(\Ev{E_{n}}\right)x^{n} = \sqrt{\dfrac{1 +
x}{1-x}}
\]

\item En remarquant que l'événement $[T = 0]$ est inclus dans $E_{n}$
pour
tout entier naturel $n$, montrer que l'on a : $P\left(\Ev{T =
0]}\right) = 0$.\\
\textit{Ainsi, presque sûrement, la particule citée en exemple, revient
à
l'origine.}
\end{noliste}
\end{noliste}

\section*{Partie III}

On considère dans cette partie une suite réelle $(a_{k})_{k\in \N}$
telle que, pour tout réel $x$ de $[0,1[$, la série de terme général
$a_{k}x^{k}$ converge. Pour tout réel $x$ de $[0,1[$, on note $f(x) =
\Sum{k = 0}{+ \infty }a_{k}x^{k}$ et l'on suppose que :
\[
\dlim{x\underset{<}{\rightarrow }1}(\sqrt{1-x}\;f(x)) = \sqrt{\pi }
\]

\begin{noliste}{1.}
 \setlength{\itemsep}{4mm}
\item 

\begin{noliste}{a)}
 \setlength{\itemsep}{2mm}
\item Pour tout entier naturel $p$, déterminer :
$\dlim{x\underset{<}{\rightarrow }1}(\sqrt{1-x}\Sum{k = 0}{+ \infty
}a_{k}x^{(p + 1)k})$.

\item Pour tout entier naturel $p$, justifier la convergence de
l'intégrale $\dint{0}{+ \infty }\dfrac{e^{-(p + 1)t}}{\sqrt{t}}dt$, et,
en utilisant
le changement de variable $u = \sqrt{2(p + 1)t}$, calculer sa valeur.

\item En déduire l'égalité 
\[
\dlim{x\underset{<}{\rightarrow }1}(\sqrt{1-x}\Sum{k = 0}{+ \infty
}a_{k}x^{(p + 1)k}) = \dint{0}{+ \infty }\dfrac{e^{-(p +
1)t}}{\sqrt{t}}dt
\]
\end{noliste}

\item Montrer que, pour toute application polynomiale réelle $Q$, on a
: 
\[
\dlim{x\underset{<}{\rightarrow }1}(\sqrt{1-x}\Sum{k = 0}{+ \infty
}a_{k}x^{k}Q(x^{k})) = \dint{0}{+ \infty
}\dfrac{e^{-t}}{\sqrt{t}}Q(e^{-t})dt
\]

\item Soit $h$ la fonction définie, pour tout réel $x$ de $[0,1[$, par
: 
\[
h(x) = \left\{ 
\begin{tabular}{ll}
$0$ & si $x\in \lbrack 0,\dfrac{1}{e}[$ \\
$\dfrac{1}{x}$ & si $x\in \lbrack \dfrac{1}{e},1[$\end{tabular}\right.
\]

\begin{noliste}{a)}
 \setlength{\itemsep}{2mm}
\item Justifier la convergence de l'intégrale $\dint{0}{+ \infty
}\dfrac{e^{-t}}{\sqrt{t}}h(e^{-t})dt$ et donner sa valeur.

\item Soit $x$ un réel de $[0,1[$. En déterminant la valeur de
$h(x^{k})$
pour $k$ assez grand, justifier la convergence de la série de terme
général $a_{k}x^{k}h(x^{k})$.
\end{noliste}

\item \textbf{On admet} l'égalité : $\dlim{x\underset{<}{\rightarrow
}1}(\sqrt{1-x}\Sum{k = 0}{+ \infty
}a_{k}x^{k}h(x^{k})) = \dint{0}{+ \infty
}\dfrac{e^{-t}}{\sqrt{t}}h(e^{-t})dt$\\
En utilisant ce résultat pour $x = e^{-\dfrac{1}{n}}$, en déduire que,
lorsque
l'entier naturel $n$ tend vers l'infini, $\Sum{k = 0}{n}a_{k}$ est
équivalent à $2\sqrt{n}$.
\end{noliste}

\section*{Partie IV}

On considère une suite $(a_{n})_{n\in \N}$ \textbf{décroissante} de
réels \textbf{positifs ou nuls} et, pour tout entier naturel $n$, on
pose : 
\[
S_{n} = \Sum{k = 0}{n}a_{k}
\]
On fait l'hypothèse que, lorsque $n$ tend vers $ + \infty $, $S_{n}$
est équivalent à $2\sqrt{n}$. On va montrer qu'alors $a_{n}$ est
équivalent à $\dfrac{1}{\sqrt{n}}$.\\
On notera $\lfloor x\rfloor $ la partie entière d'un réel $x$.

\begin{noliste}{1.}
 \setlength{\itemsep}{4mm}
\item Soit $(\alpha,\beta )$ un couple de nombres réels vérifiant :
$0<\alpha <1<\beta $. Pour tout entier naturel $n$ tel que $n\not =
\lfloor
\alpha n\rfloor $ et $n\not = \lfloor \beta n\rfloor $, justifier
l'encadrement 
\[
\dfrac{S_{\lfloor \beta n\rfloor }-S_{n}}{\lfloor \beta n\rfloor
-n}\leq a_{n}\leq \dfrac{S_{n}-S_{\lfloor \alpha n\rfloor }}{n-\lfloor
\alpha n\rfloor }
\]

\item 

\begin{noliste}{a)}
 \setlength{\itemsep}{2mm}
\item Soit $\gamma $ un réel strictement positif. Déterminer les
limites des
suites de termes généraux $\dfrac{n}{\lfloor \gamma n\rfloor }$ et
$\dfrac{S_{\lfloor \gamma n\rfloor }}{\sqrt{n}}$.

\item Soit $\varepsilon $ un réel strictement positif. Montrer que,
pour
tout entier naturel $n$ assez grand, on a 
\[
\dfrac{2(\sqrt{\beta }-1)}{\beta -1}-\varepsilon \leq \sqrt{n}a_{n}\leq
\dfrac{2(1-\sqrt{\alpha })}{1-\alpha } + \varepsilon
\]
\end{noliste}

\item En déduire qu'on a : $\dlim{n\rightarrow + \infty }\sqrt{n}a_{n}
= 1 $.
\end{noliste}

\section*{Partie V}

\begin{noliste}{1.}
 \setlength{\itemsep}{4mm}
\item 

\begin{noliste}{a)}
 \setlength{\itemsep}{2mm}
\item À l'aide des résultats obtenus dans les parties précédentes
déterminer, quand l'entier naturel $n$ tend vers l'infini, un
équivalent de $\Sum{k = 0}{n}P\left(\Ev{T>k}\right)$.

\item En déduire un équivalent de $P\left(\Ev{T>n}\right)$.
\end{noliste}

\item La variable aléatoire $T$ possède-t-elle une espérance ?

\item Pour tout réel $x$ de $[0,1]$, prouver l'égalité : 
\[
\Sum{n = 0}{+ \infty }P\left(\Ev{T = n]}\right)x^{n} = 1-\sqrt{1-x^{2}}
\]

\item Soit $n$ un entier naturel.

\begin{noliste}{a)}
 \setlength{\itemsep}{2mm}
\item Donner le développement limité au voisinage de $0$ à l'ordre $n$
de la
fonction $u\mapsto \sqrt{1 + u}$.

\item En déduire le développement limité au voisinage de $0$, à l'ordre
$2n$
de la fonction : 
\[
x\mapsto 1-\sqrt{1-x^{2}}
\]

\item Montrer que, au voisinage de $0$ on a aussi : 
\[
1-\sqrt{1-x^{2}} = \Sum{k = 0}{2n}P\left(\Ev{T = k]}\right)x^{k} +
o(x^{2n})
\]

\item En déduire que, pour tout entier naturel $n$ non nul, on a : 
\[
P\left(\Ev{T = 2n]}\right) = \dfrac{1}{2n-1}\cdot
\dfrac{C_{2n}{n}}{4^{n}}
\]
\\
\textit{On rappelle qu'il y a unicité du développement limité, au
voisinage
de }$0$\textit{, à l'ordre }$2n$\textit{\ d'une fonction.}\\
Pour tout élément $\omega $ de $\Omega $, on pose :
\[
R_{\omega }{\prime } = \{n\in \N^{\times }/\text{ }n>T(\omega )\}\text{
et }S_{n}(\omega ) = 0\text{ et }T_{2}(\omega ) = \left\{ 
\begin{tabular}{ccc}
$\min R_{\omega }{\prime }$ & si & $R_{\omega }{\prime }\not =
\emptyset $
\\
$0$ & si & $R_{\omega }{\prime } = \emptyset $\end{tabular}\right.
\quad \}
\]
On \textbf{admet} que $T_{2}$ est une variable aléatoire.\\
\textit{Ainsi }$T_{2}$\textit{\ pourrait être le temps d'attente
(aléatoire)
du deuxième retour à l'origine de la particule.}
\end{noliste}

\item 

\begin{noliste}{a)}
 \setlength{\itemsep}{2mm}
\item Pour tout entier naturel $n$ non nul, démontrer l'égalité : 
\[
P\left(\Ev{T_{2} = 2n]}\right) = \Sum{k = 0}{n}P\left(\Ev{T =
2k]}\right)P\left(\Ev{T = 2n-2k]}\right)
\]

\item En déduire la valeur de $P\left(\Ev{(T_{2} =
0}\right)\left(\Ev{T_{2} = 0}\right))$ puis, pour tout réel $x$ de
$[0,1]$, l'égalité : 
\[
\Sum{n = 0}{+ \infty }P\left(\Ev{T_{2} = n]}\right)x^{n} =
(1-\sqrt{1-x^{2}})^{2}
\]
\end{noliste}

\item Déterminer la loi de $T_{2}$.
\end{noliste}

\label{fin}

\end{document}

