\documentclass[11pt]{article}%
\usepackage{geometry}%
\geometry{a4paper,
 lmargin = 2cm,rmargin = 2cm,tmargin = 2.5cm,bmargin = 2.5cm}

\input{../../macros.tex}

\pagestyle{fancy} %
\lhead{ECE2 \hfill Mathématiques\\
} %
\chead{\hrule} %
\rhead{} %
\lfoot{} %
\cfoot{} %
\rfoot{\thepage} %

\renewcommand{\headrulewidth}{0pt}% : Trace un trait de séparation
 % de largeur 0,4 point. Mettre 0pt
 % pour supprimer le trait.

\renewcommand{\footrulewidth}{0.4pt}% : Trace un trait de séparation
 % de largeur 0,4 point. Mettre 0pt
 % pour supprimer le trait.

\setlength{\headheight}{14pt}

\title{\bf \vspace{-2cm} ESCP 2004 - voie Générale} %
\author{} %
\date{} %
\begin{document}

\maketitle %
\vspace{-1.4cm}\hrule %
\thispagestyle{fancy}

\vspace*{.2cm}


% DEBUT DU DOC À MODIFIER : tout virer jusqu'au début de l'exo


\begin{center}
{\small CHAMBRE D\E\ COMMERCE ET D'INDUSTRIE DE PARIS}

\textbf{DIRECTION DE L'ENSEIGNEMENT}

Direction des Admissions et concours

\underline{\hspace*{3cm}}

{\Large ECOLE DES\ HAUTES\ ETUDES\ COMMERCIALES}

{\Large E.S.C.P.-E.A.P.}

{\Large ECOL\E\ SUPERIEUR\E\ D\E\ COMMERC\E\ D\E\ LYON}{\large }

CONCOURS D'ADMISSION\ SUR\ CLASSES\ PREPARATOIRES

\underline{\hspace*{3cm}}

\textbf{OPTION SCIENTIFIQUE}

{\Large MATHEMATIQUES I}

\textbf{Année 2004}

\underline{\hspace*{3cm}}
\end{center}

\begin{quotation}
\noindent \textsl{La présentation, la lisibilité, l'orthographe, la
qualité
de la rédaction, la clarté et la précision des raisonnements entreront
pour
une part importante dans l'appréciation des copies.}

\noindent \textsl{Les candidats sont invités à encadrer dans la mesure
du
possible les résultats de leurs calculs.}

\noindent \textsl{Ils ne doivent faire usage d'aucun document :
l'utilisation de toute calculatrice et de tout matériel électronique
est
interdite.}

\noindent \textsl{Seule l'utilisation d'une règle graduée est
autorisée.}

\noindent \textsl{\hrulefill }
\end{quotation}

\noindent On note $E$ l'ensemble des fonctions $f$ de $\R$ dans $\R$
pour lesquelles il existe une suite réelle $s = (s_{n})_{n\in
\N^{\times }}$, dite adaptée à $f$, telle que :
\begin{equation}
\forall n\in \N^{\times },\quad \forall x\in \R,\quad
\Sum{k = 0}{n-1}f(x + \frac{k}{n}) = s_{n}f(nx) \label{def}
\end{equation}L'ensemble des fonctions de $\R$ dans $\R$ est noté
$\mathcal{F}(\R,\R)$.\\
Les polynômes considérés sont à coefficients réels, et tout polynôme
$P$
sera confondu avec la fonction polynomiale, élément de
$\mathcal{F}(\R,\R)$, qui lui est naturellement associée.\\
Pour tout entier naturel $p$ non nul, et toute fonction $p$ fois
dérivable $f $, de $\R$ dans $\R$, la dérivée $p$-ème de la fonction
$f$
est notée $f^{(p)}$ (la dérivée première de $f$ est aussi notée
$f^{\prime }$).\\
On rappelle que, $T$ étant un réel non nul, une fonction $f$ de $\R$
dans $\R$ est dite $T$-périodique lorsque :
\[
\forall x\in \R,\quad f(x + T) = f(x)
\]
L'objet du problème est de déterminer certaines des fonctions $f$
satisfaisant l'équation (\ref{def}).

\section*{Partie I Résultats généraux et exemples d'éléments de $E$}

\begin{noliste}{1.}
 \setlength{\itemsep}{4mm}
\item Soit $f$ une fonction appartenant à $E$, autre que la fonction
nulle.
\\
Montrer qu'il existe une \textit{unique} suite $s = (s_{n})_{n\in
\N^{\times }}$ adaptée à $f$, et que $s_{1} = 1$.

\item Montrer que si $f$ est une fonction dérivable appartenant à $E$,
alors
la dérivée $f^{\prime }$ de $f$ appartient à $E$.

\item Montrer que les fonctions constantes appartiennent à $E$.

\item Soit $A$ la fonction de $\R$ dans $\R$ qui à $x$
associe $x-\dfrac{1}{2}$. Établir que $A$ est élément de $E$.

\item $E$ constitue-t-il un sous-espace vectoriel du $\R$-espace
vectoriel $\mathcal{F}(\R,\R)$ ?

\item Soit $\chi $ la fonction de $\R$ dans $\R$ définie par :
\[
\forall x\in \R,\quad \chi (x) = \left\{
\begin{tabular}{ll}
$1$ & \textrm{\ }si\textrm{\ }$x\in \Z$ \\
$0$ & \textrm{\ }si\textrm{\ } $x\not\in \Z$\end{tabular}\right.
\]
Pour tout entier naturel non nul $n$ et tout réel $x$, déterminer, en
distinguant les cas $nx\in \Z$ et $nx\not\in \Z$, la valeur
de la somme $\Sum{k = 0}{n-1}\chi (x + \dfrac{k}{n})$. En déduire que
$\chi $ appartient à $E$, la suite adaptée étant constante, égale à
$1$.

\item 

\begin{noliste}{a)}
 \setlength{\itemsep}{2mm}
\item Pour tout réel $x$ et tous entiers naturels non nuls $p$ et $n$,
calculer $\Sum{k = 0}{n-1}e^{2ip\pi (x + \frac{k}{n})}$, et en déduire
que :
\[
\Sum{k = 0}{n-1}\cos (2p\pi (x + \frac{k}{n})) = \left\{
\begin{tabular}{ll}
$n\cos (2p\pi x)$ & si $p$ est multiple de $n$ \\
$0$ & sinon\end{tabular}\right. \
\]

\item Soit $u$ la fonction de $\R$ dans $\R$ qui à $x$
associe $\cos (2\pi x)$. Montrer que $u$ appartient à $E$, et préciser
la
suite adaptée à $u$.

\item Justifier, pour tout réel $x$, la convergence de la série de
terme général $\dfrac{1}{2^{q}}\cos (2^{q + 1}\pi x)$.

Soit alors $v$ la fonction de $\R$ dans $\R$ qui à $x$
associe $\Sum{q = 0}{+ \infty }\dfrac{1}{2^{q}}\cos (2^{q + 1}\pi x)$.
\\
Montrer que $v$ appartient à $E$, et préciser la suite adaptée à $v$.
\end{noliste}
\end{noliste}

\section*{Partie II Recherche des polynômes éléments de $E$}

\begin{noliste}{1.}
 \setlength{\itemsep}{4mm}
\item 

\begin{noliste}{a)}
 \setlength{\itemsep}{2mm}
\item Montrer que si $P$ est un polynôme de degré $1$ élément de $E$,
alors
la suite adaptée au polynôme $P$ est constante, égale à $1$.

\item Quels sont les polynômes de degré $1$ appartenant à $E$ ?
\end{noliste}

\item On suppose dans cette question que $P$ est un polynôme non nul
élément
de $E$, et on note $p$ le degré de $P$.

\begin{noliste}{a)}
 \setlength{\itemsep}{2mm}
\item Montrer que la suite adaptée à $P$ est la suite $s =
(s_{n})_{n\in
\N^{\times }}$ définie par :
\[
\forall n\in \N^{\times },\quad s_{n} = \dfrac{1}{n^{p-1}}
\]

\item Montrer que, si $p$ est au moins égal à 1, on a l'égalité :\quad
$\dint{0}{1}P\left(\Ev{t}\right)dt = 0$.
\end{noliste}

\item Établir que, pour tout polynôme $Q$, il existe un unique polynôme
$P$
tel que $P^{\prime } = Q$ et $\dint{0}{1}P\left(\Ev{t}\right)dt = 0$.\\
\textit{On peut donc définir une suite }$(B_{p})_{p\in \N}$\textit{\
de polynômes de la manière suivante : }
\[
\left\{
\begin{array}{lll}
B_{0} = 1 & & \\
\forall p\in \N^{\times },\quad B_{p}{\prime } = pB_{p-1} & \text{et}
 & \dint{0}{1}B_{p}(t)dt = 0
\end{array}
\right. \
\]

\item 

\begin{noliste}{a)}
 \setlength{\itemsep}{2mm}
\item Déterminer, pour chaque entier naturel $p$, le degré et le
coefficient
dominant de $B_{p}$.

\item Vérifier, pour tout réel $x$, l'égalité : $B_{1}(x) =
x-\dfrac{1}{2}$,
puis calculer $B_{2}(x)$ pour tout réel $x$.
\end{noliste}

\item On a déjà vu dans la partie I que $B_{0}$ et $B_{1}$ sont des
éléments
de $E$. Vérifier que $B_{2}$ est élément de $E$.

\item Soit $p$ un entier naturel non nul. On suppose que $B_{p-1}$ est
élément de $E$ et on veut montrer que $B_{p}$ est élément de $E$. Pour
cela, on
fixe un entier naturel non nul $n$ et on pose, pour tout réel $x$ :
\[
\varphi (x) = \Sum{k = 0}{n-1}B_{p}(x + \frac{k}{n})\text{\quad et\quad
}\psi
(x) = \frac{1}{n^{p-1}}B_{p}(nx)
\]

\begin{noliste}{a)}
 \setlength{\itemsep}{2mm}
\item Montrer que la fonction $\varphi -\psi $ est constante.

\item Calculer $\dint{0}{1/n}\varphi (x)dx\text{ et }\dint{0}{1/n}\psi
(x)dx$.

\item Établir que $\varphi -\psi = 0$ et conclure.
\end{noliste}

\item Déduire des questions précédentes que, pour tout entier naturel
$p$,
les polynômes de degré $p$ qui appartiennent à $E$ sont exactement les
polynômes $\lambda B_{p}$ obtenus lorsque $\lambda $ décrit $\R^{\times
}$.
\end{noliste}

\section*{Partie III Étude des fonctions indéfiniment dérivables de
$E$}

\begin{noliste}{1.}
 \setlength{\itemsep}{4mm}
\item Soit $\delta $ la fonction de $\mathcal{F}(\R,\R)$
dans lui-même qui, à toute fonction $\varphi $ de $\R$ dans $\R$,
associe la fonction $\delta (\varphi )$ définie par :
\[
\forall x\in \R,\quad \delta (\varphi )(x) = \varphi (x + 1)-\varphi
(x)
\]

\begin{noliste}{a)}
 \setlength{\itemsep}{2mm}
\item Montrer que $\delta $ est linéaire. Quelle propriété caractérise
les éléments de son noyau ?

\item Vérifier que, lorsque $P$ est une fonction polynomiale, il en est
de même de $\delta (P)$, puis préciser le degré et le coefficient
dominant de $\delta (P)$ lorsque $P$ est de degré $p$ supérieur ou égal
à $1$.
\end{noliste}

\item Montrer que, si $f$ est une fonction élément de $E$, de suite
adaptée $s = (s_{n})_{n\in \N^{\times }}$, on a :
\begin{equation}
\forall n\in \N^{\times },\quad \forall x\in \R,\quad
s_{n}\delta (f)(nx) = \delta (f)(x) \label{s_{n}delta}
\end{equation}

\item Soit $g$ une fonction de classe $C^{\infty }$ de $\R$ dans $\R$.
On suppose qu'il existe un réel $\alpha $ tel que :
\begin{equation}
\forall x\in \R,\quad \alpha g(2x) = g(x)
\end{equation}

\begin{noliste}{a)}
 \setlength{\itemsep}{2mm}
\item Montrer que :
\begin{equation}
\forall x\in \R,\text{\quad }\forall k\in \N,\quad \alpha
^{k}g(x) = g(\frac{x}{2^{k}})
\end{equation}

\item Montrer que si $\alpha = 0$, alors $g$ est nulle.

\item Montrer que si $\left| \alpha \right| >1$, alors $g$ est nulle.

\item On suppose $0<\left| \alpha \right| \alpha \leq
1$. Justifier l'existence d'un entier naturel $p$ et d'un réel
$\beta $ tels que :
\[
\left| \beta \right| >1\quad \text{et\quad }\forall x\in \R,\quad \beta
g^{(p)}(2x) = g^{(p)}(x)
\]

\item En déduire que, dans tous les cas, $g$ est polynomiale.
\end{noliste}

\item Dans cette question, on suppose que $f$ est une fonction de
classe $C^{\infty }$ élément de $E$, de suite adaptée $s =
(s_{n})_{n\in \N^{\times }}$, et que $\delta (f)$ n'est pas la fonction
nulle.

\begin{noliste}{a)}
 \setlength{\itemsep}{2mm}
\item Montrer que $\delta (f)$ est une fonction polynomiale non nulle;
on
note $q$ son degré.

\item À l'aide de (\ref{s_{n}delta}), montrer que pour tout entier
naturel non
nul $n$, on a :\quad $s_{n} = \dfrac{1}{n^{q}}$

puis montrer qu'il existe un réel non nul $a$ tel que :\quad $\forall
x\in
\R,\quad \delta (f)(x) = ax^{q}$.

\item Pour chaque entier naturel non nul $p$, montrer, en appliquant ce
dernier résultat à la fonction polynomiale $B_{p}$ introduite dans la
partie
\textbf{II}, qu'on a :
\[
\forall x\in \R,\quad \delta (B_{p})(x) = px^{p-1}
\]

\item Montrer qu'il existe un réel $\lambda $ non nul et un entier $p$
non
nul tels que la fonction $\delta (f-\lambda B_{p})$ soit nulle. Établir
alors que la fonction $h = f-\lambda B_{p}$ est une fonction
$1$-périodique,
de classe $C^{\infty }$ et élément de $E$, et en préciser une suite
adaptée.
\end{noliste}
\end{noliste}

\section*{Partie IV Étude des fonctions indéfiniment dérivables et
$1$-périodiques de $E$}

\begin{noliste}{1.}
 \setlength{\itemsep}{4mm}
\item Dans cette question préliminaire, on suppose que $g$ est une
fonction
de $\R$ dans $\R$, continue et $1$-périodique, telle que,
pour tout réel $x$, $g(nx)$ tend vers $0$ lorsque l'entier $n$ tend
vers $ + \infty $.

\begin{noliste}{a)}
 \setlength{\itemsep}{2mm}
\item Montrer que, pour tout entier naturel $k$, on a l'égalité : $g(k)
= 0$.

\item Montrer que $g(\dfrac{1}{2}) = 0$. Plus généralement, montrer
que, pour
tout entier relatif $p$ et tout entier naturel non nul $q$, on a
l'égalité :
$g(\dfrac{p}{q}) = 0$.

\item En déduire que $g$ est la fonction nulle.
\end{noliste}

\noindent Dans toute la suite de cette partie, on suppose que $f$ est
une
fonction de classe $C^{\infty }$ et $1$-périodique, élément de $E$, de
suite
adaptée $s = (s_{n})_{n\in \N^{\times }}$.

\item 

\begin{noliste}{a)}
 \setlength{\itemsep}{2mm}
\item Montrer que l'application qui à tout réel $x$ associe $\dint{x}{x
+ 1}f(t)dt$ est constante.

\item Pour tout réel $x$, montrer que $\dfrac{s_{n}}{n}f(nx)$ tend vers
$\dint{0}{1}f(t)dt$ lorsque l'entier $n$ tend vers $ + \infty $.
\end{noliste}

\item On suppose dans cette question que $\dfrac{\left| s_{n}\right|
}{n}$ tend vers $ + \infty $ lorsque l'entier $n$ tend vers $ + \infty
$.

Montrer, à l'aide de la question 1), que $f$ est la fonction nulle.

\item Dans cette question, on suppose plus généralement qu'il existe un
entier naturel $k$ tel que $n^{k}\left| s_{n}\right| $ tend vers $ +
\infty $ lorsque l'entier $n$ tend vers $ + \infty $.

\begin{noliste}{a)}
 \setlength{\itemsep}{2mm}
\item Montrer, en considérant une dérivée d'ordre suffisant de $f$, que
$f$
est polynomiale.

\item Montrer que $f$ est constante.
\end{noliste}

\item À l'aide du résultat final de la partie \textbf{III}, montrer que
les
fonctions de classe $C^{\infty }$ appartenant à $E$ et qui ne sont pas
$1$-périodiques sont exactement les fonctions polynomiales du type
$\lambda B_{p}$, obtenues lorsque $p$ décrit $\N^{\times }$ et $\lambda
$ décrit $\R^{\times }$.
\end{noliste}

\label{fin}

\end{document}

