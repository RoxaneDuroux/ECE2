\documentclass[11pt]{article}%
\usepackage{geometry}%
\geometry{a4paper,
 lmargin = 2cm,rmargin = 2cm,tmargin = 2.5cm,bmargin = 2.5cm}

\input{../../macros.tex}

\pagestyle{fancy} %
\lhead{ECE2 \hfill Mathématiques\\
} %
\chead{\hrule} %
\rhead{} %
\lfoot{} %
\cfoot{} %
\rfoot{\thepage} %

\renewcommand{\headrulewidth}{0pt}% : Trace un trait de séparation
 % de largeur 0,4 point. Mettre 0pt
 % pour supprimer le trait.

\renewcommand{\footrulewidth}{0.4pt}% : Trace un trait de séparation
 % de largeur 0,4 point. Mettre 0pt
 % pour supprimer le trait.

\setlength{\headheight}{14pt}

\title{\bf \vspace{-2cm} ESCP 1996 - voie Générale} %
\author{} %
\date{} %
\begin{document}

\maketitle %
\vspace{-1.4cm}\hrule %
\thispagestyle{fancy}

\vspace*{.2cm}


% DEBUT DU DOC À MODIFIER : tout virer jusqu'au début de l'exo


\begin{center}
{\small CHAMBRE D\E\ COMMERCE ET D'INDUSTRIE DE PARIS}

\textbf{DIRECTION DE L'ENSEIGNEMENT}

Direction des Admissions et concours

\underline{\hspace*{3cm}}

{\Large ECOLE DES\ HAUTES\ ETUDES\ COMMERCIALES}

{\Large E.S.C.P.-E.A.P.}

{\Large ECOL\E\ SUPERIEUR\E\ D\E\ COMMERC\E\ D\E\ LYON}{\large }

CONCOURS D'ADMISSION\ SUR\ CLASSES\ PREPARATOIRES

\underline{\hspace*{3cm}}

\textbf{OPTION SCIENTIFIQUE}

{\Large MATHEMATIQUES I}

\textbf{Année 1996}

\underline{\hspace*{3cm}}
\end{center}

\begin{quotation}
\noindent \textsl{La présentation, la lisibilité, l'orthographe, la
qualité
de la rédaction, la clarté et la précision des raisonnements entreront
pour
une part importante dans l'appréciation des copies.}

\noindent \textsl{Les candidats sont invités à encadrer dans la mesure
du
possible les résultats de leurs calculs.}

\noindent \textsl{Ils ne doivent faire usage d'aucun document :
l'utilisation de toute calculatrice et de tout matériel électronique
est
interdite.}

\noindent \textsl{Seule l'utilisation d'une règle graduée est
autorisée.}

\noindent \textsl{\hrulefill }
\end{quotation}

\noindent Dans tout le problème, $p$ désigne un entier naturel
supérieur ou égal à deux. On note $\mathfrak{M}_{p}(\R)$ l'algèbre des
matrices
carrées à coefficients réels et $I_{p}$ la matrice identité. Pour tout
élément $M$ de $\mathfrak{M}_{p}(\R)$ et pour tout couple $(i,j)$
d'entiers compris entre 1 et $p$, on note $a_{i,j}(M)$ le coefficient
de $M$
situé sur la $i^{\text{ième}}$ ligne et la $j^{\text{ième}}$ colonne.\\
Une matrice M appartenant à $\mathfrak{M}_{p}(\R)$ est dite
\emph{stochastique} si elle satisfait au deux conditions suivantes :

\begin{noliste}{1.}
 \setlength{\itemsep}{4mm}
\item Pour tout couple $(i,j)$ d'entiers compris entre 1 et $p$,
$a_{i,j}(M)\geq 0$

\item Pour tout entier $i$ compris entre $1$ et $p$, $\Sum{j =
1}{p}a_{i,j}(M) = 1$.
\end{noliste}

\noindent On dit qu'une suite indexée par $n$, $(M_{n}) =
(M_{0},M_{1},\ldots,M_{n},\ldots )$ de matrices appartenant à
$\mathfrak{M}_{p}(\R)$
converge vers un élément $M$ de $\mathfrak{M}_{p}(\R)$ si, pour tout
couple $(i,j)$, la suite des coefficients $a_{i,j}(M_{n})$ converge
vers $a_{i,j}(M)$; on dit alors que $M$ est la limite de la suite
$(M_{n})$.\\
Étant donné une matrice $A$ appartenant à $\mathfrak{M}_{p}(\R)$,
pour tout entier $n\geq 0$, on note $C_{n}$ la matrice définie par la
relation : 
\begin{equation}
C_{n} = \dfrac{1}{n + 1}\left[ I_{p} + A + A^{2} + \ldots +
A^{n}\right] \label{C_{n}}
\end{equation}On dit enfin qu'une matrice $A$ de $\mathfrak{M}_{p}(\R)$
est $r$-périodique, où $r$ est un entier strictement positif, si $A^{r}
= I_{p}$.\\
L'objectif de ce problème est d'étudier quelques propriétés des
matrices
stochastiques et notamment, la convergence de la suite $(C_{n})$
lorsque $A$
est stochastique et $r$-périodique.

\section*{Partie I :\ Étude d'exemples }

\begin{noliste}{1.}
 \setlength{\itemsep}{4mm}
\item Soit $\alpha $ un nombre réel. Pour tout entier $n\geq 0$, on
pose 
\[
\gamma_{n} = \dfrac{1}{n + 1}\left[ 1 + \alpha + \alpha ^{2} + \ldots +
\alpha ^{n}\right] 
\]

\begin{noliste}{a)}
 \setlength{\itemsep}{2mm}
\item Calculer $\gamma_{n}$, en distinguant deux cas : $\alpha \neq 1$
et $\alpha = 1$. 

\item Étudier en fonction de $\alpha $, la convergence de la suite
$(\gamma
_{n})$ et, en cas de convergence, préciser sa limite. 
\end{noliste}

\item \textbf{Premier exemple d'étude de }$(C_{n})$

On prend $p = 3$ et 
\[
A = 
\begin{smatrix}
0 & 0 & 1 \\
1 & 0 & 0 \\
0 & 1 & 0
\end{smatrix}
\]

\begin{noliste}{a)}
 \setlength{\itemsep}{2mm}
\item Calculer $A^{2}$ et $A^{3}$. En déduire $A^{k}$ pour tout entier
$k$.
On distinguera trois cas selon que $k = 3h$, $k = 3h + 1$ et $k = 3h +
2$. 

\item Pour tout entier $q$, calculer $C_{3q}$, $C_{3q + 1}$ et $C_{3q +
2}$. En déduire que la suite $(C_{n})$ converge et préciser sa limite
$C$. 

\item Soient $(e_{1},e_{2},e_{3})$ la base canonique de $\R^{3}$ et $v$
l'endomorphisme de $\R^{3}$ canoniquement associé à $C$. Déterminer le
noyau $F$ de $v$, et prouver que son image $G$ est la droite
vectorielle $\Re$ de vecteur directeur $e = \dfrac{1}{3}\left[
e_{1} + e_{2} + e_{3}\right] $. Prouver que les sous-espaces vectoriels
$F$ et $G
$ sont supplémentaires et que $v$ est le projecteur de $\R^{3}$ sur $G$
parallèlement à $F$. 
\end{noliste}

\item \textbf{Deuxième exemple d'étude de }$(C_{n})$

On prend $p = 2$ et 
\[
A = 
\begin{smatrix}
\dfrac{1}{3} & \dfrac{2}{3} \\
\dfrac{1}{2} & \dfrac{1}{2}\end{smatrix}
\]
On note $w$ l'endomorphisme de $\R^{2}$ canoniquement associé à $A$. 

\begin{noliste}{a)}
 \setlength{\itemsep}{2mm}
\item Déterminer les valeurs propres de $w$ et une base $(f_{1},f_{2})$
de
vecteurs propres de $w$. 

\item Déterminer une matrice inversible $P$ telle que : 
\[
A = P
\begin{smatrix}
1 & 0 \\
0 & -\dfrac{1}{6}\end{smatrix}
P^{-1}
\]
En déduire une expression de $A^{k}$, pour tout entier $k\geq 0$. 

\item Déterminer deux matrices $U$ et $V$ appartenant à
$\mathfrak{M}_{2}(\R)$, telles que, pour tout $k\geq 0$ : 
\[
A^{k} = U + \left( -\dfrac{1}{6}\right) ^{k}V
\]

\item Pour tout entier $n\geq 0$, exprimer $C_{n}$ en fonction de $n$,
$U$ et $V$ et déterminer la limite $C$ de la suite $(C_{n})$. 

\item Prouver que l'endomorphisme $v$ de $\R^{2}$ canoniquement
associé à C est un projecteur dont on précisera le noyau $F$ et l'image
$G$. 
\end{noliste}
\end{noliste}

\section*{Partie II : Étude de $C_{n}$ lorsque $A$ est $r$-périodique }

On désigne par $r$ un entier strictement positif. 

\begin{noliste}{1.}
 \setlength{\itemsep}{4mm}
\item Soit $(\alpha_{k})$ une suite $r$-périodique de nombres réels,
c'est à
dire telle que, pour tout entier naturel $k\geq 0$, $\alpha
_{k + r} = \alpha_{k}$. On pose : 
\[
\gamma = \dfrac{1}{r}\left[ \ \alpha_{0} + \alpha_{1} + \ldots +
\alpha_{r-1}\right] 
\]
Pour tout entier $n\geq 0$, on pose : 
\begin{equation}
\gamma_{n} = \dfrac{1}{n + 1}\left[ \ \alpha_{0} + \alpha_{1} + \ldots
+ \alpha_{n}\right] \label{gamma_{n}}
\end{equation}

\begin{noliste}{a)}
 \setlength{\itemsep}{2mm}
\item Prouver que pour tout entier $k\geq 0$, 
\[
\gamma = \dfrac{1}{r}\left[ \ \alpha_{k} + \alpha_{k + 1} + \ldots +
\alpha_{k + r-1}\right] 
\]

\item Montrer que la suite de terme général 
\[
\beta_{n} = (n + 1)\gamma_{n}-(n + 1)\gamma 
\]
est $r$-périodique. En déduire que $(\beta_{n})$ est bornée. 

\item Établir que $(\gamma_{n})$ converge et préciser sa limite. 
\end{noliste}

\item Soit $A$ une matrice $r$-périodique appartenant à $M_{p}(R)$. 

\begin{noliste}{a)}
 \setlength{\itemsep}{2mm}
\item Montrer que, pour tout couple $(i,j)$ d'entiers compris entre $1$
et $p
$, la suite de terme général $\alpha_{k} = a_{i,j}(A^{k})$ est
$r$-périodique. En déduire que la suite $(C_{n})$ converge vers : 
\[
C = \dfrac{1}{r}\left[ I_{p} + A + \ldots + A^{r-1}\right] 
\]

\item Soient $(e_{1},e_{2},\ldots,e_{p})$ la base canonique de
$\R^{p}$, $u$ et $v$ les endomorphismes de $\R^{p}$ canoniquement
associés aux matrices $A$ et $C$. Prouver que $u^{r} = I$, où $I$ est
l'endomorphisme identique de $\R^{p}$. Montrer que $v\circ u = u\circ v
$ et que $u\circ v = v$ 

\item Soit $x$ un élément de $\R^{p}$. Prouver que $u(x) = x$ si et
seulement si $v(x) = x$, puis que $x$ appartient à $\text{Im}(v)$ si et
seulement si $u(x) = x$. En déduire que $\text{Im}(v) =
\text{Ker}(u-I)$. 

\item Montrer que $v$ est le projecteur sur $G = \text{Im}(v)$
parallèlement à 
$F = \text{Ker}(v)$. 

\item Établir enfin que $\text{Ker}(v) = \text{Im}(u-I)$ : on pourra
d'abord
prouver que $\text{Im}(u-I)\subset \text{Ker}(v)$ 
\end{noliste}

\item 

\begin{noliste}{a)}
 \setlength{\itemsep}{2mm}
\item Soit $(\alpha_{k})$ une suite de nombres réels $r$-périodique
\emph{à
partir d'un certain rang} positif $m$, c'est à dire telle que pour tout
$k\geq m$, $\alpha_{k + r} = \alpha_{k}$. On définit $(\gamma_{n})$ par
la relation (\ref{gamma_{n}}). Prouver que $\gamma_{n}$ admet une
limite que
l'on précisera. Pour cela, on pourra considérer la suite $\alpha
_{k}{\prime } = \alpha_{k + m}$, observer que $(\alpha_{k}{\prime })$
est $r$-périodique, et prouver que, $\gamma_{n}{\prime }$ étant
associée à $(\alpha_{k}{\prime })$ par la relation (\ref{gamma_{n}}),
$\gamma
_{n}{\prime }-\gamma_{n}$ tend vers $0$ lorsque $n$ tend vers $ +
\infty $. 

\item Soit $A$ une matrice de $\M_{p}(R)$ $r$-périodique \emph{à
partir d'un certain rang} positif $m$, c'est à dire telle que pour tout
$k\geq m$, $A^{k + r} = A^{k}$. Prouver que la suite $(C_{n})$ admet
une
limite que l'on précisera. 
\end{noliste}
\end{noliste}

\section*{Partie III : Étude de matrices stochastiques }

On note $S_{p}$ l'ensemble des matrices stochastiques de
$\mathfrak{M}_{p}(R)
$ et $D_{p}$ l'ensemble des matrices \emph{déterministes}, c'est à dire
stochastiques et dont tous les coefficients sont égaux à $0$ ou $1$.
Enfin,
on note $\Delta_{p}$ l'ensemble des matrices déterministes et
inversibles. 

\begin{noliste}{1.}
 \setlength{\itemsep}{4mm}
\item \textbf{Matrices stochastiques} 

\begin{noliste}{a)}
 \setlength{\itemsep}{2mm}
\item Prouver que, pour tout couple $(\lambda,\mu )$ de nombres réels
tels
que $\lambda \geq 0$, $\mu \geq 0$ et $\lambda + \mu = 1$, et pour
tout couple $(M,N)$ d'éléments de $S_{p}$, $\lambda M + \mu N$
appartient
encore à $S_{p}$. 

\item Prouver que le produit $MN$ de deux éléments $M$ et $N$ de
$S_{p}$
appartient à $S_{p}$. 

\item Soit $A$ un élément de $S_{p}$. Prouver que, pour tout entier
$n\geq 0$, $C_{n}$ (définie par (\ref{C_{n}})) appartient à $S_{p}$.
Que
peut-on en déduire pour la limite $C$ de $(C_{n})$, lorsqu'elle existe
? 
\end{noliste}

\item \textbf{Matrices déterministes} 

\begin{noliste}{a)}
 \setlength{\itemsep}{2mm}
\item Montrer qu'une matrice $M$ est déterministe si et seulement si
tous
ses coefficients sont égaux à $0$ ou $1$ et si chaque ligne de $M$
contient
exactement un coefficient égal à $1$. 

\item En déduire que $D_{p}$ est un ensemble fini et préciser le nombre
de
ses éléments. 

\item Montrer que le produit $MN$ de deux éléments $M$ et $N$ de
$D_{p}$
appartient à $D_{p}$. 

\item Soit $A$ une matrice déterministe. Prouver qu'il existe un entier
$r\geq 1$ et un entier $m\geq 0$ tels que $A^{m + r} = A^{m}$. En
déduire que, dans ces conditions, $A$ est $r$- périodique \emph{à
partir de ce
rang} $m$ et que, si de plus $A$ est inversible, $A$ est
$r$-périodique. 

\item Soit $A$ une matrice déterministe inversible. Prouver que
$A^{-1}$
l'est aussi. \label{determinisme_{i}nverse} 
\end{noliste}

\item \textbf{Étude de la suite }$C_{n}$\textbf{\ associée à une
matrice }$A$\textbf{\ déterministe} 

\begin{noliste}{a)}
 \setlength{\itemsep}{2mm}
\item En utilisant les résultats de la partie II, établir le résultat
suivant : 

Soit $A$ une matrice déterministe inversible, alors $(C_{n})$ converge
vers
une matrice stochastique $C$ telle que $C^{2} = C$. 

\item Étendre ce résultat au cas où $A$ est déterministe \textbf{non
inversible}. 
\end{noliste}

\item \textbf{Matrices stochastiques inversible}\\
Soient $X$ et $Y$ des éléments de $S_{p}$ tels que $XY = I_{p}$. On se
propose
de montrer que $X$ et $Y$ sont déterministes inversibles. 

\begin{noliste}{a)}
 \setlength{\itemsep}{2mm}
\item Prouver que $Y$ est une matrice inversible, et que $X$ l'est
aussi. 

\item On pose $X = (\alpha_{i,j})$, $Y = (\beta_{i,j})$ et, pour tout
$j$
compris entre 1 et $p$, 
\[
\mu_{j} = \text{max}\{\beta_{1,j},\beta_{2,j},\ldots,\beta_{p,j}\}
\]
Prouver que $\mu_{j} = 1$. Pour cela, on pourra calculer le coefficient
$a_{j,j}(XY)$. 

\item Montrer que $\Sum{i = 1}{p}\Sum{j = 1}{p}\beta
_{i,j} = \Sum{j = 1}{p}\mu_{j}$. En déduire que tous les coefficients
de $Y$ sont égaux à $0$ ou $1$. 

\item Prouver que $Y$ et $X$ appartiennent à $\Delta_{p}$. 

\item Plus généralement, soient $U$ et $V$ deux matrices de $S_{p}$
telles
que le produit $UV$ appartienne à $\Delta_{p}$. Prouver que $U$ et $V$
appartiennent à $\Delta_{p}$. (On pourra utiliser le résultat de la
question III.\ref{determinisme_{i}nverse}) 
\end{noliste}
\end{noliste}

\label{fin}

\end{document}

