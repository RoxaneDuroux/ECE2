\documentclass[11pt]{article}%
\usepackage{geometry}%
\geometry{a4paper,
 lmargin = 2cm,rmargin = 2cm,tmargin = 2.5cm,bmargin = 2.5cm}

\input{../../macros.tex}

\pagestyle{fancy} %
\lhead{ECE2 \hfill Mathématiques\\
} %
\chead{\hrule} %
\rhead{} %
\lfoot{} %
\cfoot{} %
\rfoot{\thepage} %

\renewcommand{\headrulewidth}{0pt}% : Trace un trait de séparation
 % de largeur 0,4 point. Mettre 0pt
 % pour supprimer le trait.

\renewcommand{\footrulewidth}{0.4pt}% : Trace un trait de séparation
 % de largeur 0,4 point. Mettre 0pt
 % pour supprimer le trait.

\setlength{\headheight}{14pt}

\title{\bf \vspace{-2cm} ESCP 1989 - voie Générale} %
\author{} %
\date{} %
\begin{document}

\maketitle %
\vspace{-1.4cm}\hrule %
\thispagestyle{fancy}

\vspace*{.2cm}


% DEBUT DU DOC À MODIFIER : tout virer jusqu'au début de l'exo


\begin{center}
{\small CHAMBRE D\E\ COMMERCE ET D'INDUSTRIE DE PARIS}

\textbf{DIRECTION DE L'ENSEIGNEMENT}

Direction des Admissions et concours

\underline{\hspace*{3cm}}

{\Large ECOLE DES\ HAUTES\ ETUDES\ COMMERCIALES}

{\Large E.S.C.P.-E.A.P.}

{\Large ECOL\E\ SUPERIEUR\E\ D\E\ COMMERC\E\ D\E\ LYON}{\large }

CONCOURS D'ADMISSION\ SUR\ CLASSES\ PREPARATOIRES

\underline{\hspace*{3cm}}

\textbf{OPTION GENERALE}

{\Large MATHEMATIQUES I}

\textbf{Année 1989}

\underline{\hspace*{3cm}}
\end{center}

\begin{quotation}
\noindent \textsl{La présentation, la lisibilité, l'orthographe, la
qualité
de la rédaction, la clarté et la précision des raisonnements entreront
pour
une part importante dans l'appréciation des copies.}

\noindent \textsl{Les candidats sont invités à encadrer dans la mesure
du
possible les résultats de leurs calculs.}

\noindent \textsl{Ils ne doivent faire usage d'aucun document :
l'utilisation de toute calculatrice et de tout matériel électronique
est
interdite.}

\noindent \textsl{Seule l'utilisation d'une règle graduée est
autorisée.}

\noindent \textsl{\hrulefill }
\end{quotation}

\noindent Le but du problème est l'étude de certaines propriétés de la
fonction de répartition de la loi normale centrée réduite $(0;1)$, ce
qui
fait l'objet des trois premières parties. Une application probabiliste
est
proposée en quatrième partie.

\section*{Partie I}

On étudie dans cette partie une méthode de calcul de l'intégrale
(convergente) : 
\[
I = \dint{-\infty }{+ \infty }\exp \left( -\dfrac{u^{2}}{2}\right) du
\]
À cet effet, on considère les fonctions $f$ et $g$ définies sur $R$ par
les
relations : 
\[
f(x) = \dint{0}{1}\dfrac{\exp \left( -x(1 + t^{2})\right) }{1 +
t^{2}}dt,\quad g(x) = \dint{0}{1}\exp \left( -x(1 + t^{2})\right\ dt.
\]
(On ne cherchera pas à calculer ces deux intégrales).

\begin{noliste}{1.}
 \setlength{\itemsep}{4mm}
\item 

\begin{noliste}{a)}
 \setlength{\itemsep}{2mm}
\item Calculer $f(0)$.

\item Pour tout nombre réel positif $x$, établir l'encadrement suivant
: 
\[
\dfrac{\pi }{4}\exp (-2x)\leq f(x)\leq \dfrac{\pi }{4}\exp (-x)
\]

\item Établir un encadrement analogue pour $x$ négatif.

\item En déduire les limites de $f(x)$ lorsque $x$ tend vers $ + \infty
$ et
quand $x$ tend vers $-\infty $.
\end{noliste}

\item On se propose de montrer que la fonction $f$ est dérivable et de
calculer sa dérivée.

\begin{noliste}{a)}
 \setlength{\itemsep}{2mm}
\item Soit $a$ un nombre réel positif. En utilisant l'inégalité de
Taylor-Lagrange, montrer que, pour tout nombre réel $h$ appartenant à
$[-1;1] $ : 
\[
\left| \exp (-ah)-1 + ah\right| \leq \dfrac{a^{2}h^{2}}{2}\exp
(a).
\]

\item En déduire que, pour tout nombre réel $x$ et pour tout nombre
réel $h$
appartenant à $[-1;1]$ : 
\[
\left| f(x + h)-f(x) + hg(x)\right| \leq \dfrac{2h^{2}}{3}\exp
\left( 2|1-x|\right)
\]

\item Montrer que $f$ est dérivable et exprimer sa dérivée à l'aide de
$g$.
\end{noliste}

\item On considère la fonction numérique $\varphi $ définie sur $R$ par
la
relation : 
\[
\varphi (x) = 2f\left( \dfrac{x^{2}}{2}\right) + \left( \dint{0}{x}\exp
\left( -\dfrac{u^{2}}{2}\right) \text{d}u\right) ^{2}
\]

\begin{noliste}{a)}
 \setlength{\itemsep}{2mm}
\item Prouver que $\varphi ^{\prime }$ est nulle et déterminer l'unique
valeur prise par $\varphi $.

\item En déduire la valeur de l'intégrale $I$.
\end{noliste}
\end{noliste}

\section*{Partie II}

On étudie dans cette partie un algorithme de calcul des valeurs prises
par
la fonction de répartition $F$ de la loi normale centrée réduite. On
rappelle que : 
\[
F(x) = \dfrac{1}{\sqrt{2\pi }}\dint{-\infty }{x}\exp \left(
-\dfrac{u^{2}}{2}\right) \text{d}u
\]

\begin{noliste}{1.}
 \setlength{\itemsep}{4mm}
\item Soit $x$ un nombre réel.

\begin{noliste}{a)}
 \setlength{\itemsep}{2mm}
\item À l'aide d'une intégration par parties, prouver que : 
\[
\sqrt{2\pi }\left( F(x)-\dfrac{1}{2}\right) = x\exp \left(
-\dfrac{x^{2}}{2}\right) + \dint{0}{x}u^{2}\exp \left(
-\dfrac{u^{2}}{2}\right) \text{d}u
\]

\item Montrer par récurrence que, pour tout nombre entier naturel $n$ :

\[
\sqrt{2\pi }\left( F(x)-\dfrac{1}{2}\right) = x\exp \left(
-\dfrac{x^{2}}{2}\right) \left( 1 + \dfrac{x^{2}}{1\times 3} +
\dfrac{x^{4}}{1\times 3\times 5} + \ldots + \dfrac{x^{2n}}{1\times
3\times 5\times \ldots \times (2n + 1)}\right)
 + R_{n}(x)
\]
avec : $R_{n}(x) = \dfrac{1}{1\times 3\times 5\times \ldots \times (2n
+ 1)}\dint{0}{x}u^{2n + 2}\exp \left( -\dfrac{u^{2}}{2}\right) du$
\end{noliste}

\item Montrer que, pour tout nombre entier naturel $n$ et pour tout
nombre
réel $x$ appartenant à $[-2 ; 2]$ : 
\[
\left|R_{n}(x)\right|\leq
|x|\dfrac{x^{2}}{3}\dfrac{x^{2}}{5}\dfrac{x^{2}}{7}\ldots
\dfrac{x^{2}}{2n + 3}
\]
Trouver une valeur de $n$ telle que, pour tout nombre réel $x$
appartenant
à $[-2 ; 2]$ : 
\[
\left|R_{n}(x)\right|\leq 10^{-6}
\]

\item On considère l'algorithme suivant, dans lequel $s$ et $x$
représentent des variables de type real, $n$ et $k$ des variables de
type
integer, les valeurs de $x$ et de $n$ étant données par ailleurs :

$s : = 1$;

for $k : = n$ downto $1$ do

$s : = 1 + x\ast x\ast s/(2\ast k + 1);$

write($s$);

\begin{noliste}{a)}
 \setlength{\itemsep}{2mm}
\item Indiquer en fonction de $x$ et de $n$ l'expression finale de $s$.

\item En déduire, à l'aide des résultats obtenus dans cette partie, des
valeurs approchées à $10^{-6}$ près de : 
\[
F(0,5)\quad F(0,783)\quad F(0,784)\quad F(1)\quad F(1,5)\quad F(2)
\]
(on donnera toutes les décimales fournies par la calculatrice).
\end{noliste}

On considère l'algorithme suivant, dans lequel $S$ et $x$ représentent
des
variables à valeurs réelles, $n$ et $k$ des variables à valeurs
entières,
les valeurs de $x$ et de $n$ étant données par ailleurs (l'instruction
$A\leftarrow B$ signifiant que la valeur de la variable $B$ est
affectée à la
variable $A$) :

$S\leftarrow 1$;

Pour $k$ décroissant de $n$ à $1$ faire $S\leftarrow 1 +
\dfrac{x^{2}}{2k + 1}S$;

écrire $S$;

\item 

\begin{noliste}{a)}
 \setlength{\itemsep}{2mm}
\item Indiquer en fonction de $x$ et de $n$ l'expression finale de $S$.

\item En déduire, à l'aide des résultats obtenus dans cette partie, des
valeurs approchées à $10^{-6}$ près de : 
\[
F(0,5)\quad \quad F(0,783)\quad \quad F(0,784)\quad \quad F(1)\quad
\quad
F(1,5)\quad \quad F(2)
\]
(on donnera toutes les décimales fournies par la calculatrice).
\end{noliste}
\end{noliste}

\section*{Partie III}

On étudie dans cette partie le comportement asymptotique de la fonction
$F$
et de sa réciproque.

\begin{noliste}{1.}
 \setlength{\itemsep}{4mm}
\item Montrer que la fonction $F$ admet une fonction réciproque $G$,
laquelle est définie sur $]0;1[$.

\item Représenter sur une même figure les courbes représentatives de $F
$ et de $G$ (unité graphique $4$ cm).

On précisera notamment les valeurs de $F(0)$ et de $G(0,5)$, les
limites
de $F$ en $-\infty$ et en $ + \infty$, les limites de $G$ en $0$ et en
$1$.

\item 

\begin{noliste}{a)}
 \setlength{\itemsep}{2mm}
\item Exprimer $F(-x)$ en fonction de $F(x)$. En déduire l'expression
de $G(1-y)$ en fonction de $G(y)$ pour $0<y<1$.

\item Pour tout nombre réel strictement négatif $x$, établir
l'encadrement
suivant : 
\[
-\dfrac{\exp \left( -\dfrac{x^{2}}{2}\right) }{x\sqrt{2\pi }}\left(
1-\dfrac{1}{x^{2}}\right) \leq F(x)\leq -\dfrac{\exp \left(
-\dfrac{x^{2}}{2}\right) }{x\sqrt{2\pi }}
\]
En déduire un équivalent de $F(x)$ quand $x$ tend vers $-\infty $, puis
de $1-F(x)$ quand $x$ tend vers $ + \infty $.

\item On pose $x = G(y)$, où $x<0$ et $0<y<\dfrac{1}{2}$. Montrer que :

\[
-\dfrac{G^{2}(y)}{2}-\ln \left| G(y)\right| -\dfrac{\ln (2\pi )}{2} +
\ln \left( 1-\dfrac{1}{G^{2}(y)}\right) \leq \ln (y)\leq
-\dfrac{G^{2}(y)}{2}-\ln \left| G(y)\right| -\dfrac{\ln (2\pi )}{2}
\]
En déduire un équivalent de $G(y)$ quand $y$ tend vers $0$, puis quand
$y$
tend vers $1$.
\end{noliste}
\end{noliste}

\section*{Partie IV}

À l'issue d'un scrutin uninominal permettant à plusieurs centaines de
milliers d'électeurs de départager deux candidats $A$ et $B$
d'importances
comparables, on se propose, avant le dépouillement, de procéder à une
estimation de la proportion $p$ des voix obtenues par le candidat
$A$.\\
À cet effet, on répète $n$ fois ($n\geq 1$) l'expérience suivante : on
retire "au hasard" un bulletin des urnes ; on note s'il est ou non en
faveur
de $A$ et on le remet dans les urnes. On note $X_{n}$ la variable
aléatoire
indiquant le nombre des suffrages favorables à $A$ parmi les $n$
bulletins dépouillés ; le quotient : 
\[
Y_{n} = \dfrac{X_{n}}{n}
\]
indique donc la proportion des suffrages favorables à $A$ parmi ces $n$
bulletins. On pose enfin : 
\[
u_{n} = P\left(\Ev{ \left| Y_{n}-p\right| >0,01}\right).
\]
Soit $\epsilon $ un nombre réel appartenant à $]0;1[$. L'objectif est
de déterminer le nombre $n$ de bulletins qu'il suffit de dépouiller
ainsi pour
que $u_{n}\leq \epsilon $, c'est à dire pour connaitre $p$ à $0,01$
près avec un risque d'erreur inférieur à $\epsilon $.

\begin{noliste}{1.}
 \setlength{\itemsep}{4mm}
\item On étudie dans cette question les lois de $X_{n}$ et de $Y_{n}$.

\begin{noliste}{a)}
 \setlength{\itemsep}{2mm}
\item Déterminer la loi de $X_{n}$. Calculer les espérances et les
variances de $X_{n}$ et de $Y_{n}$.

\item Montrer que : 
\[
\V(X_{n} )\leq \dfrac{n}{4}
\]
\end{noliste}

\item \textit{Première majoration de $u_{n}$}

\begin{noliste}{a)}
 \setlength{\itemsep}{2mm}
\item En appliquant à $X_{n}$ l'inégalité de Bienaymé-Tchebychev et en
utilisant le résultat de la question 1.b., donner un majorant $M_{n}$
de $u_{n} $ ne dépendant que de $n$.

\item Comment suffit-il de choisir $n$ pour que $M_{n} \leq \epsilon$ ?

\hspace{-\leftmargin}Examiner les cas où $\epsilon = 0,1$ et $\epsilon
= 
0,05$.
\end{noliste}

\item \textit{Seconde majoration de $u_{n}$}

\begin{noliste}{a)}
 \setlength{\itemsep}{2mm}
\item En approchant $X_{n}$ par la loi normale (on justifiera la mise
en \oe\ uvre d'une telle approximation), exprimer $u_{n}$ en fonction
de $n$ et de $p$
à l'aide de $F$. En utilisant le résultat de la question 1.b., donner
un
majorant $m_{n}$ de $u_{n}$ ne dépendant que de $n$.

\item En déduire que $m_{n} \leq \epsilon$ dès que $n \geq 2500
G^{2}\left(\dfrac{\epsilon}{2}\right)$.

Examiner les cas où $\epsilon = 0,1$ et $\epsilon = 0,05$.

(On rappelle les valeurs approchées suivantes : $F(1,96) \approx 0,975$
et 
$F(1,64) \approx 0,950$)

\item Soit $n(\epsilon )$ le plus petit des nombres entiers naturels
$n$
tels que $m_{n}\leq \epsilon $. Déterminer un équivalent de $n(\epsilon
)$ lorsque $\epsilon $ tend vers $0$.
\end{noliste}
\end{noliste}

\label{fin}

\end{document}

