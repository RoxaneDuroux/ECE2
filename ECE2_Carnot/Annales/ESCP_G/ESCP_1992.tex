\documentclass[11pt]{article}%
\usepackage{geometry}%
\geometry{a4paper,
 lmargin = 2cm,rmargin = 2cm,tmargin = 2.5cm,bmargin = 2.5cm}

\input{../../macros.tex}

\pagestyle{fancy} %
\lhead{ECE2 \hfill Mathématiques\\
} %
\chead{\hrule} %
\rhead{} %
\lfoot{} %
\cfoot{} %
\rfoot{\thepage} %

\renewcommand{\headrulewidth}{0pt}% : Trace un trait de séparation
 % de largeur 0,4 point. Mettre 0pt
 % pour supprimer le trait.

\renewcommand{\footrulewidth}{0.4pt}% : Trace un trait de séparation
 % de largeur 0,4 point. Mettre 0pt
 % pour supprimer le trait.

\setlength{\headheight}{14pt}

\title{\bf \vspace{-2cm} ESCP 1992 - voie Générale} %
\author{} %
\date{} %
\begin{document}

\maketitle %
\vspace{-1.4cm}\hrule %
\thispagestyle{fancy}

\vspace*{.2cm}


% DEBUT DU DOC À MODIFIER : tout virer jusqu'au début de l'exo


\begin{center}
{\small CHAMBRE D\E\ COMMERCE ET D'INDUSTRIE DE PARIS}

\textbf{DIRECTION DE L'ENSEIGNEMENT}

Direction des Admissions et concours

\underline{\hspace*{3cm}}

{\Large ECOLE DES\ HAUTES\ ETUDES\ COMMERCIALES}

{\Large E.S.C.P.-E.A.P.}

{\Large ECOL\E\ SUPERIEUR\E\ D\E\ COMMERC\E\ D\E\ LYON}{\large }

CONCOURS D'ADMISSION\ SUR\ CLASSES\ PREPARATOIRES

\underline{\hspace*{3cm}}

\textbf{OPTION GENERALE}

{\Large MATHEMATIQUES I}

\textbf{Année 1992}

\underline{\hspace*{3cm}}
\end{center}

\begin{quotation}
\noindent \textsl{La présentation, la lisibilité, l'orthographe, la
qualité
de la rédaction, la clarté et la précision des raisonnements entreront
pour
une part importante dans l'appréciation des copies.}

\noindent \textsl{Les candidats sont invités à encadrer dans la mesure
du
possible les résultats de leurs calculs.}

\noindent \textsl{Ils ne doivent faire usage d'aucun document :
l'utilisation de toute calculatrice et de tout matériel électronique
est
interdite.}

\noindent \textsl{Seule l'utilisation d'une règle graduée est
autorisée.}

\noindent \textsl{\hrulefill }
\end{quotation}

{Objectif du problème :} on note $E$ l'espace vectoriel sur $\R$ des
fonctions à valeurs réelles, continues sur un intervalle $[a,b]$ où
$a<b$.
On rappelle qu'une fonction affine sur un intervalle $J$ est une
fonction $f$
définie sur $J$ par une relation de la forme $f(t) = \alpha t + \beta
$, où $\alpha $ et $\beta $ sont des nombres réels.\\
Étant donné un nombre entier $n\geq 2$ et une subdivision de $[a,b]$,
c'est-à-dire une suite strictement croissante $\sigma
_{n} = (x_{0},x_{1},\ldots,x_{n})$ de $n + 1$ éléments de $[a,b]$, avec
$x_{0} = a
$ et $x_{n} = b$, on note $\E(\sigma_{n})$ l'ensemble des fonctions $f$
définies sur $[a,b]$ telles que chacune des restrictions de $f$ aux
intervalles $[x_{k},x_{k + 1}]$, où $k$ prend les valeurs $0,1,\ldots
n-1$,
soit une fonction affine.\\
L'objet du problème est d'étudier l'approximation d'une fonction de
classe $\mathcal{C}{2}$ sur $[a,b]$ par des éléments de
$\E(\sigma_{n})$

\section*{Première partie : {\protect\Large Étude de
$\E(\protect\sigma_{2})$}}

Dans cette partie, on prend $n = 2$, ainsi $x_{0} = a$ et $x_{2} = b$.
On pose $c = x_{1}$.

\begin{noliste}{1.}
 \setlength{\itemsep}{4mm}
\item Montrer que $\E(\sigma_{2})$ est un sous-espace vectoriel de $E$.

\item Soit $f$ un élément de $\E(\sigma_{2})$. Calculer $f(t)$ en
fonction
de $f(a)$, $f(b)$ et $f(c)$ pour $a\leq t\leq c$, puis pour $c\leq
t\leq b$.

\item Soit $\Phi $ l'application de $\E(\sigma_{2})$ dans $\R^{3}$
définie par $\Phi (f) = (f(a),f(b),f(c))$. Montrer que $\Phi $ est un
isomorphisme d'espaces vectoriels. En déduire la dimension de
$\E(\sigma
_{2}) $.

\item On définit trois éléments $f_{0}$, $f_{1}$ et $f_{2}$ de
$\E(\sigma
_{2})$ par les conditions : 
\[
\Phi (f_{0}) = (1,0,0)\qquad \Phi (f_{1}) = (0,1,0)\qquad \Phi (f_{2})
= (0,0,1)
\]

\begin{noliste}{a)}
 \setlength{\itemsep}{2mm}
\item Représenter graphiquement les fonctions $f_{0}$, $f_{1}$ et
$f_{2}$.

\item Montrer que $(f_{0},f_{1},f_{2})$ est une base de
$\E(\sigma_{2})$.
\end{noliste}

\item On considère les fonctions $g_{0}$, $g_{1}$ et $g_{2}$ définies
sur $[a,b]$ par les relations : 
\[
g_{0}(t) = \left| t-a\right| \qquad g_{1}(t) = \left| t-c\right|
\qquad g_{2}(t) = \left| t-b\right|
\]

\begin{noliste}{a)}
 \setlength{\itemsep}{2mm}
\item Montrer que $(g_{0},g_{1},g_{2})$ est une base de
$\E(\sigma_{2})$.
Calculer les coordonnées de chacune des fonctions $g_{0}$, $g_{1}$ et
$g_{2}$
dans la base $(f_{0},f_{1},f_{2})$.

\item Soient $\alpha $, $\beta $ et $\gamma $ des nombres réels non
nuls. On
pose : 
\[
A = 
\begin{smatrix}
0 & \alpha & \beta \\
\alpha & 0 & \gamma \\
\beta & \gamma & 0
\end{smatrix}
\]
Montrer que la matrice $A$ est inversible et calculer son inverse.

\item En déduire les coordonnées de chacune des fonctions $f_{0}$,
$f_{1}$
et $f_{2}$ dans la base $(g_{0},g_{1},g_{2})$.
\end{noliste}
\end{noliste}

\section*{Deuxième partie : {\protect\Large Étude de
$\E(\protect\sigma_{n})$}}

Dans cette partie on prend $a = 0$, $b = 1$ et $x_{k} = \dfrac{k}{n}$
pour tout
nombre entier naturel $k\leq n$. L'espace vectoriel $\E(\sigma_{n})$
sera noté plus simplement $E_{n}$.

\begin{noliste}{1.}
 \setlength{\itemsep}{4mm}
\item Prouver que $E_{n}$ est un sous-espace vectoriel de l'espace
vectoriel 
$E$ des fonctions continues sur $[0,1]$.

\item Soit $\Phi $ l'application de $E_{n}$ dans $\R^{n + 1}$ définie
par $\Phi (f) = (f(x_{0}),f(x_{1}),\ldots,f(x_{n}))$. Montrer que $\Phi
$ est
un isomorphisme d'espaces vectoriels. En déduire la dimension de
$E_{n}$.

\item On définit une famille $(f_{0},f_{1},\ldots,f_{n})$ d'éléments de
$E_{n}$ par les conditions : 
\[
\Phi (f_{k}) = (0,\ldots,0,1,0,\ldots,0)
\]
le nombre $1$ étant à la $(k + 1)^{i\grave{e}me}$ place. Autrement dit
$f_{k}(x_{k}) = 1$ et $f_{k}(x_{j}) = 0$ pour $j\neq k$.

\begin{noliste}{a)}
 \setlength{\itemsep}{2mm}
\item Montrer que $(f_{0},f_{1},\ldots,f_{n})$ est une base de $E_{n}$.

\item Montrer que, pour tout élément $g$ de $E_{n}$, on a l'égalité : 
\[
g = \Sum{k = 0}{n}g(x_{k})f_{k}
\]
\end{noliste}

\item Pour tout nombre entier naturel $k\leq n$, on considère la
fonction $g_{k}$ définie sur $[0,1]$ par la relation : 
\[
g_{k}(t) = \left| t-x_{k}\right| 
\]

\begin{noliste}{a)}
 \setlength{\itemsep}{2mm}
\item Montrer que les fonctions $g_{k}$ appartiennent à $E_{n}$.

\item Soit $(\lambda_{0},\lambda_{1},\ldots,\lambda_{n})$ un élément de
$R^{n + 1}$. On pose : 
\[
g = \Sum{k = 0}{n}\lambda_{k}g_{k}
\]
Soit $j$ un nombre entier compris entre $1$ et $n-1$. Si on suppose que
$\lambda_{j}\neq 0$, montrer que la fonction $g$ n'est pas dérivable en
$x_{j}$.
\end{noliste}

\item 

\begin{noliste}{a)}
 \setlength{\itemsep}{2mm}
\item En déduire que la famille $(g_{0},g_{1},\ldots,g_{n})$ est libre.
Est-ce une base de $E_{n}$ ?

\item En déduire aussi que $f_{0}$, $f_{1}$,\dots, $f_{n}$ peuvent
s'écrire
sous la forme : 
\[
\left\{
\begin{array}{cl}
f_{0} = \lambda_{0}g_{0} + \mu_{0}g_{1} + \nu_{0}g_{n} & \\
f_{k} = \lambda_{k}g_{k-1} + \mu_{k}g_{k} + \nu_{k}g_{k + 1} & \text{
si }1\leq k\leq n-1 \\
f_{n} = \lambda_{n}g_{0} + \mu_{n}g_{n-1} + \nu_{n}g_{n} & 
\end{array}
\right.
\]
Pour les entiers $k$ compris entre $1$ et $n-1$, on commencera par
montrer
que $f_{k}$ est de la forme 
\[
f_{k} = \alpha_{k}g_{0} + \lambda_{k}g_{k-1} + \mu_{k}g_{k} + \nu
_{k}g_{k + 1} + \beta_{k}g_{n}
\]
puis que $\alpha_{k} = \beta_{k} = 0$ en considérant les restrictions
de $f_{k}
$ aux deux intervalles $[x_{0},x_{1}]$ et $[x_{n-1},x_{n}]$.

\item Calculer, à l'aide des résultats de la première partie, les
nombres $\lambda_{k}$, $\mu_{k}$ et $\nu_{k}$.
\end{noliste}

\item \textbf{Application}. On considère la matrice carrée $A_{n}$
d'ordre $n + 1$ dont l'élément de la $i^{i\grave{e}me}$ ligne et de la
$j^{i\grave{e}me}
$ colonne est $\left| i-j\right| $. Ainsi : 
\[
A_{n} = 
\begin{smatrix}
0 & 1 & 2 & \cdots & \cdots & n-1 & n \\
1 & 0 & 1 & \cdots & \cdots & n-2 & n-1 \\
2 & 1 & 0 & \cdots & \cdots & n-3 & n-2 \\
\vdots & \vdots & \vdots & \vdots & \vdots & \vdots & \vdots \\
n-2 & n-3 & \cdots & \cdots & 0 & 1 & 2 \\
n-1 & n-2 & \cdots & \cdots & 1 & 0 & 1 \\
n & n-1 & \cdots & \cdots & 2 & 0 & 1
\end{smatrix}
\]

\begin{noliste}{a)}
 \setlength{\itemsep}{2mm}
\item Calculer les coordonnées de $g_{k}$ dans la base
$(f_{0},f_{1},\ldots,f_{n})$. Interpréter la matrice : 
\[
P_{n} = \dfrac{1}{n}A_{n}
\]

\item Prouver que la matrice $A_{n}$ est inversible et calculer son
inverse.
\end{noliste}
\end{noliste}

\section*{Troisième partie}

Dans cette partie, on approche une fonction de classe $C^{2}$ sur
$[0,1]$
par des éléments de $E_{n}$.

\begin{noliste}{1.}
 \setlength{\itemsep}{4mm}
\item On considère une fonction de classe $C^{2}$ sur un intervalle
$[a,b]$,
où $a<b$, telle que $f(a) = f(b)$. Soit $c$ un élément de $]a,b[$ et
$\sigma
_{2} = (a,b,c)$.

\begin{noliste}{a)}
 \setlength{\itemsep}{2mm}
\item Soit $\varphi $ un élément de $\E(\sigma_{2})$ tel que $\varphi
(a) = \varphi (b) = 0$. Établir l'égalité : 
\[
\dint{a}{b}\varphi (t)f^{\prime \prime }(t)dt = \left(
f(c)-f(a)\right) \,\left( \varphi ^{\prime }(b)-\varphi ^{\prime
}(a)\right)
\]

\item Déterminer un élément $\varphi_{c}$ de $\E(\sigma_{2})$ tel que,
quelle que soit $f$ : 
\[
\dint{a}{b}\varphi_{c}(t)f^{\prime \prime }(t)dt = f(c)-f(a)
\]
\end{noliste}
\end{noliste}

\label{fin}

\end{document}

