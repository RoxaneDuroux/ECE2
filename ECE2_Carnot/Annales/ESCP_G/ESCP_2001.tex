\documentclass[11pt]{article}%
\usepackage{geometry}%
\geometry{a4paper,
 lmargin = 2cm,rmargin = 2cm,tmargin = 2.5cm,bmargin = 2.5cm}

\input{../../macros.tex}

\pagestyle{fancy} %
\lhead{ECE2 \hfill Mathématiques\\
} %
\chead{\hrule} %
\rhead{} %
\lfoot{} %
\cfoot{} %
\rfoot{\thepage} %

\renewcommand{\headrulewidth}{0pt}% : Trace un trait de séparation
 % de largeur 0,4 point. Mettre 0pt
 % pour supprimer le trait.

\renewcommand{\footrulewidth}{0.4pt}% : Trace un trait de séparation
 % de largeur 0,4 point. Mettre 0pt
 % pour supprimer le trait.

\setlength{\headheight}{14pt}

\title{\bf \vspace{-2cm} ESCP 2001 - voie Générale} %
\author{} %
\date{} %
\begin{document}

\maketitle %
\vspace{-1.4cm}\hrule %
\thispagestyle{fancy}

\vspace*{.2cm}


% DEBUT DU DOC À MODIFIER : tout virer jusqu'au début de l'exo


\begin{center}
{\small CHAMBRE D\E\ COMMERCE ET D'INDUSTRIE DE PARIS}

\textbf{DIRECTION DE L'ENSEIGNEMENT}

Direction des Admissions et concours

\underline{\hspace*{3cm}}

{\Large ECOLE DES\ HAUTES\ ETUDES\ COMMERCIALES}

{\Large E.S.C.P.-E.A.P.}

{\Large ECOL\E\ SUPERIEUR\E\ D\E\ COMMERC\E\ D\E\ LYON}{\large }

CONCOURS D'ADMISSION\ SUR\ CLASSES\ PREPARATOIRES

\underline{\hspace*{3cm}}

\textbf{OPTION SCIENTIFIQUE}

{\Large MATHEMATIQUES I}

\textbf{Année 2001}

\underline{\hspace*{3cm}}
\end{center}

\begin{quotation}
\noindent \textsl{La présentation, la lisibilité, l'orthographe, la
qualité
de la rédaction, la clarté et la précision des raisonnements entreront
pour
une part importante dans l'appréciation des copies.}

\noindent \textsl{Les candidats sont invités à encadrer dans la mesure
du
possible les résultats de leurs calculs.}

\noindent \textsl{Ils ne doivent faire usage d'aucun document :
l'utilisation de toute calculatrice et de tout matériel électronique
est
interdite.}

\noindent \textsl{Seule l'utilisation d'une règle graduée est
autorisée.}

\noindent \textsl{\hrulefill }
\end{quotation}

\noindent L'objet du problème est l'étude, dans certains cas, des
sous-espaces stables par un endomorphisme d'un espace vectoriel.\\
Dans tout le problème, on considère un entier naturel $n$ non nul et on
note 
$E$ le $\R$-espace vectoriel $\R^{n}$. On note $0_{E}$ le
vecteur nul de $E$ et $Id_{E}$ l'endomorphisme identité de $E$. On dira
qu'un sous-espace vectoriel $F$ de $E$ est stable par un endomorphisme
$f$
de $E$ (ou que $f$ laisse stable $F$) si l'inclusion $f(F)\subset F$
est vérifiée.\\
\textit{On observera que le sous-espace vectoriel réduit à
}$\{0_{E}\}$\textit{\ et }$E$\textit{\ lui-même sont stables par tout
endomorphisme de }$E$\textit{. }\\
On note $\R[X]$ l'espace vectoriel des polynômes à coefficients réels
et, pour tout entier naturel $k$, on note $\R_{k}[X]$ le
sous-espace vectoriel formé par les éléments de $\R[X]$ qui sont de
degré inférieur ou égal à $k$.\\
Si $f$ est un endomorphisme de $E$ on pose $f^{0} = Id_{E}$, $f^{1} =
f$, $f^{2} = f\circ f$, $f^{3} = f\circ f\circ f$, etc.\\
Si $f$ est un endomorphisme de $E$ et si $P = \Sum{k = 0}{n}a_{k}X^{k}$
est un élément de $\R[X]$, on rappelle qu'on note
$P\left(\Ev{f}\right)$
l'endomorphisme de $E$ égal à $P\left(\Ev{f}\right) = \Sum{k =
0}{n}a_{k}f^{k}$.

\section*{Partie I : {Préliminaires}}

Soit $f$ un endomorphisme de $E$.

\begin{noliste}{1.}
 \setlength{\itemsep}{4mm}
\item Soit $P$ un élément de $\R[X]$. Montrer que le sous-espace
vectoriel $\ker P\left(\Ev{f}\right)$ est stable par $f$.

\item 

\begin{noliste}{a)}
 \setlength{\itemsep}{2mm}
\item Montrer que les droites de $E$ stables par $f$ sont exactement
celles
qui sont engendrées par un vecteur propre de l'endomorphisme $f$.

\item On note $\mathcal{B} = (e_{1},e_{2},e_{3})$ la base canonique de
$\R^{3}$ et on considère l'endomorphisme $g$ de $\R^{3}$ dont
la matrice dans la base $\mathcal{B}$ est 
\[
B = 
\begin{smatrix}
1 & 1 & 0 \\
0 & 1 & 0 \\
0 & 0 & 2
\end{smatrix}
\]
Déterminer (en en donnant une base) les droites de $\R^{3}$ stables
par $g$.
\end{noliste}

\item Soit $p$ un entier naturel non nul.

\begin{noliste}{a)}
 \setlength{\itemsep}{2mm}
\item Si $F_{1},\dots,F_{p}$ sont $p$ sous-espaces vectoriels de $E$
stables par $f$, montrer qu'alors la somme $\Sum{k = 1}{p}F_{k}$ est
un sous-espace vectoriel stable par $f$.

\item Si $\lambda_{1},\dots,\lambda_{p}$ sont $p$ valeurs propres de
$f$
et si $n_{1},\dots,n_{p}$ sont $p$ entiers naturels montrer qu'alors la
somme $\Sum{k = 1}{p}\ker (f-\lambda_{k}Id_{E})^{n_{k}}$ est stable
par $f$.
\end{noliste}

\item 

\begin{noliste}{a)}
 \setlength{\itemsep}{2mm}
\item Soit $a$ un réel. Vérifier que les sous-espaces vectoriels de $E$
stables par un endomorphisme $f$ sont exactement ceux qui sont stables
par
l'endomorphisme $f-\lambda Id_{E}$.

\item Quel lien y-a-t-il entre les sous-espaces vectoriels stables par
un
endomorphisme $f$ et ceux qui sont stables par l'endomorphisme $f^{2}$
?

\item Quel lien y-a-t-il entre les sous-espaces vectoriels stables par
un
automorphisme $f$ et ceux qui sont stables par l'endomorphisme $f^{-1}$
?

\item Que dire d'un endomorphisme de $E$ laissant stable tout
sous-espace
vectoriel de $E$ ?

\item Donner un exemple d'endomorphisme de $\R^{2}$ ne laissant
stable que le sous-espace vectoriel réduit au vecteur nul et l'espace
$\R^{2}$.
\end{noliste}

\item 

\begin{noliste}{a)}
 \setlength{\itemsep}{2mm}
\item On rappelle qu'une forme linéaire sur $E$ est une application
linéaire
de $E$ dans $\R$ et qu'un hyperplan de $E$ est un sous-espace
vectoriel de $E$ de dimension $n-1$.\\
Montrer que les hyperplans de $E$ sont exactement les noyaux de formes
linéaires non nulles sur $E$. \textit{On pourra compléter une base d'un
hyperplan en une base de }$E$\textit{. }

\item Soit $\varphi $ une forme linéaire non nulle sur $E$ et $H = \ker
\varphi $.

\begin{nonoliste}{(i)}
\item Montrer que l'hyperplan $H$ est stable par $f$ si et seulement si
il
existe un élément $\lambda $ de $\R$ vérifiant l'égalité : $\varphi
\circ f = \lambda \varphi $.

\item On note $A$ la matrice de $f$ relativement à la base canonique de
$E$
et $L$ la matrice (ligne) de $\varphi $ relativement aux bases
canoniques de 
$E$ et $\R$.\\
Montrer que l'hyperplan $H$ est stable par $f$ si et seulement si il
existe
un réel $\lambda $ vérifiant l'égalité $^{t}A$ $^{t}L = \lambda $
$^{t}L$.
\end{nonoliste}

\item Déterminer (en en donnant une base) les plans de $\R^{3}$
stables par l'endomorphisme $g$ défini à la question 2).
\end{noliste}
\end{noliste}

\section*{Partie II : {Le cas où l'endomorphisme est diagonalisable}}

Dans cette partie, on considère un endomorphisme $f$ de $E$
diagonalisable
et on note $\lambda_{1},\dots,\lambda_{p}$ ses valeurs propres
distinctes
et $E_{1},\dots,E_{p}$ les sous-espaces propres correspondants.

\begin{noliste}{1.}
 \setlength{\itemsep}{4mm}
\item Que dire des sous-espaces vectoriels de $E$ stables par $f$ si $p
= 1$ ?

\item On suppose l'entier $p$ au moins égal à $2$. On considère un
sous-espace vectoriel $F$ de $E$ stable par $f$ et un élément $x$ de
$F$.

\begin{noliste}{a)}
 \setlength{\itemsep}{2mm}
\item Justifier l'existence d'un unique élément
$(x_{1},x_{2},\dots,x_{p})$
de $\prod\limits_{k = 1}{p}E_{k}$ vérifiant l'égalité : $x = \Sum{k =
1}{p}x_{k}$.

\item Montrer que le vecteur $\Sum{k = 1}{p}(\lambda_{k}-\lambda
_{1})x_{k}$ est élément de $F$.

\item Montrer que les vecteurs $x_{1},\dots,x_{p}$ sont tous dans $F$.
\end{noliste}

\item Déduire de la question précédente que les sous-espaces vectoriels
de $E $ stables par $f$ sont exactement les sous-espaces vectoriels de
la forme $\Sum{k = 1}{p}F_{k}$ où, pour tout entier $k$ vérifiant les
inégalités $1\leq k\leq p$, $F_{k}$ est un sous-espace vectoriel de
$E_{k}$.

\item Montrer que l'endomorphisme induit par $f$ sur l'un de ses
sous-espaces vectoriels stables $F$ est un endomorphisme diagonalisable
de $F $.

\item Donner une condition nécessaire et suffisante portant sur les
valeurs
propres de $f$ pour que $E$ possède un nombre fini de sous-espaces
vectoriels stables par $f$. Quel est alors ce nombre ?
\end{noliste}

\section*{Partie III : {Le cas où l'endomorphisme est nilpotent d'ordre
$n$}}

\begin{noliste}{1.}
 \setlength{\itemsep}{4mm}
\item On note $D$ l'endomorphisme de $\R_{n-1}[X]$ qui à tout polynôme
$P$ associe son polynôme dérivé $P^{\prime }$.

\begin{noliste}{a)}
 \setlength{\itemsep}{2mm}
\item Vérifier que $D^{n}$ est l'endomorphisme nul et que $D^{n-1}$ ne
l'est
pas.

\item Vérifier que les sous-espaces vectoriels de $\R_{n-1}[X]$
stables par $D$ sont, en dehors du sous-espace vectoriel réduit au
polynôme
nul, les $n$ sous-espaces vectoriels suivants :
$\R_{0}[X],\R_{1}[X],\dots,\R_{n-1}[X]$.
\end{noliste}

\item On note $\mathbf{0}$ l'endomorphisme nul de $E$ et on considère
un
endomorphisme $f$ de $E$ nilpotent d'ordre $n$, c'est-à-dire vérifiant
les
conditions : $f^{n} = \mathbf{0}$ et $f^{n-1}\not = \mathbf{0}$.

\begin{noliste}{a)}
 \setlength{\itemsep}{2mm}
\item Établir qu'il existe une base $\mathcal{B} =
(e_{1},e_{2},\dots,e_{n})$
de $E$ dans laquelle la matrice $A$ de $f$ est 
\[
A = 
\begin{smatrix}
0 & 1 & 0 & \dots & 0 \\
0 & 0 & 1 & \ddots & \vdots \\
\vdots & \ddots & \ddots & \ddots & 0 \\
\vdots & & \ddots & 0 & 1 \\
0 & \dots & \dots & 0 & 0
\end{smatrix}
\]
$A$ est donc la matrice dont le coefficient de la ligne $i$ et de la
colonne 
$j$ ($1\leq i\leq n,1\leq j\leq n$) vaut $1$ si $j = i + 1$
et $0$ sinon.

\item Montrer que la matrice $A$ est semblable à la matrice $B$
suivante 
\[
B = 
\begin{smatrix}
0 & 1 & 0 & \dots & 0 \\
0 & 0 & 2 & \ddots & \vdots \\
\vdots & \ddots & \ddots & \ddots & 0 \\
\vdots & & \ddots & 0 & n-1 \\
0 & \dots & \dots & 0 & 0
\end{smatrix}
\]
$B$ est donc la matrice dont le coefficient de la ligne $i$ et de la
colonne 
$j$ ($1\leq i\leq n,1\leq j\leq n$) vaut $i$ si $j = i + 1$
et $0$ sinon.

\item Déterminer (en en donnant une base) les sous-espaces vectoriels
de $E$
stables par $f$.
\end{noliste}
\end{noliste}

\section*{Partie IV : {Le cas où l'endomorphisme est nilpotent d'ordre
2}}

Dans cette partie on considère un endomorphisme $f$ de $E$ nilpotent
d'ordre 
$2$ c'est à dire un endomorphisme non nul de $E$ tel que $f\circ f$ est
l'endomorphisme nul.

\begin{noliste}{1.}
 \setlength{\itemsep}{4mm}
\item On considère un sous-espace vectoriel $F_{2}$ de $E$ vérifiant
$F_{2}\cap \ker f = \{0_{E}\}$.

\begin{noliste}{a)}
 \setlength{\itemsep}{2mm}
\item Justifier l'inclusion : $f(F_{2})\subset \ker f$.

\item On considère de plus un sous-espace vectoriel $F_{1}$ de $\ker f$
contenant $f(F_{2})$. Montrer que la somme $F_{1} + F_{2}$ est directe
et que
c'est un sous-espace vectoriel de $E$ stable par $f$.

\item Étant donné $A$, $B$, $C$ trois sous-espaces vectoriels de $E$,
établir l'inclusion : 
\[
(A\cap C) + (B\cap C)\subset (A + B)\cap C
\]
A-t-on nécessairement l'égalité ?

\item Déterminer l'intersection $(F_{1} + F_{2})\cap \ker f$.
\end{noliste}

\item Réciproquement on considère un sous-espace vectoriel $F$ de $E$
stable
par $f$. On pose $F_{1} = F\cap \ker f$ et on considère un sous-espace
vectoriel $F_{2}$ supplémentaire de $F_{1}$ dans $F$.\\
Vérifier l'inclusion $f(F)\subset \ker f$ et prouver que l'intersection
$F_{2}\cap \ker f$ est réduite au vecteur nul.

\item \textbf{Dans cette question}, on suppose que l'entier $n$ est
égal à $4 $ (i.e. $E = \R^{4}$) et on considère l'endomorphisme $h$ de
$E$
dont la matrice dans la base canonique $\mathcal{B} =
(e_{1},e_{2},e_{3},e_{4}) $ de $\R^{4}$ est la matrice $M$ suivante 
\[
M = 
\begin{smatrix}
1 & 1 & 0 & 0 \\
0 & 1 & 0 & 0 \\
0 & 0 & 2 & 1 \\
0 & 0 & 0 & 2
\end{smatrix}
\]

\begin{noliste}{a)}
 \setlength{\itemsep}{2mm}
\item Vérifier que les sous-espaces vectoriels $G_{1} = \ker
(h-Id)^{2}$ et $G_{2} = \ker (h-2Id)^{2}$ sont supplémentaires.

\item Montrer que les sous-espaces vectoriels stables par $h$ sont
exactement les sommes $H_{1} + H_{2}$ où $H_{1}$ (resp. $H_{2}$) est un
sous-espace vectoriel de $G_{1}$ (resp. $G_{2}$) stable par $h$.

\item Déterminer (en en donnant une base) les sous-espaces vectoriels
de $E$
stables par $h$.
\end{noliste}
\end{noliste}

\section*{Partie V : {Existence d'un plan stable par un endomorphisme}}

Soit $f$ un endomorphisme non nul de $E$.

\begin{noliste}{1.}
 \setlength{\itemsep}{4mm}
\item Justifier l'existence d'un polynôme non nul à coefficients réels
annulant $f$.\\
On note $M$ un polynôme non nul à coefficients réels de plus bas degré
annulant $f$. \textit{On observera que }$M$\textit{\ n'est pas
constant. }

\item \textbf{Dans cette question}, on suppose que le polynôme $M$ n'a
pas
de racine réelle et on note $z$ l'une de ses racines complexes.

\begin{noliste}{a)}
 \setlength{\itemsep}{2mm}
\item Vérifier que le conjugué de $z$ est aussi racine de $M$ et en
déduire
qu'il existe un polynôme du second degré à coefficients réels noté
$X^{2} + bX + c$ qui divise $M$.

\item Montrer que l'endomorphisme $f^{2} + bf + cId_{E}$ n'est pas
injectif.

\item En déduire qu'il existe un plan de $E$ stable par $f$.
\end{noliste}

\item \textbf{Dans cette question}, on suppose qu'il existe un réel
$\lambda 
$, un réel $\alpha $ non nul et un entier $p$ au moins égal à $2$
vérifiant
l'égalité : $M = \alpha (X-\lambda )^{p}$. On pose $g = f-\lambda
Id_{E}$.

\begin{noliste}{a)}
 \setlength{\itemsep}{2mm}
\item Montrer qu'il existe un vecteur $x$ de E tel que la famille
$(x,g(x),\dots,g^{(p-1)}(x))$ est libre.

\item En déduire qu'il existe un plan de $E$ stable par $f$.
\end{noliste}

\item Montrer que, dans tous les cas, il existe un plan de $E$ stable
par $f$.
\end{noliste}

\label{fin}

\end{document}

