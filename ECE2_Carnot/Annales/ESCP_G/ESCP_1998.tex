\documentclass[11pt]{article}%
\usepackage{geometry}%
\geometry{a4paper,
 lmargin = 2cm,rmargin = 2cm,tmargin = 2.5cm,bmargin = 2.5cm}

\input{../../macros.tex}

\pagestyle{fancy} %
\lhead{ECE2 \hfill Mathématiques\\
} %
\chead{\hrule} %
\rhead{} %
\lfoot{} %
\cfoot{} %
\rfoot{\thepage} %

\renewcommand{\headrulewidth}{0pt}% : Trace un trait de séparation
 % de largeur 0,4 point. Mettre 0pt
 % pour supprimer le trait.

\renewcommand{\footrulewidth}{0.4pt}% : Trace un trait de séparation
 % de largeur 0,4 point. Mettre 0pt
 % pour supprimer le trait.

\setlength{\headheight}{14pt}

\title{\bf \vspace{-2cm} ESCP 1998 - voie Générale} %
\author{} %
\date{} %
\begin{document}

\maketitle %
\vspace{-1.4cm}\hrule %
\thispagestyle{fancy}

\vspace*{.2cm}


% DEBUT DU DOC À MODIFIER : tout virer jusqu'au début de l'exo


\begin{center}
{\small CHAMBRE D\E\ COMMERCE ET D'INDUSTRIE DE PARIS}

\textbf{DIRECTION DE L'ENSEIGNEMENT}

Direction des Admissions et concours

\underline{\hspace*{3cm}}

{\Large ECOLE DES\ HAUTES\ ETUDES\ COMMERCIALES}

{\Large E.S.C.P.-E.A.P.}

{\Large ECOL\E\ SUPERIEUR\E\ D\E\ COMMERC\E\ D\E\ LYON}{\large }

CONCOURS D'ADMISSION\ SUR\ CLASSES\ PREPARATOIRES

\underline{\hspace*{3cm}}

\textbf{OPTION SCIENTIFIQUE}

{\Large MATHEMATIQUES I}

\textbf{Année 1998}

\underline{\hspace*{3cm}}
\end{center}

\begin{quotation}
\noindent \textsl{La présentation, la lisibilité, l'orthographe, la
qualité
de la rédaction, la clarté et la précision des raisonnements entreront
pour
une part importante dans l'appréciation des copies.}

\noindent \textsl{Les candidats sont invités à encadrer dans la mesure
du
possible les résultats de leurs calculs.}

\noindent \textsl{Ils ne doivent faire usage d'aucun document :
l'utilisation de toute calculatrice et de tout matériel électronique
est
interdite.}

\noindent \textsl{Seule l'utilisation d'une règle graduée est
autorisée.}

\noindent \textsl{\hrulefill }
\end{quotation}

\noindent L'objet de ce problème est l'étude d'un algorithme
d'approximation
d'une racine carrée de certains éléments de $\R$, $\C$ ou $\M_{n}(\R)$,
c'est à dire la construction d'une suite
convergeant vers un élément dont le carré est donné.

\section*{Partie I : Algorithme de Newton dans $\R$}

Soit $a$ un réel strictement positif.

\begin{noliste}{1.}
 \setlength{\itemsep}{4mm}
\item 

\begin{noliste}{a)}
 \setlength{\itemsep}{2mm}
\item Donner le tableau de variation de la fonction définie, pour $x$
élément de $\R_{+}{\times }$ par : 
\[
f(x) = \dfrac{1}{2}\left( x + \dfrac{a}{x}\right)
\]

\item Justifier rapidement l'existence de la suite réelle
$(x_{n})_{n\in 
\N}$ définie par la donnée de $x_{0} = a$ et de la relation de
récurrence : $x_{n + 1} = f(x_{n})$ pour tout entier naturel $n$.

\item Pour tout entier naturel $n$, établir les égalités : 
\[
x_{n + 1}-\sqrt{a} = \dfrac{1}{2x_{n}}(x_{n}-\sqrt{a})^{2}\qquad
\text{et}\qquad
x_{n + 2}-x_{n + 1} = \dfrac{1}{2x_{n + 1}}(a-x_{n + 1}{2})
\]

\item En déduire que la suite $(x_{n})_{n\in \N}$ converge vers
$\sqrt{a}$.
\end{noliste}

\item 

\begin{noliste}{a)}
 \setlength{\itemsep}{2mm}
\item Pour tout entier naturel non nul, prouver les inégalités : 
\[
0\leq x_{n + 1}-\sqrt{a}\leq \dfrac{1}{2\sqrt{a}}(x_{n}-\sqrt{a})^{2}
\]

\item Soit $b$ un réel strictement positif et $(u_{n})_{n\in \N^{\times
}}$ une suite de réels positifs vérifiant l'inégalité : $u_{n + 1}\leq
bu_{n}{2}$ pour tout entier naturel non nul $n$.\\
Pour tout entier naturel non nul $n$, donner une majoration de $u_{n}$
en
fonction de $n$, $b$, $u_{1}$.

\item En déduire, pour tout entier naturel $n$ non nul, une majoration
de $x_{n}-\sqrt{a}$ en fonction de $n$, $x_{1}$ et $a$.
\end{noliste}

\item 

\begin{noliste}{a)}
 \setlength{\itemsep}{2mm}
\item En décrivant pas à pas les étapes de l'algorithme, que dire du
résultat rendu par le programme suivant quand on l'exécute ?{}

\texttt{program\ racine\ carree\ ;}

\texttt{function\ rc(a,x,eps\ :real)\ :real;}

\texttt{begin}

\texttt{if\ abs(x*x-a) \TEXTsymbol{<}eps then\ rc\ : = x\ else}

\qquad \qquad \qquad \qquad \qquad \qquad \qquad \qquad \qquad
\texttt{begin}

\qquad \qquad \qquad \qquad \qquad \qquad \qquad \qquad \qquad \qquad 
\texttt{x : = 1/2*(x + a/x);}

\qquad \qquad \qquad \qquad \qquad \qquad \qquad \qquad \qquad \qquad 
\texttt{rc : = rc(a,x,eps) ;\ }

\qquad \qquad \qquad \qquad \qquad \qquad \qquad \qquad \qquad
\texttt{end;}

\texttt{end;}

\texttt{begin\ writeln(rc(2,2,1e-16))\ ;}

\texttt{end.}

\item On rappelle les inégalités : $1,4<\sqrt{2}<1,5$.\\
Montrer que, lors de l'exécution du programme précédent, le nombre de
comparaisons effectuées est inférieur ou égal à six.\textit{\ On
supposera
le type real\ suffisamment étendu pour pouvoir manipuler des nombres à
une précision d'au moins vingt décimales. }
\end{noliste}
\end{noliste}

\section*{Partie II\hspace{0.2cm} : Algorithme de Newton dans $\C$}

On se propose dans cette partie d'adapter la méthode de Newton à la
recherche d'une racine carrée d'un nombre complexe $a$, c'est à dire
d'approcher un nombre complexe dont le carré vaut $a$. Dans toute cette
partie, $a$ désigne un nombre complexe qui n'est pas un réel négatif ou
nul.\\
On note $Re(z)$ la partie réelle d'un nombre complexe $z$.

\begin{noliste}{1.}
 \setlength{\itemsep}{4mm}
\item 

\begin{noliste}{a)}
 \setlength{\itemsep}{2mm}
\item Montrer qu'il existe un unique nombre complexe $b$ de partie
réelle
strictement positive tel que $b^{2} = a$. \\
On note $\mathcal{P}_{+} = \left\{ z\in \C,\quad \mathcal{R}e\left( 
\dfrac{z}{b}\right) >0\right\} $.

\item Dans cette sous question (et uniquement ici) on suppose que $a =
2i$. Déterminer le nombre $b$ dans ce cas particulier et représenter
l'ensemble des
points $M$ du plan dont l'affixe $z$ est élément de $\mathcal{P}_{+}$.
\end{noliste}

\item On revient au cas général (où le nombre complexe $a$ n'est pas un
réel
négatif ou nul) et on considère l'application $f$ définie pour tout
nombre
complexe $z$ non nul par $f(z) = \dfrac{1}{2}\left( z +
\dfrac{a}{z}\right) $.\\
Établir l'inclusion : $f\left( \mathcal{P}_{+}\right) \subset P_{+}$.

\item On considère la suite $(z_{n})_{n\in \N}$ définie par la donnée
de $z_{0} = a$ et par la relation de récurrence : $z_{n + 1} =
f(z_{n})$ pour
tout entier naturel $n$.\\
On pose également : $w_{n} = \dfrac{z_{n}-b}{z_{n} + b}$ pour tout
entier
naturel $n$.

\begin{noliste}{a)}
 \setlength{\itemsep}{2mm}
\item Justifier l'existence des suites $(z_{n})_{n\in \N}$ et
$(w_{n})_{n\in \N}$.

\item Exprimer, pour tout entier naturel $n$ non nul, $w_{n}$ en
fonction de 
$w_{n-1}$, puis, pour tout entier naturel $n$, $w_{n}$ en fonction de
$w_{0}$
et $n$.
\end{noliste}

\item Prouver la majoration $\left| w_{0}\right| <1$. En déduire la
limite de la suite $(z_{n})_{n\in \N}$.
\end{noliste}

\section*{Partie III : Racine carrée d'une matrice}

Dans cette partie $n$ désigne un entier naturel au moins égal à 2.

\begin{noliste}{$\sbullet$}
\item On note $\mathfrak{M}_{n}(\R)$ l'espace vectoriel des matrices
carrées à coefficients réels ayant $n$ lignes et $n$ colonnes.

\item On note $\mathfrak{M}_{n,1}(\R)$ l'espace vectoriel des
matrices colonnes à coefficients réels ayant $n$ lignes et une colonne.

\item On munit $\R^{n}$ de sa structure euclidienne usuelle dont on
note $||.||$ la norme.

\item On identifie les vecteurs de $\R^{n}$ aux éléments de
$\mathfrak{M}_{n,1}(\R)$ de telle sorte que si $x =
(x_{1},x_{2},\ldots,x_{n})$ on écrira $Mx$ pour $M
\begin{smatrix}
x_{1} \\
x_{2} \\
\vdots \\
x_{n}\end{smatrix}
$

\item Soit $A$ une matrice carrée à coefficients réels. On appelle
racine
carrée de $A$ toute matrice $B$ vérifiant $B^{2} = A$.
\end{noliste}

\subsection*{A. Quelques exemples \ }

\begin{noliste}{1.}
 \setlength{\itemsep}{4mm}
\item Montrer que la matrice $\begin{smatrix}
0 & 1 \\
0 & 0
\end{smatrix}
$ n'admet pas de racine carrée.

\item On se propose, dans cette question, de généraliser le résultat de
la
question précédente.\\
On considère l'élément de $\mathfrak{M}_{n}(\R)$ suivant : 
\[
A = 
\begin{smatrix}
0 & 1 & 0 & \ldots & 0 \\
0 & 0 & 1 & \ldots & 0 \\
\vdots & & \ddots & \ddots & \vdots \\
\vdots & & & \ddots & 1 \\
0 & \ldots & \ldots & 0 & 0
\end{smatrix}
\]
$A$ est donc la matrice dont le coefficient en ligne $i$ et colonne $j$
est
nul sauf si $1\leq i\leq n-1$ et $j = i + 1$ auquel cas il vaut 1.\\
On suppose qu'il existe une matrice $B$ racine carrée de $A$ et on note
$g$
l'endomorphisme de $\R^{n}$ ayant, dans la base canonique, $B$ pour
matrice.

\begin{noliste}{a)}
 \setlength{\itemsep}{2mm}
\item L'endomorphisme $g$ est-il bijectif ?

\item Prouver que $\func{Im}(g)$ est stable par $g$ (c'est à dire que
$g(\func{Im}(g))\subset \func{Im}(g)$), puis que la restriction de $g$
à $\func{Im}(g)$ est un automorphisme de $\func{Im}(g)$.

\item Que vaut $g^{2n}$ ? En déduire que la matrice $A$ n'a pas de
racine
carrée.
\end{noliste}

\item Donner un exemple d'élément de $\mathfrak{M}_{2}(R)$ possédant
une
infinité de racines carées.
\end{noliste}

\subsection*{B. Racine carrée d'une matrice symétrique strictement
positive}

On note $\mathcal{S}_{n}$ l'ensemble des matrices symétriques de
$\mathfrak{M}_{n}(\R)$. On appelle matrice symétrique strictement
positive, tout 
élément de $\mathcal{S}_{n}$ dont les valeurs propres sont strictement
positives. On note $\mathcal{S}_{n}{+}$ l'ensemble des matrices
symétriques
réelles strictement positives. On suppose désormais que $A$ est un
élément
de $\mathcal{S}_{n}{+}$.

\begin{noliste}{1.}
 \setlength{\itemsep}{4mm}
\item Montrer que $A$ admet une racine carrée symétrique réelle
strictement
positive. On pourra commencer par le cas où $A$ est diagonale.

\item Soit $B$ et $C$ deux racines carrées symétriques réelles
strictement
positives de $A$.

\begin{noliste}{a)}
 \setlength{\itemsep}{2mm}
\item Justifier l'existence de deux matrices $P$ et $Q$ inversibles et
de
deux matrices diagonales $D$ et $\Delta $ telles que : $A = PD^{2\;t}P
= Q\Delta
^{2\;t}Q$.

\item En déduire l'existence d'une matrice inversible $R$ telle que
$RD^{2} = \Delta ^{2}R$.\\
Établir l'égalité $RD = \Delta R$. \ \textit{On comparera les
coefficients de
ligne }$i$\textit{\ et de colonne }$j$\textit{\ (}$1\leq i,j\leq
n$\textit{) de ces deux matrices} \emph{.}

\item Conclure qu'il existe une unique racine carrée de $A$ symétrique
réelle strictement positive, que l'on notera $A^{1/2}$.\vspace{0.5cm}
\end{noliste}

Jusqu'à la fin de cette partie B, on note
$\lambda_{1},\lambda_{2},\ldots,\lambda_{p}$ les valeurs propres
distinctes de $A$ et, pour tout entier $j$, $1\leq j\leq p$, $E_{j}$ le
sous espace propre de $A$ associé à
la valeur propre $\lambda_{j}$.

\item Pour tout entier $j$ tel que $1\leq j\leq p$ et pour tout réel
$x$, on pose : 
\[
L_{j}(x) = \prod_{\substack{ i = 1 \\
i\neq j}}{p}\dfrac{x-\lambda_{i}}{\lambda_{j}-\lambda_{i}}
\]

\begin{noliste}{a)}
 \setlength{\itemsep}{2mm}
\item Montrer que la famille $(L_{1},L_{2},\ldots,L_{p})$ est une base
de
l'espace vectoriel des polynômes à coefficients réels de degré
strictement
inférieur à $p$.

\item Montrer qu'il existe un unique polynôme $P$ à coefficients réels
de
degré strictement inférieur à $p$ tel que, pour tout entier $i$ de
$\{1,2,\ldots,p\}$, $P\left(\Ev{\lambda_{i}}\right) =
\sqrt{\lambda_{i}}$.

\item 

\begin{nonoliste}{(i)}
\item Pour tout entier $j$, $1\leq j\leq p$ et pour tout vecteur
$x_{j}$ de $E_{j}$, calculer
$P\left(\Ev{A}\right)\left(\Ev{x_{j}}\right)$ et en déduire l'égalité :
$P\left(\Ev{A}\right)^{2} = A$.

\item Montrer que les valeurs propres de $P\left(\Ev{A}\right)$ sont
toutes strictement
positives.

\item Conclure à l'égalité : 
\[
A^{1/2} = \Sum{i = 1}{p}\sqrt{\lambda_{i}}L_{i}(A)
\]
\end{nonoliste}
\end{noliste}

\item \textbf{Un exemple}\\
On considère les éléments de $\M_{n}(\R)$ suivants : 
\[
U = 
\begin{smatrix}
1 & 1 & \ldots & \ldots & 1 \\
1 & 1 & 1 & \ldots & 1 \\
\vdots & \ddots & \ddots & \ddots & \vdots \\
\vdots & & \ddots & \ddots & 1 \\
1 & \ldots & \ldots & 1 & 1
\end{smatrix}
\qquad \text{et}\qquad A = 
\begin{smatrix}
n & -1 & \ldots & \ldots & -1 \\
-1 & n & -1 & \ldots & -1 \\
\vdots & \ddots & \ddots & \ddots & \vdots \\
\vdots & & \ddots & \ddots & -1 \\
-1 & \ldots & \ldots & -1 & n
\end{smatrix}
\]
$U$ est donc la matrices dont tous les coefficients sont égaux à 1 et
$A$
celle dont le coefficient en ligne $i$ et colonne $j$ vaut $n$ si $i =
j$ et $-1$ sinon.

\begin{noliste}{a)}
 \setlength{\itemsep}{2mm}
\item Déterminer les valeurs propres et les sous espaces propres de $U$
et
de $A$.

\item Exprimer $A^{1/2}$ en fonction de $A$ et $I_{n}$ (matrice
identité de $\M_{n}(\R)$).
\end{noliste}
\end{noliste}

\section*{Partie IV : Algorithme de Newton dans $\mathcal{S}_{n}{+}$}

Dans toute cette partie on considère un élément $A$ de
$\mathcal{S}_{n}{+}$
et une base orthonormée $(e_{1},e_{2},\ldots,e_{n})$ de $\R^{n}$
formée de vecteurs propres de $A$, le vecteur propre $e_{i}$ étant,
pour
tout entier $i$, $1\leq i\leq n$, associé à la valeur propre
$\lambda_{i}$. (les réels $\lambda_{1},\lambda_{2},\ldots,\lambda_{n}$
n'étant pas nécessairement distincts).

\begin{noliste}{1.}
 \setlength{\itemsep}{4mm}
\item Soit $M$ un élément de $\mathcal{S}_{n}{+}$ dont
$(e_{1},e_{2},\ldots,e_{n})$ est aussi une base de vecteurs propres. On
note $\mu_{1},\mu
_{2},\ldots,\mu_{n}$ les valeurs propres correspondantes (i.e. $Me_{i}
= \mu
_{i}e_{i}$, pour tout entier $i$, $1\leq i\leq n$). \\
Montrer que $(e_{1},e_{2},\ldots,e_{n})$ est encore une base de
vecteurs
propres de la matrice $M^{\prime } = \dfrac{1}{2}(M + M^{-1}A)$. Pour
tout
entier $i$, $1\leq i\leq n$, quelle relation existe-t-il entre la
valeur propre $\mu_{i}{\prime }$ de $M^{\prime }$ associée à $e_{i}$ et
$\mu_{i}$ ?

\item 

\begin{noliste}{a)}
 \setlength{\itemsep}{2mm}
\item Déduire de la question précédente qu'il est possible de définir
une
suite $(A_{k})_{k\in \N}$ d'éléments de $\mathcal{S}_{n}{+}$ telle
que : $A_{0} = A$ et, pour tout entier naturel $k$, $A_{k + 1} =
\dfrac{1}{2}(A_{k} + A_{k}{-1}A)$.

\item Pour tout entier $i$, $1\leq i\leq n$, on note $\lambda
_{i,k}$ la valeur propre de $A_{k}$ associée à $e_{i}$. Étudier, pour
tout
entier $i$, $1\leq i\leq n$, la convergence de la suite $(\lambda
_{i,k})_{k\in \N}$.
\end{noliste}

\item On dit qu'une suite $(M_{k})_{k\in \N}$ d'éléments de
$\M_{n}(\R)$ converge vers la matrice $M$ de $\M_{n}(\R)$ s'il existe
une base $(\varepsilon_{1},\varepsilon_{2},\ldots,\varepsilon_{n})$ de
$\R^{n}$ telle que, pour tout entier $i$, $1\leq i\leq n$,
$\dlim{k\rightarrow + \infty
}||M_{k}\varepsilon_{i}-M\varepsilon_{i}|| = 0$.\\
Montrer que la suite $(A_{k})_{k\in \N}$ converge vers $A^{1/2}$.
\end{noliste}

\label{fin}

\end{document}

