\documentclass[11pt]{article}%
\usepackage{geometry}%
\geometry{a4paper,
 lmargin = 2cm,rmargin = 2cm,tmargin = 2.5cm,bmargin = 2.5cm}

\input{../../macros.tex}

\pagestyle{fancy} %
\lhead{ECE2 \hfill Mathématiques\\
} %
\chead{\hrule} %
\rhead{} %
\lfoot{} %
\cfoot{} %
\rfoot{\thepage} %

\renewcommand{\headrulewidth}{0pt}% : Trace un trait de séparation
 % de largeur 0,4 point. Mettre 0pt
 % pour supprimer le trait.

\renewcommand{\footrulewidth}{0.4pt}% : Trace un trait de séparation
 % de largeur 0,4 point. Mettre 0pt
 % pour supprimer le trait.

\setlength{\headheight}{14pt}

\title{\bf \vspace{-2cm} ESCP 1988 - voie Générale} %
\author{} %
\date{} %
\begin{document}

\maketitle %
\vspace{-1.4cm}\hrule %
\thispagestyle{fancy}

\vspace*{.2cm}


% DEBUT DU DOC À MODIFIER : tout virer jusqu'au début de l'exo


\begin{center}
{\small CHAMBRE D\E\ COMMERCE ET D'INDUSTRIE DE PARIS}

\textbf{DIRECTION DE L'ENSEIGNEMENT}

Direction des Admissions et concours

\underline{\hspace*{3cm}}

{\Large ECOLE DES\ HAUTES\ ETUDES\ COMMERCIALES}

{\Large E.S.C.P.-E.A.P.}

{\Large ECOL\E\ SUPERIEUR\E\ D\E\ COMMERC\E\ D\E\ LYON}{\large }

CONCOURS D'ADMISSION\ SUR\ CLASSES\ PREPARATOIRES

\underline{\hspace*{3cm}}

\textbf{OPTION GENERALE}

{\Large MATHEMATIQUES I}

\textbf{Année 1988}

\underline{\hspace*{3cm}}
\end{center}

\begin{quotation}
\noindent \textsl{La présentation, la lisibilité, l'orthographe, la
qualité
de la rédaction, la clarté et la précision des raisonnements entreront
pour
une part importante dans l'appréciation des copies.}

\noindent \textsl{Les candidats sont invités à encadrer dans la mesure
du
possible les résultats de leurs calculs.}

\noindent \textsl{Ils ne doivent faire usage d'aucun document :
l'utilisation de toute calculatrice et de tout matériel électronique
est
interdite.}

\noindent \textsl{Seule l'utilisation d'une règle graduée est
autorisée.}

\noindent \textsl{\hrulefill }
\end{quotation}

\noindent Le but du problème est d'étudier les fonctions polynômes P à
coefficients réels telles que, pour tout x réel :\begin{equation}
(x-1)P{"}(x) + 4xP^{\prime }(x) = \lambda P\left(\Ev{x}\right)
\label{1}
\end{equation}où $\lambda $ est un nombre réel donné et où $P^{\prime
}$ et $P{"}$ désigne
les dérivés première et seconde de $P$. Dans la \textbf{partie I}, on
détermine à partir d'une suite $(P_{0},P_{1},...,P_{n},...)$ de
solutions
particulières toutes les solutions de cette équation et on établit une
relation de récurrence satisfaite par les polynômes $P_{n}$. Dans la
\textbf{partie II}, on utilise cette relation pour obtenir le
comportement de la
suite des valeurs $(P_{n}(x))$ en un point $x$ donné, en commençant par
le
cas particulier où $x = \dfrac{5}{3}$.

\section*{PARTIE I : ÉTUDE DE L'EQUATION (\protect\ref{1})}

\begin{noliste}{1.}
 \setlength{\itemsep}{4mm}
\item Soit $P$ une solution non nulle de (\ref{1}), de degré $n$

\begin{noliste}{a)}
 \setlength{\itemsep}{2mm}
\item Montrer, en identifiant dans (\ref{1}) les termes de plus haut
degré,
que $\lambda $ est nécessairement égal à $n(n + 3)$.

\item Soit $Q(x) = (-1)^{n}P\left(\Ev{-x}\right)$. Montrer que $Q$ est
solution de (\ref{1}).
En étudiant le degré du polynôme $P-Q$, prouver que $P = Q$ et en
déduire la
parité de $P$ en fonction de $n$.
\end{noliste}

\item Inversement, on se propose de prouver qu'étant donné un entier
$n\geq 0$, il existe un polynôme $P$ à coefficients réels et un seul
dont le terme de plus haut degré est $x^{n}$ et tel que, pour tout réel
$x$ :\begin{equation}
(x-1)P + 4xP = n(n + 3)P \label{2}
\end{equation}

\begin{noliste}{a)}
 \setlength{\itemsep}{2mm}
\item Déterminer $P_{0},P_{1},P_{2}${\small.}

\item Dans le cas général, on pose : $P_{n}(x) = \Sum{k =
0}{\E(\dfrac{n}{2})}a_{2k}x^{n-2k} = a_{0}x^{n} + a_{2}x^{n-2} +... +
a_{2k}x^{n-2k} +...$ avec $a_{0} = 1$\\
où $E$ désigne le plus grand entier inférieur ou égal à
$\dfrac{n}{2}$.\\
Expliciter un système linéaire satisfait par les nombres $a_{2k}$, où
$0\leq 2k\leq n$, et montrer que ce système admet une solution et
une seule (que l'on ne demande pas d'expliciter). Donner l'expression
du
coefficient $a_{2}$.
\end{noliste}

\item À partir de la suite $(P_{n})$, déterminer, selon les valeurs de
$\lambda $, l'ensemble $E_{\lambda }$ des solutions de (\ref{1}).

\item On se propose d'établir que, pour tout réel $x$ et tout nombre
entier $n\geq 2$ :\begin{equation}
P_{n}(x)-xP_{n-1}(x) + \dfrac{n^{2}-1}{4n^{2}-1}P_{n-2}(x) = 0
\label{3}
\end{equation}

\begin{noliste}{a)}
 \setlength{\itemsep}{2mm}
\item On considère, pour $n\geq 2$, la fonction polynôme :
\[
Q_{n}(x) = (x^{2}-1)P_{n}{\prime }(x)-nxP_{n}(x).
\]
Déterminer le monône de plus haut degré de $Q$.\\
Montrer que $Q_{n}{\prime }(x) = (n + 2)(nP_{n}(x)-xP_{n}{\prime
}(x))$, et
calculer $(x^{2}-1)Q_{n}{\prime \prime }(x) + 4xQ_{n}{\prime }(x)$ en
fonction de $Q_{n}(x)$ seulement.\\
En déduire que :\begin{equation}
(x^{2}-1)P_{n}{\prime }(x)-nxP_{n}(x) + \dfrac{n(n + 2)}{2n +
1}P_{n-1}(x) = 0
\label{4}
\end{equation}

\item En dérivant (\ref{4}), et en recourant par exemple à l'expression
de
la dérivée de $Q_{n}$ obtenue précédemment, donner une relation entre
$P_{n},P_{n}{\prime }$et $P_{n-1}{\prime }$.\\
En utilisant à nouveau la relation (\ref{4}), en déduire la relation
(\ref{3}).
\end{noliste}
\end{noliste}

\section*{PARTIE II : ETUDE DU COMPORTEMENT ASYMPTOTIQUE DE LA
SUIT\E($P_{n}(x)$)}

Soit $(u_{n})_{n\geq 0}$ la suite de nombres réels définie par la
relation de récurrence :\begin{equation}
u_{n} = u_{n-1} + \dfrac{1}{9}[(u_{n-1}-u_{n-2}) +
\dfrac{3}{4n^{2}-1}u_{n-2}]
\label{5}
\end{equation}avec $n\geq 2$, et les conditions initiales $u_{0} = 1$
et $u_{1} = 1 + \dfrac{1}{9}$.

\begin{noliste}{1.}
 \setlength{\itemsep}{4mm}
\item En remarquant que :
\[
\dfrac{1}{4k^{2}-1} = \dfrac{1}{2}(\dfrac{1}{2k-1}-\dfrac{1}{2k + 1})
\]
calculer pour tout entier $n\geq 2 :\quad S_{n} = \Sum{k =
2}{2}\dfrac{1}{4k^{2}-1}$\\
Quelle est la limite de la suite $(S_{n})$ ?

\item On se propose, dans cette question, d'étudier la suite $(u_{n})$
définie ci dessus.

\begin{noliste}{a)}
 \setlength{\itemsep}{2mm}
\item Montrer, par récurrence, que, pour tout entier $n\geq
1$,\hspace{5mm}$u_{n}\geq u_{n-1}\geq 1$.

\item Prouver que l'on a pour tout entier $n\geq 2$ : 
\begin{equation}
u_{n} = u_{1} + \dfrac{1}{9}\left[ (u_{n-1}-u_{0}) + \Sum{k =
2}{n}\dfrac{3}{4k^{2}-1}u_{k-2}\right] \label{6}
\end{equation}et en déduire, pour tout entier naturel $n$, que
$u_{n}\leq \dfrac{6}{5}
$.

\item Montrer que la suite $(u_{n})$ est convergente, et, à l'aide de
(\ref{6}), donner un encadrement de sa limite $L$ permettant d'en
obtenir une
valeur décimale approchée à $0,01$ près.
\end{noliste}

\item On reprend dans cette question les notations de la première
partie et,
pour tout entier naturel $n$ et pour tout réel $t>0$, on pose : 
\[
u_{n}(t) = \dfrac{2^{n}}{e^{nt}}P_{n}(\dfrac{e^{t} + e^{-t}}{{2}})
\]

\begin{noliste}{a)}
 \setlength{\itemsep}{2mm}
\item Montrer que, pour tout réel $x>1$, il existe un nombre réel $t>0$
et
un seul tel que : 
\[
x = \dfrac{e^{t} + e^{-t}}{2}
\]

\item Calculer $u_{0}(t)$ et $u_{1}(t)$, et montrer que, pour tout
$n\geq 2$ :\begin{equation}
u_{n}(t)-u_{n-1}(t) = e^{-2t}[(u_{n-1}(t)-u_{n-2}(t)) +
\dfrac{3}{4n^{2}-1}u_{n-2}(t)] \label{7}
\end{equation}

\item Montrer par récurrence que, pour tout $n\geq 1$, les fonctions
$u_{n}$et $(u_{n}-u_{n-1})$ sont strictement positives et décroissantes
sur $]0, + \infty \lbrack $.

\item Expliciter le réel $e^{t}$ quand $x = \dfrac{5}{3}$, et montrer
que,
dans ce cas, les suites $(u_{n})$ et $(u_{n}(t))$ définies
respectivement
par les relations (\ref{5}) et (\ref{7}) sont les mêmes. \\
On suppose que $x\geq \dfrac{5}{3}$. Déduire des résultats précédents
que la suite $(u_{n}(t))$ converge vers une limite strictement positive
$L(x) $ que l'on ne demande pas d'expliciter, et en déduire un
équivalent de $P_{n}(x)$ lorsque n tend vers l'infini.
\end{noliste}
\end{noliste}

\label{fin}

\end{document}

