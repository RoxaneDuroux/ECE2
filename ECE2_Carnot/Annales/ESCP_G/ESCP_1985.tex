\documentclass[11pt]{article}%
\usepackage{geometry}%
\geometry{a4paper,
 lmargin = 2cm,rmargin = 2cm,tmargin = 2.5cm,bmargin = 2.5cm}

\input{../../macros.tex}

\pagestyle{fancy} %
\lhead{ECE2 \hfill Mathématiques\\
} %
\chead{\hrule} %
\rhead{} %
\lfoot{} %
\cfoot{} %
\rfoot{\thepage} %

\renewcommand{\headrulewidth}{0pt}% : Trace un trait de séparation
 % de largeur 0,4 point. Mettre 0pt
 % pour supprimer le trait.

\renewcommand{\footrulewidth}{0.4pt}% : Trace un trait de séparation
 % de largeur 0,4 point. Mettre 0pt
 % pour supprimer le trait.

\setlength{\headheight}{14pt}

\title{\bf \vspace{-2cm} ESCP 1985 - voie Générale} %
\author{} %
\date{} %
\begin{document}

\maketitle %
\vspace{-1.4cm}\hrule %
\thispagestyle{fancy}

\vspace*{.2cm}


% DEBUT DU DOC À MODIFIER : tout virer jusqu'au début de l'exo


\begin{center}
{\small CHAMBRE D\E\ COMMERCE ET D'INDUSTRIE DE PARIS}

\textbf{DIRECTION DE L'ENSEIGNEMENT}

Direction des Admissions et concours

\underline{\hspace*{3cm}}

{\Large ECOLE DES\ HAUTES\ ETUDES\ COMMERCIALES}

{\Large E.S.C.P.-E.A.P.}

{\Large ECOL\E\ SUPERIEUR\E\ D\E\ COMMERC\E\ D\E\ LYON}{\large }

CONCOURS D'ADMISSION\ SUR\ CLASSES\ PREPARATOIRES

\underline{\hspace*{3cm}}

\textbf{OPTION GENERALE}

{\Large MATHEMATIQUES I}

\textbf{Année 1985}

\underline{\hspace*{3cm}}
\end{center}

\begin{quotation}
\noindent \textsl{La présentation, la lisibilité, l'orthographe, la
qualité
de la rédaction, la clarté et la précision des raisonnements entreront
pour
une part importante dans l'appréciation des copies.}

\noindent \textsl{Les candidats sont invités à encadrer dans la mesure
du
possible les résultats de leurs calculs.}

\noindent \textsl{Ils ne doivent faire usage d'aucun document :
l'utilisation de toute calculatrice et de tout matériel électronique
est
interdite.}

\noindent \textsl{Seule l'utilisation d'une règle graduée est
autorisée.}

\noindent \textsl{\hrulefill }
\end{quotation}

\noindent On désigne par $a$ un nombre réel strictement supérieur à
$1$.
Pour tout nombre entier naturel non nul, on note $g_{n}$ la fonction
définie
sur $\R_{+}$ par la relation : $g_{n}(x) = x(x-1)(x-2)...(x-n)a^{-x}$\\
L'objet du problème est d'étudier la maximum de la fonction $g_{n}$ sur
l'intervalle $[n, + \infty \lbrack $.

\section*{Partie I}

Dans cette partie, on examine le cas particulier où $n = 1$.

\begin{noliste}{1.}
 \setlength{\itemsep}{4mm}
\item 

\begin{noliste}{a)}
 \setlength{\itemsep}{2mm}
\item Étudier la variation de la fonction $\dfrac{g_{1}{\prime
}}{g_{1}} $.
On notera $u$ et $v$ les valeurs de $x$ où cette fonction s'annule,
avec $u<v $.

\item Dresser le tableau de variation de $g_{1}$. Étudier la branche
infinie
du graphe de $g_{1}$.
\end{noliste}

\item Dans cette question on prend $a = 2$.

\begin{noliste}{a)}
 \setlength{\itemsep}{2mm}
\item Calculer des valeurs approchées à $10^{-5}$ de $u,v,g_{1}(u)$et
$g_{1}(v)$.

\item Construire le graphe de $g_{1}$.
\end{noliste}
\end{noliste}

\section*{Partie II}

Dans cette partie, on considère une fonction réelle $f$ de classe
$C^{2}$
sur l'intervalle $[-1,1]$. On désigne par $M$ un nombre réel tel que,
pour
tout élément $x$ de $[-1;1]$, $\left| f"(x)\right| \leq M$.
Soit $\beta $ un élément de $\R_{+}$. On se propose d'approcher
l'intégrale $\dint{0}{1}f(t)dt$ par la somme $\dfrac{1}{n}\Sum{k =
1}{n}f(\dfrac{k-\beta }{n})$. On suppose que $n\geq 
\beta $. Pour tout nombre entier naturel $k$ tel que $1\leq k\leq n
$, on pose : 
\[
R_{1}(k,n) = f(\dfrac{k}{n})-f(\dfrac{k-\beta }{n})-\dfrac{\beta
}{n}f^{\prime
}(\dfrac{k}{n})\qquad R_{2}(k,n) =
\dint{\dfrac{k-1}{n}}{\dfrac{k}{n}}f(t)dt-\dfrac{1}{n}f(\dfrac{k}{n}) +
\dfrac{1}{2n^{2}}f^{\prime }(\dfrac{k}{n})
\]

\begin{noliste}{1.}
 \setlength{\itemsep}{4mm}
\item En appliquant l'inégalité de Taylor-Lagrange à des fonctions
convenables au point $\dfrac{k}{n}$, déterminer des nombres réels $A$
et $B$
tels que, quels que soient $n$ et $k$,
\[
\left| R_{1}(k,n)\right| \leq \dfrac{A}{n^{2}}\qquad \text{et\qquad
}\left| R_{2}(k,n)\right| \leq \dfrac{B}{n^{3}}
\]

\item 

\begin{noliste}{a)}
 \setlength{\itemsep}{2mm}
\item On pose : $R_{3}(k,n) =
\dint{\dfrac{k-1}{n}}{\dfrac{k}{n}}f(t)dt-\dfrac{1}{n}f(\dfrac{k-\beta
}{n})-\dfrac{\beta
-\dfrac{1}{2}}{n}(f(\dfrac{k}{n})-f(\dfrac{k-1}{n}))$\\
Déduire de la question précédente un nombre réel $C$ tel que, quels que
soient $n$ et $k$,
\[
R_{3}(k,n)\leq \dfrac{C}{n^{3}}
\]

\item On pose : $\Delta_{n}\dint{0}{1}f(t)dt-\dfrac{1}{n}\Sum{k =
1}{n}f(\dfrac{k-\beta }{n})-\dfrac{\beta
-\dfrac{1}{2}}{n}(f(1)-f(0))$\\
Prouver que : $\left| \Delta_{n}\right| \leq \dfrac{C}{n^{2}}$
\end{noliste}
\end{noliste}

\section*{Partie III}

On revient à l'étude de la fonction $g_{n}$ dans le cas général.

\begin{noliste}{1.}
 \setlength{\itemsep}{4mm}
\item 

\begin{noliste}{a)}
 \setlength{\itemsep}{2mm}
\item Pour tout nombre réel $x$ strictement supérieur à $n$, calculer :
$h_{n}(x) = \dfrac{g_{n}{\prime }(x)}{g_{n}(x)}$.

\item Montrer que sur l'intervalle $]n, + \infty \lbrack $, la dérivée
de $g_{n}$ s'annule en un point $x_{n}$ et un seul. Étudier le signe de
$g_{n}{\prime }$ sur $]n, + \infty \lbrack $.\\
On note $M_{n} = g_{n}(x_{n})$.
\end{noliste}

\item Soit $\alpha $ un nombre réel strictement supérieur à $1$. On
considère la fonction $f_{\alpha }$ définie par la relation :
\[
f_{\alpha }(x) = \dfrac{1}{\alpha -x}
\]

\begin{noliste}{a)}
 \setlength{\itemsep}{2mm}
\item Déterminer en fonction de $a$ la valeur de $\alpha $ pour
laquelle :
\[
h_{n}(n\alpha ) = \dfrac{1}{n\alpha } + \dfrac{1}{n}\Sum{k =
1}{n}f_{\alpha }(\dfrac{k}{n})-\dint{\alpha }{1}f_{\alpha }(t)dt
\]
Dans toute la suite du problème, on donne à $\alpha $ la valeur ainsi
déterminée. (on contrôlera que si $a = 2$, alors $\alpha = 2$).

\item Vérifier que : $h_{n}(n\alpha + \beta ) = \dfrac{1}{n\alpha
}-\dfrac{\beta -\dfrac{1}{2}}{n}(f_{\alpha }(1)-f_{\alpha
}(0))-\Delta_{n}$ où $\Delta_{n}$ a été défini dans la question II-2
(avec ici $f = f_{\alpha }$).\\
Déterminer la limite lorsque $n$ tend vers $ + \infty $ de
$nh_{n}(n\alpha
 + \beta )$.
\end{noliste}

\item Montrer qu'à partir d'un certain rang, $n\alpha \leq
x_{n}\leq (n + 1)\alpha $.\\
À cet effet, on étudiera les signes de $h_{n}(n\alpha )$ et de
$h_{n}((n + 1)\alpha )$.

\item 

\begin{noliste}{a)}
 \setlength{\itemsep}{2mm}
\item On se propose de déterminer la limite lorsque $n$ tend vers $ +
\infty $
de :
\[
y_{n} = \dfrac{1}{n}\ln (\prod\limits_{k = 0}{n}\dfrac{x_{n}-k}{n})
\]
On encadrera $y_{n}$ à l'aide de l'encadrement de $x_{n}$ obtenu
précédemment. On utilisera le résultat de la question II-2-b avec une
fonction $f$
convenable et $\beta = 0$, puis $\beta = \alpha $.\\
En conclure que : $\underset{n\rightarrow + \infty }{\lim }y_{n} =
\dint{0}{1}\ln (\alpha -x)dx$

\item Calculer cette intégrale.

\item Montrer finalement que la suite de terme général
$\dfrac{1}{n}(M_{n})^{\frac{1}{n}}$ converge et déterminer sa limite.
On contrôlera que pour $a = 2$
cette limite est égale à $\dfrac{1}{e}$.
\end{noliste}
\end{noliste}

\label{fin}

\end{document}

