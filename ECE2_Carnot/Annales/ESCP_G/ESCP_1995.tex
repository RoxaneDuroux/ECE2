\documentclass[11pt]{article}%
\usepackage{geometry}%
\geometry{a4paper,
 lmargin = 2cm,rmargin = 2cm,tmargin = 2.5cm,bmargin = 2.5cm}

\input{../../macros.tex}

\pagestyle{fancy} %
\lhead{ECE2 \hfill Mathématiques\\
} %
\chead{\hrule} %
\rhead{} %
\lfoot{} %
\cfoot{} %
\rfoot{\thepage} %

\renewcommand{\headrulewidth}{0pt}% : Trace un trait de séparation
 % de largeur 0,4 point. Mettre 0pt
 % pour supprimer le trait.

\renewcommand{\footrulewidth}{0.4pt}% : Trace un trait de séparation
 % de largeur 0,4 point. Mettre 0pt
 % pour supprimer le trait.

\setlength{\headheight}{14pt}

\title{\bf \vspace{-2cm} ESCP 1995 - voie Générale} %
\author{} %
\date{} %
\begin{document}

\maketitle %
\vspace{-1.4cm}\hrule %
\thispagestyle{fancy}

\vspace*{.2cm}


% DEBUT DU DOC À MODIFIER : tout virer jusqu'au début de l'exo


\begin{center}
{\small CHAMBRE D\E\ COMMERCE ET D'INDUSTRIE DE PARIS}

\textbf{DIRECTION DE L'ENSEIGNEMENT}

Direction des Admissions et concours

\underline{\hspace*{3cm}}

{\Large ECOLE DES\ HAUTES\ ETUDES\ COMMERCIALES}

{\Large E.S.C.P.-E.A.P.}

{\Large ECOL\E\ SUPERIEUR\E\ D\E\ COMMERC\E\ D\E\ LYON}{\large }

CONCOURS D'ADMISSION\ SUR\ CLASSES\ PREPARATOIRES

\underline{\hspace*{3cm}}

\textbf{OPTION GENERALE}

{\Large MATHEMATIQUES I}

\textbf{Année 1995}

\underline{\hspace*{3cm}}
\end{center}

\begin{quotation}
\noindent \textsl{La présentation, la lisibilité, l'orthographe, la
qualité
de la rédaction, la clarté et la précision des raisonnements entreront
pour
une part importante dans l'appréciation des copies.}

\noindent \textsl{Les candidats sont invités à encadrer dans la mesure
du
possible les résultats de leurs calculs.}

\noindent \textsl{Ils ne doivent faire usage d'aucun document :
l'utilisation de toute calculatrice et de tout matériel électronique
est
interdite.}

\noindent \textsl{Seule l'utilisation d'une règle graduée est
autorisée.}

\noindent \textsl{\hrulefill }
\end{quotation}

L'objet du problème est l'étude de la fonction $f$ définie sur $[0, +
\infty
\lbrack $ par les relations : 
\[
f(x) = {\dfrac{x}{e^{x}-1}}\quad \text{si}\quad {}x>0,\qquad {}f(0) = 1
\]

Dans la partie II, on établit l'existence des \textit{moments} $I_{p} =
\dint{0}{+ \infty }x^{p}f(x)dx$ où $p$ est un entier naturel,
puis on exprime ces moments en fonction des séries $A_{p} = \Sum{k =
1}{+ \infty }${$\dfrac{1}{k^{p + 2}}$}.

Dans la partie III, on établit un procédé d'approximation des nombres
$A_{p}$.

\section*{Partie I}

\begin{noliste}{1.}
 \setlength{\itemsep}{4mm}
\item 

\begin{noliste}{a)}
 \setlength{\itemsep}{2mm}
\item Étudier la continuité de $f$ sur $[0, + \infty \lbrack $.

\item Quelle est la limite de $f$ en $ + \infty $ ?
\end{noliste}

\item 

\begin{noliste}{a)}
 \setlength{\itemsep}{2mm}
\item Calculer la dérivée $f^{\prime }$ de $f$ sur $]0, + \infty
\lbrack $.

\item Étudier la dérivabilité de $f$ en 0.

\item La fonction $f$ est-elle de classe $C^{1}$ sur $[0, + \infty
\lbrack $ ?
\end{noliste}

\item Étudier les variations de la fonction $\varphi $ définie sur $[0,
+ \infty \lbrack $ par 
\[
\varphi (x) = 1-x-e^{-x}
\]
En déduire le signe de $f^{\prime }(x)$.

\item Étudier les variations de la fonction $\psi $ définie sur $[0, +
\infty
\lbrack $ par 
\[
\psi (x) = (x + 2) + (x-2)e^{x}
\]
En déduire le signe de $f^{\prime \prime }(x)$ pour $x>0$.

\item Donner une représentation graphique de $f$.
\end{noliste}

\section*{Partie II}

Dans cette partie et jusqu'à la fin du problème, $p$ désigne un entier
naturel.

\begin{noliste}{1.}
 \setlength{\itemsep}{4mm}
\item Dans cette question, $\lambda $ est un réel strictement positif.

\begin{noliste}{a)}
 \setlength{\itemsep}{2mm}
\item Établir la convergence de l'intégrale :
\[
K(p,\lambda ) = \dint{0}{+ \infty }x^{p}e^{-\lambda x}dx
\]
et calculer $K(0,\lambda )$.

\item Établir une relation simple entre $K(p,\lambda )$ et $K(p +
1,\lambda )$\\
\textit{On utilisera une intégration par parties}.

\item En déduire par récurrence la valeur de $K(p,\lambda )$.
\end{noliste}

\item 

\begin{noliste}{a)}
 \setlength{\itemsep}{2mm}
\item Montrer que, pour $x\in \lbrack 0, + \infty \lbrack $, 
\[
f(x)\leq e^{-{\dfrac{x}{2}}}
\]

\item Démontrer que l'intégrale $\dint{0}{+ \infty }x^{p}f(x)dx$
converge.\\
Dans toute la suite du problème, on pose : 
\[
I_{p} = \dint{0}{+ \infty }x^{p}f(x)dx = \dint{0}{+ \infty
}{\dfrac{x^{p + 1}}{e^{x}-1}}dx
\]
\end{noliste}

\item Calcul de $I_{p}$.

\begin{noliste}{a)}
 \setlength{\itemsep}{2mm}
\item Établir la convergence de la série $\Sum{k = 1}{+ \infty
}${$\dfrac{1}{k^{p + 2}}$}. On note 
\[
A_{p} = \Sum{k = 1}{+ \infty }{\dfrac{1}{k^{p + 2}}}
\]

\item Pour $x\in \ ]0, + \infty \lbrack $ et $n$ entier supérieur ou
égal à 1, établir que 
\[
{\dfrac{1}{e^{x}-1}} = \Sum{k = 1}{n}e^{-kx} +
{\dfrac{e^{-nx}}{e^{x}-1}}
\]

\item En déduire que : 
\[
I_{0} = \Sum{k = 1}{n}{\dfrac{1}{k^{2}}} + \dint{0}{+ \infty
}f(x)e^{-nx}dx
\]

\item Exprimer $I_{0}$ à l'aide de $A_{0}$.

\item En adaptant la méthode précédente, exprimer $I_{p}$ en fonction
de $A_{p}$.
\end{noliste}
\end{noliste}

\section*{Partie III}

On étudie une méthode de calcul approché de $A_{p}$.

\begin{noliste}{1.}
 \setlength{\itemsep}{4mm}
\item Première approximation

\begin{noliste}{a)}
 \setlength{\itemsep}{2mm}
\item Soit $x$ un nombre réel et $g$ une fonction de classe $C^{2}$
définie
sur $\left[ x-{\dfrac{1}{2}},x + {\dfrac{1}{2}}\right] $ à valeurs
réelles.
Établir la relation : 
\[
g(x) = \dint{x-{\frac{1}{2}}}{x +
{\frac{1}{2}}}g(t)\,dt-{\dfrac{1}{2}}\dint{x}{x +
{\frac{1}{2}}}(t-x-{\dfrac{1}{2}})^{2}g^{\prime \prime
}(t)\,dt-{\dfrac{1}{2}}\dint{x-{\frac{1}{2}}}{x}(t-x +
{\dfrac{1}{2}})^{2}g^{\prime \prime }(t)\,dt
\]
\textit{On pourra intégrer par parties les deux dernières intégrales
apparaissant dans la formule}.

\item En déduire que pour $k$ entier supérieur ou égal à 1, on a : 
\[
\left| {\dfrac{1}{k^{p + 2}}}-\dint{k-{\frac{1}{2}}}{k +
{\frac{1}{2}}}{\frac{dt}{t^{p + 2}}}\right| \leq {\dfrac{(p + 2)(p +
3)}{24(k-{\dfrac{1}{2}})^{p + 4}}}
\]

\item Soit $n$ un entier supérieur ou égal à 1, montrer que : 
\[
\Sum{k = n + 1}{+ \infty }{\dfrac{1}{(k-{\dfrac{1}{2}})^{p + 4}}}\leq 
{\dfrac{1}{(p + 3)(n-{\dfrac{1}{2}})^{p + 3}}}
\]
et en déduire, à l'aide de (b), que : 
\[
\left| \Sum{k = 1}{+ \infty }{\dfrac{1}{k^{p + 2}}}-{\dfrac{1}{(p +
1)(n + {\dfrac{1}{2}})^{p + 1}}}\right| \leq {\dfrac{p +
2}{24(n-{\dfrac{1}{2}})^{p + 3}}}
\]

\item Exemple.\\
On pose :\qquad $u_{n} = \Sum{k = 1}{n}{\dfrac{1}{k^{4}}} +
{\dfrac{1}{3(n + {\dfrac{1}{2}})^{3}}}$.\\
Proposer un majorant de $\left| A_{2}-u_{n}\right| $.\\
Pour quelle valeur minimale de $n$ peut-on affirmer que : $\left|
A_{2}-u_{n}\right| <10^{-6}$ ?
\end{noliste}

\item Deuxième approximation

\begin{noliste}{a)}
 \setlength{\itemsep}{2mm}
\item On reprend les notations et les hypothèses de la question
(III-1.a) et
on suppose de plus que $g$ est de classe $C^{4}$. Montrer que :
\[
g(x) = \dint{x-{\frac{1}{2}}}{x +
{\frac{1}{2}}}g(t)\,dt-{\dfrac{g^{\prime \prime
}(x)}{24}}-{\dfrac{1}{24}}\left( \dint{x}{x +
{\frac{1}{2}}}(t-x-{\dfrac{1}{2}})^{4}g^{(4)}(t)\,dt +
\dint{x-{\frac{1}{2}}}{x}(t-x +
{\dfrac{1}{2}})^{4}g^{(4)}(t)\,dt\right)
\]

\item En déduire que pour $k$ entier supérieur ou égal à 1 : 
\[
\left| {\dfrac{1}{k^{p + 2}}}-\dint{k-{\frac{1}{2}}}{k +
{\frac{1}{2}}}{\dfrac{dt}{t^{p + 2}}} + {\dfrac{(p + 2)(p + 3)}{24k^{p
+ 4}}}\right| \leq 
{\dfrac{(p + 2)(p + 3)(p + 4)(p + 5)}{1\,920(k-{\dfrac{1}{2}})^{p +
6}}}
\]

\item Soit $n$ un entier supérieur ou égal à 1. Montrer que :
\[
\left| \Sum{k = n + 1}{+ \infty }{\dfrac{1}{k^{p + 2}}}-{\dfrac{1}{(p +
1)(n + {\dfrac{1}{2}})^{p + 1}}} + {\dfrac{(p + 2)(p + 3)}{24}}\Sum{k =
n + 1}{+ \infty }{\dfrac{1}{k^{p + 4}}}\right| \leq {\dfrac{(p + 2)(p +
3)(p + 4)}{1\,920(n-{\dfrac{1}{2}})^{p + 5}}}
\]

\item Pour $n$ entier supérieur ou égal à 1, on pose :
\[
v_{n} = \Sum{k = 1}{n}{\dfrac{1}{k^{p + 2}}} + {\dfrac{1}{(p + 1)(n +
{\dfrac{1}{2}})^{p + 1}}}-{\dfrac{p + 2}{24(n + {\dfrac{1}{2}})^{p +
3}}}
\]
En utilisant les résultats des questions III.1.c) et III.2.c), proposer
un
majorant de $\left| A_{p}-v_{n}\right| $.

\item Exemple. On fait $p = 0$.\\
Pour quelle valeur minimale de $n$ peut-on affirmer que :\quad $\left|
A_{0}-v_{n}\right| \leq 10^{-6}$ ?
\end{noliste}
\end{noliste}

\label{fin}

\end{document}

