\documentclass[11pt]{article}%
\usepackage{geometry}%
\geometry{a4paper,
 lmargin = 2cm,rmargin = 2cm,tmargin = 2.5cm,bmargin = 2.5cm}

\input{../../macros.tex}

\pagestyle{fancy} %
\lhead{ECE2 \hfill Mathématiques\\
} %
\chead{\hrule} %
\rhead{} %
\lfoot{} %
\cfoot{} %
\rfoot{\thepage} %

\renewcommand{\headrulewidth}{0pt}% : Trace un trait de séparation
 % de largeur 0,4 point. Mettre 0pt
 % pour supprimer le trait.

\renewcommand{\footrulewidth}{0.4pt}% : Trace un trait de séparation
 % de largeur 0,4 point. Mettre 0pt
 % pour supprimer le trait.

\setlength{\headheight}{14pt}

\title{\bf \vspace{-2cm} ESCP 1999 - voie Générale} %
\author{} %
\date{} %
\begin{document}

\maketitle %
\vspace{-1.4cm}\hrule %
\thispagestyle{fancy}

\vspace*{.2cm}


% DEBUT DU DOC À MODIFIER : tout virer jusqu'au début de l'exo


\begin{center}
{\small CHAMBRE D\E\ COMMERCE ET D'INDUSTRIE DE PARIS}

\textbf{DIRECTION DE L'ENSEIGNEMENT}

Direction des Admissions et concours

\underline{\hspace*{3cm}}

{\Large ECOLE DES\ HAUTES\ ETUDES\ COMMERCIALES}

{\Large E.S.C.P.-E.A.P.}

{\Large ECOL\E\ SUPERIEUR\E\ D\E\ COMMERC\E\ D\E\ LYON}{\large }

CONCOURS D'ADMISSION\ SUR\ CLASSES\ PREPARATOIRES

\underline{\hspace*{3cm}}

\textbf{OPTION SCIENTIFIQUE}

{\Large MATHEMATIQUES I}

\textbf{Année 1999}

\underline{\hspace*{3cm}}
\end{center}

\begin{quotation}
\noindent \textsl{La présentation, la lisibilité, l'orthographe, la
qualité
de la rédaction, la clarté et la précision des raisonnements entreront
pour
une part importante dans l'appréciation des copies.}

\noindent \textsl{Les candidats sont invités à encadrer dans la mesure
du
possible les résultats de leurs calculs.}

\noindent \textsl{Ils ne doivent faire usage d'aucun document :
l'utilisation de toute calculatrice et de tout matériel électronique
est
interdite.}

\noindent \textsl{Seule l'utilisation d'une règle graduée est
autorisée.}

\noindent \textsl{\hrulefill }
\end{quotation}

\noindent Dans tout le problème l'espérance d'une variable aléatoire
$Y$
sera notée $\E(Y)$. Tous les polynômes de ce problème sont à
coefficients réels.\\
Pour tout entier naturel $k$, on note $E_{k}$ l'espace vectoriel des
polynômes de degré au plus $k$. A tout entier naturel $n$ non nul et à
toute suite
$(s_{0},s_{1},\dots,s_{2n})$ de $2n + 1$ réels, on associe les
applications $\Phi_{n}$ et $S_{n}$ définies de la manière suivante :\\
pour tout élément $(A,B)$ de $E_{n}\times E_{n}$ avec $A = \Sum{i =
0}{n}a_{i}X^{i}$ et $B = \Sum{j = 0}{n}b_{j}X^{j}$, on pose
\[
\Phi
_{n}(A,B) = \Sum{i = 0}{n}\Sum{j = 0}{n}a_{i}b_{j}s_{i + j} =
\Sum{0\leq
i,j\leq n}a_{i}b_{j}s_{i + j}
\]
et, pour tout polynôme $C$ élément de $E_{2n}$, avec $C = \Sum{i =
0}{2n}c_{i}X^{i}$, on pose $S_{n}(C) = \Sum{i = 0}{2n}c_{i}s_{i}$.

\begin{noliste}{1.}
 \setlength{\itemsep}{4mm}
\item 

\begin{noliste}{a)}
 \setlength{\itemsep}{2mm}
\item Vérifier que, pour tout entier naturel $n$, $\Phi_{n}$ est une
forme
bilinéaire symétrique sur $E_{n}\times E_{n}$.

\item Vérifier que, pour tout entier naturel $n$, $S_{n}$ est une forme
linéaire sur $E_{2n}$ et, pour tout élément $(A,B)$ de $E_{n}\times
E_{n}$, prouver l'égalité : $\Phi_{n}(A,B) = S_{n}(AB)$ (on
commencera par considérer le cas où $A = X^{i}$ et $B = X^{j}$ avec
$0\leq i,j\leq n$.)
\end{noliste}

\item Deux cas particuliers

\begin{noliste}{a)}
 \setlength{\itemsep}{2mm}
\item Dans cette sous-question on suppose que $n = 1$ et $s_{0} = 1$,
$s_{1}$ et
$s_{2}$ étant quelconques. Pour tout élément $(a,b)$ de $\R^{2}$
vérifier l'égalité
\[
\Phi_{1}(aX + b,\;aX + b) = (b + as_{1})^{2} + a^{2}(s_{2}-s_{1}{2})
\]
En déduire une condition nécessaire et suffisante, portant sur les
réels $s_{1}$ et $s_{2}$, pour que l'application $\Phi_{1}$ soit un
produit
scalaire sur $E_{1}\times E_{1}$.

\item Dans cette sous-question on suppose que $n = 2$, $s_{0} = 1$ et
$s_{1} = s_{3} = 0$, $s_{2}$ et $s_{4}$ étant quelconques. Prouver que
l'application $\Phi_{2}$, associée à un tel choix de
$(s_{0},s_{1},s_{2},s_{3},s_{4})$, est un produit scalaire sur
$E_{2}\times
E_{2}$ si et seulement si les réels $s_{2}$ et $s_{4}$ vérifient les
conditions suivantes : $s_{2}>0$ et $s_{4}-s_{2}{2}>0$.
\end{noliste}

\item Deux exemples\\
Dans cette question on considère un entier naturel $n$ non nul.

\begin{noliste}{a)}
 \setlength{\itemsep}{2mm}
\item Dans cette sous-question, on se donne un entier naturel $d$ non
nul et
une variable aléatoire discrète $Y$, prenant $d$ valeurs distinctes
$\alpha
_{1},\alpha_{2},\dots,\alpha_{d}$, avec les probabilités, strictement
positives, respectives $p_{1},p_{2},\dots,p_{d}$, et on pose, pour tout
entier naturel $k$
\[
s_{k} = E(Y^{k}) = \Sum{i = 1}{d}\alpha_{i}{k}p_{i}
\]
On considère les applications $\Phi_{n}$ et $S_{n}$ associées à ce
choix de
$(s_{0},s_{1},\dots,s_{2n})$

\begin{nonoliste}{(i)}
\item Pour tout polynôme $Q$ de $E_{2n}$, vérifier l'égalité :
$S_{n}(Q) = \Sum{i = 1}{d}Q(\alpha_{i})p_{i}$.

\item En déduire une condition nécessaire et suffisante, portant sur
$n$ et $d$, pour que l'application $\Phi_{n}$ soit un produit scalaire
sur $E_{n}\times E_{n}$.
\end{nonoliste}

\item 

\begin{nonoliste}{(i)}
\item Dans cette sous-question, on considère une variable aléatoire $Y$
dont
une densité $f$ est continue sur le segment $[0,1]$ et nulle en dehors
de $[0,1]$.\\
On pose, pour tout entier naturel $k$,
\[
s_{k} = E(Y^{k}) = \dint{0}{1}t^{k}f(t)dt
\]
Vérifier que l'application $\Phi_{n}$, associée à ce choix de
$(s_{0},s_{1},\dots,s_{2n})$ est un produit scalaire sur $E_{n}\times
E_{n}$.

\item Montrer que, dans le cas où $(s_{0},s_{1},\dots,s_{2n}) =
(1,\dfrac{1}{2},\dots,\dfrac{1}{2n + 1})$, l'application $\Phi_{n}$,
associée à ce choix,
est un produit scalaire sur $E_{n}\times E_{n}$.
\end{nonoliste}
\end{noliste}

\item Dans cette question on revient au cas général où on considère un
entier naturel $n$ non nul, une suite $(s_{0},s_{1},\dots,s_{2n})$ de
$2n + 1$
réels et les applications $\Phi_{n}$ et $S_{n}$ associées à cette
suite.\\
On admet le résultat suivant : tout polynôme $P$ peut s'écrire sous la
forme
\[
P = \lambda \prod_{i = 1}{r}(X-\zeta_{i})^{m_{i}}\prod_{j = 1}{\ell
}(X^{2} + b_{j}X + c_{j})
\]
où $r$ et $\ell $ sont des entiers naturels (avec la convention que si
$r$
ou $\ell $ est nul, le produit correspondant vaut $1$), où $\lambda $
est un
réel, où, si $r$ est non nul, $\zeta_{1},\zeta_{2},\dots,\zeta_{r}$
sont
les racines réelles distinctes de $P$, de multiplicités respectives
$m_{1},m_{2},\dots,m_{r}$, et où, si $\ell $ est non nul,
$b_{1},b_{2},\dots,b_{\ell },c_{1},c_{2},\dots,c_{\ell }$ sont des
réels vérifiant $b_{j}{2}-4c_{j}<0$ pour tout entier $j$ tel que $1\leq
j\leq \ell $.\\
 Un polynôme \textbf{non nul} $P$, à
coefficients réels, est dit positif si, pour tout réel $x$,
$P\left(\Ev{x}\right)\geq 0$.

\begin{noliste}{a)}
 \setlength{\itemsep}{2mm}
\item Montrer que la multiplicité d'une racine réelle d'un polynôme
positif
est paire.

\item Montrer que tout polynôme $P$ positif de degré $2$ est somme de
deux
carrés de polynômes, c'est-à-dire qu'il existe un couple $(A,B)$ de
polynômes, tel que $P = A^{2} + B^{2}$.

\item En remarquant que, si $A$, $B$, $C$, $D$ sont quatre polynômes,
on a :
\[
(A^{2} + B^{2})(C^{2} + D^{2}) = (AC + BD)^{2} + (AD-BC)^{2}
\]
montrer que tout polynôme positif est somme de deux carrés de
polynômes.

\item Montrer que $\Phi_{n}$ est un produit scalaire sur $E_{n}\times
E_{n}$
si et seulement si, pour tout polynôme $P$ positif, élément de
$E_{2n}$, on
a : $S_{n}(P)>0$.
\end{noliste}

\item Dans cette question on suppose que $n = 2$ et
$(s_{0},s_{1},s_{2},s_{3},s_{4}) = \left(
1,\dfrac{1}{2},\dfrac{1}{3},\dfrac{1}{4},\dfrac{1}{5}\right) $.

\begin{noliste}{a)}
 \setlength{\itemsep}{2mm}
\item À l'aide du procédé d'orthonormalisation de Schmidt, construire,
à
partir de la base $(1,X,X^{2})$ une base orthonormale de $E_{2}$ pour
le
produit scalaire $\Phi_{2}$.

\item Pour tous $(a_{0},a_{1},a_{2})$ et $(b_{0},b_{1},b_{2})$,
éléments de $\R^{3}$, vérifier l'égalité :
\[
\Phi_{2}(a_{2}X^{2} + a_{1}X + a_{0},b_{2}X^{2} + b_{1}X + b_{0}) =
\text{ }{t}AMB
\]
où $A = 
\begin{smatrix}
a_{0} \\
a_{1} \\
a_{2}\end{smatrix}
$, $B = 
\begin{smatrix}
b_{0} \\
b_{1} \\
b_{2}\end{smatrix}
$, et $M = 
\begin{smatrix}
1 & \dfrac{1}{2} & \dfrac{1}{3}\\
\dfrac{1}{2} & \dfrac{1}{3} & \dfrac{1}{4}\\
\dfrac{1}{3} & \dfrac{1}{4} & \dfrac{1}{5}\end{smatrix}
$.

\item Déterminer une matrice $T$ triangulaire telle que : $^{t}TMT =
I_{3}$ ($I_{3}$ désignant la matrice identité d'ordre $3$).
\end{noliste}

\item Jusqu'à la fin du problème, on considère un entier naturel $n$
non
nul, une suite $(s_{0},\dots,s_{2n})$ de premier terme $s_{0} = 1$,
telle que
$\Phi_{n}$ soit un produit scalaire sur $E_{n}\times E_{n}$, et on note
$(P_{0},P_{1},\dots,P_{n})$ la base orthonormale de $E_{n}$ pour le
produit
scalaire $\Phi_{n}$ obtenue, par le procédé de Schmidt, à partir de la
base
$(1,X,\dots,X^{n})$, le polynôme $P_{i}$ étant de degré $i$ pour tout
entier $i$ compris entre $0$ et $n$.

\begin{noliste}{a)}
 \setlength{\itemsep}{2mm}
\item En considérant le nombre $\Phi_{n}(P_{n},1)$, prouver que le
polynôme
$P_{n}$ ne peut pas garder un signe fixe sur $\R$. En déduire que
$P_{n}$ possède au moins une racine réelle de multiplicité impaire.

\item On note $\alpha_{1},\alpha_{2},\dots,\alpha_{k}$ les racines
réelles de $P_{n}$ de multiplicité impaire. Montrer que $P_{n}$ s'écrit
sous
la forme $P_{n} = \varepsilon Q\prod\limits_{i = 1}{k}(X-\alpha_{i})$,
où $\varepsilon $ est élément de $\{-1,1\}$ et $Q$ est un polynôme
positif de $E_{n}$.\\
En considérant le nombre $\Phi_{n}(P_{n},\varepsilon
\prod\limits_{i = 1}{k}(X-\alpha_{i}))$, montrer que $k = n$.
\end{noliste}

\item On note $\alpha_{1},\alpha_{2},\dots,\alpha_{n}$ les $n$ racines
du polynôme $P_{n}$, réelles et distinctes deux à deux selon la
question précédente.\\
Pour tout élément $k$ de $\{1,2,\dots,n\}$, on note $L_{k}$ le polynôme
$L_{k} = \prod\limits_{\QATOP{1\leq i\leq n}{i\not =
k}}\dfrac{X-\alpha_{i}}{\alpha_{k}-\alpha_{i}}$.

\begin{noliste}{a)}
 \setlength{\itemsep}{2mm}
\item Montrer que $(L_{1},L_{2},\dots,L_{n})$ est une base de
$E_{n-1}$,
et, pour tout polynôme $R$ de $E_{n-1}$, justifier l'égalité : $R =
\Sum{i = 1}{n}R(\alpha_{i})L_{i}$. En déduire $\Sum{i = 1}{n}L_{i}$.

\item Soit $A$ un polynôme, élément de $E_{2n-1}$.

\begin{nonoliste}{(i)}
\item Justifier l'existence d'un couple $(Q,R)$ élément de
$E_{n-1}\times
E_{n-1}$ tel que $A = P_{n}Q + R$.

\item Vérifier que $S_{n}(A) = S_{n}(R)$, puis que $S_{n}(A) = \Sum{i =
1}{n}A(\alpha_{i})S_{n}(L_{i})$.
\end{nonoliste}

\item Pour tout élément $k$ de $\{1,2,\dots,n\}$, on pose $p_{k} =
S_{n}(L_{k})$.\\
Vérifier que $\Sum{k = 1}{n}p_{k} = 1$ et, en considérant
$S_{n}(L_{k}{2})$, montrer que $p_{k}>0$.

\item Déduire de ce qui précède qu'il existe une variable aléatoire
discrète
$Y$ vérifiant, pour tout élément $k$ de $\{0,1,\dots,2n-1\},\;s_{k} =
E(Y^{k}) $.

\item Déterminer la loi d'une telle variable aléatoire, dans le cas où
:
\[
n = 2\text{ et }(s_{0},s_{1},s_{2},s_{3},s_{4}) = \left(
1,\dfrac{1}{2},\dfrac{1}{3},\dfrac{1}{4},\dfrac{1}{5}\right)
\]
\end{noliste}
\end{noliste}

\label{fin}

\end{document}

