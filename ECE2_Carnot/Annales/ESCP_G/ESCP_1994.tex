\documentclass[11pt]{article}%
\usepackage{geometry}%
\geometry{a4paper,
 lmargin = 2cm,rmargin = 2cm,tmargin = 2.5cm,bmargin = 2.5cm}

\input{../../macros.tex}

\pagestyle{fancy} %
\lhead{ECE2 \hfill Mathématiques\\
} %
\chead{\hrule} %
\rhead{} %
\lfoot{} %
\cfoot{} %
\rfoot{\thepage} %

\renewcommand{\headrulewidth}{0pt}% : Trace un trait de séparation
 % de largeur 0,4 point. Mettre 0pt
 % pour supprimer le trait.

\renewcommand{\footrulewidth}{0.4pt}% : Trace un trait de séparation
 % de largeur 0,4 point. Mettre 0pt
 % pour supprimer le trait.

\setlength{\headheight}{14pt}

\title{\bf \vspace{-2cm} ESCP 1994 - voie Générale} %
\author{} %
\date{} %
\begin{document}

\maketitle %
\vspace{-1.4cm}\hrule %
\thispagestyle{fancy}

\vspace*{.2cm}


% DEBUT DU DOC À MODIFIER : tout virer jusqu'au début de l'exo


\begin{center}
{\small CHAMBRE D\E\ COMMERCE ET D'INDUSTRIE DE PARIS}

\textbf{DIRECTION DE L'ENSEIGNEMENT}

Direction des Admissions et concours

\underline{\hspace*{3cm}}

{\Large ECOLE DES\ HAUTES\ ETUDES\ COMMERCIALES}

{\Large E.S.C.P.-E.A.P.}

{\Large ECOL\E\ SUPERIEUR\E\ D\E\ COMMERC\E\ D\E\ LYON}{\large }

CONCOURS D'ADMISSION\ SUR\ CLASSES\ PREPARATOIRES

\underline{\hspace*{3cm}}

\textbf{OPTION GENERALE}

{\Large MATHEMATIQUES I}

\textbf{Année 1994}

\underline{\hspace*{3cm}}
\end{center}

\begin{quotation}
\noindent \textsl{La présentation, la lisibilité, l'orthographe, la
qualité
de la rédaction, la clarté et la précision des raisonnements entreront
pour
une part importante dans l'appréciation des copies.}

\noindent \textsl{Les candidats sont invités à encadrer dans la mesure
du
possible les résultats de leurs calculs.}

\noindent \textsl{Ils ne doivent faire usage d'aucun document :
l'utilisation de toute calculatrice et de tout matériel électronique
est
interdite.}

\noindent \textsl{Seule l'utilisation d'une règle graduée est
autorisée.}

\noindent \textsl{\hrulefill }
\end{quotation}

Dans tout le problème, on désigne par $W$ la fonction définie sur
$\R_{+}$ par : 
\[
W(x) = \dint{0}{\dfrac{\pi }{2}}\sin ^{x}t\,dt
\]
L'objectif est d'étudier cette fonction $W$ et d'en déduire quelques
applications.

\section*{Partie I : Étude d'une intégrale impropre}

{\normalsize On désigne par $f$ la fonction définie sur $]0,\dfrac{\pi
}{2}]$
par la relation : 
\[
f(x) = -\dint{x}{\dfrac{\pi }{2}}\ln (\sin t)\,dt
\]
}

\begin{noliste}{1.}
 \setlength{\itemsep}{4mm}
\item Montrer que, pour tout nombre réel $t$ appartenant à
$[0,\dfrac{\pi }{2}]$ :
\[
\sin t\geq \dfrac{2t}{\pi }
\]

\item À l'aide d'une intégration par parties, déterminer les primitives
de
la fonction : 
\[
t\rightarrow \ln \dfrac{2t}{\pi }
\]
En déduire la convergence et la valeur de l'intégrale : 
\[
-\dint{0}{\dfrac{\pi }{2}}\ln \dfrac{2t}{\pi }\,dt
\]

\item Établir que la fonction $f$ est monotone et bornée sur
$]0,\dfrac{\pi 
}{2}]$; en déduire la convergence de l'intégrale suivante : 
\[
L = -\dint{0}{\dfrac{\pi }{2}}\ln (\sin t)\,dt
\]

\item On se propose enfin de calculer $L$ à l'aide des deux intégrales
suivantes, dont la convergence résultera de la question a) ci-dessous :

\[
J = -\dint{\dfrac{\pi }{2}}{\pi }\ln (\sin t)\,dt\qquad
K = -\dint{0}{\dfrac{\pi }{2}}\ln (\cos t)\,dt
\]

\begin{noliste}{a)}
 \setlength{\itemsep}{2mm}
\item Obtenir des relations entre les intégrales $J$, $K$ et $L$ en
effectuant dans cette dernière les changements de variables $u = \pi
-t$ et $u = \pi /2-t$.

\item Exprimer $K + L$ en fonction de $L + J$ (on rappelle que $\sin
(2t) = 2\sin
t\cos t$).

\item En déduire les valeurs des intégrales $J$, $K$ et $L$.
\end{noliste}
\end{noliste}

\section*{Partie II : Étude de la fonction $W$}

\begin{noliste}{1.}
 \setlength{\itemsep}{4mm}
\item Soient $x$ et $y$ des nombres réels positifs tels que $x\leq y$.
Comparer $W(x)$ et $W(y)$ et en déduire le sens de variation de la
fonction $W$ sur $[0, + \infty \lbrack $.

\item On considère des nombres réels positifs $x$ et $x_{0}$.

\begin{noliste}{a)}
 \setlength{\itemsep}{2mm}
\item Montrer que, pour tout nombre réel positif $a$ : 
\[
\left| e^{-ax}-e^{-ax_{0}}\right| \leq a\left|
x-x_{0}\right|
\]

\item En déduire l'inégalité suivante : 
\[
\left| W(x)-W(x_{0})\right| \leq \dfrac{\pi \ln 2}{2}\,\left|
x-x_{0}\right|
\]

\item Établir la continuité de la fonction $W$ sur $[0, + \infty
\lbrack $
\end{noliste}

\item Pour tout nombre réel positif $x$, exprimer $W(x + 2)$ en
fonction de $W(x)$ à l'aide d'une intégration par parties (on écrira à
cet effet : $\sin
^{x + 2}t = \sin t\sin ^{x + 1}t$).

\item On se propose d'étudier le comportement asymptotique de $W$ à
l'aide
de la fonction auxiliaire $g$ définie pour $x\geq 0$ par : 
\[
g(x) = (x + 1)W(x + 1)W(x)
\]

\begin{noliste}{a)}
 \setlength{\itemsep}{2mm}
\item Établir que, pour tout nombre réel positif ou nul $x$, $g(x + 1)
= g(x)$.\\
En déduire la valeur de $g(n)$ pour tout entier naturel $n$.

\item Montrer que, pour tout nombre réel $x$ de $[0,1]$ et tout entier
naturel $n$ : 
\[
\dfrac{g(n + 1)}{n + 2}\leq \dfrac{g(x + n)}{x + n + 1}\leq
\dfrac{g(n)}{n + 1}
\]
En déduire en fonction de $n$ un encadrement de $g(x)$. En conclure que
$g$
est constante sur $[0, + \infty \lbrack $ (on explicitera son unique
valeur).

\item En remarquant que $W(x + 2)\leq W(x + 1)\leq W(x)$, montrer que
$W(x + 1)$ est équivalent à $W(x)$ lorsque $x$ tend vers $ + \infty $.

\item Déduire de ces résultats que, lorsque $x$ tend vers $ + \infty $
: 
\[
W(x)\sim \sqrt{\dfrac{\pi }{2x}}
\]
\end{noliste}
\end{noliste}

\section*{Partie III : Applications}

%TCIMACRO{%\TeXButton{TeX field}{\renewcommand\theenumi{\Alph{enumi}}
%\renewcommand\theenumii{\arabic{enumii}}
%\renewcommand\theenumiii{\alph{enumiii}}
%\renewcommand\labelenumii{\theenumii)}}}%BeginExpansion
\renewcommand\theenumi{\Alph{enumi}}
\renewcommand\theenumii{\arabic{enumii}}
\renewcommand\theenumiii{\alph{enumiii}}
\renewcommand\labelenumii{\theenumii)}%EndExpansion

\begin{noliste}{1.}
 \setlength{\itemsep}{4mm}
\item \textbf{Calcul de l'intégrale de Gauss} : 
\[
G = \dint{0}{+ \infty }\exp (-t^{2})\,dt
\]

\begin{noliste}{a)}
 \setlength{\itemsep}{2mm}
\item Montrer que, pour tout nombre réel strictement positif $x$ et
tout
nombre réel $u>-x$ : 
\[
\left( 1 + \dfrac{u}{x}\right) ^{x}\leq \exp u
\]
On pourra étudier pour $u>-x$ le signe de la fonction $u\rightarrow
u-x\ln
(1 + \dfrac{u}{x})$.

\item En intégrant l'inégalité précédente avec des valeurs convenables
de $u$, établir que, pour $x\geq 1$ : 
\[
\dint{0}{\sqrt{x}}\left( 1-\dfrac{t^{2}}{x}\right) ^{x}\,dt\leq
\dint{0}{\sqrt{x}}\exp (-t^{2})\,dt\leq
\dint{0}{+ \infty }\left( 1 + \dfrac{t^{2}}{x}\right) ^{-x}\,dt
\]

\item En posant respectivement $t = \sqrt{x}\cos u$ et $t =
\sqrt{x}\dfrac{\cos u}{\sin u}$ dans la première et la dernière de ces
intégrales, établir que,
pour $x\geq 1$ : 
\[
\sqrt{x}W(2x + 1)\leq \dint{0}{\sqrt{x}}\exp
(-t^{2})\,dt\leq \sqrt{x}W(2x-2)
\]

\item À l'aide de l'équivalent de $W$ obtenu dans la deuxième partie,
en déduire la convergence et la valeur de l'intégrale $G$, et retrouver
la valeur
de l'intégrale : 
\[
\dint{-\infty }{+ \infty }\exp \dfrac{-t^{2}}{2}\,dt
\]
\end{noliste}

\item \textbf{Calcul de valeurs approchées du nombre} $\pi $

\begin{noliste}{a)}
 \setlength{\itemsep}{2mm}
\item On pose pour tout réel positif $x$ : 
\[
h(x) = \dfrac{W(x + 1)}{W(x)}
\]

\begin{nonoliste}{(i)}
\item En remarquant que $W(x + 2)\leq W(x + 1)\leq W(x)$ et en
utilisant la relation établie à la question (2.3), établir que : 
\[
0\leq 1-h(x)\leq \dfrac{1}{x + 2}
\]

\item Exprimer $h(x)$ en fonction de $h(x-2)$ pour $x\geq 2$, et en
déduire que, pour tout nombre entier naturel $n$ : 
\[
h(2n) = \dfrac{r_{n}}{\pi }\qquad \text{avec}\qquad r_{n} = 2\prod_{k =
1}{n}\dfrac{4k^{2}}{4k^{2}-1}
\]
En déduire la limite de $r_{n}$ quand $n$ tende vers $ + \infty $ et un
encadrement de $\pi -r_{n}$.

\item Écrire en \Scilab{} un algorithme permettant le calcul de
$r_{n}$.\\
Donner des valeurs approchées de $r_{n}$ (avec 4 décimales) pour $n =
25$ et $n = 75$.
\end{nonoliste}

\item On se propose d'accélérer la convergence de la suite $(r_{n})$.

\begin{nonoliste}{(i)}
\item On pose pour tout entier $k\geq 2$ : 
\[
u_{k} = \ln r_{k}-\ln r_{k-1}
\]
En calculant $u_{n + 1} + u_{n + 2} + \cdots + u_{n + m}$ et en faisant
tendre $m$ vers 
$ + \infty $, établir que : 
\[
\ln \dfrac{\pi }{r_{n}} = -\Sum{k = n + 1}{+ \infty }\ln \left(
1-\dfrac{1}{4k^{2}}\right)
\]

\item En comparant une série à une intégrale, établir l'inégalité : 
\[
\dfrac{1}{n + 1}\leq \Sum{k = n + 1}{+ \infty }\dfrac{1}{k^{2}}\leq 
\dfrac{1}{n}
\]

\item Étudier les variations sur $]0,1[$ de la fonction $\varepsilon $
définie par : 
\[
\forall x\in \ ]0,1[,\quad \varepsilon (x) = -\dfrac{\ln (1-x) + x}{x}
\]
En déduire que, pour tout entier $k\geq n$ : 
\[
\dfrac{1}{4k^{2}}\leq -\ln \left( 1-\dfrac{1}{4k^{2}}\right) \leq
(1 + \varepsilon_{n})\,\dfrac{1}{4k^{2}}\qquad \text{avec}\qquad
\varepsilon
_{n} = \varepsilon \left( \dfrac{1}{4n^{2}}\right)
\]

\item Déduire des résultats précédents un équivalent de $\ln \dfrac{\pi
}{r_{n}}$ et prouver que : 
\[
\pi -r_{n}\sim \dfrac{r_{n}}{4n}
\]
Déduire enfin des valeurs approchées de $r_{n}$ obtenues précédemment
des
valeurs approchées de : 
\[
\left( 1 + \dfrac{1}{4n}\right) r_{n}
\]
(avec 4 décimales) pour $n = 25$ et $n = 75$.
\end{nonoliste}
\end{noliste}
\end{noliste}

\label{fin}

\end{document}

