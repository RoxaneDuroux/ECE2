\documentclass[11pt]{article}%
\usepackage{geometry}%
\geometry{a4paper,
 lmargin = 2cm,rmargin = 2cm,tmargin = 2.5cm,bmargin = 2.5cm}

\input{../../macros.tex}

\pagestyle{fancy} %
\lhead{ECE2 \hfill Mathématiques\\
} %
\chead{\hrule} %
\rhead{} %
\lfoot{} %
\cfoot{} %
\rfoot{\thepage} %

\renewcommand{\headrulewidth}{0pt}% : Trace un trait de séparation
 % de largeur 0,4 point. Mettre 0pt
 % pour supprimer le trait.

\renewcommand{\footrulewidth}{0.4pt}% : Trace un trait de séparation
 % de largeur 0,4 point. Mettre 0pt
 % pour supprimer le trait.

\setlength{\headheight}{14pt}

\title{\bf \vspace{-2cm} ESCP 2005 - voie Générale} %
\author{} %
\date{} %
\begin{document}

\maketitle %
\vspace{-1.4cm}\hrule %
\thispagestyle{fancy}

\vspace*{.2cm}


% DEBUT DU DOC À MODIFIER : tout virer jusqu'au début de l'exo


\begin{center}
{\small CHAMBRE D\E\ COMMERCE ET D'INDUSTRIE DE PARIS}

\textbf{DIRECTION DE L'ENSEIGNEMENT}

Direction des Admissions et concours

\underline{\hspace*{3cm}}

{\Large ECOLE DES\ HAUTES\ ETUDES\ COMMERCIALES}

{\Large E.S.C.P.-E.A.P.}

{\Large ECOL\E\ SUPERIEUR\E\ D\E\ COMMERC\E\ D\E\ LYON}{\large }

CONCOURS D'ADMISSION\ SUR\ CLASSES\ PREPARATOIRES

\underline{\hspace*{3cm}}

\textbf{OPTION SCIENTIFIQUE}

{\Large MATHEMATIQUES I}

\textbf{Année 2005}

\underline{\hspace*{3cm}}
\end{center}

\begin{quotation}
\noindent \textsl{La présentation, la lisibilité, l'orthographe, la
qualité
de la rédaction, la clarté et la précision des raisonnements entreront
pour
une part importante dans l'appréciation des copies.}

\noindent \textsl{Les candidats sont invités à encadrer dans la mesure
du
possible les résultats de leurs calculs.}

\noindent \textsl{Ils ne doivent faire usage d'aucun document :
l'utilisation de toute calculatrice et de tout matériel électronique
est
interdite.}

\noindent \textsl{Seule l'utilisation d'une règle graduée est
autorisée.}

\noindent \textsl{\hrulefill }
\end{quotation}

\textsl{Ce problème se compose de trois parties largement
indépendantes,
même si certains objets introduits dans la partie II se retrouvent dans
la
partie III. La partie I étudie un exemple de couple aléatoire suivant
une loi trinomiale. La partie II étudie les lois marginales d'un tel
couple. La partie III propose une caractérisation de la loi de
Poisson.}

\section*{Partie I}

On considère, dans cette partie des entiers naturels non nuls $n$, $u$,
$d$, 
$t$ et $b$, vérifiant $u + d + t = b$.\\
Une urne $\mathcal{U}$ contient $b$ boules, parmi lesquelles $u$ boules
portent le numéro $1$, $d$ le numéro $2$ et $t$ le numéro $3$.\\
Une expérience consiste en $n$ tirages successifs d'une boule de l'urne
$\mathcal{U}$ avec remise.\\
{À} chaque tirage, toutes les boules de l'urne $\mathcal{U}$ ont même
probabilité d'être tirées.\\
Le modèle choisi pour cette expérience est l'espace probabilisé
$(\Omega,\mathcal{T},P)$ dans lequel l'univers $\Omega $ est l'ensemble
$\{1,2,3\}{n} $ des $n$-uplets d'éléments de l'ensemble $\{1,2,3\}$, et
la
tribu $\mathcal{T}$ est l'ensemble $\mathcal{P}(\Omega )$ des parties
de $\Omega $, la probabilité $P$ se déduisant naturellement des
hypothèses qui
ont été ou seront formulées.\\
Aucun tirage n'influe sur les autres en cela que, si une suite
quelconque $(V_{k})_{1\leq k\leq n}$ de variables aléatoires définies
sur
l'espace probabilisé $(\Omega,\mathcal{T},P)$ est telle que, pour tout
$k\in \lbrack \hspace{-0.15em}[1,n]\hspace{-0.13em}]$, la valeur de
$V_{k}$
ne dépend que du résultat du $k$-ième tirage, alors les variables
$V_{1},V_{2},...,V_{n}$ sont mutuellement indépendantes.\\
On note $U$ (respectivement $D$, $T$) la variable aléatoire définie sur
l'espace probabilisé $(\Omega,\mathcal{T},P)$ dont la valeur est le
nombre
de boules numérotées $1$ (respectivement $2$, $3$) tirées au cours de
l'expérience.

\begin{noliste}{1.}
 \setlength{\itemsep}{4mm}
\item Montrer que la variable aléatoire $U$ suit une loi usuelle (à
préciser), donner son espérance et sa variance. Donner, de même, les
lois des
variables aléatoires $D$ et $T$, respectivement.

\item Les variables aléatoires $U$ et $D$ sont-elles indépendantes ?
\textsl{Justifiez votre réponse.}

\item Déterminer, sans calcul, la loi de la variable aléatoire $U + D$,
son espérance et sa variance.

\item En déduire que la covariance du couple $(U,D)$ est égale à
$-{\dfrac{nud}{b^{2}}}$.

\item \emph{Simulation informatique}\\
En \Scilab{}, si \texttt{i} est un entier naturel non nul,
l'instruction 
\texttt{random(i)} retourne aléatoirement un entier choisi
équiprobablement
parmi les entiers $0,1,\dots,i-1$. On considère la procédure \Scilab{}
nommée 
\texttt{simulation} déclarée comme suit : \\
\texttt{procedure simulation(var x,y,z : integer ; n : integer) ; }\\
\texttt{var k, r : integer ; }\\
\texttt{begin }\\
\texttt{\ \ x : = 0 ; y : = x ; z : = x ; }\\
\texttt{\ \ for k : = 1 to n do }\\
\texttt{\ \ \ \ begin }\\
\texttt{\ \ \ \ \ \ r : = random(6) ; }\\
\texttt{\ \ \ \ \ \ if r = 0 then x : = x + 1 else if r\TEXTsymbol{<} =
2 then y : = y + 1
else z : = z + 1 }\\
\texttt{\ \ \ \ end }\\
\texttt{end ;} \\
Que réalise l'instruction \texttt{simulation (a,b,c,12)}, les variables
\Scilab{} \texttt{a}, \texttt{b} et \texttt{c} étant toutes trois de
type
integer ? \textsl{On demande une réponse en rapport avec l'expérience
précédemment étudiée et, en particulier, que soient précisées les
valeurs des
paramètres $u$, $d$, $t$ et $n$ dans la simulation proposée.} 

\item Dans toute la suite, $m$, $i$ et $j$ étant des entiers naturels,
on
note : 
\[
\begin{smatrix}
m \\
i,j
\end{smatrix}
 = \left\{ 
\begin{array}{ccc}
\dfrac{m!}{i!j!(m-i-j)!} & \text{si} & i + j\leq m \\
0 & \text{si} & i + j>m
\end{array}
\right.
\]
On considère deux entiers naturels $k$ et $\ell $ vérifiant $k + \ell
\leq n$.\\
Soit $\omega = (x_{1},x_{2},\dots,x_{n})$ un élément donné de $\Omega $
comportant exactement $k$ " $1$ "\ et $\ell $ " $2$ ".\\
Quelle est la probabilité $P\left(\Ev{ \{\omega \}}\right) $ de
l'événement élémentaire $\{\omega \}$ ?\\
Dénombrer les $n$-uplets appartenant à l'ensemble $\Omega $ et
comportant
exactement $k$ " $1$ "\ et $\ell $ " $2$ ".\\
En déduire que la probabilité de l'événement $[U = k]\cap \lbrack D =
\ell ]$
est égale à : 
\[
\begin{smatrix}
n \\
k,\ell
\end{smatrix}
{\dfrac{u^{k}d^{\ell }t^{n-k-\ell }}{b^{n}}}
\]
Ce résultat reste-t-il vrai si $k + \ell >n$ ?
\end{noliste}

\section*{Partie II : Lois marginales d'un couple aléatoire de loi
trinomiale.}

On considère, dans cette partie, un entier naturel $n$ et l'ensemble
$I_{n}$
défini par 
\[
I_{n} = \{(k,\ell )/\;k\in \lbrack
\hspace{-0.15em}[0,n]\hspace{-0.13em}]\quad 
\text{et}\quad \ell \in \lbrack
\hspace{-0.15em}[0,n]\hspace{-0.13em}]\quad 
\text{et}\quad k + \ell \leq n\}
\]
Un espace probabilisé $(\Omega,\mathcal{T},P)$ étant donné, ainsi que
trois
réels strictement positifs $p$, $q$ et $r$ vérifiant $p + q + r = 1$,
on considère
un couple aléatoire $(X_{n},Y_{n})$ défini sur
$(\Omega,\mathcal{T},P)$, à
valeurs dans $I_{n}$, et tel que, pour tout couple $(k,\ell )\in I_{n}$
: 
\[
P\left(\Ev{ (X_{n},Y_{n}) = (k,\ell
)}\right)\left(\Ev{X_{n},Y_{n}}\right) = (k,\ell )\right) = 
\begin{smatrix}
n \\
k,\ell
\end{smatrix}
p^{k}q^{\ell }r^{n-k-\ell }
\]

\begin{noliste}{1.}
 \setlength{\itemsep}{4mm}
\item Vérifier que :\quad $\Sum{(k,\ell )\in I_{n}}\begin{smatrix}
n \\
k,\ell
\end{smatrix}
p^{k}q^{\ell }r^{n-k-\ell } = 1$.

\item Montrer que les variables aléatoires $X_{n}$ et $Y_{n}$ suivent
toutes
deux une loi binomiale (en préciser les paramètres respectifs).

\item On se propose de calculer la covariance du couple
$(X_{n},Y_{n})$.

\begin{noliste}{a)}
 \setlength{\itemsep}{2mm}
\item On suppose que $n\geq 2$. Prouver que, pour tout couple $(k,\ell
)\in I_{n}$ vérifiant $k\geq 1$ et $\ell \geq 1$, on a : 
\[
k\ell 
\begin{smatrix}
n \\
k,\ell
\end{smatrix}
 = n(n-1)\begin{smatrix}
n-2 \\
k-1,\ell -1
\end{smatrix}
\]
En déduire que $\E(X_{n}Y_{n}) = n(n-1)pq$.

\item Cette relation est-elle encore vraie si $n = 0$ ? si $n = 1$ ?

\item En déduire la valeur de la covariance $\func{cov}(X_{n},Y_{n})$
du
couple $(X_{n},Y_{n})$.
\end{noliste}

\item Les variables aléatoires $X_{n}$ et $Y_{n}$ sont-elles
indépendantes ?
\end{noliste}

\section*{Partie III : Une caractérisation de la loi de Poisson.}

Dans cette partie, la lettre $n$ ne désigne plus un entier naturel fixé
et
on considère les suites $(X_{n})_{n\in \N}$ et $(Y_{n})_{n\in 
\N}$ des variables aléatoires définies, dans la partie précédente,
pour chaque entier naturel $n$ sur l'espace probabilisé
$(\Omega,\mathcal{T},P)$.\\
On rappelle que, pour tout entier naturel $n$, les variables aléatoires
$X_{n}$ et $Y_{n}$ suivent des lois binomiales dont les paramètres ont
été
calculés en \textbf{II.2}.\\
On considère par ailleurs, une variable aléatoire $N$ \textsl{non
presque sûrement constante} définie sur le même espace probabilisé
$(\Omega,\mathcal{T},P)$, à valeurs dans l'ensemble $\N$ des entiers
naturels et indépendante de tous les couples $(X_{n},Y_{n})$, ce qui
signifie que, pour tout 
$n\in \N$ et tout $(i,j,k)\in \N^{3}$, 
\[
P\left( [\left(\Ev{X_{n},Y_{n}}\right) = (i,j)]\cap \lbrack N =
k]\right) = P\left(\Ev{
(X_{n},Y_{n}) = (i,j)}\right)\left(\Ev{X_{n},Y_{n}}\right) =
(i,j)\right) P\left(\Ev{N = k}\right)
\]
On définit les fonctions $X :\Omega \longmapsto \N$ et $Y :\Omega
\mapsto \N$ de la manière suivante : si $N$ prend la valeur $n$,
alors $X$ (respectivement $Y$) prend la même valeur que $X_{n}$
(respectivement $Y_{n}$).

\subsection*{A) Remarques générales.}

\begin{noliste}{1.}
 \setlength{\itemsep}{4mm}
\item Montrer que, pour tout $k\in \N$, 
\[
\lbrack X = k] = \dcup{n = 0}{+ \infty }\left( [X_{n} = k]\cap \lbrack
N = n]\right) = \dcup{n = k}{+ \infty }\left( [X_{n} = k]\cap \lbrack
N = n]\right)
\]
En déduire que $X$ est une variable aléatoire sur l'espace probabilisé
$(\Omega,\mathcal{T},P)$.\\
\textsl{On prouverait de même que $Y$ est une variable aléatoire sur
l'espace probabilisé $(\Omega,\mathcal{T},P)$.}

\item Montrer que, pour tout $n\in \N$, les variables aléatoires $N$
et $X_{n}$, sont indépendantes.\\
\textsl{On prouverait de même que, pour tout $n\in \N$, les
variables $\N$ et $Y_{n}$ sont indépendantes.}

\item Déduire des résultats précédents que, pour tout $k\in \N$, 
\[
P\left(\Ev{X = k}\right) = \Sum{n = 0}{+ \infty
}C_{n}{k}p^{k}(1-p)^{n-k}P\left(\Ev{N = n}\right) = \Sum{n = k}{+
\infty
}C_{n}{k}p^{k}(1-p)^{n-k}P\left(\Ev{N = n}\right)
\]
Exprimer de même, pour tout $\ell \in \N$, $P\left(\Ev{Y = \ell
}\right)$ sous forme
de somme d'une série.

\item Les variables aléatoires $N$ et $X$ sont-elles indépendantes ?\\
\textsl{On considérera deux entiers $a$, et $b$ vérifiant $0\leq a<b$,
$P\left(\Ev{N = a}\right)\not = 0$ et $P\left(\Ev{N = b}\right)\not =
0$, et on se préoccupera de l'événement $[N = a]\cap \lbrack X = b]$.}
\end{noliste}

\subsection*{B) Si $N$ suit une loi de Poisson, alors $X$ et $Y$ sont
indépendantes.}

On considère un réel strictement positif $\lambda$. On suppose que la
variable aléatoire $N$ suit la loi de Poisson de paramètre $\lambda$.

\begin{noliste}{1.}
 \setlength{\itemsep}{4mm}
\item Montrer que $X$ suit la loi de Poisson de paramètre $\lambda p$
et que 
$Y$ suit la loi de Poisson de paramètre $\lambda q$.

\item Montrer que les variables aléatoires $X$ et $Y$ sont
indépendantes.\\
\textsl{On commencera par justifier que, pour tout couple $(k,\ell )\in

\N^{2}$,} 
\[
P\left( [X = k]\cap \lbrack Y = \ell ]\right) = \Sum{n = k + \ell }{+
\infty
}P\left( [X_{n} = k]\cap \lbrack Y_{n} = \ell ]\right) P\left(\Ev{N =
n}\right)
\]
\end{noliste}

\subsection*{C) Si $X$ et $Y$ sont indépendantes, alors $N$ suit une
loi de
Poisson.}

\textsl{On ne suppose plus} a priori \textsl{que la variable aléatoire
$N$
suit une loi de Poisson.} On suppose maintenant que les variables
aléatoires $X$ et $Y$ sont indépendantes.

\begin{noliste}{1.}
 \setlength{\itemsep}{4mm}
\item Montrer que, pour tout réel $z$ appartenant à $[0,1]$, la série
$\Sum{n\in \N}z^{n}P\left(\Ev{N = n}\right)$ converge.\\
\textsl{Dans toute la suite, si $z\in \lbrack 0,1]$, la somme de cette
série
est notée $\Phi (z)$.}\\
\textsl{Un lemme de Fubini.} \\
On \textbf{admet} le résultat suivant : soit $(r_{i,j})_{(i,j)\in
\N\times \N}$ une famille de réels \textsl{positifs} ou nuls.\\
On suppose que, pour tout $j\in \N$, la série $\sum r_{i,j}$
converge ; on note $C_{j} = \Sum{i = 0}{+ \infty }r_{i,j}$. On suppose
de plus que la série $\Sum{j\in \N}C_{j}$ converge.\\
Alors :\\
\quad i) Pour tout $i\in \N$, la série $\Sum{j\in \N}r_{i,j}$ converge
; on note $L_{i} = \Sum{j = 0}{+ \infty }r_{i,j}$ sa
somme ;\\
{}\quad ii) La série $\Sum{i\in \N}L_{i}$ converge et $\Sum{i = 0}{+
\infty }L_{i} = \Sum{j = 0}{+ \infty }C_{j}$.\\
\textsl{On définit en ce cas la somme $\Sum{(i,j)\in \N\times
\N}r_{i,j}$ comme étant le nombre $\Sum{i = 0}{+ \infty
}\left( \Sum{j = 0}{+ \infty }r_{i,j}\right)
 = \Sum{j = 0}{+ \infty }\left( \Sum{i = 0}{+ \infty
}r_{i,j}\right) $.}

\item On considère deux réels $\alpha $ et $\beta $ appartenant tous
deux à $[0,1]$. On définit sur l'espace probabilisé
$(\Omega,\mathcal{T},P)$ les
variables aléatoires $A = \alpha ^{X}$ et $B = \beta ^{Y}$.

\begin{noliste}{a)}
 \setlength{\itemsep}{2mm}
\item Montrer que les variables aléatoires $A$ et $B$ admettent une
espérance (que l'on ne cherchera pas à évaluer).

\item Montrer que $0\leq p\alpha + 1-p\leq 1$ puis, en utilisant le
lemme de Fubini, que 
\[
\E(A) = \varphi (p\alpha + 1-p)
\]
\textsl{On montrerait de même que $0\leq q\beta + 1-q\leq 1$ et que 
$\E(B) = \Phi (q\beta + 1-q)$.}
\end{noliste}

\item On définit la variable aléatoire $C = AB = \alpha ^{X}\beta
^{Y}$.

\begin{noliste}{a)}
 \setlength{\itemsep}{2mm}
\item Justifier que la variable aléatoire $C$ admet une espérance (que
l'on
ne cherchera pas à évaluer).

\item Établir que $0\leq p\alpha + q\beta + r\leq 1$, puis, en
utilisant le théorème de transfert et le lemme de Fubini, que $\E(C) =
\Phi
(p\alpha + q\beta + r)$.
\end{noliste}

\item Pour quelle raison peut-on affirmer que $\E(AB) = E(A)\E(B)$ ?\\
En déduire que, pour tout couple $(\alpha,\beta )\in \lbrack 0,1]^{2}$,

\[
\Phi \left( 1-p(1-\alpha )-q(1-\beta )\right) = \Phi \left(
1-p(1-\alpha
)\right) \Phi \left( 1-q(1-\beta )\right)
\]

\item Montrer que, pour tout réel $z\in \ ]0,1]$, $\Phi (z)>0$. Que
vaut $\varphi (1)$ ?

\item On définit l'application $\varphi :\left[ 0,1\right[ \ \mapsto \R
$ par la relation $\varphi (z) = \ln \left( \Phi (1-z)\right) $.

\begin{noliste}{a)}
 \setlength{\itemsep}{2mm}
\item Montrer que, pour tous réels $a\in \lbrack 0,p]$ et $b\in \lbrack
0,q]$, $\varphi (a + b) = \varphi (a) + \varphi (b)$.\\
On pose dans toute la suite $\mu = \min (p.q)$ et $I = [0,\mu ]$.

\item Calculer $\varphi (0)$.\\
Montrer que, pour tout couple $(n,a)\in \N\times I$ vérifiant $0\leq
na\leq \mu $, on a : $\varphi (na) = n\varphi (a)$.

\item Montrer que, pour tout triplet $(n,m,a)\in \N\times \N^{\times
}\times I$ tel que $0\leq {\dfrac{n}{m}}\,a\leq \mu $, 
\[
\varphi \left( \dfrac{n}{m}\,a\right) = \dfrac{n}{m}\,\varphi (a)
\]

\item Soit un couple $(x,a)\in \R\times I$ vérifiant $0<xa<\mu $.\\
Si r est un réel, on note $\left[ r\right] $ la partie entière de
$r$.\\
Montrer que, pour tout $n\in \N^{\times }$,\quad ${\dfrac{\left[
nx\right] }{n}}\leq x<{\dfrac{\left[ nx\right] + 1}{n}}$ et que
$\dlim{n\rightarrow + \infty }{\dfrac{\left[ nx\right] }{n}} = x$.\\
Montrer que la fonction $\varphi $ décro{î}t sur $[0,1[$. En déduire
que $\varphi (xa) = x\varphi (a)$.

\item Montrer enfin qu'il existe un réel $\lambda >0$ tel que, pour
tout $x\in I$, $\varphi (x) = -\lambda x$.
\end{noliste}

\item On admet le résultat suivant :\\
\textsl{si la suite réelle $(u_{n})_{n\in \N}$ est telle que, pour
tout $z\in \lbrack 1-\mu,1]$, la série $\Sum{n\in \N}u_{n}z^{n}$
converge et est de somme nulle, alors, pour tout $n\in \N$, $u_{n} =
0$. }\\
\textsl{Montrer que la variable aléatoire $N$ suit la loi de Poisson de
paramètre $\lambda $ }
\end{noliste}

\label{fin}

\end{document}

