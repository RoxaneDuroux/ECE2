\documentclass[11pt]{article}%
\usepackage{geometry}%
\geometry{a4paper,
 lmargin = 2cm,rmargin = 2cm,tmargin = 2.5cm,bmargin = 2.5cm}

\input{../../macros.tex}

\pagestyle{fancy} %
\lhead{ECE2 \hfill Mathématiques\\
} %
\chead{\hrule} %
\rhead{} %
\lfoot{} %
\cfoot{} %
\rfoot{\thepage} %

\renewcommand{\headrulewidth}{0pt}% : Trace un trait de séparation
 % de largeur 0,4 point. Mettre 0pt
 % pour supprimer le trait.

\renewcommand{\footrulewidth}{0.4pt}% : Trace un trait de séparation
 % de largeur 0,4 point. Mettre 0pt
 % pour supprimer le trait.

\setlength{\headheight}{14pt}

\title{\bf \vspace{-2cm} ESCP 1993 - voie Générale} %
\author{} %
\date{} %
\begin{document}

\maketitle %
\vspace{-1.4cm}\hrule %
\thispagestyle{fancy}

\vspace*{.2cm}


% DEBUT DU DOC À MODIFIER : tout virer jusqu'au début de l'exo


\begin{center}
{\small CHAMBRE D\E\ COMMERCE ET D'INDUSTRIE DE PARIS}

\textbf{DIRECTION DE L'ENSEIGNEMENT}

Direction des Admissions et concours

\underline{\hspace*{3cm}}

{\Large ECOLE DES\ HAUTES\ ETUDES\ COMMERCIALES}

{\Large E.S.C.P.-E.A.P.}

{\Large ECOL\E\ SUPERIEUR\E\ D\E\ COMMERC\E\ D\E\ LYON}{\large }

CONCOURS D'ADMISSION\ SUR\ CLASSES\ PREPARATOIRES

\underline{\hspace*{3cm}}

\textbf{OPTION GENERALE}

{\Large MATHEMATIQUES I}

\textbf{Année 1993}

\underline{\hspace*{3cm}}
\end{center}

\begin{quotation}
\noindent \textsl{La présentation, la lisibilité, l'orthographe, la
qualité
de la rédaction, la clarté et la précision des raisonnements entreront
pour
une part importante dans l'appréciation des copies.}

\noindent \textsl{Les candidats sont invités à encadrer dans la mesure
du
possible les résultats de leurs calculs.}

\noindent \textsl{Ils ne doivent faire usage d'aucun document :
l'utilisation de toute calculatrice et de tout matériel électronique
est
interdite.}

\noindent \textsl{Seule l'utilisation d'une règle graduée est
autorisée.}

\noindent \textsl{\hrulefill }
\end{quotation}

\noindent Etant donné un nombre entier naturel non nul $n$ et un
ensemble $E$
à $n$ éléments, on appelle \textit{involution} de $E$ toute bijection
$f$ de
E sur lui même telle que $f\circ f = Id$, où $Id$ désigne l'application
identique de $E$. L'objectif du problème est l'étude du nombre $T_{n}$
d'involutions de $E$ et, en particulier, la recherche d'un équivalent
du
nombre $T_{n}$ quand n tend vers $ + \infty $.

\section*{Partie I Étude du nombre $T_{n}$ d'involutions de E}

\begin{noliste}{1.}
 \setlength{\itemsep}{4mm}
\item Calculer $T_{1},T_{2}$ et $T_{3}$.

\item On suppose désormais $n\geq 3$.

\begin{noliste}{a)}
 \setlength{\itemsep}{2mm}
\item Déterminer en fonction de $T_{i}$, où $1\leq i<n$ :

\begin{noliste}{$\sbullet$}
\item le nombre des involutions $\sigma $ de ${\{}1,...,n{\}}$ telles
que $\sigma (n) = n$ ;

\item le nombre des involutions $\sigma $ de ${\{}1,...,n{\}}$ telles
que $\sigma (n) = k$, où $k$ est un nombre entier donné de
${\{}1,...,n-1{\}}$
\end{noliste}

\item En déduire la relation suivante : 
\begin{equation}
T_{n} = T_{n-1} + (n-1)T_{n-2} \label{1}
\end{equation}
\end{noliste}

\item Rédiger en \textsc{\Scilab{}} un algorithme permettant le calcul
des $p$
premiers termes de la suites $(T_{n})$ pour un nombre entier donné
$p\geq 3$.\\
En programmant cet algorithme, expliciter les valeurs de $T_{n}$ pour
$n\leq 10$.
\end{noliste}

\section*{Partie II Interprétation de $T_{n}$ à l'aide d'une suite de
polynômes}

On considère la fonction numérique $u$ définie sur $\R$ par la
relation :
\[
u(x) = exp(\dfrac{x^{2}}{2})
\]

Pour tout nombre entier naturel non nul $n$, on désigne par $u^{(n)}$
la dérivée $n^{\grave{e}me}$ de u. On note $H_{n}$ la fonction
numérique définie
sur $\R$ par la relation : 
\begin{equation}
u^{(n)}(x) = H_{n}(x)u(x) \label{2}
\end{equation}

\begin{noliste}{1.}
 \setlength{\itemsep}{4mm}
\item 

\begin{noliste}{a)}
 \setlength{\itemsep}{2mm}
\item Exprimer $u^{\prime }(x)$ en fonction de $u(x)$ et $x$.\\
En déduire la relation suivante, pour tout nombre entier $n\geq 2$
:\begin{equation}
u^{(n)}(x) = xu^{(n-1)}(x) + (n-1)u^{(n-2)}(x) \label{3}
\end{equation}

\item Calculer $H_{0}$ et $H_{1}$, puis déduire des relations
précédentes
l'expression de $H_{n}(x)$ en fonction de $H_{n-1}(x)$, $H_{n-2}(x)$ et
$x$.

\item Prouver que $H_{n}$ est un polynôme dont on précisera, en
fonction de $n$, le degré, la parité et le signe sur $[0, + \infty
\lbrack $.

\item Comparer $T_{n}$ et $H_{n}(1)$.
\end{noliste}

\item 

\begin{noliste}{a)}
 \setlength{\itemsep}{2mm}
\item En dérivant la relation (\ref{2}) et en utilisant la relation
entre $H_{n + 1}(x),H_{n}(x),H_{n-1}(x)$ et $x$, établir la relation
suivante, pour
tout nombre entier naturel non nul $n$ : 
\begin{equation}
H_{n}{\prime }(x) = nH_{n-1}(x) \label{4}
\end{equation}

\item Pour tout nombre entier naturel $n$, exprimer $H_{n}(0)$ et
$H_{n}{\prime }(0)$ en fonction de $n$. (On distinguera deux cas
suivant la
parité de $n.$)
\end{noliste}

\item 

\begin{noliste}{a)}
 \setlength{\itemsep}{2mm}
\item Établir que, pour tout nombre entier naturel $n$ : 
\[
H_{n}{\prime \prime }(x) + xH_{n}{\prime }(x)-nH_{n}(x) = 0
\]
(on pourra dériver deux fois la formule (\ref{2}.)

\item Dans toute la suite du problème, pour tout nombre entier naturel
n, on
note $v_{n}$ la fonction numérique définie sur $\R$ par la relation :
\[
v_{n}(x) = H_{n}(x)\exp (\dfrac{x^{2}}{4})
\]
Étudier le signe de $v_{n}$ et de $v_{n}{\prime }$sur $[0, + \infty
\lbrack $. Calculer $v_{n}(0)$et $v_{n}{\prime }(0)$.

\item Exprimer $v_{n}{\prime \prime }(x)$ en fonction de $v_{n}(x)$et
de $x$.

\item En déduire la relation suivante, pour tout nombre entier naturel
$n$
et pour tout nombre réel $x$ appartenant à $[0,1]$ : 
\begin{equation}
(n + \dfrac{1}{2})v_{n}(x)\leq v_{n}{\prime \prime }(x)\leq (n +
\dfrac{3}{4})v_{n}(x) \label{5}
\end{equation}
\end{noliste}
\end{noliste}

\noindent Dans toute la suite du problème, on posera :

\[
\alpha_{n} = \sqrt{n + \dfrac{1}{2}}\qquad \beta_{n} = \sqrt{n +
\dfrac{3}{4}}
\]

\section*{Partie III Recherche d'un équivalent de $T_{n}$}

\textit{On étudie tout d'abord un équivalent de }$T_{n}$
\textit{lorsque
l'entier }$n = 2p$\textit{\ est pair.}

\begin{noliste}{1.}
 \setlength{\itemsep}{4mm}
\item On établit dans cette question un résultat préliminaire
permettant
d'encadrer une fonction numérique définie sur $[0,1]$ à valeurs
strictement
positives, de classe $C^{2}$ et satisfaisant aux relations : 
\begin{equation}
\alpha ^{2}f(x)\leq f"(x)\leq \beta ^{2}f(x)\quad f(0) = a\quad
f^{\prime }(0) = 0 \label{6}
\end{equation}où a, $\alpha $ et $\beta $ sont des nombres réels
strictements positifs donnés.

\begin{noliste}{a)}
 \setlength{\itemsep}{2mm}
\item Déterminer des nombres réels $\lambda $ et $\mu $ tels que la
fonction
numérique $\varphi $ définie sur $[0,1]$ par la relation : 
\[
\varphi (x) = \lambda \exp (\beta x) + \mu \exp (-\beta x)
\]
vérifie $\varphi (0) = a$ et $\varphi ^{\prime }(0) = 0$.\\
Indiquer alors le signe de $\varphi $ sur $[0,1]$ et exprimer $\varphi
^{\prime }(x)$ en fonction de $\varphi (x)$.

\item Soit $w$ la fonction numérique définie sur $[0,1]$ par la
relation :
\[
w = f\varphi ^{\prime }-\varphi f^{\prime }
\]
Calculer $w(0)$. Étudier le signe de $w^{\prime }$, puis celui de $w$.

\item En déduire, pour tout nombre réel x appartenant à $[0,1]$,
l'inégalité
\[
f(x)\leq \varphi (x)
\]

\item Établir, pour tout nombre réel $x$ appartenant à $[0,1]$,
l'inégalité
suivante :\begin{equation}
f(x)\leq \dfrac{a}{2}(\exp (\beta x) + 1) \label{7}
\end{equation}

\item Établir de même que, pour tout nombre réel x appartenant à
$[0,1]$ :\begin{equation}
\dfrac{a}{2}\exp (\alpha x)\leq f(x) \label{8}
\end{equation}
\end{noliste}

\item 

\begin{noliste}{a)}
 \setlength{\itemsep}{2mm}
\item À l'aide de la relation (\ref{5}), établir que, pour tout nombre
entier naturel $p$ :
\[
H_{2p}(0)\dfrac{\exp (\alpha_{2p})}{2}\leq \exp
(\dfrac{1}{4})H_{2p}(1)\leq H_{2p}(0)\dfrac{\exp (\beta_{2p}) + 1}{2}
\]

\item On admet la formule de Stirling : $n!\underset{n\rightarrow +
\infty }{\sim }(\dfrac{n}{e})^{n}\sqrt{2\pi n}$\\
D'après la \textbf{partie II} : $H_{2p}(0) = \dfrac{(2p)!}{2^{p}p!}$\\
En déduire que, pour $n = 2p$, on a : 
\begin{equation}
T_{n}\underset{n\rightarrow + \infty }{\sim }\dfrac{1}{\sqrt{2}}\exp
({-}\dfrac{1}{4})\exp (\sqrt{n})(\dfrac{n}{e})^{\dfrac{n}{2}} \label{9}
\end{equation}

\item Donner une valeur approchée du quotient des deux nombres de cette
expression pour $n = 10$.\\
\textit{On étudie enfin un équivalent de }$T_{n}$ \textit{lorsque
l'entier n
est impair (}$n = 2p + 1$\textit{).}
\end{noliste}

\item On établit par des méthodes analogues à celles de la question
III.1.
que, si $g$ est une fonction numérique sur $[0,1]$ à valeurs
strictement
positives sur $]0,1]$ de classe $C^{2}$ et satisfaisant aux relations
:\begin{equation}
\alpha ^{2}g(x)\leq g^{\prime \prime }(x)\leq \beta ^{2}g(x)\quad
g(0) = 0\quad g^{\prime }(0) = a \label{10}
\end{equation}où a, $\alpha $ et $\beta $ sont des nombres réels
strictements positifs donnés, alors, pour tout nombre réel $x$
appartenant à $[0,1]$ :
\[
\dfrac{a}{2\alpha }(\exp (\alpha x)-1)\leq g(x)\leq \dfrac{a}{2\beta
}\exp (\beta x)
\]
\textit{(on ne demande pas de justifier cet encadrement)}

\begin{noliste}{a)}
 \setlength{\itemsep}{2mm}
\item En déduire que, pour $n = 2p + 1$ :
\[
H_{n}(1)\underset{n\rightarrow + \infty }{\sim
}\dfrac{1}{2}\sqrt{n}\exp (-\dfrac{1}{4})\exp (\sqrt{n})H_{n-1}(0)
\]

\item En conclure que la relation (\ref{9}) est encore valable lorsque
le
nombre entier $n$ est impair.
\end{noliste}
\end{noliste}

\label{fin}

\end{document}

