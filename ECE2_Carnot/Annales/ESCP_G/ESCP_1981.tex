\documentclass[11pt]{article}%
\usepackage{geometry}%
\geometry{a4paper,
 lmargin = 2cm,rmargin = 2cm,tmargin = 2.5cm,bmargin = 2.5cm}

\input{../../macros.tex}

\pagestyle{fancy} %
\lhead{ECE2 \hfill Mathématiques\\
} %
\chead{\hrule} %
\rhead{} %
\lfoot{} %
\cfoot{} %
\rfoot{\thepage} %

\renewcommand{\headrulewidth}{0pt}% : Trace un trait de séparation
 % de largeur 0,4 point. Mettre 0pt
 % pour supprimer le trait.

\renewcommand{\footrulewidth}{0.4pt}% : Trace un trait de séparation
 % de largeur 0,4 point. Mettre 0pt
 % pour supprimer le trait.

\setlength{\headheight}{14pt}

\title{\bf \vspace{-2cm} ESCP 1981} %
\author{} %
\date{} %
\begin{document}

\maketitle %
\vspace{-1.4cm}\hrule %
\thispagestyle{fancy}

\vspace*{.2cm}


% DEBUT DU DOC À MODIFIER : tout virer jusqu'au début de l'exo


\begin{center}
{\small CHAMBRE D\E\ COMMERCE ET D'INDUSTRIE DE PARIS}

\textbf{DIRECTION DE L'ENSEIGNEMENT}

Direction des Admissions et concours

\underline{\hspace*{3cm}}

{\Large ECOLE DES\ HAUTES\ ETUDES\ COMMERCIALES}

{\Large E.S.C.P.-E.A.P.}

{\Large ECOL\E\ SUPERIEUR\E\ D\E\ COMMERC\E\ D\E\ LYON}{\large }

CONCOURS D'ADMISSION\ SUR\ CLASSES\ PREPARATOIRES

\underline{\hspace*{3cm}}

\textbf{OPTION GENERALE}

{\Large MATHEMATIQUES I}

\textbf{Année 1981}

\underline{\hspace*{3cm}}
\end{center}

\begin{quotation}
\noindent \textsl{La présentation, la lisibilité, l'orthographe, la
qualité
de la rédaction, la clarté et la précision des raisonnements entreront
pour
une part importante dans l'appréciation des copies.}

\noindent \textsl{Les candidats sont invités à encadrer dans la mesure
du
possible les résultats de leurs calculs.}

\noindent \textsl{Ils ne doivent faire usage d'aucun document :
l'utilisation de toute calculatrice et de tout matériel électronique
est
interdite.}

\noindent \textsl{Seule l'utilisation d'une règle graduée est
autorisée.}

\noindent \textsl{\hrulefill }
\end{quotation}

\section*{PREAMBULE}

Dans ce problème, on appelle " suite associée à une variable aléatoire
réelle $X$ "\ ($X$ étant quelconque), une suite infinie $(X_{i})$ de
variables
aléatoires réelles $X_{i}$ $(i\in \N^{\times })$ de même loi de
probabilité que X, telles que, pour tout $n\in \N^{\times }$ les $n$
variables aléatoires $X_{1},X_{2},\dots,X_{i},\dots,X_{n}$ soient
indépendantes.

\section*{PARTIE PRELIMINAIRE}

Soit $a>0$ un réel donné.\\
Soit $L_{1}(a)$ l'ensemble des fonctions numériques $f$, continues sur
$]0,a] $, à valeurs positives, telles que l'intégrale 
\[
\dint{0}{a}f(x)dx
\]
converge.\\
Soit $L_{2}(a)$ l'ensemble des fonctions numériques $f$, continues sur
$]0,a] $, à valeurs positives, telles que l'intégrale : 
\[
\dint{0}{a}f^{2}(x)dx
\]
converge.

\begin{noliste}{1.}
 \setlength{\itemsep}{4mm}
\item 

\begin{noliste}{a)}
 \setlength{\itemsep}{2mm}
\item Montrer que si $f_{1}$ et $f_{2}$ sont deux éléments de
$L_{2}(a)$, ${\dfrac{(f_{1}{2} + f_{2}{2})}{2}}$ et le produit
$f_{1}.f_{2}$ sont éléments
de $L_{1}(a)$.

\item En déduire que $L_{2}(a)$ est inclus dans $L_{1}(a)$.
\end{noliste}

\item $p$ et $q$ étant des entiers naturels donnés, on considère la
fonction
réelle $f_{p,q}$ de la variable réelle $x$ définie pour $x$ strictement
positif par : 
\[
f_{p,q}(x) = x^{p-1}(\ln x)^{q}
\]

\begin{noliste}{a)}
 \setlength{\itemsep}{2mm}
\item A quelle condition nécessaire et suffisante notée " $C_{0}$ "\
$f_{p,q} $ est-elle élément de $L_{2}(1)$ ?

\item La condition $C_{0}$ étant réalisée, calculer : 
\[
I_{p,q} = \dint{0}{1}f_{p,q}(x)dx
\]
\end{noliste}
\end{noliste}

\section*{Partie I}

Soit $X$ une variable aléatoire réelle, ayant une espérance
mathématique $\E(X) = \mu \not = 0$ et une variance $\V(X) = \sigma
^{2}>0$.\\
On considère la suite $(X_{i})$ associée à $X$ ainsi que la suite de
variables aléatoires $(T_{n})$ dont les éléments sont définis quel que
soit $n\in \N^{\times }$ par : 
\[
T_{n} = \alpha_{1,n}X_{1} + \alpha_{2,n}X_{2} + \cdots +
\alpha_{n,n}X_{n}
\]
où les coefficients $\alpha_{i,n}$, $(i\in \lbrack
\hspace{-0.15em}[1,n]\hspace{-0.13em}])$ sont des réels positifs ou
nuls.

\begin{noliste}{1.}
 \setlength{\itemsep}{4mm}
\item A quelle condition nécessaire et suffisante notée " $C_{1}(\mu )$
"\
a-t-on : $\E(T_{n}) = \mu $ ?

\item La condition $C_{1}(\mu )$ étant réalisée, montrer qu'une
condition nécessaire et suffisante notée " $C_{2}(\sigma ^{2})$ "\ pour
que $\dlim{n\rightarrow + \infty }\V(T_{n}) = 0$ est : 
\[
\dlim{n\rightarrow \infty }\Sum{1\leq i<j\leq
n}\alpha_{i,n}\alpha_{j,n} = {\dfrac{1}{2}}
\]

\item Les conditions $C_{1}(\mu )$ et $C_{2}(\sigma ^{2})$ étant
réalisées,
vers quelle valeur $t_{0}$ la suite de variables aléatoires $(T_{n})$
converge-t-elle en probabilité ?
\end{noliste}

\noindent Les indices $p$ et $q$, définis dans le préliminaire,
satisfaisant 
à la condition $C_{0}$ sont maintenant fixés. On pose $f_{p,q} = f$ et
$I_{p,q} = I$.

\section*{Partie II}

Soit $X$ une variable aléatoire réelle dont la loi de probabilité est
la loi
uniforme sur l'intervalle $]0,1]$. On notera $g$ sa fonction densité de
probabilité.\\
On admettra qu'il existe une variable aléatoire réelle $Y$ dont
l'espérance
mathématique et le moment d'ordre $2$ sont donnés respectivement par 
\[
\E(Y) = \dint{0}{1}f(x)g(x)dx\quad \text{et}\quad
\E(Y^{2}) = \dint{0}{1}f^{2}(x)g(x)dx
\]
Soit $(Y_{i})$ une suite associée à $Y$, on pose : 
\[
\forall n\in \N^{\times },\quad Z_{n} = {\dfrac{1}{n}}\Sum{i =
1}{n}Y_{i}
\]

\begin{noliste}{1.}
 \setlength{\itemsep}{4mm}
\item 

\begin{noliste}{a)}
 \setlength{\itemsep}{2mm}
\item Montrer que, quels que soient les entiers $i$ et $n$, $Y_{i}$ et
$Z_{n} $ satisfont à la condition $C_{1}(I)$.

\item Montrer que $Z_{n}$ satisfait à la condition $C_{2}(\V(Y))$.

\item Montrer que, quel que soit $i\in \lbrack
\hspace{-0.15em}[1,n]\hspace{-0.13em}]$ :
\[
\V(Z_{n}) = {\dfrac{1}{n-1}}\E((Y_{i}-Z_{n})^{2}).
\]
\end{noliste}

\item Montrer que l'inégalité : $\left| Z_{n}-I\right| \leq
10\sqrt{\dfrac{\V(Y)}{n}}$ est satisfaite avec une probabilité au moins
égale à 
$0.99$.

\item Lorsque n est suffisamment grand, on peut admettre vue la
variable aléatoire centrée réduite associée à $Z_{n}$ a pour loi de
probabilité la loi
normale centrée réduite. \\
Sous cette hypothèse, montrer que l'on peut améliorer l'inégalité
précédente, c'est-à-dire trouver $a<10$ et $P>0.99$ tels que
$P\left(\Ev{ \left|
Z_{n}-I\right| \leq a\sqrt{\dfrac{\V(Y)}{n}}}\right) \geq P$
\end{noliste}

\section*{Partie III}

Nota : dans cette partie la notation $X^{j}$ signifie "$X$ indice $j$"
et
NON "$X$ puissance $j$". On subdivise l'intervalle $]0,1]$ en $k$
sous-intervalles $]a_{j},a_{j + 1}]$ non vides et disjoints tels que : 
\[
\dcup{j = 1}{k}]a_{j},a_{j + 1}] = \ ]0,1]
\]
A chaque intervalle $]a_{j},a_{j + 1}]$, on associe

\begin{noliste}{$\sbullet$}
\item le nombre $I_{j} = \dint{a_{j}}{a_{j + 1}}f(x)dx$

\item la la variable aléatoire $X^{j}$ de loi de probabilité uniforme
sur
celui-ci (on notera $g_{j}$ sa densité de probabilité);

\item une suite $(X_{i}{j})_{i\in \N^{\times }}$ associée à $X^{j}$

\item une variable aléatoire $Y^{j}$ dont on admettra l'existence,
ayant
pour espérance mathématique et pour moment d'ordre $2$ : 
\[
\E(Y^{j}) = (a_{j + 1}-a_{j})\dint{a_{j}}{a_{j + 1}}f(x)g_{j}(x)dx
\]
\[
\E((Y^{j})^{2}) = (a_{j + 1}-a_{j})^{2}\dint{a_{j}}{a_{j +
1}}f^{2}(x)g_{j}(x)dx
\]

\item une suite $(Y_{i}{j})_{i\in \N^{\times }}$ associée à $Y^{j}$

\item la variable aléatoire $Z_{n_{j}}{j} = {\dfrac{1}{n_{j}}}\Sum{i =
1}{n_{j}}Y_{i}{j}$\quad où $n_{j}\in \N^{\times }$.
\end{noliste}

\noindent On pose $n = \Sum{j = 1}{k}n_{j}$ et on suppose que les
variables aléatoires

\begin{noliste}{$\sbullet$}
\item $X_{j}$ pour $j\in \lbrack
\hspace{-0.15em}[1,k]\hspace{-0.13em}]$
sont indépendantes;

\item $Y_{i}{j}$, pour $j\in \lbrack
\hspace{-0.15em}[1,n_{j}]\hspace{-0.13em}]$ sont indépendantes;

\item $Z_{n_{j}}{j}$, pour $j\in \lbrack
\hspace{-0.15em}[1,k]\hspace{-0.13em}]$ sont indépendantes;
\end{noliste}

\begin{noliste}{1.}
 \setlength{\itemsep}{4mm}
\item Montrer que, quels que soient les entiers $n_{j},i,j$ :\quad
$Y_{i}{j}
$ et $Z_{n_{j}}{j}$ satisfont à la condition $C_{1}(I_{j})$ et que
$Z_{n_{j}}{j}$ satisfait à la condition $C_{2}(\V(Y^{j}))$.

\item On considère la variable aléatoire $Z_{n}{\star
} = \Sum{j = 1}{k}Z_{n_{j}}{j}$. Montrer que $\E(Z_{n}{\star }) = I$.

\item On suppose que pour tout $j\in \lbrack
\hspace{-0.15em}[2,k]\hspace{-0.13em}]\ :\quad na_{j}\in \N^{\times }$.

\begin{noliste}{a)}
 \setlength{\itemsep}{2mm}
\item Démontrer que si $\forall j\in \lbrack
\hspace{-0.15em}[1,k]\hspace{-0.13em}],\quad n_{j} = (a_{j +
1}-a_{j})n$, alors $\V(Z_{n})\leq
\V(Z_{n}{\star })$.

\item En déduire que $\dlim{n\rightarrow \infty }\V(Z_{n}{\star }) =
0$.
\end{noliste}
\end{noliste}

\label{fin}

\end{document}

