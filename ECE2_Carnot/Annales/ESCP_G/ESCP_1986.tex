\documentclass[11pt]{article}%
\usepackage{geometry}%
\geometry{a4paper,
 lmargin = 2cm,rmargin = 2cm,tmargin = 2.5cm,bmargin = 2.5cm}

\input{../../macros.tex}

\pagestyle{fancy} %
\lhead{ECE2 \hfill Mathématiques\\
} %
\chead{\hrule} %
\rhead{} %
\lfoot{} %
\cfoot{} %
\rfoot{\thepage} %

\renewcommand{\headrulewidth}{0pt}% : Trace un trait de séparation
 % de largeur 0,4 point. Mettre 0pt
 % pour supprimer le trait.

\renewcommand{\footrulewidth}{0.4pt}% : Trace un trait de séparation
 % de largeur 0,4 point. Mettre 0pt
 % pour supprimer le trait.

\setlength{\headheight}{14pt}

\title{\bf \vspace{-2cm} ESCP 1986 - voie Générale} %
\author{} %
\date{} %
\begin{document}

\maketitle %
\vspace{-1.4cm}\hrule %
\thispagestyle{fancy}

\vspace*{.2cm}


% DEBUT DU DOC À MODIFIER : tout virer jusqu'au début de l'exo


\begin{center}
{\small CHAMBRE D\E\ COMMERCE ET D'INDUSTRIE DE PARIS}

\textbf{DIRECTION DE L'ENSEIGNEMENT}

Direction des Admissions et concours

\underline{\hspace*{3cm}}

{\Large ECOLE DES\ HAUTES\ ETUDES\ COMMERCIALES}

{\Large E.S.C.P.-E.A.P.}

{\Large ECOL\E\ SUPERIEUR\E\ D\E\ COMMERC\E\ D\E\ LYON}{\large }

CONCOURS D'ADMISSION\ SUR\ CLASSES\ PREPARATOIRES

\underline{\hspace*{3cm}}

\textbf{OPTION GENERALE}

{\Large MATHEMATIQUES I}

\textbf{Année 1986}

\underline{\hspace*{3cm}}
\end{center}

\begin{quotation}
\noindent \textsl{La présentation, la lisibilité, l'orthographe, la
qualité
de la rédaction, la clarté et la précision des raisonnements entreront
pour
une part importante dans l'appréciation des copies.}

\noindent \textsl{Les candidats sont invités à encadrer dans la mesure
du
possible les résultats de leurs calculs.}

\noindent \textsl{Ils ne doivent faire usage d'aucun document :
l'utilisation de toute calculatrice et de tout matériel électronique
est
interdite.}

\noindent \textsl{Seule l'utilisation d'une règle graduée est
autorisée.}

\noindent \textsl{\hrulefill }
\end{quotation}

\noindent Pour tout nombre entier naturel $p$, on considère la fonction
$A_{p}$ définie sur $[0; + \infty \lbrack $ par la relation : 
\[
A_{p}(x) = \Sum{k = 0}{p}(-1)^{k}\dfrac{x^{k}}{k!}
\]
Dans ce problème, on étudie la suite des nombres réels positifs
$(x_{n})$
tels que $A_{2n-1}(x_{n}) = 0$ où $n\geq 1$, ce qui fait l'objet de la 
\textbf{partie III}. Dans les \textbf{parties I et II}, on établit des
résultats auxiliaires.

\section*{Partie I}

Dans cette partie, on étudie un algorithme d'approximation de l'unique
solution de l'équation $f(t) = t$, où, pour tout nombre réel positif
$t$ : 
\[
f(t) = e^{-1-t}
\]

\begin{noliste}{1.}
 \setlength{\itemsep}{4mm}
\item 

\begin{noliste}{a)}
 \setlength{\itemsep}{2mm}
\item Construire sur une même figure les représentations graphiques de
la
fonction $f$ et de la fonction $t\mapsto t$.

\item Montrer que l'équation $f(t) = t$ admet une solution $a$ et une
seule.

\item Montrer que, pour tout couple $(x,y)$ de nombres réels positifs :

\[
\left| f(x)-f(y)\right| \leq \dfrac{1}{e}\left|
x-y\right|
\]

\item Prouver que l'intervalle $\left[ 0;\dfrac{1}{e}\right] $ est
stable
par $f$ et que $a$ appartient à cet intervalle.
\end{noliste}

\item Soit $u$ la suite numérique définie par la relation de récurrence

\[
u_{n + 1} = f(u_{n})
\]
et la condition initiale $u_{0} = 0$.

\begin{noliste}{a)}
 \setlength{\itemsep}{2mm}
\item Montrer que, pour tout nombre entier naturel $n$ : 
\[
0\leq u_{n}\leq \dfrac{1}{e}\qquad \text{et}\qquad \left|
u_{n}-a\right| \leq \dfrac{1}{e^{n + 1}}
\]
En déduire que la suite $u$ converge et donner sa limite.

\item Déterminer un nombre entier naturel $n_{0}$ tel que : 
\[
\left| u_{n_{0}}-a\right| \leq 10^{-6}
\]
Écrire des valeurs décimales approchées à la précision $10^{-6}$ des
termes $u_{n}$, où $1\leq n\leq n_{0}$.
\end{noliste}
\end{noliste}

\section*{Partie II}

On se propose d'étudier les suites $v$ et $w$ définies par les
relations : 
\[
v_{n} = \dfrac{1}{n}\sqrt[n]{n!}\qquad w_{n} = \dfrac{n^{n}}{n!}\qquad
\text{où $n\geq 1$}
\]

\begin{noliste}{1.}
 \setlength{\itemsep}{4mm}
\item Trouver une relation simple entre $\ln (v_{n})$ et $\ln (w_{n})$.

\item \textit{Minoration de }$w$\textit{\ }

\begin{noliste}{a)}
 \setlength{\itemsep}{2mm}
\item Montrer que, pour tout nombre entier naturel non nul $n$ : 
\[
\dfrac{w_{n + 1}}{w_{n}} = \left( 1 + \dfrac{1}{n}\right) ^{n}
\]

\item Montrer que, pour tout nombre entier naturel non nul $n$ : 
\[
\left( 1 + \dfrac{1}{n}\right) ^{n}\geq 2
\]

\item En déduire que, si $n\geq 6$, alors $w_{n}\geq 2^{n}$.

\item Déterminer un majorant de la suite $(v_{n})_{n\geq 6}$.
\label{maj}
\end{noliste}

\item \textit{Convergence de la suite }$v$

\begin{noliste}{a)}
 \setlength{\itemsep}{2mm}
\item Déterminer la limite de la suite : $(\ln (w_{n + 1})-\ln
(w_{n}))$.

\item Établir que, pour tout nombre réel $x\geq 0$ : 
\[
0\leq x-\ln (1 + x)\leq \dfrac{x^{2}}{2}
\]
En déduire que : 
\[
0\leq 1 + \ln (w_{n})-\ln (w_{n + 1})\leq \dfrac{1}{2n}
\]

\item Établir que, pour tout nombre réel $x$ appartenant à $[0;1[$ : 
\[
x\leq -\ln (1-x)
\]
En déduire que : 
\[
1 + \dfrac{1}{2} + \ldots + \dfrac{1}{n}\leq 1 + \ln (n)
\]

\item Prouver finalement que $\dlim{n\rightarrow + \infty
}\dfrac{1}{n}\ln (w_{n}) = 1$.\\
En déduire la limite de la suite $v$.\\
Peut-on retrouver ainsi la majoration obtenue dans la question
\textbf{II}.\ref{maj} ?
\end{noliste}
\end{noliste}

\section*{Partie III : \emph{Étude de la suite $(x_{n})$}}

\begin{noliste}{1.}
 \setlength{\itemsep}{4mm}
\item 

\begin{noliste}{a)}
 \setlength{\itemsep}{2mm}
\item Prouver que, pour tout nombre entier naturel $p$ et pour tout
nombre réel positif $x$ : 
\[
e^{-x} = A_{p}(x) + (-1)^{p + 1}I_{p}(x)\qquad \text{où}\qquad
I_{p}(x) = \dint{0}{x}e^{-t}\dfrac{(x-t)^{p}}{p!}dt
\]

\item En déduire que, pour tout nombre entier naturel $n$ et pour tout
nombre réel positif $x$ : 
\begin{equation}
A_{2n + 1}(x)\leq e^{-x}\leq A_{2n}(x) \label{exp(-x)}
\end{equation}

\item Exprimer la dérivée $A_{p + 1}{\prime }$ en fonction de $A_{p}$.

\item Prouver que, pour tout nombre entier naturel non nul $n$, la
fonction $A_{2n-1}$ est strictement décroissante et que, sur
l'intervalle $[0; + \infty
\lbrack $, l'équation $A_{2n-1}(x) = 0$ admet une solution $x_{n}$ et
une
seule.\\
Calculer $A_{2n}{\prime }(x_{n})$ et dresser le tableau de variation de
$A_{2n-1}$ et de $A_{2n}$.
\end{noliste}

\item Soit $n$ un nombre entier naturel non nul.

\begin{noliste}{a)}
 \setlength{\itemsep}{2mm}
\item Montrer que : 
\[
A_{2n}(x_{n}) = \dfrac{x_{n}{2n}}{(2n)!}\qquad \text{et}\qquad
A_{2n + 1}(x_{n}) = \dfrac{x_{n}{2n}}{(2n)!}\left( 1-\dfrac{x_{n}}{2n +
1}\right)
\]

\item En déduire que $\dfrac{x_{n}{2n}}{(2n)!}\leq 1$. À l'aide de la
majoration établie au \textbf{II}.\ref{maj}, montrer que si $n\geq 3$,
$x_{n}\leq n$. Vérifier directement que ce dernier résultat est encore
valable si $n = 1$ et si $n = 2$.

\item Montrer que $A_{2n + 1}(x_{n})>0$. En déduire que la suite
$(x_{n})$ est
strictement croissante.
\end{noliste}

\item Soit $n$ un nombre entier naturel non nul.

\begin{noliste}{a)}
 \setlength{\itemsep}{2mm}
\item À l'aide de (\ref{exp(-x)}) et de la majoration $x_{n}\leq n$,
établir l'encadrement : 
\[
1\leq \dfrac{x_{n}{2n}}{(2n)!}e^{x_{n}}\leq 2
\]

\item On pose $y_{n} = \dfrac{x_{n}}{2n}$. Montrer que : 
\[
v_{2n}\leq y_{n}e^{y_{n}}\leq 2^{1/2n}v_{2n}
\]

\item En déduire que la suite de terme général $z_{n} = y_{n}e^{y_{n}}$
converge vers $\dfrac{1}{e}$.
\end{noliste}

\item En conclure que la suite $(y_{n})$ converge vers $a$ (On étudiera
à
cet effet la fonction $y\mapsto ye^{y}$))
\end{noliste}

\label{fin}

\end{document}

