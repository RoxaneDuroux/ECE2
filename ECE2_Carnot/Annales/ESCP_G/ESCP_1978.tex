\documentclass[11pt]{article}%
\usepackage{geometry}%
\geometry{a4paper,
 lmargin = 2cm,rmargin = 2cm,tmargin = 2.5cm,bmargin = 2.5cm}

\input{../../macros.tex}

\pagestyle{fancy} %
\lhead{ECE2 \hfill Mathématiques\\
} %
\chead{\hrule} %
\rhead{} %
\lfoot{} %
\cfoot{} %
\rfoot{\thepage} %

\renewcommand{\headrulewidth}{0pt}% : Trace un trait de séparation
 % de largeur 0,4 point. Mettre 0pt
 % pour supprimer le trait.

\renewcommand{\footrulewidth}{0.4pt}% : Trace un trait de séparation
 % de largeur 0,4 point. Mettre 0pt
 % pour supprimer le trait.

\setlength{\headheight}{14pt}

\title{\bf \vspace{-2cm} ESCP 1978} %
\author{} %
\date{} %
\begin{document}

\maketitle %
\vspace{-1.4cm}\hrule %
\thispagestyle{fancy}

\vspace*{.2cm}


% DEBUT DU DOC À MODIFIER : tout virer jusqu'au début de l'exo


\begin{center}
{\small CHAMBRE D\E\ COMMERCE ET D'INDUSTRIE DE PARIS}

\textbf{DIRECTION DE L'ENSEIGNEMENT}

Direction des Admissions et concours

\underline{\hspace*{3cm}}

{\Large ECOLE DES\ HAUTES\ ETUDES\ COMMERCIALES}

{\Large E.S.C.P.-E.A.P.}

{\Large ECOL\E\ SUPERIEUR\E\ D\E\ COMMERC\E\ D\E\ LYON}{\large }

CONCOURS D'ADMISSION\ SUR\ CLASSES\ PREPARATOIRES

\underline{\hspace*{3cm}}

\textbf{OPTION GENERALE}

{\Large MATHEMATIQUES I}

\textbf{Année 1978}

\underline{\hspace*{3cm}}
\end{center}

\begin{quotation}
\noindent \textsl{La présentation, la lisibilité, l'orthographe, la
qualité
de la rédaction, la clarté et la précision des raisonnements entreront
pour
une part importante dans l'appréciation des copies.}

\noindent \textsl{Les candidats sont invités à encadrer dans la mesure
du
possible les résultats de leurs calculs.}

\noindent \textsl{Ils ne doivent faire usage d'aucun document :
l'utilisation de toute calculatrice et de tout matériel électronique
est
interdite.}

\noindent \textsl{Seule l'utilisation d'une règle graduée est
autorisée.}

\noindent \textsl{\hrulefill }
\end{quotation}

\noindent Dans ce problème, toutes les fonctions envisagées sont des
fonctions d'une variable réelle $x$ à valeurs dans $\R.$\\
Soit $F$ une telle fonction; on note, si elles existent
:\begin{eqnarray*}
F(x + 0) & = & \dlim{h\downarrow 0}F(x + h) \\
F(x-0) & = & \dlim{h\downarrow 0}F(x-h)
\end{eqnarray*}La notation $h\downarrow 0$ signifiant que $h$ tend vers
zéro par valeurs
positives. On rappelle qu'une fonction $F,$ définie au point $x$, est
continue à gauche en $x$ si et seulement si : $F(x) = F(x-0).$

\section*{PARTI\E\ I}

On considère la fonction $F$ définie par :
\[
\left\{ 
\begin{array}{cc}
F(x) = 0 & x\leq a \\
F(x) = \dfrac{x}{8} & a<x\leq b \\
F(x) = \dfrac{x^{2}}{16} & b<x\leq c \\
F(x) = 1 & c<x
\end{array}
\right. 
\]
$a,$ $b,$ $c$ étant trois nombres réels satisfaisant à : $a<b<c.$

\begin{noliste}{1.}
 \setlength{\itemsep}{4mm}
\item 

\begin{noliste}{a)}
 \setlength{\itemsep}{2mm}
\item Montrer que, quels que soient $a,$ $b$ et $c,$ $F$ est continue à
gauche sur $\R.$

\item A quelles conditions doivent satisfaire $a,$ $b,$ c pour que $F$
soit
continue pour toutes valeurs de $x$ ? Est-elle dérivable quel que soit
$x$ ?
\end{noliste}

\item Déterminer les conditions sur $a,$ $b,$ $c$ pour que $F$ soit
non-décroissante; montrer que, si ces conditions sont réalisées, $F$
peut alors être considérée comme la fonction de répartition d'une
variable aléatoire $X$ :
\[
F(x) = P\left(\Ev{X\leq x}\right)
\]
\end{noliste}

\section*{PARTI\E\ II}

Dans cette seconde partie, on pose $a = 0,$ $b = 2,$ $c = 4.$ On note
$X$ la
variable aléatoire ayant $F$ pour fonction de répartition.

\begin{noliste}{1.}
 \setlength{\itemsep}{4mm}
\item Étudier la fonction $F$ et tracer sa représentation graphique.

\item Étudier la fonction $F^{\prime }$ dérivée de $F$ par rapport à
$x.$

\item En désignant par $F_{g}{\prime }$ la dérivée à gauche de $F,$
définie
quel que soit $x,$ on note $f$ la fonction telle que :
\[
\left\{ 
\begin{array}{ll}
f(x) = F^{\prime }(x) & \text{en toute valeur }x\text{ où }F^{\prime
}\text{
est définie} \\
f(x) = F_{g}{\prime }(x) & \text{pour tout valeur }x_{i}\text{ où
}F^{\prime }\text{ n'est pas définie}
\end{array}
\right. 
\]
Montrer que $f$ est continue à gauche, qu'elle est intégrable sur tout
intervalle fermé de $\R$ et que $F(x) = \dint{0}{x}f(t)dt =
\dint{-\infty }{x}f(t)dt.$\\
La fonction $f$ apparait ainsi, et on l'admettra, comme la fonction de
densité de la variable aléatoire $X.$

\item Calculer l'espérance mathématique $\E(X)$ et la variance $\V(X)$
de $X.$

\item 

\begin{noliste}{a)}
 \setlength{\itemsep}{2mm}
\item Étudier la variation de la fonction $G$ de la variable $u$
définie
dans $\R$ par :
\[
G(u) = \dint{0}{u}[1-F(x)]dx
\]

\item Montrer qu'il existe une valeur $u_{0}$ que l'on précisera, telle
que $G(u) = E(X)$ pour $u\geq u_{0}$
\end{noliste}
\end{noliste}

\section*{PARTI\E\ III}

On suppose dans cette troisième partie que $a,$ $b,$ $c$ satisfont aux
conditions 
\[
\left\{ 
\begin{array}{l}
0<a<b \\
2<b<c<4
\end{array}
\right. 
\]
On pose $\{x_{i}\}$ l'ensemble des points $x_{i}$ de discontinuité de
la
fonction $F$ et $p_{x_{i}} = F(x_{i} + 0)-F(x_{i}-0)$ le saut de la
fonction $F$
au point de discontinuité $x_{i}.$

\begin{noliste}{1.}
 \setlength{\itemsep}{4mm}
\item 

\begin{noliste}{a)}
 \setlength{\itemsep}{2mm}
\item Expliciter les éléments de l'ensemble $\{x_{i}\}.$

\item Calculer toutes les valeurs $p_{x_{i}}.$
\end{noliste}

\item Soit $\Phi $ la fonction définie par $\Phi
(x) = \Sum{x_{i}<x}p_{x_{i}}.,$ la sommation étant étendue à tous les
points de discontinuité de $F$ strictement inférieurs à $x.$\\
Étudier la fonction $\Phi $ et tracer sa représentation graphique (on
pourra, pour ce tracé seulement, choisir $a = 1,$ $b = 2,5,$ $c = 3).$

\item On note $\Psi $ la fonction $\Psi = F-\Phi.$

\begin{noliste}{a)}
 \setlength{\itemsep}{2mm}
\item Montrer que la fonction $\Psi $ est continue et non décroissante.

\item Pour quelles valeurs de $x,$ $\Psi $ n'est-elle pas dérivable ?
\end{noliste}

\item Soit $\alpha = \dlim{x\rightarrow + \infty }\Phi (x)$ et $\beta
 = \dlim{x\rightarrow + \infty }\Psi (x).$

\begin{noliste}{a)}
 \setlength{\itemsep}{2mm}
\item Montrer que $\alpha \leq 1$ et $\beta \leq 1.$

\item Quelle relation existe-t-il entre $\alpha $ et $\beta $ ?
\end{noliste}

\item 

\begin{noliste}{a)}
 \setlength{\itemsep}{2mm}
\item Montrer que l'on peut trouver deux fonctions de répartition l'une
$F_{d}$ en escalier, l'autre $F_{c}$ continue, telles que $F$ soit
décomposable en :
\[
F = \lambda_{1}F_{d} + \lambda_{2}F_{c}
\]
où $\lambda_{1}$ et $\lambda_{2}$ sont deux réels, que l'on
déterminera,
satisfaisant aux conditions 
\[
0\leq \lambda_{1}\leq 1,\qquad 0\leq \lambda_{2}\leq
1\qquad \text{et}\qquad \lambda_{1} + \lambda_{2} = 1
\]

\item Une telle décomposition de $F$ est-elle unique ?
\end{noliste}
\end{noliste}

\label{fin}

\end{document}

