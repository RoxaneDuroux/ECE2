\documentclass[11pt]{article}%
\usepackage{geometry}%
\geometry{a4paper,
 lmargin = 2cm,rmargin = 2cm,tmargin = 2.5cm,bmargin = 2.5cm}

\input{../../macros.tex}

\pagestyle{fancy} %
\lhead{ECE2 \hfill Mathématiques\\
} %
\chead{\hrule} %
\rhead{} %
\lfoot{} %
\cfoot{} %
\rfoot{\thepage} %

\renewcommand{\headrulewidth}{0pt}% : Trace un trait de séparation
 % de largeur 0,4 point. Mettre 0pt
 % pour supprimer le trait.

\renewcommand{\footrulewidth}{0.4pt}% : Trace un trait de séparation
 % de largeur 0,4 point. Mettre 0pt
 % pour supprimer le trait.

\setlength{\headheight}{14pt}

\title{\bf \vspace{-2cm} ESCP 1987 - voie Générale} %
\author{} %
\date{} %
\begin{document}

\maketitle %
\vspace{-1.4cm}\hrule %
\thispagestyle{fancy}

\vspace*{.2cm}


% DEBUT DU DOC À MODIFIER : tout virer jusqu'au début de l'exo


\begin{center}
{\small CHAMBRE D\E\ COMMERCE ET D'INDUSTRIE DE PARIS}

\textbf{DIRECTION DE L'ENSEIGNEMENT}

Direction des Admissions et concours

\underline{\hspace*{3cm}}

{\Large ECOLE DES\ HAUTES\ ETUDES\ COMMERCIALES}

{\Large E.S.C.P.-E.A.P.}

{\Large ECOL\E\ SUPERIEUR\E\ D\E\ COMMERC\E\ D\E\ LYON}{\large }

CONCOURS D'ADMISSION\ SUR\ CLASSES\ PREPARATOIRES

\underline{\hspace*{3cm}}

\textbf{OPTION GENERALE}

{\Large MATHEMATIQUES I}

\textbf{Année 1987}

\underline{\hspace*{3cm}}
\end{center}

\begin{quotation}
\noindent \textsl{La présentation, la lisibilité, l'orthographe, la
qualité
de la rédaction, la clarté et la précision des raisonnements entreront
pour
une part importante dans l'appréciation des copies.}

\noindent \textsl{Les candidats sont invités à encadrer dans la mesure
du
possible les résultats de leurs calculs.}

\noindent \textsl{Ils ne doivent faire usage d'aucun document :
l'utilisation de toute calculatrice et de tout matériel électronique
est
interdite.}

\noindent \textsl{Seule l'utilisation d'une règle graduée est
autorisée.}

\noindent \textsl{\hrulefill }
\end{quotation}

\noindent {Objectif du problème :} dans la première partie, on approche
$\alpha = (1/2)^{1/3}$ à l'aide d'une suite numérique. Dans la seconde,
on
approche sur l'intervalle $[0,1]$ la fonction $t\rightarrow t^{1/3}$ à
l'aide d'une suite de fonctions polynomiales et on évalue la rapidité
de la
convergence.\\
On notera qu'une valeur approchée de $\alpha $ à la précision $10^{-9}$
est $0,793\,700\,526$. 

\section*{Première partie : Approximation de $\protect\alpha $ }

Soit $\lambda $ un nombre réel strictement positif. On considère la
fonction
numérique $f$ définie sur l'intervalle $[0,1]$ par la relation : 
\[
f_{\lambda }(x) = x + \lambda \left( \frac{1}{2}-x^{3}\right) 
\]

\begin{noliste}{1.}
 \setlength{\itemsep}{4mm}
\item Montrer que $\alpha $ est l'unique solution de l'équation
$f_{\lambda
}(x) = x$. 

\item 

\begin{noliste}{a)}
 \setlength{\itemsep}{2mm}
\item Calculer la dérivée de $f_{\lambda }$. Montrer que $f_{\lambda }$
est
croissante sur $[0,1]$ si et seulement si $\lambda \leq \dfrac{1}{3}$. 
\\
On suppose désormais que cette condition est satisfaite. 

\item Prouver que l'intervalle $]\alpha,1]$ est stable par $f_{\lambda
}$,
c'est à dire que : 
\[
f_{\lambda }(]\alpha,1])\subset ]\alpha,1]
\]

\item Montrer que, pour tout élément $x$ de $]\alpha,1]$ : 
\[
0\leq f_{\lambda }(x)-\alpha \leq (x-\alpha )f_{\lambda }{\prime
}(\alpha )
\]
\end{noliste}

\item Soient $c$ un élément de $]\alpha,1]$ et $v$ la suite définie par
la
relation de récurrence : 
\[
v_{n + 1} = v_{n} + \lambda \left( \frac{1}{2}-v_{n}{3}\right) 
\]

\begin{noliste}{a)}
 \setlength{\itemsep}{2mm}
\item Montrer que la suite $v$ est strictement décroissante et qu'elle
converge vers $\alpha $. 

\item Montrer que, pour tout nombre entier naturel $n$ : 
\[
0<v_{n}-\alpha \leq (c-\alpha )\left[ f_{\lambda }{\prime }(\alpha
)\right] ^{n}
\]

\item Montrer que $f_{\lambda }{\prime }(\alpha )$ est minimal si et
seulement si $\lambda = \dfrac{1}{3}$. 
\end{noliste}

\item On suppose que $\lambda = \dfrac{1}{3}$ et on prend $v_{0} = c =
0,8$.
Calculer $v_{n}$ pour $n\leq 8$.\\
Montrer que $0<c-\alpha <7.10^{-3}$ et majorer $v_{n}-\alpha $. (Dans
cette
question, on n'utilisera pas la valeur approchée de $\alpha $ donnée
dans l'énoncé.) 
\end{noliste}

\section*{Deuxième partie : Approximation polynomiale de $t\rightarrow
t^{1/3}$}

Pour tout élément $t$ de l'intervalle $[0,1]$, on considère la suite
$(u_{n}(t))$ définie par la relation de récurrence : 
\[
u_{n + 1}(t) = u_{n}(t) + \frac 13 \left[ t-u_{n}(t)^{3}\right]
\]
et la condition initiale $u_{0}(t) = 0$.

\begin{noliste}{1.}
 \setlength{\itemsep}{4mm}
\item 

\begin{noliste}{a)}
 \setlength{\itemsep}{2mm}
\item Calculer $u_{1}(t)$ et $u_{2}(t)$. 

\item Montrer par récurrence que, pour tout entier $n$, la fonction
$t\rightarrow u_{n}(t)$ est une fonction polynomiale et déterminer son
degré. 
\end{noliste}

\item 

\begin{noliste}{a)}
 \setlength{\itemsep}{2mm}
\item Montrer que, pour tout nombre entier naturel $n$ : 
\[
t^{1/3}-u_{n + 1}(t) = \left[ t^{1/3}-u_{n}(t)\right] \left(
1-\frac{1}{3}\left[
t^{2/3} + t^{1/3}u_{n}(t) + u_{n}(t)^{2}\right] \right) 
\]

\item Montrer que, pour tout nombre entier naturel $n$, $u_{n}(t)\leq
t^{1/3}$. 

\item En déduire que la suite de terme général $u_{n}(t)$ est
croissante et
positive. 

\item Montrer que la suite $(u_{n}(t))_{n\in \mathbf{N}}$ converge vers
$t^{1/3}$. 
\end{noliste}

\item Montrer que pour tout nombre entier naturel $n$ : 
\[
t^{1/3}\left( 1-t^{2/3}\right) ^{n}\leq t^{1/3}-u_{n}(t)\leq
t^{1/3}\left( 1-\frac{1}{3}t^{2/3}\right) ^{n}
\]

\item Pour tout nombre entier naturel non nul $n$, soit $\varphi_{n}$
l'application de $[0,1]$ dans $R$ définie par la relation : 
\[
\varphi_{n}(x) = x\left( 1-\frac{1}{3}x^{2}\right) ^{n}
\]

\begin{noliste}{a)}
 \setlength{\itemsep}{2mm}
\item Étudier les variations de $\varphi_{n}$ et déterminer son
maximum. 

\item On pose : 
\[
\beta_{n} = \sup_{t\in \lbrack 0,1]}\left[ t^{1/3}-u_{n}(t)\right] 
\]
Montrer que : 
\[
\beta_{n}\leq \sqrt{\frac{3}{2n}}
\]

\item Montrer qu'il existe un nombre réel strictement positif $\gamma $
tel
que, pour tout nombre entier naturel non nul $n$ : 
\[
\beta_{n}\geq \frac{\gamma }{\sqrt{n}}
\]

\item Plus précisément, soit $p$ un nombre entier naturel non nul.
Montrer,
que pour tout nombre entier naturel $n$ tel que $n\geq p$, on a : 
\[
e^{-1/2}\,\sqrt{\frac{1}{2n + 1}}<\beta_{n}\leq \left( 1-\frac{1}{2p +
1}\right) ^{p}\sqrt{\frac{3}{2n + 1}}
\]
\end{noliste}
\end{noliste}

\label{fin}

\end{document}

