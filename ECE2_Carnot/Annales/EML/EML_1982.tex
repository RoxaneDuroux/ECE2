\documentclass[11pt]{article}%
\usepackage{geometry}%
\geometry{a4paper,
 lmargin = 2cm,rmargin = 2cm,tmargin = 2.5cm,bmargin = 2.5cm}

\input{../../macros.tex}

\pagestyle{fancy} %
\lhead{ECE2 \hfill Mathématiques\\
} %
\chead{\hrule} %
\rhead{} %
\lfoot{} %
\cfoot{} %
\rfoot{\thepage} %

\renewcommand{\headrulewidth}{0pt}% : Trace un trait de séparation
 % de largeur 0,4 point. Mettre 0pt
 % pour supprimer le trait.

\renewcommand{\footrulewidth}{0.4pt}% : Trace un trait de séparation
 % de largeur 0,4 point. Mettre 0pt
 % pour supprimer le trait.

\setlength{\headheight}{14pt}

\title{\bf \vspace{- 2cm} EML 1982} %
\author{} %
\date{} %
\begin{document}

\maketitle %
\vspace{- 1.4cm}\hrule %
\thispagestyle{fancy}

\vspace*{.2cm}


% DEBUT DU DOC À MODIFIER : tout virer jusqu'au début de l'exo

%Définition et changement de valeurs de
compteurs%newcounter{cpt1}{section} compteur cpt1 remis à 0 à chaque
aumentation par stepcounter du compteur section%setcounter{cpt1}{3} on
met le compteur à 3%addtocounter{cpt1}{5} on ajoute 5 au compteur%
stepcounter{cpt1} on ajoute 1% ifthenelse{test}{alors}{sinon} (page
206) pour subordonner à une condition % whiledo{test}{commande} pour
faire une boucle (page 206 aussi) % value{cpt1} pour noter dans le
document la valeur de cpt1 
%Définition définitive d'opérateurs
mathématiques\newcommand{\ch}{\operatorname{ch}} 
\newcommand{\sh}{\operatorname{sh}}
\renewcommand{\tanh}{\operatorname{th}}
\renewcommand{\sinh}{\operatorname{sh}}
\renewcommand{\cosh}{\operatorname{ch}}
\newcommand{\argsh}{\operatorname{argsh}}
\newcommand{\argch}{\operatorname{argch}}
\newcommand{\argth}{\operatorname{argth}}
\newcommand{\ker}{\operatorname{Ker}}
\renewcommand{\im}{\operatorname{Im}}
\newcommand{\rg}{\operatorname{rg}}
\newcommand{\Id}{\operatorname{Id}}
\newcommand{\id}{\operatorname{id}}
\renewcommand{\leq}{\leq}
\renewcommand{\geq}{\geq }

%Définition de nouvelles couleurs : rgb(trois paramètres red green blue
entre 0 et 1); cmyk (quatre cyan magenta yellow black) entre 0 et 1;
gray (entre 0 et 1) et black, white, red, green, blue, cyan, magenta,
yellow% definecolor{0gris}{gray}{0.8} 
% Nouvelle commande pour encadrer le titre car shabox ne veut que d'une
seule ligne; ATTENTION A LA TAILLE; petite différence avec shadowbox ou
doublebox, voire fcolorbox ou colorbox (au lieu de shabox; laisser le
parbox tranquille sauf pour la taille de la boîte
\newcommand{\Tbox}[1]{\begin{center} \shabox{\parbox{0.6
\linewidth}{#1}} \end{center}} %[1] pour 1 paramètre ; #1 pour ce que
fait le 1er paramètre; entre accolades ce que fait la commande
%Mise en page en mode fancy : en - têtes et pieds de pages puis
définition des en - têtes et pieds de pages\pagestyle{fancy}
\lhead{ECE 2 - Mathématiques \\
Quentin Dunstetter - ENC - Bessières 2011$\backslash$2012}
\chead{}
\rhead{EML 1982}
\rfoot[ \ \thepage]{\thepage}
\cfoot{}
\lfoot{}
\thispagestyle{fancy} %Mise en page de la 1ère page en mode fancy
%Trait en bas et en haut de la page (entre en - tête et texte et texte
et pied de page)\renewcommand{\footrulewidth}{0.4pt}
\renewcommand{\headrulewidth}{0.4pt}


%DEBUT DU DOCUMENT\vspace*{3cm}

\begin{center}
{\LARG\E\textbf{BANQUE COMMUNE D'ÉPREUVES}}



{\large \textsc{CONCOURS D ADMISSION DE 1982}}



{\large \textbf{Concepteur : EML}}



\rule{2.39cm}{0.05cm}



{\Large \textbf{OPTION ÉCONOMIQUE}}



{\Large \textbf{MATHÉMATIQUES }}



{\Large Lundi 9 mai, de 14h à 18h}



\rule{2.39cm}{0.05cm}
\end{center}

\textit{La présentation, la lisibilité, l'orthographe, la qualité
de la rédaction, la clarté et la précision des raisonnements
entreront pour une part importante dans l'appréciation des copies.}

\textit{Les candidats sont invités à \textbf{encadrer} dans la mesure
du possible les résultats de leurs calculs.}

\textit{Ils ne doivent faire usage d'aucun document. L'utilisation de
toute
calculatrice et de tout matériel électronique est interdite. Seule
l'utilisation d'une règle graduée est autorisée.}

\textit{Si au cours de l'épreuve, un candidat repère ce qui lui semble
être une erreur d'énoncé, il la signalera sur sa copie et
poursuivra sa composition en expliquant les raisons des initiatives
qu'il sera
amené à prendre.}

\vspace*{3cm}

\section*{Exercice 1}

On désigne par $E$ l'espace vectoriel sur des polynômes à une
indéterminée $X
$ à coefficients réels, de degré inférieur ou égal à 2. On rappelle que
$E$
est de dimension 3 et que les trois polynômes $1,$ $X,$ $X^{2}$ en
constituent une base que l'on notera $B$.\\
On considère l'application $\varphi $, de $E$ dans $E$, définie par 
\[
\varphi :P\mapsto \varphi (P) = (X^{2} - 1)P^{\prime } - (2X + 1)P.
\]
$P^{\prime }$ désignant le polynôme dérivé de $P$.

\begin{noliste}{1.}
 \setlength{\itemsep}{4mm}
\item Démontrer que $\varphi $ est une application linéaire de $E$ dans
$E$.

\item Écrire la matrice $A$ de $\varphi $ relative à la base $B$.

\item Déterminer les valeurs propres de $\varphi $. En déduire que
$\varphi $
est un isomorphisme.

\item Déterminer les sous - espaces propres de $\varphi $ et montrer
que $\varphi $ est diagonalisable.

\item Déterminer une base $B^{\prime }$ de $E$ dans laquelle la matrice
$A^{\prime }$ de $\varphi $ est diagonale.\\
Écrire la matrice de passage $M$ de la base $B$ à la base $B^{\prime }$
et
la relation entre les matrices $A$, $A^{\prime }$, $M$ et $M^{- 1}$.

\item 

\begin{noliste}{a)}
 \setlength{\itemsep}{2mm}
\item Déduire de 5) la relation
\[
A^{k} = M(A^{\prime })^{k}M^{- 1},\qquad k\in.\N^{\times }
\]

\item Application: déterminer $\varphi ^{k}(1)$, k $\in \N^{\times }$,
sachant que $\varphi ^{k} = \varphi ^{k - -1}\circ \varphi $.
\end{noliste}
\end{noliste}

\section*{Exercice 2}

On considère la fonction numérique de variable réelle f définie sur
$\R\backslash \{0\}$ par $f(x) = 2 + \dfrac{1}{x^{2}}$

\begin{noliste}{1.}
 \setlength{\itemsep}{4mm}
\item 

\begin{noliste}{a)}
 \setlength{\itemsep}{2mm}
\item Etudier la fonction f et donner sa courbe représentative.

\item Montrer que 
\[
\text{si }\left| x\right| \geq 2,\qquad |f^{\prime
}(x)|\leq \dfrac{1}{4}.
\]
\end{noliste}

\item Montrer qu'il existe un nombre $L$ unique tel que $f(L) = L$ et
que $2\leq L\leq 2 + \dfrac{1}{4}$\\
(on pourra étudier la fonction $\varphi :x\mapsto f(x) - x$).

\item Soit un réel $x\geq 2$. En appliquant la formule des
accroissements finis à la fonction f sur $[x,L]$ ou $[L,x]$, montrer
que 
\[
\left| f(x) - L\right| \leq \left| x - L\right|.
\]

\item On considère la suite $n\mapsto u_{n}$ définie par $u_{0}\neq 0$
et la
relation $u_{n + 1} = f(u_{n})$, $n\in.\N$

\begin{noliste}{a)}
 \setlength{\itemsep}{2mm}
\item Montrer que $\forall n\geq 1$, $u_{n}>2$.

\item Montrer que $\forall n\geq 1$, $\left| u_{n + 1} - L\right|
\leq \left| u_{n} - L\right| $.

\item En déduire que, quel que soit $u_{0}\neq 0$, la suite $n\mapsto
u_{n}$
converge vers $L$.
\end{noliste}
\end{noliste}

\section*{Exercice 3}

Introduction: on s'intéresse à la transmission d'une information
binaire,
c'est - à - dire ne pouvant prendre que deux valeurs.\\
On admet que le procédé de transmission directe entre deux individus A
et B
est tel que, lorsque A émet une valeur de l'information à destination
de B,
ce dernier reçoit la valeur émise par A avec la probabilité $p$, et
donc
l'autre valeur avec la probabilité $q = 1 - p$.\\
Dans tout le problème, $p$ est supposé connu et vérifie : $0<p<1$.

\begin{noliste}{1.}
 \setlength{\itemsep}{4mm}
\item On considère des individus successifs $i_{0}$, $i_{1}$,...,\
$i_{n}
$, $n\in \N^{\times }.$\\
L'information émise par $i_{0}$ est transmise à $i_{1}$, qui transmet
la
valeur reçue à $i_{2}$, et ainsi de suite jusqu'à $i_{n}$.\\
Entre deux individus $i_{k}$ et $i_{k + 1}$, $k\in \{0,1,...,n - 1\}$,\
la\
transmission\ de\ l'information\ suit\ la\ loi\ décrite\ dans\
l'introduction.\\
On note $p_{k}$, $k\in \{1,...,n\}$,\ la\ probabilité\ que\ la\ valeur\
de\
l'information\ reçue\ par\ $i_{k}$ soit identique à celle émise par
$i_{0}$, et on pose $p_{0} = 1$.

\begin{noliste}{a)}
 \setlength{\itemsep}{2mm}
\item Trouver la relation liant $p_{k + 1}$ et $p_{k}$, $k\geq 0$
(attention : $i_{k + 1}$ peut recevoir la valeur émise par $i_{0}$ même
lorsque $i_{k}$ a reçu l'autre valeur...).

\item En déduire l'expression de $p_{n}$ en fonction de $n$ et de $p$.

\item Calculer $p_{n}$. Que conclure ?
\end{noliste}

\item Dans cette question, l'individu $i_{0}$ émet l'information à
destination directe (c'est - à - dire sans l'intermédiaire d'autres
individus)
d'une infinité d'individus $\{i_{k}\}$, dont une partie se trouve "à
l'écoute".\\
Le nombre X d'individus à l'écoute suit une loi de Poisson de paramètre
$\lambda >0$.\\
Entre $i_{0}$ et chacun des individus à l'écoute, la transmission de
l'information suit la loi décrite dans l'introduction.

\begin{noliste}{a)}
 \setlength{\itemsep}{2mm}
\item Lorsque $n$ individus sont à l'écoute, quelle est la probabilité
pour
qu'ils soient $k$, $0\leq k\leq n$, à recevoir la valeur de
l'information émise par $i_{0}$, et donc $n - k$ à recevoir l'autre
valeur ?

\item Exprimer en fonction de $p$ et $\lambda $ la probabilité pour
qu'aucun
individu ne reçoive la valeur de l'information émise par $i_{0}$ (c'est
- à - dire qu'aucun individu n'est à l'écoute, ou que tous les
individus à l'écoute reçoivent l'autre valeur d'information).

\item Soit $Y$ la variable aléatoire représentant le nombre d'individus
recevant la valeur de l`information émise par $i_{0}$.\\
Montrer que la loi de probabilité de $Y$ est une loi de Poisson de
paramètre 
$\lambda p$.\\
Calculer $\E(Y)$ et $\V(Y)$, espérance mathématique et variance de $Y$.
\end{noliste}
\end{noliste}

\section*{Exercice 4}

Soit $n$ un entier naturel supérieur ou égal à 3. On pose : 
\[
u_{n} = \dint{0}{\pi /n}\dfrac{\sin x}{1 + \tan x}dx\qquad
\text{et}\qquad v_{n} = \dint{0}{\pi /n}\dfrac{x}{1 + x}dx
\]

\begin{noliste}{1.}
 \setlength{\itemsep}{4mm}
\item Calculer $v_{n}$ et en donner un équivalent simple lorsque $n$
tend
vers l'infini.

\item 

\begin{noliste}{a)}
 \setlength{\itemsep}{2mm}
\item Montrer que, pour $x\in \lbrack 0,\dfrac{\pi }{2}[$ : 
\[
0\leq \sin x\leq x\qquad \text{et}\qquad x\leq \tan x.
\]

\item En déduire que, pour $n\geq 3$, $0\leq u_{n}\leq v_{n}$.
\end{noliste}

\item Etudier la convergence de la série de terme général $v_{n}$, puis
de
la série de terme général $u_{n}$.
\end{noliste}

\begin{center}
FIN
\end{center}

\label{fin}

\end{document}

