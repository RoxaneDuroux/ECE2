\documentclass[11pt]{article}%
\usepackage{geometry}%
\geometry{a4paper,
 lmargin = 2cm,rmargin = 2cm,tmargin = 2.5cm,bmargin = 2.5cm}

\input{../../macros.tex}

\pagestyle{fancy} %
\lhead{ECE2 \hfill Mathématiques\\
} %
\chead{\hrule} %
\rhead{} %
\lfoot{} %
\cfoot{} %
\rfoot{\thepage} %

\renewcommand{\headrulewidth}{0pt}% : Trace un trait de séparation
 % de largeur 0,4 point. Mettre 0pt
 % pour supprimer le trait.

\renewcommand{\footrulewidth}{0.4pt}% : Trace un trait de séparation
 % de largeur 0,4 point. Mettre 0pt
 % pour supprimer le trait.

\setlength{\headheight}{14pt}

\title{\bf \vspace{-2cm} EML 1991} %
\author{} %
\date{} %
\begin{document}

\maketitle %
\vspace{-1.4cm}\hrule %
\thispagestyle{fancy}

\vspace*{.2cm}


% DEBUT DU DOC À MODIFIER : tout virer jusqu'au début de l'exo

%Définition et changement de valeurs de
compteurs%newcounter{cpt1}{section} compteur cpt1 remis à 0 à chaque
aumentation par stepcounter du compteur section%setcounter{cpt1}{3} on
met le compteur à 3%addtocounter{cpt1}{5} on ajoute 5 au compteur%
stepcounter{cpt1} on ajoute 1% ifthenelse{test}{alors}{sinon} (page
206) pour subordonner à une condition % whiledo{test}{commande} pour
faire une boucle (page 206 aussi) % value{cpt1} pour noter dans le
document la valeur de cpt1 
%Définition définitive d'opérateurs
mathématiques\newcommand{\ch}{\operatorname{ch}} 
\newcommand{\sh}{\operatorname{sh}}
\renewcommand{\tanh}{\operatorname{th}}
\renewcommand{\sinh}{\operatorname{sh}}
\renewcommand{\cosh}{\operatorname{ch}}
\newcommand{\argsh}{\operatorname{argsh}}
\newcommand{\argch}{\operatorname{argch}}
\newcommand{\argth}{\operatorname{argth}}
\newcommand{\ker}{\operatorname{Ker}}
\renewcommand{\im}{\operatorname{Im}}
\newcommand{\rg}{\operatorname{rg}}
\newcommand{\Id}{\operatorname{Id}}
\newcommand{\id}{\operatorname{id}}
\renewcommand{\leq}{\leq}
\renewcommand{\geq}{\geq }

%Définition de nouvelles couleurs : rgb(trois paramètres red green blue
entre 0 et 1); cmyk (quatre cyan magenta yellow black) entre 0 et 1;
gray (entre 0 et 1) et black, white, red, green, blue, cyan, magenta,
yellow% definecolor{0gris}{gray}{0.8} 
% Nouvelle commande pour encadrer le titre car shabox ne veut que d'une
seule ligne; ATTENTION A LA TAILLE; petite différence avec shadowbox ou
doublebox, voire fcolorbox ou colorbox (au lieu de shabox; laisser le
parbox tranquille sauf pour la taille de la boîte
\newcommand{\Tbox}[1]{\begin{center} \shabox{\parbox{0.6
\linewidth}{#1}} \end{center}} %[1] pour 1 paramètre ; #1 pour ce que
fait le 1er paramètre; entre accolades ce que fait la commande
%Mise en page en mode fancy : en-têtes et pieds de pages puis
définition des en-têtes et pieds de pages\pagestyle{fancy}
\lhead{ECE 2 - Mathématiques \\
Quentin Dunstetter - ENC-Bessières 2011$\backslash$2012}
\chead{}
\rhead{EML 1991}
\rfoot[ \ \thepage]{\thepage}
\cfoot{}
\lfoot{}
\thispagestyle{fancy} %Mise en page de la 1ère page en mode fancy
%Trait en bas et en haut de la page (entre en-tête et texte et texte et
pied de page)\renewcommand{\footrulewidth}{0.4pt}
\renewcommand{\headrulewidth}{0.4pt}


%DEBUT DU DOCUMENT\vspace*{3cm}

\begin{center}
{\LARG\E\textbf{BANQUE COMMUNE D'ÉPREUVES}}



{\large \textsc{CONCOURS D ADMISSION DE 1991}}



{\large \textbf{Concepteur : EML}}



\rule{2.39cm}{0.05cm}



{\Large \textbf{OPTION ÉCONOMIQUE}}



{\Large \textbf{MATHÉMATIQUES }}



{\Large Lundi 9 mai, de 14h à 18h}



\rule{2.39cm}{0.05cm}
\end{center}

\textit{La présentation, la lisibilité, l'orthographe, la qualité
de la rédaction, la clarté et la précision des raisonnements
entreront pour une part importante dans l'appréciation des copies.}

\textit{Les candidats sont invités à \textbf{encadrer} dans la mesure
du possible les résultats de leurs calculs.}

\textit{Ils ne doivent faire usage d'aucun document. L'utilisation de
toute
calculatrice et de tout matériel électronique est interdite. Seule
l'utilisation d'une règle graduée est autorisée.}

\textit{Si au cours de l'épreuve, un candidat repère ce qui lui semble
être une erreur d'énoncé, il la signalera sur sa copie et
poursuivra sa composition en expliquant les raisons des initiatives
qu'il sera
amené à prendre.}

\vspace*{3cm}
\section*{EXERCICE 1}

Soient $\alpha \in \R_{+}{\times }$, $A(\alpha ) = \left( 
\begin{array}{ccc}
1 & 0 & \alpha \\
0 & 1 & 0 \\
\alpha & 0 & 1
\end{array}
\right) \in \mathfrak{M}_{3}(\R)$, et $f_{\alpha }$ l'endomorphisme
de $\R^{3}$ dont la matrice dans la base canonique de $\R^{3}
$ est $A(\alpha )$.

\begin{noliste}{1.}
 \setlength{\itemsep}{4mm}
\item 

\begin{noliste}{a)}
 \setlength{\itemsep}{2mm}
\item Calculer les valeurs propres de $f_{\alpha }$.

\item $f_{\alpha }$ est-il diagonalisable ?

\item Pour quelles valeurs de $\alpha $, $f_{\alpha }$ est-il un
isomorphisme ?
\end{noliste}

\item Déterminer une base de $\R^{3}$ formée de vecteurs propres de
$f_{\alpha }$.
\end{noliste}

\section*{EXERCICE 2}

\begin{noliste}{1.}
 \setlength{\itemsep}{4mm}
\item On considère la fonction définie sur $\R$ par : $
\begin{array}{ccc}
f :\R & \longrightarrow & \R \\
x & \longmapsto & f(x) = x^{2} + 4x + 2.
\end{array}
$

Étudier les variations de $f$ sur $\R$. Discuter suivant les valeurs
du paramètre réel $\mathfrak{M}$, le nombre de solutions de l'équation
$f(x) = m$. Résoudre $f(x) = -1$.

\item On considère la suite $(u_{n})_{n\in \N}$ déterminée par la
donnée de $u_{0}\in \R$, et la relation de récurrence $u_{n + 1} =
u_{n}{2} + 4u_{n} + 2$ pour tout $n$ entier naturel.

On rappelle la définition suivante : une suite $(v_{n})_{n\in \N}$
est dite stationnaire si elle vérifie : 
\[
\exists N\in \N,\forall n\in \N,\left( n\geq 
N\Rightarrow v_{n + 1} = v_{n}\right).
\]

Montrer qu'il existe trois valeurs de $u_{0}$ pour lesquelles la suite
$(u_{n})_{n\in \N}$ est stationnaire.

\item Montrer que, pour tout entier naturel $n$, $u_{n + 1} + 2 =
(u_{n} + 2)^{2}$.
En déduire la nature de la suite $(u_{n})_{n\in \N}$ suivant les
valeurs de $u_{0}$.
\end{noliste}

\section*{EXERCICE 3}

Une urne contient $p$ jetons numérotés de $1$ à $p$ ($p\geq 2$). On
effectue $N$ tirages successifs ($N\geq 1$) : chaque tirage consiste à
prendre un jeton dans l'urne, noter son numéro, puis remettre le jeton
dans
l'urne.\\
Pour tout entier $i$ compris entre $1$ et $p$, on définit les variables
aléatoires $F_{i}$ et $X_{i}$ comme suit :

\begin{noliste}{$\sbullet$}
\item $F_{i}$ est le nombre de fois où le jeton numéroté $i$ a été tiré

\item $X_{i}$ prend la valeur $0$ si le jeton numéroté $i$ n'a pas été
tiré
et prend la valeur $1$ si le jeton numéroté $i$ a été tiré au moins une
fois.
\end{noliste}

\begin{noliste}{1.}
 \setlength{\itemsep}{4mm}
\item \textbf{Étude des variables aléatoires $F_{i}$.}

\begin{noliste}{a)}
 \setlength{\itemsep}{2mm}
\item Pour tout $i$ compris entre $1$ et $p$, déterminer l'espérance et
la variance de la variable aléatoire $F_{i}$.

\item On considère la variable aléatoire $F = \Sum{i = 1}{p}F_{i}$.

Que vaut $F$ ? Calculer l'espérance et la variance de $F$.

\item Est-ce que les variables aléatoires $F_{i}$ sont deux à deux
indépendantes ?
\end{noliste}

\item \textbf{Étude des variables aléatoires $X_{i}$.}

\begin{noliste}{a)}
 \setlength{\itemsep}{2mm}
\item Pour tout $i$ compris entre $1$ et $p$, déterminer l'espérance et
la variance de la variable aléatoire $X_{i}$.

\item Soient $i$ et $j$ deux entiers distincts compris entre $1$ et
$p$.

Déterminer la probabilité pour que $X_{i} = 0$ sachant que $X_{j} = 0$.

Est-ce que les variables aléatoires $X_{i}$ et $X_{j}$ sont
indépendantes ?

\item Déterminer l'espérance de la variable aléatoire $X = \Sum{i =
1}{p}X_{i}$.
\end{noliste}

\item \textbf{Application}

Vous êtes responsable du service après-vente d'une chaîne de magasins.
Ce service est présent sur quinze sites et, au total, il reçoit en
moyenne
cinquante appels par jour.

\begin{noliste}{a)}
 \setlength{\itemsep}{2mm}
\item En utilisant le début de l'exercice pour modéliser cette
situation, donner une interprétation des variables aléatoires $F_{i}$,
$X_{i} $ et $X$.

\item Calculer des valeurs approchées à $10^{-1}$ près de
l'espérance de $F_{i}$, de l'espérance de $X_{i}$ et de l'espérance de
$X$.

Commenter brièvement ces résultats.
\end{noliste}
\end{noliste}

\section*{EXERCICE 4}

Pour tout entier $n\geq 1$, on pose 
\[
I_{n} = \dint{0}{1}x^{n}\ln (1 + x^{2})dt\quad \quad \text{ et }\quad
\quad J_{n} = \dint{0}{1}\dfrac{x^{n}}{1 + x^{2}}dx.
\]

\begin{noliste}{1.}
 \setlength{\itemsep}{4mm}
\item \textbf{Étude de la suite $(J_{n})_{n\geq 1}$.}

\begin{noliste}{a)}
 \setlength{\itemsep}{2mm}
\item Calculer $J_{1}$.

\item Montrer que, pour tout entier $n\geq 1$, $0\leq
J_{n}\leq \dfrac{1}{n + 1}$.

\item Étudier la convergence de la suite $(J_{n})_{n\geq 1}$.
\end{noliste}

\item \textbf{Étude de la suite $(I_{n})_{n\geq 1}$.}

\begin{noliste}{a)}
 \setlength{\itemsep}{2mm}
\item À l'aide d'une intégration par parties, montrer que, pour tout
entier $n\geq 1$, 
\[
I_{n} = \dfrac{\ln 2}{n + 1}-\dfrac{2}{n + 1}J_{n + 2}.
\]

\item Étudier la convergence de la suite $(I_{n})_{n\geq 1}$.

\item Déterminer un équivalent de $I_{n}$ lorsque $n$ tend vers $ +
\infty $.
\end{noliste}
\end{noliste}

\label{fin}

\end{document}

