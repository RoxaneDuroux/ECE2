\documentclass[11pt]{article}%
\usepackage{geometry}%
\geometry{a4paper,
 lmargin = 2cm,rmargin = 2cm,tmargin = 2.5cm,bmargin = 2.5cm}

\input{../../macros.tex}

\pagestyle{fancy} %
\lhead{ECE2 \hfill Mathématiques\\
} %
\chead{\hrule} %
\rhead{} %
\lfoot{} %
\cfoot{} %
\rfoot{\thepage} %

\renewcommand{\headrulewidth}{0pt}% : Trace un trait de séparation
 % de largeur 0,4 point. Mettre 0pt
 % pour supprimer le trait.

\renewcommand{\footrulewidth}{0.4pt}% : Trace un trait de séparation
 % de largeur 0,4 point. Mettre 0pt
 % pour supprimer le trait.

\setlength{\headheight}{14pt}

\title{\bf \vspace{-2cm} EML 2004} %
\author{} %
\date{} %
\begin{document}

\maketitle %
\vspace{-1.4cm}\hrule %
\thispagestyle{fancy}

\vspace*{.2cm}


% DEBUT DU DOC À MODIFIER : tout virer jusqu'au début de l'exo

%Définition et changement de valeurs de
compteurs%newcounter{cpt1}{section} compteur cpt1 remis à 0 à chaque
aumentation par stepcounter du compteur section%setcounter{cpt1}{3} on
met le compteur à 3%addtocounter{cpt1}{5} on ajoute 5 au compteur%
stepcounter{cpt1} on ajoute 1% ifthenelse{test}{alors}{sinon} (page
206) pour subordonner à une condition % whiledo{test}{commande} pour
faire une boucle (page 206 aussi) % value{cpt1} pour noter dans le
document la valeur de cpt1 
%Définition définitive d'opérateurs
mathématiques\newcommand{\ch}{\operatorname{ch}} 
\newcommand{\sh}{\operatorname{sh}}
\renewcommand{\tanh}{\operatorname{th}}
\renewcommand{\sinh}{\operatorname{sh}}
\renewcommand{\cosh}{\operatorname{ch}}
\newcommand{\argsh}{\operatorname{argsh}}
\newcommand{\argch}{\operatorname{argch}}
\newcommand{\argth}{\operatorname{argth}}
\newcommand{\ker}{\operatorname{Ker}}
\renewcommand{\im}{\operatorname{Im}}
\newcommand{\rg}{\operatorname{rg}}
\newcommand{\Id}{\operatorname{Id}}
\newcommand{\id}{\operatorname{id}}
\renewcommand{\leq}{\leq}
\renewcommand{\geq}{\geq }

%Définition de nouvelles couleurs : rgb(trois paramètres red green blue
entre 0 et 1); cmyk (quatre cyan magenta yellow black) entre 0 et 1;
gray (entre 0 et 1) et black, white, red, green, blue, cyan, magenta,
yellow% definecolor{0gris}{gray}{0.8} 
% Nouvelle commande pour encadrer le titre car shabox ne veut que d'une
seule ligne; ATTENTION A LA TAILLE; petite différence avec shadowbox ou
doublebox, voire fcolorbox ou colorbox (au lieu de shabox; laisser le
parbox tranquille sauf pour la taille de la boîte
\newcommand{\Tbox}[1]{\begin{center} \shabox{\parbox{0.6
\linewidth}{#1}} \end{center}} %[1] pour 1 paramètre ; #1 pour ce que
fait le 1er paramètre; entre accolades ce que fait la commande
%Mise en page en mode fancy : en-têtes et pieds de pages puis
définition des en-têtes et pieds de pages\pagestyle{fancy}
\lhead{ECE 2 - Mathématiques \\
Quentin Dunstetter - ENC-Bessières 2011$\backslash$2012}
\chead{}
\rhead{EML 2004}
\rfoot[ \ \thepage]{\thepage}
\cfoot{}
\lfoot{}
\thispagestyle{fancy} %Mise en page de la 1ère page en mode fancy
%Trait en bas et en haut de la page (entre en-tête et texte et texte et
pied de page)\renewcommand{\footrulewidth}{0.4pt}
\renewcommand{\headrulewidth}{0.4pt}


%DEBUT DU DOCUMENT\vspace*{3cm}

\begin{center}
{\LARG\E\textbf{BANQUE COMMUNE D'ÉPREUVES}}



{\large \textsc{CONCOURS D ADMISSION DE 2004}}



{\large \textbf{Concepteur : EML}}



\rule{2.39cm}{0.05cm}



{\Large \textbf{OPTION ÉCONOMIQUE}}



{\Large \textbf{MATHÉMATIQUES }}



{\Large Lundi 9 mai, de 14h à 18h}



\rule{2.39cm}{0.05cm}
\end{center}

\textit{La présentation, la lisibilité, l'orthographe, la qualité
de la rédaction, la clarté et la précision des raisonnements
entreront pour une part importante dans l'appréciation des copies.}

\textit{Les candidats sont invités à \textbf{encadrer} dans la mesure
du possible les résultats de leurs calculs.}

\textit{Ils ne doivent faire usage d'aucun document. L'utilisation de
toute
calculatrice et de tout matériel électronique est interdite. Seule
l'utilisation d'une règle graduée est autorisée.}

\textit{Si au cours de l'épreuve, un candidat repère ce qui lui semble
être une erreur d'énoncé, il la signalera sur sa copie et
poursuivra sa composition en expliquant les raisons des initiatives
qu'il sera
amené à prendre.}

\vspace*{3cm}


\begin{center}
{\Huge \vspace{1cm} {\LARGE PREMIER EXERCICE} }
\end{center}

On considère l'application $f :\R\rightarrow \R$ définie, pour tout
$t\in \R$ par : 
\[
f\left( t\right) = \frac{2e^{t}}{\sqrt{1 + t^{2}}} 
\]

\begin{noliste}{1.}
 \setlength{\itemsep}{4mm}
\item Dresser le tableau de variation de $f$ sur $\R$ comprenant les
limites de $f$ en $-\infty $ et en $ + \infty $.

\item 
\begin{noliste}{a)}
 \setlength{\itemsep}{2mm}
\item Établir, pour tout $t\in \left[ 0, + \infty \right[ $ :\quad
$2e^{t}-t-t^{2}>0$ $\quad $et$\quad 1 + t\geq \sqrt{1 + t^{2}}$

\item En déduire : 
\[
\forall t\in \left[ 0, + \infty \right[,\quad f\left( t\right) >t 
\]
\end{noliste}

\item On considère la suite réelle $\left( u_{n}\right)_{n\geq 0}$
définie par $u_{0} = 1$ et, pour tout $n\in \N$ : 
\[
u_{n + 1} = f\left( u_{n}\right) 
\]

\begin{noliste}{a)}
 \setlength{\itemsep}{2mm}
\item Établir que $u_{n}$ tend vers $ + \infty $ lorsque $n$ tend vers
$ + \infty $.

\item Écrire un programme en \Scilab{} qui calcule et affiche le plus
petit
entier $n$ tel que $u_{n}>10^{6}$
\end{noliste}

\item On considère l'application $G :\R\rightarrow \R$ définie, pour
tout $x\in \R$ par : 
\[
G\left( x\right) = \dint{-x}{+ x}f\left( t\right\ dt 
\]

\begin{noliste}{a)}
 \setlength{\itemsep}{2mm}
\item Montrer que $G$ est impaire.

\item Montrer que $G$ est de classe $C^{1}$ sur $\R$ et calculer
$G^{\prime }\left( x\right) $ pour tout $x\in \R$.

\item Quelle est la limite de $G\left( x\right) $ lorsque $x$ tend vers
$ + \infty $ ?

\item Étudier le sens de variation de $G$ et dresser le tableau de
variation de $G$ sur $\R$ comprenant les limites de $G$ en $-\infty $
et en $ + \infty $.
\end{noliste}
\end{noliste}

\begin{center}
{\LARGE DEUXI\`{E}ME EXERCICE}
\end{center}

On note $\M{3} $ l'espace vectoriel réel des matrices carrées d'ordre
trois à éléments réels, $I$ la matrice identité de $\M{3},$ $0$
la matrice nulle de $\M{3} $.

On considère, pour toute matrice $A$ de $\M{3} $, les ensembles
$E_{1}\left( A\right) $ et $E_{2}\left( A\right) $
suivants : 
\begin{eqnarray*}
E_{1}\left( A\right) & = & \left\{ M\in \M{3}
;A\,M = M\right\} \\
E_{2}\left( A\right) & = & \left\{ M\in \M{3}
;A^{2}M = AM\right\}
\end{eqnarray*}

\begin{center}
\textbf{Partie I}
\end{center}

\begin{noliste}{1.}
 \setlength{\itemsep}{4mm}
\item Montrer que $E_{1}\left( A\right) $ est un sous-espace vectoriel
de $\M{3} $

On admettra que $E_{2}\left( A\right) $ est aussi un sous-espace
vectoriel
de $\M{3} $

\item 
\begin{noliste}{a)}
 \setlength{\itemsep}{2mm}
\item Établir :\quad $E_{1}\left( A\right) \subset E_{2}\left( A\right)
$

\item Montrer que, si $A$ est inversible, alors $E_{1}\left( A\right)
 = E_{2}\left( A\right) $
\end{noliste}

\item 
\begin{noliste}{a)}
 \setlength{\itemsep}{2mm}
\item Établir que, si $A-I$ est inversible, alors $E_{1}\left( A\right)
 = \left\{ 0\right\} $

\item Un exemple : Soit $B = \left( 
\begin{array}{rrr}
-1 & 1 & 0 \\
0 & -1 & 1 \\
0 & 0 & 2
\end{array}
\right).$ Déterminer $E_{1}\left( B\right) $ et $E_{2}\left( B\right) $
\end{noliste}
\end{noliste}

\begin{center}
\textbf{Partie II}
\end{center}

On considère la matrice $C = \left( 
\begin{array}{rrr}
3 & -2 & -1 \\
1 & 0 & -1 \\
2 & -2 & 0
\end{array}
\right) $

\begin{noliste}{1.}
 \setlength{\itemsep}{4mm}
\item Calculer les valeurs propres et les sous-espaces propres de $C$.

\item En déduire une matrice diagonale $D$, dont les termes diagonaux
sont dans l'ordre croissant, et une matrice inversible $P$, dont les
éléments de la première ligne sont égaux à 1, telles que $C =
P\,D\,P^{-1}$.

\item Soit $M\in \M{3} $. On note $N = P^{-1}M. $

Montrer : $\qquad M\in E_{1}\left( C\right) \Longleftrightarrow N\in
E_{1}\left( D\right) $.

\item Montrer que $N\in E_{1}\left( D\right) $ si et seulement s'il
existe
trois réels $a,\,b,\,c$ tels que $N = \left( 
\begin{array}{rrr}
0 & 0 & 0 \\
a & b & c \\
0 & 0 & 0
\end{array}
\right) $.

\item En déduire l'expression générale des matrices de $E_{1}\left(
C\right) $ et déterminer une base et la dimension de $E_{1}\left(
C\right) $.

\item Donner l'expression générale des matrices de $E_{2}\left(
C\right) $ et déterminer une base et la dimension de $E_{2}\left(
C\right) $.

Est-ce que $E_{1}\left( C\right) = E_{2}\left( C\right) $ ?
\end{noliste}

\begin{center}
{\LARGE TROISI\`{E}ME EXERCICE}
\end{center}

Une urne contient des boules blanches, des boules rouges et des boules
vertes.

\begin{noliste}{$\sbullet$}
\item La proportion de boules blanches est $b$.

\item La proportion de boules rouges est $r$.

\item La proportion de boules vertes est $v$.
\end{noliste}

Ainsi, on a. : $0<b<1,\quad 0<r<1,\quad 0<v<1\quad $avec$\quad b + r +
v = 1$.

On effectue des tirages successifs avec remise et on s'arrête au
premier
changement de couleur.

Pour tout entier naturel $i$ supérieur ou égal à l, on note $B_{i}$
(respectivement $R_{i}\;;\;V_{i}$ ) l'événement "\ la $i$-ème boule
tirée est blanche (respectivement, rouge ; verte)"

On note $X$ la variable aléatoire égale au nombre de tirages effectués.

Par exemple, lorsque le résultat des tirages est $V_{1},V_{2},B_{3}$,
la
variable aléatoire $X$ prend la valeur 3.

\begin{center}
\textbf{Partie I}
\end{center}

\begin{noliste}{1.}
 \setlength{\itemsep}{4mm}
\item Préciser les valeurs possibles de $X$.

\item Montrer : $\forall k\in \mathbb{N-}\left\{ 0,1\right\},\quad
P\left(\Ev{
X = k}\right) = \left( 1-b\right) b^{k-1} + \left( 1-r\right) r^{k-1} +
\left(
1-v\right) v^{k-1}$

\item Montrer que la variable aléatoire $X$ admet une espérance et
que : 
\[
\E\left( X\right) = \frac{1}{1-b} + \frac{1}{1-r} + \frac{1}{1-v}-2 
\]
\end{noliste}

\begin{center}
\textbf{Partie II}
\end{center}

On considère la fonction $f$ de classe $C^{2}$ sur $\left] 0,1\right[
\times \left] 0,1\right[ $ définie par : 
\[
\forall \left( x,y\right) \in \left] 0,1\right[ \ \times \left]
0,1\right[,\,f\left( x,y\right) = \frac{1}{1-x} + \frac{1}{1-y} +
\frac{1}{x + y} 
\]

\begin{noliste}{1.}
 \setlength{\itemsep}{4mm}
\item Calculer, pour tout $\left( x,y\right) \in \left] 0,1\right[ \
\times \left] 0,1\right[,\,\frac{\partial f}{\partial x}\left(
x,y\right) $ et $\frac{\partial f}{\partial y}\left( x,y\right).$

\item Montrer qu'il existe un unique point $I$ de $\left] 0,1\right[ \
\times \left] 0,1\right[ $ en lequel $f$ est susceptible de posséder un
extremum local et déterminer $I.$

\item Montrer que $f$ admet en $I$ un minimum local.

\item 
\begin{noliste}{a)}
 \setlength{\itemsep}{2mm}
\item Exprimer $\E\left( X\right) $ en fonction de $f\left( b,r\right)
$.

\item Que peut-on dire de $\E\left( X\right) $ lorsque $b = r = v =
\frac{1}{3}$ ?
\end{noliste}
\end{noliste}

\begin{center}
\textbf{Partie III}
\end{center}

\begin{noliste}{1.}
 \setlength{\itemsep}{4mm}
\item Montrer que l'intégrale $\dint{2}{+ \infty }\frac{1}{3^{t}}\,dt$
est convergente et déterminer sa
valeur.

On rappelle que $ = e^{t\ln \left( 3\right) }$.

On note $\alpha = \dint{2}{+ \infty }\frac{1}{3^{t}}\,dt$ et on
considère la fonction $g$ définie sur $\R$ par : 
\[
\left\{ 
\begin{array}{cc}
g\left( t\right) = 0 & \text{si }t\in \left] -\infty ;2\right[ \ \\
g\left( t\right) =  \frac{1}{\alpha 3^{t}} & \text{si }t\in \left[ 2; +
\infty \right[
\end{array}
\right. 
\]

\item Vérifier que $g$ est une densité de probabilité.

On note $Y$ une variable aléatoire admettant $g$ comme densité.

\item Montrer que $Y$ admet une espérance et calculer cette espérance.

\item On note $Z$ la variable aléatoire égale à la partie entière de
$Y$. On rappelle que la partie entière d'un nombre réel $x
$ est le plus grand entier inférieur ou égal à $x$.

\begin{noliste}{a)}
 \setlength{\itemsep}{2mm}
\item Déterminer la loi de probabilité de $Z$.

\item Comparer la loi de probabilité de $X$ lorsque $b = r = v =
\frac{1}{3}$ et la loi de probabilité de $Z$.
\end{noliste}
\end{noliste}

\begin{center}
-FIN -
\end{center}

\end{document}

