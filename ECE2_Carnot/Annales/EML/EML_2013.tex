\documentclass[11pt]{article}%
\usepackage{geometry}%
\geometry{a4paper,
 lmargin = 2cm,rmargin = 2cm,tmargin = 2.5cm,bmargin = 2.5cm}

\input{../../macros.tex}

\pagestyle{fancy} %
\lhead{ECE2 \hfill Mathématiques\\
} %
\chead{\hrule} %
\rhead{} %
\lfoot{} %
\cfoot{} %
\rfoot{\thepage} %

\renewcommand{\headrulewidth}{0pt}% : Trace un trait de séparation
 % de largeur 0,4 point. Mettre 0pt
 % pour supprimer le trait.

\renewcommand{\footrulewidth}{0.4pt}% : Trace un trait de séparation
 % de largeur 0,4 point. Mettre 0pt
 % pour supprimer le trait.

\setlength{\headheight}{14pt}

\title{\bf \vspace{-2cm} EML 2013} %
\author{} %
\date{} %
\begin{document}

\maketitle %
\vspace{-1.4cm}\hrule %
\thispagestyle{fancy}

\vspace*{.2cm}


% DEBUT DU DOC À MODIFIER : tout virer jusqu'au début de l'exo

%Définition et changement de valeurs de
compteurs%newcounter{cpt1}{section} compteur cpt1 remis à 0 à chaque
aumentation par stepcounter du compteur section%setcounter{cpt1}{3} on
met le compteur à 3%addtocounter{cpt1}{5} on ajoute 5 au compteur%
stepcounter{cpt1} on ajoute 1% ifthenelse{test}{alors}{sinon} (page
206) pour subordonner à une condition % whiledo{test}{commande} pour
faire une boucle (page 206 aussi) % value{cpt1} pour noter dans le
document la valeur de cpt1 
%Définition définitive d'opérateurs
mathématiques\newcommand{\ch}{\operatorname{ch}} 
\newcommand{\sh}{\operatorname{sh}}
\renewcommand{\tanh}{\operatorname{th}}
\renewcommand{\sinh}{\operatorname{sh}}
\renewcommand{\cosh}{\operatorname{ch}}
\newcommand{\argsh}{\operatorname{argsh}}
\newcommand{\argch}{\operatorname{argch}}
\newcommand{\argth}{\operatorname{argth}}
\newcommand{\ker}{\operatorname{Ker}}
\renewcommand{\im}{\operatorname{Im}}
\newcommand{\rg}{\operatorname{rg}}
\newcommand{\Id}{\operatorname{Id}}
\newcommand{\id}{\operatorname{id}}
\renewcommand{\leq}{\leq}
\renewcommand{\geq}{\geq }

%Définition de nouvelles couleurs : rgb(trois paramètres red green blue
entre 0 et 1); cmyk (quatre cyan magenta yellow black) entre 0 et 1;
gray (entre 0 et 1) et black, white, red, green, blue, cyan, magenta,
yellow% definecolor{0gris}{gray}{0.8} 
% Nouvelle commande pour encadrer le titre car shabox ne veut que d'une
seule ligne; ATTENTION A LA TAILLE; petite différence avec shadowbox ou
doublebox, voire fcolorbox ou colorbox (au lieu de shabox; laisser le
parbox tranquille sauf pour la taille de la boîte
\newcommand{\Tbox}[1]{\begin{center} \shabox{\parbox{0.6
\linewidth}{#1}} \end{center}} %[1] pour 1 paramètre ; #1 pour ce que
fait le 1er paramètre; entre accolades ce que fait la commande
%Mise en page en mode fancy : en-têtes et pieds de pages puis
définition des en-têtes et pieds de pages\pagestyle{fancy}
\lhead{ECE 2 - Mathématiques \\
Quentin Dunstetter - ENC-Bessières 2011$\backslash$2012}
\chead{}
\rhead{EML 2013}
\rfoot[ \ \thepage]{\thepage}
\cfoot{}
\lfoot{}
\thispagestyle{fancy} %Mise en page de la 1ère page en mode fancy
%Trait en bas et en haut de la page (entre en-tête et texte et texte et
pied de page)\renewcommand{\footrulewidth}{0.4pt}
\renewcommand{\headrulewidth}{0.4pt}


%DEBUT DU DOCUMENT\vspace*{0.3cm}

\Tbox{ \begin{center} \Huge \textbf{EM Lyon 2013} \end{center} }

\vspace*{0.5cm}

\section*{EXERCICE 1}

\section*{Partie I - Calcul d'une intégrale dépendant d'un paramètre.}
\noindent On considère l'application $g : [ 0 ; 1] \longrightarrow \R$
définie, pour tout $t \in [ 0 ; 1]$, par : 
\[
g(t) = \left\{ 
\begin{array}{ccl}
 - t \ln t & \text{ si } & 0 < t \leq 1 \\
0 & \text{ si } & t = 0. \\
\end{array}
\right. 
\]
 

\begin{noliste}{1.}
 \setlength{\itemsep}{4mm}

\item Montrer que $g$ est continue sur $[ 0 ; 1]$. \\

\item \`{A} l'aide d'une intégration par parties, calculer, pour tout
$x \in \ ] 0 ; 1[ $, l'intégrale $\dint{x}{1} g(t)\ dt$. \\

\item En déduire que l'intégrale $\dint{0}{1} g(t)\ dt$ converge et que
: 
\[
 \dint{0}{1} g(t)\ dt = \frac{1}{4}.
\]

\end{noliste}

\section*{Partie II - Exemple de densité.}
\noindent On considère l'application $f : \R \longrightarrow \R$
définie, pour tout $t \in \R$ par :
\[
 f(t) = \left\{ 
\begin{array}{cl}
 - t \ln (t) + t^{ 1/3} & \text{ si } 0 < t < 1 \\
0 & \text{ sinon.} \\
\end{array}
\right. 
\]

\begin{noliste}{1.}
 \setlength{\itemsep}{4mm}

\item Montrer que $f$ est continue sur $]-\infty ; 1 [$ et sur $] 1 ; +
\infty [$. \\
Est-ce que $f$ est continue en 1 ? \\

\item Établir que l'intégrale $\dint{-\infty}{+ \infty} f(t)\ dt$
converge et que : $\dint{-\infty}{+ \infty} f(t)\ dt = 1$. \\

\item Montrer que $f$ est une densité. \\

\item \begin{noliste}{a)}
 \setlength{\itemsep}{2mm}

\item Montrer que $f$ est de classe $C^{2}$ sur $ ] 0 ; 1[ $ et
calculer $f'(t)$ et $f''(t)$ pour tout $t \in \ ]0; 1[$. \\

\item En déduire que l'équation $f'(t) = 0$, d'inconnue $t \in \ ] 0 ;
1[$, admet une solution et une seule, notée $\alpha$, et montrer :
$\frac{1}{e} < \alpha < 1$. \\

\item Écrire un programme en -\Scilab{} qui calcule et affiche une
valeur approchée de $\alpha$ à $10^{-3}$ près, mettant en oeuvre
l'algorithme de dichotomie. 

\end{noliste}
\end{noliste}

\section*{Partie III - Calcul d'une fonction de répartition.}
\noindent On admet qu'il existe une variable aléatoire réelle $X$ ayant
$f$ pour densité (l'application $f$ a été définie au début de la partie
II) et on note $F$ la fonction de répartition de $X$.

\begin{noliste}{1.}
 \setlength{\itemsep}{4mm}

\item Calculer, pour tout $x \in \ ] 0 ; 1[$, l'intégrale $\dint{x}{1}
f(t)\ dt$. \\
(On pourra utiliser le résultat obtenu à la question I.2.) \\

\item Calculer $F(x)$ pour tout $x \in \R$. \\

\item Tracer l'allure de la courbe représentative de $F$.

\end{noliste}

\section*{Partie IV - Étude d'extremum local pour une fonction de deux
variables réelles.}
\noindent On note $D$ l'ensemble des couples $(x,y)$ appartenant à $] 0
; + \infty[$ tels que : $x + y < 1$ et $2 x < 1$. \\
On considère l'application $G : D \longrightarrow \R$, de classe
$C^{2}$ sur l'ouvert $D$, définie, pour tout $(x,y) \in D$, par : 
\[
G(x,y) = f(x + y) - \frac{1}{2} f(2x), 
\]
l'application $f$ ayant été définie au début de la partie II.

\begin{noliste}{1.}
 \setlength{\itemsep}{4mm}

\item Représenter l'ensemble $D$. \\

\item Calculer, pour tout $(x,y) \in D$, les dérivées partielles
premières de $G$ en $(x,y)$, en fonction de $x, \ y, \ f'$. \\

\item Soit $(x,y) \in D$. Montrer que $(x,y)$ est un point critique de
$G$ si et seulement si : 
\[
f'(2x) = 0 \ \ \text{ et } \ \ f'(x + y) = 0.
\]

\item En déduire que $G$ admet un point critique et un seul, et qu'il
s'agit de $\left( \frac{\alpha}{2}, \frac{\alpha}{2} \right)$, le réel
$\alpha$ ayant été défini en II 4.b. \\

\item Est-ce que $G$ admet un extremum local ?

\end{noliste}

\section*{EXERCICE 2}
\noindent On note $A = \begin{smatrix}
0 & 0 & 0 & 2 \\
0 & 0 & 1 & 0 \\
0 & 1 & 0 & 0 \\
2 & 0 & 0 & 0 \\
\end{smatrix}
\in M_{4} (\R)$. \\

\begin{noliste}{1.}
 \setlength{\itemsep}{4mm}

\item Est-ce que $A$ est diagonalisable dans $M_{4} (\R)$ ? \\

\item Déterminer les valeurs propres de $A$ et, pour chaque valeur
propre de $A$, déterminer une base du sous-espace propre associé. \\

\item En déduire une matrice diagonale $D \in M_{4} (\R)$, à
coefficients diagonaux rangés dans l'ordre croissant, et une matrice
inversible $P \in M_{4} (\R)$, à coefficients diagonaux tous égaux à 1,
telles que $A = P D P^{-1}$, et calculer $P^{-1}$. \\
\\

On appelle commutant de $A$, et on note $C_{A}$, l'ensemble des
matrices $M$ de $M_{4}(\R)$ telles que : 
\[
AM = MA.
\]
On appelle commutant de $D$, et on note $C_{D}$, l'ensemble des
matrices $N$ de $M_{4}(\R)$ telles que : 
\[
DN = ND.
\]

\item Montrer que $C_{A}$ est un sous-espace vectoriel de $M_{4} (\R)$.
\\

\item Soit $M \in M_{4} (\R)$. On note $N = P^{-1} M P$. Montrer : 
\[
 M \in C_{A} \Longleftrightarrow N \in C_{D}. 
\]

\item Déterminer $C_{D}$, en utilisant les coefficients des matrices.
\\

\item En déduire : 
\[
 C_{A} = \left\{ \begin{smatrix}
a & 0 & 0 & b \\
0 & c & d & 0 \\
0 & d & c & 0 \\
b & 0 & 0 & a \\
\end{smatrix}
\ ; \ (a,b,c,d) \in \R^{4} \right\}. 
\]

\item Déterminer une base de $C_{A}$ et la dimension de $C_{A}$.

\end{noliste}

\section*{EXERCICE 3}
\noindent Soit $n$ un entier supérieur ou égal à 2. \\
On considère une urne $\mathcal{U}$ contenant $n$ boules numérotées de
1 à $n$ et indiscernables au toucher. \\
On effectue une suite de tirages d'une boule avec remise dans l'urne
$\mathcal{U}$.

\section*{Partie I.}
\noindent Soit $k$ un entier supérieur ou égal à 1. Pour tout $i \in
\lb 1 ; n \rb$, on note $X_{i}$ la variable aléatoire égale au nombre
d'obtentions de la boule numéro $i$ au cours des $k$ premiers tirages. 

\begin{noliste}{1.}
 \setlength{\itemsep}{4mm}

\item Soit $i \in \lb 1 ; n \rb$. DOnner la loi de la variable $X_{i}$.
Rappeler l'espérance et la variance de $X_{i}$. \\

\item Les variables aléatoires $X_{1},X_{2}, \dots, X_{n}$ sont-elles
indépendantes ? \\

\item Soit $(i,j) \in \lb 1 ; n \rb^{2}$ tel que $i \neq j$.

\begin{noliste}{a)}
 \setlength{\itemsep}{2mm}

\item Déterminer la loi de la variable aléatoire $X_{i} + X_{j}$.
Rappeler la variance de $X_{i} + X_{j}$. \\

\item En déduire la covariance du couple $(X_{i},X_{j})$. \\

\end{noliste}

\end{noliste}

\noindent Pour tout entier $k$ supérieur ou égal à 1, on note $Z_{k}$
la variable aléatoire égale au nombre de numéros distincts obtenus au
cours des $k$ premiers tirages et on note $\E(Z_{k})$ l'espérance de
$Z_{k}$.

\section*{Partie II.}

\begin{noliste}{1.}
 \setlength{\itemsep}{4mm}

\item Déterminer la loi de la variable aléatoire $Z_{1}$ et la loi de
la variable aléatoire $Z_{2}$. \\
En déduire $\E(Z_{1})$ et $\E(Z_{2})$. \\

\item Soit $k$ un entier supérieur ou égal à 1. 

\begin{noliste}{a)}
 \setlength{\itemsep}{2mm}

\item Déterminer $\Prob\left(\Ev{Z_{k} = 1}\right)$ et déterminer
$\Prob\left(\Ev{Z_{k} = k}\right)$. \\

\item Montrer, pour tout $l \in \lb 1 ; n \rb$ : $\Prob\left(\Ev{Z_{k +
1} = l }\right) = \frac{l}{n} \Prob\left(\Ev{Z_{k} = l}\right) +
\frac{n-l + 1}{n} \Prob\left(\Ev{Z_{k} = l-1}\right)$. \\

\item En déduire : $\E(Z_{k + 1} = \frac{n-1}{n} E(Z_{k}) + 1$. \\

\end{noliste}

\item \begin{noliste}{a)}
 \setlength{\itemsep}{2mm}

\item Montrer que la suite $(v_{k})_{k \geq 1}$, de terme général
$v_{k} = E(Z_{k}) - n$, est une suite géométrique. \\

\item En déduire, pour tout entier $k$ supérieur ou égal à 1 :
$\E(Z_{k}) = n \left( 1 - \left( \frac{n-1}{n} \right)^{k} \right)$.

\end{noliste}

\end{noliste}

\section*{Partie III.}

\noindent On suppose maintenant que $n = 4$; ainsi l'urne $\mathcal{U}$
contient quatre boules numérotées de 1 à 4. \\
Soit $k$ un entier supérieur ou égal à 4. On se propose de déterminer
la loi de $Z_{k}$.

\begin{noliste}{1.}
 \setlength{\itemsep}{4mm}

\item Rappeler la valeur de $\Prob\left(\Ev{Z_{k} = 1}\right)$.
Déterminer $\Prob\left(\Ev{Z_{k} \geq 5}\right)$. \\

\item Montrer : $ \ \ \ \ \Prob\left(\Ev{Z_{k} = 2}\right) = 6
\frac{2^{k} - 2}{4^{k}}$. \\

\item On note, pour tout $i$ de $\lb 1 ; 4 \rb$, $A_{i}$ l'évènement : 
\[
 \text{ \og la boule numéro } i \text{ n'a pas été obtenue au cours des
} k \text{ premiers tirages \fg}. 
\]

\begin{noliste}{a)}
 \setlength{\itemsep}{2mm}

\item Montrer : $ \ \ \Prob\left(\Ev{Z_{k} \leq 3}\right) = 4
P\left(\Ev{A_{1}}\right) - 6 P( A_{1} \cap A_{2}) + 4 P( A_{1} \cap
A_{2} \cap A_{3})$. \\

\item Calculer $ \ \ \Prob\left(\Ev{A_{1}}\right), \ \ \Prob(A_{1} \cap
A_{2})$ et $P(A_{1} \cap A_{2} \cap A_{3})$. \\

\item En déduire : $ \ \ \ \Prob\left(\Ev{Z_{k} \leq 3}\right)$, puis
$\Prob\left(\Ev{Z_{k} = 3}\right)$ et $\Prob\left(\Ev{Z_{k} =
4}\right)$.

\end{noliste}

\end{noliste}















































\end{document}