\documentclass[11pt]{article}%
\usepackage{geometry}%
\geometry{a4paper,
 lmargin = 2cm,rmargin = 2cm,tmargin = 2.5cm,bmargin = 2.5cm}

\input{../../macros.tex}

\pagestyle{fancy} %
\lhead{ECE2 \hfill Mathématiques\\
} %
\chead{\hrule} %
\rhead{} %
\lfoot{} %
\cfoot{} %
\rfoot{\thepage} %

\renewcommand{\headrulewidth}{0pt}% : Trace un trait de séparation
 % de largeur 0,4 point. Mettre 0pt
 % pour supprimer le trait.

\renewcommand{\footrulewidth}{0.4pt}% : Trace un trait de séparation
 % de largeur 0,4 point. Mettre 0pt
 % pour supprimer le trait.

\setlength{\headheight}{14pt}

\title{\bf \vspace{-2cm} EML 2009} %
\author{} %
\date{} %
\begin{document}

\maketitle %
\vspace{-1.4cm}\hrule %
\thispagestyle{fancy}

\vspace*{.2cm}


% DEBUT DU DOC À MODIFIER : tout virer jusqu'au début de l'exo

%Définition et changement de valeurs de
compteurs%newcounter{cpt1}{section} compteur cpt1 remis à 0 à chaque
aumentation par stepcounter du compteur section%setcounter{cpt1}{3} on
met le compteur à 3%addtocounter{cpt1}{5} on ajoute 5 au compteur%
stepcounter{cpt1} on ajoute 1% ifthenelse{test}{alors}{sinon} (page
206) pour subordonner à une condition % whiledo{test}{commande} pour
faire une boucle (page 206 aussi) % value{cpt1} pour noter dans le
document la valeur de cpt1 
%Définition définitive d'opérateurs
mathématiques\newcommand{\ch}{\operatorname{ch}} 
\newcommand{\sh}{\operatorname{sh}}
\renewcommand{\tanh}{\operatorname{th}}
\renewcommand{\sinh}{\operatorname{sh}}
\renewcommand{\cosh}{\operatorname{ch}}
\newcommand{\argsh}{\operatorname{argsh}}
\newcommand{\argch}{\operatorname{argch}}
\newcommand{\argth}{\operatorname{argth}}
\newcommand{\ker}{\operatorname{Ker}}
\renewcommand{\im}{\operatorname{Im}}
\newcommand{\rg}{\operatorname{rg}}
\newcommand{\Id}{\operatorname{Id}}
\newcommand{\id}{\operatorname{id}}
\renewcommand{\leq}{\leq}
\renewcommand{\geq}{\geq }

%Définition de nouvelles couleurs : rgb(trois paramètres red green blue
entre 0 et 1); cmyk (quatre cyan magenta yellow black) entre 0 et 1;
gray (entre 0 et 1) et black, white, red, green, blue, cyan, magenta,
yellow% definecolor{0gris}{gray}{0.8} 
% Nouvelle commande pour encadrer le titre car shabox ne veut que d'une
seule ligne; ATTENTION A LA TAILLE; petite différence avec shadowbox ou
doublebox, voire fcolorbox ou colorbox (au lieu de shabox; laisser le
parbox tranquille sauf pour la taille de la boîte
\newcommand{\Tbox}[1]{\begin{center} \shabox{\parbox{0.6
\linewidth}{#1}} \end{center}} %[1] pour 1 paramètre ; #1 pour ce que
fait le 1er paramètre; entre accolades ce que fait la commande
%Mise en page en mode fancy : en-têtes et pieds de pages puis
définition des en-têtes et pieds de pages\pagestyle{fancy}
\lhead{ECE 2 - Mathématiques \\
Quentin Dunstetter - ENC-Bessières 2011$\backslash$2012}
\chead{}
\rhead{EML 2009}
\rfoot[ \ \thepage]{\thepage}
\cfoot{}
\lfoot{}
\thispagestyle{fancy} %Mise en page de la 1ère page en mode fancy
%Trait en bas et en haut de la page (entre en-tête et texte et texte et
pied de page)\renewcommand{\footrulewidth}{0.4pt}
\renewcommand{\headrulewidth}{0.4pt}


%DEBUT DU DOCUMENT\vspace*{3cm}

\begin{center}
{\LARG\E\textbf{BANQUE COMMUNE D'ÉPREUVES}}



{\large \textsc{CONCOURS D ADMISSION DE 2009}}



{\large \textbf{Concepteur : EML}}



\rule{2.39cm}{0.05cm}



{\Large \textbf{OPTION ÉCONOMIQUE}}



{\Large \textbf{MATHÉMATIQUES }}



{\Large Lundi 9 mai, de 14h à 18h}



\rule{2.39cm}{0.05cm}
\end{center}

\textit{La présentation, la lisibilité, l'orthographe, la qualité
de la rédaction, la clarté et la précision des raisonnements
entreront pour une part importante dans l'appréciation des copies.}

\textit{Les candidats sont invités à \textbf{encadrer} dans la mesure
du possible les résultats de leurs calculs.}

\textit{Ils ne doivent faire usage d'aucun document. L'utilisation de
toute
calculatrice et de tout matériel électronique est interdite. Seule
l'utilisation d'une règle graduée est autorisée.}

\textit{Si au cours de l'épreuve, un candidat repère ce qui lui semble
être une erreur d'énoncé, il la signalera sur sa copie et
poursuivra sa composition en expliquant les raisons des initiatives
qu'il sera
amené à prendre.}

\vspace*{3cm}

\section*{EXERCICE 1}

On note $f :\R\rightarrow \R$ l'application définie, pour
tout $x\in \R$, par : 
\[
f\left( x\right) = \left\{ 
\begin{array}{cc}
\dfrac{x}{e^{x}-1} & \text{ si }x\neq 0 \\
1 & \text{ si }x = 0
\end{array}
\right. 
\]

\subsection*{Partie I : Étude d'une fonction}

\begin{noliste}{1.}
 \setlength{\itemsep}{4mm}
\item 
\begin{noliste}{a)}
 \setlength{\itemsep}{2mm}
\item Montrer que $f$ est continue sur $\R$.

\item Justifier que $f$ est de classe $C^{1}$ sur $\left] -\infty
;0\right[ $
et sur $\left] 0; + \infty \right[ $ et calculer $f^{\prime }\left(
x\right) $pour tout $x\in \left] -\infty ;0\right[ \ \cup \left] 0; +
\infty \right[ $.

\item Montrer : \texttt{\hspace*{1cm}}$f^{\prime }\left( x\right)
\underset{x\rightarrow 0}{\rightarrow }-\dfrac{1}{2}$

\item Établir que $f$ est de classe $C^{1}$ sur $\R$ et préciser
$f^{\prime }\left( 0\right) $.
\end{noliste}

\item 
\begin{noliste}{a)}
 \setlength{\itemsep}{2mm}
\item Étudier les variations de l'application $u :\R\rightarrow 
\R$, définie, pour tout $x\in \R$, par 
\[
u\left( x\right) = \left( 1-x\right) e^{x}-1
\]

\item Montrer :\texttt{\hspace*{1cm}}$\forall x\in \R,\ f^{\prime
}\left( x\right) <0.$

\item Déterminer les limites de $f$ en $-\infty $ et en $ + \infty $ 
\\
Dresser le tableau des variations de $f$.

\item Montrer que la courbe représentative de $f$ admet une droite
asymptote, lorsque la variable tend vers $-\infty $.

\item Tracer l'allure de la courbe représentative de $f$.
\end{noliste}
\end{noliste}

\subsection*{Partie II : Étude d'une suite récurrente associée 
à la fonction $_{.}f$}

On considère la suite $\left( u_{n}\right)_{n\in \N}$, définie par
$u_{0} = 1$ et, pour tout $n\in \N,\ u_{n + 1} = f\left(
u_{n}\right) $.

\begin{noliste}{1.}
 \setlength{\itemsep}{4mm}
\item Montrer que $f$ admet un point fixe et un seul, noté $\alpha $,
que l'on calculera.

\item 
\begin{noliste}{a)}
 \setlength{\itemsep}{2mm}
\item Établir :\texttt{\hspace*{1cm}}$ \forall x\in \left[
0; + \infty \right[,\ e^{2x}-2x~e^{x}-1\geq 0$

\item Montrer :\texttt{\hspace*{1cm}}$ \forall x\in \left]
0; + \infty \right[,\ f^{\prime }\left( x\right) + \dfrac{1}{2} =
\dfrac{e^{2x}-2x~e^{x}-1}{2\left( e^{x}-1\right) ^{2}}$

\item Montrer :\texttt{\hspace*{1cm}}$ \forall x\in \left[
0; + \infty \right[,\ -\dfrac{1}{2}\leq f^{\prime }\left( x\right) <0.$

\item Établir :\texttt{\hspace*{1cm}}$ \forall n\in \N,\ \left| u_{n +
1}-\alpha \right| \leq \frac{1}{2}\left|
u_{n}-\alpha \right| $
\end{noliste}

\item En déduire :\texttt{\hspace*{1cm}}$\forall n\in \N,\mathbb{\
}\left| u_{n}-\alpha \right| \leq \dfrac{1}{2^{n}}\left( 1-\alpha
\right) $

\item Conclure que la suite $\left( u_{n}\right)_{n\in \N}$
converge vers $\alpha $.

\item Écrire un programme en -\Scilab{} qui calcule et affiche le plus
petit entier naturel $n$ tel que$\left| u_{n}-\alpha \right| <10^{-9}
$
\end{noliste}

\subsection*{Partie III : Étude d'une fonction définie par une
intégrale}

On note $G :\R\rightarrow \R$ l'application définie, pour
tout $x\in \R,$ par :
\[
G\left( x\right) = \dint{x}{2x}f\left( t\right\ dt
\]

\begin{noliste}{1.}
 \setlength{\itemsep}{4mm}
\item Montrer que $G$ est de classe $C^{1}$ sur $\R$ et que, pour
tout $x\in \R :$
\[
G^{\prime }\left( x\right) = \left\{ 
\begin{array}{cc}
\dfrac{x\left( 3-e^{x}\right) }{e^{2x}-1} & \text{si }x\neq 0 \\
1 & \text{ si }x = 0
\end{array}
\right. 
\]

\item 
\begin{noliste}{a)}
 \setlength{\itemsep}{2mm}
\item Montrer :\texttt{\hspace*{1cm}}$\forall x\in \left[ 0; + \infty
\right[,\ 0\leq G\left( x\right) \leq x~f\left( x\right) $.\\
En déduire la limite de $G$ en $ + \infty $.

\item Montrer :\texttt{\hspace*{1cm}}$\forall x\in \left] -\infty
;0\right],\ G\left( x\right) \leq x~f\left( x\right).$\\
En déduire la limite de $G$ en $-\infty $.
\end{noliste}

\item Dresser le tableau des variations de $G$. On n'essaiera pas de
calculer $G\left( \ln 3\right) $.
\end{noliste}

\section*{EXERCICE 2}

On considère les matrices carrées d'ordre trois :\hspace{5mm} $A = 
\begin{smatrix}
0 & 1 & 3 \\
0 & 1 & 3 \\
0 & 0 & 4
\end{smatrix}
$ et $D = 
\begin{smatrix}
0 & 0 & 0 \\
0 & 1 & 0 \\
0 & 0 & 4
\end{smatrix}
$

\subsection*{ Partie I : Réduction de A}

\begin{noliste}{1.}
 \setlength{\itemsep}{4mm}
\item Est-ce que $A$ est inversible ?

\item Déterminer les valeurs propres de $A$.\\
Justifier, sans calcul, que $A$ est diagonalisable.

\item Déterminer une matrice carrée $P$ d'ordre trois, inversible,
dont tous les termes diagonaux sont égaux à 1, telle que $A =
P~D~P^{-1}$ et calculer $P^{-1}$.
\end{noliste}

\subsection*{Partie II : Résolution de l'équation $M^{2} = A$}

On se propose de résoudre l'équation $\left( 1\right) :M^{2} = A$,
d'inconnue $M$, matrice carrée d'ordre trois.\\
Soit $M$ une matrice carrée d'ordre trois. On note $N = P^{-1}M~P$. (La
matrice $P$ a été définie en \textbf{I.3.})

\begin{noliste}{1.}
 \setlength{\itemsep}{4mm}
\item Montrer :\texttt{\hspace*{1cm}}$M^{2} = A\Longleftrightarrow
N^{2} = D$.

\item Établir que, si $N^{2} = D$, alors $N~D = D~N$.

\item En déduire que, si $N^{2} = D$, alors $N$ est diagonale.

\item Déterminer toutes les matrices diagonales $N$ telles que $N^{2} =
D$.

\item En déduire la solution $B$ de l'équation $\left( 1\right) $
dont toutes les valeurs propres sont positives ou nulles.
\end{noliste}

\subsection*{Partie III : Intervention d'un polyn\^{o}me}

\begin{noliste}{1.}
 \setlength{\itemsep}{4mm}
\item Montrer qu'il existe un polyn\^{o}me $Q$ de degré deux, et un
seul, que l'on calculera, tel que :
\[
Q\left( 0\right) = 0,\quad Q\left( 1\right) = 1,\quad Q\left( 4\right)
= 2.
\]

\item En déduire :\texttt{\hspace*{1cm}}$-\dfrac{1}{6}A^{2} +
\dfrac{7}{6}A = B$. (La matrice $B$ a été définie en \textbf{II.5.})

\item Montrer, pour toute matrice carrée $F$ d'ordre trois :{}
\end{noliste}

\[
A~F = F~A\Longleftrightarrow B~F = F~B.
\]

\section*{EXERCICE 3}

Une urne contient des boules blanches et des boules noires. La
proportion de
boules blanches est $p$ et la proportion de boules noires est $q$.

Ainsi, on a : $0<p<1,$ $0<q<1$et $p + q = 1$.

\subsection*{Partie I : Tirages avec arrêt dès qu'une boule noire a 
été obtenue}

Dans cette partie, on effectue des tirages successifs avec remise et on
s'arrête dès que l'on a obtenu une boule noire.

On note $T$ la variable aléatoire égale au nombre de tirages effectués
et $U$ la variable aléatoire égale au nombre de boules
blanches obtenues.

\begin{noliste}{1.}
 \setlength{\itemsep}{4mm}
\item Reconna\^{\i}tre la loi de $T$. Pour tout entier $k\geq 1$,
donner $\Prob\left(\Ev{ T = k}\right) $ et rappeler l'espérance et la
variance de $T$.

\item En déduire que $U$ admet une espérance et une variance.
Déterminer $\E\left( U\right) $ et $\V\left( U\right) $.
\end{noliste}

\subsection*{Partie II : Tirages avec arrêt dès qu'une boule blanche
et une boule noire ont été obtenues}

Dans cette partie, on effectue des tirages successifs avec remise et on
s'arrête dès que l'on a obtenu au moins une boule blanche et au moins
une
boule noire.

On note $X$ la variable aléatoire égale au nombre de tirages effectués.

On note $Y$ la variable aléatoire égale au nombre de boules blanches
obtenues. 

On note $Z$. la variable aléatoire égale au nombre de boules noires
obtenues.

Ainsi, on peut remarquer que la probabilité de l'événement $\left( Y =
1\right) \cup \left( Z = 1\right) $ est égale à $1$. 

Pour tout entier naturel non nul $i$, on note :

\texttt{\hspace*{1cm}}$B_{i}$ l'événement "la $i$-ème boule tirée est
blanche", 

\texttt{\hspace*{1cm}}$N_{i}$ l'événement "la $i$-ème boule tirée est
noire".

\begin{noliste}{1.}
 \setlength{\itemsep}{4mm}
\item 
\begin{noliste}{a)}
 \setlength{\itemsep}{2mm}
\item Montrer, pour tout entier $k\geq 2 :\Prob\left(\Ev{ X = k}\right)
 = q~p^{k-1} + p~q^{k-1}$.

\item Vérifier :\ $ \Sum{k = 2}{+ \infty }\Prob\left(\Ev{ X = k}\right)
= 1$

\item Montrer que la variable aléatoire $X$ admet une espérance et
que : $\E\left( X\right) = \dfrac{1}{p} + \dfrac{1}{q}-1$.
\end{noliste}

\item 
\begin{noliste}{a)}
 \setlength{\itemsep}{2mm}
\item Pour tout entier $k\geq 2$, déterminer $\Prob\left( \left(
X = k\right) \cap \left( Y = 1\right) \right) $\\
\emph{(On distinguera les cas }$k = 2$\emph{\ et }$k\geq 3$\emph{.)}

\item En déduire $ :\Prob\left(\Ev{ Y = 1}\right) = q\left( 1 +
p\right) $.

\item Déterminer la loi de la variable aléatoire $Y$.
\end{noliste}

On admet que l'espérance de $Y$ existe et que : $\E\left( Y\right) =
\dfrac{1}{q}\left( 1-p + p^{2}\right) $.

\item Donner la loi de $Z$ et son espérance.

\item Montrer que les variables aléatoires $Y~Z$ et $X-1$ sont égales.

\item Montrer que le couple $\left( Y,Z\right) $ admet une covariance
et
exprimer $\rm{cov}\left( Y,Z\right) $ à l'aide de $\E\left( X\right),\
\E\left( Y\right) $ et $\E\left( Z\right).$
\end{noliste}

\label{fin}



\end{document}

