\documentclass[11pt]{article}%
\usepackage{geometry}%
\geometry{a4paper,
 lmargin = 2cm,rmargin = 2cm,tmargin = 2.5cm,bmargin = 2.5cm}

\input{../../macros.tex}

\pagestyle{fancy} %
\lhead{ECE2 \hfill Mathématiques\\
} %
\chead{\hrule} %
\rhead{} %
\lfoot{} %
\cfoot{} %
\rfoot{\thepage} %

\renewcommand{\headrulewidth}{0pt}% : Trace un trait de séparation
 % de largeur 0,4 point. Mettre 0pt
 % pour supprimer le trait.

\renewcommand{\footrulewidth}{0.4pt}% : Trace un trait de séparation
 % de largeur 0,4 point. Mettre 0pt
 % pour supprimer le trait.

\setlength{\headheight}{14pt}

\title{\bf \vspace{-2cm} EML 1989} %
\author{} %
\date{} %
\begin{document}

\maketitle %
\vspace{-1.4cm}\hrule %
\thispagestyle{fancy}

\vspace*{.2cm}


% DEBUT DU DOC À MODIFIER : tout virer jusqu'au début de l'exo

%Définition et changement de valeurs de
compteurs%newcounter{cpt1}{section} compteur cpt1 remis à 0 à chaque
aumentation par stepcounter du compteur section%setcounter{cpt1}{3} on
met le compteur à 3%addtocounter{cpt1}{5} on ajoute 5 au compteur%
stepcounter{cpt1} on ajoute 1% ifthenelse{test}{alors}{sinon} (page
206) pour subordonner à une condition % whiledo{test}{commande} pour
faire une boucle (page 206 aussi) % value{cpt1} pour noter dans le
document la valeur de cpt1 
%Définition définitive d'opérateurs
mathématiques\newcommand{\ch}{\operatorname{ch}} 
\newcommand{\sh}{\operatorname{sh}}
\renewcommand{\tanh}{\operatorname{th}}
\renewcommand{\sinh}{\operatorname{sh}}
\renewcommand{\cosh}{\operatorname{ch}}
\newcommand{\argsh}{\operatorname{argsh}}
\newcommand{\argch}{\operatorname{argch}}
\newcommand{\argth}{\operatorname{argth}}
\newcommand{\ker}{\operatorname{Ker}}
\renewcommand{\im}{\operatorname{Im}}
\newcommand{\rg}{\operatorname{rg}}
\newcommand{\Id}{\operatorname{Id}}
\newcommand{\id}{\operatorname{id}}
\renewcommand{\leq}{\leq}
\renewcommand{\geq}{\geq }

%Définition de nouvelles couleurs : rgb(trois paramètres red green blue
entre 0 et 1); cmyk (quatre cyan magenta yellow black) entre 0 et 1;
gray (entre 0 et 1) et black, white, red, green, blue, cyan, magenta,
yellow% definecolor{0gris}{gray}{0.8} 
% Nouvelle commande pour encadrer le titre car shabox ne veut que d'une
seule ligne; ATTENTION A LA TAILLE; petite différence avec shadowbox ou
doublebox, voire fcolorbox ou colorbox (au lieu de shabox; laisser le
parbox tranquille sauf pour la taille de la boîte
\newcommand{\Tbox}[1]{\begin{center} \shabox{\parbox{0.6
\linewidth}{#1}} \end{center}} %[1] pour 1 paramètre ; #1 pour ce que
fait le 1er paramètre; entre accolades ce que fait la commande
%Mise en page en mode fancy : en-têtes et pieds de pages puis
définition des en-têtes et pieds de pages\pagestyle{fancy}
\lhead{ECE 2 - Mathématiques \\
Quentin Dunstetter - ENC-Bessières 2011$\backslash$2012}
\chead{}
\rhead{EML 1989}
\rfoot[ \ \thepage]{\thepage}
\cfoot{}
\lfoot{}
\thispagestyle{fancy} %Mise en page de la 1ère page en mode fancy
%Trait en bas et en haut de la page (entre en-tête et texte et texte et
pied de page)\renewcommand{\footrulewidth}{0.4pt}
\renewcommand{\headrulewidth}{0.4pt}


%DEBUT DU DOCUMENT\vspace*{3cm}

\begin{center}
{\LARG\E\textbf{BANQUE COMMUNE D'ÉPREUVES}}



{\large \textsc{CONCOURS D ADMISSION DE 1989}}



{\large \textbf{Concepteur : EML}}



\rule{2.39cm}{0.05cm}



{\Large \textbf{OPTION ÉCONOMIQUE}}



{\Large \textbf{MATHÉMATIQUES }}



{\Large Lundi 9 mai, de 14h à 18h}



\rule{2.39cm}{0.05cm}
\end{center}

\textit{La présentation, la lisibilité, l'orthographe, la qualité
de la rédaction, la clarté et la précision des raisonnements
entreront pour une part importante dans l'appréciation des copies.}

\textit{Les candidats sont invités à \textbf{encadrer} dans la mesure
du possible les résultats de leurs calculs.}

\textit{Ils ne doivent faire usage d'aucun document. L'utilisation de
toute
calculatrice et de tout matériel électronique est interdite. Seule
l'utilisation d'une règle graduée est autorisée.}

\textit{Si au cours de l'épreuve, un candidat repère ce qui lui semble
être une erreur d'énoncé, il la signalera sur sa copie et
poursuivra sa composition en expliquant les raisons des initiatives
qu'il sera
amené à prendre.}

\vspace*{3cm}

\section*{EXERCICE 1}

Soit $A = M(f,\mathcal{B}) = \left( 
\begin{array}{ccc}
-1 & 0 & 1 \\
0 & 0 & 0 \\
1 & 0 & -1
\end{array}
\right) $, $\mathcal{B}$ base canonique de $\R^{3}$.

\begin{noliste}{1.}
 \setlength{\itemsep}{4mm}
\item Déterminer une base et la dimension de $\ker (f)$, de
$\operatorname{Im}(f)$.

\item Calculer les valeurs propres et les vecteurs propres de $f$. $f$
est-il diagonalisable ? \\
f est-il un automorphisme de $\R^{3}$ ?

\item Calculer $A^{n}$, pour $n\in \N^{\times }$.

\item Déterminer tous les réels $x$ tels que $(A-x.I)^{2} = I$ (I
matrice unité). Existe-t-il un réel $x$ tel que $(A-x.I)^{3} = I$ ?
\end{noliste}

\section*{EXERCICE 2}

Soit $f$ définie par $f(x) = \sqrt{\dfrac{x}{2-x}}$, $I$ son ensemble
de définition, $\mathcal{C}$ sa courbe représentative dans un repère
orthomormé
(unité 2 cm).

\begin{noliste}{1.}
 \setlength{\itemsep}{4mm}
\item 

\begin{noliste}{a)}
 \setlength{\itemsep}{2mm}
\item Étudier les variations de $f$. Préciser les tangentes à
$\mathcal{C}$
aux points d'abscisses 0 et 1.

\item Démontrer : $\forall x\in I,\quad f(x)\geq x.$ Cas d'égalité ?

\item Tracer $\mathcal{C}$ et $D$ d'équation $y = x$.
\end{noliste}

\item 

\begin{noliste}{a)}
 \setlength{\itemsep}{2mm}
\item Démontrer que $f$ réalise une bijection de $[0,2[$ sur $[0, +
\infty
\lbrack $.

\item Expliciter $f$ et tracer sa courbe représentative dans le même
repère
que $f$
\end{noliste}

\item On considère la suite $u$ définie par $u_{n + 1} = f(u_{n})$.

\begin{noliste}{a)}
 \setlength{\itemsep}{2mm}
\item Montrer : $\forall n\in \N,\quad u_{n}\in \lbrack 0,1]$. 

\item Prouver que u est croissante.

\item Démontrer que u converge et calculer sa limite.
\end{noliste}
\end{noliste}

\section*{EXERCIC\E\ 3}

Soit un nombre réel $a$. On considère la fonction $f$ définie sur $\R
$ par
\[
\left\{ 
\begin{array}{cc}
f(x) = a2^{x} & \text{si }x<0 \\
f(x) = a2^{-x} & \text{si }x\geq 0
\end{array}
\right. 
\]

\begin{noliste}{1.}
 \setlength{\itemsep}{4mm}
\item Déterminer a pour que $f$ soit la densité d'une variable
aléatoire $X$ 
à valeurs réelles. \\
Dans la suite de cet exercice, on prend $a = \dfrac{\ln 2}{2}$.

\item 

\begin{noliste}{a)}
 \setlength{\itemsep}{2mm}
\item Calculer, si elle existe, l'espérance de $X$. Déterminer la
fonction
de répartition $F$ de la variable aléatoire $X$. Tracer la courbe
représentative de $F$.

\item Soit un nombre réel $x$. Calculer la probabilité conditionnelle
de l'évènement $(X<x)$ sachant que l'évènement $(X\geq 1)$ est réalisé.
\end{noliste}

\item Déterminer la fonction de répartition $G$ de la variable
aléatoire $Y = 2^{X/2}$.
\end{noliste}

\section*{EXERCIC\E\ 4}

Soit $I$ la suite de terme général $I_{n} = \dint{0}{1}x^{n}e^{-x}dx$

\begin{noliste}{1.}
 \setlength{\itemsep}{4mm}
\item 

\begin{noliste}{a)}
 \setlength{\itemsep}{2mm}
\item Calculer $I_{0}$ et $I_{1}$.

\item Montrer que pour tout entier naturel $n$, $I_{n}\leq \dfrac{1}{n
+ 1}$. Étudier la convergence de la suite $I$.
\end{noliste}

\item Calcul d'une valeur approchée de $I_{15}$.

\begin{noliste}{a)}
 \setlength{\itemsep}{2mm}
\item Montrer que 
\[
\forall n\in \N,\quad I_{n + 1} = (n + 1)I_{n}-\dfrac{1}{e},\quad
\text{et}\quad :I_{n} = \dfrac{n!}{e}\Sum{k = 1}{p}\dfrac{1}{(n + k)!}
+ \dfrac{n!}{(n + p)!}I_{n + p}
\]

\item En déduire que pour tout $n$ dans $\N$ : 
\[
0\leq I_{n}-\dfrac{n!}{e}\Sum{k = 1}{p}\dfrac{1}{(n + k)!}\leq
\dfrac{n!}{(n + p + 1)!}\leq \dfrac{1}{(n + 1)^{p + 1}}
\]

\item Comment peut-on choisir $p$ pour que $0\leq
I_{15}-\dfrac{15!}{e}\Sum{k = 1}{p}\dfrac{1}{(15 + k)!}<10^{-6}$ ? \\
En déduire à l'aide de la calculatrice une valeur approchée de $I_{15}$
à $10^{-6}$ près.
\end{noliste}
\end{noliste}

\label{fin}

\end{document}

