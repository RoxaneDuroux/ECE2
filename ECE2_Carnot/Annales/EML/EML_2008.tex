\documentclass[11pt]{article}%
\usepackage{geometry}%
\geometry{a4paper,
 lmargin = 2cm,rmargin = 2cm,tmargin = 2.5cm,bmargin = 2.5cm}

\input{../../macros.tex}

\pagestyle{fancy} %
\lhead{ECE2 \hfill Mathématiques\\
} %
\chead{\hrule} %
\rhead{} %
\lfoot{} %
\cfoot{} %
\rfoot{\thepage} %

\renewcommand{\headrulewidth}{0pt}% : Trace un trait de séparation
 % de largeur 0,4 point. Mettre 0pt
 % pour supprimer le trait.

\renewcommand{\footrulewidth}{0.4pt}% : Trace un trait de séparation
 % de largeur 0,4 point. Mettre 0pt
 % pour supprimer le trait.

\setlength{\headheight}{14pt}

\title{\bf \vspace{-2cm} EML 2008} %
\author{} %
\date{} %
\begin{document}

\maketitle %
\vspace{-1.4cm}\hrule %
\thispagestyle{fancy}

\vspace*{.2cm}


% DEBUT DU DOC À MODIFIER : tout virer jusqu'au début de l'exo

%Définition et changement de valeurs de
compteurs%newcounter{cpt1}{section} compteur cpt1 remis à 0 à chaque
aumentation par stepcounter du compteur section%setcounter{cpt1}{3} on
met le compteur à 3%addtocounter{cpt1}{5} on ajoute 5 au compteur%
stepcounter{cpt1} on ajoute 1% ifthenelse{test}{alors}{sinon} (page
206) pour subordonner à une condition % whiledo{test}{commande} pour
faire une boucle (page 206 aussi) % value{cpt1} pour noter dans le
document la valeur de cpt1 
%Définition définitive d'opérateurs
mathématiques\newcommand{\ch}{\operatorname{ch}} 
\newcommand{\sh}{\operatorname{sh}}
\renewcommand{\tanh}{\operatorname{th}}
\renewcommand{\sinh}{\operatorname{sh}}
\renewcommand{\cosh}{\operatorname{ch}}
\newcommand{\argsh}{\operatorname{argsh}}
\newcommand{\argch}{\operatorname{argch}}
\newcommand{\argth}{\operatorname{argth}}
\newcommand{\Id}{\operatorname{Id}}
\renewcommand{\leq}{\leq}
\renewcommand{\geq}{\geq }

\newcommand{\dlim}{\lim}
\newcommand{\dsum}{\sum}
\newcommand{\dprod}{\prod}



%Définition de nouvelles couleurs : rgb(trois paramètres red green blue
entre 0 et 1); cmyk (quatre cyan magenta yellow black) entre 0 et 1;
gray (entre 0 et 1) et black, white, red, green, blue, cyan, magenta,
yellow% definecolor{0gris}{gray}{0.8} 
% Nouvelle commande pour encadrer le titre car shabox ne veut que d'une
seule ligne; ATTENTION A LA TAILLE; petite différence avec shadowbox ou
doublebox, voire fcolorbox ou colorbox (au lieu de shabox; laisser le
parbox tranquille sauf pour la taille de la boîte
\newcommand{\Tbox}[1]{\begin{center} \shabox{\parbox{0.6
\linewidth}{#1}} \end{center}} %[1] pour 1 paramètre ; #1 pour ce que
fait le 1er paramètre; entre accolades ce que fait la commande
%Mise en page en mode fancy : en-têtes et pieds de pages puis
définition des en-têtes et pieds de pages\pagestyle{fancy}
\lhead{ECE 2 - Mathématiques \\
Quentin Dunstetter - ENC-Bessières 2011$\backslash$2012}
\chead{}
\rhead{EML 2008}
\rfoot[ \ \thepage]{\thepage}
\cfoot{}
\lfoot{}
\thispagestyle{fancy} %Mise en page de la 1ère page en mode fancy
%Trait en bas et en haut de la page (entre en-tête et texte et texte et
pied de page)\renewcommand{\footrulewidth}{0.4pt}
\renewcommand{\headrulewidth}{0.4pt}

\begin{center}
{\huge EML Eco 2008 }
\end{center}

\section*{EXERCICE 1}

On admet l'encadrement suivant :\hspace{5mm} $2,7<e<2,8$..

\subsection*{Partie I : Étude d'une fonction}

On considère l'application $f :\left[ 0, + \infty \right[ \ \rightarrow

\R$ \ définie, pour tout $t\in \left[ 0, + \infty \right[ $ par :
\[
f\left( t\right) = \left\{ 
\begin{array}{cc}
t\ln \left( t\right) -t & \text{si }t\neq 0 \\
0 & \text{si }t = 0
\end{array}
\right. 
\]

\begin{noliste}{1.}
 \setlength{\itemsep}{4mm}
\item Montrer que $f$ est continue sur $\left[ 0, + \infty \right[ $.

\item Justifier que $f$ est de classe $C^{1}$ sur $\left] 0, + \infty
\right[ $
et calculer $f^{\prime }\left( t\right) $ pour tout $t\in \left] 0, +
\infty \right[ $

\item Déterminer la limite de $f$ en $ + \infty $.

\item Dresser le tableau des variations de $f$.

\item Montrer que $f$ est convexe sur $\left] 0, + \infty \right[ $.

\item On note $\Gamma $ la courbe représentative de $f$ dans un repère
orthonormal $\left( O\ ;\ \vec{i}\ ;\ \vec{j}\right) $

\begin{noliste}{a)}
 \setlength{\itemsep}{2mm}
\item Montrer que $\Gamma $ admet une demi-tangente en $O$.

\item Déterminer les points d'intersection de $\Gamma $ avec l'axe des
abscisses.

\item Préciser la nature de la branche infinie de $\Gamma $.

\item Tracer $\Gamma $
\end{noliste}
\end{noliste}

\subsection*{Partie II : Étude d'une fonction définie par une
intégrale}

On considère l'application $G :\left] 1\ ; + \infty \right[ \
\rightarrow 
\R$ définie, pour tout $x\in \left] 1\ ; + \infty \right[ $, par : 
\[
G\left( x\right) = \frac{1}{2}\dint{x-1}{x + 1}f\left( t\right\ dt 
\]

\begin{noliste}{1.}
 \setlength{\itemsep}{4mm}
\item Montrer que $G$ est de classe $C^{2}$ sur $\left] 1\ ; + \infty
\right[ $
et que, pour tout $x\in \left] 1\ ; + \infty \right[ $
:\begin{eqnarray*}
G^{\prime }\left( x\right) & = & \frac{1}{2}\left( f\left( x + 1\right)
-f\left(
x-1\right) \right) \\
\text{et }G^{\prime \prime }\left( x\right) & = & \frac{1}{2}\left( \ln
\left(
x + 1\right) -\ln \left( x-1\right) \right)
\end{eqnarray*}

\`{A} cet effet, on pourra faire intervenir une primitive $F$ de $f$
sans
chercher à calculer $F$.

\item 
\begin{noliste}{a)}
 \setlength{\itemsep}{2mm}
\item Montrer que $G^{\prime }$ est strictement croissante sur $\left]
1\
; + \infty \right[.$

\item Vérifier : $G^{\prime }\left( 2\right) >0$.

\item Établir que l'équation $G^{\prime }\left( x\right) = 0$,
d'inconnue $x\in \left] 1\ ; + \infty \right[ $, admet une solution et
une
seule, notée $\alpha $, et que $\alpha <2$
\end{noliste}
\end{noliste}

\subsection*{Partie III : Étude d'une fonction de deux variables
réelles}

On considère l'application $\Phi :\left] 1\ ; + \infty \right[
^{2}\rightarrow \R$ définie, pour tout $\left( x,y\right) \in \left] 1\
; + \infty \right[ ^{2}$, par : 
\[
\Phi \left( x,y\right) = \left( y-f\left( x + 1\right) \right) ^{2} +
\left(
y-f\left( x-1\right) \right) ^{2} 
\]
où l'application $f$ est définie dans la partie \textbf{I}.

\begin{noliste}{1.}
 \setlength{\itemsep}{4mm}
\item Justifier que $\Phi $ est de classe $C^{2}$ sur $\left] 1\ ; +
\infty \right[ ^{2}$ et calculer les dérivées partielles premières de
$\Phi $ en tout $\left( x,y\right) $ de $\left] 1\ ; + \infty \right[
^{2}$.

\item Vérifier que $\left( \alpha,f\left( \alpha + 1\right) \right) $
est un point critique de $\Phi $. où $\alpha $ est défini en \textbf{II
2.c}.

\item Est-ce que $\Phi $ admet un extremum local en $\left(
\alpha,f\left(
\alpha + 1\right) \right) $ ?
\end{noliste}

\section*{EXERCICE 2}

On considère les matrices carrées d'ordre trois suivantes :
\[
A = 
\begin{smatrix}
1 & 1 & 1 \\
0 & 0 & -1 \\
-2 & -2 & -1
\end{smatrix},\quad B = 
\begin{smatrix}
2 & 1 & 1 \\
-3 & -2 & -1 \\
1 & 1 & 0
\end{smatrix},\quad D = 
\begin{smatrix}
0 & 0 & 0 \\
0 & -1 & 0 \\
0 & 0 & 1
\end{smatrix}
\]

\subsection*{Partie I : Réduction simultanée de $A$ et $B$}

\begin{noliste}{1.}
 \setlength{\itemsep}{4mm}
\item Déterminer les valeurs propres et les sous-espaces propres de
$A$.

\item En déduire une matrice carrée $P$ d'ordre trois, inversible,
de deuxième ligne $\left( -1\hspace{5mm}1\hspace{5mm}1\right) $, telle
que $A = P~D~P^{-1}$, et calculer $P^{-1}$.

\item Calculer la matrice $C = P^{-1}B~P$ et vérifier que $C$ est
diagonale.
\end{noliste}

\subsection*{Partie II : Étude d'un endomorphisme d'un espace de
matrices}

On note $E$ l'espace vectoriel des matrices carrées d'ordre trois, et
on
considère l'application $f :\E\rightarrow E$ qui, à toute matrice $M$
carrée d'ordre trois, associe $f\left( M\right) = AM-MB.$

\begin{noliste}{1.}
 \setlength{\itemsep}{4mm}
\item Donner la dimension de $E$.

\item Vérifier que $f$ est un endomorphisme de $E$.

\item Soit $M\in E$. On note $N = P^{-1}M~P$, où $P$ est définie en 
\textbf{I.2}.

\begin{noliste}{a)}
 \setlength{\itemsep}{2mm}
\item Montrer : $M\in \ker \left( f\right) \Longleftrightarrow DN =
NC.$

\item Déterminer les matrices $N$ carrées d'ordre trois telles que : 
$DN = NC.$

\item Montrer que l'ensemble des matrices $N$ carrées d'ordre trois
telles que $DN = NC$ est un espace vectoriel, et en déterminer une base
et
la dimension.
\end{noliste}

\item 
\begin{noliste}{a)}
 \setlength{\itemsep}{2mm}
\item En déduire la dimension de $\ker \left( f\right) $, puis la
dimension de $\operatorname{Im}\left( f\right) $.

\item Donner au moins un élément non nul de $\ker \left( f\right) $
et donner au moins un élément non nul de
$\operatorname{Im}(f).$\newpage
\end{noliste}
\end{noliste}

\section*{EXERCICE 3}

Les parties \textbf{I} et \textbf{II} sont indépendantes.

\subsection*{Partie I : Étude d'une variable aléatoire}

\begin{noliste}{1.}
 \setlength{\itemsep}{4mm}
\item Soit $h$ la fonction définie sur l'intervalle $\left[ 0;1\right]
$
par :
\[
\forall x\in \left[ 0;1\right],\quad h\left( x\right) = \frac{x}{2-x} 
\]

\begin{noliste}{a)}
 \setlength{\itemsep}{2mm}
\item Montrer que $h$ est une bijection de $\left[ 0;1\right] $ sur
$\left[
0;1\right] $ et, pour tout $y\in \left[ 0;1\right] $, exprimer
$h^{-1}\left(
y\right).$

\item Déterminer deux réels $\alpha $ et $\beta $ vérifiant : $\forall
x\in \left[ 0;1\right],\quad h\left( x\right) = \alpha + \frac{\beta 
}{2-x}$

\item Calculer $\dint{0}{1}h\left( x\right\dx$.
\end{noliste}

\item Soit $X$ une variable aléatoire suivant la loi uniforme sur
l'intervalle $\left[ 0;1\right] $

\begin{noliste}{a)}
 \setlength{\itemsep}{2mm}
\item Donner l'espérance et la variance de la variable aléatoire $X$.

\item Pour tout réel $y$ de $\left[ 0;1\right] $ déterminer la
probabilité de l'événement $\left( \frac{X}{2-X}\leq y\right) $

\item Montrer que la variable aléatoire $Y = \frac{X}{2-X}$ admet une
densité et déterminer une densité de $Y.$

\item Montrer que $Y$ admet une espérance et déterminer $\E\left(
Y\right) $.
\end{noliste}
\end{noliste}

\subsection*{Partie II : Étude d'un temps d'attente}

Soit $n$ un entier supérieur ou égal à 2. Une réunion est prévue entre
$n$ invités que l'on note $I_{1},\ I_{2}\cdots,I_{n}$. 
\\
Chaque invité arrivera entre l'instant $0$ et l'instant $1$.

Pour tout entier $k$ tel que $1\leq k\leq n$, on modélise l'instant
d'arrivée de l'invité $I_{k}$ par une variable aléatoire $T_{k}$
de loi uniforme sur l'intervalle $\left[ 0;1\right] $. On suppose de
plus
que, pour tout réel $t$, les $n$ événements $\left( T_{1}\leq
t\right) $,\ $\left( T_{2}\leq t\right) $,\ $\cdots \left( T_{n}\leq
t\right) $,\ sont indépendants.

\begin{noliste}{1.}
 \setlength{\itemsep}{4mm}
\item Soit un réel $t$ appartenant à $\left[ 0;1\right] $. Pour tout
entier $k$ tel que $1\leq k\leq n$, on note $B_{k}$ la variable
aléatoire de Bernoulli prenant la valeur $1$ si l'événement $\left(
T_{k}\leq t\right) $ est réalisé et la valeur $0$ sinon.

On note $S_{t} = B_{1} + B_{2} + \cdots + B_{n}.$

\begin{noliste}{a)}
 \setlength{\itemsep}{2mm}
\item Que modélise la variable aléatoire $S_{t}$ ?

\item Déterminer la loi de la variable aléatoire $S_{t}$.
\end{noliste}

\item Soit $R_{1}$ la variable aléatoire égale à l'instant de la
première arrivée.

\begin{noliste}{a)}
 \setlength{\itemsep}{2mm}
\item Soit un réel $t$ appartenant à $\left[ 0;1\right] $. Comparer
l'événement $\left( R_{1}>t\right) $ et l'événement $\left(
S_{t} = 0\right) $

\item Montrer que la variable aléatoire $R_{1}$ admet une densité et
en déterminer une.
\end{noliste}

\item Soit $R_{2}$ la variable aléatoire égale à l'instant de la
deuxième arrivée.\\
Montrer que la variable aléatoire $R_{2}$ admet une densité et en
déterminer une.
\end{noliste}

\end{document}

