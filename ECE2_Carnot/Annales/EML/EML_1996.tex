\documentclass[11pt]{article}%
\usepackage{geometry}%
\geometry{a4paper,
 lmargin = 2cm,rmargin = 2cm,tmargin = 2.5cm,bmargin = 2.5cm}

\input{../../macros.tex}

\pagestyle{fancy} %
\lhead{ECE2 \hfill Mathématiques\\
} %
\chead{\hrule} %
\rhead{} %
\lfoot{} %
\cfoot{} %
\rfoot{\thepage} %

\renewcommand{\headrulewidth}{0pt}% : Trace un trait de séparation
 % de largeur 0,4 point. Mettre 0pt
 % pour supprimer le trait.

\renewcommand{\footrulewidth}{0.4pt}% : Trace un trait de séparation
 % de largeur 0,4 point. Mettre 0pt
 % pour supprimer le trait.

\setlength{\headheight}{14pt}

\title{\bf \vspace{-2cm} EML 1996} %
\author{} %
\date{} %
\begin{document}

\maketitle %
\vspace{-1.4cm}\hrule %
\thispagestyle{fancy}

\vspace*{.2cm}


% DEBUT DU DOC À MODIFIER : tout virer jusqu'au début de l'exo

%Définition et changement de valeurs de
compteurs%newcounter{cpt1}{section} compteur cpt1 remis à 0 à chaque
aumentation par stepcounter du compteur section%setcounter{cpt1}{3} on
met le compteur à 3%addtocounter{cpt1}{5} on ajoute 5 au compteur%
stepcounter{cpt1} on ajoute 1% ifthenelse{test}{alors}{sinon} (page
206) pour subordonner à une condition % whiledo{test}{commande} pour
faire une boucle (page 206 aussi) % value{cpt1} pour noter dans le
document la valeur de cpt1 
%Définition définitive d'opérateurs
mathématiques\newcommand{\ch}{\operatorname{ch}} 
\newcommand{\sh}{\operatorname{sh}}
\renewcommand{\tanh}{\operatorname{th}}
\renewcommand{\sinh}{\operatorname{sh}}
\renewcommand{\cosh}{\operatorname{ch}}
\newcommand{\argsh}{\operatorname{argsh}}
\newcommand{\argch}{\operatorname{argch}}
\newcommand{\argth}{\operatorname{argth}}
\newcommand{\ker}{\operatorname{Ker}}
\renewcommand{\im}{\operatorname{Im}}
\newcommand{\rg}{\operatorname{rg}}
\newcommand{\Id}{\operatorname{Id}}
\newcommand{\id}{\operatorname{id}}
\renewcommand{\leq}{\leq}
\renewcommand{\geq}{\geq }

%Définition de nouvelles couleurs : rgb(trois paramètres red green blue
entre 0 et 1); cmyk (quatre cyan magenta yellow black) entre 0 et 1;
gray (entre 0 et 1) et black, white, red, green, blue, cyan, magenta,
yellow% definecolor{0gris}{gray}{0.8} 
% Nouvelle commande pour encadrer le titre car shabox ne veut que d'une
seule ligne; ATTENTION A LA TAILLE; petite différence avec shadowbox ou
doublebox, voire fcolorbox ou colorbox (au lieu de shabox; laisser le
parbox tranquille sauf pour la taille de la boîte
\newcommand{\Tbox}[1]{\begin{center} \shabox{\parbox{0.6
\linewidth}{#1}} \end{center}} %[1] pour 1 paramètre ; #1 pour ce que
fait le 1er paramètre; entre accolades ce que fait la commande
%Mise en page en mode fancy : en-têtes et pieds de pages puis
définition des en-têtes et pieds de pages\pagestyle{fancy}
\lhead{ECE 2 - Mathématiques \\
Quentin Dunstetter - ENC-Bessières 2011$\backslash$2012}
\chead{}
\rhead{EML 1996}
\rfoot[ \ \thepage]{\thepage}
\cfoot{}
\lfoot{}
\thispagestyle{fancy} %Mise en page de la 1ère page en mode fancy
%Trait en bas et en haut de la page (entre en-tête et texte et texte et
pied de page)\renewcommand{\footrulewidth}{0.4pt}
\renewcommand{\headrulewidth}{0.4pt}


%DEBUT DU DOCUMENT\vspace*{3cm}

\begin{center}
{\LARG\E\textbf{BANQUE COMMUNE D'ÉPREUVES}}



{\large \textsc{CONCOURS D ADMISSION DE 1996}}



{\large \textbf{Concepteur : EML}}



\rule{2.39cm}{0.05cm}



{\Large \textbf{OPTION ÉCONOMIQUE}}



{\Large \textbf{MATHÉMATIQUES }}



{\Large Lundi 9 mai, de 14h à 18h}



\rule{2.39cm}{0.05cm}
\end{center}

\textit{La présentation, la lisibilité, l'orthographe, la qualité
de la rédaction, la clarté et la précision des raisonnements
entreront pour une part importante dans l'appréciation des copies.}

\textit{Les candidats sont invités à \textbf{encadrer} dans la mesure
du possible les résultats de leurs calculs.}

\textit{Ils ne doivent faire usage d'aucun document. L'utilisation de
toute
calculatrice et de tout matériel électronique est interdite. Seule
l'utilisation d'une règle graduée est autorisée.}

\textit{Si au cours de l'épreuve, un candidat repère ce qui lui semble
être une erreur d'énoncé, il la signalera sur sa copie et
poursuivra sa composition en expliquant les raisons des initiatives
qu'il sera
amené à prendre.}

\vspace*{3cm}

\section*{Exercice 1}

\noindent On considère les matrices carrées réelles
d'ordre 3 suivantes : 
\[
I = \left( 
\begin{array}{ccc}
1 & 0 & 0 \\
0 & 1 & 0 \\
0 & 0 & 1
\end{array}
\right) \hspace{2cm}J = \left( 
\begin{array}{ccc}
0 & 0 & 1 \\
0 & 1 & 0 \\
1 & 0 & 0
\end{array}
\right) \hspace{2cm}A = \left( 
\begin{array}{ccc}
0 & 1 & 0 \\
1 & 0 & 1 \\
0 & 1 & 0
\end{array}
\right) 
\]

\begin{noliste}{1.}
 \setlength{\itemsep}{4mm}
\item Calculer $A^{2}$ et exprimer $J$ comme combinaison linéaire de
$I$
et $A^{2}$

\item 
\begin{noliste}{a)}
 \setlength{\itemsep}{2mm}
\item Calculer les valeurs propres de $A$ (on trouvera trois réels
$\lambda_{1}$, $\lambda_{2}$ et $\lambda_{3}$ que l'on rangera de sorte
que $\lambda_{1}<\lambda_{2}<\lambda_{3}$.

\item Pour chaque entier $k$ de $\{1,2,3\}$, calculer un vecteur propre
$X_{k}$ associé à la valeur propre $\lambda_{k}$ de $A$, tel que
l'élément de la première ligne de $X_{k}$ soit égal à 1.

\item En déduire une matrice carrée réelle $P$ d'ordre 3,
inversible, de pemière ligne égale à $(1,1,1)$ telle qu'en
notant $D = \left( 
\begin{array}{ccc}
\lambda_{1} & 0 & 0 \\
0 & \lambda_{2} & 0 \\
0 & 0 & \lambda_{3}
\end{array}
\right) $, on ait $A = PDP^{-1}$.
\end{noliste}

\item Soient $a$, $b$ et $c$ des réels et $M = \left( 
\begin{array}{ccc}
a & b & c \\
b & a + c & b \\
c & b & a
\end{array}
\right) $.

\begin{noliste}{a)}
 \setlength{\itemsep}{2mm}
\item Exprimer $M$ comme combinaison linéaire de $I$, $A$ et $J$, puis
comme combinaison linéaire de $I$, $A$ et $A^{2}$.

\item En déduire une matrice diagonale réelle $\Delta $ d'odre 3
telle que $M = P\Delta P^{-1}$, où $P$ est la matrice obtenue à la
question 2.c.
\end{noliste}
\end{noliste}

\section*{Exercice 2}

\noindent Soit $f$ la fonction définie sur $\R$ par $f(x) =
\dfrac{e^{x}}{e^{2x} + 1}$

\begin{noliste}{1.}
 \setlength{\itemsep}{4mm}
\item 
\begin{noliste}{a)}
 \setlength{\itemsep}{2mm}
\item Montrer que$f$ est paire.\\
Étudier les variations de $f$ et tracer sa courbe dans repère
orthogonal 
$(O;\vec{i},\vec{j})$.\\
(unités : 2 cm sur l'axe des abscisses et 10 cm sur l'axe des
ordonnées).

\item Montrer qu'il existe un unique réel $\ell $ tel que $f(\ell
) = \ell $. Justifier : $0\leq \ell \leq \dfrac{1}{2}$ (on donne
$f\left(
1/2\right) <1/2$ )

\item Montrer que pour tout réel $x$ : $|f^{\prime }(x)|\leq f(x)\leq 
\dfrac{1}{2}$
\end{noliste}

\item On définit la suite $(u_{n})_{n\in \N}$ par : 
\[
u_{0} = 0\qquad \qquad \text{et }\forall n\in \N\qquad u_{n + 1} =
f(u_{n})
\]

\begin{noliste}{a)}
 \setlength{\itemsep}{2mm}
\item Montrer que, pour tout $n\in \N\qquad u_{n}\in [0,\dfrac{1}{2}]$

\item Montrer que, pour tout $n\in \N$ : 
\[
|u_{n + 1}-\ell |\leq \dfrac{1}{2}|u_{n}-\ell |\qquad \text{puis}\qquad
|u_{n}-\ell |\leq \dfrac{1}{2^{n + 1}}
\]

\item En déduire que la suite $(u_{n})$ converge vers $\ell $.

\item Écrire un programme \Scilab{} permettant d'obtenir une valeur
approchée de $\ell $ à $10^{-3}$ près.
\end{noliste}
\end{noliste}

\section*{Exercice 3}

\noindent Soit $f$ la fonction définie sur $\R$ par : 
\[
\left\{ 
\begin{array}{ll}
f(x) = e^{-|x|} & \text{si }-\ln 2\leq x\leq \ln 2 \\
f(x) = 0 & \text{sinon}
\end{array}
\right. 
\]

\begin{noliste}{1.}
 \setlength{\itemsep}{4mm}
\item Étudier les variations de $f$ et tracer sa représentation
graphique dans un repère orthonormé (unité 5cm).

\item Montrer que $f$ est une densité de probabilité.

\item Soit $X$ une variable aléatoire réelle admettant $f$ comme
densité.

\begin{noliste}{a)}
 \setlength{\itemsep}{2mm}
\item Déterminer la fonction de répartition $F$ de $X$.

\item Montrer que $X$ admet une espérance et calculer l'espérance
de $X$.

\item On pose $Y = |X|$.\\
Déterminer la fonction de répartition $G$ de $Y$. Montrer que $Y$
est une variable à densité et déterminer une densité $g$ de $Y$.
\end{noliste}
\end{noliste}

\section*{Exercice 4}

\noindent Pour tout entier naturel $n$, on note $I_{n} =
\dint{0}{1}x^{n}e^{-x}dx$

\begin{noliste}{1.}
 \setlength{\itemsep}{4mm}
\item 
\begin{noliste}{a)}
 \setlength{\itemsep}{2mm}
\item Montrer que : $\forall n\in \N\qquad 0\leq I_{n}\leq \dfrac{1}{n
+ 1}$

\item En déduire que la suite $(I_{n})$ converge et donner sa limite.
\end{noliste}

\item À l'aide d'une intégration par parties, établir que 
\[
\forall n\in \N\qquad I_{n} = \dfrac{1}{(n + 1)e} + \dfrac{I_{n + 1}}{n
+ 1}
\]

\item 
\begin{noliste}{a)}
 \setlength{\itemsep}{2mm}
\item En déduire que $\forall n\in \N\qquad 0\leq I_{n}-\dfrac{1}{(n +
1)e}\leq \dfrac{1}{(n + 1)(n + 2)}$

\item Trouver un équivalent simple de $I_{n}$ quand $n$ tend vers $ +
\infty $.
\end{noliste}
\end{noliste}

\end{document}

