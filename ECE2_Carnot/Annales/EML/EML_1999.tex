\documentclass[11pt]{article}%
\usepackage{geometry}%
\geometry{a4paper,
 lmargin = 2cm,rmargin = 2cm,tmargin = 2.5cm,bmargin = 2.5cm}

\input{../../macros.tex}

\pagestyle{fancy} %
\lhead{ECE2 \hfill Mathématiques\\
} %
\chead{\hrule} %
\rhead{} %
\lfoot{} %
\cfoot{} %
\rfoot{\thepage} %

\renewcommand{\headrulewidth}{0pt}% : Trace un trait de séparation
 % de largeur 0,4 point. Mettre 0pt
 % pour supprimer le trait.

\renewcommand{\footrulewidth}{0.4pt}% : Trace un trait de séparation
 % de largeur 0,4 point. Mettre 0pt
 % pour supprimer le trait.

\setlength{\headheight}{14pt}

\title{\bf \vspace{-2cm} EML 1999} %
\author{} %
\date{} %
\begin{document}

\maketitle %
\vspace{-1.4cm}\hrule %
\thispagestyle{fancy}

\vspace*{.2cm}


% DEBUT DU DOC À MODIFIER : tout virer jusqu'au début de l'exo

%Définition et changement de valeurs de
compteurs%newcounter{cpt1}{section} compteur cpt1 remis à 0 à chaque
aumentation par stepcounter du compteur section%setcounter{cpt1}{3} on
met le compteur à 3%addtocounter{cpt1}{5} on ajoute 5 au compteur%
stepcounter{cpt1} on ajoute 1% ifthenelse{test}{alors}{sinon} (page
206) pour subordonner à une condition % whiledo{test}{commande} pour
faire une boucle (page 206 aussi) % value{cpt1} pour noter dans le
document la valeur de cpt1 
%Définition définitive d'opérateurs
mathématiques\newcommand{\ch}{\operatorname{ch}} 
\newcommand{\sh}{\operatorname{sh}}
\renewcommand{\tanh}{\operatorname{th}}
\renewcommand{\sinh}{\operatorname{sh}}
\renewcommand{\cosh}{\operatorname{ch}}
\newcommand{\argsh}{\operatorname{argsh}}
\newcommand{\argch}{\operatorname{argch}}
\newcommand{\argth}{\operatorname{argth}}
\newcommand{\ker}{\operatorname{Ker}}
\renewcommand{\im}{\operatorname{Im}}
\newcommand{\rg}{\operatorname{rg}}
\newcommand{\Id}{\operatorname{Id}}
\newcommand{\id}{\operatorname{id}}
\renewcommand{\leq}{\leq}
\renewcommand{\geq}{\geq }

%Définition de nouvelles couleurs : rgb(trois paramètres red green blue
entre 0 et 1); cmyk (quatre cyan magenta yellow black) entre 0 et 1;
gray (entre 0 et 1) et black, white, red, green, blue, cyan, magenta,
yellow% definecolor{0gris}{gray}{0.8} 
% Nouvelle commande pour encadrer le titre car shabox ne veut que d'une
seule ligne; ATTENTION A LA TAILLE; petite différence avec shadowbox ou
doublebox, voire fcolorbox ou colorbox (au lieu de shabox; laisser le
parbox tranquille sauf pour la taille de la boîte
\newcommand{\Tbox}[1]{\begin{center} \shabox{\parbox{0.6
\linewidth}{#1}} \end{center}} %[1] pour 1 paramètre ; #1 pour ce que
fait le 1er paramètre; entre accolades ce que fait la commande
%Mise en page en mode fancy : en-têtes et pieds de pages puis
définition des en-têtes et pieds de pages\pagestyle{fancy}
\lhead{ECE 2 - Mathématiques \\
Quentin Dunstetter - ENC-Bessières 2011$\backslash$2012}
\chead{}
\rhead{EML 1999}
\rfoot[ \ \thepage]{\thepage}
\cfoot{}
\lfoot{}
\thispagestyle{fancy} %Mise en page de la 1ère page en mode fancy
%Trait en bas et en haut de la page (entre en-tête et texte et texte et
pied de page)\renewcommand{\footrulewidth}{0.4pt}
\renewcommand{\headrulewidth}{0.4pt}


%DEBUT DU DOCUMENT\vspace*{3cm}

\begin{center}
{\LARG\E\textbf{BANQUE COMMUNE D'ÉPREUVES}}



{\large \textsc{CONCOURS D ADMISSION DE 1999}}



{\large \textbf{Concepteur : EML}}



\rule{2.39cm}{0.05cm}



{\Large \textbf{OPTION ÉCONOMIQUE}}



{\Large \textbf{MATHÉMATIQUES }}



{\Large Lundi 9 mai, de 14h à 18h}



\rule{2.39cm}{0.05cm}
\end{center}

\textit{La présentation, la lisibilité, l'orthographe, la qualité
de la rédaction, la clarté et la précision des raisonnements
entreront pour une part importante dans l'appréciation des copies.}

\textit{Les candidats sont invités à \textbf{encadrer} dans la mesure
du possible les résultats de leurs calculs.}

\textit{Ils ne doivent faire usage d'aucun document. L'utilisation de
toute
calculatrice et de tout matériel électronique est interdite. Seule
l'utilisation d'une règle graduée est autorisée.}

\textit{Si au cours de l'épreuve, un candidat repère ce qui lui semble
être une erreur d'énoncé, il la signalera sur sa copie et
poursuivra sa composition en expliquant les raisons des initiatives
qu'il sera
amené à prendre.}

\vspace*{3cm}

\section*{Exercice 1}

\noindent Pour tout entier naturel $n$, on note : $w_{n} =
\dint{0}{\dfrac{\pi }{2}}\cos ^{n}t\ dt$.

\begin{noliste}{1.}
 \setlength{\itemsep}{4mm}
\item Calculer $w_{0}$ et $w_{1}$.

\item Montrer que la suite $(u_{n})_{n\in \N}$ est décroissante.

\item Montrer pour tout entier naturel $n$ : \qquad $w_{n}\geq 0$.\\
En déduire que la suite $(w_{n})_{n\in \N}$ est convergente.

\item Soit $n\in \N$. À l'aide d'une intégration par parties,
montrer que : 
\[
w_{n + 2} = (n + 1)\dint{0}{\dfrac{\pi }{2}}\cos ^{n}t\ \sin ^{2}t\ dt
\]
En déduire : \qquad $w_{n + 2} = \dfrac{n + 1}{n + 2}w_{n}$.

\item Montrer pour tout entier naturel $n$, en utilisant \textbf{2.} et

\textbf{4.} : 
\[
0<\dfrac{n + 1}{n + 2}\ w_{n}\leq w_{n + 1}\leq w_{n}
\]
\\
En déduire : \qquad $w_{n + 1}\sim w_{n}$ quand $n\rightarrow + \infty
$.

\item Montrer, en utilisant \textbf{4.}, que la suite $(u_{n})_{n\in
\N}$ de terme général $u_{n} = (n + 1)w_{n}w_{n + 1}$ est constante.\\
En déduire : \qquad $w_{n}\sim \sqrt{\dfrac{\pi }{2n}}$ quand
$n\rightarrow + \infty $.
\end{noliste}

\section*{Exercice 2}

\noindent On considère les éléments suivants de $\mathfrak{M}_{3}(\R)$
: 
\[
I = \left( 
\begin{array}{ccc}
1 & 0 & 0 \\
0 & 1 & 0 \\
0 & 0 & 1
\end{array}
\right) \qquad J = \left( 
\begin{array}{ccc}
0 & 1 & 0 \\
1 & 0 & 1 \\
0 & 1 & 0
\end{array}
\right) \qquad K = \left( 
\begin{array}{ccc}
0 & 0 & 1 \\
0 & 1 & 0 \\
1 & 0 & 0
\end{array}
\right) \qquad P = \left( 
\begin{array}{ccc}
1 & -1 & 1 \\
-\sqrt{2} & 0 & \sqrt{2} \\
1 & 1 & 1
\end{array}
\right)
\]

\begin{noliste}{1.}
 \setlength{\itemsep}{4mm}
\item 
\begin{noliste}{a)}
 \setlength{\itemsep}{2mm}
\item Justifier (sans calcul) que $J$ est diagonalisable, que $J$ n'est
pas
inversible, et que $0$ est valeur propre de $J$.

\item Calculer $J^{2}$ et exprimer $J^{2}$ en fonction de $I$ et $K$.
\end{noliste}

\begin{noliste}{a)}
 \setlength{\itemsep}{2mm}
\item Calculer les valeurs propres de $J$ et déterminer une base de
$\mathfrak{M}_{3,1}(\R)$ formée de vecteurs propres pour $J$.\\
En déduire que $P^{-1}\ J\ P$ est une matrice diagonale que l'on
explicitera.

\item Montrer, en utilisant \textbf{1.b.} et \textbf{2.a.} que $P^{-1}\
K\ P$
est une matrice diagonale que l'on explicitera.
\end{noliste}

\item Soit $(a,b,c)\in \R^{3}$. On considère l'élément
suivant de $\mathfrak{M}_{3}(\R)$ : 
\[
M = \left( 
\begin{array}{ccc}
a & b & c \\
b & a + c & b \\
c & b & a
\end{array}
\right)
\]

\begin{noliste}{a)}
 \setlength{\itemsep}{2mm}
\item Montrer que $M$ s'exprime simplement à l'aide de $I,J,K$ et
$a,b,c$.

\item En déduire que $P^{-1}\ M\ P$ est une matrice diagonale que l'on
explicitera.
\end{noliste}

\item Trouver une matrice $X$ de $\mathfrak{M}_{3}(\R)$ telle que : 
\[
X^{2} = \left( 
\begin{array}{ccc}
2 & 2 & 1 \\
2 & 3 & 2 \\
1 & 2 & 2
\end{array}
\right)
\]
\end{noliste}

\section*{Exercice 3}

\noindent La lettre $c$ désigne un entier naturel non nul fixé.\\
Une urne contient initialement des boules blanches et des boules
rouges,
toutes indiscernables au toucher.\\
On effectue des tirages successifs d'une boule dans l'urne selon le
protocole suivant : après chaque tirage, la boule tirée est remise
dans l'urne et on rajoute dans l'urne, avant le tirage suivant, $c$
boules
de la couleur qui vient d'être tirée.

\begin{noliste}{1.}
 \setlength{\itemsep}{4mm}
\item Dans cette question, on suppose que l'urne contient initialement
$b$
boules blanches et $r$ boules rouges, où $b$, $r$ sont des entiers
naturels non nuls.

\begin{noliste}{a)}
 \setlength{\itemsep}{2mm}
\item Quelle est la probabilité d'obtenir une boule blanche au premier
tirage ?

\item Quelle est la probabilité d'obtenir une boule blanche au deuxième
tirage ?

\item Si la deuxième boule tirée est blanche, quelle est la
probabilité que la première boule tirée ait été blanche ?
\end{noliste}

\item Pour tous entiers naturels non nuls $n$, $x$, $y$, on note
$u_{n}(x,y)$
la probabilité d'obtenir une boule blanche au $n^{eme}$ tirage, lorsque
l'urne contient initialement $x$ boules blanches et $y$ boules rouges.

\begin{noliste}{a)}
 \setlength{\itemsep}{2mm}
\item Montrer, en utilisant un système complet d'évènements
associé au premier tirage, que, pour tous entiers naturels non nuls
$n$, 
$x$, $y$, on a : 
\[
u_{n + 1}(x,y) = u_{n}(x + c,y)\dfrac{x}{x + y} + u_{n}(x,y +
c)\dfrac{y}{x + y}
\]

\item En déduire, par récurrence, que, pour tous entiers naturels
non nuls $n$, $x$, $y$, on a : 
\[
u_{n}(x,y) = \dfrac{x}{x + y}
\]
\end{noliste}

\item Dans cette question, on suppose que l'urne contient initialement
une
boule blanche et une boule rouge et que $c = 1$. Pour tout entier
naturel non
nul $n$, on note $X_{n}$ la variable aléatoire égale au nombre de
boules blanches obtenues au cours des $n$ premiers tirages.

\begin{noliste}{a)}
 \setlength{\itemsep}{2mm}
\item Donner la loi de $X_{1}$.

\item Donner la loi de $X_{2}$.

\item Montrer par récurrence, que $X_{n}$ suit une loi uniforme dont on
donnera l'espérance et la variance.
\end{noliste}
\end{noliste}

\begin{center}
- \ FIN \ -
\end{center}

\end{document}

