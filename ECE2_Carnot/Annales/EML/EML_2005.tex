\documentclass[11pt]{article}%
\usepackage{geometry}%
\geometry{a4paper,
 lmargin = 2cm,rmargin = 2cm,tmargin = 2.5cm,bmargin = 2.5cm}

\input{../../macros.tex}

\pagestyle{fancy} %
\lhead{ECE2 \hfill Mathématiques\\
} %
\chead{\hrule} %
\rhead{} %
\lfoot{} %
\cfoot{} %
\rfoot{\thepage} %

\renewcommand{\headrulewidth}{0pt}% : Trace un trait de séparation
 % de largeur 0,4 point. Mettre 0pt
 % pour supprimer le trait.

\renewcommand{\footrulewidth}{0.4pt}% : Trace un trait de séparation
 % de largeur 0,4 point. Mettre 0pt
 % pour supprimer le trait.

\setlength{\headheight}{14pt}

\title{\bf \vspace{-2cm} EML 2005} %
\author{} %
\date{} %
\begin{document}

\maketitle %
\vspace{-1.4cm}\hrule %
\thispagestyle{fancy}

\vspace*{.2cm}


% DEBUT DU DOC À MODIFIER : tout virer jusqu'au début de l'exo

%Définition et changement de valeurs de
compteurs%newcounter{cpt1}{section} compteur cpt1 remis à 0 à chaque
aumentation par stepcounter du compteur section%setcounter{cpt1}{3} on
met le compteur à 3%addtocounter{cpt1}{5} on ajoute 5 au compteur%
stepcounter{cpt1} on ajoute 1% ifthenelse{test}{alors}{sinon} (page
206) pour subordonner à une condition % whiledo{test}{commande} pour
faire une boucle (page 206 aussi) % value{cpt1} pour noter dans le
document la valeur de cpt1 
%Définition définitive d'opérateurs
mathématiques\newcommand{\ch}{\operatorname{ch}} 
\newcommand{\sh}{\operatorname{sh}}
\renewcommand{\tanh}{\operatorname{th}}
\renewcommand{\sinh}{\operatorname{sh}}
\renewcommand{\cosh}{\operatorname{ch}}
\newcommand{\argsh}{\operatorname{argsh}}
\newcommand{\argch}{\operatorname{argch}}
\newcommand{\argth}{\operatorname{argth}}
\newcommand{\ker}{\operatorname{Ker}}
\renewcommand{\im}{\operatorname{Im}}
\newcommand{\rg}{\operatorname{rg}}
\newcommand{\Id}{\operatorname{Id}}
\newcommand{\id}{\operatorname{id}}
\renewcommand{\leq}{\leq}
\renewcommand{\geq}{\geq }

%Définition de nouvelles couleurs : rgb(trois paramètres red green blue
entre 0 et 1); cmyk (quatre cyan magenta yellow black) entre 0 et 1;
gray (entre 0 et 1) et black, white, red, green, blue, cyan, magenta,
yellow% definecolor{0gris}{gray}{0.8} 
% Nouvelle commande pour encadrer le titre car shabox ne veut que d'une
seule ligne; ATTENTION A LA TAILLE; petite différence avec shadowbox ou
doublebox, voire fcolorbox ou colorbox (au lieu de shabox; laisser le
parbox tranquille sauf pour la taille de la boîte
\newcommand{\Tbox}[1]{\begin{center} \shabox{\parbox{0.6
\linewidth}{#1}} \end{center}} %[1] pour 1 paramètre ; #1 pour ce que
fait le 1er paramètre; entre accolades ce que fait la commande
%Mise en page en mode fancy : en-têtes et pieds de pages puis
définition des en-têtes et pieds de pages\pagestyle{fancy}
\lhead{ECE 2 - Mathématiques \\
Quentin Dunstetter - ENC-Bessières 2011$\backslash$2012}
\chead{}
\rhead{EML 2005}
\rfoot[ \ \thepage]{\thepage}
\cfoot{}
\lfoot{}
\thispagestyle{fancy} %Mise en page de la 1ère page en mode fancy
%Trait en bas et en haut de la page (entre en-tête et texte et texte et
pied de page)\renewcommand{\footrulewidth}{0.4pt}
\renewcommand{\headrulewidth}{0.4pt}


%DEBUT DU DOCUMENT\vspace*{3cm}

\begin{center}
{\LARG\E\textbf{BANQUE COMMUNE D'ÉPREUVES}}



{\large \textsc{CONCOURS D ADMISSION DE 2005}}



{\large \textbf{Concepteur : EML}}



\rule{2.39cm}{0.05cm}



{\Large \textbf{OPTION ÉCONOMIQUE}}



{\Large \textbf{MATHÉMATIQUES }}



{\Large Lundi 9 mai, de 14h à 18h}



\rule{2.39cm}{0.05cm}
\end{center}

\textit{La présentation, la lisibilité, l'orthographe, la qualité
de la rédaction, la clarté et la précision des raisonnements
entreront pour une part importante dans l'appréciation des copies.}

\textit{Les candidats sont invités à \textbf{encadrer} dans la mesure
du possible les résultats de leurs calculs.}

\textit{Ils ne doivent faire usage d'aucun document. L'utilisation de
toute
calculatrice et de tout matériel électronique est interdite. Seule
l'utilisation d'une règle graduée est autorisée.}

\textit{Si au cours de l'épreuve, un candidat repère ce qui lui semble
être une erreur d'énoncé, il la signalera sur sa copie et
poursuivra sa composition en expliquant les raisons des initiatives
qu'il sera
amené à prendre.}

\vspace*{3cm}

\section*{Exercice 1}

On considère les éléments suivants de $M_{3}(\R)$~ :
\[
I = \left(
\begin{array}
[c]{ccc}1 & 0 & 0\\
0 & 1 & 0\\
0 & 0 & 1
\end{array}
\right),\quad J = \left(
\begin{array}
[c]{ccc}0 & 1 & 0\\
0 & 0 & 1\\
0 & 0 & 0
\end{array}
\right),\quad K = \left(
\begin{array}
[c]{ccc}0 & 0 & 1\\
0 & 0 & 0\\
0 & 0 & 0
\end{array}
\right)
\]


On note $E$ le sous-espace vectoriel de $M_{3}(\R)$ engendré par
$I$, $J$ et $K$.

Pour toute matrice $M$ de $E$, on note $M^{0} = I$, et si $M$ est
inversible, on
note, pour tout entier naturel $k$, $M^{-k} = (M^{-1})^{k}$, et on
rappelle
qu'alors $M^{k}$ est inversible et que $(M^{k})^{-1} = M^{-k}$.

\begin{noliste}{1.}
 \setlength{\itemsep}{4mm}
\item Déterminer la dimension de $E$.

\item Calculer $J^{2}$, $JK$, $KJ$ et $K^{2}$.

\item Soit la matrice $L = I + J$.

\begin{noliste}{a)}
 \setlength{\itemsep}{2mm}
\item Montrer, pour tout entier naturel $n$~ :
\[
L^{n} = I + nJ + {\frac{n(n-1)}{2}}K
\]


\item Vérifier que $L$ est inversible et montrer, pour tout entier
relatif
$n$~ :
\[
L^{n} = I + nJ + {\frac{n(n-1)}{2}}K
\]


\item Exprimer, pour tout entier relatif $n$, $L^{n}$ à l'aide de $I$,
$L
$, $L^{2}$ et $n$.
\end{noliste}

On considère

la matrice $ = \left(
\begin{array}
[c]{ccc}0 & 2 & -1\\
1 & 0 & 1\\
2 & -3 & 3
\end{array}
\right) $

de $\M{3}$ et on note $f$ l'endomorphisme de
$\R^{3}$ représenté par la matrice $A$ dans la base canonique
de $\R^{3}$ et $e$ l'application identique de $\R^{3}$ dans lui-même.

\item Montrer que $f$ admet une valeur propre et une seule que l'on
déterminera.

Est-ce que $f$ est diagonalisable ?

\item
\begin{noliste}{a)}
 \setlength{\itemsep}{2mm}
\item Soit $w = (1,0,0)$. Calculer $v = (f-e)(w)$ et $u = (f-e)(v)$.
Montrer que
$(u,v,w)$ est une base de $\R^{3}$.

\item Déterminer la matrice associée à $f$ relativement à la
base $(u,v,w)$.

\item Montrer que $f$ est un automorphisme de $\R^{3}$ et, pour tout
entier relatif $n$, exprimer $f^{n}$ à l'aide de $e$, $f$, $f^{2}$ et
$n$.
\end{noliste}
\end{noliste}

\section*{Exercice 2}

On considère l'application $f :\R\rightarrow \R$,
définie, pour tout réel $t$, par~ :
\[
f(t) = \left\{
\begin{array}
[c]{cc}0 & \text{si }t\leq0\\
\frac{1}{(1 + t)^{2}}} & \text{si }t>0
\end{array}
\right.
\]


\begin{noliste}{1.}
 \setlength{\itemsep}{4mm}
\item Tracer l'allure de la courbe représentative de $f$.

\item Montrer que $f$ est une densité de probabilité.

\item Montrer que, pour tout réel $x$, l'intégrale
$\dint{-\infty}{x}f(t)dt$ converge, et calculer cette intégrale.

\textsl{On distinguera les cas $x\leq0$ et $x>0$.}

\item Déterminer un réel positif $\alpha$ tel que $ \int
_{0}{\alpha}f(t)dt = {\frac{1}{2}}$.

\item Soit $x\in[0, + \infty[$ fixé.

On considère la fonction $\varphi_{x}$ définie sur $[0; + \infty[$ par~
:
$ \forall u\in[0, + \infty[,\; \varphi_{x}(u) = \dint{x-u}{x +
u}f(t)dt$.

\begin{noliste}{a)}
 \setlength{\itemsep}{2mm}
\item Calculer $\varphi_{x}(0)$ et $ \dlim{u\to +
\infty}\varphi_{x}(u)$.

\item Montrer~ :\quad$ \forall(u,v)\in \left( [0, + \infty[ \ \right)
^{2},\ \;u<v\Longrightarrow \varphi_{x}(v)-\varphi_{x}(u)\geq \dint{x +
u}{x + v}f(t)dt$.

En déduire que $\varphi_{x}$ est strictement croissante sur $[0; +
\infty[$.

\item On admet que $\varphi_{x}$ est continue sur $[0; + \infty[$.
Montrer que
l'équation $ \varphi_{x}(u) = {\frac{1}{2}}$, d'inconnue $u$,
admet une solution et une seule dans $[0; + \infty[$.
\end{noliste}

On note $U :[0; + \infty \lbrack \rightarrow \R$ l'application qui, à
tout réel $x\in \lbrack0; + \infty \lbrack$, associe $U(x)$ l'unique
solution
de l'équation $\varphi_{x}(u) = \frac{1}{2}$.\\
 Ainsi, pour tout
$x\in \lbrack0; + \infty \lbrack$, on a~ : $ \dint{x-U(x)}{x +
U(x)}f(t)dt = {\frac{1}{2}}$.

\item
\begin{noliste}{a)}
 \setlength{\itemsep}{2mm}
\item Vérifier, pour tout $x\in[0;{\frac{1}{2}}[$ :\quad$U(x) = 1-x$.

\item Pour tout $x\in[{\frac{1}{2}}; + \infty[$, montrer~
:\quad$\varphi_{x}(x)\ge{\frac{1}{2}}$, puis~ : $x-U(x)\geq 0$, et en
déduire~ :
$U(x) = \sqrt{4 + (x + 1)^{2}}-2$.
\end{noliste}

\item
\begin{noliste}{a)}
 \setlength{\itemsep}{2mm}
\item Montrer que l'application $U$ est continue sur $[0; + \infty[$.

\item Étudier la dérivabilité de $U$ sur $[0; + \infty[$

\item Montrer que la droite d'équation $y = x-1$ est asymptote à la
courbe représentative de $U$.

\item Tracer l'allure de la courbe représentative de $U$.
\end{noliste}

\item On considère la suite réelle $(a_{n})_{n\in \N}$
définie par $\left\{
\begin{array}
[c]{c}a_{0} = 1\\
\forall n\in \N,a_{n + 1} = U(a_{n})
\end{array}
\right. $

\begin{noliste}{a)}
 \setlength{\itemsep}{2mm}
\item Montrer~ :\quad$ \forall n\in \N,\quad a_{n}\geq
{\frac{1}{2}}$.

\item Montrer que la suite $(a_{n})_{n\in \N}$ est décroissante.

\item En déduire que la suite $(a_{n})_{n\in \N}$ converge et
montrer que sa limite est égale à $\frac{1}{2}}$.

\item Écrire un programme en -\Scilab{} qui calcule et affiche le plus
petit
entier $n\in \N$ tel que~ :
\[
\left| {a_{n}-{\frac{1}{2}}}\right| \leq10^{-6}
\]

\end{noliste}
\end{noliste}

\section*{Exercice 3}

\begin{noliste}{1.}
 \setlength{\itemsep}{4mm}
\item \textbf{Préliminaire~ :}

Soit $x\in]0;1[$. Dans une succession d'épreuves de Bernoulli
indépendantes, de même probabilité d'échec $x$, on définit
deux suites de variables aléatoires $(S_{n})_{n\geq 1}$ et
$(T_{n})_{n\geq 1}$
de la façon suivante~ :

$\bullet$ pour tout entier naturel $n$ non nul, $S_{n}$ est la variable
aléatoire égale au nombre d'épreuves nécessaires pour obtenir
le $n$-ième succès ;

$\bullet$ $T_{1}$ est la variable aléatoire égale à $S_{1}$ et
pour tout entier naturel $n\geq 2$, $T_{n}$ est la variable aléatoire
égale au nombre d'épreuves supplémentaires nécessaires pour
obtenir le $n$-ième succès après le $(n-1)$-ième succès.

Ainsi, pour tout $n\geq 2$, $T_{n} = S_{n}-S_{n-1}$ et pour tout $n\geq
1$,
$S_{n} = T_{1} + T_{2} + \cdots + T_{n}$.

\begin{noliste}{a)}
 \setlength{\itemsep}{2mm}
\item Pour tout entier naturel $n$ non nul, déterminer la loi de
$T_{n}$
et, sans calcul, donner l'espérance et la variance de $T_{n}$.

\item Pour tout entier naturel $n\geq 2$, justifier l'indépendance des
variables aléatoires $T_{1},T_{2},\dots,T_{n}$.

\item Pour tout entier naturel $n$ non nul, montrer que l'espérance et
la
variance de $S_{n}$ sont définies et montrer :

$ E(S_{n}) = {\frac{n}{1-x}} \mbox{ et } V(S_{n}) = {\frac
{nx}{(1-x)^{2}}}$.

\item Soit $n$ un entier naturel non nul. Déterminer la loi de $S_{n}$.

Que peut-on dire, sans calcul, de la valeur de $\Sum{k = n}{+
\infty}P\left(\Ev{S_{n} = k}\right)$ ?

\item En déduire, pour tout $x\in]0;1[$ et pour tout entier naturel $n$
non nul~ :
\[
\Sum{k = n}{+ \infty}\binom{k-1}{n-1}x^{k} = {\frac{x^{n}}{(1-x)^{n}}}
\]

\end{noliste}

\item Deux joueurs $A$ et $B$ procèdent chacun à une succession de
lancers d'une même pièce. \`{A} chaque lancer, la probabilité
d'obtenir pile est $p$ ($p$ fixé, $p\in]0;1[$), et la probabilité
d'obtenir face est $q = 1-p$.

Le joueur $A$ commence et il s'arrête quand il obtient le premier pile.
On
note $X$ la variable aléatoire égale au nombre de lancers
effectués par le joueur $A$.

Le joueur $B$ effectue alors autant de lancers que le joueur $A$ et on
note
$Y$ la variable aléatoire égale au nombre de piles obtenu par le
joueur $B$.

\begin{noliste}{a)}
 \setlength{\itemsep}{2mm}
\item Rappeler la loi de $X$ et, pour tout $k\geq1$, donner la loi
conditionnelle de $Y$ sachant $X = k$.

\item Quelles sont les valeurs prises par $Y$ ?

\item Montrer~ :\quad$ P\left(\Ev{Y = 0}\right) = \Sum{k = 1}{+
\infty}pq^{2k-1} = {\frac{q}{1 + q}}$.

\item Soit $n$ un entier naturel non nul.

Montrer : \quad$ P\left(\Ev{Y = n}\right) = \Sum{k = n}{+
\infty}\binom{k}{n}p^{n + 1}q^{2k-n-1}$,

puis, en utilisant \textbf{1.e}, $ P\left(\Ev{Y = n}\right) =
{\frac{1}{(1 + q)^{2}}}({\frac{q}{1 + q}})^{n-1}$
\end{noliste}
\end{noliste}


\end{document}