\documentclass[11pt]{article}%
\usepackage{geometry}%
\geometry{a4paper,
 lmargin = 2cm,rmargin = 2cm,tmargin = 2.5cm,bmargin = 2.5cm}

\input{../../macros.tex}

\pagestyle{fancy} %
\lhead{ECE2 \hfill Mathématiques\\
} %
\chead{\hrule} %
\rhead{} %
\lfoot{} %
\cfoot{} %
\rfoot{\thepage} %

\renewcommand{\headrulewidth}{0pt}% : Trace un trait de séparation
 % de largeur 0,4 point. Mettre 0pt
 % pour supprimer le trait.

\renewcommand{\footrulewidth}{0.4pt}% : Trace un trait de séparation
 % de largeur 0,4 point. Mettre 0pt
 % pour supprimer le trait.

\setlength{\headheight}{14pt}

\title{\bf \vspace{-2cm} EML 1998} %
\author{} %
\date{} %
\begin{document}

\maketitle %
\vspace{-1.4cm}\hrule %
\thispagestyle{fancy}

\vspace*{.2cm}


% DEBUT DU DOC À MODIFIER : tout virer jusqu'au début de l'exo

%Définition et changement de valeurs de
compteurs%newcounter{cpt1}{section} compteur cpt1 remis à 0 à chaque
aumentation par stepcounter du compteur section%setcounter{cpt1}{3} on
met le compteur à 3%addtocounter{cpt1}{5} on ajoute 5 au compteur%
stepcounter{cpt1} on ajoute 1% ifthenelse{test}{alors}{sinon} (page
206) pour subordonner à une condition % whiledo{test}{commande} pour
faire une boucle (page 206 aussi) % value{cpt1} pour noter dans le
document la valeur de cpt1 
%Définition définitive d'opérateurs
mathématiques\newcommand{\ch}{\operatorname{ch}} 
\newcommand{\sh}{\operatorname{sh}}
\renewcommand{\tanh}{\operatorname{th}}
\renewcommand{\sinh}{\operatorname{sh}}
\renewcommand{\cosh}{\operatorname{ch}}
\newcommand{\argsh}{\operatorname{argsh}}
\newcommand{\argch}{\operatorname{argch}}
\newcommand{\argth}{\operatorname{argth}}
\newcommand{\ker}{\operatorname{Ker}}
\renewcommand{\im}{\operatorname{Im}}
\newcommand{\rg}{\operatorname{rg}}
\newcommand{\Id}{\operatorname{Id}}
\newcommand{\id}{\operatorname{id}}
\renewcommand{\leq}{\leq}
\renewcommand{\geq}{\geq }

%Définition de nouvelles couleurs : rgb(trois paramètres red green blue
entre 0 et 1); cmyk (quatre cyan magenta yellow black) entre 0 et 1;
gray (entre 0 et 1) et black, white, red, green, blue, cyan, magenta,
yellow% definecolor{0gris}{gray}{0.8} 
% Nouvelle commande pour encadrer le titre car shabox ne veut que d'une
seule ligne; ATTENTION A LA TAILLE; petite différence avec shadowbox ou
doublebox, voire fcolorbox ou colorbox (au lieu de shabox; laisser le
parbox tranquille sauf pour la taille de la boîte
\newcommand{\Tbox}[1]{\begin{center} \shabox{\parbox{0.6
\linewidth}{#1}} \end{center}} %[1] pour 1 paramètre ; #1 pour ce que
fait le 1er paramètre; entre accolades ce que fait la commande
%Mise en page en mode fancy : en-têtes et pieds de pages puis
définition des en-têtes et pieds de pages\pagestyle{fancy}
\lhead{ECE 2 - Mathématiques \\
Quentin Dunstetter - ENC-Bessières 2011$\backslash$2012}
\chead{}
\rhead{EML 1998}
\rfoot[ \ \thepage]{\thepage}
\cfoot{}
\lfoot{}
\thispagestyle{fancy} %Mise en page de la 1ère page en mode fancy
%Trait en bas et en haut de la page (entre en-tête et texte et texte et
pied de page)\renewcommand{\footrulewidth}{0.4pt}
\renewcommand{\headrulewidth}{0.4pt}


%DEBUT DU DOCUMENT\vspace*{3cm}

\begin{center}
{\LARG\E\textbf{BANQUE COMMUNE D'ÉPREUVES}}



{\large \textsc{CONCOURS D ADMISSION DE 1998}}



{\large \textbf{Concepteur : EML}}



\rule{2.39cm}{0.05cm}



{\Large \textbf{OPTION ÉCONOMIQUE}}



{\Large \textbf{MATHÉMATIQUES }}



{\Large Lundi 9 mai, de 14h à 18h}



\rule{2.39cm}{0.05cm}
\end{center}

\textit{La présentation, la lisibilité, l'orthographe, la qualité
de la rédaction, la clarté et la précision des raisonnements
entreront pour une part importante dans l'appréciation des copies.}

\textit{Les candidats sont invités à \textbf{encadrer} dans la mesure
du possible les résultats de leurs calculs.}

\textit{Ils ne doivent faire usage d'aucun document. L'utilisation de
toute
calculatrice et de tout matériel électronique est interdite. Seule
l'utilisation d'une règle graduée est autorisée.}

\textit{Si au cours de l'épreuve, un candidat repère ce qui lui semble
être une erreur d'énoncé, il la signalera sur sa copie et
poursuivra sa composition en expliquant les raisons des initiatives
qu'il sera
amené à prendre.}

\vspace*{3cm}

\begin{center}
{\large \textsl{\underline{Exercice 1}}}
\end{center}

La fonction logarithme népérien est notée $\ln$.

\begin{noliste}{1.}
 \setlength{\itemsep}{4mm}
\item Soit $x\in [-1;1[$.

\begin{noliste}{a)}
 \setlength{\itemsep}{2mm}
\item Montrer, pour tout $n$ de $\N$ et tout $t$ de {\large
[}$-1;1${\large [} : 
\[
\frac{1}{1-t}-\Sum{k = 0}{n}t^{k} = \frac{t^{n + 1}}{1-t}
\]

\item En déduire, pour tout $n$ de $\N$ et tout $t$ de {\large
[}$-1;x${\large ]} : 
\[
\left| \frac{1}{1-t}-\Sum{k = 0}{n}t^{k}\right| \leq \frac{|t|^{n +
1}}{1-x}
\]

\item Établir, pour tout $n$ de $\N$ : 
\[
\left| -\ln (1-x)-\Sum{k = 0}{n}\frac{x^{k + 1}}{k + 1}\right| \leq
\frac{1}{(n + 2)(1-x)}
\]

\item En déduire que la série $\Sum{n\geq 1}\frac{x^{n}}{n}$ converge
et a pour somme $-\ln (1-x)$.

En particulier, montrer : $\Sum{n = 1}{+ \infty }\dfrac{1}{n2^{n}} =
\ln
2$.
\end{noliste}

\item Un joueur lance une pièce équilibrée jusqu'à
l'obtention du premier pile. S'il lui a fallu $n$ lancers $(n\in
\N^{*})$ pour obtenir ce premier pile, on lui fait alors tirer au
hasard un
billet de loterie parmi $n$ billets dont un seul est gagnant.

Quelle est la probabilité que le joueur gagne ?
\end{noliste}

\newpage 

\begin{center}
{\large \textsl{\underline{Exercice 2}}}
\end{center}

Dans l'ensemble des matrices carrées réelles d'ordre 3, on note $I =
\left(
\begin{array}{rrr}
1 & 0 & 0 \\
0 & 1 & 0 \\
0 & 0 & 1
\end{array}
\right)$ et $A = \left(
\begin{array}{rrr}
1 & 1 & -2 \\
-1 & -1 & 2 \\
-2 & -2 & 0
\end{array}
\right)$.

\begin{noliste}{1.}
 \setlength{\itemsep}{4mm}
\item 
\begin{noliste}{a)}
 \setlength{\itemsep}{2mm}
\item Calculer $A^{2}$ et $A^{3}$.

\item En déduire que $A$ n'est pas inversible et que $A$ admet $0$ pour
unique valeur propre.

\item Déterminer une base du sous-espace propre de $A$ associé \'{y}
la valeur propre $0$.

\item La matrice $A$ est-elle diagonalisable ?
\end{noliste}

\item On note, pour tout réel $a$, $M(a) = I + 2\,a\,A +
2\,a^{2}\,A^{2}$, et 
$\E\,$ l'ensemble des matrices $M(a)$ lorsque $a$ décrit $\R$.

\begin{noliste}{a)}
 \setlength{\itemsep}{2mm}
\item Calculer, pour tout couple $(a,b)$ de réels, le produit
$M(a)M(b)$
et montrer que ce produit appartient à $E$.

\item En déduire que, pour tout réel $a$, $\,M(a)$ est inversible
et préciser son inverse.
\end{noliste}

\item Soit $a$ un réel non nul.

\begin{noliste}{a)}
 \setlength{\itemsep}{2mm}
\item Montrer que tout vecteur propre de $A$ est vecteur propre de
$M(a)$.

\item Calculer $\left( M(a)-I\right) ^{3}$.

En déduire que $M(a)$ admet 1 pour seule valeur propre.

Préciser une base du sous-espace propre de $M(a)$ associé \'{y} la
valeur propre 1.

\item La matrice $M(a)$ est-elle diagonalisable ?
\end{noliste}
\end{noliste}

\newpage 

\begin{center}
{\large \textsl{\underline{Exercice 3}}}
\end{center}

Une urne contient des boules vertes et des boules blanches,
indiscernables
au toucher. La proportion de boules vertes est $p\,$, $0\, <\, p\, <\,
1\,$;
la proportion de boules blanches est $1-p$.

On effectue une suite de tirages successifs d'une boule avec
remise. (Toute boule tirée de l'urne y est remise avant de procéder au
tirage suivant.)

\begin{noliste}{1.}
 \setlength{\itemsep}{4mm}
\item On note $N_{V}$ la variable aléatoire égale au nombre de
tirages nécessaires pour obtenir la première boule verte, et $N_{B}$
la variable aléatoire égale au nombre de tirages nécessaires
pour obtenir la première boule blanche.

\begin{noliste}{a)}
 \setlength{\itemsep}{2mm}
\item Quelles sont les lois des variables aléatoires $N_{V}$ et $N_{B}$
 ?

\item Les variables aléatoires $N_{V}$ et $N_{B}$ sont-elles
indépendantes ?
\end{noliste}

On définit le couple de variables aléatoires ($X,\,Y)$ à
valeurs dans $(\N^{*})^{2}$ de la façon suivante :

pour tout $(i,\,j)\in (\N_{n}{*})^{2},\big(X = i\mathrm{et}Y = j\big)$
est l'événement :

'' les $i$ premières boules tirées sont blanches, les $j$ suivantes
sont vertes et la $(i + j + 1)^{\mathrm{ieme}}$est blanche

\textbf{ou}

les $i$ premières boules tirées sont vertes, les $j$ suivantes sont
blanches et la $(i + j + 1)^{\mathrm{ieme}}$ est verte'' 



Par exemple, pour la suite de tirages $BBBVVBVBB\cdots $ (où $V$ est
mis
pour vert et $B$ pour blanc), on a $X = 3$ et $Y = 2$.

\item 
\begin{noliste}{a)}
 \setlength{\itemsep}{2mm}
\item Déterminer la loi de la variable aléatoire $X$.

\item Montrer que la variable aléatoire $X$ admet une espérance et
que $\E(X) = \frac{p}{1-p} + \frac{1-p}{p}$.

\item Montrer que $\E(X)$ est minimale lorsque $p = \frac{1}{2}$, et
calculer
cette valeur minimale.
\end{noliste}

\item Montrer, pour tout $(i,\,j)$ de $(\N_{n}{*})^{2}$ : 
\[
P\left(\Ev{X = i\text{ et }Y = j}\right) = p^{i + 1}(1-p)^{j} +
(1-p)^{i + 1}p^{j}
\]

\item 
\begin{noliste}{a)}
 \setlength{\itemsep}{2mm}
\item En déduire la loi de la variable aléatoire $Y$.

\item Montrer que la variable aléatoire $Y$ admet une espérance que
l'on calculera.
\end{noliste}

\item 
\begin{noliste}{a)}
 \setlength{\itemsep}{2mm}
\item Établir que, si $p\neq \frac{1}{2}$, les variables aléatoires $X$
et $Y$ ne sont pas indépendantes

(on pourra envisager $P\left(\Ev{X = 1$ et $Y = 1}\right)$ ).

\item Démontrer que, si $p = \frac{1}{2}$, les variables aléatoires $X
$ et $Y$ sont indépendantes.
\end{noliste}
\end{noliste}

\end{document}

