\documentclass[11pt]{article}%
\usepackage{geometry}%
\geometry{a4paper,
  lmargin=2cm,rmargin=2cm,tmargin=2.5cm,bmargin=2.5cm}

\input{../../macros.tex}
%\input{../../../../../../macros.tex}

\pagestyle{fancy} %
\lhead{ECE2 \\
  Mathématiques\\[.2cm]
  \hrule} %
\chead{} %
\rhead{} %
\lfoot{} %
\cfoot{} %
\rfoot{\thepage} %

\renewcommand{\headrulewidth}{0pt}% : Trace un trait de séparation
                                    % de largeur 0,4 point. Mettre 0pt
                                    % pour supprimer le trait.

\renewcommand{\footrulewidth}{0.4pt}% : Trace un trait de séparation
                                    % de largeur 0,4 point. Mettre 0pt
                                    % pour supprimer le trait.

\setlength{\headheight}{14pt}

\title{\bf \vspace{-1cm} EML 2011} %
\author{} %
\date{} %
\begin{document}

\maketitle %
\vspace{-1.2cm}\hrule %
\thispagestyle{fancy}

\vspace*{.4cm}


\section*{Exercice 1}

\noindent
On considère l'application
\[
f : \left\{
\begin{array}{ccl}
\left] 0, + \infty \right[ & \longrightarrow & \R
\\
x & \longmapsto & f(x) = \left( x + \ln (x)\right) \ee^{x-1}.
\end{array}
\right.
\]


\subsection*{Partie I : Étude et représentation graphique de $f$}

\begin{noliste}{1.}
 \setlength{\itemsep}{4mm}
\item Montrer que $f$ est dérivable sur $\left] 0, + \infty \right[ $.
On note $f'$ sa fonction dérivée.\\
Pour tout $x\in \left] 0, + \infty \right[,$ calculer $f'(x)$.

\item Établir :
\[
\forall x\in \left] 0, + \infty \right[,\quad \ln( x) + \frac{1}{x}>0
\]


\item En déduire :
\[
\forall x\in \left] 0, + \infty \right[,\quad x + \ln(x) + 1 +
\frac{1}{x}>0.
\]


\item En déduire le sens de variation de $f$.

\item Dresser le tableau de variation de $f$, comprenant la limite de
$f$ en
$0$ et la limite de $f$ en $ + \infty$.

Calculer $f\left( 1\right) $ et $f'\left( 1\right).$

\item Préciser la nature des branches infinies de la courbe
représentative $\mathcal{C}$ de $f$ dans un repère du plan.\\
\emph{Cette question est hors programme, pour notre programme actuel. 
Je vous laisse cependant la question si vous souhaitez y réfléchir.}

\item Tracer l'allure de $\mathcal{C}$. On précisera la tangente au 
point d'abscisse $1$.
\end{noliste}

Il n'est demandé ni l'étude de la convexité, ni la recherche
d'éventuels points d'inflexion.

\subsection*{Partie II : Étude d'une suite récurrente associée
à $f$.}

\noindent
On considère la suite réelle $\left( u_{n}\right) _{n\in \N}$
définie par $u_{0} = 2$ et, pour tout $n\in \N,$ $u_{n + 1} = f\left(
u_{n}\right).$

\begin{noliste}{1.}
 \setlength{\itemsep}{4mm}
 \setcounter{enumi}{7}
\item Montrer que, pour tout $n\in \N,$ $u_{n}$ existe et $u_{n}\geq2$.

\item Établir, par récurrence :\qquad$\forall n\in \N,\quad
u_{n}\geq \ee^{n}$.\\
Quelle est la limite de $(u_n)$ lorsque l'entier $n$ tend vers l'infini
?

\item Écrire un programme en \Scilab{} qui calcule et affiche le plus
petit entier naturel $n$ tel que $u_{n} \geq 10^{20}.$
\end{noliste}

\newpage

\subsection*{Partie III : Étude d'extremums locaux pour une fonction de
deux variables associée à $f$}

\noindent
On considère l'application
\[
F :
\left\{
\begin{array}{ccl}
\left] 0, + \infty \right[ & \longrightarrow & \R
\\
x & \longmapsto & F(x) = \dint{1}{x}f( t)\ dt
\end{array}
\right.
\]


\begin{noliste}{1.}
 \setlength{\itemsep}{4mm}
 \setcounter{enumi}{10}
\item Montrer que $F$ est de classe $\Cont{2}$ sur $\left] 0, + \infty
\right[$ et exprimer $F'( x) $, pour tout $x\in \left]
0, + \infty \right[ $, à l'aide de $f( x)$.\\
On considère l'application de classe $\Cont{2}$
\[
G :\left\{
\begin{array}{ccl}
\left] 0, + \infty \right[^{2} & \longrightarrow & \R
\\
(x,y) & \longmapsto & G(x,y) = F(x) + F(y)
-2\ee^{\tfrac{x + y}{2}}.
\end{array}
\right.
\]


\item Exprimer les dérivées partielles premières $\dfn{G}{1}(x,y)$ 
et $\dfn{G}{2}(x,y)$, pour tout
$( x,y) \in \left] 0, + \infty \right[^{2}$ à l'aide de
$f\left( x\right) $, $f\left( y\right) $ et $\ee^{\tfrac{x + y}{2}}.$

\item
\begin{noliste}{a)}
 \setlength{\itemsep}{2mm}
\item Montrer que $f$ est bijective.

\item Établir que, pour tout $( x,y) \in \left]
0, + \infty \right[^{2},\ (x,y) $ est un point critique de
$G$ si et seulement si :
\[
x = y\quad \text{et\quad}x + \ln x = \ee.
\]
\end{noliste}

\item Montrer que l'équation $x + \ln x = \ee$, d'inconnue $x\in \left]
0, + \infty \right[ $ admet une solution et une seule, que l'on notera
$\alpha$, et montrer que : $1<\alpha<\ee.$

\item Montrer que $G$ admet un extremum local. Préciser sa nature.
\end{noliste}

\newpage

\section*{Exercice 2}

\noindent
On considère les matrices carrées d'ordre 3 suivantes :
\[
I = 
\begin{smatrix}
1 & 0 & 0\\
0 & 1 & 0\\
0 & 0 & 1
\end{smatrix},\quad A = 
\begin{smatrix}
1 & 1 & 1\\
1 & 1 & 1\\
1 & 1 & 3
\end{smatrix}.
\]


\subsection*{Partie I : Détermination d'une racine carrée de $A$}

\begin{noliste}{1.}
 \setlength{\itemsep}{4mm}
\item Sans calcul, justifier que $A$ est diagonalisable et non
inversible.
Déterminer le rang de $A$.

\item Montrer que $0$, $1$ et $4$ sont les trois valeurs propres de $A$
et
déterminer les sous-espaces propres associés.

\item En déduire une matrice diagonale $D$ de $\M{3} $ dont les
coefficients diagonaux sont dans l'ordre
croissant, et une matrice inversible $P$ de $\M{3} $, dont les
coefficients de la première ligne sont tous
égaux à $1$, telles que : $A = PDP^{-1}.$

\item Calculer $P^{-1}$.

\item Montrer qu'il existe une matrice diagonale $\Delta$ de $\M{3} $,
dont les coefficients diagonaux sont dans
l'ordre croissant, telle que $\Delta^{2} = D$, et déterminer $\Delta$.

\item On note $R = P\Delta P^{-1}$. Montrer $R^{2} = A$ et calculer
$R$.
\end{noliste}

\subsection*{Partie II : Étude d'endomorphismes}

\noindent
On munit $\R^{3}$ de sa base canonique $\mathcal{B = }\left(
e_{1},e_{2},e_{3}\right) $ et on considère les endomorphismes $f$ et
$g$ de $\R^{3}$ dont les matrices dans $\mathcal{B}$ sont respectivement
$A$ et $R$.\\
On note $\mathcal{C = }\left( u_{1},u_{2},u_{3}\right) $ la base de
$\R^{3}$ telle que $P$ est la matrice de passage de $\mathcal{B}$
à $\mathcal{C}$.

\begin{noliste}{1.}
 \setlength{\itemsep}{4mm}
\item Déterminer les matrices de $f$ et $g$ dans la base $\mathcal{C}$.

\item
\begin{noliste}{a)}
 \setlength{\itemsep}{2mm}
\item Déterminer une base et la dimension de $\kr (f) $.

\item Déterminer une base et la dimension de $\im (f) $.
\end{noliste}

\item
\begin{noliste}{a)}
 \setlength{\itemsep}{2mm}
\item Déterminer une base et la dimension de $\kr (g).$

\item Déterminer une base et la dimension de $\im(g) 
$.
\end{noliste}

\item Trouver au moins un automorphisme $h$ de $\R^{3}$ tel que
$g = f\circ h$.\\
On déterminera $h$ par sa matrice $H$ dans la base $\mathcal{C}$, puis
on exprimera la matrice de $h$ dans la base $\mathcal{B}$ à l'aide de 
$H$ et de $P$.
\end{noliste}

\newpage

\section*{Exercice 3}

\noindent
Les deux parties sont indépendantes.\\
Soit $p\in \left] 0,1\right[.$ On note $q = 1-p.$

\subsection*{Partie I : Différence de deux variables aléatoires.}

\noindent
Soit $n$ un entier naturel non nul. On considère $n$ joueurs qui visent
une cible. Chaque joueur effectue deux tirs. À chaque tir, chaque
joueur a la probabilité $p$ d'atteindre la cible. Les tirs sont 
indépendants les uns des autres.\\
On définit la variable aléatoire $X$ égale au nombre de joueurs
ayant atteint la cible au premier tir et la variable aléatoire $Z$
égale au nombre de joueurs ayant atteint la cible au moins une fois à
l'issue des deux tirs.

\begin{noliste}{1.}
 \setlength{\itemsep}{4mm}
\item Déterminer la loi de $X$. Rappeler son espérance et sa variance.

\item Montrer que $Z$ suit une loi binomiale. Donner son espérance et
sa variance.\\
On note $Y = Z-X$.

\item Que représente la variable aléatoire $Y$ ? Déterminer la loi
de $Y$.

\item
\begin{noliste}{a)}
 \setlength{\itemsep}{2mm}
\item Les variables aléatoires $X$ et $Y$ sont-elles indépendantes ?

\item Calculer la covariance du couple $\left( X,Y\right) $.
\end{noliste}
\end{noliste}

\subsection*{Partie II : Variable aléatoire à densité
conditionnée par une variable aléatoire discrète}

\noindent
Dans cette partie, on note $U$ une variable aléatoire suivant la loi
géométrique de paramètre $p.$

\begin{noliste}{1.}
 \setlength{\itemsep}{4mm}
\item Rappeler la loi de $U$, son espérance et sa variance.\\
On considère une variable aléatoire $T$ telle que :
\qquad $\forall n\in \N^*,\ \forall t\in \left[ 0, + \infty
\right[, \ \Prob_{\Ev{ U = n} }(\Ev{ T>t}) = \ee^{-nt}.$

\item
\begin{noliste}{a)}
 \setlength{\itemsep}{2mm}
\item Montrer : $\forall t\in \left[ 0, + \infty \right[,\quad
\Prob\left(\Ev{T>t}\right) = \dfrac{p~\ee^{-t}}{1-q~\ee^{-t}}.$

\item Déterminer la fonction de répartition de la variable
aléatoire $T$.

\item En déduire que $T$ est une variable aléatoire à densité
et en déterminer une densité.
\end{noliste}

\item On note $Z = U \, T$.

\begin{noliste}{a)}
 \setlength{\itemsep}{2mm}
\item Montrer : $\forall n\in \N^*,\ \forall z\in \left[
0, + \infty \right[,\quad \Prob_{\Ev{ U = n} }\left(\Ev{
Z>z}\right) = \ee^{-z}.$

\item En déduire que la variable aléatoire $Z$ suit une loi
exponentielle dont on précisera le paramètre.

\item Montrer : $\forall n\in \N^*, \quad \forall z\in \left[
0, + \infty \right[,\quad \Prob \left( \Ev{U = n} \cap \Ev{Z>z}\right) =
\Prob\left(\Ev{U = n}\right) \Pro\left(\Ev{ Z>z}\right)$.
\end{noliste}
\end{noliste}





\end{document}
