\documentclass[11pt]{article}%
\usepackage{geometry}%
\geometry{a4paper,
 lmargin = 2cm,rmargin = 2cm,tmargin = 2.5cm,bmargin = 2.5cm}

\input{../../macros.tex}

\pagestyle{fancy} %
\lhead{ECE2 \hfill Mathématiques\\
} %
\chead{\hrule} %
\rhead{} %
\lfoot{} %
\cfoot{} %
\rfoot{\thepage} %

\renewcommand{\headrulewidth}{0pt}% : Trace un trait de séparation
 % de largeur 0,4 point. Mettre 0pt
 % pour supprimer le trait.

\renewcommand{\footrulewidth}{0.4pt}% : Trace un trait de séparation
 % de largeur 0,4 point. Mettre 0pt
 % pour supprimer le trait.

\setlength{\headheight}{14pt}

\title{\bf \vspace{-2cm} EML 2014} %
\author{} %
\date{} %
\begin{document}

\maketitle %
\vspace{-1.4cm}\hrule %
\thispagestyle{fancy}

\vspace*{.2cm}


% DEBUT DU DOC À MODIFIER : tout virer jusqu'au début de l'exo

%Définition et changement de valeurs de
compteurs%newcounter{cpt1}{section} compteur cpt1 remis à 0 à chaque
aumentation par stepcounter du compteur section%setcounter{cpt1}{3} on
met le compteur à 3%addtocounter{cpt1}{5} on ajoute 5 au compteur%
stepcounter{cpt1} on ajoute 1% ifthenelse{test}{alors}{sinon} (page
206) pour subordonner à une condition % whiledo{test}{commande} pour
faire une boucle (page 206 aussi) % value{cpt1} pour noter dans le
document la valeur de cpt1 
%Définition définitive d'opérateurs
mathématiques\newcommand{\ch}{\operatorname{ch}} 
\newcommand{\sh}{\operatorname{sh}}
\renewcommand{\tanh}{\operatorname{th}}
\renewcommand{\sinh}{\operatorname{sh}}
\renewcommand{\cosh}{\operatorname{ch}}
\newcommand{\argsh}{\operatorname{argsh}}
\newcommand{\argch}{\operatorname{argch}}
\newcommand{\argth}{\operatorname{argth}}
\newcommand{\Id}{\operatorname{Id}}
\newcommand{\id}{\operatorname{id}}
\renewcommand{\im}{\operatorname{Im}}
\renewcommand{\leq}{\leq}
\renewcommand{\geq}{\geq }

\newcommand{\dlim}{\lim}
\newcommand{\dsum}{\sum\limits}
\newcommand{\dprod}{\prod}
\newcommand{\lb}{\llbracket}
\newcommand{\rb}{\rrbracket}


%Définition de nouvelles couleurs : rgb(trois paramètres red green blue
entre 0 et 1); cmyk (quatre cyan magenta yellow black) entre 0 et 1;
gray (entre 0 et 1) et black, white, red, green, blue, cyan, magenta,
yellow% definecolor{0gris}{gray}{0.8} 
% Nouvelle commande pour encadrer le titre car shabox ne veut que d'une
seule ligne; ATTENTION A LA TAILLE; petite différence avec shadowbox ou
doublebox, voire fcolorbox ou colorbox (au lieu de shabox; laisser le
parbox tranquille sauf pour la taille de la boîte
\newcommand{\Tbox}[1]{\begin{center} \shabox{\parbox{0.8
\linewidth}{#1}} \end{center}} %[1] pour 1 paramètre ; #1 pour ce que
fait le 1er paramètre; entre accolades ce que fait la commande
%Mise en page en mode fancy : en-têtes et pieds de pages puis
définition des en-têtes et pieds de pages\pagestyle{fancy}
\lhead{ECE 2 - Mathématiques \\
Quentin Dunstetter - ENC-Bessières 2011$\backslash$2012}
\chead{}
\rhead{EML 2014}
\rfoot[ \ \thepage]{\thepage}
\cfoot{}
\lfoot{}
\thispagestyle{fancy} %Mise en page de la 1ère page en mode fancy
%Trait en bas et en haut de la page (entre en-tête et texte et texte et
pied de page)\renewcommand{\footrulewidth}{0.4pt}
\renewcommand{\headrulewidth}{0.4pt}

\indent \vspace{0.3cm}

\Tbox{\begin{center} \textbf{\Huge EML 2014} \end{center} }

\vspace{0.5cm}



\section*{Exercice 1}

\noindent On considère l'application $ \varphi :]0; + \infty[ \
\rightarrow \R, x\mapsto e^{x}-xe^{\frac{1}{x}}$. On admet $2<e<3$.
\subsection*{\bf Partie I : Étude de la fonction $\varphi$}
\begin{noliste}{1.}
 \setlength{\itemsep}{4mm}
\item Montrer que $\varphi$ est de classe $C^{3}$ sur $]0; + \infty[$,
calculer, pour tout $x$ de $]0; + \infty[$, $ \varphi'(x)$ et
$\varphi''(x)$ et montrer : $\forall x\in \ ]0; + \infty[,
\varphi'''(x) = e^{x} + \dfrac{3x + 1}{x^{5}}e^{\frac{1}{x}}$.
\item Étudier le sens de variation de $\varphi''$ et calculer
$\varphi''(1)$.\\
En déduire le sens de variation de $\varphi'$, et montrer : $\forall
x\in \ ]0; + \infty[, \varphi'(x)\geq e$.
\item Déterminer la limite de $\varphi(x)$ lorsque $x$ tend vers 0 par
valeurs strictement positives.
\item Déterminer la limite de $\dfrac{\varphi(x)}{x}$ lorsque $x$ tend
vers $ + \infty$, et la limite de $\varphi(x)$ lorsque $x$ tend vers $
+ \infty$.
\item On admet : $15<\varphi(3)<16$. Montrer : $\forall x\in[3; +
\infty[, \varphi(x)\geq ex$.\\
On note $\mathcal{C}$ la courbe représentative de $\varphi$.
\item Montrer que $\mathcal{C}$ admet un unique point d'inflexion,
déterminer les coordonnées de celui-ci et l'équation de la tangente en
ce point.
\item Dresser le tableau de variations de $\varphi$, avec les limites
en 0 et en $ + \infty$, et la valeur en 1.\\
Tracer l'allure de $\mathcal{C}$ et faire apparaître la tangente au
point d'inflexion.\\
On précisera la nature de la branche infinie au voisinage de 0 et la
nature de la branche infinie au voisinage de $ + \infty$.
\end{noliste}


\subsection*{\bf Partie II : Étude d'extremum pour une fonction réelle
de deux variables réelles}
\noindent On note $U = \R\times ]0; + \infty[$ et on considère
l'application : $f :U\rightarrow \R, (x,y)\mapsto xy-e^{x}\ln y$.
\begin{noliste}{1.}
 \setlength{\itemsep}{4mm}
\item Représenter graphiqement l'ensemble $U$.
\item Montrer que $f$ est de classe $C^{2}$ sur l'ouvert $U$ et
calculer, pour tout $(x,y)$ de $U$, les dérivées partielles premières
et les dérivées partielles secondes de $f$ au point $(x,y)$.
\item Établir que, pour tout $(x,y)$ de $U$, $(x,y)$ est un point
critique de $f$ si et seulement si :\\
\[
 x>0 \text{ et } y = e^{\frac{1}{x}} \text{ et } \varphi(x) = 0
\]
\item En déduire que $f$ admet un point critique et un seul, et qu'il
s'agit de $(1,e)$.
\item Est-ce que $f$ admet un extremum local en $(1,e)$ ?
\item Est-ce que $f$ admet un extremum local sur $U$ ?
\end{noliste}
\subsection*{\bf Partie III : Étude d'une suite et d'une série}
\noindent On considère la suite réelle $(u_{n})_{n\in \N}$ définie par
$u_{0} = 3$ et : $\forall n\in \N, u_{n + 1} = \varphi(u_{n})$.
\begin{noliste}{1.}
 \setlength{\itemsep}{4mm}
\item Montrer que, pour tout $n$ de $\N$, $u_{n}$ existe et $u_{n}\geq
3e^{n}$. (On pourra utiliser les résultats de la partie I).
\item Montrer que la suite $(u_{n})$ est strictement croissante et que
$u_{n}$ tend vers $ + \infty$ lorsque $n$ tend vers l'infini.
\item Écrire un programme en -\Scilab{} qui affiche et calcule le plus
petit entier $n$ tel que \\
$u_{n}\geq 10^{3}$.
\item Quelle est la nature de la série de terme général
$\dfrac{1}{u_{n}}$ ?\\
\end{noliste}


\section*{Exercice 2}
\noindent On considère l'espace $\M{2}$ des matrices d'ordre 2 à
coefficients réels. On définit :
\[
A = 
\begin{smatrix}
1 & 0\\
0 & 0
\end{smatrix}, B = 
\begin{smatrix}
0 & 1\\
0 & 0
\end{smatrix}, C = 
\begin{smatrix}
0 & 0\\
0 & 1
\end{smatrix}, T = 
\begin{smatrix}
1 & 1\\
0 & 1
\end{smatrix}
\]
\[
\mathcal{E} = \left\{\begin{smatrix}
a & b\\
0 & c
\end{smatrix}, (a,b,c)\in \R^{3}\right\}
\]
\begin{noliste}{1.}
 \setlength{\itemsep}{4mm}
\item Montrer que $\mathcal{E}$ est un espace vectoriel et que
$(A,B,C)$ est une base de $\mathcal{E}$.
\item Établir que $\mathcal{E}$ est stable par multiplication, c'est à
dire :
\[
\forall (M,N)\in \mathcal{E}{2}, MN\in \mathcal{E}
\]
\item Montrer que, pour toute matrice $M$ de $\mathcal{E}$, si $M$ est
inversible alors $M^{-1}\in \mathcal{E}$.\\
\\
Pour toute matrice de $\mathcal{E}$, on note $f(M) = TMT$.
\item Montrer que $f$ est un endomorphisme de $\mathcal{E}$.
\item Vérifier que $T$ est inversible et démontrer que $f$ est un
automorphisme de $\mathcal{E}$.
\item Est-ce que $T$ est diagonalisable ?\\
On note $F$ la matrice de $f$ dans la base $(A,B,C)$ de $\mathcal{E}$.
\item Calculer $f(A), f(B), f(C)$ en fonction de $(A,B,C)$ et en
déduire $F$.
\item Montrer que $f$ admet une valeur propre et une seule et
déterminer celle-ci, puis déterminer une base et la dimension du
sous-espace propre pour $f$ associé à cette valeur propre.
\item Est-ce que $f$ est diagonalisable ?
\item Soit $\lambda$ un réel différent de 1. Résoudre l'équation $f(M)
= \lambda M$, d'inconnue $M\in \mathcal{E}$.\\
On note $I = 
\begin{smatrix}
1 & 0 & 0\\
0 & 1 & 0\\
0 & 0 & 1
\end{smatrix}
$ et $H = 
\begin{smatrix}
0 & 0 & 0\\
1 & 0 & 1\\
0 & 0 & 0
\end{smatrix}
$.
\item Calculer $H^{2}$, puis pour tout $a$ de $\R$ et tout $n$ de $\N$,
$(I + aH)^{n}$.
\item Calculer, pour tout $n$ de $\N$, $F^{n}$.
\item Trouver une matrice $G$ de $\M{3}$ telle que $G^{3} = F$.
Existe-t-il un endomorphisme $g$ de $\mathcal{E}$ tel que $g\circ
g\circ g = f$ ?\\
\\

\end{noliste}

\newpage

\section*{Exercice 3}

\noindent Pour tout entier $n$ supérieur ou égal à 2, on considère une
urne contenant $n$ boules numérotées de 1 à $n$, dans laquelle on
effectue une succession de $(n + 1)$ tirages d'une boule avec remise et
l'on note $X_{n}$ la variable aléatoire égale au numéro du tirage où,
pour la première fois, on a obtenu un numéro supérieur ou égal au
numéro précédent.\\
Ainsi, pour tout entier $n$ supérieur ou égal à 2, la variables $X_{n}$
prend ses valeurs dans $[ \ \![2;n + 1]\!]$. Par exemple, si $n = 5$ et
si les tirages amènent successivement les numéros 5,3,2,2,6,3, alors
$X_{5} = 4$. Pour tout $k$ de $[ \ \![1;n + 1]\!]$, on note $N_{k}$ la
variable aléatoire égale au numéro obtenu au $k$-ième tirage.
\subsection*{Partie I : Étude du cas $n = 3$}
\noindent On suppose dans cette partie {\bf uniquement} que $n = 3$.
L'urne contient donc les boules numérotées 1, 2, 3.
\begin{noliste}{1.}
 \setlength{\itemsep}{4mm}
\item
\begin{noliste}{a)}
 \setlength{\itemsep}{2mm}
\item Exprimer l'évènement $(X_{3} = 4)$ à l'aide d'évènements faisant
intervenir les variables $N_{1}, N_{2}, N_{3}$. En déduire
$\Prob\left(\Ev{X_{3} = 4}\right)$.
\item Montrer que $\Prob\left(\Ev{X_{3} = 2}\right) = \dfrac{2}{3}$, et
en déduire $\Prob\left(\Ev{X_{3} = 3}\right)$.
\end{noliste}
\item Calculer l'espérance de $X_{3}$.
\end{noliste}
\subsection*{Partie II : Cas général}
\noindent Dans toute cette partie, $n$ est un entier fixé supérieur ou
égal à 2.
\begin{noliste}{1.}
 \setlength{\itemsep}{4mm}
\item Pour tout $k$ de $ \lb 1;n + 1 \rb$, reconnaître la loi de
$N_{k}$ et rappeler son espérance et sa variance.
\item Calculer $\Prob\left(\Ev{X_{n} = n + 1}\right)$.
\item Montrer, pour tout $i$ de $ \lb 1;n \rb$ : $\Pr_{(N_{1} =
i)}(X_{n} = 2) = \dfrac{n-i + 1}{n}$.
\item En déduire une expression simple de $\Prob\left(\Ev{X_{n} =
2}\right)$.
\item Soit $k\in \ln 2;n \rb $. Justifier l'égalité d'évènements
suivante : $(X_{n}>k) = (N_{1}>N_{2}>\ldots>N_{k})$.\\
En déduire que $\Prob\left(\Ev{X_{n}>k}\right) =
\dfrac{1}{n^{k}}\binom{n}{k}$.\\
Vérifier que cette dernière égalité reste valable pour $k = 0$ et pour
$k = 1$.
\item Exprimer, pour tout $k\in \lb 2;n + 1 \rb, \Prob\left(\Ev{X_{n} =
k}\right)$ à l'aide de $\Prob\left(\Ev{X_{n}>k-1}\right)$ et de
$\Prob\left(\Ev{X_{n}>k}\right)$.
\item En déduire : $\mathrm{E}(X_{n}) = \Sum{k = 0}{n}
\Prob\left(\Ev{X_{n}>k}\right)$. Calculer ensuite $\mathrm{E}(X_{n})$.
\item Montrer : $\forall k\in \lb 2;n + 1 \rb, \Prob\left(\Ev{X_{n} =
k}\right) = \dfrac{k-1}{n^{k}}\binom{n + 1}{k}$.
\end{noliste}
\subsection*{Partie III : Une convergence en loi}
\noindent On s'intéresse dans cette partie à la suite de variables
aléatoires $(X_{n})_{n\geq 2}$.
\begin{noliste}{1.}
 \setlength{\itemsep}{4mm}
\item Soit $k$ un entier fixé supérieur ou égal à 2. Montrer :
$\underset{n\to + \infty}{\lim} \Prob\left(\Ev{X_{n} = k}\right) =
\dfrac{k-1}{k!}$.
\item Montrer que la série $\Sum{k\geq 2}\dfrac{k-1}{k!}$ converge et
calculer sa somme.\\
On admet qu'il existe une variable aléatoire $Z$ à valeurs dans $\lb 2;
+ \infty\lb$ telle que :
\[
\forall k\in \lb2; + \infty\lb, \Prob\left(\Ev{Z = k}\right) =
\dfrac{k-1}{k!}
\]
\item Montrer que $Z$ admet une espérance et la calculer. Comparer
$\mathrm{E}(Z)$ et $\underset{n\to + \infty}{\lim} \mathrm{E}(X_{n})$.


\end{noliste}


\end{document}

