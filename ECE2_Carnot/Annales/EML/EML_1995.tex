\documentclass[11pt]{article}%
\usepackage{geometry}%
\geometry{a4paper,
 lmargin = 2cm,rmargin = 2cm,tmargin = 2.5cm,bmargin = 2.5cm}

\input{../../macros.tex}

\pagestyle{fancy} %
\lhead{ECE2 \hfill Mathématiques\\
} %
\chead{\hrule} %
\rhead{} %
\lfoot{} %
\cfoot{} %
\rfoot{\thepage} %

\renewcommand{\headrulewidth}{0pt}% : Trace un trait de séparation
 % de largeur 0,4 point. Mettre 0pt
 % pour supprimer le trait.

\renewcommand{\footrulewidth}{0.4pt}% : Trace un trait de séparation
 % de largeur 0,4 point. Mettre 0pt
 % pour supprimer le trait.

\setlength{\headheight}{14pt}

\title{\bf \vspace{-2cm} EML 1995} %
\author{} %
\date{} %
\begin{document}

\maketitle %
\vspace{-1.4cm}\hrule %
\thispagestyle{fancy}

\vspace*{.2cm}


% DEBUT DU DOC À MODIFIER : tout virer jusqu'au début de l'exo

%Définition et changement de valeurs de
compteurs%newcounter{cpt1}{section} compteur cpt1 remis à 0 à chaque
aumentation par stepcounter du compteur section%setcounter{cpt1}{3} on
met le compteur à 3%addtocounter{cpt1}{5} on ajoute 5 au compteur%
stepcounter{cpt1} on ajoute 1% ifthenelse{test}{alors}{sinon} (page
206) pour subordonner à une condition % whiledo{test}{commande} pour
faire une boucle (page 206 aussi) % value{cpt1} pour noter dans le
document la valeur de cpt1 
%Définition définitive d'opérateurs
mathématiques\newcommand{\ch}{\operatorname{ch}} 
\newcommand{\sh}{\operatorname{sh}}
\renewcommand{\tanh}{\operatorname{th}}
\renewcommand{\sinh}{\operatorname{sh}}
\renewcommand{\cosh}{\operatorname{ch}}
\newcommand{\argsh}{\operatorname{argsh}}
\newcommand{\argch}{\operatorname{argch}}
\newcommand{\argth}{\operatorname{argth}}
\newcommand{\ker}{\operatorname{Ker}}
\renewcommand{\im}{\operatorname{Im}}
\newcommand{\rg}{\operatorname{rg}}
\newcommand{\Id}{\operatorname{Id}}
\newcommand{\id}{\operatorname{id}}
\renewcommand{\leq}{\leq}
\renewcommand{\geq}{\geq }

%Définition de nouvelles couleurs : rgb(trois paramètres red green blue
entre 0 et 1); cmyk (quatre cyan magenta yellow black) entre 0 et 1;
gray (entre 0 et 1) et black, white, red, green, blue, cyan, magenta,
yellow% definecolor{0gris}{gray}{0.8} 
% Nouvelle commande pour encadrer le titre car shabox ne veut que d'une
seule ligne; ATTENTION A LA TAILLE; petite différence avec shadowbox ou
doublebox, voire fcolorbox ou colorbox (au lieu de shabox; laisser le
parbox tranquille sauf pour la taille de la boîte
\newcommand{\Tbox}[1]{\begin{center} \shabox{\parbox{0.6
\linewidth}{#1}} \end{center}} %[1] pour 1 paramètre ; #1 pour ce que
fait le 1er paramètre; entre accolades ce que fait la commande
%Mise en page en mode fancy : en-têtes et pieds de pages puis
définition des en-têtes et pieds de pages\pagestyle{fancy}
\lhead{ECE 2 - Mathématiques \\
Quentin Dunstetter - ENC-Bessières 2011$\backslash$2012}
\chead{}
\rhead{EML 1995}
\rfoot[ \ \thepage]{\thepage}
\cfoot{}
\lfoot{}
\thispagestyle{fancy} %Mise en page de la 1ère page en mode fancy
%Trait en bas et en haut de la page (entre en-tête et texte et texte et
pied de page)\renewcommand{\footrulewidth}{0.4pt}
\renewcommand{\headrulewidth}{0.4pt}


%DEBUT DU DOCUMENT\vspace*{3cm}

\begin{center}
{\LARG\E\textbf{BANQUE COMMUNE D'ÉPREUVES}}



{\large \textsc{CONCOURS D ADMISSION DE 1995}}



{\large \textbf{Concepteur : EML}}



\rule{2.39cm}{0.05cm}



{\Large \textbf{OPTION ÉCONOMIQUE}}



{\Large \textbf{MATHÉMATIQUES }}



{\Large Lundi 9 mai, de 14h à 18h}



\rule{2.39cm}{0.05cm}
\end{center}

\textit{La présentation, la lisibilité, l'orthographe, la qualité
de la rédaction, la clarté et la précision des raisonnements
entreront pour une part importante dans l'appréciation des copies.}

\textit{Les candidats sont invités à \textbf{encadrer} dans la mesure
du possible les résultats de leurs calculs.}

\textit{Ils ne doivent faire usage d'aucun document. L'utilisation de
toute
calculatrice et de tout matériel électronique est interdite. Seule
l'utilisation d'une règle graduée est autorisée.}

\textit{Si au cours de l'épreuve, un candidat repère ce qui lui semble
être une erreur d'énoncé, il la signalera sur sa copie et
poursuivra sa composition en expliquant les raisons des initiatives
qu'il sera
amené à prendre.}

\vspace*{3cm}

\section*{Exercice 1}

On considère la matrice carrée d'ordre 3 réelle : 
\[
A = 
\begin{smatrix}
5 & -1 & -1 \\
2 & 2 & -1 \\
2 & -1 & 2
\end{smatrix}.
\]

\begin{noliste}{1.}
 \setlength{\itemsep}{4mm}
\item 
\begin{noliste}{a)}
 \setlength{\itemsep}{2mm}
\item Établir que $A$ admet une valeur propre et un seule, $\lambda $,
que l'on calculera.

\item $A$ est-elle inversible ?

\item $A$ est-elle diagonalisable ?
\end{noliste}

\item On note $B = A-3I$, où $I = 
\begin{smatrix}
1 & 0 & 0 \\
0 & 1 & 0 \\
0 & 0 & 1
\end{smatrix}
$.

\begin{noliste}{a)}
 \setlength{\itemsep}{2mm}
\item Calculer $B^{2}$.

\item En déduire, pour tout entier naturel $n$, l'expression de $A^{n}$
en fonction de $n$.
\end{noliste}
\end{noliste}

\section*{Exercice2}

Soit $f :\begin{array}[t]{rcl}
\lbrack 0; + \infty \lbrack & \longrightarrow & \R \\
x & \longmapsto & x\ln \,(1 + x)
\end{array}.$

On considère la suite $(u_{n})_{n\in \N}$ définie par $u_{0}\in \ ]0; +
\infty \lbrack $ et, pour tout $n$ de $\N$, $u_{n + 1} = f(u_{n}).$

\begin{noliste}{1.}
 \setlength{\itemsep}{4mm}
\item 
\begin{noliste}{a)}
 \setlength{\itemsep}{2mm}
\item Montrer que $f$ est de classe $C^{2}$ sur $[0; + \infty \lbrack $
et
calculer, pour tout $x$ de $[0; + \infty \lbrack $, $f^{\prime }(x)$ et
$f^{\prime \prime }(x)$.

\item Étudier les variations de $f^{\prime }$, puis celles de $f$.

\item Tracer la courbe représentative de $f$ dans un repère orthonormé.
\end{noliste}

\item Résoudre l'équation $f(x) = x$, d'inconnue $x\in \lbrack
0; + \infty \lbrack.$

\item On suppose dans cette question : $u_{0}\in \left] e-1; + \infty
\right[. $

\begin{noliste}{a)}
 \setlength{\itemsep}{2mm}
\item Montrer que, pour tout $n$ de $\N :e-1<u_{n}\leq u_{n + 1}$.

\item En déduire que $u_{n}$ tend vers $ + \infty $ lorsque $n$ tend
vers $ + \infty $.
\end{noliste}

\item On suppose, dans cette question : $u_{0}\in \left]
0;e{-1}\right[.$ 
Étudier la convergence de $(u_{n})_{n\in \N}$.
\end{noliste}

\section*{Exercice 3}

Dans un jeu, il y a $n$ numéros (de 1 à $n$) dont $p$ numéros
gagnants choisis à l'avance et connus du seul meneur de jeu.

On suppose : $n\in \N^{\ast }$, $p\in \N^{\ast }$, $p\leq 
\dfrac{n}{3}.$

Dans la première phase du jeu, le joueur tire au hasard,
successivement, 
$p$ numéros différents. Le meneur de jeu dévoile alors $p$ numéros
perdants parmi les $n-p$ numéros qui n'ont pas été tirés.

Dans la deuxième phase du jeu, le joueur a le choix entre deux
stratégies.

\begin{noliste}{$\sbullet$}
\item {Stratégie }A : il garde les $p$ numéros qu'il a tirés.

\item Stratégie B : il échange les $p$ numéros qu'il a tirés
contre $p$ nouveaux numéros tirés au hasard, successivement, parmi
les $n-2p$ numéros qui n'ont été ni tirés ni dévoilés durant la
première phase.
\end{noliste}

Le but de l'exercice est de déterminer laquelle de ces deux stratégies
permet d'espérer obtenir le plus de numéros gagnants.

\begin{noliste}{1.}
 \setlength{\itemsep}{4mm}
\item \textbf{Étude directe d'un cas simple}

On suppose ici : $n = 3$, $p = 1$. Calculer la probabilité d'obtenir le
numéro gagnant avec la stratégie $A$, puis avec la stratégie $B$.

\item \textbf{Étude du cas général}

Pour $1\leq i\leq p$, on note $X_{i}$ la variable aléatoire égale 
à 1 si le $i-$ème numéro tiré dans la première phase est
gagnant, à 0 sinon.

On note $X$ la variable aléatoire égale au nombre de numéros
gagnants parmi les $p$ numéros tirés dans la première phase.

Ainsi : $X = X_{1} + \ldots + X_{p}$.

\begin{noliste}{a)}
 \setlength{\itemsep}{2mm}
\item Démontrer que, pour $1\leq i\leq p$ : $P\left(\Ev{X_{i} =
1}\right) = \dfrac{p}{n}.$
En déduire : $\E(X) = \dfrac{p^{2}}{n}.$

\item Déterminer la loi de $X$. En déduire les formules : 
\[
(1)\quad \Sum{k = 0}{p}\binom{p}{k}\binom{n-p}{p-k} =
\binom{n}{p}\qquad
\qquad (2)\quad \Sum{k = 0}{p}k\binom{p}{k}\binom{n-p}{p-k} =
\frac{p^{2}}{n}\binom{n}{p}.
\]
\end{noliste}

On suppose désormais, dans toute la suite de l'exercice, que le joueur
utilise la stratégie $B$.

Pour $1\leq i\leq p$, on note $Z_{i}$ la variable aléatoire égale 
à 1 si le $i-$ème numéro tiré durant la deuxième phase
est gagnant, à 0 sinon.

On note $Z$ la variable aléatoire égale au nombre de numéros
gagnants parmi les $p$ numéros tirés dans la deuxième phase.

\item 
\begin{noliste}{a)}
 \setlength{\itemsep}{2mm}
\item Pour tout entier naturel $k$ tel que $0\leq k\leq n$ et pour
$1\leq
i\leq p$, calculer la probabilité conditionnelle : $P_{X = k}(Z_{i} =
1)$.

\item Pour $1\leq i\leq p$, démontrer que : 
\[
P\left(\Ev{Z_{i} = 1}\right) = \frac{1}{(n-2p)\binom{n}{p}}\Sum{k =
0}{p}(p-k)\binom{p}{k}\binom{n-p}{p-k}.
\]

\item En utilisant les formules démontrées en b), vérifier que : 
\[
\E(Z) = \frac{p^{2}(n-p)}{n(n-2p)}.
\]
Des stratégies $A$ et $B$, laquelle est préférable ?
\end{noliste}
\end{noliste}

\section*{Exercice 4}

On définit la fonction 
\[
f :\begin{array}[t]{rcl}
\lbrack 2;{+ \infty }[ & \longrightarrow & {\R\xspace} \\
x & \longmapsto & \frac{1}{\sqrt{x^{2}-1}}
\end{array}
\]

\begin{noliste}{1.}
 \setlength{\itemsep}{4mm}
\item Démontrer que, pour tout réel $x$ supérieur ou égal 
à 2 : 
\[
\frac{1}{x}\leq f(x)\leq \frac{1}{\sqrt{x-1}}.
\]

\item Pour tout entier $n$ supérieur ou égal à 2, on définit
l'intégrale : 
\[
I_{n} = \dint{2}{n}{f(x)}\, \mathrm{d}x.
\]

\begin{noliste}{a)}
 \setlength{\itemsep}{2mm}
\item Démontrer que : $ \dlim{n\rightarrow + \infty
}I_{n} = + \infty $.

\item On définit la fonction 
\[
F :\begin{array}[t]{rcl}
\lbrack 2;{+ \infty }[ & \longrightarrow & {\R\xspace} \\
x & \longmapsto & {\ln \,(x + \sqrt{x^{2}-1}).}
\end{array}
\]
Calculer la dérivée de $F$, et en déduire une expression de $I_{n}$ en
fonction de $n$.

\item Déterminer la limite de $I_{n}-\ln \,(n)$ quand $n$ tend vers $ +
\infty $.
\end{noliste}

\item On définit, pour tout entier naturel $n$ supérieur ou égal 
à 2 : 
\[
S_{n} = \Sum{k = 2}{n}\frac{1}{\sqrt{k^{2}-1}}.
\]

\begin{noliste}{a)}
 \setlength{\itemsep}{2mm}
\item Montrer que : $I_{n + 1}\leq S_{n}\leq I_{n} +
\frac{1}{\sqrt{3}}.$

\item Trouver un équivalent simple de $S_{n}$ quand $n$ tend vers $ +
\infty $.
\end{noliste}
\end{noliste}

\end{document}

