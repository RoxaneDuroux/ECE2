\documentclass[11pt]{article}%
\usepackage{geometry}%
\geometry{a4paper,
 lmargin = 2cm,rmargin = 2cm,tmargin = 2.5cm,bmargin = 2.5cm}

\input{../../macros.tex}

\pagestyle{fancy} %
\lhead{ECE2 \hfill Mathématiques\\
} %
\chead{\hrule} %
\rhead{} %
\lfoot{} %
\cfoot{} %
\rfoot{\thepage} %

\renewcommand{\headrulewidth}{0pt}% : Trace un trait de séparation
 % de largeur 0,4 point. Mettre 0pt
 % pour supprimer le trait.

\renewcommand{\footrulewidth}{0.4pt}% : Trace un trait de séparation
 % de largeur 0,4 point. Mettre 0pt
 % pour supprimer le trait.

\setlength{\headheight}{14pt}

\title{\bf \vspace{-2cm} EML 1990} %
\author{} %
\date{} %
\begin{document}

\maketitle %
\vspace{-1.4cm}\hrule %
\thispagestyle{fancy}

\vspace*{.2cm}


% DEBUT DU DOC À MODIFIER : tout virer jusqu'au début de l'exo

%Définition et changement de valeurs de
compteurs%newcounter{cpt1}{section} compteur cpt1 remis à 0 à chaque
aumentation par stepcounter du compteur section%setcounter{cpt1}{3} on
met le compteur à 3%addtocounter{cpt1}{5} on ajoute 5 au compteur%
stepcounter{cpt1} on ajoute 1% ifthenelse{test}{alors}{sinon} (page
206) pour subordonner à une condition % whiledo{test}{commande} pour
faire une boucle (page 206 aussi) % value{cpt1} pour noter dans le
document la valeur de cpt1 
%Définition définitive d'opérateurs
mathématiques\newcommand{\ch}{\operatorname{ch}} 
\newcommand{\sh}{\operatorname{sh}}
\renewcommand{\tanh}{\operatorname{th}}
\renewcommand{\sinh}{\operatorname{sh}}
\renewcommand{\cosh}{\operatorname{ch}}
\newcommand{\argsh}{\operatorname{argsh}}
\newcommand{\argch}{\operatorname{argch}}
\newcommand{\argth}{\operatorname{argth}}
\newcommand{\ker}{\operatorname{Ker}}
\renewcommand{\im}{\operatorname{Im}}
\newcommand{\rg}{\operatorname{rg}}
\newcommand{\Id}{\operatorname{Id}}
\newcommand{\id}{\operatorname{id}}
\renewcommand{\leq}{\leq}
\renewcommand{\geq}{\geq }

%Définition de nouvelles couleurs : rgb(trois paramètres red green blue
entre 0 et 1); cmyk (quatre cyan magenta yellow black) entre 0 et 1;
gray (entre 0 et 1) et black, white, red, green, blue, cyan, magenta,
yellow% definecolor{0gris}{gray}{0.8} 
% Nouvelle commande pour encadrer le titre car shabox ne veut que d'une
seule ligne; ATTENTION A LA TAILLE; petite différence avec shadowbox ou
doublebox, voire fcolorbox ou colorbox (au lieu de shabox; laisser le
parbox tranquille sauf pour la taille de la boîte
\newcommand{\Tbox}[1]{\begin{center} \shabox{\parbox{0.6
\linewidth}{#1}} \end{center}} %[1] pour 1 paramètre ; #1 pour ce que
fait le 1er paramètre; entre accolades ce que fait la commande
%Mise en page en mode fancy : en-têtes et pieds de pages puis
définition des en-têtes et pieds de pages\pagestyle{fancy}
\lhead{ECE 2 - Mathématiques \\
Quentin Dunstetter - ENC-Bessières 2011$\backslash$2012}
\chead{}
\rhead{EML 1990}
\rfoot[ \ \thepage]{\thepage}
\cfoot{}
\lfoot{}
\thispagestyle{fancy} %Mise en page de la 1ère page en mode fancy
%Trait en bas et en haut de la page (entre en-tête et texte et texte et
pied de page)\renewcommand{\footrulewidth}{0.4pt}
\renewcommand{\headrulewidth}{0.4pt}


%DEBUT DU DOCUMENT\vspace*{3cm}

\begin{center}
{\LARG\E\textbf{BANQUE COMMUNE D'ÉPREUVES}}



{\large \textsc{CONCOURS D ADMISSION DE 1990}}



{\large \textbf{Concepteur : EML}}



\rule{2.39cm}{0.05cm}



{\Large \textbf{OPTION ÉCONOMIQUE}}



{\Large \textbf{MATHÉMATIQUES }}



{\Large Lundi 9 mai, de 14h à 18h}



\rule{2.39cm}{0.05cm}
\end{center}

\textit{La présentation, la lisibilité, l'orthographe, la qualité
de la rédaction, la clarté et la précision des raisonnements
entreront pour une part importante dans l'appréciation des copies.}

\textit{Les candidats sont invités à \textbf{encadrer} dans la mesure
du possible les résultats de leurs calculs.}

\textit{Ils ne doivent faire usage d'aucun document. L'utilisation de
toute
calculatrice et de tout matériel électronique est interdite. Seule
l'utilisation d'une règle graduée est autorisée.}

\textit{Si au cours de l'épreuve, un candidat repère ce qui lui semble
être une erreur d'énoncé, il la signalera sur sa copie et
poursuivra sa composition en expliquant les raisons des initiatives
qu'il sera
amené à prendre.}

\vspace*{3cm}

\section*{EXERCICE 1}

Soient $f$ et $g$ les endomorphismes de l'espace vectoriel $\R$ $^{3} $
dont les matrices, relativement à la base canonique de $\R$ $^{3}$ sont
respectivement : 
\[
F = 
\begin{smatrix}
1 & -1 & -1 \\
-2 & 2 & 3 \\
2 & -2 & -3
\end{smatrix}
\qquad G = 
\begin{smatrix}
1 & 0 & 0 \\
-2 & 3 & 4 \\
2 & -2 & -3
\end{smatrix}
\]

\begin{noliste}{1.}
 \setlength{\itemsep}{4mm}
\item 

\begin{noliste}{a)}
 \setlength{\itemsep}{2mm}
\item Déterminer les valeurs propres et les vecteurs propres de $f$.

\item Déterminer les valeurs propres et les vecteurs propres de $g$.
\end{noliste}

\item Montrer qu'il existe une base de $\R$ $^{3}$ formée de
vecteurs propres de $f$ et de $g$.

\item Pour $\alpha \in \R$, soit $H(\alpha )$ la matrice carrée
d'ordre 3 suivante : 
\[
H(\alpha ) = 
\begin{smatrix}
1 & -\alpha & -\alpha \\
-2 & 3-\alpha & 4-\alpha \\
2 & -2 & 3
\end{smatrix}
\]

\begin{noliste}{a)}
 \setlength{\itemsep}{2mm}
\item Montrer que, pour tout $\alpha \in \R$\ : $H(\alpha ) = \alpha
F + (1-\alpha )G$.

\item Calculer, pour tout $\alpha \in \R$\ et tout $n\in \N^{\times }$,
$\left( H(\alpha )\right) ^{n}$.
\end{noliste}
\end{noliste}

\section*{EXERCICE 2}

$\ln $ désigne la fonction logarithme népérien.\\
On considère la fonction $f$ définie sur $\left[ 0;\dfrac{\pi
}{2}\right] $
par \ 
\[
f(x) = 1-\dfrac{1}{4}(x + \cos x + \ln (1 + x))
\]

\begin{noliste}{1.}
 \setlength{\itemsep}{4mm}
\item 

\begin{noliste}{a)}
 \setlength{\itemsep}{2mm}
\item Calculer la dérivée $f^{\prime }$ de la fonction $f$ et dresser
le
tableau de variations de $f$. \\
En déduire que, pour tout $x\in \left[ 0;\dfrac{\pi }{2}\right] $,
$f(x)$
est compris entre $0$ et $\dfrac{\pi }{2}$.

\item Démontrer que l'équation $f(x) = x$ admet dans $\left[
0;\dfrac{\pi }{2}\right] $ une unique solution que l'on notera $a$.\\
(On pourra étudier la fonction $g$ définie par $g(x) = f(x)-x$).
\end{noliste}

\item On considère la suite $(u_{n})_{n\in \N}$ définie par : 
\[
\left\{ 
\begin{array}{l}
u_{0} = 0 \\
\forall n\in \N,u_{n + 1} = f(u_{n})
\end{array}
\right.
\]

\begin{noliste}{a)}
 \setlength{\itemsep}{2mm}
\item Montrer que $u_{0}\leq a\leq u_{1}$ et que, pour tout entier
naturel $n$, $u_{2n}\leq a\leq u_{2n + 1}$.

\item Dresser sur le modèle suivant un tableau où figurent des valeurs
approchées par défaut des nombres $u_{2n}$, des valeurs approchées par
excès
des nombres $u_{2n + 1}$, et des valeurs approchées par excès des
nombres $u_{2n + 1}-u_{2n}$.
\[
\begin{tabular}{|c|c|c|c|}
\hline
$n$ & $u_{2n}$ & $u_{2n + 1}$ & $u_{2n + 1}-u_{2n}$ \\
\hline
0 & 0 & 0,75 & 0,75 \\
\hline
1 &... &... &... \\
\hline
2 &... &... &... \\
\hline
3 &... &... &... \\
\hline
4 &... &... &... \\
\hline
\end{tabular}
\]

En déduire une valeur approchée de $a$ à $10^{-4}$ près.
\end{noliste}
\end{noliste}

\section*{EXERCICE 3}

\begin{noliste}{1.}
 \setlength{\itemsep}{4mm}
\item On rappelle que la série géométrique de terme général $x^{n}$ est
convergente pour $x\in \ ]-1;1[$, et que sa somme $\Sum{n = 0}{+ \infty
}x^{n}$
est égale à $\dfrac{1}{1-x}$. On rappelle également que la fonction $S$
ainsi définie sur $]-1;1[$ est indéfiniment dérivable et que, pour tout
entier naturel $k$, sa dérivée $k^{\text{ème}}$ $S^{(k)}$ est définie
sur $]-1;1[$ par 
\[
S^{(k)}(x) = \Sum{n = k}{+ \infty }n(n-1)...(n-k + 1)x^{n-k} =
\dfrac{k!}{(1-x)^{k + 1}}
\]
Démontrer que pour tout $x\in \ ]-1;1[$ et pour tout entier naturel
$k$, 
\[
\Sum{n = k}{+ \infty }C_{n}{k}x^{n} = \dfrac{x^{k}}{(1-x)^{k + 1}}
\]

\item Soit $p$ un nombre réel tel que $0<p<\dfrac{2}{3}$. \\
Dans un pays, la probabilité $q_{n}$ qu'une famille ait exactement $n$
enfants est de $\dfrac{1}{2}p^{n}$ quand $n\geq 1$ ; par ailleurs, la
probabilité, à chaque naissance, d'avoir un garçon est de
$\dfrac{1}{2}$.

\begin{noliste}{a)}
 \setlength{\itemsep}{2mm}
\item Calculer la probabilité $q$ qu'une famille ait au moins un
enfant.
Calculer la probabilité $q_{0}$ qu'une famille n'ait aucun enfant.

\item Soient $n$ un entier naturel tel que $n\geq 1$ et $k$ un entier
tel que $0\leq k\leq n$. On considère une famille de $n$ enfants ;
calculer la probabilité pour que cette famille ait exactement $k$
garçons.

\item Soit un entier $k\geq 1$. Calculer la probabilité pour qu'une
famille ait exactement $k$ garçons.

\item Calculer la probabilité pour qu'une famille n'ait aucun garçon.
\end{noliste}
\end{noliste}

\section*{EXERCICE 4}

$\ln $ désigne la fonction logarithme népérien.\\
Pour tout $n\in $ $\N$, on note $I_{n} =
\dint{0}{1}\dfrac{x^{n}}{\sqrt{1 + x^{2}}}dx$ et $J_{n} =
\dint{0}{1}\dfrac{x^{n + 2}}{(1 + x^{2})\sqrt{1 + x^{2}}}dx$.

\begin{noliste}{1.}
 \setlength{\itemsep}{4mm}
\item 

\begin{noliste}{a)}
 \setlength{\itemsep}{2mm}
\item Quelle est la dérivée de la fonction $f$ : $\R$ $\rightarrow 
\[
\R$\ définie par $f(x) = \ln (x + \sqrt{1 + x^{2}})$ ?

\item Calculer $I_{0}$.
\end{noliste}

\item Calculer $I_{1}$.

\item 

\begin{noliste}{a)}
 \setlength{\itemsep}{2mm}
\item Montrer que pour tout $n\in $ $\N$ : $0\leq
I_{n}\leq \dfrac{1}{n + 1}$ ; en déduire la limite de $I_{n}$ quand $n$
tend vers $ + \infty $.

\item Montrer que $J_{n}$ tend vers 0 quand $n$ tend vers $ + \infty $.
\end{noliste}

\item 

\begin{noliste}{a)}
 \setlength{\itemsep}{2mm}
\item Établir, à l'aide d'une intégration par parties, que, pour tout
$n\in $
$\N$, $I_{n} = \dfrac{1}{(n + 1)\sqrt{2}} + \dfrac{1}{n + 1}J_{n}$.

\item Quelle est la limite de $nI_{n}$ quand $n$ tend vers $ + \infty $
?
\end{noliste}
\end{noliste}

\label{fin}

\end{document}

