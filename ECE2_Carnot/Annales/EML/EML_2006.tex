\documentclass[11pt]{article}%
\usepackage{geometry}%
\geometry{a4paper,
 lmargin = 2cm,rmargin = 2cm,tmargin = 2.5cm,bmargin = 2.5cm}

\input{../../macros.tex}

\pagestyle{fancy} %
\lhead{ECE2 \hfill Mathématiques\\
} %
\chead{\hrule} %
\rhead{} %
\lfoot{} %
\cfoot{} %
\rfoot{\thepage} %

\renewcommand{\headrulewidth}{0pt}% : Trace un trait de séparation
 % de largeur 0,4 point. Mettre 0pt
 % pour supprimer le trait.

\renewcommand{\footrulewidth}{0.4pt}% : Trace un trait de séparation
 % de largeur 0,4 point. Mettre 0pt
 % pour supprimer le trait.

\setlength{\headheight}{14pt}

\title{\bf \vspace{-2cm} EML 2006} %
\author{} %
\date{} %
\begin{document}

\maketitle %
\vspace{-1.4cm}\hrule %
\thispagestyle{fancy}

\vspace*{.2cm}


% DEBUT DU DOC À MODIFIER : tout virer jusqu'au début de l'exo

%Définition et changement de valeurs de
compteurs%newcounter{cpt1}{section} compteur cpt1 remis à 0 à chaque
aumentation par stepcounter du compteur section%setcounter{cpt1}{3} on
met le compteur à 3%addtocounter{cpt1}{5} on ajoute 5 au compteur%
stepcounter{cpt1} on ajoute 1% ifthenelse{test}{alors}{sinon} (page
206) pour subordonner à une condition % whiledo{test}{commande} pour
faire une boucle (page 206 aussi) % value{cpt1} pour noter dans le
document la valeur de cpt1 
%Définition définitive d'opérateurs
mathématiques\newcommand{\ch}{\operatorname{ch}} 
\newcommand{\sh}{\operatorname{sh}}
\renewcommand{\tanh}{\operatorname{th}}
\renewcommand{\sinh}{\operatorname{sh}}
\renewcommand{\cosh}{\operatorname{ch}}
\newcommand{\argsh}{\operatorname{argsh}}
\newcommand{\argch}{\operatorname{argch}}
\newcommand{\argth}{\operatorname{argth}}
\newcommand{\ker}{\operatorname{Ker}}
\renewcommand{\im}{\operatorname{Im}}
\newcommand{\rg}{\operatorname{rg}}
\newcommand{\Id}{\operatorname{Id}}
\newcommand{\id}{\operatorname{id}}
\renewcommand{\leq}{\leq}
\renewcommand{\geq}{\geq }

%Définition de nouvelles couleurs : rgb(trois paramètres red green blue
entre 0 et 1); cmyk (quatre cyan magenta yellow black) entre 0 et 1;
gray (entre 0 et 1) et black, white, red, green, blue, cyan, magenta,
yellow% definecolor{0gris}{gray}{0.8} 
% Nouvelle commande pour encadrer le titre car shabox ne veut que d'une
seule ligne; ATTENTION A LA TAILLE; petite différence avec shadowbox ou
doublebox, voire fcolorbox ou colorbox (au lieu de shabox; laisser le
parbox tranquille sauf pour la taille de la boîte
\newcommand{\Tbox}[1]{\begin{center} \shabox{\parbox{0.6
\linewidth}{#1}} \end{center}} %[1] pour 1 paramètre ; #1 pour ce que
fait le 1er paramètre; entre accolades ce que fait la commande
%Mise en page en mode fancy : en-têtes et pieds de pages puis
définition des en-têtes et pieds de pages\pagestyle{fancy}
\lhead{ECE 2 - Mathématiques \\
Quentin Dunstetter - ENC-Bessières 2011$\backslash$2012}
\chead{}
\rhead{EML 2006}
\rfoot[ \ \thepage]{\thepage}
\cfoot{}
\lfoot{}
\thispagestyle{fancy} %Mise en page de la 1ère page en mode fancy
%Trait en bas et en haut de la page (entre en-tête et texte et texte et
pied de page)\renewcommand{\footrulewidth}{0.4pt}
\renewcommand{\headrulewidth}{0.4pt}


%DEBUT DU DOCUMENT\vspace*{3cm}

\begin{center}
{\LARG\E\textbf{BANQUE COMMUNE D'ÉPREUVES}}



{\large \textsc{CONCOURS D ADMISSION DE 2006}}



{\large \textbf{Concepteur : EML}}



\rule{2.39cm}{0.05cm}



{\Large \textbf{OPTION ÉCONOMIQUE}}



{\Large \textbf{MATHÉMATIQUES }}



{\Large Lundi 9 mai, de 14h à 18h}



\rule{2.39cm}{0.05cm}
\end{center}

\textit{La présentation, la lisibilité, l'orthographe, la qualité
de la rédaction, la clarté et la précision des raisonnements
entreront pour une part importante dans l'appréciation des copies.}

\textit{Les candidats sont invités à \textbf{encadrer} dans la mesure
du possible les résultats de leurs calculs.}

\textit{Ils ne doivent faire usage d'aucun document. L'utilisation de
toute
calculatrice et de tout matériel électronique est interdite. Seule
l'utilisation d'une règle graduée est autorisée.}

\textit{Si au cours de l'épreuve, un candidat repère ce qui lui semble
être une erreur d'énoncé, il la signalera sur sa copie et
poursuivra sa composition en expliquant les raisons des initiatives
qu'il sera
amené à prendre.}

\vspace*{3cm}

{\LARGE Exercice 1}


On considère les trois matrices de $\M{2} $ suivantes :

\[
A = \left( 
\begin{array}{cc}
0 & 1 \\
0 & 1
\end{array}
\right),\;D = \left( 
\begin{array}{cc}
0 & 0 \\
0 & 1
\end{array}
\right),\;U = \left( 
\begin{array}{cc}
1 & 0 \\
0 & 0
\end{array}
\right)
\]

\begin{noliste}{1.}
 \setlength{\itemsep}{4mm}
\item 
\begin{noliste}{a)}
 \setlength{\itemsep}{2mm}
\item Quelles sont les valeurs propres de $A$ ?

\item Déterminer une matrice inversible $P$ telle que $A =
P\,D\,P^{-1}$
\end{noliste}

\hspace{-1cm}On note $E$ l'ensemble des matrices carrées $M$ d'ordre
$2$
telles que : $A\,M = M\,D$

\item 
\begin{noliste}{a)}
 \setlength{\itemsep}{2mm}
\item Vérifier que $E$ est un sous espace vectoriel de $\M{2} $

\item Soit $M = \left( 
\begin{array}{cc}
x & y \\
z & t
\end{array}
\right) $ une matrice de $\M{2} $

Montrer que $M$ appartient à $E$ si et seulement si : $z = 0$ et $y =
t$

\item Établir que $\left( U,A\right) $ est une base de $E.$

\item Calculer le produit $U\,A.$ Est-ce que $U\,A$ est élément de $E $
?
\end{noliste}

\item On note $f :\M{2} \rightarrow 
\M{2} $ l'application définie, pour
tout $M\in \M{2},$ par :\\
$f\left( M\right) = A\,M-M\,D.$

\begin{noliste}{a)}
 \setlength{\itemsep}{2mm}
\item Vérifier que $f$ est linéaire.

\item Déterminer le noyau de $f$ et donner sa dimension.

\item Quelle est la dimension de l'image de $f$ ?

\item déterminer les matrice $M$ de $\M{2} $ telles que $f\left(
M\right) = M.$

En déduire que $1$ est valeur propre de $f.$

Montrer que $-1$ est aussi valeur propre de $f.$

\item Est-ce que $f$ est diagonalisable ?

\item Montrer que $f\circ f\circ f = f$
\end{noliste}
\end{noliste}

\begin{center}
{\LARGE Exercice 2}
\end{center}

On note $F :\R^{2}\rightarrow \R$ l'application définie
pour tout $\left( x,y\right) \in \R^{2}$ par : 
\[
F\left( x,y\right) = \left( x-1\right) \left( y-2\right) \left( x +
y-6\right)
\]

\begin{noliste}{1.}
 \setlength{\itemsep}{4mm}
\item 
\begin{noliste}{a)}
 \setlength{\itemsep}{2mm}
\item Montrer que $\left( 4,2\right) $ et $\left( 2,3\right) $ sont des
points critiques de $F.$

\item Est-ce que $F$ présente un extremum local au point $\left(
4,2\right) $ ?

\item Est-ce que $F$ présente un extremum local au point $\left(
2,3\right) $ ?
\end{noliste}

\item On note $\varphi :\R\rightarrow \R$ l'application définie, pour
tout $x\in \R$ par : 
\[
\varphi \left( x\right) = x\left( x-2\right) \left( 2x-5\right) 
\]

\begin{noliste}{a)}
 \setlength{\itemsep}{2mm}
\item Montrer : $\forall x\in \left[ 4; + \infty \right[,\quad \left(
x-2\right) \left( 2x-5\right) \geq 4$

\item En déduire : $\forall x\in \left[ 4; + \infty \right[,\quad
\varphi
\left( x\right) \geq 4x$ et $\varphi \left( x\right) \in \left[ 4, +
\infty.\right[ $
\end{noliste}

\item On considère la suite réelle $\left( u_{n}\right)_{n\in 
\N}$ définie par $u_{0} = 4$ et : 
\[
\forall n\in \N,\,u_{n + 1} = F\left( 1 + u_{n},u_{n}\right)
\]

\begin{noliste}{a)}
 \setlength{\itemsep}{2mm}
\item Exprimer $u_{n + 1}$ en fonction de $u_{n}$ à l'aide de la
fonction $\varphi.$

\item Montrer : $\forall n\in \N :\mathbb{\quad }u_{n}\geq 4^{n + 1}$

Quelle est la nature de la série de terme général $\dfrac{1}{u_{n}}$ ?

\item Écrire un programme en -\Scilab{} qui calcule et affiche le plus
petit entier naturel $n$ tel que $u_{n}\geq 10^{10}$
\end{noliste}

\item On note $g :\left[ 4, + \infty \right[ \ \rightarrow \R$
l'application définie, pour tout $x\in \left[ 4; + \infty \right[,$ par
: 
\[
g\left( x\right) = \frac{10}{\varphi \left( x\right) }
\]

\begin{noliste}{a)}
 \setlength{\itemsep}{2mm}
\item Montrer que l'intégrale $\dint{4}{+ \infty }g\left( x\right\dx$
converge.

\item Trouver trois réels $a,\;b,\;c$ tels que 
\[
\forall x\in \left[ 4; + \infty \right[,\quad g\left( x\right) =
\frac{a}{x} + \frac{b}{x-2} + \frac{c}{2x-5}
\]

\item Calculer $\dint{4}{+ \infty }g\left( x\right) \,dx$
\end{noliste}
\end{noliste}

\begin{center}
{\LARGE Exercice 3}
\end{center}

{\Large Partie A}

\begin{noliste}{1.}
 \setlength{\itemsep}{4mm}
\item Soit $U$ une variable aléatoire à densité suivant une loi
normale d'espérance nulle et de variance $\frac{1}{2}$

\begin{noliste}{a)}
 \setlength{\itemsep}{2mm}
\item Rappeler une densité de $U$

\item En utilisant la définition de la variance de $U,$ montrer que
l'intégrale $\dint{0}{+ \infty }x^{2}e^{-x^{2}}dx$ est
convergente et que $\dint{0}{+ \infty }x^{2}e^{-x^{2}}dx =
\frac{\sqrt{\pi }}{4}$
\end{noliste}

\hspace{-1cm}Soit $F$ la fonction définie sur $\R$
par : $\left\{ 
\begin{array}{c}
\forall x\leq 0,\quad F\left( x\right) = 0 \\
\forall x>0,\quad F\left( x\right) = 1-e^{-x^{2}}
\end{array}
\right. $

\item Montrer que la fonction $F$ définit une fonction de répartition
de variable aléatoire dont on déterminera une densité $f.$

\item Soit $X$ une variable aléatoire admettant $f$ pour densité.

\begin{noliste}{a)}
 \setlength{\itemsep}{2mm}
\item Montrer que $X$ admet une espérance $\E\left( X\right) $ et que
$\E\left( X\right) = \dfrac{\sqrt{\pi }}{2}.$

\item Déterminer, pour tout réel $y,$ la probabilité $\Prob\left(\Ev{
X^{2}\leq y}\right).$ \emph{On distinguera les cas }$y\leq 0$ \emph{et
}$y>0.$

\item Montrer que la variable aléatoire $X^{2}$ suit une loi
exponentielle dont on précisera le paramètre.

En déduire que $X$ admet une variance $\V\left( X\right) $ et calculer
$\V\left( X\right) $
\end{noliste}
\end{noliste}

{\Large Partie B}

\begin{noliste}{1.}
 \setlength{\itemsep}{4mm}
\item Soit $Z$ une variable aléatoire suivant une loi géométrique de
paramètre $p.$

Ainsi, pour tout $k\in \N^{*},\quad \mathrm{P}\left( Z = k\right)
 = p\left( 1-p\right) ^{k-1}$

Rappeler la valeur de l'espérance $\E\left( Z\right) $ et celle de la
variance $\V\left( Z\right) $ de la variable aléatoire $Z.$

\item Soient un entier $n$ supérieur ou égal à $2,$ et $n$
variables aléatoires indépendantes $Z_{1}$, $Z_{2}$, $\dots $, $Z_{n}
$, suivant toutes le loi géométrique de paramètre $p.$

on considère la variable aléatoire $M_{n} = \dfrac{1}{n}\left(
Z_{1} + Z_{2} + \dots + Z_{n}\right).$

\begin{noliste}{a)}
 \setlength{\itemsep}{2mm}
\item Déterminer l'espérance $m$ et l'écart-type $\sigma_{n}$
de $M_{n}$

\item Montrer que $\dlim{n\rightarrow + \infty }\Prob\left(\Ev{ 0\leq
M_{n}-m\leq \sigma
_{n}}\right) $ existe et exprimer sa valeur à l'aide de
$\dint{0}{1}e^{-\tfrac{x^{2}}{2}}dx$
\end{noliste}
\end{noliste}

\end{document}

