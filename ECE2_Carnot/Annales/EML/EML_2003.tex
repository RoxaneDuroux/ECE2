\documentclass[11pt]{article}%
\usepackage{geometry}%
\geometry{a4paper,
 lmargin = 2cm,rmargin = 2cm,tmargin = 2.5cm,bmargin = 2.5cm}

\input{../../macros.tex}

\pagestyle{fancy} %
\lhead{ECE2 \hfill Mathématiques\\
} %
\chead{\hrule} %
\rhead{} %
\lfoot{} %
\cfoot{} %
\rfoot{\thepage} %

\renewcommand{\headrulewidth}{0pt}% : Trace un trait de séparation
 % de largeur 0,4 point. Mettre 0pt
 % pour supprimer le trait.

\renewcommand{\footrulewidth}{0.4pt}% : Trace un trait de séparation
 % de largeur 0,4 point. Mettre 0pt
 % pour supprimer le trait.

\setlength{\headheight}{14pt}

\title{\bf \vspace{-2cm} EML 2003} %
\author{} %
\date{} %
\begin{document}

\maketitle %
\vspace{-1.4cm}\hrule %
\thispagestyle{fancy}

\vspace*{.2cm}


% DEBUT DU DOC À MODIFIER : tout virer jusqu'au début de l'exo

%Définition et changement de valeurs de
compteurs%newcounter{cpt1}{section} compteur cpt1 remis à 0 à chaque
aumentation par stepcounter du compteur section%setcounter{cpt1}{3} on
met le compteur à 3%addtocounter{cpt1}{5} on ajoute 5 au compteur%
stepcounter{cpt1} on ajoute 1% ifthenelse{test}{alors}{sinon} (page
206) pour subordonner à une condition % whiledo{test}{commande} pour
faire une boucle (page 206 aussi) % value{cpt1} pour noter dans le
document la valeur de cpt1 
%Définition définitive d'opérateurs
mathématiques\newcommand{\ch}{\operatorname{ch}} 
\newcommand{\sh}{\operatorname{sh}}
\renewcommand{\tanh}{\operatorname{th}}
\renewcommand{\sinh}{\operatorname{sh}}
\renewcommand{\cosh}{\operatorname{ch}}
\newcommand{\argsh}{\operatorname{argsh}}
\newcommand{\argch}{\operatorname{argch}}
\newcommand{\argth}{\operatorname{argth}}
\newcommand{\ker}{\operatorname{Ker}}
\renewcommand{\im}{\operatorname{Im}}
\newcommand{\rg}{\operatorname{rg}}
\newcommand{\Id}{\operatorname{Id}}
\newcommand{\id}{\operatorname{id}}
\renewcommand{\leq}{\leq}
\renewcommand{\geq}{\geq }

%Définition de nouvelles couleurs : rgb(trois paramètres red green blue
entre 0 et 1); cmyk (quatre cyan magenta yellow black) entre 0 et 1;
gray (entre 0 et 1) et black, white, red, green, blue, cyan, magenta,
yellow% definecolor{0gris}{gray}{0.8} 
% Nouvelle commande pour encadrer le titre car shabox ne veut que d'une
seule ligne; ATTENTION A LA TAILLE; petite différence avec shadowbox ou
doublebox, voire fcolorbox ou colorbox (au lieu de shabox; laisser le
parbox tranquille sauf pour la taille de la boîte
\newcommand{\Tbox}[1]{\begin{center} \shabox{\parbox{0.6
\linewidth}{#1}} \end{center}} %[1] pour 1 paramètre ; #1 pour ce que
fait le 1er paramètre; entre accolades ce que fait la commande
%Mise en page en mode fancy : en-têtes et pieds de pages puis
définition des en-têtes et pieds de pages\pagestyle{fancy}
\lhead{ECE 2 - Mathématiques \\
Quentin Dunstetter - ENC-Bessières 2011$\backslash$2012}
\chead{}
\rhead{EML 2003}
\rfoot[ \ \thepage]{\thepage}
\cfoot{}
\lfoot{}
\thispagestyle{fancy} %Mise en page de la 1ère page en mode fancy
%Trait en bas et en haut de la page (entre en-tête et texte et texte et
pied de page)\renewcommand{\footrulewidth}{0.4pt}
\renewcommand{\headrulewidth}{0.4pt}


%DEBUT DU DOCUMENT\vspace*{3cm}

\begin{center}
{\LARG\E\textbf{BANQUE COMMUNE D'ÉPREUVES}}



{\large \textsc{CONCOURS D ADMISSION DE 2003}}



{\large \textbf{Concepteur : EML}}



\rule{2.39cm}{0.05cm}



{\Large \textbf{OPTION ÉCONOMIQUE}}



{\Large \textbf{MATHÉMATIQUES }}



{\Large Lundi 9 mai, de 14h à 18h}



\rule{2.39cm}{0.05cm}
\end{center}

\textit{La présentation, la lisibilité, l'orthographe, la qualité
de la rédaction, la clarté et la précision des raisonnements
entreront pour une part importante dans l'appréciation des copies.}

\textit{Les candidats sont invités à \textbf{encadrer} dans la mesure
du possible les résultats de leurs calculs.}

\textit{Ils ne doivent faire usage d'aucun document. L'utilisation de
toute
calculatrice et de tout matériel électronique est interdite. Seule
l'utilisation d'une règle graduée est autorisée.}

\textit{Si au cours de l'épreuve, un candidat repère ce qui lui semble
être une erreur d'énoncé, il la signalera sur sa copie et
poursuivra sa composition en expliquant les raisons des initiatives
qu'il sera
amené à prendre.}

\vspace*{3cm}


\begin{center}
{\large \textbf{EXERCICE~1} }
\end{center}

On note $\M{3} $ l'ensemble des matrices
carrées réelles d'ordre trois et on considère les matrices
suivantes de $\M{3} $~ : {\large 
\[
I = \left( 
\begin{array}{ccc}
1 & 0 & 0 \\
0 & 1 & 0 \\
0 & 0 & 1
\end{array}
\right) \qquad A = \left( 
\begin{array}{ccc}
1 & 1 & 1 \\
1 & 0 & 0 \\
1 & 0 & 0
\end{array}
\right) 
\]
}

\textbf{I.~ Première partie}

\begin{noliste}{1.}
 \setlength{\itemsep}{4mm}
\item Calculer $A^{2}$ et $A^{3}$, puis vérifier~ : $A^{3} = A^{2} +
2A$.

\item Montrer que la famille $\left( {A,A^{2}}\right) $ est libre dans
$\M{3} $.

\item Montrer que, pour tout entier $n$ supérieur ou égal à $1$, il
existe un couple unique $\left( {a_{n},b_{n}}\right) $ de nombres
réels tel que~ : $A^{n} = a_{n}A + b_{n}A^{2}$, et exprimer $a_{n + 1}$
et $b_{n + 1}$ en fonction de $a_{n}$ et $b_{n}$.

\item Écrire un programme en -\Scilab{} qui calcule et affiche $a_{n}$
et $b_{n}$ pour un entier $n$ donné supérieur ou égal à $1$.

\item 
\begin{noliste}{a)}
 \setlength{\itemsep}{2mm}
\item Montrer, pour tout entier $n$ supérieur ou égal à 1~ : 
\[
a_{n + 2} = a_{n + 1} + 2a_{n}
\]

\item En déduire $a_{n}$ et $b_{n}$ en fonction de $n$, pour tout
entier $n$ supérieur ou égal à $1$.

\item Donner l'expression de $A^{n}$ en fonction de $A$, $A^{2}$ et
$n$,
pour tout entier $n$ supérieur ou égal à $1$.
\end{noliste}
\end{noliste}

{\large \ ~}

\textbf{\noindent II.~ Seconde partie}

On note $f$ l'endomorphisme de $\Bbb{R}{3}$ dont la matrice
relativement
à la base canonique $\left( {e_{1},e_{2},e_{3}}\right) $ de
$\Bbb{R}{3}$ est $A$.

\begin{noliste}{1.}
 \setlength{\itemsep}{4mm}
\item Déterminer une base de $\operatorname{Im}\left( {f}\right) $ et
donner la
dimension de $\operatorname{Im}\left( {f}\right) $.

\item 
\begin{noliste}{a)}
 \setlength{\itemsep}{2mm}
\item Est-ce que $f$ est diagonalisable~ ?

\item Est-ce que $f$ est bijectif~ ?
\end{noliste}

\item Déterminer les valeurs propres de $f$, et donner, pour chaque
sous-espace propre de $f$, une base de ce sous-espace propre.

\item Déterminer une matrice diagonale $D$ dont les termes diagonaux
sont dans l'ordre réel croissant, et une matrice inversible $P$ dont la
troisième ligne est formée de termes tous égaux à $1$,
telles que $A = PDP^{-1}$, et calculer $P^{-1}$.

\item Déterminer l'ensemble des matrices $M$ de $\M{3} $ telles que~ : 
\[
AM + MA = 0
\]
\end{noliste}

\begin{center}
\vspace{1.5cm}{\LARG\E\ \noindent EXERCICE~2}
\end{center}

On note $\text{e} = \exp \left( {1}\right) $, et $\Bbb{R}_{+}{*} =
\left] {0; + \infty }\right[ $.

On considère, pour tout nombre réel $a$ non nul, l'application $f_{a}
:\Bbb{R}_{+}{*}\times \Bbb{R}_{+}{*}\longrightarrow \Bbb{R}$
définie par~ : 
\[
\forall \left( {x,y}\right) \in \Bbb{R}_{+}{*}\times
\Bbb{R}_{+}{*},\quad f_{a}\left( {x,y}\right) =
\dfrac{x\mathrm{e}{-x}}{y}-\dfrac{y}{a} 
\]

Les deux parties de l'exercice sont indépendantes entre elles.

~

\textbf{\ \noindent I.~ Première partie}

Dans cette première partie, on prend $a = -\mathrm{e}$, et on note $g$
à la place de $f_{-\text{e}}$. Ainsi, l'application $g
:\Bbb{R}_{+}{*}\times \Bbb{R}_{+}{*}\longrightarrow \Bbb{R}$ est
définie
par~ : 
\[
\forall \left( {x,y}\right) \in \Bbb{R}_{+}{*}\times
\Bbb{R}_{+}{*},\quad g\left( {x,y}\right) = \dfrac{x\mathrm{e}{-x}}{y}
+ \dfrac{y}{e}
\]

\begin{noliste}{1.}
 \setlength{\itemsep}{4mm}
\item Montrer que $g$ est de classe $C^{2}$ sur $\Bbb{R}_{+}{*}\times
\Bbb{R}_{+}{*}$.

\item Calculer les dérivées partielles d'ordre $1$ de $g$ en tout
point $\left( {x,y}\right) $ de $\Bbb{R}_{+}{*}\times \Bbb{R}_{+}{*}$.

\item Montrer qu'il existe un couple unique $\left( {x,y}\right) $ de
$\Bbb{R}_{+}{*}\times \Bbb{R}_{+}{*}$ en lequel les deux
dérivées partielles d'ordre $1$ de $g$ s'annulent, et calculer ce
couple.

\item Est-ce que $g$ admet un extremum~ ?
\end{noliste}

~

\textbf{\ \noindent II.~ Seconde partie }

Dans cette seconde partie, on prend $a = 1$.

On considère, pout tout entier $n$ tel que $n\geq 1$, l'application 
$h_{n} :\left] {0; + \infty }\right[ \ \longrightarrow \Bbb{R}$ définie
par~ : 
\[
\forall x\in \left] {0; + \infty }\right[,\quad h_{n}\left( {x}\right)
 = f_{1}\left( {x,x^{n}}\right) = \dfrac{x\text{e}{-x}}{x^{n}}-x^{n} 
\]
et l'application $\varphi_{n} :\left] {0; + \infty }\right[ \
\longrightarrow \Bbb{R}$ définie par~ : 
\[
\forall x\in \left] {0; + \infty }\right[,\quad \varphi_{n}\left(
{x}\right)
 = \text{e}{-x}-x^{2n-1} 
\]

\begin{noliste}{1.}
 \setlength{\itemsep}{4mm}
\item 
\begin{noliste}{a)}
 \setlength{\itemsep}{2mm}
\item Montrer que, pour tout entier $n$ supérieur ou égal à $1$~ : 
\[
\forall x\in \left] {0; + \infty }\right[,\quad h_{n}\left( {x}\right)
 = 0\Longleftrightarrow \varphi_{n}\left( {x}\right) = 0 
\]

\item En déduire que, pour tout entier $n$ supérieur ou égal
à $1$, l'équation $h_{n}\left( {x}\right) = 0$, d'inconnue $x\in
\left] {0; + \infty }\right[ $, admet une solution et une seule, notée
$u_{n}$, et que~ : 
\[
0<u_{n}<1 
\]

\item Montrer, pour tout entier $n$ supérieur ou égal à $1$~ : $\ln
\left( {u_{n}}\right) = -\dfrac{u_{n}}{2n-1}$.

\item En déduire~ : $u_{n}\rightarrow 1$ quand $n\rightarrow + \infty $
\end{noliste}
\end{noliste}

\begin{center}
\vspace{1.5cm}{\LARG\E\ \noindent EXERCICE~3}
\end{center}

Montrer que l'intégrale $ \dint{2}{+ \infty }\dfrac{1}{x\sqrt{x}}dx$
est convergente et calculer sa valeur.

\begin{noliste}{1.}
 \setlength{\itemsep}{4mm}
\item ~

Soit $f :\Bbb{R}\longrightarrow \Bbb{R}$ la fonction définie par~ :
$\left\{ 
\begin{array}{cc}
f\left( {x}\right) = 0 & \text{si }x<2 \\
f\left( {x}\right) = \dfrac{1}{x\sqrt{2x}} & \text{si }x\geq 2
\end{array}
\right. $

~

\item Montrer que $f$ définit une densité de probabilité.

\item Soit $X$ une variable aléatoire réelle admettant $f$ pour
densité.

\begin{noliste}{a)}
 \setlength{\itemsep}{2mm}
\item Déterminer la fonction de répartition de $X$.

\item La variable aléatoire $X$ admet-elle une espérance~ ?
\end{noliste}

~

On considère trois variables aléatoires indépendantes $T_{1}$, $T_{2}$
et $T_{3}$, chacune de même loi que $X$.

~

\item On condidère la variable aléatoire $U = \inf \left(
{T_{1},T_{2},T_{3}}\right) $ définie par~ : 
\[
\forall t\in \Bbb{R},\quad \left( {U>t}\right) = \left(
{T_{1}>t}\right)
\cap \left( {T_{2}>t}\right) \cap \left( {T_{3}>t}\right) 
\]

\begin{noliste}{a)}
 \setlength{\itemsep}{2mm}
\item Déterminer la fonction de répartition $G$ de $U$.

\item Montrer que $U$ admet une densité et déterminer une
densité $g$ de $U$.

\item Montrer que $U$ admet une espérance et calculer $\E\left(
{U}\right) $.
\end{noliste}

\item On condidère la variable aléatoire $V = \sup \left(
{T_{1},T_{2},T_{3}}\right) $ définie par~ : 
\[
\forall t\in \Bbb{R},\quad \left( {\V\leq t}\right) = \left( {T_{1}\leq
t}\right) \cap \left( {T_{2}\leq t}\right) \cap \left( {T_{3}\leq
t}\right) 
\]

\begin{noliste}{a)}
 \setlength{\itemsep}{2mm}
\item Déterminer la fonction de répartition $H$ de $V$.

\item Montrer que $V$ admet une densité et déterminer une
densité $h$ de $V$.

\item La variable aléatoire $V$ admet-elle une espérance~ ?
\end{noliste}
\end{noliste}

\end{document}

