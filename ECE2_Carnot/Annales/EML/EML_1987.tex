\documentclass[11pt]{article}%
\usepackage{geometry}%
\geometry{a4paper,
 lmargin = 2cm,rmargin = 2cm,tmargin = 2.5cm,bmargin = 2.5cm}

\input{../../macros.tex}

\pagestyle{fancy} %
\lhead{ECE2 \hfill Mathématiques\\
} %
\chead{\hrule} %
\rhead{} %
\lfoot{} %
\cfoot{} %
\rfoot{\thepage} %

\renewcommand{\headrulewidth}{0pt}% : Trace un trait de séparation
 % de largeur 0,4 point. Mettre 0pt
 % pour supprimer le trait.

\renewcommand{\footrulewidth}{0.4pt}% : Trace un trait de séparation
 % de largeur 0,4 point. Mettre 0pt
 % pour supprimer le trait.

\setlength{\headheight}{14pt}

\title{\bf \vspace{-2cm} EML 1987} %
\author{} %
\date{} %
\begin{document}

\maketitle %
\vspace{-1.4cm}\hrule %
\thispagestyle{fancy}

\vspace*{.2cm}


% DEBUT DU DOC À MODIFIER : tout virer jusqu'au début de l'exo

%Définition et changement de valeurs de
compteurs%newcounter{cpt1}{section} compteur cpt1 remis à 0 à chaque
aumentation par stepcounter du compteur section%setcounter{cpt1}{3} on
met le compteur à 3%addtocounter{cpt1}{5} on ajoute 5 au compteur%
stepcounter{cpt1} on ajoute 1% ifthenelse{test}{alors}{sinon} (page
206) pour subordonner à une condition % whiledo{test}{commande} pour
faire une boucle (page 206 aussi) % value{cpt1} pour noter dans le
document la valeur de cpt1 
%Définition définitive d'opérateurs
mathématiques\newcommand{\ch}{\operatorname{ch}} 
\newcommand{\sh}{\operatorname{sh}}
\renewcommand{\tanh}{\operatorname{th}}
\renewcommand{\sinh}{\operatorname{sh}}
\renewcommand{\cosh}{\operatorname{ch}}
\newcommand{\argsh}{\operatorname{argsh}}
\newcommand{\argch}{\operatorname{argch}}
\newcommand{\argth}{\operatorname{argth}}
\newcommand{\ker}{\operatorname{Ker}}
\renewcommand{\im}{\operatorname{Im}}
\newcommand{\rg}{\operatorname{rg}}
\newcommand{\Id}{\operatorname{Id}}
\newcommand{\id}{\operatorname{id}}
\renewcommand{\leq}{\leq}
\renewcommand{\geq}{\geq }

%Définition de nouvelles couleurs : rgb(trois paramètres red green blue
entre 0 et 1); cmyk (quatre cyan magenta yellow black) entre 0 et 1;
gray (entre 0 et 1) et black, white, red, green, blue, cyan, magenta,
yellow% definecolor{0gris}{gray}{0.8} 
% Nouvelle commande pour encadrer le titre car shabox ne veut que d'une
seule ligne; ATTENTION A LA TAILLE; petite différence avec shadowbox ou
doublebox, voire fcolorbox ou colorbox (au lieu de shabox; laisser le
parbox tranquille sauf pour la taille de la boîte
\newcommand{\Tbox}[1]{\begin{center} \shabox{\parbox{0.6
\linewidth}{#1}} \end{center}} %[1] pour 1 paramètre ; #1 pour ce que
fait le 1er paramètre; entre accolades ce que fait la commande
%Mise en page en mode fancy : en-têtes et pieds de pages puis
définition des en-têtes et pieds de pages\pagestyle{fancy}
\lhead{ECE 2 - Mathématiques \\
Quentin Dunstetter - ENC-Bessières 2011$\backslash$2012}
\chead{}
\rhead{EML 1987}
\rfoot[ \ \thepage]{\thepage}
\cfoot{}
\lfoot{}
\thispagestyle{fancy} %Mise en page de la 1ère page en mode fancy
%Trait en bas et en haut de la page (entre en-tête et texte et texte et
pied de page)\renewcommand{\footrulewidth}{0.4pt}
\renewcommand{\headrulewidth}{0.4pt}


%DEBUT DU DOCUMENT\vspace*{3cm}

\begin{center}
{\LARG\E\textbf{BANQUE COMMUNE D'ÉPREUVES}}



{\large \textsc{CONCOURS D ADMISSION DE 1987}}



{\large \textbf{Concepteur : EML}}



\rule{2.39cm}{0.05cm}



{\Large \textbf{OPTION ÉCONOMIQUE}}



{\Large \textbf{MATHÉMATIQUES }}



{\Large Lundi 9 mai, de 14h à 18h}



\rule{2.39cm}{0.05cm}
\end{center}

\textit{La présentation, la lisibilité, l'orthographe, la qualité
de la rédaction, la clarté et la précision des raisonnements
entreront pour une part importante dans l'appréciation des copies.}

\textit{Les candidats sont invités à \textbf{encadrer} dans la mesure
du possible les résultats de leurs calculs.}

\textit{Ils ne doivent faire usage d'aucun document. L'utilisation de
toute
calculatrice et de tout matériel électronique est interdite. Seule
l'utilisation d'une règle graduée est autorisée.}

\textit{Si au cours de l'épreuve, un candidat repère ce qui lui semble
être une erreur d'énoncé, il la signalera sur sa copie et
poursuivra sa composition en expliquant les raisons des initiatives
qu'il sera
amené à prendre.}

\vspace*{3cm}

\section*{EXERCICE 1}

On considère les matrices suivantes : 
\[
A = \left( 
\begin{array}{ccc}
2 & 1 & 1 \\
1 & 2 & 1 \\
1 & 1 & 2
\end{array}
\right),\quad B = \left( 
\begin{array}{ccc}
1 & 1 & 1 \\
1 & 1 & 1 \\
1 & 1 & 1
\end{array}
\right),\quad \text{et\quad }I = \left( 
\begin{array}{ccc}
1 & 0 & 0 \\
0 & 1 & 0 \\
0 & 0 & 1
\end{array}
\right) 
\]

\begin{noliste}{1.}
 \setlength{\itemsep}{4mm}
\item Calculer $B^{2}$ puis $B^{n}$ pour $n\in \N$.

\item Vérifier que $A = I + B$ et en déduire l'expression de $A^{n}$
pour $n\in 
\N$.
\end{noliste}

\section*{EXERCICE 2}

On considère la fonction définie par $f :\left. 
\begin{array}{ccl}
\lbrack 0,1] & \rightarrow & \R \\
x & \mapsto & f(x) = \dfrac{e^{x}}{x + 10}
\end{array}
\right. $

\begin{noliste}{1.}
 \setlength{\itemsep}{4mm}
\item 

\begin{noliste}{a)}
 \setlength{\itemsep}{2mm}
\item Calculer $f^{\prime }$ et $f^{\prime \prime }$. Étudier les
variations
de la fonction $f$. Tracer la courbe représentative $C_{f}$ de $f$ dans
un
repère orthonormé, l'unité étant prise égale à $10\,cm$.

\item Démontrer que, $\forall x\in \left[ 0;1\right] \quad 0,09\leq
f^{\prime }\left( x\right) \leq 0,225$.

\item Démontrer que l'équation $f\left( x\right) = x$ admet une
solution
unique dans $\left[ 0;1\right] $, qui sera notée $l$.\\
Donner une valeur approchée $l_{0}$ de $l$ à $10^{-1}$ près par défaut,
par
lecture du graphique.
\end{noliste}

\item On considère la suite définie par $u_{0} = l_{0}$ et $u_{n + 1} =
f\left(
u_{n}\right) $.

\begin{noliste}{a)}
 \setlength{\itemsep}{2mm}
\item Montrer que $u_{n + 1}-u_{n}\geq 0$, $\forall n\in \N$. En
déduire que la suite $\left( u_{n}\right)_{n\in \N}$ est croissante.

\item Calculer $f\left( 0,2\right) $, et en déduire $l\leq 0,2$.
Calculer $f^{\prime }\left( 0,2\right) $.

\item Établir que, pour $n\in \N$, $0\leq \dfrac{u_{n +
1}-l}{u_{n}-l}\leq 0,11$.

\item En déduire que la suite $\left( u_{n}\right)_{n\in \N}$
converge vers $l$, et déterminer un entier $n_{0}$ tel que
$l-u_{n_{0}}\leq 10^{-3}$. Calculer $u_{n_{0}}.$
\end{noliste}
\end{noliste}

\section*{EXERCICE 3}

Tous les dés considérés sont cubiques, et les apparitions des faces
sont équiprobables.\\
On considère le dé $A$ dont les faces sont numérotées de $1$ à $6$, et
les
sept dés $D_{i}$ ($1\leq i\leq 7$) tels que, pour chaque $i$, le dé
$D_{i}$ possède $i-1$ faces blanches et $7-i$ faces noires (par
exemple, le dé $D_{3}$, contient $2$ faces blanches et $4$ faces
noires).\\
On choisit tout d'abord un numéro $i$ compris entre $1$ et $7$ en
lançant le
dé $A$ et en procédant de la façon suivante :

\begin{noliste}{$\sbullet$}
\item si le résultat du lancer est $2,3,4,5$ ou $6$, on choisit le
numéro
sorti,

\item si le résultat du lancer est $1$, on lance à nouveau le dé $A$,
et si
le nouveau résultat est $1,2$ ou $3$, on choisit le numéro $1$, sinon
on
choisit le numéro $7$.
\end{noliste}

\noindent Après avoir choisi de cette façon le numéro $i$ ($1\leq
i\leq 7$), on joue exclusivement avec le dé $D_{i}$. On lance $D_{i}$
successivement plusieurs fois de suite : les lancers successifs sont
indépendants les uns des autres.\\
L'observateur qui compte les faces noires ignore quel est le dé $D_{i}$
utilisé.

\begin{noliste}{1.}
 \setlength{\itemsep}{4mm}
\item Calculer la probabilité pour qu'il sorte une face noire au
premier
lancer.

\item Sachant qu'il est sorti une face noire aux deux premiers lancers,
calculer la probabilité qu'il sorte une face noire au troisième lancer.

\item Calculer la probabilité qu'il sorte une face noire au
$n^{\grave{e}me}$
lancer, sachant que dans les lancers précédents, il est toujours sorti
une
face noire. Calculer la limite de cette probabilité quand $n$ tend vers
$ + \infty $, et interpréter le résultat.
\end{noliste}

\section*{EXERCICE 4}

Soit $f :\left. 
\begin{array}{ccl}
\R & \rightarrow & \R \\
x & \mapsto & f(x) = \left\{ 
\begin{array}{cl}
12\left( t^{2}-t^{3}\right) & \text{si }t\in \lbrack 0,1] \\
0 & \text{si }t\in \left] -\infty ;0\right[ \ \cup \left] 1; + \infty
\right[ 
\end{array}
\right. 
\end{array}
\right. $ 

\begin{noliste}{1.}
 \setlength{\itemsep}{4mm}
\item 

\begin{noliste}{a)}
 \setlength{\itemsep}{2mm}
\item Montrer que $f$ est continue sur $\R$.

\item Est-ce que $f$ est dérivable sur $\R$ ?

\item Étudier les variations de $f$, et tracer sa courbe représentative
(repère orthonormé, unité $5\,cm$).\\
On précisera les coordonnées du point d'inflexion.

\item Calculer $\dint\limits_{0}{1}f\left( t\right) \,dt$.
\end{noliste}

\item Soit $F :
\begin{array}{ccc}
\R & \rightarrow & \R \\
x & \mapsto & \dint\limits_{0}{x}f\left( t\right) \,dt
\end{array}
$

\begin{noliste}{a)}
 \setlength{\itemsep}{2mm}
\item Calculer $F\left( x\right) $ pour tout $x\in \R$.

\item Étudier les variations de $F$, et tracer sa courbe représentative
sur
le même schéma que $f.$
\end{noliste}

\item Soit $G :
\begin{array}{lll}
\R & \rightarrow & \R \\
x & \mapsto & \left\{ 
\begin{array}{ll}
0 & \text{si }x\in \left] -\infty ;1\right[ \\
1-F\left( \tfrac{1}{x}\right) & \text{si }x\in \left[ 1; + \infty
\right[ 
\end{array}
\right. 
\end{array}
$

\begin{noliste}{a)}
 \setlength{\itemsep}{2mm}
\item Calculer $G\left( x\right) $ pour tout $x\in \R$.

\item Calculer les trois dérivées successives $G^{\prime }$, $G^{\prime
\prime }$, $G^{\prime \prime \prime }$ de $G$, et tracer la courbe
représentative de $G^{\prime }$ (repère orthonormé, unité de $5\;cm$).
\end{noliste}

\item Interpréter, dans le vocabulaire des probabilités, les résultats
des
questions \textbf{1} et \textbf{2}.
\end{noliste}

\label{fin}

\end{document}

