\documentclass[11pt]{article}%
\usepackage{geometry}%
\geometry{a4paper,
 lmargin = 2cm,rmargin = 2cm,tmargin = 2.5cm,bmargin = 2.5cm}

\input{../../macros.tex}

\pagestyle{fancy} %
\lhead{ECE2 \hfill Mathématiques\\
} %
\chead{\hrule} %
\rhead{} %
\lfoot{} %
\cfoot{} %
\rfoot{\thepage} %

\renewcommand{\headrulewidth}{0pt}% : Trace un trait de séparation
 % de largeur 0,4 point. Mettre 0pt
 % pour supprimer le trait.

\renewcommand{\footrulewidth}{0.4pt}% : Trace un trait de séparation
 % de largeur 0,4 point. Mettre 0pt
 % pour supprimer le trait.

\setlength{\headheight}{14pt}

\title{\bf \vspace{-2cm} EML 2012} %
\author{} %
\date{} %
\begin{document}

\maketitle %
\vspace{-1.4cm}\hrule %
\thispagestyle{fancy}

\vspace*{.2cm}


% DEBUT DU DOC À MODIFIER : tout virer jusqu'au début de l'exo

%Définition et changement de valeurs de
compteurs%newcounter{cpt1}{section} compteur cpt1 remis à 0 à chaque
aumentation par stepcounter du compteur section%setcounter{cpt1}{3} on
met le compteur à 3%addtocounter{cpt1}{5} on ajoute 5 au compteur%
stepcounter{cpt1} on ajoute 1% ifthenelse{test}{alors}{sinon} (page
206) pour subordonner à une condition % whiledo{test}{commande} pour
faire une boucle (page 206 aussi) % value{cpt1} pour noter dans le
document la valeur de cpt1 
%Définition définitive d'opérateurs
mathématiques\newcommand{\ch}{\operatorname{ch}} 
\newcommand{\sh}{\operatorname{sh}}
\renewcommand{\tanh}{\operatorname{th}}
\renewcommand{\sinh}{\operatorname{sh}}
\renewcommand{\cosh}{\operatorname{ch}}
\newcommand{\argsh}{\operatorname{argsh}}
\newcommand{\argch}{\operatorname{argch}}
\newcommand{\argth}{\operatorname{argth}}
\newcommand{\ker}{\operatorname{Ker}}
\renewcommand{\im}{\operatorname{Im}}
\newcommand{\rg}{\operatorname{rg}}
\newcommand{\Id}{\operatorname{Id}}
\newcommand{\id}{\operatorname{id}}
\renewcommand{\leq}{\leq}
\renewcommand{\geq}{\geq }

%Définition de nouvelles couleurs : rgb(trois paramètres red green blue
entre 0 et 1); cmyk (quatre cyan magenta yellow black) entre 0 et 1;
gray (entre 0 et 1) et black, white, red, green, blue, cyan, magenta,
yellow% definecolor{0gris}{gray}{0.8} 
% Nouvelle commande pour encadrer le titre car shabox ne veut que d'une
seule ligne; ATTENTION A LA TAILLE; petite différence avec shadowbox ou
doublebox, voire fcolorbox ou colorbox (au lieu de shabox; laisser le
parbox tranquille sauf pour la taille de la boîte
\newcommand{\Tbox}[1]{\begin{center} \shabox{\parbox{0.6
\linewidth}{#1}} \end{center}} %[1] pour 1 paramètre ; #1 pour ce que
fait le 1er paramètre; entre accolades ce que fait la commande
%Mise en page en mode fancy : en-têtes et pieds de pages puis
définition des en-têtes et pieds de pages\pagestyle{fancy}
\lhead{ECE 2 - Mathématiques \\
Quentin Dunstetter - ENC-Bessières 2011$\backslash$2012}
\chead{}
\rhead{EML 2012}
\rfoot[ \ \thepage]{\thepage}
\cfoot{}
\lfoot{}
\thispagestyle{fancy} %Mise en page de la 1ère page en mode fancy
%Trait en bas et en haut de la page (entre en-tête et texte et texte et
pied de page)\renewcommand{\footrulewidth}{0.4pt}
\renewcommand{\headrulewidth}{0.4pt}


%DEBUT DU DOCUMENT\vspace*{3cm}

\begin{center}
{\LARG\E\textbf{BANQUE COMMUNE D'ÉPREUVES}}



{\large \textsc{CONCOURS D ADMISSION DE 2012}}



{\large \textbf{Concepteur : EML}}



\rule{2.39cm}{0.05cm}



{\Large \textbf{OPTION ÉCONOMIQUE}}



{\Large \textbf{MATHÉMATIQUES }}



{\Large Lundi 9 mai, de 14h à 18h}



\rule{2.39cm}{0.05cm}
\end{center}

\textit{La présentation, la lisibilité, l'orthographe, la qualité
de la rédaction, la clarté et la précision des raisonnements
entreront pour une part importante dans l'appréciation des copies.}

\textit{Les candidats sont invités à \textbf{encadrer} dans la mesure
du possible les résultats de leurs calculs.}

\textit{Ils ne doivent faire usage d'aucun document. L'utilisation de
toute
calculatrice et de tout matériel électronique est interdite. Seule
l'utilisation d'une règle graduée est autorisée.}

\textit{Si au cours de l'épreuve, un candidat repère ce qui lui semble
être une erreur d'énoncé, il la signalera sur sa copie et
poursuivra sa composition en expliquant les raisons des initiatives
qu'il sera
amené à prendre.}

\vspace*{3cm}

\section*{EXERCICE 1}

On considère les matrices carrées d'ordre 2 suivantes :
\[
I = 
\begin{smatrix}
1 & 0\\
0 & 1
\end{smatrix},\qquad A = 
\begin{smatrix}
1 & 0\\
0 & -1
\end{smatrix}
\qquad B = 
\begin{smatrix}
2 & 2\\
1 & 3
\end{smatrix}
\]


\subsection*{Partie I : Étude de la matrice $B$}

\begin{noliste}{1.}
 \setlength{\itemsep}{4mm}
\item Déterminer les valeurs propres et les sous-espaces propres de
$B$.
Est-ce que $B$ est diagonalisable ?

\item Déterminer une matrice diagonale $D$ de $\M{2} $, dont les
coefficients diagonaux sont dans l'ordre
croissant, et une matrice inversible $P$ de $\M{2} $ dont les
coefficients de la première ligne sont tous
égaux à $1$, telles que $B = PDP^{-1}$

\item Vérifier que $D^{2} = 5D-4I$ et exprimer $B^{2}$ comme
combinaison
linéaire de $B$ et $I$.

\item Montrer que $B$ est inversible et exprimer $B^{-1}$ comme
combinaison
linéaire de $B$ et $I$.
\end{noliste}

\subsection*{Partie II : Étude d'un endomorphisme de $\M{2} $}

On considère l'application\qquad$h :\M{2} \rightarrow\M{2},$
\quad$M\longmapsto h\left( M\right) = AMB$.

\begin{noliste}{1.}
 \setlength{\itemsep}{4mm}
\item Vérifier que $h$ est un endomorphisme de $\M{2}.$

\item Montrer que $h$ est bijectif et exprimer $h^{-1}$ sous une forme
analogue à celle donnée pour $h$.

\item On se propose dans cette question de déterminer les valeurs
propres
de $h$.

\begin{noliste}{a)}
 \setlength{\itemsep}{2mm}
\item Soient $\lambda\in\R,\mathbb{\ }\M{2} $.\\
 On note $N = MP$, ou $P$ est la matrice
définie dans la question \textbf{I2.}

Montrer : $h\left( M\right) = \lambda M\Longleftrightarrow AND =
\lambda N,$
où $D$ est la matrice définie dans la question \textbf{I2.}

\item Déterminer les réels $\lambda$ pour lesquels il existe une
matrice $N$ de $\M{2} $ non nulle telle
que $A~N~D = \lambda N.$ \`{A} cet effet, on pourra noter $N = 
\begin{smatrix}
x & y\\
z & t
\end{smatrix}
$

\item En déduire les valeurs propres de $h$. Montrer que $h$ est
diagonalisable et, donner une matrice diagonale représentant $h$.

\item On note $e$ l'endomorphisme identité de $\M{2} $ et on note $0$
l'endomorphisme nul de $\M{2} $

Montrer : $\left( h-e\right) \circ\left( h + e\right) \circ\left(
h-4e\right) \circ\left( h + 4e\right) = 0$.
\end{noliste}
\end{noliste}

\section*{EXERCICE 2}

\subsection*{Partie I : Étude d'une fonction d'une variable réelle}

On considère l'application $f :\left[ 0; + \infty\right[ \rightarrow
\R$définie, pour tout $t\in\left[ 0; + \infty\right[ $ par :
\[
f\left( t\right) = \left\{
\begin{array}
[c]{cc}t\ln\left( t\right) & \text{si }t\neq0\\
0 & \text{si }t = 0
\end{array}
\right.
\]


\begin{noliste}{1.}
 \setlength{\itemsep}{4mm}
\item Montrer que $f$ est continue sur $\left[ 0; + \infty\right[ $.

\item Montrer que $f$ est de classe $C^{1}$ sur $\left] 0; +
\infty\right[ $
et calculer $f^{\prime}\left( t\right) $ pour tout $t\in\left]
0; + \infty\right[ $.

\item Dresser le tableau des variations de $f$. On précisera la limite
de
$f$ en $ + \infty$.

\item Montrer que $f$ est convexe sur $\left] 0; + \infty\right[ $.

\item On note $\Gamma$ la courbe représentative de $f$ dans un repère
orthonormal $\left( O,\vec{i},\vec{j}\right) $

\begin{noliste}{a)}
 \setlength{\itemsep}{2mm}
\item Montrer que $\Gamma$ admet une demi-tangente en $O$ et préciser
celle-ci.

\item Déterminer les points d'intersection de $\Gamma$ et, de l'axe des
abscisses.

\item Préciser la nature de la branche infinie de $\Gamma$.

\item Tracer l'allure de $\Gamma$. On admet : $0,36\leq e^{-1}<0,37.$
\end{noliste}
\end{noliste}

\subsection*{Partie Il : Étude d'une fonction de deux variables
réelles}

On considère l'application $F :\left] 0; + \infty\right[ \rightarrow
\R$, de classe $C^{2}$, définie, pour tout $\left( x,y\right)
\in\left] 0; + \infty\right[ ^{2}$ par :
\[
F\left( x,y\right) = \frac{\ln\left( x\right) }{y} + \frac{\ln\left(
y\right) }{x}
\]


\begin{noliste}{1.}
 \setlength{\itemsep}{4mm}
\item Calculer les dérivées partielles premières de $F$ en tout
$\left( x,y\right) $ de $\left] 0; + \infty\right[ ^{2}.$

\item Montrer que $\left( e,e\right) $ est un point critique de $F$.

\item Calculer les dérivées partielles secondes de $F$ en tout
$\left( x,y\right) $ de $\left] 0; + \infty\right[ ^{2}.$ Est-ce que
$F$
admet, un extremum local en $\left( e,e\right) $ ?
\end{noliste}

\section*{EXERCICE 3}

Soit$a\in\R_{+}{\ast}.$

\begin{noliste}{1.}
 \setlength{\itemsep}{4mm}
\item Montrer que, pour tout entier $n$ tel que $n\geq0$, l'intégrale
$ I_{n} = \dint{0}{+ \infty}x^{n}e^{-\tfrac{x^{2}}{2a^{2}}}dx$
\ est convergente.

\item
\begin{noliste}{a)}
 \setlength{\itemsep}{2mm}
\item Rappeler une densité d'une variable aléatoire. qui suit la loi
normale d'espérance nulle et de variance $a^{2}$.\\
 En déduire :\qquad$ I_{0} = a\sqrt{\frac{\pi}{2}}.$

\item Calculer la dérivée de l'application $\varphi :\R\rightarrow\R$
définie, pour tout $x\in\R,$ par :
$\varphi\left( x\right) = e^{-\tfrac{x^{2}}{2a^{2}}}$

En déduire : $I_{1} = a^{2}$.
\end{noliste}

\item
\begin{noliste}{a)}
 \setlength{\itemsep}{2mm}
\item Montrer, pour tout entier $n$. tel que $n\geq2$ et pour tout
$t\in\left[ 0; + \infty\right[ :$
\[
\dint{0}{t}x^{n}e^{-\frac{x^{2}}{2a^{2}}}dx =
-a^{2}t^{n-1}e^{-\frac{t^{2}}{2a^{2}}} + \left( n-1\right)
a^{2}\dint{0}{t}x^{n-2}e^{-\frac{x^{2}}{2a^{2}}}dx
\]


\item En déduire, pour tout entier $n$ tel que $n\geq2 :I_{n} = \left(
n-1\right) a^{2}I_{n-2}.$

\item Calculer $I_{2}$ et $I_{3}.$
\end{noliste}

\hspace{-1cm}On considère l'application $g_{a} :\R\rightarrow
\R$ définie, pour tout $x\in\R$, par :
\[
g_{a}\left( x\right) = \left\{
\begin{array}
[c]{cc}0 & \text{si }x\leq0\\
\dfrac{x}{a^{2}}e^{-\frac{x^{2}}{2a^{2}}} & \text{si }x>0
\end{array}
\right.
\]


\item Montrer que $g_{a}$ est une densité.

\hspace{-1cm}On considère une variable aléatoire $X$ admettant $g_{a}$
comme densité.

\item Déterminer la fonction de répartition de la variable
aléatoire $X$.

\item Montrer que la variable aléatoire $X$ admet une espérance
$\E\left( X\right) $ et que $ E\left( X\right) = a\sqrt
{\frac{\pi}{2}}$

\item Montrer que la variable aléatoire $X$ admet une variance
$\V\left(
X\right) $ et calculer $\V\left( X\right) $.

\item
\begin{noliste}{a)}
 \setlength{\itemsep}{2mm}
\item On considère une variable aléatoire $U$ suivant la loi uniforme
sur l'intervalle, $\left] 0;1\right] $. Montrer que la variable
aléatoire $Z = a\sqrt{-2\ln\left( U\right) }$ suit la même loi que la
variable aléatoire $X$

\item En déduire un programme en langage \Scilab{}, utilisant le
générateur aléatoire \Scilab{}, simulant la variable aléatoire
$X$, le réel $a$ strictement positif étant entré par l'utilisateur.
\end{noliste}
\end{noliste}

Soit un entier $n$ tel que $n\geq2.$

On dit. que les. variables aléatoires à densité $X_{1},$
$X_{2}\cdots,X_{n}$ sont indépendantes si, pour tout, $n$-uplet $\left(
x_{1},x_{2},\cdots,x_{n}\right) $ de réels, les événements
$\left( X_{1}\leq x_{1}\right),\ \left( X_{2}\leq x_{2}\right),\
\cdots\left( X_{n}\leq x_{n}\right) $ sont mutuellement indépendants.

On admet que si $n$ variables aléatoires à densité $X_{1},$
$X_{2}\cdots,X_{n}$admettent une espérance, alors la variable
aléatoire $X_{1} + \cdots + X_{n}$ admet une espérance qui est égale
à la somme des espérances.

On admet que si $n$ variables aléatoires à densité $X_{1},$
$X_{2}\cdots,X_{n}$sont indépendantes et admettent variance alors la
variable aléatoire $X_{1} + \cdots + X_{n}$ admet une variance qui est
égale à la. somme des variances.

On considère $n$ variables aléatoires indépendantes $X_{1},$
$X_{2}\cdots,X_{n}$ suivant toutes la même loi que la variable
aléatoire $X$.

\begin{noliste}{1.}
 \setlength{\itemsep}{4mm}
\item[9.] On considère la variable aléatoire $A_{n} =
\dfrac{\sqrt{2}}{n\sqrt{\pi}}\left( X_{1} + X_{2} + \cdots +
X_{n}\right) $.

\begin{noliste}{a)}
 \setlength{\itemsep}{2mm}
\item Montrer que la variable -aléatoire $A_{n}$, est un estimateur
sans
biais de $a$.

\item Déterminer le risque quadratique de l'estimateur $A_{n}$.
\end{noliste}
\end{noliste}

On définit la variable aléatoire $M_{n} = \min\left(
X_{1},X_{2}\cdots,X_{n}\right) $.

Ainsi \ : \qquad$\forall t\in\R,$\ $\left( M_{n}>t\right) = \left(
X_{1}>t\right) \cap\left( X_{2}>t\right) \cap\cdots\cap\left(
X_{n}>t\right).$

\begin{noliste}{1.}
 \setlength{\itemsep}{4mm}
\item[10.]
\begin{noliste}{a)}
 \setlength{\itemsep}{2mm}
\item Montrer, pour tout $t\left[ 0, + \infty\right[
:\quad\operatorname{P}\left( M_{n}>t\right) =
e^{-\tfrac{nt^{2}}{2a^{2}}}$.

\item En déduire la fonction de répartition de $M_{n}$.

\item Montrer que $M_{n}$ est une variable aléatoire à densité,
admettant $g_{b}$ comme densité avec $b = \dfrac{a}{\sqrt{n}}.$

\item Montrez- que la variable aléatoire $M_{n}$, admet une espérance
$\E\left( M_{n}\right) $ et une variance $\V\left( M_{n}\right) $

Calculer $\E\left( M_{n}\right) $ et $\V\left( M_{n}\right) $.
\end{noliste}

\item[11.] 

\begin{noliste}{a)}
 \setlength{\itemsep}{2mm}
\item En déduire un estimateur $B_{n}$, sans biais de $a$, de la forme
$\lambda_{n}M_{n}$ avec $\lambda_{n}\in\R.$

\item Déterminer le risque quadratique de l'estimateur $B_{n}$.
\end{noliste}
\end{noliste}


\end{document}