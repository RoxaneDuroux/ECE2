\documentclass[11pt]{article}%
\usepackage{geometry}%
\geometry{a4paper,
 lmargin = 2cm,rmargin = 2cm,tmargin = 2.5cm,bmargin = 2.5cm}

\input{../../macros.tex}

\pagestyle{fancy} %
\lhead{ECE2 \hfill Mathématiques\\
} %
\chead{\hrule} %
\rhead{} %
\lfoot{} %
\cfoot{} %
\rfoot{\thepage} %

\renewcommand{\headrulewidth}{0pt}% : Trace un trait de séparation
 % de largeur 0,4 point. Mettre 0pt
 % pour supprimer le trait.

\renewcommand{\footrulewidth}{0.4pt}% : Trace un trait de séparation
 % de largeur 0,4 point. Mettre 0pt
 % pour supprimer le trait.

\setlength{\headheight}{14pt}

\title{\bf \vspace{-2cm} EML 1994} %
\author{} %
\date{} %
\begin{document}

\maketitle %
\vspace{-1.4cm}\hrule %
\thispagestyle{fancy}

\vspace*{.2cm}


% DEBUT DU DOC À MODIFIER : tout virer jusqu'au début de l'exo

%Définition et changement de valeurs de
compteurs%newcounter{cpt1}{section} compteur cpt1 remis à 0 à chaque
aumentation par stepcounter du compteur section%setcounter{cpt1}{3} on
met le compteur à 3%addtocounter{cpt1}{5} on ajoute 5 au compteur%
stepcounter{cpt1} on ajoute 1% ifthenelse{test}{alors}{sinon} (page
206) pour subordonner à une condition % whiledo{test}{commande} pour
faire une boucle (page 206 aussi) % value{cpt1} pour noter dans le
document la valeur de cpt1 
%Définition définitive d'opérateurs
mathématiques\newcommand{\ch}{\operatorname{ch}} 
\newcommand{\sh}{\operatorname{sh}}
\renewcommand{\tanh}{\operatorname{th}}
\renewcommand{\sinh}{\operatorname{sh}}
\renewcommand{\cosh}{\operatorname{ch}}
\newcommand{\argsh}{\operatorname{argsh}}
\newcommand{\argch}{\operatorname{argch}}
\newcommand{\argth}{\operatorname{argth}}
\newcommand{\ker}{\operatorname{Ker}}
\renewcommand{\im}{\operatorname{Im}}
\newcommand{\rg}{\operatorname{rg}}
\newcommand{\Id}{\operatorname{Id}}
\newcommand{\id}{\operatorname{id}}
\renewcommand{\leq}{\leq}
\renewcommand{\geq}{\geq }

%Définition de nouvelles couleurs : rgb(trois paramètres red green blue
entre 0 et 1); cmyk (quatre cyan magenta yellow black) entre 0 et 1;
gray (entre 0 et 1) et black, white, red, green, blue, cyan, magenta,
yellow% definecolor{0gris}{gray}{0.8} 
% Nouvelle commande pour encadrer le titre car shabox ne veut que d'une
seule ligne; ATTENTION A LA TAILLE; petite différence avec shadowbox ou
doublebox, voire fcolorbox ou colorbox (au lieu de shabox; laisser le
parbox tranquille sauf pour la taille de la boîte
\newcommand{\Tbox}[1]{\begin{center} \shabox{\parbox{0.6
\linewidth}{#1}} \end{center}} %[1] pour 1 paramètre ; #1 pour ce que
fait le 1er paramètre; entre accolades ce que fait la commande
%Mise en page en mode fancy : en-têtes et pieds de pages puis
définition des en-têtes et pieds de pages\pagestyle{fancy}
\lhead{ECE 2 - Mathématiques \\
Quentin Dunstetter - ENC-Bessières 2011$\backslash$2012}
\chead{}
\rhead{EML 1994}
\rfoot[ \ \thepage]{\thepage}
\cfoot{}
\lfoot{}
\thispagestyle{fancy} %Mise en page de la 1ère page en mode fancy
%Trait en bas et en haut de la page (entre en-tête et texte et texte et
pied de page)\renewcommand{\footrulewidth}{0.4pt}
\renewcommand{\headrulewidth}{0.4pt}


%DEBUT DU DOCUMENT\vspace*{3cm}

\begin{center}
{\LARG\E\textbf{BANQUE COMMUNE D'ÉPREUVES}}



{\large \textsc{CONCOURS D ADMISSION DE 1994}}



{\large \textbf{Concepteur : EML}}



\rule{2.39cm}{0.05cm}



{\Large \textbf{OPTION ÉCONOMIQUE}}



{\Large \textbf{MATHÉMATIQUES }}



{\Large Lundi 9 mai, de 14h à 18h}



\rule{2.39cm}{0.05cm}
\end{center}

\textit{La présentation, la lisibilité, l'orthographe, la qualité
de la rédaction, la clarté et la précision des raisonnements
entreront pour une part importante dans l'appréciation des copies.}

\textit{Les candidats sont invités à \textbf{encadrer} dans la mesure
du possible les résultats de leurs calculs.}

\textit{Ils ne doivent faire usage d'aucun document. L'utilisation de
toute
calculatrice et de tout matériel électronique est interdite. Seule
l'utilisation d'une règle graduée est autorisée.}

\textit{Si au cours de l'épreuve, un candidat repère ce qui lui semble
être une erreur d'énoncé, il la signalera sur sa copie et
poursuivra sa composition en expliquant les raisons des initiatives
qu'il sera
amené à prendre.}

\vspace*{3cm}

\section*{EXERCICE 1}

On considère la matrice $A\left( a\right) $ de $\mathfrak{M}_{3}\left( 
\R\right) $ suivante : 
\[
A\left( a\right) = \left( 
\begin{array}{ccc}
1 & -1 & a^{2} \\
0 & 0 & a^{2} \\
1 & 0 & 0
\end{array}
\right)
\]

\begin{noliste}{1.}
 \setlength{\itemsep}{4mm}
\item Déterminer les valeurs propres de $A\left( a\right),$ pour $a\in 
\R.$

\item Étudier suivant les valeurs du réels $a,$ l'inversibilité de
$A\left(
a\right) $ dans $\mathfrak{M}_{3}\left( \R\right).$

\item On suppose dans cette question 3. seulement : $a\ne 0$ et $a\ne
1$ et $a\ne -1.$

\begin{noliste}{a)}
 \setlength{\itemsep}{2mm}
\item Montrer que $A\left( a\right) $ est diagonalisable.

\item Calculer, pour chacune des valeurs propres de $A,$ un vecteur
propre
de $A\left( a\right) $ associé à cette valeur propre.
\end{noliste}

\item 

\begin{noliste}{a)}
 \setlength{\itemsep}{2mm}
\item La matrice $A\left( 0\right) $ est-elle diagonalisable ?

\item calculer $\left( A\left( 0\right) \right) ^{2},$ $\left( A\left(
0\right) \right) ^{3},$ et $\left( A\left( 0\right) \right) ^{n}$ pour
tout
entier naturel $n$ non nul.
\end{noliste}
\end{noliste}

\section*{EXERCIC\E\ 2}

On considère la fonction $f$ définie sur $\left[ 0, + \infty \right[ $
par $ :f\left( x\right) = 2\sqrt{x}e^{-x}$

\begin{noliste}{1.}
 \setlength{\itemsep}{4mm}
\item 

\begin{noliste}{a)}
 \setlength{\itemsep}{2mm}
\item Dresser le tableau des variations de $f.$

\item La fonction $f$ est-elle dérivable en $0$ ?

\item Tracer la courbe représentative de $f$ (repère orthonormé, unité
5cm)

(Cette courbe admet un point d'inflexion qu'on ne cherchera pas à
déterminer)
\end{noliste}

\item 

\begin{noliste}{a)}
 \setlength{\itemsep}{2mm}
\item Montrer que l'image par $f$ du segment $\left[
0,\dfrac{1}{2}\right] $
est le segment $\left[ 0,\sqrt{\dfrac{2}{e}}\right] $.

\item On définit la fonction $\varphi :\begin{array}[t]{ccc}
\left[ 0,\dfrac{1}{2}\right] & \rightarrow & \left[
0,\sqrt{\dfrac{2}{e}}\right] \\
x & \mapsto & 2\sqrt{x}e^{-x}
\end{array}
$

Démontrer que $\varphi $ admet une fonction réciproque continue que
l'on
notera $g.$

\item Dresser le tableau des variations de $g.$

\item Démontrer que $g$ est dérivable en tout point de l'intervalle
$\left]
0,\sqrt{\dfrac{2}{e}}\right[ $

\item La fonction $g$ est-elle dérivable en 0 ? En
$\sqrt{\dfrac{2}{e}}$ ?
\end{noliste}

\item 

\begin{noliste}{a)}
 \setlength{\itemsep}{2mm}
\item Soit $n$ un entier naturel supérieur ou égal à 2. Démontrer que
l'équation d'inconnue $x$ : 
\[
\varphi \left( x\right) = \frac{1}{n}
\]
admet une unique solution dans le segment $\left[
0,\dfrac{1}{2}\right].$
On notera $a_{n}$ cette solution.

\item Montrer que la suite $\left( a_{n}\right)_{n\geq 2}$ est
décroissante.

\item Montrer que la suite $\left( a_{n}\right)_{n\geq 2}$ converge
vers $0.$
\end{noliste}
\end{noliste}

\section*{EXERCICE 3}

On suppose que le nombre $N$ de colis expédiés à l'étranger chaque jour
par
une entreprise suit une loi de Poisson de paramètre 5. Ces colis sont
expédiés indépendemment les uns des autres.\\
La probabilité qu'un colis expédié à l'étranger soit déterioré est
égale à $0,1$.\\
On s'intéresse aux colis expédiés à l'étranger un jour donné :

\begin{noliste}{$\sbullet$}
\item $N$ est la variable aléatoire égale au nombre de colis expédiés.

\item $X$ est la variable aléatoire égale au nombre de colis
déteriorés.

\item $Y$ est la variable aléatoire égale au nombre de colis en bon
état.
\end{noliste}

\noindent On a donc $N = X + Y$

\begin{noliste}{1.}
 \setlength{\itemsep}{4mm}
\item Soit $n$ un entier naturel; calculer pour tout entier naturel $k$
la
probabilité conditionnelle suivante : 
\[
P_{\left( N = n\right) }\left( X = k\right)
\]

\item Donner la loi du couple $\left( X,N\right) $ puis montrer que $X$
suit
une loi de Poisson de paramètre 0,5.

\item Déterminer la loi de $Y$.

\item 

\begin{noliste}{a)}
 \setlength{\itemsep}{2mm}
\item Si $i$ et $j$ sont deux entir naturels, calculer la probabilité $
:P\left( \left(\Ev{ X = i}\right) \cap \left( Y = j\right) \right) $

\item $X$ et $Y$ sont-elles indépendantes ?
\end{noliste}
\end{noliste}

\section*{EXERCICE 4}

On pose pour tout entier $n$ non nul 
\[
I_{n} = \dint{1}{e}\left( \ln \left( x\right) \right) ^{n}dx\quad 
\text{et\quad }I_{0} = e-1
\]

\begin{noliste}{1.}
 \setlength{\itemsep}{4mm}
\item 

\begin{noliste}{a)}
 \setlength{\itemsep}{2mm}
\item \label{rec}Établir, pour tout entier $n :I_{n + 1} = e-(n +
1)I_{n}$.

\item \label{ine}Montrer, pour tout enier $n$, que : $I_{n}\geq 0$

\item Déduire des questions \ref{rec} et \ref{ine} que pour tout $n :$
\[
0\leq I_{n}\leq \frac{e}{n + 1}
\]

\item Quelle est la lmite de la suite $(I_{n})_{n\in \N}$ ?

\item Montrer que $I_{n}\thicksim \dfrac{e}{n}$
\end{noliste}

\item Soit $a$ un réel différent de $I_{0}$. On note $(u_{n})_{n\in
\N}$ la suite réelle définie par : 
\[
\left\{ 
\begin{array}{ccc}
u_{0} & = & a \\
u_{n + 1} & = & e-\left( n + 1\right) u_{n}
\end{array}
\right.
\]
Montrer que $\left| u_{n}\right| \underset{n\rightarrow + \infty
}{\rightarrow } + \infty $\\
(On pourra considérer la suite $\left( D_{n}\right)_{n\in \N}$ de
terme général $D_{n} = \left| u_{n}-I_{n}\right| )$
\end{noliste}

\label{fin}

\end{document}

