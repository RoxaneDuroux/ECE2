\documentclass[11pt]{article}%
\usepackage{geometry}%
\geometry{a4paper,
 lmargin = 2cm,rmargin = 2cm,tmargin = 2.5cm,bmargin = 2.5cm}

\input{../../macros.tex}

\pagestyle{fancy} %
\lhead{ECE2 \hfill Mathématiques\\
} %
\chead{\hrule} %
\rhead{} %
\lfoot{} %
\cfoot{} %
\rfoot{\thepage} %

\renewcommand{\headrulewidth}{0pt}% : Trace un trait de séparation
 % de largeur 0,4 point. Mettre 0pt
 % pour supprimer le trait.

\renewcommand{\footrulewidth}{0.4pt}% : Trace un trait de séparation
 % de largeur 0,4 point. Mettre 0pt
 % pour supprimer le trait.

\setlength{\headheight}{14pt}

\title{\bf \vspace{-2cm} EML 1993} %
\author{} %
\date{} %
\begin{document}

\maketitle %
\vspace{-1.4cm}\hrule %
\thispagestyle{fancy}

\vspace*{.2cm}


% DEBUT DU DOC À MODIFIER : tout virer jusqu'au début de l'exo

%Définition et changement de valeurs de
compteurs%newcounter{cpt1}{section} compteur cpt1 remis à 0 à chaque
aumentation par stepcounter du compteur section%setcounter{cpt1}{3} on
met le compteur à 3%addtocounter{cpt1}{5} on ajoute 5 au compteur%
stepcounter{cpt1} on ajoute 1% ifthenelse{test}{alors}{sinon} (page
206) pour subordonner à une condition % whiledo{test}{commande} pour
faire une boucle (page 206 aussi) % value{cpt1} pour noter dans le
document la valeur de cpt1 
%Définition définitive d'opérateurs
mathématiques\newcommand{\ch}{\operatorname{ch}} 
\newcommand{\sh}{\operatorname{sh}}
\renewcommand{\tanh}{\operatorname{th}}
\renewcommand{\sinh}{\operatorname{sh}}
\renewcommand{\cosh}{\operatorname{ch}}
\newcommand{\argsh}{\operatorname{argsh}}
\newcommand{\argch}{\operatorname{argch}}
\newcommand{\argth}{\operatorname{argth}}
\newcommand{\ker}{\operatorname{Ker}}
\renewcommand{\im}{\operatorname{Im}}
\newcommand{\rg}{\operatorname{rg}}
\newcommand{\Id}{\operatorname{Id}}
\newcommand{\id}{\operatorname{id}}
\renewcommand{\leq}{\leq}
\renewcommand{\geq}{\geq }

%Définition de nouvelles couleurs : rgb(trois paramètres red green blue
entre 0 et 1); cmyk (quatre cyan magenta yellow black) entre 0 et 1;
gray (entre 0 et 1) et black, white, red, green, blue, cyan, magenta,
yellow% definecolor{0gris}{gray}{0.8} 
% Nouvelle commande pour encadrer le titre car shabox ne veut que d'une
seule ligne; ATTENTION A LA TAILLE; petite différence avec shadowbox ou
doublebox, voire fcolorbox ou colorbox (au lieu de shabox; laisser le
parbox tranquille sauf pour la taille de la boîte
\newcommand{\Tbox}[1]{\begin{center} \shabox{\parbox{0.6
\linewidth}{#1}} \end{center}} %[1] pour 1 paramètre ; #1 pour ce que
fait le 1er paramètre; entre accolades ce que fait la commande
%Mise en page en mode fancy : en-têtes et pieds de pages puis
définition des en-têtes et pieds de pages\pagestyle{fancy}
\lhead{ECE 2 - Mathématiques \\
Quentin Dunstetter - ENC-Bessières 2011$\backslash$2012}
\chead{}
\rhead{EML 1993}
\rfoot[ \ \thepage]{\thepage}
\cfoot{}
\lfoot{}
\thispagestyle{fancy} %Mise en page de la 1ère page en mode fancy
%Trait en bas et en haut de la page (entre en-tête et texte et texte et
pied de page)\renewcommand{\footrulewidth}{0.4pt}
\renewcommand{\headrulewidth}{0.4pt}


%DEBUT DU DOCUMENT\vspace*{3cm}

\begin{center}
{\LARG\E\textbf{BANQUE COMMUNE D'ÉPREUVES}}



{\large \textsc{CONCOURS D ADMISSION DE 1993}}



{\large \textbf{Concepteur : EML}}



\rule{2.39cm}{0.05cm}



{\Large \textbf{OPTION ÉCONOMIQUE}}



{\Large \textbf{MATHÉMATIQUES }}



{\Large Lundi 9 mai, de 14h à 18h}



\rule{2.39cm}{0.05cm}
\end{center}

\textit{La présentation, la lisibilité, l'orthographe, la qualité
de la rédaction, la clarté et la précision des raisonnements
entreront pour une part importante dans l'appréciation des copies.}

\textit{Les candidats sont invités à \textbf{encadrer} dans la mesure
du possible les résultats de leurs calculs.}

\textit{Ils ne doivent faire usage d'aucun document. L'utilisation de
toute
calculatrice et de tout matériel électronique est interdite. Seule
l'utilisation d'une règle graduée est autorisée.}

\textit{Si au cours de l'épreuve, un candidat repère ce qui lui semble
être une erreur d'énoncé, il la signalera sur sa copie et
poursuivra sa composition en expliquant les raisons des initiatives
qu'il sera
amené à prendre.}

\vspace*{3cm}

\section*{EXERCICE 1}

$f$ est l'endomorphisme de $\R^{3}$ dont la matrice dans la base
canonique de $\R^{3},$ notée$\left( \vec{i},\vec{j},\vec{k}\right) $
est $A$ : 
\[
A = \left( 
\begin{array}{rrr}
1 & 0 & -1 \\
0 & 1 & 0 \\
-1 & 2 & 1
\end{array}
\right)
\]

\begin{noliste}{1.}
 \setlength{\itemsep}{4mm}
\item 

\begin{noliste}{a)}
 \setlength{\itemsep}{2mm}
\item Calculer les valeurs propres de $f$

On les notrera $\lambda_{1},\lambda_{2},\lambda_{3}$ de sorte que
$\lambda_{1}<\lambda_{2}<\lambda_{3}$

\item En déduire, sans autre calcul, les réponses aux questions
suivantes :\\
$A$ est-elle diagonalisable ?\\
$A$ est-elle inversible ?
\end{noliste}

\item Pour tout $p\in \left\{ 1,2,3\right\},$ montrer qu'il existe un
vecteur popre $\overrightarrow{e_{p}}$ de $f$ associé à la valeur
porpre $\lambda_{p}$ dont la $p^{i\grave{e}me}$ coordonnée dans la base
canonique $\left( \vec{i},\vec{j},\vec{k}\right) $ est $1.$ Donner les
coordonnées de $\overrightarrow{e_{p}}$ dans cette base.

\item Soit $P$ la matrice de passage de $\left(
\vec{i},\vec{j},\vec{k}\right) $ à la base $\left(
\overrightarrow{e_{1}},\overrightarrow{e_{2}},\overrightarrow{e_{3}}\ri
ht).$ Écrire $P.$ Calculer $P^{-1} :$ les calculs
devront figurer sur la copie.

\item Pour $n\in \N^{\times },$ calculer $A^{n} :$ on donnera de façon
explicite les neufs coefficients de la matrice $A^{n}.$
\end{noliste}

\section*{EXERCICE 2}

Soit $f :\begin{array}[t]{lll}
\R & \rightarrow & \R \\
x & \mapsto & f\left( x\right) = {\dfrac{x + 1}{\sqrt{x^{2} + 1}}}-1
\end{array}
$

\subsection*{I. Étude de f.}

\begin{noliste}{1.}
 \setlength{\itemsep}{4mm}
\item Former le tableau de variation de $f$

\item 

\begin{noliste}{a)}
 \setlength{\itemsep}{2mm}
\item Résoudre l'équation $f\left( x\right) = x$, d'inconnue $x\in \R$

\item Résoudre l'équation $f\left( x\right) \leq x$, d'inconnue $x\in 
\R$
\end{noliste}

\item Tracer la courbe représentative $\left( C\right) $ de $f$ dans un
repère orthonormé d'unité 5cm, et préciser la position relative de
$\left(
C\right) $ et de la première bissectrice (on ne cherchera pas
d'éventuels
points d'inflexion)
\end{noliste}

\subsection*{II. Étude d'une suite récurrente.}

On considère la suite $(u_{n})_{n\in \N}$ définie par : $u_{0}\in 
\R$ et pour tout entier $n,$ $u_{n + 1} = f\left( u_{n}\right) $

\begin{noliste}{1.}
 \setlength{\itemsep}{4mm}
\item Que dire de $(u_{n})_{n\in \N}$ si $u_{0} = -1$ ou $u_{0} = 0$ ?

\item On suppose ici $u_{0}<-1$.

\begin{noliste}{a)}
 \setlength{\itemsep}{2mm}
\item Montrer que $\forall n\in \N,\quad u_{n}<-1$

\item En déduire que $\left( u_{n}\right)_{n\in \N}$ est croissante.

\item Montrer que $(u_{n})_{n\in \N}$ converge vers un réel que l'on
déterminera.
\end{noliste}

\item On suppose ici $-1<u_{0}<0$.

Montrer que $\left( u_{n}\right)_{n\in \N}$ converge et déterminer
sa limite.

\item On suppose ici $u_{0}>0$.

Sans en donner de démonstration, quel résultat obtiendrait-on
concernant la
convergence de $(u_{n})_{n\in \N}$ dans ce cas ?
\end{noliste}

\section*{EXERCICE 3}

\subsection*{Question préliminaire}

Soient $k$ et $n$ deux entiers naturels tels que $0<3k\leq n$

\begin{noliste}{1.}
 \setlength{\itemsep}{4mm}
\item 
\begin{noliste}{a)}
 \setlength{\itemsep}{2mm}
\item Démonter que\\
pour tout $i$ tel que $0\leq i\leq k-1,C_{n}{i}\leq \frac{1}{2}C_{n}{i
+ 1}$\\
puis que\\
pour tout $i$ tel que $0\leq i\leq k,$ $C_{n}{i}\leq
\frac{1}{2^{k-i}}C_{n}{k}$

\item En déduire que $C_{n}{k}\leq
\Sum{i = 0}{k}C_{n}{i}\leq 2C_{n}{k}$
\end{noliste}
\end{noliste}

\noindent Monsieur X vend des journaux, sur le marché le samedi matin;
il
propose au choix, deux quotidiens A et B et il dispose d'un stock de 40
exemplaires de A et 40 exemplaires de B.\\
On suppose :

\begin{noliste}{$\sbullet$}
\item qu'aucun client ne demande A et B

\item que si un client demande A (respectivement B) alors que le stock
de A
(respectivement B) est épuisé, il part sans demander B (respectivement
A)

\item que les demandes des clients sont indépendantes les unes des
autres.
\end{noliste}

\noindent Un samedi, 60 clients se présentent dans la matinée. chaque
client
qui demande soit A soit B avec la même probabiltié $0,5.$

\begin{noliste}{1.}
 \setlength{\itemsep}{4mm}
\item $Y$ est la variable aléatoire égale au nombre de clients qui
demandent
A dans cette matinée.\\
Déterminer la loi de $Y$\\
Donner son espérance et sa variance.

\item On note $x$ la probabilité de l'évènement \textquotedblright
Monsieur
X ne satisfait pas toutes les demandes, cette matinée\textquotedblright

\begin{noliste}{a)}
 \setlength{\itemsep}{2mm}
\item Exprimer $x$ à l'aide de la loi de $Y.$

\item Déduire de l'inégalité de Bienaymé-Tchebychev un majorant de $x.$

\item Déduire de la question préliminaire un encadrement de $x.$

\item Comparer, en utilisant des valeurs numériques approchées données
en
annexe, les résultats des questions \textbf{b} et \textbf{c}.
\end{noliste}
\end{noliste}

\noindent Annexe : $C_{60}{20}\simeq 4,192.10^{15};\quad
C_{60}{19}\simeq
2.045;10^{15};\quad C_{60}{18}\simeq 9,250.10^{14}$

\section*{EXERCICE 4}

Pour tout entier $n$ on note : 
\[
I_{n} = \dint{0}{1}e^{-x^{2}}(1-x)^{n}dx\qquad \text{et\qquad }J_{n} =
\dint{0}{1}xe^{-x^{2}}(1-x)^{n}dx.
\]

\begin{noliste}{1.}
 \setlength{\itemsep}{4mm}
\item 

\begin{noliste}{a)}
 \setlength{\itemsep}{2mm}
\item Former le tableau de variation sur $[0,1]$ de $x\rightarrow
xe^{-x^{2}} $..

\item En déduire pour tout $n$ de $\N :$
\[
0\leq J_{n}\leq \frac{1}{\sqrt{2e}(n + 1)}
\]

\item Étudier la convergence de la suite $\left( J_{n}\right)_{n\in
\N}$.
\end{noliste}

\item 

\begin{noliste}{a)}
 \setlength{\itemsep}{2mm}
\item À l'aide d'une intégration par parties, établir pour tout $n$ de
$\N$ : 
\[
I_{n} = \frac{1}{n + 1}-\frac{2}{n + 1}J_{n + 1}
\]

\item En déduire la limite de $I_{n}$ et celle de $n.I_{n}$ quand $n$
tend
vers $ + \infty $.
\end{noliste}
\end{noliste}

\label{fin}

\end{document}

