\documentclass[11pt]{article}%
\usepackage{geometry}%
\geometry{a4paper,
 lmargin = 2cm,rmargin = 2cm,tmargin = 2.5cm,bmargin = 2.5cm}

\input{../../macros.tex}

\pagestyle{fancy} %
\lhead{ECE2 \hfill Mathématiques\\
} %
\chead{\hrule} %
\rhead{} %
\lfoot{} %
\cfoot{} %
\rfoot{\thepage} %

\renewcommand{\headrulewidth}{0pt}% : Trace un trait de séparation
 % de largeur 0,4 point. Mettre 0pt
 % pour supprimer le trait.

\renewcommand{\footrulewidth}{0.4pt}% : Trace un trait de séparation
 % de largeur 0,4 point. Mettre 0pt
 % pour supprimer le trait.

\setlength{\headheight}{14pt}

\title{\bf \vspace{-2cm} EML 1986} %
\author{} %
\date{} %
\begin{document}

\maketitle %
\vspace{-1.4cm}\hrule %
\thispagestyle{fancy}

\vspace*{.2cm}


% DEBUT DU DOC À MODIFIER : tout virer jusqu'au début de l'exo

%Définition et changement de valeurs de
compteurs%newcounter{cpt1}{section} compteur cpt1 remis à 0 à chaque
aumentation par stepcounter du compteur section%setcounter{cpt1}{3} on
met le compteur à 3%addtocounter{cpt1}{5} on ajoute 5 au compteur%
stepcounter{cpt1} on ajoute 1% ifthenelse{test}{alors}{sinon} (page
206) pour subordonner à une condition % whiledo{test}{commande} pour
faire une boucle (page 206 aussi) % value{cpt1} pour noter dans le
document la valeur de cpt1 
%Définition définitive d'opérateurs
mathématiques\newcommand{\ch}{\operatorname{ch}} 
\newcommand{\sh}{\operatorname{sh}}
\renewcommand{\tanh}{\operatorname{th}}
\renewcommand{\sinh}{\operatorname{sh}}
\renewcommand{\cosh}{\operatorname{ch}}
\newcommand{\argsh}{\operatorname{argsh}}
\newcommand{\argch}{\operatorname{argch}}
\newcommand{\argth}{\operatorname{argth}}
\newcommand{\ker}{\operatorname{Ker}}
\renewcommand{\im}{\operatorname{Im}}
\newcommand{\rg}{\operatorname{rg}}
\newcommand{\Id}{\operatorname{Id}}
\newcommand{\id}{\operatorname{id}}
\renewcommand{\leq}{\leq}
\renewcommand{\geq}{\geq }

%Définition de nouvelles couleurs : rgb(trois paramètres red green blue
entre 0 et 1); cmyk (quatre cyan magenta yellow black) entre 0 et 1;
gray (entre 0 et 1) et black, white, red, green, blue, cyan, magenta,
yellow% definecolor{0gris}{gray}{0.8} 
% Nouvelle commande pour encadrer le titre car shabox ne veut que d'une
seule ligne; ATTENTION A LA TAILLE; petite différence avec shadowbox ou
doublebox, voire fcolorbox ou colorbox (au lieu de shabox; laisser le
parbox tranquille sauf pour la taille de la boîte
\newcommand{\Tbox}[1]{\begin{center} \shabox{\parbox{0.6
\linewidth}{#1}} \end{center}} %[1] pour 1 paramètre ; #1 pour ce que
fait le 1er paramètre; entre accolades ce que fait la commande
%Mise en page en mode fancy : en-têtes et pieds de pages puis
définition des en-têtes et pieds de pages\pagestyle{fancy}
\lhead{ECE 2 - Mathématiques \\
Quentin Dunstetter - ENC-Bessières 2011$\backslash$2012}
\chead{}
\rhead{EML 1986}
\rfoot[ \ \thepage]{\thepage}
\cfoot{}
\lfoot{}
\thispagestyle{fancy} %Mise en page de la 1ère page en mode fancy
%Trait en bas et en haut de la page (entre en-tête et texte et texte et
pied de page)\renewcommand{\footrulewidth}{0.4pt}
\renewcommand{\headrulewidth}{0.4pt}


%DEBUT DU DOCUMENT\vspace*{3cm}

\begin{center}
{\LARG\E\textbf{BANQUE COMMUNE D'ÉPREUVES}}



{\large \textsc{CONCOURS D ADMISSION DE 1986}}



{\large \textbf{Concepteur : EML}}



\rule{2.39cm}{0.05cm}



{\Large \textbf{OPTION ÉCONOMIQUE}}



{\Large \textbf{MATHÉMATIQUES }}



{\Large Lundi 9 mai, de 14h à 18h}



\rule{2.39cm}{0.05cm}
\end{center}

\textit{La présentation, la lisibilité, l'orthographe, la qualité
de la rédaction, la clarté et la précision des raisonnements
entreront pour une part importante dans l'appréciation des copies.}

\textit{Les candidats sont invités à \textbf{encadrer} dans la mesure
du possible les résultats de leurs calculs.}

\textit{Ils ne doivent faire usage d'aucun document. L'utilisation de
toute
calculatrice et de tout matériel électronique est interdite. Seule
l'utilisation d'une règle graduée est autorisée.}

\textit{Si au cours de l'épreuve, un candidat repère ce qui lui semble
être une erreur d'énoncé, il la signalera sur sa copie et
poursuivra sa composition en expliquant les raisons des initiatives
qu'il sera
amené à prendre.}

\vspace*{3cm}

\section*{Exercice 1}

On note $f$ l'endomorphisme de l'espace vectoriel $\R^{3}$ dont la
matrice dans la base canonique $B = (e_{1},e_{2},e_{3})$ est : 
\[
A = \left( 
\begin{array}{ccc}
3 & 1 & 0 \\
-3 & 0 & 1 \\
1 & 0 & 0
\end{array}
\right)
\]
D'autre part, on considère les trois éléments $v_{1},v_{2},v_{3}$ de
$\R^{3}$ définis par : 
\[
v_{1} = e_{1}-2e_{2} + e_{3},\quad v_{2} = -e_{2} + e_{3},\quad v_{3} =
e_{3}.
\]

\begin{noliste}{1.}
 \setlength{\itemsep}{4mm}
\item Calculer $e_{1},e_{2},e_{3}$ en fonction de $v_{1},v_{2},v_{3}$.

\item Montrer que $B^{\prime } = (v_{1},v_{2},v_{3})$ est une base de
$\R^{3}$. On note $P$ la matrice de passage de la base $B$ à la base
$B^{\prime }$ ; calculer $P$ et $P^{-1}$.

\item Déterminer la matrice $A^{\prime }$ de l'endomorphisme $f$
relativement à la base $B^{\prime }$.

\item Calculer $(A^{\prime })^{2},(A^{\prime })^{3},(A^{\prime })^{n}$
puis $A^{n}$ pour tout entier naturel non nul $n$.
\end{noliste}

\section*{Exercice 2}

On désigne par $f$ et $g$ deux fonctions continues de $\R$ dans $\R$.\\
On définit la fonction $h = f\ast g$ par : $h :\left. 
\begin{array}{ccl}
\R & \rightarrow & \R \\
x & \mapsto & h(x) = \dint{0}{x}\,f(t)g(x-t)\,dt
\end{array}
\right. $

\begin{noliste}{1.}
 \setlength{\itemsep}{4mm}
\item Montrer que $f\ast g = g\ast f$.

\item Dans cette question, on suppose que : \\
pour $x\leq 0$, $f(x) = g(x) = 0$ \\
et pour $x>0$, $f(x) = x^{p}$ et $g(x) = x^{q}$ avec $(p,q){\large \in
}\N^{2}.$\\
On pose $I_{p,q} = f\ast g$.

\begin{noliste}{a)}
 \setlength{\itemsep}{2mm}
\item Déterminer $I_{p,0}$, $I_{1,1}$ et $I_{1,2}$.

\item Pour $p$ et $q$ entiers naturels, $q\geq 1$, établir une relation
entre $I_{p,q}$ et $I_{p + 1,q-1}$ et en déduire explicitement
$I_{p,q}$.
\end{noliste}

\item On suppose que $f$ et $g$ sont définies respectivement par $f(x)
= \left| x\right| $ et $g(x) = x-1$ pour tout $x$ réel. Déterminer
la fonction $f\ast g$ et tracer sa représentation graphique dans un
repère
orthonormé ; on précisera la tangente à l'origine du repère. 
\end{noliste}

\section*{Exercice 3}

Une urne contient 3 boules (une bleue, une blanche et une rouge) dont
les
tirages sont supposés équiprobables. On effectue des tirages successifs
d'une boule, avec remise à chaque fois de la boule tirée.\\
Pour tout entier naturel $n\geq 2$, on appelle $A_{n}$ l'évènement :
les $n-1$ premiers tirages ont donné la même boule et la
$n^{\grave{e}me}$
boule tirée est différente de celle tirée lors des $n-1$ premiers
tirages.

\begin{noliste}{1.}
 \setlength{\itemsep}{4mm}
\item Déterminer $p(A_{2})$, $p(A_{3})$ et $p(A_{n})$ pour tout entier
$n$
tel que $n\geq 2$.

\item Calculer $\Sum{n = 2}{+ \infty }p(A_{n})$.

\item On appelle $X$ la variable aléatoire prenant la valeur $n$
lorsque $A_{n}$ est réalisé et $0$ lorsque $A_{n}$ n'est pas réalisé ;
calculer l'espérance et la variance de $X$.
\end{noliste}

\section*{Exercice 4}

On considère la fonction $f$ définie par $f :\left. 
\begin{array}{ccl}
\R & \rightarrow & \R \\
x & \mapsto & e^{-3x^{2}}-3x + 5
\end{array}
\right. $.

\begin{noliste}{1.}
 \setlength{\itemsep}{4mm}
\item Calculer les dérivées $f^{\prime }$ et $f^{\prime \prime }$.
Étudier
les variations de $f$ sur $\R$. Tracer la courbe représentative $C_{f}$
de la fonction $f$ dans un repère orthonormé, l'unité étant égale à
2cm.\\
Quelles sont les coordonnées des points d'inflexion de $C_{f}$.

\item Montrer qu'il existe un unique réel $\alpha $ tel que $f(\alpha )
= 0$.
Donner une valeur approchée de $\alpha $ à l'aide de $C_{f}$.

\item Donner une meilleure valeur approchée de $\alpha $ à l'aide du
théorème des accroissements finis.
\end{noliste}

\label{fin}

\end{document}

