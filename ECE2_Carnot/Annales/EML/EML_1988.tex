\documentclass[11pt]{article}%
\usepackage{geometry}%
\geometry{a4paper,
 lmargin = 2cm,rmargin = 2cm,tmargin = 2.5cm,bmargin = 2.5cm}

\input{../../macros.tex}

\pagestyle{fancy} %
\lhead{ECE2 \hfill Mathématiques\\
} %
\chead{\hrule} %
\rhead{} %
\lfoot{} %
\cfoot{} %
\rfoot{\thepage} %

\renewcommand{\headrulewidth}{0pt}% : Trace un trait de séparation
 % de largeur 0,4 point. Mettre 0pt
 % pour supprimer le trait.

\renewcommand{\footrulewidth}{0.4pt}% : Trace un trait de séparation
 % de largeur 0,4 point. Mettre 0pt
 % pour supprimer le trait.

\setlength{\headheight}{14pt}

\title{\bf \vspace{-2cm} EML 1988} %
\author{} %
\date{} %
\begin{document}

\maketitle %
\vspace{-1.4cm}\hrule %
\thispagestyle{fancy}

\vspace*{.2cm}


% DEBUT DU DOC À MODIFIER : tout virer jusqu'au début de l'exo

%Définition et changement de valeurs de
compteurs%newcounter{cpt1}{section} compteur cpt1 remis à 0 à chaque
aumentation par stepcounter du compteur section%setcounter{cpt1}{3} on
met le compteur à 3%addtocounter{cpt1}{5} on ajoute 5 au compteur%
stepcounter{cpt1} on ajoute 1% ifthenelse{test}{alors}{sinon} (page
206) pour subordonner à une condition % whiledo{test}{commande} pour
faire une boucle (page 206 aussi) % value{cpt1} pour noter dans le
document la valeur de cpt1 
%Définition définitive d'opérateurs
mathématiques\newcommand{\ch}{\operatorname{ch}} 
\newcommand{\sh}{\operatorname{sh}}
\renewcommand{\tanh}{\operatorname{th}}
\renewcommand{\sinh}{\operatorname{sh}}
\renewcommand{\cosh}{\operatorname{ch}}
\newcommand{\argsh}{\operatorname{argsh}}
\newcommand{\argch}{\operatorname{argch}}
\newcommand{\argth}{\operatorname{argth}}
\newcommand{\ker}{\operatorname{Ker}}
\renewcommand{\im}{\operatorname{Im}}
\newcommand{\rg}{\operatorname{rg}}
\newcommand{\Id}{\operatorname{Id}}
\newcommand{\id}{\operatorname{id}}
\renewcommand{\leq}{\leq}
\renewcommand{\geq}{\geq }

%Définition de nouvelles couleurs : rgb(trois paramètres red green blue
entre 0 et 1); cmyk (quatre cyan magenta yellow black) entre 0 et 1;
gray (entre 0 et 1) et black, white, red, green, blue, cyan, magenta,
yellow% definecolor{0gris}{gray}{0.8} 
% Nouvelle commande pour encadrer le titre car shabox ne veut que d'une
seule ligne; ATTENTION A LA TAILLE; petite différence avec shadowbox ou
doublebox, voire fcolorbox ou colorbox (au lieu de shabox; laisser le
parbox tranquille sauf pour la taille de la boîte
\newcommand{\Tbox}[1]{\begin{center} \shabox{\parbox{0.6
\linewidth}{#1}} \end{center}} %[1] pour 1 paramètre ; #1 pour ce que
fait le 1er paramètre; entre accolades ce que fait la commande
%Mise en page en mode fancy : en-têtes et pieds de pages puis
définition des en-têtes et pieds de pages\pagestyle{fancy}
\lhead{ECE 2 - Mathématiques \\
Quentin Dunstetter - ENC-Bessières 2011$\backslash$2012}
\chead{}
\rhead{EML 1988}
\rfoot[ \ \thepage]{\thepage}
\cfoot{}
\lfoot{}
\thispagestyle{fancy} %Mise en page de la 1ère page en mode fancy
%Trait en bas et en haut de la page (entre en-tête et texte et texte et
pied de page)\renewcommand{\footrulewidth}{0.4pt}
\renewcommand{\headrulewidth}{0.4pt}


%DEBUT DU DOCUMENT\vspace*{3cm}

\begin{center}
{\LARG\E\textbf{BANQUE COMMUNE D'ÉPREUVES}}



{\large \textsc{CONCOURS D ADMISSION DE 1988}}



{\large \textbf{Concepteur : EML}}



\rule{2.39cm}{0.05cm}



{\Large \textbf{OPTION ÉCONOMIQUE}}



{\Large \textbf{MATHÉMATIQUES }}



{\Large Lundi 9 mai, de 14h à 18h}



\rule{2.39cm}{0.05cm}
\end{center}

\textit{La présentation, la lisibilité, l'orthographe, la qualité
de la rédaction, la clarté et la précision des raisonnements
entreront pour une part importante dans l'appréciation des copies.}

\textit{Les candidats sont invités à \textbf{encadrer} dans la mesure
du possible les résultats de leurs calculs.}

\textit{Ils ne doivent faire usage d'aucun document. L'utilisation de
toute
calculatrice et de tout matériel électronique est interdite. Seule
l'utilisation d'une règle graduée est autorisée.}

\textit{Si au cours de l'épreuve, un candidat repère ce qui lui semble
être une erreur d'énoncé, il la signalera sur sa copie et
poursuivra sa composition en expliquant les raisons des initiatives
qu'il sera
amené à prendre.}

\vspace*{3cm}

\section*{EXERCICE 1}

Soit $f :\R^{3}\rightarrow \R^{3},(x,y,z)\longmapsto
(3x-2y,2x-4z,y-3z).$

\begin{noliste}{1.}
 \setlength{\itemsep}{4mm}
\item Écrire la matrice $A$ de $f$ dans la base canonique de $\R^{3}$.

\item Déterminer les valeurs propres de $A$. En déduire que $A$ n'est
pas
inversible et que $A$ est diagonalisable.

\item Calculer $A^{2}$, $A^{3}$. En déduire $A^{n}$, $n\in \N^{\times
}$.
\end{noliste}

\section*{EXERCIC\E\ 2}

Soit f la fonction de $\R$ dans $\R$ définie par $f(x) =
x-\dfrac{1}{4}(x + 1)e^{-x}.$

\begin{noliste}{1.}
 \setlength{\itemsep}{4mm}
\item Calculer $f^{\prime }$ et $f"$. Étudier les variations de
$f^{\prime }$. \\
Montrer que l'équation $f^{\prime }(x) = 0$ admet une seule solution
$\alpha $
et que $-1,3<\alpha <-1,2.$ \\
En déduire le sens de variation de $f$. Étudier les limites de $f$ en $
+ \infty $ et $-
\[
\infty $. Préciser les branches infinies de $\mathcal{C}_{f} $. Tracer
$\mathcal{C}_{f}$.

\item Soit $u$ la suite définie par $u_{0} = 0$ et $\forall n\in
\N,\quad u_{n + 1} = f(u_{n})$. \\
Montrer que pour tout entier $n>0$, on a $-1<u_{n}<0$ et que la suite
$u$
est décroissante. Trouver sa limite.
\end{noliste}

\section*{EXERCIC\E\ 3}

Une secrétaire effectue n appels téléphoniques vers n correspondants
distincts $(n\geq 2)$. Pour chaque appel, la probabilité d'obtenir le
correspondant demandé est $p$ appartenant à $]0,1[$ et la probabilité
de ne
pas l'obtenir est $q$, avec $q = 1-p$.

\begin{noliste}{1.}
 \setlength{\itemsep}{4mm}
\item Soit $X$ le nombre de correspondants obtenus lors de ces $n$
appels.
Quelle est la loi de $X$ ? Calculer l'espérance $\E(X)$ et la variance
$\V(X)$.

\item Après ces $n$ recherches, la secrétaire demande une deuxième fois
chacun des $n-X$ correspondants qu'elle n'a pas obtenus la première
fois.
Soit $Y$ le nombre de correspondants obtenus dans la deuxième série
d'appels, et $Z = X + Y$ le nombre total de correspondants obtenus.

\begin{noliste}{a)}
 \setlength{\itemsep}{2mm}
\item Quelles sont les valeurs prises par $Z$ ?

\item Calculer $p_{0} = P\left(\Ev{Z = 0}\right),p_{1} = P\left(\Ev{Z =
1}\right)$. Montrer que $p_{1} = npq^{2n--2}(1 + q).$

\item Calculer la probabilité conditionnelle $P\left(\Ev{(Y = h)/(X =
k)}\right)\left(\Ev{Y = h}\right)/\left(\Ev{X = k}\right))$, pour $k$
appartenant à $\{0,1,\dots,n\}$ et $h$ à $\{0,1,\dots,n-k\}$.

\item Démontrer $P\left(\Ev{Z = s}\right) = \Sum{k = 0}{s}P(\left(\Ev{X
= k}\right)\cap \left(\Ev{Y = s-k}\right)).$

\item Calculer $p_{s} = P\left(\Ev{Z = s}\right)$, et montrer que $Z$
suit une loi binomiale de
paramètres $n$ et $p(1 + q)$. \\
(On pourra vérifier : $C_{n}{k}C_{n-k}{s-k} = C_{n}{s}C_{s}{k}$)
\end{noliste}
\end{noliste}

\section*{EXERCIC\E\ 4}

\begin{noliste}{1.}
 \setlength{\itemsep}{4mm}
\item Vérifier : $\forall x\in \lbrack 0, + \infty \lbrack,\quad 0\leq
\ln (1 + x)\leq x.$ En déduire la limite quand l'entier $n$ tend vers $
+ \infty $ de $\dint{0}{1}\ln (1 + x^{n})dx$.

\item Soit u la suite réelle définie par $u_{n} =
\dint{0}{1}\dfrac{x^{n}}{1 + x^{n}}dx$. Montrer que pour tout entier
naturel $n$ non nul : 
\[
u_{n} = \dfrac{\ln (2)}{n}-\dfrac{1}{n}\dint{0}{1}\ln (1 + x^{n})dx
\]
(On pourra utiliser une intégration par parties.)\\
En déduire la limite de $u_{n}$ et celle de $n.u_{n}$ quand n tend vers
+ $\infty $.
\end{noliste}

\label{fin}

\end{document}

