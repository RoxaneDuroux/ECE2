\documentclass[11pt]{article}%
\usepackage{geometry}%
\geometry{a4paper,
 lmargin = 2cm,rmargin = 2cm,tmargin = 2.5cm,bmargin = 2.5cm}

\input{../../macros.tex}

\pagestyle{fancy} %
\lhead{ECE2 \hfill Mathématiques\\
} %
\chead{\hrule} %
\rhead{} %
\lfoot{} %
\cfoot{} %
\rfoot{\thepage} %

\renewcommand{\headrulewidth}{0pt}% : Trace un trait de séparation
 % de largeur 0,4 point. Mettre 0pt
 % pour supprimer le trait.

\renewcommand{\footrulewidth}{0.4pt}% : Trace un trait de séparation
 % de largeur 0,4 point. Mettre 0pt
 % pour supprimer le trait.

\setlength{\headheight}{14pt}

\title{\bf \vspace{-2cm} EML 2010} %
\author{} %
\date{} %
\begin{document}

\maketitle %
\vspace{-1.4cm}\hrule %
\thispagestyle{fancy}

\vspace*{.2cm}


\begin{center}
\underline{\bf Exercice 1}
\end{center}
{\bf Partie I : Un endomorphisme de l'espace vectoriel des matrices
carrées d'ordre 2}\\

\begin{noliste}{$\sbullet$}
\item On note $\M{2}$ l'espace vectoriel des matrices carrées d'ordre
2.
\item On note : $A = 
\begin{smatrix}
0 & 2\\
2 & 3
\end{smatrix}, \quad F = 
\begin{smatrix}
1 & 0\\
0 & 0
\end{smatrix}, \quad G = 
\begin{smatrix}
0 & 1\\
1 & 0
\end{smatrix}, \quad H = 
\begin{smatrix}
0 & 0\\
0 & 1
\end{smatrix}
$.
\item On note ${\bf \mathfrak{S}_{2}}$ l'ensemble des matrices carrées
symétriques d'ordre 2.
\end{noliste}

\begin{noliste}{1.}
 \setlength{\itemsep}{4mm}
\item Calculer $AFA$, $AGA$, $AHA$.
\item Montrer que ${\bf \mathfrak{S}_{2}}$ est un sous-espace vectoriel
de $\M{2}$ et que $(F,G,H)$ est une base de ${\bf \mathfrak{S}_{2}}$.
Déterminer la dimension de ${\bf \mathfrak{S}_{2}}$.\\

On note $u$ l'application qui, à chaque matrice $S$ de ${\bf
\mathfrak{S}_{2}}$, associe la matrice $u(S) = ASA$.\\

\item 
\begin{noliste}{a)}
 \setlength{\itemsep}{2mm}
\item Montrer : $\forall s\in {\bf \mathfrak{S}_{2}}, u(S)\in {\bf
\mathfrak{S}_{2}}$.
\item Montrer que $u$ est un endomorphisme de l'espace vectoriel ${\bf
\mathfrak{S}_{2}}$.
\item Donner la matrice de $u$ dans la base $(F,G,H)$ de ${\bf
\mathfrak{S}_{2}}$.\\
\end{noliste}
\end{noliste}

{\bf Partie 2 : Réduction d'une matrice carrée d'ordre 3}\\
\\
On note : $I = 
\begin{smatrix}
1 & 0 & 0\\
0 & 1 & 0\\
0 & 0 & 1
\end{smatrix}, \quad M = 
\begin{smatrix}
0 & 0 & 4\\
0 & 4 & 6\\
4 & 12 & 9
\end{smatrix}, \quad D = 
\begin{smatrix}
-4 & 0 & 0\\
0 & 1 & 0\\
0 & 0 & 16
\end{smatrix}
$
\begin{noliste}{1.}
 \setlength{\itemsep}{4mm}
\item Vérifier que -4, 1, 16 sont valeurs propres de $M$ et déterminer,
pour chacune de celles-ci, une base du sous-espace propre associé.
Est-ce que $M$ est diagonalisable ?
\item Déterminer une matrice $P$ carrée d'ordre 3, inversible, de
première ligne égale à $\begin{smatrix}
4 & 4 & 1
\end{smatrix}
$, telle que $M = PDP^{-1}$.
\item Vérifier que $(D + 4I)(D-I)(D-16I)$ est la matrice nulle.
\item En déduire : $\quad M^{3} = 13M^{2} + 52M-64I$.
\item Établir : $u^{3} = 13u^{2} + 52u-64e$, où $e$ désigne
l'application indentité de ${\bf \mathfrak{S}_{2}}$ et où $u$ a été
définie dans la partie I.\\
\\
\end{noliste}

\newpage

\begin{center}
\underline{\bf Exercice 2 }
\end{center}

On note $f :\R\rightarrow \R$ l'application de classe $C^{2}$, définie,
pour tout $x\in \R$, par : 
\[
f(x) = x-\ln(1 + x^{2})
\]
et $C$ la courbe représentative de $f$ dans un repère orthonormé.\\
On donne la valeur approchée : $\ln(2)\approx 0,69$.\\
\\

{\bf Partie I : Étude de $f$ et tracé de $C$}\\
\begin{noliste}{1.}
 \setlength{\itemsep}{4mm}
\item 
\begin{noliste}{a)}
 \setlength{\itemsep}{2mm}
\item Calculer, pour tout $x\in \R, f'(x)$.
\item En déduire le sens de variation de $f$.
\item Calculer, pour tout $x\in \R$, $f''(x)$.
\end{noliste}
\item Déterminer la limite de $f$ en $-\infty$ et la limite de $f$ en $
+ \infty$.
\item Déterminer la nature des branches infinies de $C$.
\item Montrer que $C$ admet deux points d'inflexion dont on déterminera
les coordonnées.
\item Tracer $C$. On utilisera un repère orthonormé d'unité graphique 2
centimètres, et on précisera la tangente à $C$ en l'origine et en
chacun des points d'inflexion.
\item Calculer $\dint{0}{1}xf(x)dx$. \\
A cet effet, on pourra utiliser le changement de variable défini par $t
= 1 + x^{2}$.\\
\end{noliste}

{\bf Partie II : Étude d'une suite et d'une série associées à $f$}\\
\\
On considère la suite $(u_{n})_{n\geq 0}$ définie par $u_{0} = 1$ et 
\[
\forall n\in \N, \quad u_{n + 1} = f(u_{n})
\]
\begin{noliste}{1.}
 \setlength{\itemsep}{4mm}
\item Montrer que $(u_{n})_{n\geq 0}$ est décroissante.
\item Établir que la suite $(u_{n})_{n\geq 0}$ converge et déterminer
sa limite.
\item Écrire un programme en -\Scilab{} qui calcule et affiche un
entier $n$ tel que $u_{n}\leq 10^{-3}$.
\item
\begin{noliste}{a)}
 \setlength{\itemsep}{2mm}
\item Établir : $\forall x\in [0;1], \quad f(x)\leq
x-\dfrac{1}{2}x^{2}$.
\item En déduire : $\forall n\in \N, u_{n}{2}\leq2(u_{n}-u_{n + 1})$.
\item Démontrer que la série $\Sum{n\geq 0}u_{n}{2}$ converge.\\
\end{noliste}
\end{noliste}

{\bf Partie III : Étude d'une fonction de deux variables réelles
associée à $f$}\\

On considère l'application $F :\R^{2}\rightarrow`\R$, définie, pour
tout $(x,y)\in \R^{2}$, par : 
\[
F(x,y) = f(x + y)-f(x)-f(y)
\]
\begin{noliste}{1.}
 \setlength{\itemsep}{4mm}
\item Montrer que $F$ est de classe $C^{1}$ sur $\R^{2}$ et exprimer,
pour tout $(x,y)\in \R^{2}$, les dérivées partielles premières de $F$
en $(x,y)$ à l'aide de $f'$, $x$ et $y$.
\item Résoudre dans $\R^{2}$ le système $\left\{ 
\begin{array}{ll}
 f'(x) = f'(y)\\
f'(x + y) = f'(x)
\end{array}
\right.$ En déduire les points critiques de $F$.
\item Est-ce que $F$ admet un minimum local ?\\
\\
\end{noliste}

\newpage

\begin{center}
\underline{\bf Exercice 3}
\end{center}

{\bf Les deux parties sont indépendantes}\\
\\
{\bf Partie 1}\\
\\
Une gare dispose de deux guichets. Trois clients notés $C_{1}, C_{2},
C_{3}$ arrivent en même temps. Les clients $C_{1}$ et $C_{2}$ se font
servir tandis que le client $C_{3}$ attend puis effectue son opération
dès que l'un des deux guichets se libère.\\
\\
On définit $X_{1}, X_{2}, X_{3}$ les variables aléatoires égales à la
durée d'opération des clients $C_{1}, C_{2}, C_{3}$ respectivement. Ces
durées sont mesurées en minutes et arrondies à l'unité supérieure ou
égale. On suppose que les variables $X_{1}, X_{2}, X_{3}$ suivent la
loi géométrique de paramètre $p$, $p\in \ ]0;1[$ et qu'elles sont
indépendantes. On note $q = 1-p$.\\

On note $A$ l'évènement : "$C_{3}$ termine en dernier son opération".\\
Ainsi l'évènement $A$ est égal à l'évènement : $(\min(X_{1},X_{2}) +
X_{3}>\max(X_{1},X_{2}))$.\\
On se propose de calculer la probabilité de $A$.\\

{\bf Partie I}\\
\begin{noliste}{1.}
 \setlength{\itemsep}{4mm}
\item Rappeler la loi de $X_{1}$ ainsi que son espérance E($X_{1}$) et
sa variance V($X_{1}$).\\
\\
On définit la variable aléatoire $\Delta = |X_{1}-X_{2}|$.\\
\item Calculer la probabilité $P\left(\Ev{\Delta = 0}\right)$.
\item Soit $n$ un entier naturel non nul.
\begin{noliste}{a)}
 \setlength{\itemsep}{2mm}
\item Justifier : $P\left(\Ev{X_{1}-X_{2} = n}\right) = \Sum{k = 1}{+
\infty}P\left(\Ev{X_{2} = k}\right)P\left(\Ev{X_{1} = n + k}\right)$.
\item En déduire : $P\left(\Ev{\Delta = n}\right) = \dfrac{2pq^{n}}{1 +
q}$.
\end{noliste}
\item
\begin{noliste}{a)}
 \setlength{\itemsep}{2mm}
\item Montrer que $\Delta$ admet une espérance E($\Delta$) et la
calculer.
\item Montrer : E($(X_{1}-X_{2})^{2}$) = 2\V($X_{1}$). En déduire que
$\Delta$ admet une variance $\V(\Delta)$ et la calculer.
\end{noliste}
\item Montrer que l'évènement $A$ est égal à l'évènement
$(X_{3}>\Delta)$.
\item 
\begin{noliste}{a)}
 \setlength{\itemsep}{2mm}
\item En déduire : P\left(\Ev{$A$}\right) = $\Sum{k = 0}{+
\infty}\text{P}(\Delta = k)\text{P}(X_{3}>k)$.
\item Exprimer $P\left(\Ev{A}\right)$ à l'aide de $p$ et $q$.\\
\end{noliste}
\end{noliste}

{\bf Partie II}\\
\\
Dans cette partie, $X$ est une variable aléatoire suivant la loi
géométrique de paramètre $p$, $p\in \ ]0;1[$ et $Y$ est une variable
aléatoire suivant la loi exponentielle de paramètre $\lambda$,
$\lambda\in \ ]0; + \infty[$. On note $q = 1-p$.\\
\\
On suppose que $X$ et $Y$ sont indépendantes, c'est à dire : 
\[
\forall k\in \N^*, \quad \forall t\in [0; + \infty[, \quad \text{P}((X
= k)\cap(Y\leq t )) = \text{P}(X = k)\text{P}(Y\leq t)
\]
\begin{noliste}{1.}
 \setlength{\itemsep}{4mm}
\item Rappeler une densité de $Y$ ainsi que son espérance et sa
variance.
\item On définit la variable aléatoire $Z = \dfrac{Y}{X}$.
\begin{noliste}{a)}
 \setlength{\itemsep}{2mm}
\item Montrer : $\forall t\in [0; + \infty[, \quad P\left(\Ev{Z\geq
t}\right) = \Sum{k = 1}{+ \infty}\text{P}(X = k)\text{P}(Y\geq kt)$.
\item En déduire : $\forall t\in [0; + \infty[, \quad P\left(\Ev{Z\geq
t}\right) = \dfrac{pe^{-\lambda t}}{1-qe^{-\lambda t}}$.
\item Montrer que la variable aléatoire $Z$ admet une densité et
déterminer une densité de $Z$.
\end{noliste}
\end{noliste}





\end{document}