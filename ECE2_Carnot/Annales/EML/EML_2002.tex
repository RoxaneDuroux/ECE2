\documentclass[11pt]{article}%
\usepackage{geometry}%
\geometry{a4paper,
 lmargin = 2cm,rmargin = 2cm,tmargin = 2.5cm,bmargin = 2.5cm}

\input{../../macros.tex}

\pagestyle{fancy} %
\lhead{ECE2 \hfill Mathématiques\\
} %
\chead{\hrule} %
\rhead{} %
\lfoot{} %
\cfoot{} %
\rfoot{\thepage} %

\renewcommand{\headrulewidth}{0pt}% : Trace un trait de séparation
 % de largeur 0,4 point. Mettre 0pt
 % pour supprimer le trait.

\renewcommand{\footrulewidth}{0.4pt}% : Trace un trait de séparation
 % de largeur 0,4 point. Mettre 0pt
 % pour supprimer le trait.

\setlength{\headheight}{14pt}

\title{\bf \vspace{-2cm} EML 2002} %
\author{} %
\date{} %
\begin{document}

\maketitle %
\vspace{-1.4cm}\hrule %
\thispagestyle{fancy}

\vspace*{.2cm}


% DEBUT DU DOC À MODIFIER : tout virer jusqu'au début de l'exo

%Définition et changement de valeurs de
compteurs%newcounter{cpt1}{section} compteur cpt1 remis à 0 à chaque
aumentation par stepcounter du compteur section%setcounter{cpt1}{3} on
met le compteur à 3%addtocounter{cpt1}{5} on ajoute 5 au compteur%
stepcounter{cpt1} on ajoute 1% ifthenelse{test}{alors}{sinon} (page
206) pour subordonner à une condition % whiledo{test}{commande} pour
faire une boucle (page 206 aussi) % value{cpt1} pour noter dans le
document la valeur de cpt1 
%Définition définitive d'opérateurs
mathématiques\newcommand{\ch}{\operatorname{ch}} 
\newcommand{\sh}{\operatorname{sh}}
\renewcommand{\tanh}{\operatorname{th}}
\renewcommand{\sinh}{\operatorname{sh}}
\renewcommand{\cosh}{\operatorname{ch}}
\newcommand{\argsh}{\operatorname{argsh}}
\newcommand{\argch}{\operatorname{argch}}
\newcommand{\argth}{\operatorname{argth}}
\newcommand{\ker}{\operatorname{Ker}}
\renewcommand{\im}{\operatorname{Im}}
\newcommand{\rg}{\operatorname{rg}}
\newcommand{\Id}{\operatorname{Id}}
\newcommand{\id}{\operatorname{id}}
\renewcommand{\leq}{\leq}
\renewcommand{\geq}{\geq }

%Définition de nouvelles couleurs : rgb(trois paramètres red green blue
entre 0 et 1); cmyk (quatre cyan magenta yellow black) entre 0 et 1;
gray (entre 0 et 1) et black, white, red, green, blue, cyan, magenta,
yellow% definecolor{0gris}{gray}{0.8} 
% Nouvelle commande pour encadrer le titre car shabox ne veut que d'une
seule ligne; ATTENTION A LA TAILLE; petite différence avec shadowbox ou
doublebox, voire fcolorbox ou colorbox (au lieu de shabox; laisser le
parbox tranquille sauf pour la taille de la boîte
\newcommand{\Tbox}[1]{\begin{center} \shabox{\parbox{0.6
\linewidth}{#1}} \end{center}} %[1] pour 1 paramètre ; #1 pour ce que
fait le 1er paramètre; entre accolades ce que fait la commande
%Mise en page en mode fancy : en-têtes et pieds de pages puis
définition des en-têtes et pieds de pages\pagestyle{fancy}
\lhead{ECE 2 - Mathématiques \\
Quentin Dunstetter - ENC-Bessières 2011$\backslash$2012}
\chead{}
\rhead{EML 2002}
\rfoot[ \ \thepage]{\thepage}
\cfoot{}
\lfoot{}
\thispagestyle{fancy} %Mise en page de la 1ère page en mode fancy
%Trait en bas et en haut de la page (entre en-tête et texte et texte et
pied de page)\renewcommand{\footrulewidth}{0.4pt}
\renewcommand{\headrulewidth}{0.4pt}


%DEBUT DU DOCUMENT\vspace*{3cm}

\begin{center}
{\LARG\E\textbf{BANQUE COMMUNE D'ÉPREUVES}}



{\large \textsc{CONCOURS D ADMISSION DE 2002}}



{\large \textbf{Concepteur : EML}}



\rule{2.39cm}{0.05cm}



{\Large \textbf{OPTION ÉCONOMIQUE}}



{\Large \textbf{MATHÉMATIQUES }}



{\Large Lundi 9 mai, de 14h à 18h}



\rule{2.39cm}{0.05cm}
\end{center}

\textit{La présentation, la lisibilité, l'orthographe, la qualité
de la rédaction, la clarté et la précision des raisonnements
entreront pour une part importante dans l'appréciation des copies.}

\textit{Les candidats sont invités à \textbf{encadrer} dans la mesure
du possible les résultats de leurs calculs.}

\textit{Ils ne doivent faire usage d'aucun document. L'utilisation de
toute
calculatrice et de tout matériel électronique est interdite. Seule
l'utilisation d'une règle graduée est autorisée.}

\textit{Si au cours de l'épreuve, un candidat repère ce qui lui semble
être une erreur d'énoncé, il la signalera sur sa copie et
poursuivra sa composition en expliquant les raisons des initiatives
qu'il sera
amené à prendre.}

\vspace*{3cm}


\begin{center}
{\LARGE Exercice 1}
\end{center}

\noindent On considère les deux matrices carrées réelles d'ordre
quatre suivantes : 
\[
I = \left( 
\begin{array}{rrrr}
1 & 0 & 0 & 0 \\
0 & 1 & 0 & 0 \\
0 & 0 & 1 & 0 \\
0 & 0 & 0 & 1
\end{array}
\right) \,\quad K = \left( 
\begin{array}{rrrr}
1 & 1 & -1 & -3 \\
1 & 1 & 1 & -2 \\
0 & -1 & 0 & 1 \\
1 & 1 & 0 & -2
\end{array}
\right) 
\]
\textsl{Les questions 2 et 3 sont indépendantes entre elles.}

\begin{noliste}{1.}
 \setlength{\itemsep}{4mm}
\item 
\begin{noliste}{a)}
 \setlength{\itemsep}{2mm}
\item Calculer $K^{2}$.

\item En déduire que la matrice $K$ est inversible et déterminer
$K^{-1}$.

\item Montrer que la matrice $K$ n'admet aucune valeur propre réelle.
\end{noliste}

\item Soient $a$ et $b$ deux nombres réels. On note $M$ la matrice
définie par $M = aI + bK$.

\begin{noliste}{a)}
 \setlength{\itemsep}{2mm}
\item Montrer : $\quad M^{2} = -(a^{2} + b^{2})I + 2aM$.

\item En déduire que, si $(a,b)\neq (0,0)$, alors la matrice $M$ est
inversible, et exprimer son inverse comme combinaison linéaire de $I$
et 
$M$.

\item \textsl{Application :} donner l'inverse de la matrice 
\[
\left( 
\begin{array}{rrrr}
1 + \sqrt{2} & 1 & -1 & -3 \\
1 & 1 + \sqrt{2} & 1 & -2 \\
0 & -1 & \sqrt{2} & 1 \\
1 & 1 & 0 & -2 + \sqrt{2}
\end{array}
\right) 
\]
\end{noliste}

\item On note $\mathcal{B} = (e_{1},e_{2},e_{3},e_{4})$ la base
canonique de $\R^{4}$, et $f$ l'endomorphisme de $\R^{4}$ associé à
la matrice $K$ relativement à la base $\mathcal{B}$. On considère
les quatre éléments suivants de $\R^{4}$ : 
\[
v_{1} = e_{1}\qquad v_{2} = f(e_{1})\qquad v_{3} = e_{3}\qquad v_{4} =
f(e_{3}) 
\]

\begin{noliste}{a)}
 \setlength{\itemsep}{2mm}
\item Montrer que la famille $\mathcal{C} = (v_{1},v_{2},v_{3},v_{4})$
est une
base de $\R^{4}$.

\item Exprimer $f(v_{1})$, $f(v_{2})$, $f(v_{3})$, $f(v_{4})$ en
fonction de 
$v_{1}$, $v_{2}$, $v_{3}$, $v_{4}$ et en déduire la matrice $K^{\prime
}$
associée à $f$ relativement à la base $\mathcal{C}$.

\item Déterminer la matrice de passage $P$ de la base $\mathcal{B}$ à
la base $\mathcal{C}$.

\item Rappeler l'expression de $K^{\prime }$ en fonction de $K$, $P$ et
$P^{-1}$.\newpage
\end{noliste}
\end{noliste}

\begin{center}
{\LARGE Exercice 2}
\end{center}

\noindent On considère, pour tout $n \in \N^*$, la
fonction polynomiale $P_{n} \ : \ [0, + \infty[ \ \longrightarrow \R$
définie pour tout $x \in [0, + \infty[$, par : 
\[
P_{n}(x) = \Sum{k = 1}{2n}{\frac{(-1)^{k} x^{k} }{k}} = -x +
{\frac{x^{2}}{2}} + \ldots + {\frac{-x^{2n-1} }{2n- 1}} +
{\frac{x^{2n}}{2n}} 
\]
\textbf{I. Étude des fonctions polynomiales $\mathbf{P_{n}}$}

\begin{noliste}{1.}
 \setlength{\itemsep}{4mm}
\item Montrer, pour tout $n\in \N^{\ast }$ et tout $x\in \lbrack
0, + \infty \lbrack $ : 
\[
P_{n}{\prime }(x) = {\frac{x^{2n}-1}{x + 1}}\quad \text{où
}P_{n}{\prime }\text{ désigne la dérivée de }P_{n}
\]

\item Établir, pour $n\in \N^{\ast }$, les variations de $P_{n}$
sur $[0, + \infty \lbrack $ et dresser le tableau de variations de
$P_{n}$.

\item Montrer, pour tout $n\in \N^{\ast }$ : $P_{n}(1)<0$.

\item 
\begin{noliste}{a)}
 \setlength{\itemsep}{2mm}
\item Vérifier, pour tout $n\in \N^{\ast }$ et tout $x\in
\lbrack 0, + \infty \lbrack $ : 
\[
P_{n + 1}(x) = P_{n}(x) + x^{2n + 1}\left( -{\frac{1}{2n + 1}} +
{\frac{x}{2n + 2}}\right) 
\]

\item En déduire, pour tout $n\in \N^{\ast }$ : $P_{n}(2)\geq 0$.
\end{noliste}

\item Montrer que, pour tout $n\in \N^{\ast }$, l'équation $P_{n}(x) =
0$, d'inconnue $x\in \lbrack 1, + \infty \lbrack $, admet une
solution et une seule notée $x_{n}$, et que : 
\[
1<x_{n}\leq 2
\]

\item Écrire un programme en langage \Scilab{} qui calcule une valeur
approchée décimale de $x_{2}$ à $10^{-3}$ près.
\end{noliste}

\textbf{II. Limite de la suite $\mathbf{\ (x_{n})_{n \in \N^*}}$}

\begin{noliste}{1.}
 \setlength{\itemsep}{4mm}
\item Établir, pour tout $n\in \N^{*}$ et tout $x\in [0, + \infty [
$ : 
\[
P_{n}(x) = \dint{0}{x}{\frac{t^{2n}-1}{t + 1}}\ dt 
\]

\item En déduire, pour tout $n\in \N^{*}$ : 
\[
\dint{1}{x_{n}}{\frac{t^{2n}-1}{t + 1}}\ dt =
\dint{0}{1}{\frac{1-t^{2n}}{t + 1}}\ dt 
\]

\item Démontrer, pour tout $n\in \N^{*}$ et tout $t\in
[1, + \infty [$ : 
\[
t^{2n}-1\geq n(t^{2}-1) 
\]

\item En déduire, pour tout $n\in \N^{*}$ : 
\[
\dint{1}{x_{n}}{\frac{t^{2n}-1}{t + 1}}\ dt\geq
{\frac{n}{2}}(x_{n}-1)^{2}\, 
\]
puis : 
\[
0<x_{n}-1\leq {\frac{\sqrt{2\ln 2}}{\sqrt{n}}} 
\]

\item Conclure quant à la convergence et à la limite de la suite
$(x_{n})_{n\in \N^{*}}$.
\end{noliste}

\begin{center}
{\LARGE Exercice 3}
\end{center}

\begin{noliste}{1.}
 \setlength{\itemsep}{4mm}
\item \textbf{Étude préliminaire}

\noindent On admet, pour tout entier $k$ et pour tout $x\in [0,1[$
que la série $ \Sum{n\geq k}\binom{n}{k}x^{n}$ est
convergente et on note $s_{k}(x) =  \Sum{n = k}{+ \infty
}\binom{n}{k}x^{n}$.

\begin{noliste}{a)}
 \setlength{\itemsep}{2mm}
\item Vérifier, pour tout réel $x$ de $[0,1[$ : 
\[
s_{0}(x) = {\frac{1}{1-x}}\qquad \hbox{ et }\qquad s_{1}(x) =
{\frac{x}{(1-x)^{2}}} 
\]

\item Pour tout couple d'entiers naturels $(n,k)$ tels que $n<k$,
montrer : 
\[
\binom{n + 1}{k + 1} = \binom{n}{k} + \binom{n}{k + 1} 
\]

\item Pour tout entier naturel $k$ et pour tout réel $x$ de $[0,1[$,
déduire de la question précédente : 
\[
s_{k + 1}(x) = xs_{k}(x) + xs_{k + 1}(x) 
\]

\item Montrer par récurrence : 
\[
\forall k\in \N\quad \forall x\in [0,1[ \ \qquad s_{k}(x) =
{\frac{x^{k}}{(1-x)^{k + 1}}} 
\]
\newpage
\end{noliste}

\item \textbf{Étude d'une expérience aléatoire}

\noindent On considère une urne contenant une boule noire et
quatre boules blanches. On effectue l'expérience aléatoire suivante :

\begin{noliste}{$\sbullet$}
\item On commence par tirer des boules de l'urne une à une avec remise
jusqu'à obtenir la boule noire (que l'on remet aussi dans l'urne ).\\
On définit la variable aléatoire $N$ égale au nombre de tirages
avec remise nécessaires pour obtenir la boule noire.

\item Puis, si $N$ prend une valeur entière positive non nulle notée 
$n$, on réalise alors une seconde série de $n$ tirages dans l'urne,
toujours avec remise.\\
On définit la variable aléatoire $X$ égale au nombre de fois où la
boule noire a été obtenue dans cette seconde série de
tirages.
\end{noliste}

\begin{noliste}{a)}
 \setlength{\itemsep}{2mm}
\item Déterminer la loi de la variable aléatoire $N$. Donner son
espérance.

\item Soient $k\in \N$ et $n\in \N^{\ast }$. Déterminer
la probabilité conditionnelle $P_{N = n}(X = k)$.

\item Vérifier que : \quad $ P\left(\Ev{X = 0}\right) = {\frac{4}{9}}$.

\item En utilisant l'étude préliminaire, montrer : 
\[
\forall k\in \N^{*}\qquad P\left(\Ev{X = k}\right) =
{\frac{25}{36}}\left( {\frac{4}{9}}\right) ^{k} 
\]

\item Montrer que $X$ admet une espérance $\E(X)$ et calculer $\E(X)$.

\item Montrer : $\qquad \forall k\in \N\qquad  P\left(\Ev{X\leq
k}\right) = 1-{\frac{5}{9}}\left( {\frac{4}{9}}\right) ^{k}$.
\end{noliste}

\item \textbf{Étude d'une variable à densité}

\noindent On note $a =  -{\frac{\ln 9-\ln 5}{\ln 9-\ln 4}}$ et on
définit la fonction $F$ sur $\R$ par 
\[
\left\{ 
\begin{array}{ll}
F(x) = 1- {\frac{5}{9}}\left( {\frac{4}{9}}\right) ^{x} & 
\hbox{ si
}x\in [a, + \infty [ \ \\
F(x) = 0 & \text{sinon}
\end{array}
\right. 
\]
On rappelle : $\qquad \forall x\in \R\quad  \left( {\frac{4}{9}}\right)
^{x} = e^{x\ln {\frac{4}{9}}}$.

\begin{noliste}{a)}
 \setlength{\itemsep}{2mm}
\item Montrer que $F$ est la fonction de répartion d'une variable à
densité, notée $Y$.

\item Déterminer une densité $f$ de $Y$.

\item Déterminer une primitive de la fonction $g$ définie par $ =
x\,e^{x\ln {\frac{4}{9}}}$.

\item Montrer que $Y$ admet une espérance $\E\left( Y\right) $ et
calculer $\E(Y)$.
\end{noliste}
\end{noliste}

\end{document}

