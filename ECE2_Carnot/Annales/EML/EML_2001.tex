\documentclass[11pt]{article}%
\usepackage{geometry}%
\geometry{a4paper,
 lmargin = 2cm,rmargin = 2cm,tmargin = 2.5cm,bmargin = 2.5cm}

\input{../../macros.tex}

\pagestyle{fancy} %
\lhead{ECE2 \hfill Mathématiques\\
} %
\chead{\hrule} %
\rhead{} %
\lfoot{} %
\cfoot{} %
\rfoot{\thepage} %

\renewcommand{\headrulewidth}{0pt}% : Trace un trait de séparation
 % de largeur 0,4 point. Mettre 0pt
 % pour supprimer le trait.

\renewcommand{\footrulewidth}{0.4pt}% : Trace un trait de séparation
 % de largeur 0,4 point. Mettre 0pt
 % pour supprimer le trait.

\setlength{\headheight}{14pt}

\title{\bf \vspace{-2cm} EML 2001} %
\author{} %
\date{} %
\begin{document}

\maketitle %
\vspace{-1.4cm}\hrule %
\thispagestyle{fancy}

\vspace*{.2cm}


% DEBUT DU DOC À MODIFIER : tout virer jusqu'au début de l'exo

%Définition et changement de valeurs de
compteurs%newcounter{cpt1}{section} compteur cpt1 remis à 0 à chaque
aumentation par stepcounter du compteur section%setcounter{cpt1}{3} on
met le compteur à 3%addtocounter{cpt1}{5} on ajoute 5 au compteur%
stepcounter{cpt1} on ajoute 1% ifthenelse{test}{alors}{sinon} (page
206) pour subordonner à une condition % whiledo{test}{commande} pour
faire une boucle (page 206 aussi) % value{cpt1} pour noter dans le
document la valeur de cpt1 
%Définition définitive d'opérateurs
mathématiques\newcommand{\ch}{\operatorname{ch}} 
\newcommand{\sh}{\operatorname{sh}}
\renewcommand{\tanh}{\operatorname{th}}
\renewcommand{\sinh}{\operatorname{sh}}
\renewcommand{\cosh}{\operatorname{ch}}
\newcommand{\argsh}{\operatorname{argsh}}
\newcommand{\argch}{\operatorname{argch}}
\newcommand{\argth}{\operatorname{argth}}
\newcommand{\ker}{\operatorname{Ker}}
\renewcommand{\im}{\operatorname{Im}}
\newcommand{\rg}{\operatorname{rg}}
\newcommand{\Id}{\operatorname{Id}}
\newcommand{\id}{\operatorname{id}}
\renewcommand{\leq}{\leq}
\renewcommand{\geq}{\geq }

%Définition de nouvelles couleurs : rgb(trois paramètres red green blue
entre 0 et 1); cmyk (quatre cyan magenta yellow black) entre 0 et 1;
gray (entre 0 et 1) et black, white, red, green, blue, cyan, magenta,
yellow% definecolor{0gris}{gray}{0.8} 
% Nouvelle commande pour encadrer le titre car shabox ne veut que d'une
seule ligne; ATTENTION A LA TAILLE; petite différence avec shadowbox ou
doublebox, voire fcolorbox ou colorbox (au lieu de shabox; laisser le
parbox tranquille sauf pour la taille de la boîte
\newcommand{\Tbox}[1]{\begin{center} \shabox{\parbox{0.6
\linewidth}{#1}} \end{center}} %[1] pour 1 paramètre ; #1 pour ce que
fait le 1er paramètre; entre accolades ce que fait la commande
%Mise en page en mode fancy : en-têtes et pieds de pages puis
définition des en-têtes et pieds de pages\pagestyle{fancy}
\lhead{ECE 2 - Mathématiques \\
Quentin Dunstetter - ENC-Bessières 2011$\backslash$2012}
\chead{}
\rhead{EML 2001}
\rfoot[ \ \thepage]{\thepage}
\cfoot{}
\lfoot{}
\thispagestyle{fancy} %Mise en page de la 1ère page en mode fancy
%Trait en bas et en haut de la page (entre en-tête et texte et texte et
pied de page)\renewcommand{\footrulewidth}{0.4pt}
\renewcommand{\headrulewidth}{0.4pt}


%DEBUT DU DOCUMENT\vspace*{3cm}

\begin{center}
{\LARG\E\textbf{BANQUE COMMUNE D'ÉPREUVES}}



{\large \textsc{CONCOURS D ADMISSION DE 2001}}



{\large \textbf{Concepteur : EML}}



\rule{2.39cm}{0.05cm}



{\Large \textbf{OPTION ÉCONOMIQUE}}



{\Large \textbf{MATHÉMATIQUES }}



{\Large Lundi 9 mai, de 14h à 18h}



\rule{2.39cm}{0.05cm}
\end{center}

\textit{La présentation, la lisibilité, l'orthographe, la qualité
de la rédaction, la clarté et la précision des raisonnements
entreront pour une part importante dans l'appréciation des copies.}

\textit{Les candidats sont invités à \textbf{encadrer} dans la mesure
du possible les résultats de leurs calculs.}

\textit{Ils ne doivent faire usage d'aucun document. L'utilisation de
toute
calculatrice et de tout matériel électronique est interdite. Seule
l'utilisation d'une règle graduée est autorisée.}

\textit{Si au cours de l'épreuve, un candidat repère ce qui lui semble
être une erreur d'énoncé, il la signalera sur sa copie et
poursuivra sa composition en expliquant les raisons des initiatives
qu'il sera
amené à prendre.}

\vspace*{3cm}

\section*{Exercice 1}

\noindent On considère la matrice carrée réelle d'ordre quatre :
\[
A = \left(
\begin{array}{rrrr}
1 & 0 & 0 & -1 \\
1 & 0 & 0 & -1 \\
0 & 1 & 0 & -1 \\
0 & 0 & 1 & -1
\end{array}
\right)
\]
et l'endomorphisme $f$ de $\R^{4}$ dont la matrice dans la base
canonique $\mathcal{B} = (e_{1},e_{2},e_{3},e_{4})$ de $\R^{4}$ est
$A$.

\begin{noliste}{1.}
 \setlength{\itemsep}{4mm}
\item[ \ \textbf{1.}] Montrer que $A$ n'est pas inversible. En déduire
que $0$
est valeur propre de $A$.

\item[ \ \textbf{2.}]

\begin{noliste}{a)}
 \setlength{\itemsep}{2mm}
\item Calculer $A^{2}$, $A^{3}$, $A^{4}$.

\item Établir que $0$ est la seule valeur propre de $f$.

\item Déterminer la dimension du noyau de $f$.

\item Est-ce que $f$ est diagonalisable ?
\end{noliste}

\item[ \ \textbf{3.}] On note $\varepsilon_{1} = e_{1}$, $\varepsilon
_{2} = f(\varepsilon_{1}),\varepsilon_{3} =
f(\varepsilon_{2}),\varepsilon
_{4} = f(\varepsilon_{3})$, et $\mathcal{C} =
(\varepsilon_{1},\varepsilon
_{2},\varepsilon_{3},\varepsilon_{4})$.

\begin{noliste}{a)}
 \setlength{\itemsep}{2mm}
\item Montrer que $\mathcal{C}$ est une base de $\R^{4}$.

\item Déterminer la matrice $N$ de $f$ relativement à la base
$\mathcal{C}$
de $\R^{4}$.
\end{noliste}

\item[ \ \textbf{4.}] Existe-t-il un automorphisme $g$ de l'espace
vectoriel $\R^{4}$ tel que $g\circ f\circ g^{-1} = f^{2}$ ?
\end{noliste}

\section*{Exercice 2}

\noindent On considère l'application $f\ :[0; + \infty \lbrack
\longrightarrow
\R$, définie, pour tout $x$ de $[0; + \infty \lbrack $, par :
\[
f(x) = \left\{
\begin{array}{cl}
\dfrac{x}{e^{x}-1} & \text{si }x>0 \\
1 & \text{si }x = 0
\end{array}
\right.
\]

\begin{noliste}{1.}
 \setlength{\itemsep}{4mm}
\item 

\begin{noliste}{a)}
 \setlength{\itemsep}{2mm}
\item Montrer que $f$ est continue sur $[0; + \infty \lbrack $.

\item Montrer que $f$ est de classe $C^{1}$ sur $]0; + \infty \lbrack
$. Pour
tout $x\in \ ]0, + \infty \lbrack $, calculer $f^{\prime }(x)$.

\item Montrer que $f^{\prime }(x)$ tend vers $-\dfrac{1}{2}$ lorsque
$x$
tend vers $0$.

\item En déduire que $f$ est $C^{1}$ sur $[0; + \infty \lbrack $.
\end{noliste}

\item 

\begin{noliste}{a)}
 \setlength{\itemsep}{2mm}
\item Montrer que $f$ est de classe $C^{2}$ sur $]0; + \infty \lbrack $
et
que : $\forall x\in \ ]0; + \infty \lbrack \qquad f^{\prime \prime }(x)
= \dfrac{e^{x}}{(e^{x}-1)^{3}}(xe^{x}-2e^{x} + x + 2)$

\item Étudier les variations de la fonction $g\ :[0; + \infty \lbrack
\longrightarrow \R$, définie, pour tout $x$ de $[0; + \infty \lbrack $,
par :
\[
g(x) = xe^{x}-2e^{x} + x + 2
\]
En déduire : \qquad $\forall x\in \ ]0; + \infty \lbrack,\quad
f^{\prime \prime
}(x)>0$.

\item En déduire le sens de variation de $f$. On précisera la limite de
$f$
en $ + \infty $. Dresser le tableau de variation de $f$.

\item Tracer l'allure de la courbe représentative de $f$.
\end{noliste}

\item[ \ \textbf{3.}] On considère la suite $(u_{n})_{n\geq 0}$ définie
par $u_{0} = 0$ et : $\forall n\in \N,\quad u_{n + 1} = f(u_{n})$.

\begin{noliste}{a)}
 \setlength{\itemsep}{2mm}
\item Montrer :
\[
\forall x\in \lbrack 0; + \infty \lbrack,\quad |f^{\prime }(x)|\leq
\dfrac{1}{2}\quad \text{et}\quad 0\leq f(x)\leq 1
\]

\item Résoudre l'équation $f(x) = x$, d'inconnue $x\in \ ]0; + \infty
\lbrack $.

\item Montrer :
\[
\forall n\in \N\quad |u_{n + 1}-\ln 2|\leq
\dfrac{1}{2}|u_{n}-\ln 2|
\]

\item Établir que la suite $(u_{n})_{n\geq 0}$ converge et
déterminer sa limite.
\end{noliste}
\end{noliste}

\section*{Exercice 3}

\begin{noliste}{1.}
 \setlength{\itemsep}{4mm}
\item[ \ \textbf{1.}] Pour tout entier naturel $n$, on considère la
fonction $f_{n}\ :\ \R\longrightarrow \R$ définie par :
\[
\forall t\in \R,\quad f_{n}(t) = \left\{
\begin{array}{cl}
\dfrac{e^{-t}t^{n}}{n!} & \text{si }t>0 \\
0 & \text{si }t\leq 0
\end{array}
\right.
\]

\begin{noliste}{a)}
 \setlength{\itemsep}{2mm}
\item Soit $n\in \N$. Montrer que $\dlim{t\rightarrow + \infty
}t^{2}f_{n}(t) = 0$.\\
En déduire que l'intégrale $\dint{0}{+ \infty }f_{n}(t)\ dt$ est
convergente.

\item Montrer : \qquad $\forall n\in \N^{\ast },\quad \forall x\in
\lbrack 0; + \infty \lbrack,\qquad \dint{0}{x}f_{n}(t)\ dt =
-\dfrac{e^{-x}x^{n}}{n!} + \dint{0}{x}f_{n-1}(t)\ dt$.

\item En déduire : \qquad $\forall n\in \N,\qquad
\dint{0}{+ \infty}f_{n}(t)\ dt = 1$

\item Montrer que, pour tout entier naturel $n$, la fonction $f_{n}$
est la
densité de probabilité d'une variable aléatoire.
\end{noliste}

\item Pour tout entier naturel $n$, on définit la variable aléatoire
$X_{n}$
admettant $f_{n}$ pour densité de probabilité.

\begin{noliste}{a)}
 \setlength{\itemsep}{2mm}
\item Montrer que, pour tout entier naturel $n$, l'espérance
$\E(X_{n})$ et
la variance $\V(X_{n})$ vérifient :
\[
\E(X_{n}) = n + 1\qquad V(X_{n}) = n + 1
\]

\item Dans cette question, on suppose que $n = 4$. On donne les valeurs
approchées à $10^{-2}$ suivantes :
\[
\dint{0}{4}f_{4}(t)\ dt\simeq 0,37\qquad
\dint{0}{6}f_{4}(t)\ dt\simeq 0,71\qquad
\dint{0}{8}f_{4}(t)\ dt\simeq 0,90
\]
Tracer l'allure de la courbe représentative de la fonction de
répartition de $X_{4}$.\\
 Déterminer une valeur décimale
approchée de la probabilité $P\left(\Ev{X_{4}>4}\right)$ et une valeur
décimale
approchée de la probabilité $P\left(\Ev{4<X_{4}\leq 8}\right)$.
\end{noliste}

\item[ \ \textbf{3.}] Pour tout réel $t>0$, on définit la variable
aléatoire $Y_{t}$ égale au nombre de voitures arrivant à un péage
d'autoroute de
l'instant $0$ à l'instant $t$.\\
On suppose que la variable aléatoire $Y_{t}$ suit une loi de Poisson de
paramètre $t$.

\begin{noliste}{a)}
 \setlength{\itemsep}{2mm}
\item Rappeler, pour tout réel $t>0$, les valeurs de l'espérance et de
la
variance de $Y_{t}$.\\
Pour tout entier naturel $n$ non nul, on définit la variable aléatoire
réelle $Z_{n}$, prenant ses valeurs dans $\R^{+}$, égale à l'instant
d'arrivée de la $n^{i\grave{e}me}$ voiture au péage à partir de
l'instant $0$.

\item Soient $t\in \ ]0; + \infty \lbrack $ et $n\in \N^{\ast }$.\\
Justifier l'égalité de l'évènement $(Z_{n}\leq t)$ et de l'évènement
$(Y_{t}\geq n)$

\item En déduire, pour tout entier naturel $n$ non nul, la fonction de
répartition de la variable aléatoire réelle $Z_{n}$.

\item Montrer que, pour tout entier naturel $n$ non nul, la variable
aléatoire $Z_{n}$ admet $f_{n-1}$ comme densité de probabilité.
\end{noliste}
\end{noliste}

\begin{center}
{- \ FIN \ -}
\end{center}

\label{fin}

\end{document}

