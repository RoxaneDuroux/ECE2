\documentclass[11pt]{article}%
\usepackage{geometry}%
\geometry{a4paper,
 lmargin = 2cm,rmargin = 2cm,tmargin = 2.5cm,bmargin = 2.5cm}

\input{../../macros.tex}

\pagestyle{fancy} %
\lhead{ECE2 \hfill Mathématiques\\
} %
\chead{\hrule} %
\rhead{} %
\lfoot{} %
\cfoot{} %
\rfoot{\thepage} %

\renewcommand{\headrulewidth}{0pt}% : Trace un trait de séparation
 % de largeur 0,4 point. Mettre 0pt
 % pour supprimer le trait.

\renewcommand{\footrulewidth}{0.4pt}% : Trace un trait de séparation
 % de largeur 0,4 point. Mettre 0pt
 % pour supprimer le trait.

\setlength{\headheight}{14pt}

\title{\bf \vspace{-2cm} EML 2007} %
\author{} %
\date{} %
\begin{document}

\maketitle %
\vspace{-1.4cm}\hrule %
\thispagestyle{fancy}

\vspace*{.2cm}


% DEBUT DU DOC À MODIFIER : tout virer jusqu'au début de l'exo

%Définition et changement de valeurs de
compteurs%newcounter{cpt1}{section} compteur cpt1 remis à 0 à chaque
aumentation par stepcounter du compteur section%setcounter{cpt1}{3} on
met le compteur à 3%addtocounter{cpt1}{5} on ajoute 5 au compteur%
stepcounter{cpt1} on ajoute 1% ifthenelse{test}{alors}{sinon} (page
206) pour subordonner à une condition % whiledo{test}{commande} pour
faire une boucle (page 206 aussi) % value{cpt1} pour noter dans le
document la valeur de cpt1 
%Définition définitive d'opérateurs
mathématiques\newcommand{\ch}{\operatorname{ch}} 
\newcommand{\sh}{\operatorname{sh}}
\renewcommand{\tanh}{\operatorname{th}}
\renewcommand{\sinh}{\operatorname{sh}}
\renewcommand{\cosh}{\operatorname{ch}}
\newcommand{\argsh}{\operatorname{argsh}}
\newcommand{\argch}{\operatorname{argch}}
\newcommand{\argth}{\operatorname{argth}}
\newcommand{\ker}{\operatorname{Ker}}
\renewcommand{\im}{\operatorname{Im}}
\newcommand{\rg}{\operatorname{rg}}
\newcommand{\Id}{\operatorname{Id}}
\newcommand{\id}{\operatorname{id}}
\renewcommand{\leq}{\leq}
\renewcommand{\geq}{\geq }

%Définition de nouvelles couleurs : rgb(trois paramètres red green blue
entre 0 et 1); cmyk (quatre cyan magenta yellow black) entre 0 et 1;
gray (entre 0 et 1) et black, white, red, green, blue, cyan, magenta,
yellow% definecolor{0gris}{gray}{0.8} 
% Nouvelle commande pour encadrer le titre car shabox ne veut que d'une
seule ligne; ATTENTION A LA TAILLE; petite différence avec shadowbox ou
doublebox, voire fcolorbox ou colorbox (au lieu de shabox; laisser le
parbox tranquille sauf pour la taille de la boîte
\newcommand{\Tbox}[1]{\begin{center} \shabox{\parbox{0.6
\linewidth}{#1}} \end{center}} %[1] pour 1 paramètre ; #1 pour ce que
fait le 1er paramètre; entre accolades ce que fait la commande
%Mise en page en mode fancy : en-têtes et pieds de pages puis
définition des en-têtes et pieds de pages\pagestyle{fancy}
\lhead{ECE 2 - Mathématiques \\
Quentin Dunstetter - ENC-Bessières 2011$\backslash$2012}
\chead{}
\rhead{EML 2007}
\rfoot[ \ \thepage]{\thepage}
\cfoot{}
\lfoot{}
\thispagestyle{fancy} %Mise en page de la 1ère page en mode fancy
%Trait en bas et en haut de la page (entre en-tête et texte et texte et
pied de page)\renewcommand{\footrulewidth}{0.4pt}
\renewcommand{\headrulewidth}{0.4pt}


%DEBUT DU DOCUMENT\vspace*{3cm}

\begin{center}
{\LARG\E\textbf{BANQUE COMMUNE D'ÉPREUVES}}



{\large \textsc{CONCOURS D ADMISSION DE 2007}}



{\large \textbf{Concepteur : EML}}



\rule{2.39cm}{0.05cm}



{\Large \textbf{OPTION ÉCONOMIQUE}}



{\Large \textbf{MATHÉMATIQUES }}



{\Large Lundi 9 mai, de 14h à 18h}



\rule{2.39cm}{0.05cm}
\end{center}

\textit{La présentation, la lisibilité, l'orthographe, la qualité
de la rédaction, la clarté et la précision des raisonnements
entreront pour une part importante dans l'appréciation des copies.}

\textit{Les candidats sont invités à \textbf{encadrer} dans la mesure
du possible les résultats de leurs calculs.}

\textit{Ils ne doivent faire usage d'aucun document. L'utilisation de
toute
calculatrice et de tout matériel électronique est interdite. Seule
l'utilisation d'une règle graduée est autorisée.}

\textit{Si au cours de l'épreuve, un candidat repère ce qui lui semble
être une erreur d'énoncé, il la signalera sur sa copie et
poursuivra sa composition en expliquant les raisons des initiatives
qu'il sera
amené à prendre.}

\vspace*{3cm}

\section*{Exercice 1}

On considère la matrice carrée d'ordre trois suivante :
\[
A = 
\begin{smatrix}
0 & \frac{1}{2} & \frac{1}{2} \\
\frac{1}{2} & 0 & \frac{1}{2} \\
\frac{1}{2} & \frac{1}{2} & 0
\end{smatrix}
\]

\begin{noliste}{1.}
 \setlength{\itemsep}{4mm}
\item Montrer, sans calcul, que $A$ est diagonalisable.

\item Déterminer une matrice diagonale $D$ et une matrice inversible et
symétrique $P$, de première ligne $\begin{smatrix}
1 & 1 & 1
\end{smatrix}
$ et de deuxième ligne $\begin{smatrix}
1 & -1 & 0
\end{smatrix}
$, telles que $A = P~D~P^{-1}$.

Calculer $P^{-1}$.

\item Déterminer, pour tout $n\in \N^{\ast },$ la matrice $A^{n}$
par ses éléments.

\item Soient $u_{0},\ v_{0},\ w_{0}$ trois nombres réels positifs ou
nuls tels que $u_{0} + v_{0} + w_{0} = 1$.

On note $X_{0} = 
\begin{smatrix}
u_{0} \\
v_{0} \\
w_{0}\end{smatrix}
$ et, pour tout $n\in \N^{\ast },\ X_{n} = 
\begin{smatrix}
u_{n} \\
v_{n} \\
w_{n}\end{smatrix}
$ la matrice colonne définie par la relation de récurrence : $X_{n} =
A~X_{n-1}.$

\begin{noliste}{a)}
 \setlength{\itemsep}{2mm}
\item Montrer, pour tout $n\in \N :X_{n} = A^{n}X_{0}$

\item En déduire, pour tout $n\in \N :$

$\left\{ 
\begin{array}{c}
 u_{n} = \frac{1}{3} + \left( u_{0}-\frac{1}{3}\right) \left(
-\frac{1}{2}\right) ^{n} \\
 v_{n} = \frac{1}{3} + \left( v_{0}-\frac{1}{3}\right) \left(
-\frac{1}{2}\right) ^{n} \\
 w_{n} = \frac{1}{3} + \left( w_{0}-\frac{1}{3}\right) \left(
-\frac{1}{2}\right) ^{n}
\end{array}
\right. $

\item Déterminer les limites respectives $u,\ v,\ w$ de $u_{n}$,
$v_{n}$, $w_{n}$ lorsque le nombre entier $n$ tend vers l'infini.

\hspace{-1cm}On note, pour tout $n\in \N :d_{n} = \sqrt{\left(
u_{n}-u\right) ^{2} + \left( v_{n}-v\right) ^{2} + \left(
w_{n}-w\right) ^{2}}$

\item Montrer, pour tout $n\in \N :d_{n}\leq \dfrac{1}{2^{n-1}}$

\item Déterminer un entier naturel $n$ tel que : $d_{n}\leq 10^{-2}$
\end{noliste}
\end{noliste}

\section*{Exercice 2}

\subsection*{Préliminaire}

On donne : $0,69<\ln 2<0,70.$

On considère l'application :
\[
g :\left] 0; + \infty \right[ \ \rightarrow \R,\quad x\mapsto g\left(
x\right) = x^{2} + \ln x
\]

\begin{noliste}{1.}
 \setlength{\itemsep}{4mm}
\item Montrer que $g$ est continue et strictement croissante sur
$\left]
0; + \infty \right[ $ et déterminer les limites de $g$ en $0$ et en $ +
\infty $

\item Montrer que l'équation $g\left( x\right) = 0$, d'inconnue $x\in
\left] 0; + \infty \right[ $, admet une solution et une seule.

\hspace{-1cm}On note $\alpha $ l'unique solution de cette équation.

\item Montrer :\qquad $\frac{1}{2}<\alpha <1$
\end{noliste}

\subsection*{Partie A}

On note $I = \left[ \ \frac{1}{2},1\right] $ et on considère
l'application : 
\[
f :I\rightarrow \R,\quad x\mapsto f\left( x\right) =
x-\frac{1}{4}x^{2}-\frac{1}{4}\ln x
\]

\begin{noliste}{1.}
 \setlength{\itemsep}{4mm}
\item 
\begin{noliste}{a)}
 \setlength{\itemsep}{2mm}
\item Montrer que $f$ est strictement croissante sur $I$.

\item Montrer : $\frac{1}{2}<f\left( \frac{1}{2}\right) <f\left(
1\right) <1.
$

\item En déduire : $\forall x\in I,\quad f\left( x\right) \in I.$
\end{noliste}

\item On considère la suite réelle $\left( u_{n}\right)_{n\in 
\N}$ définie par $u_{0} = 1$ et, pour tout $n\in \N,\
u_{n + 1} = f\left( u_{n}\right) $.

\begin{noliste}{a)}
 \setlength{\itemsep}{2mm}
\item Calculer $u_{1}$

\item Montrer : $\forall n\in \N,\ u_{n}\in I$

\item Montrer que la suite $\left( u_{n}\right)_{n\in \N}$ est
décroissante.

\item Montrer que la suite $\left( u_{n}\right)_{n\in \N}$ converge
et que sa limite est le réel $\alpha $.
\end{noliste}
\end{noliste}

\subsection*{Partie B}

On considère l'application :
\[
F :\R_{+}{\ast }\times \R\rightarrow \R,\quad \left(
x,y\right) \mapsto F\left( x,y\right) = x~e^{y} + y~\ln x
\]

\begin{noliste}{1.}
 \setlength{\itemsep}{4mm}
\item 
\begin{noliste}{a)}
 \setlength{\itemsep}{2mm}
\item Montrer que $F$ est de classe $C^{1}$ sur $\R_{+}{\ast
}\times \R$ et calculer les dérivées partielles premières de $F$ en \
tout point $\left( x,y\right) $ de $\R_{+}{\ast
}\times \R$.

\item Montrer que $F$ admet un point critique et un seul que l'on
exprimera 
à l'aide du nombre réel $\alpha $.
\end{noliste}

\item Est-ce que $F$ admet un extremum local ?
\end{noliste}

\section*{EXERCICE 3}

\begin{noliste}{1.}
 \setlength{\itemsep}{4mm}
\item On considère l'application $f :\R\rightarrow \R$ \ définie pour
tout nombre réel $x$ par : 
\[
\left\{ 
\begin{array}{cc}
f\left( x\right) = e^{-x} & \text{si }x>0 \\
f\left( x\right) = 0 & \text{si }x\leq 0
\end{array}
\right.
\]
Montrer que $f$ est une densité de probabilité.

\hspace{-1cm}On considère une variable aléatoire $X$ admettant $f$
pour densité.

\item On définit la variable aléatoire discrète $Y$ à
valeurs dans $\N$ de la façon suivante :

\begin{noliste}{$\sbullet$}
\item[$\star $] l'événement $\left( Y = 0\right) $ est égal l'événement
$\left( X<1\right) $

\item[$\star $] pour tout nombre entier strictement positif $n$,
l'événement $\left( Y = n\right) $ est égal à l'événement $\left( n\leq
X<n + 1\right) $.
\end{noliste}

\begin{noliste}{a)}
 \setlength{\itemsep}{2mm}
\item Montrer, pour tout entier naturel $n :\Prob\left(\Ev{ Y =
n}\right)
 = \left( 1-\dfrac{1}{e}\right) e^{-n}$

\item Montrer que la variable aléatoire $Y + 1$ suit une loi
géométrique dont on précisera le paramètre.

En déduire l'espérance et la variance de $Y$.

\item Recopier et compléter le programme ci-dessous pour qu'il simule
la
variable aléatoire $Y$\\
\texttt{program eml2007;}

\texttt{var y :integer; u :real;}

\texttt{begin}

\texttt{\hspace{1cm}randomize;}

\texttt{\hspace{1cm}u : = random; y : = }$^{...}$\texttt{;}

\texttt{\hspace{1cm}while... do }

\texttt{\hspace{1cm}.........}

\texttt{\hspace{1cm}writeln('y vaut ', y);}

\texttt{end.}
\end{noliste}

\item Soit $U$ une variable de Bernoulli telle que $\Prob\left(\Ev{
U = 1}\right) = P\left(\Ev{ U = 0}\right) = \frac{1}{2}$.

On suppose que les variables aléatoires $U$ et $Y$ sont indépendantes.

Soit la variable aléatoire $T = \left( 2U-1\right) Y$, produit des
variables aléatoires $2U-1$ et $Y$.

Ainsi, $T$ est une variable aléatoire discrète à valeurs dans $\Z$,
l'ensemble des entiers relatifs.

\begin{noliste}{a)}
 \setlength{\itemsep}{2mm}
\item Montrer que la variable aléatoire $T$ admet une espérance
$\E\left( T\right) $ et calculer $\E\left( T\right) $

\item Vérifier que $T^{2} = Y^{2}$

En déduire que la variable aléatoire $T$ admet une variance $\V\left(
T\right) $ et calculer $\V\left( T\right) $

\item Pour tout nombre entier relatif $n$, calculer la probabilité
$\Prob\left(\Ev{ T = n}\right) $.
\end{noliste}

\item Soit la variable aléatoire $D = X-Y$. On note $F_{D}$ la fonction
de
répartition de $D$.

\begin{noliste}{a)}
 \setlength{\itemsep}{2mm}
\item Justifier : $\forall t\in \left] -\infty ;0\right[,\ F_{D}\left(
t\right) = 0$\hspace{5mm} et :\hspace{5mm} $\forall t\in \left[ 1; +
\infty \right[,\ F_{D}\left( t\right) = 1$.

\item Soit $t\in \left[ 0;1\right[ $. Exprimer l'événement $\left(
D\leq t\right) $ à l'aide des événements $\left( n\leq X\leq
n + t\right) $, $n\in \N$

\item Pour tout nombre réel $t\in \left[ 0;1\right[ $ et pour tout
nombre entier naturel $n$, calculer la probabilité $\Prob\left(\Ev{
n\leq X\leq n + t}\right).$

\item Montrer : $\forall t\in \left[ 0;1\right[,\ F_{D}\left( t\right)
= \dfrac{1-e^{-t}}{1-e^{-1}}$

\item Montrer que $D$ est une variable aléatoire à densité. Déterminer
une densité de $D$.
\end{noliste}
\end{noliste}

\end{document}

