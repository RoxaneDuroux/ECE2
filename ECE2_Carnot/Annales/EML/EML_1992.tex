\documentclass[11pt]{article}%
\usepackage{geometry}%
\geometry{a4paper,
 lmargin = 2cm,rmargin = 2cm,tmargin = 2.5cm,bmargin = 2.5cm}

\input{../../macros.tex}

\pagestyle{fancy} %
\lhead{ECE2 \hfill Mathématiques\\
} %
\chead{\hrule} %
\rhead{} %
\lfoot{} %
\cfoot{} %
\rfoot{\thepage} %

\renewcommand{\headrulewidth}{0pt}% : Trace un trait de séparation
 % de largeur 0,4 point. Mettre 0pt
 % pour supprimer le trait.

\renewcommand{\footrulewidth}{0.4pt}% : Trace un trait de séparation
 % de largeur 0,4 point. Mettre 0pt
 % pour supprimer le trait.

\setlength{\headheight}{14pt}

\title{\bf \vspace{-2cm} EML 1992} %
\author{} %
\date{} %
\begin{document}

\maketitle %
\vspace{-1.4cm}\hrule %
\thispagestyle{fancy}

\vspace*{.2cm}


% DEBUT DU DOC À MODIFIER : tout virer jusqu'au début de l'exo

%Définition et changement de valeurs de
compteurs%newcounter{cpt1}{section} compteur cpt1 remis à 0 à chaque
aumentation par stepcounter du compteur section%setcounter{cpt1}{3} on
met le compteur à 3%addtocounter{cpt1}{5} on ajoute 5 au compteur%
stepcounter{cpt1} on ajoute 1% ifthenelse{test}{alors}{sinon} (page
206) pour subordonner à une condition % whiledo{test}{commande} pour
faire une boucle (page 206 aussi) % value{cpt1} pour noter dans le
document la valeur de cpt1 
%Définition définitive d'opérateurs
mathématiques\newcommand{\ch}{\operatorname{ch}} 
\newcommand{\sh}{\operatorname{sh}}
\renewcommand{\tanh}{\operatorname{th}}
\renewcommand{\sinh}{\operatorname{sh}}
\renewcommand{\cosh}{\operatorname{ch}}
\newcommand{\argsh}{\operatorname{argsh}}
\newcommand{\argch}{\operatorname{argch}}
\newcommand{\argth}{\operatorname{argth}}
\newcommand{\ker}{\operatorname{Ker}}
\renewcommand{\im}{\operatorname{Im}}
\newcommand{\rg}{\operatorname{rg}}
\newcommand{\Id}{\operatorname{Id}}
\newcommand{\id}{\operatorname{id}}
\renewcommand{\leq}{\leq}
\renewcommand{\geq}{\geq }

%Définition de nouvelles couleurs : rgb(trois paramètres red green blue
entre 0 et 1); cmyk (quatre cyan magenta yellow black) entre 0 et 1;
gray (entre 0 et 1) et black, white, red, green, blue, cyan, magenta,
yellow% definecolor{0gris}{gray}{0.8} 
% Nouvelle commande pour encadrer le titre car shabox ne veut que d'une
seule ligne; ATTENTION A LA TAILLE; petite différence avec shadowbox ou
doublebox, voire fcolorbox ou colorbox (au lieu de shabox; laisser le
parbox tranquille sauf pour la taille de la boîte
\newcommand{\Tbox}[1]{\begin{center} \shabox{\parbox{0.6
\linewidth}{#1}} \end{center}} %[1] pour 1 paramètre ; #1 pour ce que
fait le 1er paramètre; entre accolades ce que fait la commande
%Mise en page en mode fancy : en-têtes et pieds de pages puis
définition des en-têtes et pieds de pages\pagestyle{fancy}
\lhead{ECE 2 - Mathématiques \\
Quentin Dunstetter - ENC-Bessières 2011$\backslash$2012}
\chead{}
\rhead{EML 1992}
\rfoot[ \ \thepage]{\thepage}
\cfoot{}
\lfoot{}
\thispagestyle{fancy} %Mise en page de la 1ère page en mode fancy
%Trait en bas et en haut de la page (entre en-tête et texte et texte et
pied de page)\renewcommand{\footrulewidth}{0.4pt}
\renewcommand{\headrulewidth}{0.4pt}


%DEBUT DU DOCUMENT\vspace*{3cm}

\begin{center}
{\LARG\E\textbf{BANQUE COMMUNE D'ÉPREUVES}}



{\large \textsc{CONCOURS D ADMISSION DE 1992}}



{\large \textbf{Concepteur : EML}}



\rule{2.39cm}{0.05cm}



{\Large \textbf{OPTION ÉCONOMIQUE}}



{\Large \textbf{MATHÉMATIQUES }}



{\Large Lundi 9 mai, de 14h à 18h}



\rule{2.39cm}{0.05cm}
\end{center}

\textit{La présentation, la lisibilité, l'orthographe, la qualité
de la rédaction, la clarté et la précision des raisonnements
entreront pour une part importante dans l'appréciation des copies.}

\textit{Les candidats sont invités à \textbf{encadrer} dans la mesure
du possible les résultats de leurs calculs.}

\textit{Ils ne doivent faire usage d'aucun document. L'utilisation de
toute
calculatrice et de tout matériel électronique est interdite. Seule
l'utilisation d'une règle graduée est autorisée.}

\textit{Si au cours de l'épreuve, un candidat repère ce qui lui semble
être une erreur d'énoncé, il la signalera sur sa copie et
poursuivra sa composition en expliquant les raisons des initiatives
qu'il sera
amené à prendre.}

\vspace*{3cm}

\section*{EXERCICE 1}

L'espace vectoriel $\R^{3}$ est muni de sa base canonique $\mathcal{B =
}\left(
\overrightarrow{e_{1}},\overrightarrow{e_{2}},\overrightarrow{e_{3}}\ri
ht) $\\
Soit $\left( a_{1},a_{2},a_{3}\right) \in \R^{3}$ tel que
$a_{1}a_{2}a_{3}\neq 0$;\\
on considère

\begin{noliste}{$\sbullet$}
\item $\overrightarrow{a} = \left( a_{1},a_{2},a_{3}\right) $

\item $A$ la matrice de $\M{3} $ dont le
terme situé à la ligne $i$ et colonne $j$ vaut $a_{i}a_{j}$ pour tout
$\left( i,j\right) $ de$\left\{ 1,2,3\right\} $,

\item $f$ l'endomorphisme de $\R^{3}$ de matrice $A$ dans
$\mathcal{B}$.
\end{noliste}

\begin{noliste}{1.}
 \setlength{\itemsep}{4mm}
\item Montrer que $\overrightarrow{a}$ est vecteur propre de $f$,
associé à
une valeur propre que l'on déterminera.

\item Déterminer $\ker \left( f\right) $.

\item Montrer qu'il existe une base de $\R^{3}$ formée de vecteurs
propres de $f$ ; donner une telle base et écrire la matrice de $f$ dans
cette base.

\item Calculer $A^{n}$ pour tout $n\in \N$.
\end{noliste}

\section*{EXERCICE 2}

On note $f :\left] 1, + \infty \right[ \ \rightarrow \R$ l'application
définie par : 
\[
\forall x\in \left] 1, + \infty \right[,\quad f\left( x\right) =
\dfrac{1}{x\ln \left( x\right) }
\]

\begin{noliste}{1.}
 \setlength{\itemsep}{4mm}
\item Étudier les variations de $f$ et tracer sa courbe représentative.

\item Montrer, pour tout entier $k$ tel que $k\geq 3$ : 
\[
f\left( k\right) \leq \dint{k-1}{k}f\left( x\right\dx\leq
f\left( k-1\right)
\]

\hspace{-1cm}Pour tout $n\in \N$ tel que $n\geq 2$n, on note $S_{n} =
\Sum{k = 2}{n}f\left( k\right) $

\item 

\begin{noliste}{a)}
 \setlength{\itemsep}{2mm}
\item Montrer, pour tout $n\in \N$ tel que $n\geq 2$ :

\[
S_{n}-\dfrac{1}{2\ln \left( 2\right) }\leq \dint{2}{n}f\left(
x\right\dx\leq S_{n}-\dfrac{1}{n\ln \left( n\right) }
\]

\item En déduire, pour tout $n\in \N$ tel que $n\geq 2$ :

\[
\ln \left( \ln \left( n\right) \right) -\ln \left( \ln \left( 2\right)
\right) \leq S_{n}\leq \ln \left( \ln \left( n\right) \right) -\ln
\left( \ln \left( 2\right) \right) + \dfrac{1}{2\ln \left( 2\right) }
\]

\item établir : $S_{n}\underset{n\rightarrow + \infty }{\thicksim }\ln
\left(
\ln \left( n\right) \right) $\\
Pour tout $n\in \N$ tel que $n\geq 2$, on note 
\[
u_{n} = S_{n}-\ln \left( \ln \left( n + 1\right) \right)
\;\text{et}\;v_{n} = S_{n}-\ln \left( \ln \left( n\right) \right) 
\]
\end{noliste}

\item En utilisant le résultat de la question 2., montrer que les
suites $\left( u_{n}\right)_{n\geq 2}$ et $\left( v_{n}\right)_{n\geq
2}$
sont adjacentes. On note $\ell $ leur limite commune.

\item 

\begin{noliste}{a)}
 \setlength{\itemsep}{2mm}
\item Montrer, pour tout $n\in \N$ tel que $n\geq 2$ : 
\[
0\leq v_{n}-\ell \leq \dfrac{1}{n\ln \left( n\right) }
\]

\item En déduire une valeur approchée de $\ell $ à $10^{-2}$ près.
\end{noliste}
\end{noliste}

\section*{EXERCICE 3}

Soit $N$ un entier naturel supérieur ou égal à 2.

\begin{noliste}{1.}
 \setlength{\itemsep}{4mm}
\item Montrer les égalités suivantes : 
\[
\Sum{k = 1}{N}k = \dfrac{N\left( N + 1\right) }{2},\quad
\Sum{k = 1}{N}k^{2} = \dfrac{N\left( N + 1\right) \left( 2N + 1\right)
}{6},
\]

\item Une urne contient une boule blanche, une boule verte et $N-2$
boules
rouges. Ces boules sont indiscernables au toucher.\\
On tire successivement les $N$ boules sans remettre les boules tirées
dans
l'urne.\\
On note $X_{1}$ la variable aléatoire égale au rang du tirage de la
boule
blanche et $X_{2}$ la variable aléatoire égale au rang du tirage de la
boule
verte.

\begin{noliste}{a)}
 \setlength{\itemsep}{2mm}
\item Soient $i$ et $j$ deux entiers compris entre 1 et $N$.\\
Calculer la probabilité $P_{ij}$ pour que $X_{1} = i$ et $X_{2} = j$.\\
(On distinguera le cas $i = j$ et le cas $i\neq j$).

\item Déterminer les lois des variables aléatoires $X_{1}$ et
$X_{2}$.\\
Est-ce que les variables aléatoires $X_{1}$ et $X_{2}$ sont
indépendantes ?\\
Calculer les espérances et variances des variables aléatoires $X_{1}$
et $X_{2}$.

\item On note $X$ la variable aléatoire égale au rang du tirage où l'on
obtient pour la première fois soit la boule blanche soit la boule
verte.\\
On note $Y$ la variable aléatoire égale au rang du tirage à partir
duquel on
a obtenu la boule blanche et la boule verte.\\
Remarque : en fait $X = \inf \left( X_{1},X_{2}\right) $ et $Y = \sup
\left(
X_{1},X_{2}\right) $\\
Par exemple, si on a tiré rouge, rouge, verte, rouge, blanche, alors
$X_{1} = 5
$ et $X_{2} = 3$ et $X = 3$ et $Y = 5$\\
Déterminer les lois des variables aléatoires $X$ et $Y$.\\
Calculer les espérances des variables aléatoires $X$ et $Y$.
\end{noliste}
\end{noliste}

\section*{EXERCICE 4}

On considère la fonction $f$ définie sur $\R$ par $\left\{ 
\begin{array}{ccc}
f\left( x\right) = 0 & \text{si} & x<0 \\
f\left( x\right) = x & \text{si} & x\geq 0
\end{array}
\right. $\\
On se propose d'étudier la suite réelle $\left( u_{n}\right) $ définie
par
la donnée de son premier terme $u_{0}$ et la relation de récurrence 
\[
\forall n\in \N,\;u_{n + 1} = u_{n} + \dint{0}{1}f\left(
t-u_{n}\right\ dt
\]

\begin{noliste}{1.}
 \setlength{\itemsep}{4mm}
\item Étude du cas $0\leq u_{0}\leq 1$\\
On suppose $0\leq u_{0}\leq 1$

\begin{noliste}{a)}
 \setlength{\itemsep}{2mm}
\item Montrer que la suite $\left( u_{n}\right) $ est croissante.

\item Montrer que, si $0\leq u_{n}\leq 1,$ alors $u_{n + 1} =
\dfrac{1}{2}\left( 1 + u_{n}{2}\right).$\\
En déduire que, pour tout entier positif ou nul $n$, $u_{n}\leq 1$.

\item Montrer que la suite $\left( u_{n}\right) $ est convergente ;
déterminer sa limite.
\end{noliste}

\item Étude des cas $u_{0}<0$ et $u_{0}>1$.

\begin{noliste}{a)}
 \setlength{\itemsep}{2mm}
\item On suppose $u_{0}<0.$\\
Calculer $u_{1}$. En déduire l'étude de la suite $\left( u_{n}\right)
$.

\item On suppose $u_{0}>1$.\\
Calculer $u_{1}$, puis, pour tout entier positif $n$,$u_{n}$. Que dire
de la
suite $\left( u_{n}\right) $ ?
\end{noliste}

\item Interprétation graphique.\\
On considère la fonction $g$ définie sur $\R$ par 
\[
g\left( x\right) = x + \dint{0}{1}f\left( t-x\right\ dt
\]

\begin{noliste}{a)}
 \setlength{\itemsep}{2mm}
\item Calculer pour tout nombre réel $x$ la valeur de $g\left( x\right)
$.\\
Construire le graphe de $g$ dans un repère orthonormé (unité graphique
2 cm).

\item Représenter graphiquement les quatre premiers termes de la suite
$\left( u_{n}\right) $ dans les cas suivants : 
\[
u_{0} = -2;\quad u_{0} = 0;\quad u_{0} = 2
\]
\end{noliste}
\end{noliste}

\label{fin}

\end{document}

