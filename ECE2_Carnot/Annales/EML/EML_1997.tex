\documentclass[11pt]{article}%
\usepackage{geometry}%
\geometry{a4paper,
 lmargin = 2cm,rmargin = 2cm,tmargin = 2.5cm,bmargin = 2.5cm}

\input{../../macros.tex}

\pagestyle{fancy} %
\lhead{ECE2 \hfill Mathématiques\\
} %
\chead{\hrule} %
\rhead{} %
\lfoot{} %
\cfoot{} %
\rfoot{\thepage} %

\renewcommand{\headrulewidth}{0pt}% : Trace un trait de séparation
 % de largeur 0,4 point. Mettre 0pt
 % pour supprimer le trait.

\renewcommand{\footrulewidth}{0.4pt}% : Trace un trait de séparation
 % de largeur 0,4 point. Mettre 0pt
 % pour supprimer le trait.

\setlength{\headheight}{14pt}

\title{\bf \vspace{-2cm} EML 1997} %
\author{} %
\date{} %
\begin{document}

\maketitle %
\vspace{-1.4cm}\hrule %
\thispagestyle{fancy}

\vspace*{.2cm}


% DEBUT DU DOC À MODIFIER : tout virer jusqu'au début de l'exo

%Définition et changement de valeurs de
compteurs%newcounter{cpt1}{section} compteur cpt1 remis à 0 à chaque
aumentation par stepcounter du compteur section%setcounter{cpt1}{3} on
met le compteur à 3%addtocounter{cpt1}{5} on ajoute 5 au compteur%
stepcounter{cpt1} on ajoute 1% ifthenelse{test}{alors}{sinon} (page
206) pour subordonner à une condition % whiledo{test}{commande} pour
faire une boucle (page 206 aussi) % value{cpt1} pour noter dans le
document la valeur de cpt1 
%Définition définitive d'opérateurs
mathématiques\newcommand{\ch}{\operatorname{ch}} 
\newcommand{\sh}{\operatorname{sh}}
\renewcommand{\tanh}{\operatorname{th}}
\renewcommand{\sinh}{\operatorname{sh}}
\renewcommand{\cosh}{\operatorname{ch}}
\newcommand{\argsh}{\operatorname{argsh}}
\newcommand{\argch}{\operatorname{argch}}
\newcommand{\argth}{\operatorname{argth}}
\newcommand{\ker}{\operatorname{Ker}}
\renewcommand{\im}{\operatorname{Im}}
\newcommand{\rg}{\operatorname{rg}}
\newcommand{\Id}{\operatorname{Id}}
\newcommand{\id}{\operatorname{id}}
\renewcommand{\leq}{\leq}
\renewcommand{\geq}{\geq }

%Définition de nouvelles couleurs : rgb(trois paramètres red green blue
entre 0 et 1); cmyk (quatre cyan magenta yellow black) entre 0 et 1;
gray (entre 0 et 1) et black, white, red, green, blue, cyan, magenta,
yellow% definecolor{0gris}{gray}{0.8} 
% Nouvelle commande pour encadrer le titre car shabox ne veut que d'une
seule ligne; ATTENTION A LA TAILLE; petite différence avec shadowbox ou
doublebox, voire fcolorbox ou colorbox (au lieu de shabox; laisser le
parbox tranquille sauf pour la taille de la boîte
\newcommand{\Tbox}[1]{\begin{center} \shabox{\parbox{0.6
\linewidth}{#1}} \end{center}} %[1] pour 1 paramètre ; #1 pour ce que
fait le 1er paramètre; entre accolades ce que fait la commande
%Mise en page en mode fancy : en-têtes et pieds de pages puis
définition des en-têtes et pieds de pages\pagestyle{fancy}
\lhead{ECE 2 - Mathématiques \\
Quentin Dunstetter - ENC-Bessières 2011$\backslash$2012}
\chead{}
\rhead{EML 1997}
\rfoot[ \ \thepage]{\thepage}
\cfoot{}
\lfoot{}
\thispagestyle{fancy} %Mise en page de la 1ère page en mode fancy
%Trait en bas et en haut de la page (entre en-tête et texte et texte et
pied de page)\renewcommand{\footrulewidth}{0.4pt}
\renewcommand{\headrulewidth}{0.4pt}


%DEBUT DU DOCUMENT\vspace*{3cm}

\begin{center}
{\LARG\E\textbf{BANQUE COMMUNE D'ÉPREUVES}}



{\large \textsc{CONCOURS D ADMISSION DE 1997}}



{\large \textbf{Concepteur : EML}}



\rule{2.39cm}{0.05cm}



{\Large \textbf{OPTION ÉCONOMIQUE}}



{\Large \textbf{MATHÉMATIQUES }}



{\Large Lundi 9 mai, de 14h à 18h}



\rule{2.39cm}{0.05cm}
\end{center}

\textit{La présentation, la lisibilité, l'orthographe, la qualité
de la rédaction, la clarté et la précision des raisonnements
entreront pour une part importante dans l'appréciation des copies.}

\textit{Les candidats sont invités à \textbf{encadrer} dans la mesure
du possible les résultats de leurs calculs.}

\textit{Ils ne doivent faire usage d'aucun document. L'utilisation de
toute
calculatrice et de tout matériel électronique est interdite. Seule
l'utilisation d'une règle graduée est autorisée.}

\textit{Si au cours de l'épreuve, un candidat repère ce qui lui semble
être une erreur d'énoncé, il la signalera sur sa copie et
poursuivra sa composition en expliquant les raisons des initiatives
qu'il sera
amené à prendre.}

\vspace*{3cm}


\begin{center}
{\LARGE Exercice 1}
\end{center}

Soient $a,b,c$ trois reels tous non nuls, et $M$ la matrice carrée
d'ordre 3 suivante : $M = \left( 
\begin{array}{lll}
1 & a/b & a/c \\
b/a & 1 & b/c \\
c/a & c/b & 1
\end{array}
\right) $

\begin{noliste}{1.}
 \setlength{\itemsep}{4mm}
\item 
\begin{noliste}{a)}
 \setlength{\itemsep}{2mm}
\item Montrer $M^{2} = 3M$

\item En déduire que l'ensemble des valeurs propres de $M$ est inclus
dans $\left\{ 0,3\right\} $.
\end{noliste}

\item 
\begin{noliste}{a)}
 \setlength{\itemsep}{2mm}
\item Déterminer les valeurs propres de $M$ et, pour chaque valeur
propre, une base du sous-espaces propre associe.

\item La matrice $M$ est-elle diagonalisable ?
\end{noliste}

\hspace{-1cm}On note : 
\[
P = \left( 
\begin{array}{ccc}
a & a & a \\
b & -b & 0 \\
c & 0 & -c
\end{array}
\right) \;D = \left( 
\begin{array}{lll}
3 & 0 & 0 \\
0 & 0 & 0 \\
0 & 0 & 0
\end{array}
\right) \;Q = \left( 
\begin{array}{ccc}
1/a & 1/b & 1/c \\
1/a & -2/b & 1/c \\
1/a & 1/b & -2/c
\end{array}
\right) 
\]

\item 
\begin{noliste}{a)}
 \setlength{\itemsep}{2mm}
\item Calculer $PQ$. Montrer que $P$ est inversible. Quel est son
inverse ?

\item Vérifier $ :M = PDP^{-1}$
\end{noliste}

\item Déterminer l'ensemble des matrices $Y$ de $\M{3} $ telles que
$DY-YD = 3Y$

\item Montrer que l'ensemble des matrices $X$ de $\M{3} $ telles que
$MX-XM = 3X$ est un espace vectoriel de
dimension 2 sur $\R$.
\end{noliste}

\begin{center}
{\LARGE Exercice 2}
\end{center}

Le but de l'exercice est l'étude des extremums de la fonction $f : 
\begin{array}[t]{lll}
\R^{2} & \longrightarrow & \R \\
\left( x,y\right) & \longmapsto & x^{2}-2xy + 2y^{2} + e^{-x}
\end{array}
$

\begin{noliste}{1.}
 \setlength{\itemsep}{4mm}
\item 
\begin{noliste}{a)}
 \setlength{\itemsep}{2mm}
\item Établir que l'équation $ = x$, d'inconnue $x\in \R$, admet une
solution et une seule.

\item Montrer qu'il existe $\left( x_{0},y_{0}\right) \in \R^{2}$
unique tel que : 
\[
\left\{ 
\begin{array}{c}
 \frac{\partial f}{\partial x}\left( x_{0},y_{0}\right)
 = 0\\
 \frac{\partial f}{\partial y}\left( x_{0},y_{0}\right) = 0
\end{array}
\right. \text{ et établir que }\left\{ 
\begin{array}{c}
 x_{0}-e^{-x_{0}} = 0 \\
 y_{0} = \frac{x_{0}}{2}
\end{array}
\right. 
\]

\item Montrer que $f$ admet un extremum en $\left( x_{0},y_{0}\right)
$.
Est-ce un minimum ou un maximum ?
\end{noliste}

\item On note $g : 
\begin{array}[t]{lll}
\left[ 0, + \infty \right[ & \longrightarrow & \R \\
x & \longmapsto &  \frac{1 + x}{1 + e^{x}}
\end{array}
$

\begin{noliste}{a)}
 \setlength{\itemsep}{2mm}
\item Montrer que l'équation $g(x) = x$, d'inconnue $x\in \left[ 0, +
\infty \right[ $, admet une solution et une seule, que celle-ci est
$x_{0} $, et
que $\frac{1}{2}$ $<x_{0}<1$

\item Former le tableau des variations de $g$ et tracer sa courbe
représentative (repère orthonormé, unité : 5 cm).

\hspace{-1cm}On considère la suite $(u_{n})_{n\in \N}$ définie par
$u_{0} = 0$ et, pour tout $n$ de $\N,\;u_{n + 1} = g\left(
u_{n}\right) $

\item Établir que la suite $(u_{n})_{n\in \N}$ est croissante et
converge vers $x_{0}$.

\item 
\begin{nonoliste}{(i)}
\item Montrer : $\forall x\in \left[ \ \frac{1}{2};1\right] \;,\;\left|
g^{\prime }\left( x\right) \right|
\leq 0,125$, où $g^{\prime }$ désigne la dérivée de $g.$

On donne $g^{\prime }\left( \frac{1}{2}\right) \simeq 0,025$ et
$g^{\prime
}\left( 1\right) \simeq -0,124$

\item Établir : $\forall n\in \N^{*}\;,\;\left|
u_{n}-x_{0}\right| \leq \left( 0,125\right) ^{n-1}\cdot 0,5$

\item Écrire un programme \Scilab{} qui calcule et affiche une valeur
approchée de $x_{0}$ à $10^{-8}$ près.

\item Montrer que $f(x_{0},y_{0}) = \frac{x_{0}{2}}{2} + x_{0}$, et en
déduire une valeur approchée décimale à $10^{-7}$ près de
$f(x_{0},y_{0})$.
\end{nonoliste}
\end{noliste}
\end{noliste}

\begin{center}
{\LARGE Exercice 3}
\end{center}

On dispose d'un dé équilibré à 6 faces et d'une pièce
truquée telle que la probabilité d'apparition de ''pile'' soit égale à
$p\;,\;p\in \left] 0;1\right[ $.

On pourra noter $q = 1-p$.

Soit $N$ un entier naturel non nul fixé.

On effectue $N$ lancers du dé ; si $n$ est le nombre de " 6" obtenus,
on
lance alors $n$ fois la pièce.

On définit trois variables aléatoires $X,\;Y,\;Z$ de la manière
suivante :

\begin{noliste}{$\sbullet$}
\item $Z$ indique le nombre de ''6'' obtenus aux lancers du dé,

\item $X$ indique le nombre de ''piles'' obtenus aux lancers de la
pièce,

\item $Y$ indique le nombre de ''faces'' obtenues aux lancers de la
pièce.
\end{noliste}

Ainsi, $X + Y = Z$ et, si $Z$ prend la valeur 0, alors $X$ et $Y$
prennent la
valeur 0.

\begin{noliste}{1.}
 \setlength{\itemsep}{4mm}
\item Préciser la loi de $Z$, son espérance et sa variance.

\item Pour $k\in \N$, $n\in \N$, déterminer la
probabilité conditionnelle $P\left(\Ev{X = k/Z = n}\right)$. On
distinguera les cas : $k\le
n $ et $k>n.$

\item Montrer, pour tout couple d'entiers naturels $(k,n)$ :

\begin{noliste}{$\sbullet$}
\item si $0\leq k\leq n\leq N$ alors $ P\left(\Ev{ X = k\text{ et }Z =
n}\right) = \binom{n}{k}\binom{N}{n}\cdot p^{k}\left( 1-p\right)
^{n-k}\left( \frac{5}{6}\right) ^{N-n}$ $\left( \frac{1}{6}\right)
^{n}$

\item si $n>N$ ou $k>n$ alors $ P\left(\Ev{ X = k\text{ et }Z =
n}\right) = 0$
\end{noliste}

\item Calculer la probabilité $P\left(\Ev{X = 0}\right)$

\item Montrer pour tout couple d'entiers naturels $(k,n)$ tel que
$0\leq
k\leq n\leq N :$
\[
\binom{n}{k}\binom{N}{n} = \binom{N}{k}\binom{N-k}{n-k} 
\]
En déduire la probabilité $P\left(\Ev{X = k}\right)$.

\item Montrer que la variable aléatoire $X$ suit une loi binomiale de
paramètre $\frac{p}{6})$.

Quelle est la loi de la variable aléatoire $Y$ ?

\item Est-ce que les variables aléatoires $X$ et $Y$ sont indépendantes
?

Déterminer la loi du couple $(X,Y)$.

\item En comparant les variances de $Z$ et de $X + Y$, déterminer la
covariance du couple $(X,Y)$.
\end{noliste}

\end{document}

